\chapter{Survey of the grammars of questions in Northeast Asia}\label{sec:5}

This chapter takes a closer look at the \isi{grammar of questions} in language families of \isi{NEA} in alphabetical order, from Ainuic (\sectref{sec:5.1}) to Yeniseic (\sectref{sec:5.14}). Each section on a \isi{language family} is divided into three parts, a brief introduction that sketches its internal classification, a section on \isi{question marking}, and one on interrogatives. For practical purposes, subsections in the larger language families Indo-European (\sectref{sec:5.5}) and Trans-Himalayan (\sectref{sec:5.9}) are distinguished additionally into subbranches such as Germanic or Sinitic. The part on Yeniseic has an additional subsection on the Dene-Yeniseian hypothesis (\sectref{sec:5.13.4}). Please note that, except for perhaps \ili{Tungusic}, the classification of each \isi{language family} is not exhaustive and is mostly intended as a tool that allows to better understand the internal order of the individual subsections.
\section{Ainuic}\label{sec:5.1}
\subsection{Classification of Ainuic}

\ili{Ainuic} has three dialect groups that are named after their geographical distribution. These are the \isi{Sakhalin} dialects, \isi{Kuril} dialects, and \isi{Hokkaid\=o} dialects. Excluding the possible existence of now extinct \ili{Para-Ainuic} varieties on \isi{Honshū}, the \ili{Ainuic} \isi{language family} may roughly be classified as follows (cf. \citealt{Vovin1993}: 157, see also \figref{fig:ainu:1} in \sectref{sec:5.1.3}).

\ea\upshape%1
    \label{ex:ainu:1}
\begin{forest}  for tree={grow'=east,delay={where content={}{shape=coordinate}{}}},   forked edges  
[
    [\textsuperscript{†}\isi{Sakhalin} dialects
    ]
    [\isi{Hokkaid\=o}-\isi{Kuril}
        [\textsuperscript{†}\isi{Kuril} dialects]
        [\textsuperscript{(†)}Hokkaid\=o dialects]
    ]
]
\end{forest}   
    \z

According to \citet[4]{Shibatani1990} there is what he calls “Classical \ili{Ainu}”, the language of oral epics (\textit{\isi{yukar}}), which differs from the spoken language and allegedly represents older stages of development. But \citet[378]{NakagawaOkuda2007} claim that

\begin{quote}
it is misleading to describe the grammar of \ili{Ainu} as resting upon this distinction, because the behaviour and distribution of so-called “classical” features are actually independent from each other. There is no sound evidence to support the claim that the “classical” features are really older than “colloquial” ones in the history of this language.
\end{quote}

The following description of \isi{question marking} is mostly based on the Saru and Chitose dialects in southwestern \isi{Hokkaid\=o} as well as the Shizunai and Tokachi dialects in southeastern Hokkaid\=o. The \isi{Sakhalin} and \isi{Kuril} dialects will only be mentioned briefly. A more complex picture including almost all dialects can be drawn for the \isi{interrogative} system.

\subsection{Question marking in Ainuic}\label{sec:5.1.2}

For marking polar \isi{questions}, the \textbf{Saru} dialect of \ili{Ainu} has final rising \isi{intonation} combined with an optional final particle \textit{ya}.

\ea%2
    \label{ex:ainu:2}
    \ili{Ainu} (Saru)\\
    \gll nisatta   nupurpet  or  un  e=arpa \textbf{{ya}}?\\
    tomorrow   \textsc{pn}    place  \textsc{all}  2\textsc{sg}.S=go.\textsc{pl}  \textsc{q}\\
    \glt ‘Will you go to Noboribetsu tomorrow?’ \citep[497]{Bugaeva2012}
    \z

\noindent The online \textit{Topical Dictionary of Conversational Ainu} based on the Saru dialect contains a section called \textit{Question and Answer} from which the following example of an unmarked \isi{polar question} with a longish and slightly rising \isi{intonation} towards the end was drawn.

\ea%3
    \label{ex:ainu:3}
    \ili{Ainu} (Saru)\\
    \gll ku=ye    itak    e=raman?\\
    1\textsc{sg}.A=say  language  2\textsc{sg}.A=know\\
    \glt ‘Do you understand what I am saying?’ (\citealt{NINJAL2015})
    \z

\noindent Interrogatives are \textit{in situ} and there is usually no additional morphosyntactic marking, though the use of \textit{ya} is possible. Alternative questions exhibit \isi{double marking} with \textit{ya}.

\ea%4
    \label{ex:ainu:4}
    \ili{Ainu} (Saru)\\
    \ea
    \gll \textbf{{hunna} } ek?\\
    who     come\\
    \glt ‘Who came?’

    \ex
    \gll \textbf{{hemanta}} e=e    rusuy \textbf{{ya}}?\\
    what    2\textsc{sg}.S=\textsc{cop}  want  \textsc{q}\\
    \glt ‘What would you like to eat?’ (\citealt{Tamura2000}: 235, 236)

    \ex
    \gll ek \textbf{{ya}}, \textbf{{somo}} \textbf{{ya}}?\\
    come  \textsc{q}  \textsc{neg}  \textsc{q}\\
    \glt ‘Are (you) coming or not?’ (\citealt{NINJAL2015})
    \z
    \z 

\noindent The use of the same final particle for both polar and \isi{content question}s has an areal connection to surrounding languages (\sectref{sec:6}). Alternative and (truncated) \isi{polar question}s are both marked with \textit{=he}.

\ea%5
    \label{ex:ainu:5}
    \ili{Ainu} (Saru)\\
    \gll matci=\textbf{{he}} e=kor    rusuy  tampaku=\textbf{{he}} e=kor    rusuy?\\
    match=\textsc{q}  2\textsc{sg}.S\textit{=}have  want  cigarette=\textsc{q}  2\textsc{sg}.S\textit{=}have  want\\
    \glt ‘Do you want the matches or the cigarettes?’ \citep[234]{Tamura2000}
    \z

\ea%6
    \label{ex:ainu:6}
    \ea
    \gll toan  pe  ekte.\\
    that  thing  hand.over\\
    \glt ‘Give that to me.’

    \ex
    \gll tan  pe=\textbf{{he}}?\\
    this  thing=\textsc{q}\\
    \glt ‘This one?’ \citep[234]{Tamura2000}
    \z
    \z

\noindent In one recorded open \isi{alternative question}, perhaps because \textit{=he} cannot combine with interrogatives, only the first alternative takes the marker \textit{=he}.

\ea%7
    \label{ex:ainu:7}
    \ili{Ainu} (Saru)\\
    \gll te  ta=\textbf{{he}}, \textbf{{hunak}} ta?\\
    here  \textsc{loc}=\textsc{q}    where  \textsc{loc}\\
    \glt ‘Here or where?’ (\citealt{NINJAL2015})
    \z

\noindent An additional marking of polar \isi{questions} grammaticalized from \isi{nominalization} has parallels in \ili{Japanese} (see \sectref{sec:5.6.2}). According to \citet[497]{Bugaeva2012} the final copula \textit{ne} may be omitted following the evidential infinitive marker \textit{ruwe} (\textit{ne}) ‘it is a fact that’, which in turn seems to mark polar \isi{questions}. The same pattern can be observed in the Chitose dialect \citep[85]{Bugaeva2004}. \citet[233]{Tamura2000} claims that the same development is also possible with other evidential markers, notably \textit{hawe} (\textit{ne}) ‘it is said that’ and \textit{siri} (\textit{ne}) ‘it looks that’. It appears that the newly grammaticalized question markers may be present in polar, alternative, and \isi{content question}s.

\ea%8
    \label{ex:ainu:8}
    \ili{Ainu} (Saru)\\
    \ea
    \gll yosiku  e=ne \textbf{{ruwe}}?\\
    \textsc{pn}  2\textsc{sg}.S=\textsc{cop}  \textsc{q}\\
    \glt ‘Are you Yoshiko?’ \citep[497]{Bugaeva2012}

    \ex
    \gll na  tuyma \textbf{{ruwe}},  hanke \textbf{{ruwe}}?\\
    still  far  \textsc{q}  be.close  \textsc{q}\\
    \glt ‘Is it far or near?’

    \ex
    \gll \textbf{{makanak}} pak-no    sir-tuyma \textbf{{ruwe}}?\\
    what    till-\textsc{adv}  appearance-be.far  \textsc{q}\\
    \glt ‘How far is it?’ (\citealt{NINJAL2015})
    \z
    \z 

\noindent The translation of the three markers above was taken from \citegen[494]{Bugaeva2012} description, which contains yet another evidential marker \textit{humi} (\textit{ne}) ‘it feels that’ (see 17 and 24 below). The nominalizers or evidential markers transparently derive from nouns, namely “inferential \textit{ruw-e} (< ‘the trace of’), reportative \textit{haw-e} (< ‘the voice of’), non-visual (= semblative) \textit{hum-i} (< ‘the sound of’), visual \textit{sir-i} (< ‘the sight of’)” \citep[470]{Bugaeva2012}. The evidential markers also appear in what seem to be \isi{tag question}s, where they are followed by \textit{somo ne (ya)}. The use of the \isi{question marker} is optional.

\ea%9
    \label{ex:ainu:9}
    \ili{Ainu} (Saru)\\
    \gll e=sinki    ruwe \textbf{somo}  ne (\textbf{{ya}})?\\
    2\textsc{sg}.S\textit{=}be.tired   \textsc{inf.ev}  \textsc{neg}  \textsc{cop}  \textsc{q}\\
    \glt ‘You’re tired, aren’t you?’ \citep[233]{Tamura2000}
    \z

In addition to the question markers mentioned above, there is a special copula \textit{an} that replaces the plain copula \textit{ne} in \isi{questions}. Special \isi{interrogative} copula forms are also known from several \ili{Mongolic} languages (\sectref{sec:5.8.2}) as well as \ili{Shuri} (\sectref{sec:5.6.2}).

\ea%10
    \label{ex:ainu:10}
    \ili{Ainu} (Saru)\\
    \gll núman \textbf{{hunna}} ek    ruwe \textbf{{an}}?\\
    yesterday  who    come.\textsc{sg}  \textsc{inf.ev}  \textsc{cop}.\textsc{q}\\
    \glt ‘Who came yesterday?’ (\citealt{Bugaeva2012}: 497; \citealt{Tamura2000}: 237)
    \z

For the \textbf{Chitose} dialect \citet[88]{Bugaeva2004} mentions the fact that the special copula is usually encountered after one of the evidential markers mentioned above, though see example (\ref{ex:ainu:17}b) for a counterexample of the Tokachi variety. The copula can also appear twice in alternative \isi{questions}.

\ea%11
    \label{ex:ainu:11}
    \ili{Ainu} (Saru)\\
    \gll ooho    pet \textbf{{an}},  ohak    pet \textbf{{an}}?\\
    be.deep  river  \textsc{cop}.\textsc{q}  be.shallow  river  \textsc{cop}.\textsc{q}\\
    \glt ‘Is it a deep or a shallow river?’ (\citealt{NINJAL2015})
    \z

According to \citet[141]{Batchelor1905}, the enclitic \textit{=he} “expresses \isi{interrogation}, and is often though by no means always, followed by the verb \textit{an} ‘to be.’”

\ea%12
    \label{ex:ainu:12}
    \ili{Ainu} (Saru)\\
    \gll tan  kur  aynu  itak    eraman  kur=\textbf{{he}} \textbf{{an}}?\\
    this  person  \textsc{pn}  language  know    person=\textsc{q}  \textsc{cop}.\textsc{q}\\
    \glt ‘Does this person understand \ili{Ainu}?’ (\citealt{NINJAL2015})
    \z

A question construction specialized for inquiring about topics is \textit{hike (mak)?} ‘how about’ \citep[237]{Tamura2000}, which appears to take a sentence-final position. As we will see in \sectref{sec:5.1.3}, \textit{mak} is actually an \isi{interrogative} meaning ‘how, why’, while \textit{hike} is a conjunction with the meaning ‘and’ \citep[497]{Bugaeva2012}.

In the \textbf{Chitose} dialect, few polar \isi{questions} are marked with rising \isi{intonation} alone. In most cases it is combined with the same final \isi{question marker} \textit{ya} as seen above.

\ea%13
    \label{ex:ainu:13}
    \ili{Ainu} (Chitose)\\
    \gll tan-to    e=nepki  humi  pirka    a \textbf{{ya}}?\\
    this-day  2\textsc{sg}.S\textit{=}work  \textsc{ev.n}  be.good  \textsc{pfv.pl}  \textsc{q}\\
    \glt ‘Did you work well today?’ \citep[85]{Bugaeva2004}
    \z

\noindent Content \isi{questions} in the Chitose dialect are also said to exhibit the marker \textit{ya} more often, as compared with the Saru dialect \citep[86]{Bugaeva2004}. There are, nevertheless, \isi{content question}s without the marker.

\ea%14
    \label{ex:ainu:14}
    \ili{Ainu} (Chitose)\\
    \gll eani \textbf{{hunna}} e=ko-ysoytak?\\
    2\textsc{sg}  who    2\textsc{sg}.S\textit{=}to.\textsc{appl}-talk\\
    \glt ‘Who are you talking to?’ \citep[86]{Bugaeva2004}
    \z

Alternative \isi{questions} have the same particle \textit{he} as seen in the Saru dialect above. But Bugaeva mentions an example of an \isi{alternative question} which in addition exhibits the \isi{question marker} \textit{ya} following each alternative. Altogether there are thus four question markers. Apparently, \textit{=he} attaches to the \isi{focus}, while \textit{ya} can be found in final position after each alternative.

\ea%15
    \label{ex:ainu:15}
    \ili{Ainu} (Chitose)\\
    \gll seta=\textbf{{he}} ne \textbf{{ya}},  kamuy=\textbf{{he}} ne \textbf{{ya}}?\\
    dog=\textsc{q}    \textsc{cop}  \textsc{q}  god=\textsc{q}    \textsc{cop}  \textsc{q}\\
    \glt ‘Is it a dog or a god?’ \citep[88]{Bugaeva2004} 
    \z

The \textbf{Shizunai} dialect also has the \isi{question marker} \textit{ya} in sentence-final position, which seems to have the same \isi{semantic scope} as seen before.

\ea%16
    \label{ex:ainu:16}
    \ili{Ainu} (Shizunai)\\
    \gll numan    ekasi \textbf{{nep}} kar \textbf{{ya}}?\\
    yestderday  old.man  what  do  \textsc{q}\\
    \glt ‘What did the old man do yesterday?’ \citep[229]{Refsing1986}
    \z

As in the Saru and Chitose dialects, there is a connection of \isi{questions} to \isi{nominalizations}, i.e. \textit{ruwe}, \textit{siri}, \textit{hawe}, and \textit{pe}. The first three correspond to the Saru forms mentioned above while the last one is similarly neutral like \textit{ruwe}. The difference between the two is the level of abstractness, \textit{pe} referring to concrete and \textit{ruwe} to abstract objects (e.g., \citealt{Refsing1986}: 229f.). The copula \textit{an} is attested as well.

A recent treatment of the \textbf{Tokachi} dialect in southeastern \isi{Hokkaid\=o} mentions several \isi{questions} that exhibit no significant difference from the other dialects already mentioned.

\newpage 
\ea%17
    \label{ex:ainu:17}
    \ili{Ainu} (Tokachi)\\
    \ea
    \gll \textbf{{nep}} tap eci-ki sir okay \textbf{{ya}}?\\
    what  \textsc{emph}  2\textsc{pl}.A-do  \textsc{ev.n}  \textsc{cop}  \textsc{q}\\
    \glt ‘What are you doing?’

    \ex
    \gll oupeka ekaci e={cakoko} a ru etap \textbf{{an}}?\\
    straight    child  2\textsc{sg}.A\textit{=}teach  \textsc{pfv.pl}  \textsc{ev.n}  \textsc{emph}  \textsc{cop.q}\\
    \glt ‘Did you teach the children correctly?’

    \ex
    \gll e={cikir-i} arka ru=\textbf{{he}}?\\
    2\textsc{sg}.S\textit{=}foot-\textsc{poss}  hurt  \textsc{ev.n=q}\\
    \glt ‘Are your feet hurting?’ (\citealt{Takahashi2013}: 131)
    \z
    \z 

Based on this \isi{similarity}, one may speculate that \isi{alternative question}s presumably display \isi{double marking} with \textit{=he} and that the \isi{question marker} \textit{ya} also marks \isi{polar question}s. Tokachi \ili{Ainu} has yet another \isi{question marker} \textit{a} not encountered thus far. In all examples given, it follows the copula \textit{an} and marks \isi{content question}s.

\ea%18
    \label{ex:ainu:18}
    \ili{Ainu} (Tokachi)\\
    \gll \textbf{{nen}} tap apusta kik hum \textbf{{an}} \textbf{{a}}?\\
    who  \textsc{emph}  door  knock  \textsc{ev.n}  \textsc{cop.q}  \textsc{q}\\
    \glt ‘Who knocked on the door?’ (\citealt{Takahashi2013}: 131)
    \z

Apparently, the marker also exists in other dialects such as Saru. The following example illustrates that it can also appear in \isi{polar question}s.

\ea%19
    \label{ex:ainu:19}
    \ili{Ainu} (Saru)\\
    \gll arki    rok \textbf{{a}}?\\
    come.\textsc{pl}  \textsc{pfv}.\textsc{pl}  \textsc{q}\\
    \glt ‘Have they come?’ \citep[79]{Shibatani1990}
    \z

Other \isi{Hokkaid\=o} dialects seem to exhibit a pattern very similar to those already observed, though there usually is only little information available. For example, the Samani dialect also has the marker \textit{ya} and the special \isi{interrogative} copula \textit{’an}, but additional information on further question markers and their \isi{semantic scope} remain obscure (T. \citealt{Sato2002}: 101, 107).

For \textbf{Sakhalin} \ili{Ainu}, the materials collected by Konada \citep{Tittel1922} contain the three \isi{question marker}s \textit{a}, \textit{ya}, and \textit{he}. We have already encountered all three markers above in several Hokkaid\=o dialects. Their \isi{semantic scope} remains unclear but may be similar to \isi{Hokkaid\=o} dialects as well.

\ea%20
    \label{ex:ainu:20}
    \ili{Ainu} (\isi{Sakhalin})\\
    \ea
    \gll pirika \textbf{{a}}?\\
    be.good  \textsc{q}\\
    \glt ‘Is it alright?’

    \ex
    \gll tam-pe    ne \textbf{{ya}}?\\
    this-thing  \textsc{cop}  \textsc{q}\\
    \glt ‘Is it this thing?’
 
    \ex
    \gll e=nu=\textbf{{he}}?\\
    2\textsc{sg.A}\textit{=}hear=\textsc{q}\\
    \glt ‘Do you hear?’ \citep[85]{Tittel1922}
    \z
    \z 

For the \textbf{Kuril} dialect of \ili{Ainu}, there is a \isi{content question} that was originally recorded by Voznesenskii. Apart from the \isi{interrogative}, no marking is present.

\ea%21
    \label{ex:ainu:21}
    \ili{Ainu} (\isi{Kuril})\\
    \gll \textbf{{nie}}{-bie-gor?}\\
    what-thing-have\\
    \glt ‘What thing is there?’ \citep[199]{Vovin1993}
    \z

\noindent No information on other question types in this dialect group seems to be available.

\tabref{tab:ainu:1} summarizes the limited information of \ili{Ainuic} \isi{question marking} that we have seen above. The semantic differences between different markers of \isi{polar question}s as well as the exact \isi{semantic scope} for most forms remains obscure for now.

\begin{table}
\caption{Tentative summary of question marking in Ainuic}
\label{tab:ainu:1}
\small
\begin{tabularx}{\textwidth}{Qlll}
\lsptoprule 
\textbf{Language} & \textbf{PQ} & \textbf{CQ} & \textbf{AQ}\\
\midrule 
Chitose \ilit{Ainu} & ya\# & ya\# & 2x =he (+ 2x ya\#)\\
\isit{Kuril} \ilit{Ainu} & ? & ?- & ?\\
\isit{Sakhalin} \ilit{Ainu} & ya\#, a\# =he\# & ? & ?\\
Saru \ilit{Ainu} & ya\#, a\# =he\#, an \textsc{cop.q}, \textsc{n\#} & ya\#, an \textsc{cop.q}, \textsc{n\#} & 2x =he, 2x ya\#, 2x\textsc{n\#}\\
Shizunai \ilit{Ainu} & ya\#, an \textsc{cop.q} & ya\#, an \textsc{cop.q} & ?\\
Tokachi \ilit{Ainu} & ?ya\#, an \textsc{cop.q} & ya\#, a\#, an \textsc{cop.q} & ?\\
\lspbottomrule
\end{tabularx}
\end{table}

As usual, most question markers remain etymologically \isi{opaque}, but \ili{Ainuic} \textit{ya} could be somehow related to \ili{Old Japanese} \textit{=ya} (\sectref{sec:5.6.2}). A problem for the comparison is, however, a different morphosyntactic behavior and \isi{semantic scope} of the \ili{Old Japanese} marker that is a mobile enclitic not found in \isi{content question}s.

\subsection{Interrogatives in Ainuic}\label{sec:5.1.3}

The sets of interrogatives in the three dialects Saru, Chitose, and Shizunai mentioned in the previous section are very similar to each other (\tabref{tab:ainu:2}). For the Tokachi dialect \citet{Takahashi2013} only mentions \textit{nen} ‘who’, \textit{nep} ‘what’, \textit{nekon} ‘how’, and \textit{onon} ‘whence’.

\begin{table}[p]
\caption{Saru (\citealt{Tamura2000}\emph{\textup{;}} \citealt{NINJAL2015}), Chitose \citep{Bugaeva2004}, and Shizunai interrogatives \citep{Refsing1986}}
\label{tab:ainu:2}

\begin{tabularx}{\textwidth}{llQl}
\lsptoprule

\textbf{Meaning} & \textbf{Saru} & \textbf{Chitose} & \textbf{Shizunai}\\
\midrule
who & hunna & hunna & hunna, nen\\
where (to) & hunak (ta), hinak (ta) & hunak & hunak, neyta\\
what & hemanta, hńta & hemanta ‘what, why’ & hemanta\\
how much/many & hempak &  & hempak\\
when & hempara & hempara & hempara\\
why & hemanta ne & nep (kusu) & nepkus (ta)\\
which & inan, (h)inaan & inaan & \\
how, why & mak, makanak & neun, makanak & nekon\\
\lspbottomrule
\end{tabularx}
\end{table}

\begin{table}[p]
\caption{Sakhalin Ainu interrogatives according to Bronisław Piłsudski (\citealt{Majewicz1998}: passim) with tentative additional analysis based on \citet{Shibatani1990} and \citet{Bugaeva2012}}
\label{tab:ainu:3}
\begin{tabularx}{.8\textwidth}{Xl}
\lsptoprule
\textbf{Meaning} & \textbf{Piłsudski}\\
\midrule
which & hemanu\\
% \tablevspace
what & hemat\\
% \tablevspace
what, why, whose & hemata {\textasciitilde} hematu\\
% \tablevspace
why (kusu ‘because’) & hemata kusu\\
% \tablevspace
wherefore (kusu ‘because’) & hemata ki-kusu\\
% \tablevspace
how many & hempak\\
% \tablevspace
when, how & hempara\\
% \tablevspace
who & inki an-ku(ru)\\
% \tablevspace
whither & nakan\\
% \tablevspace
what & nex\\
% \tablevspace
what (ta ‘\textsc{loc}, \textsc{all}’) & nex-ta\\
% \tablevspace
what, where from & nejava\\
% \tablevspace
where (ta ‘\textsc{loc}, \textsc{all}’) & nejta\\
% \tablevspace
what & nep\\
% \tablevspace
what, how & temana\\
% \tablevspace
what, how, in what way & temana-ka\\
\lspbottomrule
\end{tabularx}
\end{table}


From a \isi{synchronic} point of view, the interrogatives are mostly \isi{opaque}, but at least some forms are readily analyzable. The form \textit{nep kusu} ‘why’ from the Chitose dialect consists of \textit{nep} ‘what’ and \textit{kusu} ‘because’. The Shizunai dialect in this expression has an optional locative/allative \isi{case} marker \textit{ta}. Saru has a different formation based on the \isi{interrogative} \textit{hemanta} ‘what’ followed by the translative \citep[36]{Shibatani1990} or mutative \citep[476]{Bugaeva2012} marker \textit{ne} that derives from the copula. Most forms have a \isi{resonance} in \textit{h{\textasciitilde}}. Perhaps, \ili{Ainuic} thus not only belongs to the group of languages that have what has been called the \isi{KIN-interrogative} (e.g., Saru \textit{hunna}), but also exhibits K\textit{-}interrogatives (\sectref{sec:6.2.1}). However, the presence of the forms \textit{hunna(k)} ‘who’ and \textit{hunnak-ta} ‘where’ \citep{Batchelor1905}, the latter with locative \isi{case} marker, suggests that the form underlying both may have been a selective \isi{interrogative}. \tabref{tab:ainu:3} lists some \isi{Sakhalin} \ili{Ainu} interrogatives as recorded by Bronisław Piłsudski. For the \isi{Sakhalin} dialect, \citet[77]{Tittel1922} only mentions a handful of forms that are more or less identical with those listed in \tabref{tab:ainu:3}. These data clearly show that there are also resonances in \textit{m{\textasciitilde}} and especially \textit{n{\textasciitilde}} as well (see also \citealt{Batchelor1905}).






\citet{Vovin1993} reconstructs four \isi{interrogative} stems for \ili{Proto-Ainuic}, *\textit{gEm=}, *\textit{gu[n]na}, *\textit{in[a]=}, and *\textit{nEE=}, but the situation seems to be much more complicated than that. Altogether he assumes seven interrogatives that are based on these stems as *\textit{gEm=} is thought to be the basis for the three different interrogatives *\textit{gEm=an=ta} ‘what’, *\textit{gEm=pa=ra} ‘which’, and *\textit{gEm=pak=pE} ‘how many’ (e.g., Horobetsu \textit{hemanta}, \textit{henpara}, and \textit{henpakpe}), which is in accordance with \citegen{Cysouw2005} typology and suggests an original meaning ‘which’ or maybe ‘what’. However, there are several problems with Vovin’s reconstructions. Vovin does not comment on the \isi{morphology} he reconstructs. The use of the equal sign instead of the usual hyphen for morphemes remains unclear as well. Furthermore, it is rather questionable whether an original bilabial nasal \textit{m} should have developed into an \textit{n} followed by a bilabial plosive in all dialects but one. In fact, exactly the opposite development would be expected. Perhaps the same is true for the initial consonant *\textit{g-} that in almost all dialects mentioned has the form \textit{h-}. Similarly, except for one dialect, the alleged \isi{interrogative} *\textit{nEE=} actually always has the form \textit{ne}. The stem \textit{ne} is said to mean both ‘who’ and ‘what’, which is rare from a typological perspective, but seems possible \citep{Cysouw2005,Cysouw2007}. The \isi{interrogative} *\textit{in[a]=} appears to be mistaken, as there may have been an original initial consonant, e.g. Saru \textit{(h)inaan} ‘which’. It may also be noted that \citegen{Vovin1993} list of cognates is not exhaustive. There is an older but more complete description of interrogatives by \cite[64f.]{Asai1974} that is given in \tabref{tab:ainu:4}. \figref{fig:ainu:1} indicates the geographical distribution of the personal interrogatives.

\begin{figure}[t]
\includegraphics[width=.45\textheight]{figures/fig_ainu_1.jpg}
\caption{Distribution of forms meaning ‘who’ after \cite[64f.]{Asai1974}}
\label{fig:ainu:1}
\end{figure}


\begin{table}
\caption{Distribution of forms among dialects after \cite[64f.]{Asai1974}; 1 = Yakumo, 2 = Oshamambe, 3 = Horobetsu, 4 = Piratori, 5 = Nukibetsu, 6 = Niikappu, 7 = Samani, 8 = Obihiro, 9 = Kushiro, 10 =Bihoro, 11 = Asahikawa, 12 = Nayoro, 13 = Sôya, 14 = Ochiho, 15 = Tarantomari, 16 = Maoka, 17 = Shiraura, 18 = Raichishika, 19 = Nairo, 20 = Kuril, 21 = Chitose}
\label{tab:ainu:4}

\begin{tabularx}{\textwidth}{XXl}
\lsptoprule

\textbf{Meaning} & \textbf{Form} & \textbf{Dialects}\\
\midrule
who & nen & 1-3, 7-13\\
& hunna & 4-6, 21\\
& hunat & 20\\
& naat & 14-19\\
what & nep & 1-3, 7-14\\
& hemanta & 3, 4, 6, 21\\
& hemata & 14-19\\
& hinta & 5\\
& nejanokonejanpi & 20\\
how & nekona & 1-3, 9, 10\\
& nekon & 7-13\\
& mak & 4-6\\
& manak, makanak & 21\\
& temana & 14-19\\
& uiman & 20\\
when & henpara & 1-6, 11, 12, 14-16, 18, 21\\
& henparakanne & 17, 19\\
& henpaki & 20\\
& nenpara & 7-10, 13\\
where & nejta & 1-3, 7-13\\
& nahta & 14-19\\
& hunakta & 4, 6, 21\\
& hunakun & 5, 21\\
& huija & 20\\
\lspbottomrule
\end{tabularx}
\end{table}
\newpage 