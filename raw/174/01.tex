\chapter{Introduction}

In recent years the study of \isi{linguistic diversity} took center stage in \isi{linguistic typology} (e.g., \citealt{EvansLevinson2009}). \citet[10]{Nettle1999} usefully differentiated between three types of \isi{linguistic diversity} that he called \textit{\isi{language diversity}} (the number of languages), \textit{\isi{phylogenetic diversity}} (the number of language families), and \textit{\isi{structural diversity}} (grammatical differences among languages). This study is concerned with all three kinds of diversity, but places an emphasis on the last. In this it follows \citet[2]{Nichols1992}, who postulated that “the main object of description here is not principles constraining possible human languages but principles governing the distribution of structural features among the world’s languages.” Different from a classical and purely \isi{synchronic} typological study based on a well-balanced global sample of languages, this study openly seeks the areal and \isi{genetic bias} and investigates the distribution of linguistic and especially of \isi{structural diversity} in \isi{Northeast Asia} (\isi{NEA}). Because “typological distributions are historically grown” \citep[239]{Bickel2007}, this study emphasizes the internal development in individual language families as well as their mutual relations.

\begin{quote}
The ultimate goal is to understand “\textbf{what’s where why?}”, and this makes it clear that the major contributions that typology offers are not confined to \isi{Cognitive Science} as narrowly understood. The goals of 21st century typology are embedded in a much broader \textbf{anthropological} \textbf{perspective}: to help understand how the variants of one key social institution are distributed in the world, and what general principles and what incidental events are the historical causes for these distributions. (\citealt[248]{Bickel2007}, my boldface)
\end{quote}

\noindent \citet{Bickel2015} today calls this approach \textit{distributional typology}. \citet{Nichols1992}, based on an analogy with biology, employed the term \textit{population typology} instead. \cite[1456]{Dahl2001} prefers yet another name, \textit{\isi{areal typology}}, defined as “the study of patterns in the areal distribution of typologically relevant features of languages” that “is both descriptive and explanatory” and “has both a \isi{synchronic} and a \isi{diachronic} side.” What these approaches have in common is not only their focus on the distribution of diversity, but also the desire to explain its emergence.

The holistic approach taken in this study can be tentatively characterized as an \textit{ecological typology} that is committed to an ecologically plausible understanding of language and human beings \citep[186]{Hölzl2015d}. However, in linguistics \textit{ecology} can be understood in a variety of different ways. So-called \textit{\isi{ecolinguistics}}, for instance, according to one view “is the study of the impact of language on the life-sustaining relationships among humans, other organisms and the physical environment” and “is normatively orientated towards preserving relationships which sustain life.” (\citealt[105]{AlexanderStibbe2014}) In another sense, the ecological aspect instead refers to the maintenance of languages and ensuing preservation of \isi{linguistic diversity} (e.g., \citealt{Mühlhäusler1992}). The approach followed here is less value-driven (\citealt[173f.]{Hölzl2015d}); it concentrates instead on the description and explanation of \isi{linguistic diversity}. While it shares this focus with the other approaches mentioned above, it emphasizes the importance of \isi{ecology} for an adequate understanding of language. The fundamental unit of description is the \textit{\isi{organism-environment system}}, or OES for short (e.g., \citealt{Turvey2009}; \citealt{Welsch2012}). According to \cite[329]{Järvilehto1998}, the theory of the OES maintains “that in any functional sense organism and environment are inseparable and form only one unitary system. The organism cannot exist without the environment and the environment has descriptive properties only if it is connected to the organism.” This theory has a relatively long history, which is concisely summarized in \citet{Järvilehto2009}. For example, \cite[233]{Sumner1922} employed the term \textit{organism-environment} \textit{complex} instead, but similarly claimed that “the organism and the environment interpenetrate one another through and through.” However, \citet{Järvilehto2009} did not mention a very similar concept called the \textit{life space} advocated by \citet[12]{Lewin1936}: “Every scientific psychology must take into account whole situations, \textit{i.e.,} the state of both person and environment.” Language, it will be argued, is an integral component of the human OES. Language is not restricted to the organism (e.g., the brain), but equally has an existence as a self-constructed niche (\citealt{Odling-SmeeLaland2009}; \citealt{Sinha2013}), i.e. a modification of the environment by an organism such as the web of a spider or the dam of a beaver \citep[5]{Odling-Smeeetal2013}.

\begin{quote}
Niche construction refers to the modification of both biotic and abiotic components in environments via trophic interactions and the informed (i.e., based on genetic or acquired information) physical “work” of organisms. It includes the metabolic, physiological, and behavioral activities of organisms, as well as their choices.
\end{quote}

\noindent Human \isi{niche construction} encompasses a multitude of different examples, ranging from the use of tents such as the \ili{Evenki} \textit{d’u} (similar to a tipi), over the domestication of \isi{reindeer}, the construction of railroads, or deforestation, to human-induced \isi{climate} change. In fact, given the extraordinary impact of humans on the environment, the term \textit{\isi{Anthropocene}} has been suggested as the contemporary geological epoch (e.g., \citealt{Rosol2017} and references therein). The hypothesis that language is an integral component of the \isi{organism-environment system} has important consequences for the understanding of \isi{linguistic diversity}. Of course, \isi{linguistic diversity} is neither scattered at random, nor is it without limits. Rather, there must be a \textit{reason} for the distribution of \isi{linguistic diversity} we find today (\citealt{Bickel2014}; \citealt[904f.]{Bickel2015}). However, a distinction between synchrony and diachrony is insufficient as a proper explanation. One of the most promising approaches to the \textit{natural causes of language} has recently been put forward by \citep[13ff.]{Enfield2014}, who distinguishes between a total of six \textit{causal frames} in which \textit{linguistic} processes occur.

\begin{quote}
Each of the six frames – \isi{microgenetic}, \isi{ontogenetic}, \isi{phylogenetic}, \isi{enchronic}, \isi{diachronic}, \isi{synchronic} – is distinct from the others in terms of the kinds of causality it implies, and thus in its relevance to what we are asking about language and its relation to culture and other aspects of human diversity. One way to think about these distinct frames is that they are different sources of evidence for explaining the things that we want to understand. \citep[13]{Enfield2014}
\end{quote}

\noindent These causal frames are related to, but not quite identical with, different \isi{time scales}, ranging from milliseconds to millions of years (\tabref{tab:1}). There is a certain amount of mutual interdependence and influence between these frames, each of which combines properties of both organism and environment to different degrees. Niche construction, for example, may exist at several time scales and can “accumulate over time” (\citealt[18]{Odling-Smeeetal2013}).

\begin{table}
\caption{Examples of {causal frames} loosely based on \cite[13–17]{Enfield2014} with a focus on language}
\label{tab:1}

\begin{tabularx}{\textwidth}{llQ}
\lsptoprule
\textbf{Frames} & \textbf{Timescales} & \textbf{Examples}\\
\midrule
\isi{phylogenetic} & ky--my & biological \isi{evolution}, \isi{climate} change, language \isi{evolution}\\
\isi{diachronic} & y--ky & \isi{language change}, language families, \isi{conventionalization}\\
\isi{ontogenetic} & m--y & individual biography, language \isi{acquisition}, \isi{entrenchment}\\
\isi{enchronic} & s--m & \isi{turn-taking}, conversation, \isi{question-response sequences}\\
\isi{microgenetic} & ms--s & physiological processes, \isi{action}, \isi{perception}, \isi{conception}\\
\isi{synchronic} & -- & language systems, knowledge of a given language\\
\lspbottomrule
\end{tabularx}
\end{table}

All of these frames are crucial to an explanation of \isi{linguistic diversity}, although a focus will be on some of them. Originally, \isi{linguistic typology}  was mostly concerned with the \textit{\isi{synchronic}} dimension, which is a necessary abstraction to consider individual languages as fixed entities that can be described and compared. The \textit{diachronic} frame primarily concerns \isi{language change} over a period of years or thousands of years. This study in particular investigates what will be called the \textit{grammar of questions} (GQ), i.e. those aspects of any given language that are specialized for asking \isi{questions} or regularly combine with these.\footnote{Cable’s dissertation has the title \textit{The grammar of Q} \citep{Cable2007}. However, the term itself has not been clearly defined and is grounded in generative grammar.} The ability to ask \isi{questions} as well as the existence of specialized constructions for asking \isi{questions} seem to be \isi{universal}. Questions, of course, are part of \isi{question-response sequences}, which are located in the \textit{\isi{enchronic}} frame that refers to social \isi{interaction}. Most theoretical discussions of \isi{questions}, from a \isi{speech act} perspective for example, concentrate on this frame (e.g., \citealt{Levinson2012b}). Exceptions include psychological studies (e.g., \citealt{Loewenstein1994}) or the so-called \textit{\isi{cognitive typology}} approach by \citet{Schulze2007}, which also include the \isi{microgenetic} frame. As opposed to the social dimension of the \isi{enchronic} frame, the \textit{microgenetic} perspective concentrates on the cognitive and physiological processes that take place within the \isi{organism-environment system}. The emergence of the \isi{grammar of questions} over \textit{\isi{phylogenetic}} (human and linguistic \isi{evolution}) and \textit{\isi{ontogenetic}} time-spans (individual development, especially of children), as described by \citet{Tomasello2008}, will not play an important role in this study.

Apart from the causal frames, it is important to add different \textit{loci of causes}, which can be described metaphorically as different types of \isi{ecology} that language is embedded in. A recent classification proposed by \citet[7]{SteffensenFill2014} distinguishes between four different ecologies:

\begin{quote}
\textbf{(1)} Language exists in a \textbf{\isi{symbolic ecology}}: this approach investigates the co-exis\-tence of languages or ‘symbol systems’ within a given area. \textbf{(2)} Language exists in a \textbf{\isi{natural ecology}}: this approach investigates how language relates to the biological and ecosystemic surroundings (topography, \isi{climate}, fauna, flora, etc.). \textbf{(3)} Language exists in a \textbf{\isi{sociocultural ecology}}: this approach investigates how language relates to the social and cultural forces that shape the conditions of speakers and speech communities. \textbf{(4)} Language exists in a \textbf{\isi{cognitive ecology}}: this approach investigates how language is enabled by the dynamics between biological organisms and their environment, focusing on those cognitive capacities that give rise to organisms’ flexible, adaptive behaviour. (my enumeration and boldface)
\end{quote}

\noindent Of course, a focus on language as such is only an abstraction and the above distinction merely highlights several important perspectives (\citealt{SteffensenFill2014}: 7). Each of the four different ecologies influences all three kinds of \isi{linguistic diversity}, i.e. language, \isi{phylogenetic}, and \isi{structural diversity}.

In many cases the exact influence of the four ecologies is only beginning to be understood (e.g., \citealt{DeBusser2015}), which is why only a handful of examples connected with the \isi{grammar of questions} can be given here. \textit{Symbolic} \textit{ecology} refers to the aspect of \isi{language contact} that has a central position in \isi{areal linguistics}. It encompasses phenomena such as the \isi{borrowing} of linguistic items, the creolization of languages, or \isi{language shift}. For example, many languages of \isi{China} that share a common \ili{Chinese} ad- or superstrate have borrowed the \isi{question marker} \textit{ba} \zh{吧} (see below and \sectref{sec:5.9.2.1}). \textit{Natural ecology}, too, is an aspect that should not be underestimated (e.g., \citealt{AxelsenManrubia2014}). After all, the distribution of languages even today is determined to a large degree by natural and constructed \textit{\isi{affordances}}---roughly possibilities of \isi{action} (\citealt{Lewin1936}; \citealt{Gibson1979})---of our environment such as those of rivers, mountains, roads, bridges, or borders. Climate clearly also influences all three types of \isi{linguistic diversity} (e.g., \citealt{Everett2015,Everett2016}). For example, languages that mark \isi{polar question}s with \isi{intonation} exclusively and do not have additional \isi{question marking} strategies---similar to the total number of languages---strangely cluster around the \isi{tropics} \citep{Dryer2013m}. In \isi{Northeast Asia} there are almost no such languages. The \textit{sociocultural ecology} plays an important role in \isi{language spread} as well, but also influences the relative prestige and importance of languages. This has a direct influence on \isi{language shift} and the direction of \isi{borrowing} of linguistic items in \isi{language contact} situations. As shown by \citet{Trudgill2011} the \textit{\isi{social ecology}} can have a strong influence on the \isi{complexity} of a given language, including aspects of the \isi{grammar of questions}, such as the \isi{interrogative} system (see \sectref{sec:6.3}). Furthermore, the culture and way of life of a speech community may have an impact on the structure of languages. \citet[388]{CysouwComrie2013} argued, for instance, that the languages of hunter-gatherers might have preferences for certain linguistic features such as “relatively many cases of initial interrogatives”, although this could not be confirmed for \isi{NEA}, which contains few real \isi{hunter-gatherer} groups and few languages with sentence-initial interrogatives. The last point mentioned, the \textit{cognitive ecology}, especially from a \isi{microgenetic} perspective, is an important factor in the structural properties the \isi{grammar of questions} tends to have cross-linguistically. For example, there is a recurrent structural pattern among many different languages in which a \isi{content question} is immediately followed by a polar, focus, or \isi{alternative question} (e.g., \textit{What are you doing, are you crazy?}), which can be explained by aspects of the human conceptual system (see \sectref{sec:4.4}, \sectref{sec:6.3}).

In principle, all four perspectives are crucial for a complete investigation of language as well as the \isi{grammar of questions}. Nevertheless, within this study the focus will lie on the aspect of \textit{language contact} (symbolic \isi{ecology}). Furthermore, a word of caution is in order. While most scholars would probably agree that there may be fundamental differences among individual symbolic, natural, and sociocultural ecologies, there is often a tacit assumption of the uniformity of human \isi{cognition} throughout the world. This is what \citet[397]{Levinson2012a} has rightfully called “the original sin of the cognitive sciences---the denial of variation and diversity in human \isi{cognition}.” In fact, \citet[61]{HenrichHeineNorenzayan2010} have quite convincingly shown that many previous investigations in cognitive science or psychology were strongly biased due to problematic samples of participants that do not accurately represent human diversity. This presents us with a severe problem. For instance, \isi{questions}, it might be argued, can be seen as a way to verbally resolve \isi{curiosity}. Problematically, publications on \isi{curiosity} such as \citet[453]{Reio2011} usually share this tacit assumption of universality:

\begin{quote}
Curiosity is the desire for new information and sensory experience that motivates \isi{exploratory behavior}. External stimuli with novel, complex, uncertain, or conflicting properties (i.e., \isi{collative} stimuli) create internal states of arousal that motivate exploratory behaviors to reduce the state of arousal.
\end{quote}

\noindent Curiously, there are surprisingly few scientific investigations of \isi{curiosity}. That is why this study necessarily follows this theory, which is basically a summary of \cite{Berlyne1954,Berlyne1960,Berlyne1978}. But it should be borne in mind that there are personal differences of \isi{curiosity} in both quantity and quality (e.g., \citealt{vonStummHellChamorroPremuzic2011}).

The bulk of this study is a bottom-up comparison of the grammars of questions in different languages and a tentative explanation of their similarities and differences in terms of some of the \isi{causal frames} and ecologies sketched out above. As further explained in Chapter 4, the typology of \isi{questions} proposed in this study will mostly concentrate on \isi{question marking} and interrogatives (see also \citealt{Huang1999}). This is a major difference from previous approaches that are usually based on a distinction between different question types, such as polar and \isi{content question}s. These two domains---\isi{question marking} and interrogatives---behave quite differently, for instance as regards the symbolic \isi{ecology} and \isi{diachronic} time scale. Interrogatives are known to be generally very conservative (e.g., \citealt{Diessel2003}). In many instances, an \isi{interrogative} can even remain stable for thousands of years. For example, \ili{English} \textit{where} can be directly traced back over a time span of several thousand years to \ili{Proto-Indo-European} *\textit{k\textsuperscript{w}ór} with the same meaning (\citealt[419f.]{MalloryAdams2006}). \ili{Proto-Indo-European} was probably spoken about 6500 years before present (\citealt{AnthonyRinge2015}), which means that the \isi{interrogative} is \textit{at least} of this age. \citet[649]{Diessel2003} thus correctly concludes that interrogatives (and \isi{demonstratives}) “are generally so old that their roots are not etymologically analyzable”. Theoretically, similar interrogatives can thus be employed to detect previously unknown old genetic connections between languages. In \isi{NEA} there are a few possible examples of this sort. The most striking is a personal \isi{interrogative} ‘who’ that has an uncanny \isi{similarity} in several families, even if one goes back to the respective \isi{proto-languages} (e.g., \ili{Proto-Mongolic} *\textit{ken}, \ili{Proto-Turkic} *\textit{kim {\textasciitilde} *käm}, \ili{Proto-Yukaghiric} *\textit{kin} etc.). This will be called the \textit{\isi{KIN-interrogative}} in this study (see \sectref{sec:6.2.1}). Furthermore, many languages in \isi{NEA} have what will be called \textit{\isi{K-interrogatives}}, that is, they have several interrogatives that share a so-called \textit{\isi{resonance}} (a \isi{submorpheme}, see \citealt{BickelNichols2007}: 209; \citealt{Mackenzie2009}: 1141) that has the form of a velar or uvular plosive or fricative (e.g., \ili{Nanai} \textbf{\textit{x}}\textit{aɪ} ‘what’, \textbf{\textit{x}}\textit{ado} ‘how many’, \textbf{\textit{x}}\textit{ooni} ‘how’). Given its fuzzy boundary and only partly analyzable character, a \isi{resonance} will be indicated with a tilde (e.g., \ili{Nanai} \textit{x{\textasciitilde}}) in order to keep it apart from fully analyzable morpheme boundaries written with a hyphen (e.g., \ili{Nanai} \textit{xaɪ-wa} ‘what-\textsc{acc}’). This is similar to well-known submorphemes such as \ili{English} \textit{gl{\textasciitilde}}, found in \textit{gleam}, \textit{glimmer}, \textit{glisten}, or \textit{glow}. Despite the fact that the initial consonant cluster is not clearly analyzable, the individual instances nevertheless have a vague \isi{similarity} in meaning. A \isi{resonance} usually, but not necessarily, indicates a common origin of different interrogatives within one language. It may be noted, however, that KIN- and K-interrogatives are, first and foremost, typological labels and do not necessarily indicate a common origin of different languages as was assumed by \cite[217–224]{Greenberg2000}. They are intended to be analogous to the well-known \isi{m-T-pronouns} found throughout Eurasia, such as in \ili{English} \textit{me} and \textit{thee} or \ili{Nanai} \textit{mi} ‘I’ and \textit{si} ‘you (\textsc{sg})’ (see \citealt{NicholsPeterson2013}). Interrogatives are rarely borrowed, and when they are, this usually indicates an extreme \isi{contact} situation or perhaps widespread \isi{bilingualism}. Take Mednyj \ili{Aleut}, for instance, which may be considered a truly mixed language. It exhibits interrogatives both of \ili{Aleut} (e.g., \textit{kiin} ‘who’) and of \ili{Russian} (e.g., \textit{kuda} ‘where’) origin (see \sectref{sec:5.4.3}). \cite[65f.]{Bickerton2016} and \citet{MuyskenSmith1990} argue that creole and pidgin languages may have a preponderance of synchronically analyzable interrogatives such as \ili{English} \textit{at what time}. Because most languages contain at least some instances of analyzable interrogatives, it will be argued that, in order to identify such instances, the whole \textit{\isi{interrogative} system} needs to be investigated (\citealt{MuyskenSmith1990}). In most cases of analyzable interrogatives in \isi{NEA} the actual \isi{interrogative} takes first position (e.g., \ili{Manchu} \textit{ai-ba-} ‘what-place-’). Generalizing on \citegen{Bickerton2016} and \citegen{MuyskenSmith1990} assumption, the emergence of several analyzable interrogatives can be said to be an instance of \textit{simplification} in the sense of a “regularization of irregularities”, an “increase in morphological \isi{transparency}” \citep[62]{Trudgill2011}, and a \isi{reduction} in the number of actual interrogatives. This is most likely due to a specific type of strong \isi{language contact} such as massive non-native language \isi{acquisition} (e.g., \citealt{McWhorter2007}). In sum, interrogatives may thus indicate different kinds of strong \isi{language contact} (mixing, \isi{simplification}) and perhaps very distant genetic relationships. The overall \isi{similarity} of \isi{interrogative systems} among related languages can also function as a rough proxy for their time of divergence.

Question marking behaves very differently from interrogatives. Of course, \isi{question marking} may remain stable over long time spans in some cases, but generally is much less stable and more flexible than the \isi{interrogative} system and is extremely sensitive to \isi{language contact}. In \isi{NEA} alone there are dozens of examples of borrowed question markers. One prominent example is the \ili{Chinese} marker \textit{ba} \zh{吧} that marks \isi{polar question}s with an additional moment of supposition (‘isn’t it the case that’). The marker has been borrowed by many languages spoken in \isi{China} today from diverse language families and in many different regions. Even structural \isi{question marking} such as verb-first \isi{word order} as found in \ili{Germanic} languages has been adopted by some \ili{Uralic} languages, for example \citep{Miestamo2011}. Question marking thus has the potential to indicate \isi{language contact}, and this it does quite independently of the intensity of the \isi{contact}. Even relatively light \isi{contact} may lead to the adoption of a \isi{question marker} from other languages. However, \isi{question marking} cannot suggest distant language families. Without doubt, this difference between the two domains—\isi{question marking} and interrogatives—is an example of the more general principle “that basic structural features tend to be stable, whereas pragmatically sensitive features such as politeness phenomena and evidentials tend to be unstable.” \citep[3]{Trudgill2011} But interrogatives and \isi{question marking} certainly represent the extreme ends of what may be conceptualized as a \isi{continuum}. More or less, they are in complementary distribution when it comes to genetic inheritance and different types of areal contacts. However, the type of \isi{question marking} (e.g., initial \isi{question marker}) appears to be more stable than the actual form of the \isi{question marker}. For instance, many \ili{Tungusic} languages have a \isi{tendency} for sentence-final \isi{polar question} markers despite the fact that they are etymologically unrelated and attested many thousand kilometers apart, e.g. \ili{Sibe} \textit{=na}\# at the Chinese Kazakh border or \ili{Even} \textit{=Ku}\# in northeastern \isi{Siberia}. The type of \isi{question marking} thus seems to take a position between the two extremes. Therefore, the \isi{grammar of questions} represents an ideal tool for the identification of linguistic convergence, possible middle- or long-range relationships, and instances of unusually extreme \isi{language contact}. Linguistic diversity, just like archaeological records or the \isi{human genome}, can thus function as a powerful source for the investigation of human \isi{prehistory} over time spans of hundreds and thousands of years (e.g., \citealt{Nichols1992}; \citealt{HeggartyRenfrew2014a}). In this study \isi{Northeast Asia} functions as a testing ground for this tentative methodology (see \sectref{sec:6.3}).

\newpage 
\isi{Northeast Asia} (\isi{NEA}) here is first and foremost defined geographically as the region north of the \isi{Yellow River} and east of the \isi{Yenisei} (\figref{fig:1:1}). A \isi{natural boundary} is formed in the north by the \isi{Arctic Ocean} and in the east by the Pacific. In the northeast, the \isi{Bering Strait} separates \isi{NEA} from \isi{Alaska}. \isi{NEA} includes all islands along the \isi{Pacific Rim} up to the Aleutian chain that are all located north of \isi{Taiwan}, but excludes \isi{Taiwan} itself, which has stronger ties with \isi{Southeast Asia}. The islands in the \isi{Arctic Ocean} are largely uninhabited, which renders them irrelevant for the purposes of this study. The \isi{Altai}, the \isi{Kunlun}, the \isi{Pamir}, the \isi{Karakorum}, the \isi{Tianshan}, the \isi{Qinling}, and the \isi{Tibetan Plateau} will be taken as natural boundaries to the west, southwest, and south.

\begin{figure}
\includegraphics[width=\textwidth]{figures/fig_key_1.jpg}
\caption{Some natural boundaries of Northeast Asia;
adapted from 
\url{https://en.wikipedia.org/wiki/Geography_of_Asia}, adapted from \url{http://visibleearth.nasa.gov/view_rec.php?vev1id=11656} (Accessed 2016-04-10.)}
\label{fig:1:1}
\end{figure}

Thus defined, \isi{NEA} is a vast area that covers all of \isi{Japan}, \isi{Mongolia}, and the two Koreas as well as all of the \isi{Far Eastern Federal district}, most of the \isi{Siberian Federal district} of \isi{Russia}, and northern \isi{China}, including \isi{Manchuria}, Inner \isi{Mongolia}, \isi{Xinjiang}, parts of the adjacent provinces, and certain parts of Tibet (\isi{Amdo}).

Unfortunately, \isi{Asia} is a clear concept only until one tries to define it properly. It combines cultures and languages as diverse as Israel and the Asiatic Eskimos, it is located on several distinct tectonic plates, the largest of which includes \isi{Europe} but not India, and there is no meaningful boundary of any sort that would clearly differentiate between \isi{Asia} and \isi{Europe}. Thus, in the end one is left with the two possibilities that \citet{Sinor1990} was struggling with when trying to define the cultural area of \textit{Inner Asia}. He was well aware that the term \textit{Inner Eurasia} would have been more adequate, but today the term \textit{Asia} is simply too strongly conventionalized and entrenched. This book similarly makes use of the term \textit{Northeast Asia}, even though \textit{\isi{Northeast Eurasia}} might have been the better choice. Nevertheless, this makes it compatible with previous approaches with the same name and research on neighboring areas such as \textit{Southeast Asia} (SEA).

Apart from Northern \isi{China}, \isi{Korea}, and \isi{Japan}, \isi{NEA} is extremely sparsely settled. Even Northeast \isi{China} (\isi{Manchuria}) and northern \isi{Japan} (\isi{Hokkaid\=o}) have only been settled in larger numbers within the last 150 years or so (e.g., \citealt{Janhunen1996}). In contrast with the \isi{Western Siberian Lowland} and the adjacent regions of European \isi{Russia} and Eastern \isi{Europe}, most of \isi{NEA} may be said to be generally very mountainous or at least to be located at higher altitudes. \isi{NEA} has important bodies of water, including lakes such as Lake \isi{Baikal}, which defines something like the center of \isi{NEA}, and several large rivers that play an important role for the dispersal of languages. In \isi{Russia} these are, beginning from the west, the \isi{Yenisei}, the \isi{Lena}, the \isi{Indigirka}, and the \isi{Kolyma}, all of which flow into the \isi{Arctic Ocean}. Further south, the \isi{Amur} forms the border between \isi{Russia} and \isi{China} before it bends towards the northeast and flows into the \isi{Sea of Okhotsk}. In \isi{China}, the \isi{Liao} flows into the \isi{Gulf of Bohai} from the north and the \isi{Yellow River} from the west. There are several smaller rivers such as the \isi{Yalu}, which forms the border between \isi{North Korea} and \isi{China}, or the Anadyr in \isi{Chukotka}. For the most part, \isi{NEA} is characterized by a continental \isi{climate} with cold and often dry winters but warm or hot and more humid summers. However, there are considerable regional differences ranging from a \isi{tundra} \isi{climate} in the northern parts of \isi{Russia}, to a very humid subtropical \isi{climate} in the south of \isi{Japan}, to a desert \isi{climate} in northwestern \isi{China} as well as parts of \isi{Mongolia}. The northern parts of \isi{NEA} are mostly covered by \isi{Taiga} and, further north, by \isi{tundra}. As one moves south, the \isi{Taiga} changes into mixed forests that give way to the \isi{steppes} in Inner and Outer \isi{Mongolia}, the Manchurian and North \ili{Chinese} Plain, the \isi{Ordos Plateau}, as well as the deserts \isi{Gobi} and \isi{Taklamakan} (e.g., \citealt{Taaffe1990}; \citealt{Janhunen1996}; \citealt{NarangoaCribbs2014}).

Parts of \isi{NEA} have been home to \textit{\isi{Homo erectus}}, \isi{Neanderthals}, \isi{Denisovans}, and possibly to other human (sub)species, the classification of which is still disputed. Despite the possibility that both \isi{Neanderthals} and Denisovans may have had a language comparable to languages today (e.g., \citealt{DediuLevinson2013}) and the fact that both interbred with modern humans (\citealt{SankararamanPattersonReich2016}; \citealt{Reich2018} and references therein), there is no direct evidence for the languages these extinct groups may have spoken. For this reason, only the language of \isi{anatomically modern humans} (AMH) can be investigated here. AMH reached \isi{NEA} and even the northernmost parts of it at least 45 kya (\citealt{Pitulko2016}, see also \citealt{Lbova2014}). However, the earliest records of any language in \isi{NEA} are from \ili{Old Chinese} and are only about 3250 years old and thus much younger than \ili{Sumerian} (about 5000 years old) or \ili{Ancient Egyptian} (about 4700 years old). If history is defined as that period when written language was present, in large parts of \isi{NEA} it only started several centuries ago \citep{Bellwood2013}. Linguistic reconstructions of some of the oldest \isi{proto-languages} located in or close to \isi{NEA}, such as of \ili{Austronesian}, \ili{Trans-Himalayan} (\ili{Sino-Tibetan}), \ili{Uralic}, and maybe \ili{Dene-Yeniseian}, must be several thousand years older than \ili{Old Chinese} records, but nothing comparable to the time of the first peopling of the area.

The earliest accounts of \isi{Northeast Asia} such as Nicolaas \citegen{Witsen1705} \textit{Noord en Oost Tartarye} employed the term \textit{\isi{Tatary}} (or \textit{Tartary}), but were quite inconsistent in their use of it. This name has dropped out of use today and in \ili{English} there is at present no common designation for what has been defined as \isi{NEA} above. Only in recent years has there been an increase in the West of publications bearing the name \textit{Northeast Asia} in the title. Interestingly, this is a much more common concept in \isi{Japan} (\textit{hokut\=o} \textit{ajia} \jp{北東アジア}), \isi{Korea} (\textit{dongbuk asia}), \isi{Mongolia} (\textit{züün xojd azi}), and \isi{China} (\textit{d\=o}\textit{ngběi yàzh\=o}\textit{u} \zh{东北亚洲}), but apparently less so in \isi{Russia} (\textit{severo-vostochnaja azija}). The origin of the term has recently been concisely summarized by \citet[2]{NarangoaCribbs2014}:

\begin{quote}
The term “\isi{Northeast Asia}” is relatively new. It was introduced into academic discourse in the 1930s by the American historian and political scientist Robert Kerner, who taught at the University of California. Kerner’s “\isi{Northeast Asia}” comprised the \isi{Korean Peninsula}, the \isi{Manchurian Plain}, the \isi{Mongolian Plateau}, and the mountainous regions of Eastern \isi{Siberia}, stretching from Lake \isi{Baikal} to the \isi{Pacific Ocean}.
\end{quote}

\noindent In her recent book \textit{Early modern \isi{China} and \isi{Northeast Asia}}, \citet{Rawski2015} included more or less the same region. My account adds substantial areas to this definition, especially in the north and the west. Nevertheless, my approach is similar to \citegen[2]{NarangoaCribbs2014} and \citegen{Rawski2015} in trying to break down traditional conceptions of \isi{East Asia} and a \isi{Sinocentric view}. Interestingly, an older definition by \citet[xv]{Chard1974}, which only came to my attention after the bulk of this study was already written, roughly coincides with my definition above:

\begin{quote}
The area covered comprises \isi{Siberia} from the \isi{Altai} Mountains and \isi{Yenisei} River valleys eastwards, \isi{Mongolia}, \isi{Manchuria}, \isi{Korea}, and \isi{Japan}. This area has a certain coherence. Geographically, if we except western \isi{Siberia} with its close affinities to European \isi{Russia}, it represents the steppe, forest, and \isi{tundra} zones of northern \isi{Asia}, lying beyond the loess farmland of traditional \isi{China}.
\end{quote}

\noindent The only difference concerns the exclusion of \isi{Xinjiang} and other parts of northern \isi{China}. \isi{Xinjiang} happens to be included in \isi{NEA} in this study because of its relatively old ties to central \isi{China} due to Chinese expansions and trade along the Silk Roads, the presence of a great many northwestern \ili{Mandarin} speakers today, and some linguistic connections to \isi{Amdo} and \isi{Mongolia}. \isi{Xinjiang} is also included in \citegen[25f.]{Nichols1992} concept of \textit{Northern Asia}, which coincides with my definition, except that it includes those areas between the \isi{Yenisei} and the \isi{Ural} Mountains. In his recent book \textit{The peoples of \isi{Northeast Asia} through time}, \citet[21ff.]{Zgusta2015} is not very clear about his definition of \isi{Northeast Asia}, but he puts an \isi{emphasis} on what he calls \textit{Pacific Northeast Asia}, which only includes northern \isi{Japan}, \isi{Sakhalin}, eastern \isi{Manchuria}, \isi{Kamchatka}, and \isi{Chukotka}. Here this quite useful term will be adopted to additionally include all of \isi{Japan}, \isi{Korea}, and the area around the \isi{Gulf of Bohai}, i.e. all of insular and peninsular \isi{NEA} adjacent to the Pacific.

The brief review above is not exhaustive but sufficiently illustrates a wide variety of overlapping designations and definitions of \isi{NEA}. One of the few authors who draw a more differentiated picture is \citet[284]{Janhunen2010}:

\begin{quote}
In the \textbf{widest} sense, \isi{Northeast Asia} as a geographical and ethnohistorical region can be defined as the entire northeastern part of the Eurasian continent, delimited by the \isi{Yenisei} in the west and the \isi{Yellow River} in the south. In the northeast, the region extends, in principle, to the \isi{Bering Strait}. In a somewhat \textbf{narrower} framework, \isi{Northeast Asia} may be defined as comprising the territory between the \isi{Amur} and \isi{Yellow River} basins, including the Korean Peninsula and the Japanese Islands in the Pacific coastal zone, but excluding the northeasternmost limits of what is today the Russian Far East. (my boldface)
\end{quote}

\noindent This broad definition has clearly been influenced by Chard’s point of view \citep[7]{Janhunen1996}. The narrow definition, on the other hand, is more or less identical with the perspective taken by \citet{NarangoaCribbs2014} or \citet{Rawski2015} seen above and may be more appropriately termed \textit{\isi{Greater Manchuria}} instead of \isi{Northeast Asia} \citep[6]{Janhunen1996}. Needless to say, this study is based on a wide definition of \isi{NEA}.

The addition of the part \textit{and beyond} to the title of this book has two meanings. First, some languages such as the \ili{Turkic} languages \ili{Chuvash} and \ili{Turkish} that are located outside of, but have ties to, or in these cases even originate in, \isi{NEA}, will be included as well. This problem of establishing a meaningful western boundary of, in their terms, \textit{northern East Asia} has also been observed by \citet[873]{HeggartyRenfrew2014b}:

\begin{quote}
\ili{Turkish} serves also to stress just how far the typological unity of this language area stretches beyond any geographical definition of \textit{East} \isi{Asia}. For in linguistic terms – whether in family affiliations, typology or \isi{prehistory} – northern \isi{Asia} allows of no meaningful division into eastern or western parts. This language area covers its entirety, westwards to the Urals and, as \ili{Turkish} (or \ili{Finnish}) attest, in parts beyond. Its origin and core, however, do lie firmly within our scope here.
\end{quote}

\noindent Second, despite its focus on one area, this study is still intended to be applicable to other languages. Especially Chapter 4 is a more classical approach to typology that seeks to understand what grammars of questions are cross-linguistically attested and possible (cf. \citealt{Hölzl2016a}). Therefore, it makes extensive use of data from languages outside of \isi{NEA}.

The survey of languages in \isi{Northeast Asia} is intended to be as exhaustive as possible. As \citet[2]{VoegelinVoegelin1964} put it: “In linguistic \isi{ecology}, one begins not with a particular language but with a particular area, not with selective attention to a few languages, but with \textbf{comprehensive attention to all the languages in the area}.” (my boldface) However, some individual languages are underrepresented because of a lack of data. The accuracy and amount of details of descriptions for languages and families varies considerably with my personal experience and the available literature. This book largely relies on previously published material, but several speakers and experts of individual languages were consulted as well. \ili{German} examples are based on my knowledge as a native speaker. Given my educational background, literature in \ili{Chinese}, \ili{English}, and \ili{German} form the linguistic core on which this book is based. There are a few French publications on \isi{NEA} languages that were included as well. \ili{Russian} and especially Japanese literature was consulted where possible, but not with equal intensity. Therefore, the southern part of \isi{NEA} is somewhat overrepresented in this study. Finnish, Hungarian, Korean, and Mongolian publications were necessarily excluded. Other languages play no significant role for the study of the languages of \isi{Northeast Asia}. Unfortunately, most grammatical descriptions are insufficient and only those in \ili{English} and \ili{Japanese} usually reach an international standard with adequate analyses of examples and \isi{glossing}. For a typological study, \ili{Chinese} descriptions that have a rudimentary \isi{glossing} with characters but usually lack a clear \isi{analysis}, are usually more useful than \ili{German} or \ili{Russian} publications that, with some exceptions, usually lack glosses or analyses completely. As a consequence, many of the examples found in this study have been painstakingly analyzed by myself as far as possible, by and large following the \textit{Leipzig Glossing Rules}.\footnote{See \url{https://www.eva.mpg.de/lingua/resources/glossing-rules.php} (Accessed 2016-07-06.)} Remaining uncertainties are signaled with a question mark. For most of the languages in \isi{NEA} only rather brief accounts are available. These are often limited to mentioning a handful of unexplained interrogatives with very rough translations and, with some luck, unanalyzed examples of polar and \isi{content question}s. The length of the descriptions of the languages within this study also varies due to extreme differences in the \isi{complexity} of the \isi{grammar of questions}. It is not always easy to distinguish between simplicity and a lack of information. But there certainly are extremely complex systems such as in the \ili{Yupik} languages that require several pages and tables just to give a rough outline. Some of the most complex systems can be found in \textit{\ili{Omotic}} languages (\ili{Afroasiatic}) spoken in Ethiopia (see \citealt{Amha2012}; \citealt{Köhler2013,Köhler2016}, and references therein). In comparison (i.e., relative \isi{complexity}), most languages of \isi{NEA} have much simpler and typologically more common grammars of questions (e.g., \citealt{Miestamo2008}). Given the large number of languages included in this study, the description of individual languages is necessarily somewhat superficial and experts will certainly have a lot more to say about each of them. For several reasons, \sectref{sec:5.10} on \textit{Tungusic} is somewhat more extensive than those on other language families. First, my personal knowledge of \ili{Tungusic} is better than for many other languages in this study. Second, there are extremely good descriptions of \isi{questions} in some \ili{Tungusic} languages such as \ili{Evenki} and \ili{Udihe}. Third, because of their vast distribution over almost all of \isi{NEA}, \ili{Tungusic} could potentially be crucial for this study (see Chapter 3). This study also includes several varieties that were described only from the 1980s onward by \ili{Chinese} scholars but seem to have mostly gone unnoticed outside of \isi{China}. \ili{Tungusic} languages will also sometimes be considered in other chapters to illustrate certain points.

There have been several earlier studies on \isi{questions} in the languages of \isi{NEA}. There are many good descriptions of \isi{questions} in individual languages such as \citet{ZhangDingjing1991} on \ili{Kazakh}, M. \cite{HayashiM2010} on \ili{Japanese} or \cite{Yoon2010} on \ili{Korean}, to name but a few examples. There are far fewer studies of \isi{questions} in more than one language, but still no exhaustive list can be given here. \citet{Audova1997} briefly investigates \isi{question marking} types in the northern part of \isi{NEA}, but lacks a clear \isi{analysis} and confuses \isi{interrogative} verbs (a subtype of interrogatives) with \isi{question marking}. Nevertheless, she makes some useful observations on possible areal connections. \citegen{LuoTianhua2013} dissertation is an investigation of \isi{questions} in the languages of \isi{China} and thus covers the southern half of \isi{NEA}. Unfortunately, the overview of most languages is superficial and not always reliable. For instance, only two and a half pages are devoted to all the \ili{Tungusic} languages spoken in \isi{China} (\citealt[133–135]{LuoTianhua2013}). Several names of individual languages are erroneous and \ili{Korean} is wrongly classified as a \ili{Tungusic} language. Nevertheless, there are useful insights about \isi{questions} in \ili{Mandarin} and some other languages. More problematic is \citegen[217–234]{Greenberg2000} investigation of interrogatives in so-called \textit{\ili{Eurasiatic}} languages, which compares look-alike elements in a more or less random sample of languages and claims to have proven a genetic connection among them. A high-quality description of polar \isi{question marking} in \ili{Uralic} languages, on the other hand, some of which are spoken in \isi{NEA}, is given by \citet{Miestamo2011}, which is also the most up-to-date description of polar \isi{question marking} types. Yet another very good typology of \isi{questions} in \ili{Austronesian} languages of \isi{Taiwan}, mostly excluded from this study, can be found in \citet{Huang1999}.

In sum, at its core this study is an investigation of the distribution of \isi{structural diversity} in the \isi{grammar of questions} in the limited geographical region of \isi{Northeast Asia} and beyond. The restriction to one category is necessary for reasons of space and clarity, and the process of zooming in on one region allows a higher resolution and historical accuracy than is usually the case in \isi{linguistic typology}. Some of the \isi{questions} addressed by this study are: “What does it mean to question?” \citep[561]{Sanitt2011} Are \isi{questions} indeed \isi{universal}, and if yes, why? What about \isi{questions} is variable? How can this variation be classified? What are possible motivations behind this variation? What patterns do the languages of \isi{Northeast Asia} show with respect to this classification? What roles do geography, genetic inheritance, and \isi{language contact} play in explaining these patterns? Is there \isi{convergent evidence} from other disciplines such as genetics? And finally, does the concept of \isi{Northeast Asia} make sense from the point of view of \isi{areal linguistics}?

This book is organized into seven chapters, including this Introduction. Chapters 2 and 3 briefly present the languages of \isi{NEA} from a genetic and an areal perspective, respectively. Chapter 4 introduces a somewhat new typology of \isi{questions} that is illustrated with languages from around the world. The longest chapter (Chapter 5) gives an extensive overview of the grammars of questions in the fourteen language families of \isi{NEA}. Readers only interested in the typological aspects are advised to skip over this lengthy chapter and consult Chapter 6 instead, which gives an overview of the findings of the previous chapter, illustrated with several geographical maps inspired by the \textit{\isi{World Atlas of Language Structures}} (\citealt{DryerHaspelmath2013}). Chapter 7 presents some conclusions, sketches possible avenues for further research, and briefly summarizes the tentative idea of an \textit{\isi{ecological typology}}. Following the extensive list of References, the Appendix lists the data that were used for the comparative maps of \sectref{sec:6.4}. At the end of the book there are Name, Language, and Subject Indexes.