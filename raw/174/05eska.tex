\section{\ili{Eskaleut}}\label{sec:5.4}
\subsection{Classification of \ili{Eskaleut}}\label{sec:5.4.1}

The Eskimo-\ili{Aleut} or \ili{Eskaleut} \isi{language family} may be classified as in \figref{figex:eska:1} (\citealt{Berge2006,Berge2010}; \citealt{Fortescue2013}; and especially \citealt{FortescueJacobsonKaplan2010}: xiif.).


% \ea\upshape%1
\begin{figure}
\caption{Classification of Eskaleut.}
    \label{figex:eska:1}\small
\begin{forest}  for tree={grow'=east,delay={where content={}{shape=coordinate}{}}},   forked edges  
[
    [\ili{Aleut}*
        [Eastern \ili{Aleut}*]
        [Atkan*]
        [\textsuperscript{†}Attu(an)*]
        [Copper Island (Mednyj) \ili{Aleut}*]
    ]
    [\ili{Eskimo}
        [\ili{Yupik}
        	[\ili{Naukan}(ski) \ili{Yupik}*]
            [Central Siberian \ili{Yupik}*]
            [Alutiiq Alaskan \ili{Yupik}]
            [Central Alaskan \ili{Yupik}]
        ]
        [\textsuperscript{†}Sirenik(ski) (\ili{Yupik}?)*
        ]
        [Inuit
        	[Seward Peninsula Inuit]
            [North Alaskan Inuit]
            [Western Canadian Inuit]
            [Eastern Canadian Inuit]
            [Greenlandic Inuit]
        ]
    ]
]
\end{forest}   
%     \z
\end{figure}

Languages spoken in \isi{Northeast Asia} are signaled with an asterisk, but for the purpose of better understanding, Central Alaskan \ili{Yupik} will be included in the discussion as well. For a more fine-grained classification of subdialects see \cite[xiif.]{FortescueJacobsonKaplan2010}. The primary split is between \ili{Aleut} on the one hand and Eskimo on the other. \ili{Eskimo} itself falls into two main branches, \ili{Yupik} and Inuit. However, Sirenik(ski)---usually considered a part of \ili{Yupik}---could possibly form a third branch of Eskimo (\citealt{FortescueJacobsonKaplan2010}: x). In general, the \ili{Aleut} branch must be considered the most aberrant member of the family. \ili{Aleut} historically formed a dialect \isi{continuum} with linguistic diffusion from east to west but only three main dialect groups are sufficiently attested \citep[14]{Bergsland1997}. Copper Island or Mednyj \ili{Aleut} is a truly mixed language that contains a large number of \ili{Russian} elements, including verbal \isi{morphology} (e.g., \citealt{Comrie1981}: 253; \citealt{Golovko1990}; \citealt{Sekerina1994}; \citealt{Golovko1996}; \citealt{Vakhtin1998}), but is classified with other \ili{Aleut} dialects here.

\subsection{Question marking in \ili{Eskaleut}}\label{sec:5.4.2}

\ili{Eskaleut} languages are famous for their \isi{interrogative} mood, perhaps because of the well-known description of \isi{questions} in \ili{West Greenlandic} by \citet{Sadock1984}, but this is not present in all \ili{Eskaleut} languages. \textbf{Aleut} has a mobile question particle, \textit{hi(i)’} {\textasciitilde} \textit{ii’} with final glottal stop in the Eastern dialect and \textit{ii} in Attuan and Atkan \citep[82]{Bergsland1997} that marks polar and \isi{focus question}s. It is reanalyzed here as enclitic because it freely attaches to the element in \isi{focus}. As expected, the finite verb is focal in \isi{polar question}s.

\newpage 
\ea%2
    \label{ex:eska:2}
    \ili{Aleut} (Atkan)\\
    \ea
    \gll qilagan  piitra-\^{x} hla-\^{x}  tuga-l sa\^{g}a-na-\^{x}=\textbf{ii}?\\
    yesterday  \textsc{pn-abs.sg}  boy-\textsc{abs.sg}  hit-?\textsc{cvb} \textsc{aux}-\textsc{rem}-\textsc{sg}=\textsc{q}\\
    \glt ‘Did Peter hit the boy yesterday?’
    
    \ex
      qilagan  piitra\^{x} hla\^{x} tugal=\textbf{{ii}} sa\^{g}ana\^{x}?\\
    \glt ‘Did Peter really \textit{hit} the boy yesterday?’ \citep[83]{Bergsland1997}
    \z
    \z

No \ili{Aleut} dialect \isi{content question}s have an overt \isi{question marker}. The sentence ‘Who are you?’, for example, is \textit{kiin ax\^ (}\textit{t)?} in Atkan, \textit{kiin txin?} in the Eastern dialect, and \textit{kiin tin?} in Attuan, where only the \isi{interrogative} \textit{kiin} ‘who’ marks the sentence as a question. In Atkan the \isi{interrogative} is followed by a second person form of the copula \textit{a-} ‘to be’ while in the other two dialects there is an overt second person \isi{singular} pronoun that can be analyzed as \textit{t(x)i-n} ‘\textsc{dem}-2\textsc{sg}’ (\citealt{Bergsland1997}: 57, 81, 89, 135). The marking of \isi{questions} in \ili{Aleut} is thus typologically close to the \ili{Tungusic} language \ili{Evenki}, for example, although \isi{alternative question}s contain a \isi{disjunction}.

\ea%3
    \label{ex:eska:3}
    \ili{Aleut} (Atkan)\\
    \gll ting \textbf{{asxuunulax}} txin satxa\^{x}  ta\^{g}aa\^{g}an a\^{x}s?\\
    1\textsc{sg}  or    2\textsc{sg}  gill.net    check.?  be.?\\
    \glt ‘Am I or are you going to check the gill net?’ (\citealt{Bergsland1997}: 83, passim)
    \z

\textbf{Copper Island Aleut} presents a special case because of its strong \ili{Russian} impact. Unfortunately, almost no information on \isi{interrogative} constructions is available. As far as the few examples allow any conclusions, polar and \isi{content question}s were probably unmarked.\footnote{Most elements in these examples are from \ili{Russian}, except for those underlined, which derive from \ili{Eskaleut}.}

\ea%4
    \label{ex:eska:4}
    Copper Island \ili{Aleut}\\
    \ea
    \gll ya-\uline{\^{g}\={\i}m} b\=ud-i\^{s}=min’a    s\=u{-t’?}\\
    1\textsc{sg}-?\textsc{rel}.\textsc{sg}  \textsc{aux}-2\textsc{sg=}1\textsc{sg.acc}  take-\textsc{inf}\\
    \glt ‘Will you take me with you?’
    
    \ex
    \gll \uline{\textbf{{aqú}}-\^{x}} ti  \uline{\=angicik\=ay-us\=a}-i\^{s}?\\
    what-\textsc{abs}.\textsc{sg}  2\textsc{sg}  think-\textsc{tr-}2\textsc{sg}\\
    \glt ‘What are you thinking about?’ (\citealt{Golovko1990}: 103, 104)
    \z
    \z

\noindent The auxiliary \textit{bu(d)-}, the personal ending \textit{-i\^s}, the infinitive marker \textit{-t’} as well as the pronouns \textit{ya}, \textit{min’a} (used as a verbal person marker), and \textit{ti} are of \ili{Russian} origin.

The following short dialogue between a five year old child and her mother in Sirenik was recorded in 1985. The data show a mixed language that is comparable to Copper Island \ili{Aleut} but might be more strongly based on \ili{Russian}. The short dialogue includes an \isi{alternative question} and an \isi{answer}. There is \isi{juxtaposition} of the two alternatives and there is no \isi{question marker} or \isi{disjunction}. Presumably, the question had a special \isi{intonation} contour.

\ea%5
    \label{ex:eska:5}
    ?Central Siberian \ili{Yupik}\\
    \ea
    \gll mam,  ya   èto  \uline{quuv}.a=y.u,    ya  èto  \uline{niv}.a=y.u?\\
    mom  1\textsc{sg}  this  pour.out=1\textsc{sg}    1\textsc{sg}  this  pour.into=1\textsc{sg}\\
    \glt ‘Mom, shall I pour this out or shall I pour this (into something)?’
    
    \ex
    \gll ladno,  \uline{quuv}=a.y.\\
    alright  pour.out=\textsc{imp}\\
    \glt ‘All right, pour it out.’ \citep[324]{Vakhtin1998}
    \z
    \z

As the \isi{alternative question}s appears to consist of two juxtaposed \isi{focus question}s, we may surmise that \isi{focus} and \isi{polar question}s were marked by \isi{intonation} as well.

Before focusing on \ili{Yupik} as spoken in \isi{Siberia}, let us have a brief look at the better-known \textbf{Central Alaskan Yupik} language to establish a \isi{reference point}. Polar questions in this language are marked with a second position marker that also marks \isi{focus question}s.

\ea%6
    \label{ex:eska:6}
    Central Alaskan \ili{Yupik} (General)\\
    \ea
    \gll tekít-ùq${\neq}$\textbf{{qaa}} nuk’aq?\\
    arrive-3\textsc{sg.ind}${\neq}$\textsc{q}  \textsc{pn.abs.sg}\\
    \glt ‘Has Nuk’aq arrived?’
    
    \ex
    \gll {nùk’àq${\neq}$}\textbf{{qaa}} tekit-uq?\\
    \textsc{pn.abs.sg}${\neq}$\textsc{q}    arrive-3\textsc{sg.ind}\\
    \glt ‘Has \textit{Nuk’aq} arrived?’ \citep[168]{Miyaoka2012}
    \z
    \z

Polar questions may also be marked with final rising \isi{intonation} alone. The marker \textit{qaa} has been translated as ‘right’ and may also mark \isi{tag question}s. It is also optionally found on the first alternative in \isi{alternative question}s and combines with a \isi{disjunction}.

\ea%7
    \label{ex:eska:7}
    Central Alaskan \ili{Yupik} (General)\\
    \ea
    \gll enér-pa-ŋqér-tuten, \textbf{{qaá}}/\textbf{{qáa}}?\\
    house-big-have-2\textsc{sg.ind}  \textsc{q}\\
    \glt ‘You have a big house, right?’
    
    \ex
    \gll qayar-pa-li-uq${\neq}$\textbf{qaa} \textbf{wall’u}${\neq}$  pi-cuar-mek?\\
    kayak-big-make-3\textsc{sg.ind}${\neq}$\textsc{q}  or${\neq}$    thing-small-\textsc{abm.sg}\\
    \glt ‘Is he making a big kayak or a small one?’ \citep[170]{Miyaoka2012}
    \z
    \z

Mild \isi{questions} are marked with the suffix \textit{-ɬi-} translated as ‘perhaps’ and \isi{topic question}s (polar or content) with \textit{=mi}.

\ea%8
    \label{ex:eska:8}
    Central Alaskan \ili{Yupik} (General)\\
    \ea
    \gll kai-\textbf{{lli}}-uten?\\
    hungry-\textsc{inf}-2\textsc{sg}.\textsc{ind}\\
    \glt ‘Are you hungry?’
    
    \ex
    \gll kaig-tuten=\textbf{{qaa}}?\\
    hungry-2\textsc{sg}.\textsc{ind}=\textsc{q}\\
    \glt ‘Are you hungry?’
    
    \ex
    \gll u-na=\textbf{{mi}}?\\
    this-\textsc{ex.abs}=\textsc{q}\\
    \glt ‘How about this?’
    
    \ex
    \gll \textbf{{nauwa}}=\textbf{{mi}} aata-ka?\\
    where=\textsc{q}  father-\textsc{abs}.1\textsc{sg.sg.poss}\\
    \glt ‘(But) where is my father?’ (\citealt{Miyaoka2012}: 173, 171)
    \z
    \z

Content questions uttered in soliloquy contain an enclitic \textit{=ki{̇γ}} ‘I wonder’ that attaches to the initial \isi{interrogative}.

\ea%9
    \label{ex:eska:9}
    Central Alaskan \ili{Yupik} (General)\\
    \gll \textbf{{qaillun}}=\textbf{{kiq}} tai-\textbf{{ga}}?\\
    how=\textsc{q}    come-3\textsc{sg}.\textsc{q}\\
    \glt ‘How did he come over, I wonder?’ \citep[1360]{Miyaoka2012}
    \z

\noindent Whether Central Siberian \ili{Yupik} or \ili{Naukan} \ili{Yupik} share all of these question markers remains obscure from the limited and problematic publications available to me.

As in this last example, and similar to \ili{Yukaghiric} languages (\sectref{sec:5.14.2}), \isi{content question}s exhibit an additional \isi{interrogative} mood marking on the verb that replaces declarative endings.

\ea%10
    \label{ex:eska:10}
    Central Alaskan \ili{Yupik} (General)\\
    \ea
    \gll nùk’àq    tekít-\textbf{{ùq}}.\\
    \textsc{pn.abs.sg}  arrive-3\textsc{sg.ind}\\
    \glt ‘Nuk’aq arrived.’
    
    \ex
    \gll \textbf{{ki}}-na    tekít-\textbf{{a}}-\textbf{{Ø}}?\\
    who-\textsc{abs.sg}  arrive-3\textsc{q-}3\textsc{sg.q}\\
    \glt ‘Who has arrived?’ \citep[167]{Miyaoka2012}
    \z
    \z

\noindent This last type of \isi{question marking} shows that \isi{questions} in \ili{Yupik} are much more complicated than in \ili{Aleut} as they combine special \isi{interrogative} mood suffixes with special \isi{interrogative} person endings.

Morphological \isi{question marking} in \ili{Yupik} involves two layers of suffixes. The first is an actual \isi{question marker} and attaches to the stem (\tabref{tab:eska:1}), followed by the second, which is an agreement marker of person and number specialized for \isi{questions}. Regarding the second layer, there is a distinction between intransitive and transitive paradigms. What is more, the first layer exhibits an additional distinction into different forms that depends on person as well. There are, furthermore, some complex morphonological patterns of interactions between the stem and the two layers of suffixes that cannot be dealt with here in detail, e.g. CAY \textit{niic+ta+ɣu} ‘hear+3\textsc{q}+3\textsc{sg}.S.3\textsc{sg}.O.\textsc{q}’ > \textit{niitau} ‘does (s)he hear it?’ \citep[1350]{Miyaoka2012}. \tabref{tab:eska:1} lists the first layer of \isi{question marking}. Generally, first and second persons are marked the same way, while third person receives another marker.

\begin{table}
\caption{Simplified inventory of interrogative mood endings in Central Alaskan Yupik \citep[1352]{Miyaoka2012}, Central Siberian Yupik \citep[61]{Jacobson1979}, Naukan Yupik (\citealt{Menovshchikov1975}: 240ff.), and Sirenik \citep[517]{Vakhtin2000}}
\label{tab:eska:1}

\begin{tabularx}{\textwidth}{XXXXl}
\lsptoprule
& \textbf{CAY} & \textbf{\textsc{CSY}} & \textbf{Naukan} & \textbf{Sirenik}\\
\midrule
\textsc{3sg} & C +ta {\textasciitilde} V +ɣa & +(t)a {\textasciitilde} +(g)a & -aa & -taa {\textasciitilde} -tsaa\\
\textsc{3pl} & id. & id. & id. & id.\\
\textsc{3du} & id. & id. & id. & id.\\
\textsc{1sg} & +\textsubscript{1}ci & +(t)zi & -si(i) {\textasciitilde} -jii & -sii {\textasciitilde} -tsii\\
\textsc{1pl} & id. & +ste & id. & id.\\
\textsc{1du} & id. & id. & id. & id.\\
\textsc{2sg} & id. & +(t)zi & id. & id.\\
\textsc{2pl} & id. & +ste & id. & id.\\
\textsc{2du} & id. & id. & id. & id.\\
\lspbottomrule
\end{tabularx}
\end{table}

The morphosyntactic behavior of the forms is quite complex, but has only been described in sufficient detail for Central Alaskan \ili{Yupik}: “The initial /c/ of the first- and second-person mood markers is fricativized to /z/ after a vowel if the subject is \isi{singular} and, if the subject is non-\isi{singular}, (though with some variance) after a stem that ends in a stop plus /ɨ/.” \citep[1352]{Miyaoka2012}

Intransitive \isi{interrogative} agreement forms are given in \tabref{tab:eska:2}. Apart from Sirenik, the individual affixes are very similar across the different languages.

\begin{table}
\caption{Intransitive interrogative person endings in Central Alaskan Yupik \citep[1352]{Miyaoka2012}, Central Siberian Yupik (St. Lawrence Island, \citealt{Jacobson1979}: 61), Naukan Yupik \citep[240]{Menovshchikov1975}, and Sirenik \citep[521]{Vakhtin2000}}
\label{tab:eska:2}

\begin{tabularx}{\textwidth}{XXXXl}
\lsptoprule

S & \textbf{CAY} & \textbf{\textsc{CSY}} & \textbf{Naukan} & \textbf{Sirenik}\\
\midrule
\textsc{3sg} & +Ø & +Ø & -Ø & +a\\
\textsc{3pl} & +t & +t & -t & +i\\
\textsc{3du} & +ɣ & +k & -k & -\\
\textsc{1sg} & +ŋa & +ŋa & -see, -Ø & +n’\\
\textsc{1pl} & +ta & +a & -ta & +ta\\
\textsc{1du} & +nuɣ & +uŋ & -nuŋ & -\\
\textsc{2sg} & +t & +n & -n & +Ø\\
\textsc{2pl} & +ci & +tsi & -si & +si\\
\textsc{2du} & +tɨɣ & +k & -tyk & -\\
\lspbottomrule
\end{tabularx}
\end{table}

In \ili{Naukan} \ili{Yupik} the form \textit{-see} might derive from a \isi{combination} of \textit{-si(i)} with \textit{-ŋa}. Note a parallel in the transitive paradigm below: \textit{-see} ‘\textsc{2pl.S.2sg.O}’ = CAY \textit{+ci+ŋa}. As we have just seen in \tabref{tab:eska:1}, CAY \textit{-ci} corresponds to \ili{Naukan} \ili{Yupik} \textit{-si(i)} ({\textasciitilde} \textit{-jii}).

Tables \ref{tab:eska:3}, \ref{tab:eska:4}, \ref{tab:eska:5} contain transitive \isi{interrogative} endings from Central Alaskan \ili{Yupik}, Central Siberian \ili{Yupik}, and \ili{Naukan} \ili{Yupik}.

\begin{table}
\caption{Transitive interrogative person endings in Central Alaskan Yupik \citep[1352]{Miyaoka2012}; forms in parentheses are identical with the intransitive forms}
\label{tab:eska:3}
\fittable{
\begin{tabular}{llllllllll}
\lsptoprule

A/O & \textsc{3sg} & \textsc{3pl} & \textsc{3du} & \textsc{1sg} & \textsc{1pl} & \textsc{1du} & \textsc{2sg} & \textsc{2pl} & \textsc{2du}\\
\midrule
\textsc{3sg} & +ɣu & +ki & +kɨɣ & +ŋa & +kut & +kuɣ & +tɨn & +ci & +tɨɣ\\
\textsc{3pl} & +tɣu & +tki & +tkɨɣ & +tŋa & +tkut & +tkuɣ & +tɣɨn & +tci & +ttɨɣ\\
\textsc{3du} & +ɣnɨɣu & +ɣnɨki & +ɣnɨkiɣ & +ɣŋa & +ɣkut & +ɣk+kuɣ & +ɣtɣɨn & +ɣtci & +ɣttɨɣ\\
\textsc{1sg} & (+ŋa) & (+ŋa) & (+ŋa) &  &  &  & +kɨn\footnotemark & ? & ?\\
\textsc{1pl} & (+ta) & (+ta) & (+ta) &  &  &  & ? & ? & ?\\
\textsc{1du} & (+nuɣ) & (+nuɣ) & (+nuɣ) &  &  &  & ? & ? & ?\\
\textsc{2sg} & +ɣu & +ki & +kɨɣ & +ŋa & +kut & +kuɣ &  &  & \\
\textsc{2pl} & +ci+ɣu & +ciki & +cikɨɣ & +ci+ŋa & +cikut & +cikuɣ &  &  & \\
\textsc{2du} & +tɨɣu & +tɨki & +tkɨɣ & +tɨɣŋa & +tɨɣkut & +tɨɣkuɣ &  &  & \\
\lspbottomrule
\end{tabular}
}
\end{table}

\footnotetext{ In the original table this form was given one row below, which seems to be a mistake (cf. \citealt{Miyaoka2012}: 1350). \citet{Miyaoka2012} is not sufficiently clear about the gaps marked with a question mark here. The other languages show indicative forms here that roughly correspond to CAY \textit{-mt+ɣɨn} ‘1\textsc{pl}.2\textsc{sg}’, \textit{-mɨɣtɨn} ‘1\textsc{du}.2\textsc{sg}’, \textit{-mci} ‘1\textsc{sg}.2\textsc{pl}’, \textit{-mtci} ‘1\textsc{pl}.2\textsc{pl}’, \textit{-mɨɣci} ‘1\textsc{du}.2\textsc{pl}’, \textit{-mtɨɣ} ‘1\textsc{sg}.2\textsc{du}’, \textit{-mttɨɣ} ‘1\textsc{pl}.2\textsc{du}’, and \textit{-mɨɣtɨɣ} ‘1\textsc{du}.2\textsc{du}’ \citep[1325]{Miyaoka2012}.}

\begin{table}
\caption{Transitive interrogative person endings in Central Siberian Yupik (St. Lawrence Island, \citealt{Jacobson1979}: 61, 56); forms in parentheses are identical with indicative forms}
\label{tab:eska:4}
\fittable{
\begin{tabular}{llllllllll}
\lsptoprule

A/O & \textsc{3sg} & \textsc{3pl} & \textsc{3du} & \textsc{1sg} & \textsc{1pl} & \textsc{1du} & \textsc{2sg} & \textsc{2pl} & \textsc{2du}\\
\midrule
\textsc{3sg} & +gu & +ki & +kek & +ŋa & +\textsubscript{i}nkut & +\textsubscript{i}nkuŋ & +ten & +\textsubscript{i}si & +\textsubscript{i}stek\\
\textsc{3pl} & +tgu & +tki & +tkek & +tŋa & +\textsubscript{i}nkut & +\textsubscript{i}nkuŋ & +ten & +\textsubscript{i}si & +\textsubscript{i}stek\\
\textsc{3du} & +gnegu & +gneki & +gnekek & +gneŋa & +\textsubscript{i}nkut & +\textsubscript{i}nkuŋ & +ten & +\textsubscript{i}si & +\textsubscript{i}stek\\
\textsc{1sg} & +kun & +ngi & +gngek &  &  &  & +ken & (-msi) & (-mtek)\\
\textsc{1pl} & +ggu & +ki & +kek &  &  &  & +ken & (-msi) & (-mtek)\\
\textsc{1du} & +gnegu & +gneki & +gnekek &  &  &  & +ken & (-msi) & (-mtek)\\
\textsc{2sg} & +gu & +ki & +kek & +ŋa & +kut & +kuŋ &  &  & \\
\textsc{2pl} & +ggu & +ki & +kek & +ŋŋa & +kut & +kuŋ &  &  & \\
\textsc{2du} & +gnegu & +gneki & +gnekek & +gneŋa & +gnekut & +gnekuŋ &  &  & \\
\lspbottomrule
\end{tabular}
}
\end{table}

\begin{table}
\caption{Tentative transitive interrogative person endings in Naukan Yupik (based on \citealt{Menovshchikov1975}: 241f.)}
\label{tab:eska:5}
\fittable{
\begin{tabular}{llllllllll}
\lsptoprule

A/O & \textsc{3sg} & \textsc{3pl} & \textsc{3du} & \textsc{1sg} & \textsc{1pl} & \textsc{1du} & \textsc{2sg} & \textsc{2pl} & \textsc{2du}\\
\midrule
\textsc{3sg} & -ûŋ & -ki & -kyk & -ŋa & -tykut & -tykuk & -tyn & -si & -tyk\\
\textsc{3pl} & -txuŋ & -tyki & -tykyk & -tŋa & -tykut & -tykuk & -tyn & -si & -tyk\\
\textsc{3du} & ? & ? & ? & ? & ? & ? & ? & ? & ?\\
\textsc{1sg} & -ku & -nka & -xka &  &  &  & -kyn & -msi & -mtyk\\
\textsc{1pl} & -vut & -put & -xput &  &  &  & -mtykyn & -msi & -mtyk\\
\textsc{1du} & -vuk & -puk & -xpuk &  &  &  & ? & ? & ?\\
\textsc{2sg} & -ûŋ & -ki & -kyk & -ŋa & -kut & -kuk &  &  & \\
\textsc{2pl} & -siûŋ & -siki & -sikyk & -see & -sikut & -sikuk &  &  & \\
\textsc{2du} & ? & ? & ? & ? & ? & ? &  &  & \\
\lspbottomrule
\end{tabular}
}
\end{table}

In Sirenik, intransitive first person \isi{singular} and second person \isi{plural} forms are identical to the declarative endings. Paradigms for transitive verbs are almost entirely unknown. \citet[521]{Vakhtin2000} mentions \textit{+(gy)pyn’}/+(\textcyrillic{гы)пын’} ‘2\textsc{sg}.A.1\textsc{sg}.O’, \textit{+n’}/+\textcyrillic{н’} ‘3\textsc{pl}.A.1\textsc{sg}.O’, \textit{+kyn}/+\textcyrillic{кын} ‘1\textsc{sg}.A.2\textsc{sg}.O’, \textit{+tyn}/+\textcyrillic{тын} ‘3\textsc{sg}.A.2\textsc{sg}.O’, and \textit{+gu}/+\textcyrillic{гу} ‘2/3\textsc{sg}/1\textsc{pl}.A.3\textsc{sg}.O’. Apart from the first, these seem to correspond to Central Alaskan \ili{Yupik} +\textit{(t)ŋa}, \textit{+kɨn}, \textit{+tɨn}, and \textit{+(t/ɣnɨ)ɣu}, respectively (\tabref{tab:eska:3}).

Some of the agreement forms in Central Alaskan \ili{Yupik} (\tabref{tab:eska:3}) are still analyzable in two different affixes (stem-A-O). According to this observation, the following suffixes can be extracted: \textit{-Ø-} ‘3\textsc{sg}.A’, \textit{-t-} ‘\textsc{3pl.A}’, \textit{-ɣ(nɨ)-} ‘\textsc{3du.A}’, \textit{-Ø-} ‘2\textsc{sg}.A’, \textit{-ci-} ‘\textsc{2pl.A}’, and \textit{-tɨɣ-} ‘\textsc{2du.A}’. These are related to, but not identical with the intransitive markers (\tabref{tab:eska:2}). With some exceptions, these suffixes are also present in Central Alaskan \ili{Yupik} and \ili{Naukan} \ili{Yupik}. In Central Alaskan \ili{Yupik} “gaps in the paradigm are filled in with an intransitive person marker, which is extended to transitive use, without distinguishing the object number” \citep[1350]{Miyaoka2012}. Central Siberian \ili{Yupik} on the other hand has special third person as well as second person \isi{singular} object forms and employs the indicative forms as second person \isi{plural} and \isi{dual} object endings. In Central Siberian \ili{Yupik}, the \isi{interrogative} mood marker (\tabref{tab:eska:1}) takes a form with \textit{i} instead of \textit{a} before the endings with subscript \textit{\textsubscript{i}}. “The final or semi-final vowel of these endings if often lengthened (and e changed to a) if the verb is used in a ‘yes’ or ‘no’ question.” \citep[61]{Jacobson1979} \citegen[242]{Menovshchikov1975} table of \isi{interrogative} forms on which \tabref{tab:eska:5} was based seems to be rather problematic, as it apparently shows some confusion regarding grammatical relations. My \isi{analysis} usually follows the comparison with Central Alaskan (\tabref{tab:eska:3}) and Central Siberian \ili{Yupik} (\tabref{tab:eska:4}) (see also \citealt{Menovshchikov1975}: 241). For lack of data, \isi{dual} A forms have usually been excluded. In some instances either intransitive or transitive verb endings may be employed with a slight change of meaning.

\ea%11
    \label{ex:eska:11}
    Central Alaskan \ili{Yupik}\\
    \ea
    \gll \textbf{{ca}}-mek    ner-yug-\textbf{{ci}}-\textbf{{t}}?\\
    what-\textsc{abm.sg}  eat-\textsc{des}-2\textsc{sg.q}-2\textsc{sg.intr.q}\\
    \glt ‘What (kind of food) do you want to eat?’
    
    \ex
    \gll \textbf{{ca}} ner-yug-\textbf{{ci}}-\textbf{{u}}?\\
    what    eat-\textsc{des}-2\textsc{sg.q}-2\textsc{sg.tr.q}\\
    \glt ‘What/which (specific) food do you want to eat?’ \citep[756]{Miyaoka2012}
    \z
    \z

A difference between Central Alaskan \ili{Yupik} and Central Siberian \ili{Yupik} is that in the former they are limited to \isi{content question}s while in the latter they are also encountered in \isi{polar question}s.

\ea%12
    \label{ex:eska:12}
    Central Siberian \ili{Yupik} (St. Lawrence Island)\\
    \ea
    \gll negh-yug-\textbf{{si}}-\textbf{{n}}?\\
    eat-\textsc{des}-2\textsc{sg.q}-2\textsc{sg.q}\\
    \glt ‘Do you want to eat (anything)?’
    
    \ex
    \gll \textbf{{sa}}-meŋ    negh-yug-\textbf{{si}}{-}\textbf{{n}}?\\
    what-\textsc{abm.sg}  eat-\textsc{des}-2\textsc{sg.q}-2\textsc{sg.intr.q}\\
    \glt ‘What (kind of food) do you want to eat?’\citep[60]{Jacobson1979}
    \z
    \z

\begin{table}
\caption{Summary of question marking in \ili{Eskaleut}.}
\label{tab:eska:6}
\small
\begin{tabularx}{\textwidth}{lQQl}
\lsptoprule
& \textbf{PQ} & \textbf{CQ} & \textbf{AQ}\\
\midrule
\ilit{Aleut} (Atkan) & =ii & - & asxuunulax ‘or’\\
Mednyj \ilit{Aleut} & - & - & ?\\
CAY & V${\neq}$qaa & (see Tables \ref{tab:eska:1}, \ref{tab:eska:2}, \ref{tab:eska:3}) & (A${\neq}$qaa) + wall’u ‘or’\\
CSY & (see Tables \ref{tab:eska:1}, \ref{tab:eska:2}, \ref{tab:eska:4}) & id. & ?\\
\ilit{Naukan} & (see Tables \ref{tab:eska:1}, \ref{tab:eska:2}, \ref{tab:eska:5}) & ?id. & ?\\
Sirenik & (see Tables \ref{tab:eska:1}, \ref{tab:eska:2}) & ?id. & ?\\
\lspbottomrule
\end{tabularx}
\end{table}

There is a marked contrast between \ili{Aleut} and \ili{Yupik} \isi{question marking} (\tabref{tab:eska:6}). \ili{Aleut} resembles the Northeast Asian mainstream, while \ili{Yupik} belongs to an area in the northern part of \isi{NEA} that exhibits complex \isi{interrogative} mood systems (e.g., \citealt{Audova1997}). Other languages belonging to this belt are \ili{Nganasan} (\sectref{sec:5.12.2}), \ili{Yukaghiric} (\sectref{sec:5.14.2}), and perhaps \ili{Negidal} (\sectref{sec:5.10.2}).

\subsection{Interrogatives in \ili{Eskaleut}}\label{sec:5.4.3}

The comparison of interrogatives in \ili{Yupik} languages and Sirenik is relatively straightforward (\tabref{tab:eska:7}). The \isi{interrogative} system in all four languages listed is relatively similar, but Sirenik is clearly the most aberrant. All three languages have resonances in \textit{q{\textasciitilde}} and \textit{n{\textasciitilde}}. Apart from CAY, there is an additional \isi{resonance} in \textit{s{\textasciitilde}}. \ili{Yupik} thus has K-interrogatives. The authors also mention the form PE *\textit{ay} ‘what did you say’ (CAY \textit{ai}, CSY \textit{ay}, \ili{Naukan} \ili{Yupik} \textit{ay}), but this is not a true \isi{interrogative} (\citealt{FortescueJacobsonKaplan2010}: 62). Other interrogatives such as *\textit{cuuq} ‘why’ or *\textit{qanuq} ‘how’ can only be found in Inuit (\citealt{FortescueJacobsonKaplan2010}: 98, 310). Central Siberian \ili{Yupik} interrogatives were also mostly left unexplained in \citegen[49, 57, passim]{Jacobson2001} description. Fortunately, there is a very good \isi{analysis} for Central Alaskan \ili{Yupik} by \cite[443-461]{Miyaoka2012} that can be transferred to the other languages.

\begin{table}
\caption{PE = \ili{Proto-Eskimo} and PY-S = \ili{Proto-Yupik-Sirenik} interrogatives and cognate sets according to \cite[97, 98, 190, 223, 304, 310, 318]{FortescueJacobsonKaplan2010}; not all variants and dialectal forms are shown}
\label{tab:eska:7}

\begin{tabularx}{\textwidth}{QQQQQ}
\lsptoprule

\textbf{Reconstructions} & \textbf{CAY} & \textbf{CSY} & \textbf{Naukan} & \textbf{Sirenik}\\
\midrule
PE *ki(na)\newline ‘who’ & kina,\newline \textsc{pl} kinkut & kina,\newline \textsc{pl} kinkut & kina,\newline \textsc{pl} kinkut & kina,\newline \textsc{pl} kinəki\\
\tablevspace 
PE *kitu\newline ‘which, who’ & kitu- & kitu- & kitu- & -\\
\tablevspace 
PE *qaku\newline ‘when.\textsc{fut}’ & qaku & qakun & qaku & qaku\\
\tablevspace 
PE *qaŋa, qaŋyaq\newline ‘when.\textsc{pst}’ & qaŋyaq & qavŋaq & qamvaq & qaŋ\\
\tablevspace 
PE *qavcit\newline ‘how many’ & qavcin & qafsina(t), qafsiin ‘which in a row’ & qafsit & qafsi(ɣ-)\\
\tablevspace 
PE *na-\newline ‘where’ & na- & na- ‘which’ & na- & na-\\
\tablevspace 
PE *naɣu, *na(C)uŋ\newline ‘where (is it)’ & nauxu & 3\textsc{sg} naaɣu,\newline 3\textsc{du} nakək,\newline 3\textsc{pl} naaki & naa ‘where’ & 3\textsc{sg} naɣəla\\
\tablevspace 
PE *nalli\textsc{r}\newline ‘which’ & naliq & naliq & naliq & nacəχ\\
\tablevspace 
PY-S *natə-\newline ‘which (part)’ & natə- & natə- ‘where’,\newline natu- ‘which’ & - & \mbox{natəlŋuχ ‘which’}, natu ‘where’\\
\tablevspace 
PY-S *natən\newline ‘how’ & - & natən & natən & natən\\
\tablevspace 
PY-S *caŋan, *caŋami\newline ‘why’ & ciin & 3\textsc{sg} saŋan,\newline 3\textsc{sg.refl} saŋami,\newline 1\textsc{sg} saŋama & si(i)mi & saŋami\\
\tablevspace 
PE *cu(na)\newline ‘(to do) what’ & ca- & sa- & suna,\newline sa- & saŋəca ‘what’,\newline \mbox{sa- ‘to do what’}\\
\lspbottomrule
\end{tabularx}
\end{table}

Similar to \ili{Tungusic}, \ili{Chukotko-Kamchatkan}, and \ili{Samoyedic} the CSY stem \textit{sa-} ‘(to do) what’ (CAY \textit{ca-}) may take both nominal and verbal \isi{morphology} and the forms meaning ‘why’ are derived from its verbal form. CAY has a form \textit{ciin} ‘why’ that is a contraction of \textit{ca-ŋan}, a third person \isi{singular} causal connective mood form and cognate with CSY \textit{sa-ŋan}. \ili{Naukan} \ili{Yupik} \textit{si(i)mi} has a similar phonological development and seems to correspond directly with the third person reflexive form \textit{sa-ŋami} in CAY and Sirenik. According to \citet{Jacobson2001}, \textit{sa-ŋami} is a form that requires a third person singuar subject while \textit{sa-ŋameŋ} (not listed in \tabref{tab:eska:7}) is used with third person \isi{plural} forms. An interesting speciality of \ili{Yupik} and Sirenik is the existence of two forms meaning ‘when’ for future and past actions that has no equivalent in \isi{NEA}. The suffix \textit{-ku} in \textit{qa-ku-} is a future form, but \cite[452f.]{Miyaoka2012} does not comment on the etymology of \textit{qaŋva\.{ɣ}}\textit{-} ‘when (\textsc{pst})’ (CSY \textit{qavŋaq}, \ili{Naukan} \ili{Yupik} \textit{qamvaq}) but is of the opinion that it also derives from the stem \textit{qa(ŋ)-}. CAY \textit{qavci-n} ‘how many’ (CSY \textit{qafsiin}, \ili{Naukan} \ili{Yupik} \textit{qafsit}) is a \isi{plural} absolutive form of the stem \textit{qavci\.{ɣ}}\textit{-} (Sirenik \textit{qafsi(ɣ-)}). All four languages above have KIN-interrogatives, although the stem really is \textit{ki(t)-}, \textit{ki-na} being its \isi{singular} absolutive form and \textit{kin-kut} its \isi{plural} absolutive form. The selective \isi{interrogative} \textit{naliq} is apparently an unanalyzable form, and can be inflected, e.g. \textit{nallir-put} ‘which one of us’ (cf. \citealt{Miyaoka2012}: 451). The form \textit{natən} ‘how’ is restricted to \ili{Naukan} \ili{Yupik}, CSY, and Sirenik (\citealt{FortescueJacobsonKaplan2010}: 223), while CAY has the interrogatives \textit{qaillun} {\textasciitilde} \textit{qaill’} and \textit{qayu-} instead (\citealt{Miyaoka2012}: 454f.). The special form \textit{natən} is certainly connected with the stem \textit{na-} that is ambiguous and means both ‘which’ and ‘where’. \tabref{tab:eska:8} compares locative \isi{interrogative} paradigms in CSY and CAY. Demonstratives in Central Alaskan \ili{Yupik} also have an allative ending \textit{+vɨt} and variants \citep[769]{Miyaoka2012}.

\begin{table}
\caption{CSY \citep{Jacobson2001} and CAY locative interrogatives \citep{Miyaoka2012}}
\label{tab:eska:8}

\begin{tabularx}{\textwidth}{XXl}
\lsptoprule

\textbf{Meaning} & \textbf{CSY} & \textbf{CAY}\\
\midrule
where (-\textsc{loc}) & na-ni & na-ni\\
whence (-\textsc{abl}) & na-ken & na-ken\\
whither (-\textsc{all}) & na-\textbf{vek} & na-mun\\
\lspbottomrule
\end{tabularx}
\end{table}

Because of a rather unsystematic presentation by \citet[80-83]{Bergsland1997}, no complete \isi{analysis} of interrogatives in all the \textbf{Aleut} dialects can be presented here (\tabref{tab:eska:9}). Copper Island \ili{Aleut}, in addition to \ili{Aleut} interrogatives, has borrowed a number of \ili{Russian} interrogatives (\tabref{tab:eska:10}). Similar to \ili{Eskimo} and several other languages in \isi{NEA} (Chapter 6), no other \isi{interrogative} starts with the same consonant as does \textit{kiin} ‘who’. However, the \isi{interrogative} system is quite different from \ili{Yupik} and Sirenik, although some similarities can be observed. The personal \isi{interrogative} \textit{kiin} ‘who’ (\textsc{du} \textit{kiinkux,} \textsc{pl} Eastern \textit{kiinkun,} Atkan \textit{kiinkus}), for example, is directly comparable. There is one major \isi{resonance} in \textit{q{\textasciitilde}}. The stem \textit{qana-} has the same \isi{semantic scope} over selective and locative meaning as does \textit{na-} in \ili{Yupik} and Sirenik. The stem \textit{alqu-} (Attuan \textit{aqu-}) ‘(to do/be) what, what kind/part, to be how’ is entirely absent from Eskimo, but exhibits the same ambiguity between a verbal and nominal stem as does PE *\textit{cu-} ‘(to do) what’. The causal \isi{interrogative} \textit{alqu-l} ‘why’ likewise has a verbal basis.

\begin{table}
\caption{Atkan Aleut interrogatives according to \cite[80ff.]{Bergsland1997}}
\label{tab:eska:9}

\begin{tabularx}{\textwidth}{Xl}
\lsptoprule

\textbf{Meaning} & \textbf{Form}\\
\midrule
who & kiin, \textsc{du} kiin-kux, \textsc{pl} kiin-kus\\
what (-\textsc{abs.sg}) & alqu-\^{x} \\
to do what (-\textsc{v}-) & alqu-sa-\\
which & qanan, \textsc{pl} qana-kus\\
where & qanang\\
why (-\textsc{cvb}) & alqu-l\\
how & qanamat-, qanamasix\\
how many/much & qanaang\\
when & qanayaam\\
\lspbottomrule
\end{tabularx}
\end{table}

\begin{table}
\caption{Interrogatives in Copper Island (Mednyj) Aleut \citep[26]{Sekerina1994} in comparison with Attuan Aleut (\citealt{Bergsland1997}: 80ff.) and Russian (\sectref{sec:5.5.3.3})}
\label{tab:eska:10}

\begin{tabularx}{\textwidth}{XXXl}
\lsptoprule
& \textbf{Mednyj Aleut} & \textbf{Attuan Aleut} & \textbf{Russian}\\
\midrule
who & kiin & kiin & -\\
what (-\textsc{abs.sg}) & aqu-\^{x}  {\textasciitilde} aqo-\^{x}  & aqu-\^{x}  & -\\
how & aqu-ta-l & aqu-ta- & -\\
why & aqo-li, aqu-ma & aqu- & -\\
where from & qanaaga & qanaax & -\\
how many/much & qanan & qanang & -\\
when & ka(g)da & - & kokda/\textcyrillic{когда}\\
which & kakuy & - & kakoj/\textcyrillic{какой}\\
which & katorəye & - & kotoryj/\textcyrillic{который}\\
where & kuda & - & kuda/\textcyrillic{куда}\\
\lspbottomrule
\end{tabularx}
\end{table}

\clearpage 