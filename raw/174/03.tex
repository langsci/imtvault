\chapter{Areal typology and Northeast Asia}\label{sec:3}

Chapter 2 has introduced the languages of \isi{Northeast Asia} from a genetic perspective, i.e. classified into language families. The focus in the present chapter is on \isi{language contact} instead and adds an areal perspective to the discussion. The two classifications are not always clearly separable, especially at greater time depths (e.g., \citealt{Nichols2010}; \citealt{Operstein2015}), and given the fact that, naturally, languages from one family can also have \isi{contact} with each other (e.g., \citealt{Epps2013}). Since an exhaustive presentation of all \isi{language contact} phenomena goes well beyond the possibilities of this study---“Language \isi{contact} is everywhere” \citep[8]{Thomason2001}---, there will be a focus on some points that are especially relevant.

\section{Theoretical considerations}\label{sec:3.1}

This chapter is concerned with \isi{structural diversity}, or rather, structural \isi{similarity} among languages. There are several different reasons that languages can be similar, including universals, tendencies, chance, genetic inheritance, and \isi{language contact} (e.g., \citealt{AikhenvaldDixon2001}: 1-3). It seems that all languages around the globe have specialized constructions for asking \isi{questions}, so that this is a linguistic \isi{universal} and the reason this study is possible in the first place. Interestingly, there might even be \isi{universal questions} such as ‘\isi{What is your name?}’, ‘Who are you?’, and ‘What is that?’ that are, however, expressed differently from language to language.\footnote{David Gil (p.c. 2018) informs me that he is working on a typology of the question ‘What is your name?’ on which see also \cite{Idiatov2007,Hölzl2014b} and §§\ref{sec:4.3.1}, \ref{sec:5.6.3}.} There may be yet more specific universals. \citet[1]{DingemanseTorreiraEnfield2013} have quite convincingly shown that the repair initiator \textit{\isi{huh}?} could well be a \isi{universal} word “not because it is innate but because it is shaped by selective pressures in an [\isi{enchronic}] interactional environment that all languages share: that of other-initiated repair.” In my opinion, potential universals of this kind have to be distinguished from strong tendencies, such as the fact that positive one-word answers in a great many languages around the globe contain \isi{laryngeal sounds} [h] and [ʔ] \citep{Parker2006}. Take \ili{German}, for instance, which has the word \textit{ja} ‘yes’. At first glance, this does not contain any \isi{laryngeal sounds}, but it has many different variations, among which one encounters [jaʔ] with a final glottal stop as well as ingressive [hja↓] with an initial laryngeal fricative (my knowledge). A similar \isi{tendency} is for languages to have rising \isi{intonation} in \isi{polar question}s, which is common but by no means \isi{universal}. \ili{Hawai’i Creole English}, for example, has falling \isi{intonation} instead \citep[353]{Veluppilai2012}. A factor that should not be underestimated is chance resemblance. An example from the category of \isi{questions} are the \isi{polar question} markers \textit{-(V)ʔ} in Hup (Nadahup, \citealt{Epps2008}: 784ff.) in South America and \textit{-ʔ} in \ili{Crow} (Siouan, \citealt{Graczyk2007}: 391) in North America that at least in some instances are basically identical. To my knowledge there does not appear to be a general \isi{tendency} for question markers to exhibit \isi{laryngeal sounds}, as far as we know the two languages do not share a common ancestor, and there certainly was no \isi{contact} between them. This only leaves pure coincidence to account for this \isi{similarity}. An example for a chance resemblance in the \isi{interrogative} system would be \ili{Tocharian} (\ili{Indo-European}) \textit{kos} and \ili{Dolgan} (\ili{Turkic}) \textit{kas} ‘how much’. Given that both interrogatives and question markers tend to be very short, chance resemblance is extremely hard to distinguish from genetic inheritance and \isi{language contact}. As seen in Chapter 2, genetic inheritance refers to languages of one and the same \isi{language family} that go back to one proto-language and therefore preserve features that are similar to each other. The two \ili{Tungusic} languages \ili{Evenki} and \ili{Even}, to take a random example, shared a common ancestor only several centuries ago and therefore display many similarities such as an almost identical \isi{question marker} \textit{=Ku} (\sectref{sec:5.10.2}). The last of the explanations for the \isi{similarity} between different languages is \isi{language contact}. The term \textit{language contact}, of course, is nothing but a metaphorical abstraction of what is actually an integral part of the complex \isi{interaction} of different human beings. But certainly it serves its purpose to facilitate our discourse on the topic. Language \isi{contact} presupposes a linguistic \isi{interaction} of speakers of different languages (\citealt{Thomason2001}: 1f.). Perhaps every linguistic \isi{interaction} has certain properties that qualify as \isi{language contact}. However, \isi{language contact} is usually identified through observable results such as the \isi{borrowing} of elements. Contact may be either direct or indirect. The latter can be further divided into the \isi{contact} of two languages with a transmitting language on the one hand, and a common \isi{contact} language of two languages on the other.

The outcome of \isi{language contact} differs from instance to instance. \citet[10]{Thomason2001} suggests “a hierarchical set of typologies, starting with a three-way division at the top level into contact-induced \isi{language change}, extreme language mixture (resulting in \isi{pidgins}, \isi{creoles}, and bilingual \isi{mixed languages}), and \isi{language death}.” In \isi{NEA} there are examples of all three kinds, but the details of \citegen[60]{Thomason2001} typology are too complex to be repeated here in full. Language shift, which today is an extremely common phenomenon around the globe and in \isi{NEA}, will for the most part be excluded for the lack of relevant data concerning the effects on the grammars of \isi{questions}. Examples for extreme cases of \isi{language contact} found in \isi{NEA} include two extinct \isi{pidgins} (\ili{Chinese Pidgin Russian}, \ili{Govorka}, \sectref{sec:5.5}), some creolized languages (e.g., \ili{Gangou}, \ili{Wutun}, \ili{Tangwang}, Hezhou, \sectref{sec:5.9}), several \isi{mixed languages} (\ili{Eynu}, \sectref{sec:5.11}, Mednyj \ili{Aleut}, \sectref{sec:5.4}, some \ili{Tungusic} languages \sectref{sec:5.10}, \sectref{sec:6.3}, and an \ili{Ainu-Itelmen hybrid}), and perhaps some slightly less extreme cases such as \ili{Mandarin} or \ili{Manchu} (e.g., \citealt{McWhorter2007}).

Contact-induced change has several subtypes (\textit{relabeling}, \textit{calquing} etc.), but arguably, regarding the \isi{grammar of questions}, the most important case is \textit{borrowing}, simply put the transfer of a certain element from one language to another. But how do we actually know that a linguistic item in a given language can be explained by \isi{language contact} rather than genetic inheritance? Let me illustrate this with an example from the \ili{Tungusic} language \ili{Uilta} (\sectref{sec:5.10.2}). \ili{Uilta} has a \isi{content question} marker \textit{=ga} {\textasciitilde} \textit{=ka}. A comparison with closely related languages such as \ili{Nanai} shows that the marker is not present, in fact, \isi{content question}s in \ili{Nanai} remain unmarked. The fact that related languages do not show this marker in most cases rules out an explanation in terms of genetic inheritance. Then the form might simply be an innovation found in \ili{Uilta}, but no plausible etymology is known to me. Thus, \ili{Uilta} perhaps borrowed the \isi{question marker} from a surrounding language. \ili{Uilta} is spoken on \isi{Sakhalin} where it is known to have had \isi{contact} with the neighboring language \ili{Nivkh} (\citealt{Yamada2010}, \sectref{sec:5.2.2}). In fact, \ili{Nivkh} has overt \isi{content question} markers, one of which has the form \textit{=ŋa}. Of course, \ili{Uilta} could also have borrowed the \isi{question marker} from other surrounding language families such as \ili{Japonic}. \ili{Tsuken}, for example, has a content \isi{question marker} \textit{=ga}. However, \ili{Tsuken} is spoken in the \isi{Ryūkyūan Islands} several thousand miles south of \isi{Sakhalin}. This geographical distance makes a connection extremely implausible, because the speakers of \ili{Uilta} and \ili{Tsuken} quite certainly had no direct \isi{contact} with each other. But what about \ili{Japanese}, which was once spoken on \isi{Sakhalin} and has a \isi{question marker} \textit{ka} \jp{か} that can also be found in content \isi{questions}? First of all, \ili{Uilta} had much more longstanding and intimate \isi{contact} with \ili{Nivkh} than with \ili{Japanese}. However, in order to refute this possibility, more information on \ili{Japanese} and \ili{Uilta} is in order. \ili{Old Japanese} already possessed the \isi{question marker}, which had more or less the same form, but in \ili{Uilta} there are further forms such as \textit{=ge} (alternatively written with a schwa \textit{ə} and an optional long vowel). Given that \ili{Uilta} has \isi{vowel harmony} in which \textit{a} stands opposed to \textit{e} \citep[3]{Tsumagari2009b}, this appears to be an innovation and the form might still derive from either \ili{Nivkh} or \ili{Japanese}. However, the integration of the \isi{question marker} into the morphological system suggests a relatively early \isi{borrowing}, which makes a comparison with \ili{Japanese} much less likely. Furthermore, \citet[50ff.]{Nakanome1928} mentions a form that was written as <ṅö>. The pronunciation of this form must be [ŋə], as a comparison of \citet{Nakanome1928} with \citegen{Ikegami1997} modern dictionary suggests, e.g. <önnö> = [ənnə] ‘mother’, <ṅâla> = [ŋaala] ‘hand’. The existence of the \isi{velar nasal} makes a comparison with \ili{Nivkh} much more likely than with \ili{Japanese}. The fact that both \ili{Nivkh} and \ili{Uilta}, but not the surrounding languages, overtly mark polar and \isi{content question}s differently---i.e. there is a \isi{similarity} in type---confirms this hypothesis (e.g., \citealt{Hölzl2015b}). This typological parallel has also recently been observed by \citet[59f.]{Pevnov2016}. On the contrary, \ili{Japanese} allows the marker \textit{ka} in both polar and \isi{content question}s. For reasons of space, this procedure will not be given in full detail for every potential instance of \isi{borrowing} identified in Chapter 5. A list of all the borrowed elements of the grammars of questions in \isi{NEA} found throughout this study is given in Chapter 6. In several cases the details will have to be discussed by experts of the individual languages.

One of the central concepts of \isi{areal linguistics} is the heavily disputed notion of a \textit{\isi{linguistic area}} or \textit{\isi{sprachbund}}. The best summary of previous approaches can be found in \citet[18]{Campbell2006}, whose rather skeptical conclusion is the following.

\begin{quote}
Every ‘\isi{linguistic area}’, to the extent that the notion has any meaning at all, arises from an accumulation of individual cases of ‘localized diffusion’; it is the investigation of these specific instances of diffusion, and not the pursuit of defining properties for linguistic areas, that will increase our understanding and will explain historical facts.
\end{quote}

\noindent There is a strange dissonance between theoretical approaches that usually take a negative stance on the concept (e.g., \citealt{Dahl2001}; \citealt{Bisang2010}) and the widespread use of the term for individual areas such as the \isi{Amdo Sprachbund}. This study acknowledges the fundamental theoretical problems of the concept, but takes a pragmatic approach. The term \textit{linguistic area} is taken as a useful label if it is not meant to indicate clear-cut boundaries or absolute homogeneity. Like many linguistic phenomena, linguistic convergence is obviously a \isi{matter of degree} (cf. \citealt{Langacker2008}: 13) and there is no problem in calling areas of strong convergence a \textit{sprachbund} or \textit{linguistic area}. As a rule of thumb, an area should be characterized with the help of features that are not very common cross-linguistically, and that are not shared with surrounding areas. \isi{NEA} is surrounded by several possible areas such as the \textit{\isi{Greater Himalayan Region}} to the south \citep{Kraaijenbrink2009} and the \textit{\isi{Pamir-Hindukush Sprachbund}} (\citealt[82]{Novák2014}) as well as the \textit{\isi{Araxes-Iran Linguistic Area}} to the southwest \citep{Stilo2015}, and \textit{\isi{Mainland Southeast Asia}} (\citealt{EnfieldComrie2015}) to the southeast. Unfortunately, with only the exception of MSEA (\citealt{EnfieldComrie2015}), the definition of all of these areas is quite problematic. Nevertheless, the fact that the entire southern and southeastern boundary is marked by mountains teeming with \isi{linguistic diversity} indicates that they form an \textit{accretion} or \textit{\isi{residual zone}} (\citealt{Nichols1992,Nichols1997,Nichols2015}, see \sectref{sec:3.4}) that functions as some kind of boundary. The most difficult problem is the identification of a western boundary (\citealt[873]{HeggartyRenfrew2014b}). Immediately to the west of \isi{NEA} live the speakers of the \ili{Uralic}, more precisely \ili{Finno-Ugric}, languages \ili{Khanty} and \ili{Mansi} that are sometimes collectively called \ili{Ob-Ugric}. Their genetic classification is disputed, with some arguing that they belong to a single branch and others for a classification into two different branches that had strong mutual contacts, called \ili{Khantic} and \ili{Mansic} by \citet[65]{Janhunen2009}. It is difficult to consider these two languages as forming a useful western boundary. But the \isi{Western Siberian Lowland} together with Kazakhstan to its south is a region of low \isi{linguistic diversity} (a \isi{spread zone}, \citealt{Nichols1992}), which contrasts with the adjacent areas of \isi{NEA} along the \isi{Yenisei}. Located to the west of the \isi{Ural} mountains, and thus separated from \isi{NEA} by the \isi{Western Siberian Lowland}, lies the \textit{\isi{Volga-Kama} Area} (see \citealt{Manzelli2015}). This is an area of strong linguistic convergence between several \ili{Finno-Ugric} and \ili{Turkic} languages (see \sectref{sec:5.11}, \sectref{sec:5.12}). If one was to extend \isi{NEA} to include all of the area to the east of the \isi{Ural} mountains into, say \textit{Northern Asia} (\citealt{Nichols1992}: 25f.), the \isi{Volga-Kama} area would certainly function as a better western boundary than does Ob-Ugric. Nevertheless, several languages with affinities to \isi{NEA}, notably \ili{Finnish} or \ili{Turkish} would still be located to the west of the \isi{Volga-Kama} Area.

For practical purposes, Eurasia will be treated as a macro-area (\sectref{sec:3.2}) that contrasts relatively sharply with \isi{Mainland Southeast Asia} (\sectref{sec:3.3}) and contains a meso-area called \isi{Northeast Asia} (\sectref{sec:3.4}). \isi{NEA} in turn encompasses several possible micro-areas such as the so-called \textit{\isi{Amdo Sprachbund}} (\sectref{sec:3.5}) only some of which will be mentioned in this chapter.

\section{The Eurasian macro-area}\label{sec:3.2}

As is well-known, \isi{NEA} is part of a large Eurasian area that is characterized by several dominant features (\tabref{tab:3:1}). This area includes most of \isi{Eurasia} but not \isi{Mainland Southeast Asia} (MSEA), parts of \isi{Europe} and parts of the Near East. There is some variation in the geographical distribution of these features, but notably \isi{NEA} invariably shares all of them. The variation concerns the periphery of the Eurasian area such as \isi{Europe}.

\begin{table}
\caption{Some areally dominant features of \isi{NEA} shared with neighboring areas}
\label{tab:3:1}

\begin{tabularx}{\textwidth}{>{\raggedright}p{35mm}Qp{2cm}}
\lsptoprule

\textbf{Feature} & \textbf{Approximate area} & \textbf{Source}\\
\midrule
SOV (SV \& OV) & Eurasia, excluding MSEA and \isi{Europe} & \citealt{Dryer2013c}\\
\tablevspace
noun-adposition\newline (NAdp) & Eurasia, excluding MSEA, most of \isi{Europe}, and parts of the Near East & \citealt{Dryer2013d}\\
\tablevspace
genitive-noun\newline (GenN) & Eurasia, excluding southern MSEA, western \isi{Europe}, and the Near East & \citealt{Dryer2013e}\\
\tablevspace
adjective-noun\newline (AdjN) & Eurasia, excluding MSEA, western \isi{Europe}, and the Near East & \citealt{Dryer2013f}\\
\tablevspace
demonstrative-noun\newline (DemN) & Eurasia, excluding MSEA and northwestern \isi{Europe} & \citealt{Dryer2013g}\\
\tablevspace
numeral-noun\newline (NumN) & Eurasia, excluding parts of MSEA & \citealt{Dryer2013h}\\
\tablevspace
relative clause-noun\newline (RelN) & \isi{Eurasia}, excluding parts of MSEA, parts of South \isi{Asia}, \isi{Europe}, and the Near East & \citealt{Dryer2013i}\\
\tablevspace
degree word-adjective\newline (DAdj) & Eurasia, excluding parts of MSEA & \citealt{Dryer2013j}\\
\tablevspace
non-initial \isi{interrogative} & Eurasia, excluding \isi{Europe} & \citealt{Dryer2013l}\\
\tablevspace
morphological \isi{case} marking & Eurasia, excluding western \isi{Europe} and MSEA & \citealt{Iggesen2013}\\
\tablevspace
\isit{suffixing} in inflectional \isi{morphology} & Eurasia, excluding MSEA & \citealt{Dryer2013b}\\
\tablevspace
\isi{m-T-pronouns} present & large parts of Eurasia, excluding MSEA and its adjacent areas to the northeast and west & \citealt{NicholsPeterson2013}\\
\lspbottomrule
\end{tabularx}
\end{table}

The usefulness of some of these \isi{word order} features for the identification of linguistic convergence is somewhat reduced by the existence of implicational hierarchies connecting several of them (e.g., \citealt{Bisang2010}: 422; \citealt{Dryer2013a}). Nevertheless, they define a relatively clear-cut boundary towards the southeast. A possible further trait of this Eurasian Area is the existence of \isi{K-interrogatives} (\sectref{sec:6.2.1}).

\section{Mainland Southeast Asia}\label{sec:3.3}

The sharpest contrast of \isi{NEA} with other areas is that with \isi{Mainland Southeast Asia} (MSEA), the adjacent region to the southeast, which has recently been defined as

\begin{quote}
the area occupied by present day Cambodia, Laos, Peninsular Malaysia, Thailand, Myanmar, and Vietnam, along with areas of \isi{China} south of the \isi{Yangtze} River. Also sometimes included are the seven states of Northeast India, and—although here the term ‘mainland’ no longer applies—the islands from Indonesia and Malaysia running southeast to Australia and West Papua (\citealt[1]{EnfieldComrie2015})
\end{quote}

\noindent MSEA is widely accepted as a region of strong convergence of five different language families, namely \ili{Trans-Himalayan} (\ili{Sino-Tibetan}), \ili{Tai-Kadai}, \ili{Hmong-Mien} (Miao-Yao), \ili{Austroasiatic}, and \ili{Austronesian}. My definition of \isi{NEA} excludes the \isi{Yangtze} watershed, a part of which is likely the historical \isi{homeland} of the Hmong-Mien languages \citep[241]{Ratliff2010} that clearly belong to the MSEA area. Moreover, \ili{Sinitic} languages show an internal split between northern and southern varieties (e.g., \citealt{Ramsey1987}: 19-26; \citealt{Matthews2010}: 760f.). In a certain sense, the distinction between \ili{Mandarin} and Southern \ili{Sinitic} varieties is symptomatic for the difference between \isi{NEA} and MSEA. \ili{Mandarin} is rather homogeneous and is spread over a vast area ranging from \isi{Yunnan} in the Southwest to \isi{Heilongjiang} in the Northeast and from Jiangsu in the east to \isi{Xinjiang} in the west. Southern \ili{Sinitic}, on the other hand, is limited to a much smaller geographical area but nevertheless shows extremely strong internal variation with many mutually unintelligible varieties \citep{Kurpaska2010}.

\begin{quote}
There is a qualitative difference between these two areas. The \ili{Mandarin} area, on the one hand, is unusually uniform; virtually all of the dialects spoken there are mutually intelligible---or very nearly so. [...] But the non-\ili{Mandarin} area is extremely varied, and within it sharply divergent forms of speech are often separated by only a few miles. \citep[21]{Ramsey1987}
\end{quote}

\noindent But \ili{Mandarin} also differs from most of \ili{Sinitic} in structure. Southern \ili{Sinitic} exhibits stronger affinities to Southeast Asian languages than does \ili{Mandarin}, which has been more strongly influenced by languages in \isi{NEA}. There is a debate as to whether the special structure of \ili{Mandarin} can be explained by “\isi{Altaicization}”, i.e. influence from \ili{Turkic}, Khitano-\ili{Mongolic}, and \ili{Tungusic} \citep{Hashimoto1986}, or \isi{reduction} due to non-native \isi{acquisition} of speakers of languages in nowadays northern \isi{China} (\citealt{McWhorter2007}: 104–137). But in any case, this can be labeled an areal feature that separates \ili{Mandarin} from the rest of \ili{Sinitic}. Following an extensive discussion, \citet[429]{deSousa2015} concludes the following:

\begin{quote}
Some studies on the MSEA \isi{linguistic area} leave out the languages in \isi{China}. This is unwise, as the centres of diversity for the Kra-Dai and Hmong-Mien families are still in Southern \isi{China}, and the Southern \ili{Sinitic} languages also have many MSEA linguistic traits. Studies of the MSEA \isi{linguistic area} would benefit immensely if the Southern \ili{Sinitic} languages, the Far-Southern \ili{Sinitic} languages in particular, are included in the MSEA \isi{linguistic area}.
\end{quote}

\noindent Within the \isi{human genome}, too, there is a marked difference between Northern and Southern Han populations, the dividing line of which roughly coincides with the \isi{Yangtze} river (e.g., \citealt{Zhao2014} and references therein). As is well-known, there is also a stereotypical division into North and South as perceived by the \ili{Chinese} themselves that at least in part has a basis in actual facts such as the predominant cultivation of \isi{wheat} and \isi{rice}, respectively (e.g., \citealt{Eberhard1965}: 601f.). My approach thus stands opposed to \citet[870]{HeggartyRenfrew2014b}, who classify the linguistic landscape of \textit{East Asia} around a “\ili{Chinese} core” into a northern, a \ili{Sinitic}, and a southern zone. Of course, all \ili{Sinitic} languages share certain inherited properties. Perhaps, \ili{Sinitic} and especially \ili{Mandarin} may thus be better conceptualized as a transitional zone between MSEA and \isi{NEA} (\citealt{Dryer2003}: 48ff.; \citealt{Comrie2008}). However, in stark contrast to \isi{Northeast Asia}, \isi{Mainland Southeast Asia} (MSEA) generally has the following \isi{word order} features: SVO (SV \& VO), AdpN, NGen, NAdj, NDem, NNum, NRel, AdjD (\citealt{deSousa2015}: 366). Languages in MSEA usually lack inflectional \isi{morphology} and have no sign of \isi{m-T-pronouns}. Of the features listed in \tabref{tab:3:1}, MSEA only shares the non-initial interrogatives. However, for this southeastern neighbor a much longer list of distinguishing linguistic features, such as the lack of a voiced [g] or the existence of complex tone systems, has been summarized by \cite[7f.]{EnfieldComrie2015}. At least for some of them there is no clear-cut boundary to neighboring areas. For instance, \ili{Mandarin}, \ili{Manchu}, and \ili{Japanese} share a similar syllable structure with only very few possible final consonants. In \ili{Manchu} the only exceptions are ideophones, which is yet another feature that is not unique to MSEA but shared with many languages in \isi{NEA} as well.

\section{Northeast Asia}\label{sec:3.4}

In terms of \isi{language diversity} and \isi{phylogenetic diversity}, MSEA and \isi{NEA} show strikingly different patterns as well (\tabref{tab:3:2}). \isi{Southeast Asia} is home to only five language families, but in its broadest definition encompasses almost 600 languages. During the preparation of this study it became increasingly clear that an exact number of languages cannot possibly be given for \isi{NEA}. There is a constant fluctuation of languages spoken by tourists, exchange students, foreign workers, etc. But even if one leaves aside this problem, it is by no means clear at what point a dialect should be counted a language or at what point a language should be considered extinct. For instance, the northern \ili{Tungusic} languages \ili{Evenki}, \ili{Even}, \ili{Negidal}, \ili{Oroqen}, and \ili{Solon}, each of which has strong internal dialectal variation, as well as the extinct language \ili{Arman} form a complex net of dialect continua. If one agrees with the traditional point of view and considers \ili{Arman} a dialect of \ili{Even}, the language as such never went extinct (cf. \citealt{DoerferKnüppel2013}). \ili{Evenki} alone has about 50 different dialects and experts disagree on whether \ili{Oroqen} dialects should be included in the list or not (e.g., \citealt{WhaleyLi2000}; \citealt{Janhunen2012b}: 7). Given the rapid shift of speakers of both \ili{Evenki} and \ili{Oroqen} to \ili{Russian} and \ili{Chinese}, respectively, it is often only the older generation that can speak the languages. In some cases no fluent speaker is left, but some relics of the language nevertheless remain in the form of individual expressions or passive speakers. Clear-cut distinctions in these cases are neither feasible nor desirable (cf. \citealt{Langacker2008}: 13). Leaving aside this fluctuation, most of the dialects, and clearly extinct languages, \isi{NEA} may be estimated to be home to between 120 and 150 languages. However, \isi{NEA} shows much more diversity in the number of language families than does MSEA.

\begin{table}
\caption{Comparison of language and phylogenetic diversity in MSEA (\citealt[6]{EnfieldComrie2015}) and \isi{NEA}, excluding historically attested languages (this study)}
\label{tab:3:2}

\begin{tabularx}{\textwidth}{QSSS}
\lsptoprule
& \textbf{Core MSEA} & \textbf{Greater MSEA} & \textbf{\isi{NEA}}\\
\midrule
language families & 5 & 5 & 14\\
languages & 280 & 583 & ca. 120-150\\
\lspbottomrule
\end{tabularx}
\end{table}

Of course, (Greater) \isi{Mainland Southeast Asia} actually encompasses more than five language families if one includes all small language families (or “isolates”) such as (extinct) \ili{Kenaboi}, \ili{Shom Peng} (perhaps \ili{Austroasiatic}), (extinct) \ili{Great Andamanese}, or \ili{Ongan} (Jarawa-Onge) (e.g., \citealt{Hammarström2016}). The \isi{phylogenetic diversity} of \isi{NEA} is also much higher than that of the entire landmass to the west. Excluding extinct languages such as Etruscan and the relatively recent migrations from other parts of the world, there are only representatives of five language families in \isi{Europe} today, namely \ili{Indo-European}, \ili{Uralic}, \ili{Basque}, \ili{Afroasiatic} (\ili{Maltese}), and \ili{Turkic}. The Caucasus alone adds three more families, but even so, \isi{NEA} still exhibits much more \isi{phylogenetic diversity}. \citet[137]{Anderson2010} goes so far as to call the eastern part of \isi{NEA}, where representatives of 12 of the 14 language families are spoken, a language hot spot with a “high level of unique \isi{phylogenetic} \isi{linguistic diversity} endemic to the region”. Of course, if a macro-family such as \ili{Transeurasian} (\citealt{Robbeets2015}), allegedly including five different language families, was to be proven, the \isi{phylogenetic diversity} of \isi{NEA} would be lower but still higher than in MSEA, not to mention that there are attempts to lump together language families in MSEA as well (e.g., \citealt{Sagart2016}). But \isi{linguistic diversity} in \isi{NEA} and around the globe is in retreat as many speakers are shifting to larger languages. Not only the number of languages is fading (decrease in \isi{language diversity}), but whole families such as \ili{Ainuic}, \ili{Amuric}, \ili{Tungusic}, \ili{Yeniseic}, \ili{Yukaghiric}, and perhaps \ili{Chukotko-Kamchatkan} as well as \ili{Samoyedic} will probably not survive this or the next century (decrease in \isi{phylogenetic diversity}). \ili{Eskaleut}, which will persist in other parts of the world, could disappear from \isi{NEA} as well. In other words, \isi{NEA} could be the home of languages from only six families in future times, although globalization will bring many more languages from around the world into this area as well.

A good overview of some areal traits found throughout \isi{Northeast Asia} and adjacent areas has recently been given by \citet[366]{Nichols2010}:

\begin{quote}
Interior \isi{Asia} has been a center of \isi{language spread} at least since the Neolithic. The linguistic evidence points to strong and long-term areality in the epicenter of spread, with innovations made in the center eventually showing up farther away. To judge from its distribution, the \textit{m-T} pronoun type may have spread early and then developed its strong structural parallelism in later innovations in the center; \isi{case}–number coexponence is found at the far peripheries of the area (besides \ili{Uralic} and \ili{Indo-European} it also occurs in \ili{Chukchi} and \ili{West Greenlandic}), but for at least the last few millennia the classic agglutinating type (with monoexponential and transparently segmentable suffixes) has predominated in the epicenter. Phonemic \isi{front rounded vowels} may have spread from the epicenter more recently. The consistently head-final morphosyntax of \ili{Uralic}, core \isi{Altaic}, \ili{Japanese}, etc. is more generally widespread in Eurasia and not specific to this northeastern area.
\end{quote}

\noindent In fact, perhaps one of the strongest features of \isi{NEA} are the \isi{front rounded vowels} \textit{ü} and \textit{ö}. A previous study by \citet{Maddieson2013} has shown that these are, by and large, restricted to Eurasia, but it seems that this is a relatively late expansion out of \isi{NEA} where the highest concentration of languages with these vowels can be found (\tabref{tab:3:3}). In many cases, the available descriptions are not extremely specific about the exact nature of the vowels, i.e. whether they are exactly [y] and [ø] or slightly different sounds.

\begin{table}
\caption{Front rounded vowels in Northeast Asia in comparison with \citegen{Maddieson2013} global sample; see \sectref{sec:6.4} and the Appendix for the data}
\label{tab:3:3}

\begin{tabularx}{\textwidth}{QSS}
\lsptoprule

\textbf{Type} & \textbf{\isi{NEA}} & \textbf{Global}\\
\midrule
not present (or unknown) & 47 & 525\\
high and mid & 24 & 23\\
high only & 8 & 8\\
mid only & 4 & 6\\
\midrule
Total & 83 & 562\\
\lspbottomrule
\end{tabularx}
\end{table}

The comparison of the two different samples, global and Northeast Asian, is quite revealing. While altogether 36 out of 83 languages in \isi{NEA} have at least one kind of front rounded vowel (about 43\%), \citet{Maddieson2013} found only 37 out of a sample of 562 languages (about 7\%). There are almost no languages of this type along the \isi{Pacific Rim}, i.e. in Pacific \isi{NEA}. In fact, excluding the far Northeast (\ili{Eskaleut} and \ili{Chukotko-Kamchatkan}) as well as \isi{Japan} (\ili{Japonic} and \ili{Ainuic}) results in an even larger number of 60\% (36 out of 60 languages). In \isi{NEA} all languages with \isi{front rounded vowels} are from seven language families, namely \ili{Koreanic}, Khitano-\ili{Mongolic}, \ili{Trans-Himalayan} (especially \ili{Sinitic}), \ili{Tungusic}, \ili{Turkic}, \ili{Uralic}, and \ili{Yukaghiric}. They were historically lost in many \ili{Mongolic} and especially \ili{Tungusic} languages, in the latter case possibly because of \isi{contact} with languages along the \isi{Pacific Rim} such as \ili{Amuric}. \citet{Maddieson2013} mentions only a few languages outside of \isi{NEA} with \isi{front rounded vowels}. Of these, four in the Americas, three in the Pacific region and one in Africa are of no concern for us here. But there are several languages in Eurasia, more exactly, six to the adjacent south and thirteen to the west of \isi{NEA} that also share the phenomenon. Interestingly, the languages in the west include many that have an origin further to the east or within \isi{NEA} (\ili{Hungarian}, \ili{Finnish}, \ili{Mari}, \ili{Turkish}, \ili{Azeri}, \ili{Bashkir}, \ili{Chuvash}). \tabref{tab:3:4} summarizes whether \isi{front rounded vowels} can be reconstructed to the fourteen \isi{proto-languages} of languages that are today located in \isi{NEA}. There will be no comment on the accuracy of the reconstructions and on the details of later developments here, which goes beyond the possibilities of this study. But it may be noted that \citegen{Vovin1993} \isi{reconstruction} of \ili{Proto-Ainuic} in this case is highly doubtful.

\begin{table}
\caption{Reconstructed front rounded vowels (FRV) for languages spoken in \isi{NEA}}
\label{tab:3:4}

\fittable{
\begin{tabular}{lccll}
\lsptoprule

\textbf{Language} & \textbf{High} & \textbf{Mid} & \textbf{Source} & \textbf{Comment}\\
\midrule
Proto-\ilit{Amuric} & - & - & \citealt{Fortescue2011}: 4 & \\
Proto-CK & - & - & \citealt{Fortescue2005}: 6 & \\
\ilit{Proto-Eskaleut} & - & - & \citealt{Fortescue1998}: 125 & \\
Proto-IE & - & - & \citealt{Fortson2010}: 66 & \\
\ilit{Proto-Japonic} & - & - & \citealt{Pellard2008}: 136 & \\
Old \ilit{Korean} & - & - & \citealt{Whitman2012}: 28 & \ilit{Korean} has both\\
\ilit{Old Chinese} & - & - & \citealt{BaxterSagart2014a}: 195 & \ilit{Mandarin} has high FRV\\
\ilit{Ket} & - & - & \citealt{Georg2007}: 61 & \\
Proto-\ilit{Ainuic} & ?+ & ?+ & \citealt{Vovin1993}: 42ff. & ?verbal stems only\\
\ilit{Proto-Uralic} & + & - & \citealt{Sammallahti1988}: 481 & \\
\ilit{Proto-Samoyedic} & + & + & \citealt{Janhunen1998}: 463 & \\
Proto-\ilit{Mongolic} & + & + & \citealt{Janhunen2003a}: 4 & later both mostly lost\\
\ilit{Khitan} & + & - & \citealt{Miyake2017}: 493 & \\
\ilit{Proto-Tungusic} & + & + & \citealt{Doerfer1978a}; 1978b & later both mostly lost\\
\ilit{Proto-Turkic} & + & + & \citealt{Róna-Tas1998}: 70 & \\
\ilit{Proto-Yukaghiric} & + & + & \citealt{Nikolaeva2006}: 57 & high FRV later lost\\
\lspbottomrule
\end{tabular}
}
\end{table}

I currently lack exact reconstructions for \ili{Koreanic}, \ili{Trans-Himalayan} and \ili{Yeniseic}, which is why Old \ili{Korean}, \ili{Old Chinese} and \ili{Ket} have been listed instead. The presence of \isi{front rounded vowels} in \ili{Yukaghiric} corroborates the hypothesis that this \isi{language family} historically derives from a location further to the south (\sectref{sec:2.14}). Similarly, \ili{Uralic} likely derives from a location close to or perhaps even in \isi{NEA} (\sectref{sec:2.12}). Front rounded vowels in this part of the world seem to be a “\isi{Ural-Altaic}” phenomenon (including \ili{Yukaghiric} but excluding \ili{Japonic} and \ili{Koreanic}). However, the origin of the \isi{similarity} does not necessarily lie in a common origin but may well be the result of prehistoric \isi{language contact} in southern \isi{NEA}. It seems that the historical center of the phenomenon clustered around Lake \isi{Baikal}. Perhaps, its emergence is connected to the phenomenon of vowel assimilation, i.e. \isi{vowel harmony} (e.g., \citealt{Maddieson2013}: Chapter Text). The history of \ili{German} shows that vowel assimilation (in this case umlaut) can most likely be responsible for the emergence of \isi{front rounded vowels}. The list of \isi{proto-languages} with \isi{front rounded vowels} roughly corresponds to the list of \isi{proto-languages} with KIN-interrogatives. Exceptions include \ili{Tungusic} (without \isi{KIN-interrogative}), and \ili{Eskaleut} (without \isi{front rounded vowels}). The status of both the \isi{interrogative} (\textit{hunna} ‘who’) as well as the vowels in \ili{Proto-Ainuic} is questionable.

A well-known concept of \isi{areal linguistics} is that of \textit{spread} versus \textit{residual zones} (e.g., \citealt{Nichols1992}: 13–24; \citeyear{Nichols1997}; \citeyear{Nichols2015}; \citealt{Dahl2001}: 1460f.; \citealt{Bisang2010}: 431f.). Large parts of \isi{NEA}, especially in the steppes towards the west and along the \isi{Lena} qualify as spread zones (\citealt{Nichols1992}: 13–24). In fact, the Eurasian steppe was her prime example. Spread zones are areas with low \isi{phylogenetic diversity}, low \isi{structural diversity}, and also low \isi{language diversity} per \isi{language family}. There is also no accumulation of diversity over time. In spread zones there is rapid expansion of languages over vast areas that subsequently serve as \textit{lingua francas} for and often replace languages previously spoken in that area.

\begin{quote}
Each language or dialect group spreading westward on the steppe probably took the form of a classic dialect-geographical area, with a center of innovation (in its eastern range, at least initially) and archaisms on the periphery. Certainly there were centers of political, economic, and cultural influence \citep[16]{Nichols1992}
\end{quote}

\largerpage
\noindent One prime example of \isi{language spread} is the expansion of \ili{Sinitic} from around the \isi{Yellow River} southwards towards MSEA, an event influenced by state building, complex social structures, and warfare. Beginning in the 18th century, \ili{Mandarin}, again starting from about the same area, expanded towards the regions around core \isi{China}, i.e. \isi{Manchuria}, Inner \isi{Mongolia}, \isi{Xinjiang}, Tibet, \isi{Qinghai}, and the Southwest. \ili{Mandarin} is not only used as a main language of communication in all of \isi{China} and is rapidly replacing many minority languages, but is currently also influencing or even replacing several \ili{Sinitic} varieties in the South that are the result of the earlier spread. The history of the southern parts of \isi{NEA} over thousands of years is strongly based on the emergence and spread of multicultural and multilingual confederations ranging from even before the ancient \isi{Xiongnu} (ca. 3rd century BCE to 4th century CE) to the Manchus from the 17th century onward. The moving factor behind the spread of languages and language families can often be found in cultural or technological innovations, the domestication of different plants and animals, etc. In the case of \ili{Indo-European} (except Anatolian), for instance, this possibly was the use of the wheel and wagon (\citealt{AnthonyRinge2015}). \isi{NEA} has seen a variety of spreads of languages or language families over large distances, but to my knowledge, in most cases they have not been clearly linked with such innovations yet. We do not know, for example, which language group was connected with the original domestication of the \isi{reindeer} in \isi{NEA} several thousand years ago, which happened independently of the domestication in northern \isi{Europe} (\citealt{Røed2008}). But we know that the expansion of some northern \ili{Tungusic} languages, some \ili{Samoyedic} languages, \ili{Yukaghiric}, and \ili{Chukchi} were likely connected with this innovation (e.g., \citealt{Janhunen1996}: 61ff.; \citealt{Helimski1998}: 480; \citealt{Anderson2006c,Anderson2006e}). Further to the south, the domestication of the horse about 7000 years ago was crucial for the steppe cultures, connected with several language families including \ili{Indo-European}, \ili{Turkic}, or \ili{Mongolic} \citep{Anthony2007}. The yak played a comparable role for the high altitude regions in the southern periphery of \isi{NEA} around the \ili{Tibetan} highland, but reaching as far north as the \isi{Altai} \citep{Wiener2013}. The domestication of the dog may have a relatively long history as compared to that of the other animals mentioned. A recent study found evidence “that sled dogs could have been used in \isi{Siberia} around 15,000 years ago” (\citealt{PitulkoKasparov2017}: 491). In \isi{NEA} dog sleds were used, for instance, by the \ili{Nivkh} and some surrounding \ili{Tungusic} populations, but also by Samoyeds, Yukaghirs etc. However, the spread of languages is not necessarily based on the spread of its speech community by means of growth and \isi{migration}. Another important mechanism of \isi{language spread} is \textit{language shift}, i.e. the shift of a given speech community from one language to another (e.g., \citealt{Nichols1997}: 372; \citealt{Janhunen2007a}: 74). Most cases are a \isi{combination} of these factors.

Spread zones are opposed to \textit{residual} or \textit{accretion zones} (\citealt{Nichols1997}: 369f.), which \cite[13–15]{Nichols1992} illustrated with the help of the Caucasus. These are areas that have greater \isi{phylogenetic}, language, and \isi{structural diversity}. Language families tend to be older (i.e., the age of the respective proto-language lies further in the past) and there are fewer movements of peoples and languages than in spread zones. “As in mountain areas, innovations arise in the periphery (in the lowlands) and archaisms are found in the interior (in the highlands).” \citep[14]{Nichols1992} Residual zones are areas of retreat rather than spread, usually do not show a single \textit{lingua franca} over the entire area, and have an increase of diversity. There are several possible residual zones in, or rather around, \isi{NEA}, including most of Pacific \isi{NEA} (e.g., \isi{Ryūkyūan Islands}, \isi{Hokkaid\=o}, \isi{Sakhalin}, \isi{Kuril} Islands, \isi{Kamchatka}, Aleut Islands), the lower \isi{Amur}, and many mountain ranges and high altitude regions (e.g., \isi{Yunnan}, \isi{Amdo}, the \isi{Tibetan Plateau}, the \isi{Himalayas}, the \isi{Pamir}, the \isi{Altai}). It should be borne in mind that the features of spread and residual zones mentioned above do not all apply in every case but represent valid tendencies.


Both \citet{Anderson2006a} and \citet{Pakendorf2010} grant Northern \ili{Tungusic} (more precisely \ili{Ewenic}) languages a special position for the Siberian area

\begin{quote}
The features of the Siberian linguistic macro-area cluster around those of the Northern \ili{Tungusic} languages and this is not by accident. Indeed, the highly mobile \ili{Evenki} (and to a lesser degree its sister language, \ili{Even}) both have the local \isi{bilingualism} relationships and widespread distribution necessary to make them likely vectors of diffusion for at least some of these features \citep[294]{Anderson2006a}
\end{quote}

\noindent Expanding on this proposal, one might argue that it is not only the \ili{Ewenic} branch, but all of \ili{Tungusic} that used to have a rather special position for \isi{NEA}. Today, \ili{Tungusic} languages are mostly endangered, moribund or already extinct (\citealt{Janhunen2005}; \citealt{TsumagariKurebitoEndo2007}), but one should not underestimate their historical influence over all of \isi{NEA}. Most \ili{Tungusic} languages are still located in \isi{Manchuria} where they had a certain amount of impact on \ili{Mongolic}, \ili{Amuric}, \ili{Ainuic}, and \ili{Koreanic}. While \ili{Evenki} and \ili{Even} expanded into northern \isi{NEA} and reached places as far apart as \isi{Kamchatka} and the Taimyr Peninsula, Jurchen and \ili{Manchu} played an important role for the southern half. The \isi{Jurchen} established the Jin-dynasty (1115-1234) in northern \isi{China} and the Manchus had an even more pronounced influence during their Qing-dynasty (1636-1912) that at its height not only included all of modern \isi{China}, but also what is now the \ili{Russian} Primorye region as well as \isi{Mongolia}. Most importantly, \ili{Manchu} played the role of an ad- and substrate language of \ili{Mandarin} in Peking, which later was the basis of Standard \ili{Mandarin}. Of course, there are other language families such as \ili{Khitano-Mongolic} or \ili{Turkic} that had an even stronger impact in large parts of \isi{NEA}.

\section{Subareas in Northeast Asia}\label{sec:3.5}

The following gives a brief overview of areas of linguistic convergence and \isi{contact} found in \isi{NEA}. The areas may overlap strongly with each other, which is not indicated in every case. The discussion limits itself to those areas that seem to be most important for this study.

The territory of \isi{NEA} as defined here is covered by six different countries: \isi{China}, \isi{Russia}, \isi{Mongolia}, \isi{North Korea}, \isi{South Korea}, and \isi{Japan}. Each of these countries has a \textbf{national language} that increasingly influences or even replaces all other languages and dialects within that country. In the case of North and \isi{South Korea} these are almost identical. There are, therefore, five superstrate languages that may be seen as defining special kinds of linguistic areas. The expansion of the national languages proceeds at the expense of languages and dialects alike. But even if a given language is strong enough to resist the complete loss (e.g., \ili{Amdo Tibetan}, \ili{Buryat}, \ili{Tuvan}, \ili{Uyghur}, or \ili{Yakut}), it is usually heavily influenced by the national language. Because the \isi{superstrate} is the same in the entire country, there is thus a general \isi{tendency} for all languages to become more similar to each other.

Apart from the areas of \ili{Mandarin} and \ili{Russian} influence, \textbf{Siberia} is doubtless the largest subarea of \isi{NEA}. There have been many studies on this northern half of \isi{NEA} (e.g., \citealt{Fortescue1998}; \citealt{Anderson2004,Anderson2006a}; \citealt{Skribnik2004}; \citealt{Vajda2009a}; \citealt{Comrie2013}), but its status as a \isi{linguistic area} has not been finally clarified. The best summary of areal features found throughout \isi{Siberia} has been given by \citet{Anderson2006a}: \isi{vowel harmony}, a high back unrounded vowel [ɯ], four \isi{nasals} (\textit{m}, \textit{n}, \textit{ñ}, \textit{ŋ}), an initial \isi{velar nasal} (\textit{ŋ-}), SOV \isi{word order}, morphologically marked reciprocal and desiderative, converbs, case-marked clausal subordination, many cases (especially a prolative), \isi{suffixing} \isi{morphology}, a distinction of dative and allative (but see \citealt{Pakendorf2010}: 715). Quite problematically, many of the features such as SOV \isi{word order} or \isi{suffixing} \isi{morphology} are not specific to \isi{Siberia} (\sectref{sec:3.2}). The \isi{velar nasal} indeed shows a very interesting areal pattern, but fails to define a Siberian area as well \citep{Anderson2013}. The Caucasus as well as parts of \isi{Europe} and South \isi{Asia} tend towards the absence of the \isi{velar nasal} altogether. An extremely large area covering most of central Eurasia has velar \isi{nasals} but not in initial position. Crucially, not only the northern parts of \isi{NEA}, but also MSEA have the \isi{velar nasal} in initial position. Nevertheless, \isi{NEA} has a sharp boundary to the Americas where a majority of languages lack a \isi{velar nasal} altogether. Interestingly, \ili{Mandarin} and \ili{Manchu}, both located in the southern half of \isi{NEA}, historically lost the initial \isi{velar nasal}, perhaps due to \isi{contact} with \ili{Mongolic} or \ili{Turkic}. Despite the addition of other possible features such as the special use of \isi{speech act} verbs (\citealt{Matić2013}), \isi{Siberia} clearly does not qualify as a \isi{linguistic area} comparable with \isi{Mainland Southeast Asia} \citep{Comrie2013}. Furthermore, a treatment of \isi{Siberia} without the inclusion of at least parts of \isi{Manchuria} and \isi{Mongolia} is necessarily incomplete.

Several scholars, notably \cite[261–266]{Comrie1981}, \citet{Anderson2003}, and \citet{Georg2008}, have pointed out the special position of \textbf{Yeniseic} languages in \isi{Siberia}. \citet[8]{Comrie2003} summarized the typological differences as follows.

\begin{quote}
\ili{Ket} alone has phonemic tone, and \ili{Ket} alone has a consistent gender/class system, distinguishing masculine, feminine, and neuter nouns, with assignment to masculine and feminine genders using semantic features that go well beyond a mere male/female distinction. While most neighboring languages have relatively simple, agglutinating morphological structure, \ili{Ket} has a substantially different system, making use, for instance, of internal flexion and discontinuous roots; and while the neighboring languages are at least primarily \isi{suffixing}, \ili{Ket} makes widespread use of both suffixes and prefixes.
\end{quote}

\noindent However, some languages, notably \ili{Middle Mongol}, \ili{Khitan}, \ili{Manchu}, and perhaps even the mysterious language of the \ili{Rouran} \citep{Vovin2004}, had a limited gender or sexus system. Prefixes can, furthermore, also be found in \ili{Ainuic}, \ili{Yukaghiric}, or \ili{Chukotko-Kamchatkan}. Tones are also present in \ili{Japonic}, \ili{Koreanic}, and \ili{Trans-Himalayan}.

\textbf{Altaic} is perhaps one of the most disputed proposals for a \isi{language family} worldwide. Still today, there is complete disagreement over the validity of the family and no conclusion is in sight. \citet{Robbeets2015} restricts the name \textit{Altaic} to \ili{Turkic}, Khitano-\ili{Mongolic}, and \ili{Tungusic} and uses the term \textit{Transeurasian} as a cover term for \isi{Altaic}, \ili{Japonic}, and \ili{Koreanic}. However, given that \isi{Altaic} or \ili{Transeurasian} fails to be accepted by a majority of scholars, it must be considered an \textit{unproven} (but still interesting) hypothesis. This does not mean, however, that it is not possible that some of the languages, say, Khitano-\ili{Mongolic} and \ili{Tungusic} are ultimately related to each other. \citet{Janhunen1996} proposed the name \textit{Khinganic} for the hypothetical \isi{language family} that unites \ili{Tungusic} and Khitano-\ili{Mongolic}. It should also be pointed out that there are more possibilities than genetic relatedness on the one hand and mutual \isi{contact} on the other hand (cf. \citealt{Doerfer1985}). One of several imaginable scenarios is that at least one of the \isi{proto-languages} involved really was a mixed language and thus had no clear-cut affiliation in the first place. Any theory would need to explain the fact that \ili{Turkic}, \ili{Tungusic}, and \ili{Mongolic} share similarities in the pronominal system that are both too similar to be due to chance and at the same time too similar to be of common origin, especially given the absence of a common inherited vocabulary \citep[221]{Janhunen2013}. An explanation that seems to be gaining acceptance sees the observable lexical similarities as \isi{borrowing}, especially from \ili{Turkic} to \ili{Mongolic} and from \ili{Mongolic} to \ili{Tungusic} (e.g., \citealt{Doerfer1985}; \citealt{Schönig2003}). Therefore, one of the most important tasks still is the identification of layers of loanwords in all five language families (e.g., \citealt{Khabtagaeva2017}; p.c. 2018). A bilateral relation of two of the so-called \ili{Transeurasian} languages, namely \ili{Koreanic} and \ili{Japonic}, is being advocated independently of their relation to \ili{Turkic}, \ili{Khitano-Mongolic}, or \ili{Tungusic} \citep{Whitman2012}, but awaits further discussion.
\isi{Altaic} in some parts even today is rather premature in the sense that the internal \isi{reconstruction} of the individual language families should still take priority. For instance, it does not make too much sense to compare Proto-\ili{Mongolic} with \ili{Proto-Tungusic} before the evidence by \ili{Khitan} (e.g., \citealt{Janhunen2012d}), other Para-\ili{Mongolic} languages (e.g., \citealt{Shimunek2017}), the Hüis Tolgoi inscription (e.g., \citealt{Vovin2017}), and less well-known \ili{Tungusic} languages is taken into account. The necessary first step for the Khitano-\ili{Mongolic} side must be the continuing decipherment of \ili{Khitan}, followed by, if possible, an improved \isi{reconstruction} of Proto-Khitano-\ili{Mongolic}. On the \ili{Tungusic} side, evidence from several languages such as \ili{Alchuka}, \ili{Bala}, or \ili{Kyakala} (e.g., \citealt{MuYejun1985,MuYejun1986,MuYejun1987}, \citealt{Hölzl2017c,Hölzl2018a}) as well as the dialects of \ili{Oroqen} (\citealt{WhaleyLi2000}) and \ili{Manchu} (e.g., \citealt{Hölzl2018b}) keep being neglected in most studies.

The term \textbf{Ural-Altaic} was originally a proposal for a \isi{language family} which has long since been abandoned. \citet[78]{Janhunen2007a} revived the term in an areal typological sense as

\begin{quote}
a complex of several language families covering the entire trans-Eurasian belt from Finland and Lapland in northern \isi{Europe} to \isi{Korea} and \isi{Japan} in the Far East. Other regions where \isi{Ural-Altaic} languages are spoken, or have until recently been spoken, include Pannonia, Anatolia, Western and Eastern Turkestan, \isi{Mongolia}, \is{Manchuria}Manchuria, much of \isi{Russia}, and most of \isi{Siberia}. The language families conventionally ‘classified’ as \isi{Ural-Altaic} are: \ili{Uralic}, \ili{Turkic}, \ili{Mongolic}, \ili{Tungusic}, \ili{Koreanic}, and \ili{Japonic}.
\end{quote}

\noindent This is as problematic as the use of \textit{Altaic} as a typological label (\citealt{Janhunen2007a}; \citealt{Janhunen2007b}): “areal typology would study the geographical distribution of such features, rather than the characteristics of individual areas.” \citep[1456]{Dahl2001} In fact, several of the few features mentioned by Janhunen are not characteristic of “\isi{Ural-Altaic}” languages only, but can be found in the larger Eurasian area (\sectref{sec:3.2}). A label such as “\isi{Altaic}” or "\isi{Ural-Altaic}" may not only lead to misinterpretations concerning genetic connections, but also suggests a certain typological homogeneity, which---as Janhunen is clearly aware---is not always the case. In many cases such as the pronominal similarities it is more plausible to add \ili{Yukaghiric} to \isi{Ural-Altaic} rather than \ili{Japonic} or \ili{Koreanic}.

To my knowledge, \cite{Janhunen1996,Janhunen1997} offers the only explicit treatment of languages in \textbf{Manchuria}. But whatever the exact delimitation of \isi{Manchuria}, it certainly does not qualify as a \isi{linguistic area} in any sense because it has not been defined on linguistic grounds at all. There are, however, several areas of strong convergence and \isi{bilingualism} within \isi{Manchuria} such as one around the \ili{Mongolic} language \ili{Dagur} that includes the \ili{Tungusic} languages \ili{Solon}, \ili{Oroqen}, \ili{Manchu}, and \ili{Mongolian} dialects such as \ili{Khorchin}. \citet[72]{Gusev2015b} argues for a \isi{linguistic area} “which includes the dialects of \ili{Negidal}, \isi{Amur} \ili{Tungusic}, \ili{Nivkh} and \ili{Ainu}, and in some respects may be a part of a larger area, that could embrace other varieties, such as \ili{Evenki} and \ili{Even}, Hokkaid\=o \ili{Ainu}, \ili{Japanese} and the languages of \isi{Kamchatka}”, but fails to mention any defining features for this larger area altogether (but see also \citealt{Yamada2010}). For interference between different \ili{Tungusic} languages in \isi{Manchuria} see \sectref{sec:5.10.1} and \sectref{sec:6.3}.

Usually considered one of the best examples of a linguistic subarea in \isi{NEA}, the \textbf{\isi{Amdo Sprachbund}} can be found in northwestern \isi{China} (e.g., \citealt{Dwyer1995}; \citealt{Slater2003a}; \citealt{ZhongJinwen2007}). There are many different designations for the area \citep{Janhunen2007b}, but the name \textit{\isi{Amdo Sprachbund}} has been adopted by several recent publications (e.g., \citealt{Janhunen2012a}; \citealt{Simon2015}; \citealt{SandmanSimon2016}). It is difficult to establish a clear boundary of the area, but it roughly encompasses eastern \isi{Qinghai}, parts of northern \isi{Sichuan}, and most of \isi{Gansu}. The best overview of the area has been given by \cite{Janhunen2007b,Janhunen2012a}, according to whom the area is the result of a very unique \isi{interaction} of \ili{Turkic}, \ili{Mongolic}, \ili{Tibetic}, and \ili{Sinitic} languages. However, historically the \ili{Tangut} language, usually classified as Qiangic, as well as the probably \ili{Para-Mongolic} language \ili{Tuyuhun} were at some point also spoken in the area. \cite[180ff.]{Janhunen2012a} mentions the following defining features of the area: SOV \isi{word order}, suffixes or enclitics, \isi{case} marking, verbal tense-aspect categories, converbs, postpositions, indefinite articles, perspective marking (including loss of person marking in \ili{Turkic} and \ili{Mongolic}). From this brief list alone, however, this quite clearly does not qualify as a \isi{linguistic area} at all. Most of these features are not only prevalent in adjacent areas, but are also extremely common worldwide. But it may be noted that, apart from strong interference between individual languages with each other (e.g., \citealt{Sandman2012}; \citealt{Simon2015}; \citealt{SandmanSimon2016}), there are also several creolized languages such as \ili{Gangou}, \ili{Wutun}, \ili{Tangwang}, and Hezhou \ili{Chinese}, which indicates strong \isi{language contact} in the area (\sectref{sec:5.9}). Before this background it seems even possible to extend the area towards the south to include, for example, the language \ili{Daohua}, which is a \ili{Chinese}-\ili{Tibetan} creole or mixed language spoken in western Sichuan province (e.g., \citealt{Acuo2001}; \citealt{Chen2017}). Nevertheless, the traditional \isi{conception} of the \isi{Amdo Sprachbund} is adopted here for pragmatic reasons (see \sectref{sec:6.3}).