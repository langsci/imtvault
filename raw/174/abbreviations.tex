\addchap{Abbreviations}

The {glossing} of the examples mostly follows the \textit{Leipzig Glossing Rules}. {See \url{https://www.eva.mpg.de/lingua/resources/glossing-rules.php} (Accessed 2016-07-06).}
\\ 

\medskip 

\begin{tabularx}{.45\textwidth}{lQ} 
. & old or unimportant morpheme boundary\\
{\textasciitilde} & {submorpheme}, {resonance}\\
\# & beginning or end of a sentence\\
\textsc{abm} & ablative modalis \citep{Miyaoka2012}\\
\textsc{appr} & approximative\\
AMH & {anatomically modern humans}\\
ANE & {Ancient North Eurasians}\\
\textsc{appr} & approximative \citep{Shigeno2010}\\
AQ & {alternative question}\\
Av & {Avestan}\\
\textsc{bg} & bound genitive pronoun \citep{Huang1996}\\
CAY & Central Alaskan {Yupik}\\
CDC & Common Dialectal {Chinese} \citep{Norman2014}\\
CIA & Copper Island {Aleut}\\
\textsc{cj} & conjunct\\
CK & {Chukotko-Kamchatkan}\\
\textsc{coll} & collective (\citealt{KämpfeVolodin1995})\\
\textsc{comm} & committal\\
CPR & {Chinese} Pidgin {Russian}\\
CQ & {content question}\\
CSY & Central Siberian {Yupik}\\
\textsc{dj} & disjunct\\
EC & Early {Chinese} \citep{Norman2014}\\
EOJ & Eastern {Old Japanese}\\
\end{tabularx}
\begin{tabularx}{.45\textwidth}{lQ}  
\textsc{ex} & root expander \citep{Miyaoka2012}\\
\textsc{exp} & experiential (\citealt{SunChaofen2006})\\
FQ & {focus question}\\
G & Gyeongsang\\
GQ & grammar of {questions}\\
Got & {Gothic}\\
Grk & Greek\\
\textsc{hl} & highlighter \citep{Stern2005}\\
IE & Indo-European\\
INT & {interrogative} (e.g., \textit{who}, \textit{what})\\
K & {Korean}\\
KM & \textit{kakari musubi} ({Japanese} for \textit{{focus} concord}) \citep{Shinzato2015}\\
KP & \textit{kakari (musubi)} particle \citep{Shinzato2015}\\
ky & thousand years\\
kya & thousand years ago\\
Lat & {Latin}\\
MHG & {Middle High German}\\
MK & Middle {Korean}\\
my & million years\\
\textsc{ms} & morphosyntactic separator \citep{Vajda2004}\\
MSEA & {Mainland Southeast Asia}\\
NAQ & negative {alternative question}\\
NE & New {English}\\
NEA & {Northeast Asia}\\
NHG & New High {German}\\
\end{tabularx}

\begin{tabularx}{.45\textwidth}{lQ} 
NPQ & negative {polar question}\\
\textsc{nrf} & non-referential \citep{Huang1996}\\
OAQ & open {alternative question}\\
OAv & Old {Avestan}\\
OCS & {Old Church Slavonic}\\
OE & Old {English}\\
OES & {organism-environment system}\\
OHG & Old High {German}\\
OR & Old Ry\=uky\=uan\\
\textsc{p} & participle\\
PC & Proto-{Chukotian}\\
PCK & Proto-{Chukotko-Kamchatkan}\\
PG & Proto-{Germanic}\\
PIE & {Proto-Indo-European}\\
PJ & {Proto-Japonic}\\
PM & Proto-{Mongolic}\\
PMJ & pre-modern {Japanese}\\
\textsc{pn} & personal/proper/place name\\
\textsc{post} & postterminal \citep[151]{Ragagnin2011}\\
\end{tabularx}
\begin{tabularx}{.45\textwidth}{lQ}  
PR & Proto-Ry\=uky\=uan\\
PS & {Proto-Slavic}\\
PT & {Proto-Tungusic}\\
PQ & {polar question}\\
p.c. & personal communication\\
\textsc{q}/Q & question\\
\textsc{rel} & relevance \citep{Shapiro2010}\\
\textsc{rf} & reduced forcefulness (\citealt{LiThompson1981})\\
\textsc{semf} & semi-formal\\
\textsc{sgs} & suggestion \citep{Miyara2015}\\
Skt & {Sanskrit}\\
TA & {Tocharian A}\\
TAME & tense, aspect, mood, and {evidentiality}\\
TB & {Tocharian B}\\
TH & {Trans-Himalayan} ({Sino-Tibetan})\\
\textsc{thm} & thematic vowel\\
TQ & {tag question}\\
WOJ & Western {Old Japanese}\\
x & mixed with \citep{Janhunen2012b}\\ 
\end{tabularx}
