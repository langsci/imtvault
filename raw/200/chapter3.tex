%versao de 23-MAR-2019

\chapter{Coordenação e negação}

Neste capítulo, discutiremos o significado flexível da conjun\-ção aditiva \textit{e} e da palavra negativa \textit{não}. Desenvolveremos uma análise em que as
extensões desses itens são uma família de fun\-çõ\-es, continuando a nos valer de
aplica\-ção funcional como o pilar de nosso sistema composicional.

\section{Coordenação}

Consideremos a sentença (\ref{co}) abaixo:

\begin{exe}
\ex João trabalha e Maria estuda.\label{co}
\end{exe}

\n Essa senten\-ça é formada pela coordena\-ção de duas outras
senten\-ças por meio da conjun\-ção aditiva \textit{e}. No capítulo 1, representamos essa coordenação como uma estrutura ternária em que a conjunção \textit{e} tinha como irmãos na árvore sintática as duas partes coordenadas.

\begin{figure}[H]
	\centerline{ \Tree [ \qroof{João trabalha}.S$_{1}$ e \qroof{Maria estuda}.S$_{2}$ ].S } \caption{Estrutura ternária para a coordenação }
\end{figure}



\n É possível, entretanto, adotar uma estrutura binária para essas construções, deixando as mesmas em pé igualdade com outras construções sintáticas complexas como os VPs e as sentenças do capítulo 2. Não iremos aqui argumentar empiricamente a favor dessa posição (ver, por exemplo, \cite{munn93}, capítulo 2, para possíveis indícios), apenas adotando-a por uma questão de uniformidade. Rotularemos o item \textit{e} de Coord(enação) e as projeções imediatamente acima dele de Coord$^{\prime}$ e CoordP:

\begin{figure}[H]
	\centerline{ \Tree [ \qroof{João trabalha}.S$_{1}$ [ [ e ].Coord \qroof{Maria estuda}.S$_{2}$ ].Coord$^{\prime}$ ].CoordP } \caption{Estrutura binária para a coordenação }
\end{figure}


Lembre-se, entretanto, que rótulos não desempenham nenhum papel em
nosso sistema. O que realmente importa é a estrutura hierárquica e as extensões dos seus constituintes. Nem aplicação funcional nem o princípio dos nós não ramificados fazem  menção a rótulos categoriais (NP, VP, S, ...). Portanto, se, por alguma razão extra, você preferir outros rótulos (ConjP, S$^{\prime}$, ...), sinta-se à vontade para modificá-los.

Note a semelhan\-ça dessa estrutura com a estrutura de senten\-ças
com verbos transitivos que discutimos no capítulo anterior. Tanto
a conjun\-ção \textit{e} quanto um verbo transitivo aparecem junto a dois
constituintes da mesma categoria: S no primeiro caso, e NP no
segundo (o NP sujeito e o NP objeto). A diferen\-ça é que a extensão de uma senten\-ça é de
tipo $t$, enquanto a extensão de um NP é de tipo $e$. Tratamos os
verbos transitivos como fun\-çõ\-es de tipo
$\langle e,\langle e,t \rangle\rangle$. Iremos agora propor uma
entrada lexical para a conjun\-ção  \textit{e} que a trata como uma
fun\-ção de tipo $\langle t,\langle t,t \rangle\rangle$. A
mudan\-ça nos tipos reflete a diferen\-ça que acabamos de notar.
Quanto à descri\-ção dessa fun\-ção, notemos que uma senten\-ça
coordenada é verdadeira se as senten\-ças que a compõem forem
ambas verdadeiras, e falsa se umas delas, ou ambas, forem falsas.
A entrada lexical abaixo capta essa condi\-ção:

\begin{exe}
	\ex \den{e} = $(\lambda p_{t}.\ \lambda q_{t}.\ p=1\ \text{e}\ q=1)$
\end{exe}

Antes de prosseguir, pausemos um instante para fazer algumas considerações sobre o uso da conjunção \textit{e}, já que esse uso  costuma trazer à tona outras nuances de significado, além da adição, tais como nexos causais e temporais. Considere, por exemplo, as sentenças abaixo:

\begin{exe}
	\ex João faltou ao serviço e seu patrão o demitiu. \label{dem}
	
	\ex Pedro entrou na sala e João o cumprimentou. \label{cum}	
\end{exe}

\n O uso de (\ref{dem}) comumente veicula a ideia de que a falta ao serviço causou a demissão. Já o de (\ref{cum}) passa a ideia que primeiro Pedro entrou na sala, e depois João o cumprimentou. Entretanto, esses aspectos do significado não parecem estar embutidos na contribuição semântica de \textit{e}, já que podem facilmente ser cancelados:

\begin{exe}
	\ex João faltou ao serviço e seu patrão o demitiu, mas uma coisa não teve nada a ver com a outra. \label{demc}

	\ex Pedro entrou na sala e João o cumprimentou, não necessariamente nessa ordem. \label{cumc}	
\end{exe}

Já a exigência de que ambas as orações coordenadas sejam verdadeiras tem um estatuto diferente e parece inexoravelmente atrelada a essa conjunção. Note a flagrante contradição nos casos abaixo:

\begin{exe}
	\ex João faltou ao serviço e seu patrão o demitiu, mas João não faltou ao serviço. \label{demct}
	
	\ex Pedro entrou na sala e João o cumprimentou, mas João não o cumprimentou. \label{cumct}	
\end{exe}

Não discutiremos aqui como a veiculação de nexos causais e temporais emerge de certos usos da conjunção \textit{e}. Limitamo-nos a apontar que esse foi e continua sendo um tema muito discutido em pragmática, que, muitas vezes, se caracteriza por buscar manter uma semântica mais austera, enriquecida por princípios conversacionais que regem a interação linguística entre falantes (ver sugestões de leitura ao final deste capítulo).

Para realçar esse tratamento semântico enxuto que estamos dando a \textit{e}, vamos usar na metalinguagem o conectivo lógico $\&$ (às vezes também representado por $\land$), caracterizado por exigir única e exclusivamente a verdade de ambas as sentenças coordenadas. Sua contribuição semântica costuma ser representada como na tabela \ref{tabelae}, em que $\phi$ e $\psi$ representam duas sentenças quaisquer:

\begin{table}
	\begin{tabularx}{.4\textwidth}{XXX}
		\lsptoprule
		$\phi$ & $\psi$ & $\phi\ \&\ \psi$ \\
		\midrule 
		1 & 1 & \ \ \ 1 \\  
		1 & 0 & \ \ \ 0 \\ 
		0 & 1 & \ \ \ 0 \\ 
		0 & 0 & \ \ \ 0 \\
		\lspbottomrule
	\end{tabularx}
\caption{Tabela de verdade de $\&$}
\label{tabelae}
\end{table}


\n A entrada lexical de \textit{e} será, portanto, a seguinte:

\begin{exe}
	\ex \den{e} = $(\lambda p_{t}.\ \lambda q_{t}.\ p=1\ \&\ q=1)$
\end{exe}

Feitas essas considerações, vejamos como as condi\-çõ\-es de verdade de (\ref{co}), repetida abaixo, são
derivadas:

\begin{figure}[H]
	\centerline{ \Tree [ \qroof{João trabalha}.S$_{1}$ [ [ e ].Coord \qroof{Maria estuda}.S$_{2}$ ].Coord$^{\prime}$ ].CoordP } \caption{Coordenação sentencial }
\end{figure}




\n Através de aplica\-ção funcional, obtemos o
seguinte:

\begin{exe}
	\ex \den{Coord$^{\prime}$} = \den{Coord}(\den{S$_{2}$}) \\
	= ($\lambda p_{t}.\ \lambda q_{t}.\ p=1\ \&\ q=1$)(\den{S$_{2}$}) \\
	= $\lambda q_{t}.\ $\den{S$_{2}$}$=1\ \&\ q=1$
\end{exe}

\n Como já sabemos que:

\begin{exe}
	\ex \den{S$_{2}$} = 1 \textit{sse} $\predica{estuda}{maria}$
\end{exe}

\n temos que:

\begin{exe}
	\ex \den{Coord$^{\prime}$} = $\lambda q_{t}.\ \predica{estuda}{maria}\ \&\ q=1$
\end{exe}

\n Aplica\-ção funcional é então novamente utilizada para derivar
as condi\-çõ\-es de verdade de CoordP:

\begin{exe}
	\ex \den{CoordP} = \den{Coord$^{\prime}$}(\den{S$_{1}$})\\
	= ($\lambda q_{t}.\ \predica{estuda}{maria}\ \&\ q=1$)(\den{S$_{1}$}) \\
	= 1 \textit{sse} $\predica{estuda}{maria}$ $\&$ \den{S$_{1}$} = 1
\end{exe}

\n Como já sabemos que:

\begin{exe}
	\ex \den{S$_{1}$} = 1 \textit{sse} $\predica{trabalha}{joão}$
\end{exe}

\n temos que:

\begin{exe}
	\ex \den{CoordP} = 1 \textit{sse} $\predica{estuda}{maria}\ \&\ \predica{trabalha}{joão} $
\end{exe}

\n Note que, como o conectivo \& utilizado na metalinguagem é simétrico, ou seja, a ordem dos elementos coordenados não é relevante, podemos igualmente escrever:

\begin{exe}
	\ex \den{CoordP} = 1 \textit{sse} $\predica{trabalha}{joão}\ \&\  \predica{estuda}{maria}$
\end{exe}

\n Esse é, de fato, o resultado que desejávamos.

\section{A flexibilidade da coordenação}

Analisamos acima uma senten\-ça formada pela coordena\-ção de duas senten\-ças e atribuímos uma entrada lexical para a
conjun\-ção \textit{e} de tipo $\langle t,\langle t,t\rangle\rangle$, que
nos forneceu condi\-çõ\-es de verdade adequadas.

A conjun\-ção \textit{e}, entretanto, é capaz de coordenar não só
senten\-ças mas uma variedade de constituintes, conforme atestam
os exemplos abaixo:

%xl
\begin{exe}
\ex\label{cro}
\begin{xlist}
\ex João estuda e trabalha.\hfill Coord. de  Vs intransitivos\label{croa}
\ex João abraçou e beijou Maria.\hfill Coord. de  Vs transitivos\label{crob}
\ex João mostrou e deu o livro pra Maria.\hfill Coord. de Vs bitrans.\label{croc}
\ex João abraçou Maria e beijou Paula.\hfill Coord. de  VPs\label{crod}
\ex João é médico e psicólogo.\hfill Coord. de NPs \label{croe}
\ex João está alegre e orgulhoso da Maria.\hfill Coord. de
APs\label{crof}
\ex João esteve em Paris e em Londres.\hfill Coord. de
PPs.\label{crog}
\ex João e Maria estudam.\hfill Coord. de nomes próprios\label{croh}

\end{xlist}
\end{exe}

\n Como se pode notar, todos os tipos de constituintes que vimos
até agora podem ser coordenados pela conjun\-ção \textit{e}. Porém,
nosso sistema não é capaz de interpretar nenhuma das senten\-ças
acima. A razão para isso é simples. A entrada lexical que
fornecemos para a conjun\-ção \textit{e}, por ser de tipo $\langle
t,\langle t,t\rangle\rangle$, só permite sua combina\-ção com
constituintes de tipo \textit{t}, ou seja, constituintes
sentenciais. Mas nenhum dos constituintes coordenados em
(\ref{croa})-(\ref{croh}) é de tipo \textit{t}. Os verbos
intransitivos em (\ref{croa}), por exemplo, denotam fun\-çõ\-es de
tipo $\langle e,t\rangle$, assim como os VPs em (\ref{crod}) e os
predicados não verbais em (\ref{croe})-(\ref{crog}). Já os verbos
transitivos em (\ref{crob}) denotam fun\-çõ\-es de tipo $\langle
e,\langle e,t\rangle\rangle$, enquanto os bitransitivos em
(\ref{croc}) denotam fun\-çõ\-es de tipo $\langle e,\langle e,
\langle e,t\rangle\rangle\rangle$. Por fim, os nomes próprios em
(\ref{croh}) denotam indivíduos (tipo $e$).

Uma saída para esse problema seria admitir que as senten\-ças em
(\ref{cro}) são, na verdade, versões reduzidas de senten\-ças mais
complexas envolvendo coordena\-ção sentencial. De fato, para todas
as senten\-ças em (\ref{cro}), é possível encontrar uma senten\-ça
equivalente envolvendo a coordena\-ção de duas outras senten\-ças.

%xl
\begin{exe}
\ex\label{dro}
\begin{xlist}
\ex João estuda e João trabalha.\label{droa}
\ex João abraçou Maria e João beijou Maria.\label{drob}
\ex João mostrou o livro pra Maria e João deu o livro pra Maria.\label{droc}
\ex João abraçou Maria e João beijou Paula.\label{drod}
\ex João é médico e João é psicólogo.\label{droe}
\ex João está alegre e João está orgulhoso da Maria.\label{drof}
\ex João esteve em Paris e João esteve em Londres.\label{drog}
\ex João estuda e Maria estuda.\label{droh}

\end{xlist}
\end{exe}

\n As versões reduzidas em (\ref{cro}) seriam, então, obtidas
através de transforma\-çõ\-es sintáticas a partir das senten\-ças
em (\ref{dro}), mas essas transforma\-çõ\-es não teriam impacto
semântico, ou seja, o \textit{input} para o componente semântico
seriam as versões não reduzidas em (\ref{dro}). E para a
interpreta\-ção dessas senten\-ças, nossa entrada lexical de tipo
$\langle t,\langle t,t\rangle\rangle$ é suficiente.

Há problemas, no entanto, com esse processo de redu\-ção
sintática. Nem todos os casos de coordena\-ção são equivalentes às
suas versões não reduzidas. Considere, por exemplo, a senten\-ça
em (\ref{ero}) e sua versão não reduzida em (\ref{fro}):

\begin{exe}
	\ex Alguém estuda e trabalha.\label{ero}

	\ex Alguém estuda e alguém trabalha.\label{fro}
\end{exe}

Ainda não dissemos nada sobre o pronome indefinido \textit{alguém}, e só
iremos fazê-lo mais tarde no capítulo dedicado à análise da
quantifica\-ção. Mas basta refletirmos um pouco sobre as
condi\-çõ\-es de verdade das duas senten\-ças acima para notarmos
que elas não são equivalentes. Para (\ref{ero}) ser verdadeira, é
necessária a existência de uma mesma pessoa que estude e também
trabalhe. Já no caso de (\ref{fro}), basta haver uma pessoa que
estude e uma que trabalhe, sem a necessidade de que elas sejam a mesma pessoa. Não queremos, portanto, que o
\textit{input} para a interpreta\-ção de (\ref{ero}) seja (\ref{fro}), já
que elas não têm o mesmo significado.

Busquemos, então, uma alternativa para a análise baseada em
redu\-ção sintática. Vamos manter que a estrutura das senten\-ças
em (\ref{cro}) que serve de \textit{input} ao componente semântico envolve
mesmo a coordena\-ção de constituintes menores que a senten\-ça.

A maneira mais óbvia de concretizar essa abordagem é assumir que
existem várias conjun\-çõ\-es \textit{e}, cada uma com um tipo semântico
diferente. Assim, além da versão de tipo $\langle t,\langle
t,t\rangle\rangle$, utilizada na coordena\-ção de senten\-ças,
haveria, por exemplo, uma outra de tipo $\langle et,\langle
et,et\rangle\rangle$ utilizada na coordena\-ção de constituintes
com
denota\-çõ\-es de tipo $\langle e,t\rangle$, como em (\ref{croa}).

\begin{figure}[H]
	\centerline{ \Tree [ [ [ João ].N ].NP [ [ [ trabalha ].V ].VP$_{1}$ [ [ e$'$ ].Coord [ [ estuda ].V ].VP$_{2}$ ].Coord$'$ ].CoordP ].S } \caption{Coordenação de VPs }
\end{figure}


\n Vamos chamar essa outra conjun\-ção de $e'$ para diferenciá-la
do $e$ que coordena senten\-ças. Queremos que essa nova
conjun\-ção tome as extensões dos dois VPs (duas funções de tipo
$\langle e,t\rangle$) como argumentos e retorne uma nova fun\-ção
desse mesmo tipo $\langle e,t\rangle$. Essa nova função tomará um indivíduo como argumento (fornecido pela extensão do NP sujeito) e aplicará as extensões dos VPs originais a ela. A
entrada lexical abaixo formaliza essa
intuição:

\begin{exe}
	\ex \den{e$'$} = $\lambda F_{\langle e,t\rangle}.\lambda G_{\langle e,t\rangle}.\lambda x_{e}.\ F(x)=1\ \&\ G(x)=1$
\end{exe}

\n Ou, equivalentemente, na notação de predicados:

\begin{exe}
	\ex \den{e$'$} = $\lambda F_{\langle e,t\rangle}.\lambda G_{\langle e,t\rangle}.\lambda x_{e}.\ F'(x)\ \&\ G'(x)$
\end{exe}

\n Nessa nova entrada, $F$ e $G$ correspondem às extensões dos VPs coordenados e $x$ à extensão do NP sujeito. Com isso em mente, acompanhe atentamente a deriva\-ção das condi\-çõ\-es de
verdade da estrutura acima e note que a nova entrada lexical fornece
o resultado desejado.

\begin{exe}
	\ex \den{Coord$'$} = \den{Coord}(\den{VP$_{2}$})\\
	= ($\lambda F.\lambda G.\lambda x.\ F'(x)\ \&\ G'(x)$)(\den{VP$_{2}$})
\end{exe}


\n Como já sabemos que:

\begin{exe}
	\ex \den{VP$_{2}$} = $\lambda x.\ \predica{estuda}{x}$
\end{exe}

\n e que, portanto:

\begin{exe}
	\ex \den{VP$_{2}$}($x$) = 1 \textit{sse} $\predica{estuda}{x}$
\end{exe}

\n temos que:

\begin{exe}
	\ex \den{Coord$'$} = $\lambda G.\lambda x.\ \predica{estuda}{x}\ \&\ G'(x)$
\end{exe}

\n Quanto a CoordP, valemo-nos de aplica\-ção funcional
novamente:

\begin{exe}
	\ex \den{CoordP} = \den{Coord$'$}(\den{VP$_{1}$}) \\
	= ($\lambda G.\lambda x.\ \predica{estuda}{x}\ \&\ G'(x)$)(\den{VP$_{1}$})
\end{exe}

\n Como já sabemos que:

\begin{exe}
	\ex \den{VP$_{1}$} = $\lambda x.\ \predica{trabalha}{x}$
\end{exe}

\n e que, portanto:

\begin{exe}
	\ex \den{VP$_{1}$}(\textit{x}) = 1 \textit{sse} $\predica{trabalha}{x}$
\end{exe}

\n temos que:

\begin{exe}
	\ex \den{CoordP} = $\lambda x_{e}.\ \predica{estuda}{x}\ \&\ \predica{trabalha}{x}$
\end{exe}

\n Note que essa é uma função de tipo $\langle e,t\rangle$ que leva um indivíduo $x$ ao valor de verdade 1 se, e somente se, $x$ estuda e $x$ trabalha. Note, ainda, que dada a já salientada simetria do conectivo lógico \&, podemos igualmente escrever:

\begin{exe}
	\ex \den{CoordP} = $\lambda x_{e}.\ \predica{trabalha}{x}\ \&\ \predica{estuda}{x}$
\end{exe}


\n Finalmente, temos que:

\begin{exe}
	\ex \den{S} = \den{CoordP}(\den{NP})\\
		\den{S} = $(\lambda x.\ \predica{trabalha}{x}\ \&\ \predica{estuda}{x})(\text{\textit{joão}})$\\
		\den{S} = 1 \textit{sse} $\predica{trabalha}{joão}\ \&\ \predica{estuda}{joão}$
\end{exe}

A entrada lexical proposta acima para $e'$ pode ser utilizada não
apenas em (\ref{croa}), mas também em (\ref{crod})-(\ref{crog}),
já que todos esses casos envolvem coordena\-ção de constituintes
cujas denota\-çõ\-es são fun\-çõ\-es de tipo $\langle e,t\rangle$.

Para casos envolvendo a coordena\-ção de verbos transitivos e de
verbos bi-transitivos, precisamos de outras entradas lexicais, já
que suas denota\-çõ\-es são fun\-çõ\-es de tipo $\langle e,\langle
e,t\rangle\rangle$ e $\langle e,\langle e, \langle
e,t\rangle\rangle\rangle$ respectivamente. Mas a lógica por trás
da defini\-ção dessas novas entradas lexicais é semelhante à que
utilizamos para $e'$. Queremos uma fun\-ção que, após ser aplicada
às extensões dos verbos coordenados, retorne como valor uma
fun\-ção do mesmo tipo que as extensões desses verbos. Já os
argumentos fornecidos a essa nova fun\-ção serão, por assim dizer,
transmitidos às extensões dos verbos coordenados, como
exemplificado
abaixo:

\begin{exe}
	\ex \den{e$''$} = $\lambda F_{\langle e,\langle e,t\rangle\rangle}.\lambda G_{\langle e,\langle e,t\rangle\rangle}.\lambda x_{e}.\lambda y_{e}.\ F(x)(y)=1\ \&\ G(x)(y)=1$
\end{exe}

\begin{exe}
	\ex \den{e$'''$} = $\lambda F_{\langle e,\langle e, \langle e,t\rangle\rangle\rangle}.\lambda G_{\langle e,\langle e, \langle e,t\rangle\rangle\rangle}.\lambda x_{e}.\lambda y_{e}.\lambda z_{e}.\ F(x)(y)(z)=1\ \& $ \\
	\hspace*{\fill} $ G(x)(y)(z)=1 $
\end{exe}



\n Ou, equivalentemente, na notação de predicados:


\begin{exe}
	\ex \den{e$''$} = $\lambda F_{\langle e,\langle e,t\rangle\rangle}.\lambda G_{\langle e,\langle e,t\rangle\rangle}.\lambda x_{e}.\lambda y_{e}.\ F'(y,x)\ \&\ G'(y,x)$
\end{exe}

\begin{exe}
	\ex \den{e$'''$} = $\lambda F_{\langle e,\langle e, \langle e,t\rangle\rangle\rangle}.\lambda G_{\langle e,\langle e, \langle e,t\rangle\rangle\rangle}.\lambda x_{e}.\lambda y_{e}.\lambda z_{e}.\ F'(z,x,y)\ \&\ G'(z,x,y)$
\end{exe}

Vistas sob um certo ângulo, essas entradas lexicais caracterizam uma espécie de economia de meios, em que um mesmo recurso sintático (sujeito, objeto direto, etc.) é utilizado duas vezes semanticamente.

O procedimento de postular múltiplas entradas lexicais para a
conjun\-ção \textit{e} equivale a assumir que essa conjun\-ção é ambígua,
o que, por sua vez, equivale a dizer que existem várias
conjun\-çõ\-es com significados diferentes, porém homófonas. Ainda
que tal procedimento conduza a resultados descritivamente
corretos, causa um certo desconforto o fato de que intuitivamente
estamos diante de uma mesma conjun\-ção que introduz sempre a
ideia de adi\-ção, seja essa a adi\-ção de proposi\-çõ\-es,
propriedades ou rela\-çõ\-es. Além disso, essa suposta ambiguidade
não é exclusiva do português, existindo em uma variedade de
línguas. Tudo isso leva a crer que essa flexibilidade no uso da
conjun\-ção \textit{e} não é acidental como seria de se esperar de uma
abordagem como a descrita acima, baseada na simples listagem de
múltiplas entradas lexicais de itens homófonos. Seria interessante
se pudéssemos relacionar explicitamente todas essas entradas
lexicais, de modo que a partir de uma defini\-ção básica,
derivássemos todas as demais.

Há uma maneira relativamente simples de concretizar essa proposta.
Ela se baseia em uma semelhan\-ça na forma das entradas lexicais
que propusemos acima e que repetimos abaixo por conveniência:

\begin{exe}
	\ex \den{e} = $\lambda p_{t}.\ \lambda q_{t}.\ p=1\  \&\ q=1$\\
		\den{e$'$} = $\lambda F_{\langle e,t\rangle}.\lambda G_{\langle e,t\rangle}.\lambda x_{e}.\ F(x)=1\ \&\ G(x)=1$\\
		\den{e$''$} = $\lambda F_{\langle e,\langle e,t\rangle\rangle}.\lambda G_{\langle e,\langle e,t\rangle\rangle}.\lambda x_{e}.\lambda y_{e}.\ F(x)(y)=1\ \&\ G(x)(y)=1$\\
		\den{e$'''$} = $\lambda F_{\langle e,\langle e, \langle e,t\rangle\rangle\rangle}.\lambda G_{\langle e,\langle e, \langle e,t\rangle\rangle\rangle}.\lambda x_{e}.\lambda y_{e}.\lambda z_{e}.\ F(x)(y)(z)=1\ \& $ \\ \hspace*{\fill} $G(x)(y)(z)=1$
\end{exe}


\n Para \den{e}, valemo-nos simplesmente do conectivo lógico
sentencial $\&$, já que estávamos coordenando constituintes
sentenciais. Nos demais casos, em que coordenamos constituintes com
denota\-çõ\-es funcionais, o que fizemos foi alimentar cada uma das
fun\-çõ\-es com os mesmos argumentos e conectar os resultados (na
metalinguagem) com o $\&$ sentencial. A diferen\-ça entre
\den{e$'$}, \den{e$''$} e \den{e$'''$} está apenas no número de
argumentos necessários para que as fun\-çõ\-es
envolvidas retornem um valor de verdade: 1, 2 e 3, respectivamente. 

Tendo isso em mente, podemos tomar \den{e} como forma básica, e
derivar as demais a partir dela.

\begin{exe}
	\ex \den{e} = $\lambda p_{t}.\ \lambda q_{t}.\ p=1\ \&\ q=1$\\
		\den{e$'$} = $\lambda F_{\langle e,t\rangle}.\lambda G_{\langle e,t\rangle}.\lambda x_{e}.\ \llbracket \text{e} \rrbracket (F(x))(G(x))$\\
		\den{e$''$} = $\lambda F_{\langle e,\langle e,t\rangle\rangle}.\lambda G_{\langle e,\langle e,t\rangle\rangle}.\lambda x_{e}.\lambda y_{e}.\ \llbracket \text{e} \rrbracket (F(x)(y))(G(x)(y))$\\
		\den{e$'''$} = $\lambda F_{\langle e,\langle e, \langle e,t\rangle\rangle\rangle}.\lambda G_{\langle e,\langle e, \langle e,t\rangle\rangle\rangle}.\lambda x_{e}.\lambda y_{e}.\lambda z_{e}. $ \\
		\hspace*{\fill} $\llbracket \text{e} \rrbracket (F(x)(y)(z))(G(x)(y)(z))$
\end{exe}

\n É importante notar que, para
procedermos dessa forma, é necessário que a aplicação das funções $F$ e $G$  aos argumentos $x$, $y$, $z$, cheguemos a um valor de verdade (tipo
\textit{t}), caso contrário não poderíamos usar os resultados como argumento de \den{e} na definição dessas formas derivadas. Portanto, apenas extensões de tipos semânticos
que terminam em \textit{t} podem ser coordenadas seguindo essa
receita. Exemplos de constituintes que terminam em \textit{t} são
$t$, $\langle e,t\rangle$, $\langle e,et\rangle$, $\langle
et,et\rangle$, etc. Esses tipos terminados em \textit{t}
costumam ser chamados de \textsc{tipos booleanos}, em homenagem a George Boole, um dos pais da lógica e da álgebra modernas. Não são
booleanos tipos tais como $e$, $\langle et,e\rangle$, $\langle
e,e\rangle$, etc.


Dessa forma, continuaremos assumindo que a conjun\-ção \textit{e} está associada a
múltiplos tipos semânticos. Entretanto, relacionaremos explicitamente as múltiplas extensões envolvidas. Esse procedimento costuma ser chamado de polissemia lógica ou \textsc{polimorfismo}.

Generalizando o que acabamos de ver para todos os tipos booleanos, podemos definir o seguinte esquema para a conjunção:

\begin{exe}
	\ex Versão polimórfica da conjun\-ção \textit{e}: \\
	\hspace*{18pt}\underline{Caso básico}: se $a = t$, \\
	\hspace*{30pt}$\llbracket e_{\langle a ,\langle a,a\rangle\rangle}\rrbracket = \lambda p_{t}.\ \lambda q_{t}.\ p=1\ \&\ q=1$ \\ 
	\hspace*{18pt}\underline{Casos derivados}: se $a = \langle a_{1},\langle a_{2}, ... ,\langle a_{n},t\rangle ... \rangle\rangle\ (n\geq 1)$, \\
	\hspace*{30pt} $\llbracket e_{\langle a ,\langle a,a\rangle\rangle}\rrbracket = \lambda f_{a}.\lambda g_{a}.\lambda \gamma_{a_{1}}^{1}.\lambda \gamma_{a_{2}}^{2}...\lambda \gamma_{a_{n}}^{n}.$\\ \hspace*{\fill} $\llbracket e_{\langle t\langle t,t\rangle\rangle}\rrbracket(f_{a}(\gamma_{a_{1}}^{1})(\gamma_{a_{2}}^{2})...(\gamma_{a_{n}}^{n}))(g_{a}(\gamma_{a_{1}}^{1})(\gamma_{a_{2}}^{2})...(\gamma_{a_{n}}^{n}))$
\end{exe}


\n Dado um caso concreto, basta selecionar a versão correspondente do tipo $a$ no esquema acima. Por exemplo, no caso da coordenação de verbos bitransitivos, $a$ será igual a $\langle e, \langle e, \langle e,t\rangle\rangle\rangle$ e $a_{1}=a_{2}=a_{3}=e$. Como o leitor poderá verificar, ao aplicar esses valores no esquema acima, obteremos exatamente a extensão de $e'''$ que vimos anteriormente.
    
O esquema acima consegue, portanto, dar conta de
todos os casos de coordena\-ção listados em (\ref{cro}), com
exce\-ção de (\ref{croh}). Como atribuímos aos nomes próprios o
tipo básico \textit{e}, que não é um tipo booleano, tal esquema não pode ser aplicado e nosso sistema não
consegue atribuir uma extensão ao constituinte [$_{\text{\textsc{NP}}}$ João e Maria ]. Deixaremos, por hora, essa
questão em aberto. Voltaremos a ela no capítulo sobre
quantifica\-ção, quando estaremos em uma melhor posi\-ção para
resolvê-la.

Há uma outra maneira de formalizar essa flexibilidade da conjunção \textit{e}, com a qual vale a pena estar familiarizado. Trata-se de definir recursivamente as múltiplas entradas da conjunção. Para tanto, vamos retomar rapidamente as versões para a coordenação de verbos intransitivos e transitivos que já vimos. Lembremos que, para dar conta da coordenação de verbos intransitivos, como em \textit{Maria fuma e bebe}, valemo-nos da seguinte entrada lexical:

\begin{exe}
	\ex \den{e$'$} = $\lambda F_{\langle e,t\rangle}.\lambda G_{\langle e,t\rangle}.\lambda x_{e}.\ F'(x)\ \&\ G'(x)$
\end{exe}

\n E para dar conta da coordenação de verbos transitivos, como em \textit{Maria beijou e abraçou Pedro}, valemo-nos de outra extensão:

\begin{exe}
	\ex \den{e$''$} = $\lambda F_{\langle e,\langle e,t\rangle\rangle}.\lambda G_{\langle e,\langle e,t\rangle\rangle}.\lambda x_{e}.\lambda y_{e}.\ F'(x)(y)\ \&\ G'(x)(y)$
\end{exe}

\n Atente agora para o seguinte: os três primeiros argumentos que
a entrada lexical \den{e$''$} toma correspondem às extensões de dois
verbos transitivos, \textit{F} e \textit{G}, ambas de tipo
$\langle e, et\rangle$ e à extensão do objeto direto, \textit{x},
de tipo \textit{e}. Mas com essas extensões à nossa disposição,
podemos construir, por assim dizer, duas extensões de tipo
$\langle e,t\rangle$, bastando aplicar \textit{F} e \textit{G} a
\textit{x}. Os resultados, \textit{F(x)} e \textit{G(x)} são
funções de tipo $\langle e,t\rangle$. Podemos então aplicar a
estas novas funções a extensão \den{e$'$} e obteremos o mesmo
resultado que teríamos obtido aplicando \den{e$'$} a
\textit{F}, \textit{G} e \textit{x} nesta ordem. Em termos
formais:

\begin{exe}
	\ex $\llbracket e'' \rrbracket (F)(G)(x) = \llbracket e' \rrbracket (F(x))(G(x))$
\end{exe}

\n Que esta igualdade é verdadeira pode ser facilmente verificado
a partir de \den{e$''$} e \den{e$'$}. O que fizemos acima foi 
redefinir \den{e$''$} a partir de \den{e$'$}:

\begin{exe}
	\ex $\llbracket e'' \rrbracket = \lambda F_{\langle et\rangle}.\lambda G_{\langle et\rangle}.\lambda x_{e}.\ \llbracket e' \rrbracket (F(x))(G(x))$
\end{exe}

\n O procedimento que acabamos de ilustrar pode ser generalizado, tornando possível definir a
extensão de uma conjunção que toma argumentos de tipo $\langle
\alpha,\beta\rangle$ a partir da extensão da conjunção que toma
argumentos de tipo $\beta$. E, se $\beta$ for também um tipo
funcional $\langle\gamma,\delta\rangle$, definir a conjunção que
toma argumentos de tipo $\langle\gamma,\delta\rangle$ a partir da
extensão da conjunção que toma argumentos de tipo $\gamma$, e
assim por diante. Se estivermos lidando apenas com tipos
booleanos, essa cadeia irá fatalmente terminar na conjunção que
coordena sentenças (tipo $t$), a qual podemos tomar como
sendo básica.

Formalizando tudo isso, chegamos à \textsc{definição
recursiva} abaixo, em que $a$ é um tipo booleano:

\begin{exe}
	\ex Versão recursiva da conjunção \textit{e}\\
	$\llbracket e_{\langle a ,\langle
		a,a\rangle\rangle}\rrbracket =
	\begin{cases}
	\lambda p_{t}.\ \lambda q_{t}.\ p=1\ \&\ q=1\ & \text{se } a =
	t\\
	\lambda F_{\langle b,c\rangle}.\lambda G_{\langle b,c\rangle}.\lambda x_{b}.\ \llbracket
	e_{\langle c\langle c,c\rangle\rangle}\rrbracket (F(x))(G(x)) &
	\text{se } a = \langle b,c\rangle
	\end{cases}$
\end{exe}



Definições recursivas são caracterizadas por uma ou mais definições básicas (o caso $a=t$ acima) e por uma receita para reduzir casos complexos (os tipos funcionais, na definição acima) em casos cada vez mais simples até que eventualmente se chegue a um caso básico. 

\section{Negação}

Considere a senten\-ça (\ref{neg}) a seguir, bem como sua estrutura:

\begin{exe}
	\ex João não trabalha.\label{neg}
\end{exe} 

\begin{figure}[H]
	\centerline{ \Tree [ [ [ João ].N ].NP [ não [ [ trabalha ].V ].VP ].VP$'$ ].S } \caption{Negação de VP }
\end{figure}

\n Intuitivamente, uma senten\-ça negativa formada com a palavra
\textit{não} é verdadeira se a senten\-ça afirmativa correspondente
(\textit{João trabalha}, no caso acima) for falsa; e falsa, se a senten\-ça
afirmativa for verdadeira.  No exemplo acima, sabemos que a
extensão de VP é uma fun\-ção que leva um indivíduo \textit{x}
no valor de verdade 1 se \textit{x} trabalha, ou no valor de verdade
0, se \textit{x} não trabalha. O que a nega\-ção parece fazer é
transformar essa fun\-ção em uma outra fun\-ção que, ao receber
\textit{x} como input retorna o valor 1 , se \textit{x} não trabalha,
ou 0, se \textit{x} trabalha. Para captar essa ideia, vamos assumir
a seguinte entrada lexical para a palavra \textit{não}:

\begin{exe}
	\ex \den{não} = $\lambda F_{\langle e,t\rangle }.\ \lambda x_{e}.\ F(\textit{x}) = 0$
\end{exe}

\n Ou, equivalentemente, na notação com a negação de predicados:

\begin{exe}
	\ex \den{não} = $\lambda F_{\langle e,t\rangle }.\ \lambda x_{e}.\ \neg F'(x)$
\end{exe}

\n Essa extensão é uma fun\-ção de tipo
$\langle\langle e,t \rangle,\langle e,t \rangle\rangle$.
Note que o argumento dessa fun\-ção é, ele mesmo, uma outra
fun\-ção. Vamos calcular o significado de (\ref{neg}) e verificar
se a entrada lexical que acabamos de propor fornece resultados
adequados ou não. Desta vez, procederemos de baixo para cima na
estrutura. Já sabemos a extensão de VP:

\begin{exe}
	\ex \den{VP} = \den{V} = \den{trabalha} = $\lambda x_{e}.\ \predica{trabalha}{x}$
\end{exe}

\n Para obter a extensão de VP$'$, podemos usar aplica\-ção
funcional, já que a extensão de \textit{não} é uma fun\-ção cujo domínio
é D$_{\langle e,t\rangle }$, justamente o domínio a que pertence a
extensão de VP$_{1}$.

\begin{exe}
	\ex \den{não} = $\lambda F_{\langle e,t\rangle}.\ \lambda x_{e}.\ \neg F'(x)$
\end{exe}

\begin{exe}
	\ex \den{VP$'$} = \den{não}(\den{VP}) \\
		\den{VP$'$} = ($\lambda F_{\langle e,t\rangle}.\ \lambda x_{e}.\ \neg F'(x)$)(\den{VP})
\end{exe}

\n Como sabemos que a extensão de VP é uma função que retorna o valor 1 se João trabalha
e zero se João não trabalha, temos que:

\begin{exe}
	\ex \den{VP$'$} = $\lambda x_{e}.\ \neg\predica{trabalha}{x}$
\end{exe}

\n Para chegarmos às condi\-çõ\-es de verdade de S, basta
utilizarmos aplica\-ção funcional uma vez mais, já que a extensão
de NP é de tipo \textit{e}, podendo servir como argumento para a
extensão de VP$_{2}$.

\begin{exe}
	\ex \den{NP} = \den{N} = \den{João} = \textit{joão}
\end{exe}

\begin{exe}
	\ex \den{S} = \den{VP$'$}(\den{NP}) \\
		= ($\lambda x_{e}.\ \neg\predica{trabalha}{x}$)(\textit{joão}) \\
		= 1 \textit{sse} $\neg\predica{trabalha}{joão}$
\end{exe}

\n Derivamos, assim, o resultado desejado, confirmando a adequa\-ção
da entrada lexical que propusemos acima para a nega\-ção.

A exemplo do que fizemos no caso da conjun\-ção \textit{e} na se\-ção
anterior, podemos nos perguntar se há casos em que a nega\-ção se
aplica a outros tipos de constituintes que não VPs. Exemplos como
os em (\ref{tru}) abaixo não são convincentes, já que, apesar de a
nega\-ção aparecer junto a verbos transitivos e bitransitivos, eles
podem ser analisados como em (\ref{try}) com a nega\-ção formando
um constituinte com os respectivos VPs:

%xl
\begin{exe}
\ex\label{tru}
\begin{xlist}
\ex João não beijou Maria\label{trua}
\ex João não apresentou Pedro pra Maria.\label{trub}
\end{xlist}
\end{exe}

%xl
\begin{exe}
\ex\label{try}
\begin{xlist}
\ex $[$ João [ não [$_{\text{\textsc{VP}}}$ beijou Maria ]]]\label{trya}
\ex $[$ João [ não [$_{\text{\textsc{VP}}}$ apresentou Pedro pra Maria ]]]\label{tryb}
\end{xlist}
\end{exe}

\n Para esses casos, portanto, basta a entrada lexical que já
propusemos para a nega\-ção. Porém, há casos mais complexos
envolvendo coordena\-ção e nega\-ção em uma mesma senten\-ça que
nos fazem crer na possibilidade de a nega\-ção estar mesmo
formando um constituinte com verbos e não com os respectivos
sintagmas verbais. Considere, por exemplo, as senten\-ças abaixo:

%xl
\begin{exe}
\ex\label{trw}
\begin{xlist}
\ex João não beijou e não abraçou Maria.\label{trwa}
\ex João não mostrou e não apresentou Pedro pra Maria.\label{trwb}
\end{xlist}
\end{exe}

\n Dada a presen\-ça de duas nega\-çõ\-es em cada uma das
senten\-ças acima e dada a maneira como analisamos na se\-ção
anterior suas contrapartes sem as nega\-çõ\-es, parece que estamos,
de fato, diante de casos envolvendo  nega\-ção formando um
constituinte com o verbo.

%xl
\begin{exe}
\ex\label{arg}
\begin{xlist}
\ex $[$ João [[ não beijou] e [não abraçou]] Maria ]\label{arga}
\ex $[$ João [[ não mostrou ] e [ não apresentou ]] Pedro pra Maria ]\label{argb}
\end{xlist}
\end{exe}

\n Se essa for mesmo a análise sintática correta para essas
senten\-ças, nosso sistema precisa ser reformulado, já que a
extensão que foi atribuída à nega\-ção (de tipo $\langle\langle
e,t\rangle,\langle e,t\rangle\rangle$) não permite que essa
combine diretamente com as extensões de verbos transitivos e
bitransitivos (tipos $\langle e,\langle e,t\rangle\rangle$ e
$\langle e,\langle e,\langle e,t\rangle\rangle\rangle$,
respectivamente).

Há ainda casos em que a negação parece se aplicar a constituintes com extensões de tipo $t$ e que também não podem ser analisados com a entrada lexical atual de que dispomos. Como possíveis exemplos, temos sentenças com verbos meteorológicos, sem argumentos de tipo $e$, e alguns casos de inversão entre sujeito e verbo, em que o argumento se junta ao verbo internamente a VP:

%xl
\begin{exe}
\ex\label{jjj}
\begin{xlist}
\ex Não está chovendo.\label{jjja}
\ex $[$ não [ está chovendo ]]\label{jjjb}
\end{xlist}
\end{exe}

%xl
\begin{exe}
\ex\label{grn}
\begin{xlist}
\ex Não chegaram cartas pra mim ontem.\label{grna}
\ex $[$ não [ chegaram cartas pra mim ontem ]]\label{grnb}
\end{xlist}
\end{exe}


\n Para esses casos, precisaríamos de uma entrada de tipo $\langle t,t\rangle$ para a negação. Tal entrada deve inverter o valor de verdade do constituinte que a negação toma como argumento:

\begin{exe}
	\ex \den{não$_{\langle t,t\rangle}$} = $(\lambda p_{t}.\ p=0)$
\end{exe}

\n O papel semântico dessa negação sentencial pode também ser representado como na tabela \ref{tabelaneg}:

\begin{table}
	\begin{tabularx}{.3\textwidth}{XX}
		\lsptoprule
		$\phi$ & não $\phi$ \\
		\midrule 
		1 & \ \ 0 \\  
		0 & \ \ 1 \\
		\lspbottomrule
	\end{tabularx}
	\caption{Tabela de verdade da negação}
	\label{tabelaneg}
\end{table}


\n De acordo com a entrada acima, a negação toma um valor de verdade \textit{p} como argumento e retorna 1 se \textit{p} for igual a 0, e 0 se p for igual a 1. Aplicando-a ao exemplo em (\ref{jjj}), temos:

\begin{exe}
	\ex \den{não está chovendo} = \den{não}(\den{está chovendo})\\
	\den{não está chovendo} = 1 \textit{sse} \den{está chovendo} = 0\\
	\den{não está chovendo} = 1 \textit{sse} não estiver chovendo
\end{exe}

Essa flexibilidade da negação em se associar a constituintes de diversos tipos semânticos é semelhante ao que vimos com a conjunção \textit{e}. Assim, munidos da estratégia que utilizamos nos casos de
coordena\-ção, podemos pensar em associar uma multiplicidade de tipos à
nega\-ção, definidos a partir de um tipo básico, de modo a relacionar
seus diversos usos. O esquema seguinte, construído à maneira do
esquema proposto para a conjun\-ção \textit{e} para tipos booleanos $a$, faz justamente isso:

\begin{exe}
	\ex Versão polimórfica da negação: \\
	\hspace*{18pt}\underline{Caso básico}: Se $a = t$, \\
	\hspace*{30pt} $\llbracket \text{não}_{\langle a,a\rangle}\rrbracket = (\lambda p_{t}.\ p=0)$\\ 
	\hspace*{18pt}\underline{Casos derivados}: Se $a = \langle a_{1},\langle a_{2}, ... ,\langle a_{n},t\rangle\rangle\rangle\ (n\geq 1)$ , \\
	\hspace*{30pt} $\llbracket \text{não}_{\langle a,a\rangle} \rrbracket = \lambda F_{a}.\lambda \gamma_{a_{1}}^{1}.\lambda \gamma_{a_{2}}^{2}...\lambda \gamma_{a_{n}}^{n}.\ \llbracket  \text{não}_{\langle t,t\rangle}\rrbracket(F_{a}(\gamma_{a_{1}}^{1})(\gamma_{a_{2}}^{2})...(\gamma_{a_{n}}^{n}))$
\end{exe}

\n Note que o tipo básico é $\langle t,t\rangle$. Para obtermos nossa entrada lexical original de tipo $\langle et, et\rangle$, basta tomarmos como caso derivado a versão com $n=1$ e $a_{1}=e$, o que resulta em \textit{a} sendo o tipo $\langle e,t\rangle$:

\begin{exe}
	\ex $\llbracket \text{não}_{\langle et,et\rangle} \rrbracket = \lambda F_{\langle e,t\rangle}.\lambda x_{e}.\ \llbracket  \text{não}_{\langle t,t\rangle}\rrbracket(F_{\langle e,t\rangle}(x_{e}))$\\
	$\llbracket \text{não}_{\langle et,et\rangle} \rrbracket = \lambda F_{\langle e,t\rangle}.\lambda x_{e}.\ (\lambda p_{t}.\ p=0)(F(x))$\\
	$\llbracket \text{não}_{\langle et,et\rangle} \rrbracket = \lambda F_{\langle e,t\rangle}.\lambda x_{e}.\ F(x)=0$\\
	$\llbracket \text{não}_{\langle et,et\rangle} \rrbracket = \lambda F_{\langle e,t\rangle}.\lambda x_{e}.\ \neg F'(x)$
\end{exe}

\n De maneira análoga, obtemos os demais tipos, bastando aplicar o esquema acima a eles.  

Também a exemplo do que vimos para a conjunção \textit{e}, podemos formalizar as múltiplas extensões da negação através de uma definição recursiva. Nesse caso, a ideia é que, quando o argumento da negação for de tipo funcional $\langle b,c\rangle$, podemos definir a extensão correspondente $\llbracket \text{não}_{\langle bc,bc\rangle} \rrbracket$ em função da extensão que toma argumentos de tipo $c$ $\llbracket \text{não}_{\langle c,c\rangle} \rrbracket$. Se $c$ for também um tipo funcional, repetimos o procedimento até chegarmos à versão básica de tipo $\langle t,t\rangle$. Lembre-se que estamos lidando com tipos booleanos e que, portanto, esse processo sempre terminará no caso básico. Em termos formais:

\begin{exe}
	\ex Versão recursiva da conjunção \textit{e}\\
	$\llbracket \text{não}_{\langle a,a\rangle}\rrbracket =
	\begin{cases}
	\lambda p_{t}.\ \ p=0 & \text{se } a =
	t\\
	\lambda F_{\langle b,c\rangle}.\lambda x_{b}.\ \llbracket
	\text{não}_{\langle c,c\rangle}\rrbracket (F(x)) &
	\text{se } a = \langle b,c\rangle
	\end{cases}$	
\end{exe}

\n Fica a cargo do leitor como exercício técnico aplicar a definição acima para obter as negações correspondentes a verbos intransitivos, transitivos e bitransitivos.

\bigskip

\begin{tcolorbox}[parbox=false,boxrule=0pt,sharp corners,breakable]

\section*{Sugestões de leitura}
\addcontentsline{toc}{section}{Sugestões de leitura}

\n Para um tratamento lógico da coordenação e da negação vistas como operadores sentenciais, ver \cite{gamut91}, volume 1. Para uma boa introdução à lógica em português, ver \cite{mortari16}. Para discrepâncias entre a interpretação puramente verifuncional que a lógica clássica confere a operadores sentenciais como \textit{e}, \textit{ou}, \textit{se}, \textit{não} e a interpretação e uso de suas contrapartes nas línguas naturais, ver \cite{grice89}, \cite{gamut91}. Para um tratamento flexível da coordenação, consultar \cite{gazdar80} e \cite{parroo83}. Para um estudo erudito e abrangente sobre a negação, ver \cite{horn89}.

\end{tcolorbox}

\bigskip

\begin{tcolorbox}[parbox=false,boxrule=0pt,sharp corners,breakable]


\section*{Exercícios}
\addcontentsline{toc}{section}{Exercícios}

\n\textbf{I.} Aplique a definição recursiva proposta para a conjunção \textit{e} a um caso envolvendo coordenação de verbos bitransitivos, como em \textit{João mostrou e deu o livro para Maria}. Use a extensão resultante e calcule as condições de verdade dessa sentença.\\


\n\textbf{II.} Considere a senten\-ça abaixo, que envolve a
coordena\-ção de três VPs:\\

\n (1)\ \ \ João estudou, dançou e cantou.\\


\n Assuma a seguinte estrutura sintática para essa senten\-ça:

\begin{center}
	\Tree [ [ João ].NP [ [ estudou ].VP$_{1}$ [ [ dançou ].VP$_{2}$ [ [ e ].Coord [ cantou ].VP$_{3}$ ].Coord$'$ ].Coord$''$ ].CoordP ].S
\end{center}

\n Proponha uma entrada lexical adicional para a conjun\-ção \textit{e} que permita ao nosso sistema interpretar a estrutura acima e mostre, passo a passo, como obter suas condi\-çõ\-es de verdade.\\

\n\textbf{III.} Considere a senten\-ça abaixo, que contém duas
nega\-çõ\-es, um tipo de senten\-ça bastante comum no português
falado no Brasil.\\

\n (1)\ \ \ João não trabalha não.\\


\n Assuma para essa senten\-ça a estrutura sintática abaixo:

\begin{center}
	\Tree [ [ [ João ].N ].NP [ [  não  [ [ trabalha ].V ].VP ].VP$'$ não  ].VP$''$ ].S
\end{center}

\n Sua tarefa é derivar as condi\-çõ\-es de verdade que o nosso
sistema prevê para esta estrutura. Você deve proceder passo a
passo indicando sempre quais os princípios composicionais
utilizados. Feito isso, reflita sobre as condi\-çõ\-es de verdade
obtidas e diga se elas fazem jus ou não ao significado dessa
senten\-ça. Se você achar que as condi\-çõ\-es de verdade obtidas
não estão corretas, arrisque um palpite sobre o que pode estar
acontecendo.\\

\n\textbf{IV.} A sentença abaixo apresenta um caso de ambiguidade estrutural.\\

\n (1)\ \ \ João não trabalha e estuda.\\


\n Desenhe as duas estruturas possíveis e as respectivas derivações semânticas, de acordo com o que vimos neste capítulo sobre coordenação e negação.

\end{tcolorbox}
