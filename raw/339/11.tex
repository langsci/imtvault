\chapter[Constraints on clitic climbing in Czech compared to BCS]
        {Constraints on clitic climbing in Czech compared to BCS: Theory and observations}
\label{Constraints on clitic climbing in Czech compared to Bosnian, Croatian and Serbian (theory and observations)}
\section{Introduction}

\textcolor{black}{Since the environments where CC occurs in BCS match with CC constructions in better-studied cCL languages \citep[448]{CavarWilder94}}, we start out with claims concerning constraints on CC in Czech and the very few constraints which can be found in the literature about CC in BCS. We approach the topic in the following steps. First, taking the state of the art predominantly concerning Czech as a point of departure, we try to apply the putative constraints to BCS by looking for similar structures or counterexamples in our database and by querying \{bs,hr,sr\}WaC directly.\footnote{For a detailed description of the database, see Chapter \ref{Introductory remarks to corpus studies on CC}.}\textsuperscript{,}\footnote{For more information on the corpora selected and our argumentation for choosing those and not other corpora, see Section \ref{Clitic climbing in BCS}. For the queries used see Section \ref{Operationalising the constructions in question} and for an exhaustive discussion of our methodological approach see Section \ref{Empirical approach in the current study}.} We focus on structural contexts found both in Czech and in BCS, excluding the usage of CLs in Czech structures which are not attested in BCS.\footnote{Furthermore, we do not elaborate on the difference in the structure of control complements proposed by \citet{Rezac05}. These are not constraints in a narrow sense, since they do not prevent CC per se, but lead to certain semantic and temporal differences when CC does occur. In other words, there are cases in which Czech sentences with and without CC have different semantic interpretations. However, we believe that studying semantic and temporal differences between sentences with and without CC in BCS should be a separate study, which actually cannot be conducted before the syntactic conditions for CC have been described well. That is why we neither report \citet{Rezac05} observations for Czech nor compare them with data from BCS. Readers who are interested in the subject can consult \citet{Rezac05}.} The findings from the literature and our first tentative corpus data are used to formulate further hypotheses regarding possible constraints on CC. These qualitative data are our first source of observation. Second, in order to gain some sort of negative evidence, we examine some of the examples extracted from corpora which were permuted, and then tested by at least five native speakers (informal acceptability judgments).\footnote{For more information on informal acceptability judgments see Section \ref{Introduction:4}.}\textsuperscript{,}\footnote{The most prospective hypotheses based on both sets of data (from corpora and informal acceptability judgments) underwent further rigorous tests that used psycholinguistic methodology: see Section \ref{RQ:16}.} Although at this point we do not address the question of diatopic variation, we look for corresponding data in all three varieties: Bosnian, Croatian, and Serbian. For the sake of brevity, however, we present only one or two examples for one structure. If not stated otherwise, we found examples of corresponding structures in all three languages.

As stated in Section \ref{Empirical orientation} our aim is to give a maximally adequate descriptive account of the variation in CC that we are able to detect in natural data. Following this empirical orientation, we refrain from offering our own theoretical account of the sentence structure and confine ourselves to a list of putative constraints. 

For the time being we propose six basic types of constraints. Their presentation is structured as follows. In Section \ref{Island constraints} we present island constraints, while Section \ref{Constraints related to the raising-control distinction} introduces constraints which are connected to the raising and control dichotomy of CTPs. Next, in Section \ref{Constraints related to mixed clitic clusters} we discuss constraints related to the inner structure of the mixed CL cluster. Constraints connected to the way CLs climb are taken into consideration in Section \ref{How clitics climb} Furthermore, one constraint linked to sentential negation is presented in Section \ref{Sentential negation} Finally, some constraints related to information structure can be found in Section \ref{Infinitive as a whole as the topic of a sentence}. 

We discuss how the range of constraints on CC we found can be accounted for and what the relation between selected constraints is, i.e. whether one constraint can be deduced from another, in Chapter \ref{On the heterogeneous nature of constraints on clitic climbing: complexity effects}.

\section{Island constraints}
\label{Island constraints}
\subsection{Infinitives in comparative sentences with \textit{než}/\textit{nego}}
\label{Infinitives in comparative sentences with nez}

%\footnote{We use the term island as a descriptive label: for an overview of the research literature see \textit{Docekal17} \textit{Ostrov, Slabý ostrov, Silný ostrov} in \textit{Novy Encyklopedycký slovník češtiny on-line}.}

Several authors point out that certain types of phrases seem to defy CC. \citet[245]{FranksKing00} call phrases showing some sort of locality constraint “islands for clitic climbing”. The term goes back to \citet{Ross67}, who is known to have coined a major number of syntactic terms. 

\citet[76]{Junghanns02} observes for Czech that there is no CC from infinitive comparative complements and idiomatic phrases with \textit{než} ‘than’. In example (\ref{(11.108a)}) the pronominal CL \textit{ho} ‘him’ governed by the infinitive \textit{držet} ‘hold’ cannot climb into the matrix clause: compare it with its unacceptable permutation (\ref{(11.108b)}). 

\begin{exe}\ex
\begin{xlist}
\ex[]{\label{(11.108a)}
\gll Je\textsubscript{1} pro tu chvíli snadnější\textsubscript{1} vyměnit\textsubscript{2} děcku plenu, než \textbf{ho}\textsubscript{3} držet\textsubscript{3} na hrnečku a mluvit na ně.\\
be.3\textsc{sg} for this moment easier change.\textsc{inf} baby diaper than him.\textsc{acc} hold.\textsc{inf} on potty and talk to him.\textsc{dat}\\ }
\ex[*]{\label{(11.108b)}
\gll Vyměnit děcku plenu \textbf{ho}\textsubscript{3} je\textsubscript{1} snadnější\textsubscript{1} než držet\textsubscript{3} na hrnečku a mluvit na ně. \\
 change.\textsc{inf} baby diaper  him.\textsc{acc}  be.3\textsc{sg} easier than hold.\textsc{inf} on potty and talk to him.\textsc{dat}\\ }
\end{xlist}
\glt ‘For the moment it is easier to change the baby’s diaper than to hold him on the potty and talk to him.’
\hfill (Cz; \citealt[76]{Junghanns02})
\end{exe}

\noindent Examples similar to the one in (\ref{(11.108a)}) can easily be found in \{bs,hr,sr\}WaC. Their permutations with CC (\ref{(11.109b)})--(\ref{(11.111b)}) are, as expected, unacceptable.

\begin{exe}\ex
\begin{xlist}
\ex[]{\label{(11.109a)}
\gll Nisam\textsubscript{1} imao\textsubscript{1} izbora\textsubscript{1} nego prodati\textsubscript{2} \textbf{ga}\textsubscript{2}. \\
\textsc{neg}.be.1\textsc{sg} have.\textsc{ptcp}.\textsc{sg}.\textsc{m} choice than sell.\textsc{inf} him.\textsc{acc} \\ }
\ex[*]{\label{(11.109b)}
\gll Nisam\textsubscript{1} \textbf{ga}\textsubscript{2} imao\textsubscript{1} izbora\textsubscript{1} nego prodati\textsubscript{2}. \\
 \textsc{neg}.be.1\textsc{sg} him.\textsc{acc} have.\textsc{ptcp}.\textsc{sg}.\textsc{m} choice than sell.\textsc{inf} \\ }
\end{xlist}
\glt ‘I had no choice but to sell him.’
\hfill [bsWaC v1.2]

\ex
\begin{xlist}
\ex[]{\label{(11.110a)}
\gll Što \textbf{sam}\textsubscript{1} drugo mogao\textsubscript{1}, nego poslušati\textsubscript{2} {\textbf{ga}\textsubscript{2} [\dots].} \\
what be.1\textsc{sg} else can.\textsc{ptcp}.\textsc{sg}.\textsc{m} than listen.\textsc{inf} him.\textsc{acc} \\ }
\ex[*]{\label{(11.110b)}
\gll Što \textbf{sam}\textsubscript{1} \textbf{ga}\textsubscript{2} drugo mogao\textsubscript{1}, nego poslušati\textsubscript{2} [\dots].\\
 what be.1\textsc{sg} him.\textsc{acc} else can.\textsc{ptcp}.\textsc{sg}.\textsc{m} than listen.\textsc{inf} \\ }
\end{xlist}
\glt ‘What else could I do but listen to him [\dots].’
\hfill [hrWaC v2.2]

\ex
\begin{xlist}
\ex[]{\label{(11.111a)}
\gll Nema\textsubscript{1} većeg\textsubscript{1} smora\textsubscript{1} i uzaludnijeg\textsubscript{1} procesa\textsubscript{1} nego skupljati\textsubscript{2} \textbf{ih}\textsubscript{2} u {jedno, [\dots].} \\
\textsc{neg}.have.3\textsc{sg} bigger annoyance and more.useless process than collect.\textsc{inf} them.\textsc{acc} in one\\ }
\ex[*]{\label{(11.111b)}
\gll Nema\textsubscript{1} \textbf{ih}\textsubscript{2} većeg\textsubscript{1} smora\textsubscript{1} i uzaludnijeg\textsubscript{1} procesa\textsubscript{1} nego skupljati\textsubscript{2} u {jedno, [\dots].}\\
\textsc{neg}.have.3\textsc{sg} them.\textsc{acc} bigger annoyance and more.useless process than collect.\textsc{inf} in one\\ }
\end{xlist}
\glt `There is no bigger annoyance and more useless process than collecting them into one [...].'
\hfill [srWaC v1.2]
\end{exe}

\noindent These first data suggest that clauses introduced by the comparative particle \textit{než}{\slash}\textit{ne\-go} are islands for CC in both Czech and BCS.

\subsection{Clauses with inflected verbs}
\label{Clauses with inflected verbs}
In Czech, finite clauses are islands which do not permit CC at all, \citet[69]{Junghanns02}, \citet[83]{Dotlacil04}, \citet[7, 9]{Rezac05}, and \citet[103]{Rosen14} unanimously agree. Compare the following Czech example (\ref{(11.1a)}) and its permutation (\ref{(11.1b)}) in which the finite clause is introduced by the complementiser \textit{že} ‘that’.

\begin{exe}\ex
\begin{xlist}
\ex[]{\label{(11.1a)}
\gll Řekl\textsubscript{1}, že \textbf{mi}\textsubscript{3} \textbf{ho}\textsubscript{3} můžete\textsubscript{2} ukázat\textsubscript{3}. \\
say.\textsc{ptcp}.\textsc{sg}.\textsc{m} that me.\textsc{dat} him.\textsc{acc} can.2\textsc{prs} show.\textsc{inf} \\ }
\ex[*]{\label{(11.1b)}
\gll Řekl\textsubscript{1} \textbf{mi}\textsubscript{3} \textbf{ho}\textsubscript{3}, že můžete\textsubscript{2} ukázat\textsubscript{3}. \\
 say.\textsc{ptcp}.\textsc{sg}.\textsc{m} me.\textsc{dat} him.\textsc{acc} that can.2\textsc{prs} show.\textsc{inf} \\ }
\end{xlist}
\glt ‘He said that you could show him to me.’
\hfill (Cz; \citealt[69]{Junghanns02})
\end{exe}


\noindent Similar sentences do not allow CC in BCS either \citep[cf.][448]{CavarWilder94}; compare example (\ref{(11.7a)}) with its unacceptable permutation (\ref{(11.7b)}).


\begin{exe}\ex
\begin{xlist}
\ex[]{\label{(11.7a)}
\gll Rekao\textsubscript{1} \textbf{je}\textsubscript{1} da \textbf{me}\textsubscript{2} voli\textsubscript{2} \\
say.\textsc{ptcp}.\textsc{sg}.\textsc{m} be.3\textsc{sg} that me.\textsc{acc} love.3\textsc{prs} \\ }
\ex[*]{\label{(11.7b)}
\gll Rekao\textsubscript{1} \textbf{me}\textsubscript{2} \textbf{je}\textsubscript{1}, da voli\textsubscript{2}. \\
say.\textsc{ptcp}.\textsc{sg}.\textsc{m} me.\textsc{acc} be.3\textsc{sg} that love.3\textsc{prs} \\ }
\end{xlist}
\glt ‘He said that he loved me.’
\hfill [hrWaC v2.2]
\end{exe}

\noindent However, the constraint needs further investigation in relation to the feature of “finiteness” of the verb in the complement clause. It seems that this could be a major difference between Czech and Serbian, since in both web corpora and literature on BCS we find sentences like the following (\ref{(11.2)}), where CLs climb out of a complement with an inflected verb. 

\protectedex{\begin{exe}\ex\label{(11.2)}
\gll [\dots] ali nešto \textbf{joj}\textsubscript{2} mogu\textsubscript{1} da {prigovorim\textsubscript{2} [\dots].} \\
{} but something her.\textsc{dat} can.1\textsc{prs} that object.1\textsc{prs} \\
\glt ‘[\dots] but something I can find fault with her for [\dots].’
\hfill [srWaC v1.2]
\end{exe}
}

\noindent In example (\ref{(11.2)}) the CL \textit{joj} ‘her’ climbs out of complement \textit{prigovorim} ‘I object’. However, unlike the complement in the Czech example (\ref{(11.1a)}), the Serbian complement \textit{prigovorim} is inflected for person and number, but not for tense. Some of the usages of this \textit{da}\textsubscript{2}-complement will be mentioned in this chapter in Sections~\ref{All-or-nothing constraint} and \ref{Sentential negation}.\footnote{See Section \ref{Types of complements} for basic information on \textit{da}-complement types.} We will discuss CC out of this complement in detail in Chapter \ref{A corpus-based study on CC in da constructions and the raising-control distinction (Serbian)}. 

\subsection{Phrases with gerunds}
\label{Phrases with gerunds}

\citet[70f]{Junghanns02} shows that in Czech there is no CC out of phrases with gerunds (transgressives).\footnote{Due to lack of space we cannot discuss the terms gerund and adverbial participle. Suffice to point out that the Czech forms (transgressives) show agreement whereas the BCS equivalents do not.} He provides the following Czech example (\ref{(11.3a)}), and its permutation (\ref{(11.3b)}) in which the reflexive CL \textit{se} cannot climb out of the phrase with the gerund \textit{opírajíce} ‘leaned’.

\begin{exe}\ex
\begin{xlist}
\ex[]{\label{(11.3a)}
\gll Později oba usnuli\textsubscript{1}, opírajíce\textsubscript{2} \textbf{se}\textsubscript{2} o sebe hlavami.\\
later both fall.asleep.\textsc{ptcp}.\textsc{pl}.\textsc{m} lean.\textsc{ptcp}.\textsc{prs}.\textsc{pl} \textsc{refl} on \textsc{refl} head\\ }
\ex[*]{\label{(11.3b)}
\gll Později \textbf{se}\textsubscript{2} oba usnuli\textsubscript{1}, opírajíce\textsubscript{2} o sebe hlavami.\\
later \textsc{refl} both fall.asleep.\textsc{ptcp}.\textsc{pl}.\textsc{m} lean.\textsc{ptcp}.\textsc{prs}.\textsc{pl} on \textsc{refl} head\\ }
\end{xlist}
\glt ‘Later both of them fell asleep with their heads leaning on each other.’ \\
\strut\hfill (Cz; \citealt[70]{Junghanns02})
\end{exe}


\noindent In BCS gerunds (adverbial participles) are stylistically restricted. Below we adduce examples with the present (\ref{(11.4a)}) and the past (\ref{(11.5a)}) adverbial participle. Examples such as the one presented in (\ref{(11.5a)}) would suggest that a CL governed by an adverbial participle can move away from it, since the CL, in this case the reflexive \textit{se}, is placed to the left of its governor, here the present adverbial participle \textit{žaleći} ‘complaining’. However, according to informal acceptability judgments of permuted examples (\ref{(11.4b)}), (\ref{(11.5b)}), CLs cannot climb from the adverbial participles into the main clause.\footnote{Example (\ref{(11.4a)}) shows that within the adverbial participle phrase the CL does not necessarily follow the verb as claimed by \citet[446f]{CavarWilder94}.}


\begin{exe}\ex
\begin{xlist}
\ex[]{\label{(11.4a)}
\gll [\dots] kako sa zanimanjem razgledavaju\textsubscript{1} eksponate nimalo \textbf{se}\textsubscript{2} ne žaleći\textsubscript{2} {na [\dots].}\\
{} how with interest look.at.3\textsc{prs} exhibits not.at.all \textsc{refl} \textsc{neg} complaining.\textsc{ptcp}.\textsc{adv}.\textsc{prs} on \\ }
\ex[*]{\label{(11.4b)}
\gll [\dots] kako \textbf{se}\textsubscript{2} sa zanimanjem razgledavaju\textsubscript{1} eksponate nimalo ne žaleći\textsubscript{2} {na [\dots].}\\
{} how \textsc{refl} with interest look.at.3\textsc{prs} exhibits not.at.all \textsc{neg} complaining.\textsc{ptcp}.\textsc{adv}.\textsc{prs} on \\ }
\end{xlist}
\glt ‘[\dots] how they look at the exhibits with interest, not complaining at all about [\dots].’
\hfill [hrWaC v2.2]

%solved. I keep the following comment in in case someone wants to do something with it. TBH I have no idea how I solved it.
%I can not fix this to look any better. If I delete the additional \\ in the second \gll, or try to replace them with \glt, or even try to move the } it stops compiling.
%I can not identify the problem, it has to be something in the second \ex; the \gll line has no effect at all.
%I am not touching it anymore, because it breaks when you breathe on it.
%I think it is because he chooses \twosent instead of \threesent in this example:
%Typical LaTeX Error:
%! Argument of \twosent has an extra }.
%<inserted text> 
%                \par 
%l.172 ..., not complaining at all about [\dots].’}
%
%\twosent is in cgloss4e.sty: 
%
%\gdef\twosent#1\\ #2\\{% #1 = first line, #2 = second line
%    \getwords(\lineone,\eachwordone)#1 \\%
%    \getwords(\linetwo,\eachwordtwo)#2 \\%
%    \loop\lastword{\eachwordone}{\lineone}{\wordone}%
%         \lastword{\eachwordtwo}{\linetwo}{\wordtwo}%
%         \global\setbox\gline=\hbox{\unhbox\gline
%                                    \hskip\glossglue
%                                    \vtop{\box\wordone   % vtop was vbox
%                                          \nointerlineskip
%                                          \box\wordtwo
%                                         }%
%                                   }%
%         \testdone
%         \ifnotdone
%    \repeat
%    \egroup % matches \bgroup in \gloss
%   \gl@stop}


\ex
\begin{xlist}
\ex[]{\label{(11.5a)}
\gll Slično \textbf{su}\textsubscript{1} \textbf{mu}\textsubscript{1} ponovili\textsubscript{1} moj sin Senad i učenik M.R. zamolivši\textsubscript{2} \textbf{ga}\textsubscript{2} {da [\dots].} \\
similar be.3\textsc{pl} him.\textsc{dat} repeat.\textsc{ptcp}.\textsc{pl}.\textsc{m} my son Senad and student M.R. ask.\textsc{ptcp}.\textsc{adv}.\textsc{pst} him.\textsc{acc} that\\}
\ex[*]{\label{(11.5b)}
\gll Slično \textbf{su}\textsubscript{1} \textbf{mu}\textsubscript{1} \textbf{ga}\textsubscript{2} ponovili\textsubscript{1} moj sin Senad i učenik M.R. zamolivši\textsubscript{2} {da [\dots].}\\
similar be.3\textsc{pl} him.\textsc{dat} him.\textsc{acc} repeat.\textsc{ptcp}.\textsc{pl}.\textsc{m} my son Senad and student M.R. ask.\textsc{ptcp}.\textsc{adv}.\textsc{pst} that\\}
\end{xlist}
\glt ‘My son Senad and the student M.R. repeated something similar, asking him {to [\dots].’}
\hfill [srWaC v1.2]
\end{exe}

\noindent As we can see from the examples above, there is no doubt that the constraint noticed by \citet{Junghanns02} for Czech is relevant in the case of BCS as well \citep[cf.][447]{CavarWilder94}. Adverbial participles prevent both reflexive and pronominal CLs from climbing.

\subsection{Adjective phrases}
\label{Adjective phrases}

\citet[71]{Junghanns02} points out that adjective phrases lack the feature of finiteness. If an adjective has a CL as a complement, this CL will not be able to climb out of the adjective phrase in which it was generated. This holds at least for adjective phrases in attributive position preceding a noun phrase. Below is \citepossst{Junghanns02} Czech example (\ref{(11.6a)}) and its permutation (\ref{(11.6b)}), in which the reflexive CL \textit{si} cannot climb out of the adjective phrase \textit{neznámý člověk} ‘unknown man’.\footnote{Alexandr
    Rosen (p.c.) warned us that example (\ref{(11.6a)}), from the Czech writer Ludvík Vaculík, sounds very odd and that Vaculík often uses his native Moravian dialect of Czech. According to Rosen, a much better version of the same sentence in standard Czech would be [\dots] \textit{vyšel jsem z telefonní budky jako sobě neznámý člověk}. However, Rosen does not dispute Junghanns’ observation that CLs cannot climb out of adjective phrases and offers a better example for the same constraint:

    \ea
        \ea[]{\label{(footnote1)}
        \gll Ze dvora bylo\textsubscript{1} slyšet\textsubscript{2} křik hrajících\textsubscript{3} \textbf{si}\textsubscript{3} dětí. \\
             from courtyard be.\textsc{ptcp}.\textsc{sg}.\textsc{n} hear.\textsc{inf} scream playing \textsc{refl} children \\ }
        \ex[*]{\label{(footnote2)}
        \gll Ze dvora \textbf{si}\textsubscript{3} bylo\textsubscript{1} slyšet\textsubscript{2} křik hrajících\textsubscript{3} dětí.\\
             from courtyard \textsc{refl} be.\textsc{ptcp}.\textsc{sg}.\textsc{n} hear.\textsc{inf} scream playing children \\ }
        \z
    \z
    \glt ‘You could hear the shouts of children playing in the courtyard.’
}

\begin{exe}
\ex
\begin{xlist}
\ex[]{\label{(11.6a)}
\gll [\dots] vyšel\textsubscript{1} jsem\textsubscript{1} z telefonní budky jako neznámý\textsubscript{2} \textbf{si}\textsubscript{2} člověk. \\
{} go.out.\textsc{ptcp}.\textsc{sg}.\textsc{m} be.1\textsc{sg} out phone booth as unknown \textsc{refl} man \\ }
\ex[*]{\label{(11.6b)}
\gll Vyšel\textsubscript{1} jsem\textsubscript{1} \textbf{si}\textsubscript{2} z telefonní budky jako neznámý\textsubscript{2} člověk.\\
go.out.\textsc{ptcp}.\textsc{sg}.\textsc{m} be.1\textsc{sg} \textsc{refl} out phone booth as unknown  man \\ }
\end{xlist}
\glt ‘[\dots] I came out from the phone booth as a man unknown to myself.’ \\
\strut\hfill (Cz; \citealt[71]{Junghanns02})
\end{exe}

\noindent Our corpus data suggest that the same constraint is found in BCS: see example (\ref{(11.8a)}) and its unacceptable permutation (\ref{(11.8b)}) below.

\begin{exe}\ex
\begin{xlist}
\ex[]{\label{(11.8a)}
\gll [\dots] radim\textsubscript{1} ritmom ponuđenog\textsubscript{2} \textbf{mi}\textsubscript{2} rada\textsubscript{2} za opstanak.\\
{} work.1\textsc{prs} rhythm offered me.\textsc{dat} work for survival \\ }
\ex[*]{\label{(11.8b)}
\gll [\dots] radim\textsubscript{1} \textbf{mi}\textsubscript{2} ritmom ponuđenog\textsubscript{2} rada\textsubscript{2} za opstanak.\\ 
{} work.1\textsc{prs} me.\textsc{dat} rhythm offered work for survival\\ }
\end{xlist}
%\vspace{-\baselineskip}
\glt‘[\dots] I work according to the rhythm of the job offered to me for survival.’ \\
\strut\hfill [bsWaC v1.2]
\end{exe}

\noindent However, as \citet[72]{Junghanns02} points out, adjective phrases in predicate position do allow for the extraction of CLs, like in Czech example (\ref{(11.10)}) in which the dative CL \textit{mu} ‘him’ is placed to the left of its governor \textit{vděčný} ‘grateful’. Nevertheless, we would like to point out that this is not a case of CC sensu stricto because we are dealing with a clearly mono-clausal structure with a single predicative element.

\begin{exe}\ex\label{(11.10)}
\gll Libor \textbf{mu}\textsubscript{1} byl\textsubscript{1} za dotaz v duchu vděčný\textsubscript{1}.\\
Libor him.\textsc{dat} be.\textsc{ptcp}.\textsc{sg}.\textsc{m} for question in spirit grateful \\
\glt ‘In his mind Libor was grateful to him for the question.’ \\
\hfill (Cz; \citealt[72]{Junghanns02})
\end{exe}

\noindent Our preliminary data from \{bs,hr,sr\}WaC suggest that the same holds for BCS; see examples (\ref{(11.11)})--(\ref{(11.13)}). 

\begin{exe}\ex\label{(11.11)}
\gll I još \textbf{ću}\textsubscript{1} \textbf{mu}\textsubscript{1} biti\textsubscript{1} zahvalan\textsubscript{1}.\\
and also \textsc{fut}.1\textsc{sg} him.\textsc{dat} be.\textsc{inf} grateful \\
\glt ‘And I will also be grateful to him.’
\hfill [bsWaC v1.2]

\ex\label{(11.12)}
\gll Sutra \textbf{ćeš}\textsubscript{1} \textbf{mi}\textsubscript{1} biti\textsubscript{1} zahvalna\textsubscript{1}. \\
tomorrow \textsc{fut}.2\textsc{sg} me.\textsc{dat} be.\textsc{inf} grateful \\
\glt ‘Tomorrow you will be grateful to me’
\hfill [hrWaC v2.2]

\ex\label{(11.13)}
\gll Toliko \textbf{sam}\textsubscript{1} \textbf{im}\textsubscript{1} bila\textsubscript{1} dužna\textsubscript{1}.\\
that.much be.1\textsc{sg} them.\textsc{dat} be.\textsc{ptcp}.\textsc{sg}.\textsc{f} in.debt \\
\glt ‘I was so much in their debt.’
\hfill [srWaC v1.2]
\end{exe}

\noindent As is apparent from the previous examples, in Czech and in BCS adjective phrases in predicate position allow CLs to climb.

\subsection{Depth and kind of embeddedness of infinitive phrases}
\label{Depth and kind of embeddedness of infinitive phrases}

In this subsection we will discuss several constraints which depend on the depth and kind of embeddedness. 

\subsubsection{Infinitives as complements of nouns}
\label{Infinitives as complements of nouns}

Infinitives complementing nouns are still a somewhat unclear case.\footnote{\citet[73]{Junghanns02} uses the German term “Funktionsverbgefüge” (light verb construction).} \citet[72]{Junghanns02} argues that normally CC does not occur out of infinitives embedded in a determiner phrase: compare Czech example (\ref{(11.14a)}) and its permutation (\ref{(11.14b)}).\footnote{Alexandr Rosen (p.c.) disagrees with Junghanns’ observation regarding this example: such structures have multiple attestations in the Czech National Corpus, and additionally he as a native speaker finds them acceptable.
	
\begin{exe}\ex\label{(11.16)}
\gll Policisté \textbf{mi} dali neoprávněně botičku, pokutu, neměli\textsubscript{1} \textbf{mě}\textsubscript{2} právo\textsubscript{1} {zastavit\textsubscript{2} [\dots].} \\
 Policemen me.\textsc{dat} give.\textsc{ptcp}.\textsc{pl}.\textsc{m} unjustified ticket fine \textsc{neg}.have.\textsc{ptcp}.\textsc{pl}.\textsc{m} me.\textsc{acc} right stop.\textsc{inf} \\
\glt ‘Policemen gave me an unjustified ticket, a fine, they had no right to stop me [\dots].’  \\
\hfill [Czech National Corpus]

\ex\label{(11.17)}
\gll Neměli\textsubscript{1} \textbf{ho}\textsubscript{2} právo\textsubscript{1} {dát\textsubscript{2} [\dots].} \\
 \textsc{neg}.have.\textsc{ptcp}.\textsc{pl}.\textsc{m} it.\textsc{acc} right give.\textsc{inf} \\
\glt ‘They did not have any right to give it [\dots].’
\hfill [Czech National Corpus]
\end{exe}} However, at the same time he admits that counterexamples can still be found, such as (\ref{(11.15)}) \citep[cf.][73]{Junghanns02}. 
%continuation after footnote above. don't move. no \noindent. We are mid-line.


\begin{exe}\ex
\begin{xlist}
\ex[]{\label{(11.14a)}
\gll Nemám\textsubscript{1} právo\textsubscript{1} \textit{ti}\textsubscript{2} bránit\textsubscript{2}.\\
\textsc{neg}.have.1\textsc{sg} right you.\textsc{dat} restrain.\textsc{inf} \\ }
\ex[*]{\label{(11.14b)}
\gll Nemám\textsubscript{1} \textit{ti}\textsubscript{2} právo\textsubscript{1} bránit\textsubscript{2}. \\
\textsc{neg}.have.1\textsc{sg} you.\textsc{dat} right restrain.\textsc{inf} \\ }
\end{xlist}
\glt ‘I do not have any right to restrain you.’
\hfill (Cz; \citealt[72]{Junghanns02})

\ex\label{(11.15)}
\gll Já jsem\textsubscript{1} \textbf{mu}\textsubscript{2} to neměla\textsubscript{1} čas\textsubscript{1} vysvětlit\textsubscript{2}. \\
 I be.1\textsc{sg} him.\textsc{dat} that \textsc{neg}.have.\textsc{ptcp}.\textsc{sg}.\textsc{f} time explain.\textsc{inf} \\
\glt ‘I did not have time to explain it to him.’
\hfill (Cz; \citealt[73]{Junghanns02})
\end{exe}

\noindent He offers a possible explanation for the discrepancy in the acceptability of examples presented in (\ref{(11.14b)}) and (\ref{(11.15)}). Namely, he suggests that CC is possible only in the context of CTPs in which the verbal part has almost no descriptive content while the nominal part contains substantial descriptive content (\ref{(11.15)}). If, however, both the nominal and the verbal part of the construction contain descriptive content, CC is claimed to be blocked (\ref{(11.14b)}).

\textcolor{black}{Here, we must emphasise that infinitives which are an adjunct or complement to a noun were recognized as general islands for CC in Croatian by \citet[448f]{CavarWilder94} well before \citet{Junghanns02}. However, as corpus data show}, it seems that BCS does allow CC not only with light verb constructions like \textit{imati}/\textit{nemati} \textit{pravo} ‘be right/wrong’, i.e., cases in which only the noun has descriptive content (\ref{(11.18)}), but also with constructions like \textit{pasti} \textit{na} \textit{um}/\textit{pamet} ‘cross one’s mind’ in which both the noun and the verb have descriptive content (\ref{(11.19)}). Compare also sentences with light verb constructions from \{bs,sr\}WaC in (\ref{(11.20a)})--(\ref{(11.21a)}) and their acceptable permutations (\ref{(11.20b)})--(\ref{(11.21b)}). 

\begin{exe}\ex\label{(11.18)}
\gll [\dots] a ti \textbf{ga}\textsubscript{3} imaš\textsubscript{1} pravo\textsubscript{1} odbiti\textsubscript{2} dati\textsubscript{3}. \\
{} and you it.\textsc{acc} have.2\textsc{prs} right refuse.\textsc{inf} give.\textsc{inf} \\
\glt ‘[\dots] and you have the right to refuse to give it.’
\hfill [hrWaC v2.2]

\ex\label{(11.19)}
\gll [\dots] i ne pada\textsubscript{1} \textit{mi}\textsubscript{1} \textit{je}\textsubscript{3} na\textsubscript{1} pamet\textsubscript{1} ići\textsubscript{2} buditi\textsubscript{3}.\\
{} and \textsc{neg} fall.3\textsc{prs} me.\textsc{dat} her.\textsc{acc} on mind go.\textsc{inf} wake.\textsc{inf}\\
\glt ‘[\dots] and it does not cross my mind to go and wake her up.’
\hfill [hrWaC v2.2]


\ex
\begin{xlist}
\ex\label{(11.20a)}
\gll Neki \textbf{su}\textsubscript{1} imali\textsubscript{1} potrebu\textsubscript{1} braniti\textsubscript{2} \textbf{ga}\textsubscript{2} {od [\dots].} \\
some be.1\textsc{pl} have.\textsc{ptcp}.\textsc{pl}.\textsc{m} need defend.\textsc{inf} him.\textsc{acc} from \\
\ex\label{(11.20b)}
\gll Neki \textbf{su}\textsubscript{1} \textbf{ga}\textsubscript{2} imali\textsubscript{1} potrebu\textsubscript{1} braniti\textsubscript{2} {od [\dots].} \\
some be.1\textsc{pl} him.\textsc{acc} have.\textsc{ptcp}.\textsc{pl}.\textsc{m} need defend.\textsc{inf} from \\
\end{xlist}
\glt ‘Some had the need to defend him from [\dots].’
\hfill [bsWaC v1.2]

\ex
\begin{xlist}
\ex\label{(11.21a)}
\gll Naime, mozak nije\textsubscript{1} u\textsubscript{1} stanju\textsubscript{1} prebaciti\textsubscript{2} \textbf{ih}\textsubscript{2} iz kratkoročnog u dugoročno {pamćenje [\dots].} \\
namely brain \textsc{neg}.be.3\textsc{sg} in state switch.\textsc{inf} them.\textsc{acc} from short in long memory \\
\glt 
\ex\label{(11.21b)}
\gll Naime, mozak \textbf{ih}\textsubscript{2} nije\textsubscript{1} u\textsubscript{1} stanju\textsubscript{1} prebaciti\textsubscript{2} iz kratkoročnog u dugoročno {pamćenje [\dots].} \\
namely brain them.\textsc{acc} \textsc{neg}.be.3\textsc{sg} in state switch.\textsc{inf}  from short in long memory \\
\end{xlist}
\glt ‘Namely, the brain is unable to move them from short term to long term memory [\dots].’
\hfill [srWaC v1.2] 
\end{exe}

\noindent As to BCS, our small selection of examples and their permutations seems to contradict Junghanns’ explanation. However, we would like to point out that neither for Czech nor for BCS is it known exactly which light verb constructions, i.e. infinitives as complements of a noun, allow and which block CC. This indicates that CC in the context of infinitives complementing nouns still needs to be investigated both in Czech and in BCS.

\subsubsection{Infinitives as complements of nouns in prepositional phrases}
\label{Infinitives as complements of nouns in prepositional phrases}

A case related to but slightly different from the one mentioned in the previous subsection concerns infinitives which are complements of a noun in a prepositional phrase: see Czech example (\ref{(11.22a)}). In this example the infinitive \textit{přimět} ‘bring’ is a complement of the noun in the prepositional phrase \textit{se} \textit{snahou} ‘with aim’. According to \citet[75]{Junghanns02}, CC is blocked in such cases.


\begin{exe}\ex
\begin{xlist}
\ex[]{\label{(11.22a)}
\gll [\dots] zeptal\textsubscript{1} \textbf{se} \minsp{[} se snahou]\textsubscript{2} přimět\textsubscript{3} \textbf{ho}\textsubscript{3} k odpovědi. \\
{} ask.\textsc{ptcp}.\textsc{sg}.\textsc{m} \textsc{refl} {} with aim bring.\textsc{inf} him.\textsc{acc} to answer \\ }
\ex[*]{\label{(11.22b)}
\gll [\dots] zeptal\textsubscript{1} \textbf{se}\textsubscript{1} \textbf{ho}\textsubscript{3} \minsp{[} se snahou]\textsubscript{2} přimět\textsubscript{3} k odpovědi.\\
{} ask.\textsc{ptcp}.\textsc{sg}.\textsc{m} \textsc{refl} him.\textsc{acc} {} with aim bring.\textsc{inf} to answer \\ }
\end{xlist}
\glt ‘[\dots] he asked, with the aim of getting him to answer.’ \\
\strut\hfill (Cz; \citealt[75]{Junghanns02})
\end{exe}

\noindent Below are similar examples from Bosnian (\ref{(11.23)}) and Croatian (\ref{(11.24)}) web corpora and their permutations, which were not accepted by our informants. 

\begin{exe}\ex\label{(11.23)}
\begin{xlist}
\ex[]{
\gll [\dots] i došao\textsubscript{1} \minsp{[} u situaciju]\textsubscript{2} vratiti\textsubscript{3} \textbf{se}\textsubscript{3} u meč. \\
{} and come.\textsc{ptcp}.\textsc{sg}.\textsc{m} {} in situation return.\textsc{inf} \textsc{refl} in match \\ }
\ex[*]{\label{(11.23b)}
\gll [\dots] i došao\textsubscript{1} \textbf{se}\textsubscript{3} \minsp{[} u situaciju]\textsubscript{2} vratiti\textsubscript{3} u meč. \\
{} and come.\textsc{ptcp}.\textsc{sg}.\textsc{m} \textsc{refl} {} in situation return.\textsc{inf} in match \\ }
\end{xlist}
\glt ‘[\dots] and he was in a position to come back into the match.’ 
\hfill [bsWaC v1.2]

\ex\label{(11.24)}
\begin{xlist}
\ex[]{
\gll [\dots] i pružila\textsubscript{1} ruku \minsp{[} u namjeri]\textsubscript{2} pomilovati\textsubscript{3} \textbf{me}\textsubscript{3} po {obrazu [\dots].}\\
{} and extend.\textsc{ptcp}.\textsc{sg}.\textsc{f} hand {} in intention caress.\textsc{inf} me.\textsc{acc} on cheek \\ }
\ex[*]{\label{(11.24b)}
\gll [\dots] i pružila\textsubscript{1} \textbf{me}\textsubscript{3} ruku \minsp{[} u namjeri]\textsubscript{2} pomilovati\textsubscript{3} po {obrazu [\dots].} \\
{} and extend.\textsc{ptcp}.\textsc{sg}.\textsc{f} me.\textsc{acc} hand {} in intention caress.\textsc{inf} on cheek \\ }
\end{xlist}
\glt ‘[\dots] and reached out an arm intending to caress my cheek [\dots].’ \\
\hfill [hrWaC v2.2]
\end{exe}

\noindent As the example above suggests, it seems that in BCS, just like in Czech, a CL cannot not climb out of an infinitive phrase which is a complement of a noun in a prepositional phrase. It is important to note that although these constructions share some features with the light verb constructions described in Section \ref{Infinitives as complements of nouns}, only the former seem to function as a constraint in BCS.

\subsubsection{Infinitives as complements of agreeing predicative adjectives}
\label{Infinitives as complements of agreeing predicative adjectives}

\citet[75]{Junghanns02} argues that CLs do not climb out of infinitives embedded in a predicative adjective phrase. In his example presented in (\ref{(11.25a)}) the reflexive CL \textit{se} stays in the embedding of its governor, the infinitive \textit{vyjádřit} ‘express’ which is a complement of the agreeing predicative adjective \textit{schopni} ‘able’. He emphasises that such cases should be strictly distinguished from CL positioning with predicative adjectives like in example (\ref{(11.10)}) given above.

\begin{exe}\ex
\begin{xlist}
\ex[]{\label{(11.25a)}
\gll [\dots] \textbf{jsme}\textsubscript{1} schopni\textsubscript{1} \textbf{se}\textsubscript{2} i k této věci společně vyjádřit\textsubscript{2}? \\
{} be.1\textsc{pl} able \textsc{refl} and to this matter together express.\textsc{inf}\\ }
\ex[*]{\label{(11.25b)}
\gll \textbf{Jsme}\textsubscript{1} \textbf{se}\textsubscript{2} schopni\textsubscript{1} i k této věci společně vyjádřit\textsubscript{2}?\\
 be.1\textsc{pl} \textsc{refl} able and to this matter together express.\textsc{inf}\\
}
\end{xlist}
\glt ‘Can we express ourselves together regarding this matter?’ \\
\strut\hfill (Cz; \citealt[75]{Junghanns02})
\end{exe}

\noindent However, \citet[76]{Junghanns02} admits that there are counterexamples to the constraint in question. In the following example (\ref{(11.26)}), the pronominal dative CL \textit{mu} ‘him’ climbs out of the embedded infinitive \textit{říct} ‘say’ in spite of the fact that the latter is a complement of the agreeing predicative adjective \textit{schopen} ‘able’. We would like to point out that in both examples, (\ref{(11.25b)}) and (\ref{(11.26)}), the infinitives are complements of the same agreeing predicative adjective, i.e. \textit{schopen}.

\protectedex{\begin{exe}\ex\label{(11.26)}
\gll Já \textbf{jsem}\textsubscript{1} \textbf{mu}\textsubscript{2} ted’ však nebyla\textsubscript{1} schopna\textsubscript{1} nic říct\textsubscript{2}.\\
I be.1\textsc{sg} him.\textsc{dat} now but \textsc{neg}.be.\textsc{ptcp}.\textsc{sg}.\textsc{f} able nothing say.\textsc{inf}\\
\glt ‘But I was unable to tell him anything.’
\hfill (Cz; \citealt[76]{Junghanns02})
\end{exe}
}

\noindent Junghanns assumes that in this and similar examples, the adjective moves to the verb, where it becomes incorporated. The CL can then be extracted over the V$+$A head \citep[cf.][76]{Junghanns02}. He upholds the claim that in some cases incorporation is not possible, which he supports with the unacceptable example in (\ref{(11.25b)}). However, he admits that the exact conditions of CC in such structures are yet to be clarified. We would like to point out that CL type might be responsible for the difference in the acceptability of examples (\ref{(11.25b)}) and (\ref{(11.26)}). Namely, reflexives might be blocked from climbing out of an infinitive phrase which is a complement of an agreeing predicative adjective (\ref{(11.25b)}), in contrast to pronominal CLs which might not be blocked (\ref{(11.26)}). This would be plausible since \citet[82]{Dotlacil04} shows that in the case of CC out of an infinitive as a complement of a non-agreeing predicative a similar constraint applies only to reflexive CLs (see next section). 

Let us have a look at BCS. Examples (\ref{(11.27)})--(\ref{(11.29)}) extracted from \{bs,hr,sr\}WaC suggest that in BCS pronominal CLs can be extracted out of an infinitive which complements an agreeing predicative.

\largerpage

\begin{exe}\ex\label{(11.27)}
\gll [\dots] i dužan\textsubscript{1} \textbf{ih}\textsubscript{2} \textbf{je}\textsubscript{1} naručiti\textsubscript{2} prilikom prijave putovanja. \\
 {} and obligated them.\textsc{acc} be.3\textsc{sg} order.\textsc{inf} when application travel \\
\glt ‘[\dots] and he is obligated to order them when applying to travel.’ \\
\hfill [bsWaC v1.2]
\ex\label{(11.28)}
\gll Spremni\textsubscript{1} \textbf{smo}\textsubscript{1} \textbf{ti}\textsubscript{2} pomoći\textsubscript{2} u svakoj {situaciji [\dots].} \\
ready be.1\textsc{pl} you.\textsc{dat} help.\textsc{inf} in every situation \\
\glt ‘We are ready to help you in every situation [...].’
\hfill [hrWaC v2.2]

\ex\label{(11.29)}
\gll Pored slijeđenja dužni\textsubscript{1} \textbf{smo}\textsubscript{1} \textbf{mu}\textsubscript{2} pružiti\textsubscript{2} i svoju {ljubav [\dots].}\\
besides following obligated be.1\textsc{pl} him.\textsc{dat} offer.\textsc{inf} and own love\\
\glt ‘Besides allegiance, we are obligated to offer him our love, too [\dots].’ \\
\hfill [srWaC v1.2]
\end{exe}
%Weird empty line, can't see why.

\noindent Furthermore, as our corpus-based examples (\ref{(11.30)})--(\ref{(11.32)}) show, it seems that in contrast to Czech, in BCS the abovementioned restriction does not apply to reflexives. 

\begin{exe}\ex\label{(11.30)}
\gll [\dots] i dužni\textsubscript{1} \textbf{su}\textsubscript{1} \textbf{ga}\textsubscript{2} \textbf{se}\textsubscript{2} pridržavati\textsubscript{2}. \\
{} and obligated be.3\textsc{pl} him.\textsc{gen} \textsc{refl} abide.by.\textsc{inf} \\
\glt ‘[\dots] and they are obligated to abide by it.’
\hfill [bsWaC v1.2]

\ex\label{(11.31)}
\gll Spremna\textsubscript{1} \textbf{sam}\textsubscript{1} \textbf{mu}\textsubscript{2,3} \textbf{se}\textsubscript{2} vratiti\textsubscript{2} i {oprostiti\textsubscript{3} [\dots].} \\
ready be.1\textsc{sg} him.\textsc{dat} \textsc{refl} return.\textsc{inf} and forgive.\textsc{inf} \\
\glt ‘I am ready to return to him and forgive him [\dots].’
\hfill [hrWaC v2.2]

\ex\label{(11.32)}
\gll Jesam\textsubscript{1} \textbf{li} \textbf{se}\textsubscript{2} spreman\textsubscript{1} preseliti\textsubscript{2}? \\
be.1\textsc{sg} \textsc{q} \textsc{refl} ready move.\textsc{inf} \\
\glt ‘Am I ready to move?’
\hfill [srWaC v1.2]
\end{exe}

\noindent This constraint seems to be another difference between CC in BCS and Czech. 

\subsubsection{Infinitives as complements of non-agreeing predicatives}
\label{Infinitives as complements of non-agreeing predicatives}

\citet[77]{Junghanns02} notes that in Czech it is not possible to extract CLs from postponed infinitives complementing non-agreeing predicatives.\footnote{This kind of infinitive complement is labelled as “rechtsextraponierter Subjektsatz” in \citet[77]{Junghanns02} or as “an infinitival clause being a subject” in \citet[82]{Dotlacil04}. We shall refrain from discussing the question whether a complement clause can occupy the position of the subject.} In example (\ref{(11.33a)}) and its permutation (\ref{(11.33b)}) which he provides, the reflexive CL \textit{se} and the pronominal CL \textit{mu} ‘mu’, governed by the embedded infinitive \textit{ukazovat} ‘show’, cannot climb because the latter is a complement of the non-agreeing predicative \textit{vhodné} ‘appropriate’. 

\begin{exe}\ex
\begin{xlist}
\ex[]{\label{(11.33a)}
\gll [\dots] mám pořád pocit že není\textsubscript{1} vhodné\textsubscript{1} ukazovat\textsubscript{2} \textbf{se}\textsubscript{2} \textbf{mu}\textsubscript{2} št’astní. \\
{} have.1\textsc{prs} always feeling that \textsc{neg}.be.3\textsc{sg} appropriate show.\textsc{inf} \textsc{refl} him.\textsc{dat} happy\\ }
\ex[*]{\label{(11.33b)}
\gll [\dots] že \textbf{se}\textsubscript{2} \textbf{mu}\textsubscript{2} není\textsubscript{1} vhodné\textsubscript{1} ukazovat\textsubscript{2} št’astní. \\
 {} that \textsc{refl} him.\textsc{dat} \textsc{neg}.be.3\textsc{sg} appropriate show.\textsc{inf} happy\\ }
\end{xlist}
\glt ‘I still feel it is inappropriate to look happy in front of him.’ \\
\strut\hfill (Cz; \citealt[77]{Junghanns02})
\end{exe}

\noindent \citet[82]{Dotlacil04} later examined this constraint in more detail and refined Junghanns’ statement. He claims that it is only reflexive CLs \textit{se} and \textit{si} that are blocked from climbing out of this type of infinitive complement. In contrast, this restriction does not apply to other CLs \citep[cf.][82]{Dotlacil04}. He supports his claims with examples featuring dative (\ref{(11.34)}) and accusative CLs (\ref{(11.35)}) which have climbed out of infinitive complements embedded in non-agreeing predicatives. 

\begin{exe}\ex\label{(11.34)}
\gll Myslím, že \textbf{mu}\textsubscript{2} není\textsubscript{1} možné\textsubscript{1} pomoct\textsubscript{2}. \\
 think.1\textsc{prs} that him.\textsc{dat} \textsc{neg}.be.3\textsc{sg} possible help.\textsc{inf}\\
\glt ‘I think that it is not possible to help him.’
\hfill (Cz; \citealt[82]{Dotlacil04})

\ex\label{(11.35)}
\gll Myslím, že \textbf{tě}\textsubscript{2} / \textbf{ho}\textsubscript{2} není\textsubscript{1} možné\textsubscript{1} touhle zbraní zabít\textsubscript{2}. \\
think.1\textsc{prs} that you.\textsc{acc} {} him.\textsc{acc} \textsc{neg}.be.3\textsc{sg} possible this weapon kill.\textsc{inf}\\
\glt ‘I think that it is not possible to kill you/him with this weapon.’ \\
\hfill (Cz; \citealt[82]{Dotlacil04})
\end{exe}

\noindent In \citeauthor{Junghanns02}' (\citeyear[77]{Junghanns02}) example (\ref{(11.33a)}), there are two CLs in the embedded infinitive, the reflexive CL \textit{se} and the pronominal CL \textit{mu}. Since the permutation with CC results in an unacceptable sentence (\ref{(11.33b)}), Junghanns assumes that no CL can climb out of an infinitive that is a complement of a non-agreeing predicative. In contrast, both of \citet[82]{Dotlacil04} examples have a single CL governed by the embedded infinitive. In this way, he was able to narrow down this specific CC constraint to reflexive CLs only. The reason why Junghanns’ permuted sentence presented in (\ref{(11.33b)}) is unacceptable is possibly the fact that in Czech CC seems to be an all-or-nothing phenomenon (see Section \ref{All-or-nothing constraint} for more details). Hence, in example (\ref{(11.33a)}) the reflexive CL \textit{se} does not climb since it falls under the mentioned restriction and as a consequence the pronominal CL \textit{mu} cannot climb either. 

Browsing the literature on BCS, we came across the example in (\ref{(11.36)}) from \citet[564]{Ridjanovic12} which goes against claims of \citet[77]{Junghanns02} and \citet[82]{Dotlacil04}. Namely, in this example the \textsc{refl\textsubscript{lex}} CL \textit{se} does climb out of the infinitive complement embedded in the non-agreeing adjective \textit{dobro} ‘good’. 

\protectedex{\begin{exe}\ex\label{(11.36)}
\gll Dobro\textsubscript{1} \textbf{se}\textsubscript{2} je\textsubscript{1} nadati\textsubscript{2}. \\
good \textsc{refl} be.3\textsc{sg} hope.\textsc{inf} \\
\glt ‘It is good to hope.’
\hfill (Bs; \citealt[564]{Ridjanovic12})
\end{exe}
}

\noindent Similarly, as example (\ref{(11.37)}) shows, we found examples with climbing of the reflexive CL in the same structure in the Serbian web corpus. Moreover, we found dozens of such examples in hrWaC v2.2. In (\ref{(11.38)}) we provide one of them.

\begin{exe}\ex\label{(11.37)}
\gll Dobro\textsubscript{1} \textbf{se}\textsubscript{2} \textbf{je}\textsubscript{1} raspitati\textsubscript{2} kod drugih, {ali [\dots].} \\
 good \textsc{refl} be.3\textsc{sg} ask.around.\textsc{inf} at others but \\
\glt ‘It is good to ask around, but [\dots].’
\hfill [srWaC v1.2] 

\ex\label{(11.38)}
\gll [\dots] no dobro\textsubscript{1} \textbf{se}\textsubscript{2} \textbf{je}\textsubscript{1} uvjeriti\textsubscript{2} {da [\dots].} \\
 {} but good \textsc{refl} be.3\textsc{sg} convince.\textsc{inf} that \\
\glt ‘[\dots] but it is good to convince yourself that [\dots].’
\hfill [hrWaC v2.2]
\end{exe}

\noindent Moreover, we would like to emphasise that in all three web corpora we found examples with climbing of pronominal CLs out of infinitive complements of non-agreeing predicatives – see (\ref{(11.39)})--(\ref{(11.41)}) below. 

\begin{exe}\ex\label{(11.39)}
\gll Također, dobro\textsubscript{1} \textbf{ga}\textsubscript{2} \textbf{je}\textsubscript{1} konzumirati\textsubscript{2} {za [\dots].} \\
 also good him.\textsc{acc} be.3\textsc{sg} consume.\textsc{inf} for \\
\glt ‘Also, it is good to consume it for [\dots].’
\hfill [bsWaC v1.2]

\ex\label{(11.40)}
\gll [\dots] potrebno\textsubscript{1} \textbf{ih}\textsubscript{3} \textbf{je}\textsubscript{1} pokušati\textsubscript{2} {osnovati\textsubscript{3} [\dots].} \\
 {} necessary them.\textsc{acc} be.3\textsc{sg} try.\textsc{inf} establish.\textsc{inf}\\
\glt ‘[\dots] it is necessary to try to establish them [\dots].’
\hfill [hrWaC v2.2]
\ex\label{(11.41)}
\gll U svakom slučaju, dobro\textsubscript{1} \textbf{ih}\textsubscript{2} \textbf{je}\textsubscript{1} imati\textsubscript{2}. \\
 in every case good them.\textsc{acc} be.3\textsc{sg} have.\textsc{inf} \\
\glt ‘In any case, it is good to have them.’
\hfill [srWaC v1.2]
\end{exe}

\noindent As our corpus data show, it seems that the constraint on CC out of infinitive embeddings of non-agreeing predicatives does not apply to BCS at all. Namely, in these varieties it is possible to extract not only pronominal CLs from infinitives complementing non-agreeing predicatives like in Czech, but also the reflexive CL \textit{se}. 

\subsection{Embedded wh-infinitives}
\label{Embedded wh-infinitives}

\citet[77]{Junghanns02}, \citet[83]{Dotlacil04}, and \citet[8, 9]{Rezac05} argue that although wh-infinitives generally do not present islands for syntactic movements in Czech – for instance, full prepositional phrases can be extracted from them – they do not allow the extraction of CLs. In other words, CC out of interrogative wh-infinitives is not possible.\footnote{A marginal exception poses Modal Existential Construction described by \citet{simik11}, where CC is possible both in Czech and BCS.} \citet[77]{Junghanns02} supports his claims with example (\ref{(11.42a)}) and its unacceptable permutation (\ref{(11.42b)}). From the latter it is clear that the pronominal accusative CL \textit{ho} ‘him’ cannot climb out of wh-infinitive \textit{jak} \textit{zapisovat} ‘how to record’.

\protectedex{\begin{exe}\ex
\begin{xlist}
\ex[]{\label{(11.42a)}
\gll Ale nevím\textsubscript{1} opravdu, jak \textbf{ho}\textsubscript{2} zapisovat\textsubscript{2}. \\
but \textsc{neg}.see.1\textsc{prs} really how him.\textsc{acc} write.down.\textsc{inf} \\ }
\ex[*]{\label{(11.42b)}
\gll Ale nevím\textsubscript{1} \textbf{ho}\textsubscript{2} opravdu, jak zapisovat\textsubscript{2}. \\
but \textsc{neg}.see.1\textsc{prs} him.\textsc{acc} really how write.down.\textsc{inf} \\ }
\end{xlist}
\glt ‘I do not really know how to record him.’
\hfill (Cz; \citealt[77]{Junghanns02})
\end{exe}}

\noindent \citet[3]{Aljovic04} claims that the same constraint exists in BCS; she provides the following example with a \textit{da}-complement and its permutation:

\protectedex{\begin{exe}\ex
\begin{xlist}
\ex[]{\label{(11.43a)}
\gll Ona nije\textsubscript{1}        odlučila\textsubscript{1}                   kako \minsp{(} da  \textbf{li}) da \textbf{mu}\textsubscript{2} pomogne\textsubscript{2} \minsp{(} ili ne).\\
     she \textsc{neg}.be.3\textsc{sg} decide.\textsc{ptcp}.\textsc{sg}.\textsc{f} how  {}        that \textsc{q}  that him.\textsc{dat}            help.3\textsc{prs} {} or not\\ }
\ex[*]{\label{(11.43b)}
\gll Ona \textbf{mu}\textsubscript{2} nije\textsubscript{1} odlučila\textsubscript{1} kako \minsp{(} da \textbf{li}) da pomogne\textsubscript{2} ili ne.\\
she him.\textsc{dat} \textsc{neg}.be.3\textsc{sg} decide.\textsc{ptcp}.\textsc{sg}.\textsc{f} how {} that \textsc{q} that  help.3\textsc{prs} or not\\ }
\end{xlist}
\glt ‘She did not decide how (/whether) to help him (or not)’. \\
\strut\hfill (BCS; \citealt[3]{Aljovic04})
\end{exe}}

\noindent In her subsequent paper \citet[8]{Aljovic05} provides evidence that CC out of wh-infinitives is blocked in BCS, in the form of permuted example (\ref{(11.44b)}).

\protectedex{\begin{exe}\ex
\begin{xlist}
\ex[]{\label{(11.44a)}
\gll Mila \textbf{je}\textsubscript{1} odlučila\textsubscript{1} kome \textbf{ga}\textsubscript{2} preporučiti\textsubscript{2}. \\
Mila be.3\textsc{sg} decide.\textsc{ptcp}.\textsc{sg}.\textsc{f} who him.\textsc{acc} recommend.\textsc{inf} \\ }
\ex[*]{\label{(11.44b)}
\gll Mila \textbf{ga}\textsubscript{2} \textbf{je}\textsubscript{1} odlučila\textsubscript{1} kome preporučiti\textsubscript{2}. \\
 Mila him.\textsc{acc} be.3\textsc{sg} decide.\textsc{ptcp}.\textsc{sg}.\textsc{f} who recommend.\textsc{inf} \\ }
\end{xlist}
\glt ‘Mila decided to whom to recommend it.’
\hfill (BCS; \citealt[8]{Aljovic05})
\end{exe}}

\noindent Our corpus-based examples (\ref{(11.45a)}--\ref{(11.47a)}) and their rejected permutations (\ref{(11.45b)}--\ref{(11.47b)}) confirm that this constraint indeed applies to wh-infinitives in Bosnian, Croatian, and Serbian as claimed by \citet{Aljovic04, Aljovic05}. 

\begin{exe}\ex
\begin{xlist}
\ex[]{\label{(11.45a)}
\gll \minsp{“} NAFAKA” -- bezbroj razloga zašto \textbf{ga}\textsubscript{1} pogledati\textsubscript{1}. \\
{} NAFAKA {} countless reasons why him.\textsc{acc} look.\textsc{inf} \\ }
\ex[*]{\label{(11.45b)}
\gll \minsp{“} NAFAKA” -- bezbroj razloga \textit{ga}\textsubscript{1} zašto pogledati\textsubscript{1}. \\
{} NAFAKA {} countless reasons him.\textsc{acc} why look.\textsc{inf} \\ }
\end{xlist}
\glt ‘“NAFAKA” -- countless reasons to watch it.’
\hfill [bsWaC v1.2]

\ex
\begin{xlist}
\ex[]{\label{(11.46a)}
\gll Imate\textsubscript{1} \textbf{li} ideju kako \textbf{mu}\textsubscript{2} pomoći\textsubscript{2}? \\
have.2\textsc{pl} \textsc{q} idea how him.\textsc{dat} help.\textsc{inf} \\ }
\ex[*]{\label{(11.46b)}
\gll Imate\textsubscript{1} \textbf{li} \textbf{mu}\textsubscript{2} ideju kako pomoći\textsubscript{2}? \\
 have.2\textsc{pl} \textsc{q}  him.\textsc{dat} idea how help.\textsc{inf} \\ }
\end{xlist}
\glt ‘Do you have any idea how to help him?’
\hfill [hrWaC v2.2]
\ex
\begin{xlist}
\ex[]{\label{(11.47a)}
\gll [\dots] i potrebno\textsubscript{1} \textbf{je}\textsubscript{1} znati\textsubscript{2} kako \textbf{ga}\textsubscript{3} proceniti\textsubscript{3}. \\
{} and necessary be.3\textsc{sg} know.\textsc{inf} how him.\textsc{acc} estimate.\textsc{inf} \\ }
\ex[*]{\label{(11.47b)}
\gll [\dots] i potrebno\textsubscript{1} \textbf{ga}\textsubscript{3} \textbf{je}\textsubscript{1} znati\textsubscript{2} kako proceniti\textsubscript{3}. \\
{} and necessary him.\textsc{acc} be.3\textsc{sg} know.\textsc{inf} how estimate.\textsc{inf} \\ }
\end{xlist}
\glt ‘[\dots] and it is necessary to know how to assess him.’
\hfill [srWaC v1.2]
\end{exe}

\noindent This constraint is one of the clear cases of a lack of CC in both BCS and Czech.

\section{Constraints related to the raising--control distinction}
\label{Constraints related to the raising-control distinction}
\subsection{Object-controlled complements}
\label{Object-controlled complements}
There is an intensive and quite controversial debate on a possible relationship between CC and certain types of control phenomena. Most of the authors working on CC in Czech have pointed out that, unlike in raising and subject control complements, in object-controlled complements CC is highly restricted.\footnote{For more information on the distinction between raising and control predicates see Section \ref{The control vs raising distinction}.}\textsuperscript{,}\footnote{\textcolor{black}{For some interesting examples on restrictions on CC out of infinitive complements of reflexive subject control verbs in Czech, see \citet[159f]{Lenertova04}}.} Claims have been made that CC does not completely depend on the raising--control distinction, but rather on its combination with other features like case, person, animacy, and CL type (pronominal vs reflexive), which will be discussed separately in subsequent sections.

\citet{Thorpe91} and \citet{Junghanns02} argue that in Czech CLs generally cannot climb from object-controlled infinitives, whereas \citet{Rezac99, Rezac05}, \citet{Dotlacil04}, \citet{Lenertova04}, \citet{Hana07}, and \citet{LelandToman76} do not entirely exclude this possibility. The disagreement between scholars becomes even more apparent when they quote examples which their colleagues evaluated as acceptable and mark them either as completely unacceptable (normally with *) or as somewhat questionable (usually with ?). For instance, when arguing that CLs do not climb out of object-controlled infinitives, \citet[69]{Junghanns02} quotes Rezac’s example (\ref{(11.48)}), and marks it with an asterisk, although in Rezac’s text the very same example was evaluated as acceptable.\footnote{Due to lack of space, \textcolor{black}{in examples} in all other chapters we glossed case only for personal pronouns. In this chapter and this section on the raising--control distinction we gloss the case of nominal objects in order to help readers follow the presented discussion.} 

\protectedex{\begin{exe}\ex\label{(11.48)}
\gll Marie \textbf{ho}\textsubscript{2} Petrovi\textsubscript{1} přikázala\textsubscript{1} poslat\textsubscript{2} domů. \\
Marie him.\textsc{acc} Peter.\textsc{dat} order.\textsc{ptcp}.\textsc{sg}.\textsc{f} sent.\textsc{inf} home \\
\glt ‘Marie ordered Peter to send him home.’
\hfill (Cz; \citealt{Rezac99})
\end{exe}
}

\noindent \citet[129]{Hana07}, like \citet{Rezac99, Rezac05}, thinks that the object control constraint does not apply to CC in Czech. He provides the following example (\ref{(11.49)}) in which the pronominal CL \textit{ho} ‘him’ climbs out of the infinitive complement \textit{vyhodit} ‘fire’ of the object control matrix predicate \textit{doporučila} ‘recommended’.

\protectedex{\begin{exe}\ex\label{(11.49)}
\gll [\dots] a když \textbf{ho}\textsubscript{2} i perzonalistika doporučila\textsubscript{1} šéfovi\textsubscript{1} {vyhodit\textsubscript{2} [\dots].} \\
{} and when him.\textsc{acc} and human.resources recommend.\textsc{ptcp}.\textsc{sg}.\textsc{f} boss.\textsc{dat} fire.\textsc{inf} \\
\glt ‘[\dots] and when even personnel management recommended his boss to fire him [\dots].’
\hfill (Cz; \citealt[129]{Hana07})
\end{exe}
}

\noindent Although \citet[4]{Aljovic05} does not use the term subject and object control, she indirectly comments on it when she states that in BCS CC is only possible out of complement clauses whose subject is empty and coreferential with the matrix subject. However, in a footnote she admits that CC is also possible when the subject of the embedded clause is coreferential with the matrix indirect object in the dative \citep[cf.][4]{ Aljovic05}, i.e. out of object control CTPs. She provides example (\ref{(11.50)}) in which the pronominal CL \textit{ih} ‘them’ climbed out of the infinitive \textit{posjetiti} ‘visit’ and clusterised with the dative CL \textit{nam} ‘us’, which is a complement of the object control CTP \textit{brani} ‘(she) forbids’. 

\protectedex{\begin{exe}\ex\label{(11.50)}
\gll Ona \textbf{nam}\textsubscript{1} \textbf{ih}\textsubscript{2} brani\textsubscript{1} posjetiti\textsubscript{2}. \\
she us.\textsc{dat} them.\textsc{acc} forbid.3\textsc{prs} visit.\textsc{inf} \\
\glt ‘She forbids us to visit them.’
\hfill (BCS; \citealt[4]{ Aljovic05})
\end{exe}
}

\noindent Aljović’s example shows that in BCS a dative object in the matrix clause does not necessarily have to block CC out of infinitive complements (cf. for Czech \citealt{LelandToman76}, \citealt{Dotlacil04}, \citealt{Rezac05}, and \citealt{Hana07}). Our corpus data also indicate that pronominal CLs can climb out of object-controlled infinitives: see the example in (\ref{(11.51)}).

\protectedex{\begin{exe}\ex\label{(11.51)}
\gll [\dots] koji \textbf{mi}\textsubscript{1} \textbf{ga}\textsubscript{2} pomažu\textsubscript{1} nositi\textsubscript{2}. \\
{} which me.\textsc{dat} him.\textsc{acc} help.3\textsc{prs} carry.\textsc{inf} \\
\glt ‘[\dots] which help me to carry it.’
\hfill [hrWaC v2.2]
\end{exe}
}

%A detailed discussion of CC out of \textit{da}\textsubscript{2}-complements with respect to the raising and control dichotomy based on empirical evidence can be found in Chapter \ref{A corpus-based study on CC in da constructions and the raising-control distinction (Serbian)}.

\subsection{Object control constraint related to case}
\label{Object control constraint related to case}

\citet{Rezac05} and \citet[79]{Dotlacil04} elaborate further on the control constraint. They note that object control verbs restrict the freedom of CC through constraints which are based on case. \citet[17]{Rezac05} argues that there is a coherent pattern where restructuring is blocked by object control verbs. More specifically, whether a CL climbs or not depends on the one hand on the case of the controller, and on the other hand on the case of the CL governed by the embedded infinitive.\footnote{\textcolor{black}{\citet[17]{Rezac05} argues that in contrast to object control CTPs, raising and subject control CTPs exhibit no case restrictions on CC out of their infinitive complements. In other words, both dative and accusative CLs can climb freely. However, the reader should bear in mind that there could be exceptions to this rule in Czech, for more information see \citeauthor{Lenertova04}'s (\citeyear[159f]{Lenertova04}) examples with CC out of infinitive complements of the reflexive subject control verb \textit{podařit se} ‘manage’.}} Furthermore, \citet[7]{Rezac05} claims that this constraint does not depend on whether the pronoun is in the full or CL form in a given sentence. Accordingly, object control CTPs with a dative controller only allow accusative CLs to climb from the infinitive (\ref{(11.52)}), and block climbing by dative CLs (\ref{(11.53)}).\footnote{Alexandr Rosen (p.c.) argues that CC and the resulting mixed cluster of two dative CLs in the following permuted example is acceptable to him: 
	
\ea\label{(11.53b)}
\gll Dana \textbf{mi}\textsubscript{1} \textbf{mu}\textsubscript{2} přikázala\textsubscript{1} pomoct\textsubscript{2} s mytím. \\
 Dana me.\textsc{dat} him.\textsc{dat} order.\textsc{ptcp}.\textsc{sg}.\textsc{f} help.\textsc{inf} with washing \\
\glt ‘Dana ordered me to help him with the washing.’
\hfill 
\z
} 

Example (\ref{(11.54)}) shows that CC does not occur even if the controller is expressed as a NP in the dative – in this case \textit{Martinovi} ‘(to) Martin’ \citep[cf.][17f]{Rezac05}.\footnote{Alexandr Rosen (p.c.) points out that the string itself is acceptable, as long as \textit{mu} ‘him’ is a matrix complement, like in the permuted example below. Note that in example (\ref{(11.54b)}) it is not the CL \textit{mu} ‘him’ that climbs, but the dative NP \textit{Martinovi} ‘(to) Martin’.

\ea\label{(11.54b)}
\gll Dana \textbf{mu}\textsubscript{1} Martinovi\textsubscript{2} přikázala\textsubscript{1} pomoct\textsubscript{2}. \\
 Dana him.\textsc{dat} Martin.\textsc{dat} order.\textsc{ptcp}.\textsc{sg}.\textsc{f} help.\textsc{inf} \\
\glt ‘Dana ordered him to help Martin.’
\z
}

\begin{exe}\ex\label{(11.52)}
\gll Dana \textbf{ti}\textsubscript{1} \textbf{ho}\textsubscript{2} doporučila\textsubscript{1} {poslat\textsubscript{2} [\dots].} \\
Dana you.\textsc{dat} him.\textsc{acc} recommend.\textsc{ptcp}.\textsc{sg}.\textsc{f} sent.\textsc{inf} \\
\glt ‘Dana recommended that you send him [\dots].’
\hfill (Cz; \citealt[17]{Rezac05})
\ex[*]{\label{(11.53)}
\gll Dana \textbf{jí}\textsubscript{1} \textbf{mu}\textsubscript{2} přikázala\textsubscript{1} pomoct\textsubscript{2} s mytím. \\
Dana her.\textsc{dat} him.\textsc{dat} order.\textsc{ptcp}.\textsc{sg}.\textsc{f} help.\textsc{inf} with washing \\
\glt Intended: ‘Dana ordered her to help him with the washing.’ \\}
\strut\hfill (Cz; \citealt[17]{Rezac05})
\ex[*]{\label{(11.54)}
\gll Dana \textbf{mu}\textsubscript{2} Martinovi\textsubscript{1} přikázala\textsubscript{1} pomoct\textsubscript{2}. \\
Dana him.\textsc{dat} Martin.\textsc{dat} order.\textsc{ptcp}.\textsc{sg}.\textsc{f} help.\textsc{inf} \\
\glt Intended: `Dana ordered Martin to help him.’}
\hfill (Cz; \citealt[18]{Rezac05})
\end{exe}


\noindent In addition, \citet[18]{Rezac05} argues that if there is an accusative controller (CL or NP), CC is even more restricted since it not only blocks the movement of dative CLs (\ref{(11.55)}), but also prevents accusative CLs from climbing (\ref{(11.56)}).

\begin{exe}\ex[*]{\label{(11.55)}
\gll Dana \textbf{jí}\textsubscript{2} \textbf{ho}\textsubscript{1} / Pavla\textsubscript{1} přinutila\textsubscript{1} pomoct\textsubscript{2}. \\
Dana her.\textsc{dat} him.\textsc{acc} {} Pavel.\textsc{acc} force.\textsc{ptcp}.\textsc{sg}.\textsc{f} help.\textsc{inf} \\
\glt Intended: ‘Dana forced him/Paul to help her.’
\hfill (Cz; \citealt[18]{Rezac05})}

\ex[*]{\label{(11.56)}
\gll Dana \textbf{ho}\textsubscript{2} \textbf{tě}\textsubscript{1} / Marii\textsubscript{1} přinutila\textsubscript{1} políbit\textsubscript{2}. \\
Dana him.\textsc{acc} you.\textsc{acc} {} Maria.\textsc{acc} force.\textsc{ptcp}.\textsc{sg}.\textsc{f} kiss.\textsc{inf} \\
\glt Intended: ‘Dana forced you/Maria to kiss him.’
\hfill (Cz; \citealt[18]{Rezac05}) }
\end{exe}

\noindent \textcolor{black}{Here we must warn the reader that the unacceptability of the example in (\ref{(11.55)}) may be related to the ordering restrictions of the CL \textit{ho} ‘him'. According to \citet[153f]{Lenertova04}, the Czech CL \textit{ho} cannot appear initially in a cluster, and CC is not felicitous with pairs that have the preferred inverted order. Moreover, \citet[154]{Lenertova04} provides example (\ref{(28.06.2)}), in which an accusative CL climbs to the matrix in spite of the accusative controller:}\footnote{\textcolor{black}{For more examples of CC in the context of dative and accusative controllers, we refer the reader to \citet[162]{Lenertova04}.} }

\begin{exe}\ex\label{(28.06.2)}
\gll Stejně \textit{ji}\textsubscript{1} \textit{ho}\textsubscript{2} nenechali\textsubscript{1} dokončit\textsubscript{2}. \\
anyway her.\textsc{acc} it.\textsc{acc} \textsc{neg}.let.\textsc{ptcp.pl.m} finish.\textsc{inf} \\
\glt ‘Anyway, they did not let her finish it.’
\hfill (Cz; \citealt[153f]{Lenertova04}) %added language
\end{exe}


\noindent Let us now turn to BCS, which seems to show some variation between the national variants with respect to object control constructions. Regarding CC out of object-controlled infinitives in Croatian, we found examples (\ref{(11.57)})--(\ref{(11.58)}) in which the dative CL complements \textit{nam} ‘us’ and \textit{mi} ‘me’ of the matrix verb, i.e. controller, did not prevent the accusative CLs \textit{ga} ‘him’ and \textit{ju} ‘her’ generated in an infinitive complement from climbing. 

\begin{exe}\ex\label{(11.57)}
\gll [\dots] koji \textbf{nam}\textsubscript{1} \textbf{ga}\textsubscript{2} pomaže\textsubscript{1} {upoznati\textsubscript{2} [\dots].} \\
{} which us.\textsc{dat} him.\textsc{acc} help.1\textsc{prs} meet.\textsc{inf} \\
\glt ‘[\dots] which helps us get to know him [\dots].’
\hfill [hrWaC v2.2]
\ex\label{(11.58)}
\gll [\dots] koji \textbf{su}\textsubscript{1} \textbf{mi}\textsubscript{1} \textbf{ju}\textsubscript{2} pomogli\textsubscript{1} svladati\textsubscript{2}. \\
{} which be.3\textsc{pl} me.\textsc{dat} her.\textsc{acc} help.\textsc{ptcp}.\textsc{pl}.\textsc{m} overcome.\textsc{inf} \\
\glt ‘[\dots] which helped me to overcome her.’
\hfill [hrWaC v2.2]
\end{exe}


\noindent These examples indicate that in Croatian, just like in Czech, object control CTPs with a dative CL complement do not necessarily prevent accusative CLs from climbing out of infinitive embeddings.\footnote{Although these two sentences are not the only two sentences found in hrWaC with an accusative CL that climbs out of an infinitive in spite of a dative CL controller in the matrix clause, in psycholinguistic experiment (see Chapter \ref{Experimental study on constraints on clitic climbing out of infinitive complements}) we could not corroborate this possibility as a general tendency. This might be due to the lack of ecological validity of our stimuli (see Sections \ref{Triangulation of methods} and \ref{Research validity}) or more probably to the fact that in such a context CC is lexically restricted only to certain matrix predicates such as \textit{pomoći} or \textit{pomagati} ‘help’. } It is interesting to note, however, that we could not find such examples in either bsWaC or in srWaC. One reason for this could be that in Bosnian and Serbian \textit{da}\textsubscript{2}-complements predominate with object control CTPs, and not the infinitive.\footnote{For the highly restricted possibility of CC out of \textit{da}\textsubscript{2}-complements see \citet*{JHK17a} and \ref{Discussion:da}.}
%Starting here I do not do a in examples with a and b. Only b from now on --- because sometimes the b follows later on and fixing it takes forever.
Unlike accusative CLs, dative CLs do not seem to climb out of infinitive complements in the presence of a dative controller. Permutation seems to lead to unacceptable sentences: compare example (\ref{(11.59)}) and its unacceptable permutation (\ref{(11.59b)}).\footnote{\citet*[263, 265]{HKJ18} speak, in the context of stacked infinitives, about a same case-different governors constraint, for which they provide empirical evidence. In their example the dative reflexive CL \textit{si} controller blocks climbing of a dative CL generated in the infinitive. Moreover, this tendency was also corroborated in our psycholinguistic experiment, see Section \ref{Object control predicates with a refl2nd CL si controller and with a refl2nd CL se controller}. In other words, the results of both the psycholinguistic experiment and our inspection of corpora are in line: dative CLs do not to climb out of infinitive embeddings in the presence of a dative CL controller in the matrix clause.}

\begin{exe}\ex
\begin{xlist}
\ex[]{\label{(11.59)}
\gll [\dots] da \textbf{nam}\textsubscript{1} pomognete\textsubscript{1} naći\textsubscript{2} \textbf{im}\textsubscript{2} najbolje skrbnike. \\
{} that us.\textsc{dat} help.2\textsc{prs} find.\textsc{inf} them.\textsc{dat} best guardians\\ }
\ex[*]{\label{(11.59b)}
\gll [\dots] da \textbf{nam}\textsubscript{1} \textbf{im}\textsubscript{2} pomognete\textsubscript{1} naći\textsubscript{2} najbolje skrbnike. \\
 {} that us.\textsc{dat} them.\textsc{dat} help.2\textsc{prs} find.\textsc{inf} best guardians\\ }
\end{xlist}
\glt ‘[\dots] that you help us find the best guardians for them.’
\hfill [hrWaC v2.2]
\end{exe}

\noindent Furthermore, it seems that in Croatian, like in Czech, object control CTPs with accusative CL complements block climbing of accusative CLs generated in infinitives, as example (\ref{(11.60)}) and its unacceptable permutation (\ref{(11.60b)}) suggests.

\begin{exe}\ex
\begin{xlist}
\ex[]{\label{(11.60)}
\gll Želim\textsubscript{1} \textbf{te}\textsubscript{2} naučiti\textsubscript{2} voljeti\textsubscript{3} \textbf{me}\textsubscript{3}. \\
want.1\textsc{prs} you.\textsc{acc} teach.\textsc{inf} love.\textsc{inf} me.\textsc{acc} \\ }
\ex[*]{\label{(11.60b)}
\gll Želim\textsubscript{1} \textbf{te}\textsubscript{2} \textbf{me}\textsubscript{3} naučiti\textsubscript{2} voljeti\textsubscript{3}.\\
want.1\textsc{prs} you.\textsc{acc} me.\textsc{acc} teach.\textsc{inf} love.\textsc{inf} \\ }
\end{xlist} 
\glt ‘I want to teach you to love me.’
\hfill [hrWaC v2.2]
\end{exe}

\noindent However, some caution is called for. It may be the case that \textit{me} cannot climb not only due to object control constraint related to case, but also due to phonological similarity (both \textit{me} ‘me’ and \textit{te} ‘you’ end in the vowel [ɛ]) or due to person constraint (see the next Section \ref{Object control person-case constraint} for a detailed discussion).

We would like to emphasise that in bsWaC and srWaC we could not easily find infinitive complements of object control CTPs. Therefore, we conducted a special corpus study, presented in Chapter \ref{A corpus-based study on CC in da constructions and the raising-control distinction (Serbian)}. 

At the end of this subsection we would like to point out once more that scholars differ in their opinion on CC out of object-controlled infinitives in Czech. \citet[18]{Rezac05} claims that not only CL, but also NP accusative complements in the matrix clause block all CC. In contrast, \citet[81]{Dotlacil04} leaves open the possibility that two CLs in the same case will appear in one mixed cluster as a consequence of CC (for more information, see the next Section \ref{Object control person-case constraint}). \citet[123]{Hana07} makes it clear that two morphologically and phonetically identical CLs governed by different governors cannot appear together in the same cluster. However, he does not comment on what will happen if a sentence contains CLs in the same case with different governors which are phonetically different, i.e., if there is a difference in the grammatical category of person. Nevertheless, discussing various examples involving object controlled VPs, \citet[130]{Hana07} concludes that “it seems clear that for non-reflexive clitic a more fine grained distinction of verbs than that based on control is needed.”

\subsection{Object control person-case constraint}
\label{Object control person-case constraint}

It is necessary to add one more observation concerning the object control constraint related to case, made only by \citet{Dotlacil04} for Czech. Although he does not explicitly state that he is drawing on Bonet’s (\citeyear{Bonet91, Bonet94}) person-case constraint (PCC), they clearly have some points in common.\footnote{\textcolor{black}{According to PCC rule} in a combination of a direct object and an indirect object, the direct object has to be in the third person. Both the direct object and the indirect object are phonologically weak.} He argues that in Czech, if the matrix clause has an object, the only CL which can climb is the third person accusative \citep[cf.][79ff]{Dotlacil04}. He illustrates his claims with the acceptable example in (\ref{(11.61)}) and its unacceptable permutation in (\ref{(11.61b)}). In the former, although the matrix clause has an indirect object in the dative \textit{Jirkovi} ‘Jirka’, the third person CLs in the accusative \textit{ho}, \textit{ji}, \textit{je} `him, her, them’ can freely climb from the infinitive embedding \textit{navštěvovat} ‘visit’. In contrast, in the latter example, climbing of the first and second person accusative CLs \textit{mě}, \textit{tě}, \textit{nás}, \textit{vás} ‘me, you, us, you’ leads to unacceptable sentences.

\begin{exe}\ex
\begin{xlist}
\ex[]{\label{(11.61)}
\gll Doktoři \textbf{ho}\textsubscript{2} / \textbf{ji}\textsubscript{2} / \textbf{je}\textsubscript{2} / Jirkovi\textsubscript{1} zakázali\textsubscript{1} navštěvovat\textsubscript{2}.\\
doctors him.\textsc{acc} {} her.\textsc{acc} {} them.\textsc{acc} {} Jirka.\textsc{dat} forbid.\textsc{ptcp}.\textsc{pl}.\textsc{m} visit.\textsc{inf}\\ }
\glt ‘The doctors forbade Jirka to visit him/her/them.’
\ex[*]{\label{(11.61b)}
\gll Doktoři \textbf{mě}\textsubscript{2} / \textbf{tě}\textsubscript{2} / \textbf{nás}\textsubscript{2} / \textbf{vás}\textsubscript{2} Jirkovi\textsubscript{1} zakázali\textsubscript{1} navštěvovat\textsubscript{2}.\\
doctors me.\textsc{acc} {} you.\textsc{acc} {} us.\textsc{acc} {} you.\textsc{acc} Jirka.\textsc{dat} forbid.\textsc{ptcp}.\textsc{pl}.\textsc{m} visit.\textsc{inf}\\ }
\end{xlist}
\glt Intended: ‘The doctors forbade Jirka to visit me/you/us/you.’ \\ 
\strut\hfill (Cz; \citealt[80f]{Dotlacil04})
\end{exe}

\noindent \citet[81]{Dotlacil04} summarises his observations as follows: the first important factor in blocking CC is arguments. However, not all arguments block CC: only objects do \citep[81]{Dotlacil04}. The most powerful factor in preventing CLs from climbing is an accusative object, which blocks climbing of all CLs other than the accusative third person \citep[81]{Dotlacil04}.\footnote{As we pointed out above, \citet[18]{Rezac05} does not agree with this, claiming that “Accusative controllers, clitic or NP, also block the climbing of dative clitics, but in addition block the climbing of accusative clitics as well.” Furthermore, he claims that PCC does not always hold in Czech and that this kind of restriction depends on the dative type \citep[cf.][25]{Rezac05}. PCC holds for the combination of dative and accusative CLs only in the case of argumental, benefactive or possessive dative, while in the case of dative of address the combination of dative and non-third person accusative CL is not excluded \citep[cf.][25]{Rezac05}. However, it is important to emphasise that his example does not have an embedded infinitive complement.}

Here we would like to point out that it is rather difficult to find examples or counterexamples for the object control person-case constraint in BCS. Example (\ref{(11.57)}) presented in Section \ref{Object control constraint related to case} suggests that a third person accusative CL can climb into a matrix clause which contains a dative CL. In order to test whether non-third person CLs can climb, we permuted this example and asked native speakers to perform informal acceptability judgments of (\ref{(11.57b)}) and (\ref{(11.57c)}).

\begin{exe}\ex\begin{xlist}
\ex\label{(11.57b)}
\gll [\dots] koji \textbf{nam}\textsubscript{1} \textbf{te}\textsubscript{2} pomaže\textsubscript{1} upoznati\textsubscript{2}. \\
{} which us.\textsc{dat} you.\textsc{acc} help.1\textsc{prs} meet.\textsc{inf} \\
\glt `[\dots] which helps us get to know you.'
\ex\label{(11.57c)}
\gll [\dots] koji \textbf{joj}\textsubscript{1} \textbf{me}\textsubscript{2} pomaže\textsubscript{1} upoznati\textsubscript{2}.\\
{} which her.\textsc{dat} me.\textsc{acc} help.1\textsc{prs} meet.\textsc{inf} \\
 \glt ‘[\dots] which helps her get to know me.’
\hfill 
\end{xlist}
\end{exe}

\noindent Native speakers of Croatian accepted the permuted sentences in the informal acceptability test. This result indicates that the person constraint does not hold for CC out of object-controlled infinitive complements with a dative controller. However, further testing and empirical, robust data are still necessary.

\subsection{Object control and animacy of the referent of the clitic}
\label{Object control and animacy of the referent of the clitic}
\citet{LelandToman76} show that in Czech, a CL can climb from an infinitive headed by a causative. They claim that if the matrix contains an object, i.e. if the CTP is object control, only inanimate objects can climb from the infinitive complement \citep[cf.][241]{LelandToman76}.\footnote{The example (\ref{(11.62)}) they provide contradicts the constraint proposed by \citet{Rezac05}, presented in Section \ref{Object control constraint related to case}. Note that according to \citet{Rezac05}, any kind of complement in the accusative blocks climbing of an accusative CL complement out of an infinitive, while \citet{LelandToman76} believe that only climbing of CLs with animate referents is blocked.}

They support their claims with (\ref{(11.62)}) and its unacceptable permutation (\ref{(11.62b)}). In the former, the matrix clause has the direct object \textit{Karla} ‘Karel’, but the accusative CL \textit{ji} ‘her/it’ can climb out of the object-controlled infinitive \textit{napsat} ‘write’ since it has an inanimate referent – application. In contrast, in (\ref{(11.62b)}) climbing of the very same CL \textit{ji} ‘her/it’ to the very same matrix leads to an unacceptable sentence, since it has an animate referent – that woman.\footnote{According to \citet[245]{LelandToman76} CC in the context of object control matrix predicates depends wholly on the animacy of the argument of the infinitive complement. In contrast to \citet{Dotlacil04} and \citet{Rezac05}, they argue that even if the controller is a NP in the dative, infinitive accusative CLs with an animate referent cannot climb into the matrix clause \citep[cf.][245]{LelandToman76}. However, we would like to point out that this whole discussion on animacy is a bit vague. Namely, the problem is rather that they do not test the mentioned constraint with changing referents. One may wonder: would different referents change the acceptability?}

\begin{exe}\ex
\begin{xlist}
\ex[]{\label{(11.62)}
\gll Nutili\textsubscript{1} \minsp{[} \textbf{ji}\textsubscript{2}]\textsubscript{animate$-$} Karla\textsubscript{1} napsat\textsubscript{2}. \\
force.\textsc{ptcp}.\textsc{pl}.\textsc{m} {} her.\textsc{acc} Karel.\textsc{acc} write.\textsc{inf} \\ 
\glt ‘They forced Karel to write it (application).’}
\ex[*]{\label{(11.62b)}
\gll Nutili\textsubscript{1} \minsp{[} \textbf{ji}\textsubscript{2}]\textsubscript{animate+} Karla\textsubscript{1} navštívit\textsubscript{2}. \\
force.\textsc{ptcp}.\textsc{pl}.\textsc{m} {} her.\textsc{acc} Karel.\textsc{acc} visit.\textsc{inf} \\ 
\glt Intended: ‘They forced Karel to visit her (that woman).’\\}
\hfill (Cz; \citealt[241]{LelandToman76})
\end{xlist}
\end{exe}


\noindent As to Croatian, besides examples like the one presented in (\ref{(11.64)}) in which the accusative CL which climbed out of the object-controlled infinitive complement has an inanimate referent – company, we found examples like (\ref{(11.65)}) and (\ref{(11.66)}) in which CLs with animate referents climbed as well. From the surrounding context in hrWaC it is clear that \textit{ga} ‘him’ refers to \textit{muž} ‘husband’ and that \textit{ih} ‘they’ refers to \textit{cure} ‘girls’.

\begin{exe}\ex\label{(11.64)}
\gll Pomogla\textsubscript{1} \textbf{nam}\textsubscript{1} \textbf{ih}\textsubscript{2} \textbf{je}\textsubscript{1} tiskati\textsubscript{2} {Dunea [\dots].} \\
help.\textsc{ptcp}.\textsc{sg}.\textsc{f} us.\textsc{dat} them.\textsc{acc} be.3\textsc{sg} print.\textsc{inf} Dunea \\
\glt ‘Dunea helped us to print them [\dots].’
\hfill [hrWaC v2.2]

\ex\label{(11.65)}
\gll Pomoći\textsubscript{1} \textbf{ću}\textsubscript{1} \textbf{vam}\textsubscript{1} \textbf{ga}\textsubscript{2} odnijeti\textsubscript{2} do kola. \\
help.\textsc{inf} \textsc{fut}.1\textsc{sg} you.\textsc{dat} him.\textsc{acc} carry.\textsc{inf} to car \\
\glt ‘I will help you carry him to the car.’
\hfill [hrWaC v2.2]

\ex\label{(11.66)}
\gll Pomoći\textsubscript{1} \textbf{ćemo}\textsubscript{1} \textbf{mu}\textsubscript{1} \textbf{ih}\textsubscript{2} pronaći\textsubscript{2}. \\
help.\textsc{inf} \textsc{fut}.1\textsc{pl} him.\textsc{dat} them.\textsc{acc} find.\textsc{inf} \\
\glt ‘We will help him find them.’
\hfill [hrWaC v2.2]
\end{exe}

\noindent From our examples it seems that in Croatian, at least in the case of object control CTPs with dative CLs, object animacy of a CL referent does not function as a constraint to CC. However, we must admit that it was difficult to find examples of climbing of accusative CLs, irrespective of their animacy status, out of object-controlled infinitives with an accusative CL controller in corpora. Since it is possible that animacy as a factor does play a role in the latter case, we decided to incorporate CLs with animate referents in stimuli for our psycholinguistic experiment, see Chapter \ref{Experimental study on constraints on clitic climbing out of infinitive complements}.

\newpage
\subsection{Object control reflexive constraint}
\label{Object control reflexive constraint}

Regarding the two different control types, \citet[129f]{Hana07} observes that Czech reflexive CLs can climb from subject-controlled (\ref{(11.67)}), but not from object-con\-trolled infinitives (\ref{(11.68b)}).\footnote{Classification of the matrix predicate \textit{potřebovat} ‘need’ as subject control was taken from \citet[130]{Hana07}.}\textsuperscript{,}\footnote{\citet[81]{Dotlacil04} also noted that reflexives cannot climb if there is an object in the matrix clause.}\textsuperscript{,}\footnote{Whether a reflexive climbs or not probably does not depend only on the raising--control distinction of the CTP, but also on the type of the reflexive itself. This is suggested by \citet{LesnerovaMalink08}, who examine the position of the Czech reflexive CL \textit{se} in active and passive sentences with the raising phase verb \textit{přestat} `to stop' and an infinitive complement. Their data suggest that CC is obligatory for passive sentences, i.e. CC is obligatory for the reflexive passive marker \textit{se} in Czech. In contrast, in active structures CC can but does not have to occur \citep[cf.][400f] {LesnerovaMalink08}. According to them, the lack of CC in passive sentences leads to incorrect agentive interpretations of the sentence \citep[cf.][396, 400f] {LesnerovaMalink08}. In contrast, if the free morpheme \textit{se} of agentive reflexive deponent verbs does not climb into the matrix clause with the phase verb \textit{přestat}, the resulting sentence will not be ungrammatical, but will have a marked information structure \citep[cf.][499f] {LesnerovaMalink08}. \textcolor{black}{Some observations on CC of accusative complements to a passivised matrix verb in Czech can be found in \citet[159]{Lenertova04}}.} 

\begin{exe}\ex\label{(11.67)}
\gll Martin \textbf{se}\textsubscript{2} potřebuje\textsubscript{1} zeptat\textsubscript{2} {jak [\dots].} \\
Martin \textsc{refl} need.3\textsc{prs} ask.\textsc{inf} how \\
\glt ‘Martin needs to ask how [\dots].’
\hfill (Cz; \citealt[130]{Hana07})

\ex
\begin{xlist}
\ex[]{\label{(11.68)}
\gll Martin zakázal\textsubscript{1} Petrovi dívat\textsubscript{2} \textbf{se}\textsubscript{2} na televizi. \\
Martin forbid.\textsc{ptcp}.\textsc{sg}.\textsc{m} Peter.\textsc{dat} watch.\textsc{inf} \textsc{refl} on TV \\ }
\ex[*]{\label{(11.68b)}
\gll Martin \textbf{se}\textsubscript{2} zakázal\textsubscript{1} Petrovi dívat\textsubscript{2} na televizi. \\
Martin \textsc{refl} forbid.\textsc{ptcp}.\textsc{sg}.\textsc{m} Peter.\textsc{dat} watch.\textsc{inf} on TV \\ }
\end{xlist}
\glt‘Martin forbade Peter to watch TV.’
\hfill (Cz; \citealt[129]{Hana07})
\end{exe}

\noindent In BCS, like in Czech, a reflexive can climb from raising (\ref{(11.69)}) and subject-controlled infinitives (\ref{(11.70)}), but not from object-controlled infinitives (\ref{(11.71b)}) \citep*[cf.][]{HKJ18}. This finding from the corpus study on stacked infinitives was also corroborated in our psycholinguistic experiment (see Chapter \ref{Experimental study on constraints on clitic climbing out of infinitive complements}).

\begin{exe}\ex\label{(11.69)}
\gll Mogu\textsubscript{1} \textbf{se}\textsubscript{2} samo nasmijati\textsubscript{2} tvojem neznanju. \\
can.1\textsc{prs} \textsc{refl} only laugh.\textsc{inf} your ignorance \\
\glt ‘I can only laugh at your ignorance.’
\hfill [hrWaC v2.2]

\ex\label{(11.70)}
\gll No ne znaju\textsubscript{1} \textbf{se}\textsubscript{2} svi izraziti\textsubscript{2} {lijepo [\dots].} \\
but \textsc{neg} know.3\textsc{prs} \textsc{refl} all express.\textsc{inf} nicely \\
\glt ‘But not everyone knows to express themselves nicely [\dots].’
\hfill [hrWaC v2.2]

\ex
\begin{xlist}
\ex[]{\label{(11.71)}
\gll Pa tko \textbf{im}\textsubscript{1} brani\textsubscript{1} uključiti\textsubscript{2} \textbf{se}\textsubscript{2} u politiku? \\
well who them.\textsc{dat} forbid.3\textsc{prs} enter \textsc{refl} in politics \\ }
\ex[*]{\label{(11.71b)}
\gll Pa tko \textbf{im}\textsubscript{1} \textbf{se}\textsubscript{2} brani\textsubscript{1} uključiti\textsubscript{2} u politiku?\\
well who them.\textsc{dat} \textsc{refl} forbid.3\textsc{prs} enter in politics \\ }
\end{xlist}
\glt ‘Well, who forbids them to enter the politics?’
\hfill [hrWaC v2.2]
\end{exe}

\section{Constraints related to mixed clitic clusters}
\label{Constraints related to mixed clitic clusters}
\subsection{Pseudo-twins}
\label{Pseudo-twins}

Under this section we subsume various observations about constraints on CC in relation to mixed clusters reported by scholars who worked on Czech.\footnote{For the distinction between simple and mixed clusters see Section \ref{Clitic ordering within the cluster}.} In their studies they concentrated on the relationship between two CLs which are generated by different governors and due to different constraints do not end up in a mixed CL cluster. We cover them all under one heading which we named “constraints related to mixed CL clusters”. At this point we would like to emphasise that scholars who reported constraints related to mixed CL clusters did not try to establish a meaningful connection between such clusters and the control phenomena. However, we believe that the constraints related to mixed CL clusters cannot be separated from control constraints.\footnote{We therefore present these constraints directly after the section which was dedicated to the raising and control distinction.} Namely, whenever we analyse two CLs with two different governors, the matrix CTP is of either subject or object control type (see examples in Sections \ref{Phonologically identical pronominal and reflexive clitics with different governors} and \ref{Morphologically different clitics with similar syntactic function and different governors}). \citet[79]{Junghanns02} speaks about what he calls a pseudo-twins constraint on CC, whose nature, however, is not completely clear. This category covers two cases in which a CL does not climb out of a complement:

\begin{enumerate}
\item there is a phonologically identical CL in the matrix clause;
\item the matrix clause does not contain an identical CL, but a CL with a similar syntactic function.
\end{enumerate}

However, \citet[80]{Junghanns02} warns that the constraints for pseudo-twins do not always apply. In a special context and with enough differentiation, the co-occurrence of similar expressions within one sentence is possible \citep[see][]{Lenertova04}.

\subsection{Phonologically identical pronominal and reflexive clitics with different governors}
\label{Phonologically identical pronominal and reflexive clitics with different governors}

\citet[79]{Junghanns02} argues that if the matrix clause and the embedded complement contain (phonologically) identical CL, there will be no CC.\footnote{\citet[79]{Junghanns02} does not specify what the adjective “identical” covers: phonological level, morphological level or both. As will become obvious in the following lines, CLs discussed in this section are always phonologically and sometimes also morphologically identical.} He supports this claim with example (\ref{(11.77)}) and its unacceptable permutation (\ref{(11.77b)}). In the latter, climbing of the reflexive CL \textit{se} leads to formation of a mixed cluster with two reflexive CLs and consequently to an unacceptable sentence.

\begin{exe}\ex
\begin{xlist}
\ex[]{\label{(11.77)}
\gll [\dots] a všude \textbf{jsem}\textsubscript{1} \textbf{se}\textsubscript{1} snažil\textsubscript{1} dozvědět\textsubscript{2} \textbf{se}\textsubscript{2} co {nejvíc [\dots].} \\
{} and everywhere be.1\textsc{sg} \textsc{refl} try.\textsc{ptcp}.\textsc{sg}.\textsc{m} find.out.\textsc{inf} \textsc{refl} what most\\}
\ex[*]{\label{(11.77b)}
\gll Všude \textbf{jsem}\textsubscript{1} \textbf{se}\textsubscript{1} \textbf{se}\textsubscript{2} snažil\textsubscript{1} dozvědět\textsubscript{2} co nejvíc. \\
 everywhere be.1\textsc{sg} \textsc{refl} \textsc{refl} try.\textsc{ptcp}.\textsc{sg}.\textsc{m} find.out.\textsc{inf} what most\\
}
\end{xlist}
\glt ‘[\dots] and I tried to find out as much as possible [\dots].’ \\
\strut\hfill (Cz; \citealt[79]{Junghanns02})
\end{exe}

\noindent \citet[104]{Rosen14} agrees that two reflexive \textit{se} CLs cannot appear in one mixed cluster, and underlies that this constraint is “blind” to different types of reflexives.\footnote{Here we would like to emphasise that Rosen allows two reflexives in the dative case generated by two different verbs either to haplologise within the matrix cluster or to appear next to each other, like in the following example: 

\begin{exe}\ex\label{(11.79)}
\gll Netroufla\textsubscript{1} \textbf{si}\textsubscript{1} | \textbf{si}\textsubscript{2} řict\textsubscript{2} o víc knedlíků. \\
 \textsc{neg}.dare.\textsc{ptcp}.\textsc{sg}.\textsc{f} \textsc{refl} {} \textsc{refl} ask.\textsc{inf} about more dumplings \\
\glt ‘She did not dare to ask for more dumplings.’ 
\hfill (Cz; \citealt[105]{Rosen14})
\end{exe}

\noindent In his words, this is possible if the two occurrences of \textit{si} are prosodically separated (marked above with |).}\textsuperscript{,}\footnote{For our typology of reflexive CLs see Section \ref{Different types of reflexives}.} This is exemplified in (\ref{(11.78)}): although the reflexive CL \textit{se} in the matrix clause is lexically bound (\textsc{refl\textsubscript{lex}}) and the reflexive in the complement blocks the internal argument (\textsc{refl\textsubscript{2nd}}), they cannot appear in the same cluster \citep[cf.][104f]{Rosen14}. 

\begin{exe}\ex
\begin{xlist}
\ex[*]{\label{(11.78)}
\gll Děvče \textbf{se}\textsubscript{1} \textbf{se}\textsubscript{2} stydělo\textsubscript{1} převléknout\textsubscript{2}.\\
 girl \textsc{refl} \textsc{refl} ashame.\textsc{ptcp}.\textsc{sg}.\textsc{n} change.\textsc{inf} \\ }
\ex[]{\label{(11.78b)}
\gll Děvče \textbf{se}\textsubscript{1$+$2} stydělo\textsubscript{1} převléknout\textsubscript{2}.\\
 girl \textsc{refl} ashame.\textsc{ptcp}.\textsc{sg}.\textsc{n} change.\textsc{inf} \\ }
\end{xlist} 
\glt ‘The girl was ashamed of changing (clothes).’
\hfill (Cz; \citealt[104]{Rosen14})
\end{exe}


\noindent While \citet{Junghanns02} offers splitting of reflexives, where each reflexive CL stays with its governor (\ref{(11.77)}), as the only solution in the case of utterances containing two reflexives with different governors, \citet{Rosen14} allows haplology (\ref{(11.78b)}). He treats the deletion of one reflexive as an instance of CC since, in his view, it is the matrix reflexive which is deleted \citep[cf.][106, 114]{Rosen14}.

Although the previous examples contained reflexive CLs, it is important to emphasise that the constraint in question does not concern reflexive CLs only. Rather, it is a rule which applies to all pronominal and reflexive CLs, i.e., to all CLs which can theoretically undergo the process of climbing. Basing on Rosen’s example from Czech (\ref{(11.80)}), \citet[123]{Hana07} formulated this constraint as the following rule: “A clitic cluster cannot contain two morphologically identical clitics with different governors”. In this example, the dative pronominal CL \textit{mi} ‘me’ does not climb from the infinitive embedding \textit{vrátit} ‘return’ into the matrix clause because the morphologically and phonologically identical CL \textit{mi} governed by the matrix predicate \textit{slíbila} ‘promised’ is already there.


\begin{exe}\ex
\begin{xlist}
\ex[]{\label{(11.80)}
\gll Kamila \textbf{mi}\textsubscript{1} slíbila\textsubscript{1} \textbf{mi}\textsubscript{2} \textbf{to}\textsubscript{2} vrátit\textsubscript{2}. \\
Kamila me.\textsc{dat} promise.\textsc{ptcp}.\textsc{sg}.\textsc{f} me.\textsc{dat} it.\textsc{acc} return.\textsc{inf} \\ }
\ex[*]{\label{(11.80b)}
\gll Kamila \textbf{mi}\textsubscript{1} \textbf{mi}\textsubscript{2} \textbf{to}\textsubscript{2} slíbila\textsubscript{1} vrátit\textsubscript{2}.\\
Kamila me.\textsc{dat} me.\textsc{dat} it.\textsc{acc} promise.\textsc{ptcp}.\textsc{sg}.\textsc{f} return.\textsc{inf} \\ }
\end{xlist}
\glt‘Kamila promised me to return it to me.’ \\
\strut\hfill (Cz; \citealt{Rosen01}, as cit. in \citealt[123]{Hana07})
\end{exe}

\noindent Furthermore, we would like to point out that in the three examples provided by Junghanns, Hana and Rosen, CTP predicates are of the subject control type (\textit{snažit se} ‘try hard’, \textit{stydět se} ‘be ashamed’, \textit{slíbit} ‘promise’, \textit{troufnout si} ‘dare’). Although none of these scholars seems to take into account the predicate type as a relevant factor, all three of them recognize the relevance of syntax, since they argue that the constraint is not phonological. All of them support this claim with strong arguments. In Junghanns’ (\citeyear[80]{Junghanns02}) opinion the constraint cannot be phonological in nature since the reflexive CL \textit{se} and the homonymous preposition \textit{se} can stand next to each other. In addition, to refute a purely phonological nature of the constraint, both \citet[124]{Hana07} and \citet[105]{Rosen14} provide examples of the verbal CL \textit{si} ‘are’ and the reflexive CL \textit{si} in contact position.

Querying \{bs,hr,sr\}WaC, we did not find a single occurrence of a mixed cluster containing phonologically and morphologically identical pronominal CLs with different governors. Neither did we find them in contexts of pseudodiaclisis.\footnote{We do not rule out the possibility that such sentences do exist in the queried BCS corpora. However, designing a CQL query which would be precise, would not require excessive post-hoc manual human checking, and at the same time would have a good recall is challenging.} This is in accordance with Hana's (\citeyear[123]{Hana07}) observation made for Czech that “none of the searched corpora contain such a sentence”. Conversely, in web corpora we did find examples of pseudodiaclisis of two reflexive CLs: see (\ref{(11.82)}) and (\ref{(11.84)}). As examples (\ref{(11.83)}) and (\ref{(11.85)}) suggest, haplology of one reflexive is also a possible solution. The reader has to bear in mind that the examples in (\ref{(11.82)}) and (\ref{(11.83)}) on the one hand, like those in (\ref{(11.84)}) and (\ref{(11.85)}) on the other, have identical matrix predicates and infinitive complements. In the former pair these are \textit{truditi se} ‘try’ and \textit{svidjeti se} ‘be liked’, while in the latter they are \textit{bojati se} ‘be afraid’ and \textit{odreći se} ‘give up’. 

\begin{exe}\ex\label{(11.82)}
\gll [\dots] toliko \textbf{se}\textsubscript{1} covjek trudi\textsubscript{1} svidjeti\textsubscript{2} \textbf{se}\textsubscript{2} {drugima [\dots].} \\
{} that.much \textsc{refl} man try.3\textsc{prs} like.\textsc{inf} \textsc{refl} other \\
\glt ‘[\dots] one tries hard to be liked by others [\dots].’
\hfill [bsWaC v1.2]

\ex\label{(11.83)}
\gll [\dots] da \textbf{se}\textsubscript{1$+$2} ne trudi\textsubscript{1} previse svidjeti\textsubscript{2} {drugima [\dots].} \\
{} that \textsc{refl} \textsc{neg} try.3\textsc{prs} too.much like.\textsc{inf} other \\
\glt ‘[\dots] that s/he does not try too much to be liked by others [\dots].’ \\
\hfill [bsWaC v1.2]

\ex\label{(11.84)}
\gll [\dots] koji \textbf{se}\textsubscript{1} boje\textsubscript{1} odreći\textsubscript{2} \textbf{se}\textsubscript{2} {grijeha [\dots].} \\
{} which \textsc{refl} be.afraid.3\textsc{prs} give.up.\textsc{inf} \textsc{refl} sin \\
\glt ‘[\dots] who are afraid to renounce their sins [\dots].’
\hfill [hrWaC v2.2]

\ex\label{(11.85)}
\gll [\dots] jednostavno \textbf{se}\textsubscript{1$+$2} boji\textsubscript{1} odreći\textsubscript{2} studentskog načina života.\\
{} simply \textsc{refl} be.afraid.3\textsc{prs} give.up.\textsc{inf} student way life\\
\glt ‘[\dots] he is simply afraid to give up the student way of living. ’ \\
\hfill [hrWaC v2.2]
\end{exe}

\noindent Example (\ref{(11.86)}) and its acceptable (\ref{(11.86b)}) and unacceptable (\ref{(11.86c)}) permutation suggest that \citet[106, 114]{Rosen14} might be right when he claims that in such structures haplology of reflexives is an instance of CC. Haplology of reflexives is possible only if the pronominal CL \textit{mi} ‘me’ climbs as well: compare (\ref{(11.86b)}) and (\ref{(11.86c)}), where the latter example with haplology of reflexives and without CC of the pronominal \textit{mi} is not acceptable. In the former example the pronominal CL \textit{mi} and the reflexive CL \textit{se} probably climbed together, and the latter CL took the position of the reflexive CL \textit{se} which was already present in the matrix clause. 

\begin{exe}\ex
\begin{xlist}
\ex[]{\label{(11.86)}
\gll Ne trebaju\textsubscript{1} \textbf{se}\textsubscript{2} bojati\textsubscript{2} približiti\textsubscript{3} \textbf{mi}\textsubscript{3} \textbf{se}\textsubscript{3}. \\
\textsc{neg} need.3\textsc{prs} \textsc{refl} be.afraid.\textsc{inf} approach.\textsc{inf} me.\textsc{dat} \textsc{refl} \\}
\ex[]{\label{(11.86b)}
\gll Ne trebaju\textsubscript{1} \textbf{mi}\textsubscript{3} \textbf{se}\textsubscript{$2+3$} bojati\textsubscript{2} približiti. \\
 \textsc{neg} need.3\textsc{prs} me.\textsc{dat} \textsc{refl} be.afraid.\textsc{inf} approach.\textsc{inf} \\}
\ex[*]{\label{(11.86c)}
\gll Ne trebaju\textsubscript{1} \textbf{se}\textsubscript{$2+3$} bojati\textsubscript{2} približiti\textsubscript{3} \textbf{mi}\textsubscript{3}.\\
  \textsc{neg} need.3\textsc{prs} \textsc{refl} be.afraid.\textsc{inf} approach.\textsc{inf}  me.\textsc{dat} \\
}
\end{xlist}
\glt ‘They do not have to fear approaching me.’
\hfill [hrWaC v2.2]
\end{exe}

\noindent Since this constraint concerns not only pronominal but also reflexive CLs, we would like to formulate it slightly more accurately than \citet[123]{Hana07} did, namely: a mixed CL cluster cannot contain two phonologically (and sometimes morphonologically) identical pronominal and reflexive CLs with different governors.

\subsection{Morphologically different clitics with similar syntactic function and different governors}
\label{Morphologically different clitics with similar syntactic function and different governors}

There are cases in which CLs do not necessarily have the same phonological and morphological form, but CC still does not occur. Scholars dedicated the most attention to different reflexives (\textit{se} vs \textit{si}). Based on example (\ref{(11.87)}) with the reflexive CL \textit{se} in the matrix clause and the reflexive CL \textit{si} in the embedding, \citet[79]{Junghanns02} shows that two CLs with similar syntactic functions and different governors block CC.\footnote{He does not explain what exactly is denoted by “similar syntactic function”.}

\begin{exe}\ex\label{(11.87)}
\gll Řekl \textbf{jsem} \textbf{mu}, jak \textbf{jsem}\textsubscript{1} \textit{se}\textsubscript{1} jednou rozhodl\textsubscript{1} trénovat\textsubscript{2} \textbf{si}\textsubscript{2} pamět’. \\
say.\textsc{ptcp}.\textsc{sg}.\textsc{m} be.1\textsc{sg} him.\textsc{dat} how be.1\textsc{sg} \textsc{refl} once decide.\textsc{ptcp}.\textsc{sg}.\textsc{m} train.\textsc{inf} \textsc{refl} memory\\
\glt ‘I told him how I had once decided to train my memory.’ \\
\hfill (Cz; \citealt[80]{Junghanns02})
\end{exe}

\noindent Discussing the same problem, \citet[106]{Rosen14} agrees with Junghanns that two different reflexive CLs cannot appear in the same cluster (\ref{(11.88)}). However, he does not per se rule out the possibility of CC in such structures – CC is possible if the reflexives haplologise \citep[cf.][106]{Rosen14}. In permuted examples (\ref{(11.88b)}) and (\ref{(11.89b)}) with subject control matrix predicates \textit{bát se} ‘be afraid’ and \textit{troufnout si} ‘dare’, the reflexive CLs \textit{si} and \textit{se} climbed into the matrix clause from the embeddings. In contrast to instances of haplology in which matrix reflexives are deleted, according to \citet[106]{Rosen14} deletion of embedded reflexives leads to sentences whose acceptability is questionable, see (\ref{(11.88c)}) and (\ref{(11.89c)}). 

\begin{exe}\ex
\begin{xlist}
\ex[]{\label{(11.88)}
\gll Jan \textbf{se}\textsubscript{1} bál\textsubscript{1} vzít\textsubscript{2} \textbf{si}\textsubscript{2} kravatu. \\
Jan \textsc{refl} be.afraid.\textsc{ptcp}.\textsc{sg}.\textsc{m} take.\textsc{inf} \textsc{refl} tie \\ }
\ex[]{\label{(11.88b)}
\gll Jan \textbf{si}\textsubscript{1$+$2} bál\textsubscript{1} vzít\textsubscript{2} kravatu. \\
Jan \textsc{refl} be.afraid.\textsc{ptcp}.\textsc{sg}.\textsc{m} take.\textsc{inf} tie \\ }
\ex[?]{\label{(11.88c)}
\gll Jan \textbf{se}\textsubscript{1$+$2} bál\textsubscript{1} vzít\textsubscript{2} kravatu. \\
Jan \textsc{refl} be.afraid.\textsc{ptcp}.\textsc{sg}.\textsc{m} take.\textsc{inf} tie \\ }
\end{xlist}
\glt ‘Jan was afraid to put on a tie.’
\hfill (Cz; \citealt[106]{Rosen14})

\ex
\begin{xlist}
\ex[]{\label{(11.89)}
\gll Troufla\textsubscript{1} \textbf{si}\textsubscript{1} usadit\textsubscript{2} \textbf{se}\textsubscript{2} v první řadě. \\
dare.\textsc{ptcp}.\textsc{sg}.\textsc{f} \textsc{refl} sit.\textsc{inf} \textsc{refl} in first row \\ }
\ex[]{\label{(11.89b)}
\gll Troufla\textsubscript{1} \textbf{se}\textsubscript{1$+$2} usadit\textsubscript{2} v první řadě.\\
dare.\textsc{ptcp}.\textsc{sg}.\textsc{f} \textsc{refl} sit.\textsc{inf} in first row \\ }
\ex[?]{\label{(11.89c)}
\gll Troufla\textsubscript{1} \textbf{si}\textsubscript{1$+$2} usadit\textsubscript{2} v první řadě.\\
 dare.\textsc{ptcp}.\textsc{sg}.\textsc{f} \textsc{refl} sit.\textsc{inf} in first row \\ }
\end{xlist}
\glt ‘She dared to sit in the first row.’
\hfill (Cz; \citealt[106]{Rosen14})
\end{exe}

%However, according to \citet[111]{Rosen14} \textsc{refl\textsubscript{lex}} and \textsc{refl\textsubscript{1st+2nd}} behave differently in respect to haplology. Namely, in the case of \textsc{refl\textsubscript{lex}} the preference for haplologised sentences does not depend on the distinction between the \textsc{refl\textsubscript{lex}} CL \textit{se} and \textsc{refl\textsubscript{2nd}} \textit{si}, because native speakers prefer sentences in which embedded reflexives override the reflexives in the matrix clause. In contrast, in the case of the impersonal CLs \textit{se} (\textsc{refl\textsubscript{1st+2nd}}) and the dative reflexive CL \textit{si} (\textsc{refl\textsubscript{2nd}}), CC is only possible if the impersonal \textit{se} overrides another reflexive: compare (\ref{(11.90)}) and its unacceptable permutation (\ref{(11.90b)}) with Rosen’s examples provided above in this section. Unlike in example (\ref{(11.88b)}), the Czech reflexive CL \textit{si} cannot replace the impersonal CL \textit{se} (\ref{(11.90b)}) \citep[cf.][111]{Rosen14}. 
%
%\protectedex{
%\begin{exe}\ex
%\begin{xlist}
%\ex\label{(11.90)}
%\gll Na šéfa \textbf{se}\textsubscript{2} přestalo\textsubscript{1} stěžovat\textsubscript{2}. \\
%on boss \textsc{refl} stop.\textsc{ptcp}.\textsc{sg}.\textsc{n} complain.\textsc{inf} \\
%\ex[*]{\label{(11.90b)}
%\gll Na šéfa \textbf{si}\textsubscript{2} přestalo\textsubscript{1} stěžovat\textsubscript{2}.\\
% \\
%}
%\end{xlist}
%\glt ‘People stopped complaining about the boss.’
%\end{exe}
%\vspace{\vSpaceForLanguageExamples}
%\hfill (Cz; \citealt[111]{Rosen14})
%}

\noindent BCS show some variation with respect to the abovementioned constraint between the three national variants. As presented in Section \ref{Reflexive markers se and si in BCS standard varieties} standard Bosnian and standard Serbian do not recognise the reflexive CL \textit{si}. This notwithstanding, we found examples in which the reflexives \textit{se} and \textit{si} appear in pseudodiaclisis not only in the Croatian, but also in the Bosnian web corpora.\footnote{\textcolor{black}{We do not rule out the possibility that such sentences exist also in srWaC, since the reflexive CL \textit{si} is found in dialects spoken on Serbian territory – see Section \ref{Reflexive clitic si}. However, if they exist in srWaC, they must be rarer than in bsWaC and hrWaC.}} Permutations with CC in Croatian (\ref{(11.91b)}) lead to unacceptable sentences, while permutations with CC in which the embedded reflexive \textit{si} overrides the matrix reflexive CL \textit{se} are marginally possible – see (\ref{(11.91d)}). Permutations with haplology of unlikes and without CC are marginally possible, as are those with haplology of unlikes and with CC – compare (\ref{(11.91c)}) and (\ref{(11.91d)}). 

\begin{exe}\ex
\begin{xlist}
\ex[]{\label{(11.91)}
\gll [\dots] prije nego \textbf{se}\textsubscript{1} odvažimo\textsubscript{1} priuštiti\textsubscript{2} \textbf{si}\textsubscript{2} zeru više života.\\
 {} before than \textsc{refl} dare.1\textsc{prs} afford.\textsc{inf} \textsc{refl} little more life\\ }
\ex[*]{\label{(11.91b)}
\gll [\dots] prije nego \textbf{se}\textsubscript{1} \textbf{si}\textsubscript{2} odvažimo\textsubscript{1} priuštiti\textsubscript{2} zeru više života.\\
 {} before than \textsc{refl} \textsc{refl} dare.1\textsc{prs} afford.\textsc{inf} little more life\\ }
\ex[?]{\label{(11.91c)}
\gll [\dots] prije nego \textbf{se}\textsubscript{1$+$2} odvažimo\textsubscript{1} priuštiti\textsubscript{2} zeru više života.\\
  {} before than \textsc{refl} dare.1\textsc{prs} afford.\textsc{inf} little more life\\ }
\ex[?]{\label{(11.91d)}
\gll [\dots] prije nego \textbf{si}\textsubscript{1$+$2} odvažimo\textsubscript{1} priuštiti\textsubscript{2} zeru više života.\\
  {} before than \textsc{refl} dare.1\textsc{prs} afford.\textsc{inf} little more life\\
}
\end{xlist}
\glt ‘[\dots] before we dare to allow ourselves to live life a little more fully.’ \\
\strut\hfill [hrWaC v2.2]
\end{exe}

\noindent These data indicate that the situation in Bosnian and Croatian is quite similar to the situation in Czech. The only difference is that the examples with haplology of unlikes and with CC (\ref{(11.91d)}) are just as marginally possible as examples with haplology of unlikes and without CC (\ref{(11.91c)}). In addition, examples like the following call into question whether it is possible to apply haplology of unlikes in the case of different reflexives in Bosnian and Croatian.

\begin{exe}\ex
\begin{xlist}
\ex[]{\label{(11.93)}
\gll Dozvoljavam\textsubscript{1} \textbf{si}\textsubscript{1} opteretiti\textsubscript{2} \textbf{se}\textsubscript{2} svim i svačim.\\
 allow.1\textsc{prs} \textsc{refl} burden.\textsc{inf} \textsc{refl} everything and anything\\ }
\ex[*]{\label{(11.93b)}
\gll Dozvoljavam \textbf{si}\textsubscript{1} \textbf{se}\textsubscript{2} opteretiti\textsubscript{2} svim i svačim. \\
 allow.1\textsc{prs} \textsc{refl} \textsc{refl} burden.\textsc{inf} everything and anything\\ }
\ex[*]{\label{(11.93c)}
\gll Dozvoljavam\textsubscript{1} \textbf{si}\textsubscript{1$+$2} opteretiti\textsubscript{2} svim i svačim.\\
  allow.1\textsc{prs} \textsc{refl} burden.\textsc{inf} everything and anything\\ }
\ex[*]{\label{(11.93d)}
\gll Dozvoljavam\textsubscript{1} \textbf{se}\textsubscript{1$+$2} opteretiti\textsubscript{2} svim i svačim.\\
  allow.1\textsc{prs} \textsc{refl} burden.\textsc{inf} everything and anything\\ }
\end{xlist}
\glt ‘I allow myself to burden myself with everything and anything.’ \\
\strut\hfill [hrWaC v2.2]

\ex
\begin{xlist}
\ex[]{\label{(11.94)}
\gll [\dots] pa \textbf{si}\textsubscript{1} dopustimo\textsubscript{1} utopiti\textsubscript{2} \textbf{se}\textsubscript{2} u neke druge. \\
{} so \textsc{refl} allow.1\textsc{prs} drown.\textsc{inf} \textsc{refl} in some others \\ }
\ex[*]{\label{(11.94b)}
\gll [\dots] pa \textbf{si}\textsubscript{1} \textbf{se}\textsubscript{2} dopustimo\textsubscript{1} utopiti\textsubscript{2} u neke druge.\\
 {} so \textsc{refl} \textsc{refl} allow.1\textsc{prs} drown.\textsc{inf} in some others \\ }
\ex[*]{\label{(11.94c)}
\gll [\dots] pa \textbf{si}\textsubscript{1$+$2} dopustimo\textsubscript{1} utopiti\textsubscript{2} u neke druge.\\
 {} so \textsc{refl} allow.1\textsc{prs} drown.\textsc{inf} in some others \\ }
\ex[*]{\label{(11.94d)}
\gll [\dots] pa \textbf{se}\textsubscript{1$+$2} dopustimo\textsubscript{1} utopiti\textsubscript{2} u neke druge.\\
 {} so \textsc{refl} allow.1\textsc{prs} drown.\textsc{inf} in some others \\ }
\end{xlist}
\glt ‘[\dots] so we allow ourselves to drown in other people.’
\hfill [bsWaC v1.2]
\end{exe}

\noindent In Bosnian and Croatian, if the reflexive CL \textit{si} is in the matrix clause and the reflexive CL \textit{se} is in the embedding, the only possible solution is pseudodiaclisis, since neither CC ((\ref{(11.93b)}) and (\ref{(11.94b)})) nor haplology ((\ref{(11.93c)}), (\ref{(11.94c)}), (\ref{(11.93d)}) and (\ref{(11.94d)})) lead to acceptable sentences. This is the major difference between Bosnian and Croatian on the one hand and Czech on the other.

\section{How clitics climb}
\label{How clitics climb}
\subsection{Clitic cannot climb over clitic}
\label{Clitic cannot climb over clitic}

\citet[127]{Hana07} and \citet[102]{Rosen14} claim that in Czech, CC is “monotonic”. This means that a CL can climb to a given cluster only if all CLs with a less embedded governor also climb to that cluster or a higher one ((\ref{(11.95b)}) and (\ref{(11.95c)})). This is because a CL cannot climb over another CL (\ref{(11.95d)}).\footnote{\textcolor{black}{We refer the reader to \citet[153]{Lenertova04} for examples of the lack of CC out of control constructions with CL pairs which would result in inverted CL order. In contrast, according to \citet[153]{Lenertova04} in Czech CC in the context of object control matrix verbs is not problematic as long as it concerns CL pairs which are never used in inverted order.}} 

\begin{exe}\ex
\begin{xlist}
\ex[]{\label{(11.95)}
\gll Všichni \textbf{jsme}\textsubscript{1} \textbf{se}\textsubscript{1} snažili\textsubscript{1} \textbf{mu}\textsubscript{2} pomoci\textsubscript{2} \textbf{ho}\textsubscript{3} najít\textsubscript{3}.\\
all be.1\textsc{pl} \textsc{refl} try.\textsc{ptcp}.\textsc{pl}.\textsc{m} him.\textsc{dat} help.\textsc{inf} him.\textsc{acc} find.\textsc{inf}\\ }
\ex[]{\label{(11.95b)}
\gll Všichni \textbf{jsme}\textsubscript{1} \textbf{se}\textsubscript{1} \textbf{mu}\textsubscript{2} snažili\textsubscript{1} \textbf{ho}\textsubscript{3} pomoci\textsubscript{2} najít\textsubscript{3}.\\
 all be.1\textsc{pl} \textsc{refl}  him.\textsc{dat} try.\textsc{ptcp}.\textsc{pl}.\textsc{m} him.\textsc{acc}  help.\textsc{inf} find.\textsc{inf}\\ }
\ex[]{\label{(11.95c)}
\gll Všichni \textbf{jsme}\textsubscript{1} \textbf{se}\textsubscript{1} \textbf{mu}\textsubscript{2} \textbf{ho}\textsubscript{3} snažili\textsubscript{1} pomoci\textsubscript{2} najít\textsubscript{3}.\\
 all be.1\textsc{pl} \textsc{refl}  him.\textsc{dat} him.\textsc{acc} try.\textsc{ptcp}.\textsc{pl}.\textsc{m}  help.\textsc{inf} find.\textsc{inf}\\ }
\ex[*]{\label{(11.95d)}
\gll Všichni \textbf{jsme}\textsubscript{1} \textbf{se}\textsubscript{1} \textbf{ho}\textsubscript{3} snažili\textsubscript{1} \textbf{mu}\textsubscript{2} pomoci\textsubscript{2} najít\textsubscript{3}.\\
 all be.1\textsc{pl} \textsc{refl} him.\textsc{acc} try.\textsc{ptcp}.\textsc{pl}.\textsc{m} him.\textsc{dat} help.\textsc{inf} find.\textsc{inf}\\ }
\end{xlist}
\glt ‘All of us tried to help him find it.’
\hfill (Cz; \citealt[127]{Hana07})
\end{exe}

\noindent First, we would like to emphasise that in BCS, like in Czech, if all CLs with a less embedded governor climb to a higher cluster, CLs with a more embedded governor can stay in situ. In the following example (\ref{(11.96)}) the less embedded pronominal dative CL \textit{nam} ‘us’ climbed out of the infinitive \textit{pomoći} ‘help’, whereas the more embedded pronominal accusative CL stayed in the embedding of its governor \textit{očuvati} ‘preserve’:

\begin{exe}\ex
\begin{xlist}
\ex[]{\label{(11.96)}
\gll [\dots] kako \textbf{nam}\textsubscript{2} posjetitelji i drugi dionici mogu\textsubscript{1} pomoći\textsubscript{2} očuvati\textsubscript{3} \textbf{ih}\textsubscript{3}.\\
{} how us.\textsc{dat} visitors and other contributors can.3\textsc{prs} help.\textsc{inf} preserve.\textsc{inf} them.\textsc{acc} \\ }
\ex[*]{\label{(11.96b)}
\gll [\dots] kako \textbf{ih}\textsubscript{3} posjetitelji i drugi dionici mogu\textsubscript{1} pomoći\textsubscript{2} \textbf{nam}\textsubscript{2} očuvati\textsubscript{3}.\\
{} how them.\textsc{acc} visitors and other contributors can.3\textsc{prs} help.\textsc{inf} us.\textsc{dat} preserve.\textsc{inf}\\ }
\end{xlist}
\glt ‘[\dots] how visitors and other contributors can help us preserve them.’ \\
\strut\hfill [hrWaC v2.2]
\end{exe}

\noindent Second, like in Czech, CC in BCS is monotonic. As permuted examples (\ref{(11.96b)}), (\ref{(11.97b)}) and (\ref{(11.98b)}) show, if the more embedded CL climbs and the less embedded CL stays in situ, the sentence will be unacceptable.

\begin{exe}\ex
\begin{xlist}
\ex[]{\label{(11.97)}
\gll [\dots] i mi \textbf{ćemo}\textsubscript{1} \textbf{im}\textsubscript{1} pomoći\textsubscript{1} ispuniti\textsubscript{2} \textbf{ga}\textsubscript{2}. \\
{} and we \textsc{fut}.1\textsc{pl} them.\textsc{dat} help.\textsc{inf} fill.out.\textsc{inf} him.\textsc{acc} \\ }
\ex[*]{\label{(11.97b)}
\gll [\dots] i mi \textbf{ćemo}\textsubscript{1} \textbf{ga}\textsubscript{2} pomoći\textsubscript{1} \textbf{im}\textsubscript{1} ispuniti\textsubscript{2} \\
{} and we \textsc{fut}.1\textsc{pl} him.\textsc{acc} help.\textsc{inf} them.\textsc{dat} fill.out.\textsc{inf} \\ }
\end{xlist}
\glt ‘[\dots] and we will help them to fill it out.’
\hfill [bsWaC v1.2]

\ex
\begin{xlist}
\ex[]{\label{(11.98)}
\gll [\dots] kako \textbf{bi}\textsubscript{1} \textbf{im}\textsubscript{1} pomogli\textsubscript{1} rešiti\textsubscript{2} {\textbf{ih}\textsubscript{2} [\dots].} \\
{} how \textsc{cond} them.\textsc{dat} help.\textsc{ptcp}.\textsc{pl}.\textsc{m} solve.\textsc{inf} them.\textsc{acc} \\ }
\ex[*]{\label{(11.98b)}
\gll [\dots] kako \textbf{bi}\textsubscript{1} \textbf{ih}\textsubscript{2} pomogli\textsubscript{1} \textbf{im}\textsubscript{1} rešiti\textsubscript{2}. \\
 {} how \textsc{cond} them.\textsc{acc} help.\textsc{ptcp}.\textsc{pl}.\textsc{m} them.\textsc{dat} solve.\textsc{inf} \\ }
\end{xlist}
\glt ‘[\dots] so that they would help them to solve them [\dots].’
\hfill [srWaC v1.2] 
\end{exe}

\noindent It seems that there are no differences between BCS and Czech with respect to this constraint on CC, described by \citet{Hana07}.

\subsection{All-or-nothing constraint}
\label{All-or-nothing constraint}

\citet[8]{Rezac05} claims that in Czech, if CC takes place it is an all-or-nothing phenomenon, i.e. either all the CLs of an embedded verb undergo CC or none do. Diaclisis of CLs which were generated in the same infinitive is claimed to lead to unacceptable sentences \citep[cf.][8]{Rezac05}. He illustrates this with two permuted examples. In the first (\ref{(11.99b)}), both pronominal CLs \textit{ti} ‘you’ and \textit{ho} ‘him’ generated by the infinitive \textit{ukázat} ‘show’ climbed together to the matrix clause. In the second (\ref{(11.99c)}), only the pronominal CL \textit{ti} climbed, whereas \textit{ho} stayed in the embedding. According to \citet[8]{Rezac05}, only the former is acceptable. 

\begin{exe}\ex\label{rezacsexamples}
\begin{xlist}
\ex[]{\label{(11.99)}
\gll Jana chce\textsubscript{1} ukázat\textsubscript{2} \textbf{ti}\textsubscript{2} \textbf{ho}\textsubscript{2} zejtra. \\
Jana want.3\textsc{prs} show.\textsc{inf} you.\textsc{dat} him.\textsc{acc} tomorrow \\ }
\ex[]{\label{(11.99b)}
\gll Jana \textbf{ti}\textsubscript{2} \textbf{ho}\textsubscript{2} chce\textsubscript{1} ukázat\textsubscript{2} zejtra. \\
 Jana  you.\textsc{dat} him.\textsc{acc} want.3\textsc{prs} show.\textsc{inf} tomorrow \\ }
\ex[*]{\label{(11.99c)}
\gll Jana \textbf{ti}\textsubscript{2} chce\textsubscript{1} ukázat\textsubscript{2} \textbf{ho}\textsubscript{2} zejtra. \\
  Jana  you.\textsc{dat}  want.3\textsc{prs} show.\textsc{inf} him.\textsc{acc} tomorrow \\ }
\end{xlist}
\glt ‘Jana wants to show him to you tomorrow.’
\hfill (Cz; \citealt[8]{Rezac05})
\end{exe}

\noindent Rezac does not provide further evidence; neither were we able to find corresponding hypotheses by other authors. The all-or-nothing constraint in Czech thus remains on rather shaky ground.\footnote{Alexandr Rosen (p.c.) disagrees with Rezac’s claim that CC has to be an all-or-nothing phenomenon. He claims that the following example in which CLs did not climb together is completely acceptable.

\begin{exe}\ex\label{(11.99d)}
Jana \textbf{ti} chce \textbf{ho} ukázat zejtra.
% Jana  you.\textsc{dat}  want.3\textsc{prs} him.\textsc{acc} show.\textsc{inf} tomorrow \\
\hfill 
\end{exe}

Note, however, that this does not have to be an instance of CC; see \citet[67]{Junghanns02}.
%(91d) Jana \textbf{ti} chce \textbf{ho} ukázat zejtra. %%%doesn't work at all. \ref{(11.99d)}
} Furthermore, something that is quite the opposite is claimed to be possible in the case of CC out of \textit{da}\textsubscript{2}-complements in BCS. Namely, \citet[182]{Stjepanovic04} observes that two CLs with the same governor do not have to climb together into the matrix clause. She claims that if CLs split, then the only possibility is that the dative climbs while the accusative stays in the \textit{da}\textsubscript{2}-complement (\ref{(11.100c)}), and not vice versa (\ref{(11.100d)}) \citep[cf.][182]{Stjepanovic04}.

\begin{exe}\ex
\begin{xlist}
\ex[]{\label{(11.100)}
\gll Marija želi\textsubscript{1} da \textbf{mu}\textsubscript{2} \textbf{ga}\textsubscript{2} predstavi\textsubscript{2}. \\
Marija want.3\textsc{prs} that him.\textsc{dat} him.\textsc{acc} introduce.3\textsc{prs} \\ }
\ex[]{\label{(11.100b)}
\gll Marija \textbf{mu}\textsubscript{2} \textbf{ga}\textsubscript{2} želi\textsubscript{1} da predstavi\textsubscript{2}. \\
Marija him.\textsc{dat} him.\textsc{acc} want.3\textsc{prs} that introduce.3\textsc{prs} \\ }
\ex[]{\label{(11.100c)}
\gll Marija \textbf{mu}\textsubscript{2} želi\textsubscript{1} da \textbf{ga}\textsubscript{2} predstavi\textsubscript{2}. \\
Marija him.\textsc{dat} want.3\textsc{prs} that him.\textsc{acc} introduce.3\textsc{prs} \\ }
\ex[*]{\label{(11.100d)}
\gll Marija \textbf{ga}\textsubscript{2} želi\textsubscript{1} da \textbf{mu}\textsubscript{2} predstavi\textsubscript{2}. \\
Marija him.\textsc{acc} want.3\textsc{prs} that him.\textsc{dat} introduce.3\textsc{prs} \\ }
\end{xlist}
\glt ‘Marija wants to introduce him to him.’
\hfill 
\end{exe}

\noindent This is in line with our corpus study on CC out of \textit{da}\textsubscript{2}-complements, where we find another example of two CLs in \textit{da}\textsubscript{2}-complement which do not climb together. For more information on this see Section \ref{Discussion:da}.

%Similarly, \cite[188]{JHK17a} provide an example of two CLs from the \textit{da}\textsubscript{2}-complement which do not climb together. In their example (\ref{(11.101)}) from srWaC the pronominal CL \textit{mi} ‘me’ climbed and the reflexive CL \textit{se} stayed in situ.\footnote{For more information on this see also Section \ref{Discussion:da}.}
%
%\protectedex{\begin{exe}\ex\label{(11.101)}
%\gll [\dots]i počelo\textsubscript{1} \textbf{mi}\textsubscript{2} \textbf{je}\textsubscript{1} da \textbf{se}\textsubscript{2} vrti\textsubscript{2} u glavi. \\
%and start.\textsc{ptcp}.\textsc{sg}.\textsc{n} me.\textsc{dat} be.3\textsc{sg} that \textsc{refl} spin.3\textsc{prs} in head\\
%\glt ‘[\dots] and I started to feel dizzy.’
%\hfill [srWaC v1.2]
%\end{exe}
%}

Furthermore, as our permuted examples show, it seems that the all-or-nothing constraint does not even apply to BCS infinitive complements. CLs generated in the same infinitive can undergo pseudodiaclisis. As long as the dative CL climbs, the sentence will stay acceptable. 

\begin{exe}\ex
\begin{xlist}
\ex\label{(11.102)}
\gll [\dots] a ja nisam\textsubscript{1} imao\textsubscript{1} namjeru\textsubscript{1} mijenjati\textsubscript{2} \textbf{joj}\textsubscript{2} {\textbf{ga}\textsubscript{2} [\dots].} \\
{} and I \textsc{neg}.be.1\textsc{sg} have.\textsc{ptcp}.\textsc{sg}.\textsc{m} intention change.\textsc{inf} her.\textsc{dat} him.\textsc{acc}\\
\ex\label{(11.102b)}
\gll [\dots] a ja \textbf{joj}\textsubscript{2} nisam\textsubscript{1} imao\textsubscript{1} namjeru\textsubscript{1} mijenjati\textsubscript{2} {\textbf{ga}\textsubscript{2} [\dots].} \\
 {} and I her.\textsc{dat} \textsc{neg}.be.1\textsc{sg} have.\textsc{ptcp}.\textsc{sg}.\textsc{m} intention change.\textsc{inf}  him.\textsc{acc}\\
\end{xlist}
\glt ‘[\dots] and I had no intention of changing it for her [\dots].’ 
\hfill [hrWaC v2.2]

\ex
\begin{xlist}
\ex\label{(11.103)}
\gll [\dots] kada \textbf{su}\textsubscript{1} \textbf{joj}\textsubscript{2} \textbf{ih}\textsubscript{2} zbog opasne infekcije stafilokokom morali\textsubscript{1} {izvaditi\textsubscript{2} [\dots].}\\
 {} when be.3\textsc{pl} her.\textsc{dat} them.\textsc{acc} because dangerous infection staphylococcus must.\textsc{ptcp}.\textsc{pl}.\textsc{m} remove.\textsc{inf} \\
\ex\label{(11.103b)}
\gll [\dots] kada \textbf{su}\textsubscript{1} \textbf{joj}\textsubscript{2} zbog opasne infekcije stafilokokom morali\textsubscript{1} izvaditi\textsubscript{2} {\textbf{ih}\textsubscript{2} [\dots].}\\
 {} when be.3\textsc{pl} her.\textsc{dat} because dangerous infection staphylococcus must.\textsc{ptcp}.\textsc{pl}.\textsc{m} remove.\textsc{inf} them.\textsc{acc} \\
\end{xlist}
\glt ‘[\dots] when they had to remove them from her because of a dangerous staphylococcus infection [\dots].’
\hfill [bsWaC v1.2]
\end{exe}

\noindent Thus, it seems that the all-or-nothing constraint on CC, reported for Czech by \citet{Rezac05}, is not relevant for CC in BCS. As BCS examples in this section suggest, one of the two complement CLs can climb while the other can stay in the complement as long as it is the one which comes later in the CL cluster according to the ordering rules.\footnote{For the relative order of CLs in the clitic cluster see Section \ref{Clitic ordering within the cluster}.}

\section{Sentential negation}
\label{Sentential negation}
Sentential negation has not been discussed by scholars who researched CC in Czech, but it has been noticed in the literature on CC in BCS. \footnote{This may be due to the fact that negation seems to allow CC in Czech (Alexandr Rosen, p.c.).} Aljović (\citeyear[3f]{Aljovic04}, \citeyear[6]{Aljovic05}) was the first who claimed that sentential negation blocks CC in BCS. The permutation in (\ref{(11.104b)}) demonstrates how CC out of \textit{da}\textsubscript{2}-complements with negation leads to unacceptable sentences, whereas (\ref{(11.105b)}) illustrates the same but for CC out of a negated infinitive complement.\footnote{Here we would like to comment that some scholars do not agree that such sentences are possible at all. \citet[168]{Todorovic12} claims that “In respect to negation, indicative and subjunctive \textit{da}-complements differ in that negation can precede the embedded verb in indicative complements but cannot precede the embedded verb in subjunctive complements”. In other words, she claims that negation within the \textit{da}\textsubscript{2}-complement is not possible.}

\begin{exe}\ex
\begin{xlist}
\ex[]{\label{(11.104)}
\gll Marija želi\textsubscript{1} da \textbf{ga}\textsubscript{2} ne vidi\textsubscript{2}. \\
Marija want.3\textsc{prs} that him.\textsc{acc} \textsc{neg} see.3\textsc{prs} \\}

\ex[*]{\label{(11.104b)}
\gll Marija \textbf{ga}\textsubscript{2} želi\textsubscript{1} da ne vidi\textsubscript{2}.\\
Marija him.\textsc{acc} want.3\textsc{prs} that \textsc{neg} see.3\textsc{prs} \\
 \\
}
\end{xlist}
\glt‘Marija wants to not see him.’
\hfill (BCS; \citealt[3]{Aljovic04})

\ex
\begin{xlist}
\ex[]{\label{(11.105)}
\gll Marija želi\textsubscript{1} ne sresti\textsubscript{2} \textbf{ga}\textsubscript{2} nigdje. \\
Marija want.3\textsc{prs} \textsc{neg} meet.\textsc{inf} him.\textsc{acc} nowhere \\ }
\ex[*]{\label{(11.105b)}
\gll Marija \textbf{ga}\textsubscript{2} želi\textsubscript{1} ne sresti\textsubscript{2} nigdje. \\
Marija him.\textsc{acc} want.3\textsc{prs} \textsc{neg} meet.\textsc{inf} nowhere \\
 \\ }
\end{xlist}
\glt ‘Marija wants to meet him nowhere.’
\hfill (BCS; \citealt[4]{Aljovic04})
\end{exe}

\noindent In her second paper, \citet[7]{Aljovic05} extends her claims from the first paper and argues that negation in \textit{da}\textsubscript{2}-complements always blocks CC. However, in the case of infinitives CC is blocked only if there is a negative polarity item, like \textit{nigdje} ‘nowhere’ in example (\ref{(11.105)}). Thus, negation in the infinitive without a negative polarity item does not obligatorily block CC according to \citet[7]{Aljovic05}: compare (\ref{(11.105b)}) and (\ref{(11.106b)}). 

\begin{exe}\ex\begin{xlist}
\ex\label{(11.106)}
\gll Ona više voli\textsubscript{1} ne vidjeti\textsubscript{2} \textbf{ga}\textsubscript{2}.\\
she more love.3\textsc{prs} \textsc{neg} see.\textsc{inf} him.\textsc{acc} \\
\ex\label{(11.106b)}
\gll Ona \textbf{ga}\textsubscript{2} više voli\textsubscript{1} ne vidjeti\textsubscript{2}. \\
she him.\textsc{acc} more love.3\textsc{prs} \textsc{neg} see.\textsc{inf} \\
\end{xlist}
\glt ‘She likes not seeing him more.’
\hfill (BCS; \citealt[7]{Aljovic05})
\end{exe}

\noindent To explain the difference in the behaviour of infinitives and \textit{da}\textsubscript{2}-complements with respect to CC and negation, she introduces sentential negation as a constraint to CC \citep[7]{Aljovic05}. Namely, she claims that in the case of \textit{da}\textsubscript{2}-com\-ple\-ments there is no doubt that the negation is sentential. Moreover, the same applies to negated infinitives with negative polarity items, since only sentential negation can license them \citep[cf.][7]{Aljovic05}. In the case of negated infinitives without negative polarity items, the negative particle can be interpreted as lexical negation \citep[cf.][7]{Aljovic05}. Unlike sentential negation in (\ref{(11.104b)}) and (\ref{(11.105b)}), constituent negation in (\ref{(11.106b)}) does not block CC \citep[cf.][7]{Aljovic05}. 

We agree with Aljović that CLs can climb out of negated infinitive complements without a negative polarity item, since the permutation (\ref{(11.107b)}) of the example found in hrWaC (\ref{(11.107)}) is completely acceptable to our informants. 

\begin{exe}\ex\begin{xlist}
\ex\label{(11.107)}
\gll Odlučila\textsubscript{1} \textbf{sam}\textsubscript{1} ne pokazati\textsubscript{2} \textbf{joj}\textsubscript{2} \textbf{ga}\textsubscript{2}. \\
decide.\textsc{ptcp}.\textsc{sg}.\textsc{f} be.1\textsc{sg} \textsc{neg} show.\textsc{inf} her.\textsc{dat} him.\textsc{acc} \\
\ex\label{(11.107b)}
\gll Odlučila\textsubscript{1} \textbf{sam}\textsubscript{1} \textbf{joj}\textsubscript{2} \textbf{ga}\textsubscript{2} ne pokazati\textsubscript{2}. \\
 decide.\textsc{ptcp}.\textsc{sg}.\textsc{f} be.1\textsc{sg} her.\textsc{dat} him.\textsc{acc} \textsc{neg} show.\textsc{inf}  \\
\end{xlist}
\glt ‘I decided to not show him to her.’
\hfill [hrWaC v2.2]
\end{exe}

\section{Constraints related to information structure}
\label{Infinitive as a whole as the topic of a sentence}

\subsection{Infinitive as a whole as the topic of a sentence}
\label{Infinitive as a whole as the topic of a sentence:real}

\citet{Boskovic01} was the first to notice that there is no CC in BCS if the infinitive complement is fronted. The slightly later work of \citet[182f]{Stjepanovic04} provides examples which support this claim. In example (\ref{(11.112)}) the infinitive complement \textit{sresti} ‘meet’ is fronted. Therefore the pronominal accusative CL \textit{ga} ‘him’ does not form a mixed cluster with the verbal CL \textit{je} ‘is’. 

\begin{exe}\ex\label{(11.112)}
\gll Sresti\textsubscript{2} \textbf{ga}\textsubscript{2} u Kanadi, Dragan \textbf{je}\textsubscript{1} želio\textsubscript{1}. \\
meet.\textsc{inf} him.\textsc{acc} in Canada Dragan be.3\textsc{sg} want.\textsc{ptcp}.\textsc{sg}.\textsc{m} \\
\glt ‘Dragan wanted to meet him in Canada.’
\hfill (BCS; \citealt[182]{Stjepanovic04})
\end{exe}

\noindent \citet{Junghanns02} has similar observations regarding Czech. He claims that CLs stay in their position if the embedded infinitive as a whole is the topic of a sentence \citep[cf.][78]{Junghanns02}, for which he provides the following example (\ref{(11.113)}):

\begin{exe}\ex\label{(11.113)}
\gll Chovat\textsubscript{2} \textbf{se}\textsubscript{2} v souladu se svým svědomím nemá\textsubscript{1} prý žádnou {cenu [\dots].} \\
behave.\textsc{inf} \textsc{refl} in harmony with own conscience \textsc{neg}.have.\textsc{3prs} allegedly any value \\
\glt ‘Behave in accordance with conscience has no value [\dots].’ \\
\hfill (Cz; \citealt[78]{Junghanns02})
\end{exe}

\subsection{Infinitive as a whole as the focus of a sentence}
\label{Infinitive as a whole as the focus of a sentence}

\citet[78]{Junghanns02} and \citet[98]{Dotlacil04} note regarding Czech that if the embedded infinitive as a whole (together with its CL complements) is the focus of the sentence or is a part of the focus, CC will not occur.

\begin{exe}\ex\label{(11.114)}
\gll [\dots] kteří čas od času přicházeli\textsubscript{1} \textbf{se}\textsubscript{2} \textbf{mu}\textsubscript{2} posmívat\textsubscript{2}. \\
{} which time from time come.\textsc{ptcp}.\textsc{pl}.\textsc{m} \textsc{refl} him.\textsc{dat} laugh.\textsc{inf} \\
\glt ‘[\dots] who came to mock him from time to time.’
\hfill (Cz; \citealt[79]{Junghanns02})
\end{exe}

\noindent \citet{Junghanns02} further argues that climbing of the pronominal dative CL \textit{mu} ‘him’ and the reflexive CL \textit{se} in sentence (\ref{(11.114)}) would not lead to an ungrammatical sentence, but would definitely change its informational structure.\footnote{Alexandr
    Rosen (p.c.) disagrees with Junghanns. He argues that there is no difference between (\ref{(11.114)}) and (\ref{(11.114a)}):\\

    \ea\label{(11.114a)}
    \gll [\dots] kteří \textbf{se}\textsubscript{2} \textbf{mu}\textsubscript{2} čas od času přicházeli\textsubscript{1} posmívat\textsubscript{2}. \\
        {} which \textsc{refl} him\textsc{.dat} time from time come\textsc{.ptcp.pl.m} laugh\textsc{.inf}\\
        \glt ‘[\dots] who came to mock him from time to time.’
    \z
}

Constraints on CC which are related to information structure have been noticed both by scholars studying CC in Czech and those studying it in BCS. However, we must point out that there is relatively little literature on the phenomenon. 

\section{Summary}
\label{Summary:Czech}
\subsection{Overview}

In this chapter we focused on constraints on CC which have been described in the reviewed literature on this phenomenon in Czech and/or in BCS. In our analysis, we did not take into account structures described for Czech in \citet{Junghanns02} which are not attested in BCS. 

As already stated, our aim was to give a maximally adequate descriptive account of the possible constraints on CC in BCS. Therefore, we tried to pretest constraints on CC in the natural language environment provided by \{bs,hr,sr\}WaC. Furthermore, sometimes because of the problem of negative evidence we used informal acceptability judgments where sentences in each language were evaluated by at least five native speakers. 

In addition, we would like to point out that even for Czech the inventory of constraints is based on very sparse natural data (normally just a minimal pair of sentences per constraint). We cannot but wonder if real empirical studies (corpus linguistic or psycholinguistic) would corroborate constraints on CC reported for Czech in the theoretical syntactic literature. Furthermore, the authors sometimes tried to offer explanations, but they had to admit that counterexamples could be found. We summarise our main findings in Tables \ref{T11.1} and \ref{T11.1a}. However, we are aware that these are only to be considered preliminary results. The constraints described and marked “yes” are generalisations about a potentially large set of data; however, they were created on a very small number of examples, like in the case of CC in Czech. To further validate these constraints, we have to look for appropriate, representative and bigger samples: i.e., in order to establish whether the constraints on the list marked “yes” really operate as constraints in all three South Slavonic languages or only in some of them, more robust evidence should be found. 


\subsection{Island constraints}


\citet{Junghanns02} observes that in Czech CLs do not climb out of gerund phrases (see Section \ref{Phrases with gerunds}) and adjective phrases (see Section \ref{Adjective phrases}). However, these structures are not described in the reviewed literature on CC in BCS. Therefore we permuted sentences from \{bs,hr,sr\}WaC and pretested them via informal acceptability judgments. We finally came to the same conclusion as \citet{Junghanns02}, i.e., there is no doubt that the mentioned constraints operate in both Czech and BCS for both pronominal and reflexive CLs. 

Both scholars working on Czech (e.g. \citealt{Junghanns02}, \citealt{Dotlacil04}, \citealt{Rezac05}) and on BCS \citep[e.g.][]{ Aljovic05} recognize wh-infinitives as a constraint on CC (see Section \ref{Embedded wh-infinitives}). Our corpus-based examples and their unacceptable permutations support the claims from the theoretical literature on syntax. 

\citet{Junghanns02} was the only one to note that CC from an infinitive which is a complement of a noun in a prepositional phrase is blocked (see Section \ref{Infinitives as complements of nouns in prepositional phrases}). Since this constraint was not mentioned in the literature on CC in BCS, we first queried BCS web corpora and then conducted informal acceptability judgments with native speakers. As our informants did not accept the permutations of sentences from corpora, we assume that the same constraint operates in BCS as well. 

\citet{Junghanns02} notes one more island constraint for CC in Czech (see Section \ref{Infinitives in comparative sentences with nez}). CLs cannot climb out of infinitives in comparative sentences with \textit{než}. We pretested this constraint via informal acceptability judgments. Our native speaker informants did not accept permuted CC versions of the sentences with CLs in \textit{nego} infinitives without CC found in web corpora. Therefore, on the basis of our tentative results we can assume that this constraint on CC operates in BCS as well. This is the last of the four island constraints shared by Czech and BCS. 

It seems that the infinitive as a complement of an agreeing predicative adjective (see Section \ref{Infinitives as complements of agreeing predicative adjectives}) is a constraint on CC in Czech. However, \citet{Junghanns02}, who was the first to observe this phenomenon, acknowledges that counterexamples do exist. Consequently, he admits that this restriction has to be studied more thoroughly. Based on Junghanns’ examples, our assumption is that in Czech this constraint operates only in the case of reflexives. In contrast, as our tentative corpus-based research indicates, this constraint does not operate in BCS at all. This is one of many differences in CC between BCS and Czech. 

The constraint termed: infinitives as complements of non-agreeing predicatives (see Section \ref{Infinitives as complements of non-agreeing predicatives}) was first observed for Czech by \citet{Junghanns02} and further described by \citet{Dotlacil04}. In Czech, this constraint applies only to reflexives, while pronominal CLs can freely climb out of infinitives which are complements of non-agreeing predicatives. Our data show that unlike in Czech, in BCS this restriction does not apply to reflexive CLs. As the corpus data reveal, in all three South Slavonic varieties pronominal CLs can, like in Czech, climb out of an infinitive which is a complement of a non-agreeing predicative. 

\citet{Junghanns02} was the first to observe that in Czech, CLs do not climb out of infinitives which are complements of nouns, i.e. light verb constructions (see Section \ref{Infinitives as complements of nouns}). However, he acknowledges that there are exceptions to this constraint. Namely, in Czech CC is possible if the verbal part of a light verb construction bears no or almost no content. However, from BCS examples found in web corpora it seems that the amount of semantic content in the verbal part of light verb constructions does not play a role in blocking CC. 

Clauses with an inflected verb are one more island constraint which seems to operate differently in Czech and BCS (see Section \ref{Clauses with inflected verbs}). Scholars unanimously agree that there is no CC out of finite clauses in Czech. However, both BCS scholarly literature and \{bs,hr,sr\}WaC provide sentences in which CLs climb out of a complement with a verb inflected for person, the \textit{da}\textsubscript{2}-complement. The empirical data and discussion of this question can be found in Chapter \ref{A corpus-based study on CC in da constructions and the raising-control distinction (Serbian)}. 

\subsection{Constraints related to the raising--control distinction}


Scholars do not agree whether climbing of pronominal CLs out of ob\-ject-con\-trolled infinitives is possible in Czech (see Section \ref{Object-controlled complements}). While \citet{Thorpe91} and \citet{Junghanns02} believe that it is impossible, \citet{LelandToman76}, \citet{Dotlacil04}, \citet{Rezac05}, and \citet{Hana07} allow it iff some additional conditions are fulfilled. For BCS \citet{Aljovic05} is the only scholar who claims that an indirect object in the matrix clause does not necessarily have to block climbing of pronominal CLs out of an infinitive complement. Our first tentative corpus data seem to corroborate her claim. 

\citet{Dotlacil04} argues that it is not control by itself which blocks climbing of pronominal CLs in Czech, but that person and case are the additional features which do so (see Section \ref{Object control person-case constraint}). Namely, an accusative complement in the matrix clause blocks the climbing of all CLs except the third person accusative CL. However, it seems that \citet{Rezac05} does not share Dotlačil’s opinion. Namely, he claims that the only additional feature which prevents CLs from climbing is case, i.e., accusative case of the complement in the matrix clause blocks all CC (see Section \ref{Object control constraint related to case}). Unlike \citet{Dotlacil04} and \citet{Rezac05}, \citet{LelandToman76} consider animacy of the CL referent to be the only additional feature which can stop a CL from climbing out of object controlled infinitives, i.e., only CLs with inanimate referents can climb (see Section \ref{Object control and animacy of the referent of the clitic}).\footnote{The striking difference between \citet{Dotlacil04} and \citet{LelandToman76} becomes more obvious if we compare their examples presented in (\ref{(11.61)}) and in (\ref{(11.62b)}) respectively, the latter marked by \citet[245]{LelandToman76} as incorrect. First, in both cases the embedded infinitive is \textit{navštívit}/\textit{navštěvovat} ‘visit’, once in perfective and once in its imperfective form. Second, the complements of the embedded infinitive are accusative CLs with animate referents. Why do \citet[245]{LelandToman76} evaluate their sentence as unacceptable and \citet[80]{Dotlacil04} as acceptable? This may be one of the best examples showing that syntactic theories lie on shaky grounds, being based on informal acceptability judgments made by linguists and probably deliberately chosen to support their theories. We admit that it is possible that differences in evaluations are due to differences in authors’ dialects or idiolects. However, this is exactly why we see the necessity of a serious empirical approach to syntactic problems as argued in Chapter \ref{Empirical approach to clitics in BCS}.}

While climbing of pronominal CLs is blocked by a combination of object control and other features according to \citet{Hana07} CC of reflexives out of object controlled infinitives is completely impossible in Czech (see Section \ref{Object control reflexive constraint}). We pretested this constraint using permutations of examples from web corpora. The first tentative results of informal acceptability judgments made by our informants confirm that the constraint in question operates in BCS as well. However, it is important to emphasise that in Czech, according to \citet{Hana07}, and in BCS, according to our tentative exploration of \{bs,hr,sr\}WaC, raising and subject control CTPs do not block reflexive CLs from climbing. 

\subsection{Constraints related to mixed clitic clusters}
\label{Constraints related to mixed clitic clusters:summary}

The pseudo-twins constraint (see Section \ref{Pseudo-twins}) was first described by \citet{Junghanns02} on examples with reflexive CLs and later elaborated on by \citet{Hana07} and \citet{Rosen14}. These scholars unanimously agree that the nature of the constraint in question is not phonological. 

In contrast to \citet{Junghanns02}, \citet{Rosen14} argues that in Czech, CC is possible in the case of reflexives since the more embedded reflexive overrides the less embedded one. If reflexives are phonologically identical and have different governors, CC is possible in the form of haplology (see Section \ref{Phonologically identical pronominal and reflexive clitics with different governors}). Similarly, if reflexives are morphologically different and have different governors, CC is possible in the form of haplology of unlikes (see Section \ref{Morphologically different clitics with similar syntactic function and different governors}). 

 Unlike for reflexive CLs, as \citet{Hana07} observes, CC is not possible in Czech if pronominal CLs are phonologically (and morphologically) identical and have different governors.

 So it seems that there are differences among CLs. This goes well with the hypothesis put forward by \citet[82f]{Dotlacil04}: that with respect to CC, CLs cannot be treated as one homogeneous class, since they do not behave in the same way.\footnote{“[\dots] all clitics were treated as one homogenous class and were expected to behave same. The point of both subsections is precisely against this treatment of clitic climbing.” \citep[82f]{Dotlacil04}.} The data from Czech and BCS discussed in Sections \ref{Infinitives as complements of agreeing predicative adjectives}, \ref{Infinitives as complements of non-agreeing predicatives} and \ref{Object control reflexive constraint} indicate exactly that: reflexive CLs seem to behave differently than pronominals. Furthermore, from the work of \citet{LesnerovaMalink08} and \citet{Rosen14} it has become clear that different types of reflexives do not behave in a uniform way. We agree with the scholars who pointed out that not only do pronominal and reflexive CLs differ from each other, but also that reflexive CLs form a heterogeneous group. These differences are important factors which cannot be neglected in the research on CC in BCS. Therefore, we included them as variables in our empirical studies on CC in Croatian described in Chapters \ref{A corpus-based study on clitic climbing in infinitive complements in relation to the raising-control dichotomy and diaphasic variation (Croatian)} and \ref{Experimental study on constraints on clitic climbing out of infinitive complements}.

Since these constraints were not recognised in the literature on CC in BCS, we pretested them through informal acceptability judgments of permutations of examples from web corpora. Our tentative results suggest that this constraint does apply to BCS. However, in contrast to Czech, haplology of reflexives is not always a possible solution. According to our first results for BCS, haplology can be applied only to phonologically identical reflexive CLs. 

In addition, we would like to emphasise that the situation of two CLs with different governors has to be observed coherently with respect to the distinction of two different control predicate types: subject and object. See Section \ref{Constraints related to the raising-control distinction} on features which are strongly correlated with the control constraint. 

\subsection{How clitics climb}

Two constraints are related to the way in which CLs climb (see Section \ref{How clitics climb}). According to \citet{Hana07} and \citet{Rosen14} CC in Czech is monotonic, i.e., the more embedded CLs cannot climb unless the less embedded CLs climb as well (see Section \ref{Clitic cannot climb over clitic}). \citet{Rezac05} claims that in Czech CC is an all-or-nothing phenomenon, i.e. either all CLs governed by the embedded infinitive climb or none do (see Section \ref{All-or-nothing constraint}). 

However, it is important to note that the latter constraint might apply only to Czech. Namely, according to data in the literature on CC out of \textit{da}\textsubscript{2}-complements \citep[e.g.][]{Stjepanovic04} and the results of our corpus study in Section \ref{Discussion:da} and permutations of examples of CC out of infinitive complements found in \{bs,hr,sr\}WaC, the all-or nothing constraint does not apply in BCS. 

\subsection{Sentential negation}

\citet{Aljovic04, Aljovic05} is the only scholar who elaborates on the sentential negation constraint on CC in BCS. She claims that negation in \textit{da}\textsubscript{2}-complements always blocks CC (see Section \ref{Sentential negation}). However, it is not completely clear whether such constructions are possible at all, since \citet{Todorovic12} claims that negation of \textit{da}\textsubscript{2}-complements is not possible. In contrast, as \citet{Aljovic05} and our corpus-based examples show, negation can be found within infinitive complements. However, negation does not necessarily function as a constraint on CC from infinitive embeddings in BCS. CC is blocked only if there is a negative polarity item within the infinitive clause: otherwise CLs can climb. 

\subsection{Constraints related to information structure}

Two constraints linked to the information structure of a sentence are reported in the literature on CC in Czech and BCS. \citet{Boskovic01}, \citet{Stjepanovic04}, and \citet{Junghanns02} agree that CLs do not climb out of fronted infinitive complements, i.e. out of an infinitive which is the topic of a sentence. BCS and Czech share this constraint (see Section \ref{Infinitive as a whole as the topic of a sentence:real}). In addition, \citet{Junghanns02} and \citet{Dotlacil04} argue that in Czech CLs cannot climb out of an embedded infinitive which is the focus of a sentence (see Section \ref{Infinitive as a whole as the focus of a sentence}).


\begin{table}
\caption[Overview of tentative constraints for Czech and BCS]{Overview of tentative constraints for Czech and BCS}
\label{T11.1}
\scriptsize
\begin{tabularx}{.97\textwidth}{X>{\raggedright\arraybackslash}X>{\raggedright\arraybackslash}X>{\raggedright\arraybackslash}p{0.1\linewidth}p{0.1\linewidth}}
\lsptoprule
Constraint type & Subtype & Subsubtype & Czech & BCS \\
\midrule
2. island constraints & 2.1. infinitives in comparative sentences with \textit{než}/\textit{nego} && yes & yes \\
 & 2.2 clauses with inflected verbs && yes & partially \\
 & 2.3 phrases with gerunds& & yes & yes \\
 & 2.4 adjective phrases && yes & yes \\
 & 2.5 depth and kind of embeddedness of infinitive phrases & 2.5.1. infinitives as complements of nouns & not clear & no \\
 && 2.5.2. infinitives as complements of nouns in prepositional phrases & yes & yes \\
 && 2.5.3. infinitives as complements of agreeing predicative adjectives & not clear & no \\
 && 2.5.4. infinitives as complements of non-agreeing predicatives & only for \textsc{refl} & no \\
 & 2.6. embedded interrogative wh-infinitives && yes & yes \\
\tablevspace
3. constraints related to the raising--control distinction & 3.1. object-controlled complements & 3.1.1. object control constraint related to case & yes & yes \\
 && 3.1.2. object control person-case constraint & not clear & no \\
 && 3.1.3. object control and animacy of the referent of the CL constraint & not clear & no \\
 && 3.1.4. object control reflexive constraint & yes & yes \\
 \multicolumn{2}{l}{(Continuation in Table \ref{T11.1a})}&&& \\
 \lspbottomrule
\end{tabularx}
\end{table}

 \begin{table}
\caption{Continuation of Table \ref{T11.1}}
\label{T11.1a}
\scriptsize
\begin{tabularx}{.97\textwidth}{X>{\raggedright\arraybackslash}X>{\raggedright\arraybackslash}X>{\raggedright\arraybackslash}p{0.1\linewidth}>{\raggedright\arraybackslash}p{0.1\linewidth}}
\lsptoprule
Constraint type & Subtype & Subsubtype & Czech & BCS \\
\midrule
4. constraints related to mixed CL clusters & 4.1. pseudo-twins & 4.1.1. phonologically identical pronominal and reflexive CLs with different governors & yes & yes \\
 && 4.1.2. morphologically different CLs with similar syntactic function and different governors & yes & yes \\
5. how CLs climb & 5.1. CL cannot climb over CL && yes & yes \\
 & 5.2. all-or-nothing constraint && not clear & no \\
6. sentential negation &&& not described & not clear \\
7. constraints related to information structure & 7.1. infinitive as a whole as the topic of a sentence && yes & yes \\
 & 7.2. infinitive as a whole as the focus of a sentence && yes & not described \\
\lspbottomrule
\end{tabularx}
\end{table}




\section{Further Perspectives}

In this chapter we showed that Czech and BCS share the following island restrictions on CC: gerunds or adverbial participles respectively, adjective phrases, infinitives as complements of nouns in prepositional phrases and embedded wh-infinitives, as well as one constraint caused by comparative sentences with \textit{než}{\slash}\textit{ne\-go}. Other island constraints reported for Czech, such as finite clauses, infinitives as complements of nouns (i.e. light verb CTPs), infinitives as complements of agreeing predicative adjectives, and infinitives as complements of non-agree\-ing predicatives seem mostly – in the small scope that is covered by literature and according to our first tentative results – not to operate in BCS. However, apart from the previous five relatively clear constraints on CC in BCS, there are some less clear cases. The case of the reflexive CL \textit{se}/\textit{si} reported for Czech as the pseudo-twins constraint and the control constraint turns out to be particularly intriguing. While the former constraint has to be systematically linked to the difference between subject and object control predicate types, the latter should be investigated in the context of other features such as case, person, animacy, CL type (reflexive vs pronominal). According to our first tentative results, some of these features could be important for CC in BCS as well.

Additionally, we saw that some features relevant to CC seem to interact with each other, but we do not know exactly how. These features are: predicate type (control vs raising), CL type (pronominal vs reflexive) and those related to the mixed CL cluster under the label pseudo-twins. The last set of features has not been systematically described under the control distinction, but we believe they cannot be separated. Namely, in the case of pseudo-twins, if one reflexive CL is in the matrix clause and the other in the infinitive clause, then the matrix verb can be either of the subject control type (\textit{veseliti se} ‘look forward to’, \textit{odlučiti se} ‘decide’) or object control type (\textit{prisiliti se} ‘force oneself’). Hence, if a pseudo-twins constraint is present, there is still the factor of object control, which cannot be neglected. Furthermore, since differences are reported in climbing of different CL types, even within the group of reflexive CLs, for valid results this feature has to be tested in combination with different control predicates as well. 

This discussion of Czech and BCS shows that putative blocking effects on CC in object-controlled complements in BCS are worth investigating (see Chapter \ref{Experimental study on constraints on clitic climbing out of infinitive complements}). As we have already stated, blocking effects seem to arise from the combination of control and some other features. We will discuss in more detail the link between the control (and raising) distinction and CC in Chapters \ref{A corpus-based study on CC in da constructions and the raising-control distinction (Serbian)}--\ref{Experimental study on constraints on clitic climbing out of infinitive complements}. 
