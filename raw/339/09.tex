\chapter{Parameters of variation: {C}onclusions}
\label{Parameters of variation: conclusions}
\section{Introduction}
In this chapter we summarise the findings presented in the previous three chapters on microvariation in BCS grammaticography and dialects, and in the corpus of spoken Bosnian. We focus on the relationship between language use in spoken languages and standardisation processes which include selection and prescription of features. By taking a bird's eye view, we detect global patterns of microvariation which we discuss in the following sections. Section \ref{sec:10.2.1} provides an overview of the variation in the inventory. Section \ref{sec:10.2.2} presents variation within CL clusterisation and morphonological changes which occur in CL clusters, followed by conclusions on variation in position of CLs or CL clusters. Section \ref{sec:10.2.3} discusses the absolute first position, Section \ref{sec:10.2.4}, second position, delayed placement and phrase splitting, and Section \ref{sec:10.2.5}, the heaviness of constituents. As already mentioned, clitic climbing will be dealt with separately in Part \ref{part3} because the data from the grammar handbooks, dialectological sources and our corpus of spoken language are too limited to allow for any sound conclusions. As we are aware that it might be hard to follow our discussion on microvariation detected between codified standard varieties on the one hand and BCS Štokavian dialects and the spoken variety of Bosnian on the other hand, we provide Figure \ref{M10.1}. which shows the occurrence of the discussed features in the individual dialects or dialect groups. At the same time, readers who are not familiar with the geographical background can clearly see the borders of the ex-Yugoslavian countries. On the map we indicate only features from dialects. First, we use them to discuss which features present on the territory of certain countries were or were not selected to be features of the respective codified standard variety. Secondly, we compare features attested in dialects spoken on Bosnian territory with features attested in the spoken Bosnian variety. However, variation which was attested in spoken Bosnian is not indicated on Figure \ref{M10.1}. Since data on the origins of interviewees was anonymised, we could not trace possible dialectal influences in their speech. 

\begin{figure}
\caption{Map showing the dialects and the distribution of selected features. Author: Branimir Brgles.}
\label{M10.1}
\includegraphics[width=\textwidth]{FinalV1.pdf}
\end{figure}

\begin{table}
\caption{Legend to Figure \ref{M10.1}}
\label{T10.1}
\begin{tabular}{|l|l|}
\hline
&\\ %first image it too high (deletes parts of the hline), needs extra space. looks a little weird. ok-ish
\includegraphics[width=0.3\textwidth]{LEGENDA_OP58.pdf}
&
\includegraphics[width=0.6\textwidth]{legendaFinal.pdf} %just trying to make them equal in size without using em, fits ok-ish
\\\hline
\end{tabular}
\end{table}


\section{Parameters of microvariation: Global patterns}
\subsection{Clitic inventory}
\label{sec:10.2.1}

Our findings on variation in the CL inventory can be summed up as follows. Our analysis of the standard grammar books shows differences in the inventory of the standard varieties. Only Croatian grammarians accept the reflexive CL \textit{si} as standard. In this respect \citet[440]{Ridjanovic12} even claims that the \textsc{refl\textsubscript{2nd}} CL \textit{si}, which is widely used in Croatian, can hardly be found elsewhere in BCS territory. The analysis of the dialectological literature, however, yields a more varied picture. Figure \ref{M10.1} clearly shows that the \textit{si }form (transparent pentagon) is found not only on Croatian, but also on Bosnian and Serbian language territory. Namely, it is attested in scattered areas comprising some idioms of Montenegro, South Eastern Serbia, Western Herzegovina and Northern Bosnia. The presence of the \textsc{refl\textsubscript{2nd}} CL \textit{si} in idioms of Western Herzegovina and Northern Bosnia explains why this CL is also attested in the corpus of spoken Bosnian. Nevertheless, we would like to emphasise that our data suggest that this form is very rare in the spoken Bosnian variety. 

Further, Croatian and Serbian authors differ in their recommendations for the usage of the third person singular feminine accusative CL \textit{ju}. According to some Croatian authors, \textit{ju} can be treated as a separate unit of the inventory, which is not restricted only to realisations of the CL cluster sequence with the third person singular feminine accusative and the third person singular present tense of the verb \textit{biti} ‘be’. In contrast, in standard Serbian the third person feminine accusative pronoun \textit{ju} can be used only in the case of suppletion. If we want to compare its situation in standard BCS varieties with the situation attested in dialects spoken in BCS language territory in Figure \ref{M10.1}, we clearly see that the \textit{ju }form (black pentagon) is attested in many idioms of Old, Middle and Neo-Štokavian dialects. The spatial distribution of the CL \textit{ju} is not limited only to Croatian language territory. Moreover, it stretches from the West (\textit{Zapadni},\textit{ Srednobosanski}) to the Southeast (\textit{Timočko-lužnički},\textit{ Kosovsko-resavski} and \textit{Prizrensko-južnomoravski}) and covers also Bosnian, Montenegrin and Serbian language territory. However, unlike in dialects spoken on Bosnian language territory, the CL \textit{ju} is not attested at all in the data from the corpus of spoken Bosnian. 

No variation as to the inventory of verbal CLs is found in standard BCS varieties. For the few varying forms that appear only in dialects, see Chapter \ref{Clitics in dialects}. Notably, in many dialects the conditional auxiliary form \textit{bi} is used for all persons (\textit{Istočnohercegovački}, \textit{Zapadni}, \textit{Šu\-ma\-dij\-sko-voj\-vo\-đan\-ski}, \textit{Slavonski} and \textit{Kosovsko-resavski}). Moreover, our data from the corpus of spoken Bosnian corroborate Peco's (\citeyear[331]{Peco07b}) claim that the CL form \textit{bi} used for all persons is spreading as a trait from dialects into spoken varieties. 

At the end of this subsection we conclude that the selection and prescription of certain features related to CLs in standard BCS varieties do not correlate with their distribution on BCS language territory. Namely, although the CL \textit{si} is present on Bosnian and Serbian language territory, it has not found its way into their standard varieties. Similarly the CL \textit{ju} which, while attested in Bosnian and Serbian language territory, is restricted only to suppletion contexts in the relevant standard varieties. All three standard varieties are equally strict in their treatment of the conditional CL form \textit{bi} used for all persons as a feature limited exclusively to non-standard varieties. 

\subsection{Clitic cluster and morphonological processes}
\label{sec:10.2.2}
The maximum size of CL clusters in standard BCS varieties has been discussed by the Serbian authors \citet[451f]{PiperKlajn14} and the Bosnian author \citet[558]{Ridjanovic12}. They claim that the CL cluster usually consists of two or three elements and that groups of five or more CLs are quite infrequent. Whereas there is still no solid empirical data on the maximum size of CL clusters in standard BCS varieties, in Chapter \ref{Clitics in a corpus of a spoken variety} we provide empirical data from the corpus of spoken Bosnian. Here, clusters consisting of only two elements are by far the most representative (99\%). In contrast to the claims made for standard Serbian and Bosnian varieties, our empirical data show that in spoken Bosnian CL clusters with three components are an exception (1\%). Moreover, strings of four or more CLs in a cluster are not attested at all. 

Let us discuss the variation with respect to CL ordering in the cluster. The ordering of the reflexive CL \textit{se} and the verbal CL \textit{je} is a further clear case of microvariation. Whereas in standard Bosnian and Croatian both haplology of unlikes and CL clusters with the sequence \textit{je} \textit{se} are allowed, the Serbian authors of a normative grammar book \citet[452]{PiperKlajn14} acknowledge only the former as a feature of standard Serbian. Unlike in standard BCS varieties, both in dialects and in the spoken Bosnian variety we find ample evidence for the reversed CL order. The situation in BCS dialects with respect to the \textit{je} \textit{se} sequence is clearly visible on Figure \ref{M10.1}: it is attested in central BCS territory of \textit{Šumadijsko-vojvođanski}, \textit{Zapadni}, \textit{Slavonski}, \textit{Srednjobosanski} and \textit{Istočnohercegovački}. Thus, it is attested on Bosnian language territory. This is in accordance with the data from the corpus of spoken Bosnian where the \textit{je} \textit{se} cluster sequence is four times more frequent than \textit{se} \textit{je} prescribed in the standard Bosnian and Croatian varieties.  

A second case of variation in CL ordering in the cluster concerns both diaphasic and diatopic variation. Namely both in BCS dialects (\textit{Šumadijsko-vojvođanski} and \textit{Svrljiško-zaplanjski}) and in the corpus of spoken Bosnian we find sentences in which the verbal CL \textit{je} precedes pronominal accusative CLs. Since this order has been attested in the spoken Bosnian variety, we assume that it is very probably present in dialects spoken in Bosnian language territory too. Moreover, not only sentences in which the verbal CL \textit{je} precedes accusative pronominal CLs, but also those with pronominal CLs in other cases are attested in various dialects: for more information and examples see Section \ref{Clitic ordering within the cluster:8}.

Regarding the morphonological process of haplology of unlikes in the context of co-occurrence of the reflexive CL \textit{se} and the verbal CL \textit{je}, we would like to put forward our empirical data from the corpus of spoken Bosnian. Although both haplology of unlikes and the \textit{se} \textit{je} sequence are allowed in the standard Bosnian variety, the data from the corpus of spoken Bosnian show that haplology of unlikes is far more common than the co-occurrence of these two CLs. Namely, we find haplology (with only the CL \textit{se} occurring) in 68.8\% of cases, in 25.4\% of cases a CL cluster (the CLs co-occur, the sequence \textit{je} \textit{se} is more frequent than \textit{se} \textit{je}) with \textit{je}, and in 5.8\% of cases the reflexive CL \textit{se} and the verbal CL \textit{je} appear in diaclisis.

\largerpage
\subsection{Absolute first position and clitics after the conjunctions \textit{a} and \textit{i}}
\label{sec:10.2.3}
We start with our findings concerning absolute 1P, i.e., CLs which are placed at the beginning of the clause. According to the prescribed language norms of all three standard varieties, this CL position is ungrammatical. However, the dialectal map in \ref{M10.1} shows that absolute 1P (black circle) is attested in idioms of the \textit{Šumadijsko-vojvođanski}, \textit{Kosovsko-resavski}, \textit{Prizrensko-južnomoravski} and \textit{Ti\-moč\-ko-lužnički} dialects. It is important to note that the former two are in language contact with Romanian, while the latter two are in language contact with Macedonian. We did not find CLs in the absolute 1P in the corpus of spoken Bosnian. We only came across sentences in which CLs follow insertions, DSEs and retrospectives. These findings strongly suggest that absolute 1P is likely to occur in Štokavian contact varieties.

Our dialectological data indicate that at least some Štokavian idioms, including even idioms of \textit{Istočnohercegovački}, allow CL positioning directly after the coordinative conjunctions \textit{a} and \textit{i}, unlike standard BCS varieties. Figure \ref{M10.1} shows that this feature, represented by a transparent circle, is also attested in \textit{Šuma\-dij\-sko-vojvođanski}, spoken mainly in Serbia.\footnote{As mentioned in Chapter \ref{Clitics and variation in grammaticography and related work}, \textit{Istočnohercegovački} served as a dialectal base for all three standard BCS varieties.}

\subsection{Second position, delayed placement and phrase splitting}
\label{sec:10.2.4}
In this subsection we would like to highlight the following facts. While it seems that in standard Croatian and standard Bosnian the second position rule is understood as 2W, in the literature on standard Serbian it is emphasised that 2P is normally understood as the position posterior to the first phrase. Moreover, in the latter variety splitting the initial phrase is less preferred than placing CLs after initial compound phrases of two content words. 

In contrast to standard Serbian, in which phrase splitting is uncontroversial only in very few cases, Croatian and Bosnian standards allow the insertion of CLs in far more contexts. Our dialectological data show that splitting of forename and family name, of conjoined NPs and of quantificational phrases is not only widespread in Bosnian and Croatian territory, but can also be found in Serbian language territory. Similarly, the corpus of spoken Bosnian contains ample evidence for phrase splitting. Moreover, we would like to emphasise that in the spoken Bosnian variety not only subject phrases, but also prepositional phrases can be split. As can be seen in Figure \ref{M10.1} (black triangle), the latter is also attested in the \textit{Istočnohercegovački} dialect spoken on Bosnian language territory.  

Furthermore, we would like to comment on the disagreement among theoretical syntacticians with respect to the number of CLs taking part in phrase splitting. Dialectal data from \textit{Šumadijsko-vojvođanski} and \textit{Istočnohercegovački} show that two CLs can be inserted into a phrase. Since this feature is attested in a dialect spoken on Bosnian language territory (see transparent triangle on Figure \ref{M10.1}) it should come as no surprise that the corpus of spoken Bosnian also contains such instances. To conclude, both dialectal data and the data on the spoken Bosnian variety clearly contradict \citet{Progovac96} and \citet{RadanovicKocic88, RadanovicKocic96}, who claim that clusters are not used as splitting elements.

Moreover, BCS normativists disagree in their evaluations of DP. While Bosnian and Croatian authors recommend delaying the placement of CLs as a better alternative to placing CLs after compound phrases, Serbian authors propose quite the opposite. Delayed placement is widespread in dialects (see black square on Figure \ref{M10.1}). We find such cases in the \textit{Slavonski}, \textit{Istočnohercegovački} and \textit{Šu\-ma\-dij\-sko-vojvođanski} dialects spoken not only in Croatia and Bosnia, but also in Serbia.

\subsection{Heaviness of the initial constituent}
\label{sec:10.2.5}
Several authors mention the heaviness of the initial constituent as a factor responsible for DP. However, exact information on how to distinguish initial constituents which are heavy and cause DP from those which allow 2P can be found neither in grammar books nor in the dialectological literature. Therefore, we conducted an empirical study based on the measurement of heaviness proposed by the Czech linguists \citet{KCN18}. The chapter on spoken Bosnian provides some first hints on the heaviness of a constituent in the spoken variety. As to the nature of 2P in spoken Bosnian, we saw a strong tendency towards 2W; in 77\% of all observations (single CLs and clusters) the CL occupies a position after the first word. The typical CL position in the sentence is after the first word, which is most frequently two graphemes long. The most frequent initial constituent in DP is three graphemes long, but in general its length is not limited, while the most frequent host in DP is four graphemes long, and thus longer than the initial constituent. 
