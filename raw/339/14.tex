\chapter[A corpus-based study on clitic climbing out of infinitive complements]
        {A corpus-based study on clitic climbing out of infinitive complements in relation to the raising--control dichotomy and diaphasic variation (Croatian)}
\label{A corpus-based study on clitic climbing in infinitive complements in relation to the raising-control dichotomy and diaphasic variation (Croatian)}
\section{Introduction}

The present chapter is a further empirical study on CC out of infinitive complements with a specific focus on the raising--control distinction, which we showed to be a relevant factor for CC out of \textit{da}\textsubscript{2}-constructions in Serbian in Chapter \ref{A corpus-based study on CC in da constructions and the raising-control distinction (Serbian)}.\footnote{Some results from this chapter have been previously discussed in \citet*{KJH19}.}\textsuperscript{,}\footnote{For basic information on the raising--control dichotomy see Section \ref{The control vs raising distinction}.}\textsuperscript{,}\footnote{See also \citet*{JHK17b}.} Here, we broaden the empirical base for the investigation of this dichotomy by specifically examining Croatian infinitive constructions. In addition, we zoom in on the diaphasic dimension of variation as a factor influencing the probability of CC occurring. This type of variation, as explained in Section \ref{Systemic vs functional microvariation}, reflects different modes of language use in different situations. To illustrate, the examples provided below contain the same CTP \textit{morati} ‘must’ and infinitive \textit{odlučiti se} ‘decide’. However, whereas in example (\ref{(15.2)}) extracted from the corpus of the standard Croatian variety the \textsc{refl\textsubscript{lex}} CL \textit{se} climbs out of the infinitive complement, in example (\ref{(15.1)}) extracted from the Forum subcorpus of the Croatian web corpus the very same CL stays in situ.\footnote{For more information on our typology of reflexives see Section \ref{Conclusion: how many types of se do we need to distinguish?}.} In this chapter we thus examine whether these differences in CL positioning are due to chance or whether they can be ascribed to diaphasic variation.

\begin{exe}\ex\label{(15.1)}
\gll Pa ako već morate\textsubscript{1} odlučiti\textsubscript{2} \textbf{se} za  {jaslice [\dots].} \\
well if already must.\textsc{2prs} decide.\textsc{inf} \textsc{refl} for nursery \\
\glt ‘Well if you have to opt for nursery [\dots].’
\hfill [hrWaC v2.2]

\ex\label{(15.2)}
\gll A da \textbf{se} morate\textsubscript{1} odlučiti\textsubscript{2}? \\
and if \textsc{refl} must.\textsc{2prs} decide.\textsc{inf} \\
\glt ‘And if you have to decide?’
\hfill [Riznica]
\end{exe}

\noindent The rest of this chapter is structured as follows: Section \ref{Clitic climbing and diaphasic variation} describes the importance of diaphasic variation for CC in Spanish and Portuguese. Spanish and Portuguese are of interest because their CL systems have many features in common with Croatian and show diaphasic variation. Next we present basic information on diaphasic variation in Croatian. Our research questions are presented in Section \ref{Research questions}. The choice of data and the collection process are explained in Section \ref{Methods}, while Section \ref{Results and discussion} describes the results in detail. It is followed by the final Section \ref{Conclusions:inf}, which draws conclusions.

\section{Clitic climbing and diaphasic variation}
\label{Clitic climbing and diaphasic variation}
\subsection{Clitic climbing and diaphasic variation in Romance languages}

Although in Chapter \ref{Approaches to clitic climbing} we avoid comparison between BCS CLs and those in Romance languages, we do make it here. As the relationship between CC and diaphasic variation has never been the topic of any study on a Slavonic language, it is worth looking at variationist work on Romance languages. All the more so as Spanish and Portuguese are languages with CLs which can climb. In the literature on variation in Spanish CC, several authors (e.g. \citealt{Davies95}, \citealt{Cacoullos99}) point out the relevance of register: generally it can be said that CC is less frequent in Spanish written texts than in Spanish spoken texts.\footnote{A deeper look at those papers reveals that authors who worked on the impact of register on CC in Spanish and Portuguese use the concept of register to refer to different things. In its broadest sense, register is a language variety defined by the context of usage \citep[31]{Colak15}.} 

\citet{Davies95} investigated CC on the basis of a corpus composed of texts from ten Spanish-speaking countries. He reported a consistent difference between registers. His data show that the distance between registers with respect to CC can be as high as 30\% \citep[cf.][373f]{Davies95}. \citet{Cacoullos99} studied CC in Mexican Spanish using similar methodology. Register once again turned out to be an important factor: sociolinguistic interviews had higher rates of CC than essays (89\% versus 68\%) \citep[cf.][]{Cacoullos99}.

\citet{Andrade10} replicated the results of studies on Spanish CC for European Portuguese data. He analysed CC in 1000 Portuguese sentences, which were annotated as formal (newspaper interviews and novels) or informal (sociolinguistic interviews). Using basic statistical correlation testing \citet[99]{Andrade10} showed that the CC rates in those two registers differ significantly. 

It is worth mentioning that he also analysed, but only on the data from the formal register, how language-internal factors such as CL type and grammatical function, syntactic context, the presence of intervening elements between CTP and infinitive, and the frequency of the CTP influence CC. His results on the importance of CL type and grammatical status are in accordance with claims concerning those factors made in the theoretical literature on CC in Czech.\footnote{For more information see Sections \ref{Object control person-case constraint} and \ref{Object control reflexive constraint}.} Like in Czech, in European Portuguese CL type and grammatical function are important factors for CC. Specifically, in Portuguese CC is much more frequent in the case of datives than in the case of accusatives. While the CC rate is 51.6\% for ethic and possessive datives and 50.7\% for argumental datives, for accusatives it is only 32.6\% \citep[cf.][101]{Andrade10}. 

\subsection{Diaphasic variation in Croatian}

Although we are aware of the differences in the stratification of the languages mentioned above, we treat these results as a point of departure for addressing diaphasic variation in Croatian. Due to lack of space, we cannot give a full account of the stratification of Croatian. We only refer to \citet[10--17]{FHM06}, who distinguish the following diatopic strata in Croatian: local idioms or Croatian dialects; urban idioms (substandard idioms or jargon) and the Croatian standard language. The latter is an abstract system based on three dialects – not only Štokavian but also Čakavian and Kajkavian \citep[cf.][22f]{FHM06}. Furthermore, the literature (e.g. \citealt[230]{FHM06}) acknowledges the following diaphasic strata in Croatian: scientific (scholarly), administrative, journalistic, literary and colloquial.

As we can see, on the one hand we have standard Croatian and on the other, non-standard Croatian conventionally labelled as spoken, colloquial, dialectal, rural, etc. \citep[cf.][32f]{Murelli11}. The latter variety comprises various idioms with elements which are not codified or are rejected in the standard. 
%punkt und so fehlt? or does this continue?

However, in this particular study of CC, we are interested only in the standard Croatian variety and in the non-standard variety termed “everyday colloquial language”, “conversational standard” or “informal spoken standard”. Because everyday colloquial Croatian as a non-standard idiom is in fact a sub-variety of standard language with elements which are not a part of the norm (\citealt[cf.][13--17]{Marle97}, \citealt[30]{LangstonPeti14}), this non-standard variety shares more similarities with the Croatian standard than Croatian local idioms (i.e. dialects) do. In the remainder of our paper, the term \textsc{colloquial} \textsc{Croatian} (variety) will be used to refer to this particular non-standard variety of Croatian. 

\section{Research questions}
\label{Research questions}
Based on the considerations presented in Sections \ref{Clitic climbing and diaphasic variation} and \ref{Constraints related to the raising-control distinction}, we explore the claim that CC varies with respect to both the raising--control dichotomy and register. As already mentioned in Section \ref{Choice of matrix verbs} we expand the typology of CTPs to include reflexive subject control verbs, as suggested by \citet*[266]{HKJ18}, and address the following research questions: 

\begin{enumerate}[label=Q\arabic*:]
\item Does clitic climbing out of single infinitive complements in Croatian depend on CTP type with respect to the raising--control distinction?
\item Does CC out of single infinitive complements in Croatian depend on CL type?
\item Does CC out of single infinitive complements in Croatian depend on CL case?
\item Does CC out of single infinitive complements appear equally frequently in the standard and colloquial Croatian variety if the type of CTP verb (raising vs control) as a variable remains constant? 
\end{enumerate}

\section{Methods}
\label{Methods}
In order to answer the research questions we quantitatively analysed data obtained from three corpora. Forum, a subcorpus built from a hrWaC subdomain forum.hr (Forum) represents the informal register, while CNC and Riznica (Standard) are used as the source of formal data strongly influenced by prescriptive norms.\footnote{For more information on corpora available for Croatian and their detailed descriptions see Chapter \ref{Corpora for Bosnian, Croatian and Serbian}, where we also explain our choices concerning particular studies.} We present information on the matrix verbs chosen for the study as well as details of data retrieval in Chapter \ref{Introductory remarks to corpus studies on CC}. 

As mentioned in that chapter, we investigated variants with and without CC for 24 CTPs in two types of corpora representing standard and colloquial language varieties.\footnote{For more information on structure variants see Table \ref{T12.1}.} We limit ourselves to the analysis of raising and subject control verbs only. The study could have been extended to object control verbs. However, finding appropriate observations in corpora is very costly for several reasons, such as the frequency of particular lexemes in comparison to the size of the population of all object control lexemes, as well as the higher grade of complexity of object control predicates (in comparison to other CTPs), which necessarily encode two arguments: subject and object.\footnote{The manual revision of data for Chapter \ref{A corpus-based study on CC in da constructions and the raising-control distinction (Serbian)} taught us that CQL queries for object control matrices perform poorly in retrieval of CC. The CLs that appear in the matrix are not CLs which climb out of infinitive complements, but CLs which are complements of object control matrix predicates. In other words, manual revision of CC structure variants with object control matrix predicates would be extremely time consuming and would ultimately result in a small number of observations which we would not be able to analyse using the logistic regression model. These kinds of predicates are extensively studied in Chapter \ref{Experimental study on constraints on clitic climbing out of infinitive complements} since the experimental approach allows fast collection of the necessary amount of the observations.} We obtained 96 samples, upon which we built a logistic regression.\footnote{Our aim was to obtain fully crossed data (144 samples), that is, samples of the size of 100 for three variants (see Sections \ref{Operationalising the constructions in question} and \ref{Data collection Ch 12}) from two corpora, but this task turned impossible for some CTPs.} The model contains the following variables to be investigated with respect to the research questions: type of corpus, type of CTP, type and case of infinitive CL. These variables and their levels are summarised in Table \ref{T15.1} below.

\begin{table}
\caption{Variables used in the regression model\label{T15.1}}
\begin{tabular}{lll}
\lsptoprule
Variable name & Levels & Class \\
\midrule
presence of CC & 1 (CC present) & dependent \\
 & 0 (CC absent) & \\
 \tablevspace
corpus type & forum & independent \\
 & standard & \\
 \tablevspace
CTP type & raising & independent \\
 & subject control & \\
 & reflexive subject control & \\
 \tablevspace
case of infinitive CL & accusative & independent\\
 & dative & \\
 & no case & \\
 \tablevspace
type of infinitive CL & pronominal & independent \\ 
 & \textsc{refl\textsubscript{lex}} & \\
 & \textsc{refl\textsubscript{2nd}} & \\
\lspbottomrule
\end{tabular}
\end{table}

In the study we did not control for infinitive complements, but we ensured they did not influence the results (see more in Section \ref{Results and discussion}).

\section{Results and discussion}
\label{Results and discussion}
\subsection{Data distribution}

We now discuss the distributions of the independent variables in the context of the studied dependent variable, which was the presence of CC. The analysed data set comprised 2337 observations in total. CC occurred in 1850 cases, while in 477 cases, that is 20\%, it did not occur. 1566 observations originated from Forum, and 761 from Riznica and CNC. This difference in the number of retrieved examples also corresponds to the difference in the size of the corpora used. 

From the list of CTPs, we did not retrieve occurrences of \textit{stidjeti se} ‘be ashamed’ in any of the three queried patterns, while the verbs \textit{sramiti se} ‘be ashamed’ and \textit{kretati} ‘go, to start’ were identified only in Forum. The size of samples obtained for different CTPs differed drastically. We identified 1027 observations of raising CTPs, 1118 of simple subject control and only 182 of reflexive subject control predicates. The frequencies for individual lexemes are shown below in Figure \ref{F15.1}. 

The distribution is quite proportional to the absolute frequency of the lexemes in the whole hrWaC presented in Table \ref{T12.3}, but it does not precisely follow the same order. Simple subject control predicates are generally less frequent (with the exception of the verbs \textit{željeti} ‘wish’ and \textit{znati} ‘know’) than raising predicates, and reflexive subject control predicates are even less frequent than simple subject control predicates (with the exception of the verb \textit{truditi se} ‘try’). Because of the overall differences in the frequencies of particular syntactic types, it is completely impossible to build frequency triplets with the three types of predicates. We elaborate further on that problem in Section \ref{Selection of matrix verbs}.

\begin{figure}
\caption{Distribution of different CTP lexemes in the data set: abscissa – number of observations per CTP lexeme, ordinate – CTP lexemes chosen for study.}
\label{F15.1}
\includegraphics[width=.60\textwidth]{F151}
\end{figure}

Since infinitive complements were not restricted in the query, we did not use them as independent variables, but we examined their distribution in order to exclude the possibility of their significant impact on our results (e.g. we checked whether a clear pattern for a particularly frequent complement did not dominate the data). In the data we identified 837 distinct infinitive complements, the five most frequent being: \textit{baviti se} ‘be occupied with’ (3\%), \textit{vratiti se} ‘return’ (3\%), \textit{držati} ‘hold’ (1.7\%), \textit{nositi} ‘carry’ (1.4\%) and \textit{dati} ‘give’ (1.3\%).

Figure \ref{F15.2} shows the CL position for all studied CTPs. The plots on the left represent Forum, the plots on the right, Standard (Riznica $+$ CNC). Raising and simple subject control predicates show a strong tendency to appear in CC constructions, in contrast to reflexive subject control predicates, which show the opposite trend. This is particularly visible in the case of the only well-represented verb, \textit{truditi se} ‘try’, which has the same strong preference for not climbing in both types of corpora. Also, at first glance the climbing seems  more frequent in the observations from standard language corpora for all three types of predicates. The only exceptions seem to be the reflexive subject control matrix predicate \textit{truditi se} ‘try’ and \textit{libiti se} `hesitate', which occur more often in noCC than in CC structures even in standard corpora. Furthermore, it is worth pointing out that structures without CC in the Forum subcorpus are distinctly more frequent in the case of the raising CTPs \textit{trebati} ‘have to’ and \textit{moći} ‘can’, and the simple subject control CTP \textit{željeti} ‘wish/want’.

\begin{figure}
\caption{Verb-specific CL positioning across different CTP types and corpora}
\label{F15.2}
\includegraphics[width=.97\textwidth]{F152}
\end{figure}

In standard corpora, the raising CTP \textit{smjeti} ‘be allowed’ and simple subject control CTP \textit{znati} ‘know’ are attested only in CC structures, as in examples (\ref{(15.3)}) and (\ref{(15.5)}) below. In Forum, both variants are attested for these verbs. Examples (\ref{(15.4)}) and (\ref{(15.6)}) show the noCC structures.

\begin{exe}\ex\label{(15.3)}
\gll [\dots] ne smije\textsubscript{1} \textbf{se}\textsubscript{2} više baviti\textsubscript{2} čokoladom. \\
{} \textsc{neg} be.allowed.\textsc{3prs} \textsc{refl} more occupy.\textsc{3sg} chocolate \\
\glt ‘[\dots] he is not allowed  to be in the chocolate business anymore.’
\hfill [Riznica]

\ex\label{(15.4)}
\gll Apsolutno smiju\textsubscript{1} baviti\textsubscript{2} \textbf{se}\textsubscript{2} {znanošću  [\dots].} \\
absolutely be.allowed.\textsc{3prs} occupy.\textsc{inf} \textsc{refl} science \\
\glt ‘They are absolutely allowed to pursue science [\dots].’
\hfill [hrWaC v2.2]

\ex\label{(15.5)}
\gll [\dots] koja \textbf{se}\textsubscript{2} ne zna\textsubscript{1} vratiti\textsubscript{2} u svoje toplo gnijezdo. \\
 {} which \textsc{refl} \textsc{neg} know.\textsc{3prs} come.back.\textsc{inf} in own warm nest \\
\glt ‘[\dots] which does not know how to come back to its warm nest.’
\hfill [Riznica]

\ex\label{(15.6)}
\gll Mačke obično znaju\textsubscript{1} vratiti\textsubscript{2} \textbf{se}\textsubscript{2} {kući  [\dots].} \\
 cats usually know.\textsc{3prs} come.back.\textsc{inf} \textsc{refl} home  \\
\glt  ‘Cats usually know how to come back home [\dots].’
\hfill [hrWaC v2.2]
\end{exe}

\noindent Reflexive subject control predicates \textit{sramiti} \textit{se} ‘be ashamed’ and \textit{sjetiti se} ‘remember’ were attested only in noCC structures in Forum, as in examples (\ref{(15.7)}) and (\ref{(15.8)}). The CTP \textit{sramiti se} ‘be ashamed’ was attested only twice in our data, and only in Forum. The two utterances with \textit{sramiti se} represent pseudo-twin structures – both matrix predicate and infinitive complement are reflexive (see example (\ref{(15.7)})). As explained in some detail in Section \ref{Pseudo-twins}, these structures allow only pseudodiaclisis or haplology, and not a mixed cluster with two reflexive CLs.\footnote{For basic information on pseudodiaclisis see Section \ref{Diaclisis and pseudodiaclisis}.}\textsuperscript{,}\footnote{For basic information on haplology see Section \ref{Morphonological processes within the cluster}.}\textsuperscript{,}\footnote{This was also confirmed in our psycholinguistic study presented in Chapter \ref{Experimental study on constraints on clitic climbing out of infinitive complements}.} The second predicate, \textit{sjetiti se} ‘remember’, was also attested in a pseudo-twins structure in Forum, and additionally for infinitive complements with pronominal CLs, as in (\ref{(15.8)}), but not in constructions with mixed clusters containing reflexive matrix and pronominal infinitive CLs. Mixed clusters were attested with this CTP in the standard Croatian variety, see example (\ref{(15.9)}). Moreover, in the case of \textit{sjetiti se} ‘remember’, in standard corpora of Croatian mixed clusters like (\ref{(15.9)}) are more frequent than pseudodiaclisis structures.

\begin{exe}\ex\label{(15.7)}
\gll [\dots] ili \textbf{se}\textsubscript{1} ti sramiš\textsubscript{1} maknuti\textsubscript{2} \textbf{se}\textsubscript{2} {iz  [\dots].} \\
{} or \textsc{refl} you be.ashamed.2\textsc{prs} move.away.\textsc{inf} \textsc{refl} from  \\
\glt ‘[\dots] or you are ashamed to move away from [\dots].’
\hfill [hrWaC v2.2]

\ex\label{(15.8)}
\gll [\dots] a sjetim\textsubscript{1} \textbf{se}\textsubscript{1} upisati\textsubscript{2} \textbf{ga}\textsubscript{2} na {forum  [\dots].}\\
{} and remember.1\textsc{prs} \textsc{refl} register.\textsc{inf} him.\textsc{acc} on forum\\
\glt ‘[\dots] and I remember to register it on the forum [\dots].’
\hfill [hrWaC v2.2]

\ex\label{(15.9)}
\gll [\dots] i rijetko \textbf{ga}\textsubscript{2} \textbf{se}\textsubscript{1} tko sjeti\textsubscript{1} pozvati\textsubscript{2}  na  premijeru. \\
 {} and rarely him.\textsc{acc} \textsc{refl} who remember.3\textsc{prs} invite.\textsc{inf} on premiere \\
\glt ‘[\dots] and people rarely remember to invite him to a premiere.’
\hfill [CNC]
\end{exe}

\noindent We retrieved only one occurrence of the reflexive subject control predicate \textit{libiti se} ‘hesitate’ from standard corpora. It was attested as a noCC structure in CNC: see example (\ref{(15.10)}). However, in Forum this predicate was attested not only in noCC, but also in CC structures, as shown in example (\ref{(15.11)}).

\begin{exe}\ex\label{(15.10)}
\gll [\dots] da \textbf{se}\textsubscript{1} ne libi\textsubscript{1} upotrijebiti\textsubscript{2} {\textbf{ga}\textsubscript{2}  [\dots].} \\
  {} that \textsc{refl} \textsc{neg} hesitate.3\textsc{prs} use.\textsc{inf} him.\textsc{acc}  \\
\glt ‘[\dots] that he should not hesitate to use it [\dots].’
\hfill [CNC]

\ex\label{(15.11)}
\gll [\dots] pa da \textbf{mu}\textsubscript{2} \textbf{se}\textsubscript{1} ne libe\textsubscript{1} staviti\textsubscript{2} brnjicu. \\
{} so that him.\textsc{dat} \textsc{refl} \textsc{neg} hesitate.3\textsc{prs} put.\textsc{inf} muzzle \\
\glt ‘[\dots] so that they should not hesitate to muzzle him.’
\hfill [hrWaC v2.2]
\end{exe}

\noindent We now move to the type and case of infinitive complement CL. Figure \ref{F15.3}. presents CL type distribution across CTP types and corpora. In all, the \textsc{refl\textsubscript{lex}} CL \textit{se} is the most frequent ($N=1026$), while the pronominal ($N=691$) and the \textsc{refl\textsubscript{2nd}} CLs \textit{se} and \textit{si} have similar distributions ($N=610$).

\begin{figure}
\caption{Type-specific CL positioning across different CTP types and corpora}
\label{F15.3}
\includegraphics[width=.97\textwidth]{F153}
\end{figure}

If we consider the size of the retrieved samples, all three types of CLs are used similarly frequently in both types of corpora. The main difference in distribution concerns predicate type. In the sample of reflexive subject CTPs, we retrieved mainly structures with pronominal infinitive CLs. Sentences with a reflexive subject CTP and reflexive infinitive CL such as the one in (\ref{(15.12)}) are very rare in our data. This example contains the only occurrence of the \textsc{refl\textsubscript{lex}} infinitive CL \textit{se} appearing in pseudodiaclisis in the reflexive subject control sentences retrieved from standard corpora. 

\protectedex{\begin{exe}\ex\label{(15.12)}
\gll [\dots] te \textbf{se}\textsubscript{1}  usuđuje\textsubscript{1} oprijeti\textsubscript{2} \textbf{se}\textsubscript{2} Tvom pozitivnom  {nalogu [\dots].}  \\
{} and \textsc{refl} dare.3\textsc{prs} withstand.\textsc{inf} \textsc{refl} your positive ordering \\
\glt ‘[\dots] and it (council) dares to oppose your express orders [\dots].’
\hfill [Riznica]
\end{exe}
}

\noindent The possible reasons for this may be the syntactic rarity of the combination of a reflexive subject control predicate with a lexical reflexive complement or the usage of a competing construction such as haplology or \textit{da}\textsubscript{2}-construction.

CC dominates for all types of CLs in the case of raising and simple subject control predicates. The noCC structures seem to be slightly more pronounced in Forum. In the sample of reflexive subject control predicates, pronominal CLs tend not to climb (only 39 of 154 CLs climb, that is, 25\%), see examples (\ref{(15.8)}) and (\ref{(15.10)}). However, unlike pronominal CLs which can climb out of infinitive complements of reflexive subject control predicates, reflexive CLs do not climb at all: compare examples (\ref{(15.9)}), (\ref{(15.11)}), (\ref{(15.13)}), and (\ref{(15.14)}) on the one hand with examples (\ref{(15.7)}) and (\ref{(15.12)}) on the other hand.

\begin{exe}\ex\label{(15.13)}
\gll A  ja \textbf{ih}\textsubscript{2} \textbf{se}\textsubscript{1} bojim\textsubscript{1} drzati\textsubscript{2} skupa. \\
and I them.\textsc{acc} \textsc{refl} be.afraid.3\textsc{prs} keep.\textsc{inf} together \\
\glt  ‘And I am afraid to keep them together.’
\hfill [hrWaC v2.2]

\ex\label{(15.14)}
\gll [\dots] ali bojim\textsubscript{1} \textbf{joj}\textsubscript{2} \textbf{se}\textsubscript{1} jos davati\textsubscript{2} ljudsku {hranu [\dots].} \\
{} but be.afraid.3\textsc{prs} her.\textsc{dat} \textsc{refl} still give.\textsc{inf} human food \\
\glt ‘[\dots] but I am still afraid to give her food for humans [\dots].’
\hfill [hrWaC v2.2]
\end{exe}

\noindent These differences suggest that CL type is a constraint on CC in the case of reflexive subject control predicates. This is tested further in the next section

Finally, we present the distribution of case across predicate types and corpora. In general, accusative CLs ($N=963$) appeared three times as often as dative CLs ($N=338$). Since reflexive CLs seem to be distributed differently for reflexive subject control verbs, in the plot we distinguish the case of pronominal CLs and of \textsc{refl\textsubscript{2nd}} CLs too. This is shown in Figure \ref{F15.5}.

\begin{figure}
\caption{Case-specific CL positioning across different CTP types and corpora}
\label{F15.5}
\includegraphics[width=.97\textwidth]{F155}
\end{figure}

When examining CL type, we see that cases are not distributed equally across types – dative is more frequent as a pronominal case ($N=252$) than as the case of \textsc{refl\textsubscript{2nd}} CLs ($N=86$). Accusative is used 439 time as the case of pronominal CLs, and 524 times for \textsc{refl\textsubscript{2nd}} CLs. In general, the usage of CL cases is quite similar in both corpora. Nevertheless, the prevailing part of observations concerning dative reflexive CLs is from Forum ($N=78$), whereas the standard corpora yielded only 8 occurrences. Closer inspection of the data reveals further interesting differences. While in the standard corpora the \textsc{refl\textsubscript{2nd}} CL \textit{si} is a complement of infinitives which have an obligatory dative argument, such as \textit{dopustiti} ‘allow’ (\ref{(15.15)}) and \textit{priuštiti} ‘afford’ (\ref{(15.16)}), the same CL is used in Forum as a complement of infinitives such as \textit{kupiti} ‘buy’ (\ref{(15.17)}) and \textit{obnoviti} ‘renew’ (\ref{(15.18)}). In the Croatian standard variety a dative complement of these infinitives is not obligatorily expressed when it refers to the subject itself, but it is usually inferred.\footnote{Petar Vuković (p.c.) claims that precisely such constructions \textcolor{black}{with overtly expressed dative complement} are features of the non-standard variety.}

\begin{exe}\ex\label{(15.15)}
\gll [\dots] ne mogu\textsubscript{1} \textbf{si}\textsubscript{2} vise dopustiti\textsubscript{2} {luksuz [\dots].} \\
{} \textsc{neg} can.3\textsc{prs} \textsc{refl} more allow.\textsc{inf} luxury  \\
\glt ‘[\dots] they cannot allow themselves the luxury [\dots] .’
\hfill [Riznica]

\ex\label{(15.16)}
\gll [\dots] Hrvatice \textbf{si}\textsubscript{2} smiju\textsubscript{1} priustiti\textsubscript{2} {poraz [\dots].} \\
{} Croatians \textsc{refl} be.allowed.3\textsc{prs} afford.\textsc{inf} defeat \\
\glt  ‘[\dots] Croatians can afford to be defeated [\dots] .’
\hfill [Riznica]

\ex\label{(15.17)}
\gll [\dots] a i ja \textbf{si}\textsubscript{2} planiram\textsubscript{1} kupiti\textsubscript{2} {jedan [\dots].} \\
{} and and I \textsc{refl} plan.1\textsc{prs} buy.\textsc{inf} one \\
\glt ‘[\dots] and I am planning to buy myself one too [\dots] .’
\hfill [hrWaC v2.2]

\ex\label{(15.18)}
\gll [\dots] i moram\textsubscript{1} \textbf{si}\textsubscript{2} malo obnoviti\textsubscript{2} garderobu. \\
 {} and must.1\textsc{prs} \textsc{refl} little renew.3\textsc{prs} wardrobe.\textsc{inf} \\
\glt ‘[\dots] and I have to renew my wardrobe a little bit.’
\hfill [hrWaC v2.2]
\end{exe}

\noindent Further, a closer look at our data reveals that in standard corpora the \textsc{refl\textsubscript{2nd}} CL \textit{si} was not attested at all with reflexive subject control predicates, and with subject control and raising predicates it was attested only in CC structures (for the latter see example (\ref{(15.15)})). In Forum this CL was attested in both CC and noCC structures with raising and subject control predicates. Additionally, in Forum the \textsc{refl\textsubscript{2nd}} CL \textit{si} was also attested in a sentence with a reflexive subject control predicate, but, as expected, in a noCC structure: see example (\ref{(15.19)}) below. 

\protectedex{\begin{exe}\ex\label{(15.19)}
\gll [\dots] tko \textbf{se}\textsubscript{1} usudjuje\textsubscript{1} dati\textsubscript{2} \textbf{si}\textsubscript{2} nadimak Master {of [\dots].} \\
{} who \textsc{refl} dares.3\textsc{prs} give.\textsc{inf} \textsc{refl} alias Master of \\
\glt ‘[\dots] who dares call himself Master of  [\dots].’
\hfill [hrWaC v2.2]
\end{exe}
}

\noindent Moreover, we would like to point out that Figure \ref{F15.5} does not reveal any striking differences between the accusative and the dative in relation to CC. CC is the more frequent construction for both cases, in both types of corpora for raising and simple subject control predicates, and the less frequent construction for both cases in both corpora types for reflexive subject control predicates. Summing up, we observe that the behaviour of CLs belonging to complements of reflexive subject control predicates shows an opposite trend as to CC than the other two types of CTPs. Reflexive CLs are generally rare and do not climb to the matrix at all. We see that CC is a slightly more unified mechanism in corpora representing standard language than in Forum. CL case does not seem to make any difference to CC as long as the CTP type is held constant, but climbing of reflexive CLs in the group of reflexive subject control CTPs does not seem to occur.

\subsection{Testing correlations with a logistic regression model}
\label{Testing correlations with a logistic regression model}
\subsubsection{Complement-taking predicate type and corpus type}
In order to statistically test the relationships between CC, CTP type, infinitive CL type and case, and corpus type discussed in the previous subsection, we used logistic regression models with CTP lexemes as random effects. For our calculations we used the generalised linear mixed model fit by maximum likelihood from the lme4 R-package \citep*{BKVB15}. The first model covered CTP type and corpus type. The remaining variables, type and case of the infinitive CL, were tested separately for two reasons. First, a model that includes case should include only CLs marked for case to avoid interaction with CL type. Second, we have very few observations for reflexive CLs of infinitive complements in sentences with reflexive subject control CTPs. The results are reported in Table \ref{T15:r}.\footnote{The model formula is: $\text{CC} \sim \text{CtpType} * \text{CorpusType} + (1|\text{CtpVerb})$; for explanation of statistical measures in Table \ref{T15:r} \textcolor{black}{and significance codes} see Appendix \ref{s}.}

\begin{table}
\caption{Generalised mixed effects regression model\label{T15:r}}
\begin{tabular}{lrrrr@{\,}l}
\lsptoprule
Random effects & \\
Groups  Name & \multicolumn{1}{c}{Var.} & \multicolumn{1}{c}{SD} &&& \\\midrule
CtpVerb (Intercept) &  0.4854 & 0.6967 &&& \\
\multicolumn{5}{l}{Number of obs: 2327, groups:  CtpVerb, 23} \\\midrule
Fixed effects &Est.& \multicolumn{1}{c}{SE}& \multicolumn{1}{c}{$z$} & $\text{Pr}(>|z|)$ &\\\midrule
(intercept CC-yes, raising, forum) & 1.7498 &  0.2918 & 5.997 & $<0.001$ & *** \\
simple subject  & $-0.1784$ & 0.3969 & $-0.449$ & 0.653 & \\
reflexive subject & $-2.9451$ & 0.5012 & $-5.877$ & $<0.001$ & ***\\
standard  & 1.5790 & 0.2489 & 6.343 & $<0.001$ & *** \\
simple subject and standard & $-0.7625$ & 0.3306 & $-2.307$ & 0.021 & * \\
reflexive subject and standard & $-0.7246$ & 0.4978 & $-1.456$ & 0.146 & \\
%\tablevspace
%\multicolumn{6}{l}{Signif. codes: 0 '***' 0.001 '**' 0.01 '*' 0.05 '.' 0.1 ' ' 1} \\
\lspbottomrule
\end{tabular}
\end{table}

The results of the first model confirm our preliminary observations – corpus type and predicate type (reflexive subject control versus others) influence the probability of CC occurring in a sentence. We elaborate shortly on them. 

The intercept in the model is CC occurring for raising CTPs in Forum and is used as a reference level for effects. The estimate of the intercept which is log odd can be recalculated to probability.\footnote{The formula looks as follows: $P=e^{\log O} / (1+e^{\log O})$, where P -- probability, O -- odds.} That is, the chance of CC occurring when a raising CTP is used in colloquial Croatian is 0.85. The other estimates refer to the change in log odds when particular effects are compared with the intercept. Thus, in colloquial Croatian there is no substantial difference between raising and simple subject CTPs, but there is only a 0.23 chance of CC occurring with a reflexive subject control CTP. 

The change from colloquial to standard Croatian is significant, and has a positive effect on CC in the presence of raising CTPs. Namely, chances of CC increase to 0.96. This increase in probability also applies to simple subject control verbs, but the increase is significantly lower than for raising CTPs: the probability of CC is only 0.94. The change from the Forum subcorpus to standard Croatian has little impact on probability of CC in sentences with reflexive subject control verbs.

\subsubsection{Infinitive clitic case and type}
We built separate models for type and case of infinitive CLs; however, neither of them yielded any significant differences. Thus, for raising and simple subject control predicates neither the case of pronominal and reflexive infinitive CLs nor the type of infinitive CL appears to be a relevant factor influencing CC. The small number of observations for infinitive reflexive CLs in constructions with reflexive subject predicates leads to the conclusion that in the case of infinitive complements these CLs are haplologised (i.e., omitted), or that an alternative construction, for example, with a \textit{da}\textsubscript{2}-complement, is used.

\section{Conclusions}
\label{Conclusions:inf}
This study gives the following answers to our research questions: 

\begin{enumerate}[label=A\arabic*:]
\item The difference between raising and simple subject control CTPs is statistically significant for CC only in standard Croatian, while CC with reflexive subject control CTPs is significantly less frequent than with the other two types of CTPs regardless of diaphasic variety. 

\item The type of infinitive CL does not influence CC out of single infinitive complements either in standard or in colloquial Croatian.

\item The case of the infinitive CL is not a relevant factor for CC out of single infinitive complements either in standard or in colloquial Croatian.

\item Diaphasic variation is a significant factor influencing the probability of CC. Unlike in Romance languages, in Croatian CC appears more frequently in the standard than in the colloquial variety. 
\end{enumerate}

These findings allow some tentative observations to be made, which should feed into future research. Although in standard Croatian CC out of a single infinitive complement appears highly probable with raising CTPs, it does not seem to be absolutely obligatory (pace \citealt{Aljovic05}, in accordance with \citealt*{HKJ18}). Colloquial language in particular allows the lack of CC to a certain degree. This tendency, however, is not universal to CC languages since Romance languages exhibit the opposite trend – CC is significantly more frequent in colloquial language, whereas in formal language CLs are more likely to appear in noCC constructions.

Furthermore, our assumption that a differentiation of simple and reflexive subject control CTPs hitherto neglected in theoretical syntactic research on CL could actually shed new light on mechanisms of CC is justified. In order to get more data, we further explore the possibilities of CC in the context of reflexive subject control CTPs in Chapter \ref{Experimental study on constraints on clitic climbing out of infinitive complements}, where we report a psycholinguistic experiment. As in the case of reflexive subject control CC inevitably leads to mixed CL clusters, we might conjecture that there could exist a strategy to avoid such mixed clusters. Therefore, in order to broaden the database to include structures which might lead to such mixed clusters, in the next chapter object control predicates are studied in addition to reflexive subject control predicates.
