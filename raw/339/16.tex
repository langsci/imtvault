\chapter[On the heterogeneous nature of constraints on clitic climbing]
        {On the heterogeneous nature of constraints on clitic climbing: Complexity effects}
\label{On the heterogeneous nature of constraints on clitic climbing: complexity effects}
\section{Introduction}

In this chapter we summarise our main findings concerning constraints on CC. We also offer an explanation of this phenomenon in more general terms. We draw our conclusions from the theoretical literature and informal judgments described in Chapter \ref{Constraints on clitic climbing in Czech compared to Bosnian, Croatian and Serbian (theory and observations)}, as well as the empirical studies in Chapters \ref{A corpus-based study on CC in da constructions and the raising-control distinction (Serbian)}--\ref{Experimental study on constraints on clitic climbing out of infinitive complements}. Note that our empirical studies focus exclusively on CLs in the context of matrix embedding structures. Triangulation of methods, in which we compare observations from other studies with empirical results from our corpus studies and from a psycholinguistic experiment, gives us a very interesting picture of language production (real language usage) and language comprehension (what is judged acceptable).\footnote{For more information \textcolor{black}{on triangulation of methods} see Section \ref{Triangulation of methods}.} As we explained in  Section \ref{Pros and cons of judgment data}, the psycholinguistic experiment is necessary since it allows rare phenomena to be investigated and negative data to be obtained \citep[cf.][100]{Hoffmann13}. We are also interested in the relationship between frequencies obtained in the corpus studies and the acceptance rates from the psycholinguistic experiment, as the connection between low frequency and acceptability is a contentious issue in linguistics (see \citealt{BermelKnittl12}, \citealt{Divjak17}).

 \textcolor{black}{As we discussed in Chapter \ref{Approaches to clitic climbing} scholars working within formal-theoretical frameworks disagree as to optionality \citep[][]{Progovac93, Progovac96, CavarWilder94, Stjepanovic04} and obligatoriness \citep{Aljovic05} of CC in BCS upon restructuring. In our empirical studies, we identified a certain variation which lets us conclude that even upon restructuring contexts CC is not obligatory. We observe that the significant factors influencing the probability of CC are very heterogeneous in their nature. Therefore, we go away from established methods of analysing the mechanism of clitic climbing towards the new paths which probabilistic syntax \citep{Manning02} offers. In our view, constraints on CC are best interpreted in the context of complexity, which allows us to include also non-systemic sources of variation. It also enables constructing a series of hierarchies where the factors relevant for predicting clitic climbing interact with each other.}

This chapter is structured as follows. In Section \ref{Complexity} we describe the concept of complexity. In the next three sections we discuss in detail the results of our study in the context of different types of complexity. In Section \ref{Systemic constraints related to the embedding type} we focus on systemic constraints related to embedding type. In Section \ref{Systemic constraints related to the interaction of matrix and embedding} we proceed to systemic constraints related to the interaction of matrix and embedding. Section \ref{Non-systemic factors related to the diaphasic dimension} explores the non-systemic factors constraining CC. In Section \ref{Interaction, optionality an preferences} we summarise our findings and draft a model of constraints based on the notion of complexity.

\section{Complexity}
\label{Complexity}
\subsection{The complexity of a system}
\label{The complexity of a system}
We put forward the hypothesis that the mechanism behind CC and its constraints could best be explained as a certain type of complexity effect in the sense that the complexity of the sentence structures involved is the driving force for CC or blocking of CC.

Before we present our thoughts on these complexity effects, we have to discuss the somewhat elusive term of ``complexity'' as such. As \citet[][118]{Pallotti15} shows, in the linguistic literature complexity has many competing meanings which tend to be confused. For instance, the formal properties of a construction are frequently identified with issues of difficulty or costs for the language user or learner. To avoid this polysemy, we follow \citet[][vii]{KMS08}, who use the philosophical approach by \citet[][]{Rescher98} in order to disentangle the bewildering range of studies on linguistic complexity. The point of departure is Rescher’s definition of the \textsc{complexity} of a system:

\begin{quotation}
A system’s complexity is a matter of the quantity and variety of its constituent elements and of the interrelational elaborateness of their organizational and operational make-up.\hfill\hbox{\citet[][1]{Rescher98}}\hbox{}
\end{quotation}

\citet[][vii]{KMS08} deplore that this approach has not come to the fore in linguistic debates on complexity. We agree with the authors that it is well suited to application to syntactic structures (and beyond). Therefore, we will try to apply this precise and comprehensive approach to linguistic data, especially to clause structures relevant to the constraints on CC. Further, \citet[][]{Rescher98} makes a primary distinction between the complexity of systems themselves (ontological modes of complexity) and the complexity of how knowledge about systems can be presented (epistemic modes of complexity).

\subsection{Ontological modes of complexity}
\label{Ontological modes of complexity}
In the current discussion we focus mainly on the ontological modes of complexity. Three perspectives may be offered on a system’s complexity: compositional, structural and functional. Compositional complexity refers to parts of a system, structural complexity applies to the way the parts of a system can be combined, while functional complexity measures the variety of roles and contexts a system can be applied to. Below, we list the subtypes of complexity relevant to the following discussion as defined by \citet[][9]{Rescher98} together with the modes of complexity they belong to:

\begin{enumerate}\sloppy
\item \textsc{constitutional} \textsc{complexity} defined as “the number of constituent elements or components” (a subtype of compositional complexity);
\item \textsc{organisational} \textsc{complexity} defined as “variety of different possible ways of arranging components in different modes of interrelationship” (a subtype of structural complexity);
\item \textsc{operational} \textsc{complexity} understood as “variety of modes of operation or types of functioning” (a subtype of functional complexity). 
\end{enumerate}

From the above, it is clear that there is no one absolute mode of complexity, but instead different modes of complexity interact with each other. For example, we can say that the constitutional complexity of pronominal CLs is higher than that of reflexive CLs. This is because they encode case and number, and for the third person singular also gender. This leads to a great number of unique items, and thus constitutional complexity decreases operational complexity of pronominal CLs, since each pronominal CL form has a narrow scope of reference. In contrast, the operational complexity of reflexive CLs is higher than that of pronominal CLs, as the number of functional contexts in which they can appear is greater (see Section \ref{Different types of reflexives}).

We use complexity-related terms exclusively in relation to clause structures. We do not say anything about comprehension difficulty or about “overall complexity” ascribed to a language as a whole. With respect to structural complexity we can state that structures where both CC and noCC are possible are characterised by a higher grade of organisational complexity in comparison to those where only CC or only noCC is allowed. In the following, we claim that certain types of complexity in a construction decrease the organisational complexity related to CL positioning. Next, we would like to discuss how complexity effects can explain differences in positioning of CLs belonging to embedded structures, i.e., CC vs noCC structures.

As outlined in Section \ref{Systemic vs functional microvariation}, we distinguish systemic and non-systemic microvariation. In the particular case of CC, the former is defined as variation between the dependent variable CL position (CC vs noCC) and independent variables encoded in the linguistic context. The latter is defined as variation between the dependent variable CL position and independent variables related to space (diatopic variation) or to the modes of language use in different situations (e.g. oral vs written, diaphasic variation). Accordingly, we identify systemic and non-systemic constraints. 

Following these lines of thought we propose two different types of systemic constraints. First, constraints related to the syntactic environment of the embedding. Second, constraints related to the matrix predicate which can potentially open a slot for a climbing CL. The constraints related to the matrix predicate are further subdivided into constraints related to predicate type with respect to the raising--control dichotomy and constraints related to the slot in the CL cluster in the matrix. 

\citet[][11]{Rescher98} points out that “[a] complex system that embodies subsystems can be organized either hierarchically through their subordination relations among its elements or coordinatively through their reciprocal interrelationships”. Here, we argue that some constraints on CC form hierarchical relationships: that is, they nest other constraints. Others operate as coordinative structures: that is, they interact with each other. The former applies to the type of embedding discussed in the next section. The latter applies to the constraints where predicate type is involved. These constraints, however, should not be seen as ultimate laws blocking CC, but rather as factors which each have a certain impact on the likelihood that CC will occur. Based on the findings from previous chapters we now discuss the individual types of constraints.

\section{Systemic constraints related to the embedding type}
\label{Systemic constraints related to the embedding type}
\subsection{Islands}
\label{Islands}

In Chapter \ref{Constraints on clitic climbing in Czech compared to Bosnian, Croatian and Serbian (theory and observations)}, following \citet[][245]{FranksKing00} we used the concept of island for phrases showing a specific locality constraint on CC. In further chapters we observed that the spectrum of syntactic variation between embeddings as to the constraints on CC is not necessarily dichotomous. Between structures which completely block CC and structures like infinitive complements, which are the most suitable syntactic contexts for climbing, there is space for \textit{da}\textsubscript{2}-con\-struc\-tions.\footnote{For more information on \textit{da}\textsubscript{2}-complements see Section \ref{Types of complements}.} Though less conducive to CC than the latter type of phrase, they do allow it to some extent. Therefore, we propose to distinguish two types of islands which we call \textsc{tied} \textsc{islands} and \textsc{true islands}. 

\subsection{Tied islands}
\label{Tied islands}

In Chapter \ref{A corpus-based study on CC in da constructions and the raising-control distinction (Serbian)} we analysed the behaviour of CLs in \textit{da}\textsubscript{2}-constructions in more detail. Our data revealed that such clauses, with verbs inflected for person and number but not for tense, should not generally be seen as islands preventing CLs from climbing. CC turned out to be marginally possible, but only with raising and subject control matrices like in (\ref{(14.11rep)}) where the pronominal CL \textit{im} ‘them’ climbs out of the \textit{da}\textsubscript{2}-complement.

\protectedex{\begin{exe}\ex\label{(14.11rep)}
\gll [\dots] počeo\textsubscript{1} \textbf{im}\textsubscript{2} \textbf{je}\textsubscript{1} da govori\textsubscript{2} o dolasku ove grupe.\\
 {} start.\textsc{ptcp}.\textsc{sg}.\textsc{m} them.\textsc{dat} be.3\textsc{sg} that speak.3\textsc{prs} about arrival this group\\
\glt ‘[\dots] he began to speak to them about the arrival of this group.’ \\
\strut \hfill [srWaC v1.2]
\end{exe}
}

\noindent Therefore, in this case the term ``island'' coined by \citet[][]{Ross67} is not really appropriate. Extending his metaphor somewhat further, we would like to introduce a new term: \textsc{tied island}. Like pieces of land surrounded by water which are connected to the mainland by a tombolo, i.e. a spit of beach materials, syntactic tied islands allow only very restricted movement of CLs. 

\subsection{True islands}
\label{True islands}

\textsc{True islands} are attested not only in matrix embedding structures but also in adjuncts (gerund phrases) and adjective phrases in the attributive function. We analyse true islands on the basis of a qualitative comparison with Czech, supplemented with some data from informal acceptability judgments. As our first tentative data from Chapter \ref{Constraints on clitic climbing in Czech compared to Bosnian, Croatian and Serbian (theory and observations)} suggest, CLs cannot climb out of the following embeddings:

\begin{enumerate}
\item infinitives in comparative sentences with \textit{nego} ‘than’

\begin{exe}\ex
\begin{xlist}
\ex[]{\label{(11.109)}
\gll Nisam\textsubscript{1} imao\textsubscript{1} izbora\textsubscript{1} nego prodati\textsubscript{2} \textbf{ga}\textsubscript{2}. \\
 \textsc{neg}.be.1\textsc{sg} have.\textsc{ptcp}.\textsc{sg}.\textsc{m} choice than sell.\textsc{inf} him.\textsc{acc} \\ }
\ex[*]{\label{(11.109brep)}
\gll Nisam\textsubscript{1} \textbf{ga}\textsubscript{2} imao\textsubscript{1} izbora\textsubscript{1} nego prodati\textsubscript{2}. \\
\textsc{neg}.be.1\textsc{sg} him.\textsc{acc} have.\textsc{ptcp}.\textsc{sg}.\textsc{m} choice than sell.\textsc{inf}  \\}
\end{xlist}
\glt ‘I had no choice but to sell him (a football player).’
\hfill [bsWaC v1.2]
\end{exe}

\item embedded wh-infinitives

\begin{exe}\ex
\begin{xlist}
\ex[]{\label{(11.44)}
\gll Mila \textbf{je}\textsubscript{1} odlučila\textsubscript{1} kome \textbf{ga}\textsubscript{2} preporučiti\textsubscript{2}. \\
 Mila be.3\textsc{sg} decide.\textsc{ptcp}.\textsc{sg}.\textsc{f} who him.\textsc{acc} recommend.\textsc{inf}\\}
\ex[*]{\label{(11.44brep)}
\gll Mila \textbf{ga}\textsubscript{2} \textbf{je}\textsubscript{1} odlučila\textsubscript{1} kome preporučiti\textsubscript{2}.\\
 Mila him.\textsc{acc} be.3\textsc{sg} decide.\textsc{ptcp}.\textsc{sg}.\textsc{f} who recommend.\textsc{inf}\\}
\end{xlist} 
\glt ‘Mila decided to whom to recommend him.’
\hfill (BCS; \citealt[][8]{Aljovic05})
\end{exe}

\noindent Permutations of both structures show that CC leads to unacceptable sentences. Neither climbing of the accusative CL \textit{ga} ‘him’ generated in the \textit{nego} infinitive (\ref{(11.109brep)}) nor climbing of the same CL generated in the embedded infinitive headed by the wh-word \textit{kome} ‘whom’ is possible (\ref{(11.44brep)}). 

\item \textit{da}\textsubscript{1}-complements 

\noindent Although we do not discuss it in any detail, we assume that \textit{da}\textsubscript{1}-com\-ple\-ments, which unlike \textit{da}\textsubscript{2}-complements are also inflected for tense, function as an additional true island.\footnote{For more information on \textit{da}\textsubscript{1}-complements see Section \ref{Types of complements}.}\textsuperscript{,}\footnote{\textcolor{black}{The assumption that \textit{da}\textsubscript{1}-complements function as a true island is based on the study conducted by \citet*{HKJ16}}.} The \textsc{refl\textsubscript{lex}} CL \textit{se} cannot climb out of a future-tense \textit{da}\textsubscript{1}-complement and form a mixed cluster with the matrix CLs \textit{mi} ‘me’ and \textit{je} ‘is’ – see example presented in (\ref{(2.29rep)}) and its permutation (\ref{(2.29b)}).

\begin{exe}\ex
\begin{xlist}
\ex[]{\label{(2.29rep)}
\gll On \textbf{mi}\textsubscript{1} \textbf{je}\textsubscript{1} obećao\textsubscript{1} da \textbf{će}\textsubscript{2} \textbf{se}\textsubscript{2} vratiti\textsubscript{2} u {Kragujevac [\dots].} \\
 he me.\textsc{dat} be.3\textsc{sg} promise.\textsc{ptcp}.\textsc{sg}.\textsc{m} that \textsc{fut}.3\textsc{sg} \textsc{refl} return.\textsc{inf} in Kragujevac\\ }
\ex[*]{\label{(2.29b)}
\gll On \textbf{mi}\textsubscript{1} \textbf{se}\textsubscript{2} \textbf{je}\textsubscript{1} obećao\textsubscript{1} da \textbf{će}\textsubscript{2} vratiti\textsubscript{2} u {Kragujevac [\dots].} \\
 he me.\textsc{dat} \textsc{refl} be.3\textsc{sg} promise.\textsc{ptcp}.\textsc{sg}.\textsc{m} that \textsc{fut}.3\textsc{sg} return.\textsc{inf} in Kragujevac\\ }
\end{xlist}
\glt ‘He promised me that he would come back to Kragujevac [\dots].’ \\
\strut\hfill [srWaC v1.2]
\end{exe}
\end{enumerate}

\subsection{Complexity effects in embeddings}
\label{Complexity effects in embeddings}

We argue that in the case of true islands 1, 2 and 3 we are dealing with phrases which show a higher degree of constitutional complexity than simple infinitive complements in matrix complement structures. Namely, if we look closer at the examples in (\ref{(11.109)}) and (\ref{(11.44)}), we see that the number of constituent elements or components is higher than in simple infinitive complements. In example (\ref{(11.109)}) the infinitive phrase is headed by the comparative marker \textit{nego}, and in (\ref{(11.44)}) the infinitive phrase is headed by a wh-element. Furthermore, we can see structural similarities between the tied island described in Section \ref{Tied islands} and the three true islands: all four of them contain phrases headed with an element in initial position:

\begin{itemize}
\item \textit{da} $+$ semifinite verb (\textit{da}\textsubscript{2}-complement),
\item \textit{nego} $+$ infinitive,
\item wh-element $+$ infinitive,
\item \textit{da} $+$ finite verb (\textit{da}\textsubscript{1}-complement).
\end{itemize}

Further, we would argue that constitutional complexity also explains the general differences in the behaviour of the different types of complements in respect to CC. Infinitive complements are less complex than \textit{da}\textsubscript{2}-complements because they do not contain agreement marking. In other words, unlike in \textit{da}\textsubscript{2}-complements the infinitive is not marked for number and person. There are thus two grammatical markers less. This is a difference in constitutional complexity. In a \textit{da}\textsubscript{1}-complement such as the one in (\ref{(17.3)}) the verb is even more complex as it additionally contains a tense marker.\footnote{Our analysis correlates with works which assume that CLs climb from domains that are “functionally poor” like \citet[][]{Aljovic05}.} As an island it totally blocks CC.

The fact that \textit{da}\textsubscript{2}-complements do allow CC to a certain degree seems to contradict our claim concerning constitutional complexity. At second glance, however, it can be explained by the assumption that in contrast to the \textit{da} in \textit{da}\textsubscript{1}-complements, the wh-element and \textit{nego}, the \textit{da} in \textit{da}\textsubscript{2}-complements has lost its status as a complementiser. It has become a modal particle followed by a subjunctive, like the Albanian \textit{të}, Bulgarian \textit{da}, Greek \textit{na}, and Romanian \textit{să} \citep[see][]{Turano17}. As shown by \citet[][passim]{Joseph83} the replacement of the infinitive construction with a subjunctive introduced by a modal particle started spreading north from Middle Greek. In BCS it first developed in the eastern varieties, but later spread to other areas as well. \citet[][33f]{Markovic55}, for instance, observed a drastic increase in the frequency of the structure on Bosnian language territory in the last century. The development of the complementiser into the subjunctive particle can be explained as a process of contact-induced grammaticalisation; for more details and further studies on this change see \citet[][80--83]{WiemerHansen12}.

This assumption is supported by the fact that although both islands 1 and 2 are discussed above in the context of infinitives, they can also appear in the context of \textit{da}\textsubscript{2}-complements, as shown in (\ref{(17.2)}) and (\ref{(17.3)}) and in their respective permutations:

\begin{enumerate}
\item comparative sentences with \textit{nego}

\begin{exe}\ex
\begin{xlist}
\ex[]{\label{(17.2)}
\gll [\dots] i nije\textsubscript{1} \textbf{joj}\textsubscript{1} ostalo\textsubscript{1} drugo nego da \textbf{mu}\textsubscript{2} dozvoli\textsubscript{2} da ide\textsubscript{3} na Olimp. \\
{} and \textsc{neg}.be.\textsc{3sg} she.\textsc{dat} remain.\textsc{ptcp}.\textsc{sg}.\textsc{m} else than that he.\textsc{dat} allow.\textsc{3prs} that go.\textsc{3prs} on Olimp \\ }
\ex[*]{\label{(17.2b)}
\gll [\dots] i nije\textsubscript{1} \textbf{joj}\textsubscript{1} \textbf{mu}\textsubscript{2} ostalo\textsubscript{1} drugo nego da dozvoli\textsubscript{2} da ide\textsubscript{3} na Olimp \\
 {} and \textsc{neg}.be.\textsc{3sg} she.\textsc{dat} he.\textsc{dat} remain.\textsc{ptcp}.\textsc{sg}.\textsc{m} else than that allow.\textsc{3prs} that go.\textsc{3prs} on Olimp \\
\\ }
\end{xlist}
\glt ‘[\dots] and she had no choice but to allow him to climb Olympus.’  \\
\strut\hfill [srWaC v1.2]
\end{exe}

\item embedded wh-\textit{da}\textsubscript{2}-complements

\begin{exe}\ex
\begin{xlist}
\ex[]{\label{(17.3)}
\gll [\dots] i razmišljam\textsubscript{1} kome da \textbf{ga}\textsubscript{2} poklonim\textsubscript{2}. \\
{} and think.\textsc{1prs} who.\textsc{dat} that him.\textsc{acc} donate.\textsc{1prs} \\ }
\ex[*]{\label{(17.3b)}
\gll [\dots] i razmišljam\textsubscript{1} \textbf{ga}\textsubscript{2} kome da poklonim\textsubscript{2}. \\
{} and think.\textsc{1prs} him.\textsc{acc} who.\textsc{dat} that donate.\textsc{1prs} \\
}
\end{xlist}
\glt ‘[\dots] and I am thinking about whom I should give it to.’\\
\strut\hfill [srWaC v1.2]
\end{exe}
\end{enumerate}

\section{Systemic constraints related to the interaction of matrix and embedding}
\label{Systemic constraints related to the interaction of matrix and embedding}
\subsection{Constraints in the light of empirical evidence}
\label{Constraints in the light of empirical evidence}

Our discussion on systemic constraints related to the interaction of the matrix and the embedding is based on empirical evidence. We examined the behaviour of infinitive complement CLs of raising, simple subject control, reflexive subject control and object control predicates in different contexts and from various perspectives. Below we recapitulate the results of the corpus-based study (Figure \ref{F17.1}) and of the psycholinguistic experiment (Figure \ref{F17.2}), which are our primary points of reference in this section. 

Figure \ref{F17.1} prepared on the basis of corpora shows the predicted probability of CC in the context of different matrix verbs. The result is driven by frequency of usage, and therefore, the figure models production. Figure \ref{F17.2} prepared on the basis of the experiment shows the predicted probability of a sentence with CC (right) or without CC (left) being accepted by a native speaker; hence, it models comprehension. It accounts for CTP type and CL type. The factor missing from both figures is CL case, because its impact is either insignificant for the model or was tested separately. In the following subsections we discuss each of the factors mentioned independently.

\begin{figure}
\caption{Results of the corpus-based study from Chapter \ref{A corpus-based study on clitic climbing in infinitive complements in relation to the raising-control dichotomy and diaphasic variation (Croatian)}}
\label{F17.1}
\includegraphics[width=0.6\textwidth]{f171}
\end{figure}

\begin{figure}
\caption{Main results of the psycholinguistic experiment from Chapter \ref{Experimental study on constraints on clitic climbing out of infinitive complements}}
\label{F17.2}
\includegraphics[width=\textwidth]{figures/Fig.16.2.pdf}
\end{figure}

In addition to the two studies on constructions with infinitive complements, we draw on further evidence on systemic constraints related to the matrix from the study on \textit{da}\textsubscript{2}-constructions and the raising--control distinction (Chapter \ref{A corpus-based study on CC in da constructions and the raising-control distinction (Serbian)}). Since the data obtained in this study are more modest, they cannot be analysed with comparable quantitative methods. Therefore, in the argumentation which follows we use them only as a supplementary source.

We will now discuss each of the relevant factors (predicate type, CL type, CL case, mixed cluster effects).

\subsection{Predicate type (CTP)}
\label{Predicate type (CTP)}

The models summarised in Figures \ref{F17.1} and \ref{F17.2} allow us to conclude that raising and simple subject control predicates do not differ drastically with respect to CC. CC is slightly more frequent and also more acceptable with raising than with simple subject control predicates. However, we observe a significant difference between reflexive subject and object control predicates on the one hand, and the other two types on the other. In the model built on experimental data, we additionally see an interaction with the CL type discussed in more detail in the next subsection. As our corpus studies and experimental data clearly show, neither reflexive subject control nor object control predicates allow reflexive CLs to climb.

\subsection{CL type}
\label{CL type}
In the corpus data for raising and simple subject control verbs, the type of the infinitive CL plays no role in CC. In other words, climbing of all types of CLs is similarly frequent for both types of predicates, and the differences in acceptability of such structures are insignificant. However, in the context of reflexive subject control matrix predicates, climbing of reflexive CLs seems completely impossible: see the example in (\ref{(11.84rep)}) without CC.

\begin{exe}\ex
\begin{xlist}
\ex[]{\label{(11.84rep)}
\gll [\dots] koji \textbf{se}\textsubscript{1} boje\textsubscript{1} odreći\textsubscript{2} \textbf{se}\textsubscript{2} {grijeha [\dots].} \\
 {} which \textsc{refl} be.afraid.\textsc{3prs} give.up.\textsc{inf} \textsc{refl} sin \\ }
\ex[*]{\label{(11.84b)}
\gll [\dots] koji \textbf{se}\textsubscript{1} \textbf{se}\textsubscript{2} boje\textsubscript{1} odreći\textsubscript{2} {grijeha [\dots].} \\
 {} which \textsc{refl} \textsc{refl} be.afraid.\textsc{3prs} give.up.\textsc{inf} sin\\
}
\end{xlist}
\glt ‘[\dots] who are afraid to renounce their sins [\dots].’
\hfill [hrWaC v2.2]
\end{exe}

\noindent We have not found evidence for examples similar to (\ref{(11.84b)}) in corpora, nor were they accepted by native speakers in the psycholinguistic experiment. Furthermore, for sentences with reflexive subject control predicates even noCC versions with reflexive infinitive CLs (as in example (\ref{(11.84rep)})) are extremely rare in corpora. However, the probability of speakers accepting such a construction is 0.75. This appears to be a major difference between production and comprehension.

We have found some evidence for climbing of pronominal CLs occurring with reflexive subject control CTPs in corpora, but its distribution is different than for raising and simple subject control predicates, which appear with CC in the majority of sentences. In corpora, sentences with reflexive subject control predicates are more frequent without CC, whereas in the experiment both versions ((\ref{(16.14arep)}) and (\ref{(16.14brep)})) are equally acceptable. 

\begin{exe}\ex\begin{xlist}
\ex\label{(16.14arep)}
\gll Ipak \textbf{se}\textsubscript{1} trudim\textsubscript{1} konkurirati\textsubscript{2} \textbf{joj}\textsubscript{2} na drugim područjima. \\
 still \textsc{refl} try.\textsc{1prs} compete.\textsc{inf} her.\textsc{dat} on other areas \\
\ex\label{(16.14brep)}
\gll Ipak \textbf{joj}\textsubscript{2} \textbf{se}\textsubscript{1} trudim\textsubscript{1} konkurirati\textsubscript{2} na drugim područjima. \\
 still her.\textsc{dat} \textsc{refl} try.\textsc{1prs} compete.\textsc{inf} on other areas \\
\end{xlist}
\glt ‘Still, I am trying to compete with her in other areas.’
\end{exe}


\noindent Sentences with object control predicates were not retrieved from the corpus, as this turned out to be an extremely hard and costly task.\footnote{Note, however, that this does not mean that such sentences are not retrievable from corpora. In Section \ref{Object control constraint related to case} we give a number of examples with object control verbs and CC such as (\ref{(11.51rep)}), although the experiment reveals that such sentences are not usually accepted: 

\begin{exe}\ex\label{(11.51rep)}
\gll [\dots] koji \textbf{mi}\textsubscript{1} \textbf{ga}\textsubscript{2} pomažu\textsubscript{1} nositi\textsubscript{2}. \\
 {} which me.\textsc{dat} him.\textsc{acc} help.3\textsc{prs} carry.\textsc{inf} \\
\glt ‘[\dots] which help me to carry it.’\hfill\hbox{[hrWaC v2.2]}
\end{exe}} Like sentences with reflexive subject control predicates, sentences with CC and any kind of object control predicates are not acceptable, and we observe a further drop in acceptability for pronominal CLs. The CC version is somewhat likely to be accepted only in object control sentences with the reflexive controller \textit{se}, like (\ref{(16.20arep)}), with a probability slightly over 0.5. However, sentences without climbing, like (\ref{(16.20brep)}), are accepted with a probability of about 0.8.

\begin{exe}\ex\begin{xlist}
\ex\label{(16.20arep)}
\gll Sada \textbf{se}\textsubscript{1} prisiljavate\textsubscript{1} zahvaliti\textsubscript{2} \textbf{im}\textsubscript{2} na nesebičnoj pomoći. \\
 now \textsc{refl} force.2\textsc{prs} thank.\textsc{inf} them.\textsc{dat} on unselfish help \\
\ex\label{(16.20brep)}
\gll Sada \textbf{im}\textsubscript{2} \textbf{se}\textsubscript{1} prisiljavate\textsubscript{1} zahvaliti\textsubscript{2} na nesebičnoj pomoći. \\
 now them.\textsc{dat} \textsc{refl} force.2\textsc{prs} thank.\textsc{inf} on unselfish help \\
\end{xlist}
\glt ‘Now you are forcing yourselves to thank them for their unselfish help.’
\end{exe}

\noindent We see that the CL type factor operates in combination with the predicate type factor. Namely, whereas it does influence CC in sentences with reflexive subject control matrix predicates, it is not significant at all for raising and simple subject control matrix predicates. As may be seen in Figure \ref{F17.2}, the probability that a sentence without CC will be accepted is always above 0.5 but mostly below 0.8.

When analysing the CL type factor, we observe some correspondence between the results for CC in the corpora and in the experiment. The drop in frequency of CC for individual CL types in the context of reflexive subject predicates in corpora corresponds to the decrease in acceptability of such sentences in the experiment. However, this is not the case for sentences without CC. In standard corpora we rarely find variants with pseudodiaclisis for reflexive subject control matrices and the infinitive reflexive CL (similar to the structure presented in (\ref{(11.84rep)})).\footnote{For more information on pseudodiaclisis see Section \ref{Diaclisis and pseudodiaclisis}.} Additionally, we have little corpus data for both variants with reflexive subject control predicates in general. Conversely, such sentences are acceptable to the participants of the experiment, but not at a level of 90--100\% but rather a level of 50--80\%. We can speculate that this is due to the form of the complement: even Croatian object control predicates seem to demand \textit{da}\textsubscript{2}-complements and not the infinitive, and something similar might be true also in the case of reflexive subject control predicates. Notice also that we did not examine haplology, which might be another highly acceptable structure, either in the experiment or in the corpus.

\subsection{CL case}
\label{CL case}
Neither in the corpus studies nor in the experiment did we find evidence that the case of the infinitive CL might play a significant role for CC in the context of raising and subject control predicates. As it is very hard to search for examples of CC with object control predicates in corpora, we have only experimental data for this predicate type.\footnote{For an explanation \textcolor{black}{why is it very hard to search for examples of CC with object control predicates in corpora} see Section \ref{Methods}.} It turns out that the case of the infinitive CL is significant only for object control predicates with pronominal CL controllers in the dative (of the \textit{naređivati} ‘give an order’ type) and the accusative (of the \textit{prisiljavati} ‘force’ type). Namely, sentences with object control matrix predicates which have pronominal CL controllers are less acceptable in their CC version if the infinitive CL is a pronoun in the dative. Case is thus a factor influencing CC only in combination with predicate type. 

\subsection{Mixed cluster effects}
\label{Mixed cluster effects}
We observe that there are constraints on CC that manifest in the context of mixed clusters, haplology and pseudodiaclisis. In Section \ref{Clitic ordering within the cluster} we distinguished simple and mixed clusters. The latter clusters contain CLs of at least two different governors. The following example shows a mixed cluster consisting of the dative CL \textit{mi} ‘to me’, which is a complement of the matrix predicate \textit{pomoći} ‘help’, and the accusative CL \textit{ih} ‘them’, which is the direct object of the infinitive complement \textit{riješiti} ‘solve’.

\protectedex{\begin{exe}\ex\label{(17.1)}
\gll [\dots] neće \textbf{mi}\textsubscript{1} \textbf{ih}\textsubscript{2} pomoći\textsubscript{1} riješiti\textsubscript{2} ni oni. \\
 {} \textsc{neg}.\textsc{fut}.3\textsc{pl} me.\textsc{dat} them.\textsc{acc} help.\textsc{inf} solve.\textsc{inf} \textsc{neg} they \\
\glt ‘[\dots] not even they will help me solve them.’
\hfill [hrWaC v2.2]
\end{exe}
}

\noindent We found the following three types of mixed cluster effects triggered by different types of matrix predicates (CTPs): 

\begin{enumerate}\sloppy
\item Raising and simple subject control CTPs do not introduce their own pronominal or reflexive CLs and thus freely allow CC from the embeddings in the absence of verbal and interrogative CLs. 
\item Reflexive subject control CTPs have the \textsc{refl\textsubscript{lex}} CL \textit{se} taking its slot in the cluster. In this context, CC is restricted: climbing of pronominal CLs is marginally possible in production, but quite comprehensible. Climbing of reflexive CLs involves haplology.\footnote{For more on CC in the context of haplology see Section \ref{Pseudo-twins}.}
\item Object control CTPs have an argument position which can be instantiated as a pronominal CL in the dative or the accusative, or as reflexive CLs \textit{se} and \textit{si}. In the former case, like in the case of an argument which is instantiated as the reflexive CL \textit{si}, no climbing is allowed from the embeddings, either for reflexives or for dative pronominal CLs. In the context of object control matrix predicates, climbing of accusative CLs is more acceptable than climbing of dative CLs. Nevertheless, sentences with object control matrix predicates whose accusative CL climbs are accepted with a probability below 0.5.
\end{enumerate}

We found some further evidence for the mixed cluster effect with the tied-island \textit{da}\textsubscript{2}-complement mentioned above. First, if two CLs are generated in a \textit{da}\textsubscript{2}-complement and occur in pseudodiaclisis, it is the pronominal that climbs and the reflexive that stays in the \textit{da}\textsubscript{2}-complement. Second, it seems that the reflexive CL \textit{se} does not climb with the pronominal CL if there is a verbal CL in the matrix clause: see example (\ref{(14.13rep)}). 

\protectedex{\begin{exe}\ex\label{(14.13rep)}
\gll [\dots] i počelo\textsubscript{1} \textbf{mi}\textsubscript{2} \textbf{je}\textsubscript{1} da \textbf{se}\textsubscript{2} vrti\textsubscript{2} u glavi.\\
 {} and start.\textsc{ptcp}.\textsc{sg}.\textsc{n} me.\textsc{dat} be.3\textsc{sg} that \textsc{refl} spin.\textsc{3prs} in head \\
\glt ‘[\dots] and I started to feel dizzy.’
\hfill [srWaC v1.2]
\end{exe}
}

\subsection{Complexity effects related to the interaction of matrix and embedding}
\label{Complexity effects related to the interaction of matrix and embedding}
Summing up these findings on constraints related to the interaction of the matrix and the embedding, which were best observed in the context of mixed clusters, pseudodiaclisis and haplology, we see a strong interaction of the individual factors. We put forward the hypothesis that they can be described by the following types of ontological complexity:

\begin{enumerate}
\item Constitutional complexity measured by the number of arguments of the CTP.
\item Organisational complexity measured by the number and types of slots available in the CL cluster. These slots can potentially be already taken by CLs belonging to the CTP and consequently not be available for climbing CLs.
\item Constitutional complexity related to each CL appearing in a structure, measured by the number of grammatical categories it encodes.
\item Operational complexity related to each CL appearing in a structure, measured by the number of functions that CL serves.
\end{enumerate}

We apply these measures to our studies. Taking the typology of CTPs and the structure of the CL cluster, we see that the different predicate types may introduce their own CLs that fill the slots in the CL cluster sequence. As shown in Section \ref{Clitic ordering within the cluster}, we assume the following slots and their relative order in the CL cluster:

\begin{exe}\sn
\textit{li} ${}>{}$ \textsc{verbal} ${}>{}$ \textsc{pron}\textsubscript{\textsc{dat}} ${}>{}$ \textsc{pron}\textsubscript{\textsc{acc}} ${}>{}$ \textsc{pron}\textsubscript{\textsc{gen}} ${}>{}$ \textsc{refl} ${}>{}$ \textit{je}
\end{exe}

A cluster may contain the polar marker \textit{li}, but it is not generated by a predicate. In the studies conducted, \textit{li} does not show any variation in its behavior and regularly appears in 2P. Cases where \textit{li} does not form clusters with other CLs are extremely rare. We thus have no grounds to assume that its presence is a large constraint on CC.
CTPs differ as to the number and type of slots they can potentially cause to be occupied, as shown in Table \ref{T17.1}.

\begin{table}\small
\caption{Complexity related to CTPs. Non-obligatory CL types are given in brackets. }
\label{T17.1}

\begin{tabularx}{\textwidth}{Xrrl}
\lsptoprule
CTP type& \parbox[t]{1.6cm}{No. of\newline arguments\newline of CTP} &\parbox[t]{2.2cm}{No. of slots\newline in cluster\newline related to CTP} &\parbox[t]{2.4cm}{Type of slot in cluster related\newline to CTP}\\
\midrule
raising& 1& 1 &(V)\\
simple subject control &2& 1 &(V)\\
reflexive subject control &2 &2 &(V)$+$\textsc{refl}\\
object control &3 &2& (V)$+$(\textsc{refl}/\textsc{pron})\\
\lspbottomrule
\end{tabularx}
\end{table}

Accordingly, raising predicates are the least complex of all CTPs as they have only one semantic argument into the complement whereas simple control predicates additionally have a semantic subject argument. As may be seen in Table \ref{T17.1}, these predicate types potentially have only one, verbal position in the CL cluster to fill – when the matrix verb is in the past and future tense or in the conditional. Since the slots in the cluster reserved for pronominal and reflexive CLs remain free, CC is possible and very likely to take place.\footnote{We are aware that there are subject control predicates CTPs denoting commissive speech acts (e.g. \textit{obećati} ‘promise’) which can potentially fill not only the verbal slot but also the pronominal slot. Such verbs, however, were not surveyed in our studies. More information on those verbs and an example can be found in Section \ref{The control vs raising distinction}.} 

\begin{sloppypar}
Like simple subject control predicates, reflexive subject control predicates have two semantic arguments and they can potentially fill one position in the CL cluster with a verbal CL. However, they also fill the position of the reflexive CL in the cluster. Since the slot reserved for a reflexive CL in the CL cluster is already occupied, CC is restricted. Namely, only pronominal CLs can climb and fill the free positions reserved for them in the CL cluster, whereas reflexive CLs must either stay in situ or haplologise. Therefore, the constitutional complexity of CTPs restricts organisational complexity with respect to the position of infinitive CLs.
\end{sloppypar}

In terms of constitutional complexity, object control predicates are the most complex predicates as they have three semantic arguments. Moreover, they potentially fill two slots in the CL cluster with verbal and pronominal or reflexive CLs. Although a controller can be expressed as an NP, note that we have only studied structures in which it was expressed as a CL. In the studied structures, object control predicates always filled one position: either of the pronominal or of the reflexive CL. 

Additionally, as we already pointed out in Section \ref{Ontological modes of complexity}, pronominal CLs increase the constitutional complexity of a structure as a whole since they encode case and number (and for the third person singular also gender), while reflexive CLs increase its operational complexity due to their polyfunctionality. 

As regards object control predicates with reflexive CL controllers, reflexive CLs may increase both constitutional complexity (CL \textit{si}) and operational complexity. In general, they are unlikely to climb. The constructions with object control predicates that have reflexive CL controllers show some similarity to those with reflexive subject control predicates, as climbing of pronominal CLs is quite probable, though to a lesser degree than in the case of reflexive subject control CTPs. This can be considered an argument supporting the claim that the polyfunctionality of a CL (operational complexity) is an important factor. A CL belonging to a reflexive subject control CTP is “poorer” in that respect compared to other types of reflexive CLs. Further, in the light of constitutional complexity the \textsc{refl\textsubscript{lex}} CL \textit{se} belonging to the reflexive subject control CTP is also less complex than the \textsc{refl\textsubscript{2nd}} CL \textit{se} belonging to the reflexive object control CTP since the latter encodes case.\footnote{This difference in constitutional complexity and, accordingly, in CC between the two studied reflexive types \textsc{refl\textsubscript{lex}} and \textsc{refl\textsubscript{2nd}} becomes even more apparent in the case of the \textsc{refl\textsubscript{2nd}} CL \textit{si}. The difference in constitutional complexity between the \textsc{refl\textsubscript{lex}} \textit{se} and \textsc{refl\textsubscript{2nd}} \textit{se} is not immediately apparent since they coincide phonologically (which also explains why they behave in a similar but not the same way). In contrast, the difference in constitutional complexity between the \textsc{refl\textsubscript{lex}} \textit{se} and \textsc{refl\textsubscript{2nd}} \textit{si} is more pronounced since their morphological differences have additional support in their different phonological realisation. }

Finally, in Section \ref{CL case} we noted that climbing of pronominal CLs in the context of object control predicates with a pronominal dative controller is less likely to be accepted than climbing of other CLs. This result can be explained by the fact that structures with mixed clusters containing two pronominal CLs represent the highest level of constitutional complexity, and compared to other cases the dative increases operational complexity, as it is the most polyfunctional case \citep[cf.][219f, 223]{SilicPranjkovic07}. 

The combinations of complexity measures nicely explain the differences in frequency and acceptance of all four types of CTPs, including the less pronounced difference between raising and simple subject control predicates. 

\section{Non-systemic factors related to the diaphasic dimension}
\label{Non-systemic factors related to the diaphasic dimension}
We did not test all the factors and their interactions as regards diatopic microvariation in all three BCS varieties. Nevertheless, in our earlier study on stacked infinitives in web corpora we did not find statistically relevant differences in CC between Croatian, Bosnian and Serbian \citep*[cf.][]{HKJ18}.\footnote{Although there are differences in the frequency of stacked infinitives we have not found language-specific differences in the distributions of constructions with and without CC.} Therefore, based on what we know, we have no reason to expect diatopic variation to be involved as a constraint on CC out of stacked infinitives. However, this does not exclude the possibility that diatopic microvariation does affect CC in the case of other factors. To establish this empirically further studies are needed.\footnote{For instance, we do not have empirical data on the difference in CC out of \textit{da}\textsubscript{2}-complements between Bosnian and Serbian.}

Apart from the systemic factors triggering microvariation in the domain of CC, we detected that the non-systemic diaphasic factor has an impact on CC at least for Croatian. In other words, there is a higher frequency of CC in the standard Croatian variety than in informal Croatian as presented in web fora: see Chapter \ref{A corpus-based study on clitic climbing in infinitive complements in relation to the raising-control dichotomy and diaphasic variation (Croatian)}. Namely, CC is used significantly more frequently in the standard than in informal language, in particular in the case of raising verbs. However, we have to point out that this is not a universal tendency as Spanish and European Portuguese, which have a pronominal CL system with CC phenomena, show the reverse tendency. In these Romance languages, CC is less frequent in written than in spoken texts. 

In terms of complexity, we might argue that diaphasic variation is related to operational complexity: formal language is more codified, following rules, so operational complexity understood as a variety of modes should be lower. Low operational complexity means little variety, hence little flexibility in the way linguistic units can be combined. Necessarily, organisational complexity decreases too. In terms of CC, this results in the availability of only one position for CLs. As the study in Chapter \ref{A corpus-based study on clitic climbing in infinitive complements in relation to the raising-control dichotomy and diaphasic variation (Croatian)} suggests, while in Croatian the prescribed variant is CC, in Romance it is noCC. 

\section{Interaction, optionality and preferences}
\label{Interaction, optionality an preferences}
Our findings for BCS fully corroborate the claim of \citet[][205]{Rosen01} that word order properties of (Czech) CLs defy straightforward explanation due to the following two facts: 

\begin{enumerate}
\item ``several factors interact to determine their position [\dots]”,
\item “[o]nly some generalizations concerning their ordering behaviour can be expressed by strict rules, while other properties have to be stated as mere preferences”. 
\end{enumerate}

We have seen that most factors interact with each other. The factors CL type and CL case interact with the factor of matrix predicate type, but they are not active on their own. As to preferences, we saw certain patterns of CC which show graded acceptability. A nice case of peripheral usage is CC out of \textit{da}\textsubscript{2}-constructions, which is rare as such, albeit possible in certain contexts (i.e. certain combinations of factors). This is why we proposed the new term tied island. 

In relation to the question raised by some scholars of whether CC is obligatory, we have solid empirical grounds for regarding CC as optional. Namely, the acceptability rates for noCC versions of sentences with raising and simple subject control predicates are around 50\%; that is, they reach the threshold of acceptability.

If we recall the theoretical accounts of CC briefly discussed in Chapter \ref{Approaches to clitic climbing}, we come to the conclusion that our findings are compatible with Junghanns’ (\citeyear[85f]{Junghanns02}) claim that CC does not take place if the CL cannot reach the corresponding landing site in the matrix. We would argue that the effects responsible for blocking these landing sites can best be explained in terms of ontological complexity. 

\begin{table}[b]
\caption{The probability of CC with regard to the type of embedding (constitutional complexity)\label{new17.2}}
\begin{tabularx}{\textwidth}{llclcl}
\lsptoprule
Embedding & No island & ${}>{}$ & Tied Island & ${}>{}$ & True island \\\midrule
Probability of CC & Very high & ${}>{}$ & Very low & ${}>{}$ & 0 \\
\lspbottomrule
\end{tabularx}
\end{table}
We can now show the changes in probabilities of CC based on the different types of complexity described above, starting from the most pervasive island constraint, as shown in Table \ref{new17.2}. In the case of no islands and tied islands the probability of CC varies depending on the systemic subordinate types of complexities (Tables \ref{new17.3} and \ref{new17.4}) and on non-systemic factors. 








According to our studies, the factors CL type and CL case are relevant only in the context of mixed clusters. Here, the operational complexity arising from the polyfunctionality of CLs appears to play a certain role. Since the exact cognitive processes behind the formation of mixed clusters are unknown, we refrain from drawing any strong conclusions.

We conclude that CC is not ruled by a single strict constraint. As a sort of perspective we put forward the hypothesis that the functioning of CC and its constraints could best be explained as a type of complexity effect in the sense that the constitutional (or operational) complexity of the involved sentence structures is the driving force for CC or for blocking it.




\begin{table}[t]
\caption{The probability of CC with regard to the type of CTP (constitutional complexity)}
\label{new17.3}
\small
\begin{tabularx}{\linewidth}{Q@{~}l@{~}p{1.2cm}@{~}Q@{~}Q@{~}Q}
\lsptoprule
CTP type &Raising& Simple subject control& Reflexive subject\newline control &Object control with \textit{se} controller& Object control \mbox{with pronominal} and reflexive CL controller clearly marked for case \\
\midrule
Probability of CC & \mbox{High ${}>{}$} & \mbox{High ${}>{}$} & \mbox{Middle-low ${}>{}$} & \mbox{Middle-low ${}>{}$} & Low \\
\tablevspace
Coordination with other systemic factors& NA & NA & Infinitive CL type & Infinitive CL type and mixed cluster & Infinitive CL type and case, mixed cluster \\
\lspbottomrule
\end{tabularx}
\end{table}


\begin{table}[t]
\caption{The probability of CC with regard to the mixed cluster effects (organisational complexity)\label{new17.4}}
\begin{tabularx}{\textwidth}{llQ}
\lsptoprule
Slot occupied& Yes & No \\\midrule
Probability of CC & 0 & Regulated by other factors, in particular complexity related to other CLs in cluster\\
\lspbottomrule
\end{tabularx}
\end{table}










