\chapter[A corpus-based study on clitic climbing out of \textit{da}\textsubscript{2}-construction]
        {A corpus-based study on clitic climbing out of \textit{da}\textsubscript{2}-construction and the raising--control distinction (Serbian)}
\label{A corpus-based study on CC in da constructions and the raising-control distinction (Serbian)}
\section{Introduction}

In this chapter we address CC out of \textit{da}-complements.\footnote{Some results from this chapter were previously discussed in  \citet*{JHK17a}.} In many languages, CC is only attested in clauses with infinitive complements; cross-linguistically, CC out of complements with inflected verbs is a rare phenomenon.\footnote{A comparison of CC out of complements with inflected verbs in Czech and BCS can be found in Section \ref{Clauses with inflected verbs}.} %first footnote with bibliography text???

In Serbian, infinitive complements compete with the so-called \textit{da}-complement, that is, a verb marked for person and number which is introduced by an element usually treated as a complementiser.\footnote{See Section \ref{Types of complements} for basic information on complement types in BCS.}  As an illustration, compare the sentence presented in (\ref{(14.1)}) with the infinitive complement \textit{naći} ‘find’, and the sentence with the \textit{da}-complement \textit{nađete} `(you) find' in (\ref{(14.2)}).

\begin{exe}\ex\label{(14.1)}
\gll Ako ne možete naći udoban {položaj [\dots].} \\
if \textsc{neg} can.\textsc{2prs} find.\textsc{inf} comfortable position \\
\glt ‘If you cannot find a comfortable position [\dots].’
\hfill [srWaC v1.2]
\ex\label{(14.2)}
\gll [\dots]  na celoj toj teritoriji ne možete da nađete 500 stanovnika.  \\
{} on whole that territory \textsc{neg} can.\textsc{2prs} that find.\textsc{2prs} 500 inhabitants\\
\glt ‘[\dots] on that whole territory you cannot find 500 inhabitants.’ \\
\strut\hfill [srWaC v1.2]
\end{exe}

\noindent However, it is rather unclear to what extent and under what circumstances CC out of \textit{da}-complements is possible. The present chapter approaches this problem empirically using corpus-based methods. Section \ref{The da-complement and CC in Serbian} refers to the discussion on CC out of \textit{da}-complements in Serbian. Section \ref{Results:da} describes the results of the corpus queries in detail, while in Section \ref{Discussion:da} we analyse  and discuss them. The final Section \ref{Conclusions:da} draws conclusions from the main results and offers a suggestion for further research.

\section{The \textit{da}-complement and CC in Serbian}
\label{The da-complement and CC in Serbian}

As discussed in Section \ref{Clitics and microvariation}, the research on the syntax of BCS is divided into descriptive empirical studies on the one hand, and works with a formal theoretical orientation on the other. Therefore it comes as no surprise that in the literature we find largely contradictory statements concerning CC out of \textit{da}-complements. \citet[174ff]{Stjepanovic04} argues that \textit{da}-complements and infinitives allow CC in a similar way. Nevertheless, discussing examples of CC out of \textit{da}-complements, she writes imprecisely that those “are acceptable sentences, however, they are short of perfect” \citep[cf.][201]{Stjepanovic04}. Similarly, according to \citet[243]{FranksKing00}, movement out of the finite complement is only “marginally possible”. On the opposite side of the spectrum, \citet[41]{CavarWilder94} and \citet[41]{Browne03} argue that CC out of finite complements is completely impossible. \textcolor{black}{Moreover, \citet[448]{CavarWilder94} claim that CC is not possible even out of semi-finite complements of subject control verbs like \textit{ht(j)eti} ‘will’.}\footnote{\textcolor{black}{\citet[448]{CavarWilder94} address causative constructions as the only exception to that rule.}} Finally, \citet[146]{Progovac05} admits that “some speakers of Serbian” do not accept CC in the presented contexts. All the above-mentioned authors rely exclusively on self-constructed examples.\footnote{For more information on the traps of studies based on self-constructed examples and intuition, see Section \ref{Introduction:4}.}

However, as explained in Section \ref{Types of complements} we have to bear in mind that \textit{da}-com\-ple\-ments do not behave in a uniform way since they differ with regard to tense marking. Based on \citet{Todorovic12}, we assume that if CC is possible, this is only the case for \textit{da}\textsubscript{2}-complements, which are marked only for person and number. One hypothetical reason why some scholars reject the possibility of CC out of \textit{da}\textsubscript{2}-complements is its extreme rarity in comparison to equivalent constructions without CC. An early empirical work concerning CC is \citet{Markovic55}, which assumes that the variation in CL positioning is closely related to the (at that time) new and growing tendency to replace infinitives with \textit{da}\textsubscript{2}-complements.\footnote{\textcolor{black}{\citet{Markovic55}} does not use the term clitic climbing.} Marković addresses the question of diatopic and diaphasic variation with respect to CC out of such constructions.\footnote{More information on diaphasic and diatopic variation can be found in Section \ref{Systemic vs functional microvariation}.} As to the former dimension he claims that ekavian Serbian speakers, who (at that time) had already almost completely replaced the infinitive with \textit{da}\textsubscript{2}-complements, preferred keeping the pronominal CL directly after the \textit{da} particle, i.e. no CC.\footnote{In BCS languages speakers of ekavian, ikavian and ijekavian dialects may be distinguished. More information on this can be found in Section \ref{An overview of BCS Štokavian dialects}.} Regarding the two types of variation, he stated that at that time, CC was common in journalistic ijekavian texts published in Sarajevo \citep[cf.][35--40]{Markovic55}. 

Very few papers recognise the importance of the raising--control dichotomy for CC in BCS. \citet{Aljovic05} observes that in BCS, CC is only possible out of complements whose subject is empty and coreferential with the matrix subject, although in a footnote she acknowledges that CC is also possible when the subject of the embedded complement is coreferential with the matrix indirect object in the dative, i.e. out of object-controlled infinitives.\footnote{\textcolor{black}{Bear in mind that complements whose subject is empty and coreferential with the matrix subject} are complements of raising and subject control CTPs.} For Czech a range of various constraints on CC closely connected with object control was described in the theoretical literature \citep{Rezac05, Dotlacil04, Hana07}. 

As explained in Section \ref{Introduction:11}, in this study we focus exclusively on Serbian, and not on Bosnian or Croatian, because \textit{da}-complements are much more frequently used in Serbian than in Bosnian and Croatian, especially in the context of modal verbs in non-epistemic meanings as in example (\ref{(14.2)}) above.

Therefore, in this chapter we address the following research question: 

\begin{enumerate}[label=Q\arabic*:]
\item To what extent is CC out of \textit{da}\textsubscript{2}-complements possible in Serbian?
\end{enumerate}

If CC out of \textit{da}\textsubscript{2}-complements is possible, the question arises which syntactic features enable or block climbing. To start with, we investigate the potential link between CC and the raising--control distinction, usually held to be crucial for categorising different types of sentences with verbal complements.\footnote{For more information on the distinction between control and raising predicates see Section \ref{The control vs raising distinction}.} The point of departure for the present study is divergent statements on the link between CC and the raising--control dichotomy in Czech.\footnote{For more information on this topic see Section \ref{Constraints related to the raising-control distinction}.} 

\citet*{HKJ18} demonstrate that CC out of stacked infinitives, that is, multiply embedded infinitives, is not obligatory in BCS. However, empirical data for CC out of \textit{da}-complements are still lacking. Based on this, we formulate the second research question:

\begin{enumerate}[label=Q\arabic*:]
\setcounter{enumi}{1}
\item Does CC out of \textit{da}\textsubscript{2}-complements in Serbian depend on verb type with respect to the raising--control distinction?
\end{enumerate}

\begin{sloppypar}We approach the questions stated above by investigating 17 matrix verbs whose choice is explained in Section \ref{Choice of matrix verbs}. The data come from srWaC, which is due to its size the most reliable source for tracking rare phenomena in BCS, such as CC out of \textit{da}\textsubscript{2}-complements in Serbian \citep*{JHK17a}.\footnote{For more information on the corpora selected and our argumentation for choosing those and not other corpora, see Section \ref{Clitic climbing in BCS}. See Section \ref{Operationalising the constructions in question} for the queries used and Section \ref{Empirical approach in the current study} for an exhaustive discussion of our methodological approach.} Accordingly, although \citet{Markovic55} argued that CC out of \textit{da}\textsubscript{2}-complements is more frequent on Bosnian than on Serbian language territory, we decided to conduct the study on Serbian material and extract data from srWaC since it is almost twice as big as bsWaC.\end{sloppypar}

\section{Results}
\label{Results:da}
We present the results of corpus queries in detail in Tables \ref{T14.1}--\ref{T14.1b}. The results still posed problems due to the number of retrieved queries and their precision. Since we noticed that not all retrieved sentences correspond to the target structures, we decided to conduct a manual check.\footnote{The target structures can be found in Table \ref{T12.1}.}  


\begin{table}
\caption{Position of CL with respect to \textit{da}-complementiser in sentences with raising predicates\label{T14.1}}
\begin{tabularx}{\textwidth}{lYYYX}
\lsptoprule
Construction & Retrieved sentences CQL & Correct target structures in sample & Estimated frequency &  Conf. interval\\\midrule
\textit{moći} `can' &&&& \\
CTP da PRS CL &2527  &0 &0  &0 \\
CTP da CL PRS &44691 &91 &37363 &0.83602 \\
CTP CL da PRS &1843  &2 &5 &0.00243 \\
CL CTP da PRS &9637  &5 &158 &0.01643 \\
\multicolumn{3}{l}{relative frequency of CC out of \textit{da}\textsubscript{2}} & 0.0043& \\
\tablevspace
\textit{nastaviti } `continue' &&&& \\
CTP da PRS CL &83 &1 &1 &0 \\
CTP da CL PRS &1120  &97 &1025 &0.91482 \\
CTP CL da PRS &24 &1 &1 &0 \\
CL CTP da PRS &255 &2 &2 &0.00243 \\
\multicolumn{3}{l}{relative frequency of CC out of \textit{da}\textsubscript{2}} & 0.0029& \\
\tablevspace
\textit{početi} `start' &&&&\\
CTP da PRS CL &427 &1 &1 &0.00025 \\
CTP da CL PRS &6724 &100 &6480 &0.96378 \\
CTP CL da PRS &80 &5 &5 &0 \\
CL CTP da PRS &956 &12 &61 &0.06357 \\
\multicolumn{3}{l}{relative frequency of CC out of \textit{da}\textsubscript{2}} & 0.01&\\
\tablevspace
\textit{prestati} `stop' &&&&\\
CTP da PRS CL &82 &2 &2 &0 \\
CTP da CL PRS &1298 &97 &1187 &0.91482 \\
CTP CL da PRS &13 &2 &2 &0 \\
CL CTP da PRS &213 &11 &12 &0.05621 \\
\multicolumn{3}{l}{relative frequency of CC out of \textit{da}\textsubscript{2}} & 0.0116& \\
\tablevspace
\textit{sm(j)eti } `be allowed' &&&&\\
CTP da PRS CL &86 &1 &1 &0 \\
CTP da CL PRS &2134 &99 &2018 &0.94554 \\
CTP CL da PRS &56 &0 &0 &0 \\
CL CTP da PRS &338 &6 &8 &0.02233 \\
\multicolumn{3}{l}{relative frequency of CC out of \textit{da}\textsubscript{2}} & 0.01& \\
\lspbottomrule
\end{tabularx}
\end{table}


\begin{table}
\caption{Position of CL with respect to \textit{da}-complementiser in sentences with subject control predicates.}
\label{T14.1a}
{\scriptsize
\begin{tabularx}{.97\textwidth}{lYYYX}
\lsptoprule
Construction & Retrieved sentences CQL & Correct target structures in sample & Estimated frequency &  Conf. interval \\
\midrule
\textit{nam(j)eravati} `intend' &&&&\\
CTP da PRS CL &12  &1 &1 &0 \\
CTP da CL PRS &506  &97 &463 &0.91482 \\
CTP CL da PRS &1 &0 &0 &0 \\
CL CTP da PRS &34 &1 &1 &0 \\
\multicolumn{3}{l}{relative frequency of CC out of \textit{da}\textsubscript{2}} & 0.0021& \\
\textit{nastojati} `strive' &&&&\\
CTP da CL PRS &23 &1 &1 &0 \\
CTP da CL PRS &797 &96 &718 &0.90074 \\
CTP CL da PRS &0 &0 &0 &0 \\
CL CTP da PRS &55 &2 &2 &0 \\
\multicolumn{3}{l}{relative frequency of CC out of \textit{da}\textsubscript{2}} & 0.0027& \\
\tablevspace
\textit{pokušati} `try' &&&&\\
CTP da PRS CL  &126 &2 &2 &0.00243 \\
CTP da CL PRS  &5044 &99 &4769 &0.94554 \\
CTP CL da PRS  &33 &4 &4 &0 \\
CL CTP da PRS  &296 &12 &19 &0.06357 \\
\multicolumn{3}{l}{relative frequency of CC out of \textit{da}\textsubscript{2}} & 0.0047& \\
\tablevspace
\textit{um(j)eti} `be able to' &&&&\\
CTP da PRS CL &59 &0 &0 &0 \\
CTP da CL PRS &1277 &99 &1207 &0.94554 \\
CTP CL da PRS &53 &0 &0 &0 \\
CL CTP da PRS &381 &2 &2 &0.00243 \\
\multicolumn{3}{l}{relative frequency of CC out of \textit{da}\textsubscript{2}} & 0.0016& \\
\tablevspace
\textit{usp(j)eti} `succeed' &&&&\\
CTP da PRS CL &173 &0 &0 &0 \\
CTP da CL PRS &4655 &98 &4327 &0.92962 \\
CTP CL da PRS &74 &0 &0 &0 \\
CL CTP da PRS &673 &2 &2 &0.00243 \\
\multicolumn{3}{l}{relative frequency of CC out of \textit{da}\textsubscript{2}} & 0.009& \\
\lspbottomrule
\end{tabularx}
}
\end{table}

\begin{table}
\caption{Position of CL with respect to \textit{da}-complementiser in sentences with object control predicates}
\label{T14.1b}
{\scriptsize
\begin{tabularx}{.97\textwidth}{lYYYX}
\lsptoprule
Construction & Retrieved sentences CQL & Correct target structures in sample & Estimated frequency &  Conf. interval \\
\midrule
\textit{dozvoliti} `allow' &&&&\\
CTP da PRS CL &125 &1 &1 &0.00025 \\
CTP da CL PRS &2805 &96 &2527 &0.90074 \\
CTP CL da PRS &1932 &0 &0 &0 \\
CL CTP da PRS &3142 &0 &0 &0 \\
\multicolumn{3}{l}{relative frequency of CC out of \textit{da}\textsubscript{2}} & 0& \\
\tablevspace
\textit{narediti} `order' &&&&\\
CTP da PRS CL &22 &1 &1 &0 \\
CTP da CL PRS &1242 &99 &1174 &0.94554 \\
CTP CL da PRS &721 &0 &0 &0 \\
CL CTP da PRS &814 &0 &0 &0 \\
\multicolumn{2}{l}{relative frequency of CC out of \textit{da}\textsubscript{2}} & 0& \\
\tablevspace
\textit{nat(j)erati} `force' &&&&\\
CTP da PRS CL &39 &0 &0 &0 \\
CTP da CL PRS &558 &96 &503 &0.90074 \\
CTP CL da PRS &64 &0 &0 &0 \\
CL CTP da PRS &3930 &0 &0 &0 \\
\multicolumn{3}{l}{relative frequency of CC out of \textit{da}\textsubscript{2}} & 0& \\
\tablevspace
\textit{primorati} `force' &&&&\\
CTP da PRS CL &10 &0 &0 &0 \\
CTP da CL PRS &257 &100 &248 &0.96378 \\
CTP CL da PRS &209 &0 &0 &0 \\
CL CTP da PRS &474 &0 &0 &0 \\
\multicolumn{3}{l}{relative frequency of CC out of \textit{da}\textsubscript{2}} & 0& \\
\textit{pomoći} `help' &&&&\\
CTP da PRS CL &68 &0 &0 &0 \\
CTP da CL PRS &874 &48 &331 &0.37901 \\
CTP CL da PRS &1963 &0 &0 &0 \\
CL CTP da PRS &7528 &0 &0 &0 \\
\multicolumn{3}{l}{relative frequency of CC out of \textit{da}\textsubscript{2}} & 0& \\
\tablevspace
\textit{pustiti} `let' &&&&\\
CTP da PRS CL &76 &2 &2 &0 \\
CTP da CL PRS &685 &86 &532 &0.77627 \\
CTP CL da PRS &2087 &0 &0 &0 \\
CL CTP da PRS &2271 &0 &0 &0 \\
\multicolumn{3}{l}{relative frequency of CC out of \textit{da}\textsubscript{2}} & 0& \\
\tablevspace
\textit{zamoliti } `ask' &&&&\\
CTP da PRS CL &24 &1 &1 &0\\
CTP da CL PRS &1784 &95 &1583 &0.88717\\
CTP CL da PRS &1380 &0 &0 &0\\
CL CTP da PRS &2116 &0 &0 &0\\
\multicolumn{3}{l}{relative frequency of CC out of \textit{da}\textsubscript{2}} & 0& \\
\tablevspace
\lspbottomrule
\end{tabularx}
}
\end{table}


As no gold standards have been broadly acknowledged, we decided to follow some suggestions by \citet{Wallis14}, and accordingly we estimated the precision of queries through sampling. From all sentences retrieved, with the Sample function in NoSketch Engine we took random samples of 100 sentences and checked all of them manually. The number of correct target structures can be seen in Tables \ref{T14.1}--\ref{T14.1b}. A sample of this size should usually give no more than a 10\% margin of error at a confidence level of 95\% regardless of the population size. We calculated the binomial probability confidence interval (“conf. interval” column in Tables \ref{T14.1}--\ref{T14.1b}) using the Clopper-Pearson exact method. On the basis of the worst-case scenario for the obtained confidence intervals, we recalculated raw frequencies (“retrieved sentences CQL” column in Tables \ref{T14.1}--\ref{T14.1b}) into estimated frequencies (“estimated frequency” column in Tables \ref{T14.1}--\ref{T14.1b}). The relative frequency of CC out of \textit{da}\textsubscript{2}-complements in these tables refers to the proportion of the estimated frequency of CC out of \textit{da}\textsubscript{2}-complements to the estimated frequency of all \textit{da}\textsubscript{2}-complements for the given CTP. These are analysed in the next section.

\section{Discussion: Constraints on CC from \textit{da}\textsubscript{2}-complements}
\label{Discussion:da}

Although the following discussion is based on a worst-case scenario, our material provides empirical evidence that CC out of \textit{da}\textsubscript{2}-complements into matrix clauses is indeed possible, but it is most likely a marginal phenomenon.\footnote{CC out of \textit{da}\textsubscript{2}-complements into matrix clauses is also attested in dialects: for more information and examples see Section \ref{Clitic climbing:8}.}

Our samples yielded 69 correct sentences with CC originating from 42 different top-level domains. From that we estimated a worst-case scenario of 286 CC cases in the whole examined population in srWaC. The frequencies of CC normalised to the frequency of a \textit{da}\textsubscript{2}-complement for a particular verb are presented as part of Tables \ref{T14.1}--\ref{T14.1b} and in Figure \ref{F14.1}. Analysis of the frequencies shows that CC out of \textit{da}\textsubscript{2}-complements occurs with verbs of different frequencies. The Chi-square test of dependence between syntactic type and CC yields a significant result ($p<0.001$), so the null-hypothesis that there is no relation between CC and the type of CTP  can be rejected.


\begin{figure}
\caption[Relative frequencies of CC for the retrieved CTPs]{Relative frequencies of CC for the retrieved CTPs\label{F14.1}}
\includegraphics[width=.7\textwidth]{F141}
\end{figure}


Figure \ref{F14.1} shows that the two phasal verbs \textit{prestati} ‘stop’ and \textit{početi} ‘start’ have the highest relative frequency of CC out of \textit{da}\textsubscript{2}-complements, followed by the subject control predicate \textit{pokušati} ‘try’, and raising verbs \textit{moći} ‘can’, \textit{sm(j)eti} ‘be allowed’ and \textit{nastaviti} ‘continue’. An interesting finding is that object control CTPs with both dative and accusative controllers are highly unlikely to allow CC. We did not find a single example for the predicates we selected.

Although the probability that CC out of a \textit{da}\textsubscript{2}-complement will occur is generally low, we can conclude that it is additionally influenced by the syntactic type of the CTP. Thus, it is lower for subject control verbs than for raising predicates, and not retrievable from corpus data for object control predicates.

As explained in Chapter \ref{Introductory remarks to corpus studies on CC}, in the case of CC out of \textit{da}\textsubscript{2}-complements we distinguish four different CL positions. In Table \ref{T12.1}, it may be seen that as orientation points for the CL positions we use the particle \textit{da} and matrix predicates. Tables \ref{T14.1}--\ref{T14.1b} show that sentences in which the CL is placed to the right of the verb of the \textit{da}\textsubscript{2}-complement are extremely rare (\ref{(14.3)})--(\ref{(14.5)}), albeit possible for all three investigated types of CTPs (pace \citealt[41]{Browne03}, \citealt[41]{CavarWilder94}).\footnote{For basic information on CL placement after complementisers in BCS standard varieties see Section \ref{Placement with regard to different types of hosts in BCS standard varieties}.} In example (\ref{(14.3)}) with the raising matrix predicate \textit{prestao} ‘(I) stopped’ the pronominal CL \textit{te} ‘you’ is placed to the right of its governing semi-finite verb \textit{volim} ‘(I) love’. The same CL positioning can be observed in examples with subject (\ref{(14.4)}) and object (\ref{(14.5)}) control matrix predicates. 

\begin{exe}
\ex\label{(14.3)}
\gll [\dots] ali nikada nisam\textsubscript{1} prestao\textsubscript{1} da volim\textsubscript{2} {\textbf{te}\textsubscript{2} [\dots].} \\
{} but never \textsc{neg}.be.\textsc{1sg} stop.\textsc{ptcp.sg.m} that love.\textsc{1prs} you.\textsc{acc} \\
\glt ‘[\dots] but I never stopped loving you [\dots].’
\hfill [srWaC v1.2]

\ex\label{(14.4)}
\gll Mihajlo htede\textsubscript{1} da pokuša\textsubscript{2} da urazumi\textsubscript{3} {\textbf{ga}\textsubscript{3} [\dots].} \\
Mihajlo want.\textsc{3aor} that try.\textsc{3prs} that bring.to.reason.\textsc{3prs} him.\textsc{acc} \\
\glt  ‘Mihajlo wanted to bring him to reason [\dots].’
\hfill [srWaC v1.2]

\ex\label{(14.5)}
\gll [\dots] i dozvoli\textsubscript{1} da osetim\textsubscript{2} {\textbf{te}\textsubscript{2} [\dots].}  \\
{} and allow.\textsc{2imp} that feel.\textsc{1prs} you.\textsc{acc} \\
\glt ‘[\dots] and let me feel you [\dots].’
\hfill [srWaC v1.2]
\end{exe}

\noindent It is also very clear that regardless of the CTP type, CLs tend to be placed directly after the \textit{da} particle, as in the examples with raising (\ref{(14.6)}), subject (\ref{(14.7)}) and object control (\ref{(14.8)}) CTPs below. This is the CL position which some scholars (e.g. \citealt[41]{Browne03}, \citealt[41]{CavarWilder94}) assumed to be the only possible and correct one.\footnote{In varieties which are not under direct influence of prescriptive norms, i.e. dialects and spoken varieties, CLs do not always follow the \textit{da} particle. For more information and examples, see Section \ref{Delayed placement of clitics}.}

\begin{exe}\ex\label{(14.6)}
\gll Možete da \textbf{ga}\textsubscript{2} {podignete\textsubscript{2} [\dots].}\\
can.\textsc{2prs} that him.\textsc{acc} lift.\textsc{2prs}\\
\glt  ‘You can lift it [\dots].’
\hfill [srWaC v1.2]

\ex\label{(14.7)}
\gll Nastojim da	\textbf{ih}\textsubscript{2}	{razumem\textsubscript{1}	[\dots].}\\
try.\textsc{1prs} that them.\textsc{acc} understand.\textsc{1prs}\\
\glt  ‘I try to understand them [\dots].’
\hfill [srWaC v1.2]

\ex\label{(14.8)}
\gll Ova okolnost pomogla\textsubscript{1} \textbf{je}\textsubscript{1} Pavlu da \textbf{ga}\textsubscript{2} uspešno {prati\textsubscript{2} [\dots].}\\
this circumstance help.\textsc{ptcp.sg.f} be.\textsc{3sg} Pavle that him.\textsc{acc} successfully follow.\textsc{3prs}\\
\glt ‘This circumstance helped Pavle to follow him successfully [\dots].’ \\
\strut\hfill [srWaC v1.2]
\end{exe}

\noindent Furthermore, in the case of CC, CLs tend to be placed left of the matrix verb as in (\ref{(14.9)}). However, they can appear between the CTP and the \textit{da} particle as well, as in (\ref{(14.10)}). Both examples contain the raising CTP form \textit{mogu} ‘(I) can’.

\begin{exe}\ex\label{(14.9)}
\gll [\dots] i zato \textbf{te}\textsubscript{2} ne mogu\textsubscript{1} da primim\textsubscript{2}.\\
{} and therefore you.\textsc{acc} \textsc{neg} can.\textsc{1prs} that welcome.\textsc{1prs} \\
\glt ‘[\dots] and therefore I cannot welcome you.’
\hfill [srWaC v1.2]

\ex\label{(14.10)}
\gll [\dots] ne mogu\textsubscript{1} \textbf{ih}\textsubscript{2} da {napišem\textsubscript{2} [\dots].}\\
{} \textsc{neg} can.\textsc{1prs} them.\textsc{acc} that write.\textsc{1prs} \\
\glt ‘[\dots] I cannot write them.’
\hfill [srWaC v1.2]
\end{exe}

\largerpage
\noindent If auxiliaries belonging to the CTP appear, the climbing CLs can form mixed clusters with them.\footnote{For more information and examples of simple and mixed CL clusters, see Section \ref{Clitic ordering within the cluster}.} In example (\ref{(14.11)}), the pronominal CL \textit{im} ‘them’ climbed out of the \textit{da}\textsubscript{2}-complement ‘(he) speaks’ and formed a mixed cluster with the auxiliary CL \textit{je} ‘is’ which was present in the matrix clause. We observe a similar situation in example (\ref{(14.13)}) where the matrix auxiliary CL \textit{je} formed a mixed cluster with the pronominal CL \textit{mi} ‘me’ which climbed out of the \textit{da}\textsubscript{2}-complement. These examples allow us to reject \citet[166]{Todorovic12} claim that “if the matrix verb is in the past or future tense, whose auxiliary clitics carry the tense feature, no clitic climbing is allowed out of the subjunctive \textit{da}-complement”. 

\begin{exe}\ex\label{(14.11)}
\gll [\dots] počeo\textsubscript{1} \textbf{im}\textsubscript{2} \textbf{je}\textsubscript{1} da govori\textsubscript{2} o dolasku ove grupe.\\
{} start.\textsc{ptcp.sg.m} them.\textsc{dat} be.\textsc{3sg} that speak.\textsc{3prs} about arrival this group\\
\glt ‘[\dots] he began to speak to them about the arrival of this group.’ \\
\hfill [srWaC v1.2]
\end{exe}

\noindent A reflexive CL \textit{se} can either climb with the pronominal CL, as in (\ref{(14.12)}), or it can stay in the \textit{da}\textsubscript{2}-complement, as in (\ref{(14.13)}).

\begin{exe}
\ex\label{(14.12)}
\gll U poslednje vreme \textbf{mi}\textsubscript{2} \textbf{se}\textsubscript{2} pocelo\textsubscript{1}  da desava\textsubscript{2} da {cujem\textsubscript{3} [\dots].} \\
in past time me.\textsc{dat} \textsc{refl} start.\textsc{ptcp.sg.n} that happen.\textsc{3prs} that hear.\textsc{1prs} \\ 
\glt ‘Recently, it has started happening to me that I hear [\dots].’
\hfill [srWaC v1.2]

\ex\label{(14.13)}
\gll [\dots] i počelo\textsubscript{1} \textbf{mi}\textsubscript{2} \textbf{je}\textsubscript{1} da \textbf{se}\textsubscript{2} vrti\textsubscript{2} u glavi.\\
{} and start.\textsc{ptcp.sg.n} me.\textsc{dat} be.\textsc{3sg} that \textsc{refl}  spin.\textsc{3prs} in head\\
\glt ‘[\dots] and I started to feel dizzy.’
\hfill [srWaC v1.2]
\end{exe}

\noindent The fact that two CLs that were generated by the same verb do not have to climb together out of \textit{da}\textsubscript{2} complement was observed already by \citet[182]{Stjepanovic04}. Her examples, however, concern only two pronominal CLs and not the reflexive CL \textit{se} in combination with a pronominal CL. \citet[182]{Stjepanovic04} concludes that in the case of pseudodiaclisis  only a dative CL climbs, while an accusative CL stays in the \textit{da}\textsubscript{2}-complement. We additionally argue that if two CLs are generated in the \textit{da}\textsubscript{2}-complement and occur in pseudodiaclisis, it is the pronominal that climbs, while the reflexive tends to stay in the \textit{da}\textsubscript{2}-complement, like in example (\ref{(14.13)}). In addition, since in the latter example the two CLs do not climb together, we can conclude that in Serbian there is no all-or-nothing constraint on CC  out of \textit{da}\textsubscript{2}-complements \citep[pace][8]{Rezac05}.

Moreover, it is worth mentioning that when the reflexive CL \textit{se} climbs with a pronominal CL in the matrix clause, the auxiliary CL \textit{je}  ‘is’ from the matrix clause is omitted. In other words, haplology of unlikes occurs.\footnote{See Section \ref{Morphonological processes within the cluster} for basic information and examples of haplology of unlikes. It is claimed that haplology of unlikes is obligatory in the standard Serbian variety – see Section \ref{Haplology of unlikes}. However, in dialects spoken on Serbian territory, haplology of unlikes is not obligatory – for more information and examples see Section \ref{Clitic ordering within the cluster:8}.} Since we did not find examples with three CLs (auxiliary, pronominal and reflexive) in a cluster, we may speculate that whenever there are three CLs in a sentence, the reflexive tends to stay in the \textit{da}\textsubscript{2}-complement.

Finally, it is worth mentioning that CC has not been attested for the third person accusative/genitive singular feminine CL \textit{je}. This needs further investigation, but could be due to error in tagging, i.e., if the pronominal CL \textit{je} was tagged as the verbal CL \textit{je} ‘is’. 

\section{Conclusions}
\label{Conclusions:da}
In this chapter we addressed the syntactic mechanism of CC in the context of \textit{da}\textsubscript{2}-complements. These complements are characterised by the presence of a verb inflected for person and number. This is an interesting topic because it is claimed for Czech, for instance, that finite complements block CC. The point of departure of our study was the observation that there is large disagreement as to the acceptability of CC out of \textit{da}\textsubscript{2}-complements. Whereas \citet{Stjepanovic04} allows the grammaticality of CC out of \textit{da}\textsubscript{2}-complements mainly within a unified formal theory of CC in BCS, most other authors reject the grammaticality of this structure outright. Our data allow the following answers to be given to our research questions from Section \ref{Research questions}:

\begin{enumerate}[label=A\arabic*:]
\item Serbian \textit{da}\textsubscript{2}-complements do marginally allow CC. In these cases, the CL that climbed can form a mixed cluster with the auxiliary CL of the matrix verb. We thus in principle agree with  \citet{Stjepanovic04}, but have to point out that we are dealing with a highly marginal construction. Examples supporting the occurrence of CC for all CL forms were not found. In contrast, CC out of infinitive complements in Croatian (see next chapter) and even out of stacked infinitives in Bosnian, Croatian, and Serbian \citep*{HKJ18} is not a marginal phenomenon at all.

\item CC is possible in raising and in subject control contexts. It is, however, most probably blocked in the case of object control. This is in line with what has been claimed for Czech. Moreover, the results of our corpus study on CC out of \textit{da}\textsubscript{2}-complements in Serbian are in line with our psycholinguistic study on CC out of infinitive complements in Croatian presented in Chapter \ref{Experimental study on constraints on clitic climbing out of infinitive complements} and with the corpus study on CC out of stacked infinitives in Bosnian, Croatian, and Serbian by \citet*{HKJ18}.
\end{enumerate}

In addition we would like to comment on some further evidence for the following two constraints. First, it seems that the reflexive CL \textit{se} does not climb out of the \textit{da}\textsubscript{2}-complement if there is an auxiliary CL in the matrix clause. Second, if the \textit{da}\textsubscript{2}-complement is reflexive and governs the pronominal CL and if those CLs appear in pseudodiaclisis, it is the pronominal one that climbs and the reflexive that stays in the complement. First, this suggests that the pronominal CL and reflexive \textit{se} behave differently, which leads to the conclusion that CC is not a unified syntactic mechanism. Second, examples of pseudodiaclisis in the context of CC out of the \textit{da}\textsubscript{2}-complement indicate that CC is not an all-or-nothing phenomenon, which is in line with Stjepanović’s (\citeyear[182]{Stjepanovic04}) observations \citep[pace][8]{Rezac05}. Finally, we were able to reject Todorović’s (\citeyear{Todorovic12}) hypothesis that past tense or future auxiliaries block CC.
