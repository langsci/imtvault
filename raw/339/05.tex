\part{Parameters of  variation}
\label{part2}
\chapter[Parameters of variation]{Parameters of variation: Inventory, internal organisation of cluster, and position}
\label{Parameters of variation: Inventory, internal organisation of cluster and position}




	Part \ref{part2} is dedicated to variation in the diatopic and the diaphasic dimensions. It is based on the first two steps of our methodological approach, that is, on intuition/theory and observation. The first step always involves a thorough analysis of the existing research literature, independently of the relevant theoretical framework.\footnote{A detailed description of our methodological approach can be found in Section \ref{Empirical approach in the current study}.}  



	We start with the literature-based Chapter \ref{Clitics and variation in grammaticography and related work}, which compares and sums up the treatment of CLs in the three standard varieties. First we review the principal prescriptive handbooks. This information on CLs is complemented by related theoretical studies on CLs. Although in the latter studies BCS is usually treated as one abstract system, they, like the prescriptive handbooks, help us detect differences in BCS standard varieties, mostly through contradictory statements on the acceptability of certain structures.\footnote{Although the authors of these works treat BCS as one abstract system when discussing whether certain structures are possible or not, i.e. acceptable or not, they usually use their own sense of language, their own dialects or idiolects as a baseline for comparison. Consequently, readers can easily find contradictory statements on the CL system when comparing such works.} Furthermore, in Chapter \ref{Clitics and variation in grammaticography and related work} we compile information that some authors give on variation in the CL system with respect to different registers, i.e. diaphasic variation. 



	Chapter \ref{Clitics in dialects} which follows provides complementary information. As there are no studies dealing specifically with CLs in dialects, in this chapter, like in the preceding one, we apply only the first two steps of our methodological approach: intuition/theory and observation. We summarise and synthesise data from the extensive dialectological literature, which usually consists of holistic descriptions of the grammatical and lexical systems of small local idioms. We consider these data to be highly valuable as they provide insights into the spoken idioms which might influence not only the colloquial varieties but also the standard norms. The CL system in Kajkavian and Čakavian differs significantly from the CL system in Štokavian dialects. Furthermore, some Štokavian dialects served as the base for the contemporary BCS standard varieties. We therefore focus mainly on Štokavian dialects. Nevertheless, we sometimes touch on the CL system in the Kajkavian and Čakavian dialects, mostly to comment on features which appear in contact dialects. 



	Chapter \ref{Clitics in a corpus of a spoken variety} goes one methodological step further as it describes an empirical study on the usage of CLs in spoken variety based on corpus data. We would like to emphasise that this is the first ever, pilot study on CLs in spoken BCS. In it we develop an annotation scheme for spoken data which takes into consideration the peculiarities of the syntax of spoken language. These are for instance disfluency phenomena which make it more difficult to establish clause boundaries and consequently to determine the position of the CL or the CL cluster in the clause. This study brings to the fore not only interesting findings concerning the CL inventory, internal organisation of the cluster, and morphonological processes within it, but also insights into usage-based patterns of CL placement and, most importantly, into the heaviness of the host (or in the case of DP, of the host and the initial phrase) in spoken variety. This is the very first study on the heaviness of phrases preceding CLs in BCS that goes beyond linguists' intuitive judgments. It is based on \citeposst{KCN18} approach to measuring heaviness.



	In Chapter \ref{Parameters of variation: conclusions}, we summarise the findings presented in the previous three chapters and identify some global patterns of microvariation. We present new findings concerning language prescription in the normative handbooks, which is based on the conscious selection of only some of the features attested in non-standard varieties. 
