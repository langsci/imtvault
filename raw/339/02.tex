\chapter[Terms and concepts in the light of theoretical approaches]
        {Terms and concepts in the light of theoretical approaches to the study of clitics in BCS}
\label{Our terms and concepts}
\label{Theoretical Aproaches to the Study of Clitics in BCS}
%section CC -- the definition shall be rewritten
\section{Introduction}

The goal of this chapter is to present the most important terms and concepts used throughout the monograph \textcolor{black}{in the light of existing approaches}.  \textcolor{black}{As pointed out by \citet[233]{SpencerLuis12}, there are \textsc{phonological}, \textsc{morphological} and \textsc{syntactic approaches} to the study of CLs. In phonological approaches CL positioning is defined in terms of phonological phrasing, which often interacts with information structure. Morphological approaches treat CLs as morphological units, usually as a specific type of affix, whereas syntactic approaches define CLs as function words which are associated with specific syntactic positions. As we will see below, some authors propose \textsc{mixed approaches} combining, for example, phonological and syntactic rules.}

\textcolor{black}{
In BCS, CLs have been studied by scholars of two major lines of research, which tend to ignore each other. On the one hand, CLs have been the subject of a large number of theory-driven studies by US-based linguists (for an overview, see \citealt{Boskovic00, Boskovic04}). On the other hand, some aspects of CLs have been discussed with respect to stylistic and prescriptive factors (e.g. \citealt{Reinkowski01}, \citealt{PetiStantic07b}). Most theoretically oriented studies are attempts to explain the principles of second position (see Section  \ref{Position of the clitic or the clitic cluster}) and CL ordering within the framework of a formal grammar theory (e.g. \citealt{RadanovicKocic88, RadanovicKocic96}, \citealt{Schutze94}, \citealt{Progovac96}, \citealt{Boskovic00, Boskovic04}; see Section \ref{Clitic ordering within the cluster}).}

\textcolor{black}{However, a}s mentioned in Chapter \ref{Introduction with overview}, our aim is to prepare a data-oriented, empirical in-depth study of variation in the system of CLs in Bosnian, Croatian, and Serbian. We mainly focus on systemic microvariation and on selected cases of sociolinguistic variation in the diatopic and the diaphasic dimensions.\footnote{\textcolor{black}{For more information on (micro)variation see Section \ref{Systemic vs functional microvariation}.}} These research aims mean that we are not a priori bound to a specific syntactic theory; we thus strive for descriptive category labels and terms that are maximally compatible with different theoretical approaches. Therefore, our objective is to explore the range of variation of the CL system, which might or should inform future theoretical accounts.

The rest of this chapter is structured as follows: Section \ref{Clitics} offers a cross-linguistic definition of CLs, whose properties are exemplified on BCS language material. In Section \ref{Systemic vs functional microvariation} we present our approach to the terms systemic and functional (micro)variation and describe which features they refer to. Section \ref{Parameters of microvariation} focuses on parameters of CL microvariation: inventory, internal organisation of the CL cluster, position of the CL or CL cluster, CC, diaclisis, and pseudodiaclisis. Syntactic categories relevant to the description of microvariation such as complement-taking predicates, complement types, and reflexive types are presented in Section \ref{Syntactic categories relevant for the description of microvariation}. 


\section{Clitics}
\label{Clitics}
\largerpage[2]

As already mentioned in Chapter \ref{Introduction with overview}, CLs can be defined as ``elements with some of the properties characteristic of independent words and some characteristic of affixes, in particular, inflectional affixes within words. Such elements act like single-word syntactic constituents in that they function as heads, arguments, or modifiers within phrases, but like affixes in that they are ``dependent'', in some way or another, on adjacent words'' \citep[xii]{Zwicky94}. CLs cannot bear an accent of their own and therefore need an accented word form, the so-called host, to form an independent syntactic word. In sharp contrast to affixes, CLs exhibit low selectivity towards their host, attaching to very different kinds of hosts (\textsc{promiscuous attachment}). In the following examples the CLs attach to the personal pronoun \textit{on} ‘he’ (\ref{(2.1)}), the adverb \textit{jučer} ‘yesterday’ (\ref{(2.2)}) and the adjective \textit{svakog} ‘every’ within the noun phrase (\ref{(2.3)}):

\begin{exe}
\ex\label{(2.1)}
\gll \minsp{[} On] \textbf{ga} \textbf{je} potvrdio. \\
{} he him\textsc{.acc} be\textsc{.3sg} confirm\textsc{.ptcp.sg.m} \\
\glt ‘He confirmed it.’
\hfill [hrWaC v2.2]

\ex\label{(2.2)}
\gll \minsp{[} Jučer] \textbf{ga} \textbf{je} Marin snimao na središnjem terenu. \\
{} yesterday him\textsc{.acc} be\textsc{.3sg} Marin record\textsc{.ptcp.sg.m} on central site \\
\glt ‘Yesterday Marin was recording him at the central site.’
\hfill [hrWaC v2.2]

\ex\label{(2.3)}
\gll \minsp{[} Svakog] \textbf{ga} \textbf{je} dana do studija vozio odani šofer {Clifton [\dots].}   \\
{} every him\textsc{.acc} be\textsc{.3sg} day to studio drive\textsc{.ptcp.sg.m} loyal chauffeur Clifton  \\
\glt ‘Every day his loyal chauffeur Clifton drove him to the studio [\dots].’ \\
\xspace \hfill [hrWaC v2.2]
\end{exe}\clearpage

\noindent In general, CLs can attach either to the left of a host or to its right; in the first case they are called enclitics, in the latter proclitics. For the sake of brevity, throughout this book we use the term clitic exclusively to denote enclitics. 

 \textcolor{black}{As the focus of the present monograph is on parameters of variation, we will not discuss the plethora of existing approaches and definitions of the CL category. Instead, we refer to the handbooks by \citet{SpencerLuis12} and \citet{FranksKing00}, which offer thorough overviews of the state of the art in the field.}

In the following, we will outline the most important terms and concepts used throughout the book. 

\section{Systemic vs functional microvariation}
\label{Systemic vs functional microvariation}
First and foremost, we need to clarify what we mean by the term microvariation in relation to the syntax of CLs. Variationist linguistics has been developing as an independent research paradigm  in the wake of William Labov’s pioneering work on the social stratification of English and has brought to the fore a large number of variationist studies on English, but has not yet gained firm ground in South Slavistics, where prescriptive attitudes prevail among scholars and where the field of variation is still dominated by descriptive dialectology. In work on South Slavic syntax, variation does not therefore play an important role.

\textcolor{black}{According to \citet[440]{Walker13}  linguistic variation can be, informally speaking, understood as ``different ways of saying the same thing''. 
As a matter of fact, it is more challenging to demonstrate an instance of syntactic than of phonological variation. This is because the former involves the non-trivial question of whether the given grammatical variants really present different ways of saying the same things or whether there are fine semantic or functional differences between them \citep[441]{Walker13}.
As \citet[442]{Walker13} points out, the crucial methodological step in variation analysis is, therefore, defining precisely what is understood under “the same thing”, that is, circumscribing the variable context where the speaker has a true choice between forms. 
This can be achieved by following a form-based or a function-based approach, depending on the type of variable and the purposes of the study. As we do not intend to discuss in any detail the assumption of form-meaning isomorphism (where exactly one form corresponds to one meaning and vice versa), we avoid  the function-based approach. Instead, in the present work, we follow the form-based approach. Hence, in order to examine syntactic variants, we extract CL forms that alternate with each other in a single (i.e. non-complementary) variable context or that are used for a single, identical meaning \citep[443]{Walker13}.\footnote{Notice that we do not understand the term \textit{context} in the informal way as items (words or passages) which precede and follow the studied item. Instead, we treat it as a construct of variables held constant in the study.} Identical communicative functions may or may not be present.}
 This can be exemplified by the following two sentences which show variation as to the position of the CL \textit{ga} `him'. \textcolor{black}{In these two sentences the variable context is the same: in the matrix clause, it is the same subject control complement-taking predicate \textit{žel(j)eti} `wish/want' complemented with the same complement type \textit{da}\textsubscript{2}.}\footnote{\textcolor{black}{For more information on complement-taking predicates see Section \ref{Complement-taking predicates}.}}\textsuperscript{,}\footnote{\textcolor{black}{For more information on complement types in BCS see Section \ref{Types of complements}.}} We index complement-taking predicates and their respective CLs with 1 and complements and their respective CLs with 2. 

\begin{exe}\ex
\begin{xlist}
\ex \label{(2.4a)}
\gll Mila želi\textsubscript{1} da \textbf{ga}\textsubscript{2} vidi\textsubscript{2}.\\
 Mila want\textsc{.3prs} that him\textsc{.acc} see\textsc{.3prs}\\
\ex \label{(2.4b)}
\gll Mila \textbf{ga}\textsubscript{2} želi\textsubscript{1} da vidi\textsubscript{2}.\\
 Mila him\textsc{.acc} want\textsc{.3prs} that see\textsc{.3prs}\\
\end{xlist}
\glt `Mila wants to see him.'
\hfill (BCS; \citealt[11]{Aljovic05})
\end{exe}

\noindent For the purposes of the present study, we would like to distinguish between systemic and functional factors. This distinction is meant to capture the fact that there are two basic types of conditioning factors:\footnote{\textcolor{black}{See} the discussion on syntactic variables in \citet{Romaine81}. She emphasises that purely syntactic variables differ from phonological variables because the latter always imply a social or stylistic factor \citep[cf.][15]{Romaine81}.} First, \textsc{systemic microvariation}, which is defined as purely language-internal, i.e. as variation between a dependent variable (e.g. CL position) and an independent variable encoded in the linguistic context. Second, variation in the traditional sociolinguistic sense, which depends on features relating to space (\textsc{diatopic}), to social groups (\textsc{dia\-stra\-tic}) or to the modes of language use in different situations (e.g. oral vs written, \textsc{diaphasic}).\footnote{This distinction, which most variationist linguists can probably agree on, goes back to \citet[111]{Coseriu80}: ,,Es gibt nämlich in einer historischen Sprache zumindest drei Arten der inneren Verschiedenheit, und zwar: \textit{diatopische} Unterschiede (d.h. Unterschiede im Raume), \textit{diastratische} Unterschiede (Unterschiede zwischen den sozial-kulturellen Schichten) und \textit{diaphasische Unterschiede}, d.h. Unterschiede zwischen den Modalitäten des Sprechens je nach der Situation desselben (einschließlich der Teilnehmer am Gespräch).“} The focus of the present monograph is on the range of systemic factors and only secondarily on the conditional sociolinguistic factors. In some places we discuss the link between the two, but we refrain from a systematic sociolinguistic variationist account. We are mainly interested in the range and the limits of microvariation determined by linguistic contexts. The choice between variants is governed by what one might call variable rules of grammar. We follow \citet[141]{Walker10} who argues that for a full understanding of variation we have to take formal or structural considerations into account. It goes without saying that we are  unable to cover the whole area of variation delineated by the three dimensions in question. 

We present an in-depth empirical study on the syntactic microvariation in the area of CC, for which we identify structural factors (i.e. constraints).\footnote{For basic information on CC see Section \ref{Clitic climbing} below and for in-depth information on CC see Part \ref{part3}.} Among the diatopic conditioning factors, we mainly deal with variation between the standard norms of Croatian, Serbian, and to a lesser degree of Bosnian, as described in publications relevant for language corpus planning like authoritative reference handbooks used in schools, universities, and in the media. Furthermore, we give a literature-based account of variation in the BCS dialects spoken in Serbia, Croatia, Bosnia and Herzegovina, Montenegro, and Kosovo.

It is well known that the three standards show major differences in their lexicons, which includes cases where one and the same lexical unit belongs to different diastratic or diaphasic layers of the language.\footnote{A good reference source for this kind of difference is \citet{Samardzija15}.}  As to core grammar, the differences are much more subtle. \citet[547]{Piper09}, who discusses the differences between the Croatian and Serbian standards, points out that both varieties or languages have the same parts of speech, grammatical categories, grammemes, and morphonological processes. Following \citet[542f]{Piper09} we can distinguish:
\begin{enumerate}
\item differences in the inventory of grammatical constructions (e.g. the reflexive impersonal construction with the second argument in the accusative),\footnote{For more information on the reflexive impersonal construction see Section \ref{Different types of reflexives} in this chapter.}
 \item differences concerning variants of forms (e.g. dative forms of \textit{ko/tko} `who': \textit{kome} in Serbian and \textit{kome}, \textit{komu} or \textit{kom} in Croatian), 
 \item differences in the frequency of specific forms (short forms of the adjective in oblique cases),
 \item differences in the stylistic value of specific forms (e.g. the preposition \textit{u} plus genitive is perceived as archaic in Serbian but neutral in Croatian).
\end{enumerate}

  It is noteworthy that in his overview \citet{Piper09} mentions CLs as a feature dividing Serbian and Croatian. He observes that the frequency of specific forms and the stylistic values of specific forms vary. The Croatian usage of the impersonal reflexive and the CL form \textit{si} are recognised as a difference in the inventory of grammatical constructions. In addition, he also notes the difference with respect to phrase splitting, which ``in modern standard Serbian [is] less common or felt as regionalism'' \citep[546]{Piper09}. 

As far as possible, we mark all examples in our study with abbreviations indicating the national varieties: Croatian (Cr), Serbian (Sr) and Bosnian (Bs). When an example has been culled from the web corpora we restrict ourselves to the corpus names hrWaC, srWaC, and bsWaC. It goes without saying that some authors stick to the glossonym Serbo-Croatian, for which we use the label BCS.

We also record variants discussed in the normative literature which do not gain approval as ``good'' or ``correct'' language use. These data can be interpreted as variants determined by diatopic, diastratic or diaphasic factors. Furthermore, we dedicate one chapter to the use of CLs in a spoken variety, specifically in Bosnian, taking into account the diaphasic dimension of variation. The diaphasic dimension is additionally addressed in a corpus study based on a web subcorpus containing texts from a Croatian forum. As our empirical approach is based on data from the literature, from web and oral corpora, and finally from psycholinguistic experiment, we have nothing to say about variation related to social factors.\footnote{Our empirical approach is presented in more detail in Chapter \ref{Empirical approach to clitics in BCS}.} This is a separate research question which would require a completely different research design. 

We acknowledge that due to the lack of space and available human and language resources we are not able to study all three national variants of BCS with the same analytical depth.\footnote{We refer to ressources such as available electronically stored and morphosyntactically annotated corpora of a sufficient size.} Furthermore, not all investigated phenomena are equally common in all varieties. We therefore concentrate on varieties in which the most data for certain structures were available or easily accessible. The monograph thus has a certain bias towards Croatian. 

\section{Parameters of microvariation}
\label{Parameters of microvariation}
In this section, we present the dimensions or parameters of variation relevant for the CL systems of BCS, \textcolor{black}{and discuss previous approaches}. Note that we understand the term \textcolor{black}{parameters} not in the sense of Universal Grammar but as a set of contexts and variables pertaining to CLs.

\subsection{Inventory}

An important parameter of (micro)variation is the inventory of CLs in the Bosnian, Croatian, and Serbian standard languages and their non-standard varieties. The inventory of CLs encompasses the following four types:

\begin{enumerate}
	\item	personal pronouns,
	\item	verbal CLs, 
	\begin{enumerate}
		\item	copula/past tense auxiliary \textit{biti},
		\item	conditional auxiliary,
		\item	future auxiliary,
	\end{enumerate}
	\item	the reflexive marker(s) \textit{se} and \textit{si},
	\item	the polar question marker \textit{li}.
\end{enumerate}

As already mentioned in Chapter \ref{Introduction with overview}, in our project we cover mainly verbal, pronominal and reflexive CLs. We thus exclude proclitic elements like prepositions and only touch upon the polar question marker \textit{li}. The question marker differs from pronominal, reflexive and verbal CLs in its syntactic function and lack of a nonclitic equivalent. Whereas we discuss the variation within each CL type in Chapters \ref{Clitics and variation in grammaticography and related work}, \ref{Clitics in dialects}, and in \ref{Clitics in a corpus of a spoken variety}, in this chapter we discuss the reflexive marker in more detail because due to its multifunctionality it turns out to be an important factor in microvariation (see Section \ref{Different types of reflexives} below). 

The BCS CLs, except the polar question marker \textit{li}, belong to what since the seminal work by \citet{Zwicky77} has been called special clitics. This term has been contested and its usefulness has been called into question (see the discussion in \citealt[41--45]{SpencerLuis12}). For our purposes, it suffices to point out that CLs in BCS have ``a significantly different distribution from their non-clitic counterpart'' \citep[6]{FranksKing00}. Whereas the full forms of verbs and personal pronouns can change their position in the sentence depending on information structure, the CLs in question have a much more fixed position in the sentence. An important feature setting CLs apart from their stressed counterparts is coordination, which is possible with the full forms but completely ruled out with CLs. There does not seem to be variation in this respect; see the sentence presented in (\ref{(2.5a)}) and its permuted counterpart (\ref{(2.5b)}):\footnote{We did not find a single instance of the string \textit{i te i ga} in hrWaC, srWaC or bsWaC.}

\begin{exe}\ex
\begin{xlist}
\ex[]{\label{(2.5a)}
\gll Vara i \minsp{[} tebe] i \minsp{[} njega]. \\
 cheat\textsc{.3prs} and {} you\textsc{.acc} and {} him\textsc{.acc}\\}
\ex[*]{\label{(2.5b)}
\gll Vara i \textbf{te} i \textbf{ga}. \\
 cheat\textsc{.3prs} and you\textsc{.acc} and him\textsc{.acc} \\}
\end{xlist}
\glt ‘(S/he) is cheating on both you and him.’ 
\hfill [hrWaC v2.2]
\end{exe}

\noindent The question of placement concerns the internal organisation of CL clusters (see the next subsection) on the one hand and the position of the cluster in the sentence on the other. 

\subsection{Internal organisation of the clitic cluster}
\label{Internal organisation of the clitic cluster}
\subsubsection{Clitic ordering within the cluster}
\label{Clitic ordering within the cluster}
\label{Approaches to clusters: syntax vs morphology}
If several CLs occur in one clause, they \textcolor{black}{usually} occur in a cluster, i.e. ``a string of clitics that neither allows insertion of non-clitic elements nor permutation of clitics, when they are contiguous'' \citep[181]{ZimmerlingKosta13}.\footnote{\textcolor{black}{For more information on diaclisis, i.e. situations where one CL can be in clausal second position (or in delayed placement), while an additional clusterising CL is placed to its right, see Sections \ref{Diaclisis and pseudodiaclisis}, \ref{Diaclisis:8}, and \ref{Diaclisis:9}.}} The CLs occupy a fixed slot within the cluster, available only to this particular CL or type of CL \citep[182]{ZimmerlingKosta13}.
\textcolor{black}{All theoretical models face a significant problem in this relative order of CLs, since it does not correlate with any other ordering rule in BCS.}
Authors arguing in favour of a morphologically oriented approach to the ordering of CLs assume a morphological template similar to affix order within a word, whereas in syntactic approaches the linear order within the cluster is explained in terms of syntactic positions. 

\textcolor{black}{\citet[63]{Boskovic01} claims that the generative syntactic account of CL order is more principled, since under this account CL order within the cluster matches the structural height or position of the CLs in the clause structure. He further argues that conversely, morphological template analysis merely provides a formal way of stating the idiosyncrasies of BCS CL ordering \citep[64]{Boskovic01}.}
\textcolor{black}{The proponents of the syntactic approach tend to seek general explanations relating the positions to universal syntactic heads. 
A major problem arises because ordering within CL clusters differs cross-linguistically. Even a closely related language like Czech shows different ordering (e.g. reflexive before pronominal).\footnote{See also the putative generalisations concerning CL ordering in Romance languages discussed in \citet{HeggieOrdonez05}.}  A second typologically interesting feature of the ordering sequence is that unlike e.g. clusters in Romance languages it contains not only pronominal and reflexive elements, but also verbal elements.
In contrast, proponents of a morphological approach usually refrain from such generalisations and may explain a given pattern аs “a caprice of history as any property of the language faculty” \citep[319]{SpencerLuis12}.}

Slightly revising the proposal in \citet[29]{FranksKing00}, we argue in favour of the following ordering within the cluster for BCS: 

\begin{exe}\sn
\textit{li} ${}>{}$ \textsc{verbal}* ${}>{}$ \textsc{pron\textsubscript{\textsc{dat}}} ${}>{}$ \textsc{pron\textsubscript{\textsc{acc}}} ${}>{}$ \textsc{pron\textsubscript{\textsc{gen}}} ${}>{}$ \textsc{refl} ${}>{}$ \textit{je}\\
* except \textit{je} = \textsc{prs.3sg} of \textit{biti} `be'
\end{exe}

In contrast to \citet{FranksKing00}, we do not use the label \textsc{aux} because the ordering sequence does not seem to distinguish between the copula and the auxiliary uses of the forms of \textit{biti} `be'.


The most puzzling feature of this ordering sequence \textcolor{black}{and without doubt a major challenge for any theory of BCS CL ordering is presented by the position of the present tense third person singular verbal CL \textit{je} ‘is’. Unlike other verbal CLs it follows the pronominal (and reflexive) CLs, and  appears in cluster-final position as a} sort of outlier.\footnote{In standard BCS varieties the verbal CL \textit{je} is omitted after the reflexive CL \textit{se}, but this is not always the rule in non standard varieties, for more information see Sections \ref{Haplology of unlikes}, \ref{Haplology:8}, and \ref{Chapter9:Morphonological processes within the cluster}. Moreover, in non-standard varieties the reversed order where the CL \textit{je} precedes the reflexive CL \textit{se} is attested, see Sections \ref{Clitic ordering within the cluster:8} and \ref{Internal organisation of the clitic cluster:9}.} The verbal CLs are thus split between the left and the right periphery of the cluster.

\textcolor{black}{In order to remain consistent with the above-mentioned idea that CL order should match the structural height or position of the CLs in the clause structure, \citet{MiseskaTomic96}, \citet{Progovac05}, and \citet{Franks17} account for differences in the slot verbal CLs occupy in the cluster with the existence of distinct heads: one for \textit{je}, lower than for pronominal CLs, and one for other verbal CLs, higher than for pronominal CLs.}

\textcolor{black}{Although this solution sounds very attractive, \citet[126]{Boskovic01} presents examples against it. He bases his argumentation on the observation that it is possible to insert a constituent between the pronominal CLs and \textit{je} and that in the case of verbal phrase ellipsis, verbal phrase fronting, and parenthetical placement \textit{je} behaves like other verbal CLs and precedes pronominal (and reflexive) CLs.\footnote{\textcolor{black}{Interestingly, \citet[224]{Franks17} starts his explanation from exactly the opposite statement: nothing can be inserted between pronominal CLs and be.\textsc{3sg} CL \textit{je} -- but his example comes from Bulgarian.}} This serves as evidence that \textit{je} is higher in the syntax than pronominal CLs: that is, it does not diverge from other verbal CLs in terms of generation place, only its  phonological form is the last to occur on the surface, after pronominal CLs. As to the reasons why \textit{je} must be pronounced in the tail of the cluster, \citet[130f]{Boskovic01} suggests that \textit{je} is in the process of losing its clitichood.}

\textcolor{black}{Finally, \citet{Migdalski20} proposes a syntactic approach in which CL variants of \textit{biti} are pure phi-feature bundles. In this approach \textit{je} specifies only the number feature (which is also present in the participle structure), whereas other CLs also carry the person feature, which results in different projections (Aux\textsuperscript{0} for \textit{je} and T\textsuperscript{0} for the others) and leads to different ordering of the CLs. This, however, does not explain why only \textit{je} is affected and not biti.\textsc{3pl} \textit{su}.}

\textcolor{black}{It must be pointed out that \textit{je} behaves peculiarly also in other respects, and not only regarding the slot it occupies. It is morphologically different from other CLs since the cliticised form originates from the root and not from the ending as is the case for all other verbal clitics (as pointed out e.g. by \citealt{MiseskaTomic96}). Secondly, it participates in the morphonological processes of suppletion, omission, and haplology of unlikes (see the next section) to a far greater extent than some other verbal CLs.\footnote{\textcolor{black}{For more information on omission see \citet{MeermannSonnenhauser16}.}}\textsuperscript{,}\footnote{\textcolor{black}{This range of variation is hard to explain using purely formal approaches and would require a separate extensive study including a diachronic perspective. In fact, the explanation for the variation in the ordering and idiosyncrasy of the position of the verbal CL \textit{je} within the cluster could be connected to the relative age of CLs, as suggested by \citeauthor{Grickat72} (\citeyear[95]{Grickat72}; cf. also \citealt[189]{ZimmerlingKosta13}). \citet[60]{Pavlovic13} claims that in 12\textsuperscript{th}--13\textsuperscript{th} century Old Serbian vernacular texts the hierarchy of the CLs within the cluster was as follows: 1. the interrogative particle \textit{li}, 2. the conditional forms of the verb ‘be’, 3. the dative pronominal CLs, 4. the accusative pronominal CLs, and 5. the present tense forms of the verb ‘be’.}} In the current work we do not strive to explain the slot taken by \textit{je} in the cluster, but we do take a closer look at morphonological processes.}




We distinguish \textsc{simple} \textsc{clusters} and \textsc{mixed} \textsc{clusters}. In the former, CLs originate in one clause like in example (\ref{(2.6)}):\footnote{In non-standard varieties various reversed orders of CLs within a cluster are attested: see Chapters \ref{Clitics and variation in grammaticography and related work}, \ref{Clitics in dialects}, and \ref{Clitics in a corpus of a spoken variety}.}

\protectedex{
\begin{exe}\ex\label{(2.6)}
\gll I stalno \textbf{smo}\textsubscript{1} \textbf{mu}\textsubscript{1} \textbf{se}\textsubscript{1} vraćali\textsubscript{1}. \\
 and constantly be\textsc{.1pl} him\textsc{.dat} \textsc{refl} return\textsc{.ptcp.pl.m}\\
\glt ‘And we kept returning to him.'
\hfill [hrWaC v2.2] 
\end{exe}
}

\noindent In the latter, CLs originate in the matrix clause and in its infinitive or \textit{da}\textsubscript{2}-com\-ple\-ment, as in the case of clitic climbing. In the following example (\ref{(2.7)}), in the cluster \textit{si ga} the accusative pronoun \textit{ga} `him' depends on the embedded infinitive \textit{ubiti} `to kill' and the auxiliary \textit{si} on \textit{mogla} `could'.


\begin{exe}\ex\label{(2.7)}
\gll Kako \textbf{si}\textsubscript{1} \textbf{ga}\textsubscript{3} mogla\textsubscript{1} dati\textsubscript{2} ubiti\textsubscript{3}?\\
 how be\textsc{.2sg} him\textsc{.acc} can\textsc{.ptcp.sg.f} give\textsc{.inf} kill\textsc{.inf}\\
\glt ‘How could you have him get killed?’
\hfill [hrWaC v2.2]
\end{exe}


\noindent In some cases, CLs do not show up in a cluster but occupy separate positions (see Section \ref{Diaclisis and pseudodiaclisis} below).

\subsubsection{Morphonological processes within the cluster}
\label{Morphonological processes within the cluster}
The ordering of CLs is not restricted only to the positioning of each CL: certain combinations of CLs within the cluster are subject to \textsc{morphonological processes}. As \citet[685]{NeelemanKoot06}  note, many languages exhibit a resistance against accidental repetition of morphemes (\textsc{repeated} \textsc{morph} \textsc{constraint}). One solution is the avoidance of such repetitions. In BCS, three types of such morphonological processes can be found. The first is called \textsc{suppletion}: either morpheme is associated ``with a different realization, typically based on a subset or a superset of its features'' \citep[686]{NeelemanKoot06}. This is observed when the homophonous CLs, prono­un her.\textsc{acc} \textit{je} and be.3\textsc{sg} \textit{je}, co-occur as in example (\ref{(2.8a)}); the string \textit{je} \textit{je} is altered to \textit{ju je}, see example (\ref{(2.8b)}).\footnote{The sequence \textit{je} \textit{je} can be labelled as incorrect only in standard BCS varieties, since it is attested not only in Štokavian, but also in Čakavian dialects; for more information and examples see Section \ref{Haplology:8}.}

\begin{exe}\ex
\begin{xlist}
\ex[*]{\label{(2.8a)}
\gll On \textbf{je} \textbf{je} čekao u njihovoj ulici.\\
 he her\textsc{.acc} be\textsc{.3sg} wait\textsc{.ptcp.sg.m} in their street\\}
\ex[]{\label{(2.8b)}
\gll On \textbf{ju} \textbf{je} čekao u njihovoj ulici.\\
 he her\textsc{.acc} be\textsc{.3sg} wait\textsc{.ptcp.sg.m} in their street\\ }
\end{xlist}
\glt ‘He was waiting for her in their street.’
\hfill [srWaC v1.2]
\end{exe}

\noindent The second type is identified for the co-occurrence of the reflexive \textit{se} (\textsc{pseudo-twins}, \citealt[79]{Junghanns02}). In the example presented in (\ref{(2.9)}) we have two lexical reflexive verbs (\textit{bojati se} `be afraid', \textit{vratiti se} `return'), which would result in the repetition of \textit{se}. Since only one reflexive CL \textit{se} is present in the sentence, it is an instance of \textsc{haplology}, i.e. the deletion of one \textit{se}:

\protectedex{
\begin{exe}\ex\label{(2.9)}
\gll Boji\textsubscript{1} \textbf{se}\textsubscript{1$+$2} vratiti\textsubscript{2} u svoje rodno Cetinje [\dots].\\
 fear\textsc{.3prs} \textsc{refl} return\textsc{.inf} in own birth Cetinje \\
\glt ‘He is afraid to return to his hometown Cetinje [\dots].’
\hfill [hrWaC 2.2]
\end{exe}
}

The third type involves the combination of the verbal CL \textit{je} and the reflexive CL \textit{se}. In this case \textit{je} is deleted: see the example presented in (\ref{(2.10b)}).\footnote{\textcolor{black}{This type of morphonological process within the cluster is actually a feature of standard language, for more information on deletion or non-deletion of the verbal CL \textit{je} in standard varieties, dialects, and spoken Bosnian see Sections \ref{Haplology of unlikes}, \ref{Haplology:8}, and \ref{Chapter9:Morphonological processes within the cluster}. Moreover, this type of so-called morphonological process does not have a purely morphonological nature. For instance, \citet[564]{Ridjanovic12} shows that the verbal CL \textit{je} which is a copula, will not be omitted, see Section \ref{Haplology of unlikes}. The partial syntactic nature of this constraint is also observable in the fact that it does not affect the homophonous pronoun her.\textsc{gen} \textit{je}, which occurs in the reversed CL order \textit{je} \textit{se}. The genitive pronoun \textit{je} cannot be deleted since it is an argument, while the auxiliary verb \textit{je} can be and to a high degree in standard varieties is deleted.}} It is interesting to note that here the deletion affects phonologically similar but not identical morphs. This means that haplology can occur when CLs are not phonologically identical (\textsc{haplology} \textsc{of} \textsc{unlikes}, \citealt{RosenHana17}). 

\begin{exe}\ex
\begin{xlist}
\ex[?]{\label{(2.10a)}
\gll Idol \textbf{se} \textbf{je} nalazio u Brandenburgu. \\
 idol \textsc{refl} be\textsc{.3sg} find\textsc{.ptcp.sg.m} in Brandenburg\\}
\ex[]{\label{(2.10b)}
\gll Idol \textbf{se} nalazio u Brandenburgu. \\
 idol \textsc{refl} find\textsc{.ptcp.sg.m} in Brandenburg\\}
\end{xlist}
\glt ‘The idol was located in Brandenburg.’
\hfill [srWaC v1.2]
\end{exe}

\noindent Haplo­logy does not seem to affect the homophonous pronoun her.\textsc{gen} \textit{je}, which occurs in the reversed CL order \textit{je} \textit{se}: see the example presented in (\ref{(2.11)}). 

\protectedex{
\begin{exe}\ex\label{(2.11)}
\gll Bojim \textbf{je} \textbf{se}. \\
 fear\textsc{.1prs} her\textsc{.gen}  \textsc{refl} \\
\glt ‘I am afraid of her.’
\hfill [bsWaC v1.2]
\end{exe}
}

\noindent When it comes to such morphonological processes, CL clusters behave more like affixes than like words \citep[121f]{SpencerLuis12}.

\subsection{Position of the clitic or the clitic cluster}
\label{Position of the clitic or the clitic cluster}

\subsubsection{Second position}
\label{Second position}
As mentioned above, the CLs in BCS are special clitics. This means that they are subject to word order restrictions characteristic of this and only of this category. The single CL or CL cluster occupies what is frequently called the Wackernagel or second position in the clause (2P). There is a long debate on what 2P actually is.

CLs are positioned within a clause with respect to a constituent which serves as a host. In this book, we use the term 2P in a narrow sense as referring to the position after the first constituent of the clause. This covers the position after a full phrase and after a complementiser. \textcolor{black}{Delayed placement may be triggered by so-called heavy phrases.}\footnote{For more information on heavy phrases see the next Section \ref{Barriers and delayed placement of clitics (DP)}.} Generally speaking, CLs can attach to any type of phrase to which they bear or do not bear a syntactic relationship (promiscuous attachment, mentioned above). The possessive dative is a special case because it has a fixed position in the sentence: it either has to follow the noun/phrase denoting the ``possessed entity'' (\ref{(2.12a)}) or it comes after the first stressed word in the phrase which it modifies (\ref{(2.12b)}) \citep[cf.][559]{Ridjanovic12}; see our transformation of Ridjanović’s example (\ref{(2.12c)}):

\begin{exe}\ex
\begin{xlist}
\ex[]{\label{(2.12a)}
\gll Starija sestra \textbf{mu} pjeva u horu. \\
 older sister he\textsc{.dat} sing\textsc{.3prs} in choir \\}
\ex[]{\label{(2.12b)}
\gll Starija \textbf{mu} sestra pjeva u horu. \\
 older he\textsc{.dat} sister sing\textsc{.3prs} in choir \\}
\ex[*]{\label{(2.12c)}
\gll Pjeva \textbf{mu} starija sestra u horu. \\
  sing\textsc{.3prs} he\textsc{.dat} older sister in choir \\}
\end{xlist}
\glt ‘His older sister sings in a choir.’ 
\hfill (Bs; \citealt[559]{Ridjanovic12})
\end{exe}

\noindent This is a case of syntactic microvariation which is discussed in Section \ref{Positioning of single clitics and clitic clusters}.

\subsubsection{Approaches to 2P effects: syntax, phonology and information structure}
\label{Approaches to 2P effects: syntax, phonology and information structure}
\label{Barrier-template theory}

\textcolor{black}{Many studies on BCS are attempts to explain the principles of 2P and CL ordering within the framework of Minimalism. The discussion essentially concerns the division of labour between syntactic structure on the one hand and phonology or prosody on the other. According to \citet{Boskovic00}, three different schools can be distinguished among generative models:\footnote{\textcolor{black}{Recent developments include parametric approaches. Here we can mention the somewhat controversial suggestion of \citet{Runic14} and \citet{Boskovic16} that 2P effects relate to the lack of a determiner phrase layer in languages and the lack of articles. This restrictions seems to be too general and are criticised and modified by \citet{Migdalski21}, who also provides an alternative (parametric) approach related to the loss of verbal morphology \citep{JungMigdalski15, Migdalski16, Migdalski20}. As explained in Section \ref{Clitics and microvariation} we refrain from both overall typological generalisations, and diachronical perspective. Thus, we do not discuss these two ideas in the current work.}}\textsuperscript{,}\footnote{\textcolor{black}{Although 2P cliticisation is rarely a topic of non-generative works, for Czech see \citet{Fried94}}.}}
\begin{enumerate}

\item \textcolor{black}{The strictly syntactic approach explains 2P effects exclusively by syntactic mechanisms \citep[e.g.][]{Progovac96, Progovac93c, Franks97}. In these accounts it is argued that ``clitic placement is a syntactic phenomenon and should be assimilated to other more familiar types of syntactic movement rules, rather than involving a special kind of phonological clitic placement operation. Clitics are syntactic entities—in particular, functional heads—and they move as such'' \citep[111]{Franks97}. In Minimalism a clause is assumed to be headed by several functional projections, which hierarchically dominate the lexical projection of the verb. Accordingly, the CL is moved to the left-hand periphery in the syntax, where it leans to the right of the element that is in the so-called complementiser position. \citet[412]{Progovac96} argues that CLs move in syntax – their distribution is constrained not by phonological, but by syntactic principles. She claims that the strongest argument that the placement of CLs is sensitive to syntax/semantics comes from subjunctive-like complements \citep[422f]{Progovac96}. While in in\-di\-ca\-tive-like complements CLs are strictly clause-bound and must attach to the local complementiser, in subjunctive-like complements CLs attach either to the local complementiser or to the matrix complementiser position.}  

\item \textcolor{black}{The strictly phonological approach postulates that 2P is governed exclusively by phonological rules. This position is mainly represented by \citet{RadanovicKocic88, RadanovicKocic96}. According to this approach the target of the movement is not a syntactically defined constituent or syntactic position, but the intonational phrase \citep[441]{RadanovicKocic96}.\footnote{\textcolor{black}{\citet[441]{CavarWilder94} argue against a purely phonological approach to the 2P phenomenon. In their view it is not desirable to assume phonological rules which have the power to move material around in phonological representations in order to capture marginal cases like phrase splitting \citep[cf.][441]{CavarWilder94}.}} Nevertheless, bear in mind that even in her so-called ``strictly phonological approach'' the position of the CL or the CL cluster is hard to explain only within the domain of phonology. As we show in Section \ref{The limits of phrase splitting in BCS standard varieties}, \citet{RadanovicKocic88} uses syntax to explain variation and constraints on phrase splitting. For instance, she argues that whether CLs are placed after the first word of a two-word initial subject or after the whole phrase depends on the structure of the subject phrase \citep[cf.][112]{RadanovicKocic88}. Further, she claims that there is an important difference between initial two-word subject phrases and non-subject phrases, and concludes that only subject phrases can be split, whereas others cannot \citep[111]{RadanovicKocic88}. Hence, it is more than obvious that in her ``strictly phonological approach'' \citet{RadanovicKocic88} uses syntax to explain the limits of phrase splitting in BCS.} 

\item \textcolor{black}{In mixed approaches, both the syntactic and the phonological components of the language system are responsible for the positioning of the CL. For example, \citet{Schutze94} assumes that the CL is moved by syntax, but specific contexts also permit phonological movements (weak syntax approach). In contrast, \citet[]{Boskovic00, Boskovic01}
 assigns the dominant role to phonology. According to the so-called weak phonology approach, movement takes place in the syntax, but in addition a phonological filter is in operation.} \textcolor{black}{In other words, the 2P is actually a constraint on phonological form representations which filters out all constructions where CLs are found in any other position of their intonational phrase than the second \citep{Boskovic00}.}
 
 \textcolor{black}{Within the mixed approach \citet[431]{CavarWilder94} treat CL forms as both syntactic CLs and phonological enclitics. More specifically, they consider CL placement in Croatian to be a syntactic process \citep[431]{CavarWilder94}. According to them the 2P effect can be best accounted for syntactically. However, they attribute the ill-formedness of the 1P to a phonological (prosodic) property of CLs \citep[431]{CavarWilder94}. According to \citet{WilderCavar94a} and \citet{WilderCavar94b} the CL 2P effect results from the interaction between a syntactic CL placement rule and a phonological filter.} 
 
 \textcolor{black}{A similar view is presented in \citet{Franks00}. Second position CLs as verbal features on their way up the verbal extended projection form a syntactic cluster which ends up in the highest functional position of the clause \citep{Franks00}. If syntax leaves CLs without a proper host, a lower copy of the CL cluster is pronounced. In other words, phonological form plays a filtering role \citep{Franks00}. Similar claims can be found in \citet[264]{Boskovic95}: syntax proposes structures to phonology, which discards some syntactically well-formed structures since they violate certain phonological form requirements. In other words, the role of phonology is to filter out the output of the syntactic component \citep[264]{Boskovic95}. This is actually in contradiction to \citet{Boskovic00}, where it is argued that no special syntactic procedure is involved in CL placement.}
 
\end{enumerate}

\textcolor{black}{It is worth noticing that there is one major problem with the phonological and mixed approaches. Namely, as \citet[70]{DFZ09} point out, the advocates of the idea of an intonational phrase do not provide any experimental acoustic evidence for the postulated pauses or intonation units.} \textcolor{black}{ Moreover, as \citet[136]{CavarSeiss11} put it, ``all these accounts have in common that they cannot motivate or explain the intra-linguistic variation, i.e. the alternations of the different constructions''. Finally, as \citet[175]{ZFD17} observe, regardless of the theoretical frame of reference, only main clauses with initial arguments have been investigated.}\footnote{\textcolor{black}{\citeposst{Browne75} detailed description is the only exception to this.}} \textcolor{black}{The question is whether this somewhat impoverished empirical landscape can indeed give valid formal accounts of the bifurcation into two 2P types: 2W and after the first phrase \citep[175]{ZFD17}.}

\textcolor{black}{A factor which might be worth studying in more detail in future is \textsc{information structure} which, however, has not received much attention in the existing placement analyses (\citealt[cf.][134]{CavarSeiss11}, \citealt[71f]{DFZ09}). Several authors \citep[e.g.][]{DFZ09, Diesing10, CavarSeiss11, ZFD17, DiesingZec11, DiesingZec17} discuss the possibility that information structure may have an influence on the positioning of CLs in simple clauses (see below).}

\largerpage
\textcolor{black}{In \citeposst{DFZ09} acceptability judgment experiment, object argument sentences were more likely to be accepted with the CL after the first word when the first word was a demonstrative. The difference in the acceptance rate between split object argument constituents with adjectives and demonstratives was statistically significant \citep[69]{DFZ09}. Based on the reported findings, \citet[69f]{DFZ09} believe that the preferred status of demonstratives over adjectives as first word CL hosts suggests potential differences in information structure. This conjecture is based on the status of demonstratives as deictic and/or specific determiners in languages that do not otherwise have determiners \citep[70]{DFZ09}. More specifically, they argue that it is more likely for a demonstrative than an adjective to be a point of contrast in Serbian \citep[cf.][70]{DFZ09}. \citeposst{DFZ09} hypothesis is ``that clitic positioning is an interface phenomenon, in the broadest sense of the term, with at least prosody, syntax, and information structure contributing to the selection between the competing configurations'' in both the argument- and predicate-initial main clauses. This is further elaborated on in \citet[13]{DiesingZec17} with the conclusion that in the predicate case, prosody alone is responsible for the selection of hosts for 2W placement, while in the argument case, prosody interfaces with syntax and the information structure in the selection of hosts for 2W placement.}


\textcolor{black}{\citet[139]{CavarSeiss11} explicitly argue that different word order positions of CLs are related to differences in their specific information theoretic properties. More specifically, they claim that both 2P types, i.e. 2W and 2P after the first phrase, can be best explained in purely syntactic terms \citep[133]{CavarSeiss11}. In their approach, the assumed cases of phonological CL placement in the 2W type of placement are analysed as instances of split constituent constructions \citep[133, 136, 141]{CavarSeiss11}. According to \citet[145]{CavarSeiss11} the CL or CL cluster always attaches after the first syntactic constituent, which in information structure terms can be a topic or a contrastive focus. If the first syntactic constituent is a split part of a syntactic constituent, it triggers a contrastive focus reading and consequently requires a specific intonational contour \citep[145]{CavarSeiss11}. In other words, according to \citet[133]{CavarSeiss11} word order variation is related to information structure: it implies scope differences in a hierarchical (syntactic) representation and not the scope-neutral phonological processes. In their analysis the prosody-syntax interface remains quite simple, since they do not utilize complex word rearrangement mechanisms outside of syntax, or at the level of phonological representation \citep[133]{CavarSeiss11}.}

\textcolor{black}{\citet{AvgustinovaOliva95} discuss CL positioning in Czech and propose an explanation for 2P which is based on the approach to the communicative structure of the sentence proposed among others by \citet{Sgalletal86}. According to this approach, the first position is defined as “preceding lexical material as a single substantial communicative segment” \citep[25]{AvgustinovaOliva95}.}

\textcolor{black}{A typological approach to CLs which combines syntactic and morphological features with information structure has been elaborated mainly by Anton Zimmerling on the basis of data from various Slavic languages. It is based on three principles:} 

\begin{enumerate}
	
\item \textcolor{black}{the Template Principle,}
\item \textcolor{black}{Constituency Conditions predicting the choice of the CL host,}
\item \textcolor{black}{Barrier Rules generating derived word orders with clusterising CLs \citep[203]{ZimmerlingKosta13}.}

\end{enumerate}

\largerpage
 \textcolor{black}{\citet[194]{ZimmerlingKosta13} argue that the description of word order systems of clausal CLs “should base on syntactic constraints and be maximally independent from conjectures about restrictions imposed by allegedly purely phonetic or lexical properties of clitics”. In languages like BCS a class of clause-level CLs form ordered clusters which, following \citet{FranksKing00}, are defined as “contiguous strings of clitics arranged in a rigid order according to language-specific rules called ‘Clitic Templates’”. A cluster is understood as “a string of clitics that neither allows insertion of non-clitic elements nor permutation of clitics, when they are contiguous” \citep[181]{ZimmerlingKosta13}. Clusters are formed according to rules that are independent from other rules of ordering; in this sense, they arrange elements in an idiosyncratic order. The authors propose what we could call a \textsc{barrier-template theory}. They claim that this theory, introduced by \citet[287]{Zaliznjak93}, is the only approach which explains delayed placement of clusters and diaclisis by one and the same underlying mechanism. Basically, CLs in BCS have a fixed position in the clause, i.e. they attach to the clause-initial element (2P). The authors note that this, however, holds only for communicatively unmarked sentences, and thus they  integrate information structure into their model. There are two main deviations from this basic 2P order. First, the whole CL cluster can end up to the right of clausal 2P (this corresponds to what we have labelled delayed placement). Second, some clusterising CLs remain in clausal 2P, while other clusterising CLs end up to the right of it (we use the term diaclisis) \citep[196]{ZimmerlingKosta13}. The main hypothesis is that the sentence-initial phrase hosting the CLs may have properties of a barrier and move all or some clusterising CLs to the right of clausal 2P. The first option is referred to as a “blind” or “indiscriminating” barrier, the second option is referred to as a “selective” barrier \citep[196]{ZimmerlingKosta13}: “[\dots] in 2P languages sentence-initial Barriers are either blind and move all clusterizing clitics \textit{n} steps to the right of clausal 2P or selective and split the clusters by moving some clusterizing CLs n steps to the right of clausal 2P” \citep[197]{ZimmerlingKosta13}. Both blind and selective barriers can be optional or obligatory. Furthermore, the authors distinguish communicative and grammaticalised barriers. Communicative barriers are phrases that affect the position of CLs due to the communicative status they acquire in a given sentence \citep[198]{ZimmerlingKosta13}. }

\subsubsection{Barriers and delayed position of clitics}
\label{Barriers and delayed placement of clitics (DP)}
A second type of placement is when for some reason the initial phrase(s) is not selected as the host and the CL cluster attaches to a phrase further to the right in the sentence. This phenomenon is sometimes referred to as ``delayed clitic placement'' \citep{InkelasZec90}, ``clitic third'' \citep{CavarWilder94, Schutze94}, ``late placement of clusters'' \citep{ZimmerlingKosta13}, ``Endstellung'' \citep{Reinkowski01} or ``resumptive RSC'' (Rhythmic Structure Constituent) \citep{Alexander08, Alexander09}. As there are cases like in example (\ref{(2.13)}) where the CL attaches not to the second but even to the third phrase, we prefer the broader term \textsc{delayed position} (DP). 

\begin{exe}\ex\label{(2.13)}
\gll \minsp{[} Pod uvjetima iz stavka 1. ovoga članka]\textsubscript{phrase1} \minsp{[} pravna osoba]\textsubscript{phrase2} \minsp{[} kaznit]\textsubscript{phrase3} \textbf{će} \textbf{se} za kaznena djela propisana Kaznenim {zakonom [\dots].} \\
{} under conditions from paragraph 1 this article {} legal person {} punish\textsc{.inf} \textsc{fut.3sg} \textsc{refl} for criminal acts regulated\textsc{.pass.ptcp} criminal law  \\
\glt ‘Under the conditions from paragraph 1 of this article a legal entity will be punished for criminal acts prescribed by Criminal Law [\dots].’  \\
\hfill [hrWaC v2.2]
\end{exe}\clearpage


\noindent \textcolor{black}{According to \citet[439]{CavarWilder94} delayed placement appears only in embedded infinitives and in main, i.e. root clauses, and is not found in subordinate clauses.\footnote{Our examples in Chapters \ref{Clitics in dialects} and \ref{Clitics in a corpus of a spoken variety} do not corroborate this claim.}}

Without taking an a priori stance as to the structural or functional nature of the DP of CLs or as to whether we are dealing with exceptions to 2P, we use the term \textsc{barriers} as a descriptive label for the preceding phrases. There are two main divergences from the basic order:
\begin{enumerate}	
 \item delayed placement of clusters, 
 \item diaclisis. 
\end{enumerate}

 According to \citet[196]{ZimmerlingKosta13} in DP ``the whole clitic cluster ends up to the right of clausal 2P''. We leave open the question whether in BCS it is the barriers that move a CL \textit{n} steps to the right of the CL host or whether some other mechanism is in play.

As we show in Chapter \ref{Clitics and variation in grammaticography and related work}, normative grammar handbooks tend to argue that CLs cannot or should better not be placed after phrases separated by a comma \citep[cf.][132]{Reinkowski08}. For such instances \citet[435]{RadanovicKocic96}, who proposes a purely prosodic account of 2P, suggests the term ``heavy constituent''. \textcolor{black}{It is worth pointing out that \citet{RadanovicKocic96} does not provide a precise definition of the ``heavy constituent'' concept. A similar observation on DP can be found in \citet[264]{Boskovic95}, where it is claimed that when the constituents preceding a CL within a clause are heavy, the CL does not have to occur in the 2P of its clause. Following \citet{Schutze94}, \citet[264]{Boskovic95} adds that phonologically heavy constituents such as preposed PPs form separate intonational phrases and as such they are followed by an intonational phrase boundary. 
However, an inspection of the theoretical literature suggests that the situation is not so clear cut. \citet[373]{InkelasZec90} argue that the ``p-constituent is heavy iff it branches''. They elaborate on branching conditions and claim that branching at the syntactic constituent level is neither a sufficient nor a necessary condition for heaviness and delayed placement \citep[374f]{InkelasZec90}. This is exemplified with the help of the following two sentences:}

\begin{exe}\ex
\begin{xlist}
\ex[*]{\label{(28.06.1a)}
\gll Sa Petrom razgovarala je samo Marija.  \\
 with Peter talk.\textsc{ptcp.sg.f} be.\textsc{3sg} only Mary \\
\glt Intended: ‘To Peter, only Mary spoke.’ }
\ex[]{\label{(28.06.1b)}
\gll Sa tim čovekom razgovarala je samo Marija.  \\
 with that man talk.\textsc{ptcp.sg.f} be.\textsc{3sg} only Mary \\
 \glt ‘To that man, only Mary spoke.’\hfill \citep[374]{InkelasZec90}\hbox{}}
\end{xlist}
\end{exe}

\noindent
Although the first constituent \textit{sa Petrom} branches at the syntactic constituent level, it does not branch at the prosodic constituent level. Therefore, according to \citet[374]{InkelasZec90} the example in (\ref{(28.06.1a)}) is ill-formed. In contrast, the first constituent in (\ref{(28.06.1b)}), \textit{sa tim čovekom}, branches not only at the syntactic constituent level, but also at the prosodic constituent level, and therefore DP of the CL \textit{je} does not result in an ill-formed sentence \citep[374]{InkelasZec90}.

We would like to point out two facts regarding DP and heavy constituents. First, the initial phrases (or constituents) involved in DP are not necessarily ``heavy'' in a phonological sense of containing a large number of phonemes: compare our example (\ref{(2.14)}) in which the initial phrase \textit{o\-va\-kva} \textit{vrsta} \textit{pretrage} `this kind of search' containing 19 phonemes does not host the verbal CL \textit{će}, with the example presented in (\ref{(2.15)}) in which the initial prepositional phrase \textit{do zime} `by winter' containing only six phonemes does not host the reflexive CL \textit{se}.  

\begin{exe}\ex\label{(2.14)}
\gll \minsp{[} Ovakva vrsta pretrage] \minsp{[} bit] \textbf{će} dostupna za čitav {HNK [\dots].} \\
{} this kind search {} be\textsc{.inf} \textsc{fut.3sg} available for entire HNK\\
\glt ‘This kind of search will be available for the entire HNK [\dots].’\\
\hfill [hrWaC v2.2]

\ex\label{(2.15)}
\gll \minsp{[} Do zime] \minsp{[} planira] \textbf{se} završiti asfaltiranje građevine [\dots].\\
{} by winter {} plan\textsc{.3prs} \textsc{refl} finish\textsc{.inf} paving building \\
\glt ‘It is planned that paving the building will be finished by winter [\dots].’\\
\hfill [hrWaC v2.2]\\
\end{exe}

\noindent Second, the example with DP provided in (\ref{(2.15)}) would not be well-formed if the heavy constituent concept were understood like in \citet[374f]{InkelasZec90}: compare their example in (\ref{(28.06.1a)}) and our example in (\ref{(2.15)}). Since, as we demonstrated, neither Bošković's (\citeyear[264]{Boskovic95}) nor \citeauthor{InkelasZec90}' (\citeyear[373ff]{InkelasZec90}) measure of heaviness seems to be applicable to the language data which can be observed in corpora, we turn to the measure of heaviness expressed by the number of graphemes, as proposed by \citet{KCN18}. 

The approach taken by \citet{KCN18} is in accordance with information found in \citet[90--93,  as well as in references therein]{Stefanowitsch20} on measuring the weight/length of linguistic units (syllables, words or phrases) in corpus studies. Operationalising (word) length for measurement purposes poses difficulties in itself \citep[90]{Stefanowitsch20}. A number of solutions can be found in the literature, e.g. number of letters \citep[cf.][]{Wulff03}, number of phonemes \citep[cf.][]{Sobkowiak93} and number of syllables \citep[cf.][]{Sobkowiak93, Stefanowitsch03}.\footnote{Applicability in the domain of corpus linguistics limits the options for length operationalisation named here. However, other definitions do exist, such as phonetic length or mean phonetic length \citep[90f]{Stefanowitsch20}.}

For the application of the measure proposed by \citet{KCN18}  see our empirical study based on a corpus of spoken language in Chapter \ref{Clitics in a corpus of a spoken variety}.


\subsubsection{Clitic first}
\label{Clitic first (1P)}
\citet[225--234]{FranksKing00} discuss another type that departs from strict 2P, in which the requirement that an element precede the CL seems to be violated. They adduce cases where a form which usually lacks stress and attaches to the preceding element shows up in sentence-initial position (\textsc{clitic} \textsc{first}, 1P), like in example (\ref{(2.16)}) presented below.

\begin{exe}\ex\label{(2.16)}
\gll \textbf{Su} bíli u célo  sèlo.\\
 be\textsc{.3pl} be\textsc{.ptcp.pl.m} in entire village\\
\glt ‘They were in the entire village.’ 
\hfill (Sr; \citealt[148]{Okuka08})
\end{exe}

\noindent We will deal with cases of 1P in the chapters on variation in the standard languages (Chapter \ref{Clitics and variation in grammaticography and related work}), in the dialects (Chapter \ref{Clitics in dialects}), and in spoken language (Chapter \ref{Clitics in a corpus of a spoken variety}).

\subsubsection{Phrase splitting}
\label{Phrase splitting (2W)}
BCS differs typologically for example from Modern Czech, as the CL can attach not only to the first phrase but also to the first word of a phrase, allowing for example a noun phrase to be split (as in modified example (\ref{(2.17b)})):

\begin{exe}\ex
\begin{xlist}
\ex\label{(2.17a)}
\gll \minsp{[} Običnim ljudima] \textbf{je} dosta rata i žele živjeti u miru.  \\
{} ordinary people be\textsc{.3sg} enough war and want\textsc{.3prs} live\textsc{.inf} in peace \\
\ex\label{(2.17b)}
\gll \minsp{[} Običnim \textbf{je} ljudima] dosta rata i žele živjeti u miru.  \\
{} ordinary be\textsc{.3sg} people enough war and want\textsc{.3prs} live\textsc{.inf} in peace \\
\end{xlist}
\glt ‘Ordinary people have had enough of war and want to live in peace.’ \\
\strut\hfill [hrWaC v2.2]
\end{exe}

\noindent It is worth pointing out that splitting is independent of 2P or DP. In permuted example (\ref{(2.17b)}) above it is the first phrase which is split, while in example (\ref{(2.18)}) below it is the second phrase:

\protectedex{
\begin{exe}\ex\label{(2.18)}
\gll \minsp{[} Prema izračunu]\textsubscript{phrase1}, \minsp{[} prosječan \textbf{je} broj sunčanih sati]\textsubscript{phrase2} godišnje čak 2500.  \\
{} according.to calculations {} average be\textsc{.3sg} number sunny hours yearly even 2500  \\
\glt ‘According to calculations, the average number of sunny hours yearly is as high as 2500.’
\hfill [hrWaC v2.2]
\end{exe}
}

\noindent Splitting occurs for adverb phrases, adjective phrases, noun phrases and prepositional phrases. We will discuss the possibility of phrase splitting based on the existing research literature in Chapters \ref{Clitics and variation in grammaticography and related work} and \ref{Clitics in dialects}, and based on our empirical study, in Chapter \ref{Clitics in a corpus of a spoken variety}. 

\subsection{Clitic climbing}
\label{Clitic climbing}
The main focus of our empirical studies on microvariation is on the 2P ordering rule for which the term \textsc{clitic climbing} (CC) was established. CC occurs in sentences consisting of a matrix clause and an embedded verbal complement. Descriptively speaking, CC refers to a phenomenon whereby a CL that depends on the embedded complement appears in the matrix clause (see discussion in Chapter \ref{Approaches to clitic climbing}). Note that throughout this monograph we stick to the established term \textit{climbing} even though we do not necessarily assume any movement operations. In example (\ref{(2.19b)}) the pronominal CL \textit{ga} `him', which fills an argument position of the infinitival verb \textit{vidjeti} `to see', is realised in the second position of the matrix clause. In some theories, it is assumed that the CL ``climbs'' from the verbal complement into the matrix clause.\footnote{More information on verbal complements in BCS can be found in Section \ref{Types of complements} below.} Throughout this book we annotate the relationship between CL and the governing predicate with small subscript numbers: in example (\ref{(2.19a)}) below both infinitive and pronominal CLs are annotated with subscript number 2, which means that the infinitive \textit{vidjeti} generated the pronominal CL \textit{ga}. A matrix predicate is always annotated with subscript number 1 and every further verbal complement (infinitive or \textit{da}\textsubscript{2}-complement), with the next number.\footnote{For more information on \textit{da}-complements and the distinction between \textit{da}\textsubscript{1} and \textit{da}\textsubscript{2} see the next section.}

\begin{exe}\ex
\begin{xlist}
\ex[*]{\label{(2.19a)}
\gll Milan mora\textsubscript{1} / želi\textsubscript{1} vidjeti\textsubscript{2} \textbf{ga}\textsubscript{2}.\\
 Milan must {} want\textsc{.3sg} see\textsc{.inf}  him\textsc{.acc} \\}
\ex[]{\label{(2.19b)}
\gll Milan \textbf{ga}\textsubscript{2} mora\textsubscript{1} / želi\textsubscript{1} vidjeti\textsubscript{2}.\\
 Milan him\textsc{.acc} must {} want\textsc{.3sg} see\textsc{.inf}\\}
\end{xlist}
\glt ‘Milan must/wants to see him.’
\hfill (BCS; \citealt[179f]{Stjepanovic04})
\end{exe}

\noindent An important question is how to detect CC. A clear case of CC is when the CL stands to the left of the matrix predicate (like \textit{ga} before \textit{mora} in example (\ref{(2.19b)}) above). \citet[67]{Junghanns02}, however, warns that if we have the surface word order $\text{matrix predicate} + \text{CL} + \text{infinitive}$ (like in example (\ref{(2.20)}) below) where the CL \textit{ga} `him' occurs directly before the infinitive \textit{upoznati} `to get to know', it cannot be ruled out that the CL is still in the complement. 

\protectedex{
\begin{exe}\ex\label{(2.20)}
\gll Moram\textsubscript{1} \textbf{ga}\textsubscript{2} upoznati\textsubscript{2}. \\
 must\textsc{.1prs} him\textsc{.acc} get.to.know\textsc{.inf} \\
\glt ‘I have to get to know him.’ 
 \hfill [hrWaC v2.2]
\end{exe}
}

\noindent CC is a central source of both sociolinguistic and systemic variation in CL usage. Among others, it involves cases where CLs show up in two clusters. This happens e.g. in constructions with stacked infinitives where the CL could climb but for some reason stays in situ, leading to a split of the CLs between two positions, like in example (\ref{(2.21)}) presented below (see Section \ref{Diaclisis and pseudodiaclisis} below). 

\protectedex{
\begin{exe}\ex\label{(2.21)}
\gll [\dots] mogao\textsubscript{1} \textbf{je}\textsubscript{1} pokušati\textsubscript{2} spasiti\textsubscript{3} \textbf{nas}\textsubscript{3} od  navale hohštaplera.  \\
 {} can\textsc{.ptcp.sg.m} be\textsc{.3sg} try\textsc{.inf} save\textsc{.inf} us\textsc{.acc} from attack conman  \\
\glt ‘[\dots] he could have tried to save us from the attack of the conmen.’  \\
\hfill [hrWaC v2.2]
\end{exe}
}

\noindent Although selected examples of CC have generated controversy in the theoretical literature, hitherto the rules and especially the constraints on CC have not been adequately described. Most tellingly, there is not even an established linguistic term for CC in Serbian or Croatian. \citet{HRK13} proposed the ad hoc translation \textit{uspon zanaglasnica} which, however, has not (yet) gained ground in Croatian linguistics. It is no exaggeration to say that the range of microvariation and the possible constraints on CC are a seriously understudied field of BCS syntax. Therefore, we will give a detailed and empirically valid account of this phenomenon in Part \ref{part3}.

\subsection{Diaclisis and pseudodiaclisis}
\label{Diaclisis and pseudodiaclisis}
As mentioned above, in certain contexts one CL can be in clausal 2P (or in DP), while an additional clusterising CL is placed to its right \citep[cf.][196]{ZimmerlingKosta13}. As this phenomenon has not been discussed \textcolor{black}{extensively} in the literature on BCS, we use the cover term \textsc{diaclisis} which we borrowed from Greek linguistics.\footnote{The term is used e.g. by \citet[270]{Janse98} in work on CLs in Cappadocian Greek. In order to avoid confusion with phrase splitting we do not use the term ``splitting'' as proposed by \citet[196]{ZimmerlingKosta13}.} We use the term for two different types: one for true inner clause diaclisis:

\protectedex{
\begin{exe}\ex\label{(2.22)}
\gll [\dots] po gradovima \textbf{su}\textsubscript{1} predsednici opština \textbf{se}\textsubscript{1} odjednom {opredeljivali\textsubscript{1} [\dots].}  \\
 {} in cities be\textsc{.3pl} presidents counties \textsc{refl} suddenly decide  \\
\glt ‘In the cities, the county presidents were suddenly deciding [\dots].’  \\
\hfill [Bosnian Interviews, BH]
\end{exe}
}

\noindent and one for diaclisis happening in the context of the matrix predicate and its verbal complement(s), as in example (\ref{(2.21)}) above. The latter case is labelled \textsc{pseudodiaclisis}. If the difference between the two types is not relevant, we use diaclisis as a cover term for the sake of brevity. We discuss this phenomenon in more detail in Chapter \ref{Clitics in a corpus of a spoken variety} and in Chapters \ref{A corpus-based study on CC in da constructions and the raising-control distinction (Serbian)}--\ref{Experimental study on constraints on clitic climbing out of infinitive complements}.

\section{Syntactic categories relevant for the description of microvariation}
\label{Syntactic categories relevant for the description of microvariation}
\subsection{Complement-taking predicates}
\label{Complement-taking predicates}
As mentioned above, CC occurs in constructions involving a matrix clause that embeds a second verbal element. As there is no agreement as to the status of the embedded element (clause or non-clause – see discussion in Chapter \ref{Approaches to clitic climbing}), we would like to avoid the term ``clause-embedding predicate'' proposed in \citepossst{Stiebels15} work on control predicates. Instead, we prefer the well-established and more general term \textsc{complement taking predicate} (CTP) used in the prominent typological work on complementation by \citet{Noonan85}.\footnote{``By complementation we mean the syntactic situation that arises when a notional sentence or predication is an argument of a predicate'' \citep[42]{Noonan85}.} CTP is a more suitable term than ``clause-embedding predicate'' because it covers both control and raising predicates and leaves open the question whether the embedded predicate has clausal status or not.\footnote{This is a correction of our terminology used in \cite*{JHK17b}.} A second feature which needs clarification concerns the relationship between the matrix and the embedded verbal predicate. We assume that CC is possible only in the case of complements and not of adjuncts. This means that we also treat the embedded structural element of verbs of motion as semantically obligatory complements and not as a final clause which is usually treated as an adjunct: see CC in the following example (\ref{(2.23)}) where the verb \textit{doći} `to come' is complemented by the infinitive phrase \textit{očistiti peć} `to clean the oven':

\protectedex{
\begin{exe}\ex\label{(2.23)}
\gll [\dots] jer \textbf{mu}\textsubscript{3} \textbf{je}\textsubscript{1} u subotu trebala\textsubscript{1} \textbf{doći}\textsubscript{2} \textbf{očistiti}\textsubscript{3} peć.  \\
 {} because him\textsc{.dat} be\textsc{.3sg} in Saturday have.to\textsc{.ptcp.sg.f.} come\textsc{.inf} clean\textsc{.inf} oven  \\
\glt ‘[\dots] because she was supposed to come on Saturday to clean his oven.’ \\
\hfill [bsWaC v1.2]
\end{exe}
}

\noindent Throughout the monograph we use the terms CTP and matrix interchangeably.



\subsection{The control vs raising distinction}
\label{The control vs raising distinction}
In our study on CC, we especially focus on the dichotomy between control and raising CTPs. Due to lack of space, we confine ourselves to some basic empirical observations discussed in various theoretical frameworks dealing with control and raising. Many syntactic theories draw a systemic distinction between \textsc{raising} and \textsc{control}. In HPSG- and Construction Grammar-related frameworks, the raising--control distinction is understood as a sort of mismatch between different levels of representation; for example, \citet[34]{PrzepiorkowskiRosen05} give a very concise characterisation of this dichotomy based on the idea of structure sharing (exemplified by the English verbs \textit{seem} and \textit{try}): 

\begin{quotation}
(i) semantically, raising verbs have one argument fewer than the corresponding control verbs, e.g. \textit{seem} is a (semantically) 1-argument verb, while \textit{try} is a (semantically) 2-argument verb; (ii) structurally, the raised argument and the subject of the infinitival verb are the same element (so-called structure sharing; [\dots]), while the controller and the subject of the infinitival verb are two different elements.
\end{quotation}

Accordingly, in raising constructions (with \textit{seem}) the subject does not receive its semantic role directly from the matrix predicate but from the embedded predicate. In a control construction (with \textit{try}), in contrast, the matrix verb and the embedded verb each assign a subject role \citep[64f]{FriedOstman04}. In Principles and Parameters accounts, control constructions are characterised by the presence of two syntactic arguments: a surface subject and a non-overt infinitival subject called big PRO \citep[600]{Wurmbrand99}. Control always involves a relationship of obligatory (full or partial) co-reference between the non-overt first argument of the complement predicate (\textsc{controllee}) and one of the arguments of the matrix predicate (\textsc{controller}). In the following example, the first argument of the verb in the complement is interpreted as co-referential with the subject of the matrix clause (marked with \textsubscript{X}):

\protectedex{
\begin{exe}\ex\label{(2.24)}
\gll [\dots] Dalibor\textsubscript{X} \textbf{im}\textsubscript{Y} \textbf{je} obećao\textsubscript{X}  pomoći.\\
 {} Dalibor them\textsc{.dat} be\textsc{.3sg} promise\textsc{.ptcp.sg.m}    help\textsc{.inf}\\
\glt ‘[\dots] Dalibor promised to help them.’ 
\hfill [hrWaC v2.2]
\end{exe}
}

\noindent \citet[4--8]{DaviesDubinsky04} list relatively robust, cross-linguistically applicable tests proposed in the literature in order to distinguish raising from control constructions:

\begin{enumerate}
\item In raising constructions the subject argument of the matrix predicate has the same semantic role as the subject argument of the embedded predicate. In example (\ref{(2.25)}), the subject argument \textit{poslodavac} `employer' receives its semantic role of agent from \textit{poništiti} `repeal' (raising) whereas in example (\ref{(2.26)}) the subject \textit{operater} `operator' receives it from the matrix predicate \textit{pokušao} `tried' (control).

\begin{exe}
\ex Raising
\begin{xlist}
\ex{\label{(2.25)}
\gll Poslodavac može poništiti  {rješenje [\dots].}\\
 employer can\textsc{.3prs} repeal\textsc{.inf}  settlement \\
\glt ‘The employer can repeal the settlement [\dots].’}
\ex{\label{(2.25+)}
\gll Rješenje može biti poništeno. \\
 settlement can\textsc{.3prs} be\textsc{.inf} repeal\textsc{.pass.ptcp} \\
\glt ‘The settlement can be repealed (by the employer).’}
\end{xlist}\hfill [hrWaC v2.2]

\newpage
\ex Control
\begin{xlist}
\ex[]{\label{(2.26)}
\gll Operater \textbf{je} pokušao ručno  obustaviti {reaktor [\dots].}\\
 operator be\textsc{.3sg} try\textsc{.ptcp.sg.m} manually stop\textsc{.inf} reactor\\
\glt ‘The operator manually tried to stop the reactor [\dots].’ }
\ex[*]{\label{(2.26+)}
\gll Reaktor je pokušao ručno biti obustavljen.\\
 reactor be\textsc{.3sg} try\textsc{.ptcp.sg.m} manually be\textsc{.inf} stop\textsc{.pass.ptcp}\\
\glt Intended: ‘An attempt was made to manually stop the reactor (by the operator).’ 
\hfill [hrWaC v2.2]}
\end{xlist}
\end{exe}

\item In subject control constructions the subject argument of the raising predicate does not show any selectional restrictions. For example, the raising verb \textit{trebati} `have to' does not determine the selectional restriction $+/-$ human: compare examples presented in (\ref{(2.27)}) and (\ref{(2.28)}):

\begin{exe}\ex\label{(2.27)}
\gll A  oni\textsubscript{human$+$} trebaju platiti za ono što su napravili.  \\
 and they need\textsc{.3prs} pay\textsc{.inf} for that what be\textsc{.3pl} do\textsc{.ptcp.pl.m}  \\
\glt ‘And they have to pay for what they did.’ 
\hfill [hrWaC v2.2]

\ex\label{(2.28)}
\gll \minsp{[} Idejna rješenja]\textsubscript{human$-$} trebaju biti poslana u JPEG i PDF obliku na sljedeću e-mail adresu [\dots].  \\
 {} idea solutions need\textsc{.3prs} be\textsc{.inf} sent in JPEG and PDF format on following e-mail address  \\
\glt ‘Ideas for a solution should be sent in JPEG and PDF format to the following e-mail address [\dots].’ 
\hfill [hrWaC v2.2]
\end{exe}

\item In raising contructions passivisation does not change the propositional meaning of the sentences as shown in the permutations of sentences (\ref{(2.25+)}) and (\ref{(2.26+)}).
\end{enumerate}

\begin{sloppypar}
A distinction is made between subject and object control constructions depending on the argument selected as controller (first or second argument). Whereas predicates that have only one individual argument besides the predicative (verbal) argument are always subject control predicates, polyvalent predicates may show either a subject or an object control reading.\footnote{We do not want to discuss the special cases of partial, split or switch control. For a more detailed account of control see \citet{Stiebels07, Stiebels15}, \citet{Landau00}, \citet{Moskovljevic07}, and \citet{Slodowicz08}.} According to \citet[422]{Stiebels15}, verbs denoting commissive speech acts (e.g. \textit{obećati} `promise') are typical \textsc{subject} \textsc{control} predicates, whereas predicates which refer to directive speech acts (e.g. \textit{zamoliti} `request') or which have a causative component belong to the canonical class of \textsc{object} \textsc{control} predicates, exemplified here in (\ref{(2.29)}) and (\ref{(2.30)}) by sentences with the so-called \textit{da}-construction:
\end{sloppypar}

\begin{exe}\ex\label{(2.29)}
\gll On\textsubscript{X} \textbf{mi}\textsubscript{Y} \textbf{je} obećao da \textbf{će}\textsubscript{X} \textbf{se} vratiti u {Kragujevac [\dots].} \\
 he me\textsc{.dat} be\textsc{.3sg} promise\textsc{.ptcp.sg.m} that \textsc{fut.3sg} \textsc{refl} return\textsc{.inf} in Kragujevac \\
\glt ‘He promised me to come back to Kragujevac, [\dots].’ 
\hfill [srWaC v1.2]
\ex\label{(2.30)}
\gll Dekan\textsubscript{X} \textbf{je} sve prisutne\textsubscript{Y} zamolio da kažu\textsubscript{Y} svoje {utiske [\dots].}  \\
 dean be\textsc{.3sg} all present ask\textsc{.ptcp.sg.m} that say\textsc{.3prs} own impressions  \\
\glt ‘The Dean kindly asked the attending members to share their impressions [\dots].’       
\hfill [srWaC v1.2]
\end{exe}

\noindent The raising--control distinction as outlined above is orthogonal to the distinction of matrix verbs proposed by \citet{Progovac93} and applied by \citet{Todorovic15}. In the following we explain why we do not use this classification, although it has been developed and used by scholars dealing with BCS CLs. 

\citet[116]{Progovac93} distinguishes two basic groups of verbs: those which select opaque complements (I-verbs, or indicative-selecting verbs) and those which select transparent complements, allowing for domain extension (S-verbs, which select subjunctive-like complements).\footnote{\textcolor{black}{A similar distinction is applied by \citet{Landau04} to Balkan languages and Hebrew.}} I-verbs are mostly verbs of saying, believing and ordering, such as \textit{kazati} (`tell'), \textit{v(j)erovati} (`believe') or \textit{narediti} (`order'). S-verbs are mainly verbs of wishing and requesting, such as \textit{žel(j)eti} (`want/wish'), \textit{ht(j)eti} (`want/will'), \textit{moći} (`be able to') and \textit{tražiti} (`ask for'). 

According to \citet[116]{Progovac93}, “the following local dependencies in Serbo-Croatian are clause bound with I-verbs, but can cross clause boundaries with S-verbs: licensing of negative polarity items (NPIs), clitic climbing, and topic preposing”. 

We would like to point out that the distinction between S- and I-verbs might not be as clear as it appears at first sight, i.e. as presented by \citet{Progovac93}. First, these semantic verb classes are quite heterogeneous: verbs of ordering, like the mentioned \textit{narediti} `order', in fact select subjunctive-like complements which do not allow past or future tense. 

Second, there are cases where the dependency relation seems to go in the opposite direction. That is, it seems that a complement can change the class of a verb. For instance, if verbs of saying co-occur with subjunctive-like complements, semantic coercion occurs. A verb of saying is interpreted as a verb of ordering, as in the following example:

\begin{exe}\ex\label{(19.05.0)}
\gll Rekao \textbf{sam} \textbf{im} da budu oprezni, objasnio tko \textbf{sam}, što \textbf{sam}.\\
say.\textsc{ptcp.sg.m} be.\textsc{1sg} them.\textsc{dat} that be.\textsc{3pl} careful explain.\textsc{ptcp.sg.m}  who be.\textsc{1sg} what be.\textsc{1sg}\\
\glt ‘I told them they should be careful and explained who I am, what I am.’  \\
\hfill [hrWaC v2.2]
\end{exe}


\noindent Even though the I- and S-classification of verbs seems too simplistic, a claim in \citet[119]{Progovac93} and \citet[146]{Progovac05} has particular relevance to our study: that S-verbs allow CC, whereas I-verbs do not. Progovac discusses the following minimal pairs of sentences:

\begin{exe}
\ex\begin{xlist}
\ex[]{\label{(19.05.1a)}
\gll Milan kaže\textsubscript{1} da \textbf{ga}\textsubscript{2} vidi\textsubscript{2}. \\
Milan say.\textsc{3prs} that him.\textsc{acc} see.\textsc{3prs} \\\hfill I-Verb
\glt ‘Milan says that he sees him.’}
\ex[*]{\label{(19.05.1b)}
\gll Milan \textbf{ga}\textsubscript{2} kaže\textsubscript{1} da vidi\textsubscript{2}.\\
 Milan him.\textsc{acc} say.\textsc{3prs} that see.\textsc{3prs}\\
\glt Intended: ‘Milan says that he sees him.’
\hfill (BCS; \citealt[119]{Progovac93})}
\end{xlist}

\ex
\begin{xlist}
\ex[]{\label{(19.05.2a)}
\gll Milan želi\textsubscript{1} da \textbf{ga}\textsubscript{2} vidi\textsubscript{2}.  \\
Milan want.\textsc{3prs} that him.\textsc{acc} see.\textsc{3prs} \\\hfill S-Verb}
\ex[?]{\label{(19.05.2b)}
\gll Milan \textbf{ga}\textsubscript{2} želi\textsubscript{1} da vidi\textsubscript{2}.\\
 Milan him.\textsc{acc} want.\textsc{3prs} that see.\textsc{3prs}\\
\glt ‘Milan wants to see him.’
\hfill (BCS; \citealt[119]{Progovac93})}
\end{xlist}
\end{exe}


\noindent However, as we show in Chapter \ref{A corpus-based study on CC in da constructions and the raising-control distinction (Serbian)}, the claim that “with S-verbs clitics originating in the embedded clause can optionally climb to the second position of the matrix clause” \citep[119]{Progovac93}, which concerns structures such as those in (\ref{(19.05.2b)}), is somewhat problematic. \citet[119]{Progovac93} herself is unsure whether the sentence is grammatically correct, since she uses the question mark. Moreover, in a footnote in her later publication \citep[146]{Progovac05}, she admits that some speakers of Serbian, including the linguist Vesna Radanović-Kocić, do not accept CC. In Chapter \ref{A corpus-based study on CC in da constructions and the raising-control distinction (Serbian)} we give an empirical answer to the question of the extent to which the structure presented in example (\ref{(19.05.2b)}) is possible, i.e. used by native speakers of Serbian. Finally, we would like to emphasise that a more fine-grained subclassification of S-verbs, as offered by the raising--control distinction, is called for.

\subsection{Types of complements}
\label{Types of complements}
As mentioned above, CC involves structures with a matrix and an embedded complement. In BCS, the latter can be encoded either by a phrase with an infinitive (as in (\ref{(2.31a)})) or by a phrase introduced by the element \textit{da} (as in (\ref{(2.31b)})) sometimes treated as a complementiser; see the examples from \citet[174]{Stjepanovic04}.\footnote{Note that in the glossing of our examples we do not account for the polyfunctionality of the glossed morpheme \textcolor{black}{\textit{da} and that we simply gloss it lexically as ‘that’}.} 


\begin{exe}\ex
\begin{xlist}
\ex\label{(2.31a)}
\gll Marija \textbf{ga}\textsubscript{2} mora\textsubscript{1} / želi\textsubscript{1} posjetiti\textsubscript{2}.\\
 Marija him\textsc{.acc} must {} want\textsc{.3prs} visit\textsc{.inf} \\
\ex\label{(2.31b)}
\gll Marija \textbf{ga}\textsubscript{2} mora\textsubscript{1} / želi\textsubscript{1} da posjeti\textsubscript{2}.\\
 Marija him\textsc{.acc} must {} want\textsc{.3prs} that visit\textsc{.3prs} \\
\end{xlist}
\glt ‘Marija must/wants to visit him.’ 
\hfill (BCS; \citealt[174]{Stjepanovic04}) 
\end{exe}

\noindent It has long been known that \textit{da}-complements do not behave in a uniform way. \citet{Ivic70} proposes to distinguish two complement types headed by \textit{da} depending on tense marking: complements with ``mobile present tense'' and complements with ``immobile present tense'', the former being regularly marked for tense and the latter not. This distinction goes back to \citet{Golab64} and was further elaborated on by \citet[and earlier]{Browne03} who uses the labels \textsc{\textit{da}\textsubscript{1}-} and \textsc{\textit{da}\textsubscript{2}-complement}. Here is an example from \citet[39]{Browne03} with the CTP \textit{saznati} `find out', which allows present (\ref{(2.32a)}), past (\ref{(2.32b)}) or future tense marking (\ref{(2.32c)}).

\begin{exe}\ex
\begin{xlist}
\ex\label{(2.32a)}
\gll Saznao \textbf{sam} da crtaš zmiju. \\
 find.out\textsc{.ptcp.sg.m} be\textsc{.1sg} that draw\textsc{.2prs} snake\\
 \glt ‘I found out that you were drawing a snake.’\\
\ex\label{(2.32b)}
\gll Saznao \textbf{sam} da \textbf{ste} crtali zmiju.\\
  find.out\textsc{.ptcp.sg.m} be\textsc{.1sg} that be\textsc{.2pl} draw\textsc{.ptcp.pl.m} snake\\
  \glt ‘I found out that you had been drawing a snake.’\\
\ex\label{(2.32c)}
\gll Saznao \textbf{sam} da \textbf{ćeš} crtati zmiju.\\
  find.out\textsc{.ptcp.sg.m} be\textsc{.1sg} that fut\textsc{.2sg} draw\textsc{.inf} snake\\
  \glt ‘I found out that you would draw a snake.’ 
\hfill (BCS; \citealt[39]{Browne03})
\end{xlist}
\end{exe}

\noindent In contrast, \textit{da}\textsubscript{2}-complements only allow the verbal form coinciding with the present tense; other tenses are impossible. \citet{Ivic72} speaks about the ``immobility'' of the present tense (\textit{nemobilnost prezenta}) whereas \citet[119]{Dukanovic94} assumes that tense marking is blocked (\textit{vremenska umrtvljenost}, \textit{neovremenjenost}). It is claimed that \textit{da}\textsubscript{2}-complements occur with CTPs with volitional meaning, e.g. with the verb \textit{žel(j)eti} `wish/want' in the following example (\ref{(2.33a)}) and in its two permutations.

\begin{exe}\ex
\begin{xlist}
\ex[]{\label{(2.33a)}
\gll Želim da\textsubscript{2} crtaš zmiju. \\
 want\textsc{.1prs} that draw\textsc{.2prs} snake \\
 \glt ‘I want you to draw a snake.’\\}
\ex[*]{\label{(2.33b)}
\gll Želim da\textsubscript{2} \textbf{si} crtala zmiju. \\
 want\textsc{.1prs} that be\textsc{.2sg} draw\textsc{.ptcp.sg.f} snake \\
\glt Intended: ‘I want you to have drawn a snake.’ \\}
\ex[*]{\label{(2.33c)}
\gll Želim da\textsubscript{2} \textbf{ćeš} crtati zmiju. \\
 want\textsc{.1prs} that \textsc{fut}\textsc{.2sg} draw\textsc{.inf} snake \\
\glt Intended: ‘I want you to draw a snake in the future.’ \\}
\strut\hfill (BCS; \citealt[39]{Browne03})
\end{xlist}
\end{exe}

\noindent \citet{Todorovic15} proposes to make a distinction between indicative and subjunctive complements. As we do not want to discuss the link between the two complement types and any semantic (i.e. modal) features, we stick to the terms \textit{da}\textsubscript{1}- vs \textit{da}\textsubscript{2}-complement. For the relationship between the semantics of the CTP and the selection of the \textit{da}-complement we refer to the in-depth empirical study by \citet*{HWK18}, who show that the semantics of the CTP does not directly determine the selection of the complement type (contra \citealt{Todorovic15}).

\subsection{Different types of reflexives}
\label{Different types of reflexives}
Veering away from problems directly relating to the inventory of CLs, we would like to discuss a wider issue concerning the \textsc{reflexive} CL \textit{se}. Like other Slavic languages, BCS displays a wide range of usages of the reflexive marker and we assume that these may differ in their syntactic behaviour, e.g. with respect to CC. In example (\ref{(2.34)}) \textit{se} has a different status than in example (\ref{(2.35)}) because in the first case its function is to hide the first argument (impersonal use), while in the second it is used to indicate reciprocity. 

\begin{exe}\ex\label{(2.34)}
\gll Reklo \textbf{bi} \textbf{se} da \textbf{je} to srebrn upaljač.\\
 say\textsc{.ptcp.sg.n} \textsc{cond.3sg} \textsc{refl} that be\textsc{.3sg} that silver lighter\\
\glt ‘It seemed to be a silver cigarette lighter.’ (=One would say [\dots]) \\
\hfill (Cr; \citealt[][106]{Moulton15})

\ex\label{(2.35)}
\gll Hrvati i mi ne moramo da \textbf{se} volimo, ali moramo da \textbf{se} poštujemo. \\
 Croats and we \textsc{neg} must\textsc{.1prs}  that  \textsc{refl} love\textsc{.1prs} but must\textsc{.1prs} that  \textsc{refl} respect\textsc{.1prs} \\
\glt ‘Croats and we do not need to love each other, but to respect each other.’ \\
\hfill (Sr; \citealt[][103]{Moulton15})
\end{exe}

\noindent As it is not our aim to either give an exhaustive overview of the existing research literature or to develop our own theoretical account of the different types of constructions, we restrict ourselves to the identification of some syntactic types of constructions with the element \textit{se} which may differ as to CC. As a matter of fact, there is a considerable body of research dealing with reflexives in the Slavic languages in general and in BCS in particular. The topic has attracted the attention of different scholars working in both formal and cognitive-functional frameworks. For our typology we draw on the recent study of reflexives from a broader Slavistic perspective, \citet{FJL10}. Among the studies specifically dealing with reflexives in BCS, we turn to the cognitivist work by \citet{Moulton15}.

\subsubsection{The approach of Fehrmann, Junghanns, and Lenertová (2010)}
\label{The approach of Fehrmann, Junghanns and Lenertová (2010)}
\largerpage[2]

In this section, we will discuss the reflexive markers based on the first two steps of our triangulation of empirical methods outlined in Chapter \ref{Empirical approach to clitics in BCS} (intuition/theory – observation – experiment). We start with data from the literature – in this case data from \citet{FJL10} – and verify them by searching for qualitative empirical data in corpora in the sense of a corpus-illustrated approach. Keeping  our research question in view, we will focus on evidence for microvariation in the use of reflexive markers.

As our study deals with variation in CL positioning and not with different semantic or structural types of the reflexive marker, we will restrict ourselves to testing a small number of types for differences especially in relation to CC. For the purposes of our study, it suffices to identify several types of reflexive constructions. \citet{Moulton15} distinguishes six semantic types which partially overlap with the list of surface configurations of reflexives in ten Slavic languages discussed by \citet{FJL10}.\footnote{\citet{Moulton15} distinguishes reflexive verbs, possessive reflexive verbs, reciprocal verbs, passive constructions, impersonal constructions, and middle verbs.} They present a unified account of two different lexical reflexive markers \textsc{refl1/refl2} based on the framework of a two-level semantics.\footnote{This framework distinguishes Semantic Form and Conceptual Structure, where the former mediates between the latter and the syntax (originally going back to \citealt{Bierwisch86}).} The authors exclude ``the relatively small group of reflexive verbs that synchronically have no non-reflexive counterparts''. These are stored in the lexicon as a unit (verbs like \textit{smijati se} `laugh'). 

\citet{FJL10} analyse seven descriptive surface types which differ among others as to the argument affected (first vs second), the argument blocking vs argument binding distinction and the presence of additional semantic features.\footnote{\textcolor{black}{\citet{FJL10} do not use the terms ``first'' and ``second argument'', they} use the terms ``external'' and ``internal argument'' \textcolor{black}{instead}. The distinction first vs second argument is found among others in Role and Reference Grammar.} Based on the possibility or exclusion of so-called by-phrases, they argue that two reflexives ``are necessary, but also sufficient, for the analysis of all [\dots] uses, regardless of whether an external or an internal argument is affected'' \citep[206]{FJL10}. The term by-phrase is used as a label covering very different surface manifestations including prepositional phrases, dative phrases and others. The main idea is that a \textsc{refl} affects one of the arguments of the verbal predicate preventing the canonical realisation of this argument as subject or object \citep[208]{FJL10}. Put simply, if a semantic specification of the affected argument is possible (via a so-called by-phrase as mentioned above), the authors propose \textsc{refl1} as an argument-blocking device. In contrast, if a semantic specification of the affected argument is not possible they refer to \textsc{refl2} as an argument-binding device. In the latter case the affected argument receives an arbitrary human interpretation. This distinction is claimed to be of a categorical nature which does not seem to allow for microvariation. In the following, we will show that this claim does not withstand closer scrutiny. 

\largerpage
Leaving aside the formal machinery used in the two-level semantics approach, we condense the main ideas and apply the stipulated distinctions to the diverse usages of \textit{se} in BCS. We critically discuss the question whether the distinction between \textsc{refl1} and \textsc{refl2} is sufficient for a typology of \textit{se} usages in BCS. The point of departure is the following list of surface configurations with \textit{se} proposed by \citet{FJL10}:

\begin{enumerate}
\setcounter{enumi}{0}
\item ``Reflexive passive'' where the second argument is realised in the nominative. The passive is restricted to transitive verbs and, as the authors claim, does not allow the so-called by-phrase in the form \textit{od strane} for the expression of the first argument. In the following example (\ref{(2.36a)}) the agent (i.e. the builder) allegedly remains unspecified.

\begin{exe}\ex
\begin{xlist}
\ex[]{\label{(2.36a)}
\gll Kuća \textbf{se} gradi.  \\
 house \textsc{refl} build\textsc{.3prs}  \\
\glt `The house is being built.'\\}
\ex[*]{\label{(2.36b)}
\gll Kuća \textbf{se} gradi od strane radnika.\\
 house \textsc{refl} build\textsc{.3prs} from side builders\\
\glt Intended: ‘The house is being built by builders.’\\}
\strut\hfill (BCS; \citealt[205]{FJL10})
\end{xlist}
\end{exe}

\noindent However, if we verify this claim with selected data from the web corpora, we do find instances of the by-phrase \textit{od} \textit{strane} specifying the reference of the first argument – see our example presented in (\ref{(2.37)}).\footnote{This possibility seems to have been noted in Croatian grammaticography. While \citet[257]{BHMV99} allow for the insertion of the subject only in the case of participle passive with an animate subject (which they call \textit{agens u užem smislu}), \citet[318]{SilicPranjkovic07} claim that it is generally possible to insert subjects in passive sentences.}

\begin{exe}\ex\label{(2.37)}
\gll Koristile su \textbf{se} razne metode od strane vladajuće stranke i njihovih {poslušnika [\dots].} \\
 use\textsc{.ptcp.pl.f} be\textsc{.3pl}  \textsc{refl} various methods from side ruling party and their minions \\
\glt ‘Various methods were used by the ruling party and their minions [\dots].’
\hfill [hrWaC v2.2]
\end{exe}


\item ``Reflexive impersonal'' where the first argument is affected; in standard Croatian and Serbian the second argument has to be realised in the nominative as in (\ref{(2.38a)}), whereas – as \citet{FJL10} claim – in colloquial Croatian it can also appear in the accusative as in example (\ref{(2.38b)}) below \citep[cf.][146]{Katicic86}.\footnote{Both \citet[146]{Katicic86} and \citet[260]{BHMV99} agree that such constructions with an object in the accusative belong to Croatian substandard; see ,,Preoblika obezličenja ne primjenjuje se na prelazne glagole s izrečenim objektom u pomnije dotjeranom hrvatskom književnom jeziku i zato je to oznaka nešto manje brižna izražavanja'' \citep[146]{Katicic86}.} In contrast to the passive, the impersonal is not restricted to transitive verbs. Based on example (\ref{(2.39a)}), \citet[214]{FJL10} assume that in BCS no by-phrase is possible: see their permutation of the example in (\ref{(2.39b)}). This is in line with the claims of \citet[318]{SilicPranjkovic07}.\footnote{It is possible to find similar examples in hrWaC v2.2, for instance.

\ea\label{13102021}
\gll [\dots] čuje \textbf{se} vodu kako lupa o zidove {suđerice [\dots].}\\
 {} hear.\textsc{3prs} \textsc{refl} water how hit.\textsc{3prs} about walls dishwasher\\
\glt `[\dots] one hears the water splashing against the dishwasher walls [\dots].'
\hfill [hrWaC v2.2]
\z}

\begin{exe}\ex
\begin{xlist}
\ex\label{(2.38a)}
\gll Čuje \textbf{se} kiša.\\
 hear\textsc{.3prs} \textsc{refl} rain\\
\ex\label{(2.38b)}
\gll Čuje \textbf{se} kišu. \\
 hear\textsc{.3prs} \textsc{refl} rain\\
 \end{xlist}
\glt ‘One hears the rain.’ 
\hfill (Cr; \citealt[214]{FJL10})

\ex
\begin{xlist}
\ex[]{\label{(2.39a)}
\gll Plesalo \textbf{se} sve do zore.  \\
 dance\textsc{.ptcp.sg.n} \textsc{refl} all to dawn \\
 \glt ‘One danced until dawn.’\\}
\ex[*]{\label{(2.39b)}
\gll Plesalo \textbf{se} sve do zore od strane žena.\\
 dance\textsc{.ptcp.sg.n} \textsc{refl} all to dawn from side women\\
\glt Intended: ‘One danced until dawn.’ \\}
\strut\hfill (BCS; \citealt[223]{FJL10},\\\strut\hfill adapted from \citealt[72]{Progovac05})
\end{xlist}
\end{exe} 

\noindent However, as in the case of the passive, in our corpora we were able to find examples containing the by-phrase, like the one presented in (\ref{(2.40)}). This would speak in favour of an interpretation as \textsc{refl1} and not \textsc{refl2} in the terms of \citet{FJL10}.  

\begin{exe}\ex\label{(2.40)}
\gll Gdje \textbf{ste} Vi bili tada, kada \textbf{se} prozivalo narod od strane bivše premijerke Jadranke {Kosor [\dots]?} \\
 where be\textsc{.3pl} you be\textsc{.ptcp.pl.m} then when \textsc{refl} call.out\textsc{.ptcp.sg.n} nation from side former prime.minister Jadranka Kosor \\
\glt ‘Where were you then, when the nation was called out by the former prime minister, Jadranka Kosor [\dots]?’ 
\hfill [hrWaC v2.2]
\end{exe}

\largerpage
\item Genuine reflexive and reciprocal where the second argument is coreferential with the first; the semantic specification of the affected second argument is obtained via identification or coindexation with the referent of the subject. In example (\ref{(2.41)}) the person referred to by the subject washes him or herself; example (\ref{(2.35)}) from \citet[103]{Moulton15} provided above at the beginning of the section is a case of a reciprocal interpretation. In these cases a by-phrase with \textit{od} \textit{strane} is ruled out.


\begin{exe}\ex\label{(2.41)}
\gll Pažljivo \textbf{se} umivam i nakon toga nanosim hidratantnu kremu.   \\
 carefully \textsc{refl} wash.face\textsc{.1prs} and after that apply\textsc{.1prs} hydrating cream   \\
\glt ‘I carefully wash my face and apply a hydrating cream afterwards.’ \\
\hfill [hrWaC v2.2]
\end{exe}\clearpage

\item Antipassive where the blocked second argument is usually interpreted as arbitrary; this construction allows for the overt expression of the affected argument in the form of a prepositional phrase which leads to a non-arbitrary interpretation, see the example presented in (\ref{(2.42)}). Antipassives are restricted to a small group of verbs. \citet[110]{Moulton15}, who uses the term agent attributive, shows that \textit{on se tuče} `he gets into fights' allows a by-phrase: \textit{on} \textit{se} \textit{tuče} \textit{s} \textit{drugima}, `he beats others up.' 

\begin{exe}\ex\label{(2.42)}
\gll Kada \textbf{se} dete mnogo  tuče, nadgledajte \textbf{ga}  češće.\\
 when \textsc{refl} child much hit\textsc{.3prs} oversee\textsc{.imp.2pl} him more often\\
\glt ‘When a child hits a lot (= is a frequent hitter), watch him closely.’  \\
\hfill (Sr; \citealt[111]{Moulton15})
\end{exe}

\item Middles look like passives but involve a non-episodic reading. ``The subject is interpreted as having properties that do or do not allow for the action expressed by the predicate to be potentially performed on the subject by an implicit, generic agent in a specific way expressed by adverbial means.'' \citep[227]{FJL10}. The first argument – i.e. the washer in example (\ref{(2.43)}) below – is bound. No by-phrase seems possible.

\begin{exe}\ex\label{(2.43)}
\gll Ova \textbf{se} haljina lako pere.\\
 this \textsc{refl} dress easily wash\textsc{.3prs} \\
\glt ‘This dress is easily washed.’ 
\hfill (Cr; \citealt[214]{Kucanda98})
\end{exe}


\item Decausatives affect the second argument and denote situations with an unagentive interpretation which is derived from the identification of the second and the first argument – see example (\ref{(2.44)}) provided below. In some cases a by-phrase with \textit{od} and the genitive is possible – see the example presented in (\ref{(2.45)}).


\begin{exe}\ex\label{(2.44)}
\gll Potopio \textbf{se} brod, poginulo 36 ljudi, među njima i trudnica. \\
 sink\textsc{.ptcp.sg.m} \textsc{refl} ship die\textsc{.ptcp.sg.n} 36 people among them and pregnant.woman \\
\glt ‘A ship sank, 36 people, including a pregnant woman, perished.’  \\
\hfill (Cr; \citealt[109]{Moulton15})

\ex\label{(2.45)}
\gll Čarobničine tamnozelene oči zamute \textbf{se} \minsp{[} od suza]. \\
 sorceress dark.green eyes blur\textsc{.3prs} \textsc{refl} {} from tears\\
\glt ‘The sorceress’ dark green eyes blurred from tears.’ \\
\hfill (Cr; \citealt[145]{Katicic86})
\end{exe}
\end{enumerate}

\noindent In addition to these six types \citet{FJL10} distinguish a further type which they call the ``involuntary state construction'', where the first argument is encoded in the dative and the predicate receives a stative reading. This type is attested, for example, in Polish, but not in BCS where there is a related but distinct construction for which we propose the term \textsc{feel-like construction}. Among others, it involves ``a dispositional interpretation (`x feels like V-ing') but no overt dispositional element''. \citet{MarusicZaucer06} suggest (for Slovene) that a null-psych-verb is present. This construction contains a dative phrase expressing the first argument interpreted as an experiencer. We consider structures with a nominative subject to be feel-like constructions as well – see example (\ref{(2.46)}) provided below.

\begin{exe}\ex\label{(2.46)}
\gll Marku \textbf{se} igrala košarka.\\
 Marko\textsc{.dat} \textsc{refl} play\textsc{.ptcp.sg.f} basketball\\
\glt ‘Marko felt like playing basketball.’ 
\hfill (Sr; \citealt[249]{StanojcicPopovic02})
\end{exe}

\noindent This discussion can be summarised as follows: the (slightly revised) list of surface configurations is indeed useful for capturing the range of usages of \textit{se} in BCS. The \textsc{refl1}/\textsc{refl2} dichotomy, however, turns out to be built on shaky empirical ground.\footnote{In a brief digression, we would like to comment on middles and the feel-like construction. More in-depth analyses explaining the additional semantic elements mentioned above are certainly needed, but we do not agree with \citet[228]{FJL10} who assume the presence of a modal operator of possibility in the structure of middles. Generally, we would argue against a broad understanding of the term modality. Real modal constructions differ both in form and function. Neither the ``feel-like'' nor the ``involuntary action'' semantic component (e.g. for Polish) belongs to the semantic domain of possibility as a subdomain of modality sensu stricto \citep[cf.][]{AuweraPlungian98}. We think we are dealing with a meaning usually associated with so-called psych-verbs. As to the referential status of argument expression in the case of middles, the sentence interpretation is always generic, while in feel-like constructions it can be either specific or generic. Furthermore, we would like to point out that the status of the dative phrase (whether it is an external argument or not) needs a more elaborate discussion. Due to lack of space, we will refrain from deeper analysis and refer the reader to works on non-canonical subjects in Croatian by \citet{Kucanda98} and \citet*{HWK18}.}

\subsubsection{Conclusion: how many types of \textit{se} do we need to distinguish?}
\label{Conclusion: how many types of se do we need to distinguish?}
To conclude, the preceding discussion of the approach to a typology of reflexive markers proposed by \citet[228]{FJL10} provides a mixed picture. On the one hand, the authors propose a list of surface configurations which seems to be applicable to BCS and claim that in Slavic a crucial role should be assigned to the availability of a by-phrase. They convincingly argue for the distinction between argument blocking and argument binding. However, our first tentative empirical test of their claims reveals that this distinction becomes blurred because in natural language use we found evidence that many more reflexive constructions allow the use of the by-phrase than the authors who refer to the prescriptive norms of standard Croatian and Serbian claim. Our data indicate that in BCS only middles are a clear case for \textsc{refl2}. All the remaining usages of \textit{se} are either unclear or evidently belong to the \textsc{refl1} argument-blocking type. These usages, however, vary considerably. This leaves us with the sobering conclusion that for an empirically validated, full typology of syntactic types of reflexive markers much more work has to be done. 

Therefore, we draw on the crucial observations by \citet{FJL10}, but use an additional feature in our typology. The main idea is that a \textsc{refl} affects one of the arguments of the verbal predicate, preventing the canonical realisation of this argument as subject or object \citep[218]{FJL10}. With regard to our empirical data, we, however, do not base our typology on the availability of the by-phrase. Instead, we propose a much simpler, robust typology referring to which argument (first vs second) is affected. Based on the discussion of the seven surface types above, we thus reach a threefold distinction:

\begin{enumerate}
	\item \textsc{first argument affected},
	\item \textsc{second argument affected},
	\item \textsc{first and second argument affected}.
\end{enumerate}

Additionally, we fully acknowledge the status of lexically determined usages of the reflexive marker. These reflexive verbs are of major interest to our study on CC as they appear both in CTPs (matrix verbs) and complements (both finite and semifinite). As to the grammatically determined usages, we more or less accept the list of surface types proposed by \citet{FJL10}.

\begin{enumerate}
\item \textsc{refl\textsubscript{lex}} – \textit{se} stored in the lexicon; in contrast to \citet{FJL10} we assume that this group comprises not only reflexiva tantum (reflexive verb as lemma, e.g. \textit{smijati se} `laugh'), but also verbs which have a slightly different meaning than their non-reflexive counterpart (reflexive verb as lexeme, e.g. \textit{šetati} `stroll/walk' vs \textit{šetati} \textit{se} `take a walk');
\item \textsc{refl\textsubscript{1st}} – constructions where only the first argument is affected:
non-standard impersonal;
\item \textsc{refl\textsubscript{2nd}} – constructions where only the second argument is affected:
reflexive/reciprocal, antipassive;
\item \textsc{refl\textsubscript{1st${}+{}$2nd}} – constructions where both the first and second argument are affected:
passive, standard impersonal, middles, feel-like and decausative.
\end{enumerate}

For the purposes of the present study, it is sufficient to distinguish these four types of usages of the reflexive marker \textit{se}. In Parts \ref{part2} and \ref{part3} we comment on the reflexive CL \textit{si} which can occur in the same contexts as the \textsc{refl\textsubscript{2nd}} CL \textit{se}. We use this tentative typology, including the corresponding abbreviations \textsc{refl\textsubscript{lex}}, \textsc{refl\textsubscript{1st}}, \textsc{refl\textsubscript{2nd}} and \textsc{refl\textsubscript{1st${}+{}$2nd}}, throughout the book. The main focus of the psycholinguistic test is on lexical reflexives (\textsc{refl\textsubscript{lex}}) and genuine reflexives (subtype of \textsc{refl\textsubscript{2nd}}). We hypothesise that the  lexical vs grammatical usage of \textit{se} plays a role in CC.
