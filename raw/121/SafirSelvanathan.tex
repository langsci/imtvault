\documentclass[output=paper]{langsci/langscibook} 
\title{Niger-Congo transitive reciprocal constructions and polysemy with reflexives} 
\author{%
Ken Safir\affiliation{Rutgers University}\lastand 
Naga Selvanathan \affiliation{Rutgers University} 
}
\chapterDOI{10.17169/langsci.b121.496} %will be filled in at production
\shorttitlerunninghead{Niger-Congo reciprocal and reflexive polysemy}
 
\abstract{
The transitive reciprocal constructions in a large number of Niger-Congo languages are known to allow both reciprocal and reflexive meanings. Reflexive/reciprocal polysemy induced by verb affixes (or clitics) is common, but reflexive/reciprocal polysemy with full direct object (DO) anaphors appears to be relatively uncommon outside of Africa. Many other languages of Africa (e.g. Babanki, Bafut, {Fe’efe’e}, Ga, Ibibio, Limbum, Saari, Urhobo and Yoruba) have transitive reciprocal polysemy of this kind, but if so, the source of reciprocal interpretation remains a puzzle. We propose a novel type of reciprocal construction that can account for this pattern. In such Transitive Reciprocal Constructions (TRC), we posit that the direct object is a variable with no contribution to reciprocal meaning, but that it may have reciprocal form by virtue of shape concord with a reciprocal marker (RCM) on little v (also \citealt{Bruening2006}). Otherwise it spells out as an anaphor that does not impose a specific kind of anaphoric reading. The source of reciprocal meaning is actually the RCM which can be covert in some languages. We further show that our proposal has far-reaching implications for theories of reciprocity and how identity-dependent meaning can be composed in natural language. 
}

% Keywords: Polysemy, Niger-Congo, Detransitivizers, Reciprocals, Reflexives, Anaphors

\maketitle
\begin{document}



\section{Introduction}
\largerpage
The prevailing idea about the typology of reciprocal constructions is that there are two major types: the constructions which encode reciprocity with a periphrastic anaphor and those that encode reciprocity within the verb itself (\citealt{KönigGast2008}; \citealt{Siloni2012}; among others).

\ea\label{ex:safir:1}
\ea\label{ex:safir:1a}
The men hit each other.  \\
\ex\label{ex:safir:1b}
The men collided.\\
\z
\z

As means of illustration, \REF{ex:safir:1a} shows a construction in which \textit{each other} is commonly assumed to be the source of reciprocal meaning (eg. \citealt{HeimEtAl1991}). In \REF{ex:safir:1b}, however, the source of reciprocal meaning is lexicalized within the verb itself \citep{Siloni2012}.  There are, of course, further divisions within each strategy but each strategy is considered distinct from the other and each is given a separate account. A specific natural language, thus, employs either or both to convey reciprocity. In general terms, the main distinction between \REF{ex:safir:1a} and \REF{ex:safir:1b} comes down to a difference in verb valency. Unlike the verb in \REF{ex:safir:1a} which is transitive, the verb in \REF{ex:safir:1b} is intransitive.

In this paper, based on data from various African languages, it is proposed that all demonstrably transitive reciprocal sentences have the same structure, including \REF{ex:safir:1a}, and this analysis of \textit{Transitive Reciprocal Constructions} (TRCs) extends to cases where the direct object appears to be a polysemic anaphor. TRCs are argued to have a reciprocal marker (RCM) hosted on little v and it is posited that it is the RCM which contributes the reciprocal meaning. The direct object in a TRC is thus simply a variable that \textit{can} be spelled out with a reciprocal form but carries no reciprocal meaning \citep{Safir2014}. Among other consequences, our proposal predicts a long noticed, but unexplained, aspect of reciprocals: unlike reflexives, reciprocal relationships are almost always clausal. This will be shown to arise as a natural consequence of our proposal.
 
The paper is organized as follows. In \sectref{sec:safir:2}, we outline the empirical motivation for this distinct type of reciprocal strategy, the TRC. In \sectref{sec:safir:3}, we outline the proposal in more detail and present evidence to support our analysis. In \sectref{sec:safir:4}, we take a closer look at \ili{Lubukusu} to substantiate the various aspects of the proposal. We then conclude with a list of implications that arise from our proposal.


\section{Motivating the TRC: A formal account of polysemic anaphors}\label{sec:safir:2}



Our main motivation for our proposal arises out of an attempt to account for reflexive/reciprocal polysemy of argument position anaphors, which as far as we know has received little to no satisfactory formal account (although see \citealt{Heine1999} who provides a diachronic analysis of ambiguous anaphors). As an example of such polysemy, consider the following transitive constructions from \ili{Yoruba}, \ili{Gungbe} and \ili{Fe’efe’e}.\footnote{Data with an ID number have been accessed from the Afranaph Project Database (Safir, ongoing) on June, 2015, and can be found at this url: \url{http://www.africananaphora.rutgers.edu/afranaph-database-mainmenu-130}. Language data not accompanied with an ID number were personally communicated to us.} The anaphors in the examples below are highlighted.



  
\ea\label{ex:safir:2}
\ili{Yoruba} \\
\gll Àwọn     obìnrin   náà  rí   \textbf{ara}   \textbf{wọn}. \\
they     women   the   see   body   their\\
\glt  ‘The women see themselves/each other/their bodies.’ {(ID1912)}
\z

  
\newpage 
\ea\label{ex:safir:3}
\ili{Gungbe} \\
\gll Náwè     lɔ  lɛ  ná  hù  \textbf{yé-ɖé}. \\
woman    \textsc{det}  \textsc{pl}  \textsc{fut}  kill  \textsc{3pl}-ɖé \\
\glt ‘The women will kill themselves/each other.’ (ID3597) 
\z


\ea\label{ex:safir:4}
\ili{Fe’efe’e} \\
\gll Pʉ-nʒwīɛ   lɒfɛn     \textbf{nā}   \textbf{jɒɒ}. \\
\textsc{c}{2-woman}   {\textsc{pst2}}-sell   body   \textsc{poss.det.c1}\\
\glt ‘The women betrayed themselves/each other.’ {(ID8237)} 
\z


Examples \REF{ex:safir:2}, \REF{ex:safir:3} and \REF{ex:safir:4} illustrate ambiguous sentences that can have a reciprocal meaning if the context permits. The \ili{Gungbe} and \ili{Fe’efe’e} anaphors are two-way ambiguous between a reciprocal and a reflexive reading. The \ili{Yoruba} anaphor in \REF{ex:safir:2} even permits a literal interpretation ‘their bodies’. Such polysemy is attested in several other African languages (many represented in the Afranaph database, including \ili{Babanki}, Bafut, Ga, Ibibio, \ili{Limbum}, Saari and Urhobo).

What is the right formal account of this polysemy? Given the current understanding of how reciprocal constructions are formed (as illustrated in \REF{ex:safir:1}), we have one of two choices. Either we have to accept that the highlighted anaphors are adjuncts and that reflexive and reciprocal meanings are mediated by a null verbal affix (under the intransitive hypothesis), or that the anaphors themselves are ambiguous with reciprocal and reflexive meaning (under what is taken to be the transitive hypothesis). 

Given that the counterparts of (\ref{ex:safir:2}--\ref{ex:safir:4}) would be unacceptable if no direct object is lexically expressed (which would be the case if there were a null, detransitivizing affix), the sentences in question must be transitive, so we reject the former option. This means that the most plausible analysis is some form of the transitive analysis, such as an analysis where the source of reciprocity in these constructions is in each case the direct object itself. On this version of the transitive analysis, there must be (at least) two distinct lexical entries for the underlined phrases in (\ref{ex:safir:2}--\ref{ex:safir:4}), one of them being a reciprocal entry and another being a reflexive entry. In \ili{Yoruba}, there would have to be three entries. This story quickly gets rather clumsy when we look at related languages. Consider Urhobo, a language that represents the extreme side of the scale. \citet{AzizaSafir2006} report that in this language, there are four different types of anaphoric direct objects and all four can be used to convey both reciprocal and reflexive meanings. 


\ea\label{ex:safir:5}
\ili{Urhobo} \\
  \ea\label{ex:safir:5a}
  \gll {Ayen}   {mrẹ}     \textbf{{ohwo-ohwo}}. \\
	they   see.\textsc{pst}   person-person \\
  \glt   ‘They saw themselves/each other.’ (ID2874)

  \ex\label{ex:safir:5b}
  \gll Emo    na  fa    \textbf{{oma-oma-rẹ-ayen}}.\\
  children  the  flog.\textsc{pst}  body-body-of-theirs-\textsc{3pl}\\
  \glt   ‘The children flogged themselves/each other.’ (ID2974)

  \ex\label{ex:safir:5c}
  \gll Avware   vwo  eguọnọ   kẹ   \textbf{oma-rẹ-avware-e}.\\
	We     have   love     for   body-of-ours-\textsc{neg}  \\
  \glt   ‘We hate ourselves/each other.’  (ID2828)

  \ex\label{ex:safir:5d}
  \gll Emo    na    gbon     \textbf{oma-obo-rẹ-ayen}. \\
	children  the  criticize.\textsc{pst}   body-hand-of-\textsc{3pl}\\
  \glt   ‘The children criticize themselves/each other.’ (Rose Aziza personal communication)
  \z
\z
 
If all the highlighted phrases in \REF{ex:safir:5} are the source of reciprocal/reflexive meaning, this would mean that each of them would have a reciprocal and a reflexive lexical entry – a substantial redundancy in the lexicon. Note that our consultant (Rose Aziza) reports that there is nothing determining a particular anaphor use. For example, \REF{ex:safir:5} will have the same exact meaning even if one were to use the direct object forms in \REF{ex:safir:5a} and \REF{ex:safir:5b}. 

We propose an alternative version of the transitive analysis that we believe can explain the polysemy illustrated above more satisfactorily. We claim that the highlighted phrases do not contribute any reciprocal meaning at all and are just variables which have been spelled out differently. This idea is adopted from \citet{Safir2014}, who argues that there is universally only one abstract anaphoric form in natural language, but the “one true anaphor” can be spelled out differently depending on the phase in which its antecedent is found. Despite having no reciprocal/reflexive meaning themselves, these variables are merged as direct objects in order to satisfy the c-selectional needs of the verb. This is what we claim is also happening in these \ili{Niger-Congo} languages. The source of reciprocal meaning in these constructions is, instead, posited to be a reciprocal marker (RCM) head that does not affect transitivity. This posited head is covert in \ili{Yoruba}, \ili{Gungbe}, \ili{Fe’efe’e} and Urhobo but can be overt as well, as it is in \ili{Limbum}, for example.


\ea\label{ex:safir:6}
\ili{Limbum}\\
\gll Wèe  tʉr  à  rjà’-*(\textbf{se})    \textbf{{mnyor}}. \\
\textsc{2pl}  have  \textsc{inf}  \textsc{inf}.help-\textsc{pl.rcm}  \textsc{c}6-bodies \\
\glt ‘We (\textsc{pl}) must help each other.’ {(}Francis \ili{Ndi} Wepngong pers. comm.{)} 
\z


As seen in \REF{ex:safir:6}, \ili{Limbum} also has a direct object anaphor which does not appear to have any compositional reciprocal meaning, much like the languages in (\ref{ex:safir:2}--\ref{ex:safir:5}). However, \ili{Limbum} differs from these other languages in having an overt RCM, realized as \textit{-se}, which is necessary for reciprocal meaning. When this affix is absent, \REF{ex:safir:6} only has a reflexive interpretation. A unified explanation for \ili{Limbum} and the other languages we have seen naturally follows from our proposal. We claim that \ili{Limbum} reveals what is actually happening in the languages in (\ref{ex:safir:2}--\ref{ex:safir:5}). In those languages, the polysemic nature of the anaphor is only illusory. There is in fact a covert head (an RCM) that determines when the sentence is interpreted as a reciprocal construction. In the absence of the RCM, the variable is simply determined as any other variable, namely, a thematic argument dependent on an antecedent, a local antecedent in this instance, to yield a reflexive reading. We, thus, conclude that the data in (\ref{ex:safir:2}--\ref{ex:safir:6}) represent a distinct type of reciprocal construction, a TRC.\footnote{\citet{MoysieFaurie2008} shows similar data in some \ili{Oceanic} languages. Paicî, for example, has the following construction (\citealt[121]{MoysieFaurie2008}, her example (34)):

\ea\label{ex:safir:i}
\gll  \textup{Ru}  \textup{\textbf{pi}-uc̑a-\textbf{rî}  \textbf{ru}}.\\
 \textsc{3du}  \textsc{pref}-look-\textsc{tr}  \textsc{3du}\\
\glt  ‘They observe each other.’
\z 
\largerpage[-1]

This reciprocal construction requires an obligatory verb prefix as well as the direct object pronoun. Note that this pronoun is just a simple 3rd person pronoun, much like the subject pronoun. The fact that this is a transitive construction is indicated by the transitive verbal suffix -\textit{rî}, which occurs when the post-verbal pronoun is included. This suffix indicates that the verb is not detransitivized by the reciprocal prefix \textit{pi}-. TRCs with overt verb affixes and direct objects are clearly not unique to African languages. We would like to thank Patricia Cabredo Hofherr for bringing this paper to our attention.} We outline the proposal in more detail in the next section.

\section{Our proposal} \label{sec:safir:3}

We propose the following base structure for TRCs crosslinguistically:

\ea\label{ex:safir:7}
%Base structure for Transitive Reciprocal Construction (TRC) \\
\begin{forest}baseline
[vP
	[DP\textsubscript{antecedent}] [v'
		[v-RCM] [VP
			[V] [DP\textsubscript{anaphor}]
		]
	]
]
\end{forest}
\z 


The tree in \REF{ex:safir:7} shows the relevant structure in which we claim the main dependency and spell out forms are established. This is the little vP phase in which the reciprocal direct object and its antecedent are base generated. The other aspects of the proposal are as follows:


\begin{itemize}[noitemsep]
\item The RCM is hosted on little v (see \citealt{Bruening2006} for a similar view).

\item The subject DP (in spec, vP) binds the variable in object position and the RCM forms a mapping relation between the atoms in the subject DP. The object in TRCs does not contribute any reciprocal meaning.

\item The relation between v-RCM and the DP anaphor determines the morphological shape of the anaphor (which we call \textit{shape concord}).

\end{itemize}

\largerpage[-2]

In this paper, we argue that the RCM is the little v head (see also \citealt{BakerEtAl2013}). The second aspect of the proposal is what allows us to account for reciprocal/reflexive polysemy that we saw in \sectref{sec:safir:2} in a straightforward way. In these sentences, reciprocal meaning comes from a null RCM. The various spell out forms (morphological shape) of the direct object that results are language-particular. 

There is also an interesting cross-linguistic difference that our proposal can capture with a minor modification. In the languages that we have seen so far, it is difficult to see how the anaphoric form could be the source of reciprocal meaning. For example, in \ili{Limbum} shown in \REF{ex:safir:6}, the anaphoric form \textit{mynor} just means ‘bodies’. However, we claim that our proposal can account for even constructions in which the direct object appears to be compositionally reciprocal. Consider \ili{Lubukusu}.



\ea\label{ex:safir:8}
\ili{Lubukusu}\\
\gll Bob  ne  Billi  ba-a-bon-*(\textbf{an})-a     (\textbf{ba-b-eene}  \textbf{khu} \textbf{b-eene}). \\
Bob   and   Bill   \textsc{sm.c2-pst}-see-\textbf{\textsc{rcm}}\textsc{-fv}   \textsc{c2-c2}-own on \textsc{c2}-own \\
\glt ‘Bob and Bill saw each other.’  (ID1377, 1379) 
\z


Like \ili{Limbum} in \REF{ex:safir:6}, \ili{Lubukusu} also has a sentence with an overt RCM and direct object. However, the RCM is obligatory for the reciprocal meaning to be realized. However, unlike \ili{Limbum}, \ili{Lubukusu} appears to have an anaphoric object which could plausibly encode reciprocal meaning. Nonetheless, we claim that it is only the RCM that encodes reciprocal meaning even in \ili{Lubukusu}. Note that like \ili{Limbum}, \ili{Lubukusu} requires the presence of the RCM for the reciprocal meaning to emerge. To account for why TRCs in languages like \ili{Lubukusu} have a reciprocal anaphor (RCA) as opposed to TRCs in languages like \ili{Limbum} which have a non-specific anaphor (ANA), we propose reciprocal concord. We claim that \ili{Lubukusu} anaphors can have an uninterpretable reciprocal feature. When such an anaphor is merged, this uninterpretable feature must be checked by the interpretable reciprocal feature of the RCM. This reverse \textsc{Agree} account is similar to \citegen{Zeijlstra2004} account of negative concord. Thus, the difference in spell-out form of the direct objects in \ili{Limbum} and \ili{Lubukusu} TRCs boils down to the feature difference in \REF{ex:safir:9}.


\begin{multicols}{2}\ea\label{ex:safir:9}
\ea\label{ex:safir:9a}\ili{Lubukusu}\\
\hspace*{-1cm} %put root under language name
\begin{forest}baseline, for tree={align=center,base=top}
	[vP
	[DP\textsubscript{antecedent}] [v'
		[v-RCM\\ \relax {[+reciprocal]}] [VP
			[V] [RCA\\ \relax {[-reciprocal]}]
			]	
		]	
	]
\end{forest}
\ex\label{ex:safir:9b}\ili{Limbum}\\
\hspace*{-1cm} %put root under language name
\begin{forest}baseline, for tree={align=center,base=top}
	[vP
	[DP\textsubscript{antecedent}] [v'
	[v-RCM\\ \relax {[+reciprocal]}] [VP
	[V] [ANA]
	]	
	]	
	]
\end{forest}
\z
\z\end{multicols}

In \REF{ex:safir:9a} and \REF{ex:safir:9b}, the RCM has interpretable reciprocal features. \ili{Lubukusu} and \ili{Limbum} differ in that only the former has an anaphoric form that has uninterpretable reciprocal features. By checking this feature, the anaphor in \ili{Lubukusu} spells out as a form that \textit{looks} reciprocal. In \ili{Limbum}, there is no such uninterpretable feature, thus it spells out as a non-specific anaphor.\footnote{There are alternative ways without appealing to feature checking to derive the same result. In a Distributed Morphology (DM) approach, concord with the antecedent adds a reciprocal feature to the anaphor. It will be pronounceable if there is special morphology for it, which is then inserted in the tree. If there is no morphology for it, it is not inserted as a reciprocal anaphor, but as a non-specific one. On this version of a DM account, there is always reciprocal concord, but not always morphology to show it.} 

Our proposal is thus able to account uniformly for reciprocal constructions which on the surface look quite different. These surface patterns are represented schematically in \REF{ex:safir:10}.


\ea\label{ex:safir:10}
\glll a. DP\-\textsubscript{antecedent}      V-RCM  DP\textsubscript{RCA}  {} {} {} {} {Lubukusu} \\
b. DP\-\textsubscript{antecedent}      V-RCM  DP\textsubscript{ANA}  {} {} {} {} {Limbum} \\
c. DP\-\textsubscript{antecedent}      V    DP\textsubscript{ANA}  {} {} {} { } {Yoruba}   \\
\z

The (10a) pattern shows \ili{Lubukusu} type languages which have an overt RCM and overt RCA. The (10b) pattern shows \ili{Limbum} type languages which have an overt RCM and an overt ANA.\footnote{A reviewer asks whether an expletive can be merged as the DO in TRCs, but if so, the construction would simply not be transitive. Our claim that the DO has no reciprocal semantics does not mean that it has no thematic weight, but quite the opposite, as the DO has to be an argument of the verb. Thus, we would not expect to see an expletive in the DO position of a TRC much like we do not see sentences like the following in English, \textit{He devoured there}, where \textit{there} has no thematic content.}   Finally, the (10c) pattern shows \ili{Yoruba} type languages which have no overt RCM and an overt ANA. However, underlyingly, all three types of languages are argued to have the same structure shown in \REF{ex:safir:7}. They differ only in whether the RCM is overt or covert and in whether the anaphor has uninterpretable reciprocal features.  In existing accounts, a uniform analysis for all these constructions is not possible and this is a further advantage of our proposal. 

In the next section, we focus on \ili{Lubukusu} which provides evidence for the two main aspects of the proposal: 1) that these reciprocal constructions are in fact transitive (even though the direct object can be omitted) and 2) that the position of the RCM is little v.

\section{Supporting the details: \ili{Lubukusu} case study}\label{sec:safir:4}
\subsection{The RCA is not an adjunct}

It is crucial to our analysis of TRCs that we show that the anaphoric element is not an adjunct but rather a genuine direct object. This is necessary because \citet{Mchombo2004} (adopting \citealt{ReinhartReuland1993}) has shown that in some \ili{Bantu} languages, the RCM is a valence reducer. If the same applies to \ili{Lubukusu}, then our claim that \ili{Lubukusu} has TRCs is unsupported. However, we can independently show that \ili{Lubukusu} RCAs are direct objects and not adjuncts. There are two pieces of evidence. The first has to do with possible interpretations in ellipsis contexts and the other has to do with anaphor positions when they are focused.

\subsection{Sloppy and contrastive readings in ellipsis contexts}

\citet[106]{Mchombo2004} shows that in \ili{Chichewa}, reflexive constructions without an overt object are actually transitive, and that reciprocal constructions without an overt object are intransitive. He does this using the availability of sloppy and contrastive readings in ellipsis contexts. This argumentation is taken from \citet{SellsEtAl1987}.



\ea\label{ex:safir:11}
\ili{Chichewa}\\
\gll {}[Aleje    á-ma-\textbf{{dzi}}{-nyóz-á}] ku-pósá   asodzi. \\
{\db}\textsc{c}2.hunters   \textsc{c2-hab-}\textsc{rfm}-despise-\textsc{fv}   \textsc{inf}-exceed   \textsc{c}2.fishermen\\
\glt i. ‘The hunters despise themselves more than the fishermen despise themselves.’ \textit{sloppy} \\
\glt ii. ‘The hunters despise themselves more than the fishermen despise them.’    \textit{contrastive}
\z


In the reflexive construction in \REF{ex:safir:11}, both sloppy and contrastive readings are possible. While the sloppy reading is possible regardless of whether there is a direct object, the presence of the contrastive reading indicates that the antecedent clause must have a transitive verb. 


\ea\label{ex:safir:12}
\ili{Chichewa}  \\
\gll {}[Aleje     á-ma-nyóz-\textbf{án}-á]    ku-pósá   asodzi. \\
{\db}\textsc{c}2.hunters   \textsc{c2-hab}-despise-\textsc{rcm}\textsc{-fv}     \textsc{inf}-exceed   \textsc{c}2.fishermen \\
\glt
‘The hunters despise each other more than the fishermen despise each other.'  \textit{sloppy}  
\z


By contrast, \REF{ex:safir:12} shows that only the sloppy reading is possible in a \ili{Chichewa} reciprocal construction. The absence of the contrastive reading is taken to indicate that the antecedent clause truly lacks a direct object. This comparison between the \ili{Chichewa} RCM and reflexive marked (RFM) constructions thus provides a consistent way to test whether \ili{Lubukusu} reciprocal constructions are indeed transitive. As expected under our proposal, the contrastive reading is possible.


\ea\label{ex:safir:13}
\ili{Lubukusu}\\
\gll [Ba-saani   ba-a-biyil-\textbf{{an}}-a]   khu-khil-a   ba-khasi. \\
{\db}\textsc{c}2-man   \textsc{c2-pst}-hate-\textsc{rcm}\textsc{-fv}   \textsc{c}15-defeat-\textsc{fv}   \textsc{c}2-woman  \\
\glt i. `The men hate each other more than the women hate each other.'  \textit{sloppy} \\
\glt ii. `The men hate each other more than the women hate the men.'  \textit{contrastive}                       (ID5036)\\
\z

Example \REF{ex:safir:13} shows that in the same type of ellipsis context as in \ili{Lubukusu}, the elided clause can have the contrastive reading. Following Mchombo's conclusion for \ili{Chichewa} reflexive constructions, we have to conclude that the reciprocal construction in \ili{Lubukusu} is indeed transitive. The RCM, thus, cannot be a detransitivizer and the RCA (when it occurs) is a direct object.

\subsubsection{Argument/Adjunct information focus asymmetries}

In this section, we outline a novel argument in further support of our claim that the reciprocal construction in \ili{Lubukusu} is indeed transitive. In \ili{Lubukusu}, it appears that post-verbal elements that are arguments have to be \textit{immediately} post-verbal if they have informational focus. However, post-verbal adjuncts need not be immediately post-verbal. This can be seen in the following contexts.



\ea\label{ex:safir:14}
\settowidth\jamwidth{(DO-Adverb)}
\ili{Lubukusu}  \\
\gll Naanu  ni-ye    ba-saani   ba-a-pa   lukali? \\
who  that-\textsc{agr}   \textsc{c}2-man   \textsc{c}2-\textsc{tns}-beat   fiercely\\
\glt ‘Who did the men beat fiercely?’ \\
\ea[]{
\gll A1:  Ba-saani  ba-a-pa  \textbf{{Yohana}}  lukali.     \\ 
     ~  \textsc{c}2-man   \textsc{c2-tns}-beat   John     fiercely\\ \jambox{(DO-Adverb)}
\glt    ‘The men beat John fiercely.'
}
\ex[]{
A2:  \#Ba-saani  ba-a-pa  lukali    \textbf{Yohana}.  \jambox{(Adverb-DO)}  
}
\z
\z


Example \REF{ex:safir:14} shows a question in \ili{Lubukusu}, seeking the identity of the person who received a beating. This question places information focus on \textit{Yohanna.} As indicated by our informant (Justine Sikuku), as a response to the question, the adverb cannot intervene between the verb and the DO. Our consultant further reports that the order is only possible if the adverb in A2 is prosodically marked. This is quite surprising, given that in neutral contexts both the declaratives in A1 and A2 are possible without any additional prosodic marking on either the adverb or DO. This is what we see if the question asks for the manner in which the beating took place as in ‘How did the men beat John?’ Here, the response can have either order although information focus is on the adverb. 

The difference between arguments and adjuncts also manifests with locative adjuncts versus locative arguments as shown in \REF{ex:safir:15}.  


\ea\label{ex:safir:15}
\settowidth\jamwidth{(PP adj-Adverb)}
\ili{Lubukusu}  \\
\gll Wahena  ni-yo    Yohana  a-a-kona  bwangu? \\
where     that-\textsc{agr}   John     \textsc{c1-tns}-sleep  quickly \\
\glt ‘Where did John sleep quickly?’ \\
\ea[]{
\gll A1:  Yohana  a-kona    \textbf{{mu-sikuri}} {bwangu}.   \\
     ~  John     \textsc{c1-tns}-sleep  \textsc{c}18-field   quickly   \\\jambox{(PP adj-Adverb)}
\glt   ‘John slept in the field quickly.’
}
\ex[]{
 A2:  Yohana  a-kona    bwangu  \textbf{mu-sikuri}.  \jambox{(PP adj-Adverb)}
}
\z
\z 

The answers A1 and A2 to the question in \REF{ex:safir:15} illustrate information focus on an adjunct PP. In line with the pattern above, the PP adjunct can be either immediately post-verbal or have an intervening adverb. This does not hold when the PP is an argument of the verb as shown below.  

\ea\label{ex:safir:16} 
\settowidth\jamwidth{(PP arg-Adverb)}
\ili{Lubukusu}\\
\gll Wahena  ni-yo    Yohana   a-a-ra     si-tabu          bwangu? \\
      where     that-\textsc{agr}   John     \textsc{c1-tns}-put   \textsc{c}7-book       quickly\\
\glt ‘Where did John put the book quickly?’  
\ea[]{
\gll A1:  Yohana    a-a-ra  si-tabu    \textbf{{khu-mesa}}  bwangu.      \\
     ~  John       \textsc{c1-tns-}put  \textsc{c}7-book   \textsc{c}17-table   quickly\\\jambox{(PP arg-Adverb)}
\glt   ‘John put the book on the table quickly.’   \\ 
}
\ex[]{
 A2.  \#Yohana   a-a-ra   si-tabu      bwangu   \textbf{khu-mesa}.        \jambox{(Adverb-PP arg)}
}
\z
\z


In \REF{ex:safir:16}, the answer to the question places information focus on an argument PP. Unlike with the PP adjunct, the adverb cannot intervene between the verb and the PP argument here. However, in a neutral context, where the locative argument is not focused, both orders in A1 and A2 are reported to be perfectly fine.

As a final illustration of this pattern, note that the restriction is not dependent on whether the intervener is an argument or adjunct. This can be seen with ditransitives. 


\ea\label{ex:safir:17} 
\settowidth\jamwidth{(DO-IO)}
\ili{Lubukusu}\\
\gll   Naanu    ni-ye    ba-saani   ba-rum-ir-a     bi-tabu?\\
     who     that-\textsc{agr}   \textsc{c}2-man   \textsc{c}2-send-\textsc{appl-fv}   \textsc{c}8-book\\
\glt ‘Who did the men send the books to?’
\ea[]{
\gll A1:  Ba-saani   ba-rum-ir-a    \textbf{{Maria}}     bi-tabu.       \\
     ~  \textsc{c}2-men   \textsc{c2.tns}-send-\textsc{appl-fv}   Mary      \textsc{c}8-book  \\\jambox{(IO-DO)}
\glt    ‘The men sent Mary books.’
}
\ex[]{
 A2.  \#Ba-saani  ba-rum-ir-a      bi-tabu   \textbf{Maria}.   \jambox{(DO-IO)}
}
\z
\z

The IO-DO and the DO-IO orders are both possible in neutral contexts, but the IO-DO order in \REF{ex:safir:17} is necessary when the IO has information focus as in this case. This shows that there is no sensitivity to whether the intervener is an argument or an adjunct. As long as some argument of the verb has information focus, it has to occur in an immediately post-verbal position and nothing can occur between it and the verb. This positional requirement does not hold of adjuncts with information focus.

This generalization in \ili{Lubukusu}, which appears to be quite robust, thus gives us another reliable way to test whether the direct objects in reciprocal constructions are arguments or adjuncts. And as we will see, this test also points to the fact that the RCAs in \ili{Lubukusu} reciprocal constructions are indeed  arguments.

First note that when the RCA as a direct object has information focus, it must be immediately post-verbal. 


\ea\label{ex:safir:18}
\settowidth\jamwidth{(Adv-RCA)}
\ili{Lubukusu}\\
\gll Naanu  ni-ye     ba-saani   ba-fumi-a   lukali?\\
who   that-\textsc{agr}  \textsc{c}2-men   \textsc{c}2-praise-\textsc{fv}   fiercely\\
\glt ‘Who did the men praise fiercely?’
\ea[]{
  A1: 
\gll  Ba-saani ba-fumi-an-a      \textbf{ba-be-ene}    \textbf{khu} \textbf{be-ene} lukali.\\  
   \textsc{c}2-men   \textsc{c}2-praise-\textsc{rcm-fv}       \textsc{c}2-\textsc{ c}2-own  on   \textsc{c}2-own  fiercely\\\jambox{(RCA-Adv)}
\glt    `The men praised each other fiercely.'
}
\ex[]{
  A2:   \#Ba-saani ba-fumi-an-a   lukali\textbf{ba-be-ene khu be-ene}. \jambox{(Adv-RCA)}   
}
\z
\z


Here, the adverb cannot intervene between the RCA and the adverb. This indicates that the RCA is an argument and not an adjunct. In addition, when the RCA is an IO with informational focus, only the IO-DO order is possible.


\ea\label{ex:safir:19}
\settowidth\jamwidth{(RCA-DO))}
\ili{Lubukusu}  \\
\ea[]{
\gll  Ba-saani  ba-rum-ir-an-a    \textbf{ba-be-ene}  \textbf{khu} \textbf{be-ene}  bi-tabu.\\
\textsc{c}2-men   \textsc{c2.tns}-send-\textsc{appl-rcm-fv}     \textsc{c2-c2}-own on   \textsc{c}2-own   \textsc{c}8-book\\\jambox{(RCA-DO)}
\glt    ‘The men sent each other books.’ 
}         
\ex[]{
   \#{Ba-saani   ba-rum-ir-an-a  bi-tabu}   \textbf{ba-be-ene} \textbf{khu} \textbf{be-ene}. \jambox{(DO-RCA)}
}
\z
\z

This evidence indicates once again that the RCA is an argument and not an adjunct of the verb. The data in this sub-section clearly converges with the conclusion based on the ellipsis facts above. The RCA in \ili{Lubukusu} reciprocal constructions must be an argument. What this means is that the RCM itself cannot be a detransitivizer. The fact that the RCM and an argument RCA co-occur in \ili{Lubukusu} receives a straightforward explanation in our proposal. \ili{Lubukusu} reciprocal constructions are TRCs with the derivation shown in \REF{ex:safir:7}. 

 
\subsection{The RCM is in little v}


In this section, we will reproduce some evidence from \citet{BakerEtAl2013} in favor of the claim that the RCM is located in little v; although it should be noted that our claim that the RCM is hosted on little v is not a novel one (for example, see \citealt{Bruening2006}). Our evidence here comes from the split comitative construction in \ili{Lubukusu}. 

We will now review the evidence in some detail. First, note the agreement paradigms that are evident in split comitative constructions and the corresponding readings. 

  
\ea\label{ex:safir:20}
\ili{Lubukusu}\\
\ea\label{ex:safir:20a}
\gll   O-mu-ndu        ne   e-m-bwa  \textbf{{bi}}{-a-rekukh-a}. \\
       \textsc{c1-c1}-person   and  \textsc{c9-c9}-dog   \textsc{c}\textbf{8}-\textsc{pst}-leave-\textsc{fv} \\
\glt   ‘The man and the dog left.’

\ex\label{ex:safir:20b}
\gll   E-m-bwa  \textbf{{yi}}-a-rekukh-a     ne   o-mu-ndu.\\
       \textsc{c9-c9}-dog  \textsc{c}\textbf{9}-\textsc{pst}-leave-\textsc{fv}   with   \textsc{c1-c1}-person\\
\glt   ‘The dog left with the man.’/‘The dog and the man left.’

\ex\label{ex:safir:20c}
\gll   O-mu-ndu   \textbf{a}-rekukh-a     ne   e-m-bwa.\\
       \textsc{c1-c1}-person   \textsc{c}\textbf{1}-\textsc{pst}-leave-\textsc{fv}   with   \textsc{c9-c9}-dog\\
\glt   ‘The man left with the dog.’/‘The man and the dog left.’
\z
\z

Example \REF{ex:safir:20a} shows an intact conjunct in the subject position of the sentence. Here, the noun class agreement is default class 8 for mixed class conjuncts, as shown in bold face. The (b) and (c) sentences show what happens when the conjunct splits. In \REF{ex:safir:20b}, the subject shows class 9 agreement and in \REF{ex:safir:20c} the subject shows class 1 agreement. This is the case even if the sentences have the conjoined readings. This shows that the verb only agrees with whatever is in the syntactic subject position, presumably Spec, TP. Baker et al. account for the availability of the conjoined reading together with subject agreement by hypothesizing that the whole conjunct is base-generated in Spec, vP as the logical subject but that split occurs from this position. The higher conjunct moves to Spec, TP and verb movement to T results in the surface order.  

The split comitative with an RCM is consistent with this general picture. Example \REF{ex:safir:21} shows this.


\ea\label{ex:safir:21}
\ili{Lubukusu}  \\
\ea\label{ex:safir:21a}
\gll  O-mu-hayi   ne   e-twika   bi-a-bon-\textbf{an}-a.  \\
       \textsc{c1-c1}-hunter   and   \textsc{c}9-giraffe   \textsc{c8-tns}-see-\textsc{rcm}\textsc{-fv} \\
\glt   ‘The hunter and the giraffe saw each other.’

\ex\label{ex:safir:21b}
\gll   O-mu-hayi     a-a-bon-\textbf{an}-a     ne   e-twika.\\
       \textsc{c1-c1}-hunter    \textsc{c1-pst}-see-\textsc{rcm}\textsc{-fv}  with  \textsc{c}9-giraffe   \\
\glt   ‘The hunter and the giraffe saw each other.’
\z
\z

Regardless of whether the conjunct is intact or split, the conjoined reading is possible and reciprocal interpretation imposed by the RCM makes it necessary in (\ref{ex:safir:21a},b). What this indicates is that the RCM must be able to access the conjunct before the split occurs. If it is true that the split occurs in the Spec of little v, then the RCM must not be located in a position higher than little v either. However, the RCM cannot also be lower than V, given that verb movement collects the RCM as a suffix. The most plausible location for the RCM is thus little v itself. 

As a control, observe the pattern in a split comitative with a reflexive marker on the verb. 

  
\ea\label{ex:safir:22}
\ili{Lubukusu}  \\
\ea\label{ex:safir:22a}
\gll  O-mu-hayi   ne   e-twika   bi-a-\textbf{{i}}{-bon-a}.  \\
       \textsc{c1-c1}-hunter   and   \textsc{c}9-giraffe   \textsc{c}8-\textsc{pst-}\textsc{rfm}-see-\textsc{fv} \\
\glt   ‘The hunter and the giraffe saw themselves.’

\ex\label{ex:safir:22b}
\gll  O-mu-hayi   a-a-\textbf{i}-bon-a     ne   e-twika. \\
       \textsc{c1-c1}-hunter  \textsc{c}1-\textsc{pst-}\textsc{rfm}-see-\textsc{fv}   with   \textsc{c}9-giraffe \\
\glt   ‘The hunter saw himself with the giraffe.’
\glt   (*‘The hunter and the giraffe saw themselves.’)
\z
\z

Unsurprisingly, when the whole conjunct is present in the subject position, as in \REF{ex:safir:22a}, the conjoined reflexive reading is possible with default agreement on the verb. However, when split occurs, only the split reading is possible (only the surface subject is the antecedent of the reflexive relation) with corresponding agreement on the verb. What this indicates is that the RFM has to be in a position higher than little v. After all, if it was in little v itself, we would expect \REF{ex:safir:22b} to have the conjoined reading like with the RCM. This analysis predicts that the RFM and RCM can co-occur given that they occur in different positions, and this is borne out as seen in \REF{ex:safir:23}.


\ea\label{ex:safir:23}
\ili{Lubukusu}  \\
\gll Ba-khasi   ba-a-\textbf{i}-yeet-\textbf{an}-a     (ba-b-eene khu ba-b-eene). \\
\textsc{c}2-woman   \textsc{c2-pst-}\textbf{\textsc{rfm}}-help-\textsc{rcm}\textsc{-fv}   \textsc{c2-c}2-own on  \textsc{c2-c2}-own\\
\glt ‘The women helped each other and themselves.’ (ID1533)
\z


Example \REF{ex:safir:23} shows a prefixal RFM and a suffixal RCM occurring on the same verb. This is the clear evidence that the RFM and RCM cannot be located in the same position and this converges with the evidence from comitative split. Ultimately, the data we have seen here supports our claim that the RCM is in little v. 

One may ask why it could not be the case that the RCM could be higher than little v, since the RCM could still “see” the whole conjunct before it splits (i.e., the conjunct is in its c-command domain and are presumably within the same phase), in which case there could be an alternative account of why conjoined readings are still possible with a split comitative like that in \REF{ex:safir:21b}. However, we have both empirical and theoretical reasons for ruling out this possibility. Empirically, this is not a desirable alternative, as the distinction between \REF{ex:safir:21b} (with the RCM) and \REF{ex:safir:22b} (with the RFM) would be eliminated. If a RCM that is higher than little v could see the whole conjunct through c-command, why can't the RFM? After all, the RFM also c-commands the little v position if it is, or is attached to, a higher head on the verb. There is also a theoretical reason against having the RCM higher than little v. We assume that the relationship between the external argument antecedent and the RCM is one of dependency, where the RCM depends on the antecedent for its interpretation. We further assume the Independence Principle \citep{Safir2004}. 


\ea\label{ex:safir:24}
Independence Principle \citep[3]{Safir2004}\\
  If x depends on y, x cannot c-command y.
\z

Given that the RCM depends on the external argument, the Independence Principle will rule out any configuration in which the RCM c-commands the external argument. These considerations thus support the view that the RCM is in little v and not higher than it. 

  

\section{Implications and conclusion}\label{sec:safir:5}
 

In this paper, we have argued for a distinct reciprocal strategy which we have called the TRC in \REF{ex:safir:7} above. 

The main motivation for this new strategy was to provide a simple straightforward explanation for how reciprocal interpretations arise for argument position anaphors that show reflexive/reciprocal polysemy. In our analysis, the reciprocal anaphor is simply a variable that can be spelled out with a reciprocal shape in some languages. The actual reciprocal meaning comes from the RCM which can be covert in some languages. We have posited further that the RCM and/or the direct object anaphor can be covert, which generates a typology of surface patterns that we summarize in \tabref{tab:Safir:1}.
 
\begin{table}
\caption{Typology of TRCs.     (Bolded elements are overt, unbolded elements with < > are covert.)}
\label{tab:Safir:1}

\begin{tabularx}{\textwidth}{llX}
\lsptoprule
a.&
[\textsubscript{vP} \textbf{DP\textsubscript{antecedent}} [ v-\textbf{RCM}   [\textsubscript{VP} \textbf{V}  \textbf{DP\textsubscript{anaphor}} ]]]
&\ili{Limbum}, \ili{Lubukusu}
\\
b.&
[\textsubscript{vP} \textbf{DP\textsubscript{antecedent}} [ v-\textbf{RCM}   [\textsubscript{VP} \textbf{V}  <DP\textsubscript{anaphor}> ]]]   &
\ili{Lubukusu}  
\\
c.&  [\textsubscript{vP} \textbf{DP\textsubscript{antecedent}} [ <v-RCM>   [\textsubscript{VP} \textbf{V}  \textbf{DP\textsubscript{anaphor}} ]]]  &
Urhobo
\\
d. & [\textsubscript{vP} \textbf{DP\textsubscript{antecedent}} [ <v-RCM>   [\textsubscript{VP} \textbf{V}  <DP\textsubscript{anaphor}> ]]] &
--
\\
\lspbottomrule
\end{tabularx}
\end{table} 

The surface realization in line (b) of \tabref{tab:Safir:1}, where the non-specific anaphor is a null direct object, was argued for on the basis of ellipsis data in \REF{ex:safir:13}. Of the four possibilities, the pattern in line (d) is presumably ruled out on functional grounds, as there will be no evidence for the hearer that a reciprocal construction is being used as opposed to a regular intransitive sentence. Also note that the direct object DP could be an RCA or ANA. Since English reciprocal sentences with \textit{each other} are transitive, we are committed to the analysis that such sentences in English are TRCs. \citet{Williams1991}, for instance, argues (contra \citealt{HeimEtAl1991}) that \textit{each other} is not compositional, that is, they are just non-compositional surface forms; and if this is right, then English reciprocal constructions could just be an instance of the pattern in line (c).\footnote{We do not defend this claim here due to space limitations, but it is worth noting that \textit{The boys told each girl a different story} is ambiguous (as many stories as girls, or they all hear one story different from some other in the discourse); while \textit{The boys told each other a different story} does not allow the distributed reading. This is explicable if \textit{each other} is a non-decomposable idiom which surfaces as a result of shape concord with a null RCM on English verbs.} The implication here is that a reciprocal construction may appear to be intransitive because an object is missing or may appear to have a reciprocal anaphor that is contributing to semantic interpretation; but on close analysis, the construction may actually turn out to be a TRC. In \ili{Lubukusu}, we showed this by demonstrating that the RCM must occur for the reciprocal meaning to emerge and by showing that the RCA is indeed an argument and not an adjunct. Although it is unrealistic to expect the same tests for transitivity to work in all languages, we call into question any claim that it is the RCA, rather than overt or covert RCM, that contributes the semantic meaning of reciprocity.\footnote{As part of the wider cross-linguistic applicability of our proposal, a reviewer asks how our analysis could be applied to languages with “SE anaphors” (\citealt{ReinhartReuland1993}) which also show reflexive/reciprocal polysemy which disappears in certain contexts. For example, while \ili{German} SE anaphors within a preposition phrase (PP) can only be construed as a reflexive, \ili{Polish} SE anaphors still retain the reflexive/reciprocal polysemy within PPs. One explanation for this contrast between \ili{German} and \ili{Polish} that our analysis could shed light on is by exploiting whether PP is a phase. Recall that in our analysis, the DO is spelled out as a RCA or ANA within the phase. In TRC contexts, this is possible because the vP contains the antecedent for the DO in this phase. Thus, if PPs in \ili{German} were a phase as well, then the SE anaphor construed as a reciprocal within the phasal PP will be left without an antecedent. On the other hand, if \ili{Polish} PPs were not phases, then the SE anaphor construed as a reciprocal will not be spelled out until the higher vP (which contains the antecendent) is completed. The inability of \ili{German} PPs to allow pseudopassive formation, as opposed to English, would be consistent with the assumption that \ili{German} PPs are phases and (at least some) English PPs are not. However, the reviewer also mentions another fact which we do not see how to account for, though we do not know of an account that does better. The polysemy of \ili{German} SE-anaphors also disappears when the DO is stressed. In such contexts, only the reflexive meaning is possible. Thus, \textit{Sie hassen \textsc{sich}} (‘they hate themselves’) only has the reflexive reading, whereas \textit{Sie \textsc{hassen} sich} (‘they hate themselves/each other’) can be reflexive or reciprocal. First, we do not agree with the reviewer that this indicates that the \ili{German} SE-anaphor has two functions (one as a reciprocal valence marker and one as a reflexive argument). Nonetheless, this hints at another source of parametric variation between \ili{German} and English, as the latter clearly allows a stressed RCA (\textit{They saw \textsc{each other}}). We will have to leave detailed extensions of our theory to other language groups for future research.} 

Our proposal also enables us to explain a notable difference between reflexives and reciprocals. While long distance reflexives are relatively common, long distance reciprocals are almost completely absent.\footnote{That is, with the exception of the so called assisted long distance reciprocals as in \textit{John and Paul said they saw each other.} In this sentence, the crucial reading is one where John said he saw Paul and Paul said he saw John. Although this has been analyzed as a scope phenomenon, see \citet{Dimitriadis2000} who proposes a non-scope solution which crucially relies on local antecedenthood. For discussion of the locality of reciprocal constructions crosslinguistically, see, for example, \citet{Everaert1999}.}  We believe that our proposal accounts for this quite naturally if the RCM on little v, overt or not, is the only source of reciprocal readings. When the direct object is transferred to the interfaces as the vP containing it is completed, the object’s spell out form and interpretation has to be determined as well. Given that the only possible binder of the variable at this point is the subject DP in Spec, vP, it follows that any antecedent of the direct object must be locally found as well, and the relationship must be reciprocal.  For the logic of this approach to go through, we must assume that reflexive interpretations do not always require the mediation of an RFM; that is, there are languages where reflexive readings are possible just because a coargument variable is bound by a local antecedent (as in typical long distance bound variable anaphora). The latter demonstration is beyond the scope of this article, however.

In conclusion, our proposal for the syntactic structure of TRCs has the following explanatory advantages. It accounts for the availability of both reciprocal/reflexive meanings with the same direct object shape, without requiring redundant stipulations in the lexicon for languages like Urhobo, which have multiple forms that allow both reflexive and reciprocal readings. It also explains how reciprocal constructions that include an overt reciprocal operator on the verb and an overt anaphor in an argument position are formed. Cross-linguistically, our analysis provides a coherent and empirically supported typology of TRCs, thus unifying the syntactic distribution of reciprocal readings across a wide range of apparently different constructions. Finally, our analysis suggests an account for the general absence of long distance reciprocals by contrast with long distance reflexives.

\section*{Acknowledgments}

We would like to thank all the consultants of the Afranaph Project, especially those whose native speaker judgments are reported here: Justine Sikuku (\ili{Lubukusu}), Rose Aziza (Urhobo), Oluseye Adesola (\ili{Yoruba}), Enoch Aboh (\ili{Gungbe}), Djomeni Gabriel Delmon (\ili{Fe’efe’e}), and Francis \ili{Ndi} Wepngong (\ili{Limbum}). We would also like to acknowledge the support of NSF BCS 0303447, NSF BCS 0523102, NSF BCS 0919086 and NSF BCS 1324404. We also thank the audience of ACAL 46 as well as two anonymous reviewers for comments and suggestions.  

\section*{Abbreviations}
\begin{multicols}{2}\begin{tabbing}
c1, c2, c3 etc. \= class markers\kill
\textsc{1, \!2, \!3}  \>  1\textsuperscript{st}, 2\textsuperscript{nd}, 3\textsuperscript{rd} \\ \> person features\\
\textsc{agr}  \>  Agreement\\
\textsc{appl}  \>  applicative\\
c1, \!c2, \!c3 \!etc. \> noun class markers\\
\textsc{det}  \>  determiner\\
\textsc{do} \>   direct object\\
\textsc{du} \>   dual\\
\textsc{fut}  \>  future tense\\
\textsc{fv}  \>  final vowel\\
\textsc{hab} \>   habitual\\
\textsc{inf}  \>  infinitive \\
\textsc{neg}  \>  negation\\
\textsc{perf}  \>  perfective\\
\textsc{pl}  \>  plural\\
\textsc{poss} \>   possessive\\
\textsc{pref}  \>  verb prefix\\
\textsc{rca}  \>  reciprocal anaphor\\
\textsc{rcm}  \>  reciprocal marker\\
\textsc{rfm}  \>  reflexive marker\\
\textsc{se} \>   simplex expression\\  
\textsc{tns} \>   tense\\
\textsc{tr}  \>  transitive suffix\\
\end{tabbing}\end{multicols}



\printbibliography[heading=subbibliography,notkeyword=this]

\end{document}