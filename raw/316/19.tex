\documentclass[output=paper]{langscibook} 
\ChapterDOI{10.5281/zenodo.5082486}

\author{Mina Giannoula\affiliation{University of Chicago}}
\title{Two kinds of `much' in Greek}
\abstract{Τhe English element \textit{much} has an NPI use (see \citealt{bolinger1972degree, isreal1996, solt2015q}). In Greek, the degree modifier \textit{poly-} ‘much’ displays a polarity-sensitive distribution as well. Unlike its free counterpart \textit{poly} ‘a lot/much’, the bound morpheme \textit{poly-} ‘much’ functions as an NPI occurring only in antiveridical environments. The main research question that this study addresses is why the bound morpheme \textit{poly-} ‘much’, but not its independent form \textit{poly} ‘a lot/much’, is an NPI. In other words, why does \textit{poly-} appear only in negative sentences, as opposed to \textit{poly}, which appears both in negative and affirmative contexts? In my paper, I present a syntactic analysis for
the licensing of the degree modifier \textit{poly-} ‘much’ as an NPI. Following \citet{giannaki1997dissert,giannaki2007even} and \citet{hedde2004phd,hedde2008}, I argue that its polarity licensing happens syntactically as an Agree relation between its formal
uninterpretable [uNeg] feature and the interpretable [Neg] feature of the
antiveridical operator. I also posit that the two kinds of `much' in Greek, i.e., the free \textit{poly}
and the bound \textit{poly-}, are generated in different positions in the syntactic
structure.

\keywords{negative polarity items, negation, much, nonveridicality, degree modifier, Greek}}

\begin{document}
\SetupAffiliations{mark style=none}
\maketitle

\section{Introduction} 

Negative polarity items (NPIs) -- a term attributed to \citet{baker1970} -- are context-sensitive elements appearing in specific environments, like negation, but are excluded from the affirmative ones. Though \citet{buyssens1959} first lists items sensitive to negation, the scientific research on NPIs began with the works by \citet{klima1964}, \citet{horn1972prop}, \citet{fauconnier1975a, fauconnier1975b}, and \citet{ladusaw1979}.

The element \textit{much} is one of the classic NPIs in English:\largerpage[-1]

\ea\label{gia:ex1}
        \ea[]{Joanne did not read much last night.}\label{gia:ex1a}
        \ex[*]{Joanne read much last night.}\label{gia:ex1b}
\z\z

\noindent As the grammaticality of sentence \REF{gia:ex1a} shows, \textit{much} appears under the scope of negation. However, affirmative environments, i.e., those lacking negation, affect the well-formedness of the sentence in \REF{gia:ex1b}.

Its Greek counterpart, the free morpheme \textit{poly} ‘much/a lot’ belongs to the category of adverbs of degree that show no restricted distribution, as seen in \REF{gia:ex2}:

\begin{exe}
\ex\label{gia:ex2} \begin{xlist}
        \ex\label{gia:ex2a} \gll I Ioanna dhen kimithike poly xthes vradi. \\
        the Joanne not slept.\textsc{3sg} much last night \\
        \trans `Joanne didn't sleep much last night.'
        \ex\label{gia:ex2b} \gll I Ioanna kimithike poly xthes vradi. \\
        the Joanne slept.\textsc{3sg} a.lot last night \\
        \trans `Joanne slept a lot last night.'
    \end{xlist}
\end{exe}

\noindent Regarding the degree of Joanne’s sleeping, what the speaker implies by uttering \REF{gia:ex2a} is that she slept  sufficiently, but not a lot, as she did in \REF{gia:ex2b}. In other words, the degree of Joanne’s sleeping in \REF{gia:ex2a} is less than a lot.

Like the free \textit{poly} ‘a lot/much’, its bound counterpart, the item \textit{poly-} ‘much’, is also used as a degree modifier in Greek. However, its distribution is restricted only to negative contexts, as the ungrammaticality of the affirmative sentence in \REF{gia:ex3b} shows, proving that \textit{poly} is an NPI.

\begin{exe}
\ex\label{gia:ex3} \begin{xlist}
        \ex[]{\gll I Ioanna dhen poly-kimithike xthes vradi. \\
        the Joanne not much-slept.\textsc{3sg} last night \\
        \trans `Joanne didn't sleep much last night.'}\label{gia:ex3a}
        \ex[*] {\gll I Ioanna poly-kimithike xthes vradi. \\
        the Joanne much-slept.\textsc{3sg} last night \\
        \trans Intended: `Joanne slept a lot last night.'}\label{gia:ex3b}
    \end{xlist}
\end{exe}

\noindent By uttering \REF{gia:ex3a}, what the speaker conveys is that Joanne slept only a little, contrary to \REF{gia:ex2a}, where in that case Joanne slept sufficiently, but not a lot.

The fact that the morphologically constructed modification of verbs with the bound element \textit{poly-} is licit only under the scope of negation has drawn some attention in the Greek literature \citep{delvevass1999,efthimiougavriili2003,ralli2004,dimelameli2009}. Focusing on the phonological, semantic and structural properties of the element, it has been pointed out that this bound element combines only with verbal bases in negative sentences to form compounds. Here, I will go one step further arguing that the bound degree modifier \textit{poly-} `much' is a strong NPI only being licensed by the antiveridical negation and \textit{without}-clauses, as opposed to its free counterpart \textit{poly} `a lot/much'.

This study addresses two main research questions: (i) Why is the bound \textit{poly-} ‘much’, but not its free form \textit{poly} ‘a lot/ much’, an NPI? In other words, why does \textit{poly-} appear only in negative sentences, as opposed to \textit{poly}, which appears both in negative and affirmative environments? (ii) Why is the meaning of the bound \textit{poly-} different from that of the free \textit{poly}? In other words, why does \textit{poly-} mean ‘a little’ but not ‘sufficiently’, as the free morpheme does?

The research is based on the (non)veridicality theory of polarity (\citealt{giannaki1997dissert,giannaki1998,giannaki2001free} et seq.), which accounts for elements exhibiting restrictions on their licensing environments, as the English \textit{anyone} and the Greek \textit{kanénas}, and places no categorial restrictions on the items showing NPI behavior.

The paper is organized as follows. In \sectref{gia:sec:nonveridicality}, I discuss briefly the (non)veridicality theory of polarity, the distinction between strong and weak NPIs (\sectref{gia:sub:framework}), and show that, based on this theory, the bound degree modifier \textit{poly-} ‘much’ is a strong NPI (\sectref{gia:sub:strongpoly-}). In \sectref{gia:sec:syntax}, I show that the bound \textit{poly-} is licensed only locally in the domain of sentential negation (super strong licensing) (\sectref{gia:sub:super}), and I claim that its licensing is accomplished syntactically due to the uninterpretable [uNeg] feature of \textit{poly-} (\sectref{gia:sub:structure}). In \sectref{gia:sec:meanings}, I answer the question how the meaning of \textit{poly-} differs from the meaning of \textit{poly} by giving the semantics of each element. \sectref{gia:sec:conclusion} concludes.

\section{Nonveridicality, NPIs, and the Greek \textit{poly-}}\label{gia:sec:nonveridicality}

\subsection{The framework} \label{gia:sub:framework}

The framework followed in the current research is the \textsc{(non)veridicality theory of polarity} (\citealt{giannakidou1994licensing,giannaki1997dissert,giannaki2001free} et seq.), which captures (i) the environments in which NPIs appear and (ii) the distinction between different kinds of NPIs. For years, it was difficult to identify the properties of NPIs and explain their polarity sensitive behavior. Under the (non)veridicality theory of polarity, which was motivated by the distribution of the NPIs \textit{kanénas} `anyone, anybody' (non-emphatic)\slash\textit{KANENAS} `no one, nobody' (emphatic) in Modern Greek and is supported crosslinguistically, Giannakidou provides a semantic account for the distribution of NPIs, i.e., for all the environments under which the property of (non)veridicality is applied.\footnote{For a discussion on emphatic/non-emphatic \textit{kanénas}, see \citet{giannaki1997dissert, giannaki1998, giannaki2000n}.} \textsc{(non)veridicality} is a semantic property under which the truth of a proposition $p$ embedded under an operator $F$ is entailed or presupposed:

\ea\label{gia:ex4}
\textit{Veridicality and nonveridicality} \citep[33]{giannaki2002licensing}
\ea {A propositional operator $F$ is \textit{veridical} iff $F\. p$ entails $p$: $F\. p \rightarrow p$;\\ otherwise, $F$ is \textit{nonveridical}.}
     \ex {A nonveridical operator $F$ is \textit{antiveridical} iff $F\. p$ entails not $p$:\\ $F\. p \rightarrow\neg p$.}
\z
\z

\noindent She also defines NPIs as linguistic expressions sensitive to (non)veridicality, that is, being licensed in non-veridical contexts:

\ea\label{gia:ex5} \textit{Polarity item} \citep[669]{giannaki2001free} \\
A linguistic expression \textit{α} is a polarity item iff:
\ea The distribution of \textit{α} is limited by sensitivity to some semantic property \textit{β} of the context of appearance, and
\ex \textit{β} is non-veridical, or a subproperty thereof: \textit{β} $\in$ \{veridicality, nonveridicality, antiveridicality, modality, intensionality, extensionality, episodicity, downward entailingness\}.
\z\z

\noindent Under this definition, NPIs are taken to be elements that appear in non-veridical contexts and are excluded from affirmative environments. They can be divided into two classes: strong NPIs and weak NPIs. \textit{Strong} NPIs are elements showing restricted distribution, being licensed only in antiveridical contexts, such as that of negation and \textit{without}-clauses, and are excluded from non-veridical environments:

\begin{exe}
\ex\label{gia:ex6}
\textit{Strong NPI}\\
An NPI is a strong NPI iff it appears only in antiveridical environments.
\end{exe}

\noindent On the other hand, \textit{weak} NPIs are elements that occur in non-veridical contexts, namely questions, conditionals, modal verbs, imperatives, generics, habituals, and disjunctions, in addition to antiveridical ones:

\begin{exe}
\ex\label{gia:ex7}
\textit{Weak NPI}\\
An NPI is a weak NPI iff it can appear in nonveridical environments.
\end{exe}

\noindent In Greek, the distinction between weak and strong NPIs is captured by non-emphatic NPIs, on the one hand, and emphatic NPIs and minimizers, on the other \citep{giannaki1997dissert,giannaki1998}.\footnote{As \citet{giannaki1997dissert,giannaki1998} indicates, Greek minimizers differ from English ones (e.g., \textit{drink a drop, sleep a wink}). Unlike the former, the latter exhibit wider distribution, appearing also in nonveridical contexts, such as questions and conditionals, among others.} Non-emphatic NPIs are the unaccented \textit{n}-words (e.g., \textit{kanenas} `anyone, anybody'), whereas the emphatic ones are the accented \textit{n}-words (e.g., \textit{KANENAS} `no one, nobody').\footnote{\citet{velou19834} is the first one to note the emphatic accent of \textit{n}-words in Greek.}

\subsection{\textit{Poly-} as a strong NPI} \label{gia:sub:strongpoly-}

Given that the bound degree modifier \textit{poly-} cannot appear in affirmative contexts, unlike its free counterpart \textit{poly}, a question that arises now is what kind of NPI it is. I~argue that, according to the (non)veridicality theory of polarity, \textit{poly-} is a strong NPI exhibiting a restricted distribution: it appears with the antiveridical licensers of negation, \textit{xoris} ‘without’ and \textit{prin} `before', but not with non-veridical licensers, namely imperatives, modal verbs, conditionals, questions, generics, habituals, and disjunctions.

\subsubsection{Negation}

\noindent Like all NPIs, \textit{poly-} occurs with sentential negation marked by negative operators, like \textit{dhen}, as in \REF{gia:ex8a}, and is excluded from affirmative contexts, as in \REF{gia:ex8b} (repeated from \ref{gia:ex3}):

\begin{exe}
\ex\label{gia:ex8} \begin{xlist}
        \ex[]{\gll I Ioanna dhen poly-kimithike xthes vradi. \\
        the Joanne not much-slept.\textsc{3sg} last night \\
        \trans `Joanne didn't sleep much last night.'\label{gia:ex8a}}
        \ex[*] {\gll I Ioanna poly-kimithike xthes vradi. \\
        the Joanne much-slept.\textsc{3sg} last night \\
        \trans Intended: `Joanne slept a lot last night.'}\label{gia:ex8b}
    \end{xlist}
\end{exe}

\subsubsection{`Without'-clauses}

\noindent \textit{Poly-} also appears in \textit{xoris} `without'-clauses:

\begin{exe}
        \ex\label{gia:ex9} \gll I Ioanna egrapse dhiagonisma xoris na poly-diavasi. \\
        the Joanne wrote.\textsc{3sg} exam without \textsc{sbjv} much-study.\textsc{3sg} \\
        \trans `Joanne took an exam without studying much.'
\end{exe}

\subsubsection{`Before'-clauses}

\noindent In addition, \textit{poly-} occurs with the antiveridical \textit{prin} `before':\footnote{\citet{giannaki1997dissert,giannaki1998} argues that \textit{prin} `before’ is context-sensitive and can be analyzed as antiveridical with respect to its second argument (see \citealt[143]{giannaki1998}).}

\begin{exe}
        \ex\label{gia:ex10} \gll I Ioanna kimithike prin na poly-diavasi. \\
        the Joanne slept.\textsc{3sg} before \textsc{sbjv} much-studied.\textsc{3sg} \\
        \trans `Joanne slept before studying much'
\end{exe}

\subsubsection{Imperatives}

\noindent On the contrary, and like many strong NPIs, \textit{poly-} does not appear in imperatives:

\begin{exe}
        \ex[*] {\gll Poly-dhiavase ghia to diagonisma! \\
        much-study.\textsc{2sg.imp} for the exam \\
        \trans Intended: `Study much for the exam!'}\label{gia:ex11}
\end{exe}

\subsubsection{Modal verbs}

\noindent Sentences with \textit{poly-} under the scope of modal verbs are ill-formed:

\begin{exe}
        \ex[*] {\gll I Ioanna bori na poly-diavasi. \\
        the Joanne may \textsc{sbjv} much-study \\
        \trans Intended: `Joanne may study much.'}\label{gia:ex12}
\end{exe}

\subsubsection{Conditionals}

\noindent Like other strong NPIs, \textit{poly-} does not allow well-formed sentences when occurring in the antecedent of conditionals:

\begin{exe}
        \ex[*] {\gll An I Ioanna poly-diavasi, tha pari A. \\
        if the Joanne much-study.\textsc{3sg} will get A  \\
        \trans Intended: `If Joanne studies much, she will get an A.'}\label{gia:ex13}
\end{exe}

\subsubsection{Questions}

\noindent In \textit{yes-no} questions, the bound \textit{poly-} does not allow well-formed sentences:

\begin{exe}
        \ex[*]{\gll Poly-dhiavase i Ioanna? \\
        much-studied.\textsc{3sg} the Joanne \\
        \trans Intended: `Did Joanne study much?'}\label{gia:ex14}
\end{exe}

\subsubsection{Generics}

\noindent Sentences with generics, which are about non-referential expressions, such as \textit{kathe fititis} `every student' in \REF{gia:ex15}, cannot license the occurrence of \textit{poly-}:

\begin{exe}
        \ex[*]{\gll Kathe fititis poly-diavazi. \\
        every student much-study.\textsc{3sg} \\
        \trans Intended: `Every student studies much.'}\label{gia:ex15}
\end{exe}

\subsubsection{Habituals}

\begin{sloppypar}
\noindent Habitual sentences with Q-adverbs of varying force (e.g., `usually', `often', `rarely', `sometimes', `never') and \textit{poly}-verbs are ill-formed:
\end{sloppypar}

\begin{exe}
        \ex[*]{\gll I Ioanna sinithos poly-maghirevi. \\
        the Joanne usually much-cook.\textsc{3sg} \\
        \trans Intended: `Joanne usually cooks much.'}\label{gia:ex16}
\end{exe}

\subsubsection{Disjunctions}

\noindent The context of disjunctions, mainly in the sense of individual disjuncts taken separately, as in \REF{gia:ex17}, comply with the bound degree modifier \textit{poly-}:

\begin{exe}
        \ex[*] {\gll I itan tixheros ke perase tin eksetasi i poly-dhiavase. \\
        either was lucky and passed.\textsc{3sg} the exam or much-studied.\textsc{3sg} \\
        \trans `Either he was lucky and passed the exam or he studied much.'}\label{gia:ex17}
\end{exe}

\noindent Therefore, as its narrow distribution shows, \textit{poly-} clearly belongs to the category of strong NPIs, only occurring under the scope of negation and the antiveridical \textit{xoris} ‘without’ and \textit{prin} `before'.

% Just uncomment the input below when you're ready to go.

\section{The syntax of \textit{poly} and \textit{poly-}} \label{gia:sec:syntax}

\subsection{Super strong licensing} \label{gia:sub:super}

Given that \textit{poly-} `much' is a strong NPI, a question that arises now, based on its restricted distribution, is whether it is licensed locally by negation (strong licensing) or it permits long-distance dependencies (weak licensing), in other words, whether \textit{poly-} needs to be in a local relation with the negative operators or not. \citet{giannaki1995subj,giannaki1997dissert,giannaki1998} and \citet{giannakiquer1995,giannakiquer1997} associate strong NPIs with strong licensing: they cannot be licensed by the negation of the main clauses when appearing in subjunctive clauses embedded by \textit{oti} `that' and \textit{pu} `that', but they allow long-distance licensing when appearing in subjunctive clauses with \textit{na}. Here, I argue that \textit{poly-} is associated with super strong licensing, showing that it can only be licensed locally in the domain of sentential negation.

More specifically, \textit{poly}- can only be licensed locally by the negative operator \textit{dhen} when appearing in indicative embedded clauses with the complementizer \textit{oti}, as \REF{gia:ex18} shows:

\begin{exe}
\ex\label{gia:ex18} \begin{xlist}
        \ex[]{\gll Ipa oti dhen poly-dhiavases ghia tin eksetasi. \\
        said.\textsc{1sg} that not much-studied.\textsc{2sg} for the exam \\
        \trans `I said that you didn't study much for the exam.'\label{gia:ex18a}}
        \ex[*] {\gll Dhen ipa oti poly-dhiavases ghia tin eksetasi. \\
        not said.\textsc{1sg} that much-studied.\textsc{2sg} for the exam \\
        \trans `I didn't say that you studied much for the exam.'}\label{gia:ex18b}
    \end{xlist}
\end{exe}

\noindent Embedded clauses with the complementizer \textit{pu} do not allow long-distance dependencies of \textit{poly-} on the negative operator \textit{dhen}:

\begin{exe}
\ex\label{gia:ex19} \begin{xlist}
        \ex[]{ \gll Mu ipe pu dhen poly-dhiavazis. \\
        me told.\textsc{1sg} that not much-study.\textsc{2sg} \\
        \trans `He told me that you don't study much.'\label{gia:ex19a}}
        \ex[*] {\gll Dhen mu ipe pu poly-dhiavazis. \\
        not me told.\textsc{1sg} that much-study.\textsc{2sg} \\
        \trans `He didn't tell me that you study much.'}\label{gia:ex19b}
    \end{xlist}
\end{exe}

\noindent Regarding subjunctive embedded domains with the complementizer \textit{na}, where the negative operator \textit{min} is used instead of \textit{dhen}, \citet{giannaki1997dissert, giannaki1998} shows that emphatics, which are strong NPIs, can be licensed even when the negative operator is in the main clause. However, unlike emphatics, \textit{poly}- does not allow long-distance licensing when occurring in subjunctive clauses with \textit{na}, as the ungrammaticality of \REF{gia:ex20b} shows:\footnote{\citet{giannakiquer1997} also point out cases of subjunctive embedded domains which are opaque, as in Catalan.}

\begin{exe}
\ex\label{gia:ex20} \begin{xlist}
        \ex[]{\gll Bori na min poly-dhiavases ghia tin eksetasi. \\
        might \textsc{sbjv} not much-studied.\textsc{2sg} for the exam \\
        \trans `It may be the case that you didn't study much for the exam.'\label{gia:ex20a}}
        \ex[*] {\gll Dhen bori na poly-dhiavases ghia tin eksetasi. \\
        not might \textsc{sbjv} much-studied.\textsc{2sg} for the exam \\
        \trans `It can't be the case that you studied much for the exam.'}\label{gia:ex20b}
    \end{xlist}
\end{exe}

\noindent I conclude here that \textit{poly-} is licensed only locally when occurring in \textit{oti-} and \textit{pu-}indicative and \textit{na-}subjunctive embedded clauses, restricting its distribution to the boundaries of mono-clausal structures. On the other hand, given that its free counterpart, the degree modifier \textit{poly} `a lot/much', is not an NPI, it appears in \textit{oti-}  and \textit{pu-} indicative and \textit{na}-subjuctive embedded clauses, whether the negative operators \textit{dhen} and \textit{min} are in the main or embedded clause:

\begin{exe}
\ex\label{gia:ex21} \begin{xlist}
        \ex\label{gia:ex21a} \gll Ipa oti dhen dhiavases poly ghia tin eksetasi. \\
        said.\textsc{1sg} that not studied.\textsc{2sg} much for the exam. \\
        \trans `I said that you didn't study much for the exam.'
        \ex\label{gia:ex21b} \gll Dhen ipa oti dhiavases poly ghia tin eksetasi. \\
        not said.\textsc{1sg} that studied.\textsc{2sg} much for the exam \\
        \trans `I didn't say that you studied much for the exam.'
    \end{xlist}

\ex\label{gia:ex22} \begin{xlist}
        \ex\label{gia:ex22a} \gll Mu ipe pu dhen dhiavazis poly. \\
        me told.\textsc{2sg} that not study.\textsc{2sg} much \\
        \trans `He told me that you don't study much.'
        \ex\label{gia:ex22b} \gll Dhen mu ipe pu dhiavazis poly. \\
        not me told.\textsc{1sg} that study.\textsc{2sg} much \\
        \trans `He didn't tell me that you study much.'
    \end{xlist}

\ex\label{gia:ex23} \begin{xlist}
        \ex\label{gia:ex23a} \gll Bori na min dhiavases poly ghia tin eksetasi. \\
        might \textsc{sbjv} not studied.\textsc{2sg} much for the exam \\
        \trans `It can be the case that you didn't study much for the exam.'
        \ex\label{gia:ex23b} \gll Dhen bori na dhiavases poly ghia tin eksetasi. \\
        not might \textsc{sbjv} much-studied.\textsc{2sg} for the exam \\
        \trans `It can't be the case that you studied much for the exam.'
    \end{xlist}
\end{exe}

\subsection{\textit{Poly} and \textit{poly-} in structure} \label{gia:sub:structure}

So far, I have shown that \textit{poly}- ‘much’ is a strong NPI, being grammatical in a sentence where it is licensed by antiveridical operators, like negation and \textit{without}-clauses. Moreover, its licensing by negative operators can only happen locally (super strong licensing). Here, I propose an analysis for its licensing which answers the first question set out above: although \textit{poly}-, like all NPIs, is sensitive to its semantic environment, I argue that its licensing is accomplished syntactically.

Before I give the syntax of the bound \textit{poly}- ‘much’, it is instructive to see the lexical features and the position of the free \textit{poly} ‘a lot/much’ in syntactic structure, which is of the category of adverbs, as its lexical entry in \REF{gia:ex24} shows:

\begin{exe}
\arraycolsep=.5\tabcolsep
\ex\label{gia:ex24}
\textit{poly} 
$\left[\begin{array}{@{}lcl@{}}
\text{\textsc{cat}} & : & \text{[Adv]} \\
\text{\textsc{infl}} & : & \text{[--]} \\
\text{\textsc{sel}} & : & \text{[$\langle - \rangle]$} \\
\end{array}\right]$
\end{exe}

\noindent For a sentence with the free degree modifier \textit{poly}, as in \REF{gia:ex25}, I assume the syntactic derivation in \figref{gia:t:basicstructure}.

\begin{exe}
\ex\label{gia:ex25}
    \gll O Petros dhen dhiavase poly. \\
    the Peter not studied.\textsc{3sg} much \\
    \trans `Peter didn't study much.'
\end{exe}

\begin{figure}[H]
\begin{footnotesize}
\begin{forest}
for tree={l sep=.5em, s sep=3em}
	[TopP, [DP [the Peter, roof, name=Pet]]
	[NegP,edge=dashed [{\hspace{1em}}]
	[Neg$'$ [Neg \\ not]
	[TP [{\hspace{1em}}]
	[T$'$ [T \\ {[+\textsc{pst}]} \\ {[$\varphi$: 3sg]} \\ studied, name=T]
	[DegP [AdvP [much, roof]]
	[Deg$'$, s sep=4em [Deg, name=Deg]
	[\textit{v}P, s sep=2em [{\hspace{1em}}]
	[\textit{v}$'$ [\textit{v},name=v]
	[VP [V \\ \textit{t}\textsubscript{V}, name=V]
	]]]]]]]]]]
	\draw[-] (V) to[out=south,in=south] (v);
	\draw[-] (v) to[out=south,in=south] (Deg);
	\draw[->] (Deg) to[out=south,in=south] (T);
\end{forest}
\end{footnotesize}
\caption{Syntactic representation of \REF{gia:ex25}}\label{gia:t:basicstructure}
\end{figure}

Following \citet{cinque1999}, I argue that the free \textit{poly} is generated in the specifier of the functional phrase Deg[ree]P, i.e., AdvP.\footnote{The obligatory or optional presence of DegP in the clausal structure does not seem to have immediate consequences for the proposed analysis.} The negative operator \textit{dhen} occupies the head of Neg[ation]P\footnote{In Greek, NegP is situated above TP (\citealt{agouraki1991, tsoulas1993, rivero1994, phillippaki1994} among others).}. The verb moves, via Head Movement \citep{travis1984}, to \textit{v} and then T to get subject-agreement and tense.\footnote{Following \citet{spyroprev2009past}, I assume that T is subject to fusion between T and Agr. I omit discussing other functional categories in the verbal projection, such as Voice and Aspect (see \citealt{merchant2015howmuch} for relevant discussion). Moreover, the subject is in its surface position, i.e., in the specifier of Topic Phrase (TopP).} That \textit{poly} sits in the specifier position of DegP comes from the fact that it is not incorporated with the verb, allowing the latter to move to T. Moreover, \textit{poly} together with other elements, such as \textit{para} `very' in \REF{gia:ex26}, form a complex head:

\begin{exe}
\ex\label{gia:ex26}
    \gll O Petros dhen dhiavase para poly. \\
    the Peter not studied.\textsc{3sg} very much \\
    \trans `Peter didn't study very much.'
\end{exe}

\noindent On the other hand, as seen in \sectref{gia:sub:super}, the bound degree modifier \textit{poly}- ‘much’ needs to be licensed locally by antiveridical operators, such as negation. The licensing of \textit{poly-}, like other Greek NPIs, is similar to the case of \textsc{negative concord} (NC). In NC languages, negation is expressed with more than one negative element in a clause (mainly, a negative marker and an \textit{n}-word), although it is interpreted only once \citep{giannaki1997dissert,giannaki1998,giannaki2002licensing,hedde2004phd,giannakihedde2017}. Working on the Greek NPI \textit{oute} ‘even’, \citet{giannaki2007even} proposes that its licensing is related to the local relation it has with negation and the uninterpretable negative feature [uNeg] \textit{oute} hosts. This feature, a characteristic it shares with other strong NPIs, needs to be checked by the interpretable [Neg] feature of sentential negation \citep{giannaki1997dissert,giannaki2007even,hedde2004phd}. Following this account, I assume that \textit{poly}- contains a formal uninterpretable feature [uNeg] that requires the presence of a matching categorial interpretable feature [Neg] in order for the sentence to be grammatical. This interpretable [Neg] feature is found in the negative operator \textit{dhen} ‘not’, as the lexical entries of the elements show:

\begin{exe}
\arraycolsep=.5\tabcolsep
\ex\label{gia:ex27}
\textit{dhen} 
$\left[\begin{array}{@{}lcl@{}}
\text{\textsc{cat}} & : & \text{[Neg [Neg]]} \\
\text{\textsc{infl}} & : & \text{[--]} \\
\text{\textsc{sel}} & : & [\langle \text{TP} \rangle] \\
\end{array}\right]$

\ex\label{gia:ex28}
\textit{poly-} 
$\left[\begin{array}{@{}lcl@{}}
\text{\textsc{cat}} & : & \text{[Deg]} \\
\text{\textsc{infl}} & : & \text{[uNeg]} \\
\text{\textsc{sel}} & : & [ \langle \text{\textit{v}P} \rangle] \\
\end{array}\right]$
\end{exe}

\noindent Unlike its free counterpart, the bound \textit{poly-} belongs to the category of Deg. I argue that its licensing is accomplished syntactically via the operation of Agree \citep{chomsky2000min,chomsky2001deriv}. The negative operator \textit{dhen} `not' with the interpretable [Neg] feature c-commands \textit{poly-} with the uninterpretable [uNeg] feature. Given that, the [uNeg] feature is checked and eliminated by the [Neg] feature of \textit{dhen}. Therefore, the agreement happens via c-command, as schematically illustrated in \figref{gia:t:NegPQPoption2}.

\begin{figure}[h!]
% \begin{footnotesize}
\begin{forest}
for tree={l sep=.5em, s sep=3em}
[NegP [{\hspace{1em}}]
[Neg$'$ [Neg \\ not % dhen 
\\ {[Neg]}]
[DegP, , edge=dashed [{\hspace{1em}}] 
[Deg \\ much- % poly- 
\\ {[\sout{uNeg}]}]]]]
\end{forest}
% \end{footnotesize}
\caption{Licensing of \textit{poly-}} \label{gia:t:NegPQPoption2}
\end{figure}

As \figref{gia:t:NegPQPoption2} shows, \textit{poly-} remains under the scope of negation. Its licensing happens in situ, thus no movement for checking is needed. Moreover, the fact that \textit{poly-} with the uninterpretable [uNeg] feature is licensed by the interpretable [Neg] feature of negation can also explain the impossibility of \textit{poly-} being licensed by non-veridical operators, such as questions and imperatives. Since non-veridical operators lack the [Neg] feature, the [uNeg] feature of \textit{poly-} cannot be checked.\footnote{The direction of probing in the assumed Agree operation is different from the one standardly assumed (cf. \citealt{chomsky2000min} et seq.): the element with the uninterpretable feature (probe), here \textit{poly-}, is c-commanded by the element with the interpretable feature (goal), here \textit{dhen} (see \citealt{hedde2004phd} et seq.).}

Since \textit{poly-} is also licensed by the antiveridical \textit{xoris} `without', I argue that the latter also has the interpretable [Neg] feature. However, the co-occurrence of the negative operator \textit{dhen} and \textit{xoris} `without' in a sentence is impossible, showing that \textit{poly-} with the uninterpretable [uNeg] feature needs the presence of only one element with an interpretable [Neg] feature in a sentence to be licensed:

\begin{exe}
\ex[*]{\gll I Ioanna dhen kimithike xoris na poly-fai. \\
    the Joanne not slept.\textsc{3sg} without \textsc{sbjv} much-ate.\textsc{3sg} \\
    \trans Intended: `Joanne didn't sleep without eating much.'}\label{gia:ex29}
\end{exe}

\noindent For a sentence with the bound \textit{poly-}, as in \REF{gia:ex30}, I propose the syntactic derivation in \figref{gia:t:polydhiavaze}.

\begin{exe}
\ex\label{gia:ex30}
    \gll O Petros dhen poly-dhiavase. \\
    the Peter not much-studied.\textsc{3sg} \\
    \trans `Peter didn't study much.'
\end{exe}

\begin{figure}
\begin{footnotesize}
\begin{forest}
for tree={l sep=.5em, s sep=3em}
	[TopP, [DP [the Peter, roof, name=Pet]]
	[NegP,edge=dashed [{\hspace{1em}}]
	[Neg$'$ [Neg \\ not]
	[TP [{\hspace{1em}}]
	[T$'$ [T \\ {[+\textsc{pst}]} \\ {[$\varphi$: 3sg]} \\ much-studied, name=T]
	[DegP [{\hspace{1em}}]
	[Deg$'$, s sep=4em [Deg \\ \textit{t}\textsubscript{Deg}, name=Deg]
	[\textit{v}P, s sep=2em [{\hspace{1em}}]
	[\textit{v}$'$ [\textit{v},name=v]
	[VP [V \\ \textit{t}\textsubscript{V}, name=V]
	]]]]]]]]]]
	\draw[-] (V) to[out=south,in=south] (v);
	\draw[-] (v) to[out=south,in=south] (Deg);
	\draw[->] (Deg) to[out=south,in=south] (T);
\end{forest}
\end{footnotesize}
\caption{Syntactic representation of \REF{gia:ex30}}\label{gia:t:polydhiavaze}
\end{figure}

I argue that \textit{poly-} is obligatorily generated in the head of the functional phrase DegP, unlike the free \textit{poly}, which is generated in SpecDegP. Sitting in that position, \textit{poly}- triggers the Head Movement of the verb to form a complex unit with it. I assume that the formation of the verbal complex happens as a subject of Head Movement \citep{travis1984}: the verb moves to the Deg-head, where the bound morpheme is generated, creating a complex unit. Later on, the complex head moves even higher, to T.\footnote{See \citet{alexanagn1998} and \citet{merchant2015howmuch} for V-to-T movement in Greek.}

So, how are \textit{poly-}verbs formed? \citet{rivero1992} discusses this phenomenon of adverb-verb word formation in Modern Greek as a subject to Incorporation providing a syntactic account.\footnote{A morphological analysis of the phenomenon of Incorporation in Modern Greek is proposed by \citet{smirnjoseph1998}. See also \citet{kakouriotis1997incorp}.} She proposes that adverbs functioning as complements, i.e., being internal to VP, can incorporate into the governing V-head considering this syntactic process an instance of Adverb Incorporation. However, treating adverbs that can be incorporated as VP-complements requires them to be obligatorily selected by the verb, which is not the case. If it was true that a verb subcategorizes for the adverb \textit{poly-} as its complement, then we would expect \textit{poly}-verbs not to take direct objects or sentences without the degree modifier \textit{poly} to be ungrammatical. As seen in \REF{gia:ex31a}, a verb like \textit{thelo} ‘want’ also takes the DP \textit{ti Maria} ‘Mary’ as its complement, whereas the absence of \textit{poly} does not render the sentence in \REF{gia:ex31b} ungrammatical.

\begin{exe}
\ex\label{gia:ex31} \begin{xlist}
        \ex\label{gia:ex31a} \gll O Yanis dhen theli poly ti Maria. \\
        the John not wants much the Mary \\
        \trans `John doesn't really want Mary.'
        \ex\label{gia:ex31b} \gll O Yanis dhen theli ti Maria. \\
        the John not wants the Mary \\
        \trans `John doesn't want Mary.'
    \end{xlist}
\end{exe}

\noindent Moreover, evidence that \textit{poly}-verb formation does not derive from the unincorporated \textit{poly} functioning as a complement to the verb comes from the fact that the formation of a \textit{poly}-verb is ungrammatical in affirmative environments. more specifically, if we follow \citeauthor{rivero1992}'s account that the degree modifier \textit{poly} ‘much’ incorporates into the verb \textit{theli} ‘wants’ to form the complex unit \textit{poly-theli}, then we expect to get the same results in positive sentences. However, this is not possible, as the ungrammaticality of \REF{gia:ex32b} shows:

\begin{exe}
\ex\label{gia:ex32} \begin{xlist}
        \ex[]{\gll O Yanis theli poly ti Maria. \\
        the John wants much the Mary \\
        \trans `John really wants Mary.'\label{gia:ex32a}}
        \ex[*] {\gll O Yanis poly-theli ti Maria. \\
        the John much-wants the Mary \\
        \trans Intended: `John really wants Mary.'}\label{gia:ex32b}
    \end{xlist}
\end{exe}

\noindent Thus, this is evidence that the formation of \textit{poly}-verbs is not a subject to Adverb Incorporation. In addition, this proves that the free degree modifier \textit{poly} and the bound degree modifier \textit{poly-} generate in different positions in the syntactic derivation and have different lexical entries, as discussed above, with the latter, but not the former, owning an inflectional uninterpretable [uNeg] feature.

\section{The meaning of \textit{poly} and \textit{poly-}} \label{gia:sec:meanings}

In this section, I answer the second question my study addresses, i.e., why the meaning of the bound degree modifier \textit{poly-} differs from that of the free degree modifier \textit{poly}, arguing that this difference can be explained by the semantics of the morphemes themselves. In other words, since both kinds of `much' in Greek are elements of category Deg but one of them projects fully to a DegP, whereas in the case of the other the projection stops at some lower level, this is related to the different meanings (values) such forms can be mapped to on a degree scale.

 As I have already presented from the very beginning of this study, both Greek degree modifiers, the free \textit{poly} and the bound \textit{poly-}, occur under the scope of negation:

\begin{exe}
\ex\label{gia:ex33} \begin{xlist}
        \ex\label{gia:ex33a} \gll O fititis dhen dhiavase poly. \\
        the student not studied.\textsc{3sg} a.lot \\
        \trans `The student didn't study a lot.'
        \ex\label{gia:ex33b} \gll O fititis dhen poly-dhiavase. \\
        the student not much-studied.\textsc{3sg} \\
        \trans `John doesn't really want Mary.'
    \end{xlist}
\end{exe}

\noindent However, its polarity-sensitive behavior identifies \textit{poly-} as an NPI, something that also affects its meaning. To capture the difference, I assume the scale of degree for gradable predicates in \REF{gia:ex34}:

\begin{exe}
\ex\label{gia:ex34} \textit{Scale of degree}\\
$\langle$excessively, a lot, sufficiently, little, very little$\rangle$
\end{exe}

\noindent In the scale in question, the value \cnst{sufficiently} is the threshold representing the value close to the norm. The scale of degree itself is sensitive to contextual factors, and the threshold \cnst{sufficiently}, like all scalar predicates, does not have a fixed value, but rather it is context-sensitive \citep{kennedy2007vag}. By uttering \REF{gia:ex33a} with the free \textit{poly} under the scope of negation, what the speaker means is that the student did not study a lot. Therefore, the degree of the student's studying is below the degree \cnst{a lot}, close to the value \cnst{sufficiently}. This means that the student studied sufficiently, but not a lot. On the other hand, by uttering the negative sentence in \REF{gia:ex33b} with the bound \textit{poly-}, what the speaker actually means is that the student studied little or even less than little. Here it is not the case that the student studied much or sufficiently. Instead, the degree of the student’s studying moves below the contextually dependent threshold, at the degree \cnst{little}, or even close to the lowest values on the scale.

In order to capture the difference in the meaning of the free \textit{poly} and the bound \textit{poly-}, I propose a semantic analysis under which there is a different denotation for each degree modifier. Starting with the free \textit{poly} ‘a lot/much’, I provide the structure in \figref{gia:t-sempoly} as a simplified version of the sentence in \REF{gia:ex33a}, where the subject is reconstructed to a lower position, i.e., below negation.

\begin{figure}
% \begin{footnotesize}
\begin{forest}
% pretty nice empty nodes,
  for tree={%
    l sep-=.2em,
   s sep=.2em,
    align=center,
    nice empty nodes,
  }
[ [not]
[ [the student]
[ [studies] [much]]]]
\end{forest}
% \end{footnotesize}
\caption{Simplified structure of sentence \REF{gia:ex33a}}\label{gia:t-sempoly}
\end{figure}

I argue that the negative sentence in \REF{gia:ex33a} is true if and only if the degree of the student’s studying is below the quantity of \cnst{a lot}. Formally, the denotation for the free degree modifier \textit{poly} is given in \REF{gia:ex35}. The semantics is a construction that involves a degree. It corresponds to the well-known generalized quantifier-style denotation that can also capture the presence of individuals. The free \textit{poly} is a relation that takes a scalar predicate \textit{P} and an individual argument \textit{x} and returns True if and only if there exists a degree \textit{d} such that \textit{x P} above the degree \cnst{sufficiently}:

\begin{exe}
\ex\label{gia:ex35}
\sib{poly}${}= \lambda P\lambda x.\exists d [P (x) (d) \wedge (d > \cnst{sufficiently})]$
\end{exe}

\noindent The analysis is built on the following denotations. In particular, the DP \textit{o fititis} ‘the student’ denotes a unique student:\footnote{The denotation for the DP \textit{o fititis} is derived by the denotations of the definite determiner \textit{o} and the noun \textit{fititis} by function application and \textit{β}-reduction:\\
(iii) \sib{fititis}${}= \lambda x[\textsc{student}(x)]$\\ 
(iv) \sib{o}${}= \lambda Q[\iota x [Q(x)]]$
}

\begin{exe}
\ex\label{gia:ex36}
\sib{o fititis}${}= \iota x [\textsc{student}(x)]$
\end{exe}

\noindent The denotation I propose for intransitive verbs like \textit{dhiavazo} ‘study’ is not the standard one. Here, intransitive verbs denote a function that takes an individual \textit{x} and a degree \textit{d}, which is assigned to the denotation of the free \textit{poly}:

\begin{exe}
\ex\label{gia:ex37}
\sib{dhiavazi}${}= \lambda d\lambda x[\textsc{study}(x)(d)]$

\ex\label{gia:ex38}
\sib{dhiavazi poly}${}= \lambda x.\exists d[\textsc{study}(d)(x) \wedge (d > \cnst{sufficiently})]$
\end{exe}

\noindent Finally, the standard denotation of the negative marker \textit{dhen} ‘not’ is given in \REF{gia:ex39}, where negation is a function that returns the opposite of the truth value of the proposition:

\begin{exe}
\ex\label{gia:ex39}
\sib{dhen}${}= \lambda p [\neg p]$
\end{exe}

\noindent Given the denotations above, the compositional semantics of the sentence in \REF{gia:ex33a} with the free degree modifier \textit{poly} is unremarkable and proceeds by function application and β-reduction as follows:

\begin{exe}
\ex\label{gia:ex40}
\sib{S}${}= \neg \exists d[\textsc{study}(\iota x [\textsc{student}(x)]) (d) \wedge (d > \cnst{sufficiently})]$
\end{exe}

\noindent The meaning of the negated sentence shows that the degree of the student's studying is not above the degree \cnst{sufficiently}. Instead, it is equal to the degree \cnst{sufficiently} or even below.

Moving to the bound \textit{poly-}, I present in \figref{gia:t-sempoly-} a simplified structure of the sentence in \REF{gia:ex33b}, where the subject is reconstructed to a lower position, i.e., below the negative operator \textit{dhen} ‘not’.

\begin{figure}
% \begin{footnotesize}
\begin{forest}
% pretty nice empty nodes,
  for tree={%
    l sep-=.2em,
   s sep=.2em,
    align=center,
    nice empty nodes,
  }
[ [not]
[ [the student]
[ [much-] [studies]]]]
\end{forest}
% \end{footnotesize}
\caption{Simplified structure of sentence \REF{gia:ex33b}} \label{gia:t-sempoly-}
\end{figure}

The denotation I propose for the bound degree modifier \textit{poly-} is given in \REF{gia:ex41}. It is similar to that of the independent form, though the degree maps to a different part on the scale. In particular, \textit{poly-} is a function that takes a scalar predicate \textit{P} and an individual argument \textit{x} and returns True iff there exists a degree \textit{d} such that \textit{x P} above the degree \cnst{little}.

\begin{exe}
\ex\label{gia:ex41}
\sib{poly-}${}=\lambda P\lambda x.\exists d [P (x) (d) \wedge (d > \cnst{little})]$
\end{exe}

\noindent The verbal complex \textit{polydhiavazi} ‘much-studied’ has the following denotation:

\begin{exe}
\ex\label{gia:ex42}
\sib{polydhiavazi}${}= \lambda x.\exists d[\textsc{study}(d)(x) \wedge (d > \cnst{little})]$
\end{exe}

\noindent Finally, given the denotation in \REF{gia:ex42}, and assuming the same denotations for definite nouns in \REF{gia:ex36} and negation in \REF{gia:ex39}, the compositional semantics of the sentence in \REF{gia:ex33b} proceeds by function application and \textit{β-}reduction as follows:

\begin{exe}
\ex\label{gia:ex43}
\sib{S}${}= \neg \exists d[\textsc{study}(\iota x [\textsc{student}(x)]) (d) \wedge (d > \cnst{little})]$
\end{exe}

\noindent Given that the sentence combines with the negative operator, the direction of the degree of the bound modifier \textit{poly-} changes and the degree maps to a value equal to \cnst{a little} on a scale like the one I provided in \REF{gia:ex34}.

Therefore, my analysis derives the correct meaning for the Greek degree modifiers \textit{poly} and \textit{poly-}. The boundedness of the latter is captured not only syntactically, as seen in \sectref{gia:sub:structure}, but also semantically with the denotations I proposed.

\section{Conclusion} \label{gia:sec:conclusion}

In this paper I presented a syntactic analysis for the licensing of the Greek NPI \textit{poly-} ‘much’, whereas the difference in meaning between the free degree modifier \textit{poly} and the bound degree modifier \textit{poly-} is captured semantically. My analysis made use of the (non)veridicality theory of polarity (\citealt{giannakidou1994licensing,giannaki1997dissert,giannaki1998} et seq.). Based on that, I have shown that, while its free counterpart, the degree modifier \textit{poly} ‘much/ a lot’, exhibits no restricted distribution, the bound element \textit{poly-} ‘much’ shows polarity behavior belonging to the category of strong NPIs only being licensed by antiveridical operators.

To answer the question of its polarity-sensitive behavior, I argued that the bound \textit{poly-} is associated with super strong licensing, i.e., it is licensed locally by an antiveridical operator. I claimed that its licensing is an Agree relation between its formal uninterpretable [uNeg] feature and the interpretable [Neg] feature of the antiveridical operator. In contrast, given that the free \textit{poly} does not have a [uNeg] feature, it does not need to be licensed by negation, and thus, can appear in both negative and affirmative environments. Moreover, the syntactic analysis I proposed illustrates the operation of Head Movement that \textit{poly-} needs to be attached to the verb stem. With respect to the second research question of this paper, i.e., the difference in meaning between \textit{poly} and \textit{poly-}, I provided distinct semantic denotations for each element indicating that the value of the NPI \textit{poly-} is mapped to the lowest values on a degree scale.

\section*{Abbreviations}

\begin{tabularx}{.5\textwidth}{@{}lX@{}}
\textsc{1/2/3}&1st/2nd/3rd person\\
\textsc{cat}&category\\
\textsc{imp}&imperative\\
\textsc{infl}&inflection\\
\end{tabularx}%
\begin{tabularx}{.5\textwidth}{@{}lX@{}}
\textsc{pst}&{past tense}\\
\textsc{sbjv}&subjunctive\\
\textsc{sel}&selection\\
\textsc{sg}&singular\\
\end{tabularx}


\section*{Acknowledgements}

I wish to thank Anastasia Giannakidou, Jason Merchant, Stephanie Solt, Hedde Zeijlstra, Erik Zyman, Nikos Angelopoulos, Carlos Cisneros, Natalia Pavlou, Ye\-nan Sun, as well as the audience at SinFonIJA 12 (Masaryk University), for discussion and comments. All errors are solely my responsibility.

{\sloppy\printbibliography[heading=subbibliography,notkeyword=this]}

\end{document}
