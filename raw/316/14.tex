\documentclass[output=paper]{langscibook} 
\ChapterDOI{10.5281/zenodo.5082476}

\author{Yuta Tatsumi\affiliation{University of Connecticut}}
\title{The architecture of complex cardinals in relation to numeral classifiers}  
\abstract{This paper investigates properties of multiplicative and additive complex cardinals in several languages. The starting point in the discussion will be recent work by \citet{IoninMatushansky2018}, who show that complex cardinals are not primitive units without complex structure. This paper observes some data that are problematic for their analysis. Based on the data, I argue that in multiplicative complex cardinals, a multiplicand is a syntactic head used for measurement and a multiplier is a phrase appearing in the specifier position of the phrase headed by the multiplicand. Building on the proposed analysis of multiplicative complex cardinals, I further argue that additive complex cardinals can have a non-coordinate structure in some languages, in addition to the coordination structure proposed by \citet{IoninMatushansky2018}. I propose that in non-coordinate additive complex cardinals, which do not include a coordinator syntactically, a lower-valued cardinal is an adjunct to a higher-valued cardinal.

\keywords{multiplicative complex cardinals, additive complex cardinals, numeral classifiers, left-branch extraction, nominal ellipsis, split topicalization}}



\begin{document}
\SetupAffiliations{mark style=none}
\maketitle


\section{Introduction}\label{tat:sec:intro}
This paper investigates two types of complex cardinals: \textsc{multiplicative complex cardinals} like \REF{tat:ex:ex1a} and \textsc{additive complex cardinals} like \REF{tat:ex:ex1b}.\footnote{In this paper, I use quotation marks for number concepts and italics for numerical expressions. For instance, \textit{three} denotes \doublequotes{three} in English.}%$^,$\footnote{Unless indicated otherwise, examples are Serbo-Croatian.}

\ea
\ea\label{tat:ex:ex1a}
\gll Ivan je pozvao \minsp{[} tri stotine] studenata.\\
     Ivan is invited {} three hundred student.\textsc{gen.pl.m}\\\hfill Multiplicative
\glt `Ivan invited three hundred students.'\hfill (Serbo-Croatian)
\ex\label{tat:ex:ex1b}
 \gll Ivan je vidio \minsp{[} dvadeset (i) pet] studenata.\\
      Ivan is seen {} twenty and five students.\textsc{gen.pl.m}\\\hfill Additive
 \glt `Ivan saw twenty five students.'\hfill (Serbo-Croatian)
 \z
\z

\noindent In \REF{tat:ex:ex1a}, the numeral \doublequotes{three} functions as a \textsc{multiplier} and \doublequotes{hundred} as a \textsc{multiplicand}. In \REF{tat:ex:ex1b}, the augend (\doublequotes{twenty}) appears with the addend (\doublequotes{five}).

\citet{IoninMatushansky2018} argue that multiplicative complex cardinals have the cascading structure represented in \REF{tat:ex:cas}. 

\ea\label{tat:ex:cas}
{[}\textsubscript{NP} three [\textsubscript{NP} hundred [\textsubscript{NP} student ] ] ]\hfill \citep{IoninMatushansky2018}
\z

\noindent Building on their analysis, this paper argues that multiplicative complex cardinals can also have a non-cascading structure in some languages. 

Regarding additive complex cardinals, \citeauthor{IoninMatushansky2018} pursue an analysis in which additive complex cardinals have an NP coordination structure. According to their analysis, additive complex cardinals are derived by deletion of a noun phrase, as in \REF{tat:cord}. This analysis is supported by the fact that additive complex cardinals can include an overt coordinator in some languages, as shown in \REF{tat:ex:ex1b}. 

\ea
\ea three hundred three girls
\ex\label{tat:cord} {[}\textsubscript{\&P} [\textsubscript{NP} three [\textsubscript{NP} hundred [\textsubscript{NP} \textst{girls}]] \& [\textsubscript{NP} three [\textsubscript{NP} girls]] ]\\\hfill \citep{IoninMatushansky2018}
\z\z

\noindent Although I follow \citet{IoninMatushansky2018} regarding the existence of the coordinate structure of additive complex cardinals, I argue in this paper that in addition to the coordinate structure as in \REF{tat:cord}, additive complex cardinals can also have a non-coordinate structure. Specifically, I propose that a lower-valued cardinal (\doublequotes{three} in \doublequotes{three hundred three}) can directly adjoin to a higher-valued cardinal (\doublequotes{three hundred} in \doublequotes{three hundred three}). The major motivation for the existence of the non-coordinate structure comes from the human classifier \textit{ri} in Japanese and contracted forms of Chinese cardinals.

The paper is organized as follows. In \sectref{tat:sec:mul.const}, I provide data which pose problems for \citeposst{IoninMatushansky2018} analysis. \sectref{tat:sec:prop} presents an analysis which can capture the data discussed in \sectref{tat:sec:mul.const}. \sectref{tat:sec:add} shows that the proposed analysis of multiplicative complex cardinals is compatible with \citeauthor{IoninMatushansky2018}'s analysis of additive complex cardinals. Moreover, I argue that in addition to the coordinate structure proposed by \citeauthor{IoninMatushansky2018}, additive complex cardinals can also have a non-coordinate structure in some languages. \sectref{tat:sec:sumsum} is the conclusion.

%=========================================================================
\section{Multiplicative complex cardinals and constituency tests}\label{tat:sec:mul.const}
In a cascading structure like \REF{tat:ex:cas}, the multiplicand and the main noun form a constituent to the exclusion of the multiplier. According to this analysis, a multiplicative complex cardinal should not behave as a single constituent since there is no syntactic constituent which directly corresponds to a multiplicative complex cardinal. However, I will show in this section that this prediction is not borne out, by investigating two types of split constructions; left-branch extraction and split topicalization. 

%=========================================================================
\subsection{Left-branch extraction}\label{tat:sec:lbe}
Some languages such as Latin and most Slavic languages allow movement of the leftmost constituent of an NP \citep{Ross1986}. Sentences in \REF{tat:lbe} are examples of \textsc{left-branch extraction (lbe)} in Serbo-Croatian, taken from \citet{Boskovic2005}.

\ea\label{tat:lbe}
\ea\label{tat:ex:lbea}
\gll Ta\textsubscript{$1$} je vidio [$\Delta$\textsubscript{$1$} kola].\\  
     that is seen {} car\\ 
\glt `That car, he saw.'\hfill (Serbo-Croatian)

\ex\label{tat:ex:lbeb}
\gll Lijepe\textsubscript{$1$} je vidio [$\Delta$\textsubscript{$1$} kuće].\\  
     beautiful is seen {} houses\\ 
\glt `Beautiful houses, he saw.'\hfill (Serbo-Croatian)
\z\z	

\noindent What is important is that in Serbo-Croatian, a multiplicative complex cardinal can undergo LBE, as shown in \REF{tat:lbe2}.

\ea
\ea
\gll Ivan je pozvao \minsp{[} {tri} {stotine} studenata].\\  
     Ivan is invited {} three hundred.\textsc{acc.f} students.\textsc{gen.m}\\ 
\glt `Ivan invited three hundred students.'\hfill (Serbo-Croatian)

\ex\label{tat:lbe2}
\gll \minsp{[} {Tri} {stotine}]\textsubscript{$1$} je Ivan pozvao [$\Delta$\textsubscript{$1$} studenata].\\  
     {} three hundred.\textsc{acc.f} is Ivan invited {} students.\textsc{gen.m}\\ 
\glt `Three hundred students, Ivan invited.'\hfill (Serbo-Croatian)
\z\z	

\noindent Following \citet{Corver1992}, I assume that LBE can be applied only to a phrasal constituent. Given this, the acceptability of \REF{tat:lbe2} shows that a multiplier and a multiplicand can form a phrasal constituent, excluding the main noun. Notice also that \textit{je} in \REF{tat:lbe2} is a second position clitic; as such it can follow only one constituent (see \citealt{Boskovic2001} and references therein). The presence of \textit{je} in \REF{tat:lbe2} then also indicates that \REF{tat:lbe2} is not derived by multiple LBE, where \textit{tri} and \textit{stotine} would undergo LBE separately.

One may consider that \REF{tat:lbe2} involves NP fronting and scattered deletion (cf. \citealt{FanselowCavar2002}). However, it has been argued that LBE and the scattered deletion construction behave differently in some respects. As discussed in \citet{Boskovic2014}, one of the main characteristics of the scattered deletion construction is that the remnant must be backgrounded and left in situ as in \REF{tat:ex:bos2}. As shown in \REF{tat:ex:bos3}, this is not the case with LBE.

\ea\label{tat:ex:bos2}{\textit{NP-fronting + Scattered deletion}}
\ea[?*]{
\gll \minsp{[} Onu \v{z}utu] mu kuću pokazuje.\\  
     {} that yellow him house is-showing\\ }
\ex[]{
\gll \minsp{[} Onu \v{z}utu] mu pokazuje kuću.\\  
     {} that yellow him is-showing house\\ }
\z
\glt `He is showing him that yellow house.'\\
\hfill (Serbo-Croatian; \citealt[421]{Boskovic2014})
\ex\label{tat:ex:bos3}{\textit{Left-branch extraction}}
\ea[]{
\gll \minsp{[} \v{Z}utu] mu kuću pokazuje.\\  
     {} yellow him house is-showing\\ }
\ex[]{
\gll \minsp{[} \v{Z}utu] mu pokazuje kuću.\\  
     {} yellow him is-showing house\\ }
\z
\glt `He is showing him the yellow house.'\\
\hfill (Serbo-Croatian; \citealt[421]{Boskovic2014})
\z

\noindent \REF{tat:lbe2} patterns with LBE in this respect. As shown in \REF{tat:bos4}, the remnant main noun can appear in the pre-verbal position. \REF{tat:lbe2} thus should not be analyzed as a scattered deletion construction.

\ea\label{tat:bos4}
\gll \minsp{[} Tri stotine] je Ivan studenata pozvao.\\  
     {} three hundred.\textsc{acc.f} is Ivan students.\textsc{gen.m} invited\\ 
\glt `Three hundred students, Ivan invited.'\\
\hfill(Serbo-Croatian; \v{Z}eljko Bo\v{s}kovi\'c, p.c.)
\z

\noindent One may also argue that \REF{tat:lbe2} is derived by movement of the main noun out of the complex cardinal expression followed by movement of the remnant phrase. However, if this kind of remnant movement were available in Serbo-Croatian, it is not clear why \REF{tat:mul.adj.lbe} is unacceptable. 

\ea[*]{\label{tat:mul.adj.lbe}
\gll  Visoke lijepe je on vidio [$\Delta$ $\Delta$ djevojke\textnormal{]}.\\  
      tall beautiful is he watches {} {} girls\\ 
\glt `He is watching tall beautiful girls.'\hfill (Serbo-Croatian; \citealt[2]{Boskovic2005})}
\z

\noindent Attributive adjectives can undergo LBE in Serbo-Croatian, as shown in \REF{tat:ex:lbeb}. However, when a noun is modified by two attributive adjectives, LBE of the two adjectives is impossible as in \REF{tat:mul.adj.lbe} \citep{Boskovic2005}. The contrast between \REF{tat:lbe2} and \REF{tat:mul.adj.lbe} is not expected under the remnant movement analysis. (For arguments against the remnant movement analysis of LBE more generally, see \citealt{Boskovic2005,Stjepanovic2010, Stjepanovic2011,Despic2011,Talic2017}, and references therein.)

Given these considerations, I conclude that the fronted multiplicative complex cardinal in \REF{tat:lbe2} must be a single phrasal constituent. The acceptability of \REF{tat:lbe2} then raises a problem for the cascading structure in \REF{tat:ex:cas} advanced by \citet{IoninMatushansky2018}, in which multiplicative complex cardinals cannot be the target of a syntactic operation as a single constituent.

%=============================================================================
\subsection{Nominal ellipsis}\label{tat:sec:ne}
Nominal ellipsis also provides an argument against \citeposst{IoninMatushansky2018} cascading structure. In \REF{tat:elipb} and \REF{tat:elipc}, the second sentence has an elided part.

\ea\label{tat:elip}
\ea\label{tat:elipa}
\gll Juan tomó seis cientas fotos, y Maria tomó tres cientas fotos.\\  
     Juan took six hundred pictures and Maria took three hundred pictures\\ 
\glt `Juan took 600 pictures, and Maria took 300 pictures.'

\ex\label{tat:elipb}
\gll Juan tomó seis cientas fotos, y Maria tomó tres cientas.\\  
    Juan took six hundred pictures and Maria took three hundred\\ 
\glt `Juan took 600 pictures, and Maria took 300 pictures.'

\ex\label{tat:elipc}
\gll Juan tomó seis cientas fotos, y Maria tomó tres.\\  
    Juan took six hundred pictures and Maria took three\\ 
\glt Unavailable: `Juan took 600 pictures, and Maria took 300 pictures.'\\
Available: `Juan took 600 pictures, and Maria took 3 pictures.'\\
\hfill (Spanish; Gabriel Martínez Vera, p.c.)
\z
\z

\noindent The elided part in \REF{tat:elipb} can receive the same interpretation as the one in \REF{tat:elipa}. On the other hand, the ellipsis in \REF{tat:elipc} cannot mean \quotes{three hundred pictures}. Instead, it is interpreted as \quotes{three pictures}. The contrast between \REF{tat:elipb} and \REF{tat:elipc} is unexpected under \citeauthor{IoninMatushansky2018}'s analysis, because the cascading structure in \REF{tat:ex:cas2} should be available for the multiplicative complex cardinals in \REF{tat:elip}.

\ea\label{tat:ex:cas2}
{[}\textsubscript{NP} three [\textsubscript{NP} hundred [\textsubscript{NP} pictures ] ] ]\hfill\citep{IoninMatushansky2018}
\z

\noindent Under their analysis, the ellipsis in \REF{tat:elipb} can be derived from the structure in \REF{tat:ex:cas2} by deleting the main NP (\textit{fotos} `pictures'). However, we may then also expect that the same deletion operation can be applied to the intermediate NP consisting of the multiplicand and the main NP, resulting in the ellipsis in \REF{tat:elipc}. This in fact is possible for adjectives in Serbo-Croatian. In \REF{tat:inter.ellipsis}, the object noun phrase in the second sentence is interpreted as \quotes{a small, square table}.  

\ea\label{tat:inter.ellipsis}
\gll Ivan je kupio veliki \v{c}etvrtasti sto, a Petar je kupio mali $\Delta$.\\
   Ivan is bought big square table and Peter is bought small {}\\
\glt `Ivan is bought a big square table and Peter is bought a small, square table.' \hfill(Serbo-Croatian; Željko Bošković, p.c.)
\z

\noindent Given these data, it seems to me that \citet{IoninMatushansky2018} need an account for the fact that the ellipsis in \REF{tat:elipc} cannot mean \quotes{three hundred pictures}.\footnote{I have examined the data regarding nominal ellipsis in English. Some of my consultants found that although there is a contrast between \REF{tat:ellib} and \REF{tat:ellic}, it is not completely impossible for \textit{two} in \REF{tat:elli} to be interpreted as \quotes{two hundred books}. \citet[338]{IoninMatushansky2006} also reported a similar observation in a footnote. 

\ea\label{tat:elli}
\ea John read three hundred books, but Mary read [ two hundred books ].
\ex\label{tat:ellib} John read three hundred books, but Mary read [ two hundred ].
\ex\label{tat:ellic} John read three hundred books, but Mary read [ two ].
\z\z

\noindent This suggests that at least for some speakers, English multiplicative complex cardinals have the cascading structure as in \REF{tat:ex:cas2}. I leave this issue for future research.}

%=============================================================================
\subsection{Split topicalization}\label{tat:sec:top}
Another potential problem for the cascading structure in \REF{tat:ex:cas} comes from split topicalization in German. As shown in \REF{tat:topc}, the main noun alone can undergo split topicalization, while leaving a multiplicative complex cardinal in situ. However, the main noun and a multiplicand cannot move together, leaving a multiplier in situ, as shown in \REF{tat:topd}.
 
\ea\label{tat:top}
\ea[]{
\gll Hans kaufte \minsp{[} {acht} {tausend} B\"{u}cher].\\  
     Hans bought {} eight thousand books\\ }
\ex[]{\label{tat:topb}
\gll \minsp{[} {Acht} {tausend} B\"{u}cher]\textsubscript{$1$} kaufte Hans $\Delta$\textsubscript{$1$}\\  
     {} eight thousand books bought Hans \\ }
\ex[]{\label{tat:topc}
\gll B\"{u}cher\textsubscript{$1$} kaufte Hans \minsp{[} {acht} {tausend} $\Delta$\textsubscript{$1$}]\\  
      books bought Hans {} eight thousand\\ }
\ex[*]{\label{tat:topd}
\gll \minsp{[} {Tausend} B\"{u}cher]\textsubscript{$1$} kaufte Hans \minsp{[} {acht} $\Delta$\textsubscript{$1$}]\\  
     {} thousand books bought Hans {} eight\\ }
\z
\glt (Intended:) `Hans bought eight thousand books.' 
     \\\hfill(German; Sabine Laszakovits, p.c.)
 \z

\noindent Split topicalization in German has received close attention in the literature (\citealt{Riemsdijk1989,FanselowCavar2002,vanHoof2006,Ott2011, Ott2015}, among others). The problem here is that the unacceptability of \REF{tat:topd} seems to be unexpected under \citeauthor{IoninMatushansky2018}'s analysis, regardless of the details of the analysis of split topicalization. Under \citeauthor{IoninMatushansky2018}'s analysis, the object phrase in \REF{tat:top} has the structure in \REF{tat:ex:cas3}.

\ea\label{tat:ex:cas3}
{[}\textsubscript{NP} eight [\textsubscript{NP} thousand [\textsubscript{NP} books ] ] ]\hfill\citep{IoninMatushansky2018}
\z

\noindent The acceptability of \REF{tat:topb} and \REF{tat:topc} shows that either the topmost NP in \REF{tat:ex:cas3} or the lowest NP (i.e. the main noun) can be a target of topicalization in German. We may then expect that the intermediate NP in \REF{tat:ex:cas3} can also undergo topicalization. (It should also be noted that \citeauthor{IoninMatushansky2018} propose that both multipliers and multiplicands are of type $\stb{\stb{e,t},\stb{e,t}}$.) It is not clear how to account for the unacceptability of \REF{tat:topd} under \citeauthor{IoninMatushansky2018}'s analysis.



%======================================================
%======================================================
%======================================================
\section{Proposal}\label{tat:sec:prop}
In \sectref{tat:sec:mul.const}, I showed that \citeauthor{IoninMatushansky2018}'s cascading structure faces some problems. To solve the problems, I pursue an analysis in which multiplicative complex cardinals can in principle have two structures cross-linguistically.

First, I propose that multiplicands are syntactic heads used for measurement whereas multipliers are phrases appearing in the specifier position of a phrase headed by the multiplicand, cross-linguistically. The noun phrase \textit{three hundred students} in English has the structure given in \figref{tat:comp} under the present analysis. What is important is that multipliers and multiplicands are syntactically different from each other. 

\begin{figure}[h]
\centering
    \begin{forest}
    for tree={s sep=1cm, inner sep=0, l=0}
    [XP [YP, name=three [three, roof] ]
        [X$'$  [X [hundred, name=hundred] ] [NP, [students, roof] ] ] ]
    % \draw[->] (trace-DP) to[out=south west, in=south west, looseness=1.2] (what);
    % \draw[->] (trace-T) to[out=south west, in=south west] (are);
    \end{forest}
    \caption{Complementation structure}
    \label{tat:comp}
\end{figure}

In \figref{tat:comp}, the multiplicand is a syntactic head taking the main NP as the complement. Structurally, \figref{tat:comp} is similar, at least in spirit, to \citeposst{IoninMatushansky2018} analysis given in \REF{tat:ex:cas} in the sense that a multiplicand takes the main NP as its complement. However, the present analysis departs from \citeauthor{IoninMatushansky2018}'s analysis with regard to the syntactic status of multipliers and multiplicands. I propose that multipliers are phrases whereas multiplicands are heads in multiplicative complex cardinals, cross-linguistically.

\begin{sloppypar}
Regarding semantics, I propose that multipliers are of type $n$, as in \REF{tat:mulp}, whereas multiplicands such as \textit{hundred} are of type $\stb{\stb{e,t},\stb{n,\stb{e,t}}}$, as in \REF{tat:mulc}.\footnote{In this respect, the proposed analysis is similar to a series of works by \citet{Rothstein2013, Rothstein2017}, where multipliers and multiplicands have different semantic types. However, the present analysis is also different from \citeauthor{Rothstein2013}'s analysis in several crucial aspects. For instance, Rothstein assumes that multiplicands are of type $\stb{n,\stb{e,t}}$, not $\stb{\stb{e,t},\stb{n,\stb{e,t}}}$. Moreover, my proposal given in \REF{tat:mulc} does not include any arithmetic functions such as $\times$, unlike \citeauthor{Rothstein2013}'s. \citeauthor{IoninMatushansky2018} argue against \citeauthor{Rothstein2013}'s assumption regarding the presence of arithmetic functions in semantics. However, this issue does not arise under the current analysis.} A multiplicand used in multiplicative complex cardinals includes a measurement function $\mu$. The denotation of the multiplicand \doublequotes{hundred} is given in \REF{tat:mulc}.\end{sloppypar}

\ea
\ea\label{tat:mulp} \sib{three} = 3
\ex\label{tat:mulc} \sib{hundred}\\
= $\lambda P.\lambda n.\lambda x.\exists S.[\Pi(S)(x) \wedge \mu(x)=n$\\ \hspace*{\fill}${}\wedge \forall y \in S.[
\underbrace{\vert\{z: z\leq_{\cnst{at}}y\}\vert = 100}_{\text{cardinality restriction}}
{}\wedge\forall z\leq_{\cnst{at}}y.[P(z)]]]$
% \begin{tikzpicture}[overlay]
%         \node at ($(x) + (-42mm,-4mm)$) {Cardinality Restriction};
% \end{tikzpicture}
\z
\z

\noindent Following \citeauthor{IoninMatushansky2018}, I make use of the cover $S$ and the partition function $\Pi$ defined in \REF{tat:pi}, to prevent multiple counting of the same members of $S$. In addition, multiplicands have a restriction on the cardinality of the set of atomic individuals in the cover $S$.

\ea\label{tat:pi}
$\Pi(S)(x)$ is true iff \hfill (\citealt[13]{IoninMatushansky2018})
\ea $S$ is a \textit{cover} of $x$, and
\ex $\forall z, y \in S [z = y \vee \neg\exists a [a \leq_i z \wedge a \leq_i y]]$ 
\z
\z

\noindent The topmost XP in \figref{tat:comp} has the denotation in \REF{tat:xp}. 

\ea\label{tat:xp}
\sib{[\textsubscript{XP} three hundred students ]}\\
= $\lambda x.\exists S.[\Pi(S)(x) \wedge\mu(x) = 3$\\ 
\hspace*{\fill}${}\wedge\forall y\in S.[\vert\{z: z\leq_{\cnst{at}}y\}\vert = 100\wedge\forall z\leq_{\cnst{at}}y.[\textsc{student}(z)]]]$
\z

\noindent What is important is that the current proposal is different from \citeauthor{IoninMatushansky2018}'s analysis in that the former assumes that multipliers and multiplicands are different syntactically and semantically.

Recall that in \sectref{tat:sec:mul.const}, I showed that the acceptability of LBE of a multiplicative complex cardinal is not expected under \citeauthor{IoninMatushansky2018}'s analysis. To solve the problems, I propose that multiplicative complex cardinals can occur in the adjunction structure as represented in \figref{tat:adjunct}, in addition to \figref{tat:comp}. 

\begin{figure}[h]
\centering
\begin{forest}
    for tree={s sep=1cm, inner sep=0, l=0}
    [NP [XP [YP [three, roof] ]
        [X$'$  [X [hundred] ] [NP, [\textsc{number}, roof] ] ] ] [NP [students, roof] ] ]
     \end{forest}
    \caption{Adjunction structure}
    \label{tat:adjunct}
\end{figure}

In \figref{tat:adjunct}, the multiplicand takes the silent \textsc{number} as the complement, instead of an overt common noun like \textit{students} (see \citealt{Kayne2005a} and \citealt{Zweig2006} for an independent argument for the presence of the silent numerical noun). However, the structural relation between the multiplier and the multiplicand is the same as in \figref{tat:comp}. The multiplier occurs in the specifier position of the phrase head by the multiplicand. 

With regard to the semantics, I assume that the silent \textsc{number} is interpreted as a property of being a number (i.e. $\lambda x[\textsc{number}(x)]$). The topmost XP in \figref{tat:adjunct} has the following denotation.

\ea\label{tat:sem.adj}
\sib{[\textsubscript{XP} three hundred \textsc{number} ]}\\
= $\lambda x.\exists S.[\Pi(S)(x) \wedge \mu(x) = 3$\\
\hspace*{\fill}${}\wedge\forall y\in S.[\vert\{z: z\leq_{\cnst{at}}y\}\vert = 100 \wedge \forall z\leq_{\cnst{at}}y.[\textsc{number}(z)]]]$
\z

\noindent Following \citet{Rothstein2013, Rothstein2017}, I assume that the topmost XP in \figref{tat:adjunct} can be converted into a singular term of type $n$ by the \textsuperscript{$\cap$} function \citep{Chierchia1985}. In \REF{tat:sem.adj}, each atomic individual of S has the property of being a number. When the \textsuperscript{$\cap$} function applies, the topmost XP, which is of type $\stb{e,t}$, becomes a numerical expression of type $n$ as in \REF{tat:sem.multipl.fin}.\footnote{When the XP including the silent \textsc{number} is modified by the \textsuperscript{$\cap$} function, it functions as a numerical expression of type n. Therefore, the multiplicative complex cardinal \textit{three hundred} can be used as a multiplier, combining with another multiplicand as in (i).

\ea\label{tat:3000}
\ea {[}\textsubscript{X1P} \textsuperscript{$\cap$}[\textsubscript{X2P} three [\textsubscript{X2$'$} hundred \textsc{number} ]] [\textsubscript{X1$'$} thousand students ]]
\ex
\sib{three hundred thousand students}\\
= $\lambda x.\exists S.[\Pi(S)(x) \wedge \mu(x) = 300$\\
\hspace*{\fill}$\wedge \forall y \in S.[\vert\{z: z\leq_{\cnst{at}}y\}\vert = 1000 \wedge \forall z\leq_{\cnst{at}}y.[\textsc{student}(z)]]]$
\z\z
}

\ea\label{tat:sem.multipl.fin}
\sib{\textsuperscript{$\cap$}XP} = 300
\z

\noindent In order to modify a noun phrase, cardinals of type $n$ need the covert measurement function $\epsilon$ defined as in \REF{tat:ep}.\footnote{The covert function $\epsilon$ is also used when a noun phrase is modified by a numerical expression in the absence of a multiplicand. For instance, the denotation of \textit{three students} is given in \REF{tat:simple}. (See \citealt{Scontras2014} (\textsc{card}) and \citealt{Champollion2017} (\textsc{many}) for a similar covert element in the numeral construction.)

\ea
\ea {[} [\textsubscript{YP} three] [ $\epsilon$ [\textsubscript{NP} students]]]
\ex\label{tat:simple} \sib{three $\epsilon$ students} \\
$\lambda x.\exists S.[\Pi(S)(x) \wedge \mu(x) = 3 \wedge \forall y \in S.[\vert\{z: z\leq_{\cnst{at}}y\}\vert = 1 \wedge \forall z\leq_{\cnst{at}}y.[\textsc{student}(z)]]]$
\z\z

\noindent Note also that the covert function $\epsilon$ must be unavailable in obligatory classifier languages, where classifiers are generally indispensable in numerical expressions. I speculate in this paper that the existence of numeral classifiers blocks the covert function $\epsilon$ in obligatory classifier languages. (See \citealt{Chierchia1998} for a similar blocking effect.)
}

\ea
\ea\label{tat:ep}
\sib{$\epsilon$} \\
= $\lambda P.\lambda x.\exists S.[\Pi(S)(x) \wedge \mu(x) = n$\\
\hspace*{\fill}$\wedge \forall y \in S.[\vert\{z: z\leq_{\cnst{at}}y\}\vert = 1 \wedge \forall z\leq_{\cnst{at}}y.[P(z)]]]$

\ex\label{tat:ep2}
\sib{[\textnormal{[\textsubscript{XP}} three hundred \textnormal{\textsc{number}]} \textnormal{[}$\epsilon$ students]]}\\
 = $\lambda x.\exists S.[\Pi(S)(x) \wedge \mu(x) = 300$\\
\hspace*{\fill}$\wedge \forall y \in S.[\vert\{z: z\leq_{\cnst{at}}y\}\vert = 1 \wedge \forall z\leq_{\cnst{at}}y.[\textsc{student}(z)]]]$
\z\z

\noindent Although the denotation in \REF{tat:ep2} is different from the one in \REF{tat:xp}, they denote the same set; a set of students whose cardinality is \doublequotes{three hundred} in total. Importantly, the topmost XP in \figref{tat:adjunct} can be the target of syntactic operations such as LBE, while keeping the main noun intact, as discussed below.


%=========================================================================
\subsection{Left-branch extraction}\label{tat:sec:lbe.prop}
The acceptability of \REF{tat:lbe2}, repeated here as \REF{tat:ex:LBE.1.rep}, in which a multiplicative complex cardinal undergoes LBE, can be captured under the proposed analysis. 

\ea\label{tat:ex:LBE.1.rep}
\gll \minsp{[} {Tri} {stotine}]\textsubscript{$1$} je Ivan pozvao [$\Delta$\textsubscript{$1$} studenata].\\  
    {} 3 100.\textsc{acc.f} is Ivan invited {} students.\textsc{gen.m}\\ 
\glt `Three hundred students, Ivan invited.'\hfill(Serbo-Croatian)
\z	

\noindent Under the current analysis, the multiplicative complex cardinal in \REF{tat:ex:LBE.1.rep} can be an adjunct to the main NP, as represented in \REF{tat:ex:st.LBE.SC} (cf. \figref{tat:adjunct}).

\ea\label{tat:ex:st.LBE.SC}
{[}\textsubscript{NP} \textsuperscript{$\cap$}[\textsubscript{XP} three hundred ] [\textsubscript{NP} students ]]
\z

\noindent The XP in \REF{tat:ex:st.LBE.SC} can undergo LBE, while leaving the main noun in situ. 

%=========================================================================
\subsection{Nominal ellipsis}\label{tat:sec:ne.prop}

The current analysis can also account for the (im)possible interpretations of elliptical examples. The crucial example is repeated here as \REF{tat:elli.rep}.

\ea\label{tat:elli.rep}
\gll Juan tomó seis cientas fotos, y Maria tomó tres.\\  
    Juan took six hundred pictures and Maria took three\\ 
\glt Unavailable: `Juan took 600 pictures, and Maria took 300 pictures.'\\
Available: `Juan took 600 pictures, and Maria took 3 pictures.' \hfill(Spanish)
\z

\noindent What is important is that the elided part in \REF{tat:elli.rep} cannot be interpreted as \quotes{three hundred pictures}. The current proposal can capture the interpretation of the elliptical example in \REF{tat:elli.rep}. The structure of the object phrases in \REF{tat:elli.rep} is represented in \REF{tat:eli.st} (cf. \figref{tat:comp}). 

\ea\label{tat:eli.st}
{[}\textsubscript{XP} three [\textsubscript{X$'$} hundred [\textsubscript{NP} pictures ] ] ]
\z

\noindent The elliptical example in \REF{tat:elli.rep} cannot be derived from the structure in \REF{tat:eli.st} because there is no phrasal constituent that can undergo ellipsis in \REF{tat:eli.st}, to the exclusion of the multiplier \doublequotes{three}.\footnote{I assume that X$'$-level cannot be a target of ellipsis.} The present analysis can thus capture the fact that the elliptical part in \REF{tat:elli.rep} cannot mean \quotes{three hundred pictures}.

%=============================================================================
\subsection{Split topicalization}\label{tat:sec:top.prop}
The data about split topicalization in German can also be captured under the current analysis. What is problematic for \citeauthor{IoninMatushansky2018}'s analysis is the unacceptability of \REF{tat:topd}, repeated here as \REF{tat:top.rep}.

\ea[*]{\label{tat:top.rep}
\gll \minsp{[} {Tausend} B\"{u}cher]\textsubscript{$1$} kaufte Hans \minsp{[} {acht} $\Delta$\textsubscript{$1$}]\\  
     {} thousand books bought Hans {} eight\\ 
      \glt Intended: `Hans bought eight thousand books.'\hfill(German)}
 \z

The contrast in question is expected by assuming that the multiplicative complex cardinal in \REF{tat:top.rep} has the structure given in \REF{tat:top.st} underlyingly.

\ea\label{tat:top.st}
{[}\textsubscript{XP} eight [\textsubscript{X$'$} thousand [\textsubscript{NP} books ] ] ]
\z

\noindent The NP \textit{B\"{u}cher} can be a target of split topicalization because it is a phrasal constituent. On the other hand, the constituent composed of the multiplicand and the main noun cannot be a target of topicalization because it is not a phrasal projection.

It is worth noting here that numeral classifiers in Mandarin and Vietnamese behave like multiplicands in German regarding leftward movement, as shown in \REF{tat:chi.top} and \REF{tat:viet.top}.

\ea\label{tat:chi.top}
\ea[]{
\gll Qiang mai le \minsp{[} {wu} {tiao} xianglian\textnormal{]}.\\  
     Qiang buy \textsc{asp} {} five \textsc{cls} necklace\\ }

\ex[]{
\gll xianglian\textsubscript{$1$} Qiang mai le \minsp{[} {wu} {tiao} $\Delta$\textsubscript{$1$}\textnormal{]}.\\  
     necklace Qiang buy \textsc{asp} {} five \textsc{cls} {}\\ }

\ex[*]{
\gll \textnormal{[}{tiao} xianglian\textnormal{]}\textsubscript{$1$} Qiang mai le \minsp{[} {wu} $\Delta$\textsubscript{$1$}\textnormal{]}.\\  
     \hspace*{1mm}\textsc{cls} necklace Qiang buy \textsc{asp} {} five {}\\ }
\z
\glt (Intended:) `Qiang bought five necklaces.'\hfill(Mandarin; Shengyun Gu, p.c.)
\ex\label{tat:viet.top}
\ea[]{
\gll Khanh mua \minsp{[} {n\v{a}m} {cuón} sách\textnormal{]}.\\  
     Khanh bought {} five \textsc{cls} book\\ }

\ex[]{
\gll sách\textnormal{\textsubscript{1}} Khanh mua \minsp{[} {n\v{a}m} {cuón} $\Delta$\textnormal{\textsubscript{1}}\textnormal{]}.\\  
     book Khanh bought {} five \textsc{cls}\\ }

\ex[*]{
\gll {[cu\'{\^{o}}n} sách\textnormal{]\textsubscript{1}} Khanh mua \minsp{[}    {n\v{a}m} $\Delta$\textnormal{\textsubscript{1}}\textnormal{]}.\\  
     \hspace*{1mm}\textsc{cls} book Khanh bought {} five\\}
\z
\glt (Intended:) `Khanh bought five books.'\hfill(Vietnamese; Thuy Bui, p.c.)
\z

\noindent As shown in the b-examples of \REF{tat:chi.top} and \REF{tat:viet.top}, the main noun moves to the sentence initial position, while leaving the cardinal and the numeral classifier in situ. However, it is impossible to move the numeral classifier and the main noun together, as in the c-examples in these classifier langugages. 

The current analysis can capture the similarity between numeral classifiers and multiplicands in German. \citet{HuangOchi2014} propose that Chinese numeral classifiers project their own phrases, taking a noun phrase as its complement. I assume that the classifier phrases in Chinese and Vietnamese have the complementation structure given in \REF{tat:ho}.\footnote{See however \citet{Nguyen2004} for a different analysis of classifier phrases in Vietnamese. See also \citet{Zhang2013} and references therein for a detailed syntactic analysis of Chinese numeral classifier phrases.}

\ea\label{tat:ho}
{[}\textsubscript{XP} five [\textsubscript{X$'$} [\textsubscript{X} \textsc{cls} ] [\textsubscript{NP} ... ] ] ]
\z

\noindent The c-examples in \REF{tat:chi.top} and \REF{tat:viet.top} are unacceptable because the non-maximal projection (i.e. X$'$) cannot be a target of the relevant movement, similarly to split topicalization in German. 

One piece of supporting evidence for the structure in \REF{tat:ho} comes from the fact that it is impossible to move a cardinal and a numeral classifier while leaving the main noun in situ, as shown in \REF{tat:ex:chi.bad} and \REF{tat:ex:viet.bad}.

\ea[*]{\label{tat:ex:chi.bad}
\gll {[wu} {tiao}]\textsubscript{1} Qiang mai le [$\Delta$\textsubscript{1} xianglian].\\  
     {\hspace{1mm}five} \textsc{cls} Qiang buy \textsc{asp} {} necklace\\ 
\glt Intended: `Qiang bought three necklaces.'\hfill(Mandarin; Shengyun Gu, p.c)}
\ex[*]{\label{tat:ex:viet.bad}
\gll {[n\v{a}m} {cuón}]\textsubscript{1} Khanh mua [$\Delta$\textsubscript{1} sách].\\  
     {\hspace{1mm}five} \textsc{cls} Khanh bought {} book\\ 
\glt Intended: `Khanh bought five books.'\hfill(Vietnamese; Thuy Bui, p.c.)}
\z

\noindent The unacceptability of \REF{tat:ex:chi.bad} and \REF{tat:ex:viet.bad} follows from the current analysis. They are unacceptable because there is no constituent composed of the cardinal and the classifier to the exclusion of the NP in \REF{tat:ho}. Notice that multiplicative complex cardinal in German cannot undergo split topicalization while leaving the main noun in situ, as in \REF{tat:top.ger}.

\ea[*]{\label{tat:top.ger}
\gll {[Acht} {tausend}]\textsubscript{$1$} kaufte Hans [$\Delta$\textsubscript{$1$} B\"{u}cher].\\  
     {\hspace{1mm}eight} thousand bought Hans {} books\\ 
     \glt Intended: `Hans bought eight thousand books.'\\\hfill(German; Sabine Laszakovits, p.c.)}
 \z

\noindent The unacceptability of \REF{tat:top.ger} indicates that multiplicative complex cardinals in German do not appear in the adjunction structure as in \figref{tat:adjunct}.

It should be noted here that it is possible to front a cardinal and a numeral classifier together in some classifier languages such as Ch'ol and Japanese, as shown in \REF{tat:ex:chol} and \REF{tat:ex:japa}. 

\ea
\ea
\gll Ta' jul-i-y-ob \minsp{[} {ux-tyikil} x'ixik]\textsubscript{$1$}.\\  
     \textsc{pfv} arrive-\textsc{itv-ep-pl} {} three-\textsc{cls} woman\\ 
\glt `Three women arrived.'

\ex\label{tat:ex:chol}
\gll \minsp{[} {Ux-tyikil}]\textsubscript{$1$} ta' jul-i-y-ob [$\Delta$\textsubscript{$1$} x'ixik].\\  
     {} three-\textsc{cls} \textsc{pfv} arrive-\textsc{itv-ep-pl} {} woman\\ 
\glt `[Three]\textsubscript{foc} women arrived.' \hfill(Ch'ol; \citealt[19]{Baleetal2019})
\z\ex
\ea
\gll kyoositsu-ni \minsp{[} zyosei {san-nin}]-ga toochaku-sita .\\  
     classroom-\textsc{loc} {} woman three-\textsc{cls}-\textsc{nom} arrive-\textsc{did} \\ 
\glt `Three women arrived at the classroom.'

\ex\label{tat:ex:japa}
\gll \minsp{[} {san-nin}]\textsubscript{$1$} kyoositu-ni \minsp{[} zyosei $\Delta$\textsubscript{$1$}]-ga toochaku-sita .\\  
     {} three-\textsc{cls} classroom-\textsc{loc} {} woman {}\hspace*{5mm}-\textsc{nom} arrive-\textsc{did}\\ 
\glt `[Three]\textsubscript{foc} women arrived at the classroom.'\hfill(Japanese)
\z\z

\noindent Following \citet{HuangOchi2014}, I assume that there are in principle two structures for numeral classifier phrases; the complementation structure as in \REF{tat:ho} and the adjunction structure as in \REF{tat:ho2}.\footnote{See \sectref{tat:sec:add} for further references and discussion regarding Japanese numeral classifiers in relation to additive complex cardinals.}

\ea\label{tat:ho2}
[\textsubscript{NP} [\textsubscript{XP} \textsc{three} [\textsubscript{X} \textsc{cls} ] ] [\textsubscript{NP} ... ] ]
\z

\noindent I take the acceptability of \REF{tat:ex:chol} and \REF{tat:ex:japa} as evidence that numeral classifier phrases in these languages make use of the adjunction structure in \REF{tat:ho2}. The XP in \REF{tat:ho2} can be a target of the relevant movement operation, similarly to LBE in Serbo-Croatian.

%=============================================================================
\subsection{Section summary}\label{tat:sec:sum}
In the present paper, I assume that the two structures are in principle available for multiplicative complex cardinals; the complementation structure \figref{tat:comp} and the adjunction structure \figref{tat:adjunct}. The current analysis differs from \citeposst{IoninMatushansky2018} analysis regarding the treatment of multipliers and multiplicands. I have proposed in this section that multiplicands are syntactic heads used for measurement, whereas multipliers are phrases appearing in the specifier position of the phrase headed by a multiplicand. In addition, I have shown some similarities and differences between multiplicands and numeral classifiers, on the basis of the data about topicalization and fronting. The cross-linguistic data are summarized in \tabref{tat:tab:1:frequencies}.

\begin{table}
\caption{Multiplicative complex cardinals \& numeral classifier phrases}
\label{tat:tab:1:frequencies}
 \begin{tabular}{lll}
  \lsptoprule
            & multiplicands & numeral classifiers\\
  \midrule
  complementation  & German &  Mandarin Chinese, Vietnamese\\
  adjunction  &   Serbo-Croatian &  Ch'ol, Japanese\\
  \lspbottomrule
 \end{tabular}
\end{table}

Building on the proposed analysis of multiplicative complex cardinals, I will investigate additive complex cardinals in the next section. 

%=============================================================================
\section{Additive complex cardinals}\label{tat:sec:add}
In this section, I discuss \citeauthor{IoninMatushansky2018}'s treatment of additive complex cardinals, showing that the proposed analysis of multiplicative complex cardinals is compatible with their analysis of additive complex cardinals. \citeauthor{IoninMatushansky2018} pursue an analysis in which additive complex cardinals have an NP coordination structure. According to their analysis, additive complex cardinals are derived by deletion of a noun phrase, as in \REF{tat:IandM.additive.structure2}.

\ea\label{tat:IandM.additive.structure2}
\ea three hundred three girls \hfill \citep{IoninMatushansky2018}
\ex {[}\textsubscript{\&P} [\textsubscript{NP} three [\textsubscript{NP} hundred [\textsubscript{NP} \textst{girls}]] \& [\textsubscript{NP} three [\textsubscript{NP} girls]]] 
\z\z

\noindent The current analysis of multiplicative complex cardinals is compatible with the coordination analysis of additive complex cardinals. For instance, \textit{three hundred three students} has the coordinate structure given in \figref{tat:my.coord}.


\begin{figure}[h]
\centering
    \begin{forest}
    for tree={s sep=1cm, inner sep=0, l=0}
    [\&P [X1P [YP [three, roof] ] [X1$'$ [X1 [hundred ] ] [\sout{NP1} [\sout{student}, roof] ] ] ] [\&$'$ [\& ] [X2P [YP [three, roof] ] [X2$'$ [X2 [$\epsilon$ ] ] [NP2 [student, roof] ] ] ] ] ]
    \end{forest}
    \caption{Coordinate structure under the present analysis}
    \label{tat:my.coord}
\end{figure}

The first conjunct in \figref{tat:my.coord} is headed by the multiplicand \textit{hundred}, and the X1P has the complementation structure of multiplicative complex cardinals. In the second conjunct (X2P), the simplex cardinal \textit{three} appears in the specifier of X2P. Recall that the covert function $\epsilon$ is used for simplex cardinals in non-classifier languages, as in \figref{tat:my.coord}. 

Although I follow \citet{IoninMatushansky2018} regarding the existence of the coordinate structure of additive complex cardinals, I argue in this section that in addition to the coordinate structure as in \REF{tat:IandM.additive.structure2}, additive complex cardinals can also have a non-coordinate structure. Specifically, I propose that a lower-valued cardinal (\doublequotes{three} in \doublequotes{three hundred three}) can directly adjoin to a higher-valued cardinal (\doublequotes{three hundred} in \doublequotes{three hundred three}). The major motivation for the existence of the non-coordinate structure comes from the human classifier \textit{ri} in Japanese and contracted forms of Chinese cardinals.



%==========================================================================
\subsection{The human classifier \textit{ri} in Japanese}\label{tat:sec:add.ja}\largerpage
Firstly, I consider human classifiers in Japanese. Japanese is an obligatory classifier language, and cardinals must co-occur with an appropriate classifier to modify a noun phrase. Japanese has two classifiers for common nouns referring to human beings; \textit{nin} and \textit{ri}. Crucially, the classifier \textit{ri} has a contextual restriction regarding the type of a cardinal it combines with. It co-occurs with the native Japanese cardinals \textit{hito} `one' and \textit{huta} `two' as in \REF{tat:ex:jap.cls.selection.1.2}, but not with the Sino-Japanese cardinals \textit{ichi} `one' and \textit{ni} `two', as shown in \REF{tat:ex:jap.cls.selection.1.2}.


\ea
\ea\label{tat:ex:jap.cls.selection.1.2}
\gll\minsp{\{} hito / huta\textnormal{\}}-ri-no gakusei\\  
      {} one {} two-\textsc{cls-gen} student\\ 
\glt `\{one/two\} student(s)'

\ex
\gll \minsp{\{*} ichi / \minsp{*} ni\textnormal{\}}-ri-no gakusei\\  
      {} one {} {} two-\textsc{cls-gen} student\\ 
\glt `\{one/two\} student(s)' \hfill(Japanese)
\z\z

\noindent I assume that the noun phrase in \REF{tat:ex:jap.cls.selection.1.2} has the adjunction structure as in \REF{tat:jap.cls} (cf.~\ref{tat:ho2} in \sectref{tat:sec:top.prop}).\footnote{See \citet{SaitoMurasugi1990} and \citet{HuangOchi2014} for the adjunct status of pre-nominal classifier phrases in Japanese.}

\ea\label{tat:jap.cls}
[\textsubscript{NP} [\textsubscript{XP} \{one / two\} [\textsubscript{X} \textsc{cls} ] ] [\textsubscript{NP} student ] ]
\z

\noindent In Japanese, when a nominal modifier precedes a noun phrase, the genitive linker \textit{no} intervenes between the pre-nominal modifier and the noun phrase (e.g. \textit{gen\-go\-gaku-no gakusei} `students of linguistics', lit. linguistics-\textsc{gen} student). Following \citet{KitagawaRoss1982}, and \citet{Watanabe2006}, I assume that the genitive linker \textit{no} is inserted, post-syntactically. 

I propose that the classifier \textit{ri} is selected as an exponent of the classifier head when the human classifier head is a sister of \textit{hito} or \textit{huta}. In \REF{tat:jap.cls}, the cardinal is a sister of Cls and the relevant contextual restriction is satisfied.

Crucially, the contextual restriction is violated when a cardinal occurs in an additive complex cardinal, as in \REF{tat:sec.ada}. In this environment, the classifier \textit{nin}, which is the elsewhere exponent of the classifier head dedicated to human beings \citep{Watanabe2010}, must be used together with the Sino-Japanese cardinals, as shown in \REF{tat:sec.adb}.

\ea\label{tat:ex:jap.cls.selection.additive}
\ea\label{tat:sec.ada}
\gll \minsp{[} yon zyuu \minsp{\{*} hito / \minsp{*} huta\textnormal{\}]}-ri-no gakusei\\  
     {} four ten {} one {} {} two-\textsc{cls-gen} student\\ 
\glt `forty \{one / two\} students'
\ex\label{tat:sec.adb}
\gll \minsp{[} yon zyuu \minsp{\{} ichi / ni\textnormal{\}]}-nin-no gakusei\\  
     {} four ten {} one {} two-\textsc{cls-gen} student\\ 
\glt `forty \{one / two\} students'\hfill(Japanese)
\z\z

\noindent The coordination analysis predicts that the additive complex cardinal in \REF{tat:sec.ada} includes the structure in \REF{tat:jap.cls} as the second conjunct of the coordinate structure. Therefore, the coordination analysis does not expect the contrast between \REF{tat:ex:jap.cls.selection.1.2} and \REF{tat:sec.ada}. 

However, if a non-coordinate structure is available for Japanese additive complex cardinals, the contrast can be accounted for. Specifically, I propose that \REF{tat:sec.ada} has the non-coordinate structure as in \REF{tat:ex:jap.additive.structure}.

\ea\label{tat:ex:jap.additive.structure}
Non-coordinate additive complex cardinal
\\{[}\textsubscript{NP} [\textsubscript{X2P} \fbox{[[\textsubscript{XP} four [\textsubscript{X$'$} ten \textsc{number} ]] \{one / two\}]} [\textsubscript{X2} \textsc{cls} ]] [\textsubscript{NP} ... ] ]
\z

\noindent In \REF{tat:ex:jap.additive.structure}, the lower-valued cardinal (i.e. \{one / two\}) combines directly with the higher-valued cardinal (i.e. XP), which includes the silent \textsc{number}. The lower-valued cardinal is not a sister of the classifier, and the relevant contextual restriction cannot be satisfied in \REF{tat:ex:jap.additive.structure}. This problem does not arise when \textit{hito} and \textit{huta} do not occur in complex cardinals. In the non-complex cardinal construction, a cardinal is a sister of the classifier head and nothing intervenes between them, as shown in \REF{tat:jap.cls}. The contrast between \REF{tat:ex:jap.cls.selection.1.2} and \REF{tat:sec.ada} can thus be accounted for by assuming the non-coordinate structure of additive complex cardinals.

It should be noted here that it seems that Japanese additive complex cardinals can have the coordinate structure in some cases. As shown in \REF{tat:ex:jap.additive.and}, Japanese additive complex cardinals can contain the overt coordinator \textit{to} `and' (\citet{Hiraiwa2016ms}). What is important is that the contextual restriction of the classifier \textit{ri} is respected in the presence of \textit{to}. 

\ea\label{tat:ex:jap.additive.and}
\gll \textnormal{[} yon zyuu to \minsp{\{} hito / huta\textnormal{\}]}-ri-no gakusei\\  
      {} four ten and {} one {} two-\textsc{cls-gen} student\\ 
\glt `forty and \{one / two\} students'\hfill(Japanese)
\z

\noindent I assume that when an additive complex cardinal contains the overt coordinator, it has the coordinate structure as in \REF{tat:ex:jap.coord.st} (see \figref{tat:my.coord}).

\ea\label{tat:ex:jap.coord.st} 
{[}\textsubscript{\&P} [\textsubscript{X1P} four [\textsubscript{X1'} [\textsubscript{X1} ten] \sout{NP} ]] \& [\textsubscript{NP} [\textsubscript{X2P} \{one / two\} [\textsubscript{X2} \textsc{cls} ]] student]]
\z

\noindent In \REF{tat:ex:jap.coord.st}, the lower-valued cardinal is a sister of the classifier head in the second conjunct. The contextual restriction is therefore satisfied in \REF{tat:ex:jap.coord.st}. (The Japanese conjunctive particle \textit{to} appears between two nominal conjuncts, e.g. \textit{Yuta to Hiro} `Yuta and Hiro'.)

\citet{IoninMatushansky2018} propose that additive complex cardinals generally involve coordinate structures, and a coordinator can be overtly realized in some languages. In fact, the presence/absence of an overt coordinator seems to be superficial in some languages such as Serbo-Croatian (see \ref{tat:ex:ex1b}). However, I showed in this section that Japanese additive complex cardinals have different structures, according to the presence/absence of an overt coordinator, which makes a significant difference regarding morphosyntactic behaviors.

%=========================================================================
\subsection{Contracted forms in Mandarin Chinese}\label{tat:sec:add.ma}
Contracted forms of Chinese cardinals also offer supporting evidence for the existence of non-coordinate additive complex cardinals. Chinese is an obligatory classifier language, and a cardinal must appear with an appropriate classifier when it modifies a noun. Mandarin Chinese has a contracted form consisting of \textit{san} `three' and the general classifier \textit{ge}; \textit{sa}, as shown in \REF{tat:sa}.\footnote{\textit{liang} `two' also has a contracted form; \textit{lia}. Since \textit{lia} behaves like \textit{sa}, I use examples with \textit{sa} in this paper.} 

\ea\label{tat:ex:chi.contraction.simple}
\ea\label{tat:}
\gll san-ge xuesheng\\  
      three-\textsc{cls} student\\ 
\glt `three students'

\ex\label{tat:sa}
\gll sa xuesheng\\  
      three.\textsc{cls} student\\ 
\glt  `three students'\hfill(Mandarin)
\z\z


\noindent However, as observed by \citet{He2015}, the contracted form cannot appear in additive complex cardinals, as in \REF{tat:ex:chi.contraction.mul}. 


\ea\label{tat:ex:chi.contraction.mul}
\ea[]{
\gll \minsp{[} si-shi san]-ge xuesheng\\  
     {} four-ten three-\textsc{cls} student\\ 
\glt `forty three students'}
\ex[*]{
\gll\minsp{[} si-shi sa] xuesheng\\  
     {} four-ten three.\textsc{cls} student\\ 
\glt `forty three students'\hfill(Mandarin)}
\z\z


\noindent I propose that additive complex cardinals in Mandarin Chinese have the non-coordinate structure. First, let us consider the simplex cardinal in \REF{tat:ex:chi.contraction.simple}. I assume that the nouns in \REF{tat:ex:chi.contraction.simple} have the structure represented in \REF{tat:ex:chi.contraction.st1}.\footnote{For a detailed syntactic analysis of Chinese classifier phrases, see \citet{Zhang2013}, \citet{HuangOchi2014} and references therein.} Here, the numeral \doublequotes{three} appears in SpecXP headed by the numeral classifier \textit{ge} (cf. \ref{tat:ho}).

\ea\label{tat:ex:chi.contraction.st1}
{[}\textsubscript{XP} three [\textsubscript{X'} [\textsubscript{X} ge ] [\textsubscript{NP} student ] ] ]
\z

\noindent Suppose that \textit{san} `three' and the classifier \textit{ge} can be fused only when they are in a Spec-Head relation. In \REF{tat:ex:chi.contraction.st1}, they can then undergo morphological fusion without any problems. 

On the other hand, when \textit{san} `three' appears inside an additive complex cardinal, \textit{sishi} `forty' and \textit{san} `three' form a constituent, resulting in the non-co\-or\-di\-nate structure in \REF{tat:ex:chi.contraction.st2}.\footnote{This line of approach is also taken taken by \citet{He2015}. However, the details are different from the current analysis. For instance, I assume that a higher-valued cardinal includes the silent \textsc{number} based on my analysis of multiplicative complex cardinals.}

\ea\label{tat:ex:chi.contraction.st2}
Non-coordinate additive complex cardinal
\\{[}\textsubscript{X2P} \fbox{[[\textsubscript{XP} four [\textsubscript{X$'$} ten \textsc{number} ]] three]} [\textsubscript{X2$'$} [\textsubscript{X2} \textsc{cls} ] [\textsubscript{NP} student ]]]
\z

\noindent In \REF{tat:ex:chi.contraction.st2}, \textit{san} `three' adjoins directly to XP, which contains the silent \textsc{number}. In this case, morphological fusion cannot take place because \textit{san} and \textit{ge} are not in a Spec-Head relation. The non-coordinate structure can thus account for the unavailability of a contracted form in Mandarin Chinese, similarly to the Japanese data discussed in \sectref{tat:sec:add.ja}. 

It should be noted here that the coordinate structure of additive complex cardinals should be unavailable in Mandarin Chinese. If the coordinate structure as in \REF{tat:ex:chi.IandM.structure} were available in Mandarin Chinese additive complex cardinals, the numeral \doublequotes{three} and the general classifier \textit{ge} would be able to undergo morphological fusion, contrary to the fact.

\ea\label{tat:ex:chi.IandM.structure}
{[}\textsubscript{\&P} [\textsubscript{X1P} four [\textsubscript{X1$'$} [\textsubscript{X1} ten] \sout{NP} ]] \& [\textsubscript{X2P} three [\textsubscript{X2$'$} [\textsubscript{X2} \textsc{cls} ] student]]]
\z

\noindent In fact, additive complex cardinals in Mandarin Chinese do not allow the presence of an overt coordinator, as in \REF{tat:ex:chi.overt.coord}, in contrast to Japanese additive complex cardinals (cf. \ref{tat:ex:jap.additive.and}). 

\ea\label{tat:ex:chi.overt.coord}
\gll \llap{*}si-shi he san-ge xuesheng\\  
      four-ten and three-\textsc{cls} student\\ 
\glt `forty three students'\hfill(Mandarin)
\z

\noindent The unacceptability of \REF{tat:ex:chi.overt.coord} indicates that the coordinate structure of additive complex cardinals is unavailable in Chinese.\footnote{There are certain cardinals that cannot occur in complex cardinals, cross-linguistically. \citeauthor{IoninMatushansky2018} discuss Polish examples in Chapter 6 and 7. \citet{Hurford2003} observes that in German, the non-agreeing counting form \textit{eins} `one' must be used in compounding cardinals like \doublequotes{one hundred one}, instead of \textit{ein} `one', which agrees with the main noun. He also reports that the presence of an overt coordinator changes the agreement pattern (e.g. *\textit{hundert eine Frau(en)} vs. \textit{hundert und eine Frauen}, p. 616). A similar pattern is observed in Mandarin Chinese. Mandarin has two forms of the cardinal \doublequotes{two}; \textit{liang} and \textit{er}. However, \textit{liang} cannot be used in additive complex cardinals (e.g. *\textit{si-shi liang-ge xuesheng} `forty two students', lit. `four-ten two-\textsc{cls} student', vs. \textit{liang-ge xuesheng} `two students', lit. `two-\textsc{cls} student'). I thank an anonymous reviewer for bringing this point to my attention.}


%=============================================================================
\section{Summary}\label{tat:sec:sumsum}
This paper examined properties of complex cardinals in several languages, in order to determine what kind of cascading structure is available for numerical expressions cross-linguistically. I focused on multiplicative complex cardinals and additive complex cardinals. 

I argued that in multiplicative complex cardinals, a multiplicand is a syntactic head used for measurement and a multiplier is a phrase appearing in the specifier position of the phrase headed by the multiplicand. Moreover, I proposed that multiplicands and numeral classifiers can in principle appear in the two different structures: the complementation structure and the adjunction structure.

Based on the proposed analysis of multiplicative complex cardinals, I argued that additive complex cardinals can have the non-coordinate structure in some languages such as Japanese and Chinese, in addition to the coordination structure proposed by \citet{IoninMatushansky2018}. In non-coordinate additive complex cardinals, which do not include a coordinator syntactically, a lower-valued cardinal is an adjunct to a higher-valued cardinal.


%=============================================================================
%=============================================================================
% Just uncomment the input below when you're ready to go.
%\input{example-osl.tex}

\section*{Abbreviations}

\begin{tabularx}{.5\textwidth}{@{}lX@{}}
\textsc{acc}&{accusative}\\
\textsc{asp}&{aspect}\\
\textsc{cls}&{classifier}\\
\textsc{f}&feminine\\
\end{tabularx}%
\begin{tabularx}{.5\textwidth}{@{}lX@{}}
\textsc{gen}&genitive\\
\textsc{m}&masculine\\
\textsc{nom}&{nominative}\\
\textsc{pl}&plural\\
\end{tabularx}

\section*{Acknowledgements}
I would like to thank Thuy Bui, Shengyun Gu, Ivana Jovović, Sabine Laszakovits, Aida Talić, Gabriel Martínez Vera, Shuyan Wang, Ting Xu, Muyi Yang, Xuetong Yuan and the audience at SinFonIja 12 for their comments and data. Special thanks also go to \v{Z}eljko Bo\v{s}kovi\'c for his helpful comments and discussion. Final thanks to two anonymous reviewers for their thoughtful comments on the earlier draft of this paper. Examples not attributed to any source are from my consultants and all errors in this paper are mine.

{\sloppy\printbibliography[heading=subbibliography,notkeyword=this]}


\end{document}
