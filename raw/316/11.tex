\documentclass[output=paper]{langscibook} 
\ChapterDOI{10.5281/zenodo.5082470}

\author{Nina Haslinger\affiliation{Georg-August-Universität Göttingen} and Viola Schmitt\affiliation{Humboldt-Universität zu Berlin}}
\title{Distinguishing belief objects}  
\abstract{The problem of intentional identity \citep{Geach:1967} has a counterpart that concerns the notion of \textsc{distinctness} for intentional objects. It arises when expressions linked to distinctness, like plurals or numerals, occur in the scope of intensional operators. Focussing on plurals in belief contexts that have a cumulative reading relative to a plural attitude subject, we argue for a notion of distinctness that appeals to the attitude subjects' counterfactual beliefs: two partial individual concepts count as sufficiently distinct if each attitude subject believes that if both were instantiated, they would yield different individuals. After providing a general paraphrase of cumulative belief sentences, we outline potential advantages of this approach over analyses of intentional identity that appeal to real-world ``causes'' of the intentional objects, or to notions of attitude content that are sensitive to discourse referents.

\keywords{intentional distinctness, plurals, attitudes, counterfactuals}}

\begin{document}
\maketitle



\section{Introduction} 

Some natural language expressions are sensitive to \textsc{identity} or \textsc{distinctness}. Pronouns, for instance, are linked to identity since they can be construed as co-varying with their antecedents: on one reading, \REF{sch-has:ex:1} says a witch blighted Bob's mare and \textit{that same witch} killed Cob's sow. Numerals and plurals are another class of such expressions: \REF{sch-has:ex:2} requires \textit{two distinct} monsters to roam the castle.

\ea
\ea\label{sch-has:ex:1} A witch blighted Bob's mare and she killed Cob's sow.
\ex\label{sch-has:ex:2} Two monsters were roaming the castle.
\z\z

\noindent In extensional contexts as above, the relevant notions of identity and distinctness seem to be based on pre-theoretically given relations between real-world objects.\footnote{Yet, in extensional contexts, certain plural and quantificational expressions are arguably sensitive to spatiotemporal configurations of the parts of an object \citep{Wagiel:2018}, which suggests that even there the notion \textsc{individual} should not be a primitive of semantic theory.} But \citet{Geach:1967} noted that the notion of identity becomes non-trivial  in certain cases of anaphoric relations in \textsc{intensional contexts}. To see the point, consider \REF{sch-has:ex:3}, where the pronoun and its potential antecedent \textit{a witch} are both embedded under attitude predicates -- each of which has a different subject.

\ea\label{sch-has:ex:3} Hob thinks a witch blighted Bob's mare, and Nob thinks she killed Cob's sow.  \hfill (\citealt[1, (1)]{Edelberg:1986}, adapted from \citealt[628, (3)]{Geach:1967}) \z

\eanoraggedright
\eanoraggedright\label{sch-has:ex:4} \textit{Scenario:} The newspaper reports that a witch called ``Sue'' has been blighting farm animals. There is no witch: the animals all died of natural causes.  Hob and Nob both read the newspaper and believe the stories about the witch. Hob thinks Sue blighted Bob's mare. Nob thinks Sue killed Cob's sow. (adapted from \citealt[2]{Edelberg:1986}) \phantom{.}\hfill \REF{sch-has:ex:3} \textsc{true}
\ex\label{sch-has:ex:5} \textit{Scenario:}  Hob and Nob each read newspaper articles about three wit\-ches. There are no witches. Hob believes one of the witches blighted Bob's mare, but has no idea which one. Bob believes one of the witches killed Cob's sow, but has no idea which one.  \hfill \REF{sch-has:ex:3} \textsc{not true}
\z\z

\noindent \citet{Geach:1967} observed that \REF{sch-has:ex:3} can be true in scenarios like \REF{sch-has:ex:4}, where there are no real-world witches. This raises the problem of how the anaphoric relation can be established at all, as the antecedent and the pronoun are hidden in the ``privacy'' of different belief contexts. But such sentences give rise to a second, related problem: the relevant reading is only possible if the object of Hob's belief can be ``identified'' with the object of Nob's belief. This is illustrated by the fact that \REF{sch-has:ex:3} is false in scenario  \REF{sch-has:ex:5} -- intuitively because, unlike in \REF{sch-has:ex:4}, we cannot be sure that Hob's and Nob's beliefs are about `the same witch'. The truth conditions of such examples thus depend on an identity relation, but in the absence of real-world witches, this relation must hold between belief objects or, more generally, intentional objects. The notion of an intentional object is further spelled out in \sectref{sch-has:sec:3}; here, we just note that an intentional object (i) picks out different individuals in different worlds and (ii) does not have to correspond to any individual in the actual world. \citeposst{Geach:1967} observation then raises the question of when two intentional objects are ``similar enough to count as one'' for semantic purposes.\largerpage[1.75]

This paper makes two points: first, we argue that Geach's puzzle is a special case of a more general problem that surfaces whenever the grammar requires a semantic identity or distinctness relation to hold between intentional objects associated with different intensional operators. This means (i) that this problem is not specific to anaphora and (ii) that apart from the question of when two intentional objects count as identical, we have to answer the potentially different question of when two intentional objects count as distinct. Second, concentrating on plurals in belief contexts, we develop a preliminary notion of distinctness based on the content of the attitude subjects' \textsc{counterfactual beliefs}. This distinctness relation does not appeal to discourse referents or real-world causes of the beliefs (often invoked for Geach's puzzle), which we argue is supported by the data.


\section{A more general problem}\largerpage[1.75]

We now show that the problem goes beyond Geach's original examples. First, it is not just identity between intentional objects that is truth-conditionally relevant, but also distinctness. Second, the puzzle extends to other intensional predicates and to non-pronominal DPs embedded under them. Thus, identity and distinctness between intentional objects play a systematic role in grammar.

\subsection{Plurals embedded under attitudes}\label{sch-has:sec:2.1}

Why is distinctness of intentional objects truth-conditionally relevant? \citet{Schmitt:2019a} notes that sentences like \REF{sch-has:ex:6}, where a plural is embedded under an attitude verb with a plural subject, can be true in scenarios like \REF{sch-has:ex:7}  (cf. \citealt{Pasternak:2018a} for similar data).\footnote{In both English and German, not all speakers accept this reading. This variation might be due to the fact that \REF{sch-has:ex:6} involves a cumulative relation across a finite clause boundary. Our claims here apply to varieties like our own, in which the cumulative reading is possible. For a general discussion of cumulative readings of non-individual-denoting expressions, see \citet{Schmitt:2019}.} Such sentences thus have a cumulative reading: neither Abe nor Bert believe that \textit{two} monsters were roaming the castle, but their beliefs ``add up'' to a belief about two monsters, in the same way that \REF{sch-has:ex:6a} is true in scenario \REF{sch-has:ex:7a} because the books Abe read and those Bert read add up to three. Moreover, as no actual monsters exist in the scenario, we face a problem very similar to that of anaphora across beliefs in Hob-Nob cases: cumulation -- the parallel ``adding up'' of pluralities -- must access objects hidden in different belief contexts -- the monster Abe `believes in' and the monster Bert `believes in'.

\eanoraggedright
\eanoraggedright \label{sch-has:ex:6} Abe and Bert  believed that two monsters were roaming the castle! % MW: sorry but boldface for highlighting in examples is not allowed :( 
  \ex\label{sch-has:ex:7} \textit{Scenario:} Abe believes in zombies, Bert in griffins. Neither exist. Both spent the night at Roy's castle.  Around midnight, Abe thought he heard a zombie in his room. A little later, Bert believed he saw a griffin  on his bed. They didn't discuss it with each other.  %but each took Roy aside and told him what she believed was going on. Roy tells me: \textit{ Well, I had invited Ada and Bea to the castle. Bad idea... I know it can be  a little spooky here, but ....} 
\hfill \REF{sch-has:ex:6} \textsc{\%true}
\z\ex
\ea \label{sch-has:ex:6a} Abe and Bert read three books. % MW: sorry but boldface for highlighting in examples is not allowed :( 
\ex \label{sch-has:ex:7a} \textit{Scenario:} Abe read books 1 and 2. Bert read book 3. \hfill \REF{sch-has:ex:6a} \textsc{true} 
\z\z

\noindent Note that the analogy between \REF{sch-has:ex:6} and \REF{sch-has:ex:6a} is not universally accepted: \citet{Pasternak:2018a} does not treat the relevant reading of \REF{sch-has:ex:6} as cumulative, rejecting the analogy with \REF{sch-has:ex:6a}. His basic idea is that Abe and Bert can \textsc{collectively believe} a proposition $p$ if the conjunction of Abe's relevant beliefs and Bert's relevant beliefs entails $p$, so that examples without plurals in the embedded clause should have analogous readings. This is correct for some of Pasternak's examples, but does not generalize \citep{Marty:2019, Schmitt:2019a}: \REF{sch-has:ex:fn2} is not true in scenario \REF{sch-has:ex:fn1} although Ada's and Bea's relevant beliefs jointly entail its embedded clause. Since collective belief in Pasternak's sense is thus subject to constraints that are not well understood, we will continue to assume a separate, plural-sensitive semantic mechanism in cases like \REF{sch-has:ex:6}.

\eanoraggedright
\eanoraggedright \label{sch-has:ex:fn1} \textit{Scenario:} Ada is looking forward to Sue's party: She believes every man at the party will fall in love with her. Bea is also looking forward to it: She hates men and is certain that only one man will attend: Roy. Sue tells me: `Ada and Bea are looking forward to the party\ldots'
\ex \label{sch-has:ex:fn2} They believe that Roy will fall in love with Ada. They are crazy! \\
\null \hfill \textsc{false} in \REF{sch-has:ex:fn1}
\z\z

\noindent As in the Hob-Nob case, the existence of this reading gives rise to second, related problem, namely how the constraints on this reading should be characterized. This is illustrated by the judgment that \REF{sch-has:ex:6} is not true in scenario \REF{sch-has:ex:8}: the reading just sketched is possible only if the monsters are intuitively ``different enough''. Pre-theoretically, we can be sure that we are talking about two different monsters in \REF{sch-has:ex:7}, but not in \REF{sch-has:ex:8}. 

\eanoraggedright\label{sch-has:ex:8} \textit{Scenario:}   (Roy's castle, no monsters\ldots) Around midnight, Abe thought it was 1 am and that he heard a monster in his room. A little later, Bert believed it was 2 am and that he heard a monster in his room. (They didn't discuss it\ldots) \z

\noindent Since monsters do not exist in either scenario, this intuitive distinctness relation must again hold between intentional objects.  Semantic theory therefore has to answer the question of when two intentional objects count as distinct.

\subsection{Plural objects of intensional transitive verbs}

The case of  \textsc{intensional transitive verbs (itv)}  like  German \textit{suchen} `look for' shows that 
 the puzzle in \REF{sch-has:ex:6} affects intensional contexts more generally and not just attitude complements. Indefinite objects of such verbs, like \textit{ein Gespenst} in \REF{sch-has:ex:12}, do not come with existential entailments, so ITV are usually assumed to take quantifier or property arguments (e.g.,~\citealt{Montague:1974,Zimmermann:1993}).\footnote{See, e.g.,~\citet{Schwarz:2021} and \citet{Deal:2008} for arguments that at least a certain subclass of ITV, including \textit{look for}, do not take covert sentential complements.} 

\ea \label{sch-has:ex:12} {\gll Abe hat in der Nacht ein Gespenst gesucht. \\
Abe has in the night a ghost sought \\
\glt `At night, Abe was looking for a ghost.' \hfill (German)}  \z

\noindent Indefinite \textit{plural} objects of \textit{suchen} can be in a cumulative relation with a plural subject even if they lack an existential entailment: \REF{sch-has:ex:10} is true in the cumulative scenario \REF{sch-has:ex:9}, where no ghosts exist. As with cumulative belief, the numeral is only licensed if the ghost Abe looked for is somehow ``distinct'' from the one Bert looked for: \REF{sch-has:ex:10} seems to be false in scenario \REF{sch-has:ex:11} since no further properties of the ghosts are specified and so we cannot conclude that Abe's and Bert's search goals are distinct. The contrast becomes even clearer with \textit{unterschiedlich} `different' \REF{sch-has:ex:17}.

\eanoraggedright \label{sch-has:ex:9ff}
\eanoraggedright \label{sch-has:ex:10} {\gll Abe und Bert haben nachts zwei Gespenster gesucht. \\
Abe and Bert have at.night two ghosts sought \\
\glt `At night, Abe and Bert were looking for two ghosts.'}
\ex \label{sch-has:ex:9} \textit{Scenario:} Last weekend, Abe and Bert stayed at Roy's castle. They both wrongly believe the castle is haunted by ghosts. At night, Abe went out to look for the ghost of its previous owner, who died in 1980. Bert looked for the ghost of its first owner, who died in 1400.\hfill \REF{sch-has:ex:10} \textsc{true}
\ex \label{sch-has:ex:11} \textit{Scenario:} (Roy's castle, no ghosts\ldots) At night, Abe went outside and tried to find some ghost of a previous owner of the castle (he doesn't care which one). Bert also went out to look for some ghost of a previous owner. \hfill \REF{sch-has:ex:10} \textsc{not true}
\z\ex \label{sch-has:ex:17} \gll Abe und Bert haben nachts zwei unterschiedliche Gespenster gesucht.\\
Abe and Bert have at.night two different ghosts sought\\
\glt `At night, Abe and Bert were looking for two different ghosts.' \hfill (German) \z

\noindent We can think of `the ghost Abe is looking for' as an intentional object that picks out a ghost in each world in which Abe's search is successful, but does not pick out anything in the evaluation world. If so, cumulativity in \REF{sch-has:ex:10} and \REF{sch-has:ex:17} is sensitive to a distinctness relation for intentional objects, just like cumulative belief.\footnote{ \citet{Condoravdi:2001} raise an analogous puzzle, arguing that \REF{sch-has:ex:13} has a reading on which three `specific' strikes were prevented. This could be true even if three other strikes occurred. 

\ea \label{sch-has:ex:13} Negotiations prevented three strikes. \hfill \citep[(2)]{Condoravdi:2001}\z

\noindent This raises the question of when potential strikes that did not occur count as distinct. Here, we focus on predicates of search for simplicity, as the downward-monotonicity of the most prominent reading of \textit{prevent} raises additional issues.}

\subsection{Relative clauses with intensional transitive verbs}

We considered two semantic phenomena that are sensitive to a notion of \textsc{intentional distinctness}. Neither involves anaphora, but semantic mechanisms motivated by anaphora -- particularly discourse referents -- underlie several accounts of the Hob-Nob puzzle (see \sectref{sch-has:sec:4.2}). This mismatch could lead to two  contrasting conclusions: (i) that cumulative sentences are unrelated to Geach's puzzle, or (ii)~that the connection between Geach's puzzle and discourse referents is less deep than commonly thought. We choose the latter option, based on the following observation: relative-clause constructions with the gap in the object position of an ITV, like \REF{sch-has:ex:14}, are sensitive to intentional identity (not distinctness!) in a way similar to Geach's puzzle, but do not involve discourse anaphora.

\eanoraggedright
\eanoraggedright \label{sch-has:ex:14} {\gll Abe hat nachts ein Gespenst gesucht, das Bert auch gesucht hat. \\
Abe has at-night a ghost sought \REL{} Bert also sought has \\
\glt `At night, Abe looked for a ghost that Bert also looked for.'} \hfill (German)
\ex \label{sch-has:ex:15} \textit{Scenario:} (Roy's castle, no ghosts\ldots) At night, Abe went outside to look for the ghost of the previous owner, who died in 1980. Independently, Bert (who Abe has never met) also went outside to look for the ghost of the previous owner\ldots \hfill \REF{sch-has:ex:14} \textsc{true}
\z\z

\noindent \REF{sch-has:ex:14} must have an intensional reading since it can be true in scenario \REF{sch-has:ex:15}. Yet this reading does not just require that Abe and Bert are each looking for a ghost, or that there is some property $P$ such that they each want to find a $P$ ghost: \REF{sch-has:ex:14} does not seem true in scenario \REF{sch-has:ex:11}, where Abe and Bert each want to find the ghost of some previous owner of the castle, but don't care which. Like anaphora in the Hob-Nob case, the construction in \REF{sch-has:ex:14} is only licensed if we are justified in ``identifying'' the ghost Abe looked for with the one Bert looked for. Abe's and Bert's searches must be directed towards intentional objects ``similar enough to count as one''.\footnote{The nature of the individuation problem in such relative-clause constructions depends on the DP. \citet{Zimmermann:2006} discusses examples like \REF{sch-has:ex:16} with the ``dummy noun'' \textit{-thing}, arguing they involve quantification over the ITV's property argument: \REF{sch-has:ex:16} roughly means there is some property $P$ such that Abe is trying to find an arbitrary $P$ and Bert is trying to find an arbitrary $P$. \citet{Haslinger:2019} argues this is correct for such ``higher-order DPs'' (\textit{something}, \textit{two things}), but not for DPs with lexical head nouns (\textit{a ghost}, \textit{two ghosts}): unlike \REF{sch-has:ex:14}, the German counterpart of \REF{sch-has:ex:16} is true in scenarios like \REF{sch-has:ex:11}, where the conditions for intentional identity are not met. This suggests that, while the relevant reading of \REF{sch-has:ex:14} is intensional, the DP quantifies over intentional objects picking out at most one individual per world, not over properties or kinds.

\ea \label{sch-has:ex:16} Abe was looking for something Bert was looking for (too). \z


} The contrast becomes even clearer with \textit{dasselbe} `the same':

\ea \label{sch-has:ex:18} {\gll Abe hat in der Nacht dasselbe Gespenst gesucht, das Bert gesucht hat. \\
Abe has in the night the.same ghost sought \REL{} Bert sought has \\
\glt `At night, Abe looked for the same ghost that Bert looked for.'} \hfill (German) \z

\noindent In sum, the Hob-Nob puzzle belongs to a broader class of configurations where semantic identity or distinctness relations required by certain expressions (plurals, numerals, anaphoric pronouns, relativization, \textit{same, different}, \ldots) cut across two intensional contexts with different subjects. The remainder of this paper concentrates on one special case -- cumulative belief sentences -- and gives a description quite different from existing analyses of the Hob-Nob puzzle. While it does not generalize straightforwardly to the intentional identity puzzles (\ref{sch-has:ex:3}, \ref{sch-has:ex:14}, \ref{sch-has:ex:18}), we hope it will serve as a first step towards a new unified analysis of the pattern.

\section{Distinctness in cumulative belief sentences}\label{sch-has:sec:3}

We will now develop a paraphrase of sentences like \REF{sch-has:ex:19} (=\ref{sch-has:ex:6}) under the reading discussed in \sectref{sch-has:sec:2.1}. Our starting point is a notion of cumulative belief that appeals to ``parts'' of the embedded proposition -- ``parts'' determined by distinct monster-concepts $f$, $g$. The difficulty is to specify when $f$ and $g$ count as distinct: properties the attitude subjects would consider relevant for individuation must be distinguished from those they would consider irrelevant. But this is hard to implement in a standard attitude semantics based on accessibility relations, as a subject can judge two monster-concepts as distinct without believing that they are both instantiated. We therefore take distinctness to involve counterfactual attitudes: for \REF{sch-has:ex:6}/\REF{sch-has:ex:19}, two monster-concepts $f$, $g$ count as distinct if both Abe and Bert believe that if both $f$ and $g$ existed, they would be distinct individuals.

\ea \label{sch-has:ex:19} Abe and Bert  believed that two monsters were roaming the castle! \z  % MW: sorry but boldface for highlighting in examples is not allowed :( 

%\ea \label{sch-has:ex:19a} Abe believes that $f$ is roaming the castle and Bert believes that $g$ is roaming the castle and both Abe and Bert believe that if both $f$ and $g$ existed, they would be distinct. \z

\subsection{Global incompatibility of belief states?}\label{sch-has:sec:3.1}

We first discuss a ``straw man'' proposal that will help clarify the truth conditions of cumulative belief sentences. One might think that the ``zombie vs.~griffin'' scenario \REF{sch-has:ex:7} makes \REF{sch-has:ex:6}/\REF{sch-has:ex:19} true because it suggests that Abe's relevant beliefs are incompatible, globally, with Bert's relevant beliefs. ``Relevant'' here is meant to ensure that conflicting beliefs unrelated to monsters (say, about the weather) do not license distinct belief objects (cf.~also \citealt{Pasternak:2018a}). This generalization faces two problems. First, incompatibility of the subjects' ``relevant'' beliefs is not necessary for distinctness: in scenario \REF{sch-has:ex:20}, a variant of \REF{sch-has:ex:7}, Abe's and Bert's beliefs are  compatible with a world where both a zombie and a griffin are at the castle. Yet, this does not make the cumulative reading of \REF{sch-has:ex:6}/\REF{sch-has:ex:19} less acceptable.\footnote{If one takes the relevant attitudes to be \textit{de se}, this issue might not arise as Abe does not self-ascribe the property of seeing a griffin in \REF{sch-has:ex:20} -- but our other arguments would still apply.} Further, a generalization based on global (in)compatibility of belief states predicts \REF{sch-has:ex:20} to pattern with the `1 am vs.~2 am' scenario \REF{sch-has:ex:21}, which seems incorrect.

\eanoraggedright
\eanoraggedright \label{sch-has:ex:20} \textit{Scenario:} (Roy's castle, no monsters\ldots) Around midnight, Abe thought he heard a zombie in his room. A little later, Bert believed he saw a griffin on his bed. Abe and Bert both consider it possible that both griffins and monsters are at the castle\ldots \hfill \REF{sch-has:ex:19} \%\textsc{true} 
\ex \label{sch-has:ex:21} \textit{Scenario:} (Roy's castle, no monsters\ldots) Around midnight, Abe thought it was 1 am and he heard a monster in his room. A little later, Bert believed it was 2 am and he heard a monster in his room. They both consider it possible that the monster they heard was roaming the castle all night\ldots \hfill \REF{sch-has:ex:19} \%\textsc{not true} \z\z

\noindent Second, incompatible beliefs are not sufficient for distinctness:  Abe's and Bert's beliefs are logically incompatible  in scenario \REF{sch-has:ex:22}, yet \REF{sch-has:ex:6}/\REF{sch-has:ex:19} is false. We might claim that beliefs about the total number of monsters are irrelevant, but then our problem would just be shifted to the problem of characterizing relevance.

\eanoraggedright\sloppy
\label{sch-has:ex:22} \textit{Scenario:} (Roy's castle, no monsters\ldots) Around midnight, Abe hears a strange sound. He believes there are exactly four monsters living in the area and concludes he must have heard one of them. A little later, Bert also hears a strange sound. He thinks there are five monsters living in the area and concludes it must be one of them. \hfill \REF{sch-has:ex:19} \textsc{false}
\z

\begin{sloppypar}
\noindent Such examples suggest the standard linguistic conception of belief contents, which relies on an accessibility relation, is not fine-grained enough. A  common response  -- notions of semantic content sensitive to discourse referents -- is addressed in \sectref{sch-has:sec:4}. Here, we will introduce a different conception of attitude contents that is richer than usually assumed, but still relies on possible worlds semantics. The puzzle posed by \REF{sch-has:ex:6}/\REF{sch-has:ex:19} then has two aspects, which we address in turn: what does it mean to have a cumulative belief ``about'' certain intentional objects? And how do we paraphrase distinctness without relying on the relation between Abe's and Bert's respective belief worlds in the way just described?
\end{sloppypar}

\subsection{Individual concepts and cumulative belief}\label{sch-has:sec:3.2}

We start by developing a general paraphrase for cumulative belief sentences of the type in \REF{sch-has:ex:23} (where $P$ is a distributive predicate) that simultaneously captures the cumulative relation between the higher DP and the plural indefinite and the non-extensional reading of the plural indefinite. In particular, we need to capture the fact that the relevant reading does not require NP to have a nonempty extension in the evaluation world.

\ea\label{sch-has:ex:23} DP believe that [[two NP] $P$] \z

\noindent The paraphrase relies on \citeposst{Schmitt:2019} semantics for plurals in intensional contexts. As suggested by the analogy between \REF{sch-has:ex:6}/\REF{sch-has:ex:19} and \REF{sch-has:ex:6a}, she generalizes \citeposst{Hintikka:1969} semantics for \textit{believe} to a cumulative relation between a plurality of individuals (\sib{DP}) and a plurality of propositions. The notion of cumulatively believing a plurality of propositions is independently motivated by cumulative readings of conjoined complement clauses, as in \REF{sch-has:ex:24}:

\eanoraggedright
\eanoraggedright \label{sch-has:ex:24} The Paris agency called and the one from Berlin. [\ldots] The agencies believe [$_p$ that Macron is considering resignation] and [$_q$ (that) Merkel is becoming paranoid], but neither had anything to say about Brexit.\\\null\hfill (adapted from \citealt[(18)]{Schmitt:2019})% adapted from \citep[(18)]{Schmitt:2019}  % MW: sorry but boldface for highlighting in examples is not allowed :( 
\ex \label{sch-has:ex:25} \textit{Scenario:} The Paris agency believes Macron might resign. The Berlin agency believes  Merkel is becoming paranoid.  \hfill \REF{sch-has:ex:24} \textsc{true}\z\z



\noindent Crucially, neither agency in scenario \REF{sch-has:ex:25} has to believe both conjuncts. From such data, \citet{Schmitt:2019} concludes that sentential conjunctions denote pluralities of propositions, which stand in a one-to-one correspondence to nonempty sets of propositions. The idea is that the set $A_{\stb{s, t}}$ of propositions in the usual sense -- partial functions from worlds to truth values -- is closed under a sum operation $\bigoplus_{\stb{s, t}}$ to form the full domain of atomic and plural propositions. $\bigoplus_{\stb{s, t}}$ maps any nonempty subset of $D_{\stb{s, t}}$ to its unique sum, in analogy to the operation $\bigoplus_{e}$ that sums up a set of individuals. Instead of giving a set-theoretic construction of $D_{\stb{s, t}}$, we simply assume that $D_{\stb{s, t}}$ must have the algebraic structure of the set $(\mathcal{P}(A_{\stb{s, t}}) \setminus \{\emptyset\}, \bigcup)$ of nonempty sets of propositions, with $\bigoplus_{\stb{s, t}}$ isomorphic to set union. Propositional conjunction denotes the binary counterpart $\oplus_{\stb{s, t}}$ of $\smash{\bigoplus_{\stb{s, t}}}$. For instance, for the propositions $p$ = \sib{Macron is considering resignation}, $q$ = \sib{Merkel is becoming paranoid} and $r$ = \sib{Brexit will be called off} in $A_{\stb{s, t}}$, we have $p \oplus_{\stb{s, t}} q = \bigoplus_{\stb{s, t}}(\{p, q\})$, the counterpart of $\{p, q\}$ in $D_{\stb{s, t}}$, and $(p \oplus_{\stb{s, t}} q) \oplus_{\stb{s, t}} r = \bigoplus_{\stb{s, t}}(\{p \oplus_{\stb{s, t}} q, r\}) = \bigoplus_{\stb{s, t}}(\{p, q, r\})$, the counterpart of $\{p, q, r\}$.\footnote{See \citet{Schmitt:2019a} for the technical details and more independent motivation.} The atomic parts of a propositional plurality are the elements of the set of atomic propositions it corresponds to; thus, if $\leq_a$ denotes the atomic-part relation, $p, q, r \leq_a p \oplus_{\stb{s, t}} q \oplus_{\stb{s, t}} r$, but $p \oplus_{\stb{s, t}} q \not\leq_a p \oplus_{\stb{s, t}} q \oplus_{\stb{s, t}} r$. This extended plural ontology now permits us to define cumulative belief:

\eanoraggedright \label{sch-has:ex:26} A (possibly plural) individual $x \in D_e$ \textsc{cumulatively believes} a (possibly plural) proposition $p \in D_{\stb{s, t}}$ in a world $w$ iff 
\ea for every $y \leq_a x$, there is a $q \leq_a p$ such that \sib{believe}$(w)(q)(y)$
\ex and for every $q \leq_a p$, there is a $y \leq_a x$ such that \sib{believe}$(w)(q)(y)$. \z\z

\noindent \REF{sch-has:ex:26} correctly predicts that in scenario \REF{sch-has:ex:25}, the agencies cumulatively believe $p \oplus_{\stb{s, t}}q$. But to apply this definition to our motivating example \REF{sch-has:ex:6}/\REF{sch-has:ex:19}, we need a way of deriving plural propositions from an embedded clause like \textit{two monsters are roaming the castle}. \citet{Schmitt:2019} outlines such a system; we just give the basic idea for the subcase where the predicate in the embedded clause is distributive. We adopt a simple formalization of intentional objects as partial individual concepts \REF{sch-has:ex:27}; e.g., \textit{two monsters} ranges over \sib{monster}-concepts, partial functions mapping each world $w$ in their domain to a monster in $w$.\footnote{It should be pointed out that letting quantifiers and pronouns range over partial individual concepts is not enough to solve the Hob-Nob puzzle. In particular, \citeauthor{Edelberg:1986}'s (\citeyear{Edelberg:1986,Edelberg:1992})  %\citeposst{Edelberg:1986,Edelberg:1992} 
arguments against a ``substitutional'' approach to the Hob-Nob puzzle based on definite descriptions carry over to analyses based on individual concepts. See \citet{Schwager:2007} for a discussion of the overgeneration problem raised by partial individual concepts in another context.}

\eanoraggedright \label{sch-has:ex:27} For a predicate $P \in D_{\stb{s, et}}$, a $P$-\textsc{concept} is a partial function $f$ from the set $W$ of possible worlds to the set $A_e$ of atomic individuals such that for any $w \in \cnst{dom}(f)$, $P(w)(f(w)) = 1$.
\z

\noindent  \REF{sch-has:ex:31} gives a preliminary semantics for \textit{two monsters}. We gloss over the internal composition (see \citealt{Schmitt:2019}), but the idea is that we form pluralities of \textsc{monster}-concepts, based on a notion of sum for individual concepts defined in the way just described for propositions, and that the numeral filters out the \textsc{monster}-concept pluralities of the right cardinality. Note that \REF{sch-has:ex:31} still involves a ``place-holder'' for the condition that the atoms in each plurality be distinct. % \footnote{We first assume that, before an NP combines with the plural morpheme, it is affixed with an operator $\mathcal{F}$, defined in \REF{sch-has:ex:28}, that returns the set of \sib{NP}-concepts. The plural morpheme forms pluralities of such \sib{NP}-concepts, based on a notion of sum for individual concepts that can be defined in the way just described for propositions. (This preliminary definition in \REF{sch-has:ex:29} won't account for the distinctness requirement yet, as we will see below.) Finally, the numeral filters out the $P$-concept pluralities of the right cardinality \REF{sch-has:ex:30}. The full DP meaning is given in \REF{sch-has:ex:31}.}

%\ea \label{sch-has:ex:28ff}
%\ea \label{sch-has:ex:28} \sib{$\mathcal{F}$} = $\lambda P_{\stb{s, et}}.\lambda f \in A_{\stb{s, e}}.f\ \text{is a}\ P\text{-concept}$
%\ex \label{sch-has:ex:29} to be revised: \sib{\textsc{pl}} = $\lambda P_{\stb{\stb{s, e}, t}}.\lambda g_{\stb{s, e}}.\forall f[f \leq_a g \rightarrow P(f)]$ 
%\ex \label{sch-has:ex:30} \sib{two} = $\lambda P_{\stb{\stb{s, e}, t}}.\{g \in D_{\stb{s, e}}\ |\ P(g) \land |\{f\ |\ f \leq_a g\}| = 2\}$

\ea \label{sch-has:ex:31} \sib{two monsters} = $\{f+g\ |\ f, g \in A_{\stb{s, e}} \land f\ \text{is a \textsc{monster}-concept} \land g\ \text{is a \textsc{monster}-concept} \land f\ \text{is distinct from}\ g\}$\z

\noindent The assumption that plural indefinites denote sets of pluralities (see \citealt{Schmitt:2019a} for motivation) is a generalization of Alternative Semantics approaches to indefinites \citep{Kratzer:2002}. As in alternative-based semantics for focus and questions, semantic composition proceeds ``pointwise'' for each member of the alternative set. However, \citeposst{Schmitt:2019a} semantics follows this principle both at the level of the alternative set and at the level of each plurality: composing \REF{sch-has:ex:31} with the distributive predicate \sib{roam the castle} yields the set of all propositional pluralities obtained by taking an element of \REF{sch-has:ex:31}, composing each of its atomic parts with the predicate and summing up the results \REF{sch-has:ex:32}.

\ea \label{sch-has:ex:32} \sib{two monsters are roaming the castle} = $\{(\lambda w.\textsc{roam}(w)(f(w)))+(\lambda w.\textsc{roam}(w)(g(w)))\ |\ f, g \in A_{\stb{s, e}} \land f\ \text{is a \textsc{monster}-concept} \land g\ \text{is a \textsc{monster}-concept} \land f\ \text{is distinct from}\ g\}$\z

\noindent We can now combine this semantics for plural sentences with our definition of cumulative belief in \REF{sch-has:ex:26} to obtain a general paraphrase for cumulative belief sentences, \REF{sch-has:ex:33}. \REF{sch-has:ex:34} gives the truth conditions this paraphrase predicts for \REF{sch-has:ex:6}/\REF{sch-has:ex:19}.

\ea \label{sch-has:ex:33} \sib{DP believe that [[two NP] $P$]}$(w) = 1$ iff there is a propositional plurality $p \in \{(\lambda w.P(w)(f(w)))+(\lambda w.P(w)(g(w)))\ |\ f, g \in A_{\stb{s, e}} \land f, g\ \text{are \sib{NP}-concepts} \land f\ \text{is distinct from}\ g\}$ such that
\ea for every $x \leq_a$ \sib{DP}, there is a $q \leq_a p$ such that \sib{believe}$(w)(q)(x)$
\ex and for every $q \leq_a p$, there is an $x \leq_a$ \sib{DP} such that \sib{believe}$(w)(q)(x)$.
\z\z

\ea \label{sch-has:ex:34} There are two \textsc{monster}-concepts $f$, $g$ such that $f$ is distinct from $g$, Abe and Bert each believe at least one of the propositions $\lambda w.\text{\sib{roam the castle}}(w)(f(w))$ and $\lambda w.\text{\sib{roam the castle}}(w)(g(w))$, and for each of these propositions, at least one of Abe and Bert believes it. \z

\noindent Importantly, since $f$ and $g$ can be partial, they do not have to be defined in the evaluation world, which accounts for the indefinite's lack of existential commitment. However, if we assume a semantics for \textit{believe} that requires the propositional complement to be defined in each of the subject's belief worlds \REF{sch-has:ex:37a}, a propositional plurality based on \textsc{monster}-concepts $f$ and $g$ can only satisfy \REF{sch-has:ex:34} if $f$ and $g$ are each defined in all of Abe's or in all of Bert's belief worlds (or both).

\ea \label{sch-has:ex:37a} \sib{believe} = $\lambda w.\lambda p_{\stb{s, t}}.\lambda x_e : \cnst{dox}(w)(x) \subseteq \cnst{dom}(p).\forall w'[w' \in \cnst{dox}(w)(x) \rightarrow p(w')]$\z

\noindent In the ``zombie vs.~griffin'' scenario in \REF{sch-has:ex:20}, the two individual concepts \sib{the zombie that was in Abe's room} and \sib{the griffin that was on Bert's bed}, among others, meet condition \REF{sch-has:ex:34}. But since our preliminary semantics for \textit{two monsters} does not require $f$ and $g$ to be distinct enough to count as two, so do the concepts \sib{the monster roaming the castle at 1 am} and \sib{the monster roaming the castle at 2 am} in scenario \REF{sch-has:ex:21}, where \REF{sch-has:ex:6}/\REF{sch-has:ex:19} is intuitively less acceptable. Even worse, we fail to rule out the ``four monsters vs.~five monsters'' scenario \REF{sch-has:ex:22}; the concepts in \REF{sch-has:ex:35a} and \REF{sch-has:ex:35b} verify condition \REF{sch-has:ex:34} in that scenario.

\ea \label{sch-has:ex:35}
\ea \label{sch-has:ex:35a} $\lambda w :\ $there are exactly four monsters in $w$ and Abe heard exactly one monster in $w$ . the monster Abe heard in $w$
\ex \label{sch-has:ex:35b} $\lambda w :\ $there are five monsters in $w$ and Bert heard exactly one monster in $w$ . the monster Bert heard in $w$ \z\z

\noindent To fix this problem,  \sib{two monsters} should contain only pluralities of pairwise ``distinct'' individual concepts. But how do we specify this distinctness relation? Note that the most obvious notion of distinctness for partial individual concepts, on which two concepts $f$, $g$ count as distinct iff there is no world $w$ such that $f(w) = g(w)$, won't work. It makes good predictions for the ``zombie vs.~griffin'' scenario (if Abe and Bert consider it impossible for a single individual to be both a zombie and a griffin). But on closer inspection, it does not improve on our straw man analysis from \sectref{sch-has:sec:3.1} since it is trivially satisfied if $f$ and $g$ have disjoint domains. Thus, \REF{sch-has:ex:35a} and \REF{sch-has:ex:35b} above count as distinct due to their incompatible presuppositions, which wrongly predicts \REF{sch-has:ex:6}/\REF{sch-has:ex:19} to be true in the ``four monsters vs.~five monsters'' scenario. Another wrong prediction is that the acceptability of \REF{sch-has:ex:6}/\REF{sch-has:ex:19} in the ``1 am vs.~2 am'' scenario should improve if Abe and Bert are assumed to have incompatible beliefs about a different topic like the weather. This would make their sets of belief worlds disjoint, so that \REF{sch-has:ex:36a} and \REF{sch-has:ex:36b} count as distinct.

\ea \label{sch-has:ex:36}
\ea \label{sch-has:ex:36a} $\lambda w :\ w$ is compatible with Abe's beliefs . the monster roaming the castle at 1am in $w$
\ex \label{sch-has:ex:36b} $\lambda w :\ w$ is compatible with Bert's beliefs . the monster roaming the castle at 2am in $ w$ \z\z

\noindent In sum, we can now paraphrase cumulative belief sentences via an independently motivated notion of propositional pluralities. To derive plausible parts for these pluralities, we analyzed plural indefinites in terms of pluralities of partial individual concepts --  but this partiality threatens to trivialize the notion of distinctness.

\subsection{A counterfactual-based paraphrase}

To see how we can avoid this problem, let us take a step back.  The data  suggest the distinctness relation should rely only on those contrasts that the attitude subjects consider relevant for individuation: what intuitively sets the ``zombie vs. griffin'' scenario apart from the ``1 am vs. 2 am'' scenario is that while it is plausible that both Abe and Bert would consider a griffin distinct from a zombie, they wouldn't necessarily consider a monster that shows up at 1 am to be distinct from a monster that shows up at 2 am. If so, our paraphrase should rely on the \textsc{distinctness criteria} of the attitude subjects. But these criteria cannot be derived (only) from Abe's and Bert's respective sets of belief worlds: it seems they can have opinions concerning the distinctness of two \textsc{monster}-concepts even if they believe the monsters under consideration do not exist. For instance, our sentence in \REF{sch-has:ex:6}/\REF{sch-has:ex:19} is as good in scenario \REF{sch-has:ex:36c} as in scenario \REF{sch-has:ex:7}. Crucially, in  \REF{sch-has:ex:36c}, there are no griffins in Abe's belief words and no zombies in Bert's belief worlds.

\eanoraggedright \label{sch-has:ex:36c} (Roy's castle\ldots) Abe believes in zombies, but believes that griffins don't exist. Bert believes  in griffins, but thinks that zombies don't exist. Around midnight, Abe thought he heard a zombie in his room. A little later, Bert believed he saw a griffin sitting on his bed. \hfill \REF{sch-has:ex:19} \textsc{true}  \z

\noindent A more adequate paraphrase of the subjects' distinctness judgments must thus appeal to worlds outside of their belief states -- i.e., to counterfactual beliefs: In \REF{sch-has:ex:36c}, both subjects could still believe that \textit{if} a zombie and a griffin existed, they would be distinct individuals. So in order to make individual concepts comparable even in cases like  \REF{sch-has:ex:36c},  we appeal to the condition in \REF{sch-has:ex:37}.\footnote{\REF{sch-has:ex:37} is misleading in one respect: usually, for a subject to believe a counterfactual, they have to believe that its antecedent is false. But a cumulative belief sentence based on concepts $f$ and $g$ can still be true if both subjects consider it possible that both $f$ and $g$ are instantiated.}

\eanoraggedright \label{sch-has:ex:37}  Two individual concepts $f$, $g$ count as distinct relative to belief-subjects $a,b$ iff  both $a$ and $b$ believe the counterfactual that if both $f$ and $g$ were instantiated, their values would be distinct.\z

\noindent In \sectref{sch-has:sec:3.2}, we saw that the natural notion of distinctness for individual concepts -- not returning the same value in any world -- is trivialized if Abe's set of belief worlds is disjoint from Bert's. We observed in \sectref{sch-has:sec:3.1} that the logical relation between Abe's and Bert's belief worlds is not crucial for our judgements of distinctness. But if we require the restrictor of the counterfactual in \REF{sch-has:ex:37} to be non-empty, then \REF{sch-has:ex:37} guarantees that there are worlds where both $f$ and $g$ are defined -- and they have different values in at least some of them. Since these worlds are not necessarily among Abe's or Bert's belief worlds, this is a non-trivial condition regardless of whether Abe and Bert believe $f$ and $g$ both exist.

Since the relevant notion of distinctness cannot be defined in terms of the attitude subjects' belief worlds, it is worth asking whether it should be relativized to a subject's belief state at all.\footnote{Thanks to Magdalena Kaufmann, Sarah Zobel and a reviewer for discussion of this issue.} For instance, we suggested that \REF{sch-has:ex:6}/\REF{sch-has:ex:19} is not judged true in the ``1 am vs.~2 am'' scenario \REF{sch-has:ex:21} because we can assume that Abe and Bert wouldn't necessarily consider a monster that shows up at 1 am distinct from a monster that shows up at 2 am. But this reasoning seems to rely on the general principle that we can perceive the same individual at different times, rather than anything specific to Abe's and Bert's belief states. So couldn't we derive the same judgment if we simply required that the utterance context, rather than the subjects' belief states, has to support the truth of the relevant counterfactual (\REF{sch-has:ex:rev4} in scenario \REF{sch-has:ex:21})?

\eanoraggedright \label{sch-has:ex:rev4} If there were a monster roaming the castle at 1 am and a monster roaming the castle at 2 am, they would be distinct.\z

\begin{sloppypar}
\noindent However, this alternative would make problematic predictions for examples where the speaker and the attitude subjects disagree on the pertinent individuation criteria. For example, consider \REF{sch-has:ex:rev5a}, where the subjects believe ghosts can be distinguished on the basis of their appearance, while the speaker doesn't share this belief. It seems to us that the German discourse in \REF{sch-has:ex:rev5} is acceptable and coherent in this scenario, contrary to the predictions of a theory on which the relevant counterfactual, \REF{sch-has:ex:rev5b}, is always evaluated relative to the speaker's beliefs or the utterance context.\footnote{A reviewer suggests that intentional distinctness might instead depend on whether the relevant counterfactual is objectively true in the evaluation world. This would presumably still predict cumulative belief sentences to not be fully acceptable if the individuation criteria for the belief objects are subject to debate: assuming that there are no ghosts in the actual world, it is not obvious what the actual truth value of \REF{sch-has:ex:rev5b} is.} That said, further empirical investigation of such examples is needed and may well show that the utterance context or the speaker's individuation criteria have some effect on intentional distinctness.
\end{sloppypar}

\eanoraggedright 
\eanoraggedright \label{sch-has:ex:rev5a}\textit{Context:} Abe and Bert believe in ghosts and think that ghosts cannot change their appearance. At 1 am, Abe thinks he saw a tall, red-haired ghost. At 2 am, Bert thinks he saw a short, black-haired ghost. They tell Roy about their beliefs. Roy isn't sure whether ghosts exist, but he is convinced that \textit{if} ghosts exist, they can shape-shift. Roy says:
\ex \label{sch-has:ex:rev5}
{\gll Abe und Bert glauben, dass zwei Geister im Schloss waren. Aber selbst wenn sie wirklich jeder einem Geist begegnet sind, war es wahrscheinlich ein und derselbe. \\
Abe and Bert believe that two ghosts in.the castle were but even if they really each a ghost encountered are was it probably one and the.same \\
\glt `Abe and Bert believe two ghosts were at the castle. But even if they really each encountered a ghost, it was probably the same one.'}  \z
 
\ex \label{sch-has:ex:rev5b}If there existed a ghost that was tall and red-haired at 1 am and a ghost that was short and black-haired at 2 am, they would be distinct.
\z %\z

\noindent Let us now return to spelling out the intuition behind \REF{sch-has:ex:37}. We have to specify in \textit{which} worlds $f$ and $g$ must yield distinct values, so we need a semantics for counterfactual beliefs. This is independently needed for overt counterfactuals in belief contexts as in \REF{sch-has:ex:38}. We follow \citet{Lewis:1973} in analyzing counterfactuals in terms of a partial ordering on worlds: when evaluating \REF{sch-has:ex:39}, we only consider the ``most plausible''  worlds where a zombie was present; for all those it must hold that there was a noise.

\eanoraggedright
\eanoraggedright \label{sch-has:ex:38} Abe thinks that if a zombie had been present, there would have been a noise. 
\ex \label{sch-has:ex:39} If a zombie had been present, there would have been a noise. \z\z

\noindent How does embedding under attitude predicates as in \REF{sch-has:ex:38} affect this ordering?  As \REF{sch-has:ex:40} is non-contradictory, it seems different subjects can have different opinions regarding the ``most plausible'' zombie-behavior. We model this by letting attitude predicates shift the ordering so that it is relativized to the attitude subject.\footnote{\citet{Arregui:2008} uses different examples that point in the same direction.}

\ea \label{sch-has:ex:40} Abe thinks that if a zombie had been present, there would have been a~noise, but Bert thinks that it would have been quiet!\z

\noindent More precisely, we associate each attitude subject $x$ and world $w$ with a weak partial ordering $\preceq_{x, w}$ that orders a subset of the possible worlds with respect to their degree of ``plausibility'' according to $x$'s belief state in $w$. We assume that the usual accessibility relation for a subject $x$ can be reconstructed from the $\preceq_{x, w'}$ relations for different worlds $w'$ as follows: the elements of $\cnst{dox}(w')(x)$ are the minimal elements of $\preceq_{x, w'}$. The meaning of the non-embedded counterfactual \REF{sch-has:ex:39} relative to a discourse context $c$ can then be paraphrased roughly as in \REF{sch-has:ex:41}: we assume that $c$ makes available an ordering relation $\preceq_c$ such that the worlds in the context set of $c$ are exactly the minimal elements of $\preceq_c$ (cf.~\citealt{Yalcin:2007}, who argues for a similar assumption wrt.~epistemic modals). The counterfactual then entails (and arguably presupposes) that its antecedent is false in those worlds \REF{sch-has:ex:41a}. Importantly though, its consequent is evaluated in the lowest-ranked worlds wrt.~$\preceq_c$ that verify the antecedent, and these worlds are \textit{not} in the context set. 

\ea \label{sch-has:ex:41}
\ea \label{sch-has:ex:41a} For all $\preceq_{c}$-minimal worlds $w'$, no zombie was present in $w'$, 
\ex \label{sch-has:ex:41b} $\&$ for all worlds $w'$ such that a zombie  was present in $w'$ $\&$ there is no $w''$ such that $w'' \prec_{c} w'$ $\&$ a zombie was present in $w''$, there was a noise in $w'$. \z\z

\noindent The truth conditions for the embedded case \REF{sch-has:ex:38} when evaluated in a world $w$ are similar. Yet, when in the scope of the attitude predicate \textit{thinks}, the counterfactual is evaluated wrt.~the subject-dependent ordering $\preceq_{\text{Abe}, w}$, rather than the ordering tied to the discourse context. The presupposition of the counterfactual -- there was no zombie -- is then required to hold in all of Abe's belief worlds \REF{sch-has:ex:42a}, but the consequent is evaluated in the ``most plausible'' worlds according to Abe's criteria where a zombie \textit{was} present, which are not among Abe's belief worlds.\largerpage

\ea \label{sch-has:ex:42}
\ea \label{sch-has:ex:42a} For all $\preceq_{\text{Abe}, w}$-minimal worlds $w'$, no zombie was present in $w'$, 
\ex \label{sch-has:ex:42b} $\&$ for all $w'$ such that a zombie was present in $w'$ $\&$ there is no $w''$ such that $w'' \prec_{\text{Abe}, w} w'$ $\&$ a zombie was present in $w''$, there was a noise in $w'$. \z\z

\noindent These paraphrases suggest that the semantics of attitudes is richer than usually assumed: a counterfactual in the scope of an attitude must have access to the attitude subject's entire $\preceq$-ordering, not just to the belief worlds. This is exactly what we need to give a precise paraphrase for our distinctness condition: two individual concepts $f$ and $g$ count as distinct for a subject if their values are distinct in all worlds that are minimal wrt.~the subject's $\preceq$-ordering \textit{among the worlds where $f$ and $g$ are both defined} \REF{sch-has:ex:43}. Crucially, these worlds don't have to be minimal in the global sense, and thus don't have to be among the subject's belief worlds (but they can --  \REF{sch-has:ex:43}, as opposed to  \REF{sch-has:ex:42}, does not require the antecedent of the counterfactual to be false in the relevant belief worlds). This captures the intuition that subjects may have beliefs about whether or not two ``potential monsters'' are distinct even if they do not believe that both of them exist.\footnote{We assume that such beliefs require each subject's $\preceq_{x, w}$-ordering to contain at least one world in which both \textsc{monster}-concepts are defined. A reviewer suggests scenarios like \REF{sch-has:ex:rev1} as a potential problem for this condition. Our predictions for \REF{sch-has:ex:rev1} hinge on the interpretation of \textit{impossible}.

\eanoraggedright\label{sch-has:ex:rev1} Abe believes that zombies exist, but that it is impossible for other monsters to exist. Bert believes that griffins exist, but that it is impossible for other monsters to exist.\z

\noindent The example is unproblematic if epistemic modals in  belief contexts quantify over the attitude subject's belief worlds. As only the minimal elements of $\preceq_{x, w}$ are among $x$'s belief worlds, there could then still be non-minimal worlds for each subject in which both types of monsters exist. This point carries over to other analyses of \textit{impossible} as a restricted modal quantifier: worlds excluded from the quantificational domain of \textit{impossible} may still be in the set ordered by $\preceq_{x, w}$, since they are needed to interpret overt embedded counterfactuals. While the reviewer's argument does go through for a `metaphysical' interpretation of \textit{impossible} as an unrestricted modal quantifier, such modalities poses a more general challenge for the possible-worlds approach.} Our original cumulative-belief example \REF{sch-has:ex:6}/\REF{sch-has:ex:19} then receives the full paraphrase in \REF{sch-has:ex:44}.

\ea 
\ea \label{sch-has:ex:43} Two partial individual concepts $f, g$ are distinct for a subject $x$ in $w$ -- $\cnst{distinct}_{x, w}(f, g)$ -- iff $\cnst{dom}(f) \cap \cnst{dom}(g) \neq \emptyset$ and for all worlds $w'$ such that $w' \in \cnst{dom}(f) \cap \cnst{dom}(g)$ and there is no $w''$ such that $w'' \in \cnst{dom}(f) \cap \cnst{dom}(g)$ and $w'' \prec_{x, w} w'$, $f(w') \neq g(w')$.
\ex \label{sch-has:ex:44} There are two \textsc{monster}-concepts $f$ and $g$, such that 
\ea \label{sch-has:ex:44a} $\cnst{distinct}_{\text{Abe}, w}(f, g)$ and $\cnst{distinct}_{\text{Bert}, w}(f, g)$ 
\ex Abe and Bert each believe at least one of the propositions $\lambda w.\text{\sib{roam the castle}}(w)(f(w))$ and $\lambda w.\text{\sib{roam the castle}}(w)(g(w))$, \ex and for each of these propositions, Abe or Bert believes it. \z\z\z

\noindent As suggested in \sectref{sch-has:sec:3.2} above, we build the distinctness condition \REF{sch-has:ex:44a} into the semantics of plural indefinites like \textit{two monsters}. It is worth noting that this gives rise to a compositionality puzzle beyond the scope of this paper: the condition in \REF{sch-has:ex:44a} requires access to each subject's entire $\preceq$-ordering for the evaluation world.\footnote{But see \citet{Haslinger:2020} for a compositional implementation of the paraphrase in \REF{sch-has:ex:44} that relies on a generalized version of \citeposst{Yalcin:2007} domain semantics.} But a standard attitude semantics as in \REF{sch-has:ex:37a} evaluates the complement distributively for each belief world. This raises the question of how the indefinite can access the relevant $\preceq$-orderings, which is particularly urgent given \citeposst{Schmitt:2019a} arguments that the lower plural in cumulative belief sentences must be interpreted \textit{in situ}, within the complement clause.

%Also mention distinctness for objects of \textit{suchen}

\section{Alternative proposals}\label{sch-has:sec:4}

While the proposal just presented concerns distinctness, not identity, and does not easily generalize to Hob-Nob sentences, it is worth asking how it differs conceptually from recent analyses of the Hob-Nob puzzle. Here, we discuss two ideas shared by many analyses of Hob-Nob sentences that do \textit{not} inform our approach. The first one is that the relevant identity relation relies on real-world individuals or events that are causally related to both belief objects. The second idea is that the identity problem requires an enriched notion of attitude contents that is sensitive to discourse referents. We submit that there is no clear evidence that an analysis of cumulative belief sentences should draw on either of these ideas. (For reasons of space, we focus on these general claims here and therefore cannot do justice to the details of the specific proposals in the literature.)

\subsection{Real-world objects}\label{sch-has:sec:4.1}

The first idea (\citealt{Rooy:1997,Dekker:1998}; see also \citealt{Cumming:2007}) is that identity between belief objects in Hob-Nob sentences requires a common real-world ``source'' of the belief objects: Abe's belief object $x$ can be identified with Bert's belief object $y$ only if there is a real-world individual or event  involved in causing Abe to form the belief that $x$ exists, and in causing Bert to form the belief that $y$ exists. Real-world events with this causal role may include linguistic utterances, like the newspaper reports in \REF{sch-has:ex:4}. Translating this to the problem of distinctness in cumulative belief sentences, two belief objects would count as distinct iff the causal chains leading the subjects to form their respective beliefs are unrelated. Yet,  this lack of a common causal source seems neither necessary nor sufficient for judgments of distinctness. In scenario \REF{sch-has:ex:50}, the same real-world sound causes Abe and Bert to form their beliefs. If distinct belief objects had to have distinct real-world sources,  we would expect our example \REF{sch-has:ex:6}/\REF{sch-has:ex:19} to be false in \REF{sch-has:ex:50}, but it isn't.\footnote{There don't have to be any obvious external sources: we would still consider the sentence true if Abe and Bert only hallucinate sounds. One could argue that hallucinations have  real-world causes (e.g., neural events), but then the question arises why the distinctness condition isn't met in scenario \REF{sch-has:ex:51} -- the relevant neural events in Abe's and Bert's brains would be distinct.} Intuitively, the belief objects are ``individuated'' by the properties ascribed to them. Scenario \REF{sch-has:ex:51}, on the other hand, involves different real-world ``causes'' for the two belief objects. Nevertheless, this is not enough to make \REF{sch-has:ex:6}/\REF{sch-has:ex:19} true. Intuitively, despite the different real-world sources, the properties ascribed to the belief objects are not sufficient to individuate them.

\eanoraggedright \label{sch-has:ex:50} \textit{Scenario:} (Roy's castle\ldots) At 1 am, the pipes make a sound. Abe hears the sound. He thinks  it is caused by a zombie in his room. Bert, in the other room, also hears the sound: He thinks it is caused by a griffin on his bed. \\\null\hfill  \REF{sch-has:ex:19} \textsc{\%true}\ex \label{sch-has:ex:51} \textit{Scenario:} (Roy's castle\ldots) At 1 am, the pipes make a sound. Abe wakes up and thinks it is a monster, but isn't sure what kind. At 2 am, the fridge makes a sound. Bert wakes up and thinks it is a monster, but isn't sure what kind.\hfill \REF{sch-has:ex:19} \textsc{\%not true} \z

\noindent Based on these judgments, there is no reason to extend the externalist identity criteria proposed for the Hob-Nob puzzle to \textit{distinctness} in cumulative belief sentences. We leave open if such criteria still play a role in intentional \textit{identity} (but see \citealt{Edelberg:1992} for interesting arguments that they do not).

\subsection{Discourse referents}\label{sch-has:sec:4.2}

Several approaches to the Hob-Nob puzzle \citep{Dekker:1998,Cumming:2007} assign a crucial role to discourse referents in mediating between the identity relation and the semantics of attitudes. The claim we address here (most explicit in \citealt{Cumming:2007}) is that the semantics of attitudes should be sensitive to the number and identity of the discourse referents the complement clause introduces. Any discourse referents free within that clause are taken to correspond to constituents of the belief subject's mental representation of their belief state. The identity relation is then defined on these mental symbols.

For example, to the extent we understand \citeposst{Cumming:2007} proposal, it involves an externalist identity relation of the kind discussed in \sectref{sch-has:sec:4.1}, but this relation holds between ``mental discourse referents'': there must be a real-world individual/event that was involved in causing Hob to form a mental symbol corresponding to the discourse referent introduced by \textit{a witch}, and also in causing Nob to form a mental symbol corresponding to the one picked up by \textit{she}. If so, Hob-Nob sentences make claims about the structure of Hob's and Nob's mental representations that go beyond their propositional contents: two belief sentences introducing different sets of discourse referents may make distinct claims about the subject's mental state even if the embedded clauses are truth-conditionally equivalent. This is the aspect we are skeptical about: while the analysis of Hob-Nob sentences may involve discourse referents, this is because the anaphoric relation in such examples is constrained by grammar just like other instances of anaphora. Thus, the judgments on Partee's marble example \citep{Heim:1982}, which shows that truth-conditionally equivalent sentences may have different dynamic meanings, do not seem to change when it is embedded in a Hob-Nob context:

\eanoraggedright \label{sch-has:ex:52}
\eanoraggedright \label{sch-has:ex:52a} \textit{Context:} Hob and Nob read in the papers that there are 10 witches in Austria. They each  believe that nine of the witches live in Vienna. Nob believes that the tenth witch lives in his neighborhood in Graz.
\ex \label{sch-has:ex:52b} Hob thinks all of the ten witches except one live in Vienna. Nob thinks she lives in Graz.
\ex \label{sch-has:ex:52c} Hob thinks nine of the ten witches live in Vienna. \#Nob thinks she lives in Graz. \z
\z

\noindent This example shows that the first sentences in \REF{sch-has:ex:52b} and \REF{sch-has:ex:52c} have different dynamic meanings, but it does not follow that these sentences make distinguishable claims about Hob's mental state. A semantics for \textit{believe} that is sensitive to the number and identity of free discourse referents in its scope would permit these sentences to differ in truth value. We think this prediction is not borne out %\footnote{Disregarding the possibility that Hob is ignorant of the fact that ten minus one is nine.} 
and conclude that the distribution of anaphora in \REF{sch-has:ex:52} is sensitive not to the structure of Hob's belief state, but to the way the belief is reported. So, while Hob-Nob sentences might involve discourse referents ranging over intentional objects, the right way of revising the semantics of \textit{attitude predicates} to model these objects is not to make it sensitive to discourse referents. This is in line with our approach to cumulative belief sentences, which requires an enriched attitude semantics (the $\preceq$-relations), but does not relate this enrichment to discourse referents. 

\section{Conclusion and outlook}

We argued that grammar is sensitive not only to intentional identity, but also to intentional distinctness, and that the grammatical phenomena sensitive to such relations are much more varied than usually assumed. We then considered the relevant notion of distinctness in cumulative belief sentences in more detail, arguing that it relies on counterfactual beliefs of the attitude subjects, so that criteria of individuation are relativized to the subjects. Apart from the question of how this can be implemented compositionally, our claim leaves open two other crucial issues: first, it remains to be seen whether the same  kind of treatment is warranted in cases involving intentional identity rather than distinctness, and, if so, how to specify it in this case. 

Second, our approach to cumulative belief should be extended to intensional predicates that don't straightforwardly involve a belief component, such as other attitude verbs like \textit{want}, but also ITV like \textit{look for}. The following data, pointed out by a reviewer, suggest that something similar to our distinctness constraint might be at work in the interpretation of plurals under \textit{look for}. 

\eanoraggedright
\eanoraggedright \label{sch-has:ex:rev2a}\textit{Scenario:} Abe and Bert occasionally go out to pick up litter in order to
keep their neighbourhood tidy. Yesterday, Abe went outside and tried to
find a piece of litter (he doesn’t care what he finds). Bert also went out to
look for a piece of litter.
\ex \label{sch-has:ex:rev2}Abe and Bert went looking for two pieces of litter. \hfill \textsc{true} in \REF{sch-has:ex:rev2a}
\z\z

\noindent This fits well with an analysis of \textit{look for} as a quantifier over worlds in which the search is successful (see e.g.,~\citealt{Zimmermann:1993,Zimmermann:2006}). To evaluate our counterfactual distinctness condition for two individual concepts $f$ and $g$ (e.g.,~$\lambda w.$the first piece of litter Abe picks up in $w$ and $\lambda w.$the first piece of litter Bert picks up in $w$), we would need to consider the closest worlds wrt.~some $\preceq$-ordering where \textit{both} search events succeed. Assuming that it is implausible for Abe and Bert to pick up exactly the same piece, $f$ and $g$ will have distinct values in these worlds. This predicts that such examples should be less acceptable if there are plausible scenarios in which Abe and Bert find the same thing. Indeed, it seems to us that the German counterpart of \REF{sch-has:ex:rev3} is not true in scenario \REF{sch-has:ex:rev3a}.

\eanoraggedright 
\eanoraggedright \label{sch-has:ex:rev3a} \textit{Scenario:} Abe and Bert are at a museum that is claimed to have ancient oil paintings. In fact, there is no such thing. Abe and Bert each want to see at least one such painting before they leave, but do not care which one.
\ex \label{sch-has:ex:rev3} Abe and Bert went looking for two ancient oil paintings. \z\z

\noindent However, a closer empirical investigation of plurals under ITV would be needed to whether this analogy with attitude verbs generalizes.

\section*{Abbreviations}
\begin{tabularx}{.5\textwidth}{@{}lQ}
ITV & intensional transitive verb \\
\end{tabularx}%
\begin{tabularx}{.5\textwidth}{lQ@{}}
\textsc{rel} & relative pronoun \\
\end{tabularx}

\section*{Acknowledgements}

Thanks to Maria Barouni, Stergios Chatzikyriakidis, Rory Harder, Winfried Lechner, Magdalena Kaufmann, Rick Nouwen, Rob Pasternak, Orin Percus, Johannes Schmitt, Frank Sode, Giorgos Spathas, Clemens Steiner-Mayr, Tim Stowell, Peter Sutton, Mayo Thompson, Marcin Wągiel, Thomas Weskott, Henk Zeevat, Ede Zimmermann, Sarah Zobel, two anonymous reviewers for this volume and the audiences at SinFonIJA 12 and the 21st Workshop on the Roots of Pragmasemantics. Viola Schmitt's research was funded by the Austrian Science Fund (FWF), project P 32939 \textit{The typology of cumulativity}. 

{\sloppy\printbibliography[heading=subbibliography,notkeyword=this]}

\end{document}
