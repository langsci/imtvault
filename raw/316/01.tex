\documentclass[output=paper]{langscibook} 
\ChapterDOI{10.5281/zenodo.5082450}

% \author{Authors\affiliation{Institution}}
\author{Marcin Wągiel\affiliation{Masaryk University} and  Mojmír Dočekal\affiliation{Masaryk University}}
\title{Number in natural language from a formal perspective}  
\abstract{In this introduction, we provide a general overview of a variety of phenomena related to the encoding of the cognitive category of \textsc{number} in natural language, e.g., number-marking, collective nouns, conjunctions, numerals and other quantifiers, as well as classifiers, and show how Slavic data can contribute to our understanding of these phenomena. We also examine the main strands of the study of number in language developed within formal lingusitics, linguistic typology, and psycholinguistics. Finally, we introduce the content of this collective monograph and discuss its relevance to current research.

\keywords{number, plurality, numerals, quantifiers, formal linguistics}}

\begin{document}
\SetupAffiliations{mark style=none}
\maketitle

\section{Introduction}

The goal of this monograph is to explore the relationship between the cognitive notion of \textsc{number} and various grammatical devices expressing this concept in natural language. The book aims at investigating different morphosyntactic and semantic categories including plurality and number-marking, individuation and countability, cumulativity, distributivity and collectivity, numerals, numeral modifiers and classifiers, as well as other quantifiers. It gathers contributions tackling the main themes from different theoretical and methodological perspectives in order to contribute to our understanding of cross-linguistic patterns both in Slavic and non-Slavic languages.

In this chapter, we will provide a brief introduction to various approaches to the study of the concept of number in natural language. We will mainly focus on the issues whose better understanding this book directly contributes to. First, in \sectref{doc-wag:sec:number-in-language}, we will discuss a variety of phenomena related to the expression of number in language. Then, in \sectref{doc-wag:sec:approaches-to-number}, we will review the major strands in linguistic research dedicated to explaining these phenomena. Finally, in \sectref{doc-wag:sec:the-contribution-of-this-book} we will introduce the content of this book and briefly explain its contribution.

\section{Number in language}\label{doc-wag:sec:number-in-language}

% \section{The research so far}\label{doc-wag:sec:the-research-so-far}

The nature of the relationship between number as a cognitive category and language %in question 
is highly complex, and thus the literature on the topic is vast. In this section, we will introduce a number of topics that are of relevance for the linguistic phenomena explored in this book and briefly discuss why they are important for a better understanding of how humans conceive of quantity and number.

\subsection{Number sense}\label{doc-wag:sec:number-sense}

It is well-documented that humans possess what is often called \textsc{number sense}, i.e., an intuitive understanding of numbers and their magnitude as well as various numerical relations and operations (see, e.g., \citealt{dehaene1997number} for an overview). The human number sense involves two distinct cognitive systems, namely the object tracking system, which enables an immediate enumeration of small sets, and the approximate number system, which supports the estimation of the magnitude of a collection of objects without relying on symbolic representations (see, e.g., \citealt{hyde2011two} for an overview). This mental ability is argued to provide an endowed predisposition for developing the concept of exact number and simple arithmetic and to facilitate the acquisition of lexical categories related to quantity, such as numerals (e.g., \citealt{gelman_gallistel1978child, wynn1990children}). Therefore, it seems that already in early childhood the language faculty interacts with that part of human mind that generates number sense. 

\subsection{Linguistic expression of the cognitive notion of number}\label{doc-wag:sec:lingusitic-expression-of-the-cognitive-notion-of-number}

Most languages of the world have formal means to express the conceptual distinction between `one' and `more than one'. A cross-linguistically widespread morphosyntactic device dedicated for that purpose is the category of \textsc{grammatical number} (e.g., \citealt{corbett2000number}). This category is typically expressed by an affix on the noun and/or by the agreement it triggers on other lexical items. The overall range of its values includes singular, dual (for two), trial (for three), paucal (for few, as opposed to many), plural and greater plural (for an excessive number). Though languages typically encode only two or three of those values, there are also languages with more complex number systems as well as ones that do not mark those distinctions morphologically at all. An example of a language with a rich number system is Bayso, see \REF{doc-wag:ex:number-bayso}, which distinguishes between number-neutral, singular, paucal and plural forms of the noun.

\ea\label{doc-wag:ex:number-bayso} 
\ea \gll lúban foofe\\
lion.\textsc{gnrl} watched\textsc{.1.sg}\\
\glt `I watched a lion/lions.'
\ex \gll lubán-titi foofe\\
lion-\textsc{sg} watched\textsc{.1.sg}\\
\glt `I watched a lion.'
\ex \gll luban-jaa foofe\\
lion-\textsc{pau} watched\textsc{.1.sg}\\
\glt `I watched a few lions.'
\ex \gll luban-jool foofe\\
lion-\textsc{pl} watched\textsc{.1.sg}\\
\glt `I watched (a lot of) lions.' \hfill \citep[Bayso, Cushitic;][11, adapted]{corbett2000number}
\z
\z

\noindent In Slavic, a complex number system including singular, dual and plural is attested in certain dialects of Slovenian as well as in Lower and Upper Sorbian, see \REF{doc-wag:ex:number-sorbian}. 

\ea\label{doc-wag:ex:number-sorbian}
\ea \gll hród\\ 
palace.\textsc{sg}\\
\glt `palace/castle' 
\ex \gll hrod-aj\\ 
palace-\textsc{du}\\
\glt `two palaces/castles'
\ex \gll hrod-y\\ 
palace-\textsc{pl}\\
\glt `palaces/castles' \hfill \citep[Upper Sorbian;][20, adapted]{corbett2000number}
\z
\z

\noindent In these languages, dual %is a morphosyntactic category that 
triggers obligatory agreement with determiners, adjectives and verbs, as demonstrated in \REF{doc-wag:ex:number-slovenian}. Its semantic relationship with the singular and plural as well as its interplay with the meaning of numerals have been subject to important theoretical considerations (e.g., \citealt{dvorak_sauerland2006semantics, marti2020dual}).

\ea\label{doc-wag:ex:number-slovenian} \gll T-a dv-a stol-a st-a polomljen-a.\\
these-\textsc{du.m.nom} two-\textsc{du.m.nom} chair-\textsc{du.m.nom} be-\textsc{3.du.prs} broken-\textsc{du.m.nom}\\
\glt `These two chairs are broken.' \hfill \citep[Slovenian;][168, adapted]{derganc2003dual}
\z

\noindent Though in Slavic and other Indo-European languages grammatical number is typically marked through suffixation and inflection, other cross-linguistically common means include apophony, i.e., a word-internal sound change, as in the English pair \textit{man} $\sim$ \textit{men}, and suppletion, e.g., \textit{čelovek} `man' $\sim$ \textit{ljudi} `men' in Russian. Yet another frequent grammatical device employed for number marking across languages is reduplication (e.g., \citealt{moravcsik1978reduplicative, corbett2000number}). For instance, the repeated initial syllable in \REF{doc-wag:ex:reduplication} functions as a morphological plural marker.

\ea\label{doc-wag:ex:reduplication}
\ea
\gll kuna\\  
     husband\\
\glt `husband'
\ex \gll kuu-kuna\\  
     \textsc{red}-husband\\ 
\glt `husbands' \hfill \citep[Papago, Uto-Aztecan;][308, adapted]{moravcsik1978reduplicative}
\z
\z

\noindent A related phenomenon attested cross-linguistically is known as syntactic reduplication (e.g., \citealt{travis2001syntax}, \citetv{chapters/04}), where the repeated material preceding and following the proposition gives raise to a plural interpretation, as illustrated in \REF{doc-wag:ex:syntactic-reduplication}.

\ea Jon washed plate after plate for hours after the party. \hfill \citep[457]{travis2001syntax}\label{doc-wag:ex:syntactic-reduplication}
\z 

\noindent Though grammatical number often expresses the semantic concepts of \textsc{singularity} and \textsc{plurality}, there are many well-studied mismatches between the two notions. % notions of \textsc{singularity} and \textsc{plurality}, there are many well-studied mismatches between form and meaning. 
First, the plural does not always mean `more than one' (e.g., \citealt{sauerland2003new, spector2007aspects, zweig2009number}). For instance, \REF{doc-wag:ex:mismatch-guns} does not mean that only carrying multiple guns is illegal in Illinois. Similarly, \REF{doc-wag:ex:mismatch-aliens} cannot be true in a scenario where a single alien has walked the earth.

\ea \ea Carrying guns is illegal in Illinois.\label{doc-wag:ex:mismatch-guns} 
\ex No aliens have ever walked the earth.\hfill \citep[267]{nouwen2016plurality}\label{doc-wag:ex:mismatch-aliens}
\z
\z

\noindent Furthermore, there is an intriguing relationship between bare singular nominals and \textsc{number neutrality} (e.g., \citealt{rullmann_you2006general, dayal2011hindi}, \citetv{chapters/06}). For instance, the bare direct object in \REF{doc-wag:ex:bare} is not specified with respect to whether it refers to a single individual or to a plurality of individuals.

\ea \gll anu bacca sambhaaltii hai\\
Anu child look-after-\textsc{ipfv} be-\textsc{prs}\\
\glt `Anu looks after (one or more) children.' \hfill \citep[Hindi;][127, adapted]{dayal2011hindi}\label{doc-wag:ex:bare}
\z

\noindent Furthermore, a question arises whether the semantics of bare noun phrases in languages with articles like English and German is the same as in articleless languages such as most Slavic languages (e.g., \citealt{geist2010bare, heim2011definiteness}). Though it has been proposed that articleless languages employ other morphological or syntactic devices in order to express definiteness, e.g., word order, aspect and number marking, novel evidence suggests the meaning of bare nouns in Slavic is different than expected under standard theories of uniqueness and maximality (e.g., \citetv{chapters/07}).

The grammatical category of plural marking is closely related to \textsc{countability}, often known also as the mass/count distinction illustrated by the contrast in \REF{doc-wag:ex:countability}. While standard theories of mass and count tend to model this distinction in binary terms (e.g., \citealt{link1983logical, chierchia1998plurality, chierchia2010mass}), there is convincing evidence that nouns can be countable to various degrees forming a scale of the mass/count spectrum (e.g., \citealt{allan1980nouns}, \citetv{chapters/03}). 

\ea\label{doc-wag:ex:countability}
\ea[]{Thirty three \{tables/stars/pieces of that pizza\}.}
\ex[*]{Thirty three \{bloods/waters/golds\}. \hfill \citep[104, adapted]{chierchia2010mass}}
\z 
\z 

\noindent Naturally, what counts as `one' and what counts as `many' relates to a deep philosophical problem of individuation, i.e, a criterion of numerically distinguishing the members of a kind (e.g., \citealt{grimm2012number, wagiel2018subatomic}). The problem of individuation becomes even more perplexing if we consider the class of abstract entities, e.g., \textit{fact} and \textit{information} (e.g., \citealt{grimm2014individuating, sutton2020informational}), and belief objects, e.g., imaginary individuals such as monsters (e.g., \citealt{geach1967intentional}, \citetv{chapters/11}).

Across languages, there is also a distinct class of nominal expressions known as \textsc{collective nouns}, e.g., \textit{committee} and \textit{pile}.\footnote{Sometimes they are also referred to as group or bunch nouns.} Though such nouns are singular in terms of their morphosyntax, they denote a plurality of objects (e.g., \citealt{landman1989groupsi, barker1992group, pearson2011new, henderson2017swarms}). This is evidenced by the fact that similar to plurals, but unlike singulars, collectives are compatible with predicates calling for plural arguments such as \textit{meet}, see \REF{doc-wag:ex:collectives}.

\ea\label{doc-wag:ex:collectives}
\ea The \{men/\#man\} met on Tuesday.
\ex The committee met on Tuesday. \hfill \citep[80, adapted]{barker1992group}
\z
\z

\noindent Interestingly, Slavic languages with their rich nominal systems have many types of derived collectives, e.g., Czech \textit{list} `leaf' $\rightarrow$ \textit{listí} `foliage'.\footnote{Note that the form \textit{listí} `foliage' is not the plural of \textit{list} `leaf', which is \textit{listy} `leaves'.} This fact makes them an especially valuable source of data regarding the ways in which the semantic notion of plurality can be encoded in derivational morphology (e.g., \citetv{chapters/08}).

Another class of expressions designating number consists of \textsc{quantifiers} such as \textit{some}, \textit{most} and \textit{all}. The nature of the lexical representations of their meanings as well as the psychological mechanisms involved in the interpretation of those meanings have been a puzzling question not only in linguistics but also in cognitive science (e.g., \citealt{pietroski2009meaning, lidz_et-al2011interface}, \citetv{chapters/17}).

A well-known property of quantifiers is that they give rise to scalar implicatures, i.e., implicit inferences suggesting that the speaker had a reason for not using a stronger, i.e., more informative, term on the same scale (e.g., \citealt{horn1984toward}). For instance, uttering \REF{doc-wag:ex:scalar-implicature} implies that the addressee did not eat all of the cookies.

\ea You ate some of the cookies. \hfill \citep[14]{horn1984toward}\label{doc-wag:ex:scalar-implicature}
\z

\noindent In this context, what is of particular interest is children's understanding of quantifiers and their computation of scalar implicatures, which seem to differ from what we find in adults (e.g., \citealt{noveck2001children, papafragou2004children}, \citetv{chapters/18}).

Yet another intriguing feature of quantifiers is that some of them enter non-trivial interactions with other phenomena such as negative polarity (e.g., \citealt{isreal1996polarity, solt2015q-adjectives}, \citetv{chapters/19}). For instance, items like \textit{much} can only appear in specific environments, such as negation, and are incompatible with affirmative contexts, as demonstrated by the contrast in \REF{doc-wag:ex:much}.

\ea\label{doc-wag:ex:much}
\ea[]{Albert didn't get much sleep.}
\ex[*]{Albert got much sleep. \hfill \citep[620]{isreal1996polarity}}
\z
\z

\noindent A unique subset of lexical items dedicated to expressing quantity are \textsc{cardinal numerals}. Though traditionally they were assumed to form a natural class with quantifiers such as \textit{some} and \textit{all}, there are good reasons to believe that in fact numerals are linguistic objects of a different type (e.g., \citealt[Ch.~2]{landman2004indefinites}, \citealt[Ch.~2]{rothstein2017semantics}). As witnessed in \REF{doc-wag:ex:numerals}, nominals modified by numerals can appear in predicate position while nominals involving other quantifiers cannot (on a non-partitive reading). Furthermore, numerals can also co-occur with the definite article and \textit{every}, e.g., \textit{the four cats} and \textit{every two students}, respectively. 

\ea\label{doc-wag:ex:numerals}
\ea[]{The inhabitants of the barn are four cats.}
\ex[\#]{The guests are \{some/most\} students. \hfill \citep[18, adapted]{rothstein2017semantics}}
\z
\z

\noindent The internal syntax and semantics of cardinal numerals as well as relationships between basic and complex numerals have been an important topic in the study of these expressions (e.g., \citealt{rothstein2013fregean, ionin_matushansky2018cardinals, wagiel_caha2020universal}, \citetv{chapters/13}, \citetv{chapters/14}). One of the questions is whether the meaning and syntactic status of \textit{six} is the same also in \textit{sixty} and \textit{six hundred}.

Though for a long time the mainstream research has been mostly focused on cardinals, like the ones described above, in recent years some attention has also been dedicated to puzzling semantic properties of numerals referring to numbers that are not positive integers like \textit{zero} (e.g., \citealt{bylinina_nouwen2018zero}) as well as fractions such as \textit{one third} (⅓) and decimals like \textit{two point five} ($2.5$) (e.g., \citealt{salmon1997}, \citetv{chapters/12}). A deeper understanding of how the mechanism responsible for quantification over parts of entities might also shed light on more general issues of individuation discussed above.

Furthermore, numerals can be modified by various modifiers including comparative modifiers such as \textit{more than} as well as superlative modifiers such as \textit{at least}. Though at first sight these two seem entirely synonymous only the latter give rise to ignorance inferences (e.g., \citealt{krifka1999least, nouwen2010two}, \citetv{chapters/15}). To illustrate, consider the contrast in \REF{doc-wag:ex:ignorance} in the scenario when the speaker knows that a hexagon has exactly six sides.

\ea\label{doc-wag:ex:ignorance}
\ea[]{A hexagon has more than three sides.}
\ex[\#]{A hexagon has at least three sides. \hfill \citep[4, adapted]{nouwen2010two}}
\z 
\z 

% \ea\label{doc-wag:ex:ignorance}
% \ea I have more than 2 children
% \ex I have at least 3 children. \hfill \citep[]{}
% \z 
% \z 

\noindent Interestingly, in many languages across the world numerals cannot combine with nouns directly. For this purpose a special category of \textsc{classifiers} is required, see \REF{doc-wag:ex:classifier} (e.g., \citealt{aikhenvald2000classifiers, bale_coon2014classifiers}). Classifiers sort nouns based on the type of their referents and provide means of the individuation thereof.
 
\ea \gll liǎng \minsp{*(} zhāng) zhuōzi\\
two {} \textsc{cl} table\\
\glt `two tables' \hfill \citep[Mandarin Chinese;][695]{bale_coon2014classifiers}\label{doc-wag:ex:classifier}
\z

\noindent A puzzling property of some classifier systems is their optionality (e.g., \citetv{chapters/16}). For instance, the classifier in  \REF{doc-wag:ex:optional-classifier} can but need not be used, which raises questions with respect to its semantic contribution.

\ea \gll sa(-tangkai) bungo\\
one-\textsc{cl} flower\\
\glt `one flower' \hfill \citep[Minangkabau, Malayic;][190, adapted]{aikhenvald2000classifiers}\label{doc-wag:ex:optional-classifier}
\z

\noindent Though classifiers are a rather marginal category in Slavic, there are a small number thereof in languages such as Bulgarian and Russian (e.g., \citealt{cinque_krapova2007note, khrizman2016functional}). For instance, the Russian classifier \textit{čelovek} for counting persons can appear optionally in constructions like \REF{doc-wag:ex:classifier-russian}. % the  classifiers are a , e.g., Russian \textit{štuka} `item', \textit{čelovek} `person' and \textit{golova} `head' (e.g., \citealt{khrizman2016functional}) as well as Bulgarian \textit{duši} `persons' and \textit{broj} `item' (e.g., \citealt{cinque_krapova2007note}).

\ea \gll pjat' \minsp{(} čelovek) stroitelej\\
five {} \textsc{cl} builders\textsc{.gen}\\
\glt `five builders' \hfill \citep[Russian;][4, adapted]{khrizman2016functional}\label{doc-wag:ex:classifier-russian}
\z

\noindent Another grammatical device dedicated to encoding plurality is \textsc{conjunction}. Interestingly, coordinated phrases as well as other plurality-denoting expressions give rise to an ambiguity between the collective, the distributive and the cumulative interpretation (e.g., \citealt{scha1981distributive, link1983logical, beck_sauerland2000cumulation, landman2000events}, \citetv{chapters/10}, \citetv{chapters/09}). For instance, \REF{doc-wag:ex:conjunction} on the collective reading is true if John and Bill together gave one flower to Mary, Sue, Ann and Jane as a group. On the distributive reading, John gave a flower to the girls and so did Bill. Finally, the cumulative scenario could look like this: John gave a flower to Mary and Ann, whereas Bill gave a flower to Sue and Jane.

\ea John and Bill gave a flower to Mary, Sue, Ann and Jane. \\ \hfill \citep[362]{beck_sauerland2000cumulation}\label{doc-wag:ex:conjunction}
\z

\noindent In this respect Slavic languages have proved to be a valuable source of data since they grammaticalized a special category of collective numerals, which rule out the distributive reading (e.g., \citealt{docekal2012atoms, wagiel2015sums}). For instance, while \REF{doc-wag:ex:czech-numeral} receives both the collective and the distributive interpreation, \REF{doc-wag:ex:czech-collective} allows only for the collective reading, i.e., the total of written letters is one.

\ea \ea \gll Tři chlapci napsali dopis.\\
three boys wrote\textsc{.pl} letter\textsc{.acc}\\
\glt `Three boys wrote a letter.'\label{doc-wag:ex:czech-numeral} % \hfill $\checkmark$\textsc{coll}, $\checkmark$\textsc{distr}
\ex \gll Troj-ice chlapců napsala dopis.\\
three-\textsc{coll.f} boys\textsc{.gen} wrote\textsc{.sg.f} letter\textsc{.acc}\\ %\hfill $\checkmark$\textsc{coll}, \#\textsc{distr}
\glt `A group of three boys wrote a letter.' \\ \hfill \citep[Czech;][113, adapted]{docekal2012atoms}\label{doc-wag:ex:czech-collective}
\z 
\z 

\noindent So far, we have discussed various ways in which the cognitive distinction between `one' and `more than one' is expressed by nouns and their modifiers. However, the expression of number is by no means restricted to the nominal domain. Many languages display the category of verbal number often termed as \textsc{pluractionality} (e.g., \citealt[Ch.~13]{lasersohn1995plurality}). This grammatical device indicates that the action designated by the verb was performed more than once or that there is more than one participant involved in that action. For instance, the contrast in \REF{doc-wag:ex:pluractionality} shows that the semantic contribution of the pluractional marker, realized here as \textit{tu}, is that the agent and the theme were involved in a plurality of pushing events.

\ea\label{doc-wag:ex:pluractionality} \ea \gll ʔiʃa-ʔ ʔinanta-siʔ ʔi=tuʛʛuur-ay\\
he-\textsc{nom} girl-\textsc{def} \textsc{3}=push-\textsc{pfv}\\
\glt `He pushed the girl.'  
\ex \gll ʔiʃa-ʔ ʔinanta-siʔ ʔi=tu-tuʛʛuur-ay\\
he-\textsc{nom} girl-\textsc{def} \textsc{3}=\textsc{plu}-push-\textsc{pfv}\\
\glt `He pushed the girl more than once.' \\ \hfill \citep[Konso, Cushitic;][adapted]{orkaydo2013category}
\z 
\z

\noindent Verbal number is also related to \textsc{aspect}, which expresses how an event or a state denoted by the verb extends over time. Since Slavic languages are renowned for their rich aspectual systems, they have attracted a lot of attention in this area (e.g., \citealt{filip1999aspect, borik2006aspect}). For instance, morphologically marked iterative forms of verbs in West Slavic express repetitive events, as illustrated in \REF{doc-wag:ex:iterative}.

\ea \gll Irenka \minsp{(} często) chadz-a-ła do biblioteki.\\
Irenka {} often walk-\textsc{iter}-\textsc{pst} to library\textsc{.gen}\\
\glt `Irenka often walked to the library.' \hfill \citep[Polish;][469, adapted]{pinon1997verbs}\label{doc-wag:ex:iterative}
\z

\noindent Moreover, it is known that the grammatical number of the noun phrase interacts non-trivially with the telicity of the entire verb phrase (e.g., \citealt{verkuyl1972compositional, krifka1998origins, de-swart2006aspectual}, \citetv{chapters/01}). While in sentences with a singular indefinite object the predicate gets a telic interpretation, see \REF{doc-wag:ex:telicity-singular}, its counterpart with a plural indefinite object is atelic, see \REF{doc-wag:ex:telicity-plural}.\footnote{Notice, however, that not all predicates behave like this, e.g., \textit{find} and \textit{kill} do not.}

\ea\label{doc-wag:ex:telicity}
\ea[\#]{Koos and Robby ate a sandwich for hours.}\label{doc-wag:ex:telicity-singular} 
\ex[]{Koos and Robby ate sandwiches for hours. \hfill \citep[49--50]{verkuyl1972compositional}}\label{doc-wag:ex:telicity-plural} 
\z 
\z 

\noindent The discussion of various grammatical and lexical devices dedicated to expressing the cognitive notion of number presented above by no means exhausts the potential of natural language. There are also various complex numerical expressions such as \textit{two-fold} and \textit{double} (e.g., \citealt{wagiel2018subatomic}), frequency adjectives such as \textit{occasional} and \textit{frequent} (e.g., \citealt{gehrke_mcnally2015distributional}), quantificational adverbials such as \textit{two times} (e.g., \citealt[Ch.~11]{landman2004indefinites}, \citealt{docekal_wagiel2018event}) and \textit{often} (e.g., \citealt{doetjes2007adverbs}) and many more. Nonetheless, we believe that this short presentation gives an overall idea of how elusive and multi-layered the relationship between number sense and grammar is. In the next section, we will briefly discuss various linguistic approaches that attempt to shed more light on the relationship in question.

\section{Approaches to number}\label{doc-wag:sec:approaches-to-number}

The phenomena described above have puzzled linguists, philosophers and psychologists for a long time. In this section, we briefly introduce three main research traditions that attempt at explaining the relationship between number and grammar. 

In the last thirty years, formal linguistics has been heavily influenced by studies addressing the vexing questions concerning the proper treatment of grammatical number, conjunction, numerals, the mass/count distinction and a number of other related topics that can be vaguely summarized under the label \textsc{theories of plurality}. The usual starting point is referenced as \citet{link1983logical}, but of course, there are many influential pre-runners such as \citet{bennett1979mass}, \citet{ter_meulen1980substances}, and \citet{scha1981distributive}. If we focus on the last three decades of the research on pluralities, we can identify several central frameworks which address the issues in question and offer heuristically intriguing paths to follow. At the end of the previous century, there appeared first proposals of the formalization of various interpretations of plurality-denoting noun phrases. Since then the study of number and plurality has become one of the central topics in linguistics.

The theories of plurality proposed so far differ in many respects. While some are more semantically oriented and develop models grounded in lattice-theory (e.g., \citealt{krifka1989nominal, landman1989groupsi, landman2000events, champollion2017parts}), others take a more pragmatic stance and base their formalizations on sets (e.g., \citealt{schwarzschild1996pluralities, winter2001flexibility}). Furthermore, after the seminal work of \citet{link1983logical} the mainstream research has agreed upon a more parsimonious approach to ontological domains, though authors diverge in the way they formalize the cognitive distinction between objects and substances (see, e.g., \citealt{krifka1989nominal, chierchia1998plurality, chierchia2010mass, rothstein2010counting, landman2011count, landman2016iceberg}). Moreover, already in the early years of semantic research the notion of plurality was extended to the domain of eventualities (e.g., \citealt{bach1986algebra}) and then expanded to even more abstract categories. Another significant strand of the research pursued in formal theories of plurality focuses on the proper treatment of numerals and classifiers (e.g., \citealt{krifka1995common,krifka1999least, landman2004indefinites, ionin_matushansky2006composition, ionin_matushansky2018cardinals, bale_gagnon_khanjian2011crosslinguistic, bale_coon2014classifiers, rothstein2017semantics}). Finally, a growing body of literature concerns bounded and unbounded interpretations of numerals and the semantic contribution of numeral modifiers (e.g., \citealt{geurts2006take, nouwen2010two, kennedy2015fregean}). 

Independently to the research pursued in formal linguistics, the distribution and grammar of number and numerals has received a lot of attention in the typological literature (e.g., \citealt{corbett1978universals, corbett2000number, greenberg1978generalizations, hurford1987language, hurford1998interaction}). Similarly, significant work has been carried out in the domain of classifiers (e.g., \citealt{dixon1982noun, aikhenvald2000classifiers}). What these broad cross-linguistic inquiries have revealed is that across languages there is a surprisingly rich diversity in meaning-form correspondences related to number and plurality. Yet, the exact nature of these correspondences remains unclear and the discovered variation often poses a challenge for the theoretical work described above.

Finally, for a couple of decades the way in which plurality and numerosity are linguistically expressed and cognitively processed has been a topic of interest for psycholinguists and cognitive scientists. This strand of research investigates experimentally different ways in which speakers refer to quantities in natural language. The key issues relate to countability, pluralization, quantity comparison and the mental representation of number magnitude (see, e.g., \citealt{henik1982three, shipley_shepperson1990countable, dehaene1993mental, barner_snedeker2005quantity, melgoza_pogue_barner2008broken}). Another important topic concerns the nature of the lexical representations of quantifiers alongside the psychological mechanisms involved in their interpretation (e.g., \citealt{pietroski2009meaning, lidz_et-al2011interface}). Finally, acquisition studies have pursued to understand how children acquire the capacity to perceive, comprehend and use those parts of language that are dedicated to expressing quantity (e.g., \citealt{noveck2001children, papafragou2004children}). Despite intriguing experimental results, it is often still unclear how to account for the psycholinguistic findings in formal models.

Though all of these traditions are very insightful and have produced significant results, so far to a great extent they seem to be developing independently, and thus many important more general issues related to number and plurality remain elusive. We feel it is time to attempt to shed more light on the topic by proposing a monograph whose aim is to combine different empirical, methodological and theoretical perspectives. We hope that as a result the field will gain a better understanding of the relationship between the cognitive notion of number and different ways it is reflected in grammar. The research pursued in the course of the last decade proves that focusing on Slavic is a good place to start (see, e.g., \citealt{docekal2012atoms}, \citealt{wagiel2015sums}, \citealt{matushansky2015}, \citealt{khrizman2016functional}, \citealt{arsenijevic2017gender}).

\section{The contribution of this book}\label{doc-wag:sec:the-contribution-of-this-book}

This monograph consists of four parts covering coherent topics within the study of number in natural language: (I)~\textit{Plurality, number and countability}, (II)~\textit{Collectivity, distributivity and cumulativity}, (III)~\textit{Numerals and classifiers} and (IV)~\textit{Other quantifiers}. Each part includes 3--6 chapters investigating different aspects of the main subject. In sum, the book consists of 19 chapters (including this introduction) related to each other by virtue of the general topic as well as formal linguistic frameworks adopted as their background. While being part of a broader whole, each chapter focuses on a particular problem from a different perspective, be it formal morphology, syntax or semantics, linguistic typology, experimental investigation or a combination of these. Concerning the empirical coverage, 11 out of the total of 19 chapters focus on Slavic data, often in comparison with other languages. The remaining 8 contributions either explore more general theoretical issues or investigate relevant linguistic phenomena in non-Slavic languages, which could also shed new light on the research on number and plurality in Slavic.

The first part, \textit{Plurality, number and countability}, is dedicated to the study of grammatical number and its correspondence to the semantic notion of plurality including the mass/count distinction. Empirically, it covers Slavic as well as Germanic, Turkic, Afro-Asiatic and Niger-Congo languages. The contribution by Piotr Gulgowski \& Joanna Błaszczak opens the volume by investigating experimentally the conceptual representation of grammatical and lexical number. This is pursued from the perspective of the perceptual processing of singular, plural and collective nouns in Polish. Subsequently, Scott Grimm, Ellise Moon and Adam Richman argue for a more fine-grained theory of countability by investigating strongly non-countable nouns in English such as \textit{fatherhood} and \textit{eyesight}. Based on the evidence from an extensive corpus search carried out on the COCA, they present a challenge for current approaches to the mass/count distinction, pointing to the need for a more general theory. Wiktor Pskit investigates (primarily) syntactic properties of English and Polish reduplicated constructions such as \textit{goal after goal}. A Slavic perspective is insightful since it allows the correlation of grammatical aspect with the pluractional interpretation of the expressions in question. Dorota Klimek-Jankowska \& Joanna Błaszczak relate plurality in the domain of objects and events. The experiment discussed in their chapter brings evidence in favor of the underspecification approach to the imperfective morphological aspect in Slavic. Suzana Fong explores the syntax of plural marking by examining bare nouns in Wolof. Her results suggest that the number interpretation of such nominals arise as a result of syntactic structures of a different size. Finally, Radek Šimík \& Christoph Demian examine the correlation in Polish and German between uniqueness and maximality on the one hand, and grammatical number on the other. Based on a production experiment, they argue that Polish word order alternations are not semantic correlates of German articles.

The second part, \textit{Collectivity, distributivity and cumulativity}, brings together contributions investigating distributive and non-distributive, i.e., cumulative and collective, interpretations of different types of nominals from a broad cross-lin\-guistic perspective. Marcin Wągiel investigates the morpho-semantics of two different types of Slavic collective nouns arguing that the manner in which parts are related to the whole is often grammaticalized. The discussed data call for a mereoto\-pological approach under which spatial collectives are interpreted as properties of spatial clusters, whereas social collectives are treated as properties of social clusters. Magdalena Roszkowski provides novel evidence from Polish concerning non-distributive interpretations of (allegedly) obligatorily distributive conjunction particles. The data are challenging for current theories of distributivity and demonstrate how careful exploration of Slavic data can help us to fine-tune the theories of plurality. Nina Haslinger, Eva Rosina, Magdalena Rosz\-kow\-ski, Viola Schmitt \& Valerie Wurm test the cross-linguistic predictions of different theories of cumulativity with respect to morphological marking. Based on a typological sample covering 22 languages from 7 language families (including Slavic), they conclude that no obligatory markers for cumulative readings were attested. Finally, Nina Haslinger \& Viola Schmitt explore contextual restrictions on intentional identity. Their research tackles an intriguing question, namely when are two intensions treated as distinct in natural language, by examining evidence from cumulative belief sentences.

The third part, \textit{Numerals and classifiers}, explores theoretical challenges related to the categories in question and discuss data from a wide variety of languages including Slavic and Germanic as well as Hungarian and obligatory classifier languages such Mandarin Chinese and Japanese. Andreas Haida \& Tue Trinh open this part of the book by convincingly showing that traditional theories of numeral denotations break down once we move beyond the usual examples including cardinals. They propose a more inclusive theory of numerals that could also account for decimals like \textit{two point five} (2.5) by postulating a mereological subpart counting component. Heidi Klockmann investigates the syntactic status of base numerals in Polish and English. Her analysis provides an account for different types of numeral bases as well as insights concerning language change in the domain of numerals. On the other hand, Yuta Tatsumi provides a syntactic analysis of complex cardinals by building on parallels between multiplicands and numeral classifiers in a number of languages (including Slavic). The data discussed pose a challenge for mainstream theories of complex numerals while the developed analysis proposes a unified account for numeral constructions in both classifier and non-classifier languages. Flóra Lili Donáti \& Yasutada Sudo explore the problem of defining alternatives for modified numerals from a theoretical perspective. Their account for the unacceptability of sentences with superlative numeral modifiers accompanied with scalar particles such as \textit{even} brings a novel piece of evidence concerning the nature of such alternatives and provides insight into the strength of the additivity presupposition. Finally, Brigitta R. Schvarcz \& Borbála Nemes investigate sortal individuating classifiers in Hungarian and their relationship with plurality and kind denotation. Their findings support analyses postulating that nouns are born as kind-denoting expressions and then can undergo a shift to predicates.

As already indicated by the title \textit{Other quantifiers}, the last part of the book focuses on other types of quantifying expressions. Barbara Tomaszewicz-Özakın discusses how the verification procedure of an agent parsing sentences containing quantifiers is directly determined by the particular formal properties of the respective quantifiers. The findings of an eye-tracking experiment on four Polish quantifiers extend the results of previous behavioral studies on the topic. Katalin É. Kiss, Lilla Pintér \& Tamás Zétényi present new evidence stemming from an acquisition study on Hungarian children's grasp of an existential plural determiner corresponding to English \textit{some}. The reported results of their experiments seem to corroborate previous studies suggesting that at least some pragmatic interpretative resources are acquired later in the course of language acquisition. Finally, Mina Giannoula brings some intriguing data concerning a previously observed fact that in some languages \textit{much} behaves in certain contexts as a weak negative polarity item. Based on a grammaticalized distinction in Greek, she argues that one of the two Greek equivalents of \textit{much} behaves like a strong negative polarity item in the sense of veridicality-based approaches.

% The book concludes with the summary of novel observations and generalizations presented in the monograph. We will comment on the puzzles tackled as well as point out new ones, which arise as a result of the reported investigations. 
We believe that the broad multi-dimensional empirical and methodological perspective of this collective monograph will be of interest to researchers focusing on how certain cognitive distinctions concerning number and related issues are represented in grammar, be it linguists, philosophers or cognitive psychologists. The reader will find data not only from Slavic languages, which constitute the main empirical focus of the book, but also from a number of typologically and genetically diverse languages including, e.g., English, German, Spanish, Greek, Japanese, Mandarin Chinese, Hungarian, Turkish as well as Wolof. Thus, we believe the book will be valuable not only to linguists working on Slavic, but also to those interested in broader cross-linguistic research and typology. 

\section*{Abbreviations}

\begin{tabularx}{.5\textwidth}{@{}lX@{}}
\textsc{1}&{first person}\\
\textsc{3}&{third person}\\
\textsc{acc}&{accusative case}\\
\textsc{cl}&{classifier}\\
\textsc{coll}&{collective marker}\\
\textsc{def}&{definite marker}\\
\textsc{du}&{dual number}\\
\textsc{f}&{feminie gender}\\
\textsc{gen}&{genitive case}\\
\textsc{gnrl}&{general number}\\
\textsc{ipfv}&{imperfective aspect}\\
\end{tabularx}%
\begin{tabularx}{.5\textwidth}{@{}lX@{}}
\textsc{iter}&{iterative aspect}\\
\textsc{m}&{masculine gender}\\
\textsc{nom}&{nominative case}\\
\textsc{pau}&{paucal number}\\
\textsc{pfv}&{perfective aspect}\\
\textsc{pl}&{plural number}\\
\textsc{plu}&{pluractional marker}\\
\textsc{prs}&{present tense}\\
\textsc{pst}&{past tense}\\
\textsc{red}&{reduplication}\\
\textsc{sg}&{singular number}\\
% \textsc{}&{}\\
\end{tabularx}

\section*{Acknowledgements}
We would like to sincerely thank Berit Gehrke and Radek Šimík for their help and support in the process of editing this book as well as for their comments on the form and content of this introduction (though of course the standard disclaimer applies). We gratefully acknowledge that the research was supported by a Czech Science Foundation (GAČR) grant to the Department of Linguistics and Baltic Languages at the Masaryk University in Brno (GA17-16111S).

{\sloppy\printbibliography[heading=subbibliography,notkeyword=this]}

\end{document}
