\documentclass[output=paper]{langsci/langscibook} 
\ChapterDOI{10.5281/zenodo.1441331}
\title{Introduction} 
\author{Ingo Feldhausen\affiliation{Goethe-Universität Frankfurt am Main}\and Jan Fliessbach\affiliation{Freie Universität Berlin}\lastand Maria del Mar Vanrell\affiliation{Universitat de les Illes Balears}}
% \chapterDOI{} %will be filled in at production

\abstract{\noabstract}

\maketitle\rohead{Introduction}
\begin{document}\label{chap:intro}
The field of \isi{prosody research} belongs to those linguistic disciplines that have developed rapidly in recent decades. This is mainly due to the appreciation it has received in theoretical studies of grammar, but also to the technological revolution that has resulted in the widespread availability of software dedicated to conducting analyses and calculating statistics. All stages of experimental work have been affected: data collection, data processing, acoustical and statistical analysis, as well as the development of stimuli for perceptual experiments. Carefully controlled methodologies have become a standard in linguistic research for gathering empirical evidence and their development plays an increasing role in \ili{Romance} linguistics. The same holds for corpus research, where large speech databases are used to investigate \isi{prosody} and automatic tools are applied for analyzing the data. A number of recent publications document these developments and show how these fields of research are becoming ever more dynamic and innovative (see, e.g., \citealt{Sudhoff.2006,Cohn.2012,Durand.2014}). 

With the proliferation of methodological options and the emergence of research traditions defined not only by their object of interest, but also by their methodological choices (e.g. \textit{Laboratory Phonology} or \textit{Corpus Linguistics}), making reasoned methodological decisions is becoming an increasingly difficult task. Yet only detailed descriptions of problems encountered in the investigation of specific phenomena can provide the practical advice needed to avoid difficulties before they arise. Thus, the goal of this book is to encourage the reader to pause for a moment and to reflect on (some of) the methods used in our field.

While the existing volumes on methodological reflections address this issue either from a general point of view by considering different linguistic subdisciplines (e.g. \citealt{Ender.2012,Podesva.2013}) or by dealing with specific methodological approaches in the entire field of phonology, including \isi{prosody} (e.g. \citealt{Cohn.2012,Nguyen.2013,Durand.2014}), the present book concentrates specifically on methods in \isi{prosody} and \isi{intonation} research. Consequently, the book has parallels with the seminal volume by \citet{Sudhoff.2006}. In contrast to that book, however, the present volume concentrates on \ili{Romance} languages and languages in contact with \ili{Romance} languages (\ili{Catalan}, \ili{Italian}, \ili{Portuguese}, \ili{Spanish}, and \ili{Quechua}) – thus languages that play an important role in \isi{prosody research}. Furthermore, it includes some of the latest developments in the field. Finally, the present volume embraces contributions that evaluate specific methods both with and without the presentation of new data.

Pausing and carrying out methodological reflections is an important step in scientific research and the relevance of that issue in \isi{prosody research} can be seen – apart from the present volume – in the increasing number of papers (e.g. \citealt{Niebuhr2015,Cole.2016}), conference sessions, or summer schools dedicated to that issue (e.g. Aix-en-Provence 2016\footnote{ Aix-en-Provence (France): Aix Summer School on Prosody 2016: “Methods in Prosody and Intonation Research: Data, Theories, Transcription”. ~\href{https://aixprosody2016.weebly.com}{https://aixprosody2016.weebly.com}.}, Vienna 2016\footnote{Vienna (Austria): Sommerschule des Deutschen Romanistenverbandes 2016: “Gesprochene Sprache in der Romania: Von der Theorie zur Empirie”. \href{https://romanistik2016.univie.ac.at/}{https://romanistik2016.univie.ac.at/}.}). This book is based on the session \textit{Methods in empirical \isi{prosody} research} from the 34\textsuperscript{th} Romanistentag, the biannual conference of the \ili{German} Association of \ili{Romance} Philologists (\href{http://www.deutscher-romanistenverband.de/}{Deutscher Romanistenverband}, DRV) held between July 26\textsuperscript{th}--29\textsuperscript{th}, 2015 in Mannheim (Germany). The session was organized by Ingo Feldhausen, Uli Reich, and Maria del Mar Vanrell. The seven double-blind peer reviewed contributions to this volume represent a selection from the talks given at this session and bring together some of the most distinguished researchers of \isi{prosody} working on \ili{Romance} languages, united in the attempt to place methodological reflections at the center of their respective chapter, while also providing insight into the current state of the research projects that apply these methods.\largerpage

The methodological paradigms covered in this book include the study of pro\-sody with large corpora and \isi{spontaneous speech} employing different approaches, more controlled \isi{prosodic} analyses, and questions of \isi{prosodic} data collection, manipulation and elicitation. The book consists of three different parts, which in turn are organized into different chapters. The first part of the book is entitled \textit{Large corpora and spontaneous speech} and consists of those papers mainly dealing with these aspects of \isi{prosody research}. The second part, \textit{Approaches to \isi{prosodic} analysis}, comprises those chapters addressing different considerations relevant in analyzing \isi{prosodic} data (e.g. the combination of production and perception experiments, the role of \isi{phonetic} analyses or multimodal analyses). The third part is entitled \textit{Elicitation methods} and focuses on the critical assessment of current elicitation methods. A detailed appreciation and overview of the three parts of the book and the different chapters is given in the guest foreword by Pilar \citeauthor{Prieto.2018}.

\largerpage Lastly, we, the editors, would like to thank several people who have helped and supported us in creating the present volume. First, we are highly indebted to Pilar Prieto for her comments and recommendations, and most importantly for her foreword that adds a magnificent finishing touch to this collective piece of work. We also extend our gratitude to each participant and contributor of the above-mentioned session at the congress of the \ili{German} Association of \ili{Romance} Philologists as well as to the authors of the individual chapters. Uli Reich deserves special mention and gratitude for his co-participation in the organization of the congress session, his relentless and fundamental logistical support, and his insightful advice on many different occasions. We also want to thank the anonymous reviewers whose critical, valuable, and insightful comments have helped to improve the present volume: Mathieu Avanzi, Stefan Baumann, Elisabeth Delais-Roussarie, Andreas Dufter, Wendy Elvira-García, Eduardo García-Fernandez, Nicholas Henriksen, José Ignacio Hualde, Conxita Lleó, Judith Meinschaefer, Trudel Meisenburg, Antje Muntendam, Oliver Niebuhr and Paul Warren. Next, we would like to thank our student assistants Julia Otto and Magalí del Valle Bertola for their different kind of help and support. Our gratitude also goes to the series editors of \textit{Studies in Laboratory Phonology} for their interest in our volume and for providing constructive criticism and sound advice during the publication process. Furthermore, we are in debt to Sebastian Nordhoff and Felix Kopecky for their patient help with the details of producing the final manuscript and to Adrien Barbaresi, Amir Ghorbanpour, Andreas Hölzl, Daniela Kolbe-Hanna, Eleni Koutso, Audrey MacDougall, Hella Olbertz, Brett Reynolds, Aysel Saricaoglu, and Jeroen van de Weijer for proofreading. 

\largerpage
Finally, we hope that the various contributions and the breadth of topics they deal with make the present volume a source of inspiration and insight for the linguistic research community and help to highlight the importance of profound methodological reflections. 

{\sloppy
\printbibliography[heading=subbibliography,notkeyword=this]
}
\end{document}
