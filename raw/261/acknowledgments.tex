\addchap{Acknowledgments}
\begin{refsection}

This volume is the result of a collaboration between Stockholm University and the University of Helsinki, funded by the Faculty of the Humanities at Stockholm University (SU FV-5.1.2-0757-15). As part of this collaboration, Henrik Bergqvist co-organized, with Seppo Kittilä, a two-day symposium on Evidentiality, Egophoricity, and Engagement at Stockholm University in April 2016. Participants at this Symposium were Ellen Basso, Henrik Bergqvist, Martine Bruil, Connie Dickinson, Karolina Grzech, Minna Jaakola, Lotta Jalava, Seppo Kittilä, Dominique Knuchel, Norikazu Kogura, Henrik Liljegren, Stéphane Robert, Jonathon Lum, Timur Maisak, Geraldine Quartararo, Johan Rooryk, Erika Sandman and Manuel Widmer. Nicholas Evans and Alan Rumsey were invited plenary speakers. On the second day of the symposium, it was agreed that there should be a proceedings volume with contributions from participants who wished to write up their presentations for publication. The decision to approach Language Science Press as the preferred venue for publication was also made then. In the end, eight contributions make up the present volume and we would like to extend our thanks to all who contributed and to all who presented at the symposium. We also wish to thank the anonymous reviewers for their work and to Elena Moser for her instrumental role in preparing the contributions for publication by formatting them to \LaTeX. Lastly, we wish to thank Martin Haspelmath and Sebastian Nordhoff at Language Science Press for accepting our proposal and for agreeing to publish the finished book. Henrik wishes to acknowledge the support of the Swedish Research Council (dnr. 2017-01969) and Åke Wiberg Stiftelse (postdoctoral grant) and Seppo expresses his gratitude to Academy of Finland (grant number 262951) for financial support. We hope that the papers of this volume will contribute to the ongoing exploration of epistemic marking in language from the point of view of first-hand language data from minority languages of the Americas, Papua New Guinea and Central Asia.  


Henrik Bergqvist is Associate Professor in Linguistics at Stockholm University. He received his PhD in Field Linguistics from the Endangered Languages Academic Programme at SOAS (University of London) in 2008. His thesis was based on field work on the Mayan language Lacandón and since 2009 Henrik has also worked with speakers of Kogi and Ika, two Chibchan languages spoken in Colombia. Henrik’s research has focused on various forms of epistemic marking, such as evidentiality, egophoricity, and engagement, and he has published several papers on these topics.    

Seppo Kittilä is Senior Lecturer in Linguistics at the University of Helsinki. His main fields of interest include transitivity, grammatical voice, ditransitives and evidentiality. He has published extensively on these fields in leading journals (such as \emph{Linguistic Typology}, \emph{Linguistics} and \emph{Studies in Language}), and edited volumes (by OUP, Mouton de Gruyter and Benjamins), and he is also the co-author (together with Fernando Zúñiga) of the recent textbook Grammatical Voice (Cambridge University Press). Seppo is a member of the Swedish Research Council’s evaluation panel for linguistics and he has received major grants from the Academy of Finland.


%\printbibliography[heading=subbibliography]
\end{refsection}

