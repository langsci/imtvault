\documentclass[output=paper ,collection	  ,collectionchapter ,biblatexbackend=biber   ]{langscibook}                               
\ChapterDOI{10.5281/zenodo.1314321}

  
\def\sh{ɕ}
% \def\hw{\textturnh}
 
\author{Jeff Good\affiliation{University at Buffalo}}

\title{East Benue-Congo noun classes, with a focus on morphological behavior} 
 
\abstract{Comparative studies of noun class systems in East Benue-Congo languages go back at least as far as the mid-nineteenth century work on comparative Bantu undertaken by Wilhelm Bleek. In the wider Benue-Congo context, the most significant work is \namecite{deWolf1971}, which reconstructs a noun class system for Proto-Benue-Congo and remains the most detailed study on the topic available today. This paper summarizes the results of De~Wolf and also looks at various morphosyntactic aspects of Benue-Congo noun class systems en route to consideration of the possibilities for reconstruction of abstract features of the noun class system of the proto-language. These include patterns of change in the structure of these systems, the fact that both prefixing and suffixing noun class systems are attested in the family, domains of noun class concord, different series of noun class markers appearing on different morphological hosts, and the issue of how attested classes can be linked to reconstructed classes.}


\begin{document}


\maketitle
\label{sec:2} 

%%%%%%%%%%%%%%%%%%%%%%%%%%%%%%%%%%%%%%%%%%%%%%%%%%%%%%%%%%%%%%%%%%%%%%%%%%%%%%%%
\section{Overview of previous comparative work \label{Overview}}
\label{sec:good:1}
%%%%%%%%%%%%%%%%%%%%%%%%%%%%%%%%%%%%%%%%%%%%%%%%%%%%%%%%%%%%%%%%%%%%%%%%%%%%%%%%
\largerpage[-1]
Comparative studies of noun class systems in \ili{East Benue-Congo} languages go
back at least as far as the mid-nineteenth century work on comparative \ili{Bantu}
undertaken by Wilhelm Bleek \cite[13--14]{Maho1999}. In the wider \ili{Benue-Congo}
context, the most significant work, by far, has been that of
\namecite{deWolf1971}, who reconstructed a noun class system for
\ili{Proto-Benue-Congo} on the basis of an examination of representatives from
languages of the Plateau, \ili{Jukunoid}, \ili{Cross River}, and \ili{Bantoid} subgroups
\cite[19--20]{deWolf1971}.{\footnote{There are complications in using the term \ili{Benue-Congo}
that are hard to avoid in a work like this one. While the focus of this chapter
is \ili{East Benue-Congo}, the group of languages referred to by this term has also
been labeled \ili{Benue-Congo} \cite[30--31]{WilliamsonB2000}. Here, I will generally
refer to \ili{East Benue-Congo}, over \ili{Benue-Congo}, in reference to the language
group of primary focus here, though \ili{Benue-Congo} will be used in some places
where the larger group is under consideration. For the sake of brevity,
\ili{Proto-Benue-Congo} will be used throughout to refer to the proto-language
associated with \ili{East Benue-Congo}. Many points made here for \ili{East Benue-Congo} 
will apply to larger groups, such as \ili{Benue-Congo} or \ili{Niger-Congo},
though this is not generally made explicit given that the focus of this volume
is on the properties of \ili{East Benue-Congo} rather than the similarities between
\ili{East Benue-Congo} and the superordinate groups that it is associated with.}}
While this work was only intended to serve as a starting point
\cite[21]{deWolf1971}, rather than a definitive reconstruction, it remains the
most detailed study on the topic available
today.{\footnote{\namecite{Williamson1993} is a notable attempt to amend
\quotecite{deWolf1971} work. However, it appears to have received relatively
little attention.}} Without intending to detract from his efforts in any way,
the reason for this is not that further work was deemed superfluous but, rather,
as has so often been the case with \ili{Niger-Congo} studies, the limited energies of
specialists have been spent on other topics.

The core comparative results of \namecite{deWolf1971} center on the
reconstruction of a number of noun class prefix forms (for nominal prefixes and
some concords), typical singular/plural pairings for these noun classes, and a
set of nouns belonging to each class. This reconstruction is summarized in
\tref{PBCDeWolf} and draws in part on the presentation provided in
\namecite[38--39]{Williamson1989nc}, in particular with respect to the
assignment of class number labels. The full treatment can be found in
\namecite[50--59]{deWolf1971}, and any reader interested in the full details is
advised to consult the original, where additional complications are
discussed.{\footnote{For instance, \namecite[52--53]{deWolf1971} suggests that
Classes 5, 6, 8, 9, 10, 12, 14, and 15 may have also occurred as unpaired
classes, alongside Class 6a which is reconstructed as only being an unpaired
class, and that there may have been an additional possible Class 7/6 pairing.}}
Partial results are given here since they should be sufficient for illustrating
the most important points regarding the reconstructions and to allow more space
to be devoted to other areas of \ili{East Benue-Congo} noun class system
reconstruction. The table gives (i) canonical singular/plural pairings for the
various reconstructed classes or indicates if the class is one that is not
associated with clear singular/plural pairings (e.g., for mass nouns), (ii)
reconstructed nominal prefix and concord forms (including indication of tone in
some cases), and (iii) exemplary reconstructed nouns for each of the various
classes.


\begin{table} 
\fittable{
\begin{tabular}{llllll} 
\lsptoprule
{\scshape{label}}		&	{\scshape{pfx}}					&	{\scshape{conc}}					&	{\scshape{pfx}}								&	{\scshape{conc}}					&	{\scshape{example}}				\\
\midrule
1/2				&	*\emph{ù-}, *\emph{ò-}		&	*\emph{gwu-}, *\emph{à-}	&	*\emph{bà-}								&	*\emph{ba-}					&	*\emph{-lume} `man'			\\	
5/6				&	*\emph{li-}					&	*\emph{zí-} (?)				&	*\emph{à-}								&	*\emph{ga-}, *\emph{a-}		&	*\emph{-tama} `cheek'		\\
7/8				&	*\emph{ki-}, *\emph{ke-}	&	*\emph{ki-}					&	*\emph{bì-}, *\emph{bè-}				&	*\emph{bi-}					&	*\emph{-kupe} `bone'		\\
9/10			&	*\emph{e-}, *\emph{i-}		&	*\emph{zì-}					&	*\emph{í-}								&	*\emph{í-}, *\emph{zi-} (?)	&	*\emph{-name} `animal'		\\
3				&	*\emph{ú-}					&	*\emph{u-}, *\emph{gu-}		&	\mc{2}{c}{(plural 10 or 13)}											&	*\emph{-sene} `back'		\\ 
11				&	*\emph{lu-}					&	*\emph{lu-}					&	\mc{2}{c}{(plural 10)}													&	*\emph{-z(w)ana} `sun'		\\
12				&	*\emph{kà-}					&	*\emph{ka-}					&	\mc{2}{c}{(plural 10 or 13)}											&	*\emph{-kama} `monkey' 		\\
13				&	\mc{2}{c}{(singular 3 or 12)}							&	*\emph{ti-}								&	*\emph{ti-}					&	*\emph{-kwon} `tree'		\\
14				&	*\emph{bù-}					&	*\emph{bu-}					&	\mc{2}{c}{(plural 6 or 10)}									&	*\emph{-su}	`face'; *\emph{-bogo} `fear'		\\
15				&	*\emph{ku-}					&	*\emph{ku-}					&	\mc{2}{c}{(plural 6 or 10)}												&	*\emph{-tuŋ⁠i} `ear'		\\
6a				&	*\emph{mà-}, *\emph{nà-}	&	*\emph{ma-}, *\emph{nà-}	&	\mc{2}{c}{(unpaired)}													&	*\emph{-luŋ⁠} `blood' \\
\lspbottomrule
\end{tabular}
}
\caption{Overview of \quotecite{deWolf1971} Proto-Benue-Congo noun class system \label{PBCDeWolf}}
\end{table}

The class numbering conventions in \tref{PBCDeWolf} draw heavily on those
associated with \ili{Proto-Bantu} noun class reconstructions (see, e.g.,
\namecite[246--255]{Maho1999} and \namecite[104]{Katamba2003}) and are not
found in \namecite{deWolf1971}. They are included here due to the long-standing
significance of \ili{Proto-Bantu} for comparative studies of \ili{East Benue-Congo} under
the assumption that \ili{Bantu} languages are relatively morphologically conservative
(see, e.g., \namecite{Williamson1985}, \namecite[29]{Jungraithmayr1990}, and
\namecite{Hyman2004}). I have collapsed a possible Class~4 with Class~10 in the
presentation since \namecite{deWolf1971} does not explicitly separate these,
though see \namecite[38]{Williamson1989nc}. Class~4 will be referred to here
in reference to specific noun class system analyses which treat it as
distinct from Class~10, even in cases where they are formally equivalent, or
nearly equivalent, as is the case with \ili{Noni} \cite[33]{Hyman1981}, to be
discussed below.

While \tref{PBCDeWolf} does not present the full range of the details
of the reconstructions of \namecite{deWolf1971}, it should
suffice to give a general impression of his proposals. Since there has not been
extensive debate regarding these reconstructions since their publication, it is
difficult to know which should be considered more or less secure as representing
key parts of the \ili{Proto-Benue-Congo} noun class system. However, to the extent
that a number of these pairings have close analogs in other branches of
\ili{Niger-Congo}, e.g., Classes 1/2, 3/4, and 6a (see
\namecite[38--39]{Williamson1989nc}), they seem quite likely to have been
present in \ili{Proto-Benue-Congo} as well.

\Citet{deWolf1971} does not discuss semantic patterns with respect to the
noun classes in detail, though his reconstructions of specific nouns as
belonging to the various classes do indicate that some of the classes would have
had fairly clear-cut semantics. On the whole, these class semantics are not particularly
surprising from a \ili{Niger-Congo} perspective (see, e.g., the semantic labels given
to the various classes and pairings across \ili{Niger-Congo} presented in
\namecite[38--39]{Williamson1989nc}). The Class~1/2 pairing is associated with
nouns referring to humans, and \namecite[53]{deWolf1971} even suggests two of
the Class~1/2 noun reconstructions *\emph{-tata} `father' and *\emph{-mama}
`mother' were likely to have not been coded with a prefix in the singular,
thereby implicitly reconstructing something along the lines of the class
designated as Class~1a in the Bantuist literature (see, e.g.,
\namecite[74]{Maho1999} and \namecite{VandeVelde2006}). The Class~1a is used to
classify nouns not showing the usual Class~1 coding but otherwise behaving like
Class~1 nouns with respect to concord. The next most robust semantic
associations are those of the Class~9/10 pairing with animals (though not
exclusively so) and Class~6a with liquids.{\footnote{The convention of referring
to the unpaired class with a nasal consonant and associated with liquid
substances as 6a is due to \namecite[163]{Welmers1974} and is connected to the
fact that Classes 6 and 6a are homophonous in \ili{Bantu}.}} The Class~5/6 pairing
contains many body parts, as does the Class~15/6 pairing. While
\namecite[59]{deWolf1971} only gives four reconstructed nouns for unpaired
Class~14 (with the meanings `fear', `life', `pain/ache', and `witchcraft'),
these can all be interpreted as referring to abstract entities.

\largerpage
The other classes and pairings do not show such straightforward semantic
categorization, and, on the whole, it appears that the results of studies of the
semantics of \ili{Bantu} noun classes can also be applied to the \ili{East Benue-Congo}
languages, which have not seen as detailed investigation in this domain (see
\namecite[55--88]{Maho1999} for the most recent detailed survey of work on the
semantics of \ili{Bantu} noun classes). Specifically, while it is
not difficult to identify semantic tendencies in the distributions of nouns into
various classes and class pairings, it has proven impossible to devise a set of
semantic principles that fully cover these distributions, and a degree of
lexical arbitrariness in assignment seems unavoidable.{\footnote{Of course, we
must also allow for the possibility that, in some cases, formal factors may have
played a role in class assignment. This probably accounts, for instance, for the
fact that loanwords from languages like English referring to non-humans can be
placed in the Class~1/2 pairing in some \ili{East Benue-Congo} languages (see,
e.g., \namecite[118--119]{LovegrenMungbam} on \ili{Mungbam}). Since English nouns
will not begin with any sort of class prefix (unlike potential borrowings
between other \ili{East Benue-Congo} languages), they formally resemble Class~1a
nouns, presumably accounting for such Class~1/2 assignments. Formal factors have
also been implicated with respect to class assignment patterns in \ili{Bantu}
languages (see, e.g, \namecite[91]{Schadeberg2009LW}).}} To the extent that
domains of semantic regularity are interspersed with varying degrees of
arbitrariness throughout \ili{East  Benue-Congo}, this seems to be the most
reasonable way to reconstruct the system of the proto-language. If there ever
was a time when the system was semantically regular, it would have presumably
been at a more ancient time depth.

The reconstructions seen in \tref{PBCDeWolf} were undertaken at a time when
crucial data had begun to become available, but \namecite[21]{deWolf1971} still
felt the data he had access to was insufficient in various ways. By contrast,
today, the problem would not seem to be a lack of data -- quite a lot on the noun
class systems of \ili{East Benue-Congo} languages has been published in intervening
decades (see, e.g., \namecite{Hyman1980borderland}, \namecite{HymanVoorhoeve1980} for
two collections of studies coming out in the decade following
\namecite[21]{deWolf1971}, which represent merely the tip of the iceberg in
this regard). Rather, the problem is that the data has, on the one hand, not
been properly synthesized given the relatively low priority of comparative work
in the field of linguistics in the last half century, and, on the other hand,
simply fails to yield straightforward patterns. It does seem clear that progress
could be quickly made within low-level subgroups if this was deemed a priority.
The work of \namecite{Connell1987} on the reconstruction of the \ili{Lower Cross River}
noun class system is exemplary in this regard, though work of this type does not
appear to be particularly common. At the same time, we must acknowledge that
local patterns of language contact among multilingual populations should be
expected to obscure genealogical signals in many cases throughout the \ili{East 
Benue-Congo} area. This means that any procedure assuming a simple path for the
reconstruction of \ili{Proto-Benue-Congo} via a series of discrete intermediate
subgrouping nodes is bound to run into difficulties (see, e.g.,
\namecite{DiCarloGoodIdeology} for a relevant case study and
contextualization). This is not to say that such work should not be undertaken.
Rather, it is simply important to take into account the realities of language
use and development in the \ili{East Benue-Congo} area when engaging in efforts at
reconstruction.

The most emblematic phenomenon seen in \ili{East Benue-Congo} noun class data that
has resisted straightforward analysis via subgrouping concerns the historical
status of nasal consonants in some of the class markers associated with Classes
1, 3, 4, 6, 9, and 10. These classes show nasals in their nominal prefixes in
\ili{Bantu} languages, leading to their reconstruction with nasals for \ili{Proto-Bantu},
but the distribution of these nasals in \ili{non-Bantu East Benue-Congo} (and
beyond) is much more complex, and there is, as yet, no consensus on their status
in \ili{Proto-Benue-Congo} (see \namecite{Hyman1980} and \namecite{Miehe1991} for
discussion; see also Hyman, chapter 7, this volume, for a current evaluation of 
these nasal classes and their possible origins).

No attempt will be made here to revise the specific reconstructions of
\namecite{deWolf1971}. This is partly because the time that would be required
to do so would be prohibitive and properly reporting on any such efforts would
almost certainly necessitate monograph-level discussion. However, there is also
a more principled reason for this. Consistent with practices of the time, the
scope of \quotecite{deWolf1971} reconstructions is relatively limited: Specific
forms and pairings are proposed, but it must be recognized that, in \ili{East 
Benue-Congo} languages, these are merely elements of a larger noun class
\emph{system} (see \namecite{Good2012}), which is associated with a range of
morphosyntactic properties. Moreover, while there has not been comprehensive
work specifically reconstructing the broader morphosyntactic properties of the
\ili{Proto-Benue-Congo} noun class system, there has, in many cases, been enough work
to allow for preliminary proposals to be made -- or at least for promising
possible alternatives to be outlined.

The rest of the discussion here, therefore, will look at various morphosyntactic
features of \ili{East Benue-Congo} noun class systems where available work makes it
possible to seriously consider issues of reconstruction. Specifically, \sectref{sec:good:2} considers 
the general direction of change assumed for \ili{East Benue-Congo} noun class 
systems, \sectref{sec:good:3} examines the significance of the presence of noun class suffixes 
(as opposed to prefixes) in the family, \sectref{sec:good:4} discusses which morphosyntactic domains 
were most likely to be domains of concord, \sectref{sec:good:5} raises issues with respect to the
presence of different form classes for concordial elements, and \sectref{sec:good:6}
looks at cases where a noun class's identity may be difficult to uniquely
reconstruct due to complex patterns of change. A brief synthesizing conclusion
is offered in \sectref{sec:good:7}. These topics are not chosen because they exhaust
all the points of potential interest with respect to \ili{East Benue-Congo} noun
classes. Rather, they represent features where significant work has already been
done and which seem to be especially revealing with respect to coming to a
better understanding of the system as a whole.{\footnote{In choosing to focus on
possibilities for system-level morphosyntactic reconstruction here, I do not
mean to suggest that continued work on reconstructing the phonological shapes of
specific class markers is not also an important endeavor within comparative
\ili{Benue-Congo} studies. I see these two lines of inquiry as complementary rather
than being in opposition.}}



%%%%%%%%%%%%%%%%%%%%%%%%%%%%%%%%%%%%%%%%%%%%%%%%%%%%%%%%%%%%%%%%%%%%%%%%%%%%%%%%
\section{``Drift'' in Benue-Congo noun class systems \label{Drift}}\label{sec:good:2}
%%%%%%%%%%%%%%%%%%%%%%%%%%%%%%%%%%%%%%%%%%%%%%%%%%%%%%%%%%%%%%%%%%%%%%%%%%%%%%%%

A remarkable fact about \ili{Benue-Congo} noun class systems is that languages of the
family range from having some of the most elaborated such systems in the world
(as evidenced by many \ili{Bantu} languages) to having, in effect, no synchronic noun
classes (see \namecite{Good2012} for detailed discussion in a \ili{Niger-Congo}
context). Languages wholly lacking in noun classes are more strongly associated
with \ili{West Benue-Congo} (e.g., \ili{Yoruba}, \ili{Igbo}, or \ili{Edo}), than \ili{East 
Benue-Congo}.{\footnote{Information on the (either remnant or lack of) noun
classes in these \ili{West Benue-Congo} languages can be found in
\namecite[32--39]{Ogunbowale1970} for \ili{Yoruba}, \namecite[13--20]{GreenIgwe1963}
for \ili{Igbo}, and \namecite[207]{Dunn1968} for \ili{Edo}.}} However, highly
reduced systems in \ili{East Benue-Congo} are present as well, as evidenced, for 
example, by the \ili{Bantu} language \ili{Komo}, which is reported to have no noun classes
(\citeboth[42]{Guthrie1971}, \citeboth{Thomas1992}: 4), or the \ili{Bangangte}
variety of the \ili{Grassfields Bantu} language \ili{Bamileke} described by
\namecite{Voorhoeve1968}, which shows a highly reduced concord system with only
five formally distinct classes that have become disconnected from the system of
nominal singular/plural marking. Remarkable in this regard is the variation that
one finds in closely related languages, like the small \ili{Ogoni}
group, where, for instance, one language of the group, \ili{Eleme}, makes extensive
use of class prefixes on nouns, two others, \ili{Ogoi} and \ili{Khana}, show traces of noun
prefixes, and a final language, \ili{Gokana}, shows no evidence of noun prefixes
\cite[436--440]{Williamson1985} {\footnote{See
\namecite{HymanVT1970,Faraclas1986,Connell1987,Gerhardt1994} and
\namecite{Storch1997} for further discussion on specific \ili{East Benue-Congo}
subgroups.}}

While the earliest work on \ili{Niger-Congo} languages proposed that languages with
minimal class systems represented an early ``primitive'' state of language
development (see, e.g., \namecite{Jungraithmayr1990}), the present, quite
stable, consensus treats relatively elaborated systems as closer to the
historical situation. This is clearly seen in the reconstructions in
\tref{PBCDeWolf}. In this regard, the reconstructed \ili{Proto-Bantu} noun class
system can be considered relatively close to the \ili{Proto-Benue-Congo} one from a
broad typological perspective. However, it would be inappropriate to equate
\ili{Proto-Bantu} with \ili{Proto-Benue-Congo} since the evidence from the group as a whole
does not support \ili{Proto-Benue-Congo} having as elaborated a system as
\ili{Proto-Bantu}. In fact, the latter group appears to have innovated a number of its
noun classes, in particular with respect to less canonically nominal categories,
such as those associated with locative meanings, i.e., Classes 16, 17, and 18
(see \namecite[37]{Williamson1989nc}).{\footnote{However, one does find
instances of apparently ``extended'' noun classes with locative meanings in
\ili{East Benue-Congo} outside of \ili{Bantu}, such as in \ili{Mungbam}
\cite[265]{LovegrenMungbam} and \ili{Noni} \cite[15--16]{Hyman1981}, both
non-Grassfields languages spoken at the northern edge of the \ili{Grassfields Bantu}
area. \namecite[243--244]{Watters2003} gives more detailed discussion on this
point (see also \namecite{Gregoire1983}). This suggests that, if the
development of such locative classes is treated as the result of a single
innovation taking place after the breakup of \ili{Proto-Benue-Congo}, this would have
to be of an older time depth than \ili{Proto-Bantu} (with the usual disclaimers
regarding the possibility of areal diffusion applying).}} Thus, \ili{Proto-Bantu} is
generally treated as having around twenty noun classes
\cite[51]{Maho1999}, while \quotecite{deWolf1971} reconstruction of
\ili{Proto-Benue-Congo} has only fifteen.

There have been statements in the literature attributing the presence of reduced
noun class systems in \ili{Niger-Congo} in general, and \ili{Benue-Congo} more specifically,
to be the result of ``drift{\ldots}in the direction of the simplification of
the nominal classification system'' \cite[9]{Greenberg1966Africa} (see also
\namecite[188]{deWolf1971} and \namecite[161--162]{Jungraithmayr1969}). This
assessment is presumably connected to the fact that one sees reduced systems in
the majority of \ili{Benue-Congo} groups (to varying degrees), while it is much more
difficult to find languages that evince the total number of reconstructed noun
classes (see, e.g., \namecite[188]{deWolf1971} on \ili{Benue-Congo} and
\namecite[51]{Maho1999} on \ili{Bantu}).

However, there are reasons to doubt the validity of ``drift'' as an explanatory
factor in the development of \ili{Benue-Congo} noun class systems. First, there is no
obvious general historical mechanism that can be associated with drift. So, its
utility as a label for patterns of change is not clear. Second, as discussed in
\namecite[322--324]{Good2012}, there are a number of distinct mechanisms
involved in the breakdown of noun classes that are not obviously interconnected,
suggesting that their reduction is not due to some general pattern of ``loss''
but, rather, to independent changes which happen to co-occur in some \ili{Benue-Congo}
languages. Third, much of the apparent drift can be more concretely attributed
to areal patterns affecting \ili{Niger-Congo} languages in the \ili{Kwa-Benue-Congo}
subregion of the so-called \ili{Macro-Sudan Belt} (see
\namecite{GueldemannMacroSudan}, as well as
\namecite[37]{ClementsRialland2008}).{\footnote{See \namecite{GoodNCareal} for
an overview of areal linguistic patterns in \ili{Niger-Congo}.}} \ili{Niger-Congo} languages
in this region have been generally subject to processes of morphological
reduction, in some cases clearly triggered by independent patterns of
phonological reduction (see \namecite{Hyman2004} and \namecite{Good2012}),
but these are probably relatively recent in nature when set against the broader
genealogical diversification of \ili{Niger-Congo} (see \namecite{Hyman2011}). This
suggests that many of the observed reductions are not attributable to a gradual
process of ``drift'' but, rather, more recent effects of contact. Finally, it
is worth mentioning that one can only characterize \ili{Benue-Congo} noun class systems as
tending towards reduction if one ignores \ili{Bantu} languages, where the pattern, if
anything, goes in the opposite direction.

\largerpage[-1]
To these remarks, one might raise a possible methodological concern: Could it be
the case that the application of the comparative method in the domain of noun
class systems may accidentally tend towards the reconstruction of larger systems
over smaller ones? Indeed, it is striking that both the \ili{Proto-Benue-Congo}
reconstructions and the \ili{Proto-Bantu} ones give a relatively high number of noun
classes when set against attested patterns in the daughter languages. One must
wonder to what extent this reflects historical reality as opposed to being an
epiphenomenon of a reconstruction methodology which might cause a proto-language
to ``accrete'' features over the course of comparative analysis. This is not to
say that reduction of noun class systems within \ili{East Benue-Congo} is not a
historically real process, as evidenced by languages showing highly reduced
systems or entirely lacking in functioning systems discussed above. Rather, it
is to suggest that one must be cautious when assuming that a relatively robust
attested noun class system is necessarily reduced because it may lack some
distinctions reconstructed for some earlier historical stage.

In any event, given the extensive body of work in linguistics on language
contact and linguistic areas since the time of \namecite{deWolf1971}, a
fruitful direction for near-term studies of high-level patterns of change in
\ili{Benue-Congo} noun class systems would be to explore their development in
terms of areal linguistic patterns in Africa, in particular looking for evidence
of their differential development in distinctive cultural regions where
\ili{Benue-Congo} languages are found. Once the descriptive picture is better
established in this regard, the stage would be set for an examination of
genealogical patterns which takes areal insights appropriately into account.




%%%%%%%%%%%%%%%%%%%%%%%%%%%%%%%%%%%%%%%%%%%%%%%%%%%%%%%%%%%%%%%%%%%%%%%%%%%%%%%%
\section{Prefixal and suffixal morphology \label{PrefSuf}}\label{sec:good:3}
%%%%%%%%%%%%%%%%%%%%%%%%%%%%%%%%%%%%%%%%%%%%%%%%%%%%%%%%%%%%%%%%%%%%%%%%%%%%%%%%

A general puzzle for the reconstruction of noun class systems in \ili{Niger-Congo} is
the fact that languages of the family do not consistently show only noun class
prefixes, but can also show noun class suffixes, or a complex mix of prefixes
and suffixes (see, among others, \namecite[252--254]{Hoffman1967},
\namecite[180--182]{deWolf1971}, \namecite[15]{Welmers1971},
\namecite{Greenberg1977,Greenberg1978Gender}, \namecite{Childs1983},
\namecite[31--37]{Williamson1989nc}, and
\namecite[378--381]{Dimmendaal2001ATR}). While this is an issue that is general
to \ili{Niger-Congo} rather than being specific to \ili{East Benue-Congo}, \ili{East 
Benue-Congo} is also implicated given that one finds both prefixing and
suffixing patterns in the family. Prefixing patterns unquestionably dominate
(even if we were to exclude the mostly exclusively prefixing systems of the
\ili{Bantu} languages), and this is presumably why \ili{Proto-Benue-Congo} has been
reconstructed as prefixing in its noun class system. However, this does not mean
that the presence of suffixing patterns does not raise significant questions for
the reconstruction of the properties of the noun class system on the whole nor
that suffixing noun class marking, or even circumfixal class marking (as
suggested by \namecite[205--210]{Welmers1974}) -- whether throughout the system
or only in part of it -- should not be considered a possibility for
\ili{Proto-Benue-Congo}.{\footnote{Resolving this issue would be more straightforward
if \ili{East Benue-Congo} subgrouping were more secure so that work could reference
clear-cut instances of innovation rather than relying on a “majority-rules”
approach for linguistic reconstruction.}}

\newpage 
Of particular interest are languages where the presence of nominal prefixes or
suffixes is dependent on a noun's morphosyntactic context. For instance, in the
\ili{Kainji} language \ili{C'lela} (\ili{Dakarkari}), nouns in citation forms will show a prefix,
as in \emph{d-hyí} `head', whereas this prefix is not present when the noun is
followed by a concordial element, such as a demonstrative, as in \emph{hyí
d{\'ə}hnà} `this head'. (For this noun, the relevant noun class is associated
with a \emph{d}, whether on the noun itself or the demonstrative
\cite[247]{Hoffman1967}). While the \ili{C'lela} pattern is a minority one within
\ili{East Benue-Congo}, it is not unique. Similar patterns are seen, for instance,
in the \ili{Grassfields Bantu} language \ili{Aghem}
\cite[56--58]{Hyman1979phonology}.{\footnote{Apparent dropping of prefixes along the
lines of what is seen in languages such as \ili{C'lela} and \ili{Aghem} is, to the best of my
knowledge, essentially unreported for \ili{Bantu} languages with the exception of what
is described for \ili{Sesotho} in \namecite{DemuthEtAl2009}.}} In the \ili{Cross River}
language \ili{Efik} one sees the ``reverse'' of this pattern, where a limited set of
nouns, when modified by adjectives, appear with a prefix that is not found in
isolating forms (\namecite[45]{Faraclas1986}, citing
\namecite[179--181]{Cook1969}).

As pointed out at least as early as \namecite[253]{Hoffman1967} (see
\namecite[380]{Dimmendaal2001ATR} for a recent overview), the nature of
\ili{Niger-Congo} noun class systems, where concordial elements such as demonstratives
can frequently be found adjacent to a noun, opens up possibilities for the
reanalysis of the concordant segments as coding class on the noun itself. Thus,
when one considers a phrase like the \ili{C'lela} expression \emph{hyí d{\'ə}hnà}
`this head', just cited above, a resegmentation of the phrase along the lines of
\emph{hyíd {\'ə}hnà} could, in principle, result in a noun coded for its class
suffixally. This sort of resegmentation would presumably be more likely in
contexts where prefixes are not present on the noun since, otherwise, it would
result in multiple exponence of class on nouns via a less typical circumfixal
structure. Therefore, it would seem to make sense to see patterns of prefix
absence and the presence class suffixes as potentially interrelated phenomena.
At the same time, it must be admitted that there are cases where the
distribution of prefixing and suffixing patterns does not point in any clear
direction regarding their historical relationship. This is seen, for instance,
in the \ili{Mambiloid} language \ili{Vute}, where nouns can appear with both prefixing and
suffixing elements that are relatable to \ili{Proto-Benue-Congo} noun class markers
but which do not appear to interact with each other \cite[69--71]{Thwing1987}
(see also \namecite[111--112]{Blench1993} for further discussion of suffixing
class markers in \ili{Mambiloid}).{\footnote{For instance, while some nouns are marked
with prefixes, plurals are generally formed via suffixation, and it appears from
\quotecite{Thwing1987} description that the addition of a suffix to a noun to
code plurality is not associated with the loss of a prefix historically
associated with singular coding.}}

\newpage 
In this context, it is worth revisiting a tendency in the literature to view
cases such as \ili{C'lela} prefix absence as involving dropping of the prefix
(see, e.g., \namecite[246]{Hoffman1967} or \namecite[27]{Hyman1979phonology}). This is
presumably based on an intuition that the citation forms of nouns are in some
sense more morphologically ``basic'' than modified forms. However, there is no
logical reason why prefixed forms could not be considered to be augmented 
with a prefix treated as coding a category such as ``lack of modification''.
And, in fact, such an analysis becomes more plausible given the well-known
presence of a formative commonly referred to as an \emph{augment} (or
\emph{pre-prefix}) in many \ili{Bantu} languages (see
\namecite[107--108]{Katamba2003} for an overview discussion,
\namecite{deBlois1970} for a detailed survey, and \namecite{Williamson1993}
for consideration of the augment in the context of \ili{Benue-Congo} reconstruction).
This element immediately precedes the class prefix on nouns and often has a form
that copies the prefix in whole or part. It is difficult to assign it a
unique, general function. Its appearance can be determined by apparent
referential factors (e.g., definiteness) but can also exhibit a degree of
sensitivity to grammatical control (e.g., being sensitive to whether or not a
verb is negated) (see, e.g., \namecite{HymanKatamba1993} for a detailed
investigation of the functions of the augment in \ili{Ganda}).{\footnote{Within the
\ili{East Benue-Congo} area, \namecite[74--75]{Boum1980} describes a similar
pattern of double prefixation in two languages of the \ili{Menchum} subgroup of
\ili{Grassfields Bantu} where nouns in certain classes show evidence of being coded
with two prefixes in citation forms, with the initial one of these not appearing
in locative and possessive contexts (see also \namecite[241]{Watters2003}
and \namecite{Hyman2005}).}}

The general prevalence of the marked nominative language type in Africa is also
relevant here (see K\"onig \citeyear{Konig2006},
\citeyear{Konig2008}: 138--203). In effect, forms in languages of this type
associated with more ``nominative'' domains (such as subjects) are
morphologically more complex than forms used in more ``accusative'' domains, are
found in a more functionally restricted range of environments, or show both
classes of properties. This suggests, in general, that we should be wary of
assuming that classificatory heuristics from European languages (such as
``citation is the same as basic'') will naturally carry over into \ili{East
Benue-Congo} languages. Furthermore, as discussed in
\namecite{CreisselsConstruct}, while it has not yet been widely explored, one
seems to find relatively frequently in Africa cases of head-marking in noun
phrases where what is coded is that the head is associated with some dependent
in its phrase. This indicates that we may want to view cases of apparent prefix
dropping in a language like \ili{C'lela} not as one noun form being derived from
another but, rather, as evincing a kind of inflectional nominal paradigm of some
kind, where each form of the noun is actively coding a specific morphosyntactic
category with respect to its relationship to a larger syntactic construction.

\newpage %longdistance
One factor that may have obscured this as a potential analysis is the fact that
the functional range of such paradigmatic oppositions does not map neatly onto
categories familiar from analyses of European languages, such as definiteness or
case.{\footnote{To pick one well-described example, \quotecite{Schadeberg1986}
description of tonal cases in the \ili{Bantu} language \ili{Umbundu} includes the category
of Common Case which covers such functions as subject, second complement of
ditransitive verb, object of a negative verb, and object of a progressive verb,
among others \cite[433--437]{Schadeberg1986}.}} Another reason that such an
analysis has presumably not been actively proposed is that variable prefix
presence has not been reported in most \ili{East Benue-Congo} languages (especially
if we include Narrow \ili{Bantu} languages in this category), meaning that an abstract
analysis of this kind would not be motivated by direct evidence in the majority
of cases.{\footnote{I am thankful to John Watters for the latter observation.
Whether the analysis of prefixes as being part of some kind of inflectional
nominal paradigm of the sort just suggested should be applied to all \ili{East Benue-Congo}
languages with productive noun classes or just that subset showing
variable prefix presence is a question of synchronic analysis that lies outside
the scope of the present chapter.}} While these remarks pertain more directly to
synchronic analysis than historical concerns, a more accurate understanding of
these synchronic systems can play an important role in reconstructing a
\ili{Proto-Benue-Congo} noun class system that is more reflective of the actual
morphosyntax of the \ili{East Benue-Congo} parent language.

\largerpage[-1]
When we come back to consideration of these patterns in the broader \ili{Benue-Congo}
picture, the question arises as to whether or not we should view the
\ili{Proto-Benue-Congo} system, as depicted in \tref{PBCDeWolf}, as relatively
well-behaved, adhering to a \ili{Bantu}-like canon (even if there are fewer overall
classes) where noun classes are almost exclusively coded with some prefix,
excluding narrow and systematic exceptions of the sort associated, for instance,
with \ili{Bantu} Class~1a (see \sectref{sec:good:1}). Alternatively, we might want to
consider what features of \ili{Proto-Benue-Congo} could have resulted in relatively
distant languages such as C'lela and \ili{Aghem} (one spoken in northwest Nigeria and
the other in northwest Cameroon) to have developed in similar directions with
respect to alternations between prefixed and non-prefixed nouns. There has not
been any general survey on patterns of prefix absence to the best of my
knowledge, and, if anything, it is probably underreported since it is not a
pattern necessarily easily detected in basic elicitation, such as when
collecting wordlists. It is also important in this regard to consider the
relatively well attested pattern where an \ili{East Benue-Congo} language may be
primarily prefixing but also show some suffixal or circumfixal noun class
marking (whether appearing on nouns or as concords). Such patterns were
recognized by \namecite[181]{deWolf1971}, and new examples have since been
attested, as seen in, for example, the overview of the noun class systems of
\ili{Naki} (Mekaf), \ili{Mungbam} (Missong), and \ili{Noni} (all non-Grassfields \ili{Bantu} Bantoid
languages spoken in the north of the Grassfields area) as presented in
\namecite[87--88]{Hombert1980}.
 
\Citet[182]{deWolf1971} appears to view the issue of understanding the
suffixing patterns through a dichotomous lens where \ili{Proto-Benue-Congo} would be
viewed as either prefixing \emph{or} suffixing. Given such a choice, it seems
likely that \ili{Proto-Benue-Congo} was much closer to a prefixing prototype than a
suffixing one. But, we might still consider whether \ili{Proto-Benue-Congo} may have
allowed for prefixes on nouns to be dropped in certain contexts, thereby
creating favorable conditions for the rise of suffixing class patterns in some
cases. In other words, as part of the reconstruction of the noun class
\emph{system} of \ili{Proto-Benue-Congo}, we should bear in mind that its properties
clearly resulted in the potential for its daughter languages to develop suffixal
class-marking patterns and consider what sort of system would have been likely
to have promoted such developments. This remains an important open area
of research on comparative \ili{East Benue-Congo} noun class systems.

\largerpage[-2]
A final point worth raising in this regard is the possibility for reconstructing
word order within the noun phrase in \ili{Proto-Benue-Congo}. I am not aware of this
topic having received much attention, perhaps because of the relative
homogeneity of \ili{East Benue-Congo} languages in key domains, such as a strong
tendency towards head-initial structures, resulting in patterns such as
Noun-Demonstrative order being well-attested (see, e.g.,
\namecite{Dryer2013DemN}). However, there are cases reported of alternative
orders being possible in specific contexts (for instance to encode emphasis). In
such cases, one may find Demonstrative-Noun ordering in languages where the
reverse order generally predominates. For example, \namecite{VandeVelde2005}
discusses this in some \ili{Bantu} languages, and \namecite[254--255]{Watters1981}
and \namecite[182]{LovegrenMungbam} give attestations of this in \ili{Bantoid}
languages (see also \citealt[248]{Watters2003}). This seems likely to be a
relatively common pattern, though I am not aware of any systematic study of it.
To the extent that noun phrases in \ili{Proto-Benue-Congo} probably tended to be
head-initial, grammaticalization processes could be expected to more often
create innovative suffixal class-marking patterns along the lines of what was
outlined for \ili{C'lela} above. However, less common word order patterns where a
concordial element such as a demonstrative may have preceded the noun could
allow for new prefixal class marking to develop, perhaps helping us understand
the rise of, for instance, the pre-prefixing augments found in \ili{Bantu} languages,
just discussed (see also \citealt[99]{Meeussen1967}).{\footnote{This
possibility raises broader questions about the role of augmentation in
accounting for the shape of noun class prefixes in \ili{East Benue-Congo}
languages, whether in the form of the so-called augment, just discussed above,
or some other kind of morpheme which would result in something comparable to the
augment in terms of form, if not necessarily function (see, e.g.,
\namecite[337]{Hyman2005}). \namecite[381--382]{Dimmendaal2001ATR} discusses
evidence suggesting that the presence of the augment is quite old within
\ili{Niger-Congo} (see also \namecite{Williamson1993}). This would open up the
possibility for it to have played a role in shaping noun class prefixes
throughout \ili{East Benue-Congo} via parallel developments in different branches.
The details of such processes, at this stage, remain somewhat speculative.}}




%%%%%%%%%%%%%%%%%%%%%%%%%%%%%%%%%%%%%%%%%%%%%%%%%%%%%%%%%%%%%%%%%%%%%%%%%%%%%%%%
\section{Domains of concord \label{Concord}}\label{sec:good:4}
%%%%%%%%%%%%%%%%%%%%%%%%%%%%%%%%%%%%%%%%%%%%%%%%%%%%%%%%%%%%%%%%%%%%%%%%%%%%%%%%
\largerpage[-2]
As discussed in \sectref{sec:good:2}, \ili{Bantu} languages are generally taken to be
conservative with respect to maintenance of the general structure of the
\ili{Proto-Benue-Congo} noun class system, though they may have innovated certain
classes. A comparison between \ili{Bantu} and the rest of \ili{East Benue-Congo} is also
relevant in this regard with respect to the domains where noun class concord is
found. That there must have been some kind of agreement relation between head
nouns and certain classes of associated elements is without question. However, what is
not fully resolved is which grammatical classes those elements would have
belonged to.

\citet[182--185]{deWolf1971} gives an overview of where concord was found in
the languages he examined most carefully in his study, providing an
exceptionally fine-grained list of environments where it was attested.
Generalizing over his categories, throughout the family as a whole, the
following domains are relevant: (i)~nominal dependents, including
demonstratives, adjectival elements (to the extent that they are present),
numerals, possessive pronouns, and modifying interrogatives, (ii)~verbs and
verb-like elements (e.g., copulas), where subject concord is often found (to be
discussed further below), (iii)~pronouns of various kinds, and in particular
anaphoric pronouns, where a prominent feature of many \ili{East Benue-Congo}
concord systems is a large class of third-person pronouns agreeing with the
class of their referent, and (iv)~associative markers and relativizers, which
can agree with the noun preceding them.{\footnote{The associative marker and
relativizer are possibly analyzable as nominal dependents, therefore belonging
to class (i) above, though in their role as connective elements between
syntactic constituents, their dependency relationships are not as obvious as for
elements such as demonstratives and adjectives, which is why they are given
their own category here.}} While not a domain of concord, per se, to this we
might also add another domain of marking: (v)~nouns themselves, specifically
when they show overt marking of their class via some sort of affixal coding. The
logic for adding this final class is that, from the perspective of a formal
reconstruction of the properties of the \ili{Proto-Benue-Congo} noun class system,
the presence/absence of class marking on nouns can vary more or less along the
same lines as its presence/absence in more properly syntactic domains in the
daughter languages.

No more thorough follow-up study of the domains of concord at the level of
\ili{East Benue-Congo} appears to have been undertaken, and this would seem to be
an area where a more detailed survey would lead to worthwhile results, perhaps
leading to robust generalizations regarding where concord is more likely to be
maintained or lost. Still, even a cursory examination of the results in
\namecite[184]{deWolf1971} shows that absolute patterns are unlikely to be
uncovered, given that a wide range of logical possibilities for combinations of
class coding across domains are attested. Of particular relevance for purposes
of reconstruction is work such as that of \namecite[467]{DemuthEtAl1986}, who
propose that class coding on concordial forms is more resistant to loss as a
result of language change than nominal class coding.
\namecite[381]{Dimmendaal2001ATR} further puts forth the idea that, when coding
on nouns survives where agreement is lost, this can be explained as the effect
of contact. \namecite{Good2012} presents a more equivocal picture about the
relative historical stability of these two types of noun class marking. However,
if more systematic studies revealed the robustness of concord as a significant
tendency, it would suggest that future work on reconstruction of the noun class
system of \ili{Proto-Benue-Congo} should privilege evidence from patterns of
agreement over class marking on nouns as more likely to represent archaic
features. A useful step forward for further examination of this issue would be
to arrive at a more detailed understanding of areal patterns of nominal class
coding and class concord, including consideration of languages where only
remnant patterns are found in order to clarify if any apparent typological
generalizations may be better understood as contact effects.


\largerpage[-2]
Even if we accept that noun classes are more robustly coded via patterns of
agreement than via nominal prefixes, there is still the question of which
precise domains would have shown agreement in \ili{Proto-Benue-Congo}. In some cases,
such as demonstratives, third-person pronouns (see Hyman, Chapter 6, this volume, 
on third-person pronouns in Grassfields), and possessive pronouns, concord
is found in a sufficiently diverse range of the family's languages (see, e.g.,
\namecite[184]{deWolf1971}) that it seems necessary to reconstruct it for the
proto-language given that concord has to be reconstructed somewhere in the
system.{\footnote{See \namecite{Kiessling2013} for discussion of attested
numeral classifier systems in \ili{Niger-Congo} languages, including many \ili{East Benue-Congo}
languages, which can potentially serve as models for the initial
development of the \ili{Niger-Congo} noun class system at some ancient stage of the
language. By the time we can sensibly speak of \ili{Proto-Benue-Congo}, however, it
seems necessary to assume that a strongly grammaticalized noun class system was
already present.}} Nominal prefixes are comparable in this regard, since the
alternative would be to posit an improbably massive number of parallel processes
of grammaticalization resulting in nominal prefixes in languages throughout the
family.

At the same time, if we assume \ili{Proto-Benue-Congo} had a fairly transparent
noun class system in some domains of its grammar, it is also clear that
processes of analogical extension and grammaticalization could have served to
extend noun classes to domains where they might not have been found in the
proto-language. Here, data from \ili{Bantu} languages becomes useful simply by virtue
of their degree of morphological elaboration and the fact that their comparative
linguistics is relatively well understood.
\namecite[386]{GueldemannHabilitation}, for instance, gives a reconstruction of
a grammaticalization pathway for a \ili{Proto-Bantu} element \hbox{*\emph{-ti}},
associated with quotative marking \cite[105]{Guthrie19703}, where it began as
an uninflecting manner-marking element (perhaps comparable to English
\emph{like}) but later developed verbal properties. One of these properties is
an ability to appear with subject concord marking, as generally found for \ili{Bantu}
verbs. Another such example involves a complementizer in the \ili{Bantu} language
\ili{Lwena}, which shows suffixal concord with the subject of its matrix clause (see
\namecite[453]{GueldemannHabilitation}, drawing on the description of
\namecite[181--182]{Horton1949}). (\namecite[832--836]{Idiatov2010} offers
more general discussion of this kind of agreement.) It may be possible to
reconstruct some degree of subject concord for *\emph{-ti} in \ili{Proto-Bantu}.
However, it is hard to consider the subject coding found in \ili{Lwena}
complementizers as representing anything other than an innovation in \ili{Bantu} terms
for this part of speech given its suffixing form. Thus, contrary to the
implications of the drift metaphor (see \sectref{sec:good:2}), we must admit the
possibility that \ili{Proto-Benue-Congo} may have exhibited concord in more limited
domains than what is found in the daughter languages, with its appearance in
other domains due to later changes. That is, morphological coding of noun
classes should not automatically be understood to represent the conservative
situation. Working out the details, however, will have to await further,
targeted study.

\largerpage[-1]
A comparatively controversial case of a concord domain in this regard involves
subject coding on the verb by means of a prefix.
\namecite[123--129]{Gueldemann2011} (see also
\namecite[184--185]{Gueldemann2003}), for instance, argues that the pattern of
subject concord (as well as object concord) on the verb seen in \ili{Bantu} languages
should be historically interpreted not as evidence for the historical presence
of such an agreement pattern in a higher-level grouping such as \ili{East Benue-Congo}
but, rather, as the result of a comparatively recent process of
grammaticalization. Specifically, an S-Aux-O-V syntagm provided the seeds for the
development of a verbal structure which is prefixally inflected for subject
concord and tense-mood-aspect marking, as well as object marking.{\footnote{The
coding of the class of object arguments on verbs in \ili{Bantu} languages is often not
clearly an example of concord since the appearance of the so-called object
markers is not obligatory in all languages in cases where an overt object is
present, suggesting that these markers behave more along the lines of
pronominals. See \namecite[124]{Bearth2003}. From a diachronic perspective,
this suggests their appearance may result from a comparatively recent process
of entrapment of object pronouns into a univerbating verbal complex, at least
when set against subject coding on the verb, which is much more strongly
associated with ``true'' grammatical agreement.}} (See
\namecite{Gueldemann2007} for general discussion of such preverbal object
structures in \ili{Benue-Congo}.) \namecite[21--40]{Hyman2011}, by contrast, provides
evidence supporting a treatment of prefixal inflection on verbs in \ili{Niger-Congo}
(and, by extension, \ili{East Benue-Congo} as having a comparatively old time
depth. While he does not propose specific reconstructions regarding subject
concords, there is a clear implication that he believes that the possibility
that they were present at a genealogical level well above Narrow \ili{Bantu} should be
seriously considered.

While \namecite{Gueldemann2011} is focused on Narrow \ili{Bantu}, the core of
his argument could apply just as well to \ili{East Benue-Congo} languages showing
phonologically fused instances of subject marking that strongly suggest a
prefixal analysis is appropriate, such as the \ili{Cross River} language \ili{Eleme}
\cite{Bond2010}. This then raises the question: Given that grammaticalization
scenarios could be developed where other domains of concord (such as
demonstratives or third-person pronouns, just discussed) could be viewed as
arising from more analytic structures, why treat subject concord differently? In
this case, significant considerations would seem to be as follows: On the one
hand, concord must be reconstructed in some domains unless we set aside the idea
that it is one of the defining historical features of \ili{East Benue-Congo}, and
the pervasiveness of concord in domains such as demonstratives and third-person
pronouns makes them strong candidates for having been concord domains in \ili{East 
Benue-Congo} as mentioned above. On the other hand, there are clear
constructional sources through which subject and object concord could have
developed, and these are found even in contemporary languages lacking such
concord. Potential sources of other kinds of concord elements are otherwise
unclear (or, at least, require more speculation).

\newpage 
Nevertheless, as made clear by the discussion in \namecite[29--40]{Hyman2011},
there are reasons to doubt any overly simplistic story for the presence/absence
of concord in any particular domain in a group as old as \ili{East Benue-Congo},
and the issue of whether or not subject concord was present must be considered
unresolved, even if some plausible hypotheses can be put forward. We are, thus,
left with an analytical problem: There is a reasonable diachronically shallow
pathway that can be proposed for the development of subject concord in \ili{East 
Benue-Congo} languages, but there are also patterns that suggest verbal prefixal
morphology may be quite old. At this point, one can merely say that \ili{East 
Benue-Congo} might have showed subject concord but that this is a less likely
concord domain than that of, say, demonstratives or third-person pronouns.



%%%%%%%%%%%%%%%%%%%%%%%%%%%%%%%%%%%%%%%%%%%%%%%%%%%%%%%%%%%%%%%%%%%%%%%%%%%%%%%%
\section{Concord form classes \label{FormClasses}}\label{sec:good:5}
%%%%%%%%%%%%%%%%%%%%%%%%%%%%%%%%%%%%%%%%%%%%%%%%%%%%%%%%%%%%%%%%%%%%%%%%%%%%%%%%

In addition to the issue of where concord was present in \ili{Proto-Benue-Congo},
there is a further concern regarding how many different series of noun class
markers there might have been. The most prominent classes where this question is
relevant are almost certainly those associated with the \ili{Bantu} nasal classes (see
\sref{Overview}), where a nasal is found in the consonantal position of CV-
nominal-marking class prefixes but not in other class-marking domains such as
verbal person-coding prefixes (see, e.g, \namecite[97--98]{Meeussen1967}). In
the {East Benue-Con\-go} case, the possibility of different series of class
markers can be seen directly in \quotecite{deWolf1971} reconstructions of a
distinctive nominal prefix series and concordial series as presented in
\tref{PBCDeWolf}. It seems reasonable, therefore, to assume that there were at
least two distinct series of noun class markers in
\ili{Proto-Benue-Congo}.\footnote{This 
  is not to say that some variety preceding
  \ili{Proto-Benue-Congo} necessarily had two distinct series since, at least for some
  cases, it would be straightforward to apply internal reconstruction to
  \quotecite{deWolf1971} \ili{Proto-Benue-Congo} system to propose an earlier stage
  with less variation. (This could, in particular, involve proposing that certain
  class prefixes on nouns, such as Class 6, were subject to initial consonant loss
  which did not affect consonants in all concordial forms.)
  } In many cases, the
markers for specific classes would be formally identical, creating alliterative
patterns of concord.{\footnote{Patterns of alliterative concord are still found
in noun class systems throughout \ili{East Benue-Congo}, though reconstructed
alliteration for any given noun class can often be lost due to historical
processes such as sound change. New patterns of alliteration can also emerge in
cases where new noun classes develop analogically on the basis of existing ones,
as appears to be the case, for instance, for the \ili{Bantu} locative classes
discussed in \sectref{sec:good:2}.}} Nevertheless, in some classes there seems to have
been partial formal divergence, with Classes 1, 3, and 6, and perhaps 9 and 10,
being the most likely candidates for this, as indicated in \tref{PBCDeWolf}.
These are also classes associated with the historically problematic nasal
classes in \ili{Bantu} just discussed, and, presumably, this is not a
coincidence.{\footnote{Class~4 is also such a class, not listed here, but this
should be understood as an artifact of the presentation scheme where I
associated \ili{Bantu} class numbers with \quotecite{deWolf1971} reconstructions in a
way that collapsed a possible Class~4/Class~10 distinction on formal grounds.
See also \sectref{sec:good:6}.}}

The possibility that more than two series of concords may need to be reconstructed for 
\ili{Proto-Benue-Congo} does not appear to have received detailed attention. 
It is unambiguously the case that the noun class systems of
some languages of the family can only be described by implicitly assuming more
than two series of concord marking insofar as there is a need to present
separate concord sets for a number of word classes, e.g., demonstratives,
numerals, and adjective-like elements. This is seen in the overview
of the \ili{Noni} noun class system given in \namecite[33]{Hyman1981}. Eight series
of noun class markers are given for this language representing the following
domains: (i)~nominal prefixes, (ii)~person pronominal elements, (iii)~possessive
marking for nouns (involving, among other things, an associative marker),
(iv)~possessive pronouns, (v)~determiners, (vi)~quantifier-like elements,
(vii)~adjective-like elements, and (viii)~numerals, and even this extensive list
abstracts away from various complications for elements within these series.

Often, it is straightforward in such cases to view a wealth of concord series as
the result of various processes of change (especially sound change) impacting
different kinds of concord-stem combinations, creating a system where concord
variants need to be explicitly listed synchronically but which can be easily
seen as deriving from a simpler historical system. For instance, \ili{Noni} Class~4
forms all contain a palatal consonantal element, but this is realized as a
modification to a stem-initial consonant in some cases rather than as a true
prefix. Thus, forms for the word `new' in \ili{Noni} are based on a stem
\emph{-fε} and can appear with an unambiguous prefix as in the Class~2 form
\emph{bɔfε} or with a modified consonant in the Class~4 form as
\emph{fiε} \cite[26]{Hyman1981}. This Class~4 form can be set against the
Class~4 word for `this' \emph{yin} (based on a stem with a shape of -V\emph{n}),
where a full palatal consonant is found \cite[23]{Hyman1981}. There is,
however, no reason to view this as evidence for the reconstruction of a
plain  and mutating series of Class~4 concords in \ili{Proto-Benue-Congo}
given that the overall pattern is one where a full palatal consonant is found
before agreeing stems beginning with a vowel and a consonant modified with
palatalization is found for stems beginning with a consonant. This simply
suggests a sound change where a former segmental prefix with a palatal quality
(presumably along the lines of \emph{i}) before consonantal stems metathesized
and fused with the following consonant, while appearing as a palatal glide
before a vowel. Indeed, this change seems to be an instance of a localized areal
pattern found in the part of the Cameroonian \ili{Grassfields} where \ili{Noni} is spoken,
as discussed in \namecite{KiesslingMutations2010}.

Nevertheless, the fact that we can explain some of the attested complications in
series of concords as the result of straightforward processes of sound change
does not mean that we should not also consider the possibility that
\ili{Proto-Benue-Congo} had more than two series or that, in some cases,
morphophonological processes had been applied to its concord system which
would have created some forms that were partly unpredictable based purely on
knowledge of the general form of a concord prefix and the stem it attached to. I
am not aware of specific work having been done on this question, however, and it
must remain an open issue for further research.



%%%%%%%%%%%%%%%%%%%%%%%%%%%%%%%%%%%%%%%%%%%%%%%%%%%%%%%%%%%%%%%%%%%%%%%%%%%%%%%%
\section{Noun class identity and class pairing consistency \label{Identity}}\label{sec:good:6}
%%%%%%%%%%%%%%%%%%%%%%%%%%%%%%%%%%%%%%%%%%%%%%%%%%%%%%%%%%%%%%%%%%%%%%%%%%%%%%%%

Implicit in much of the discussion on noun classes in \ili{Proto-Benue-Congo} is the
idea that a noun class is a relatively stable entity, associated with a
consistent form, even if subject to different patterns of change (e.g., sound
change or analogical change). Moreover, it is easy to assume that
the singular-plural pairings may be more stable than they are in reality. To be
sure, there are pockets of stability. For instance, while I am not aware of a
study systematically verifying this, the Class~1/2 pairing seems robust both in
terms of the fact that each of its component classes is well attested and
the fact that the pairing itself is well-attested for certain nouns referring to
humans. This is presumably explainable by reference to the semantic cohesiveness
of a subclass of Class~1/2 nouns, their likely frequency of use, and the general
salience of the category \emph{human}.

However, complications to this simplified picture are not hard to find. The
clearest of these is a general lack of rigidity in singular-plural pairings.
This can be seen in \quotecite{deWolf1971} reconstructions, as schematized in
\tref{PBCDeWolf}, where, for instance, he was unable to propose a consistent
singular for Class~13, indicating it as functioning as a plural for either
Class~3 or Class~12. It is important to bear in mind in this context that
patterns of singular-plural pairing are seen as (at least partly) diagnostic in
some descriptions of the presence of a distinct class itself. This is found, for
example, in the reconstruction of distinct Classes 4 and 5 for \ili{Western
Grassfields Bantu} in \namecite[183]{Hyman1980}, which are formally identical
but differentiated by virtue of their status as coding singular versus plural
and associated patterns of pairing. It is also seen in a divergence in the
schematization of \quotecite{deWolf1971} reconstructions given in
\tref{PBCDeWolf} and the summary presented in
\namecite[38--39]{Williamson1989nc}, where she gives a distinct Class~4, which
is not seen here, alongside a formally identical Class~10.
\citet[52]{deWolf1971} does not appear to make a statement on the
relationship of the relevant \ili{Proto-Bantu} classes associated with these numbers
to his class given with form *\emph{í}. These two classes are reconstructed as
formally distinct in \ili{Proto-Bantu}, and either could be historically connected to
a class associated with *\emph{í} in \ili{Proto-Benue-Congo}.

In fact, \namecite{deWolf1971} proposes pairings consistent with the presence
of something like the Class~3/4 pairing in \ili{Proto-Benue-Congo}, as well as the
Class~9/10 pairing given in \tref{PBCDeWolf}. This leaves open the question as
to whether we should view this as evidence for a distinct Class~4 in
\ili{Proto-Benue-Congo} or whether we should treat the plurals of the relevant words
as involving something like a Class~3/10 pairing, under the assumption that there
is just one plural class with a form associated with \emph{í}. (Class~10 is
picked over Class~4 in this case due to the fact that there is greater evidence
for reconstructing a Class~9/10 pairing than a Class~3/4 one.) Obviously, the
criteria one uses as diagnostic for a distinct noun class can have a significant
outcome on the apparent consistency (or inconsistency) of singular/plural
pairings, and the resolution of cases like these requires a less than
canonical system either by proposing multiple homophonous classes with simpler
pairings or less consistent patterns of pairing with fewer classes. The
``ideal''  analysis is probably more a matter for morphological theory than
historical reconstruction. From the latter perspective, of greater
interest here, the most important point to bear in mind is that the
reconstructed noun class system for \ili{Proto-Benue-Congo} almost certainly had
non-canonical pairing structure for at least some of its classes.

It may also be the case that some of the apparent variability in class pairings
could be due to the presence of ``imperialistic'' classes (see
\namecite[167]{Gerhardt1994}) within \ili{East Benue-Congo} languages, variants
of which were perhaps even found in \ili{Proto-Benue-Congo} itself.{\footnote{In the
formulation of \namecite[167]{Gerhardt1994}, an imperialistic class would not
only be a generally ``open'' class but would also be the typical class for the
incorporation of loanwords and be morphophonologically ``less marked''. While
these patterns may be generally correlated for apparently imperialistic classes,
\namecite[137]{LovegrenMungbam} notes the existence of a plural class which
appears to draw in plural nouns from other classes despite being
morphophonologically ``marked'' by virtue of employing circumfixal class
encoding on the noun.}} These are classes which, for whatever reason, tend to
historically ``absorb'' nouns from other classes. Based on
\namecite{deWolf1971}, a possible candidate for such a noun class in
\ili{Proto-Benue-Congo} may be the *\emph{í} class (here labelled Class~10), due to
its ability to serve as a plural for various singular classes as indicated in
\tref{PBCDeWolf}. To come to a better understanding of these patterns, detailed
studies of noun class distribution across dialects and low-level language
clusters would be useful. These would give us some measure of the rate and
degree to which noun class pairings can shift within languages of the family.
\namecite[306--308]{Watters1981} provides a relevant example in his description
of a clinal shift in the distribution of nouns within a Class~5/8 pairing versus
a Class~5/6 pairing, where the former pairing loses ground to the latter as one
moves west and south within the area associated with the Bantoid language
\ili{Ejagham}.

Other languages suggest additional complications that would be difficult to
reliably reconstruct to \ili{Proto-Benue-Congo} itself but whose presence within it
cannot be ruled out and which certainly raise problems when using the
comparative method to reconstruct the proto-language. These problems center
around the fact that the formal structure of the \ili{East Benue-Congo} noun class
prefixes, consisting of just CV- or V- shapes and typically making use of only a
limited range of a language's available vowel contrasts, makes them relatively
prone to different types of sub-morphemic reanalysis and analogical
contamination, where the form of one class is influenced by that of another.

Consider, for instance, patterns of prefix reduction found in the \ili{Abar} variety
of the Bantoid language \ili{Mungbam} as seen in \tref{AbarRed}
\cite[136]{LovegrenMungbam}. An optional process applies to noun class prefixes
in this variety wherein they lose their initial consonant. In cases where the
vowel of the CV form of the prefix is \emph{ə}, the reduced prefix shows the
vowel \emph{a}. From the standpoint of historical sound change, this pattern of
consonant loss is not obviously remarkable, but, when looked at in light of the
overall noun class system of the variety, it is striking that the reduced
prefixes are formally identical to non-reduced prefixes associated with other
classes. For instance, four non-reduced noun classes posited for this variety
show a prefix with a segmental form of \emph{i} (specifically, Classes 4, 5, and
10) and two show a prefix with a segmental form of \emph{u} (specifically
Classes 1 and 3), with additional tonal complications in some cases
\cite[111]{LovegrenMungbam}. As can be seen in \tref{AbarRed}, three of the
reduced prefixes have a segmental shape of \emph{i} as well and one shows an
\emph{u}, thus adding additional surface homophony to the system.


\begin{table}  
	\renewcommand{\gl}[1]{`#1'} 
	{ 
	\begin{tabular}{l>{\em}l>{\em}ll}
	\lsptoprule
	 {\scshape{class}}	&	{\scshape{full form}}	 		&	{\scshape{reduced form}}			&	{\scshape{gloss}}		\\
	 \midrule
	 6				&	{m\'ə-ŋ⁠{ka̋n}}	&	{á-ŋ⁠ka̋n}			&	{\gl{hand}}		\\
	 6a				&	{m\=ə-mba̋lɔ}	&	{ā-mba̋lɔ}			&	{\gl{oil}}		\\
	 13				&	{kí-la̋m}			&	{í-la̋m}				&	{\gl{tongue}}	\\
	 8				&	{bí-ɲ⁠ű}		&	{í-ɲ⁠ű}				&	{\gl{thing}}	\\	%	from prayer\\
	 12				&	{k\`ə-jì}			&	{à-jì}					&	{\gl{god}}		\\
	 14				&	{bú-tse̋}			&	{ú-tse̋}				&	{\gl{witchcraft}}\\
	 19				&	{{\sh}í-bûs}			&	{í-bûs}				&	{\gl{cat}}		\\
	\lspbottomrule
	\end{tabular}}
	\caption{Prefix reduction in the Abar variety of Mungbam \label{AbarRed}}
\end{table}

Patterns like those in \tref{AbarRed} would clearly allow for a reanalysis of
the structure of CV- prefixes as being morphologically complex, consisting of
something along the lines of C-V-, and thus opening the door to various
morphological developments and complications that would otherwise be unexpected.
For instance, in the Munken variety of the same language, one can find apparent
instances of ``mixed'' agreement, such as those presented in \pref{MunkenDis}. The
word for `day', which most frequently is seen in the Class~14 form
\emph{būtù}, here, shows a form that would normally be associated with
Class~3. Moreover, this apparent Class~3 marking of the form is found not only
on the noun itself but also on the following demonstrative modifier
\emph{w\'ən}. However, the following word \emph{bū}, the object of a
postposition, shows the expected Class~14 form, resulting in an inconsistent
class coding pattern. The most straightforward interpretation of this pattern is
to see it as resulting from a kind of ``confusion'' of classes triggered by
their formal similarities and facilitated by processes of sound change, such as
initial consonant loss, that would result in surface homophony of the sort
just dicussed above for the \ili{Abar} variety of this language.

\ea
{\label{MunkenDis}
	\gll  {À} {humiliation}  {ūtù}  {w\'ən}  {bū}   {ŋ⁠\=ən}. \\
		 {\scshape{ds}} humiliation 3.day 3.{\scshape{dem}} 14.{\scshape{obj}} {\scshape{loc}} \\
		\glt {``There is humility on this day.''}}
\z

While this sort of class confusion and contamination was not likely to have been
a feature of \ili{Proto-Benue-Congo} itself, its noun class system clearly provided
the seeds for it. This means, when attempting to reconstruct the system from
attested data, one must consider the possibility that the daughter languages may
have been impacted not only by comparatively regular processes, such as sound
change or typical kinds of analogical extension, but also by more complex forms
of analogical change, such as those triggered by sub-morphemic analysis.

\newpage 
%%%%%%%%%%%%%%%%%%%%%%%%%%%%%%%%%%%%%%%%%%%%%%%%%%%%%%%%%%%%%%%%%%%%%%%%%%%%%%%%
\section{Towards a reconstruction of the noun class \emph{system} \label{System}}\label{sec:good:7}
%%%%%%%%%%%%%%%%%%%%%%%%%%%%%%%%%%%%%%%%%%%%%%%%%%%%%%%%%%%%%%%%%%%%%%%%%%%%%%%%

An important theme of this chapter has been that we should consider the problem
of reconstruction of the \ili{Proto-Benue-Congo} noun classes not simply as an
exercise in arriving at a set of forms which can be associated with various
class markers but, rather, in terms of the reconstruction of an entire noun
class system, paying attention, in particular, to the morphosyntactic properties
of the system, such as whether class marking on nouns may have ever been optional
in \ili{Proto-Benue-Congo} (see \sref{PrefSuf}) or how many distinct series of
concords may have been present (see \sref{FormClasses}), among other questions.
The reason for doing this is, on the one hand, the fact that even in the absence
of a resolution on the shapes of specific forms, progress might still be made
with respect to the reconstruction of these more abstract properties of the
proto-system. On the other hand,  a better understanding of these properties is
ultimately likely to yield significant insight into why attested \ili{East Benue-Congo}
noun class systems are the way they are, even at the formal level.

Moreover, if there is a general consideration that emerges from this overview,
it is that we should probably not assume the \ili{Proto-Benue-Congo} noun class
behaved as regularly as tabular presentations such as the one in
\tref{PBCDeWolf} might be taken to imply. We can expect there to have been
opacity in the principles of class assignment, variability in singular/plural
pairings, differences in concord realization across various morphosyntactic
constructions, and so on. Whether some of these "irregularities" should be modeled as variability in the usage
of particular speakers or representative of dialect diversity among whatever
community we can identify with \ili{Proto-Benue-Congo} may not prove completely
reconstructible, though reconstructing significant dialect diversity would be
completely reasonable given that, within the \ili{East Benue-Congo} area, salient
dialect diversity within speaker communities seems to be the norm. Furthermore,
while \namecite{deWolf1971} does not appear to make an explicit statement about
this, it is worth bearing in mind that an examination of the specific historical
scenarios relating his \ili{Proto-Benue-Congo} reconstructions to the noun class
systems of his sample languages shows that they are not reducible to simple
statements of sound change or clear-cut analogical changes. Rather, one has the
impression of systems often being generally maintained while combinations of
regular, semi-regular, and apparently irregular changes impact them.

I would like to close by briefly considering how we might move forward in our
efforts to understand the nature of the \ili{Proto-Benue-Congo} noun class system. As
mentioned in \sectref{sec:good:1}, if the goal is to improve on the efforts begun by
\namecite{deWolf1971}, then the most natural step would involve reconstructing
the noun class systems of low-level subgroups and working upwards in systematic
fashion.{\footnote{Bruce Connell (personal communication) suggests that the
available data on \ili{Upper Cross}, for example, should now make it quite feasible to
reconstruct its noun class system. (See \namecite[190--195]{Dimmendaal1978} for
initial work in this direction.)}} Our dataset has improved to a point where
quick progress could be made for many such groups, even if reconstructing
higher-level positions in the tree might still be somewhat elusive. If the goal
is more generally historical in nature, namely using language as a means to
understand \ili{Niger-Congo} prehistory, then this approach is probably too limited,
and increased knowledge of the structural and typological characteristics of the
system is likely to be more worthwhile, especially since these are likely better
windows into patterns of language contact and areal influence than purely formal
reconstructions. This survey has emphasized the latter approach over the former.
On the one hand, this should be viewed as reflecting 
changing priorities in the field since \namecite{deWolf1971}, especially given
the explosion of work on language contact phenomena since the publication of
\namecite{ThomasonKaufman1988}. On the other hand, it also follows a
general expository goal here of laying out a ``bigger picture'' view of possible
directions for future work on \ili{East Benue-Congo} noun class systems, rather
than presupposing that one way forward is to be inherently preferred over
another.
\section*{Acknowledgements}

I would like to thank Larry Hyman, John Watters, and an anonymous reviewer for feedback on a draft of this
paper. Portions of the research underlying this proposal were supported by U.S. National Science Foundation award number BCS-1360763.
 
{\sloppy
\printbibliography[heading=subbibliography,notkeyword=this]
}
\end{document}

