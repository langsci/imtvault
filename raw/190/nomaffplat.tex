\documentclass[output=paper]{langsci/langscibook} 
\ChapterDOI{10.5281/zenodo.1314325}
\author{Roger M. Blench\affiliation{McDonald Institute for Archaeological Research, University of Cambridge}} 
\title{Nominal affixes and number marking in the Plateau languages of Central Nigeria }
\shorttitlerunninghead{Nominal affixes and number marking in the Plateau languages}
 
\abstract{The Plateau branch of East Benue-Congo consists of between sixty and eighty languages spoken in central Nigeria, spreading from Lake Shiroro to the banks of the Benue River. Proto-Plateau is usually considered to have a system of alternating nominal affixes marking number combined with alliterative concord. The paper presents an overall internal classification and then reviews the evidence for affix systems by subgroup, taking a specific language as an exemplar, with a view to linking these to broader hypotheses about Niger-Congo nominal classes. It appears that Plateau has undergone extensive affix renewal, and thus only fragments of any more coherent system are still present. Plateau languages originally had a rich noun class system with CV- and V- prefixes and alliterative concord, but a wave of renewal and analogical re-alignment led to many of the CV- prefixes disappearing or becoming unproductive and replaced by a much smaller set of V- prefixes.
}
\maketitle

\begin{document}  
 
\label{sec:4}
\section{Introduction: Plateau languages} %1. /

The \ili{Plateau} branch of East Benue-Congo consists of between sixty and eighty languages spoken in central Nigeria, spreading from Lake Shiroro to the banks of the Benue River (\figref{fig:nomaffplat:1}). Although most Plateau populations are small (2-10,000 speakers), there are probably more than a million speakers of \ili{Plateau} languages, with the bulk of the numbers made up from large groups such as \ili{Berom} and \ili{Eggon}. Some \ili{Plateau} languages, such as \ili{Sambe} and \ili{Yangkam}, are moribund and others are severely threatened, such as \ili{Ayu}. 

Plateau languages represent one of the four major branches of \ili{East Benue-Congo} outside \ili{Bantoid}, together with \ili{Kainji}, \ili{Jukunoid} and \ili{Cross River}. Internally, they are divided into a large number of subgroups, whose inter-relations are not well understood. \ili{Plateau} languages remain extremely poorly studied, with no complete grammar or dictionary for any language except \ili{Berom} (which is in \ili{French} and thus inaccessible to Nigerians). Basic overviews of their ethnography and ethnic distribution can be found in \citet{Temple1922}, \citet{Meek1925,Meek1931}, \citet{Gunn1953,Gunn1956} and \citet{CrozierBlench1992}.

The most striking feature of \ili{Plateau} is its morphological and typological diversity. It is usually assumed that \ili{Proto-Plateau} would have had a system of paired nominal prefixes with semantic associations and alliterative concord, similar to but probably less elaborate than \ili{Bantu}. However, these systems have collapsed and been rebuilt or in some cases disappeared completely. Compensatory morphology has evolved, including highly complex consonantal inventories and rich tone-systems. Synchronically, \ili{Plateau} languages display systems of consonant mutation, contrastive length, as well as palatalisation, labialisation, lateralisation of initial consonants and combinations of all these.
  
\begin{figure}  
\includegraphics[width=\textwidth]{figures/nomaffplat-map.pdf}
\caption{The Plateau languages by subgroup \hfill {\tiny © OpenStreetMaps contributors}}
\label{fig:nomaffplat:1}
\end{figure} 


\clearpage 
The origin of these diverse outcomes may lie in the characteristic marriage systems within the \ili{Plateau} region. These systems often involved linguistic exogamy, which, combined with marital instability, resulted in the pervasive multilingualism which almost certainly played a role in the evolution of complex ‘mixed’ systems of number marking. Plateau marriage systems are discussed in considerable detail in \citet{Muller1982}. Many different systems co-existed, and there is evidence that when absolute numbers of speakers were much lower (often as little as 500 per language in the pre-colonial era) this principle of outmarriage was strongly enforced.

Plateau languages are assumed to form part of the broader unit represented by \ili{East Benue-Congo} (\citealt{Williamson1971,Williamson1989,WilliamsonBlench2000}). The earliest source for \ili{Plateau}, \citet{Koelle1854} gives wordlists of \ili{Ham} (\ili{Hyam}), \ili{Koro of Lafia} (\ili{Migili}) and \ili{Yasgua} (\ili{Yeskwa}). \citet{Westermann1927} assigned the few languages for which he had data to a ``\ili{Benue-Cross}'' family, corresponding to present-day \ili{East Benue-Congo}, although later in \citet{WestermannBryan1952} these were classified as ``isolated units''. However, the modern subclassification of \ili{Plateau} derives principally from the work of Joseph \citet{Greenberg1963} who proposed dividing Westermann’s ``\ili{Benue-Cross}'' languages into seven co-ordinate groups (including modern-day \ili{Kainji} and \ili{Jukunoid}). With numerous emendations and additions these have been reprised in almost all subsequent works (notably \citealt{WilliamsonShimizu1968,Williamson1971,Williamson1972,Williamson1989,Maddieson1982,Gerhardt1989,CrozierBlench1992,Blench1998,Blench2000wocal}). \citet{Blenchn.d.a} reflects the most recent understanding of \ili{Plateau} subgrouping.

Comparative studies of number marking in the \ili{Plateau} languages are scarce; \citet{Bouquiaux1967} represents an initial attempt to discern commonalties across a small number of languages. 
Some \ili{Plateau} languages retain complex systems of nominal affixes and alliterative concord, notably \ili{Kulu} and other members of the northwest cluster and \ili{Tarok}. However, many languages, such as \ili{Cara} and the \ili{Ninzic} cluster, include some affix alternation as part of a repertoire of number marking strategies, while subgroups such as \ili{Ndunic} and \ili{Ake}, have completely lost these systems. Other languages, such as \ili{Izere}, have systems which look elaborate at first sight, but when segmental and tonal allomorphs are taken into account, the underlying number of pairings is considerably reduced. The existence of these systems certainly suggests that alternating affixes and concord were a feature of \ili{Proto-Plateau}, but actual segmental correspondences between affixes are few, pointing to a continuing process of renewal. \ili{Plateau} also has frequent nasal prefixes, as well as numerous examples of unproductive nasals preceding C\textsubscript{1} of the stem \citep{Miehe1991}. Some of these are reflected more widely in other branches of \ili{Benue-Congo} or even further afield in \ili{Kwa}. However, the correspondences between noun classes and semantic subsets (humans, trees, animals, paired things) characteristic of \ili{Bantu} are more tenuous.

The analytic challenge of \ili{Plateau} is to account for synchronic number marking systems through the lens of the erosion of affix alternation. In the light of this, the confident assertions of authors such as  \citet{deWolf1971} in reconstructing the prefixes of \ili{Proto-Benue-Congo} seem very optimistic. Such reconstructed forms reflect a prior knowledge of \ili{Bantu} and a large pool of miscellaneous data from which exemplars can be selectively chosen. This paper is an overview of nominal affixing in the \ili{Plateau} languages, based principally on my own fieldwork materials.\footnote{Lexical and grammatical materials are available on the author’s website \citet{Blenchn.d.b}}. It describes the systems in individual subgroups and then asks what evidence these provide for the situation in \ili{Proto-Plateau}.

\largerpage
None of the authors who have classified \ili{Plateau} languages have presented evidence for their classifications. This is not a criticism; faced with large arrays of data it is easier to set out what appears to be the case impressionistically than to write a monograph demonstrating it. The series of publications on \ili{Plateau} subgroups, especially Plateau II and IV, by
\citeauthor{Gerhardt1969a} (%
\citeyear*{Gerhardt1969a,
Gerhardt1969b,
Gerhardt1971,
Gerhardt19723a,
Gerhardt19723b,
Gerhardt19734,
Gerhardt1974,
Gerhardt1983a,
Gerhardt1983b,
Gerhardt1988a,
Gerhardt1988b,
Gerhardt1989,
Gerhardt1994}%
)
assume the boundaries of these groups. A particular issue in the internal classification of \ili{Plateau} and \ili{Jukunoid} is the notion of a ‘\ili{Benue}’ grouping. \citet[415]{Shimizu1975} proposed that some branches of \ili{Plateau} should be classified with \ili{Jukunoid}. In particular, he argued that \ili{Eggon} (and by implication the other Plateau V languages, including \ili{Nungu} and \ili{Yeskwa}) and \ili{Tarokoid} (at that time consisting only of \ili{Yergam} (=\ili{Tarok}) and \ili{Basherawa} (=\ili{Yaŋkam})) formed a group together with \ili{Jukunoid}. This emerged from his lexico-statistical tables and was further supported by five isoglosses, the words for ‘drink’, ‘tail’, ‘meat’, ‘fire’, and ‘four’. This expanded group he christened ``\ili{Benue}''. \citet{Gerhardt1983b} questioned Shimizu’s hypothesis noting both that his own lexico-statistical work (\citealt{GerhardtJockers1981}) did not support this, and casting doubt on the five isoglosses proposed by Shimizu. The ‘\ili{Benue}’ group continued in a sort of half-life, appearing in \citet{Gerhardt1989} as a subgrouping of \ili{Jukunoid} and \ili{Tarokoid} against the rest of \ili{Plateau}. \citet{Blench2005} has presented evidence that there is a genuine boundary between \ili{Plateau} and \ili{Jukunoid}, drawing on lexical and morphological evidence.

\largerpage[-1]
This uncertainty is a reflection of a more general problem, the evidence for a bounded group ``\ili{Plateau}'' in opposition to \ili{Kainji}, \ili{Jukunoid}, \ili{Dakoid} or \ili{Mambiloid}, other members of the \ili{Benue-Congo} complex. The relationships between \ili{Plateau} languages, their coherence as a grouping and their links with \ili{Jukunoid} and \ili{Kainji} remain undetermined. \citet{Rowlands1962} was the first to suggest that there was a dichotomy between certain languages of the Jos area, which he linked to West \ili{Kainji}, and the remainder, but his short wordlists were far from constituting linguistic proof. Comparative analysis has produced some tentative evidence for isoglosses defining \ili{Plateau}, but so far no phonological or morphological innovations that would define the group have been proposed. Some of this diversity is undoubtedly due to long-term interactions with the mosaic of \ili{Chadic} languages also occurring on the Jos Plateau \citep{Blench2003}.

With these caveats, \figref{fig:nomaffplat:2} presents a new subclassification of \ili{Plateau}, within the context of \ili{East Benue-Congo}. Evidence for this subgrouping is presented in Blench (in press). The majority of evidence is lexical, but some subgroups, such as Northwest, clearly also share considerable common elements in the noun-class system. This ``tree'' is clearly not final, as there are too many co-ordinate branches and too little internal structure. But until further analysis is undertaken, provisional versions of \ili{Plateau} which do not promote too many unwarranted assumptions are the best that can be produced.
 

\begin{figure}   
% \includegraphics[width=\textwidth]{figures/nomaffplat-img2.png} 
% \footnotesize
 \begin{forest}  for tree={grow'=east,delay={where content={}{shape=coordinate}{}}},   forked edges 
 [\ili{Benue-Congo} 
    [\ili{Kainji}, tier=kainji]
    [\ili{Plateau}, tier=plateau
	[	  
	  [\ili{Tarokoid}, tier=kainji]
	  [\ili{Southern} 
	    [\ili{Eggonic}]   
	    [\ili{Jilic}]
	  ]    
	  [\ili{Bo-Rukul}~~~~~~~, name=br, anchor=west]
	  [\ili{East}]
	  [\ili{Horom-Fyem}, name=hf, anchor=west]
	  [\ili{Alumic}]
	  [\ili{Ninzic}]
	  [\ili{Ndunic}]
	  [\ili{West-Central}]
	  [\ili{Beromic}]
	  [\ili{North-West}]
	]
      [\ili{Eloyi}, tier=kainji]
    ]
    [\ili{Jukunoid}, tier=kainji]
    [\parbox{6cm}{\mbox{\ili{Cross River}, \ili{Mambiloid}, \ili{Dakoid}, \ili{Bantoid}}},edge=dashed,tier=plateau]
]  
 \draw[<->,dotted, thick] (hf) to [out=north,in=south] (br); 
\end{forest}    
\caption{Proposed classification of Benue-Congo languages}
\label{fig:nomaffplat:2} 
\end{figure}
 
This paper is organised using these \ili{Plateau} subgroups and listed approximately left to right. The summary Tables \ref{tab:nomaffplat:1}--\ref{tab:nomaffplat:1z} also gives a list of all known \ili{Plateau} languages. The great majority of material presented here is either from my own fieldwork since 1980, from manuscript sources, with a relatively small amount from published work, cited in the reference list. Where no source is cited, it can be taken this is my own data. All original wordlists can be found on my website. Some of the earliest data is not tone-marked, and the segmental transcription may be less reliable. Most \ili{Plateau} languages have a three-level tone system and by convention the mid-tone is not marked. Therefore, if the data is tone-marked, a vowel without a tone is deemed to be mid. Where a standardised orthography exists, for example in the case of \ili{Mada}, only the high tone is unmarked. I have noted deviations from the standard tone marking in relation to relevant examples.

\largerpage[-1]
The sample wordlist is usually five hundred items and of these some 350 are nouns with singular and plural recorded. Allowing for entries that cannot be elicited, the nouns available for analysis amount to around 300. Where the data has not been collected by the author, the sample may be smaller, whereas in the case of dictionaries prepared by the author, for example \ili{Berom}, \ili{Izere}, \ili{Mada} and \ili{Tarok} the sample is usually well over a thousand. For most languages only singular/plural pairs are available, but where a grammar sketch has been prepared, we also have an overview of the concord system. The reader should refer to the original wordlist for examples of the noun-class pairings, where these are summarised in the tables below.

\ili{Plateau} languages exhibit extensive allomorphy in their nominal affix systems. Allomorphs are here defined as ‘one of two or more complementary morphs which manifest a morpheme in its different phonological or morphological environments’ \citet{LoosEtAl2003}. In \ili{Plateau}, tonal allomorphs are common because the tone of the prefix may be driven by the tone of the stem, thus the V of a CV prefix may have one of three tone levels, as for example in \ili{Kulu}. Sometimes claims of allomorphy in less-well-studied languages are only educated guesses, based on parallels with better known languages. The term alternation is used to apply to the change in prefix on a stem marking number, i.e. singular and either one or two plurals. Throughout the text, the tables present a summary of the prefix alternations occurring in the data, mostly wordlists. This is not ideal, as we have no evidence for the patterns of concord in many languages, but it provides a preliminary guide to the synchronic system.

\section{Plateau languages by subgroup} %2. /

Tables \ref{tab:nomaffplat:1}--\ref{tab:nomaffplat:1z} show a comprehensive list of \ili{Plateau} languages, by subgroup, and a summary of the system of number-marking, as far as it is known. Where there is a published reference on a specific language, it is given, although I do not always agree with the analysis and the text presents my own hypothesis. No entry in the reference column means the summary is based on my own fieldwork. The names of the branches are proposed by the author, since the classification is at variance with previous proposals in many areas. Further justification can be found in \citet{Blench2000wocal}.

\begin{table}
\small
\caption{Synthesis of nominal affixing: Northwest}
\label{tab:nomaffplat:1}
\begin{tabularx}{\textwidth}{lQl}
\lsptoprule 
  Language 	& Comments 	&   Reference\\
\midrule
\ilit{Eda} 	& Reduced alternating prefixes, concord 	& \\\
\tablevspace
\ilit{Edra} 	& Reduced alternating prefixes, concord 	&  \\ 
\tablevspace
\ilit{Acro} 	& Reduced alternating prefixes 	& \\
\tablevspace
\ilit{Obiro} 	& Reduced alternating prefixes 	& \\
\tablevspace
\ilit{Kulu} 	& Extensive alternating prefixes, elaborate allomorphy, concord 	& \citet{Seitz1993}\\
\tablevspace
Ẹjẹgha\il{Ejegha@Ẹjẹgha} [\ilit{Idon}] 	& Extensive alternating prefixes, elaborate allomorphy, concord 	& \\
\tablevspace
\ilit{Doka} 	& Data very poor 	& \\
\tablevspace
Ẹhwa\il{Ehwa@Ẹhwa} [\ilit{Iku-Gora-Ankwe}] 	& Reduced alternating prefixes 	& \\
\lspbottomrule
\end{tabularx}
\end{table}

\begin{table}
\caption{Synthesis of nominal affixing: Beromic}
\small
\begin{tabularx}{\textwidth}{lQl}
\lsptoprule
Language 	& Comments 	&   Reference\\
\midrule
\ili{Berom} 	& Reduced alternating prefixes, consonant mutation, concord 	& \citet{Wolff1963}\newline\citet{Bouquiaux1970}\\
\tablevspace
\ili{Cara}  	& Restricted alternating prefixes, stem-tone change, consonant mutation, concord \\
\tablevspace
\ili{Iten}  	& Reduced alternating prefixes, consonant mutation, concord 	& \citet{Bouquiaux1964}\\
\tablevspace
\ili{Shall-Zwall} 	& Data very poor but affix system apparently heavily eroded 	& \\
\lspbottomrule
\end{tabularx}
\end{table}

\begin{table} 
\small
\caption{Synthesis of nominal affixing: West-Central} 
\begin{tabularx}{\textwidth}{p{3cm}Qp{2.8cm}@{}}
\lsptoprule
 \textbf{\ilit{Izeric}}\\
\midrule
  \ilit{Izere} of Fobur \il{Izere} of Fobur 	& Restricted alternating prefixes, and extensive stem-tone changes. 	& \citet{Blench2000b}\\
\raggedright	  \ilit{Icèn}, \ilit{Ganàng}, \ilit{Fəràn} 	& Similar to others in group 	&   \\
\tablevspace
 \textbf{\ilit{Rigwe}}\\
\midrule
  \ilit{Rigwe} \il{Rigwe} 	& Innovative system, with  residual concord 	&{\raggedright  {\citet{Anon2006}} \mbox{\citet{BlenchGya2012}}}\\
\tablevspace
\multicolumn{3}{l}{\textbf{\ilit{Southern Zaria}}}\\
\midrule
  \ilit{Jju} \il{Jju} 	& Innovative prefix system,\newline suffixed elements 	&  {\citet{McKinney1979}}\newline\citet{Hyuwa1986}\\
\tablevspace
\textbf{ \ilit{Tyapic}}\\
\midrule
  \ilit{Tyap} \il{Tyap} 	& Innovative prefix system, suffixed elements 	& {\citet{Follingstad1991}} \\
% \tablevspace
	  \ilit{Gworok} 	& Innovative prefix system, suffixed elements 	& \citet{AdwiraahHagen1983}\\
% \tablevspace
\raggedright	  \ilit{Atakar}, \ilit{Kacicere}, \ilit{Sholyo}, \ilit{Kafancan} 	& Similar to others in group 	& \\
\tablevspace
\textbf{\ilit{Koro}}\\
\midrule
  \ilit{Ashe} 	& Very reduced affix alternation 	& \\
% \tablevspace
\raggedright	  \ilit{Tinɔr} (\ilit{Waci-Myamya})	& Very reduced affix alternation 	&  \\
% \tablevspace
	  \ilit{Idũ}, \ilit{Gwara} 	& Very reduced affix alternation 	& \\
% \tablevspace
	  \ilit{Nyankpa-Barde} 	& Very reduced affix alternation 	& \\
% 	  \lspbottomrule
% 	  \end{tabularx}
%   \end{table}
%  
% \begin{table} 
% \small
% \caption{Synthesis of nominal affixing: West-Central (continued)}
% \begin{tabularx}{\textwidth}{lQl}
% \lsptoprule
 \textbf{\ilit{Hyamic}}\\
\midrule
 \ilit{Shamang} 	& As \ilit{Hyam cluster} 	& \\
% \tablevspace
\ilit{Cori} 	& As \ilit{Hyam cluster} 	& \citet{Dihoff1976}\\
% \tablevspace
       Hyam\il{Hyam cluster} 	&Nominal prefixes almost lost and replaced by consonant mutation and stem-tone change 	& \citet{Jockers1982} \\
% \tablevspace
       \ilit{Zhire} 	& As \ilit{Hyam cluster} 	& \\
% \tablevspace
       \ilit{Shang}\il{Shang} 	& Small number of alternating prefixes but probably borrowed from \ilit{Koro} languages 	& \\
\tablevspace
 \textbf{\ilit{Gyongic}}\\
\midrule
  Gyong\il{Gyong} (=\ilit{Kagoma}) 	& Very restricted alternating prefixes, palatalisation, concord 	& \citet{Hagen1988}\\
% \tablevspace
      \ilit{Kamanton} 	& Similar to \ilit{Gyong} 	& \\
\lspbottomrule
\end{tabularx}
\end{table}
\begin{table}
\caption{Synthesis of nominal affixing: Ninzic}
\small
\begin{tabularx}{\textwidth}{p{2.5cm}Ql}
\lsptoprule
 Language 	& Comments 	&   Reference\\
\midrule 
Ninzo\il{Ninzo} 	& Very restricted alternating prefixes 	& \\
\tablevspace
Ce\il{Ce} 	& Elaborate alternating prefixes and concord 	& \citet{Hoffman1976}\\
\tablevspace
\ilit{Bu-Niŋkada} 	& No morphological plurals 	&   \\
\tablevspace
Mada\il{Mada} 	& Very restricted alternating prefixes, some concord, multiple other number-marking strategies 	& \citet{Price1989}\\
\tablevspace
\ilit{Numana-Nunku-Gwantu-Numbu} 	& Information inadequate 	& \\
\tablevspace
\ilit{Ningye-Ninka} 	& Alternating prefixes lost, tone plurals 	& \\
\tablevspace
\ilit{Anib} 	& Very restricted alternating prefixes 	& \\
\tablevspace
\ilit{Ninkyob} 	& Very restricted alternating prefixes 	& \\
\tablevspace
\ilit{Nindem} 	& Very restricted alternating prefixes 	& \\
\tablevspace
Nungu\il{Nungu} 	& Information inadequate 	& \\
\tablevspace
\ilit{Ayu}\il{Ayu}\footnote{Ayu is of uncertain Ninzic affiliation} 	& a. prefix alternation or addition
	    \newline
	    b. consonant mutation
	    \newline
	    c. tone-change
	    \newline
	    d. nasal insertion 	& \\
\lspbottomrule
\end{tabularx}
\end{table}

\begin{table}
\caption{Synthesis of nominal affixing: Ndunic}
\small
\begin{tabularx}{\textwidth}{lQl}
\lsptoprule
 Language 	& Comments 	&   Reference\\
\midrule
Ndun\il{Ndun-Nyeng-Shakara} [=\ilit{Tari}] 	& Extremely reduced system, retaining \ilit{Niger-Congo} a/ba person class 	& \citet{RueckEtAl2008}\\
\lspbottomrule
\end{tabularx}
\end{table}

\begin{table}
\caption{Synthesis of nominal affixing: Alumic}
\small
\begin{tabularx}{\textwidth}{lQl}
\lsptoprule
 Language 	& Comments 	&   Reference\\
\midrule
\ilit{Toro}, \ilit{Alumu-Təsu} 	& No functioning noun-prefixes and a single plural suffix.  	& \\
\tablevspace
\ilit{Hasha} 	& Innovative system, reduplicating first syllable of stem 	& \\
\tablevspace
\ilit{Sambe} (†) 	& No functioning noun-prefixes and a single plural suffix. 	& \\
\lspbottomrule
\end{tabularx} 
\end{table}


\begin{table}
\caption{Synthesis of nominal affixing: Southern}
\small
\begin{tabularx}{\textwidth}{lQp{3.4cm}}
\lsptoprule
 Language 	& Comments 	&   Reference\\
\midrule
\textbf{\ilit{Eggonic}}\\
\midrule
{Eggon}\il{Eggon} 	& Very reduced nominal affix pairings and concord, evolution of single pluralising prefix. 	& \citet{Maddieson1982,Maddiesonnda,Sibomana1985} \\
\tablevspace
{Ake}\il{Ake} 	& No functioning noun-prefixes 	& \\
\midrule
\textbf{\ilit{Jilic}}\\
\midrule
{\ilit{Jili}}\il{Jili} 	& Elaborate alternating prefixes and concord 	& \citet{Stofberg1978}\\
\tablevspace
{\ilit{Jijili}} 	& Elaborate alternating prefixes and concord 	& \\
\lspbottomrule
\end{tabularx} 
\end{table}

\begin{table}
\caption{Synthesis of nominal affixing: Southeastern}
\small
\begin{tabularx}{\textwidth}{lQl}
\lsptoprule
 Language 	& Comments 	&   Reference\\
\midrule
\ilit{Fyem} 	& Very reduced nominal affix pairings, suffixing, stem initial syllable reduplication 	& \citet{Nettle1998a}\\
\tablevspace
\ilit{Horom} 	& Very reduced nominal affix pairings, circumfixing 	& \citet{Nettle1998b}\\
\tablevspace
\ilit{Bo-Rukul} 	& Alternating prefixes with extensive allomorphy and concord 	& \citet{Nettle1998b} \\
\lspbottomrule
\end{tabularx} 
\end{table}

\begin{table}
\caption{Synthesis of nominal affixing: Tarokoid}
\small
\begin{tabularx}{\textwidth}{lQp{2.8cm}}
\lsptoprule
 Language 	& Comments 	&   Reference\\
\midrule
\ilit{Tarok} 	& Alternating prefixes and concord 	& {\raggedright \citet{Sibomana1981},\newline \citet{Longtau2008}}\\
\tablevspace
\ilit{Pe} [=\ilit{Pai}] 	& Very reduced nominal affix pairings and concord 	& \\
\tablevspace
\ilit{Kwang-Ya-Bijim-Legeri} 	& Very reduced nominal affix pairings and concord 	& \\
\tablevspace
\ilit{Yaŋkam} [=\ilit{Bashar}] 	& Fragmentary nominal affix pairings, may be a problem of informant recall 	&  \\
\tablevspace
\ilit{Sur} [=\ilit{Tapshin}] 	& No functioning noun-prefixes 	& \\
\lspbottomrule
\end{tabularx} 
\end{table}

\begin{table}
\caption{Synthesis of nominal affixing: Eloyi}
\label{tab:nomaffplat:1z}
\small
\begin{tabularx}{\textwidth}{lQp{2.8cm}}
\lsptoprule
 Language 	& Comments 	&   Reference\\
\midrule
\ilit{Eloyi} 	& Elaborate alternating prefixes and concord 	& \citet{Armstrong1964},\newline \citet{Mackay1964}\\
\lspbottomrule
\end{tabularx}
\end{table}

\newpage  
  
\subsection{Northwest Plateau} %2.1 /

\ili{Northwest Plateau} consists of \ili{Eda}/\ili{Edra}, \ili{Acro-Obiro} [=\ili{Kuturmi}], \ili{Kulu}, \ili{Idon}, Do\-ka\il{Doka} and \ili{Iku-Gora-Ankwe}. No new data has been published since this group was set up, although a wordlist of \ili{Kulu} has been circulated (\citealt{Moser1982}, analysed in \citealt{Seitz1993}) and \citet{Shimizu1996} has posted a grammar sketch on the Internet. Recent interest in Ẹda\il{Eda@Ẹda} [=\ili{Kadara}] language has resulted in an unpublished dialect survey, a preliminary alphabet book and the launching of an alphabet chart in 2009. \ili{Kadara} is known to its speakers as ‘\il{Eda@Ẹda}Ẹda’ and there is a closely related lect, Ẹdra\il{Edra@Ẹdra} (which is presumably the source of the common \ili{Hausa} name). Two other lects for which information is recorded, Ẹjẹgha\il{Ejegha@Ẹjẹgha} and \il{Ehwa@Ẹhwa}Ẹhwa,\footnote{Thanks to Zac Yoder for sound files of wordlists of 384 items of these languages. Retranscribed by the author.} correspond to \ili{Idon} and the \ili{Iku-Gora-Ankwe} clusters (as named in the \ili{Benue-Congo} Comparative Wordlist in \citealt{WilliamsonShimizu1968,Williamson1972}). The wordlists are so different from each other and from Ẹda\il{Eda@Ẹda} that they clearly deserve separate language status. \ili{Northwest Plateau} remains a high priority for further research.

\tabref{tab:nomaffplat:2} shows the singular/plural prefix pairings recorded in \ili{Kulu} including tonal variants, based on \citet{Moser1982} and \citet{Seitz1993}. A postulated ‘underlying’ prefix is given together with its allomorphs. The mid-front vowel shows harmony with the stem-vowel. The bracketed nasals in the plural prefixes show their sporadic appearance. They are homorganic with the following consonant and only follow /i/.
 
\begin{table}
\caption{Kulu prefix pairings. Re-analysis by author of \citet{Moser1982} and \citet{Shimizu1996}.}
\label{tab:nomaffplat:2}
\begin{tabular}{>{\itshape}l>{\itshape}l@{\qquad}>{\itshape}l>{\itshape}l}
\lsptoprule
\multicolumn{2}{p{4cm}}{\centering Singular} 	&\multicolumn{2}{p{3cm}}{\centering Plural}\\
\upshape Underlying  	& \upshape   Surface    	& \upshape Underlying  	& \upshape  Surface\\
\midrule 
E- 	&   è, e, é, ɛ̀, ɛ   	& bE-, a-    	& bè. bɛ, a        \\
dì-	&    dì, di         	&  a, be- e- 	&   a, be, è, e    \\
gE-	&    gè,ge,gé,gɛ̀,gɛ 	& bE-        	&  be,bɛ           \\
gì-	&    gì,gi,gí       	&  E-, Ni-   	&  be,i(m),  i(ŋ), nì,  ni, nǐ, ní(n), í(n)\\
gù-	&    gù,gu	&      E-, Ni- 	&   ɛ,è,e,ì,i(n)    \\
ì- 	&   ì     	& Ni-          	&   m̀,mì(n)     \\
ù- 	&   ù,u   	&   bE-, i-    	&  be, i            \\
\lspbottomrule
\end{tabular}
\end{table}

Tonal variation in prefixes is driven by the stem-tone (as in many \ili{Plateau} languages, cf. \citealt{Blench2000b}) and the different surface tones do not in themselves mark distinct pairs marked for number. The numerous forms of a \textit{gV}- prefix presumably point to these all originally having a single underspecified vowel which has gradually diverged.\footnote{One reviewer queries the directionality of this process. However, if instead this were a case of convergence, this would require ten different surface forms to come together, which is hardly an economical explanation.} The presence of an underspecified vowel in the V of a prefix is very common in the \ili{East Kainji} languages with which \ili{Kulu} is in contact and it is possible this is a borrowing.

\ili{Kulu} has frequent doubled /l/ in stem-initial position, assumed to derive from nasal prefixes which have been first fossilised and then assimilated to an initial lateral. For example \REF{ex:nomaffplat:1}:

\ea\label{ex:nomaffplat:1}
\ea  Doubled /l/ in stem-initial position
\ex  \textit{gɛ́-llam}    ‘water’ 
\ex  \textit{gu-llúrú} ‘storm’ 
\ex \textit{gɛ-llán}    ‘chin’
\ex \textit{gɛ̀-llìbì}    ‘hyena’
\z
\z

Semantic associations in \ili{Kulu} are weak, but the majority of nouns for persons show \textit{E-/bE-} prefix alternation. Most domestic animals have a \textit{gV-} singular prefix but no consistent plural marking. Wild animals, on the other hand, almost all have their singular and plural forms distinguished only by tonal differences in the stem. Trees, body parts, abstracts and even mass nouns do not form consistent sets marked by paired affixes. The \textit{ni-} prefix for noun plurals is uncommon and surprisingly, it is strongly correlated with household items as in \tabref{extab:nomaffplat:2}.

\begin{table}
\caption{The \textit{ni-} plural prefix in Kulu}
 \label{extab:nomaffplat:2}

\begin{tabularx}{\textwidth}{QQQ}
\lsptoprule
{{Gloss}} 	&{  Singular} 	& {Plural}\\
\midrule
{‘knife’} 	& \itshape gí-ŋmáŋ 	& \itshape ní-ŋmáŋ\\
{‘bag’}	& {\itshape gi-mpak} 	& \itshape ni-mpak\\
{‘mortar (wood)’}  	&{\itshape gí-ŋklu} 	& \itshape ní-ŋklu\\
{‘pot (generic)’} 	&{\itshape gí-nugu} 	& \itshape nìí-nugu\\
{‘head-board’}  	&{\itshape gí-ŋgwel} 	& \itshape ní-ŋgwel\\
{‘basket (generic)’}  	&{\textit{gi-nʤili\textsuperscript{+}}} 	& \textit{ni-nʤili\textsuperscript{+}}\\
{‘spoon’} 	&{\itshape gi-nʧàk} 	& \itshape ni-nʧàk\\
{‘fish-trap’} 	&{\itshape gí-sak} 	& \itshape nín-sak\\ 
\lspbottomrule
\end{tabularx} 
\end{table}


However, \ili{Kulu} does operate a principle of using prefixes to assign semantics, such as the parts of a tree, by means of prefixes, as for example in \REF{ex:nomaffplat:3}:
 
\ea\label{ex:nomaffplat:3}
\ili{Kulu} prefixes used to assign semantics e.g. parts of tree\\
\textit{gi-n-yoŋ}  ‘locust tree’\\
\textit{gɛ́-n-yoŋ}  ‘locust fruit’ \\
\textit{u-yoŋ}    ‘locust pod powder’\\
\z

 
\subsection{Beromic}

The term ‘\ili{Beromic}’ has been adopted here to cover former Plateau 2 languages. \ili{Beromic} now consists of \ili{Berom}, \ili{Iten}, \ili{Cara} and two closely related lects, \ili{Shall} and \ili{Zwall}, geographically distant in Bauchi State. The principal publications on \ili{Berom} are \citet{Bouquiaux1970,Bouquiaux2001} and \citet{KuhnDusu1985}, and on \ili{Iten}, \citet{Bouquiaux1964}. Recent unpublished materials are dictionaries of \ili{Berom} and \ili{Iten}. \ili{Cara} (\ili{Teriya}) was reported in a mimeo paper by \citet{Shimizu1975b}, who first proposed a link with \ili{Berom}. \ili{Shall} and \ili{Zwall} were previously classified with the \ili{Ninzic} languages (Plateau 4), but are better placed with \ili{Beromic} \citet{Blenchn.d.c}. \ili{Beromic} languages show a broad range of number-marking systems, although none have a full noun-class system and \ili{Shall-Zwall} has lost all nominal affixing, perhaps under the influence of \ili{Chadic}. A summary of \ili{Beromic} number marking is shown in \tabref{tab:nomaffplat:3}.

\begin{table}
\caption{\label{tab:nomaffplat:3} Number-marking in Beromic.\\ {Source: All analyses of Beromic by author based on personal fieldwork.}}
\begin{tabularx}{\textwidth}{lQ}
\lsptoprule
  Language 	&   Summary of number marking\\
\midrule
\ili{Berom} 	& Very restricted prefix alternations, incipient consonant mutation\\
\ili{Cara} 	& Restricted prefix alternations, complex consonant mutation, tone and length contrasts\\
\ili{Iten} 	& Prefix alternations, complex consonant mutation\\
\ili{Shall-Zwall} 	& Nominal prefixes entirely lost\\
\lspbottomrule
\end{tabularx}
\end{table}


\ili{Berom} itself has a complex internal structure. \ili{Central Berom} includes the Du dialect described by \citet{Bouquiaux1970,Bouquiaux2001} as well as both \ili{Vwang} (\ili{Vom}) and \ili{Ryom} (\ili{Riyom}). The speakers of \ili{Vwang} are the most numerous, but the main dia\-lect used for literacy and bible translation is the Eastern dialect, roughly centred on Foron, spoken by only a minority. The other minority dialect is \ili{Rim}, south and east of the main centres. Data on \ili{Berom} presented here is based on long-term fieldwork on the \ili{Foron} dialect and shows marked differences with the \ili{Du} of Bouquiaux.

\ili{Berom} noun pluralisation strategies are extremely varied. The most common are:

\begin{itemize}
 \item[a)] prefix addition or alternation
 \item[b)] tone-raising
 \item[c)]  (de)labialisation
 \item[d)] consonant alternation
 \item[e)]  number marking in verbal nouns replicating corresponding verbal plurals
\end{itemize}

In some cases, two procedures can be applied to mark a plural, suggesting the dynamics of renewal. The great majority of \ili{Berom} singular nouns have no prefix, while on nouns that are marked for plural, the \textit{be}- prefix is predominant. \ili{Berom} shows ‘echo’ concord, where a small subset of concordial adjectives exactly copy the nominal prefix of the noun they qualify. Where the noun has no prefix, the adjective shows no concord. Berom also has just three suppletive plurals in a dictionary which includes more than 2000 nouns. Due to this relatively large database, it is possible to estimate the frequency of nominal prefix alternations in \ili{Eastern Berom} seen in \tabref{tab:nomaffplat:4}.

\begin{table}
\caption{Nominal prefix alternations in Eastern Berom}
\label{tab:nomaffplat:4}
\begin{tabularx}{\textwidth}{llL{2cm}Q}
\lsptoprule
  {Sg.} 	&   {Pl.} 	&   Incidence 	&   Semantics \\
  \midrule
\itshape ø- 	& \itshape be-/pe- 	& common 	& loanwords, miscellaneous \\
\tablevspace
\itshape ø- 	& \itshape ba- 	& occasional 	& body parts, grasses \\
\tablevspace
\itshape ø- 	& \itshape nè- 	& common 	& miscellaneous \\
\tablevspace
\itshape kè- 	& \itshape nè- 	& common 	& diminutives \\
\tablevspace
\textit{ne-/n-/ŋ-/m-} 	& \itshape ø- 	& common & colours, abstracts, mass nouns, diminutives  \\
\tablevspace
\itshape *ra-, re-, rɛ- 	& \itshape ba- 	& common 	& body parts, miscellaneous \\
\tablevspace
\itshape se- 	& \itshape ø-, ba-, ne- 	& rare 	& unpaired class marks abstracts, paired classes miscellaneous \\
\tablevspace
\itshape -w- 	& \itshape ø- 	& common 	& miscellaneous \\
\tablevspace
\itshape wò- 	& \itshape be- 	& occasional 	& ‘person of, from’ \\
\tablevspace
\itshape -y- 	& \itshape ø- 	& common  &  miscellaneous \\
\lspbottomrule
\end{tabularx}
\end{table}

\largerpage  
Tone-marks show the most characteristic tone for this class, with mid-tone unmarked. However, there are numerous unexplained exceptions, which may reflect interaction with the stem-vowel. \textit{ra-} is not attested synchronically as a productive prefix, since all singular nouns in current \ili{Berom} with stem-initial \textit{ra}- have a zero singular prefix and a plural prefix \textit{be-}. However, many words have \textit{ra}- as a first syllable, such as \textit{rato} ‘head’ where the \textit{ra}- is not historically part of the root, because –\textit{to} is widely attested across \ili{Benue-Congo} for ‘head’.


The alternation \textit{wo-/be-}, in \REF{ex:nomaffplat:4}, is the ethnonym for the \ili{Berom} people, and is probably not originally a noun class pair. \textit{wo} is a personal pronoun and \textit{be}- a generic plural marker.


\ea\label{ex:nomaffplat:4}
{  \textit{Wòrom}  ‘Berom person’ \\        \textit{Berom}   ‘Berom people’}\\
\z

The labial and palatal infixes \textit{-w-} and \textit{-y-} almost certainly originally derive from \textit{u-} and \textit{i-} prefixes which have been incorporated into the stem, as in many other \ili{Plateau} languages. \citet{KiesslingMutations2010} has described analogous processes in the languages of the \ili{Grassfields} of Cameroun. Tonal changes accompany number marking suggest that the tone of the lost prefix vowel affected the stem tone of the noun.

The nasal prefixes form a complex set. It is most likely there is a diminutive marker \textit{ne}- which shows up both as a plural prefix and unpaired in non-count nouns, as well as in \textit{ke}-/\textit{ne}- alternations marking small entities in \REF{extab:nomaffplat:5}. The \textit{ke}- is probably cognate with \ili{Bantu} \textit{ka}- which has a similar diminutive function \citep[88]{Maho1999}.


\begin{table}
 \caption{\textit{ke-/ne-} alternation in Berom} 
\label{extab:nomaffplat:5}
\begin{tabularx}{.75\textwidth}{XXl}
\lsptoprule
Gloss 	& {Sg.} 	& {Pl.}\\
\midrule
‘small calabash’ 	& \itshape kèkyɔ́k 	& \itshape nèkyɔ́k\\
‘any small bird’ 	& \itshape kènòn 	& \itshape nènòn\\
‘little town’ 	& \itshape kèrèpomo 	& \itshape nèbàpomo\\
\lspbottomrule
\end{tabularx}
\end{table}

 
\begin{table}[p]
\caption{\textit{ø-/ne-} alternation in Berom} 
\label{extab:nomaffplat:6}
\begin{tabularx}{.75\textwidth}{lXXl}
\lsptoprule
   Gloss 	&     {Sg.} 	&     {Pl.} 	&   Also\\
   \midrule
‘knife’ 	&  \itshape bá 	&  \itshape nebá 	& \\
‘lie’ 	&  \itshape bɔs 	&  \itshape nebɔs 	& \\
‘household head’ 	&  \itshape dá lɔ 	&  \itshape beda nelɔ 	& \\
‘limb, place’ 	&  \itshape dèm 	&  \itshape nedem 	& \\
‘soil being dug’ 	&  \textit{fòŋol}  	&  \textit{nèfòŋol}  	&  \textit{ǹfòŋol}\\
‘spirit’ 	&  \itshape gabik 	&  \itshape nègabik 	&  \itshape begabik\\
‘place’ 	&  \itshape kwɔ́n 	&  \itshape nèkwɔ̄n 	& \\
\lspbottomrule
\end{tabularx}
\end{table}

\begin{table}[p]
\caption{ Unpaired \textit{n-, ne-} prefix in Berom} 
\label{extab:nomaffplat:7}
\begin{tabularx}{.8\textwidth}{Xl}
\lsptoprule
Gloss 	& \ilit{Berom}\\
\midrule
‘brownness’ 	& \textit{nèrós}\\
‘blackness’ 	& \textit{nèsi}\\
‘redness’ 	& \textit{nèsinàng}\\
‘stubbornness’ 	& \textit{nèshágárák}\\
‘intense sweetness’ 	& \itshape nèrɔ́krɔ́k\\
‘dirtiness’ 	& \textit{nèrwǐk}\\
‘friendship, fellowship’ 	& \textit{nèsá}\\
‘blood’ 	& \textit{nèmí}\\
‘milk (of animal or human being)’ 	& \textit{nèvasal}\\
‘local salt (made from acca straw)’ 	& \textit{ǹtow}\\
‘urine’ 	& \itshape ǹtyɛ̌k\\
\lspbottomrule
\end{tabularx}
\end{table}

\begin{table}[p]
\caption{An optional \textit{se-} abstract prefix in Berom}
\label{extab:nomaffplat:8}
\begin{tabularx}{.8\textwidth}{Xl}
\lsptoprule
Gloss 	& \ilit{Berom}\\
\midrule
‘leprosy’ 	&  \itshape (se-)kwa\\
‘madness’ 	&  \textit{(se-)loloŋ}\\
‘slavery’ 	&  \itshape (se-)sesàm\\
‘fascination, temptation’ 	&  \itshape setɔ́gɔ́s\\
\lspbottomrule
\end{tabularx}
\end{table}

  
   
\textit{ne}- is also a plural marker for a set of miscellaneous nouns in \tabref{extab:nomaffplat:6}.
 
\ili{Berom} also has an \textit{n-, ne-} unpaired marker for liquids, colours and abstracts as in \tabref{extab:nomaffplat:7}, comparable to the \textit{ma}- class 6 in \ili{Niger-Congo}.


An optional \textit{se-} prefix, noted with parentheses in \tabref{extab:nomaffplat:8}, marks abstract states.




There is no trace of either \ili{Bantu} class 3, \textit{mù}- for trees and plants, or Class 9, \textit{nì}- for animals. \ili{Berom} has a small set of nouns showing initial consonant mutation in \tabref{extab:nomaffplat:9} 

\begin{table}
\caption{Consonant mutation in Berom}
\label{extab:nomaffplat:9}
\begin{tabularx}{.8\textwidth}{XXl}
\lsptoprule
Gloss 	& {Sg.} 	&  {Pl.}\\
\midrule
‘hut for pounding’ 	& \textit{fwaŋ} 	& \textit{tàŋ}\\
‘cave’ 	& \itshape fware 	& \itshape tàre\\
‘thigh’ 	& \itshape fwa 	& \itshape tà\\
\lspbottomrule
\end{tabularx}
\end{table}


Presumably these originally had a singulative, \textit{fu}-, and the stem-initial \textit{t}- was deleted, converting the high back vowel into a labial.

\subsection{West-Central} %2.3 /
\subsubsection{General}

\ili{West-Central Plateau} consists of what used to be known as the ‘\ili{Southern Zaria}’ languages. Published and manuscript sources include \citet{Koelle1854,Gerhardt1971,Gerhardt1974,Gerhardt1983a,Gerhardt1994,AdwiraahHagen1983,Adwiraah1989,McKinney1979,McKinney1983,Norris1984,Norris1990,Follingstad1991,Follingstadnd}. Although these languages are clearly linked, no published evidence supports their coherence as a group. The languages \ili{Nandu} [=\ili{Ndun}] and \ili{Tari} [=\ili{Shakara}] were listed in \citet{CrozierBlench1992} as part of this group. This is erroneous; \ili{Ndun-Shakara}, together with the newly discovered \ili{Nyeng}, form their own group, \ili{Ndunic} (\sectref{sec:nomaffplat:2.4}). The \ili{West-Central Plateau} languages are a coherent geographical clustering and undoubtedly show numerous links with one another, but their genetic unity is unproven. \citet[67ff]{Gerhardt1983a} presents a comparative wordlist showing cognates between \ili{Rigwe}, \ili{Izere} and \ili{Tyap}. However, with both new insights into the phonology of these languages, and in particular the large number of lects still unrecorded at that period, a new comparative analysis is still to be undertaken. \figref{fig:nomaffplat:3} presents the known groups of \ili{West-Central Plateau} as a flat array.

\begin{figure}{}
 %%\includegraphics[width=\textwidth]{nomaffplat-img3}
 \begin{forest}  for tree={delay={where content={}{shape=coordinate}{}}},   forked edges  
 [\ili{West-Central Plateau}
  [\ili{Rigwe}]
  [\ili{Tyapic}]
  [\ili{Izeric}]
  [\ili{Hyamic}]
  [\ili{Koro}]
  [\ili{Gyongic}]
  ]
\end{forest}  
 \caption{\label{fig:nomaffplat:3} West-Central Plateau subgroups}
\end{figure}

 
\subsubsection{Rigwe}  %2.3.2 /

The \ili{Rigwe} language is spoken southeast of Jos. It is notable for an extremely complex phonology \citep{Anon2006}. Any former system of extensive alternating affixes has been replaced by a standard pluralising prefix or by a variety of tonal changes. Analysis of \ili{Rigwe} was undertaken by the author in co-operation with Daniel Gya. \tabref{tab:nomaffplat:5} lists the strategies for plural marking in \ili{Rigwe} with their allomorphs.

\begin{table}
\caption{\label{tab:nomaffplat:5}  Nominal plural marking in Rigwe}
\fittable{
\begin{tabular}{lll}
\lsptoprule
  No. 	&  {Strategy} 	&   Allomorph\\
\midrule
I. 	& addition of \textit{ɾè}- prefix 	&   \\
\tablevspace 
II. 	& \textit{ɾV}-/\`{N} alternation 	& ø-/\`{N}- alternation\\
\tablevspace 
III. 	& tone-raised on initial nasals with low tone 	& + stem-tone raising\\
\tablevspace 
IV. 	& extra-low tone initial nasal raised to mid 	& extra-low stem-tone-raising\\
\lspbottomrule
\end{tabular}
}
\end{table}


Class II nouns have a \textit{ɾV-/\`{N}}- alternation. \textit{rV-} is realized as \textit{ɾi-} when the stem vowel is front, and as \textit{ɾu-} when the stem-vowel is back. \textit{\`{N}-} is realised as \textit{ɲ-} before palatals \textit{ɲ-} and \textit{j-} and as \textit{ǹ-} elsewhere. \tabref{extab:nomaffplat:10} presents examples of the operation of this class.


\begin{table}
\caption{\textit{ɾV-/\`{N}}-  alternations in Rigwe nouns}
\label{extab:nomaffplat:10}
\begin{tabularx}{\textwidth}{XXl}
\lsptoprule
 Gloss 	& {Sg.} 	& {Pl.}\\
 \midrule
‘head’ 	& \itshape ɾitʃí 	& \itshape ǹtʃì\\
‘eye’ 	& \itshape ɾijiî 	& \itshape ɲ̀jì\\
‘tooth’ 	& \itshape ɾiɲiî 	& \itshape ɲ̀ɲì\\
‘horn’ 	& \itshape ɾité 	& \itshape ǹtè\\
‘hole’ 	& \itshape ɾuvɔ́ 	& \itshape ǹvɔ̀\\
\lspbottomrule
\end{tabularx}
\end{table}


\newpage 
As the glosses show, the nasal prefix is associated with human and animal body parts, which seems to be innovative. The Class III alternation in \ili{Rigwe} is \textit{ø-/\`{N}-}, where \textit{\`{N}-} is homorganic with the following consonant, realised as \textit{ɲ-} before palatals, \textit{ŋ-} before velars and \textit{n-} elsewhere. Only /a/, /e/ and /u/ have been recorded as stem vowels in Class III. Unlike the other classes, the stem tone changes and is always low, regardless of the tone in the singular. \tabref{extab:nomaffplat:11} presents examples of this class. This class is equally associated with body parts but is otherwise miscellaneous.

\begin{table}
\caption{\textit{ø-/\`{N}-} alternation in Rigwe  nouns}
\label{extab:nomaffplat:11}
\begin{tabularx}{.8\textwidth}{XXl}
\lsptoprule
Gloss 	& {Sg.} 	& {Pl.}\\
\midrule
‘bone’ 	& \itshape kú 	& \itshape \`{ŋ}kù\\
‘corpse’ 	& \itshape kʷé 	& \itshape \`{ŋ}kʷè\\
‘firewood’ 	& \itshape ekʷé 	& \itshape \`{ŋ}kʷè\\
‘food’ 	& \itshape jâ 	& \itshape  ɲ̀jà\\
‘hand’ 	& \itshape vá 	& \itshape ǹvà\\
‘leg’ 	& \itshape tá 	& \itshape ǹtà\\
‘part of’ 	& \itshape klá 	& \itshape ǹklà\\
 \lspbottomrule
\end{tabularx}
\end{table}


In Class IV, an extra-low nasal prefix is raised to mid, and an extra-low stem-tone becomes falling, shown in \tabref{extab:nomaffplat:12}.

\begin{table}
\caption{Extra-low nasal raising in Rigwe plurals}
\label{extab:nomaffplat:12}
\begin{tabularx}{.8\textwidth}{XXl}
\lsptoprule
Gloss 	& {Sg.} 	& {Pl.}\\
\midrule
‘chair’ 	& \itshape n̏ʨȕ    	& \itshape \={n}ʨû\\
‘chief’ 	& \itshape ŋ̏gʷɛ̏	& \itshape \={ŋ}gʷɛ̂\\
‘agama lizard’ 	& \itshape n̏dȁ 	& \itshape \={n}dâ\\
‘scar’ 	& \itshape ŋ̏mgbɛ ̏ 	& \itshape \={ŋ̄}mgbɛ̂\\
‘boyfriend’ 	& \itshape n̏ʨȁ 	& \itshape \={n}ʨâ\\
\lspbottomrule 
\end{tabularx}
\end{table}


\ili{Rigwe} has innovated in nominal affixing to such an extent that no obvious connection with postulated classes for either \ili{Niger-Congo} or \ili{Bantu} can be discerned.

\subsubsection{Tyapic} %2.3.3 /


\begin{table}
\footnotesize
\caption{\label{tab:nomaffplat:6} Tyap nominal affixes and concord  \citep[72]{Follingstad1991}}
\begin{tabularx}{\textwidth}{p{0.5cm}p{1cm}@{~}>{\itshape}l@{}p{1.8cm}@{}>{\itshape}p{1.5cm} Q>{\itshape}p{1.8cm}@{}>{\itshape}l}
\lsptoprule
Noun class 	& Number 	& \upshape Prefix 	&  \raggedright Tone~change on root	& \upshape {\raggedright\mbox{Post-concord} Element} 	&  Gloss 	& \upshape {Sg.} 	& \upshape {Pl.}\\
\midrule
1 	& Sg.  	& ə̀ 	&  	& wu 	& ‘hare’\newline ‘chief’ 	& ə̀sòm  wù\newline ə̀gwàm wù  \\
\tablevspace
2a 	& Pl.  	& ə̀yə 	&  	& ba 	& ‘hares’ 	&  	& ə̀yə́som bà\\
2b 	& Pl.  	& ø 	& + 	& ba 	& ‘chiefs’ 	&  	& ə̀gwam bà\\
\tablevspace
3 	& Sg.  	& ø 	&  	& ji 	& ‘cricket’\newline
									‘place’ 	& jèt jì\newline
                                    			  tyàn jì  \\
\tablevspace
4a 	& Pl.  	& ø 	& + & ji 	& ‘crickets’ 	&  	& jet jî  \\
4b 	& Pl.  	& redup.&  	& jí 	& ‘places’ 	&  	& tityàn jí\\
\tablevspace
5 	& Sg.  	& ə̀ 	&  	& ka 	& ‘tree’\newline
									‘farm’\newline ‘tooth’ 	& ə̀kən ka\newline 
                                    							ə̀bin ka\newline
                                                                ə̀nyuŋ kâ 	 \\
\tablevspace
6a 	& Pl.  	& ə̀kə̀ 	&  	& na 	& ‘trees’ 	&  	& ə̀kə̀kwə̀n nà\\
6b 	& Pl.  	& ə̀ + redup. 	&  	& hu 	& ‘farms’ 	&  	&bibin hu\\
6c 	& Pl.  	& ə̀ + redup. 	&  	& ba 	& ‘teeth’ 	&  	& ə̀nyûnyuŋ ba\\
6d  	& Pl.    & ø             & + 	& na\\
\tablevspace
7 	& Sg.  	& ø 	        &  	& hu 	& ‘hand’\newline 
											‘root’ 	& bwak hu\newline 
													  ə̀nan ka\\
\tablevspace
8a 	& Pl.  	& N- 	&  	& na 	& ‘hands’ 	&  	& mbwàk na\\
8b 	& Pl.  	& ə̀ + redup. 	&  	& ba 	& ‘roots’ 	&  	& ə̀nɨnan bâ\\
\tablevspace
9 	& sg/pl. & ə̀ 	&  	& na 	& ‘water’ 	& \mbox{ə̀sə̀khwôt~nà}  \\
\lspbottomrule 
\end{tabularx}
\end{table}

The \ili{Tyapic} languages are named for \ili{Tyap}, or \ili{Kataf} in older sources. The group consists of six languages (\ili{Tyap}, \ili{Gworok}, \ili{Atakar}, \ili{Kacicere}, \ili{Sholyo}, and \ili{Kafancan}), with the closely related \ili{Jju}\footnote{It is usual to list \ili{Jju} separately from the \ili{Tyap} cluster but this seems increasingly to reflect ethnic separation rather than linguistic reality.}. Only \ili{Tyap} itself is well-described \citep{Follingstad1991}. The prefixed elements appear to be innovative and consist of a (\textit{Ca}-) and its allomorphs. However, the noun is also followed a variety of alternating CV suffixes. These are almost certainly noun-class affixes, now placed after the stem. \tabref{tab:nomaffplat:6} shows a summary of \ili{Tyap} nominal affixes and concord as well as examples of nominal pairs.

The elements marked ‘post-concord’ in \tabref{tab:nomaffplat:6} were almost certainly former CV prefixes which have been copied at the end of the word, a procedure attested elsewhere in \ili{Niger-Congo}. They are written in the orthography as distinct words as they do not show phonological merger with the root they follow.


Plurals reduplicate by doubling the first syllable of the root. Thus \REF{ex:nomaffplat:13}:


\ea\label{ex:nomaffplat:13}
{ Plurals reduplicate the first root syllable}\\

  \textit{ə̀kwənka}    ‘tree’     \textit{ə̀kə̀kwə̀n nà}   ‘trees’
\z

Classes 1/2, with the suffixes \textit{wu/ba}, probably corresponds to \ili{Bantu} class 1/2 and includes many \ili{Tyap} nouns for human beings. Class 9, which is unpaired, includes liquids such as \textit{ə̀sə̀khwôt nà} ‘water’ and \textit{ə̀bààn na} ‘milk’ which is semantically similar to \ili{Niger-Congo} Class 6. The homorganic plural nasal prefix in Class 8a is possibly to be compared with \ili{Bantu} Class 6 where it is the plural of Class 5 ‘paired things’, e.g. \textit{mbwàk na} ‘hands’.

\citet[79]{Follingstad1991} shows that concord in \ili{Tyap} is much reduced with only a few adjectives and lower numerals showing any agreement. The agreement is of the ‘direct-copy’ or ‘echo’ type, where the numeral has the same prefix as the noun it agrees with.

\subsubsection{Izeric} %2.3.4 /

The \ili{Izeric} languages consist of northwest \ili{Izere}, northeast \ili{Izere}, \ili{Cèn}, \ili{Ganàng} and \ili{Fəràn}.\footnote{These last three are essentially single settlements, whereas the others represent clusters of villages, hence the rather asymmetric geographical names.} The language which is best-known is \ili{Izere of Fobur} but wordlists suggest that the affix pairings in the other languages are broadly similar.\footnote{Analysis of \ili{Izere} was undertaken by the author in collaboration with Bitrus Kaze.} \citet{Blench2000b} is a more detailed description of \ili{Izere} number marking. Nominal plurals in \ili{Izere of Fobur} are formed in four ways:

 
\begin{itemize}
 \item[a.]  affix alternation
 \item[b.]  stem-tone alternation
 \item[c.]  deverbal nouns that copy the alternations of verb stems
 \item[d.]  suppletion
\end{itemize}

Affix alternation and stem-tone alternation are frequently combined producing a very large number of plural formations. \ili{Izere of Fobur} has a relatively restricted set of segmental noun-class prefixes. \tabref{tab:nomaffplat:7} shows \ili{Izere} nominal affix pairings.

\begin{table}
\caption{\label{tab:nomaffplat:7} Izere nominal affix pairings}
\begin{tabularx}{\textwidth}{XXl}
\lsptoprule
  Singular 	&   Plural 	&   Semantics\\
  \midrule
\itshape a- 	& \itshape a- 	& persons, loanwords\\
\itshape i- 	& \itshape i- 	& miscellaneous\\
\itshape ka-, ki- 	&  	& diminutive\\
\itshape ka- 	& \itshape na- 	& birds, trees, miscellaneous\\
\itshape ku- 	& \itshape a-, i- 	& miscellaneous\\
\itshape nà- 	& \itshape ø- 	& liquids, solids, abstracts\\
\itshape ri- 	& \itshape a- 	& miscellaneous\\
\lspbottomrule
\end{tabularx} 
\end{table}

Tone cannot be specified for most \ili{Izere} prefixes, since it reflects the tone of the stem. The unpaired mass noun prefix, corresponding to \ili{Niger-Congo} Class 6, is always low tone. \textit{ka}- and its allomorph \textit{ki}-, realised when the noun stem contains a palatal, can function as a diminutive prefix. Paired \textit{ka}- and \textit{ku}- were probably allomorphs of one another historically, since there is a tendency for stem-vowels following \textit{ka}- to be front or central and those following \textit{ku}- to be back. However, exceptions now abound, suggesting a historical class split. 

\ili{Izere} has an unpaired \textit{nà}- prefix for liquids and solids which probably corresponds to the \textit{ma}- prefix in \ili{Niger-Congo}, shown in \tabref{extab:nomaffplat:14}.

\begin{table}
\caption{Examples of Izere unpaired prefix \textit{nà-}}
\label{extab:nomaffplat:14}
\begin{tabularx}{\textwidth}{XX@{\qquad}Xl}
\lsptoprule
Gloss 	&  {\ilit{Izere}} 	& Gloss 	&  {\ilit{Izere}}\\
\midrule
‘breast-milk’ 	& \itshape nàbàsang 	& ‘tears’ 	& \itshape nànyìsi\\
‘poison, venom’ 	& \textit{nàdɔm} 	& ‘dirt, fertiliser’ 	& \itshape nàrìk\\
‘gum’ 	& \itshape nàgàng 	& ‘blood’ 	& \itshape nàsɔ̀k\\
‘oil, pomade’ 	& \itshape nàmè 	& ‘local potash’ 	& \itshape nàtɔ̀k\\
‘dew’ 	& \itshape nàming 	& ‘sap’ 	& \itshape nàwùn\\
\lspbottomrule
\end{tabularx}
\end{table}

There is no evidence for a link between the common \textit{na}- prefix in \ili{Izere} and \ili{Bantu} nasal prefixes.
 
\subsubsection{Hyamic} %2.3.5 /

The \ili{Hyamic} languages are spoken between Kwoi and Nok, southwest of Jos and are now central to the prehistoric Nok culture. The members of the \ili{Hyamic} cluster are as follows:

\begin{itemize}
\item \ili{Cori}
\item \ili{Hyam} cluster (incl. Kwyeny, Yaat, Sait, Dzar, \ili{Hyam} of Nok)
\item \ili{Shamang}
\item \ili{Zhire-Shang}
\end{itemize}

Many of these languages are very poorly known and existing descriptions are tonally and phonologically inadequate (e.g. \citealt{Dihoff1976,Jockers1982}).

\ili{Hyam} has a wide range of strategies to mark nominal plurals. Analysis of \ili{Hyamic} languages is based solely on fieldwork by the author. The most important are shown in \tabref{tab:nomaffplat:8}:

\begin{table}
\caption{\label{tab:nomaffplat:8} Examples of Hyam noun pluralisation strategies}
\begin{tabularx}{\textwidth}{XXXl}
\lsptoprule
  Strategy 	& Gloss 	& {Sg.} 	& {Pl.}\\
\midrule
Tone-raising 	& `tree' 	& \itshape ki 	& \itshape kí\\
Prefix addition 	& `leaf' 	& \itshape ʤàŋ 	& \itshape maʤàŋ\\
	& `person' 	& \itshape nèt 	& \itshape mò-nèt\\
Prefix alternation 	& `blacksmith' 	& \textit{na-naa} 	& \textit{fu-naa}\\
Palatalisation 	& `vine' 	& \itshape rik 	& \textit{ryǐk}\\
Depalatalisation 	& `seed/grain' 	& \itshape ʃaŋ 	& \textit{sáŋ}\\
Labialisation 	& `fear/fright' 	& \itshape hyoŋ 	& \textit{hywoŋ}\\
Consonant mutation 	& `path' 	& \itshape fwor 	& \textit{swor}\\
\lspbottomrule
\end{tabularx}
\end{table} 

Transcription of tone is best described as schematic; \ili{Hyam} has a highly complex tone-system which is far from being fully understood, but which includes multiple contour tones, combining different levels of the underlying three-tone system.

All of these point to the former existence of nominal prefix alternation and palatalisation and labialisation to incorporated \textit{i}- and \textit{u}- prefixes. The \textit{ma}- prefix on ‘leaf’ is exceptional and not linked with the Class 6 prefix. The \textit{mò}- prefix is applied to most humans, large animals and reptiles, but not other animals, and a small scatter of miscellaneous lexemes. The \textit{na}-/\textit{fu}- singular/plural alternation is only recorded for a few nouns related to occupations as in \REF{ex:nomaffplat:15}, and may be some sort of reassigned relative marker (‘one who’) rather than a relic nominal affix.

\protectedex{
\ea\label{ex:nomaffplat:15}
\textit{na}-/\textit{fu}- singular/plural alternation\\
\medskip 
\noindent 
\parbox{3cm}{\textit{na-hywes}}
\parbox{2cm}{‘witch’}
\parbox{3cm}{\textit{fu-hywes}}
\parbox{2cm}{‘witches’}
\\
\parbox{3cm}{\textit{na-kyat kpyo}}
\parbox{2cm}{‘sorcerer’}   
\parbox{3cm}{\textit{fu-kyat kpyo}}
\parbox{2cm}{‘sorcerers’}
\\
\parbox{3cm}{\textit{na-naa}}
\parbox{2cm}{‘blacksmith’}
\parbox{3cm}{\textit{fu-naa}}
\parbox{2cm}{‘blacksmiths’}
\z
}

Almost all verbs and adjectives have obligatory plural forms and many undergo the same phonological shifts or mutations as nouns. Adjectives agree in number, i.e. where the noun is plural, the plural adjective is obligatory, but they do not show the type of alliterative concord characteristic of noun-class languages.

The \ili{Shang} language, while lexically \ili{Hyamic}, has a nominal affix system resembling \ili{Tinɔr} and similar \ili{Koro} languages (\sectref{sec:nomaffplat:2.3.7}) \citet{Blenchn.d.d}. \ili{Shang} has a reduced system of nominal affixes. The main noun-class pairs are between zero affixes in the singular and plural \textit{a}- and \textit{i}- prefixes seen in \tabref{tab:nomaffplat:9}. Rare plural prefixes include \textit{ka}-, \textit{u}- and \textit{ru}-. No singular affix, either productive or fossil, has been recorded. Some nouns referring to persons have a singular/plural alternation \textit{nè}-/\textit{fú}- (as in \ili{Hyam}) but these are probably not old affixes but compounded terms for ‘person’. The tone on the vowel of the plural affix always appears to be low.

\begin{table}
\caption{\label{tab:nomaffplat:9} Shang nominal affix pairings}
\begin{tabularx}{\textwidth}{XXXl}
\lsptoprule
 Affix 	& {Sg.} 	& {Pl.} 	& Gloss\\
\midrule
\itshape ø-/a- 	& \itshape ʤàŋ 	& \itshape à-ʤàŋ 	& ‘leaf’\\
\itshape ø-/i- 	& \itshape tàà 	& \itshape ì-taa 	& ‘stone’\\
\itshape ø-/u- 	& \itshape xá 	& \itshape ù-xá 	& ‘load’\\
\itshape ø-/ka- 	& \itshape kwè 	& \itshape kà-kwè 	& ‘nose’\\
\itshape à-/ru- 	& \itshape à-bin 	& \itshape rù-bin 	& ‘thing’\\
\lspbottomrule
\end{tabularx}
\end{table}

\largerpage
Semantic correlations are not very clear for most of these pairings. However, there is a strong predominance of body parts with the \textit{ka}- plural affix. Most nouns relating to persons have an \textit{a}- prefix in the plural, but since this is statistically the most common prefix, this may not be significant. There is no trace of nasal prefixes.

\subsubsection{Gyongic} %2.3.6 /

Gyongic is the closest relative of \ili{Hyamic} and consists of two languages, \ili{Gyong} [\ili{Kagoma}] and \ili{Angan} [\ili{Kamanton}]. Neither language is well-known but there is a description of \ili{Gyong} which includes information on noun-classes \citep{Hagen1988}. According to this, \ili{Gyong} marks plurals with prefix alternation, palatalisation alternation and tone. The data tables below follow her presentation. The reduced prefix system is as follows in \tabref{extab:nomaffplat:16}.

\begin{table}
\caption{Gyong nominal affixes}
\label{extab:nomaffplat:16}
 \begin{tabularx}{.66\textwidth}{Xl}
 \lsptoprule
 {Sg.} 	& {Pl.}\\ 
\midrule
\itshape ø- 	& \itshape bɔ̀, kì-\\
\itshape kì- 	& \itshape ø-, rì-\\
\lspbottomrule
 \end{tabularx}
\end{table}

Interestingly, liquids, abstracts and mass nouns fall into the unpaired \textit{kì}- class in \tabref{extab:nomaffplat:17}.

\begin{table}
\caption{Gyong mass noun \textit{ki-} prefix}   
\label{extab:nomaffplat:17}

\begin{tabularx}{.66\textwidth}{Xl}
\lsptoprule
Gloss 	&    {\ili{Gyong}}\\
\midrule
`blood' 	& \itshape kìdzí\\
`water' 	& \itshape kìmàláŋ\\
`oil' 	& \itshape kìtsɛ̀s\\
`death' 	& \itshape kìkpó\\
`ashes' 	& \itshape kìtɔ̀ŋ\\
`smoke' 	& \itshape kìdzɔ̀ŋ\\
`jealousy' 	& \itshape kìɣwúp\\
\lspbottomrule
\end{tabularx}
\end{table}


Stem-tone changes multiply the possible number-marking strategies. \citet[139]{Hagen1988} gives examples of adjectival agreement in \REF{ex:nomaffplat:18}.


\ea\label{ex:nomaffplat:18}
{ \ili{Gyong} adjectival agreement} \\

\gll kìpɛ̀ndɛ̀m kìlúm ~~~  pɛ̀ndɛ̀m rúm\\
large farm ~ large farms \\
\z

The data is not extensive enough to fully understand the system. Demonstratives do show alliterative concord in \REF{ex:nomaffplat:19}:

\ea\label{ex:nomaffplat:19}
{\ili{Gyong} demonstrative concord}\\ 
\gll  kìhá    kìhɔ́nà ~~~~~~ rìhá   rìdú(nà) \\
  house that ~ house those \\
\z

\subsubsection{Koro}\label{sec:nomaffplat:2.3.7}

The \ili{Koro} cluster consists of five languages spoken in Central Nigeria, north of Keffi. The published literature is sparse and based on limited data (\citealt{Gerhardt19723b,Goroh2000}). All the material presented on the \ili{Koro} languages is based on fieldwork by the author. \figref{fig:nomaffplat:4} shows a tentative outline classification of the languages in the \ili{Koro} cluster.

\begin{figure}
% \includegraphics[width=\textwidth]{figures/nomaffplat-img4}
\begin{forest}  for tree={delay={where content={}{shape=coordinate}{}}},   forked edges 
[\ili{Proto-Koro}
    [
      [\mbox{\ili{Waci}[=\ili{Begbere}]}]
      [\ili{Ashe}]
    ]
    [[\ili{Nyankpa-Barde}]]
    [
      [\ili{Idũ}]
      [\ili{Gwara}]
    ]	
]  
\end{forest}  

\caption{Classification of the Koro languages}
\label{fig:nomaffplat:4} 
\end{figure}

\ili{Waci} has retained a much richer noun-class system than any of the related languages, although it is in decay – see \tabref{tab:nomaffplat:10}. There is a strong tendency to cite some types of nouns, especially those to do with living things, without a singular prefix and to reduce the pluralisation marker to an \textit{a-} prefix. Moreover, there are a very large number of singular/plural pairings, many of them only occurring once suggesting a complex process of re-analysis is under way. Some prefixes have several allomorphs, probably prefiguring class merger. There is some semantic correlation with prefix pairings: for example, humans commonly have \textit{u-/bV}- prefixes and animals most often \textit{ì/i-}, but the correlation is far from perfect.

\begin{table}
\caption{\label{tab:nomaffplat:10} Waci nominal affix pairings}
\begin{tabularx}{\textwidth}{lQl}
\lsptoprule
Singular 	& Plural 	& Semantics\\
\midrule
\itshape ø-, ì-, ù- 	& \itshape bV- 	& human beings\\
\itshape i- 	& \itshape i- 	& large or salient animals, trees\\
\itshape i- 	& \itshape a-, bV-, ri- 	& miscellaneous\\
\itshape gV- 	& \itshape ru-, ro- 	& miscellaneous\\
\itshape gV- 	& \itshape bV- 	& miscellaneous\\
\itshape o- 	& \itshape i- 	& miscellaneous\\
\itshape wu- 	& \itshape a-, E-, O- 	& miscellaneous\\
\itshape yV- 	& \itshape bV- 	& miscellaneous\\
\lspbottomrule
\end{tabularx} 
\end{table}


There is no evidence for a distinctive mass noun prefix. Some liquids, such as water (\textit{bàm}) and blood (\textit{bèʤí}) show no singular/plural prefix alternation, while others, such as tears, saliva and urine, have diverse singular/plural affix pairs.

The \textit{bV}- plural prefix almost always marks persons and is usually, but not always paired with \textit{u}- singular \tabref{extab:nomaffplat:20}. The vowel is underspecified and very often copies the stem vowel, although \textit{b}+ high vowel (i.e. \textit{bi}- and \textit{bu}-) is apparently not permitted.


\begin{table}
\caption{\textit{bV-} plural prefixes in Waci}
\label{extab:nomaffplat:20}
\begin{tabularx}{\textwidth}{QQl}
\lsptoprule
Gloss 	& {Sg.} 	& {Pl.}\\
\midrule
‘person/people’ 	&  \itshape ù-ndìrà 	&  \itshape bà-ndìrà\\
‘husband’ 	&  \itshape ù-sá 	&  \itshape bá-sà\\
‘wife’    	& \textit{ù-cɛ́} 	&  \itshape bɛ̀-cɛ́\\
‘masquerade type’	&  \itshape keberè 	&  \itshape be-keberè\\
‘leper’ 	&  \itshape ì-kpíŋ 	&  \itshape bè-kpíŋ\\
‘masquerade type’	&  \itshape ú-kù 	&  \itshape bó-kù\\
‘brother’ 	&  \itshape ù-cɔ́bɔ̀ 	&  \itshape bɔ̀-cɔ́bɔ̀\\
‘friend’ 	&\itshape ù-dɔ̃́rĩ̀	&  \itshape bɔ̀-dɔ̃́rĩ̀\\
 \lspbottomrule
\end{tabularx}
\end{table}


There is a tendency for the V- of other plural prefixes to copy the ±ATR properties of the stem vowel where these are mid. See \tabref{extab:nomaffplat:21}.

\begin{table}
\caption{(C)V- prefixes in Waci, illustrating ±ATR vowel copying}
\label{extab:nomaffplat:21}
\begin{tabularx}{\textwidth}{XXl}
 \lsptoprule
Gloss 	& {Sg.} 	& {Pl.}\\
\midrule
‘death’ 	&  \itshape gà-pú 	&  \itshape rù-pú\\
‘Senegal coucal’ 	&  \itshape gbodotǔtǔ 	&  \itshape o-gbodotǔtǔ\\
‘story’ 	&  \itshape wù-sɔ́sɔ̀gɔ̀ 	&  \itshape  ɔ̀-sɔ́sɔ̀gɔ̀\\
‘wound’ 	&  \itshape wù-sɔ̀ 	&  \itshape ɔ̀-sɔ̀\\
 \lspbottomrule
\end{tabularx}
\end{table}

 
\newpage  
But there are exceptions as in \REF{ex:nomaffplat:22}:

\eabox{
\label{ex:nomaffplat:22}
\begin{tabularx}{\textwidth}{lll} 
‘song’ 	&  \itshape wù-vʷɔ́m 	&  \itshape ò-vʷɔ́m\\ 
\end{tabularx}
}

\textit{wu}- (\textit{gu}- in some speakers) is a very common prefix which can be paired with almost any plural \textit{V-} prefix as in \tabref{extab:nomaffplat:23}.

\begin{table}
\caption{Waci \textit{wu-} singular prefix and its pairings}
\label{extab:nomaffplat:23}
\begin{tabularx}{\textwidth}{XXl}
\lsptoprule 
Gloss 	& {Sg.} 	& {Pl.}\\
\midrule
‘leaf’ 	&  \itshape wù-yí(í) 	&  \itshape à-yí(í)\\
‘root’ 	&  \itshape wù-náŋ 	&  \itshape à-náŋ\\
‘rubbish-heap’ 	&  \itshape wù-rírí 	&  \itshape ɛ̀-rírí\\
‘village/settlement’ 	&  \itshape wù-sɛ́p 	&  \itshape  ɛ́-sɛ̀p\\
‘arm, hand’ 	&  \itshape wù-bɔ́	&  \itshape ɔ̀-bɔ́\\
‘story’ 	&  \itshape wù-sɔ́sɔ̀gɔ̀ 	&  \itshape ɔ̀-sɔ́sɔ̀gɔ̀\\
‘wall (of room)’ 	&  \itshape wù-gúgò 	&  \itshape ò-gúgò\\
 \lspbottomrule
\end{tabularx}
\end{table}


\textit{u}- may also be an allomorph of \textit{wu}- in \tabref{extab:nomaffplat:24}.

\begin{table}
\caption{Waci \textit{u-} prefix and its plural pairings}
\label{extab:nomaffplat:24}
\begin{tabularx}{\textwidth}{XXl}
\lsptoprule 
Gloss 	&  {Sg.} 	&  {Pl.}\\
\midrule
‘large river’ 	&  \itshape ù-hɛ́k	 	&  \itshape ɛ́-hɛ̀k\\
‘thing’ 	&  \itshape ù-bín 	&  \itshape è-bín\\
‘wart-hog’ 	&  \itshape ù-jì 	&  \itshape e-ji\\
‘tail’ 	&  \itshape ù-sáp 	&  \itshape ì-sáp\\
‘load’ 	&  \itshape ù-cá 	&  \itshape ì-cá\\
‘day’ 	&  \itshape ù-nɔ́m	 	&  \itshape í-nɔ̀m\\
‘night’ 	&  \itshape ù-ʃ\'{ĩ} 	&  \textit{ɛ́-ʃĩ̀ } \\
‘bark (of tree)’ 	&  \itshape ù-gùgúb 	&  \itshape ɔ̀-gùgúb\\
 \lspbottomrule
\end{tabularx}
\end{table}


\textit{i}- prefixes alternating with other prefixes than \textit{i}- are quite rare and somewhat inconsistent in \REF{ex:nomaffplat:25}:

\eabox{\label{ex:nomaffplat:25}
\begin{tabularx}{\textwidth}{lll}
‘thorn’ 	&  \itshape  ì-dìdɔ́k	&  \itshape bà-dìdɔ́k\\
‘year’ 	&  \itshape ì-yɛ́	 	&  \itshape gɛ̀-yɛ́\\
\end{tabularx}
}

The \ili{Waci} prefix \textit{yV}- where V is always a front vowel is usually paired with \textit{bV}- in the plural seen in \tabref{extab:nomaffplat:26}, although these nouns do not refer to persons as might be expected by analogy to the pairing of \textit{mu-/ba-} (classes 1/ 2) for persons in \ili{Bantu}.

\begin{table}
\caption{\textit{yV-} prefixes in Waci}
\label{extab:nomaffplat:26}
\begin{tabularx}{\textwidth}{XXl}
\lsptoprule
Gloss 	& {Sg.} 	& {Pl.}\\
\midrule
‘star(s)’ 	&  \textit{gè-j\'{ĩ} {\textasciitilde} yì-j\'{ĩ}}  	&  \itshape bà-j\'{ĩ}\\
‘fire’ 	&  \textit{gì-rá {\textasciitilde} yì-rá}  	&  \itshape bà-rá\\
‘boil’ 	&  \itshape yì-kpì 	&  \itshape bè-kpì\\
‘pygmy mouse’ 	&  {\itshape yì-kìríko} 	&  \textit{bò-kìríko} \\
‘bird (generic)’ 	&  \itshape yɛ̀-nɔ̀	 	&  \itshape bà-nɔ̀\\
	&  	& \\
but: 	&  	& \\
‘faeces’ 	&  \itshape yè-bì 	&  \itshape ru-bi\\
\lspbottomrule
\end{tabularx}
\end{table}


One of the most striking alternations is \textit{gV-/rV-}, which does not seem to have any immediate parallel in other \ili{Koro} languages. The \textit{-V-} in \textit{gV-} can be any vowel except the high back vowels. The vowel quality in the \textit{gV-} prefix partly reflects stem vowels although the correlation is not perfect. Similarly, most plurals have \textit{rV-} with a few exceptions (\tabref{extab:nomaffplat:27}). Some \textit{yV-} prefixes, such as ‘faeces’ in \tabref{extab:nomaffplat:26} may well be allomorphs of \textit{gV-} to judge by the \textit{rV-} plurals.

The \ili{Waci} nominal affix system seems to have undergone major renewal. Apart from a class pair for persons and a rather weak animal class, there is no evidence for an unpaired non-count noun prefix and no evidence for semantically clustered prefix pairs elsewhere.


\begin{table}
\caption{\textit{gV-} prefixes in Waci}
\label{extab:nomaffplat:27}
\begin{tabularx}{\textwidth}{XXl}
\lsptoprule 
Gloss 	& {Singular} 	& {Plural}\\
\midrule
‘compound’ 	&  \itshape gá-hà 	&  \itshape rú-hà\\
‘forest’ 	&  \itshape gà-kwéy 	&  \itshape rù-kwéy\\
‘death’ 	&  \itshape gà-pú 	&  \itshape rù-pú\\
‘stick’ 	&  \itshape gá-tɪ̀	&  \itshape ró-tɪ̀\\
‘gecko’ 	&  \itshape ge-mɛ́ kpikpi 	&  \itshape  ru-mɛ́ kpikpi\\
‘tongue’ 	&  \itshape gɛ̀-ɽɛ́m 	&  \itshape rù-ɽɛ́m\\
‘rope’ 	&  \itshape gɛ-ri 	&  \itshape ru-ri\\
‘genet cat’ 	&  \itshape gibíkɔn 	&  \itshape bɛ̀bikɔn\\
‘thigh’ 	&  \itshape gì-cáy ùdà 	&  \itshape à-cáy àdà\\
‘stomach’ 	&  \itshape gɪ̀-nɪ́ 	&  \itshape bà-nɪ́\\
‘sandfly’ 	&  \itshape gì-zù 	&  \itshape bò-zù\\
‘bag’ 	&  \itshape gò-gúr 	&  \itshape rù-gúr\\
‘snake (generic)’ 	&  \itshape go-sʊ 	&  \itshape ru-sʊ\\
\lspbottomrule
\end{tabularx}
\end{table}

\subsection{Ndunic (=Ahwai)} \label{sec:nomaffplat:2.4}

\ili{Ndunic} is a new name proposed here for the languages previously called ‘\ili{Nandu-Tari}’. Existing sources list two languages, but a third language, \ili{Ningon}, was first recorded in 2003. The \ili{Ndunic} languages are spoken in a small area southwest of Fadan Karshi. The correct names for these languages are \ili{Ndun} (\ili{Nandu}), \ili{Shakara} (\ili{Tari}) and \ili{Ningon}. The languages are extremely close to one another. The Ndunic peoples have recently adopted the name ‘\ili{Ahwai}’ as a cover term for all three languages \citep{RueckEtAl2008}. \ili{Shakara} has a much reduced set of nominal affixes, but \ili{Ndun} has numerous nominal singular/plural affix pairs. All the tables for \ili{Ndunic} languages are based on fieldwork by the present author. The main attested noun-class pairings of \ili{Ndun} are shown in \tabref{tab:nomaffplat:11}.

However, there are also numerous plurals created by tonal change and by presence and absence of labialisation and palatalisation. Sporadic nasalisation appears between the stem and the prefix as a result of fossil nominal prefixes, although \ili{Ndun} still preserves a few productive nasal prefixes. \ili{Ndun} has many noun-class pairings that only occur once, in part due to the underspecified vowels. The tones are too insecurely marked to be sure that there are no additional contrasts on the \textit{V-} prefixes.

\begin{table}
\caption{\label{tab:nomaffplat:11}Ndun nominal affix pairings}
\begin{tabularx}{.8\textwidth}{Xl}
\lsptoprule
{Sg.} 	& {Pl.}\\
\midrule
\itshape ø- 	& \itshape e-, i-, i(Cy)-, -y-\\
\itshape a-, a(n)- 	& \itshape i-, me-, na-\\
\itshape e- 	& \itshape ø-, be-, i(n)-\\
\itshape i- 	& \itshape be-\\
\itshape m-, ma-, me(n)- 	& \itshape ø-\\
\itshape n- 	& \itshape be-\\
\itshape u- 	& \itshape e-, i(Cy)-, n-\\
\itshape -y- 	& \itshape ø-\\
\lspbottomrule
\end{tabularx}
\end{table}

Palatalisation can be applied to almost any initial consonant in singular/plural formation, often combined with primary affix alternation as in \tabref{extab:nomaffplat:28}. The likely historical explanation is that there was an initial \textit{i}- prefix which was incorporated into the stem and then a new plural affix (ironically sometimes a new \textit{ i}- prefix) was applied subsequently.


\begin{table}
\caption{Ndun nominals with contrastive palatalization}
\label{extab:nomaffplat:28}
\begin{tabularx}{.8\textwidth}{XXl}
\lsptoprule
Gloss 	& {Sg.} 	& {Pl.}\\
\midrule
‘dream’ 	&  \itshape nári 	&  \itshape ínyári\\
‘relations’ 	&  \itshape ùgap 	&  \itshape ìgyàp\\
‘song’ 	&  \itshape úhwá 	&  \itshape ihywa\\
‘body’ 	&  \itshape ilyak 	&  \itshape ilak\\
\lspbottomrule 
\end{tabularx}
\end{table}

\ili{Ndun} also shows numerous examples of sporadic inserted nasals in affix alternations as in \tabref{extab:nomaffplat:29}. 


\begin{table}
\caption{Ndun nominals with sporadic inserted nasals in prefixes}
\label{extab:nomaffplat:29}
\begin{tabularx}{.8\textwidth}{XXl}
\lsptoprule
 {Gloss} 	& {Sg.} 	&   {Pl.}\\
\midrule‘cheek’ 	&  \itshape upǎŋ 	&  \itshape empaŋ\\
‘grandparent’ 	&  \itshape ìnìnkyer 	&  \itshape íníkyer\\
‘chief’ 	&  \itshape ètùm 	&  \itshape entûm\\
‘horn’ 	&  \itshape anshem 	&  \itshape meshèm\\
‘spider’ 	&  \itshape tìntàn 	&  \itshape intíntàn\\
\lspbottomrule
\end{tabularx}
\end{table}


Only a single example of an alternating \textit{n-} prefix showing alternation has been recorded, shown in \tabref{extab:nomaffplat:30}.

\begin{table}
\caption{Single example of alternating n-prefix}
\label{extab:nomaffplat:30}
\begin{tabularx}{.8\textwidth}{XXl}
\lsptoprule
Gloss 	& {Sg.} 	& {Pl.}\\
\midrule
‘thorn’ 	&  \itshape ùshayí 	&  \itshape ǹshayî\\
\lspbottomrule
\end{tabularx}
\end{table}
 
\newpage  
In addition there are many nouns with initial homorganic nasals (\textit{m-, n-, ŋ-}) which seem to have been incorporated during an earlier wave of prefix incorporation.

It is not uncommon for \ili{Ndun} nouns for persons to be \textit{-r} final \tabref{extab:nomaffplat:31}:

\begin{table}
\caption{Ndun nominals with final \textit{-r}}
\label{extab:nomaffplat:31}
\begin{tabularx}{.8\textwidth}{XXl}
\lsptoprule
Gloss 	& {Sg.} 	& {Pl.}\\
\midrule
‘person/people’ 	&  \itshape ènèr 	&  \itshape bénèr\\
‘man’ 	&  \itshape èromir 	&  \itshape béromír\\
‘grandparent’ 	&  \itshape ìnìnkyer 	&  \itshape íníkyer\\
‘friend’ 	&  \itshape èsamir 	&  \itshape bésamir\\
 \lspbottomrule
\end{tabularx}
\end{table}


 In one case, the final \textit{-r} alternates with a final nasal as in \REF{ex:nomaffplat:32}.


\eabox{\label{ex:nomaffplat:32}
\begin{tabularx}{\textwidth}{lll}
‘woman’ 	&  \itshape nyaan 	&  \itshape nyaar\\
\end{tabularx}
}

These are probably the traces of former prefixes which have moved to final position and have almost lost their class pair alternation. Semantic correlations with noun-class affix pairings are weak at best. The \textit{e-/be-} prefix pair includes many nouns referring to persons (\tabref{extab:nomaffplat:33}).


\begin{table}
\caption{Ndun \textit{e-/be-} prefixes marking persons}
\label{extab:nomaffplat:33}
\begin{tabularx}{.8\textwidth}{XXl} 
\lsptoprule
Gloss 	& {Sg.} 	& {Pl.}\\
\midrule
‘person/people’ 	&  \itshape ènèr 	&  \itshape bénèr\\
‘man’ 	&  \itshape èromir 	&  \itshape béromír\\
‘father’ 	&  \itshape èdâ 	&  \itshape bédâ\\
‘friend’ 	&  \itshape èsamir 	&  \itshape bésamir\\
‘guest/stranger’ 	&  \itshape èkyen 	&  \itshape békyen\\
\lspbottomrule
\end{tabularx}
\end{table}

Most liquids have initial \textit{m-} or  \textit{mV-} and this presumably reflects \ili{Niger-Congo} Class 6 \tabref{extab:nomaffplat:34}. 

\begin{table}
\caption{Ndun \textit{mV-} prefixes marking liquids}
\label{extab:nomaffplat:34}
\begin{tabularx}{.66\textwidth}{Xl}
\lsptoprule
Gloss 	&  {\ilit{Ndun}}\\
\midrule
‘water’ 	&  \itshape mákúrì\\
‘blood’ 	&  \itshape mémiŋ\\
‘tear’ 	&  \itshape mémil\\
‘saliva’ 	&  \textit{méntí} \\
‘sweat’ 	&  \itshape ḿfɔɔr\\
‘urine’ 	&  \itshape ménfìrì\\
\lspbottomrule
\end{tabularx}
\end{table}


However, where \textit{mV}- appears as a plural number marker it seems to show no semantic correlation. No other \ili{Ndun} prefixes show any tendency to reflect semantic classes such as body parts, trees or salient animals.

Shakara now has a much reduced system, but \ili{Proto-Ndunic} clearly had a wide range of nominal affix pairs, with fragmentary evidence for a suffix alternation to do with persons. Nasal prefixes were clearly very common but have become so generalised across the system it is now difficult to discern what part they may have played in the original affix alternations.

\subsection{Ninzic} %2.5 /

\ili{Ninzic}, formerly Plateau IV, is probably the most difficult group to characterise and weak data on several languages make it unclear whether certain peripheral languages really belong to it. The name \ili{Ninzic} is introduced here, reflecting the element \textit{nin}-, which is part of many ethnonyms. The \ili{Ninzic} languages are spoken south of Fadan Karshi in Plateau, Nassarawa and Kaduna States. The membership of \ili{Ninzic} has changed quite significantly between various publications noted in \tabref{tab:nomaffplat:12}.

\begin{table}
\caption{\label{tab:nomaffplat:12}Changing composition of the Ninzic language group.\\
Key: Blank = not listed; + = assigned to group; -- = assigned to another group; ? thus in source.}
\begin{tabularx}{\textwidth}{l QQQQQ@{\qquad}}
\lsptoprule
	&
\rotatehead[3cm]{\mbox{\citet{Greenberg1963}}} 	& 
\rotatehead{\mbox{\citet{HansfordEtAl1976}}} 	&
\rotatehead{\mbox{\citet{Gerhardt1989}}} 	&
\rotatehead{\mbox{\citet{CrozierBlench1992}}} 	& 
\rotatehead{\mbox{This paper}}\\
\midrule
\ili{Ce} [=\ili{Rukuba}]   	& + 	& + 	& + 	& + 	& +\\
\ili{Ninzo}   [=\ili{Ninzam}] 	& + 	& + 	& + 	& + 	& +\\
\ili{Mada} 	& + 	& + 	& + 	& + 	& +\\
\ili{Nko} 	&  	&  	&  	&  	& +\\
\ili{Katanza} 	&  	&  	&  	&  	& +\\
\ili{Bu-Niŋkada} 	&  	& – 	& – 	& – 	& +\\
\ili{Ayu} 	& + 	& + 	& + 	& ? 	& ?\\
\ili{Nungu} 	&  	& - 	& - 	& - 	& +\\
\ili{Ninkyob} [=\ili{Kaninkwom}] 	& + 	& + 	& + 	& + 	& +\\
\ili{Anib} = \ili{Kanufi} 	&  	& + 	& + 	& + 	& +\\
\ili{Nindem} 	&  	& + 	& + 	& + 	& +\\
\ili{Gwantu cluster} 	&  	& + 	& + 	& + 	& +\\
\ili{Ningye} 	&  	&  	&  	&  	& +\\
\ili{Ninka} 	&  	&  	&  	&  	& +\\
\ili{Kwanka-Boi-Bijim-Legeri} 	&  	& + 	& + 	& + 	& –\\
\ili{Shall-Zwall} 	&  	& + 	&  	& ? 	& –\\
\ili{Pe}[=\ili{Pai}] 	&  	& - 	& + 	& – 	& –\\
\lspbottomrule
\end{tabularx}
\end{table}

General overviews can be found in \citet{Gerhardt19723a,Gerhardt1983a} and materials on specific languages in \citet{Hoffman1976,Hörner1980,Price1989,Wilson2003}.

The number marking systems of \ili{Ninzic} must originally have been paired affixes with alliterative concord, as fragments of such systems are found across the group. However, in most languages the system has broken down or become severely eroded and compensatory strategies have evolved. This section uses examples from \ili{Ninzo} based on \citet{Hörner1980,NinzoLanguageProjectCommittee1999} and fieldwork in Fadan Wate in 1995 \citet{Blenchn.d.e}. \ili{Ninzo} prefix pairings are in \tabref{extab:nomaffplat:35}.

\begin{table}%35
\caption{Ninzo prefix pairings}
    \label{extab:nomaffplat:35} 
\begin{tabularx}{\textwidth}{XX}
\lsptoprule 
{Sg.} 	& {Pl.}\\
\midrule
\itshape ø- 	& \itshape à-, ì-\\
\itshape i- 	& \itshape à-\\
\itshape ù- 	& \itshape à-, ì-\\
\lspbottomrule
\end{tabularx}
\end{table}


Many words have unproductive prefixes and singular and plural is now marked only by tone. Some \textit{u-/a-} prefix alternations are co-associated with \textit{u-/i-} alternations in the first vowel of the stem in \tabref{extab:nomaffplat:36}.

\begin{table}
\caption{Ninzo \textit{u-/i-} alternations in first vowel of stem} 
\label{extab:nomaffplat:36}
\begin{tabularx}{\textwidth}{XXl}
\lsptoprule
Gloss 	& {Sg.} 	& {Pl.}\\
\midrule
‘man’ 	&  \itshape ù-nùru 	&  \itshape a-nirú\\
‘old person’ 	&  \itshape ù-tuce 	&  \itshape a-tice\\
‘senior in age’ 	&  \itshape ù-nunku 	&  \itshape a-ninku\\
‘bow’ 	&  \textit{ù-tuta\textsuperscript{+}} 	&  \itshape i-tita\\
 \lspbottomrule
\end{tabularx}
\end{table}



\begin{table}
\caption{Ninzo prefix pairs \textit{u-nV-/a-bV-} } 
\label{extab:nomaffplat:37}
\begin{tabularx}{\textwidth}{XXl}
 \lsptoprule
Gloss 	& {Sg.} 	& {Pl.}\\
\midrule
‘guest/stranger’ 	&  \itshape ù-ni-cir 	&  \itshape a-bi-cir\\
‘doctor’ 	&  \itshape ù-ni-fù 	&  \itshape a-bi-fù\\
‘hunter’ 	&  \itshape ù-nì-zhá 	&  \itshape à-bì-zhá\\
‘thief’ 	&  \itshape ù-nà-yí 	&  \itshape à-bà-yí\\
‘blacksmith’ 	&  \itshape u-nì-là 	&  \itshape a-bí-lá\\
	&  	& \\
But: 	&  	& \\
‘witch’ 	&  \itshape ù-nu-tri 	&  \itshape a-da-tri\\
\lspbottomrule
\end{tabularx}
\end{table}


Other \textit{u/a-} prefix alternations also incorporate alternations of CV syllables of the stem as in \tabref{extab:nomaffplat:37}, particularly \textit{u-nV-/a-bV-}. These suggest an unusual process, the retention of a former \textit{ni-/bi}- alternation with the addition of an innovative prefix system preceding it. The \textit{bV}- plural marker is reminiscent of \ili{Niger-Congo} \textit{ba}- but this may be coincidence; the core lexemes for persons in \ili{Ninzo} do not have this alternation. A partial development from this is the formation of plural with \textit{VnV}- prefixes \tabref{extab:nomaffplat:38}. For example, \textit{à}- and \textit{ì}- singular prefixes alternate with \textit{ànV}- plural prefixes.


\begin{table}
\caption{Ninzo prefix pairs \textit{V-/anV-}}
\label{extab:nomaffplat:38}
\begin{tabularx}{\textwidth}{XXl}
\lsptoprule
Gloss 	& {Sg.} 	& {Pl.}\\
\midrule
‘death’ 	&  \itshape ì-kfu 	&  \itshape áni-kfu\\
‘leopard’ 	&  \itshape ì-ce 	&  \itshape áni-ce\\
‘guinea-fowl’ 	&  \itshape ì-tsì 	&  \itshape áni-tsì\\
‘kob antelope’ 	&  \itshape à-kùrù 	&  \itshape  áná-kúrú\\
‘cat’ 	&  \itshape à-músâ 	&  \itshape àna-músâ\\
‘chameleon’ 	&  \itshape a-kanda 	&  \itshape anu-kanda\\
\lspbottomrule 
\end{tabularx}
\end{table}

As \tabref{extab:nomaffplat:38} shows there is quite a strong correlation between animals and the \textit{anV}- plural prefix, which is highly reminiscent of the \ili{Bantu} Class 9 \textit{nì}- singular prefix for animals. \ili{Ninzo} shows no obvious active or fossil morphology for non-count nouns although the word for ‘water’, \textit{amasíɽ}, has inherited the \textit{ma}- affix from related Plateau languages.

\newpage 
A common number marking process, which can be combined with prefix alternations, is reduplication of the first syllable of the root seen in \tabref{extab:nomaffplat:39}. The vowel of the reduplicated syllable is usually \textit{/i/}, but /\textit{u/} in two unexplained cases. 

\begin{table}
\caption{Plural marking with reduplication in Ninzo}
\label{extab:nomaffplat:39}
\begin{tabularx}{\textwidth}{XXl}
  \lsptoprule
Gloss 	& {Sg.} 	& {Pl.}\\
\midrule
‘senior in status’ 	&  {\itshape àŋkpyè} &  \textit{aŋkpikpyè} \\
‘ankle’ 	&  \itshape í-gblédzá 	&  \itshape à-gbígblédzá\\
‘navel’ 	&  \itshape í-mgbèkù 	&  \itshape í-mgbímgbèkè\\
‘liver’ 	&  \itshape ì-sur 	&  \itshape ì-sisur\\
‘animal (bush)’ 	&  \itshape í-názhù 	&  \itshape í-nínazhù\\
‘hoe’ 	&  \itshape à-kla 	&  \itshape í-kikla\\
‘termite’ 	&  \itshape í-yó 	&  \itshape í-yíyó\\
‘knife (small)’ 	&  \itshape á-njî 	&  \itshape í-njínjî\\
‘gown, small’ 	&  \itshape à-nkru 	&  \itshape í-nkinkru\\
‘basket (generic)’ 	&  \itshape à-sà 	&  \itshape í-sísà\\
‘arrow’ 	&  \itshape à-wyírr 	&  \itshape i-wyiwyírr\\
‘friend’ 	&  \itshape ù-kpà 	&  \itshape á-kpukpà\\
‘king’ 	&  \itshape ù-ʈû 	&  \itshape á-túʈù\\
\lspbottomrule
 \end{tabularx}
\end{table}


 
\ili{Ninzic} languages have highly diverse nominal morphology and space precludes describing all of them. Many have a non-count noun prefix, but this seems to vary from one group to another. For example, \tabref{extab:nomaffplat:40} shows the prefix for liquids in \ili{Ce}, \textit{bə-}, which is quite consistent, but which seems to be segmentally unrelated to \ili{Niger-Congo} Class 6, usually \textit{mV-}.

\begin{table}[t]
\caption{C\textit{e} prefix for liquids \textit{bə-}}
\label{extab:nomaffplat:40}
\begin{tabularx}{.66\textwidth}{Xl}
\lsptoprule
Gloss 	&    {C\textit{e}}\\
\midrule
‘oil’ 	&  \textit{bə̀-nyì}\\
‘fat/grease’ 	&  \textit{bə̀-nhyʊ̀}\\
‘boiled sorghum’ 	&  \textit{bə̀-kʊ̀}\\
‘potash’ 	& \itshape   bə̀-tòk	 \\
‘sorghum-beer’ 	&  \textit{bə̀-hi}\\
‘milk’ 	& \itshape  bə̀-nsə\\
‘sweat’ 	&  \textit{bə̀-cilí}\\
\lspbottomrule
\end{tabularx}
\end{table}



This is an example of metatypy, the copying of a structural feature without the associated segments.

The \ili{Mada} language has undergone a striking collapse of characteristic affix alternations, which have then been rebuilt using grammaticalisation strategies, which have resulted in highly idiosyncratic marking of nominal plurals. These can be divided into six categories:

\ea
\begin{enumerate}
\item[I]    tone-change
\item[II]   initial syllable reduplication
\item[III]  prefix addition
\item[IV]   person nouns grammaticalised as pseudo-prefixes
\item[V]    diminutives grammaticalised as pseudo-prefixes
\item[VI]   suppletives
\end{enumerate}
\z

Prefixes marking size can alternate with non-prefixed nouns creating a pleth\-o\-ra of additional forms. Some nouns usually take diminutive prefixes in speech, but these are not easy to predict. The historical layering of these number marking strategies can be detected through the existence of multiple forms, sometimes with, for example, tone-raising applied to a noun formerly which also has first syllable reduplication or prefix addition. The consequence of this has been that the tone-plurals of \ili{Mada} show extremely low levels of predictability as in \tabref{tab:nomaffplat:13}. 

\begin{table}
\caption{\label{tab:nomaffplat:13}First syllable reduplication in Mada nouns}

\begin{tabularx}{\textwidth}{lllX}
\lsptoprule
Pattern 	&  {Sg.} 	&   {Pl.} 	&  gloss\\
\midrule
\textit{be→bə} 	& \textit{bě }	& \textit{be}, \textit{bəbe}	&   ‘seed’ \\
\tablevspace	
\textit{bwɔ→bə} 	& \textit{bwɔ̌} 	& \textit{bə̄bwɔ} 	&   	‘pocket’\\
\tablevspace	
\textit{cu→cu} 	& \itshape cūn 	& \itshape mə̀cùn, mə̀cūcùn	& ‘chief’\\
\tablevspace	
\textit{gbu→gbu}	& \itshape gbù 	& \itshape gbūgbu 	& ‘town, hill’\\
\tablevspace	
\textit{gyə→gi} 	& \itshape gyə̌r 	& \itshape gigyər 	& ‘mother’\\
\tablevspace	
\textit{kpa→kpə}	& \itshape kpān 	& \itshape kpə̄kpàn 	& ‘friend’\\
\tablevspace	
\textit{kri→kə} 	& \itshape krì 	& \itshape kə̄krì 	& ‘yam’\\
\tablevspace	
\itshape lɔ→lə 	& \itshape lɔn 	& \itshape mə̄lə̄lɔn 	& ‘husband’\\
\tablevspace	
\textit{ci→ci} 	& \itshape mə̀cī 	& \textit{mə̄cici} 	& ‘father-in-law’ \\
\tablevspace	
\textit{mbə→mbə}	& \itshape mbə̄ 	& \itshape mbə̀mbə̄ 	& ‘wife, woman’\\
\tablevspace	
\textit{mgba→mə}	& \itshape mgban	& \itshape mə̀mgbǎn 	& ‘armpit’\\
\tablevspace	
\itshape mkpi→mkpə	& \itshape mkpìr	& \itshape mkpə̄mkpìr 	& ‘hip’\\
\tablevspace	
\itshape mla→mə 	& \itshape mlà 	& \itshape mə̀mlǎ 	& ‘first born’\\
\tablevspace	
\textit{mpa→mpə}	& \itshape mpā 	& \itshape mpə̄mpà 	& ‘sore, wound’\\
\tablevspace	
\itshape nci→nci	& \itshape nci 	& \itshape ncīnci 	& ‘traditional district’\\
\tablevspace	
\itshape nji→nji	& \itshape njī 	& {\itshape njīnji} 	&‘knife’ \\
\tablevspace	
\textit{njo→nju}	& \itshape njò 	& \itshape njūnjo 	& ‘horn’\\
\tablevspace	
\textit{nkɔ→nkɔ}	& \itshape nkɔ̀n 	& \itshape nkɔ̄n, nkɔ̄nkɔ̄n	& ‘road, way, door’\\
\tablevspace	
\textit{ri→ri} 	& \itshape rì 	& \itshape rīrī 	& ‘day’\\
\tablevspace	
\textit{te→tə} 	& \itshape tè 	& \itshape te, tə̄te 	& ‘father’\\
\tablevspace	
\itshape tse→tsɛ	& \itshape tse 	& \itshape tsə̄tse 	& ‘town’\\
\lspbottomrule
\end{tabularx}
\end{table}
  

To give a sense of the variety of number marking strategies in \ili{Mada}, \tabref{tab:nomaffplat:13} above shows the operation of first syllable reduplication in \ili{Mada} nouns, and selected examples in \tabref{extab:nomaffplat:41} and \tabref{extab:nomaffplat:42} below display recently adopted plural strategies.

\begin{table} 
\caption{Mada \textit{mə̀-} prefixes where stem tone is conserved}
\label{extab:nomaffplat:41}
\begin{tabularx}{\textwidth}{XXX}
\lsptoprule
  {Sg.}  	&       {Pl.} 	&         Gloss\\
  \midrule
\itshape bān 	& \itshape mə̀bān 	& ‘law’\\
\itshape gɔ̄n 	& \itshape mə̀gɔ̄n 	& ‘back’\\
\itshape gā 	& \itshape mə̀gā 	& ‘shoulder’\\
\itshape gbrīn 	& \itshape mə̀gbrīn, gbə̄gbrìn 	& ‘spirit’\\
\itshape jūjū 	& \itshape mə̀jūjū & ‘hole’\\
\itshape kpə̄ 	& \itshape mə̀kpə̄ 	& ‘female agama lizard’\\
 \lspbottomrule
\end{tabularx}
\end{table}

\begin{table} 
\caption{Mada \textit{mə̄-} prefixes where stem tone is conserved}
\label{extab:nomaffplat:42}
\begin{tabularx}{\textwidth}{XXX}
\lsptoprule
 Sg.     	&    Pl.  	&       Gloss\\
\midrule
\itshape brɛ 	& \itshape mə̄brɛ 	& ‘grave’\\
\itshape lənggə 	& \itshape mə̄lənggə 	& ‘enemy’\\
\itshape mla 	& \itshape mə̄mla 	& ‘relation’\\
\itshape n\=e 	& \itshape mə̄n\=e   & ‘person’\\
\itshape və̄nggə̄ 	& \itshape mə̄nggə̄  	& ‘girl’\\
\lspbottomrule
\end{tabularx}
\end{table}


The most recent addition to the \ili{Mada} repertoire of plural strategies is probably the \textit{mə-} prefix. This appears to have two realisations, \textit{mə̀-} and \textit{mə̄-}. The low-tone form seems to have no strongly-defined semantic field in \tabref{extab:nomaffplat:41}, but mid-tone \textit{mə̄-} is applied quite strictly to persons in \tabref{extab:nomaffplat:42}. The examples in these tables and in other sections show the prefix has been added, sometimes subsequently to other strategies, such as tone-raising or reduplication, providing evidence for its recent genesis. Most nouns taking a \textit{mə̀-} prefix conserve stem-tone in \tabref{extab:nomaffplat:41}.


\ili{Mada} provides a striking example of how rapidly a nominal affix system can break down and then be rebuilt using processes of grammaticalisation, thereby illustrating the difficulties of tracing synchronic affixes back to a presumed proto-system.

\subsection{Alumic} \label{sec:nomaffplat:2.6}

One subgroup of Plateau languages spoken in Central Nigeria has effectively no published data. These languages are \ili{Hasha} [=\ili{Yashi}], \ili{Sambe}, \ili{Alumu-Təsu} and \ili{Toro} [=\ili{Turkwam}]. Except for \ili{Sambe}, they have apparently been classified in previous lists on the basis of geographical proximity. \ili{Sambe} is moribund, as there were only two speakers over 90 in 2005, and none remain in 2017. The rest have at most a few hundred speakers. All data and analyses given here were the result of fieldwork by the author \citet{Blenchn.d.f}.

The group is here named \ili{Alumic}, after the language with the most speakers, but this term can be regarded as provisional. The \ili{Alumic} languages are now scattered geographically, and isolated among the \ili{Ninzic} (=Plateau IV) languages. The very different sociolinguistic histories may explain their striking morphological diversity. The internal structure of the \ili{Alumic} group is shown in \figref{fig:nomaffplat:5}.

\begin{figure} 
% \includegraphics[width=\textwidth]{figures/nomaffplat-img5.png}
 \caption{\label{fig:nomaffplat:5} The relation of Sambe to Hasha and the Alumic languages}
\begin{forest}  for tree={delay={where content={}{shape=coordinate}{}}},   forked edges 
[ 
    [
      [\ili{Sambe}]
      [\ili{Hasha}]
    ]
    [
      [\ili{Alumu-əsu}]
      [\ili{Toro}]
    ]	
]  
\end{forest}  
\end{figure}
 
\ili{Alumu}, \ili{Toro} and \ili{Sambe} no longer have functioning noun class systems, but the nouns have transparent fossil prefixes. \ili{Hasha} has developed a highly idiosyncratic system of reduplicating the first syllable of the stem to mark plurality in both nouns and verbs, apparently under the influence of a neighbouring \ili{Chadic} language, \ili{Sha}. \ili{Təsu} has entirely converted to a system of a single plural suffix, with no functioning noun-prefixes. Nonetheless, these can be recovered in part from the existing nouns, especially by comparison with cognate forms in other \ili{Plateau} languages. Although many nouns have zero prefixes, fossil V- and N- prefixes are quite widespread. The most common prefix is \textit{à-} and \textit{ə̀-} is probably its allomorph. \tabref{extab:nomaffplat:43} shows some characteristic examples.

\begin{table} 
\caption{Təsu \textit{à-/ə-} prefixes}
\label{extab:nomaffplat:43}
\begin{tabularx}{.8\textwidth}{lQl lll}
 \lsptoprule
\itshape a- 	& Gloss 	&    {\ilit{Təsu}} 	& \textit{ə-} 	& Gloss 	&    {\ilit{Təsu}}\\
\midrule
	& ‘tree (generic)’  	& \textit{à-gbè} 	&  	& ‘song’ 	&  \itshape ə̀-humu  \\
% \tablevspace	
	& ‘mushroom’ 	&  \itshape à-wá 	&  	& ‘leaf’ 	&  \itshape ə̀-ʃu\\
% \tablevspace	
	& ‘thorn’ 	&  \itshape à-tɔ̀tɔ̀ 	&  	& ‘road’ 	&  \itshape ə̀-ki\\
% \tablevspace	
	& ‘sand’ 	&  \textit{à-seŋge} 	&  	&  	& \\
% \tablevspace	
	& ‘farm’ 	&  \itshape à-yi 	&  	&  	& \\
 \lspbottomrule
\end{tabularx}
\end{table}

  
  
Nouns for persons typically have an \textit{à}- prefix as in \tabref{extab:nomaffplat:44}.

\begin{table}
\caption{Təsu \textit{à-} prefix for persons}
    \label{extab:nomaffplat:44} 
\begin{tabularx}{.8\textwidth}{Xl}
 \lsptoprule
Gloss 	& Sg. \\
\midrule
‘man, husband’ 	&  \itshape à-tsìɥà\\
‘child’        	& \itshape à-ɥà\\
‘woman, wife’ 	&  \itshape à-meré\\
‘father’ 	&  \itshape à-da\\
\lspbottomrule
\end{tabularx}
\end{table}


Other fossil prefixes are given in \tabref{extab:nomaffplat:45}.

\begin{table}
\caption{Fossil prefixes in Təsu}
\label{extab:nomaffplat:45} 
\begin{tabularx}{.8\textwidth}{lQQlQl}
\lsptoprule
\multicolumn{6}{l}{\ilit{Təsu} \textit{e-} \& \textit{i-}}\\
\itshape e- 	& Gloss 	&   {\ili{Təsu}} 	& \itshape    i- 	& Gloss 	&  {\ili{Təsu}}\\
\midrule 
		  & ‘cloud’ 	&  \itshape è-vírí 	&  	& ‘algae’ 	&  \itshape ì-bu\\
		  & ‘mouth’ 	&  \itshape è-né 	&  	& ‘tomorrow’ 	&  \itshape í-kyá\\
		  & ‘grasshopper’&  \itshape é-sɔ 	&  	& ‘large stone’ 	&  \itshape ì-tre\\
		  &  ‘spear’ 	&  \itshape é-mbè 	&  	& ‘cloud’ 	&  \itshape ì-ve 	  \\
 
\tablevspace
\multicolumn{6}{l}{\ili{Təsu} \textit{\`{N}-} \& \textit{u-}}\\   
\textit{\`{N}-} 	& Gloss 	&  {\ilit{Təsu}} 	& \itshape u- 	& Gloss 	&  {\ili{Təsu}}\\
\midrule
  & ‘smoke’ 	& \textit{ǹ-zu} 	&  	& ‘bush-fowl’ 	& \itshape úgrɔ́\\
  & ‘evening’ 	& \textit{ŋ-viʃi} 	&  	&  	& \\
  & ‘work’ 	& \textit{ǹ-dɔmɔ} 	&  	&  	& \\
  & ‘land/country’ 	& \textit{ǹ-zimbɔrɔ} 	&  	&  	& \\
  & ‘navel’ 	& \textit{ŋ-bu} 	&  	&  	& \\
  \lspbottomrule
\end{tabularx}

\end{table}


There is no trace of a semantic association for other prefixes. Liquids and non-count nouns show no characteristic morphological pattern.

\ili{Sambe} no longer has a functioning noun-class system, perhaps a consequence of the switch to \ili{Ninzo}. However, it clearly existed until recently and many words were cited with fossil prefixes. Indeed, sometimes a word would be cited in one elicitation with the prefix and again without it, showing the language in transition prior to its inevitable death. The tones marked are best characterised as approximate, with speakers varying between elicitation sessions. Three prefixes can be discerned in the data, \textit{kV}-, \textit{bV}- and \textit{tV}-, each with an underspecified or ‘hollow’ vowel. In some languages this shows concord with the stem vowel, but this does not seem to have been the case with Sambe. 

The most common prefix is \textit{k\`{V}-} in \tabref{tab:nomaffplat:14}.

\begin{table}
\caption{\label{tab:nomaffplat:14}  {\textit{k}\`{\textit{V}}-} fossil prefixes in Sambe}
\begin{tabularx}{\textwidth}{XXX}
\lsptoprule
  Prefix 	& Gloss 	&   Attestation\\
  \midrule
\itshape ka- 	&  ‘basket’ 	&  \itshape kàjese\\
\tablevspace
\itshape ke- 	&  ‘jar for local ‘beer’ 	&  \itshape kèɥa\\
	&  ‘head’ 	&  \itshape kècu\\
\tablevspace
\itshape ki- 	&  ‘spear’ 	&  \itshape kìnkwar\\
	&  ‘divination (types)’ 	&  \itshape kìtsu\\
\tablevspace
\itshape ku- 	&  ‘winnowing tray’ 	&  \itshape kùhûn\\
	&  ‘mortar (wood)’ 	&  \itshape kùtù\\
	& ‘skink’ 	&  \itshape kùva 	 \\
	&  ‘faeces’ 	&  \itshape kùbwà\\
\lspbottomrule
\end{tabularx} 
\end{table}


  
\tabref{tab:nomaffplat:15} shows words with a \textit{bV}- fossil prefix.

\begin{table}
\caption{\label{tab:nomaffplat:15} \textit{bV}-  fossil prefixes in Sambe}
\begin{tabularx}{\textwidth}{XXX}
\lsptoprule
  Prefix 	& Gloss 	&   Attestation\\
  \midrule
\itshape ba- 	& ‘sorghum-beer’ 	&  \itshape bàʃù\\
	& ‘ant (generic)’ 	&  \itshape bàtúnú\\
	& ‘ancestors’ 	&  \itshape bàgúgó\\
\tablevspace
{\itshape be-} & ‘fat/grease’ 	&  \itshape bènkun \\
\tablevspace
\itshape bi- 	& ‘small hoe’ 	&  \itshape bíkíta\\
\tablevspace
\itshape bu- 	& ‘today’ 	&  \itshape búrùmi\\
	& ‘salt’ 	&  \itshape bùwan\\
	& ‘rib’ 	&  \itshape bùkyɛ́\\
\lspbottomrule
\end{tabularx}
\end{table}

\newpage 
\tabref{tab:nomaffplat:16} shows words with a \textit{tV}- fossil prefix.

\begin{table}
\caption{\label{tab:nomaffplat:16} \textit{tV}-  prefixes in Sambe}
\begin{tabularx}{\textwidth}{XXX}
\lsptoprule
  Prefix 	& Gloss 	&   Attestation\\
\midrule
\itshape ta- 	& ‘name’ 	&  \itshape tánásè\\
	     	& ‘breath’ 	&  \itshape tawùrì \\
\tablevspace
\itshape ti- 	& ‘guinea-fowl’ 	&  \itshape tìmìsì\\
	      	& ‘word’ 	&  \itshape tìmǐvə̀n\\
	      	& ‘squirrel (tree)’ 	&  \itshape títɔ\\
\tablevspace
\itshape to- 	& ‘hippo’ 	&  \itshape tòbárì\\
\lspbottomrule
\end{tabularx}
\end{table}


\ili{Sambe} probably also had a nasal prefix which was homorganic with the following consonant, see \tabref{extab:nomaffplat:46}.

\begin{table}
\caption{\textit{N-} prefixes in Sambe}
    \label{extab:nomaffplat:46} 

\begin{tabularx}{\textwidth}{XX}
\lsptoprule
Gloss 	&    \ilit{Sambe}\\
\midrule
‘hair’ 	&  \itshape mfu\\
‘brother/sister’ 	&  \itshape mlànà\\
‘Senegal coucal’ 	&  \itshape mpàlàn\\
‘leaf’ 	&  \itshape ŋgbá ʃì\\
‘cock’ 	&  \itshape ŋgwà\\
‘vervet monkey’ 	&  \itshape njînjɛ̀hun\\
‘sheep’ 	&  \itshape ntùmà\\
\lspbottomrule
\end{tabularx}
\end{table}


Many nouns referring to persons have an \textit{a}- prefix and some which are naturally plural, such as ‘ancestors’ have a \textit{ba}- prefix, see \tabref{extab:nomaffplat:47}.

\begin{table}
\caption{\textit{a/ba-} prefixes in Sambe}
    \label{extab:nomaffplat:47} 
\begin{tabularx}{\textwidth}{Xl}
 \lsptoprule
Gloss 	&                        \ilit{Sambe}\\
\midrule
‘man’ 	&  \itshape àróro\\
‘woman’ 	&  \itshape àhìn\\
‘father’ 	&  \itshape adídá\\
‘mother’ 	&  \itshape aya\\
‘relations’ 	&  \itshape bàruhwin ninamláni\\
‘ancestors’ 	&  \itshape bàgúgó\\
\lspbottomrule
\end{tabularx}    
\end{table}

From this we can conclude that Sambe originally had an \textit{a-/ba}- noun class pair for humans. No other fossil prefixes have any semantic associations, and neither mass nouns nor liquids show any common features. The strong presence of CV- prefixes with underspecified vowels is extremely rare in this area, although common in \ili{Kainji} languages (Blench, Chapter \ref{sec:3} this volume).

\subsection{East}

The three languages constituting Greenberg’s Plateau 6, \ili{Fyem}, \ili{Bo-Rukul} [=\ili{Mabo-Barkul}] and \ili{Horom} were placed together as \ili{Southeastern Plateau} in the Benue-Congo Comparative Wordlist \citet{WilliamsonShimizu1968,Williamson1972}. Although named Southeastern (e.g. in \citealt{CrozierBlench1992}) it is here named '\ili{East Plateau}' as a better reflection of its direction in relation to the Plateau centre of gravity. However, it is highly uncertain that they do indeed form a coherent group as \ili{Bo-Rukul} is very distinct from \ili{Fyem} and \ili{Horom}. In \figref{fig:nomaffplat:1} they have been separated as branches of \ili{Plateau} with a tentative linkage marked. \citet{Nettle1998a} is a sketch grammar of \ili{Fyem}, and \citet{Nettle1998b} short wordlists of all three languages, but \ili{Bo-Rukul} and \ili{Horom} remain virtually unknown (although see \citealt{Blench2003} for their relation with the \ili{Ron} (\ili{Chadic}) languages). Since \ili{Horom} has the most elaborate system of nominal affixing, it is discussed in detail in this section. Data and analysis are based on fieldwork by the author.

Number marking in \ili{Horom} nouns is characterised by a great diversity of strategies. V-/CV- prefix alternation is the most characteristic process and the possibilities are numerous. Of these, the \textit{i-} plural prefix is applied in the majority of cases. The singular and plural class/pairings identified so far are shown in \tabref{tab:nomaffplat:17}.

\begin{table}
\caption{\label{tab:nomaffplat:17} Singular/plural affix pairings in Horom}
\begin{tabularx}{\textwidth}{XXl}
\lsptoprule
  Singular 	&   Plural 	&   Comment\\
  \midrule
\itshape ø- 	& \itshape à-, bà-, bɛ̀-, ɗì-, ì-, ù- 	& \\
\itshape a- 	& \itshape bà-, i- 	& \\
\itshape ɗi- 	& \itshape a-, bà- 	& \\
\itshape ɗu- 	& \itshape à-, bà-, be- 	& \\
\itshape ì- 	& \itshape bà- 	& \\
\itshape nà- 	& \itshape bɛ̀nɛ̀ 	&  A single example\\
\itshape ò- 	& \itshape bà- 	&  A single example\\
\itshape ù- 	& \itshape à-, bà-,  bɛ̀- 	& \\
\lspbottomrule
\end{tabularx}
\end{table}
 
\ili{Horom} also demonstrates some striking semantic unities with respect to plural markers. Singulars are diverse, but almost all animals, from mammals to insects, have \textit{i}- plural prefixes. Similarly, nouns referring to persons have a \textit{ba}- prefix (and sometimes a suffix) but with no corresponding singular prefix. Mass nouns and liquids have no defining morphological character. \ili{Horom} shows no evidence for nasal prefixes; in one apparent case the widespread Plateau root for ‘person’ has grammaticalised as an affix.

The most striking typological feature of \ili{Horom} is the evolution of a nominal suffixing system, characterised either by vowels or –NV structures. The singular nouns are diverse, with either zero or a wide array of prefixes. The plurals are all prefixed with \textit{ba}-, and a vocalic or –NV segment. \tabref{tab:nomaffplat:18} on the following page shows the nouns so far recorded with both prefixes and suffixes. 


\ili{Horom} also has ``broken plurals''. In words with stems of CVCCV(C) structure, an epenthetic vowel, either \textit{-i-} or \textit{-ə-}, is inserted between the two syllables of the stem as in \tabref{extab:nomaffplat:48}.


\begin{table}
\caption{Horom ‘broken’ plurals}
\label{extab:nomaffplat:48}
\begin{tabularx}{\textwidth}{lXl}
\lsptoprule
Gloss 	& {Sg.} 	& {Pl.}\\
\midrule
‘okra’ 	&  \itshape zabla 	&  \itshape i-zab-i-la\\
‘shoe’ 	&  \itshape paksak 	&  \itshape i-pak-ə-sak\\
‘sweet potato’ 	&  \itshape damʃik 	&  \itshape i-dam-ə-ʃik\\
‘gourd-bottle (\textit{L. siceraria})’ 	&  \itshape yóktál 	&  \itshape í-yók-tí-tál\\
 \lspbottomrule
\end{tabularx}
\end{table}


These may be infixes or simply a phonological extension of the syllable. None of these words are transparent compounds, but this may be their historical origin, in which case each element of the compound would have retained its plural prefix, with the second prefix undergoing centralisation in some environments.


\begin{table}
\caption{\label{tab:nomaffplat:18}{Horom nominal suffixes}
} 
\begin{tabularx}{\textwidth}{lQQl}
\lsptoprule
  Suffix 	& Gloss 	& {Sg.} 	& {Pl.}\\
\midrule
\itshape a 	& ‘river’ 	& \itshape u-lap 	& \itshape ba-lab-a\\
\itshape a 	& ‘bush’ 	& \itshape ù-háp 	& \itshape bà-háb-à\\
\itshape á 	& ‘bundle’ 	& \itshape ɗí-bwát 	& \itshape bá-bwád-á\\
\itshape e 	& ‘moon/month’ 	& \itshape u-fel 	& \itshape ba-pel-e\\
\itshape è 	& ‘song’ 	& \itshape u-sem 	& \itshape ba-sem-e\\
\itshape è 	& ‘sore / wound’ 	& \itshape u-cel 	& \itshape ba-cel-e\\
\itshape ɛ̀		& ‘compound’ 	& \itshape kyɛ̀n  	& \itshape bà-kyɛ̀n-ɛ̀ \\
\itshape ɛ̀		& ‘door’ 	& \itshape kɛ̀n kubok 	& \textit{bà-kɛ̀n-ɛ̀ kubok}\\
\itshape i 	& ‘root’ 	& \itshape u-liŋ 	& \itshape bè-liŋ-i\\
\itshape i 	& ‘fireplace’ 	& \itshape a-fik 	& \itshape ba-fik-i\\
\itshape ì 	& ‘mat (cornstalk)’ 	& \textit{ú-jír} 	& \itshape bá-jír-ì \\
\itshape ì 	& ‘canoe’ 	& \itshape u-bit 	& \itshape ba-bit-i\\
\itshape ye 	& ‘needle (thatching)’ 	& \itshape bwi 	& \itshape ba-bwi-ye\\
\itshape ɔ̀ 	& ‘skin’ 	& \itshape hɔ̀r 	& \itshape bà-hɔ̀r-ɔ̀\\
\itshape ɔ̀ 	& ‘rope’ 	& \itshape ù-zɔ̀r 	& \itshape bà-zɔ̀r-ɔ̀\\
\itshape ɔ̀ 	& ‘sorghum’ 	& \itshape pɔ̀l 	& \itshape bà-pɔ̀l-ɔ̀\\
\itshape u 	& ‘knife’ 	& \itshape mbok 	& \itshape ba-mbuk-u\\
\itshape nɛ̀ 	& ‘mother’ 	& \itshape wɔ̀ 	& \itshape bà-wɔ̀-nɛ̀\\
\itshape nɛ̀ 	& ‘father, grandfather’ 	& \itshape tɛ̀ 	& \itshape bà-tɛ̀-nɛ̀\\
\itshape mɔ̀ 	& ‘friend’ 	& \itshape ɗìsì 	& \itshape bà-ɗìsì-mɔ̀\\
\lspbottomrule
\end{tabularx} 
\end{table}

 
\subsection{South}  %2.8 /
\subsubsection{General} %2.8.1 /

\ili{South Plateau} is named for two language groups, \ili{Jilic} and \ili{Eggonic}, which are here put together. ``Southern'' was applied to \ili{Jilic} alone in \citet{CrozierBlench1992}. \figref{fig:nomaffplat:6} shows this new proposal.

 \begin{figure} 
% \includegraphics[width=\textwidth]{figures/nomaffplat-img6.png} 
\begin{forest}  for tree={delay={where content={}{shape=coordinate}{}}},   forked edges 
[\ili{South Plateau} 
    [\ili{Jilic}
      [
	[\ili{Ujijili}]
	[\ili{Koro-Ija}]
	[\ili{Koro-Zuba}]
      ]   
      [[\ili{Mijili}]]
    ]
    [\ili{Eggonic}
      [[\ili{Ake}]]
      [[\ili{Eggon}]]
    ]
]  
\end{forest}  
\caption{Classification of the Jilic-Eggonic languages}
\label{fig:nomaffplat:6}
\end{figure} 

The \ili{Jilic} or \ili{Koro} languages are spoken in scattered communities across a wide swathe of Central Nigeria and this is usually attributed to persistent slave-raiding in the nineteenth centuries. As speakers have lost contact with one another, their languages have rapidly diversified.

\subsubsection{Jilic} %2.8.2 /

\ili{Jilic} consists of at least two languages, \ili{Mijili} [=\ili{Koro of Lafia}] and \ili{Ujijili} [=\ili{Koro} \ili{Huntu}], now separated by a considerable geographic distance, but clearly related \citet{Blenchn.d.g}. There is a microfiched grammar of \ili{Mijili} by \citet{Stofberg1978}, while \ili{Ujijili} is known from an unpublished wordlist. \ili{Koro Ija} and \ili{Koro Zuba}, two languages spoken northwest of Abuja, are said to be nearly intelligible with \ili{Ujijili}, although no language data exists to demonstrate this. This section will focus on \ili{Mijili} as described by \citet{Stofberg1978}, but with additional material from fieldwork in 2003. \ili{Mijili} has a system of number marking on nouns based on prefix alternations. \tabref{tab:nomaffplat:19} is a matrix showing the possible pairings of singular and plural prefixes.

\begin{table}
\caption{\label{tab:nomaffplat:19} Matrix showing matching of singular and plural prefixes in Mijili. 
Adapted from \citet[316]{Stofberg1978}}
\begin{tabularx}{.8\textwidth}{l l| llllllllll} 
	&\multicolumn{11}{p{6cm}}{\hfill Plural prefixes}\\
\multirow{13}{*}{\rotatebox[origin=c]{90}{Singular prefixes~~~~~~~~~~~~}}
	&	& \itshape     á- 	&  \itshape     à- 	&  \itshape àmà- 	&  \itshape       í- 	&  \itshape    mí- 	& \itshape    mì- 	&  \itshape  mú- 	&  \itshape  mù- 	&   \itshape     Ń-\\
\cline{2-11}
	&\itshape cù- 	&  	&  	&  	&  	&  	&  	& +	&  	& {}\\
	&\itshape jì- 	&  	&  	&  	&  	&  	& +	&  	&  	& {}\\
	&\itshape kí- 	& +	&  	&  	& +	&  	&  	&  	&  	& {}\\
	&\itshape kú- 	& + &  	&  	&  	&  	&  	&  	&  	&  \\
	&\itshape lú- 	&  	&  	&  	& +	&  	&  	&  	&  	& {+}\\
	&\itshape mí- 	&  	&  	&  	& +	&  	&  	&  	&  	& {}\\
	&\itshape mú- 	&  	&  	&  	& +	&  	&  	&  	&  	& {}\\
	&\itshape Ń- 	&  	&  	&  	& +	&  	&  	&  	&  	& {}\\
	&\itshape \`{N}- 	&  	&  	& +	&  	& +	&  	& +	&  	& {}\\
	&\textit{ò-/ɔ-} 	& +	&  	&  	& +	&  	&  	& +	& +	& {}\\
	&\itshape rí- 	& +	&  	&  	&  	&  	&  	&  	&  	& { +}\\
	&\itshape rú- 	&  	&  	&  	&  	&  	&  	&  	&  	& { +}\\
	&\itshape ø- 	& +	& +	&  	& +	& +	&  	& +	& +	& {}\\ 
\end{tabularx}
\end{table}


Once allomorphy of the prefixes is taken into account, the number of underlying prefixes is considerably reduced. As elsewhere in Plateau, singular nouns referring to human beings have variable morphology. Many nouns for persons have a former \textit{ɲV}- prefix, now apparently lexicalised, but still in alternation in one root, the word for `young man’ in \tabref{extab:nomaffplat:49}.  Plural prefixes in \ili{Mijili} nouns for persons are either \textit{mV}- or \textit{a}-.


\begin{table}
\caption{Singular and plural prefixes for person nouns in Mijili} 
\label{extab:nomaffplat:49}
\begin{tabularx}{\textwidth}{XXl}
\lsptoprule
Gloss 	& {Sg.} 	& {Pl.}\\
\midrule
‘old person’ 	& \itshape nyɛkúk\'{ɔ̃}  	& \itshape mínyɛkúk\'{ɔ̃} \\
‘in-laws’ 	& \itshape nyɛ́lɔ́ 	& \itshape mínyɛ́lɔ́\\
‘doctor’ 	& \itshape nyɛmũgá 	& \itshape minyɛmũgá\\
‘man’ 	& \itshape nyɛvɛlɛ̀ 	& \itshape mínyɛvɛlɛ\\
‘guest/stranger’	& \itshape nyɛ̀z\`{ɔ̃}  	& \itshape minyɛ̀z\`{ɔ̃} \\
‘young man’ 	& \itshape nyɛ́zhò 	& \itshape ázhò\\
‘woman’ 	& \itshape nyinyr\`{ã} 	& \itshape mínyinyr\`{ã}\\
‘uncle’ 	& \itshape òcã 	& \itshape múcã\\
‘male ancestor’ 	& \itshape òco 	& \itshape múco\\
	  \\
but: 	 \\
‘thief’ 	& \itshape oyi 	& \itshape áyi\\
\lspbottomrule
\end{tabularx} 
\end{table}

The \textit{ɲV}- prefix in singulars is unlikely to be a ``true'' prefix but a recent grammaticalisation of the nouns for ‘person’ \REF{ex:nomaffplat:50}:

\ea
\label{ex:nomaffplat:50}
\textit{ǹnyɛ}  ‘person’ ~~~~   \textit{mínyɛ}     ‘person/people’
\z


Almost all liquids and non-count nouns have an unpaired \textit{ń}- prefix as in \tabref{extab:nomaffplat:51}.

\begin{table}
\caption{Mass nouns with \textit{n-} prefixes in Jili} 
\label{extab:nomaffplat:51}
\begin{tabularx}{\textwidth}{XXXl}
\lsptoprule 
  \ilit{Jili} 	& Gloss 	&   \ilit{Jili} 	& Gloss\\
\midrule  
\itshape ńc\`{ɛ̃}	& ‘saliva’ 	& \itshape ńs\'{ã} 	& ‘salt’\\
\itshape ńjɛ̃ 	& ‘fat/grease’ 	& \itshape ńsí 	& ‘tear’\\
\itshape ńkwálɛ̀ 	& ‘water’ 	& \itshape ńswàná 	& ‘hair’\\
\itshape ńnoro 	& ‘mud’ 	& \itshape ńzɛ̃ 	& ‘blood’\\
\itshape ńnɔ 	& ‘oil’ 	& \itshape ńzɔ̃ 	& ‘smoke’\\
\lspbottomrule
\end{tabularx} 
\end{table} 


No other semantic correlations with noun class pairs have been detected.

\subsubsection{Eggonic} %2.8.3 /

\ili{Eggonic} consists of just two languages, \ili{Eggon} and \ili{Ake}, spoken around Akwanga. These have previously been put together with \ili{Ninzic}, although this is more a supposition based on geography than historical linguistics. The \ili{Eggon} people are numerous and their language is divided into numerous dialects, while \ili{Ake} (=Aike) is spoken in only three villages. Although the languages share enough common glosses to be put together, they are still quite distant from one another. \ili{Eggon} has a limited system of nominal morphology, while \ili{Ake} has lost its system entirely. All data and analyses in this section are based on fieldwork by the author.

\ili{Ake} nouns no longer have morphologically marked plurals, with a few exceptions in the case of persons. However, there is considerable evidence for prior systems of CV prefixes, many of which survive in frozen form preceding the stem. The key to detecting such affixes is external cognates. Many words appear with different prefixes in related languages. Thus, although \ili{Proto-Ake} almost certainly had a \textit{ki-} prefix, in the word \textit{kipindye} ‘village/settlement’ the \textit{ki-} is not a prefix, since it is cognate with forms in remote \ili{Plateau} languages such as \ili{Hyam} \textit{khep}, \ili{Jili} \textit{kúp\`{ɔ̃} }, and the –\textit{ndye} element would then be a compounded element. Such evidence is not available for all the terms with potential affixes, so only more elaborated morphological comparisons will increase certainty. The former V- prefixes often have two distinct tones and may therefore be ultimately of different origins or it may be that this is the result of a now-lost morphophonemic process. However, since they exist in high-low pairs for almost all the hypothetical prefixes reconstructed in \tabref{tab:nomaffplat:20}.

\begin{table}
\caption{\label{tab:nomaffplat:20} Ake fossil noun prefixes} 
\begin{tabularx}{\textwidth}{XX}
\lsptoprule
  Prefix 	&   Allomorphs\\
\midrule
\itshape a- 	& \itshape à-,á-\\
\itshape i- 	& \itshape ì-,í-\\
\itshape kV- 	& \itshape kà-, kè-, kì-, kí-, kù-, kú-\\
\itshape mu- 	& \itshape mù-, mú-\\
\itshape O- 	& \itshape ɔ̀-, ɔ́-, ò-, ó-\\
\itshape rV-	& \itshape rì-,rí-, rù-,rú-\\
\itshape u- 	& \itshape ù-, ú-\\
\lspbottomrule
\end{tabularx}
\end{table}

\largerpage[-1]
\ili{Ake} has a variety of \textit{kV}- prefixes which constitute possible evidence for an original affix with an underspecified vowel, such as occur both in Sambe (\sectref{sec:nomaffplat:2.6}) and \ili{East Kainji} languages such as Boze (see \citealtv{Blench2018kainjitv}, \sectref{sec:kainji:3.5.2}). Examples are given in \tabref{extab:nomaffplat:52}.

\begin{table}[t]
\caption{Ake \textit{kV-} prefixes}
\label{extab:nomaffplat:52}
\begin{tabularx}{\textwidth}{XX}
 \lsptoprule
Gloss 	&    {\ilit{Ake}}\\
\midrule
‘world’ 	&  \itshape kàyùnzà\\
‘ground’ 	&  \itshape kàʃe\\
‘masquerade’ 	&  \itshape kàŋgìrì\\
‘grave’ 	&  \itshape kèmì\\
 \lspbottomrule
\end{tabularx}
\end{table}


An intriguing feature of \ili{Ake} prefixes, not apparently found in related or nearby \ili{Plateau} languages, is semantic clustering around specific segments. Some examples are found in \REF{ex:nomaffplat:53}:

% \todo{check how to typeset this}
\ea\label{ex:nomaffplat:53}
 {\ili{Ake} prefixes semantic clustering around specific segments}\\

  \ea \textit{ɔ̀-/ɔ́-}\\
  This prefix is strongly associated with body parts:\\
  \begin{tabularx}{\textwidth}{lX}
  % Gloss 	&    {Ake}\\ 
  ‘mouth’ 	&  \itshape ɔ̀mu  \\
  ‘tongue’ 	&  \itshape ɔ̀lɛ́  \\
  ‘neck’ 	&  \itshape ɔ̀lwa  \\
  ‘shoulder’ 	&  \itshape ɔ́kyɛ  \\
  ‘armpit’ 	&  \itshape ɔ́ŋgwɔ  \\  
  \end{tabularx}

\newpage   
  \ex \textit{ò-/ó-}\\
  This prefix is strongly associated with animals:\\
  \begin{tabularx}{\textwidth}{lX}
  %  Gloss 	&   \ili{Ake}\\
  ‘calf’ 	& òyèna\\
  ‘castrated small ruminant’ 	&  \textit{òkì} \\
  ‘colobus monkey’ 	& \textit{òkpesɛ̃̀ }\\
  ‘hare’ 	&  \itshape òzwè\\
  ‘electric fish’ 	&  \itshape òrǐ\\
  ‘fish sp.’ 	&  \itshape ópò\\
  \end{tabularx}
  
  \ex \textit{\`{ŋ}-} 
  \z
\z

Strikingly, and in contrast to most other \ili{Plateau} languages, the velar nasal prefix is not homorganic synchronically. Almost all the words with \textit{ŋ-} prefixes are in the same semantic area, reptiles, crustaceans and insects. See \REF{ex:nomaffplat:54}:

\ea%54
    \label{ex:nomaffplat:54}  
         \ili{Ake} \textit{ŋ-} prefix 
\begin{tabularx}{\textwidth}{XX}
% Gloss 	&  {   Ake}\\
‘hammer’ 	&  \itshape \`{ŋ}bùkù 	  \\
‘fish sp.’ 	&  \itshape \`{ŋ}gásə́ré 	  \\
‘river turtle’ 	&  \itshape \`{ŋ}gyáklà 	  \\
‘skink’ 	&  \itshape \`{ŋ̄}bɔ́klɔ́ 	 \\
‘toad’ 	&  \itshape \`{ŋ}báwù 	  \\
 
\end{tabularx}
    \z
\ili{Ake} has almost certainly reprefixed stems with former velar nasal prefixes in words such as those in \tabref{extab:nomaffplat:55}.

\begin{table}
\caption{Reprefixed stems in Ake}
\label{extab:nomaffplat:55}
\begin{tabularx}{.75\textwidth}{Xl}
\lsptoprule 
Gloss 	& \ilit{Ake}\il{Ake}\\
\midrule
‘chameleon’ 	&  \textit{íŋbrǔ} \\
‘bee’ 	&  \itshape ìŋwè\\
‘giant snail’ 	&  \itshape ìŋgìrà\\
 \lspbottomrule
\end{tabularx}
\end{table}
 
It is conceivable this is related \ili{Bantu} Class 9, \textit{nì}-, for animals, although large salient species in \ili{Ake} do not have an \textit{ŋ-} prefix.

There is weak evidence for an \textit{mV-} prefix defining liquids in \tabref{extab:nomaffplat:56}.

\begin{table}
\caption{Ake \textit{mV-} prefix defining liquids}
\label{extab:nomaffplat:56}
\begin{tabularx}{.66\textwidth}{Xl}
\lsptoprule
Gloss 	&     \ilit{Ake}\\
\midrule
‘blood’ 	&  \itshape mìʃe\\
‘tear(s)’ 	&  \itshape mínyi\\
‘urine’ 	&  \itshape màŋgbà\\
\lspbottomrule
\end{tabularx}
\end{table}

Nouns referring to persons do not have any morphologically unifying characteristics.

\newpage 
\subsection{Tarokoid} %2.9 /

The \ili{Tarokoid} languages consist of four distinct languages and the \ili{Kwang} cluster. \ili{Tarok} is numerically the most dominant, spoken in a large area around Langtang, while the others are spoken in small communities isolated from one another between Langtang and Jos. \ili{Yangkam} is moribund, spoken only by men over fifty years of age. \figref{fig:nomaffplat:7} shows the internal structure of \ili{Tarokoid}.

   
\begin{figure}
% \includegraphics[width=\textwidth]{figures/nomaffplat-img7.png}
\begin{forest}  for tree={delay={where content={}{shape=coordinate}{}}},   forked edges 
[\ili{Proto-Tarokoid}
  [
    [\ili{Kwang cluster}]
    [\ili{Sur}]
  ]
  [
    [\ili{Yangkam}]
    [
      [\ili{Tarok}]
      [\ili{Pe}]
    ]
  ]   
]  
\end{forest}  
\caption{Internal structure of Tarokoid}
\label{fig:nomaffplat:7}
\end{figure} 

Within \ili{Tarokoid} there is a considerable range of nominal morphology. \ili{Tarok} itself has both the most complete prefix system and alliterative concord. \ili{Yangkam} has lost functioning affix alternation but has partially developed a system of reduplicating the initial syllable of the stem. Sur has also lost any functioning affixes without the evolution of a compensatory process, perhaps under the influence of the \ili{Chadic} language \ili{Ngas}. 

\ili{Kwang} marks number with singular/plural prefix pairings, but these are extremely reduced compared with \ili{Tarok} or \ili{Pe} \citet{Blenchn.d.h}. With very few exceptions, all plurals are marked with an \textit{à}- prefix.  \ili{Kwang} has a small number of nouns where plurality is marked with a tone-change, Low/High or Mid/High, and some irregular plurals which may be examples of residual consonant mutation. Changes in the stem vowel occur in the plurals of some lexemes connected with persons. \tabref{tab:nomaffplat:21} shows the nominal prefixes of \ili{Kwang}, and by far the most common singular prefix is ø- and plural \textit{à-}.

\begin{table}
\caption{\label{tab:nomaffplat:21} Kwang nominal prefixes}
\begin{tabularx}{\textwidth}{XX}
\lsptoprule
  Singular 	&   Plural\\
\midrule
\itshape ø- 	& \itshape   à-\\
\itshape ì- 	& \itshape   kí-\\
\itshape kì- 	& \\
\itshape ǹ- 	& \\
\lspbottomrule
\end{tabularx}
\end{table}


However, the large number of palatalised and labialised stems in \ili{Kwang} suggests that \textit{i-} and \textit{u-} prefixes were formerly present. By far the most common singular/ plural alternation is \textit{ø-/à-}, as shown in \tabref{extab:nomaffplat:57}.

\begin{table}
\caption{Kwang \textit{ø-/à-} prefix pairing}
\label{extab:nomaffplat:57}
\begin{tabularx}{\textwidth}{XXX}
 \lsptoprule
Gloss 	& Sg.  	& Pl. \\
\midrule
 {‘}root’ 	& \itshape   liŋ 	& \itshape   àliŋ\\
‘bark’ 	& \itshape   púr 	& \itshape   àpur\\
‘thorn’ 	& \itshape   rèk 	& \itshape   àrèk\\
‘mountain’ 	& \itshape   ʃiʃ 	& \itshape   àʃiʃ\\
‘relations’ 	& \itshape   zyɛŋ 	& \itshape   àzyɛŋ\\
‘name’ 	& \itshape   sàk 	& \itshape   àsàk\\
 \lspbottomrule
\end{tabularx}
\end{table}


\ili{Kwang} seems to have deleted the prefixes on singular nouns very extensively, as most of these now show only the root with no fossil morpheme. Nouns referring to persons all take \textit{à}- plurals, but the singulars have no distinctive features. Some nouns copy the number marker at the end of the word, and assimilate the stem vowel if it is not the same –\textit{a} as in \tabref{extab:nomaffplat:58}.

\begin{table}
\caption{Kwang \textit{ø-/à-} prefix pairing with suffixed copy vowel}
\label{extab:nomaffplat:58}
\begin{tabularx}{.75\textwidth}{XXl} 
\lsptoprule
Gloss 	& Sg.  	& Pl. \\
\midrule
‘wall (of room)’ 	& \itshape lâr 	& \itshape àlárà\\
‘person/people’ 	& \itshape sùm 	& \itshape àsùmà\\
‘strength’ 	& \itshape sɔ̀tɔn 	& \itshape sɔ̀tɔna\\
‘woman’ 	& \itshape yì 	& \itshape àya\\
‘husband’ 	& \itshape dìmà lɔ̀g 	& \itshape àdàmà lɔ̀g\\
 \lspbottomrule
\end{tabularx} 
\end{table}


The next most common pairing is \textit{ǹ-/á-} as in \tabref{extab:nomaffplat:59}.

\begin{table}
\caption{Kwang \textit{ǹ-/á-}  prefix pairing} 
\label{extab:nomaffplat:59}
\begin{tabularx}{.75\textwidth}{XXl}
\lsptoprule
Gloss 	& Sg.  	& Pl. \\
\midrule
‘feather’ 	& \itshape ǹzùna 	& \itshape ázùna\\
‘gum/glue’ 	& \itshape ǹdúr 	& \itshape ádúr\\
‘forehead’ 	& \itshape ǹjan 	& \itshape ájan\\
‘catfish (spp.)’ 	& \itshape ǹdurum 	& \itshape ádurum\\
‘blood’ 	& \itshape ǹjì 	& \itshape —–\\
 \lspbottomrule
\end{tabularx}
\end{table}



\ili{Kwang} also has occasional stem-initial consonant mutation as in \tabref{extab:nomaffplat:60}.

\begin{table}
\caption{Kwang stem-initial consonant mutation}
\label{extab:nomaffplat:60}
\begin{tabularx}{.75\textwidth}{XXl}
\lsptoprule
Gloss 	& Sg.  	& Pl. \\
\midrule
‘child’ 	& \itshape fàn 	& \itshape àmàn\\
‘young girl’ 	& \itshape fàyì 	& \itshape àwàyì\\
 \lspbottomrule
\end{tabularx}
\end{table}



Despite the example of ‘blood’ there is no association between mass nouns or liquids and nasal prefixes.

The noun classes of \ili{Tarok} have been described in \citet{Sibomana1981} with additional material in \citet{Longtau2008}. \citet{Sibomana1981} sets up 6 singular and plural noun classes for \ili{Tarok} (\figref{figtab:nomaffplat:22}).

\begin{figure}
\caption{\label{figtab:nomaffplat:22} Tarok noun-class pairings}
\begin{tabular}{lllll}
1. 	& ù- 	& $\to$ 	& 2. 	& o- \\
3. 	& ì- 	& $\to$ 	& 4. 	&i- \\
5. 	& m̀/ǹ- 	& $\to$ 	& 6. 	& m/n- \\
7. 	& a- 	& $\to$ 	& 8. 	& agá - \\
9. 	& i- 	& $\to$ 	& 10. 	& igá - \\
11.	& m/n- 	& $\to$ 	& 12. 	& m/nggá - \\
\end{tabular}  
% \includegraphics[width=\textwidth]{figures/nomaffplat-img8.png}
\end{figure}
 

The order of the numbers seems less than ideal, but since it is reprised in \citet{Longtau2008} it is also used here. Historically, these pairings must result from the merger of a more complex system, as in many singular/plural pairs there are changes in the stem tone. Some of these seem to show semantic correlations, others do not, again suggesting class merger. \ili{Tarok} also has a rich inventory of adjectives with concordial prefixes. Nasal prefixes are homorganic with the following consonant, with \textit{m-} preceding bilabials and \textit{n-} all others. 

Persons in \ili{Tarok} are almost exclusively in Class 1/2, i.e. with a \textit{u-/o-} prefix alternation, as shown in \tabref{extab:nomaffplat:61}.

\begin{table}
\caption{Tarok \textit{u-/o-} prefix alternations} 
\label{extab:nomaffplat:61}
\begin{tabularx}{.75\textwidth}{XXl}
 
\lsptoprule
Gloss 	& Sg.  	& Pl. \\
\midrule
‘husband’ 	& \textit{ùɓar} 	& \textit{oɓar}\\
‘wife, woman’ 	& \itshape ùcár 	& \itshape ocár\\
‘ancestors’ 	& \itshape ùkà 	& \itshape okà\\
‘mother’ 	& \itshape ùnaŋ 	& \itshape onáŋ\\
‘man, person’ 	& \textit{ùnə̀m} 	& \textit{onəm}\\
‘father’ 	& \itshape ùpò(n) 	& \itshape opó\\
‘ancestor’ 	& \textit{ùrìm} 	& \itshape orìm \\
‘soldier’ 	& \itshape ùshózhà 	& \itshape oshózhà\\
‘child’ 	& \itshape ùyèn 	& \itshape ován\\
 \lspbottomrule
\end{tabularx}
\end{table}


There is a strong tendency for mass nouns, liquids and abstracts to have the unpaired homorganic \textit{N-} prefix as in \tabref{extab:nomaffplat:62}.

\begin{table}
\caption{Tarok \textit{N-} prefixes on mass nouns} 
\label{extab:nomaffplat:62}
\begin{tabularx}{.75\textwidth}{Xl}
 \lsptoprule
Gloss 	&  {\ilit{Tarok}}\\
\midrule
‘oil’ 	& \textit{m̀mì}\\
‘urine’ 	& \textit{m̀pə̀ng}\\
‘fat’ 	& \textit{m̀pì}\\
‘blood’ 	& \textit{ǹcìr}\\
‘water’ 	& \textit{ǹdəng}\\
‘smoke’ 	& \textit{\`{ŋ}gù}\\
 \lspbottomrule
\end{tabularx}
\end{table}
 
 \largerpage
No other semantic set, such as large animals, trees or body parts, shows a tendency to cluster around a particular prefix pairing.

\subsection{Eloyi} %2.10 /

The \ili{Eloyi} or \ili{Afo} language is spoken in about twenty villages in Nassarawa State, Nigeria. The principle sources on the language are \citet{Mackay1964} and \citet{Armstrong1964,Armstrong1983,Armstrong1984}.\footnote{Despite the title of the 1984 publication, this is about Eloyi.} The classification of \ili{Eloyi} has been disputed, all the more so because the lexical database for comparison has been so weak. All the preliminary sources classified \ili{Eloyi} as Plateau 2, i.e. together \ili{Izere}, \ili{Tyap} etc. (e.g. \citealt{Greenberg1963,WilliamsonShimizu1968,deWolf1971}). \citet{Armstrong1983} set out the case for classifying \ili{Eloyi} as \ili{Idomoid}, which is a \ili{West Benue-Congo} or \ili{Volta-Niger} subgroup, classified together with \ili{Yoruba}, \ili{Igbo}, \ili{Nupe} and \il{Edo@Ẹdo}Ẹdo. However, in Armstrong’s (\citeyear{Armstrong1984}: 29) final published discussion of the subject he expresses some doubts, concluding “Eloyi does not now seem as close to \ili{Idoma} as it did when only Varvil’s list was available”. \ili{Eloyi} has a rich system of alternating nominal prefixes, in contrast to the remainder of \ili{Idomoid}, and is provisionally treated here as \ili{Plateau}, though with significant influence from \ili{Idomoid}. The analysis here is based on the cited published sources, and an unpublished wordlist collected by Barau Kato at the request of the author \citet{Blenchn.d.i}.


\begin{table}[t]
\caption{\label{tab:nomaffplat:23} Nominal prefix pairings in Eloyi}
\begin{tabularx}{\textwidth}{XXXl}
\lsptoprule
  Underlying 	& sg. 	&   Underlying 	& Pl. \\
\midrule
\itshape a- 	& \itshape a- 	& \itshape mba- 	& \itshape mba-\\
\itshape ɛ- 	& \itshape ɛ- 	& \itshape mbɛ- 	& \itshape mbɛ-\\
\itshape O- 	& \itshape o-, ɔ- 	& \itshape a- 	& \itshape a-\\
\itshape O- 	& \itshape o-, ɔ- 	& \itshape E- 	& \itshape e-, ɛ-\\
\itshape O- 	& \itshape o- 	& \itshape i- 	& \itshape i-\\
\itshape u- 	& \itshape u- 	& \itshape a- 	& \itshape a-\\
\itshape u- 	& \itshape u- 	& \itshape i- 	& \itshape i-\\
\itshape kV- 	& \itshape kO- 	& \itshape a- 	& \itshape a-\\
\itshape kV- 	& \itshape ko- 	& \itshape e- 	& \itshape e-\\
\itshape kV- 	& \itshape ko- 	& \itshape lV- 	& \itshape lo-\\
\itshape kV- 	& \itshape kɔ- 	& \itshape O- 	& \itshape ɔ-\\
\itshape kV- 	& \itshape ku- 	& \itshape E- 	& \itshape e-, ɛ-\\
\itshape kV- 	& \itshape ka- 	& \itshape lV- 	& \itshape lɔ-\\
\itshape kV- 	& \itshape ki- 	& \itshape lV- 	& \itshape lu-\\
\itshape kV- 	& \itshape ke- 	& \itshape lV- 	& \itshape lo-\\
\itshape kV- 	& \itshape kɛ- 	& \itshape lV- 	& \itshape lu-\\
\itshape rE- 	& \itshape rɛ- 	& {\itshape a-} 	&\itshape a- \\
\itshape rE- 	& \itshape re- 	& \itshape e- 	& \itshape e-\\
\lspbottomrule
\end{tabularx}
\end{table}

Despite the complex affix-pairings, many words have zero prefixes, perhaps due to the impact of extensive bilingualism with \ili{Idomoid} languages. Many alternations have only one or two cases so far recorded, which makes setting up the system highly provisional. \tabref{tab:nomaffplat:23} shows the nominal prefix pairings in Eloyi.

\textit{kV}- prefixes are probably the most common in \ili{Eloyi} singulars and \textit{lV}- for plurals. Although usually some type of stem harmony would be expected to operate there is no evidence for this in Eloyi. 


\ili{Eloyi} noun-class pairings do not show much semantic clustering. Most nouns referring to persons have diverse singulars, and plurals in \textit{a}- or \textit{e}-. Mother and father have an exceptional class prefix pair which may reflect the \ili{Niger-Congo} persons class seen in \tabref{extab:nomaffplat:63}. There is no evidence for a distinctive morphology for non-count nouns and no trace of nasal prefixes.

\begin{table}
\caption{Eloyi prefixes \textit{V-/mba-}}
\label{extab:nomaffplat:63}
\begin{tabularx}{\textwidth}{XXl}
\lsptoprule 
Gloss 	& Sg.  	& Pl. \\
\midrule
‘father’ 	& \itshape á-da 	& \itshape mbá-da\\
‘mother’ 	& \itshape ɛ́nɛ́ 	& \itshape mb-ɛ́nɛ́\\
\lspbottomrule
\end{tabularx}
\end{table}
 
\section{Conclusion: Plateau nominal affixing} %3. /

The numerous examples illustrate the problems of making any generalisations about nominal affixes in \ili{Plateau} and only weak conclusions can be drawn about its relationship with other branches of \ili{Benue-Congo}. This represents a common problem of historical linguistics in such a significant contact zone. Traces of prefixes familiar from \ili{Bantu} and \ili{Niger-Congo} are found scattered across the family, and in the light of external data it might seem likely that these were present in \ili{Proto-Plateau}. However, on the basis of synchronic data in \ili{Plateau} alone it would be rash to reconstruct them. Taking the data as a whole we can conclude that: 

\largerpage
\begin{enumerate}
\item[a)] \ili{Plateau} languages originally had a rich noun class system with CV- and V- prefixes and alliterative concord
\item[b)] A wave of renewal and analogical re-alignment led to many of the CV- prefixes disappearing or becoming unproductive and replaced by a much smaller set of V- prefixes.
\item[c)] There is some evidence for underspecified vowels in CV- prefixes showing concord with stem vowels although this is too rare to be conclusive.
\item[d)] There is evidence for a class pair for persons, probably \textit{V-/bV-}, although the segment in the singular prefix are less certain (e.g. Tables \ref{extab:nomaffplat:33} and \ref{extab:nomaffplat:37}). This can be compared with the \ili{Niger-Congo} person class.
\item[e)] \ili{Proto-Plateau} almost certainly had an unpaired nasal class marking liquids, mass nouns and abstracts, corresponding to \ili{Niger-Congo} (e.g. examples 14, 34). Unlike \ili{Kainji}, this is rarely realised as \textit{ma}- and several branches of Plateau have \textit{nV}-. \ili{Ndun} in \tabref{extab:nomaffplat:34}  does display \textit{ma-, me-, and m-.} Other unpaired classes exhibit quite different segments which may be innovative. 
\item[f)] There is strong evidence that \ili{Proto-Plateau} had \textit{N}- prefixes, homorganic with the following consonant, and present in most branches. However, there is no evidence for any consistent semantic association.
\item[g)] There is weak evidence that the \ili{Bantu} Class 9 prefix, \textit{nì}-, existed in early \ili{Plateau} (cf. \tabref{extab:nomaffplat:38}).
\end{enumerate}

Based on the synchronic evidence from \ili{Plateau}, the connection with \ili{Niger-Congo} noun classes remains tenuous. Only the non-count nouns and the person class show similarities and even these are obscured by innovative affixes. Similarly, there is no single affix alternation that provides evidence for the genetic unity of Plateau. This can only be deduced from lexical isoglosses
(e.g. in \citealt{Blench2000wocal}).
The paper presents a summary of what is known about number marking strategies on nouns in the \ili{Plateau} languages. Further work will enrich the picture, but it is unlikely to contribute to a coherent reconstruction, as affix renewal has been very extensive. 
 
\section*{Acknowledgements} 
It would be impossible to list all those who have acted as informants, but 
Barau Kato, 
Daniel Gya and 
Selbut Longtau 
have been my principal co-workers on field data collection. 
Bitrus Kaze, 
Deme Dang, 
Ruth Adiwu,
Barnabas Dusu (†), 
Gideon Asuku, 
Alex Maikarfi and 
Wayo Bai 
have been crucial to the development of extended materials in their languages. Staff members at NBTT and SIL Jos have been always helpful in giving me access to unpublished materials and to discuss issues relating to particular languages. I would particularly like to thank Mark Gaddis for arranging workshops on the \ili{Koro} cluster languages.
\section*{Abbreviations and conventions}

\begin{tabularx}{.45\textwidth}{lX} 
A 	& any central vowel\\
C 	& consonant\\
E 	& any mid-front vowel\\
N 	& any nasal\\
\end{tabularx}
\begin{tabularx}{.45\textwidth}{lX} 
O 	& any mid-back vowel\\
S 	& s or ʃ\\
V 	& vowel\\ 
\\
\end{tabularx}
 
\sloppy
\printbibliography[heading=subbibliography,notkeyword=this] 
\end{document}