\documentclass[output=paper]{langsci/langscibook} 
\author{Roger M. Blench\affiliation{McDonald Institute for Archaeological Research, University of Cambridge}}
\title{Nominal affixing in the Kainji languages of northwestern and central Nigeria}

% \ChapterDOI{} %will be filled in at production

\epigram{Change epigram in chapters/03.tex or remove it there }
\abstract{The Kainji languages of northwest and central Nigeria remain little-researched and sparsely described. Their nominal morphology strongly resembles Bantu typologically, but finding segmental cognates remains problematic. They show systems of alternating prefixes and alliterative concord, as well as diminutive and augmentative prefixes and CV- prefixes with underspecified vowels, where the -V of the prefix harmonises with the stem vowel. The limited segmental cognates point to radical restructuring through affix loss and renewal. Indeed one language, Shen, has lost all nominal morphology and it is severely reduced in some branches. Reshe is typologically similar to other Kainji languages, but the affixes seem to have been completely restructured. The paper presents an overview of the literature on Kainji and then describes the nominal affixing in individual branches. It concludes by suggesting what hypotheses can be made about the Kainji system as a whole.
}
\maketitle

\begin{document}

% Keywords: Historical linguistics; \ili{Kainji} languages; \ili{Niger-Congo}; noun classes.

\section{1. Introduction: the Kainji languages} 

\ili{Kainji} (formerly Plateau 1a,b) is a family of some eighty languages or lects spoken in northwestern and central Nigeria. A large subset of these, the \ili{East} \ili{Kainji} languages, are spoken north and west of the Jos Plateau and are geographically separate from the other branches. \citet{Rowlands1962}, \citet{Greenberg1963}, \citet{Gerhardt1989} and Crozier \& \citet{Blench1992} treat ‘\ili{East}’ and ‘West’ \ili{Kainji} as a primary division of the family, but there is no linguistic evidence to support this. \ili{Kainji} languages are characterised by an extremely diverse morphology and relatively low percentages of common lexical items. It is only comparatively recently that their unity and distinctiveness have been recognised. They form one branch of the \ili{East} \ili{Benue}-Congo family\footnote{This term has a tortuous history. Originally ‘\ili{Benue}-Congo’ included Plateau, \ili{Kainji}, \ili{Jukunoid}, \ili{Cross}-River and Bantoid. Later ‘\ili{Benue}-Congo’ was expanded to include ‘\ili{Eastern} \ili{Kwa}’, i.e. Yoruboid, Edoid, etc. Williamson and \citet{Blench2000} subsequently divided \ili{Benue}-Congo into two branches: West and \ili{East}. The West branch consisted of the previous ‘\ili{Eastern} \ili{Kwa}’ while the \ili{East} branch consisted of the previous ‘\ili{Benue}-Congo’ languages. Thus, ‘\ili{East} \ili{Benue}-Congo’ used here is equivalent to the original ‘\ili{Benue}-Congo’ used in the literature from the 1960s to the 1990s.}, itself a major division of \ili{Niger-Congo}, and their nearest relatives are Plateau and \ili{Jukunoid} (\citealt{Williamson1971}, 1989; Williamson \& \citealt{Blench2000}).

Typologically, \ili{Kainji} languages are difficult to characterise, but the more conservative branches have both nominal and verbal morphology highly reminiscent of \ili{Bantu} (as indeed the –\textit{tu} root for ‘person’). The nominal systems are characterised by alternating affixes and concord on adjectives and some numerals. In some branches these affixes have either collapsed or been heavily restructured, resulting in contrastive consonant length as well as alternating C- prefixes, and rare systems of double-affixing. At least one language, \ili{Shen}, has lost all trace of nominal affixes and has compensated by evolving a complex tonal inventory. The alternating affixes of one language, \ili{Reshe}, show almost no segmental cognates with the remainder of the group and an innovative system must somehow have developed. Some branches have complex verbal morphology highly reminiscent of \ili{Bantu}, with verbs taking long strings of suffixes. Word order is typically S (AUX) V O. \ili{Kainji} languages are grossly under-represented in standard typological sources such as WALS and the summaries of existing material are quite inaccurate.

Most of the West \ili{Kainji} languages are still commonly spoken, which is surprising, given that some are encapsulated by \ili{Hausa} (McGill \& \citealt{Blench2012}). However, \ili{East} \ili{Kainji} languages, with few exceptions, are severely threatened and some have disappeared in recent decades. A few \ili{Kainji} languages have significant numbers of speakers, but most populations are under 10,000. Western \ili{Kainji} languages have been the subject of numerous literacy projects and these community initiatives appear to be sustainable, but \ili{Kainji} languages otherwise have a very low profile in the media.

  
%%please move the includegraphics inside the {figure} environment
%%\includegraphics[width=\textwidth]{KainjifinalversionforLaTexconversion-img1.png}
 

\textbf{\figref{fig:key:1}: The \ili{Kainji} languages}

The human geography of \ili{Kainji}-speaking peoples is very striking. As \figref{fig:key:1} shows, there are outliers of \ili{Kainji} spoken near Makurdi, far from the likely homeland area in the northwest. It is likely that the dispersal of the \ili{Basa} peoples is a consequence of the destructive effects of the nineteenth century slave-raiding era, although this is not confirmed by recorded oral traditions. However, the twentieth century has also seen important migrations. The \ili{Hun}-\ili{Saare} peoples have formed a number of colonies outside their home area to take advantage of more fertile farmland.

\ili{Kainji} languages are spoken in quite inaccessible areas, and even today, the home areas of many languages can be reached only through arduous motorbike trails. This explains why a comprehensive list of these languages is still to be established. The first attempt to compile a comparative \ili{Kainji} wordlist was the work of Clark Regnier, a young SIL linguist who began surveys in the late 1980s. Clark was unfortunately the victim of a fatal motor accident in 1992. From the dry season of 2010, a joint programme to physically visit and record the speech of as many \ili{Kainji} communities as possible has been undertaken by Roger Blench and Stuart McGill. Much of the data used in this paper was collected by the author and Stuart McGill in 2010-2012 and remains unpublished, although an extensive comparative wordlist is available online. At the same time, there has been considerable progress in the development of literacy in individual languages, strongly associated with literacy and bible translation projects (McGill \& \citealt{Blench2012}).

The first lexical material on a \ili{Kainji} language appears to be the \textit{Kambali} lists in \citet{Koelle1854}. Johnston (1919-22, I:732-746) noted that the noun-class systems of the ‘Semi-\ili{Bantu}’ languages of northwestern Nigeria showed marked resemblances to those of the \ili{Bantu} languages and published comparative wordlists linking \ili{Kamuku}, \ili{Gurmana} and \ili{Basa}. Thomas (in \citealt{Meek1925}, II:137) put the known West \ili{Kainji} languages into ‘Nigerian Semi-\ili{Bantu}’ but joined \ili{Lopa} and \ili{Laru} with \ili{Bariba} in ‘Volta’ i.e. \ili{Gur}. In the 1950s, Westermann and \citet[70]{Bryan1952} largely followed Thomas, although recognising that \ili{Kambari}, \ili{Hun}-\ili{Saare} [Duka], and possibly \ili{Kamuku} and Lela [Dakakari] were grouped together. These languages were then listed in a catch-all category ‘class languages’ under the general heading of ‘isolated units’. The recognition that the group now known as West \ili{Kainji} forms a genetic unit is due to \citet[264-6]{Bertho1952} who asserted its coherence on the basis of unpublished wordlists. Bertho rejected the \ili{Gur} affiliations of \ili{Lopa} and \ili{Laru} proposed by Thomas and stated that the affiliations of the ‘\textit{groupe Kamberi}’ were with central Nigerian Plateau languages. A nearly simultaneous classification was proposed by \citet{Greenberg1955} who created a large Plateau group encompassing what would now be called \ili{East} and West \ili{Kainji} (as Plateau 1a and b) as well as \ili{Tarokoid} and \ili{Jukunoid}. The term \ili{Kainji} was informally introduced in the 1980s but was established in an article on Plateau in the reference volume on \ili{Niger-Congo} published at the end of the decade \citep{Gerhardt1989}. No evidence was put forward to support the classification published. Since that date there has been a significant expansion of field data, most of it still in manuscript. The major unpublished sources are listed in \tabref{tab:key:26} in the Appendix A.

As our knowledge of the \ili{Kainji} languages has improved, we can better characterise their internal structure and relationships. The main points are:

1. The distinction established in \citet{Rowlands1962} and \citet{Greenberg1963} between ‘\ili{East}’ and ‘West’ \ili{Kainji} (1a and 1b in Greenberg) has never been demonstrated and seems unlikely to be valid.

2. \ili{Kainji} divides into a number of distinct subgroups, each with highly marked but extremely diverse morphological characteristics.

3. Although Proto-\ili{Kainji} has structural properties similar to proto-\ili{Bantu}, segmental cognates of morphology are difficult to establish

\figref{fig:key:2} shows an abbreviated high-level subclassification of the \ili{Kainji} languages, which proposes names for nodes at different classificatory levels. If further work confirms the tree outlined here then these names can either be adopted or replaced by something more culturally appropriate. More detailed subclassifications of each major branch are given in the relevant numbered sections of §3 below.

\figref{fig:key:2} arranges the subgroups of \ili{Kainji} roughly west to east, except for \ili{East} \ili{Kainji}, and the arrangement of §3 follows the same ordering. For reasons of space, information about other aspects of these languages is very reduced and the material is strictly confined to the data available for actual languages and what can be reasonably reconstructed.

This chapter\footnote{This chapter could not have been prepared without access to a substantial body of unpublished data, and much of the material has been re-analysed from this. I would particularly like to thank Stuart McGill, David Crozier and John Nengel for long-term collaboration, and Appollos Agamalafiya, Clark Regnier (†), Steve Dettweiler, David Heath, Becky Paterson, Katherine Mort, Jennifer Davey for access to data. A significant update on the \ili{East} \ili{Kainji} material has been possible following fieldwork in Nigeria in February-\citealt{May2016}. Thanks to Luther Hon and the Elm House Survey Department for help with logistics and community relations.} provides an overview of \ili{Kainji} nominal affixes and associated concord systems. These are very similar to those described

for \ili{Bantu} and consist of (usually) a prefix on a noun root which marks number and which changes in the plural. Typically there is alliterative concord, where the corresponding affix on a qualifier (adjectives, demonstratives, quantifiers, lower numerals) shows agreement with the prefix. A couple of examples illustrate how this operates. The first example is from the \ili{Ut-Ma}’in [\ili{Fakai}] languages, described in \citet{Smith2007}. A typical alternating prefix would be 

%%1st subexample: change \ea\label{...} to \ea\label{...}\ea; remove \z  
%%further subexamples: change \ea to \ex; remove \z  
%%last subexample: change \z to \z\z 
\langinfo{}{}{\ili{Ut-Ma}’in: alternating prefix}\\
\end{itemize}

\begin{tabularx}{\textwidth}{XXXX}
\lsptoprule
\textit{ər-t\=aʔ\=ar} & ‘stone’ & \textit{ət-t\=aʔ\=ar} & ‘stones’\\
\lspbottomrule
\end{tabularx}
The class prefix is C4 and both the quantifier and the demonstrative show alliterative concord with ‘shea tree’.

\ea\label{ex:key:}
%%1st subexample: change \ea\label{...} to \ea\label{...}\ea; remove \z  
%%further subexamples: change \ea to \ex; remove \z  
%%last subexample: change \z to \z\z 
\langinfo{}{}{\ili{Ut-Ma}’in: alliterative concord}\\
\end{itemize}

\begin{tabularx}{\textwidth}{XXXX}
\lsptoprule
\textit{əs-fàr} & \textit{əs-bɛ:t} & \textit{sɛ} & \textit{hɛ:g}\\
C4-shea.tree & C4-all & C4.Dem & fall.PST\\
\multicolumn{4}{c}{‘all the shea trees fell’}\\
\lspbottomrule
\end{tabularx}
\ili{Cicipu}, a language in the \ili{Kambari} cluster, has an extremely transparent agreement system (Mc\citealt{Gill2007}).

\ea\label{ex:key:}
%%1st subexample: change \ea\label{...} to \ea\label{...}\ea; remove \z  
%%further subexamples: change \ea to \ex; remove \z  
%%last subexample: change \z to \z\z 
\langinfo{}{}{\ili{Cicipu} agreement system}\\
\end{itemize}

\begin{tabularx}{\textwidth}{XX}
\lsptoprule
\textit{màdíyá mè-pénâu} & ‘big hare’\\
\textit{ìndíyá ìm-pénàu} & ‘big hares’\\
\lspbottomrule
\end{tabularx}
Various publications and theses have described the noun class systems for individual languages (e.g. \citet{Crozier1984}, Mc\citet{Gill2009}, \citet{Paterson2012}) but little has been written concerning the overall pattern they form. The chapter begins with a summary table of nominal affixing systems and then goes through what is known about each branch. A tentative model of the situation that can be attributed to proto-\ili{Kainji} is given in a final section together with a summary of the evidence for nasal affixes in \ili{Kainji}. The numbers assigned to noun classes are those in the source materials. Analysis is far from the point where a standard system of numbers can be established for \ili{Kainji} languages.

It cannot be emphasised too strongly that the quality of data for different branches is very uneven and that as the great majority is unpublished it should be treated as preliminary. It is unfortunate that a lack of pressure to publish means that preliminary language analyses circulate in manuscript and are made available by the authors on an informal basis. In particular, individual authors use affix numbering devised for a specific language and thus comparison across languages is more difficult. Tone-marking in particular is somewhat impressionistic. In general, in three-tone languages such as \ili{Reshe}, mid-tone is unmarked. Where the data is too poor to mark tone with certainty, this is flagged in the text.

  
%%please move the includegraphics inside the {figure} environment
%%\includegraphics[width=\textwidth]{KainjifinalversionforLaTexconversion-img2}
 

\textbf{\figref{fig:key:2}: Subclassification of \ili{Kainji} languages} 

\section{2. Nominal affixes: overview}

Given the prevalence of alternating affixes and concord in some \ili{Kainji} languages, it is reasonable to suppose that a system of this type was present in proto-\ili{Kainji}. Nonetheless, the synchronic diversity within the family is such that these systems are lost or severely reduced in many languages. Tables 1a \& 1b summarise the situation in various \ili{Kainji} subgroups.


\textbf{{\tabref{tab:key:1}a: Nominal affixing in \ili{Kainji} Languages}}


================================================

\tablehead{\textbf{Branch} & \textbf{Subgroup} & \textbf{Language} & \textbf{Comment}\\
}
\begin{tabularx}{\textwidth}{XXXX}
\lsptoprule
\ili{Reshe} &  & \ili{Reshe} & Alternating affixes and concord\\
Upper Niger & Rerang & \ili{Rop} & Alternating affixes and concord\\
& \ili{Laru} & \ili{Shuba} & Affix system very reduced\\
&  & \ili{Shen} & Affix system absent\\
&  &  & \\
Northwest & Lela & cLela & Reduced affix system with C- prefixes and concord\\
& \ili{Hun} & t\ili{Hun}/s\ili{Saare} & Reduced affix system with C- prefixes and concord\\
& Gwamhi & Gwamhyə, Wurə, \ili{Mba} & Reduced affix system with C- prefixes and concord\\
& ut-Ma’in & All & Alternating affixes and concord\\
& ? & Damakawa & Moribund\\
&  &  & \\
\ili{Kambari} &  & CiShingini & Alternating affixes and concord\\
&  & Tsivaɗi & Alternating affixes and concord\\
&  & Baangi & Alternating affixes and concord\\
&  & Tsɨkimba & Alternating affixes and concord\\
&  & Agwara & Alternating affixes and concord\\
&  & \ili{Cicipu} & Alternating affixes and concord\\
&  &  & \\
\ili{East} &  & All & Alternating affixes and concord\\
&  & Shama & Alternating affixes and concord\\
\lspbottomrule
\end{tabularx}
\textbf{===================================================}

\textbf{\tabref{tab:key:1}b: Nominal affixing in \ili{Kainji} Languages}

 \textbf{(continuation of \tabref{tab:key:1}a)}

\tablehead{\textbf{Branch} & \textbf{Subgroup} &  \textbf{Language} &  \textbf{Comment}\\
}
\begin{tabularx}{\textwidth}{XXXX}
\lsptoprule
\ili{Kamuku} &  &  & \\
&  & Rogo-Shyabe & Alternating affixes and concord\\
&  & Səgəmuk & Alternating affixes and concord\\
&  & \ili{Cinda} & Alternating affixes and concord\\
&  & Regi & Alternating affixes and concord\\
&  & Kuki & Alternating affixes and concord\\
&  & Zubazuba & Alternating affixes and concord\\
&  & Hungw[1DD?]ry[1DD?] & Complex alternating affixes and concord\\
&  &  & \\
\ili{Shiroro} &  & \ili{Fungwa} & Alternating affixes and concord\\
&  & \ili{Rin} & Alternating affixes and concord\\
&  & Wəgə & Unclear since moribund\\
&  & \ili{Gurmana} & Alternating affixes and concord\\
& \ili{Baushi} & All & Affix system in partial breakdown\\
&  &  & \\
\ili{Basa} &  & \ili{Basa} Kontagora & Extinct\\
&  & \ili{Basa}-Gumna & Extinct\\
&  & Kɔrɔmba & Affix system functional\\
&  & \ili{Basa}-Gurara & No information\\
&  & \ili{Basa}-\ili{Benue} & Three-term alternating affixes and concord\\
&  & \ili{Basa}-Makurdi & Affix system in breakdown\\
\lspbottomrule
\end{tabularx}
\section{\textmd{================================================}}
\section{3. Nominal affixes by subgroup}
\subsection{{3.1 Reshe}\footnotemark{}}
\footnotetext{ {Throughout this chapter, class prefixes are deleted in language and ethnic group names to create a uniform reference term.}}

Tsureshe, the language of the \ili{Reshe} people, is spoken at the northern   end of Lake \ili{Kainji} (Dettweiler \& \citealt{Dettweiler1993b}). \ili{Reshe} has a characteristic \ili{Niger-Congo} noun class system, reminiscent of \ili{Bantu} in several ways, although the class pairings are much reduced\footnote{Work on \ili{Reshe} was conducted jointly between the author and Appollos Agamalafiya in 2010 and 2011. See also the unpublished Boettger \& \citet{Boettger1967}}. The noun stem is preceded by a class marker, either V- or CV-, which alternates between singular and plural and shows concord with adjectives and other parts of speech. There are six paired classes, four of which clearly have semantic motivation: those containing human beings, animals, body parts and mass nouns, although the class including body parts is more weakly defined than the others (\tabref{tab:key:2}). Class 6, which is invariant, includes mass nouns such as liquids, powders and similar items. Membership of the other two noun classes appears to be arbitrary. \tabref{tab:key:2} summarises \ili{Reshe} noun-class pairings:

\begin{tabularx}{\textwidth}{XXXXX}
\lsptoprule
\multicolumn{5}{c}{
\begin{tabularx}{\textwidth}{X}
\lsptoprule
\itshape \textbf{{\tabref{tab:key:2}: \ili{Reshe} noun-class affix pairings} }\\
\lspbottomrule
\end{tabularx}
}\\
\textbf{No.} & \textbf{sg.} & \textbf{No} & \textbf{pl.} & \textbf{Semantic content}\\
1 & \textit{u{\textasciitilde}w} & 2 & \textit{bV-} & Human\\
3 & \textit{hi{\textasciitilde}hy-} & 4 & \textit{i{\textasciitilde}y-} & animals and borrowed words\\
6 & \textit{mV-} &  & invariant & mass nouns\\
7 & \textit{ú-} & 8 & \textit{á-} & body parts\\
9 & \textit{ú{\textasciitilde}w-} & 10 & \textit{tʃ{\textasciitilde}ts(u){\textasciitilde}}\textit{[F04F?]{}-} & Miscellaneous\\
11 & \textit{ri{\textasciitilde}ry-} & 12 & \textit{a-} & Miscellaneous\\
\lspbottomrule
\end{tabularx}
==============================================

The tones of the prefixes are highly variable and it seems cannot be assigned an underlying height. There is no evidence for tonal changes in the stem between singular and plural and the tone of the plural prefix is always the same as the singular.

\ili{Reshe} has a complete set of object pronouns which correspond to the nominal affixes. However, where the pronoun refers to something unknown or despised, \textit{là}, a generic pronoun not marked for number is used.

%%1st subexample: change \ea\label{...} to \ea\label{...}\ea; remove \z  
%%further subexamples: change \ea to \ex; remove \z  
%%last subexample: change \z to \z\z 
\langinfo{}{}{\ili{Reshe}: generic pronoun \textit{là}}\\
\end{itemize}
\ea\label{ex:key:}
\langinfo{}{}{\textit{ù sárì là}}\\
\gll              {he cut  it/them}\\
                  ‘s/he cut it/them’\\
This is an allomorph of the subject pronoun for inanimates, \textit{lə}.

There are a small number of unusual items, shown in \tabref{tab:key:3}, that do not form part of the noun class pairings given above. These are invariant nouns, either mass nouns or inherently plural.

\begin{tabularx}{\textwidth}{XX}
\lsptoprule
\multicolumn{2}{c}{\textbf{Table 3: Extra-systemic \ili{Reshe} nouns}}\\
\multicolumn{2}{c}{\textbf{=============================}}\\
\textbf{Tsureshe} & \textbf{Gloss}\\
\textit{[1EBD?]hɛ} & ‘tears’\\
\textit{èena} & ‘waves’\\
\textit{ə{}-ʃìmà} & ‘fat’\\
\textit{ə{}-rira} & ‘river’\\
\lspbottomrule
\end{tabularx}
Surprisingly, if they are replaced by a pronoun in a sentence, the pronoun is \textit{əbə,} usually associated with humans.

\ili{Reshe}, like many languages in this region, has distinctive incorporated possessives for kin terms and related nouns for persons. The affixes appear on the surface to have class-pair alternation, but the associated concord is that of the underlying noun. So, for example, in the word for ‘age-mate’, the \textit{mu-/ba-} alternation strongly recalls \ili{Bantu} prefixes, but in \ili{Reshe} these probably originate with possessives:

\ea\label{ex:key:}
%%1st subexample: change \ea\label{...} to \ea\label{...}\ea; remove \z  
%%further subexamples: change \ea to \ex; remove \z  
%%last subexample: change \z to \z\z 
\langinfo{}{}{\ili{Reshe}: \textit{mu-/ba-} alternation recalls \ili{Bantu} prefixes}\\
\end{itemize}

\begin{tabularx}{\textwidth}{XXXX}
\lsptoprule
 \textbf{sg.} &  \textbf{pl.} &  \textbf{Gloss} &  \textbf{Literal}\\
 \textit{mú-banɛ} &  \textit{bà-} & ‘age-mate,   colleague’ & lit. ‘my   another’\\
\lspbottomrule
\end{tabularx}
The singular first person possessive is \textit{mú}:

\ea\label{ex:key:}
%%1st subexample: change \ea\label{...} to \ea\label{...}\ea; remove \z  
%%further subexamples: change \ea to \ex; remove \z  
%%last subexample: change \z to \z\z 
\langinfo{}{}{Reshe}\\
\end{itemize}

\begin{tabularx}{\textwidth}{XX}
\lsptoprule
 \textit{mú} & \textit{úlɔ}\\
  ‘my & friend’\\
\lspbottomrule
\end{tabularx}
which has been paired with the usual class 1/2 plural prefix \textit{bà-}.

\subsection{{3.2 Upper Niger (Shen and Rerang)}}

Like the \ili{Reshe}, the \ili{Laru} (\ili{Shen}) and \ili{Lopa} (Rerang) are fishing peoples who live around the edge of Lake \ili{Kainji}. Research in 2011 and 2012 showed that ‘\ili{Lopa}’ is in fact two distinct languages. Even more surprisingly, despite the ethnic label Rerang and the assertion of a common culture between the \ili{Rop} and the \ili{Shuba}, \ili{Shuba} is clearly a conservative type of \ili{Shen}, but which still retains at least some nominal morphology. The correct terminology for the Upper Niger languages is shown in \tabref{tab:key:4}:

\begin{tabularx}{\textwidth}{XXXXXX}
\lsptoprule
\multicolumn{6}{c}{{\itshape \textbf{{Table}}\textbf{ {4: Ethnonyms and reference names of the Upper Niger Group}}}

\itshape \textbf{{==========================================================}}}\\
\textbf{Usual name} & \textbf{Group name} & \textbf{One person} & \textbf{People} & \textbf{Language} & \textbf{Reference Name}\\
\ili{Laru} &  & \textit{shen} &  & \textit{shen gwe} & \ili{Shen}\\
\ili{Lopa} & Rerang & \textit{dɔɾìɾã\'{} ŋ} & \textit{òːɾìɾã\'{} ŋ} & \textit{òlːèɾã\'{} ŋ} & \\
& \ili{Rop} & \textit{dɔɾóp} & \textit{òːɾɔp} & \textit{òlːɔp} & \ili{Rop}\\
& \ili{Shuba} &  &  &  & \ili{Shuba}\\
\lspbottomrule
\end{tabularx}
=================================================

\ili{Shen} exhibits a virtually complete loss of the nominal morphology system. All nouns either have no plural, or a plural suffix \textit{bà(u)}. \ili{Shen} has come under heavy influence from the \ili{Busa} language, which is \ili{Mande} and thus also has similar characteristics. Despite their different morphology, \ili{Shuba} and \ili{Shen} clearly share a significant amount of common lexicon.

By contrast, \ili{Shuba} has not only a relatively rich system of nominal affixation, but demonstrates reprefixing, with unproductive prefixes now incorporated into the stem. \ili{Shuba}, like many other \ili{Kainji} languages, has underspecified vowels in CV- prefixes which frequently show harmony with the stem vowels. The following examples\footnote{Unfortunately, when the data was collected, tones were not marked} show typical singular//plural pairs.

\ea\label{ex:key:}
%%1st subexample: change \ea\label{...} to \ea\label{...}\ea; remove \z  
%%further subexamples: change \ea to \ex; remove \z  
%%last subexample: change \z to \z\z 
\langinfo{}{}{\ili{Shuba} prefixation}\\
\end{itemize}
\ea\label{ex:key:}
\langinfo{}{}{a. ø\textbf{{}-}/SV-}\\
\begin{tabularx}{\textwidth}{XXX}
\lsptoprule
\textbf{Gloss} & \textbf{sg.} & \textbf{pl.}\\
Tree (generic) &   \textit{ʃə} &  \textit{ʃi-ʃə}\\
Leaf &  \textit{fwã}  &  \textit{sə-fwã} \\
\lspbottomrule
\end{tabularx}
b. ø-/a-

\begin{tabularx}{\textwidth}{XXX}
\lsptoprule
\textbf{Gloss} & \textbf{sg.} & \textbf{pl.}\\
Moon/month &  \textit{’yuuru} &  \textit{a-’yuuru}\\
Sun &  \textit{gwi} &  \textit{a-gwi}\\
\lspbottomrule
\end{tabularx}
c. rV/a-

\begin{tabularx}{\textwidth}{XXX}
\lsptoprule
\textbf{Gloss} & \textbf{sg.} & \textbf{pl.}\\
Field &  \textit{ra-hãi} &  \textit{a-hãi}\\
Seed/stone/pip &  \textit{re-kero} &  \textit{a-kero}\\
Mountain &  \textit{ri-yam} &  \textit{a-yam}\\
\begin{tabularx}{\textwidth}{XXX}
\lsptoprule
Nose &  & \\
\lspbottomrule
\end{tabularx}
 &  \textit{ro-hɔro} &  \textit{a-hɔro}\\
\lspbottomrule
\end{tabularx}
The word for ‘nose’ is an interesting example of double affixing, which probably arises through the copying of demonstratives (see \citealt{Hoffmann1967} for examples from cLela). \ili{Shuba} is cognate with t\ili{Hun} \textit{r-ho} for ‘nose’ and the prefix has been copied as a suffix.

d. fV/a-

\begin{tabularx}{\textwidth}{XXX}
\lsptoprule
\textbf{Gloss} & \textbf{sg.} & \textbf{pl.}\\
Rubbish-heap &  \textit{fɔ-k\~uh\~u} &  \textit{a-k\~uh\~u}\\
Tooth &  \textit{fo-yefə} &  \textit{a-yefə}\\
Farm &  \textit{fu-tuma} &  \textit{a-tuma}\\
\lspbottomrule
\end{tabularx}
e. sV/a-

\begin{tabularx}{\textwidth}{XXX}
\lsptoprule
\textbf{Gloss} & \textbf{sg.} & \textbf{pl.}\\
Dew &  \textit{sə-myem} &  \textit{a-myem}\\
Room &  \textit{su-rukwə} &  \textit{a-rukwə}\\
\lspbottomrule
\end{tabularx}
f. N/a-

\begin{tabularx}{\textwidth}{XXX}
\lsptoprule
\textbf{Gloss} & \textbf{sg.} & \textbf{pl.}\\
Water &  \textit{m-mi} &  \textit{a-mi}\\
Sorghum-beer &  \textit{ŋ-kwa} &  \textit{a-kwa}\\
\lspbottomrule
\end{tabularx}
g. do-/bV-

\begin{tabularx}{\textwidth}{XXX}
\lsptoprule
\textbf{Gloss} & \textbf{sg.} & \textbf{pl.}\\
Person/people &  \textit{do-h\~umwa} &  \textit{bo-h\~umwa}\\
Man &  \textit{do-rumburu} &  \textit{bu-rumburu}\\
\lspbottomrule
\end{tabularx}
h. ø\textbf{{}-}/bV-

\begin{tabularx}{\textwidth}{XXX}
\lsptoprule
\textbf{Gloss} & \textbf{sg.} & \textbf{pl.}\\
Child &  \textit{bi} &  \textit{bu-bi}\\
Chief/ruler &  \textit{tɔ\~\iʃa} &  \textit{bə-tɔ\~\iʃa}\\
\lspbottomrule
\end{tabularx}
But:

i. ø\textbf{{}-}/-bə-

\begin{tabularx}{\textwidth}{XXX}
\lsptoprule
\textbf{Gloss} & \textbf{sg.} & \textbf{pl.}\\
Father &  \textit{metõ} &  \textit{mebətõ}\\
Friend &  \textit{medo} &  \textit{mebədo}\\
\lspbottomrule
\end{tabularx}
The infixing of a \textit{{}-bə-} sequence is probably a special case of \textit{ø}\textbf{\textit{{}-}}\textit{/bV-.} The \textit{me}{}- is probably a fused possessive, cognate with \ili{Reshe} \textit{mú}{}- (see above).

As an example of how reprefixing works, the word for ‘vulture’ is almost certainly a borrowing from \ili{Nupe} \textit{gùl[1D4?]}. When first borrowed, it seems to have been attributed an \textit{sV}{}- prefix, rather like ‘tree’ above. However, it was then re-analysed as part of the sV/a- class, hence the current synchronic form. Similarly with ‘pot’ which has an old rV- prefix, fused with the stem and also copied as a final syllable.

\ea\label{ex:key:}
%%1st subexample: change \ea\label{...} to \ea\label{...}\ea; remove \z  
%%further subexamples: change \ea to \ex; remove \z  
%%last subexample: change \z to \z\z 
\langinfo{}{}{\ili{Shuba} reprefixing}\\
\end{itemize}

\begin{tabularx}{\textwidth}{XXX}
\lsptoprule
\textbf{Gloss} & \textbf{sg.} & \textbf{pl.}\\
‘hooded  vulture’ &  \textit{saguru} &  \textit{a-saguru}\\
‘pot’ &  \textit{ruburu} &  \textit{a-ruburu}\\
\lspbottomrule
\end{tabularx}
This diversity suggests that many of the prefixes are innovative. The nasal in mass nouns recalls the Class 6 prefix and the plural of \textit{do-/bV-} Class 2, the plural of ‘persons’. \figref{fig:key:3} summarises the \ili{Shuba} singular/plural affix alternations: The merger of many plural affixes to \textit{a-} resembles the universal plural prefix \textit{a-} in the \ili{Gbari} languages (Hyman \& \citealt{Magaji1970}).


[Warning: Draw object ignored] rV-



[Warning: Draw object ignored] fV-



[Warning: Draw object ignored][Warning: Draw object ignored] sV\textbf{{}-} a-



[Warning: Draw object ignored] N\textbf{{}-}



[Warning: Draw object ignored][Warning: Draw object ignored][Warning: Draw object ignored] ø\textbf{{}-} sV-



 {}-bə-



[Warning: Draw object ignored] do\textbf{{}-} bV-



[Warning: Draw object ignored] \textbf{{}-}bi sV-



[Warning: Draw object ignored] \textbf{{}-}ø {}-zi



\textbf{{\figref{fig:key:3}: \ili{Shuba} noun-class affix pairings}}


\subsection{{3.3 Northwest Kainji}}

{Lela (as \ili{Dakarkari}) is often used as a cover-term for the peoples of the region between Rijau and Donko in reference books such as Gunn and \citet{Conant1960}. This name is now generally rejected, and it is here proposed to adopt the term ‘Northwest \ili{Kainji}’ to cover this branch, which consists of the cLela, \ili{Hun}-\ili{Saare}, \ili{Kag} clusters and the Wurə-Gwamhyə-\ili{Mba} languages. The group is unified by a striking morphological feature, the reduction of nominal prefixes to single consonants. A consequence of this is the loss of harmony between prefix and stem vowels. Nominal affixing in the Northwest \ili{Kainji} languages is relatively well-described, with analyses for cLela Dettweiler (ined.), \ili{Hun}-\ili{Saare} (Bendor-Samuel} {et al.} {1973) and} {U}{t-Ma’in \citep{Patterson2012}. A particular feature of this group is affix copying (first noted in \citealt{Hoffmann1967}) which results in suffixes in animate classes (cf. an example in §3.3.1 below).}


\figref{fig:key:4} shows a subclassification of the Northwest \ili{Kainji} languages, based on lexical innovations. Damakawa is a moribund language recorded by McGill (pers. comm.) for which the data is too fragmentary to classify it with certainty.

[Warning: Draw object ignored][Warning: Draw object ignored][Warning: Draw object ignored][Warning: Draw object ignored][Warning: Draw object ignored][Warning: Draw object ignored][Warning: Draw object ignored][Warning: Draw object ignored][Warning: Draw object ignored][Warning: Draw object ignored]  
%%please move the includegraphics inside the {figure} environment
%%\includegraphics[width=\textwidth]{KainjifinalversionforLaTexconversion-img3}
 


\textbf{{Figure} }\textbf{{4: Subclassification of the Northwest \ili{Kainji} languages}}



\subsubsection{3.3.1 cLela (Dakarkari)}

{CLela has ten noun classes marked by six consonant prefixes:} c-, d-, k-, m-, s-, v-{; three vowel prefixes:} a-, i-, u-{; and a common noun} {}-n {suffix (\citealt{Hoffmann1967}; Dettweiler ined.). Number is marked on inanimate nouns with prefixes; while in animates plurality is indicated by an -nV suffix. Classes 1-9 are all inanimates, while animates are all grouped in a single class.} 



{\tabref{tab:key:5} summarises the noun class affixes of cLela in \citet{Dettweiler2015}:}



\textbf{{\tabref{tab:key:5}: cLela noun class affixes}}



\textbf{{================================================}}



\textbf{ {Singular            Plural} }



\textbf{{No.  Prefix      No.   Prefix      Suffix}}



\textbf{ {(Inanimate)    (Animate)}}



\textbf{{———————————————————————-}}



{1  a-      2  c-      –}



{3  u-      2  c-      –}



{4  d-      2  c-      –}



{5  k-      2  c-      –}



{6  v-      7  s-      –}



{8  i-      9  m-      –}



{9  m-              –}



{10  ø-      10  –      {}-nV-}



{==================================================}


A published dictionary of cLela provides a broad variety of examples of noun-class pairings \citep{RikotoEtAl2001}. The singular and plural affixes are exemplified below in \tabref{tab:key:6}:

\begin{tabularx}{\textwidth}{XXXXXX}
\lsptoprule
\multicolumn{6}{c}{{\itshape \textbf{{\tabref{tab:key:6}: Examples of cLela noun class affix pairings}}}

\itshape \textbf{{============================================}}}\\
\textbf{Class} & \textbf{Singular} & \textbf{Gloss} & \textbf{Class}  & \textbf{Plural} & \textbf{Gloss}\\
  1 &  a-cù &  ‘face’ &  2 &   c-cù &  ‘faces’\\
  3 &  u-bèlà &  ‘farm‘ &  2 &  c-bela &  ‘farms’\\
  4 &  d-isá &  ‘eye‘ &  2 &  c-isá &  ‘eyes’\\
  5 &  k-wècé &  ‘cloud‘ &  2 &  c-wècé &  ‘clouds’\\
  6 &  v-hw\textcyrillic{ѐn} &  ‘rope‘ &  7 &   s-hw\textcyrillic{ѐn}  &  ‘ropes’\\
  8 &  i-hònò &  ‘calabash‘ &  9 &   m-hònò &  ‘calabashes’\\
  9 &  m-hò &  ‘water‘ &  &  & \\
  10 &  nàamá &  ‘cow‘ &  10 &  nàam.ná &  ‘cows’\\
 10 &  \textstyleLexeme{\textmd{nètà}} &  ‘woman‘ &  10 &  \textstyleLexeme{\textmd{nètà.ná}} &  ‘women’\\
\lspbottomrule
\end{tabularx}
     =============================================

The animates class \textit{ø-/-nV} is related to the forms in the neighbouring t\ili{Hun} language but is innovative within \ili{Kainji}. Only the mass noun prefix \textit{m-} (here class 9) corresponds to \ili{Niger-Congo} 6, but has merged with the plural of Class 8 above. 

Like \ili{Shuba} and \ili{Reshe}, cLela has \textit{mV-} for inalienable possession of kin \citep{Dettweiler2015}. Thus:

%%1st subexample: change \ea\label{...} to \ea\label{...}\ea; remove \z  
%%further subexamples: change \ea to \ex; remove \z  
%%last subexample: change \z to \z\z 
\langinfo{}{}{cLela: inalienable possession}\\
\end{itemize}

\begin{tabularx}{\textwidth}{XXXX}
\lsptoprule
\textit{hən-mí} & ‘sibling-my’ & \textit{hən-mí-nì} & ‘sibling-my-PL’\\
\textit{ʧèt-mé} & ‘father-my’ & \textit{ʧèt-mé-nè} & ‘fathers-my-PL’\\
\lspbottomrule
\end{tabularx}
where \textit{nV}{}- is the Class 2 plural for persons.

A striking feature of Northwest \ili{Kainji} is double-affixing in compounds, first noted in \citet{Hoffmann1967}. This arises when the prefix of the head noun is displaced to the associated noun and precedes its own prefix. Thus:

\ea\label{ex:key:}
%%1st subexample: change \ea\label{...} to \ea\label{...}\ea; remove \z  
%%further subexamples: change \ea to \ex; remove \z  
%%last subexample: change \z to \z\z 
\langinfo{}{}{cLela: double-affixing in compounds}\\
\end{itemize}

\begin{tabularx}{\textwidth}{XXX}
\lsptoprule
\textbf{Gloss} & \textbf{cLela} & \textbf{Components}\\
‘spine’ & \textit{tɛl k-ədcìnə} & \textit{k-tɛlɛ} ‘bone’ + \textit{d-cìnə} ‘back’\\
\lspbottomrule
\end{tabularx}
This displacement also occurs on demonstratives:

\ea\label{ex:key:}
%%1st subexample: change \ea\label{...} to \ea\label{...}\ea; remove \z  
%%further subexamples: change \ea to \ex; remove \z  
%%last subexample: change \z to \z\z 
\langinfo{}{}{cLela: displacement of nominal prefix to demonstratives}\\
\end{itemize}

\begin{tabularx}{\textwidth}{XXXX}
\lsptoprule
\textit{c-gyàŋ} & ‘eggs’ & \textit{gyàn cəhnà} & ‘these eggs’\\
\lspbottomrule
\end{tabularx}
\subsubsection{3.3.2 Hun-Saare (Duka)}

The \ili{Hun}-\ili{Saare} people live directly south of the Lela, straddling the border of Niger and Kebbi States. They are conventionally divided into two groups, the \ili{Hun} and the \ili{Saare}, but are commonly known in \ili{Hausa} as Duka and their language as \ili{Dukanci} (Dettweiler \& \citealt{Dettweiler1993a}). They are first mentioned by \citet[96-100]{Temple1922}. The noun-phrase is described in Bendor-Samuel \textit{et al.} (1973). An electronic dictionary and grammar of t\ili{Hun} (Dukawa) is available, associated with the translation and literacy project (Heath p.c. a,b).

The presentation of \ili{Hun}-\ili{Saare} nominal affixing is far from transparent and is moreover, given in orthographic representations. The class marker can move from before to after the noun root. When the marker is before the noun it is the object of the verb and when it follows, the noun is the subject. \tabref{tab:key:7} shows the system of t\ili{Hun} nominal affixes and concord, based on Heath (p.c.). Note that Bendor-Samuel \textit{et al.} (1973) give a somewhat different presentation.

\begin{tabularx}{\textwidth}{XXXXXXX}
\lsptoprule
\multicolumn{7}{c}{{\itshape \textbf{{Table} }\textbf{{7: t\ili{Hun} nominal affixes and concord}}}

\itshape \textbf{{=================================================}}}\\
& \textbf{Singular} &  &  & \textbf{Plural} &  & \\
\textbf{No.} & \textbf{Affix} & \textbf{Pronoun} & \textbf{No.} & \textbf{Affix} & \textbf{Pronoun} & \textbf{Semantics}\\
1 & o- & wə & 6 & {}-nɛ & ɛ & persons\\
2 & {}-ər- & ɔ & 7 & {}-ɛgɛ-,

   {}-ɛ{}- & yo & miscellaneous\\
3 & {}-m- & yo & 8 & {}-ət- & sɛ & miscellaneous\\
4 & ø- & de & 9 & {}-ər- & rɔ & miscellaneous\\
5 & {}-m- & mɔ & 10 & {}-m- & mɔ & mass, uncountable\\
\lspbottomrule
\end{tabularx}
=================================================

An example of the movement of affixes in relation to intraclausal position is the following (Heath n.d.):

%%1st subexample: change \ea\label{...} to \ea\label{...}\ea; remove \z  
%%further subexamples: change \ea to \ex; remove \z  
%%last subexample: change \z to \z\z 
\langinfo{}{}{t\ili{Hun}: nominal affix on subjects and objects} \\
\end{itemize}

\begin{tabularx}{\textwidth}{XXX}
\lsptoprule
\textit{hɔ.m} & \textit{ʃo’ostɛ} & \textit{o.wak}\\
water.CM & filled & CM.swamp\\
‘water & filled & the swamp’\\
\lspbottomrule
\end{tabularx}
As with cLela, a class pair marking persons and a mass noun affix can be discerned, but otherwise, t\ili{Hun} shows few cognates with other systems.

\subsubsection{3.3.3 The Kag (Ut-Ma’in or Fakai) cluster}

The first mention of the languages of the \ili{Kag} cluster is \citet[89]{Temple1922} who refers to ‘Kelinchi’ [? = Kelanci, i.e. Ker-ni]. \citet{Rowlands1962} gives short lists of nouns in ‘Fakawa’, Kelawa and \ili{Zusu}. \citet{Regnier2003} conducted a sociolinguistic survey among five of the eight named \ili{Fakai} cluster members in 1991-1992. \citet{Paterson2012} represents new in-depth fieldwork on the \ili{Ror} language, now named ut-Ma’in by its speakers. \tabref{tab:key:8} shows the peoples and languages of this cluster:

\begin{tabularx}{\textwidth}{XXX}
\lsptoprule
\multicolumn{3}{c}{{\itshape \textbf{{Table} }\textbf{{8: Peoples and Languages of the \ili{Kag} Cluster}}}

\itshape \textbf{{========================================}}}\\
\textbf{\ili{Hausa} Name} & \textbf{People} &  \textbf{Language}\\
Fakkawa &  \ili{Kag}-ne &  [1DD?]t-\ili{Kag}\\
Fakkawa &  əs-Us &  [1DD?]t-Us\\
Gelawa &  a-Jiir &  [1DD?]t-Jiir\\
Zuksun &  a-Zuksun &  [1DD?]t-Zuksun\\
Kukumawa &  əs-Fer &  [1DD?]t-Fer\\
Kelawa &  Kər-ni &  [1DD?]t-Kər\\
Tuduwa &  aor &  [1DD?]t-maor\\
Kuluwa &  a-Koor &  [1DD?]t-ma-Koor\\
\lspbottomrule
\end{tabularx}
  =========================================

\tabref{tab:key:9} shows the thirteen noun classes in ut-Ma’in, following \citet{Paterson2012}. Three classes share the same \textit{ū-} prefix, but their distinct concords suggest class merger. Four classes have a null ø- prefix, but with similarly diverse agreement morphemes. I have added Class 2b, which is the \textit{{}-nɛ}\textbf{ }plural suffix marking some persons, cognate with similar \textit{nV-} suffixes in cLela and t\ili{Hun}. The first column gives a class affix number, corresponding to \ili{Bantu} where possible.{}  For the diminutive and augmentative classes of ut-Ma’in the labels DIM and AUG are used. The second column shows the nominal prefix and the third column represents the agreement targets, indicated by the object pronoun. The last two columns give sample lexemes from each class.

As elsewhere in the group, the 1/2 class pairing marks persons and the \textit{əm-} prefix marks uncountable nouns. A common feature of Northwest \ili{Kainji} is the suffix -\textit{nV} marking the plural of nouns for ‘persons’. In \ili{Kainji} languages and elsewhere in \ili{Benue}-Congo (cf. the \ili{Benue}-Congo Comparative Wordlist) -\textit{net} is a common word for person and it is possible the suffix is an old compound which has been generalised across the group. If so this creates a certain amount of redundancy. For example:

\begin{itemize}
\item \ea\label{ex:key:}
\langinfo{}{}{\ili{Ut-Ma}’in: suffix \textit{–nV} for plural of nouns for ‘persons’}\\
\end{itemize}

    ‘person’ \textit{nɛt}    ‘persons’  \textit{nɛtnɛ}

No other clear source for this suffix has been identified.

\textbf{\tabref{tab:key:9}: \ili{Ut-Ma}’in noun classes} Source: \citet{Paterson2012}

  ================================================

\begin{tabularx}{\textwidth}{XXXXX}
\lsptoprule
{\textbf{Class}} & \textbf{Prefix} & \textbf{Object Pronoun} & \textbf{\ili{Ut-Ma}’in} & {\textbf{Gloss}}\\
{1a} & \textit{ū-} & \textit{ú/wá} & \textit{ū-mákt} & {‘barren woman’}\\
&  &  & \textit{ū-rāg} & {‘stupid person’}\\
{1b} & \textit{ø-} & \textit{wá} & \textit{ø-hám[258?]t} & {‘visitor’}\\
&  &  & \textit{ø-zʷàr} & {‘young man’}\\
{2a} & \textit{ø-} & \textit{ɛ} & \textit{ø-ná} & {‘oxen, bovines’}\\
&  &  & \textit{ø-hʲ[258?]} & {‘guinea corn (pl)’}\\
&  &  & \textit{ø-rɛgɛr} & {‘stars’}\\
{2b} & \textit{ø-} & \textit{{}-nɛ} & \textit{ø-nɛtnɛ} & {‘people’}\\
{3a} & \textit{\=u-} & \textit{ɔ} & \textit{ū-bù} & {‘house’}\\
&  &  & \textit{ū-k[2B0?]óm} & {‘arm’}\\
&  &  & \textit{ū-sɛp} & {‘song’}\\
{3b} & \textit{ø-} & \textit{ɔ} & \textit{ø-bòʔ}  & {‘dream’}\\
&  &  & \textit{ø-[2A4?]āb} & {‘heart’}\\
&  &  & \textit{ø-sʷás} & {‘fish trap’}\\
{4} & \textit{[258?]s-} & \textit{sɛ} & \textit{[258?]s-bòʔ} & {‘dreams’}\\
&  &  & \textit{[258?]s-rā} & {‘muscle’}\\
&  &  & \textit{[258?]s-bàːt} & {‘medicine’}\\
{5} & \textit{[258?]r-} & \textit{dɛ} & \textit{[258?]r-kɔk} & {‘calabash’}\\
&  &  & \textit{[258?]r-[2A4?]āb} & {‘liver’}\\
&  &  & \textit{[258?]r-hí} & {‘head’}\\
{6a} & \textit{[258?]t-} & \textit{tɔ} & \textit{[258?]t-kɔk} & {‘calabashes’}\\
&  &  & \textit{[258?]t-ís} & {‘eyes’}\\
&  &  & \textit{[258?]t-rīn} & {‘charcoal’}\\
{6b} & \textit{[258?]m-} & \textit{mɔ} & \textit{[258?]m-nɔːg} & {‘oil’}\\
&  &  & \textit{[258?]m-hʲ[258?]} & {‘blood’}\\
&  &  & \textit{[258?]m-hʲ[258?]r[258?]g} & {‘sand’}\\
{7a} & \textit{\=u-} & \textit{já} & \textit{ū-ná} & {‘bovine’}\\
&  &  & \textit{ū-ʧān} & {‘feather’}\\
&  &  & \textit{ū-nín} & {‘tooth’}\\
{7b} & \textit{ø-} & \textit{já} & \textit{ø-tʃāmpá}  & {‘man’}\\
&  &  & \textit{ø-mārímárí} & {‘the dead’}\\
&  &  & \textit{ø-rʲâm} & {‘cripple (n)’}\\
{\textsc{aug}} & \textit{ā-} & \textit{á} & \textit{ā-kɔk} & {‘huge calabashes’}\\
&  &  & \textit{ā-bà} & {‘big lake’}\\
{\textsc{dim}} & \textit{ī-} & \textit{ɛ} & \textit{ī-kɔk} & {‘tiny calabash’}\\
&  &  & \textit{ī-gʷá} & {‘tiny (piece of) grass’}\\
&  &  & \textit{ī-ràndí} & {‘thread’}\\
\lspbottomrule
\end{tabularx}
==================================================

\subsection{{3.4 Kambari}}
\subsubsection{3.4.1 Introduction}

The \ili{Kambari} are perhaps the largest of the \ili{Kainji} subgroups, numerically. Their languages have been studied more extensively than others in the group although much research has never been completely published. \ili{Kambari} (\ili{Kamberi}, Cumbri etc.) is an outsiders’ name, but since there is no overall name for the group it is retained here.  

\begin{tabularx}{\textwidth}{XXXXX}
\lsptoprule
\multicolumn{5}{c}{{\itshape \textbf{{Table} }\textbf{{\stepcounter{Table}{\theTable}}}\textbf{{: The \ili{Kambari} languages}}}

\itshape \textbf{{================================================}}}\\
\textbf{Usual Name} & \textbf{Other Names} & \textbf{One person} & \textbf{People} &  \textbf{Language}\\
\textbf{\ili{Kambari} I} &  &  &  & \\
Agadi & \ili{Kakihum} &  & \textit{aGaɗi} & \textit{tsi}\textit{Gaɗi}\\
Abadi, Evadi & Ibeto &  & \textit{aVaɗi} & \textit{tsi}\textit{Vaɗi}\\
Bangawa &  & \textit{vuBaangi} & \textit{aBaangi} & \textit{ciBaangi}\\
& \ili{Salka} & \textit{sShíngíní or}

\textit{məShíngíní} & \textit{əShingini} & \textit{ciShingini}\\
\textbf{\ili{Kambari} II} &  &  &  & \\
Agaushi & Auna, Wara &  & \textit{aGaushi} & \textit{tsi}\textit{Gaushi}\\
Kimba &  &  & \textit{aKimba} & \textit{tsɨkimba}\\
Ngwunci & Agwara & \textit{maWunci} & \textit{ŋWənci} & \textit{tsuWənci}\\
&  &  &  & \\
\textbf{Cicipu} & \ili{Acipawa} & \textit{Cípù pl.}  & \textit{Àcípù} & \textit{Cìcípù}\\
\lspbottomrule
\end{tabularx}
\-\-\-\-\-\-\-\-\-\-\-\-\-\-\-\-\-\-\-\-\-\-\-\-\-=================================================

Present studies suggest that \ili{Kambari} has two major divisions, usually referred to as \ili{Kambari} I and II. These crudely correspond to east and west, but in some regions the two are territorially intertwined \citep{Blench1982}. \tabref{tab:key:10} shows the common names of the various \ili{Kambari} sub-groups and the correct names of the people and language. The initial consonant of the root is marked with upper case.

\subsubsection{3.4.2 Cicipu}

\ili{Cicipu}, the Western \ili{Acipa} language, was formerly considered part of the \ili{Kamuku} cluster, along with eastern \ili{Acipa}. 

\tabref{tab:key:11} lists the \ili{Cicipu} noun classes and corresponding prefixes:

%%please move \begin{table} just above \begin{tabular
\begin{table}
\caption{Cicipu noun class prefixes}
\label{tab:11}
\end{table}


================================================


\tablehead{\textbf{Class} & \textbf{Noun prefix} & \textbf{Agreement prefix} & \textbf{Example} & \textbf{Gloss}\\
}
\begin{tabularx}{\textwidth}{XXXXX}
\lsptoprule
1 & kA- & kA- & \textit{kà-bárá}

\textit{kɔ-kɔ\'{} ɔ}

\textit{kò-jóo}

\textit{kè-téré}

\textit{kɔ-ɔ[253?]í} & elder

egg

lizard

bone

he-goat\\
2 & A- & A- & \textit{à-bárá}

\textit{ɔ-kɔ\'{} ɔ}

\textit{ò-jóo}

\textit{è-téré} & elders

eggs

lizards

bones\\
3a & i-/y- & i-/y- & \textit{ì-námà}

\textit{yɔ-ɔmɔ} & meat

monkeys\\
3b & ri- &  & \textit{rì-hyã\'{} 'ã\`{} }

\textit{rú-usì} & arrow

rainy season\\
4 & mA- & mA- & \textit{mà-díyá}

\textit{mɔ-tɔɔ}

\textit{mò-kóotó}

\textit{mè-pésé} & hare

chick

kitchen hut

twin\\
5 & N-, mi- & N-, mi- & \textit{ǹ-díyá}

\textit{ǹ-tɔɔ}

\textit{m-pésé}

\textit{mì-nnú} & hares

chicks

twins

birds\\
6 & ti-, tu-, ci-, cu- & ti-, tu- & \textit{tì-sĩ\'{} 'ĩ\`{} }

\textit{tù-mócì}

\textit{cì-lúu}

\textit{cù-kúlú} & hair

friendship

leopard

tortoise\\
7 & u-/w- & u-/w- & \textit{ù-pépí}

\textit{wɔ-ɔvɔɔ} & wind

fear\\
8 & Ø-, C-, v- & Ø-, C-, v- & \textit{Ø-cìccérè}

\textit{c-cɔ'ɔ}

\textit{d-dɔɔ}

\textit{z-zá}

\textit{vɔ-ɔmɔ} & star

sheep

horse

person

monkey\\
9 & ku-/kw- & ku-/kw- & \textit{kù-cígà}

\textit{kwé-etú} & cockerel

medicine\\
\multicolumn{5}{c}{Source: adapted from Mc\citet{Gill2009}}\\
\lspbottomrule
\end{tabularx}
Class 1, 3b, 4, 6, 7 and 9 prefixes occur with singular nouns.

Class 2, 3a and 5 prefixes occur with plural nouns.

Class 8 prefixes can occur with either singular or plural nouns.

Dettweiler \& \citet{Dettweiler1995} present a comparative wordlist for three lects spoken in the towns \ili{Kumbashi}, \ili{Kakihum} and \ili{Karisen}. In this report they point out that ‘Western \ili{Acipa}’ is so different from all the other languages in the group that it would be better to assign it to a separate branch. Stuart McGill (2007, 2009, 2010) proposed that this language has been misclassified and is in fact part of the \ili{Kambari} group. Alternatively, it could have come under extremely strong influence from \ili{Kambari} (not impossible since the two languages are neighbours in \ili{Kakihum}). However, now that a more in-depth description of the grammar and morphology of \ili{Cicipu} is available, this seems less likely. 

The \ili{Cicipu} noun class system is similar to the \ili{Kambari} languages, and so the numbering system used by \citet{Hoffmann1963} and \citet{Crozier1984} for Central \ili{Kambari} is followed. \ili{Cicipu} has a very coherent system of underspecified vowels in noun prefixes, usually copying V\textsubscript{1} of the root. Where C[2081?] is palatalised, the vowel of the prefix is -i.

\figref{fig:key:5} shows \ili{Cicipu} noun-class affix pairings. The dotted lines indicate pairings only rarely attested.

\begin{tabularx}{\textwidth}{XXX}
\lsptoprule
\textbf{Singular} &  & \textbf{Plural}\\
kV- & [Warning: Draw object ignored] & \\
i-/y- & [Warning: Draw object ignored] & \\
[Warning: Draw object ignored][Warning: Draw object ignored]ri- (unpaired) & [Warning: Draw object ignored][Warning: Draw object ignored] & V-\\
mV- & [Warning: Draw object ignored][Warning: Draw object ignored] & N-, mi-\\
ti-, tu-, ci-, cu- &  & \\
u-/w- & [Warning: Draw object ignored] & ø-, C-, v-\\
ø-, C-, v- & [Warning: Draw object ignored] & i-/y-\\
ku-/kw- &  & \\
\multicolumn{3}{c}{{\itshape \textbf{{Figure}}\textbf{ {5: \ili{Cicipu} noun-class affix pairings}}\textbf{ }}

{\itshape {Source:}\textbf{ }{Adapted from Mc\citet{Gill2009}}}

}\\
\lspbottomrule
\end{tabularx}
There appears to be no semantic unity in the noun classes and that even common \ili{Niger-Congo} classes such as Class 6 for mass nouns\footnote{Or else the Class 4 \textit{mV-} prefix has been re-assigned.} and 1/2 for persons are absent. If we count the number of noun classes by the prefix on the noun there are ten morphological classes. However, two of these (3a and 3b) share the same agreement markers and should possibly be merged.

\subsubsection{ 3.4.3 Cishingini (Salka Kambari)}


{The nominal affixing of \ili{Cishingini}, the \ili{Kambari} of \ili{Salka} is described in \citet{Hoffman1963} and \citet{Crozier1984}. \tabref{tab:key:12} shows the} {noun class and concordial prefixes in \ili{Cishingini} as summarised in \citet{Crozier1984}.}


\begin{tabularx}{\textwidth}{XXXXXX}
\lsptoprule
\multicolumn{6}{c}{{\itshape \textbf{{Table} }\textbf{{12:}}\textbf{ {Noun class and concordial prefixes in Cishingini}}}

\itshape \textbf{{================================================}}}\\
\textbf{Class} & \textbf{Number} & \textbf{Prefix} & \textbf{1. Polar Tone} & \textbf{2. Low Tone 2} & \textbf{3. Low Tone 2}\\
1 & sg. & a:- & a:- & à:- & [2CB?]{}- à:{}-[2CA?] \\
2a &  & a- & a- & à- & [2CB?]{}- à{}-[2CA?] \\
2b & pl. & naN- &  &  & \\
3 & sg. +/- pl. & i:- & i:- & ì:- & [2CB?]{}- ì:{}-[2CA?] \\
4 & sg. & mV- & ma- & mà- & [2CB?]{}- mà{}-[2CA?] \\
5 & pl. & N- & N- & [1F9?]{}- & [2CB?]{}- [1F9?]{}- [2CA?] \\
6 & sg. +/- pl. & tsɨ- & tsɨ- & tsɨ- & [2CB?]{}- tsɨ{}-[2CA?] \\
7 & sg. & u:- & u:- & ù:- & [2CB?]{}- ù:{}-[2CA?]\\
&  & C- &  &  & \\
8a & sg. +/- pl. & vɨ- & C- & [2CB?]C- & [2CB?]{}- [2CB?]C{}-[2CA?] \\
&  & li- & vɨ- & vɨ- & [2CB?]{}- vɨ{}-[2CA?] \\
8b & sg. & 0- &  &  & \\
\lspbottomrule
\end{tabularx}
\begin{tabularx}{\textwidth}{X}
\lsptoprule
==================================================

Source: adapted from \citet{Crozier1984}

\figref{fig:key:6} illustrates the pattern of \ili{Cishingini} noun-class affix pairings:

\begin{tabularx}{\textwidth}{XXXX}
\lsptoprule
\multicolumn{4}{c}{}\\
\multicolumn{4}{c}{}\\
\textbf{Class} & \textbf{Singular} & \textbf{Class} & \textbf{Plural}\\
1 & [Warning: Draw object ignored][Warning: Draw object ignored][Warning: Draw object ignored][Warning: Draw object ignored]a:- & 2a,b & a- \\
& [Warning: Draw object ignored] &  & naN- \\
3 & [Warning: Draw object ignored]sg. +/- pl. &  & i:- \\
4 & [Warning: Draw object ignored]mV- & 5 & N- \\
6 & tsɨ- &  & \\
7 & u:- &  & \\
& C- &  & \\
8a & vɨ- &  & \\
& li- &  & \\
8b & 0- &  & \\
\multicolumn{4}{c}{\textbf{Figure 6: \ili{Cishingini} noun-class affix pairings}

Source: Adapted from \citet{Crozier1984}}\\
\lspbottomrule
\end{tabularx}
The classes have not been renumbered, but the unpaired \textit{ma-} and \textit{tsɨ{}-} classes, containing mass nouns, language names and nouns of manner and style, correspond to Proto-\ili{Bantu} classes 6 \textit{*ma}\textit{{}-} and 7 \textit{*ki}\textit{{}-}.  Unlike \ili{Bantu}, \ili{Cishingini} classes 3 and 8a occur as both singular and plural when paired with other classes. The class pair 1/2a includes the majority of nouns. In contrast to \ili{Bantu}, \ili{Cishingini} has only three sets of concordial prefixes. \citet{Crozier1984} analysed noun semantics and showed that the majority of humans are associated with the affix pairs 8/2, while other animates fall into 4/5 and 8a/3. Inanimates are common in 1/2a. The \textit{mV-} prefix Class 4 shows harmony between the prefix and stem vowel and corresponds to a syllabic nasal prefix, class 5. The majority of words in this class pair seem to be animals and plants.

\subsection{{3.5 East Kainji} }
\subsubsection{3.5.1 General}
The \ili{East} \ili{Kainji} languages are a poorly studied group of some 35 languages spoken north and west of the Jos Plateau in Central Nigeria. Compared with the branches of West \ili{Kainji}, which have undergone a wide variety of morphological changes, the \ili{East} \ili{Kainji} languages for which data exist are comparatively similar to one another. Shimizu (1979, 1982a,b) collected numerous short wordlists of \ili{East} Kanji languages and sketched the noun-class prefix pairs that could be extracted from this material. The two languages for which detailed information on nominal prefixing exist are \ili{Map} (Di\citealt{Luzio1972}/3; \citealt{Anderson1980}) and Boze [=Buji] (Blench \& BLC unpublished). \citet{Shimizu1968} is a sketch of the noun-class system of iBunu. Data for many languages consists of fragmentary wordlists, often orthographic with no tones or plurals. Tables 13a and 13b show the languages and internal structure of \ili{East} \ili{Kainji} as far as can be gauged from existing data and \figref{fig:key:8} represents this as a graphic.

\textbf{\tabref{tab:key:13}a:} \textbf{\ili{East} \ili{Kainji} languages and their internal structure}

\textbf{=================================================}

\textbf{Sub-group}

      \textbf{Language name}

\textbf{————————————————————————————-}

Tsamic

  Bishi [=Piti]

  \ili{Ngmgbang}

  Tsam [=Chawai]

Amic

  \ili{Map}

Jos group

Northern

Ningi cluster

  Kudu-Camo (almost extinct)

  Gamo-Ningi (Butu-Ningi†)

Lame cluster

  Gyem

  Shau (extinct)

Lere cluster

  Si (extinct)

  Gana (extinct)

  Takaya (extinct)

Sheni cluster

  \ili{Ziriya} (extinct)

  Kere (extinct)

  Sheni (almost extinct)

===============================================

\textbf{\tabref{tab:key:13}b:} \textbf{\ili{East} \ili{Kainji} languages and their internal structure}

 \textbf{(continuation of 13a)}

\textbf{=================================================}

\textbf{Sub-group}

      \textbf{Language name}

\textbf{————————————————————————————-}

   North-central cluster

  \ili{Zora} (=Cokobo) (moribund)

  Lemoro

  Sanga

  Janji

  $\varepsilon $Boze (=Buji) –iGusu -iZele(=Jere) -iBunu(=Ribina) -iPanawa-  iLoro

  Iguta

  Tunzu (=Duguza)

\ili{Kaduna}

  Gbiri-Niragu

  Shuwa–Zamani

  Vori [=Surubu]

  Kurmi [=Kurama]

  Mala-Ruma

  Bin

  Kono

  Kaivi

  Vono

  Tumi

  Nu [=Kinuku]

  Dungu

==================================================

Some of the names are new, representing languages first recorded in 2016. Former names are given in square brackets, but languages such as \ili{Ngmgbang} will not be found in standard references.

The threat to \ili{East} \ili{Kainji} languages cannot be emphasised too strongly; many have only a few speakers and are rapdily switching to \ili{Hausa}. Others are only now remembered and can be recovered by urgent fieldwork.  \figref{fig:key:7} shows Sarkin Yakubu, the last rememberer of \ili{Ziriya}, interviewed in 2003. No more information can now be recovered about the \ili{Ziriya} language.

  
%%please move the includegraphics inside the {figure} environment
%%\includegraphics[width=\textwidth]{KainjifinalversionforLaTexconversion-img4.jpg}
 

{\itshape \textbf{{\figref{fig:key:7}: Sarkin Yakubu, the last rememberer of \ili{Ziriya}, 2003}}}

A new wordlist of Ti\ili{Zora} was taken in \citealt{March2016}, which can be compared with the one collected by \citet{Shimizu1980} in 1973. During this period Ti\ili{Zora} went from being spoken on a daily basis to one spoken between men over seventy in scattered settlements, under heavy pressure from \ili{Hausa}. As a consequence, although speakers are quite fluent, the noun-class system has undergone systematic collapse. This is described briefly in §3.5.4 to exemplify the rapidity with which morphological change can occur.

{\itshape   
%%please move the includegraphics inside the {figure} environment
%%\includegraphics[width=\textwidth]{KainjifinalversionforLaTexconversion-img5}
 \textbf{ {Figure} }\textbf{{8: Tentative 'tree' of \ili{East} \ili{Kainji} languages}}}

\subsubsectionBoze [=Buji{]}]{3.5.2 ɛBoze [=Buji]}
The ɛBoze language is spoken in some seven villages west and northwest of Jos. The language has been threatened by the spread of \ili{Hausa} but has recently undergone a significant revival. Boze has a rich inventory of prefixes with underspecified vowels and a variety of realisations\footnotemark{}. \tabref{tab:key:14} shows a summary table of the underlying affixes and their allomorphs.

\textbf{\tabref{tab:key:14}: Prefixes and their allomorphs in ɛBoze nouns}

\begin{tabularx}{\textwidth}{XXXXXX}
\lsptoprule

 \textbf{No.} & \multicolumn{2}{c}{ \textbf{Singular}} & \multicolumn{2}{c}{ \textbf{Plural}} & \\
& \textbf{Prefix} & \textbf{Allomorphs} & \textbf{Prefix} & \textbf{Allomorphs} & \textbf{Semantics}\\
1a & O- & ɔ{}-, o-, u- & a- &  & persons\\
1b & [2CB?] VnV- & \begin{tabularx}{\textwidth}{X}
\lsptoprule
ono, OnO-, \\
unu-, uno-\\
\lspbottomrule
\end{tabularx}
 & an[2CA?]\'{V}- & ana-, ano-,

anu- & persons\\
2 & b[2CB?] V- & \begin{tabularx}{\textwidth}{X}
\lsptoprule
be-, bɛ-, \\
bə-, bi- \\
\lspbottomrule
\end{tabularx}
 & i- &  & \begin{tabularx}{\textwidth}{X}
\lsptoprule
animals,\\
people,\\
tools\\
\lspbottomrule
\end{tabularx}
\\
3 & àa- &  & t\`{V}- & \begin{tabularx}{\textwidth}{X}
\lsptoprule
t-, te-, tɛ-,\\
 ti-, tu- \\
\lspbottomrule
\end{tabularx}
 & miscellaneous\\
4 & \begin{tabularx}{\textwidth}{X}
\lsptoprule
ɛ{}-, (ə-),\\
 i-\\
\lspbottomrule
\end{tabularx}
 &  & N- & n-, ŋ-, m- & \begin{tabularx}{\textwidth}{X}
\lsptoprule
abstracts,\\
miscellaneous\\
\lspbottomrule
\end{tabularx}
\\
5 & ø- &  & tV- & \begin{tabularx}{\textwidth}{X}
\lsptoprule
t-, te-, tɛ-,\\
 ti-, tu- \\
\lspbottomrule
\end{tabularx}
 & \begin{tabularx}{\textwidth}{X}
\lsptoprule
insects,\\
 reptiles\\
\lspbottomrule
\end{tabularx}
\\
6a & \begin{tabularx}{\textwidth}{X}
\lsptoprule
màa-, \\
m-, n-\\
\lspbottomrule
\end{tabularx}
 &  & ø- &  & \begin{tabularx}{\textwidth}{X}
\lsptoprule
mass nouns,\\
 abstracts\\
\lspbottomrule
\end{tabularx}
\\
6b & \begin{tabularx}{\textwidth}{X}
\lsptoprule
màa-, \\
m-, n-\\
\lspbottomrule
\end{tabularx}
 &  & i- &  & miscellaneous\\
7 & \`{O}- & ɔ{}-, o- & tV- & \begin{tabularx}{\textwidth}{X}
\lsptoprule
t-, te-, tɛ-,\\
 ti-, tu- \\
\lspbottomrule
\end{tabularx}
 & objects\\
8 & ùu- &  & ti- &  & \begin{tabularx}{\textwidth}{X}
\lsptoprule
plants, foods,\\
 tools\\
\lspbottomrule
\end{tabularx}
\\
9a & r\`{V}- & re-, ri- & a- &  & miscellaneous\\
9b & r\`{V}- & re-, ri- & sV- & se-, sɛ-, si- & miscellaneous\\
10 & \begin{tabularx}{\textwidth}{X}
\lsptoprule
ka-, kɔ-, \\
ku-\\
\lspbottomrule
\end{tabularx}
 &  &  &  & diminutive\\
11 & A- & a-, ə- & a- &  & verbal nouns\\
\multicolumn{6}{c}{}\\
\lspbottomrule
\end{tabularx}
Marking tone on ɛBoze prefixes is something of a hostage to fortune and they are only noted in the table where the evidence is quite strong. Broadly speaking, ɛBoze has a rule where the singular prefix is low and the plural (both prefix and stem) tones are one level higher. However, there are many unexplained irregularities resulting from the influence of the stem tone on the prefix.

The table only represents common pairings, but ɛBoze has numerous examples of unexpected pairings, where the singular/plural gender is only represented by one or two attestations. The vowel in prefixes often harmonises with the stem, although vowels tend to be either front or back; only the allomorphs of \textit{tV-} show the broad range of vowels. It is striking that in Class 2, the class which includes persons, the singular prefix is \textit{bV-}, where it might be expected to mark plural.

ɛBoze shows occasional signs of a feature much more common in Plateau, the ‘intrusive nasal’. In common examples a nasal is inserted between the prefix and the stem vowel in either the singular or the plural:

\begin{itemize}
\item ɛBoze: intrusive nasal\end{itemize}
    sg. \textit{ituma},  ‘work’

    pl. \textit{intúmá} ‘works’\\
\lspbottomrule
\end{tabularx}
\footnotetext{ Work on ɛBoze has been conducted since the early 2000s in conjunction with John Nengel and the Boze Literacy Committee (BLC)}

The likely explanation is that n- was originally a nominalisation prefix applied to a verb stem. When the verbal noun was incorporated into the nominal system, it acquired a new prefix, without the previous one being deleted.

\subsubsection{3.5.3 Map [=Amo]}

The correct name for the language of the \ili{Map} people is Ti\ili{Map}. Its noun classes are described [under the name \ili{Amo}] in (Di\citealt{Luzio1972}/3; \citealt{Anderson1980}). \tabref{tab:key:15} shows Anderson’s (1980: 156) summary of ti\ili{Map} noun classes and concord. \tabref{tab:key:16} shows the ti\ili{Map} nominal prefix pairings and their semantics, where these can be identified. Nasal prefixes in ti\ili{Map} do not appear to be homorganic and do not change in relation to the following consonant.

\begin{tabularx}{\textwidth}{XXXXX}
\lsptoprule
\multicolumn{5}{c}{{\itshape \textbf{{Table} }\textbf{{15: ti\ili{Map} nominal prefixes and concord}}}

\itshape \textbf{{===================================}}}\\
\textbf{Class} & \textbf{Prefix} & \textbf{Map} & \textbf{Gloss} & \textbf{Concord}\\
1 & ù- & ù-là & fire & u\\
2 & à- & à-fà & leaf & a\\
3 & kù- & kù-fà & leaves & ku\\
4 & tè- & tè-là & fires & te\\
5 & lè- & lè-kpì & rat & le\\
6 & [1F9?]{}- & [1F9?]{}-fép & breath & mi\\
7 & kì- & kì-té & place & ki\\
8 & nì- & nì-té & places & ni\\
9 & fè- & fè-ʃù & bee & fe\\
10 & ì- & ì-ʃù & bees & i\\
11 & kà- & kà-vín & goat & ka\\
12 & mà- & mà-ví & big goats & ma\\
\lspbottomrule
\end{tabularx}
        ===================================== 

\begin{tabularx}{\textwidth}{XXX}
\lsptoprule
\multicolumn{3}{c}{{\itshape \textbf{{\tabref{tab:key:16}}}\textbf{{: ti\ili{Map} nominal prefix pairings and semantics}}}

\itshape \textbf{{=========================================}}}\\
\textbf{Class Pair} & \textbf{Prefix} & \textbf{Semantics}\\
1/ 2 & ù-/ à- & mostly humans\\
1/ 4 & ù-/ tè- & unclear\\
3/2 & kù-/ à- & unclear\\
5/2 & tè-/ à- & body parts and diverse\\
5/4 & lè-/ tè- & diverse\\
6/4 & [1F9?]{}-/ tè- & mass nouns\\
7/8 & kì-/ nì- & diverse\\
9/10 & fè-/ ì- & animals, crops and diverse\\
11/8 & kà-/ nì- & domestic animals and diverse\\
1a & ù- & uncountable\\
2a & à- & uncountable\\
4a & tè- & uncountable\\
6a & [1F9?]{}- & uncountable\\
\lspbottomrule
\end{tabularx}
         ============================================

As with other Kaniji languages, ti\ili{Map} has a diminutive and an augmentative. However, in striking contrast to Boze (see above) it has a very static concord system with the prefixes copying the nominal affixes directly. The underspecified vowel in Boze has been lost and ti\ili{Map} prefixes are all static. 


\subsubsection{3.5.4 ìZora}

The ì-\ili{Zora} language was recorded in 1973 by \citet{Shimizu1973} when it still had a functioning system of noun classes. \tabref{tab:key:17} shows the nominal prefix pairings which were functioning at that time.

\begin{tabularx}{\textwidth}{XXX}
\lsptoprule
\multicolumn{3}{c}{{\itshape \textbf{{\tabref{tab:key:17}}}\textbf{{:} }\textbf{{Ì-Zora}}\textbf{ {nominal prefix pairings}}}

\itshape \textbf{{===============================}}}\\
\textbf{Paired classes} & \textbf{sg.} & \textbf{pl.}\\
& ù- & à-, ì-, ʃì-, m-, mV-\\
& lV- & à-, ʃì-\\
& ì- & í-, bì-\\
& rì- & à-, ʃì-, sù-\\
& bì- & ì-, à-\\
& ø- & à-\\
& \`{N}- & ì-\\
Unpaired classes & à & \\
& ì- & \\
& ò- & \\
& ù- & \\
& ø- & \\
& mà- & \\
& mɨ- & \\
\lspbottomrule
\end{tabularx}
    ================================

Unpaired class prefixes associated with uncountable nouns are highly diverse but there is a strong correlation between the \textit{mɨ-} prefix and liquids:

%%1st subexample: change \ea\label{...} to \ea\label{...}\ea; remove \z  
%%further subexamples: change \ea to \ex; remove \z  
%%last subexample: change \z to \z\z 
\langinfo{}{}{i-\ili{Zora}: \textit{mi-} prefix and liquids}\\
\end{itemize}

\begin{tabularx}{\textwidth}{XX}
\lsptoprule
‘blood’ & \textit{mɨ-[272?]ùŋu}\\
‘oil’ & \textit{mɨ-ʃiyà}\\
‘water’ & \textit{mɨ-[272?]f}\\
\lspbottomrule
\end{tabularx}
\citealt{By2016}, the situation had changed radically. Only nineteen individuals now speak ì\ili{Zora}, and they do not live in the same location. To record the language, they had to be brought together (\figref{fig:key:9}). Less than ten per cent of nouns were remembered as having any plural, and the majority of nouns had acquired a ‘default’ singular prefix \textit{ù-} and a plural \textit{à-}. \tabref{tab:key:18} shows a comparison between the forms recorded in 1973 and in 2016.

\begin{tabularx}{\textwidth}{XXX}
\lsptoprule
\multicolumn{3}{c}{{\itshape \textbf{{\tabref{tab:key:18}}}\textbf{{:} }\textbf{{Ì-Zora}}\textbf{ {nominal prefix pairings compared}}}

\itshape \textbf{{=======================================}}}\\
\textbf{Gloss} & \textbf{\citet{Shimizu1973}} & \textbf{\citet{Blench2016}}\\
meat & \textit{[1F9?]{}-námá} pl. ì- & \textit{ùnámá}\\
fish & \textit{ù-kpàlà} pl. ì & \textit{ùkwálá}\\
stone & \textit{rì-kyàlé} pl. à- & \textit{ùkélé}\\
\lspbottomrule
\end{tabularx}
  ========================================

This also illustrates other changes, including the change from labial-velars to labialised velars (kp → kʷ) and the loss of palatalised consonants (k[2B8?] → ke). This likely reflects the pressure from the phonology of \ili{Hausa}, in which all speakers are bilingual and illustrates how rapidly morphological systems can change in particular sociolinguistic contexts.

  
%%please move the includegraphics inside the {figure} environment
%%\includegraphics[width=\textwidth]{KainjifinalversionforLaTexconversion-img6.jpg}
 


\textbf{{\figref{fig:key:9}:}}\textbf{ {Recording the \ili{Zora} language, 2016}}


\subsection{{3.6 The Kamuku group}}
\subsubsection{3.6.1 Introduction}

The \ili{Kamuku} peoples, following Gunn \& \citet{Conant1960} and \citet{Rowlands1962} have conventionally been divided into ‘\ili{Acipa}’ and ‘Ucinda’. The \ili{Acipawa}, correctly the \ili{Acipu}, are linguistically part of the \ili{Kambari} cluster, and are treated in §3.4 The whole \ili{Kamuku} area consists of a complex of related languages, and each lect traces its origin to the individual hills in the Mariga area. A study of \ili{Kamuku} lects has added a great deal to our understanding of these languages but also added many new possible languages \citep{YoderEtAl2008}. Several languages seem to be either extinct or moribund, but their names and locations are known. The two languages for which there are descriptions in some depth are \ili{Hungwəryə} (Hackett \& \citealt{Davey2009}) and \citet{Mort2012}. \figref{fig:key:10} shows the likely subgrouping of the \ili{Kamuku} languages, based on speakers’ impressions: 

[Warning: Draw object ignored][Warning: Draw object ignored][Warning: Draw object ignored][Warning: Draw object ignored][Warning: Draw object ignored][Warning: Draw object ignored][Warning: Draw object ignored][Warning: Draw object ignored][Warning: Draw object ignored][Warning: Draw object ignored][Warning: Draw object ignored][Warning: Draw object ignored][Warning: Draw object ignored]  
%%please move the includegraphics inside the {figure} environment
%%\includegraphics[width=\textwidth]{KainjifinalversionforLaTexconversion-img7}
 


\textbf{{Figure} }\textbf{{10: Subclassification of the \ili{Kamuku} languages}}



\subsubsectionryə [=Ngwoi{]}]{3.6.2 \ili{Hungwəryə} [=Ngwoi]}

The cə\ili{Hungwəryə} language [Ungwai, Ngwoi in older sources] has been described in Hackett \& \citet{Davey2009}. \ili{Hungwəryə} has between 13 and 17 noun class affixes which encode both number and size. The feature marking of size is also reported for t\ili{Hun} (Bendor-Samuel et al. 1973) and its optionality may mean that it is more common than is recognised. The class marker indicates whether the referent is small, normal-sized, or large. Other features distinguish the \ili{Hungwəryə} system from its neighbours, including leftwards nasal harmony of the prefixes. Where the stem vowel is nasalised, this feature spreads to the prefix vowel. In many plural prefixes the vowel has been lost and a C- prefix now abuts the stem directly, a typological change which has developed and been generalised in the Northwest \ili{Kainji} languages.

\ili{Hungwəryə} is characterised by extensive allomorphy of its prefixes. \tabref{tab:key:19} shows the singular and plural class markers, re-arranged from the data in Hackett \& \citet{Davey2009}. To match the mass noun affix in class 6 and the person class 1, as well as merging classes where the prefixes seem to be allomorphs, I have been obliged to re-assign their numbering. 

\tablehead{\multicolumn{5}{c}{{\itshape \textbf{{\tabref{tab:key:19}a:}}\textbf{ {Hungwəryə}}\textbf{ {singular and plural class markers}}}

\itshape \textbf{{=================================================}}}\\
}
\begin{tabularx}{\textwidth}{XXXXX}
\lsptoprule
\textbf{Number} & \textbf{Singular} & \textbf{Plural} & \textbf{Example} & \textbf{Gloss}\\
1a & bu-

bo- & ə{}-

a- & \textit{bú-ləɡəsə, ə-ləɡəsə}

\textit{bó-mátã\`{} , á-mátã\`{} } & boy

father-in-law\\
1b & bʷ- & s- & \textit{bʷ-áːrʲè, s-áːrʲè} & husband\\
1c & bi-

bʲẽ- & i-

ẽ- & \textit{bí-tʃítʃí, í-tʃítʃí}

\textit{bʲẽ\`{} -ɾʲẽ\'{} , ẽ\`{} -ɾʲẽ\'{} } & caterpillar

mouse\\
2a & i-

e-

ʔɛ{}-

j- & mu-

mo- 

mɔ- 

mʷ- & \textit{í-pəpì, mú-pəpì} 

\textit{é-káŋɡàzà, mó-káŋɡàzà}

\textit{ʔẽ\'{} -hɔ\'{} , mɔ\'{} -hɔ\'{} } 

\textit{j-ã\v{} rɔmà, mʷ-ã\v{} rɔmà} & bat

girl

day

chick\\
2b & i- & h-ː \textsuperscript{1} & \textit{í-jɛlà, h-ɛːlà} & tooth\\
3 & ə{}-

a-

ə{}-/a- & sə- 

sə-

tʃə- & \textit{ə{}-ɡúbə, sə-ɡúbə} 

\textit{á-tābɔ, sə-tābɔ}

\textit{á-mʷɔnʲé, ʧə-mʷɔnʲé} & hawk 

spoon

hemp leaf\\
\lspbottomrule
\end{tabularx}
================================================

\tablehead{\multicolumn{5}{c}{{\itshape \textbf{{\tabref{tab:key:19}b: Hungwəryə}}\textbf{ {singular and plural class markers} }}

{\itshape \textbf{ {(continuation of \tabref{tab:key:19}a)}}}

\itshape \textbf{{=================================================}}}\\
}
\begin{tabularx}{\textwidth}{XXXXX}
\lsptoprule
\textbf{Number} & \textbf{Singular} & \textbf{Plural} & \textbf{Example} & \textbf{Gloss}\\
4a & u-

o-

ʔũ{}-

ʔɔ\textsuperscript{3}

w- & hə-

ha-

hə-

ha-

h- & \textit{ú-kʷəgəː, hə-kʷəgəː}

\textit{ó-bʷɔmbá, há-bʷɔmbá}

\textit{ʔũ\'{} -wə\'{} , hə\'{} -wə\'{} } 

\textit{ʔɔ{}-tá, há-tá}

\textit{w-ələmí, h-ələmí} & chameleon

leaf

water monitor lizard

bow

teacher \\
4b & w- & s- & \textit{w-ã\'{} rɔmà, s-ã\'{} rɔmà} & chicken\\
5a & ø- & sə- & \textit{{}-wâː, sə-wâː} & arm\\
5b & ø- & i- & \textit{{}-bʷɔná, í-bʷɔná} & leg\\
5c & ø- & ha- & \textit{{}-bʲát[33C?]ɔ, há-bʲát[33C?]ɔ} & medicine\\
6 & m- &  & \textit{m-ĩ\'{} jə\'{} }

\textit{m-əhūt[33C?]ù} 

\textit{m-àɾʲé}

\textit{m-úhʲúwə} 

\textit{m-ɔnʲégʷà} & water

burning embers

food

smoke

meat\\
7 & ʧi- &  & \textit{tʃí-lã\={} põ\`{} } & shirt\\
8a & ka-

ka-

kə- &  & \textit{ká-tʃɛbà}

\textit{ká-tābɔ}

\textit{kə-zəgí} & mousetrap (karaku)

medium spoon

small loud drum\\
8b & kə- & \textbf{sə-} & \textit{kə-gúbə} & medium hawk\\
8c & ki- & mu- & \textit{kí-pəpì} & small bat\\
8d & ku- & \textbf{hə-} & \textit{kú-kʷəgəː} & large chameleon\\
\multicolumn{5}{c}{==================================================Restructured from Davey \& \citet{Hackett2009}}\\
\lspbottomrule
\end{tabularx}
In some cases, what must have been a high back vowel in the prefix has now become labialisation. Thus:

%%1st subexample: change \ea\label{...} to \ea\label{...}\ea; remove \z  
%%further subexamples: change \ea to \ex; remove \z  
%%last subexample: change \z to \z\z 
\langinfo{}{}{High back prefix vowel becoming labialisation}\\
\end{itemize}

\begin{tabularx}{\textwidth}{XXXX}
\lsptoprule
\textbf{Class pair} & \textbf{sg.} & \textbf{pl.} & \textbf{Gloss}\\
y-/mʷ- & \textit{y-ã\v{} rɔmà} & \textit{mʷ-ã\v{} rɔmà} & ‘chick’\\
\lspbottomrule
\end{tabularx}
Notable features are the unpaired class 7, which has few members and the prefix marking size in class 8. In other languages \textit{kV-} is always a diminutive, but in \ili{Hungwəryə} there appears to be a relationship between vowel quality and size. Where the \textit{{}-V-} is back, a larger size of the referent is marked, while central and front vowels seem to denote small and medium referents. \tabref{tab:key:20} presents hypothesised abstract underlying forms for the allomorphs of singular and plural prefxes.

\begin{tabularx}{\textwidth}{X}
\lsptoprule
{\itshape \textbf{{Table} }\textbf{{20: \ili{Hungwəryə} underlying nominal affix pairings}}}

\itshape \textbf{{==============================================}}\\
\lspbottomrule
\end{tabularx}
\begin{tabularx}{\textwidth}{XXXX}
\lsptoprule
\textbf{Underlying} & \textbf{Singular allomorphs} & \textbf{Underlying} & \textbf{Plural allomorphs}\\
A- & a-, ə-, ø- & S- & sə-, ʧə-\\
I- & i-, e-, ʔɛ-, y- & MU- & mu-, mo-, mɔ-, mʷ-\\
U- & u-, o-, ʔũ{}-, ʔɔ-, w-, ø- & hV- & hə-, ha-, hə-, h-\\
U- & w- & S- & s-\\
ø- & ø- & I- & i-\\
bU- & bu-, bo- & A- & ə{}-, a-\\
bU- & bʷ- & S- & s-\\
\lspbottomrule
\end{tabularx}
  ===============================================

Nasalisation, although phonemic, is not treated as a feature of the underlying form. There is no trace of the nasal classes characteristic of \ili{Bantu} and Bantoid.

\subsubsection{ 3.6.3 The Kamuku complex}

{The following discussion is based on the description of \ili{Cinda} in \citet{Mort2012}. \ili{Cinda} noun-classes are defined by their agreement markers, shown in \tabref{tab:key:21} but renumbered to represent allomorphy and to align the mass noun prefix with Class 6. I have entered the semantics based on lexical evidence from wordlists. The affixes are grouped according to whether} they are used for singular, plural, uncountable and as derivational prefixes.

\textbf{\tabref{tab:key:21}:} \textbf{\ili{Cinda} noun class and agreement markers}

\textbf{=================================================}

\begin{tabularx}{\textwidth}{XXXXXXX}
\lsptoprule
 \textbf{No.} & \textbf{sg.} & \textbf{Allomorphs} & \textbf{No.} & \textbf{pl.} & \textbf{Allomorphs} & \textbf{Semantics}\\
 1 & ʔA- & a-, ɨ-, ɨ:- & 9 &  ʃE- & ʃe-, ʃɛ-, ʃi-, ʃi:- & Miscellaneous but includes numerous animals\\
 2 & E- & ɛ{}-, ɛ:-, i-, i:- & 10 &  mO- & mo-, mo:-, mɔ-, mu-, mu:- & Miscellaneous\\
 3 & O- & o-, o:-, ɔ-, 

u-, u:- &  &  &  & Plants and animals\\
 4 & bE- & bɛ-, bi- & 11 &  E- & ɛ{}-, i- & Plants and animals\\
 5 & bO- & bu-, bo-, bɔ-, bʷ- & 12 &  A- & a-, a:-, ə-, ɨ-, ø- & Persons\\
 6 & mA- & ma-, mɨ- &  &  &  & Mass, uncountable\\
 7 & tV- & tɛ-, tɔ-, tu- &  &  &  & Miscellaneous but includes body parts\\
 8a & kA- & ka-, kɨ- &  &  &  & Rare\\
 8b & kE- & kɛ-, ki- &  &  &  & Rare\\
 8c & kO- & kɔ-, ku-, kʷ- &  &  &  & Rare\\
\lspbottomrule
\end{tabularx}
{==================================================}

{Class markers harmonise for height with the first vowel of the root or word where they are prefixed. There may be an additional small class similar to class 1, containing singular nouns with} {\textit{ʔ}}\textit{A-} agreement markers, but with \textit{A-} class markers on the noun. However, there is some variability between speakers, and even for the same speaker.{} 

{There is a loose semantic basis for grouping noun roots into classes and genders. Class 6 contains non-count} nouns, such as \textit{mɨ-ní} ‘water’, \textit{mà-nɛbɛ} ‘oil’. {The gender 5/12 is used almost solely for people. Classes 2, 4, 7 and 8, forming} genders 2/8, 4/8 and 4/7 are broadly associated with smaller items, although some larger things are also included, for example ‘cows’ \textit{bɛ-ná}  \textit{ɛ{}-ná} ( gender 4/7). {Class 7} {\textit{tU-}}{ is a derivational prefix commonly attached to a verb to create a noun, but can also be attached to a noun to derive another noun. The resultant noun behaves like othe}r nouns, with the class marker \textit{tU-}. This class is occasionally used for uncountable nouns which have no obvious derivation from a verb or another noun, like \textit{tɛgá} ‘porridge’.{} {Classes 8a-c are rare, with only a total of eight examples recorded to date. The most common of these is} {\textit{k}}\textit{ɔ{}-[270?]ágɔ} ‘food’ {which probably derives from} \textit{[270?]a} ‘to eat’.

\subsection{{3.7 The Shiroro languages}}

The \ili{Shiroro} group consists of four languages, usually known as \ili{Rin}, \ili{Fungwa}, \ili{Baushi} and \ili{Gurmana}\footnote{For reasons that are unclear (perhaps typographical error?), \citet{Gerhardt1989} placed \ili{Fungwa} and \ili{Rin} with \ili{Kamuku} in opposition to \ili{Baushi} and \ili{Gurmana}. The present group was proposed and provided with some justification in \citet{Blench1988} and has been confirmed by more detailed work (Dettweiler and \citealt{Dettweiler1995}; see especially their footnote 11).}. \ili{Baushi} can be considered as language cluster with six members. The name proposed here is based on the proximity to \ili{Shiroro} lake. The \ili{Shiroro} languages have previously been treated as part of the \ili{Kamuku} cluster, but there is no evidence for this and here they are treated as an independent branch of \ili{Kainji}. The \ili{Rin} (=R\~\i, formerly Pongu) language was surveyed by Dettweiler and \citet{Dettweiler1992} and MacDonell \& \citet{Smith2004} have circulated a phonology and grammar of \ili{Rin}. For the other languages there is only wordlist data. The \ili{Rin} system of nominal prefixes is quite reduced, with a bV- singular prefix predominant, and several class pairings with a zero singular prefix. Unlike many other \ili{Kainji} languages, the correspondences with \ili{Niger-Congo} classes have been somewhat better preserved. \figref{fig:key:11} shows the likely subgrouping of the \ili{Shiroro} languages;

[Warning: Draw object ignored][Warning: Draw object ignored][Warning: Draw object ignored][Warning: Draw object ignored][Warning: Draw object ignored][Warning: Draw object ignored][Warning: Draw object ignored][Warning: Draw object ignored][Warning: Draw object ignored][Warning: Draw object ignored][Warning: Draw object ignored][Warning: Draw object ignored][Warning: Draw object ignored]


\ili{Fungwa}



\ili{Gurmana}



\ili{Rin}



\ili{Baushi}



Hɨpɨna



Ndəkə



Samburu ?



Wãyã



Rubu



Mɨɨn


% [Warning: Draw object ignored]   
Proto-\ili{Shiroro}

\textbf{\figref{fig:key:11}: The \ili{Shiroro} languages}

\tabref{tab:key:22} is a summary table of \ili{Rin} nominal affixes, re-arranged from the data in MacDonell \& \citet{Smith2004} with a column listing the allomorphs of the singular prefixes which are reflected in different affix pairings. Tone is not marked in the source.

\begin{tabularx}{\textwidth}{XXXXX}
\lsptoprule
\multicolumn{5}{c}{\itshape \textbf{{\tabref{tab:key:22}:}}\textbf{ {\ili{Rin} nominal affixes}}}\\
\textbf{No.} & \textbf{Singular} & \textbf{Allomorphs} & \textbf{Plural} & \textbf{Semantics}\\
1 & bV- & bi- & N- &  animal, object\\
&  & bi- & i- &  animal\\
&  & bu- & a- &  human\\
& Ø & Ø & a- &  object\\
&  & Ø & a- &  animal\\
&  & Ø & N- &  animal\\
& a- & a- & su- &  animal\\
6 & ma- &  & — &  mass, non-count\\
& ri- & ɾi- & a- &  object\\
& tV- & tə- & — &  mass, non-count\\
& u- &  & N- &  object\\
&  &  & a- &  object\\
\lspbottomrule
\end{tabularx}
\ili{Rin} has retained the \ili{Niger-Congo} Class 6 prefix for liquids and mass nouns and a possible trace of the persons class (1/2). The predominance of the \textit{a-} plural affix recalls the \ili{Kambari} languages and the \textit{tV-} prefix for mass nouns resembles the \textit{tsV-} prefix also found in \ili{Kambari}.

\subsection{{3.8 The Basa cluster}}

The \ili{Basa} languages are spread across a wide area of central Nigeria, scattered among unrelated languages. This is probably the result of nineteenth century slave raiding. In many of the communities in the northwest, the language is moribund or only remembered by elderly speakers. As far as the fragmentary evidence goes, the \ili{Basa} languages are all closely related, with \tabref{tab:key:23} showing a cluster rather than a set of distinct languages. 

\textbf{\tabref{tab:key:23}: The \ili{Basa} languages}

\textbf{================================================}

     \textbf{\ili{Basa} language cluster}

\_\_\_\_\_\_\_\_\_\_\_\_\_\_\_\_\_\_\_\_\_\_\_\_\_\_\_\_\_\_\_\_\_\_\_\_\_\_\_\_\_\_\_\_\_\_\_\_\_\_\_\_

\ili{Basa}-Kontagora (†)

 \ili{Basa}-Gumna (†)

 Kɔrɔmba (formerly \ili{Basa}-\ili{Gurmana})

 \ili{Basa}-Gurara

 \ili{Basa}-Kwali

 \ili{Basa}-\ili{Benue} (formerly \ili{Bassa}-Kwomu)

 \ili{Basa}-Makurdi

=================================================

The \ili{Basa} languages probably fall into seven groups as shown in \figref{fig:key:12}. The subclassification is based on impressions of lexical differences. 


  
%%please move the includegraphics inside the {figure} environment
%%\includegraphics[width=\textwidth]{KainjifinalversionforLaTexconversion-img8}
 \textbf{ {Figure} }\textbf{{12: The subclassification of the \ili{Basa} languages}}


Fieldwork was conducted on \ili{Basa}-\ili{Benue} in conjunction with Paul Imoh and the late Robert Hyslop in 1984; for other languages in the cluster only wordlist data exists. \citet{Imoh2002} is a preliminary phonology and morphology of \ili{Basa} which differs somewhat from the account presented here. Tones were not marked when this data was collected.

One of the early names for West \ili{Kainji} was ‘\ili{Basa}-\ili{Kamuku}’, mainly because these were the languages for which data was accessible. However, the impression has remained that \ili{Basa} and \ili{Kamuku} have a privileged relationship, for which there seems to be no evidence. \ili{Basa}-\ili{Benue} noun prefixes are unique among \ili{Kainji} languages in having developed three-way number marking, although some traces of this exist in the \ili{Kambari} cluster. As with Nilo-Saharan, the middle term or first plural applies to the item in general, or in an undefined quantity. In other words it is a non-countable plural where an additional plural has developed. Only a minority of nouns have three-term number marking, and those with a nasal prefix on the primary plural seem to be excluded. The second plural is countable and applies to groups or clusters of the item and, but can also be used as a distributive. The singular is a singulative, i.e. it implies a unique item. So:

\ea\label{ex:key:}
\langinfo{}{}{(5) \ili{Basa}-\ili{Benue} noun prefixes: three-way number marking}\\
       a. Chili pepper\\
\glt   \textit{i-kpekpe} ‘single chili pepper’
\z

\ea\label{ex:key:}
\langinfo{}{}{ \textit{o-kpokpo}  ‘chili pepper(s), generic’}\\
\gll   ʃi-kpokpo  {‘piles of chili peppers’}\\
\glt b. Broom
\z

\ea\label{ex:key:}
\langinfo{}{}{    \textit{bi-ʃoʃo} ‘single broom’}\\
\ea\label{ex:key:}
\langinfo{}{}{ \textit{i-ʃeʃe} ‘broom(s), generic’ }\\
\ea\label{ex:key:}
\langinfo{}{}{    \textit{n-ʃoʃo} ‘groups of brooms’}\\
The prefixes are either V- or CV-. \ili{Basa} permits a large number of nominal affixes and pairings, probably the consequence of the breakdown and re-analysis of the three-way number marking.

\begin{tabularx}{\textwidth}{XXXX}
\lsptoprule
\multicolumn{4}{c}{{\itshape \textbf{{\tabref{tab:key:24}:}}\textbf{ {a/E vowel alternations in \ili{Basa} number marking}}}

\itshape \textbf{{========================================}}}\\
\textbf{Gloss} & \textbf{Singular} & \textbf{Middle} & \textbf{Plural}\\
grass sp. &  \textit{bu-baza} &  \textit{tu-baza} &  \textit{i-bɛzɛ}\\
Horse &  \textit{bu-dakwa} & \textit{—} &  \textit{i-dɛkwɛ}\\
Hand &  \textit{u-ala} & \textit{—} &  \textit{i-ɛlɛ}\\
Bow &  \textit{u-ta} & \textit{—} &  \textit{i-tɛ}\\
&  &  & \\
Dog &  \textit{u-wɛwɛ} & \textit{—} &  \textit{ʃi-wawa}\\
Termite &  \textit{u-da} & \textit{—} &  \textit{i-de}\\
antelope sp. &  \textit{bɛ-ʃɛmba} & \textit{—} &  \textit{i-ʃimbɛ}\\
Tree &  \textit{u-’wu’wu} & \textit{—} &  \textit{i-’wɛ’wɛ}\\
\lspbottomrule
\end{tabularx}
=================================================

In addition, the -V in nominal prefixes in \ili{Basa} can affect both the C[2081?] of the stem and the stem vowels. \tabref{tab:key:24} shows the far from transparent relationship between a singular stem \textit{{}-a-} and \textit{{}-E-} in the plural.

However, as the second set shows, there are a variety of isolated possibilities for vowel change which do not seem to be predictable. A less common alternation is \textit{o$\leftrightarrow $(w)e}. The absence of phonological conditioning is shown by the pair of words in (25):


\textbf{{\tabref{tab:key:25}: non-phonologically motivated vowel alternations} }



\textbf{{=============================================}}


\begin{tabularx}{\textwidth}{XXX}
\lsptoprule
\textbf{Gloss} & \textbf{Singular} & \textbf{Plural}\\
‘chick’ &  \textit{bi-yoyo} &  \textit{o-yoyo}\\
‘goat’ &  \textit{bi-yoyo} &  \textit{i-yweywe}\\
and: &  & \\
‘rope’ &  \textit{u-hwohwo} &  \textit{i-hwehwe}\\
‘bicycle’ &  \textit{i-cece} &  \textit{n-coco}\\
\lspbottomrule
\end{tabularx}
  ============================================

Where the first syllable of the stem begins with either a palatal or a labial, the -V of the prefix can act both to delete the semi-vowel and sometimes cause changes in the vowel. Thus (6):

\ea\label{ex:key:}
\langinfo{}{}{ (6) \ili{Basa}-\ili{Benue}: prefixes on stems with initial palatal or labial consonants} \\
\gll     \textbf{{Gloss    Singular  Plural}}\\
\begin{tabularx}{\textwidth}{XXX}
\lsptoprule
‘guinea-fowl’ &  \textit{u-yogwu} &  \textit{ʃa-igwu}\\
‘child’ &  \textit{yɛ-u} &  \textit{myà-wɔ}\\
\lspbottomrule
\end{tabularx}
Other examples of vowel mutation are more difficult to explain. \ili{Basa} can also manifest intrusive nasals in the plural stem as in (7), a phenomenon more common in Plateau languages. 

\ea\label{ex:key:}
\langinfo{}{}{(7) \ili{Basa}-\ili{Benue}: intrusive nasals in plural stems} \\
\begin{tabularx}{\textwidth}{XXX}
\lsptoprule
‘large bowl’ &  \textit{u-gba[2A4?]o} &  \textit{o-gbon[2A4?]o}\\
‘canoe, vehicle, boat’ &  \textit{ù-hantɔ} &  \textit{i-hɛntu}\\
‘road’ &  \textit{ù-hwãna} &  \textit{ì-hwɛnɛ}\\
\lspbottomrule
\end{tabularx}
The vowel in some CV- prefixes is underspecified and can change in order to harmonise with the ‘underlying’ second vowel in the noun stem. This is most marked in the case of the \textit{ʃV-} plural prefix. The prefix allomorphs are shown in \tabref{tab:key:25}: 

\begin{tabularx}{\textwidth}{XXX}
\lsptoprule
\multicolumn{3}{c}{\itshape \textbf{{\tabref{tab:key:25}:}}\textbf{ {Prefix and stem harmony in Basa}}}\\
\multicolumn{3}{c}{\itshape \textbf{{==================================}}}\\
& \textbf{Example} & \textbf{Gloss}\\
ʃa- & \textit{ʃa-luma} & hens\\
ʃɛ{}- & \textit{ʃɛ{}-mbɛ} & grasses\\
ʃe- & \textit{ʃe-je[292?]e} & rays\\
ʃi- & \textit{ʃi-lala} & pestles\\
ʃo- & \textit{ʃo-rubo} & francolins\\
\lspbottomrule
\end{tabularx}
        ====================================

There are no cases of \textit{ʃɔ-} and \textit{ʃu-} at present recorded. The \textit{ʃi-} prefix is most common and can apparently occur with any stem vowel, synchronically. This prefix is widespread in related languages and is probably the underlying form inherited from proto-\ili{Basa}. 

A similar plural prefix \textit{tV}{}- has a more limited range of variants. In this case, the \textit{tu-} form is dominant and again this corresponds to a similar prefix in other languages. The exact logic of the prefix vowels remains to be understood.

\ea\label{ex:key:}
\langinfo{}{}{(8) \ili{Basa}-\ili{Benue}: variation with plural prefix \textit{tV-}}\\
\begin{tabularx}{\textwidth}{XXX}
\lsptoprule
\textbf{Form} & \textbf{Example} & \textbf{Gloss}\\
tɛ- &  \textit{tɛ-jɛrɛka} &  stone wedges\\
ti- &  \textit{ti-kpeku} &  hills\\
tu- &  \textit{tu-zogu} &  bush-melons\\
&  & \\
\lspbottomrule
\end{tabularx}
With a few exceptions, words that have singular prefixes beginning in m-, s-, or t- do not form plurals. These affixes may originally have been applied only to uncountable nouns, such as liquids, but presently they seem to have no semantic unity and may have been generalised by analogy to countable nouns. 

\textbf{4. Conclusions}

The \ili{Kainji} languages demonstrate clear evidence for an original system of noun classes defined by nominal affixing and alliterative concord. However, the potential to reconstruct a proto-system is limited by the sparsity or absence of descriptions for many subgroups. Beyond that, however, the affix systems seem to show remarkable diversity, with only limited correspondences between branches. The observations of Mc\citet{Gill2009} on the noun-class system of \ili{Cicipu} could apply to much of \ili{Kainji}:

\begin{quote}
{It will be clear to anyone familiar with the \ili{Benue}-Congo or \ili{Bantu} literature that, superficially at least, the \ili{Cicipu} system is very different to both the suggested Proto-\ili{Benue}-Congo (PBC) reconstructions (e.g. De \citealt{Wolf1971}) and the present-day \ili{Bantu} systems. There are fewer classes, and the forms of the original PBC prefixes have in some cases changed beyond recognition. Nevertheless, there are also striking similarities, in particular the robust and ubiquitous alliterative agreement ... Much the same could be said about the other \ili{Kainji} languages for which we have data – the prefixes and class pairings are much changed from PBC, but the mechanics of the agreement system have been retained.}
\end{quote}

The systems have eroded and been renewed in a variety of ways in different subgroups, and in particular some languages seem to have evolved highly divergent ‘new’ prefixes. One of the distinctive features of \ili{Kainji} languages is the apparently random way singular and plural affixes shift their number marking. Thus \ili{Reshe} has a \ili{Bantu}-like u-/bu- (1/2) person marking affix pairing. \ili{Shuba} has bV- marking plural persons but the singular prefix is the unfamiliar do-. In \ili{Hungwəryə} the singular class marker for persons is bu- now paired with a plural ə-. A preliminary hypothesis to explain this would be that the three-way number marking found in \ili{Basa} languages was formerly more widespread, and as the classes collapsed innovative class pairings resulted.

The following generalisations about proto-\ili{Kainji} seem to be supported by the data.

a) Proto-\ili{Kainji} had a rich system of nominal prefixes and alliterative concord. It is possible the affixing originally showed a three-way distinction, still attested in \ili{Basa}.

b) Proto-\ili{Kainji} had the bilabial unpaired affix \textit{mV-} for liquids and other mass nouns attested widely in \ili{Niger-Congo} and usually assigned to Class 6.

c) Proto-\ili{Kainji} had a class pair for persons, perhaps \textit{u-/ba-} which can be treated as cognate with \ili{Bantu} 1/2.

d) Proto-\ili{Kainji} had underspecified vowels in a \textit{kV-} and possibly also \textit{tV-} and \textit{SV-} nominal prefixes, whereby the \textit{{}-V} shows harmony with the stem vowel.

e) Proto-\ili{Kainji} had a diminutive (and perhaps augmentative) affix marker \textit{kV-} (also found in some Plateau languages) which has become homophonous with a separate \textit{kV-} marker.

f) Proto-\ili{Kainji} allowed prefix swapping to indicate characteristics of the noun, marking qualities such as length or personhood.

g) If Proto-\ili{Kainji} had a homorganic plural nasal prefix, the evidence is now hard to discern, since it is only clearly attested in some \ili{East} \ili{Kainji} and \ili{Kambari} cluster languages.

Once languages where the affixes are eroded are discounted, there remains the problem of whether \ili{Reshe} can be said to be part of the system. There are almost no correspondences between the \ili{Reshe} system and the other branches, suggesting it is a renewed system of unknown origin. Understanding \ili{Kainji} should be a priority goal in the light of its importance in the reconstruction of Proto-\ili{East}-\ili{Benue}-Congo, but this will require a great deal more data collection and analysis.

\section{Acknowledgments}

Individual assistants and informants for particular languages are listed in the Appendix and acknowledged in particular sections, as well as in the Appendix. However, I would like to thank a large number of people for discussion, access to unpublished materials, and help in the field. These include Steve and Sonia Dettweiler, David Heath, David \& Liz Crozier, Robert (†) and Joyce Hyslop, John Nengel, Barau Kato, James McDonell, Gareth and Katherine Mort and Stuart McGill. Larry Hyman acted as a reviewer and made many helpful comments, as well as suggesting comparisons and extensions of these ideas.

\section{Abbreviations and conventions}

\begin{tabularx}{\textwidth}{lQ}
\lsptoprule
A & any central vowel\\
C & consonant\\
E & any mid-front vowel\\
N & any nasal\\
O & any mid-back vowel\\
S & s or ʃ\\
V & Vowel\\
\lspbottomrule
\end{tabularx}

\section{Appendix A. Data sources for the Kainji Languages}

\tabref{tab:key:26} shows the principal unpublished data sources for the \ili{Kainji} languages.

\textbf{\tabref{tab:key:26}a:} \textbf{Unpublished sources for \ili{Kainji} Languages}

\textbf{Branch  Subgroup  Language    Sources}

———————————————————————————–

\begin{tabularx}{\textwidth}{XXXX}
\lsptoprule
\ili{Reshe} &  & \ili{Reshe} & Harris, mss., Agamalafiya, Blench,  Dettweilers\\
&  &  & \\
Upper Niger & Rerang & \ili{Rop} & Meek, Blench, McGill\\
& \ili{Laru} & \ili{Shuba} & Blench, McGill\\
&  & \ili{Shen} & Meek, Sterk, Blench, McGill\\
&  &  & \\
Northwest & Lela & cLela \ili{Zuru} & Hoffmann, Rikoto,  Dettweilers, Regnier, Blench\\
&  & cLela Ribah & Blench\\
& \ili{Hun} & t\ili{Hun} & Skitch \& Cressman, Regnier,  Dettweilers, Heath\\
&  & s\ili{Saare} & Regnier,  Dettweilers, Blench\\
\lspbottomrule
\end{tabularx}
\textbf{\tabref{tab:key:26}b: Unpublished sources for \ili{Kainji} Languages}

 \textbf{(continuation of 26a)}

\textbf{Branch  Subgroup  Language    Sources}

———————————————————————————–

\begin{tabularx}{\textwidth}{XXXX} & Gwamhi & Gwamhyə & Regnier, Rowlands, Blench, McGill\\
\lsptoprule
&  & Wurə & Regnier, Blench, McGill\\
&  & \ili{Mba} & Blench, McGill\\
& ut-Ma’in & \ili{Kag} & Blench, Regnier\\
&  & Fer & Regnier\\
&  & Jiir & Regnier\\
&  & Kər & Regnier\\
&  & Koor & None\\
&  & \ili{Ror} & Smith, Regnier\\
&  & Us & Regnier\\
&  & Zuksun & Rowlands\\
& ? & Damakawa & McGill\\
\ili{Kambari} &  & CiShingini & Hoffmann, Crozier, Stark et al.\\
&  & Tsivaɗi & Lovelace, Blench\\
&  & Baangi & Blench\\
&  & Tsɨkimba & Blench, Stark et al.\\
&  & Agwara & Mierau, Stark et al.\\
&  & \ili{Cicipu} & McGill,  Dettweilers\\
\ili{East} &  & Gbiri & Wenger\\
&  & Boze & Blench\\
&  & Sheni & Blench\\
&  & Moro & Blench\\
\lspbottomrule
\end{tabularx}
\textbf{\tabref{tab:key:26}c: Unpublished sources for \ili{Kainji} Languages}

 \textbf{(continuation of 26b)}

\textbf{Branch  Subgroup  Language    Sources}

———————————————————————————–

\begin{tabularx}{\textwidth}{XXXX}
\lsptoprule
\ili{Kamuku} &  & Shama & Regnier, Yoder et al., McGill\\
&  & Rogo-Shyabe & Regnier, Yoder et al., Blench, McGill\\
&  & Səgəmuk & Regnier\\
&  & \ili{Cinda} & Regnier, Blench, Mort, Yoder et al.\\
&  & Regi & Regnier, Omanor, Yoder et al.\\
&  & Kuki & Regnier, Blench, Yoder et al.\\
&  & Zubazuba & Yoder et al., Blench, McGill\\
&  & Kagare & Yoder et al.\\
&  & Hungwǝryǝ & Davey\\
\todo[inline]{check Unicode for schwa}\\
\ili{Shiroro} &  & \ili{Fungwa} & Blench, McGill\\
&  & \ili{Rin} & Rowlands, Regnier,  Dettweilers, Blench, MacDonell \& Smith\\
&  & Wəgə & Blench, McGill\\
&  & \ili{Gurmana} & Johnston, Blench, McGill\\
& \ili{Baushi} & Ndəkə & Regnier\\
&  & Hɨpɨna & McGill\\
&  & Rubu & None\\
&  & Mɨɨn & Gimba, Blench\\
&  & Samburu & None\\
&  & Wãyã & Dettweiler\\
\lspbottomrule
\end{tabularx}
\textbf{\tabref{tab:key:26}d: Unpublished sources for \ili{Kainji} Languages}

 \textbf{(continuation of 26c)}

\textbf{Branch  Subgroup  Language    Sources}

———————————————————————————–

\begin{tabularx}{\textwidth}{XXXX}
\lsptoprule
\ili{Basa} &  & \ili{Basa} Kontagora & Rowlands, Blench\\
&  & \ili{Basa}-Gumna & Blench\\
&  & Kɔrɔmba & Blench\\
&  & \ili{Basa}-Gurara & Sterk\\
&  & \ili{Basa}-\ili{Benue} & Blench\\
&  & \ili{Basa}-Makurdi & Blench\\
\lspbottomrule
\end{tabularx}
\begin{verbatim}%%move bib entries to  localbibliography.bib
Anderson, Stephen C. 1980. The noun class system of \ili{Amo}. In: \textit{Noun classes in the Grassfields \ili{Bantu} borderland}. Hyman, Larry (ed.), 155-178. SCOPIL 8. Los Angeles: UCLA.

@@book{Bendor-Samuel1973,
	address = {\textit{Duka sentence, clause and phrase.} Zaria},
	author = {Bendor-Samuel, John  and  Skitch, Donna  and  Cressman, Esther.},
	publisher = {Institute of Linguistics},
	year = {1973}
}

@@article{Bertho1952,
	author = {Bertho, J},
	journal = {Institut français de l'Afrique Noire} \textit{(IFAN)}},
	pages = {259--271},
	title = {Aperçu d’ensemble sur les dialectes de l’ouest de Nigéria \textit{Bull},
	volume = {14},
	year = {1952}
}

@@article{Blench1982,
	author = {Blench, Roger M},
	journal = {\textit{Cambridge Anthropology}, 7},
	pages = {19--30},
	title = {Social structure and the evolution of language boundaries in {Nigeria}},
	volume = {3},
	year = {1982}
}

@@misc{Boettger1967,
	author = {Boettger, E.  and  Boettger, V},
	title = {\textit{Tsureshe Grammar}. {{U}}npublished ms.},
	year = {1967}
}

@@book{Crozier1984,
	address = {\textit{A study in the discourse grammar in \ili{Cishingini}.} Ibadan},
	author = {Crozier, David.},
	publisher = {Department of Linguistics and Nigerian Languages, University of Ibadan. (Ph.D. dissertation)},
	year = {1984}
}

@@book{Crozier1992,
	address = {Dallas},
	author = {Crozier, David  and  Blench, Roger M.},
	publisher = {Summer Institute of Linguistics},
	title = {\textit{An Index of {Nigerian} Languages}. {E}dition 2},
	year = {1992}
}

@@misc{De1971,
	author = {De Wolf, Paul P},
	title = {\textit{The noun class system of Proto-\ili{Benue}-{Congo}}. {T}he Hague, Mouton de Gruyter.},
	year = {1971}
}

Dettweiler, Stephen H. ined. \textit{C’Lela grammar sketch}. Electronic ms.

Dettweiler, Stephen H. \& Dettweiler, Sonia G. 1992. [electronic version dated 2002]. \textit{Level One Sociolinguistic Survey of the Pongu people.} ms., Ilorin. \url{http://www.sil.org/silesr/2002/SILESR2002-040.pdf}

Dettweiler, Stephen H. \& Dettweiler, Sonia G. 1993a. [electronic version dated 2003]. \textit{Level One Sociolinguistic Survey of the Duka (\ili{Hun}-\ili{Saare}) people.} ms., Ilorin. \url{http://www.sil.org/silesr/2003/silesr2003-014.pdf}

Dettweiler, Stephen H. \& Dettweiler, Sonia G. 1993b. [electronic version dated 2002]. \textit{Sociolinguistic survey (level one) of the \ili{Reshe} people}. \url{http://www.sil.org/silesr/2002/SILESR2002-042.pdf}

Dettweiler, Stephen H. \& Dettweiler, Sonia G. 1995. [electronic version dated 2002]. \textit{Sociolinguistic survey (level one) of the \ili{Kamuku} language cluster}. \url{http://www.sil.org/silesr/2003/silesr2003-003.pdf}

@@article{Di1972,
	author = {Di Luzio, \citealt{Aldo},
	journal = {\textit{Afrika und Übersee}},
	pages = {3--61},
	title = {}/3. {P}reliminary description of the \ili{Amo} language},
	volume = {56},
	year = {1972}
}

Gerhardt, \citealt{Ludwig1989}. \ili{Kainji} and \ili{Platoid}. In: \textit{The \ili{Niger-Congo} languages}. Bendor-Samuel, John. (ed.), 359-376. Lanham: Universities Press of America.

@@book{Greenberg1955,
	address = {\textit{Studies in African linguistic classification.} New Haven},
	author = {Greenberg, Joseph H.},
	publisher = {Compass Publishing},
	year = {1955}
}

@@misc{Greenberg1963,
	author = {Greenberg, Joseph H},
	title = {\textit{The Languages of {Africa}.} Mouton, the Hague for Indiana University, Bloomington.},
	year = {1963}
}

@@misc{Gunn1960,
	author = {Gunn, Harold D.  and  Conant, Francis Paine},
	title = {\textit{Peoples of the Middle {Niger} Region of Northern {Niger}ia.} IAI, London.},
	year = {1960}
}

\begin{styleLDLTReferences}
@@misc{Hackett2009,
	author = {Hackett, Chris  and  Davey, Niffer},
	title = {\textit{A phonological sketch of the C’Hungwere  language}. {{E}}lectronic ms.},
	year = {2009}
}
\end{styleLDLTReferences}

\begin{styleLDLTReferences}
@@article{Hoffmann1963,
	author = {Hoffmann, \citealt{Carl},
	journal = {\textit{Journal of African Languages}, 2},
	pages = {160--169},
	title = {}. {T}he noun-class system of {Central} Kambari},
	volume = {2},
	year = {1963}
}
\end{styleLDLTReferences}

\begin{styleLDLTReferences}
Hoffmann, \citealt{Carl1967}. An outline of the \ili{Dakarkari} noun class system and the relation between prefix and suffix noun-class systems. In : \textit{La Classification nominale dans les langues Négro-africaines}. Manessy, Gabriel (ed.), 237-259. Paris: CNRS.
\end{styleLDLTReferences}

@@misc{Hyman1970,
	author = {Hyman Larry M.  and  Magaji, Daniel J},
	title = {\textit{Essentials of \ili{Gwari} grammar}. {I}nstitute of {African} Studies, University of Ibadan, Occasional Publication No. 27.},
	year = {1970}
}

Imoh, Philip \citealt{Manda2002}. \textit{The Phonology and Morphology of \ili{Bassa} Language Spoken in Kogi, Nassarawa, Niger States and the Federal Capital Territory Of Nigeria}. Jos: Department Of Languages and Linguistics, University Of Jos, Nigeria. (B.A. thesis)

Johnston, Harry H. 1919-1922. \textit{A Comparative Study of the \ili{Bantu} and Semi-\ili{Bantu} Languages.} Vols. 1 \& 2. Oxford, Clarendon Press.

@@book{Koelle1854,
	address = {London \& Salisbury},
	author = {Koelle, Sigismund W.},
	publisher = {Church Missionary House},
	title = {\textit{Polyglotta Africana}},
	year = {1854}
}

@@misc{MacDonell2004,
	author = {MacDonell, James  and  Smith, Philip},
	title = {\textit{A Phonological and Grammatical Sketch of the Pongu Language}. {{E}}lectronic ms.},
	year = {2004}
}

\begin{styleLDLTReferences}
@@article{McGill2007,
	author = {McGill, \citealt{Stuart},
	journal = {\textit{Journal of West African Languages,}},
	number = {2},
	pages = {51--90},
	title = {}. {T}he \ili{Cicipu} noun class system},
	volume = {34},
	year = {2007}
}
\end{styleLDLTReferences}

\begin{styleLDLTReferences}
McGill, \citealt{Stuart2009}. \textit{Gender and person agreement in \ili{Cicipu} discourse}. London: School of Oriental and African Studies. (Ph.D. dissertation) [Available online: \url{http://www.cicipu.org/papers/gender_and_person_agreement_in_cicipu_discourse}\href{http://www.cicipu.org/papers/gender_and_person_agreement_in_cicipu_discourse.pdf}{.pdf}].
\end{styleLDLTReferences}

\begin{styleLDLTReferences}
McGill, \citealt{Stuart2010}. Person and gender: competing agreement paradigms in \ili{Cicipu}. In: Pozdniakov, Konstantin  \& Vydrin, Valentine \& Zheltov, Alexander (eds.), \textit{Proceedings of the Personal Pronouns in \ili{Niger-Congo} Workshop}, 79-87. St. Petersburg: St. Petersburg State University Press. [Available online: \url{http://mandelang.kunstkamera.ru/files/mandelang/mcgill.pdf}].
\end{styleLDLTReferences}

McGill, Stuart \& Blench, Roger M. 2012. Documentation, development, and ideology in the northwestern \ili{Kainji} languages. In: Austin, Peter K. \& McGill, Stuart (eds.) \textit{Language Documentation and Description,} 11: 91-136. London: SOAS.

@@misc{Meek1925,
	author = {Meek, Charles K},
	title = {\textit{The {Northern} Tribes of Nigeria.} Vols. 1 \& 2. {H}umphrey Milford, London for OUP.},
	year = {1925}
}

\begin{styleLDLTReferences}
@@misc{Mort2012,
	author = {Mort, \citealt{Katharine},
	title = {}. \textit{A Phonological Description of the \ili{Cinda} [tʃìndə] dialect of the \ili{Kamuku} language}. {{E}}lectronic ms.},
	year = {2012}
}
\end{styleLDLTReferences}

Paterson, \citealt{Rebecca2012}. The semantics of \ili{Ut-Ma}’in noun classes. In: Blench, Roger M. \& McGill, Stuart (eds.) \textit{Advances in minority language research in Nigeria}. Vol. 1, 239-272. Köln: Rudiger Köppe.

Regnier, Clark D.A. 2003. \textit{Sociolinguistic Survey of the People of \ili{Fakai} District}. ms. SIL International. http://www.sil.org/silesr/2003/silesr2003-021.pdf

@@book{Rikoto2001,
	address = {s.l.},
	author = {Rikoto, Bulu Doro \textit{et al.}},
	publisher = {Lelna Language Development/Translation Project},
	title = {\textit{K’Batksa C’{Lela}-C’Anasara-C’Gana}},
	year = {2001}
}

Rowlands, E.C. 1962. Notes on some class languages of Northern Nigeria. \textit{African Language Studies}, III:71-83.

@@book{Shimizu1968,
	address = {\textit{An outline of the I-búnú noun class system}.(ms.) Ibadan: Ibadan},
	author = {Shimizu, Kiyoshi.},
	publisher = {Dept. of Linguistics and Nigerian Languages. (Diploma dissertation)},
	year = {1968}
}

@@article{Shimizu1979,
	author = {Shimizu, Kiyoshi},
	journal = {\textit{Afrika und Übersee}},
	number = {4},
	pages = {253--271},
	title = {Five wordlists with analyses from the {Northern} Jos group of Plateau languages},
	volume = {62},
	year = {1979}
}

@@article{Shimizu1982,
	author = {Shimizu, Kiyoshi},
	journal = {\textit{Afrika und Übersee}},
	number = {1},
	pages = {97--134},
	title = {Ten more wordlists with analyses from the {Northern} Jos group of Plateau languages},
	volume = {65},
	year = {1982a}
}

@@article{Shimizu1982,
	author = {Shimizu, \citealt{Kiyoshi},
	journal = {\textit{Afrika und Übersee}},
	number = {2},
	pages = {161--210},
	title = {}. {D}ie Nord-Jos-Gruppe der Plateausprachen Nigerias},
	volume = {65},
	year = {1982b}
}

@@book{Temple1922,
	address = {\textit{Notes on the Tribes, Provinces, Emirates and States of the Northern Provinces of Nigeria.} Capetown},
	author = {Temple, Olive.},
	publisher = {Argus Printing and Publishing Co},
	year = {1922}
}

@@book{Westermann1952,
	address = {\textit{Languages of West Africa},
	author = {Westermann, Diedrich  and  Bryan, Margaret A.},
	publisher = {Part II.} Oxford University Press for the International African Institute, London},
	year = {1952}
}

Williamson, \citealt{Kay1971}. The \ili{Benue}-Congo languages \& [1ECA?]j[1ECD?]. In: \textit{Current trends in Linguistics 7,}  Thomas Sebeok (ed.), 245-306. The Hague: Mouton.

Williamson, \citealt{Kay1989}. \ili{Benue}-Congo Overview. In: Bendor-Samuel John (ed.). \textit{The \ili{Niger-Congo} languages}. 247-276. Lanham: Universities Press of America.

Williamson, Kay \& Blench, Roger.M. 2000. \ili{Niger-Congo}. In: \textit{African languages: an introduction}. Heine, Bernd \& Nurse, Derek (eds.), 11-42. Cambridge: Cambridge University Press.

@@book{Yoder2008,
	address = {Jos},
	author = {Yoder, Zachariah  and  Hannelová  and  Magnusson, Carol  and  Oro, Yakubu  and  Otronyi, Linus  and  Rueck, Michael J.  and  Spencer, Katharine.},
	publisher = {Wycliffe Nigeria. (Unpublished survey report)},
	title = {\textit{Sociolinguistic Survey of the ‘Yara (\ili{Kamuku}) people, {Niger} and \ili{Kaduna} States, {Niger}ia}},
	year = {2008}
}

\end{verbatim}
 
\section*{Abbreviations}
\section*{Acknowledgements}
\sloppy
\printbibliography[heading=subbibliography,notkeyword=this] 
\end{document}