\documentclass[output=paper,modfonts,nonflat]{langsci/langscibook}
\ChapterDOI{10.5281/zenodo.4018370}

\author{Giovanni Rossi\affiliation{Department of Sociology, University of California, Los Angeles}\and Simeon Floyd\affiliation{Department of Anthropology, Universidad San Francisco de Quito}\lastand N. J. Enfield\affiliation{Department of Linguistics, The University of Sydney}}
\title{Recruitments and pragmatic typology}
\abstract{\noabstract}

\IfFileExists{../localcommands.tex}{
  \input{../localpackages}
  %Copy this to localcommands.tex

\usepackage[english]{babel}
\usepackage{amsmath}
\usepackage{amssymb,amsfonts,textcomp}
\usepackage{array}
\usepackage{hhline}
\usepackage{hyperref}

\newenvironment{styleStandard}{}{}
\newenvironment{stylelsAbstract}{}{}
\newenvironment{stylelsSectioni}{}{}
\newenvironment{stylelsSectionii}{}{}
\newenvironment{stylelsBulletList}{}{}
\newenvironment{styleBibliographyi}{}{}
\newenvironment{listWWNumxxvleveli}{}{}
\newenvironment{listWWNumxxvlevelii}{}{}
\newenvironment{listWWNumxxvleveliii}{}{}
\newenvironment{listWWNumxxvleveliv}{}{}
\newenvironment{listWWNumixleveli}{}{}
\newenvironment{listWWNumixlevelii}{}{}
\newenvironment{listWWNumixleveliii}{}{}
\newenvironment{listWWNumixleveliv}{}{}

\newcommand\textstyleListLabelxvi[1]{#1}
\newcommand\labellistWWNumxxvleveli{\thelistWWNumxxvleveli.}
\newcommand\labellistWWNumxxvlevelii{\thelistWWNumxxvlevelii.}
\newcommand\labellistWWNumxxvleveliii{\thelistWWNumxxvleveliii.}
\newcommand\labellistWWNumxxvleveliv{\thelistWWNumxxvleveliv.}
\newcommand\labellistWWNumixleveli{[F0B7?]}
\newcommand\labellistWWNumixlevelii{\textstyleListLabelxvi{o}}
\newcommand\labellistWWNumixleveliii{[F0A7?]}
\newcommand\labellistWWNumixleveliv{[F0B7?]}

\newcounter{listWWNumxxvleveli}
\newcounter{listWWNumxxvlevelii}[listWWNumxxvleveli]
\newcounter{listWWNumxxvleveliii}[listWWNumxxvlevelii]
\newcounter{listWWNumxxvleveliv}[listWWNumxxvleveliii]
\newcounter{itemize} 
  \input{../localhyphenation}
  \addbibresource{../localbibliography.bib}
  \togglepaper[1]%%chapternumber
}{}

\begin{document}
\maketitle
\label{sec:intro}

\section{Introduction}\label{sec:intro:1}

Getting others to do things is a central part of social interaction in any human society. Language is our main tool for this purpose. In this book, we show that sequences of interaction in which one person’s behavior solicits or occasions another’s assistance or collaboration share common structural properties that provide a basis for the systematic comparison of this domain across languages. The goal of this comparison is to uncover similarities and differences in how language and other conduct are used in carrying out social action around the world, including different kinds of requests, orders, suggestions, and other actions brought together under the rubric of \textit{recruitment} (see \sectref{sec:intro:4} below). The project constitutes an exercise in pragmatic typology. We map out a possibility space for linguistically-mediated social actions and we use that possibility space as a grid for comparison between languages. This allows us to look for universals and cross-linguistic variation in this pragmatic domain. While other multi-authored publications present comparative findings from this project (e.g. \citealt{FloydEtAl2018}), this book lays out the conceptual and methodological background for the project (Chapters 1--2) and presents the findings language-by-language (Chapters 3--10). The book is intended to serve as a reference source for those interested in primary data on the phenomenon of recruitments in a diverse set of the world’s languages.

\section{Background on research on getting others to do things}\label{sec:intro:2}

A landmark in research on requests and similar speech acts is \citet{Searle1969,Searle1975}, who built on \citet{Austin1962}. For Searle, speech acts have felicity conditions, which need to be met if the act is to succeed. For a request, the utterance should refer to a future act of the recipient and the speaker should believe that the recipient can do the requested act, among other conditions \citep[66]{Searle1969}. Many of Searle’s felicity conditions refer to mental states of participants. Building on this and the cognitively-grounded theory of \citet{Grice1975}, \citet{BrownLevinson1978,BrownLevinson1987} developed a theory of politeness in which requests featured prominently. The theory began with observations of similarities in pragmatic strategies in three unrelated languages and cultures (Tamil, Tzeltal and English). A theory of face -- people’s public self-image \citep{Goffman1967} -- suggested universal pressures affecting social behavior, particularly in “face-threatening acts” such as requests.

Researchers in psychology engaged with the ideas of Searle, Grice, Brown \& Levinson, seeking to test them with experimental methods. One puzzle concerned the literal meaning of an utterance (e.g. \textit{Can you pass the salt?}) in the comprehension of the intended request, that is, whether or not the literal meaning must be computed first before inferring that a request is being made \citep{ClarkLucy1975,Clark1979,Gibbs1979,ClarkSchunk1980,Gibbs1983,Gibbs1986a}. Another puzzle concerned variation of request forms in terms of a single general principle: when making a request, a speaker first assesses what reason there might be for the recipient not complying, and then formulates an utterance to deal with the “greatest potential obstacle” they can anticipate (\citealt{Gibbs1985,FrancikClark1985,Gibbs1986b,GibbsMüller1988,Clark1996}). The obstacle mentioned may be generic, such as the recipient’s inability to do what is requested (e.g. \textit{Can you tell me what time it is?}), or more specific, such as the availability of a relevant object (e.g. \textit{Do you happen to have a watch?}). This is closely related to the ideas of preconditions discussed by \citet{Searle1969} and by \citet{GordonLakoff1971}.

Linguists have studied the grammatical structures and pragmatic properties of the basic sentence types, all of which are used in requesting: imperatives, interrogatives, declaratives (\citealt{GordonLakoff1971,SadockZwicky1985,AikhenvaldDixon2017}). They have also studied the connections between alternative linguistic forms and social variables in events of requesting (\citealt{SinclairCoulthard1975,Ervin-Tripp1976,Ervin-Tripp1981,GordonErvin-Tripp1984,Ervin-TrippEtAl1990}), including how these variables may affect the comprehension of the request.

A large body of research in the subfield of “cross-cultural pragmatics” (e.g. \citealt{Blum-KulkaEtAl1989}) has been devoted to the comparative study of linguistic patterns associated with requests across many languages. A unifying element of this tradition of research is a standardized methodology based on “discourse completion tasks” \citep{Blum-Kulka1982}. In \sectref{sec:intro:5} below, we further discuss this tradition of research in relation to our pragmatic typological approach.

Research in the fields of conversation analysis, ethnomethodology, and interactional linguistics is perhaps closest to the approach taken here, for a few reasons. One is that the empirical source of data is recordings of informal interaction. Another is that the units of analysis are not clauses or sentences but moves in conversational sequences (\citealt{Wootton1981,Wootton1997,Lindström2005,VinkhuyzenSzymanski2005,Heinemann2006,CurlDrew2008,CravenPotter2010,ZinkenOgiermann2013,DrewCouper-Kuhlen2014a}, among many others). These inductive approaches are grounded in the sequential development of interaction. Most conversation-analytic and interactional-linguistic studies of requesting to date have concentrated on a particular language. Comparison between languages has been only possible by drawing on results of distinct studies, each with their own particular focus and goals. Also, most conversation-analytic research tends not to be transparently quantitative. But structured quantitative analysis built on the back of a qualitative analysis has been shown to greatly enhance the analytic possibilities of comparative conversation analysis (e.g. \citealt{FoxEtAl2009,RossanoEtAl2009,StiversEtAl2009,DingemanseEtAl2015}).

The pragmatic domain of getting others to do things has been thought of in different ways. A first distinction is often made between getting someone to carry out a practical action and getting someone to provide information. Some work in the philosophy of language (e.g. \citealt{Searle1969}) and in psycholinguistics (e.g. \citealt{Clark1979,ClarkSchunk1980}) tended to merge the two. But most work has distinguished between soliciting practical action and information, and has studied them as separate phenomena. 

Another distinction has to do with the categorization of types or subtypes of social action. Two main approaches can be identified here. The first is to treat the domain of getting others to do things as a family of related but distinct speech acts or actions (e.g. directives, requests, hints) on the basis of distinct semantic/pragmatic features, for example those defining different degrees of forcefulness or coerciveness (e.g. \citealt{Searle1976,Wierzbicka1991,CravenPotter2010}). The second approach is to treat the domain as a single, generic type of social action, and to treat variations in the way this is implemented as pertaining to the level of linguistic practice (e.g. \citealt{Ervin-TrippEtAl1990,Wootton1997,Rossi2012}). 

Yet another distinction is drawn in terms of the temporality of the practical action being solicited: whether the action is carried out immediately, at the same place and time -- such as passing a knife in a kitchen setting -- or distally, at a later time and possibly different place -- such as picking somebody up from work (see, e.g., \citealt{Lindström1999}). Although much research encompasses both, studies of telephone calls obviously privilege the latter, whereas studies of face-to-face interaction privilege the former.

\section{Background to the project and studies presented in this book}\label{sec:intro:3}

The conceptual, empirical, and analytical work on recruitments presented in this volume was carried out by a team of researchers under the auspices of the \textit{Human Sociality and Systems of Language Use} (HSSLU) project, a European Research Council Starting Grant awarded to Nick Enfield (2010--2014). The Recruitments Subproject was coordinated by Simeon Floyd and Giovanni Rossi, and was convened at the Max Planck Institute for Psycholinguistics in Nijmegen, the Netherlands, in collaboration with other members of the Interactional Foundations of Language project in Stephen Levinson’s Language and Cognition Department (also encompassing Levinson’s European Research Council Advanced Grant INTERACT, 2011--2015).

The authors of the chapters of this book each contributed to the comparative study in a number of ways. All contributed to the conceptual development of the project, including the content of the coding scheme, administering the coding scheme, and analyzing the results. Collection of video corpora used in the study was carried out by Julija Baranova (Russian), Joe Blythe (Murrinhpatha), Mark Dingemanse (Siwu), N. J. Enfield (Lao), Simeon Floyd (Cha’palaa), Giovanni Rossi (Italian, English) and Jörg Zinken (Polish).\footnote{Part of the English data came from the Language and Social Interaction Archive created by Leah Wingard, available from San Francisco State University (\url{http://www.sfsu.edu/~lsi/}).} Steve Levinson provided the context for this project to thrive, and was a key interlocutor at all points throughout the project. As an external collaborator, Paul Drew was present for many of the research meetings, and contributed much to the methodology and conception of the project. Séan G. Roberts provided crucial advice and assistance in developing the quantitative aspects of the comparative analysis (not reported in this volume). The coding and data analysis workflow built on and extended work in a closely related subproject of HSSLU that developed a pragmatic typology of other-initiated repair, coordinated by Mark Dingemanse and Nick Enfield. We also owe a debt to conceptual collaborators in discussion and data analysis over the life of the project: Lorena Pool Balam, Penelope Brown, Tyko Dirksmeyer, Paul Drew, Rósa S. Gísladóttir, Gertie Hoymann, Stephen C. Levinson, Lilla Magyari, Elizabeth Manrique, Ruth Parry, Séan G. Roberts, Jack Sidnell, Tanya Stivers and Francisco Torreira.

The development of the recruitments concept and the timeline of the project work and findings is as follows.\footnote{See \url{http://recruitments.nickenfield.org/timeline/} for PDFs of the documents and memos mentioned in the rest of this section.\label{fn:intro:2}}

The HSSLU project, which began in January 2010, featured three subprojects. One of these centered on actions of getting people to do things. On 7--9 October 2010, team members discussed requests and similar kinds of social actions in a UCLA workshop on “Action Ascription in Social Interaction”. At a follow-up workshop on the same topic in Nijmegen on 18--19 March 2011, Enfield presented a first working definition of “recruitment” \citep{Enfield2011a}. This was a reference point for a one-week intensive data workshop on recruitments held later that month (March 21--25, 2011), in which team members, together with Paul Drew as an external collaborator, delved into data and initial qualitative analysis of candidate recruitment sequences in the languages represented in this volume.

This collaborative work, along with a subsequent session on recruitments at a HSSLU project retreat on 20 April 2011 \citep{Enfield2011b}, resulted in a first draft of the coding scheme for this volume, authored by Floyd and Rossi, and circulated within the project team on 25 October 2011. The project team met (on 27 March 2012) to discuss the first draft coding scheme. Notes by Floyd and Rossi were then circulated, followed by circulation of an updated coding draft on 16 April 2012.

In October 2012, a subgroup of team members -- Enfield, Floyd, Rossi, and Dingemanse -- carried out a first pilot study using the coding scheme. On November 2, 2012, Floyd presented the ongoing results of the recruitments project at a retreat at Schloss Ringberg, Germany. Later that month, the group met to discuss and plan a second pilot study (with a new coding scheme draft version 1.3), this time with all team members participating. During the week of November 20--27, 2012, the full team carried out a pilot of the coding scheme on all of the languages included in this volume. In December 2012, team members identified cases in their respective corpora for coding, and on January 20, 2013, the coding scheme was finalized (version 2.2). The full team then held an intensive coding week (January 21--25, 2013), following a shared set of further instructions circulated to the team by Floyd.

After the coding was completed, a coding consistency check was done, followed by a test for coder reliability across the team members. This process was overseen by Floyd and Rossi. Because each team member worked with data in languages that others had no access to, our coder reliability check was carried out using a reference set of English data. The reliability check established that some questions were coded in the same ways across the group, but it also revealed that some questions had not been coded consistently. The final step was for the team to carry out a re-coding of those questions in order to ensure coder reliability. The re-coding took place in December 2013.

Results of the comparative study were publicly presented in February 2014 by Floyd and Rossi at the UCLA workshop “About Face” (\citealt{FloydEtAl2014a}), and then in June 2014 at the International Conference on Conversation Analysis at UCLA \citep{FloydEtAl2014b}. The June 2014 presentation not only publicized the empirical findings of the comparative project, it also presented the key conceptual elements of our collective development of the concept of recruitment sequences. Other publications in which these ideas and findings have been discussed include \citet{Enfield2014b}, \citet{DrewCouper-Kuhlen2014b}, \citet{Rossi2015b}, \citet{KendrickDrew2016}, \citet{ZinkenRossi2016}, and \citet{Floyd2017}.

While comparative findings from this project are presented in multi-authored publications (\citealt{FloydEtAl2014b,FloydEtAl2018}), the present collection was convened as an opportunity for the individual researchers to lay out the project findings specific to their language of study.

\section{Recruitment sequences defined}\label{sec:intro:4}

Our use of the term \textit{recruitment} reflects a shift from an approach centered around the speech act of requesting to one addressing the interactional process of getting others to act. The two main alternatives to our \textit{recruitments} approach are: i) a definition of the phenomenon based on intentional states, such as someone’s desire to have another do something, and ii) a definition of the phenomenon based on linguistic form, for example, focusing on imperatively-formatted utterances. While the former presents problems of evidence, the latter overly limits the scope, as we know that other types of strategies can be used in seeking assistance or collaboration. Our functional approach based on recruitment sequences in recorded social interaction makes the identification of cases more objective and replicable, the analysis more falsifiable, and comparison across languages easier thanks to the natural control provided by sequential structure (\citealt{DingemanseFloyd2014}), without having to restrict the scope of relevant linguistic patterns beforehand.

As we define it, a recruitment is a basic cooperative phenomenon in interaction consisting of a sequence of two moves with the following characteristics:\footnote{ {By} {\textit{move}} {we intend}{ }{a unit of communicative behavior that may include language and/or other conduct} {(\citealt{Enfield2013}: chap. 6; cf. \citealt{Goffman1981}). This is a related but distinct concept to} {\textit{turn}}{, which we understand as a move involving primarily language} {(see \citealt[3--7]{Schegloff2007}}).}

\begin{description}
\item[Move A:] participant A says or does something to participant B, or that B can see or hear;
\item[Move B:] participant B does a practical action for or with participant A that is fitted to what A has said or done.
\end{description}

Crucial to this phenomenon is the nature of the behavior instigated in Move B: a \textit{practical} action involving physical work, typically the transfer of an object, the performance of a manual task, or the alteration of an ongoing bodily movement.

In this project, we restrict the target phenomenon by focusing on the recruitment of practical actions to be performed here and now. The recruitment of information and of future practical actions are excluded. At the same time, we are inclusive of any communicative behavior that causes someone to do something, independently of its verbal or nonverbal construction, and of whatever the speaker’s exact intention may be. The identification of cases does not turn on the form of the instigating behavior but on the nature of the behavior instigated, and on the causal relation between the two.

Because of our focus on here-and-now cooperation, Move A and Move B must be temporally adjacent. This means that B must begin to deal with what A has said or done in the next few moments. In some cases, the provision of assistance or collaboration may be displaced because B initiates repair or defers fulfillment on some grounds (e.g. because they are momentarily busy). What is important is that the first response addresses the relevance of immediate cooperation. This obtains also when B refuses to fulfill the recruitment. Finally, there are cases in which B may ignore Move A and produce no response, or a response that does not address the relevance of their immediate cooperation. In these cases, additional measures are taken to preserve objectivity in the identification and inclusion of cases. When there is no uptake of the recruitment, we only consider cases in which: i) Move A involves an explicit, on-record practice, typically a linguistic practice, that is known to regularly solicit compliance with recruitment (e.g. imperatives, explicit interrogatives such as \textit{Can you x}, etc.); ii) Move A is repeated, either in the same or in another form, showing pursuit of response. These criteria mean that we exclude cases of implicit, off-record practices that could have potentially led to someone doing something but did not, and were not pursued. Finally, note that in cases in which the recruitment is fulfilled or granted in Move B, the sequence may be minimally expanded with a further move  --  Move C  --  registering appreciation or satisfaction, in other words, acknowledgment by A.

Recruitment sequences encompass a broad range of actions and interactional events that result in someone providing assistance or collaboration. Our definition does not restrict the focus to request-related events but includes offer-related events, where assistance is provided in response to someone’s trouble \citep{Curl2006}. The coding scheme, presented in \chapref{sec:coding}, helps to distinguish offer-related events from others, while providing a framework for treating both as part of the broader phenomenon of recruitment.\footnote{The coding scheme and the extensive collaborative work that contributed to its development (see \sectref{sec:intro:3} and fn. \ref{fn:intro:2}) predate a study by \citet{KendrickDrew2016} that built on this collaborative work and used data collected within the project.} The scheme also suggests criteria for discriminating between offers that are occasioned by another’s trouble and offers that are not (p. \pageref{par:coding:offers}). 

Across the volume, we examine different kinds of recruitment events, from ones in which someone is told or asked to do something, to others in which someone responds to another’s statement of need (individual or collective), to yet others in which someone responds to a wordless gesture requiring cooperation. Each chapter surveys a range of actions, sequential structures, and social-interactional relations between recruiter and recruitee. The cooperative events we examine include both ones in which a participant is recruited to assist another in the realization of an individual goal and ones in which a participant is recruited to collaborate in the realization of a joint goal, involving shared commitments and responsibilities \citep{Enfield2014b,ZinkenRossi2016}.

The components of a recruitment sequence and the terminology to describe them can be summarized as follows.

A \textsc{recruitment sequence} minimally involves:

\begin{itemize}
\item two participants: \textsc{A} (the \textsc{recruiter}) producing the instigating action, and \textsc{B} (the \textsc{recruitee}) responding to it;
\item \textsc{move a}: the instigating or recruiting action;
\item \textsc{move b}: an action addressing the relevance of immediate cooperation as a result of the previous instigating action, including:

\begin{itemize}
\item \textsc{fulfillment}: a practical action involving physical work performed for or with A;
\item \textsc{rejection}: the conveyance of inability or unwillingness to fulfill the recruitment;
\item \textsc{deferment} of fulfillment;
\item \textsc{initiation of repair} (often leading to fulfillment after repair).
\end{itemize}

If B \textsc{ignores} Move A, then the sequence must include an explicit, on-record practice of initiating recruitment and/or pursuit of response in order to be included.

The sequence can be expanded by a \textsc{move c}, doing \textsc{acknowledgment}.
\end{itemize}

The coding scheme (\chapref{sec:coding}) provides detailed commentary and examples, elaborating on each element of recruitment sequences and on the criteria for their identification presented in summary form here. % of this volume

\section{Pragmatic typology}\label{sec:intro:5}

Pragmatic typology is the comparative study of language use. It brings together conceptual and analytic tools from a range of disciplines including linguistics, conversation analysis, gesture studies, and anthropology. A key innovation of the approach in relation to tools for analysis in linguistics is the reference to features of a sequentially ordered exchange of actions in conversation, including the temporal unfolding of such exchange, and its social and normative context. Further, because of the reliance on video corpora, it incorporates both verbal and nonverbal conduct in the analysis of sequences of action. We are faced with the challenge that faces any comparative linguist, namely the need to distinguish between language-particular descriptive categories and language-independent comparative categories or “comparative concepts” \citep{Haspelmath2010}. We submit that our appeal to features of conversational organization -- outside the usual realm of “concepts” in the semantic sense -- is an advance in the search for tools for linguistic comparison.

A landmark effort to carry out comparative pragmatics was the subfield of “cross-cultural pragmatics”, launched in the 1980s. Building on speech act theory and politeness theory, cross-cultural pragmatics has studied the realization of requests across a large number of languages.\footnote{See \citet{HouseKasper1981}, \citet{Blum-KulkaEtAl1989}, \citet{Sifianou1992}, \citet{Lee-Wong1994}, \citet{LePair1996}, \citet{MárquezReiter2000}, \citet{TsuzukiEtAl2005}, \citet{RueZhang2008}, \citet{FélixBrasdefer2009}, \citet{Ogiermann2009}, \citet{Peterson2010}, among many others.} This work was motivated, on the one hand, by a search for similarities and differences in the use of language across cultures, and on the other hand, by an interest in the acquisition and development of pragmatic competence (see \citealt{Woodfield2008} for a review). Studies in this tradition have provided insights into culture-specific features of politeness and directness, and produced rich inventories of request realization patterns. However, these advances in systematic comparison of speech acts across languages have been limited by their methodology. By using written elicitation, in the form of a “discourse completion task”, this work relied on speakers' metalinguistic beliefs about appropriate usage, rather than on direct observation of actual usage in situ. A first problem with this is that we cannot be sure if speakers' intuitions match with what they do in practice. A second problem is that such an approach is relatively low in ecological validity, and does not provide access to the kinds of empirical evidence that direct and repeated observation of behavior in recordings can provide.

Our approach to pragmatic typology has two fundamental elements:
(i) the empirical analysis of verbal and nonverbal behavior in video recordings of naturally occurring interaction across languages, and
(ii) a coding-based methodology for systematic comparison (see also \citealt{DingemanseEnfield2015}; \citealt{Rossi2020a}). In some previous comparative work in conversation analysis and interactional linguistics, the comparison emerges from the cumulative results of distinct studies, each with their own particular focus and goals; this applies, for instance, to collections of studies of questioning (\citealt{SteensigDrew2008}), person reference \citep{EnfieldStivers2007} and change-of-state tokens (\citealt{HeinemannKoivisto2016}). In other cases, the comparison is designed in advance and carried out jointly by reference to a common focus. Studies of this kind have examined, among other phenomena, the intersection of self-repair and turn-taking (\citealt{FoxEtAl1996}), other-initiated repair (\citealt{Egbert1996,EgbertEtAl2009}), epistemically authoritative second assessments (\citealt{SidnellEnfield2012}), and requests (\citealt{Zinken2016}). Finally, some structured comparisons involve a combination of qualitative and quantitative methods, including a coding scheme with a battery of standardized questions asked of hundreds of cases for each language; these include studies of gaze behavior (\citealt{RossanoEtAl2009}), turn-taking (\citealt{StiversEtAl2009}), self-repair (\citealt{FoxEtAl2009,FoxEtAl2010}), question-answer sequences (\citealt{EnfieldEtAl2010}), and other-initiated repair (\citealt{DingemanseEtAl2015}). The project reported on in this book falls within the third group. Ours is a mixed methodology in six defined steps, outlined in the next section.

\section{A six-step method}\label{sec:intro:6}

We now describe our six-step method for comparative team-based pragmatic typological research. We outline the process in idealized form, in part as a description of what we have done in this project and in part as a recipe for carrying out subsequent pragmatic typological team projects on other aspects of social interaction.

\begin{description}
\sloppy
    \item[Step 1. Record:] Project members carry out sustained field expeditions to village, home, and equivalent community settings, making high-quality video re\-cord\-ings of everyday interaction. This step is the foundation of each language’s corpus within the project’s comparative empirical work. High quality is paramount. Common practical and ethics protocols for the collection of conversational materials in fieldwork are crucial (see \citealt{Enfield2014a}; \citealt{DingemanseEnfield2015}). To guarantee that the highest quality materials are captured, and as a way of maximizing the investment in field research, each field researcher collects a large number of hours of raw material which ensures availability of sufficient quality data, as well as providing extensive materials for later research if needed. This step assumes significant background work on the part of the researcher, who has likely already established the appropriate type and degree of familiarity and integration in a host community and with the relevant language.
\fussy
    \item[Step 2. Transcribe:] Project members then work with native speakers in the field to transcribe and translate the recordings collected. This is a lengthy and involved phase of the research, and represents a major commitment of research resources, but with a major payoff in result. Full transcription and translation of one minute of recorded social interaction takes approximately three hours; about an hour for fine-grained transcription and about two hours for full translation (when the researcher is not a native speaker of the language being transcribed and translated). Securing an accurate and complete account of what is being said in a free-flowing conversation is difficult and time-consuming. Many team members are working on languages that are not their first languages. These hours of transcription and translation may also require more general investigation of the language as necessary background to the analysis of the corpora, along with the relevant biographical and ethnographic background. This means that the “Transcribe” step will likely require between 9 and 12 months of dedicated fieldwork. This is a valuable investment with broader payoff. An important outcome of this step is that these corpora will then be available for further research in the future.
    \item[Step 3. Confer:] Team members work together in intensive internal group meetings over a sustained period, in which all members of the team share data and observations from the corpora relating to the phenomena of interest; these are hands-on intensive meetings, carried out at close quarters with the goal of identifying and operationalizing the empirical phenomena for quantitative investigation in Step 4 below, and articulating their relation to the project’s research questions. This step is important for the project’s conceptual and theoretical outcomes, and it ensures coherence and clarity of the outcomes in subsequent steps. An important goal of this step is to ensure that the team members become so steeped in the empirical materials, not just from their own field language but from all languages in the project, that the team develops deep and shared intuitions for the phenomena at the core of the project. These sessions also have the specific goal of producing a coding scheme to be used in Step 4.
    \item[Step 4. Code:] Team members carry out quantitative coding based on Step 3 outcomes. Coding schemes should eventually be published, so as to allow the international research community to apply them in extensions and adaptations of the research (see \chapref{sec:coding}, as well as \citealt{StiversEnfield2010}, \citealt{DingemanseEtAl2016}, for examples). The coding step can be done within a few weeks, and is done in an intensive block-out work period, with all team members in daily contact to discuss and iron out coding issues while working through the data. % of this volume
    \item[Step 5. Check:] After coding of individual language corpora is completed, there is a check for coding reliability. This ensures that the coding done by each team member of data in different languages is done consistently across the project. The procedure is to use a sample of data from a language common to all team members (e.g. English) and have everybody independently code the same data, using the coding scheme from Step 4, in order to then carry out a test of reliability and consistency in coding. It is then possible to report with confidence that the coding of different languages by different researchers in Step 4 was done in the same way (as \citealt{DingemanseEtAl2015} do for other-initiated repair).
    \item[Step 6. Model:] This final step involves statistical modeling of the coding results from Step 4. The data resulting from the large-scale coding scheme enables statistical modeling for quantitative assessment of patterns of association and interdependence between the phenomena coded for (as formulated in Step 3 and executed in Step 4). It is necessary to use multivariate statistics to control for interdependence among these variables (see \citealt{DingemanseEtAl2015} for an example of this). Steps 5 and 6 do not take a long time, but require special expertise.
\end{description}

This protocol requires a team science approach. Given the demanding combination of fieldwork (Steps 1, 2), expertise in comparative linguistics, interactional linguistics, and conversation analysis (Steps 3, 4), and quantitative approaches (Steps 5, 6), this could never have been done in any way other than by a team. Team science in linguistics is still rare and we were fortunate to have had the opportunity to do this here.

The findings reported in the language-specific chapters in this volume are primarily the product of qualitative analysis but also include quantitative findings particular to each data set.

\section{Data}\label{sec:intro:7}

This study is based on the analysis of corpora of audiovisual recordings of informal everyday language usage in social interaction in eight languages from five continents (see \figref{fig:intro:1}). In building these corpora, we placed unattended cameras in household and community contexts to record social interactions as they were occurring naturally, using high standards for audio and video quality (see Step 1 discussed in \sectref{sec:intro:6} for more details).

\begin{figure}
\caption{World map showing locations of data collection for the eight languages involved in the study (Credit: satellite composition of Earth's surface by NASA).\label{fig:intro:1}}
\includegraphics[height=.37\textheight]{figures/Introduction_Figure1.png}
\end{figure}

The data were transcribed and translated by a language expert (see \tabref{tab:intro:1}) with assistance from native speakers. The corpora range in size from about ten to over ninety hours of footage. In some cases, the corpus represents the largest available database for the language, especially in the case of unwritten minority languages like Cha’palaa (\chapref{sec:floyd}), Murrinhpatha (\chapref{sec:blythe}), and Siwu (\chapref{sec:dingemanse}). For larger-scale national languages like Italian, Lao, Polish, and Russian, other corpora may be available to some degree, but most of these are limited to written language, due to the intensive demands of transcription of spoken language.

Sampling procedures and criteria for inclusion/exclusion are detailed in \chapref{sec:coding}.

\begin{table}
\small
\begin{tabularx}{\textwidth}{llQQQ}
\lsptoprule
{Language} & {Language family} & {Location} & {Data collected by} & {Data coded by}\\
\midrule
Cha’palaa & Barbacoan & Ecuador & Simeon Floyd & Simeon Floyd\\
\tablevspace
English & IE (Germanic) & United Kingdom, United States & Giovanni Rossi, LSI archive \url{http://www.sfsu.edu/~lsi/} & Kobin H. Kendrick\\
\tablevspace
Italian & IE (Romance) & Italy & Giovanni Rossi & Giovanni Rossi\\
\tablevspace
Lao & Tai & Laos & N. J. Enfield & N. J. Enfield\\
\tablevspace
Murrinhpatha & Southern Daly & Northern Australia & Joe Blythe & Joe Blythe\\
\tablevspace
Polish & IE (Slavic) & Poland & Eva Ogiermann, Jörg Zinken & Jörg Zinken\\
\tablevspace
Russian & IE (Slavic) & Russia & Julija Baranova & Julija Baranova\\
\tablevspace
Siwu & Kwa & Ghana & Mark Dingemanse & Mark Dingemanse\\
\lspbottomrule
\end{tabularx}
\caption{Languages covered in this volume, data sources, coding credits.\label{tab:intro:1}}
\end{table}

\section{This book}\label{sec:intro:8}

The goal of this book is two-fold: to document the conceptual and methodological framework of our project (especially here and in \chapref{sec:coding}) and to provide detailed qualitative/quantitative analyses of recruitment sequences in each of the eight languages: Cha’palaa, English, Italian, Lao, Murrinhpatha, Polish, Russian, and Siwu. Each language-specific study gives an overview of linguistic, gestural, sequential, and contextual features of recruitment sequences, following the categories defined in the coding scheme. While written to stand independently, the eight chapters adopt the coding scheme’s common reference structure to facilitate navigation and comparison. At the same time, the chapters develop aspects and topics that are specific to each language and data.

By focusing on the phenomenon of recruitments, this large-scale collaborative study examines a domain of social action in interaction in which social relations are exploited, maintained, and potentially tested. We find that cross-linguistic diversity in this pragmatic domain is relatively low, considerably lower than the diversity observed in phonological, morphosyntactic, and semantic systems. This is in line with the idea that a species-wide infrastructure for interaction underpins the use of language, largely independent of the specific shape of that language (see \citealt{Levinson2000,Levinson2006,Schegloff2006,Enfield2013,EnfieldSidnell2013,StiversEtAl2009,DingemanseEtAl2015}). This is not to say that these pragmatic systems are identical. The chapters of this book show that there are differences. But we are struck by the commonalities that our approach reveals in a domain of language where many might expect to find radical variation.

\section*{Acknowledgments}
Our biggest thanks go to all contributors for their sustained work on the project, and for their patience with the editors. Thank you also to our collaborators for their input and intellectual engagement (see \sectref{sec:intro:3}) and to our reviewers who helped to improve the volume enormously with their constructive commentary.
We owe a particular debt of gratitude to Sebastian Nordhoff who directed the production of the volume with great generosity and patience.
We gratefully acknowledge our funding bodies: the European Research Council (INTERACT and HSSLU projects), the Max Planck Society (Language and Cognition Department, directed by Stephen C. Levinson), the Dutch Research Council (NWO), the Australian Research Council, and the Academy of Finland.

{\sloppy\printbibliography[heading=subbibliography,notkeyword=this]}
\end{document}
