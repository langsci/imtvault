\documentclass[output=paper]{langsci/langscibook}
\ChapterDOI{10.5281/zenodo.4018384}

\author{Jörg Zinken\affiliation{Department of Pragmatics, Leibniz-Institute for the German Language}}

\title{Recruiting assistance and collaboration in Polish}

\abstract{This chapter describes the resources that speakers of Polish use when recruiting assistance and collaboration from others in everyday social interaction. The chapter draws on data from video recordings of informal conversation in Polish, and reports language-specific findings generated within a large-scale comparative project involving eight languages from five continents (see other chapters of this volume). The resources for recruitment described in this chapter include linguistic structures from across the levels of grammatical organization, as well as gestural and other visible and contextual resources of relevance to the interpretation of action in interaction. The presentation of categories of recruitment, and elements of recruitment sequences, follows the coding scheme used in the comparative project (see \chapref{sec:coding} of the volume). This chapter extends our knowledge of the structure and usage of Polish with detailed attention to the properties of sequential structure in conversational interaction. The chapter is a contribution to an emerging field of pragmatic typology. }
\maketitle
\label{sec:zinken}
\begin{document}

\section{Introduction}

This chapter provides an overview of many of the practices for recruiting another person's assistance or collaboration in Polish.  The data for this overview come from a corpus of video recordings of informal everyday interactions in the homes of families living in urban areas of Poland.  As this chapter will show, recruitment practices in Polish follow many of the regularities that we have observed for other languages in the larger project reported in the present volume.  Some distinctive aspects of Polish recruitments, such as the diverse imperative, impersonal, and infinitive formats of recruiting moves, are also discussed.

\subsection{The Polish Language}

Polish is an Indo-European language that belongs to the West-Slavic branch of the Slavic language family.  Polish is spoken by about 40 million people worldwide, of whom about 37 million live in the Republic of Poland in Central Europe.

Polish has a long tradition of grammatical description (comprehensive grammars are \citealt{bak2010} and \citealt{Strutyński2006}).  Although it is characterized by relatively free word order, its basic word order is SVO. There is a rich tradition of pragmatic work in Polish linguistics, but work on the basis of recorded interaction has been virtually absent until recently (though see \citealt{Labocha1985,Labocha1986}).  Grammatical features relevant to recruitment practices include verbal aspect, the absence of interrogative syntax, a relatively elaborate imperative paradigm, and impersonal modal constructions with the verbs \textit{trzeba} ‘it is necessary to’ and \textit{moż}\textit{na} ‘it is possible to’.

\subsection{Data collection and corpus}

The Polish corpus of video recordings was built outside the comparative project this volume reports on.  Most recordings were made in 2009 as part of a comparative project on \textit{Sharing responsibilities in English and Polish families}, funded by the British Economic and Social Research Council (ESRC).  For that project, matched corpora of video recordings of everyday life in British and Polish families were collected.  Participants were asked to record everyday activities, such as mealtimes, cooking, or playing with their children.  The Polish corpus from that project includes 10 hours of recordings made by six families.  Other recordings have been made during field visits since then.  These further data amount to 3.5 hours of recordings made by three families. The restriction to family interaction distinguishes the Polish data from other languages examined in this volume, which include recordings of informal interaction beyond family contexts. The reader might want to keep this caveat in mind when comparing the results across languages.

The recordings were made in the capital city Warsaw and in Lublin, a university city in the southeast of Poland.  This means that all recordings come from Eastern regions in Poland.  The data considered for comparison consisted of coded samples from the recordings, with the goal of coding at least 200 recruitments.  Most of the families had young children, and many recruitment sequences found in the data included a child, either as “recruiter” or as “recruitee”.  To maximize comparability of the data across languages, only recruitments in which both participants are adults were considered for the study reported on here.  Six hours and thirty minutes of recordings were sampled to identify 215 recruitment sequences.  

Transcripts may include up to three tiers for each line. The first tier represents the original talk and/or other conduct following the conventions of conversation analysis \citep{Jefferson2004}; the second tier gives a word-by-word or morpheme-by-morpheme gloss of the talk following the Leipzig glossing rules and abbreviations \citep{comrie_leipzig_2020};\footnote{In addition to the standard abbreviations, I also use \textsc{hrt} for hortative and \textsc{ptc} for particle.} the third tier gives a more idiomatic English translation. 

\section{Basics of recruitment sequences}

As defined in \chapref{sec:intro}, \sectref{sec:intro:4}, a recruitment is a basic cooperative phenomenon in social interaction consisting of a sequence of two moves with the following characteristics:

\begin{description}
\item[Move A:] participant A says or does something to participant B, or that B can see or hear;
\item[Move B:] participant B does a practical action for or with participant A that is fitted to what A has said or done.
\end{description}

A recruitment sequence can have a minimal shape, consisting only of two moves, or it can have a more complex shape. I begin with examples illustrating this difference. In the transcripts, ▶ and ▷ designate Move A and Move B, respectively.

\subsection{Minimal recruitment sequence}

\extref{ex:zinken:1} provides an example of a minimal recruitment sequence.  The participants are seated at the table for supper.  At lines 1--2, Ilona asks Jacek to pass her the salad bowl (Move A); at line 3, Jacek passes her the salad bowl (Move B).

\transheader{ex:zinken:1}{PP2-1\_2224980}\vspace{-1mm}
%
\begin{mdframednoverticalspace}[style=firstfoc]
\begin{transbox}{1}{ilo}
\begin{verbatim}
wiesz    co   podaj    mi kochanie jeszcze
know.2SG what pass.IMP me dear     still
\end{verbatim}
you know what, pass me some more
\end{transbox}
\end{mdframednoverticalspace}
%
\begin{transbox}{2}{~}
\begin{verbatim}
sałatki
salad.GEN
\end{verbatim}
salad, dear
\end{transbox}
%
\begin{mdframednoverticalspace}[style=secondfoc]
\begin{transbox}{3}{jac}
\begin{verbatim}
bardzo proszę ((passes salad bowl))
very   plead.1SG
\end{verbatim}
here you are
\end{transbox}
\end{mdframednoverticalspace}

\subsection{Non-minimal recruitment sequence}

Recruitment sequences are non-minimal when the recruiting move is done again (e.g. to clarify it, to make it more forceful, or because the recruiting speaker is not certain whether it has been heard) before it is complied with or rejected.  In \REF{ex:zinken:2}, Piotr is sitting at the dinner table with a baby on his lap.  He tells Aga at line 1 that cheese has dropped to the floor (Move A).  Aga arrives at the table and puts her coffee cup down.  This might, but need not be, a move preliminary to doing the target action of picking up the cheese (line 2 is not marked with ▷ to allow for this uncertainty).  Piotr then redoes the recruiting move in truncated form (line 3), and Aga picks up the cheese from the floor immediately thereafter.

\transheader{ex:zinken:2}{PP5-4\_0154810a}\vspace{-1mm}
%
\begin{mdframednoverticalspace}[style=firstfoc]
\begin{transbox}{1}{pio}
\begin{verbatim}
tutaj ser    u- (0.4) tu    ser    upadł             jeszcze
here  cheese dr-      here  cheese drop.down.3SG.PST still
\end{verbatim}
here cheese has- \hspace{1cm} here cheese has dropped down also
\end{transbox}
\end{mdframednoverticalspace}
%
\xtransbox{2}{aga}{((arrives at table, places coffee cup on the table))}
%
\begin{mdframednoverticalspace}[style=firstfoc]
\begin{transbox}{3}{pio}
\begin{verbatim}
ser    upadł
cheese drop.down.3SG.PST
\end{verbatim}
cheese has dropped down
\end{transbox}
\end{mdframednoverticalspace}
%
\begin{mdframednoverticalspace}[style=secondfoc]
\xtransbox{4}{aga}{((picks up cheese))}
\end{mdframednoverticalspace}\vspace{-2mm}

\subsection{Subtypes of recruitment sequence}

In the comparative project, we distinguish between four types of recruitments: i) sequences in which B provides a service, ii) sequences in which B gives an object to A, iii) sequences in which B alters the trajectory of his or her current behavior, and iv)  sequences in which B does something to address A’s current or anticipatable trouble.

We have already seen examples for two of these: \REF{ex:zinken:2}, where A recruits B to pick up something that has dropped to the floor, was an example of a recruitment the point of which was that B provide a service; \REF{ex:zinken:1}, where A recruits B to pass the salad bowl, was an example of a recruitment the point of which was that B give an object to A.  Passing an object can be a particular kind of service, but we consider object requests separately, because such requests are numerous and they constitute a distinct domain \citep[see also][]{Zinken2015}.

\extref{ex:zinken:3} is an example of a recruitment the point of which is to alter some current conduct by B.  Ilona is putting sugar into Jacek’s tea.  At line 4, Jacek’s recruiting turn \textit{już=już} `already already' (or, more idiomatically, `enough enough') gets Ilona to stop sweetening the tea further (cf. \citealt{Stivers2004} on multiple sayings as a practice for indicating that some course of action should be halted).

\transheader{ex:zinken:3}{PP2-5\_949800}\vspace{2mm}
%
\begin{transbox}{1}{ilo}
\begin{verbatim}
proszę:: ((spoons sugar into A’s tea))
plead.1SG
\end{verbatim}
here you are
\end{transbox}
%
\begin{transbox}{2}{jac}
\begin{verbatim}
dzię[kuję  bardzo
thank.1SG  very
\end{verbatim}
thank you very much
\end{transbox}
%
\begin{transbox}{3}{ilo}
\begin{verbatim}
    [słodzę::    [ci      mężu
     sweeten.1SG  you.DAT husband.VOC
\end{verbatim}
\hspace{0.6cm} I sweeten it for you, my husband
\end{transbox}
%
\begin{mdframednoverticalspace}[style=firstfoc]
\begin{transbox}{4}{jac}
\begin{verbatim}
                 [już=     już
                  already  already
\end{verbatim}
\hspace{2.6cm} enough enough
\end{transbox}
\end{mdframednoverticalspace}
%
\begin{mdframednoverticalspace}[style=secondfoc]
\xtransbox{5}{ilo}{((stops putting sugar into tea))}
\end{mdframednoverticalspace}

Finally, \REF{ex:zinken:4} is a case in which Move B addresses some current or anticipatable problem of A’s.  Piotr is trying to cut pizza, but he is also holding a baby on his lap.  The baby has started to pull the table mat with the pizza plate on it towards himself, and Piotr is in the difficult position of having to juggle trying to cut pizza, holding the baby, and controlling the baby’s hands, all at the same time.  Piotr’s trouble is both visible and audible (\textit{kurczę}, loosely translatable as `damn', line 3).  The two children, Łukasz and Przemek, laugh at Piotr’s predicament (lines 4 and 5), but his wife Aga announces help and shortly thereafter comes to the table and takes the baby from Piotr.

\transheader{ex:zinken:4}{PP5-4\_0134460}\vspace{2mm}
%
\xtransbox{1}{pio}{((cuts pizza on his plate))}
%
\begin{mdframednoverticalspace}[style=firstfoc]
\xtransbox{2}{bab}{((pulls the table mat))}
\end{mdframednoverticalspace}
%
\begin{mdframednoverticalspace}[style=firstfoc]
\begin{transbox}{3}{pio}
\begin{verbatim}
ku::rcze no,
EXPL     PTC
\end{verbatim}
damn \textit{no}
\end{transbox}
\end{mdframednoverticalspace}\vspace{2mm}
%
\begin{transbox}{4}{łuk}
\begin{verbatim}
Hh::
\end{verbatim}
\end{transbox}
%
\begin{transbox}{5}{prz}
\begin{verbatim}
A: hhahaha .H
\end{verbatim}
\end{transbox}
%
\begin{mdframednoverticalspace}[style=secondfoc]
\begin{transbox}{6}{aga}
\begin{verbatim}
już     go      ci      biorę         stamtąd
already him.ACC you.DAT take.IPFV.1SG from.there
\end{verbatim}
already I’m taking him for you from there
\end{transbox}
\end{mdframednoverticalspace}
%
\begin{mdframednoverticalspace}[style=secondfoc]
\xtransbox{7}{~}{((comes to the table and takes the baby))}
\end{mdframednoverticalspace}

Cases of assistance with current or anticipatable trouble can also often be analyzed as eliciting a service of some sort.  What separates them as a category, however, is that A might not have designed their conduct to recruit assistance. Nonetheless, B’s practical action is occasioned by some conduct in what then becomes Move A.

\section{Formats in Move A: The recruiting move}

\subsection{Fully nonverbal recruiting moves}\label{sec:zinken:3.1}

Sometimes recruiting moves are fully nonverbal.  For example, a person can simply point to an object that they want to be given, or they can reach out to receive an object, or -- as in the following case -- a person can hold out an object and thereby recruit another person to take it and do something with it.  In \REF{ex:zinken:5}, Marta and Karol are searching for a particular medication in the fridge.  Marta takes a package out of the fridge, inspects it, and then holds it out for Karol to take it from her.  Karol then takes the package from Marta (\figref{fig:zinken:1}).

\transheader{ex:zinken:5}{(PP6-3\_1920720)}%\vspace{-1mm}
%
\begin{mdframednoverticalspace}[style=firstfoc]
\xtransbox{1}{mat}{((holds out package))}
\end{mdframednoverticalspace}
%
\begin{mdframednoverticalspace}[style=secondfoc]
\xtransbox{2}{kar}{((takes package))}
\end{mdframednoverticalspace}
%
\begin{figure}
\caption{Karol takes the package from Marta (\extref{ex:zinken:5}, line 2).}
\includegraphics[height=0.5\textwidth]{figures/polish-img1.jpg}
\label{fig:zinken:1}
\end{figure}

Fully nonverbal recruiting moves can be successful when the context provides a rich scaffold that secures the other person’s attention and makes the point of the recruitment transparent (\citealt{Rossi2014} and \chapref{sec:rossi}, \sectref{sec:rossi:3.1}; see also Kendrick, \chapref{sec:kendrick}, \sectref{sec:kendrick:4.1.3}; Baranova, \chapref{sec:baranova}, \sectref{sec:baranova:3.1}; Dingemanse, \chapref{sec:dingemanse}, \sectref{sec:dingemanse:3.4}). The recruitment in \REF{ex:zinken:5} occurs in a context in which Marta and Karol are already engaged in the activity of inspecting various packages of medicine they have in the fridge. Fully nonverbal practices make up 6.5\% (\textit{n}=14) of all recruiting moves.

\subsection{Nonverbal behavior in composite recruiting moves}

Moves initiating recruitment in face-to-face interaction often involve a combination of verbal and nonverbal conduct.  For example, when A asks B to pass the butter, A might also gaze towards B, a practice that can serve to address the relevant person \citep{Lerner2003}.  Here, however, we coded only conduct that aids the recipient in identifying the target object and/or action. \tabref{tab:zinken:1} provides an overview of the types of nonverbal behavior found in composite recruiting moves in the Polish data.

\begin{table}
\begin{tabularx}{0.75\textwidth}{Xrr}
\lsptoprule
Nonverbal behavior & Count & Proportion \\
\midrule
Pointing gesture & 26 & 34\%\\
Holding out object & 9 & 11.5\%\\
Reaching to receive object & 9 & 11.5\%\\
Iconic gesture & 0 & 0\%\\
Other & 33 & 43\%\\
\lspbottomrule
\end{tabularx}
\caption{Types of nonverbal behavior in composite recruiting moves (\textit{n}=77).}
\label{tab:zinken:1}
\end{table}

%right column originally labelled % of total nonverbal behaviors
Aga’s turn in \REF{ex:zinken:6} includes two recruiting components, both of which are accompanied by relevant nonverbal conduct.  In line 2, when Aga formulates a request to be given the baby, she also stretches out her arm to receive him (\figref{fig:zinken:2}\textit{a}).  This is an example of a “reach to receive” gesture, although it also has an iconic element, because Aga would not actually grasp the baby with her outstretched hand (when Piotr hands Aga the baby, she takes him with both hands).  After the completion of this recruiting move, Aga immediately launches the next element in her turn: another request for Piotr to sit down (line 4). As she formulates this request, Aga also slightly pushes back the chair (\figref{fig:zinken:2}\textit{b}).  Such manipulation of objects involved in the target event was coded as an “other” form of nonverbal conduct.

\transheader{ex:zinken:6}{PP5-5\_28800a}\vspace{2mm}
%
\begin{transbox}{1}{pio}
\begin{verbatim}
nakarmimy    jego    tutaj?
feed.PFV.1PL him.ACC here
\end{verbatim}
will we feed him here?
\end{transbox}
%
\begin{mdframednoverticalspace}[style=firstfoc]
\begin{transbox}{2}{aga}
\begin{verbatim}
wiesz    co  (.) daj     [mi     go      na chwileczkę=
know.2SG what    give.IMP me.DAT him.ACC on moment.ACC
\end{verbatim}
you know what \hspace{0.3cm} give him to me for a moment
\end{transbox}
\end{mdframednoverticalspace}
%
\begin{mdframednoverticalspace}[style=firstfoc]
\begin{transbox}{3}{~}
\begin{verbatim}
                         [((stretches out arm))
\end{verbatim}
\end{transbox}
\end{mdframednoverticalspace}
%
\begin{mdframednoverticalspace}[style=firstfoc]
\begin{transbox}{4}{~}
\begin{verbatim}
=siedź [sobie tutaj sam  ja go      nakarmię=
 sit.IMP REFL here  self I  him.ACC feed.1SG
\end{verbatim}
\hspace{0.07cm} sit yourself down here, I will feed him
\end{transbox}
\end{mdframednoverticalspace}
%
\begin{mdframednoverticalspace}[style=firstfoc]
\begin{transbox}{5}{~}
\begin{verbatim}
       [((pushes chair back))
\end{verbatim}
\end{transbox}
\end{mdframednoverticalspace}
%
\begin{mdframednoverticalspace}[style=firstfoc]
\begin{transbox}{6}{~}
\begin{verbatim}
=owocową
 fruit.ADV.INS
\end{verbatim}
\hspace{0.07cm} fruit (soup)
\end{transbox}
\end{mdframednoverticalspace}
%
\begin{mdframednoverticalspace}[style=secondfoc]
\xtransbox{7}{pio}{((walks to Aga, hands over baby))}
\end{mdframednoverticalspace}

\begin{figure}
\caption{Frames from \extref{ex:zinken:6}.}
\subfigure[Aga stretches out arm (line 3).]{
  \includegraphics[height=.18\textheight]{figures/polish-img2.jpg}
}
\subfigure[Aga pushes back chair (line 5).]{
  \includegraphics[height=.18\textheight]{figures/polish-img3.jpg}
}
\label{fig:zinken:2}
\end{figure}

% \begin{figure}
% \includegraphics[width=0.75\textwidth]{figures/polish-img2.jpg}
% \caption{Aga stretches out arm (\extref{ex:zinken:6}, line 3).}
% \label{fig:zinken:2}
% \end{figure}
% 
% \begin{figure}
% \caption{Aga pushes back chair (\extref{ex:zinken:6}, line 5).}
% \includegraphics[width=0.75\textwidth]{figures/polish-img3.jpg}
% \label{fig:zinken:3}
% \end{figure}

By pushing back the chair, Aga indicates where Piotr should sit down, but also makes sitting down more straightforward for Piotr. Like pointing gestures, ``other'' forms of nonverbal conduct often indicate a relevant object. However, they also commonly make the object more useable for the intended purpose, and thereby increase the transparency of the recruiting move.  In another example, \REF{ex:zinken:15} below, Bogusia places a salad bowl on a small counter between the kitchen and the living room, saying \textit{jeszcze proszę sałatę} ‘also please the salad’.  Placing the salad bowl there makes it more easily accessible to the others, and, in conjunction with the verbal turn, constitutes a transparent request for somebody to bring the salad to the living room table in preparation for the meal.

\subsection{Verbal elements: construction types and subtypes}\label{sec:zinken:3.3}

Verbal elements in recruiting moves were classified with reference to three cross-linguistically common sentence types: declarative, interrogative, and imperative.  As there is no interrogative syntax in Polish, questions are recognizable through intonation, the use of question particles, and what one speaker knows that the other knows (for a conversation-analytic discussion, see \citealt{Weidner2013a}). Further construction types are recruiting turns without a predicate and others with a verb in the infinitive. Imperative recruiting turns are the most common. At the same time, we find other formats in more than half of recruiting moves with a verbal component (see \tabref{tab:zinken:2}).\footnote{The missing 0.5\% is due to one case in which the verbal component of Move A is inaudible.}

\begin{table}
\begin{tabularx}{0.66\textwidth}{Xrr}
\lsptoprule
Construction type & Count & Proportion\\
\midrule
Imperative & 93 & 47\%\\
Declarative & 46 & 23\%\\
No predicate & 40 & 20\%\\
Question format & 14 & 7\%\\
Infinitive & ~6 & 3\%\\
\lspbottomrule
\end{tabularx}
\caption{Construction type of recruiting moves including spoken elements (\textit{n}=199).}
\label{tab:zinken:2}
\end{table}

\subsubsection{Imperatives}\label{sec:zinken:3.3.1}

%As in the other languages examined in the project, verbal recruiting moves in Polish mostly involve imperative constructions.  
Polish has a relatively complex imperative paradigm.  Morphological imperatives exist for the second person singular and plural, and the first person plural.  Periphrastic hortative constructions exist for the third persons and the first person singular, as well as for formal (V-form) second person reference (on imperatives and hortatives, see \citealt{VanderAuwera2013}).  

In the examined corpus, nearly all imperatives are in the second person.  The only exception is found in \REF{ex:zinken:7} where Henio uses the third person hortative construction \textit{niech} + verb (roughly, `may it x').\footnote{Another instance of this format can be found in \REF{ex:zinken:30}, where the recruitment is initiated but not completed.} Henio’s move at lines 4--5 recruits Bogusia to leave the camera on by saying \textit{niech to jeszcze ten} `may this still that one', where \textit{ten} `that one' is a demonstrative pronoun that in spoken Polish often functions as a dummy term. Here, it stands in for an otherwise expectable third person singular verb, such as \textit{filmuje} or \textit{nagrywa} `it records'.  Bogusia complies at lines 8--9, reformulating the hortative utterance, this time with the “missing” third person predicate.

\transheader{ex:zinken:7}{PP3-1\_2348380a}\vspace{2mm}
%
\xtransbox{1}{bog}{((moving to turn off camera))}
%
\xtransbox{2}{hen}{((getting up from the table))}
%
\begin{transbox}{3}{~}
\begin{verbatim}
dobrze   to   jeszcze jeszcze to   póki Magda
good.ADV then still   still   then while Magda
\end{verbatim}
alright then while Magda is still
\end{transbox}
%
\begin{mdframednoverticalspace}[style=firstfoc]
\begin{transbox}{4}{~}
\begin{verbatim}
je      to   niech to   jeszcze ten
eat.3SG then HRT   this still   that.one
\end{verbatim}
eating then may this still that one
\end{transbox}
\end{mdframednoverticalspace}
%
\emptytransbox{5}{(0.4)}
%
\begin{mdframednoverticalspace}[style=secondfoc]
\begin{transbox}{6}{bog}
\begin{verbatim}
tak¿> osta- [Magda ostatnia od[chodzi   od
yes   las-   Magda last     leave.3SG   from
\end{verbatim}
yes? Las- Magda leaves the table last?
\end{transbox}
\end{mdframednoverticalspace}
%
\begin{transbox}{7}{hen}
\begin{verbatim}
            [(no)             [tak
              PTC              yes
\end{verbatim}
\hspace{2cm} \textit{no} \hspace{2.1cm} yes
\end{transbox}
%
\begin{mdframednoverticalspace}[style=secondfoc]
\begin{transbox}{8}{bog}
\begin{verbatim}
stołu     tak,= niech będzie
table.GEN yes   HRT   be.3SG
\end{verbatim}
right, may it be
\end{transbox}
%
\begin{transbox}{9}{~}
\begin{verbatim}
sfilmowane  no
filmed.PASS PTC
\end{verbatim}
 filmed \textit{no}
\end{transbox}
\end{mdframednoverticalspace}
%
\begin{transbox}{10}{hen}
\begin{verbatim}
niech będzie  sfilmowane
HRT   be.3SG  filmed.PASS
\end{verbatim}
 may it be filmed
\end{transbox}\bigskip

In the remainder of this section, I discuss only second person morphological imperatives.  Among these, I distinguish three turn formats: imperatives with perfective aspect marking, imperatives with imperfective marking, and the (perfective) double imperative, \textit{weź zrób x} (`take do x').  Perfective imperatives are by far the most common in the corpus (\textit{n}=68), followed by imperfective imperatives (\textit{n}=13),\footnote{One of these is the monoaspectual \textit{siedź} ‘be/remain sitting’.} and by the double imperative (\textit{n}=12).

\paragraph{Perfective imperatives.}\label{sec:zinken:3.3.1.1.} Perfective imperatives are the most common subtype of imperative recruiting format in the data.  Work on the selection of imperatives for requesting action demonstrates that such recruiting moves convey an expectation of compliance \citep{Wootton1997,Goodwin2006,CravenPotter2010,Kent2011,kent_compliance_2012,Rossi2012}. This expectation is, in informal interaction among friends and family, typically grounded in the fact that the requested action is integral to a wider activity to which the recipient is already committed \citep{Wootton1997,Rossi2012}.  The two examples of recruiting moves with perfective imperatives that we have seen so far, \REF{ex:zinken:1} and \REF{ex:zinken:6}, illustrate this. In \REF{ex:zinken:1}, Jacek is available for jobs such as passing the salad bowl on the basis of his participation in the mealtime event, and the imperative orients to this availability.  In \REF{ex:zinken:6}, Piotr is already engaged in finding arrangements for feeding his son (see his question in line 1) and the imperative recruiting move is designed as a step in this wider activity (see also \citealt{ZinkenDeppermann2017}).  Recruiting moves formatted as perfective imperatives will be discussed repeatedly in later sections (Extracts \ref{ex:zinken:8}, \ref{ex:zinken:19}, \ref{ex:zinken:22}, \ref{ex:zinken:26}, \ref{ex:zinken:32}, \ref{ex:zinken:34}, \ref{ex:zinken:35}, \ref{ex:zinken:37}) and I therefore do not provide further examples here.

\paragraph{Imperfective imperatives.}\label{sec:zinken:3.3.1.2.} ~ Imperfective imperatives have repeatedly \mbox{concerned} linguists working on Slavic languages \citep[see][]{Forsyth1970,Lehmann1989,benacchio2010}.\footnote{This literature is mostly concerned with Russian data but is relevant also to Polish.} From the perspective of sequential analysis, it is striking that imperfective imperatives are used in positions where the relevant action has already been brought into play by the other person, or is the direct consequence of what has occupied the interaction in the just prior turns.  In other words, although we might think of requests and directives as good examples of sequentially first actions \citep{SorjonenRaevaaraCouperKuhlen2017}, imperfective imperative turns in Polish are never textbook examples of first pair-parts \citep{Schegloff2007}. In fact, imperfective imperatives are often used in second position to accomplish actions such as giving a go-ahead \citep{Lehmann1989,ZinkenDeppermann2017}.  In the domain of recruitments, imperfective imperatives treat an action as already ``authored'' by the other \citep[chap. 8]{Zinken2016}.

In \REF{ex:zinken:8}, Ania is urging her mom Ela to start dinner, because she has to leave in ten minutes (lines 1--2).  The turn-initial \textit{no} in Ela’s agreeing response (line 3) conveys that it is obvious to Ela that the meal is to start now (on turn-initial \textit{no}, see \citealt{Weidner2013b}), while her \textit{już} `already' conveys that in fact everything is on track. She then extends her turn to address a directive to Ania, namely to serve the food for the younger children (line 5).  This recruiting turn is designed with an imperfective imperative.  It begins, again, with the particle \textit{no}, which here expresses Ela’s stance that serving the food is the obvious consequence of Ania’s wish to speed things up.  In response, Ania begins serving the food.

\newpage
\transheader{ex:zinken:8}{PP1-1\_0145540}\vspace{2mm}
%
\begin{transbox}{1}{ani}
\begin{verbatim}
do:bra:  mo- czy możemy  zjeść¿ =bo      ja muszę
good.ADV ca- Q   can.1PL eat.INF because I  must.1SG
\end{verbatim}
okay, can we eat, because I’ll have to
\end{transbox}
%
\begin{transbox}{2}{~}
\begin{verbatim}
za    dziesięć minut   wyjść.
after ten      minutes go.out.INF
\end{verbatim}
leave in ten minutes
\end{transbox}
%
\begin{transbox}{3}{ela}
\begin{verbatim}
no  już:=
PTC already
\end{verbatim}
\textit{no} already
\end{transbox}
%
\begin{transbox}{4}{tad}
\begin{verbatim}
=◦jest już◦
  is   already
\end{verbatim}
\hspace{0.2cm} it’s already done
\end{transbox}
%
\begin{mdframednoverticalspace}[style=firstfoc]
\begin{transbox}{5}{ela}
\begin{verbatim}
↑no  to   już     nakładaj       im.
 PTC then already serve.IMP.IPFV them
\end{verbatim}
\hspace{0.07cm} \textit{no} then serve them (their food) already
\end{transbox}
\end{mdframednoverticalspace}
%
\emptytransbox{6}{(0.5)}
%
\begin{mdframednoverticalspace}[style=firstfoc]
\begin{transbox}{7}{ela}
\begin{verbatim}
mięsko weź           na stół¿ (.) surówka jedna druga¿
meat   take.IMP.PFV on table      slaw    one   other
\end{verbatim}
put the meat on the table, the one salad and the other
\end{transbox}
\end{mdframednoverticalspace}
%
\begin{mdframednoverticalspace}[style=secondfoc]
\xtransbox{8}{ani}{((puts salads on the table))}
\end{mdframednoverticalspace}

We can think of recruiting moves as having a deontic side (telling the other that they should do something) and an information side (telling them what to do).  Imperfective imperatives add little to the information side of a recruiting move. Insofar as the recruitment concerns a new action at all, that action, as in line 5 of \REF{ex:zinken:8}, is framed as a direct consequence of what has come before. Imperfective imperatives mainly deal with the deontic side of the prospective action, that is, they give the go-ahead to, insist on, or prohibit an action that already concerns the other or is inferably relevant (see also \extref{ex:zinken:37}) \citep[chap. 8]{Zinken2016}.  Note that Ela extends her turn with another directive detailing what exactly Ania is supposed to serve the younger children (line 7), reformulating the previous recruiting component \textit{nakładaj} `serve them (the food)' (line 5) in a more informative manner -- this reformulation is done with a perfective imperative.

\paragraph{Double imperative.} A double imperative construction that is recurrent in spoken Polish takes the form of the perfective verb `take' (\textit{wziąć}, imperative: \textit{weź}) plus the relevant action verb, also with perfective aspect.  In \REF{ex:zinken:9}, Jacek is talking to his children about observations the children have made at the local swimming pool.  The turn in line 1 belongs to that conversation.  His wife, Ilona, is in the process of clearing the table, and she is scraping the last bits of a vegetable salad onto Jacek’s plate.  Some of the salad is sticking to the spoon and Ilona recruits Jacek to scrape it off (lines 3--5) while holding out the spoon for him to take.

\transheader{ex:zinken:9}{PP2-5\_2002280}\vspace{2mm}
%
\begin{transbox}{1}{jac}
\begin{verbatim}
trudno    powiedzieć
difficult say.INF
\end{verbatim}
difficult to say
\end{transbox}
%
\begin{transbox}{2}{~}
\begin{verbatim}
dlaczego akurat [(  )
why      exactly
\end{verbatim}
why exactly
\end{transbox}
%
\begin{mdframednoverticalspace}[style=firstfoc]
\begin{transbox}{3}{ilo}
\begin{verbatim}
                [weź      kochanie
                 take.IMP dear
\end{verbatim}
\hspace{2.5cm}take dear
\end{transbox}
\end{mdframednoverticalspace}\vspace{1.5mm}
%
\begin{mdframednoverticalspace}[style=firstfoc]
\begin{transbox}{4}{~}
\begin{verbatim}
                [((holds out spoon for Jacek to take))
\end{verbatim}
\end{transbox}
\end{mdframednoverticalspace}
%
\begin{mdframednoverticalspace}[style=firstfoc]
\begin{transbox}{5}{~}
\begin{verbatim}
zgarnij     z    łyżki
collect.IMP from spoon.GEN
\end{verbatim}
gather (it) from the spoon
\end{transbox}
\end{mdframednoverticalspace}
%
\begin{mdframednoverticalspace}[style=secondfoc]
\xtransbox{6}{jac}{((takes salad spoon, scrapes salad onto his plate))}
\end{mdframednoverticalspace}

Ilona’s recruiting move allows for a serial interpretation (`take the spoon and scrape off the salad'). However, \textit{weź} is grammaticalized to the extent that it is unproblematically used to solicit actions that do not involve taking anything. In other words, it functions as a particle rather than as the imperative of the verb ‘take’ (see \citealt[chap. 7]{Zinken2016}).  Simple perfective imperatives are often selected to recruit actions as part of an established joint project, as outlined in \sectref{sec:zinken:3.3.1.1.}. However, the ‘take-V2’ format is selected in situations in which B’s commitment to the relevant project is not evident, although it often is expectable at a more general level. Here, Jacek is not involved in clearing the table when Ilona makes her request, but he is one of the adults responsible for organizing the mealtime event at a more general level (see also Extracts \ref{ex:zinken:17}, \ref{ex:zinken:30}, \ref{ex:zinken:31}) (\citealt[chap. 7]{Zinken2016}; \citealt{ZinkenDeppermann2017}).

\subsubsection{Declaratives}\label{sec:zinken:3.3.2}

Declarative recruiting turns make up nearly a quarter of all the attempts that contain a verbal element (\textit{n}=46). These can be further divided into subtypes. One group are non-modal descriptions in the third person (\textit{n}=12) as in \REF{ex:zinken:2} (\textit{ser upadł} ‘cheese has dropped down’). Sometimes, third person descriptions can also recruit B by telling another, third person what B will do.  In the present corpus, these are cases where one parent says to a child what the other parent will do, thereby recruiting the adult for that action (such cases are not part of the group of non-modal descriptions, which are restricted to objects requiring action).  In \REF{ex:zinken:10}, for example, Ilona first proposes to the child that ‘we’ can put a special ointment on a scratch that the child has.  Both parents seek and receive approval from the child for this course of action.  At that point, the question remains as to which of the parents will go to get the ointment.  It would seem that Ilona is in a better position to do so, as Jacek has the child on his lap and is feeding him.  However, at line 1, Ilona formulates the target turn, addressed to the child: \textit{tatuś posmaruje takim kremem} ‘dad will put on this ointment’ and in so doing mobilizes Jacek to put the child on an adjacent chair and leave to get the cream.

\transheader{ex:zinken:10}{PP2-1\_3410860}\vspace{-1mm}
%
\begin{mdframednoverticalspace}[style=firstfoc]
\begin{transbox}{1}{ilo}
\begin{verbatim}
tatuś po- posmaruje      takim    kremem
dad       smear.PFV.3SG  such.INS cream.INS
\end{verbatim}
daddy wi- will apply this ointment
\end{transbox}
\end{mdframednoverticalspace}
%
\begin{mdframednoverticalspace}[style=secondfoc]
\xtransbox{2}{jac}{((puts child on adjacent chair, leaves))}
\end{mdframednoverticalspace}

Another group of recruiting turns in declarative format involve impersonal constructions with a verb expressing deontic modality (\textit{n}=9) (see also Floyd, \chapref{sec:floyd}, \sectref{sec:floyd:3.3.4}; Rossi, \chapref{sec:rossi}, \sectref{sec:rossi:3.3.4}; Baranova, \chapref{sec:baranova}, \sectref{sec:baranova:3.3.3}).  In Polish, turns with the impersonal modal verb \textit{trzeba} ‘it is necessary to / one has to’ are a practice for recruiting another person’s collaboration \citep{ZinkenOgiermann2011}.  In \REF{ex:zinken:11}, the family have been crafting together; some glue remains on a piece of paper in the center of the table.  When Marta tells her daughter not to play with the glue, this becomes an occasion for Karol to formulate what `is necessary' to do, namely to throw the glue away (line 5).  In overlap with Karol’s turn, Marta begins extending her arm and then picks up the paper sheet with the glue on it and throws it away (see \citealt[chap. 6]{Zinken2016}).

\transheader{ex:zinken:11}{PP6-1\_4228840}\vspace{2mm}
%
\begin{transbox}{1}{mar}
\begin{verbatim}
to   już     tym      klejem   się  Gabrysiu
this already this.INS glue.INS REFL Gabrysia
\end{verbatim}
don’t play with this glue
\end{transbox}
%
\begin{transbox}{2}{~}
\begin{verbatim}
nie baw           wiesz:¿
not play.IPFV.IMP know.2SG
\end{verbatim}
already Gabrysia
\end{transbox}
%
\emptytransbox{3}{(.)}
%
\begin{transbox}{4}{mar}
\begin{verbatim}
[(on już     troszkę)           ]
  it already a.bit
\end{verbatim}
\hspace{0.2cm} it has already somewhat
\end{transbox}
%
\begin{mdframednoverticalspace}[style=firstfoc]
\begin{transbox}{5}{kar}
\begin{verbatim}
[to   już     trzeba    wyrzucić]
 this already necessary throw.away.INF
\end{verbatim}
\hspace{0.07cm} it is necessary to throw this away already
\end{transbox}
\end{mdframednoverticalspace}
%
\begin{transbox}{6}{mar}
\begin{verbatim}
(.) [zasechł
     dry.3SG.PST
\end{verbatim}
\hspace{0.7cm} dried out
\end{transbox}
%
\begin{mdframednoverticalspace}[style=secondfoc]
\begin{transbox}{7}{~}
\begin{verbatim}
    [((picks up paper sheet with glue))
\end{verbatim}
\end{transbox}
\end{mdframednoverticalspace}


The modal verb \textit{móc} ‘can’ is sometimes used with person marking in turns that recruit another person.  In \REF{ex:zinken:12}, Olek addresses his daughter Kasia with a proposal to give her toddler son (Olek's grandson) something to eat.  He uses an infinitive recruiting form (see \sectref{sec:zinken:3.3.4} below).  Kasia responds by asking her son whether he would like to eat something (line 2) but then turns back to her father and recruits him to mount a contraption designed to hold small children, a kind of cloth child chair, on an ordinary chair (line 3).  This recruiting turn is in the form of a declarative with second person singular marking.\footnote{Note that Kasia extends her recruiting turn at line 6 with a predicate-less unit (see \sectref{sec:zinken:3.3.3} below).}

\transheader{ex:zinken:12}{MiBrApr2012\_0459322}\vspace{2mm}
%
\begin{transbox}{1}{ole}
\begin{verbatim}
jakiejś  wędlinki    może  mu      dać
some.GEN sausage.GEN maybe him.DAT give.INF
\end{verbatim}
maybe (to) give him some sausage
\end{transbox}
%
\begin{transbox}{2}{kas}
\begin{verbatim}
(Józienko)  chcesz    coś       zjeść
 NAME.VOC   want.2SG  something eat.INF
\end{verbatim}
Joseph you want to eat something?\
\end{transbox}
%
\begin{mdframednoverticalspace}[style=firstfoc]
\begin{transbox}{3}{~}
\begin{verbatim}
wiesz    co   możesz  mu     (.)  zmontować ((to Olek))
know.2SG what can.2SG him.DAT     mount.INF
\end{verbatim}
you know what, you can mount for him
\end{transbox}
\end{mdframednoverticalspace}
%
\begin{transbox}{4}{~}
\begin{verbatim}
to   siedzenie (    )
this seat
\end{verbatim}
this seat (\hspace{0.7cm})
\end{transbox}
%
\begin{mdframednoverticalspace}[style=secondfoc]
\xtransbox{5}{ole}{((gets up))}
\end{mdframednoverticalspace}
%
\begin{transbox}{6}{kas}
\begin{verbatim}
tylko    mu  jakąś    poduszkę.
only him.DAT some.ACC cushion.ACC
\end{verbatim}
just (also use) a cushion for him
\end{transbox}
%
\begin{mdframednoverticalspace}[style=secondfoc]
\xtransbox{7}{ole}{((begins mounting child seat on chair))}
\end{mdframednoverticalspace}

Recruiting moves in this format build on the other’s displayed or assumed readiness to contribute to the relevant matter (see also \extref{ex:zinken:27}), in this case on Olek’s suggestion or proposal that the child should or could eat something (line 1).

Turns with a performative verb in the first person are also used to recruit another person’s action (\textit{n}=7).  A turn format that is specialized for recruiting another person to provide an object is built with the verb \textit{prosić/poprosić} ‘plead, ask' (\textit{poprosić} is the perfective form) in the first person plus the item as direct object in the accusative (see \citealt{ogiermann2015}; for a usage of this form outside object requests, see \citealt{Weidner2015}).  In \REF{ex:zinken:13}, Kasia asks Dorota to pass the horseradish. Kasia's talk in line 1 closes a prior, unrelated interaction.

\transheader{ex:zinken:13}{MiBrApr2012\_0643192}\vspace{2mm}
%
\begin{transbox}{1}{kas}
\begin{verbatim}
dobrze   na razie Józio nie chce      siedzieć
good.ADV on time  NAME  not want.3SG  sit.INF
\end{verbatim}
okay, for now Józio doesn’t want to sit
\end{transbox}
%
\begin{mdframednoverticalspace}[style=firstfoc]
\begin{transbox}{2}{~}
\begin{verbatim}
po- poproszę      m- chrzan¿
pl- plead.PFV.1SG m- horseradish.ACC
\end{verbatim}
I ask (for) m- horseradish
\end{transbox}
\end{mdframednoverticalspace}
%
\begin{mdframednoverticalspace}[style=secondfoc]
\xtransbox{3}{dor}{((passes horseradish))}
\end{mdframednoverticalspace}

Announcements in the first person (singular or plural, \textit{n}=4) can also recruit another person’s collaboration. A type that occurs a few times in the corpus involves a family member announcing that `we will say grace' (\textit{pomodlimy się}), which recruits the others to move into the appropriate posture.  Other declarative formats are attested as single cases, such as that of the second person non-modal declarative turn in \REF{ex:zinken:14}. Paweł and Klaudia are preparing a salad and in line 1 Klaudia brings a peeled cucumber to where Paweł is standing, for him to slice.  In line 2, Paweł recruits Klaudia to give him a bowl; in response, Klaudia turns to the cupboard and gets a bowl out.

\transheader{ex:zinken:14}{PP4-1\_0812980}\vspace{2mm}
%
\emptytransbox{1}{(4.0) ((Klaudia walks towards Paweł))}
%
\begin{mdframednoverticalspace}[style=firstfoc]
\begin{transbox}{2}{paw}
\begin{verbatim}
dałabyś             mi     m- (.) miskę,
give.2SG.F.PST.COND me.DAT m-     bowl.AKK
\end{verbatim}
you’d give me a b- (.) bowl
\end{transbox}
\end{mdframednoverticalspace}
%
\begin{mdframednoverticalspace}[style=secondfoc]
\xtransbox{3}{kla}{(turns to cupboard, gets bowl))}
\end{mdframednoverticalspace}

At first glance, this turn looks just like a request formulated as a second person yes/no question (see \sectref{sec:zinken:3.3.4} below).  However, in this sequential position, the prosody of the turn -- with stress on the first syllable of \textit{dałabyś} ‘you’d give’ and level turn-final intonation -- clearly marks it as a statement.

\subsubsection{No predicate}\label{sec:zinken:3.3.3}

Recruiting turns without a predicate are common in the Polish corpus (see also \extref{ex:zinken:28}).  These are most often names of objects (\textit{n}=16) requiring some action.  In \REF{ex:zinken:15}, the family are busy laying the table for supper.  Talk is about a near-accident that the family dog has had with a car (lines 1--5).  Bogusia is taking things out of the fridge.  At line 7, she puts a bowl of salad onto the worktop and says \textit{jeszcze proszę sałatę} ‘also, please, salad’, recruiting an unspecified family member to take the salad and put it on the table.  \textit{Proszę} ‘I plead, please’ is the imperfective form of the same verb that we have encountered in the object request in \REF{ex:zinken:13}.   This imperfective form is commonly used in actions of passing or offering an object to another person.  In this function, it is best translated as ‘please’ or ‘here you are’. \textit{Sałatę} ‘salad’ here is not an argument of \textit{proszę} ‘I plead, here you are’ but a stand-alone item naming the object that has been made available for taking by somebody.

\transheader{ex:zinken:15}{PP3-2\_0338665a}\vspace{2mm}
%
\begin{transbox}{1}{bog}
\begin{verbatim}
nie zauważyła      samochodu¿
not notice.3SG.PST car.GEN
\end{verbatim}
she didn’t notice the car?
\end{transbox}
%
\begin{transbox}{2}{mag}
\begin{verbatim}
nie zauważyła      bo      ona siedz[iała  tyłem
not notice.3SG.PST because she sit.3SG.PST back.INS
\end{verbatim}
she didn’t notice cause she was sitting backward
\end{transbox}
%
\begin{transbox}{3}{bog}
\begin{verbatim}
                                    [ona zawsze
\end{verbatim}
\hspace{5.4cm} she always
\end{transbox}
%
\begin{transbox}{4}{~}
\begin{verbatim}
ucieka  przed samochodem
run.3SG from  car.INS
\end{verbatim}
runs away from the car
\end{transbox}
%
\begin{transbox}{5}{mag}
\begin{verbatim}
znaczy   była    tyłem,   znaczy
mean.3SG was.3SG back.INS mean.3SG
\end{verbatim}
that is, she was with her back, that is,
\end{transbox}
%
\begin{transbox}{6}{~}
\begin{verbatim}
[tyłem    była
 back.INS was.3SG
\end{verbatim}
\hspace{0.07cm} with her back
\end{transbox}
%
\begin{mdframednoverticalspace}[style=firstfoc]
\begin{transbox}{7}{bog}
\begin{verbatim}
[jeszcze proszę    sałatę¿ ((puts salad bowl onto work top))
 also    plead.1SG salad.ACC
\end{verbatim}
\hspace{0.07cm} also, please, salad
\end{transbox}
\end{mdframednoverticalspace}
%
\xtransbox{8}{~}{((remaining family members look at and talk to the dog))}\\

Naming an object does not select a particular person for the job at hand.  A generic danger of such an “untargeted” recruiting move is that others can choose not to feel addressed (unless addressing is done in other ways, e.g., through gaze).  This is what happens here: all the remaining family members have turned to the dog, and the recruiting move remains unanswered (and is pursued by Bogusia a few moments later).

Some recruiting turns without a predicate only ``activate'' another person with a vocative, leaving the required action to be inferred.\footnote{In this project, we use the term “vocative” to refer to proper names addressing the recruiting move to a person, and not just in relation to vocative case.}  In \REF{ex:zinken:16}, the family are preparing for a craft activity with their children, making an earthworm.  This preparation involves getting the children to come to sit down at the table (lines 1--3) and making space on it.  Marta is in the process of stowing things away in a cupboard; Karol is on the other side of the table.  He picks up a piece of crockery that is on the table and places it closer to Marta, saying \textit{mamuś}\textit{ka} ‘mommy’ (line 6).  This initiates a recruitment for Marta to stow away the crockery as well and thereby make more space on the table.  It might be that Marta first misunderstands Karol’s recruiting turn as summoning her to the table (in extension to his directives towards his daughter a bit earlier, lines 1 and 3).  Her initial response (line 7) is fitted to either recruitment -- to sit down or to clear away the piece of crockery -- but her subsequent, redone response (line 9) is specifically fitted to a recruitment to clear away the crockery.

\transheader{ex:zinken:16}{PP6-1\_8650}\vspace{2mm}
%
\begin{transbox}{1}{kar}
\begin{verbatim}
siadaj       Julka
sit.IPFV.IMP Name
\end{verbatim}
sit down, Julka
\end{transbox}
%
\emptytransbox{2}{(0.2)}
%
\begin{transbox}{3}{kar}
\begin{verbatim}
Julka siadaj       będziemy robić    dżdżownicę
Name  sit.IPFV.IMP will.1PL make.INF worm.ACC
\end{verbatim}
Julka, sit down, we’ll make the worm
\end{transbox}
%
\begin{transbox}{4}{jul}
\begin{verbatim}
no  wsz::yscy razem
PTC all       together
\end{verbatim}
\textit{no} all together
\end{transbox}
%
\emptytransbox{5}{(0.2)}
%
\begin{mdframednoverticalspace}[style=firstfoc]
\begin{transbox}{6}{kar}
\begin{verbatim}
mamuśka ((places piece of crockery closer to Marta))
mom.DIM
\end{verbatim}
mommy 
\end{transbox}
\end{mdframednoverticalspace}
%
\begin{mdframednoverticalspace}[style=secondfoc]
\begin{transbox}{7}{mar}
\begin{verbatim}
no  już
PTC already
\end{verbatim}
\textit{no} already / just a second
\end{transbox}
\end{mdframednoverticalspace}
%
\xtransbox{8}{~}{((50 seconds omitted))}
%
\begin{mdframednoverticalspace}[style=secondfoc]
\begin{transbox}{9}{mar}
\begin{verbatim}
sprzątnę
clean.PFV.1SG
\end{verbatim}
I’ll clear it
\end{transbox}
\end{mdframednoverticalspace}
%
\xtransbox{~}{~}{((50 seconds omitted))}
%
\begin{mdframednoverticalspace}[style=secondfoc]
\xtransbox{10}{mar}{((removes crockery from table))}
\end{mdframednoverticalspace}\vspace{-2mm}

\subsubsection{Question formats}\label{sec:zinken:3.3.4}

Since Polish does not have interrogative morphosyntax, I speak more generally of ``question formats'' in this section.  Question formats are morphosyntactically equivalent to declaratives in Polish.  But they become recognizable as questions through prosody and the distribution of epistemic rights among participants \citep{Weidner2013a} or the use of question words in the case of content questions. The particle \textit{czy} can be used turn-initially to mark a polar question but is rarely used in spoken Polish and not at all in the data at hand. About 7\% of all recruiting moves in the Polish corpus (\textit{n}=15) have a question format.\footnote{In addition, a few of the cases described below as infinitives could also be counted as “questions”.}

Nine recruiting moves in question format project a polar response.  Polar questions are sometimes used to indirectly recruit B for some action.  In \REF{ex:zinken:17}, the family have sat down for supper, and at line 1 one of the sons implores the parents (both of them, using a second person plural double imperative) to turn on the TV.  His mom, Aga, turns to dad, Piotr, with a question: \textit{włączymy¿} `do we turn it on?'.  A yes-response to this question would imply that somebody should now turn on the TV.  In the current situation, seating arrangements are such that Piotr is best placed to do such a job, while Aga is also feeding the baby.  Immediately upon completion of the question, Piotr turns his gaze towards the TV and quickly moves up his torso, presumably getting up from the table.  Ultimately, though, he interrupts that movement and rejects the proposal to turn on the TV, because this would make it more difficult for the researchers to listen to the recording (line 7).

\transheader{ex:zinken:17}{PP5-5\_47880b}\vspace{2mm}
%
\begin{transbox}{1}{prz}
\begin{verbatim}
weźcie           włączcie             (   )
take.PFV.IMP.2PL turn.on.PFV.IMP.2PL
\end{verbatim}
 come on, turn it on (you two)
\end{transbox}
%
\emptytransbox{2}{(0.6)}
%
\begin{mdframednoverticalspace}[style=firstfoc]
\begin{transbox}{3}{aga}
\begin{verbatim}
włączymy¿ ((gazes at Piotr))
turn.on.PFV.1PL
\end{verbatim}
(do) we turn it on?
\end{transbox}
\end{mdframednoverticalspace}
%
\begin{mdframednoverticalspace}[style=secondfoc]
\begin{transbox}{4}{pio}
\begin{verbatim}
((quick upward movement, gaze to TV))
\end{verbatim}
\end{transbox}
\end{mdframednoverticalspace}
%
\begin{transbox}{5}{prz}
\begin{verbatim}
°no  włącz°
PTC  turn.on.PFV.IMP
\end{verbatim}
\textit{no} turn (it) on
\end{transbox}
%
\xtransbox{6}{pio}{((sits back down, gaze to Aga))}\vspace{-1mm}
%
\begin{mdframednoverticalspace}[style=secondfoc]
\begin{transbox}{7}{~}
\begin{verbatim}
nie:: bo      nie będą     nas    słyszeli
no    because not will.3PL us.ACC hear.PST.3PL
\end{verbatim}
no because they won’t be able to hear us
\end{transbox}
\end{mdframednoverticalspace}\bigskip

Recruiting turns that, at least at first glance, merely ask for a decision or for information are in danger of being treated as just that.  In \REF{ex:zinken:18}, Karol and his two daughters are sitting at a table, preparing for a crafts activity.  Marta is not sitting yet but standing behind one of the daughters, doing the girl’s hair.  At line 1, she asks her husband, who is sitting at the other side of the table, \textit{masz tam wolne jedno krzesełko} ‘do you have one free stool there?’.  It might be evident to Karol that Marta is asking that question because she still needs a stool to sit on at her side of the table.  However, he does not take the opportunity to hand a stool to her -- that is, he does not take up her question as a recruiting move -- but instead treats ‘do you have one free stool there?’ merely as a request for information.

\transheader{ex:zinken:18}{PP6-1\_0520400}\vspace{2mm}
%
\emptytransbox{1}{(3.4)}
%
\begin{mdframednoverticalspace}[style=firstfoc]
\begin{transbox}{2}{mar}
\begin{verbatim}
masz     tam   wolne jedno krzesełko?
have.2SG there free  one   stool
\end{verbatim}
do you have one free stool there?
\end{transbox}
\end{mdframednoverticalspace}
%
\begin{mdframednoverticalspace}[style=secondfoc]
\begin{transbox}{3}{kar}
\begin{verbatim}
mam
have.1SG
\end{verbatim}
I do
\end{transbox}
\end{mdframednoverticalspace}
%
\xtransbox{4}{mar}{((walks around table, carries stool to her place))}

Questions that (ostensibly) ask whether B can or will do some action are used conventionally in various languages to recruit another person (see, e.g., Kendrick, \chapref{sec:kendrick}, \sectref{sec:kendrick:4.2.1} on English).  These question formats are understood as recruiting moves also in Polish.  In the present Polish corpus, however, such recruiting moves are rare (\textit{n}=4). What is more, they are resisted in three out of four cases, either by ignoring the recruiting move altogether (see below, \sectref{sec:zinken:4}) or by overtly displaying annoyance while complying.  This indicates that, in Polish, this format might be restricted to attempts at recruitment that are judged by A to be particularly sensitive (see also \citealt[chap. 4]{Rossi2015a} and \chapref{sec:rossi} of this volume, \sectref{sec:rossi:3.3.3} on ‘can you x’ requests in Italian).  On the other hand, in the one case where the recruitment is not resisted, shown below as \REF{ex:zinken:19}, the question format is treated as overly cautious (see also \citealt{ZinkenOgiermann2013} on a similar case).  

Dorota wants Wiesia to take a plate out of the cupboard for Dorota’s granddaughter.\footnote{\citet[chap. 4]{Zinken2016} provides a more detailed discussion of this case.} She first formulates a perfective imperative turn (line 4), but seeing that Wiesia has already started moving toward the table to sit down (line 3) as just suggested by Kasia (line 1), Dorota immediately changes the formulation of her recruiting turn.  This is in polar question format (line 5), selected here possibly because complying will now require Wiesia to depart from her current trajectory of sitting down \citep[see][]{Wootton1997,Rossi2012}.  The verbal response accompanying the nonverbal compliance (line 6) begins with the particle \textit{no}, which in turn-initial position can indicate that the previous turn communicated something that is obvious \citep{Weidner2013b}, followed by \textit{tak} `yes' with marked prosody involving high pitch onset and lengthening (see also \citealt{bolden2017} for Russian).  As a whole, this verbal response seems to indicate that Wiesia’s compliance with the request is obvious and need not have been questioned.

\transheader{ex:zinken:19}{Pa02Apr2012\_0823880}\vspace{2mm}
%
\begin{transbox}{1}{kas}
\begin{verbatim}
siadasz?
sit.IPFV.2SG
\end{verbatim}
are you sitting down?
\end{transbox}
%
\emptytransbox{2}{(0.6)}
%
\xtransbox{3}{wie}{[((starts moving towards table, then stops))}
%
\begin{mdframednoverticalspace}[style=firstfoc]
\begin{transbox}{4}{dor}
\begin{verbatim}
[↑daj         jej     talerz mamo    (przepraszam)
 give.PFV.IMP her.DAT plate  mom.VOC (apologize.1SG)
\end{verbatim}
\hspace{0.07cm} give her a plate mom (I’m sorry)
\end{transbox}
\end{mdframednoverticalspace}
%
\begin{mdframednoverticalspace}[style=firstfoc]
\begin{transbox}{5}{~}
\begin{verbatim}
dasz         jej¿
give.PFV.2SG her.DAT
\end{verbatim}
(will) you give her?
\end{transbox}
\end{mdframednoverticalspace}
%
\begin{mdframednoverticalspace}[style=secondfoc]
\begin{transbox}{6}{wie}
\begin{verbatim}
no  t↑a::k ((turns to cupboard for plates))
PTC yes
\end{verbatim}
\textit{no} yes
\end{transbox}
\end{mdframednoverticalspace}
%
\begin{mdframednoverticalspace}[style=secondfoc]
\xtransbox{7}{~}{((walks towards cupboard))}
\end{mdframednoverticalspace}

Other recruiting turns in question format are used even more rarely, and are attested only as single cases in the corpus. For example, a speaker can try to get another person to stop doing something by (ostensively) demanding an account (\textit{po co robisz x} `why are you doing x').\footnote{See \REF{ex:zinken:35} for a use of this format as a way of rejecting recruitment.} Or they might ask `who will do x' to get somebody to volunteer (\textit{kto wyjmuje naczynia ze zmywarki} `who is taking the dishes out of the dishwasher').

\subsubsection{Infinitive}\label{sec:zinken:3.3.5}

Infinitive constructions are functionally versatile in a way that is particularly relevant to the domain of recruitments. Depending on context, prosody, and lexical turn construction, they can embody various ``directive-commissive'' actions from requests to offers to suggestions to proposals (see \citealt{couper-kuhlen_what_2014}). Similar to turns in the no-predicate category, infinitive turns cannot be categorized for sentence type. One way of thinking about this construction is to treat it as an elliptical construction that has developed out of a modal (declarative or interrogative) sentence, e.g. ‘[you must] tie your laces’, ‘[shall we] make a salad?’ etc. (see \citealt{Deppermann2006} on such ``deontic infinitives'' in German).  Striking features of this construction as a recruiting move are its modal vagueness (it is not always clear whether the relevant action is something that must or could or should be done) and its impersonality: it does not formally specify who should or must do the relevant action.  Consider \REF{ex:zinken:20}, presented earlier as \REF{ex:zinken:12}.  In line 1, Olek suggests to his daughter that her son (his grandson) should or could be given some sausage to eat, using an infinitive turn.

\transheader{ex:zinken:20}{MiBrApr2012\_0456292}\vspace{-1mm}
%
\begin{mdframednoverticalspace}[style=firstfoc]
\begin{transbox}{1}{ole}
\begin{verbatim}
jakiejś  wędlinki    może  mu      dać
some.GEN sausage.GEN maybe him.DAT give.INF
\end{verbatim}
maybe to give him some sausage / maybe he should be given some sausage
\end{transbox}
\end{mdframednoverticalspace}
%
\begin{mdframednoverticalspace}[style=secondfoc]
\begin{transbox}{2}{kas}
\begin{verbatim}
(Józienko) chcesz   coś       zjeść
 NAME.VOC  want.2SG something eat.INF
\end{verbatim}
Joseph you want to eat something?
\end{transbox}
\end{mdframednoverticalspace}
%
\begin{transbox}{3}{~}
\begin{verbatim}
wiesz    co   możesz  mu     (.)  zmontować
know.2SG what can.2SG him.DAT     mount.INF
\end{verbatim}
you know what, you can mount for him
\end{transbox}
%
\begin{transbox}{4}{~}
\begin{verbatim}
to   siedzenie (z    y::)
this seat       with INTJ
\end{verbatim}
this seat (\hspace{0.8cm})
\end{transbox}\bigskip

Olek’s turn in line 1 could be a strong suggestion that Kasia could give the child some sausage, or it could equally be a tentative proposal that she give the child some sausage. In response, Kasia asks her son whether he would like to eat, but then moves to a counter-request for Olek to mount the child seat in preparation for the child’s meal.  All cases of (deontic) infinitives in the corpus are mitigated with \textit{może} `maybe', which gives them the quality of a suggestion or proposal, rather than of a blunt order (see also \citealt{Wierzbicka1991,Krolakrudnicka2006}).

\subsection{Additional verbal elements}

Like recruitment sequences, individual recruiting moves can be more or less complex.  In this section, I consider verbal elements beyond those required by the argument structure of the predicate.  These include mitigators or strengtheners, vocatives, the provision of reasons in a turn with multiple turn-constructional units, benefactives, or adverbs that suggest a connection of the recruited action to ongoing activities.

\subsubsection{Mitigators and strengtheners}

The previous case \REF{ex:zinken:20} already provided an example of an additional verbal element, namely the mitigator \textit{może} `maybe', which softened the deontic force of the infinitive.  The following recruiting move has the form of an impersonal declarative (see also \sectref{sec:zinken:3.3.2} above).  After Ilona and Jacek have agreed to swap childcare duties so that Jacek could finish his meal (lines 1--7), Ilona formulates the target turn, initiating recruitment for Jacek to turn off the camera (lines 8--9).  This turn contains elements that mitigate the recruiting move and seem to acquiesce to an earlier suggestion (\textit{chyba} `probably' and \textit{faktycznie} `really').

\transheader{ex:zinken:21}{PP2-2\_2315590}\vspace{2mm}
%
\begin{transbox}{1}{ilo}
\begin{verbatim}
może  ja się  nim    zajmę      skończysz  co¿
maybe I  REFL he.INS occupy.1SG finish.2SG what
\end{verbatim}
maybe I take care of him, you finish, okay?
\end{transbox}
%
\emptytransbox{2}{(1.0)}
%
\begin{transbox}{3}{ilo}
\begin{verbatim}
skończysz  y::  z:jeść.
finish.2SG INTJ eat.INF
\end{verbatim}
you finish eh eating
\end{transbox}
%
\emptytransbox{4}{(.)}
%
\begin{transbox}{5}{jac}
\begin{verbatim}
dobrz[e
good.ADV
\end{verbatim}
okay
\end{transbox}
%
\begin{transbox}{6}{ilo}
\begin{verbatim}
     [skończysz¿
      finish.2SG
\end{verbatim}
\hspace{0.8cm} you finish?
\end{transbox}
%
\emptytransbox{7}{(.)}
%
\begin{mdframednoverticalspace}[style=firstfoc]
\begin{transbox}{8}{ilo}
\begin{verbatim}
dobrze   ↑to   chyba    już     można
good.ADV  then probably already possible
\end{verbatim}
okay, then it is probably really already possible
\end{transbox}
\end{mdframednoverticalspace}
%
\begin{mdframednoverticalspace}[style=firstfoc]
\begin{transbox}{9}{~}
\begin{verbatim}
wyłączyć faktycznie
turn.off really
\end{verbatim}
to turn (it) off
\end{transbox}
\end{mdframednoverticalspace}
%
\begin{mdframednoverticalspace}[style=secondfoc]
\xtransbox{10}{jac}{((turns camera off))}
\end{mdframednoverticalspace}\vspace{-2mm}

\subsubsection{Vocatives}

Vocatives are present in roughly 12\% of all recruiting moves (\textit{n}=27). Vocatives can be inserted at the beginning of the recruiting move to single out the addressed party and mobilize the addressee’s attention.  In \REF{ex:zinken:22}, Jacek is involved in a conversation with his children, Asia and Bolek, with his body facing them.  Ilona’s recruiting move in line 6 begins with a substantial portion of talk that is preliminary to the request, and that can serve to ascertain that Jacek will be attending to Ilona’s talk by the time the request is formulated.

\transheader{ex:zinken:22}{PP2-5\_1423040}\vspace{2mm}
%
\begin{transbox}{1}{jac}
\begin{verbatim}
ale (.) każdy (0.4) wia[domo że   woli       (.)
but     every       known    that prefer.3SG
\end{verbatim}
but everybody, it’s clear, prefers
\end{transbox}
%
\begin{transbox}{2}{asi}
\begin{verbatim}
                       [ma       swoją intymność
                        have.3SG their intimacy
\end{verbatim}
\hspace{3.5cm} has their privacy
\end{transbox}
%
\begin{transbox}{3}{jac}
\begin{verbatim}
tak ma       swoją intymność i   woli
yes have.3SG their intimacy  and prefer.3SG
\end{verbatim}
yes, has their private sphere and prefers
\end{transbox}
%
\begin{transbox}{4}{bol}
\begin{verbatim}
˚↑wiem¿˚
  know.1SG
\end{verbatim}
\hspace{0.07cm} I know
\end{transbox}
%
\begin{transbox}{5}{jac}
\begin{verbatim}
czasem    żeby    go  nie oglądali      wszyscy
sometimes so.that him not watch.PST.3PL all
\end{verbatim}
sometimes that everybody doesn’t look at them
\end{transbox}
%
\begin{mdframednoverticalspace}[style=firstfoc]
\begin{transbox}{6}{ilo}
\begin{verbatim}
y    wiesz    co   kochanie ↑podaj        mi: serwetkę¿
INTJ know.2SG what love.VOC  pass.IMP.PFV me  napkin
\end{verbatim}
eh you know what dear, pass me a napkin
\end{transbox}
\end{mdframednoverticalspace}
%
\begin{transbox}{7}{asi}
\begin{verbatim}
[mogą    też  się  śmiać.
 can.3PL also REFL laugh
\end{verbatim}
\hspace{0.07cm} they can also be laughing
\end{transbox}
%
\begin{transbox}{8}{jac}
\begin{verbatim}
[↑wiadomo że  (0.4.) wiadomo że  ↑y:   nie to  [nie o
 known    that       known   that INTJ no  this not about
\end{verbatim}
\hspace{0.07cm} it’s clear that, it’s clear that eh no it, that’s
\end{transbox}
%
\begin{transbox}{9}{asi}
\begin{verbatim}
                                               [am:::
                                                INTJ
\end{verbatim}
\end{transbox}
%
\begin{transbox}{10}{jac}
\begin{verbatim}
to   nawet [cho:dchi ↓ale=
this even    go.3SG    but
\end{verbatim}
not really the point but
\end{transbox}
%
\begin{mdframednoverticalspace}[style=secondfoc]
\begin{transbox}{11}{~}
\begin{verbatim}
           [((passes tissue))
\end{verbatim}
\end{transbox}
\end{mdframednoverticalspace}\vspace{-2mm}
%
\begin{transbox}{12}{ilo}
\begin{verbatim}
           [˚o   dziękuję˚
            PTC  thank.1SG
\end{verbatim}
\hspace{1.8cm} \textit{o} thanks
\end{transbox}\bigskip

Sometimes, vocatives are inserted at the end (\textit{n}=7) or in the middle (\textit{n}=6) of a recruiting turn-constructional unit (TCU).  \extref{ex:zinken:1}, reproduced here as \REF{ex:zinken:23}, illustrates such a case, where a vocative is inserted after a move has become recognizable as a recruiting one but before the TCU’s possible completion.  Jacek is gazing at Ilona, and when she turns her gaze to him, she formulates a request for the salad bowl.  The request turn begins with a turn-initial element, \textit{wiesz co} `you know what' (line 2), like \REF{ex:zinken:22} did.  The vocative \textit{kochanie} `dear' (line 2) comes after Ilona has told Jacek to give her something, but before telling him what to give her.  Such a vocative can do work to disambiguate between potential addressees, although this does not seem to be the case here: Jacek is already being addressed through gaze, and the only other people present are two young children who are engaged in a separate conversation.  Turn-final and turn-medial vocatives might rather be doing some affiliational work in recruitments, as we will see in \sectref{sec:zinken:6} below (cf. \citealt{Lerner2003}). Possible functional differences between these two positions will require further research to be elucidated.

\transheader{ex:zinken:23}{PP2-1\_2224980}\vspace{2mm}
%
\xtransbox{1}{jac}{((gaze to Ilona))}
%
\begin{mdframednoverticalspace}[style=firstfoc]
\begin{transbox}{2}{ilo}
\begin{verbatim}
wiesz    co   podaj    mi kochanie jeszcze
know.2SG what pass.IMP me dear     still
\end{verbatim}
you know what, pass me some more
\end{transbox}
\end{mdframednoverticalspace}
%
\begin{transbox}{3}{kas}
\begin{verbatim}
sałatki
salad.GEN
\end{verbatim}
salad, dear
\end{transbox}
%
\begin{mdframednoverticalspace}[style=secondfoc]
\begin{transbox}{4}{jac}
\begin{verbatim}
bardzo proszę ((passes salad bowl))
very   plead.1SG
\end{verbatim}
here you are
\end{transbox}
\end{mdframednoverticalspace}

\subsubsection{Reasons}
\largerpage
Sometimes, speakers give a reason for recruitment (\textit{n}=21).  Reasons can be given to make a request easier to understand and comply with \citep{BaranovaDingemanse2016}. In \REF{ex:zinken:24}, the recruiting move might be barely intelligible without the appended reason.  Aga is holding her baby Feliks in her arms, and the baby has fallen asleep. Piotr, the family father, is admonishing the two sons, Przemek and Łukasz, to stop mucking about.  At line 3, Aga admonished the others to be quiet -- a recruitment that might be difficult to make sense of, and be hardly acceptable to the others without the subsequent reason.

\transheader{ex:zinken:24}{PP5-1\_301160}\vspace{2mm}
%
\begin{transbox}{1}{prz}
\begin{verbatim}
hehehe
\end{verbatim}
\end{transbox}
%
\begin{transbox}{2}{pio}
\begin{verbatim}
je::dz       (że)   Łukasz n[o:
eat.IPFV.IMP (that) Lukasz PTC
\end{verbatim}
eat now Lukasz \textit{no}
\end{transbox}
%
\begin{mdframednoverticalspace}[style=firstfoc]
\begin{transbox}{3}{aga}
\begin{verbatim}
                            [sz::
                             sh
\end{verbatim}
\end{transbox}
\end{mdframednoverticalspace}
%
\begin{mdframednoverticalspace}[style=firstfoc]
\begin{transbox}{4}{~}
\begin{verbatim}
bo      Feliks mi     zasnął
because NAME   me.DAT fall.asleep.3SG.PST
\end{verbatim}
because Feliks has fallen (me) asleep
\end{transbox}
\end{mdframednoverticalspace}
%
\begin{mdframednoverticalspace}[style=secondfoc]
\emptytransbox{5}{((Piotr, Przemek, Łukasz gaze at Aga))}
\end{mdframednoverticalspace}

But reasons can also have other interactional motivations.  In \REF{ex:zinken:25}, the provision of a reason seems to be mainly a vehicle for doing affiliational work between partners. The pair’s toddler, Staś, has been pleading to get a dummy for some time (also in line 1).  In line 3, his mom Ilona gives in.  She recruits her partner Jacek to bring the dummy, and she expands this recruiting move with a reason that expresses her exasperation in a humorous way.

\transheader{ex:zinken:25}{PP2-2\_1616090}\vspace{2mm}
%
\begin{transbox}{1}{sta}
\begin{verbatim}
khykhy Hha .Hh::=monia?
                 dummy?
\end{verbatim}
\end{transbox}
%
\emptytransbox{2}{(0.8)}
%
\begin{mdframednoverticalspace}[style=firstfoc]
\begin{transbox}{3}{ilo}
\begin{verbatim}
monia. monia=tatusiu przy[nieś tego
dummy  dummy daddy   bring.IMP this
\end{verbatim}
dummy, dummy, daddy get that
\end{transbox}
\end{mdframednoverticalspace}
%
\begin{mdframednoverticalspace}[style=secondfoc]
\begin{transbox}{4}{jac}
\begin{verbatim}
                         [już.
                          already
\end{verbatim}
\end{transbox}
\end{mdframednoverticalspace}
%
\begin{mdframednoverticalspace}[style=firstfoc]
\begin{transbox}{5}{ilo}
\begin{verbatim}
monia bo      ja dostanę:  [choroby nerwowej
dummy because I  get.1SG    illness nervous
\end{verbatim}
dummy because I am having a nervous breakdown
\end{transbox}
\end{mdframednoverticalspace}
%
\begin{mdframednoverticalspace}[style=secondfoc]
\begin{transbox}{6}{jac}
\begin{verbatim}
                           [((puts down cutlery))
\end{verbatim}
\end{transbox}
\end{mdframednoverticalspace}\vspace{-1mm}
%
\begin{mdframednoverticalspace}[style=secondfoc]
\begin{transbox}{7}{~}
\begin{verbatim}
((gets up))
\end{verbatim}
\end{transbox}
\end{mdframednoverticalspace}

Reasons are not always introduced with a \textit{bo} `because' and appended to the recruiting component.  In \REF{ex:zinken:26}, Kasia starts her turn with an observation: the toddler fed by Wiesia has a runny nose.  This observation then becomes the grounds on which Kasia incrementally builds an extended recruiting turn.\footnote{In fact, the observation might have been sufficient to mobilize Dorota to get a tissue: Dorota starts getting up after the first word of the recruiting TCU, before Kasia has formulated the object she wants to be passed.}

\transheader{ex:zinken:26}{Pa02Apr2012\_1127560}\vspace{2mm}
%
\begin{transbox}{1}{wie}
\begin{verbatim}
czekaj        mniejszy kawa[łek
wait.IPFV.IMP smaller  piece
\end{verbatim}
wait, a smaller piece
\end{transbox}
%
\begin{mdframednoverticalspace}[style=firstfoc]
\begin{transbox}{2}{kas}
\begin{verbatim}
                           [katar
                            cold/runny nose
\end{verbatim}
\end{transbox}
\end{mdframednoverticalspace}
%
\begin{mdframednoverticalspace}[style=firstfoc]
\begin{transbox}{3}{~}
\begin{verbatim}
↑podaj       [husteczkę ((point towards tissues))
pass.PFV.IMP  tissue.ACC
\end{verbatim}
pass a tissue 
\end{transbox}
\end{mdframednoverticalspace}
%
\begin{mdframednoverticalspace}[style=secondfoc]
\begin{transbox}{4}{dor}
\begin{verbatim}
             [((gets up))
\end{verbatim}
\end{transbox}
\end{mdframednoverticalspace}\vspace{-2mm}

\subsubsection{Benefactives}

Speakers sometimes formulate the beneficiary of the recruitment, which may be the recruiter (\textit{n}=14) or another participant (\textit{n}=28), usually a child. However, as \REF{ex:zinken:26} illustrated, formulating the beneficiary is not obligatory in spoken Polish even with recruitments that involve ‘giving’ or ‘passing’ something.  The question therefore arises as to what function benefactives serve.  One context in which benefactives are used is contrastive, as shown in the next example, where Bogusia is getting Magda her promised dessert, biscuits, and Henio recruits Bogusia to get something else for him (line 4).

\transheader{ex:zinken:27}{PP3-1\_1236810}\vspace{2mm}
%
\begin{transbox}{1}{bog}
\begin{verbatim}
dobrze    dobrze   już     wyjmę            te    pieguski=
good.ADV. good.ADV already take.out.PFV.1SG these cookies
\end{verbatim}
okay okay, I’m already taking the cookies out
\end{transbox}
%
\begin{transbox}{2}{mag}
\begin{verbatim}
=pieguski marki:zy >pieguski mark[izy
 cookies   biscuits  cookies  biscuits
\end{verbatim}
\end{transbox}
%
\begin{transbox}{3}{hen}
\begin{verbatim}
                                 [to-
                                  then-
\end{verbatim}
\end{transbox}
%
\begin{mdframednoverticalspace}[style=firstfoc]
\begin{transbox}{4}{~}
\begin{verbatim}
a   mi     możecie dać      tego     piernika
and me.DAT can.2PL give.INF this.GEN gingerbread.GEN
\end{verbatim}
and to me you can give that gingerbread
\end{transbox}
\end{mdframednoverticalspace}
%
\emptytransbox{5}{(0.8)}
%
\begin{mdframednoverticalspace}[style=secondfoc]
\begin{transbox}{6}{bog}
\begin{verbatim}
dobrze
good.ADV
\end{verbatim}
okay
\end{transbox}
\end{mdframednoverticalspace}
%
\begin{mdframednoverticalspace}[style=secondfoc]
\xtransbox{7}{~}{((brings gingerbread to the table))}
\end{mdframednoverticalspace}\vspace{-2mm}

\subsubsection{Adverbs embedding the recruitment in a larger activity}

Adverbs such as \textit{jeszcze} `still, also', \textit{też} `also', and \textit{już} `already' can connect the recruitment to a larger activity (\textit{n}=8).  In \REF{ex:zinken:28}, Klaudia and Paweł are preparing a meal.  On his way to the fridge, Paweł stops and turns around, looking at the oven, apparently unsure about what to do next.  A moment later, the \textit{jeszcze} `still, also' in Klaudia’s recruiting move (line 3) marks the recruited action as part of the larger activity of gathering ingredients for the meal they are preparing (see also Extracts \ref{ex:zinken:1}, \ref{ex:zinken:2}, \ref{ex:zinken:15}, and \ref{ex:zinken:32}).

\transheader{ex:zinken:28}{PP4-1\_620160}\vspace{2mm}
%
\begin{transbox}{1}{paw}
\begin{verbatim}
a- °czekaj°
    wait.IPFV.IMP
\end{verbatim}
a- wait
\end{transbox}
%
\emptytransbox{2}{(1.0)((Paweł stops, turns towards oven))}
%
\begin{mdframednoverticalspace}[style=firstfoc]
\begin{transbox}{3}{kla}
\begin{verbatim}
sera       jeszcze
cheese.GEN also
\end{verbatim}
(we need) cheese still
\end{transbox}
\end{mdframednoverticalspace}
%
\begin{mdframednoverticalspace}[style=secondfoc]
\begin{transbox}{4}{paw}
\begin{verbatim}
prosz: ((opens fridge, passes cheese))
plead.1SG
\end{verbatim}
here you are
\end{transbox}
\end{mdframednoverticalspace}
%
\begin{transbox}{5}{kla}
\begin{verbatim}
dzię:ki¿
thanks
\end{verbatim}
thanks
\end{transbox}\bigskip

The temporal adverb \textit{już} `already' can connect the recruitment to a larger course of action by marking the requested action out as a temporal milestone (e.g. the endpoint) within that activity.  In \REF{ex:zinken:29}, Ela has been offering her daughter Gabi various items of food.  Tadek requests at line 4 that she stop distracting the daughter from eating what she has on her plate (a recruiting move that Ela disregards at line 6).

\transheader{ex:zinken:29}{PP1-1\_1230310}\vspace{2mm}
%
\begin{transbox}{1}{ela}
\begin{verbatim}
Gabi może  chcesz   ka- tego     brokułka?
NAME maybe want.2SG po- this.GEN broccoli.GEN
\end{verbatim}
Gabi maybe you want some of this broccoli
\end{transbox}
%
\emptytransbox{2}{(0.8)}
%
\begin{transbox}{3}{gab}
\begin{verbatim}
nie::
no::
\end{verbatim}
\end{transbox}
%
\begin{mdframednoverticalspace}[style=firstfoc]
\begin{transbox}{4}{tad}
\begin{verbatim}
nie mieszaj          już     jej
not confuse.IPFV.IMP already her.DAT
\end{verbatim}
don’t confuse her now / stop confusing her now
\end{transbox}
\end{mdframednoverticalspace}
%
\begin{mdframednoverticalspace}[style=firstfoc]
\begin{transbox}{5}{~}
\begin{verbatim}
[niech ona je       to   co
 may   she eats.3SG this what
\end{verbatim}
\hspace{0.07cm} she should eat (let her eat) what-
\end{transbox}
\end{mdframednoverticalspace}
%
\begin{mdframednoverticalspace}[style=secondfoc]
\begin{transbox}{6}{ela}
\begin{verbatim}
[a   może  dać      ci      marchewkę.
 and maybe give.INF you.DAT carrot.ACC
\end{verbatim}
\hspace{0.07cm} or maybe you want a carrot
\end{transbox}
\end{mdframednoverticalspace}

\section{Formats in Move B: The responding move}\label{sec:zinken:4}

The space of possible next actions by participant B after a recruiting move by participant A can be partitioned into two nested sets.  At one level, B can either produce some response to the recruiting move or not respond to it at all; if B responds to the recruiting move, the response can either work towards complying with the recruitment or embody non-compliance.

Let us consider the first of these distinctions: responding in some way vs. not responding at all. A lack of response to the recruiting move is not uncommon in the Polish data: there are 23 such cases in the corpus (11\%) in which B ``ignores'' the recruiting move as it were (cf. Blythe, \chapref{sec:blythe}, \sectref{sec:blythe:4.2.4}).  A closer look, however, reveals that many of these cases are more benign.\footnote{These cases are more difficult to quantify in that it is not always clear whether a person has genuinely not heard a request or simply does not want to hear it (more on this below).}  A potential recruitee who is already involved in some work -- especially if it is work related to a wider activity within which the recruitment emerges -- might claim some allowance for not attending to the recruitment “just now”. The clearest cases of this come from nonverbal requests.  For example, think back to the activity of two people checking the medicines they have in the fridge (\extref{ex:zinken:5}).  Marta repeatedly takes a package of medicines from the fridge, inspects it, and then holds it out for Karol to take.  In that interaction, there are two instances where Marta holds the package out for Karol to take, but Karol is still inspecting the package he was given previously.  Noticing that Karol is not attending to her gesture, Marta puts the new package on the table, from where Karol eventually takes it. It seems plausible that potential recruitees can also use their being occupied strategically as a way to avoid responding to a recruiting move. \extref{ex:zinken:30} might be an example of this (also discussed in the context of deontic infinitives, see \extref{ex:zinken:20} above).  Olek is beginning to mount a kind of child seat for his toddler grandson, following a request by Kasia to do so (lines 3--4).  At line 6, Kasia incrementally extends her request with another one, namely that Olek should also put a cushion onto the chair (so that the toddler would sit high enough to be securely held in place by the child seat, and to be able to reach the table).  At this time, Olek has already begun mounting the child seat, and there is no response to this subsequent recruiting move.  At lines 8--9, Wiesia reformulates the request made by Kasia at line 6, but again, Olek does not respond to this but carries on trying to unravel parts of the child seat.  At lines 11--12, Wiesia incrementally extends the request and thus provides another occasion for Olek to provide a response, which he does not do (line 13).  At line 14, Wiesia announces that she will bring this cushion herself.  Shortly after this, Olek puts the part of the child seat that he has been wrestling with down on the table and starts walking towards the next room, at the same moment as Wiesia.  Seeing that Olek is now (presumably) on his way to get the cushion, Wiesia stops and walks back to where she was working in the kitchen, and formulates another increment to the request, specifying the kind of cushion (lines 17--18).

\newpage
\largerpage[2]
\transheader{ex:zinken:30}{MiBrApr2012\_0456292}\vspace{1mm}
%
\begin{transbox}{1}{ole}
\begin{verbatim}
jakiejś  wędlinki    może  mu      dać
some.GEN sausage.GEN maybe him.DAT give.INF
\end{verbatim}
maybe (to) give him some sausage
\end{transbox}
%
\begin{transbox}{2}{kas}
\begin{verbatim}
(Józienko)  chcesz   coś       zjeść
 NAME.VOC   want.2SG something eat.INF
\end{verbatim}
Joseph you want to eat something?
\end{transbox}
%
\begin{transbox}{3}{~}
\begin{verbatim}
wiesz    co   możesz  mu     (.)  zmontować
know.2SG what can.2SG him.DAT     mount.INF
\end{verbatim}
you know what, you can mount for him
\end{transbox}
%
\begin{transbox}{4}{~}
\begin{verbatim}
to   siedzenie (znaczy)
this seat       mean.3SG
\end{verbatim}
this seat
\end{transbox}
%
\begin{transbox}{5}{ole}
\begin{verbatim}
((gets up))
\end{verbatim}
\end{transbox}
%
\begin{mdframednoverticalspace}[style=firstfoc]
\begin{transbox}{6}{kas}
\begin{verbatim}
tylko mu      jakąś    poduszkę
only  him.DAT some.ACC cushion.ACC
\end{verbatim}
just (also use) a cushion for him
\end{transbox}
\end{mdframednoverticalspace}
%
\xtransbox{7}{ole}{((begins mounting child seat on chair)) (1.8)}
%
\begin{mdframednoverticalspace}[style=firstfoc]
\begin{transbox}{8}{wie}
\begin{verbatim}
weź      tylko ten  (jakąś)    weź      jakąś
take.IMP only  this (some.ACC) take.IMP some.ACC
\end{verbatim}
take only this (some), take some
\end{transbox}
\end{mdframednoverticalspace}
%
\begin{mdframednoverticalspace}[style=firstfoc]
\begin{transbox}{9}{~}
\begin{verbatim}
poduszkę     połóż   mu
cushion.ACC  put.IMP him.DAT
\end{verbatim}
cushion put (on the chair) for him
\end{transbox}
\end{mdframednoverticalspace}
%
\emptytransbox{10}{(0.2)}
%
\begin{mdframednoverticalspace}[style=firstfoc]
\begin{transbox}{11}{wie}
\begin{verbatim}
tego     najlepiej taką grubą:=u  ciebie
this.ACC best      this thick  at you.GEN
\end{verbatim}
this, ideally a thick one, in your (room)
\end{transbox}
\end{mdframednoverticalspace}
%
\begin{mdframednoverticalspace}[style=firstfoc]
\begin{transbox}{12}{~}
\begin{verbatim}
jest taka gruba poduszka
is   this thick cushion
\end{verbatim}
there is a thick cushion
\end{transbox}
\end{mdframednoverticalspace}
%
\emptytransbox{13}{(3.2)}
%
\begin{transbox}{14}{wie}
\begin{verbatim}
zaraz       przyniosę
in.a.moment bring.1SG
\end{verbatim}
I’ll bring it in a moment
\end{transbox}
%
\emptytransbox{15}{(1.6) ((Olek puts child seat down on table))}
%
\begin{mdframednoverticalspace}[style=secondfoc]
\emptytransbox{16}{((Olek and Wiesia both start walking towards bedroom)) (0.8)}
\end{mdframednoverticalspace}
%
\begin{mdframednoverticalspace}[style=firstfoc]
\begin{transbox}{17}{wie}
\begin{verbatim}
[u  ciebie  ta   gruba taka z    kwiatkami
 at you.GEN this thick such with flowers
\end{verbatim}
\hspace{0.07cm} in your place, the thick one with flowers
\end{transbox}
\end{mdframednoverticalspace}
%
\begin{transbox}{18}{~}
\begin{verbatim}
[((halts, returns to kitchen))
\end{verbatim}
\end{transbox}
%
\begin{mdframednoverticalspace}[style=firstfoc]
\begin{transbox}{19}{~}
\begin{verbatim}
żeby    była gruba taka wysoka
so.that was  thick such high
\end{verbatim}
so that it would be thick, the high one
\end{transbox}
\end{mdframednoverticalspace}
%
\emptytransbox{20}{((Olek comes back with cushion after some time))}\clearpage

In short, there is a series of recruiting moves here, and Olek does not produce an on-record response to any of them.  Instead, he starts a move that is conceivably the complying response (lines 15--16) in a position where it is contiguous to a prior turn that was \textit{not} a recruiting move (line 14).  ``Ignoring'' another person’s requests would seem to be a socially sensitive matter.  However, this might be mitigated here by the fact that Olek is already involved in work on the child seat.  The cushion might only be required once the child seat itself is fixed to the chair.  In other words, Olek’s non-responsiveness may not be treated as ``ignoring'' if it can be accounted for as him being busy with step 1 of the project of preparing a seat for the toddler (securing the child seat to the chair) before moving on to step 2 (providing a cushion).

Another context in which B sometimes does not respond, but may not be fully held accountable for ``ignoring'' the recruiting move, is when this is formatted as an impersonal declarative (see \sectref{sec:zinken:3.3.2} above; see also \citealt{RossiZinken2016}).  Consider \REF{ex:zinken:31}, where the family are at the dinner table and Jacek is feeding his toddler son on his lap (line 1 is part of that interaction).  At lines 2--3, Ilona formulates an impersonal declarative: \textit{Stasiowi by się przydał widelczyk}, roughly: `a fork for Staś would be useful'.  This turn is prefaced with \textit{wiesz co} `you know what', which marks it as being addressed to some individual \citep{Lerner2003}.  However, Ilona does not use any formal resources that would convey who is to get the required fork (she is cutting food on her plate and gazes down throughout her turn). It is the fact that the recruitment attempt is concerned with a childcare matter that makes the turn relevant for her partner.  However, Jacek does not provide any response.  Ilona fills the emerging silence with another short turn thinking out loud (line 5).  When Jacek still does not begin any response to the recruiting move, engaging instead in a short exchange with his son (lines 6--7), Ilona begins a new turn, which explicitly addresses the recruiting move, in different form, to her daughter, Iza (lines 8, 9, 11).

\transheader{ex:zinken:31}{PP2-2\_241620a}\vspace{2mm}
%
\begin{transbox}{1}{jac}
\begin{verbatim}
proszę    bardzo
plead.1SG very
\end{verbatim}
here you are
\end{transbox}
%
\begin{mdframednoverticalspace}[style=firstfoc]
\begin{transbox}{2}{ilo}
\begin{verbatim}
.h:: wiesz    co::¿ Stasiowi by   się  przydał
     know.2SG what  Staś     COND REFL suit.PST.3SG
\end{verbatim}
.h:: you know what, Staś could use a
\end{transbox}
\end{mdframednoverticalspace}
%
\begin{mdframednoverticalspace}[style=firstfoc]
\begin{transbox}{3}{~}
\begin{verbatim}
wid↓elczyk
fork.DIM
\end{verbatim}
fork
\end{transbox}
\end{mdframednoverticalspace}
%
\emptytransbox{4}{(0.8)}
%
\begin{transbox}{5}{ilo}
\begin{verbatim}
˚zaraz˚
right.now
\end{verbatim}
just a moment
\end{transbox}
%
\begin{transbox}{6}{sta}
\begin{verbatim}
odział
\end{verbatim}
\end{transbox}
%
\begin{transbox}{7}{jac}
\begin{verbatim}
orzeł?
eagle
\end{verbatim}
eagle?
\end{transbox}
%
\begin{mdframednoverticalspace}[style=firstfoc]
\begin{transbox}{8}{ilo}
\begin{verbatim}
Iza¿  ↑weź     przynieś      ten  malutki
Iza   take.IMP bring.PFV.IMP this small
\end{verbatim}
Iza bring this little
\end{transbox}
\end{mdframednoverticalspace}
%
\begin{mdframednoverticalspace}[style=firstfoc]
\begin{transbox}{9}{~}
\begin{verbatim}
Stasia   widelczyk wie:sz   który ten  biały [taki z=
Staś.GEN fork      know.2SG which this white  such with
\end{verbatim}
fork of Staś’s you know which one, the white one made
\end{transbox}
\end{mdframednoverticalspace}
%
\begin{transbox}{10}{iza}
\begin{verbatim}
                                             [>a<
\end{verbatim}
\end{transbox}
%
\begin{mdframednoverticalspace}[style=firstfoc]
\begin{transbox}{11}{ilo}
\begin{verbatim}
=melaminy.
 melamine
\end{verbatim}
\hspace{0.07cm} from melamine
\end{transbox}
\end{mdframednoverticalspace}
%
\begin{mdframednoverticalspace}[style=secondfoc]
\xtransbox{12}{iza}{((gets up and leaves))}
\end{mdframednoverticalspace}

We now turn to recruiting moves that receive a response.

\subsection{Fully nonverbal responses}
\largerpage
One way -- arguably the basic way -- of responding to a recruiting move is to do the relevant action.  There are many such cases among the examples discussed so far (Extracts
\ref{ex:zinken:2}--\ref{ex:zinken:3},
\ref{ex:zinken:5}--\ref{ex:zinken:6},
\ref{ex:zinken:8}--\ref{ex:zinken:14},
\ref{ex:zinken:21}--\ref{ex:zinken:22},
\ref{ex:zinken:24}, \ref{ex:zinken:26}).
Fully nonverbal compliance is common when a recruited action can be performed quickly and easily \citep{RauniomaaKeisanen2012}: passing a knife across the table, picking up something that has dropped to the floor etc.  Out of 69 cases of such quick compliance in the data, 50 (72\%) come without any verbal element.  What is maybe more surprising is that fully nonverbal compliance is also common in cases where doing the relevant action takes more time, where it is necessary to create certain conditions for the requested action first: going to the kitchen in order to fetch a spoon, for example.  There are 72 cases in the data where B’s next move after a recruiting one is the first step of a compliant response, but where that compliance takes a bit longer (or might become stalled after that first move).  Of these recruitments, 45 cases (63\%) do not involve any verbal response. \extref{ex:zinken:32} illustrates such a case.  Jacek and Ilona are talking to their son about possible places where he could search for his lost ball (line 1 is a contribution to this conversation).  At lines 2 and 5, Ilona recruits Jacek to also look ‘here’, that is, in a corner of the room.  In response, Jacek takes steps in that direction and begins moving back some furniture to look for the ball.  His response is not accompanied by any verbal turn.

\transheader{ex:zinken:32}{PP2-1\_3936480}\vspace{2mm}
%
\begin{transbox}{1}{jac}
\begin{verbatim}
może    być    też
can.3SG be.INF also
\end{verbatim}
it can also be (there)
\end{transbox}
%
\begin{mdframednoverticalspace}[style=firstfoc]
\begin{transbox}{2}{ilo}
\begin{verbatim}
i   może    jeszcze Jace::k¿
and can.3SG still   NAME
\end{verbatim}
and maybe also, Jacek?
\end{transbox}
\end{mdframednoverticalspace}
%
\emptytransbox{3}{(.)}
%
\begin{transbox}{4}{jac}
\begin{verbatim}
°hm°
\end{verbatim}
\end{transbox}
%
\begin{mdframednoverticalspace}[style=firstfoc]
\begin{transbox}{5}{ilo}
\begin{verbatim}
[y::  rzuć          okiem   [o   tutaj w:: (°°   °°)
 INTJ throw.PFV.IMP eye.INS  PTC here  in
\end{verbatim}
\hspace{0.07cm} eh, have a look \textit{o} here, in (\hspace{1cm})
\end{transbox}
\end{mdframednoverticalspace}
%
\begin{transbox}{6}{~}
\begin{verbatim}
[((head nod))               [((head nod))
\end{verbatim}
\end{transbox}
%
\begin{mdframednoverticalspace}[style=secondfoc]
\xtransbox{7}{jac}{((turns and searches for ball))}
\end{mdframednoverticalspace}\vspace{-2mm}

\subsection{Verbal elements of responses}

Verbal elements accompanying complying responses to recruitments can be ordered according to their grammatical complexity (cf. \citealt{ThompsonFoxCouperKuhlen2015}). The simplest verbal responses are polar responses that indicate (upcoming) compliance or reject the recruitment.  As mentioned above, such responses might be more relevant for recruited actions that are not quick and easy.  However, the relevance of a particular type of verbal response might also depend upon the form of the recruiting move.  Recruiting moves in polar question format grammatically project a polar response that accepts the recruitment (\citealt{Raymond2003}).\footnote{But cf. \citet{ThompsonFoxCouperKuhlen2015} for an argument against this view.} Quick and easy compliance can diminish the usefulness of accepting -- after all, acceptance should occur before the actual compliance. Out of the four conventional request moves in polar question format in the Polish data, one receives a polar response (\textit{no tak} `\textsc{PTC} yes', \extref{ex:zinken:19}), while the other three are “problematic” recruitments (see \sectref{sec:zinken:3.3.4} above).  Out of 93 imperative recruiting moves, only one receives a polar response: a flat-out rejection with \textit{nie} `no' (\extref{ex:zinken:34}, see below).  The action of accepting conveyed by a positive polar response does not seem to be relevant in response to imperative recruiting moves in Polish (see also \citealt{CravenPotter2010,Rossi2012} for English and Italian). This does not mean that there are no verbal responses to imperatives. However, these verbal responses emphasize compliance rather than accepting the recruitment, e.g. \textit{proszę bardzo} `here you are', \textit{masz} ‘here you are’, literally `you have', \textit{już przyniosę} `already I bring it' (see \citealt[chap. 5]{Zinken2016}, for a discussion). 

Another response token is \textit{dobra} or \textit{dobrze} `okay', which appears to indicate compliance “in principle”, in a situation where maybe immediate compliance is not possible, or the recruitee does not know how to go about the recruited action (see Extract \ref{ex:zinken:27}).  A practice found repeatedly in the corpus is to begin a verbal response with a temporal adverb, for example, \textit{już} `already'.  In response to an on-record request, \textit{już} can be produced as a response not just to indicate compliance but also to treat the request as urgent (see Extract \ref{ex:zinken:25}).  Clausal responses with a turn-initial \textit{już} enact stronger agency and initiative on the part of the recruitee (see Extract \ref{ex:zinken:4}).

In general, clausal responses do more than simply indicating compliance.  In \REF{ex:zinken:33}, Ania notices that the sauce she requested earlier has not been poured on her food, and she asks for it again (line 2), designing her turn as “having to ask again”.  In other words, her turn is formatted not just (and maybe not primarily) as a request, but as a complaint.  The clausal response in line 4 is fitted to this ``double-barreled'' first action (\citealt[76]{Schegloff2007}; see also \citealt{KitzingerEtAl2013,Rossi2018}). The response indicates not only or not so much compliance with the request, but manages the disaffiliational undercurrent through a relatively elaborate verbal offering of the sauce.

\transheader{ex:zinken:33}{PP1-1\_0509630b}\vspace{2mm}
%
\emptytransbox{1}{(1.6)}
%
\begin{mdframednoverticalspace}[style=firstfoc]
\begin{transbox}{2}{ani}
\begin{verbatim}
ale ja jeszcze poprosiłam      (.)  y    °sosiku°      =moment
but I  still   ask.PFV.PST.1SG      INTJ sauce.DIM.GEN moment
\end{verbatim}
but I still asked for some sauce \hspace{1.2cm} wait a moment
\end{transbox}
\end{mdframednoverticalspace}
%
\emptytransbox{3}{(1.0) ((Ania picks up her plate, walks toward cooker))}
%
\begin{mdframednoverticalspace}[style=secondfoc]
\begin{transbox}{4}{ela}
\begin{verbatim}
↑no  to   ma:sz.
PTC  then have.2SG
\end{verbatim}
\textit{no} then here you have (some)
\end{transbox}
\end{mdframednoverticalspace}
%
\begin{mdframednoverticalspace}[style=secondfoc]
\xtransbox{5}{~}{((serves Ania sauce))}
\end{mdframednoverticalspace}\vspace{-2mm}

\subsection{Types of rejections}

Recruitments are rejected in 23 cases in the corpus (nearly 11\%).  Overt rejection with just the response particle \textit{nie} `no', however, occurs in only one case (\extref{ex:zinken:34}).  Klaudia and Paweł are having supper, and when Paweł moves up his fork with melted cheese sticking to it, Klaudia pleads with him to give her the cheese.  Paweł responds curtly with a `no', and then turns to the family dog squealing at his legs.
\newpage

\transheader{ex:zinken:34}{PP4-1\_2301200}\vspace{2mm}
%
\xtransbox{1}{paw}{((moves up fork with melted cheese sticking to it))}
%
\begin{mdframednoverticalspace}[style=firstfoc]
\begin{transbox}{2}{kla}
\begin{verbatim}
da::j        mi     tego     żółtego    sera
give.PFV.IMP me.DAT this.GEN yellow.GEN cheese.GEN
\end{verbatim}
(do) give me some of that yellow cheese
\end{transbox}
\end{mdframednoverticalspace}
%
\begin{mdframednoverticalspace}[style=secondfoc]
\begin{transbox}{3}{paw}
\begin{verbatim}
nie¿
no
\end{verbatim}
no
\end{transbox}
\end{mdframednoverticalspace}
%
\xtransbox{4}{~}{((turns to dog))}

It is questionable whether Klaudia’s request in \REF{ex:zinken:34} was serious. It is more likely that she did not really expect Paweł to scrape the cheese from his own food and pass it to her.  Her plea for the cheese might more plausibly be part of some kind of tease between the two, and this also puts Paweł’s seemingly blunt rejection in a different light.  In any case, rejections are overwhelmingly done in ways that avoid being blunt in one way or another.

One way of rejecting a recruitment is to question the need for the requested action (see \citealt{ZinkenOgiermann2011}).  A format for rejection in Polish that at least ostensibly does this is \textit{po co} `what for'.  However, this format does not really seem to question the need for the requested action -- a reason is never provided in response, and is never pursued.  Instead, questioning the need in this format works as a practice for rejecting a recruitment (cf. \citealt{BoldenRobinson2011} on account solicitations with \textit{why}).  In  \REF{ex:zinken:35}, Wiesia is walking around the flat with her toddler granddaughter.  At line 2, Dorota, who is sitting at the kitchen table, recruits Wiesia to turn on the light in the corridor where she and the toddler are (in fact, Wiesia had just switched the light off, but Dorota might not have noticed).  Wiesia does not respond to this recruiting move, continuing instead a turn addressed to her granddaughter (lines 1 and 3).  Dorota repeats her recruiting move in line 5 and, after some silence, Wiesia rejects the recruitment with \textit{a po co (światło)} ‘but why (light)’ (line 8).

\transheader{ex:zinken:35}{Pa02Apr2012\_0725770b}\vspace{2mm}
%
\begin{transbox}{1}{wie}
\begin{verbatim}
chodź         (   )
come.IPFV.IMP
\end{verbatim}
come (here)
\end{transbox}
%
\begin{mdframednoverticalspace}[style=firstfoc]
\begin{transbox}{2}{dor}
\begin{verbatim}
zapal           tam   światło mamunia
turn.on.PFV.IMP there light   mom.DIM.VOC
\end{verbatim}
turn the light on there, mommy
\end{transbox}
\end{mdframednoverticalspace}
%
\begin{transbox}{3}{wie}
\begin{verbatim}
może  coś  zjesz
maybe what eat.PFV.2SG
\end{verbatim}
maybe you’ll eat something
\end{transbox}
%
\emptytransbox{4}{(.)}
%
\begin{mdframednoverticalspace}[style=firstfoc]
\begin{transbox}{5}{dor}
\begin{verbatim}
zapal           tam   ↑światło.=°mamusiu°
turn.on.PFV.IMP there light      mom.DIM.VOC
\end{verbatim}
(do) turn the light on there mommy
\end{transbox}
\end{mdframednoverticalspace}
%
\emptytransbox{6}{(0.4) ((Wiesia walking towards kitchen with toddler))}
%
\begin{mdframednoverticalspace}[style=firstfoc]
\begin{transbox}{7}{dor}
\begin{verbatim}
Pol[uniu:¿
NAME.DIM.VOC
\end{verbatim}
Polly
\end{transbox}
\end{mdframednoverticalspace}
%
\begin{mdframednoverticalspace}[style=secondfoc]
\begin{transbox}{8}{wie}
\begin{verbatim}
   [a       po  co  [(światło)
    and.but for what (light)
\end{verbatim}
\hspace{0.5cm} but why (light)
\end{transbox}
\end{mdframednoverticalspace}
%
\begin{transbox}{9}{dor}
\begin{verbatim}
                    [Poluniu      zjesz        jeszcze salami?
                     Pola.DIM.VOC eat.PFV.2SG  still   salami
\end{verbatim}
\hspace{3cm} Polly will you eat some (more) salami?
\end{transbox}\bigskip

A common element in turns rejecting a recruitment is an informing TCU that can be taken as providing an explanation for not complying.  Sometimes such turns begin with a rejection token (\textit{nie} `no', \textit{n}=2) but more commonly they do not.

\extref{ex:zinken:36} is a case where a recruitment is rejected with a \textit{nie} `no' plus explanation.  This case comes from the same setting as the previous \extref{ex:zinken:35}.  Dorota is asking Wiesia and the toddler, who are walking around the flat, to come to the table to eat something  as the rest of the family are having breakfast.  Wiesia initiates repair at line 3, and Dorota redoes the recruiting move, addressing it now only to the toddler (line 4).  However, Wiesia apparently does not notice this and responds with a rejection token (\textit{nie} `no') and an appended explanation (line 5).

\transheader{ex:zinken:36}{Pa02Apr2012\_0714730a}\vspace{-1mm}
%
\begin{mdframednoverticalspace}[style=firstfoc]
\begin{transbox}{1}{dor}
\begin{verbatim}
cho::dźcie        zjeść   z    nami
come.IPFV.IMP.2PL eat.INF with us.INS
\end{verbatim}
 come (you two) eat with us
\end{transbox}
\end{mdframednoverticalspace}
%
\emptytransbox{2}{(0.2)}
%
\begin{mdframednoverticalspace}[style=secondfoc]
\begin{transbox}{3}{wie}
\begin{verbatim}
proszę¿
plead.1SG
\end{verbatim}
excuse me?
\end{transbox}
\end{mdframednoverticalspace}
%
\begin{mdframednoverticalspace}[style=firstfoc]
\begin{transbox}{4}{dor}
\begin{verbatim}
chodź         córuś        może  zjesz   coś.
come.IPFV.IMP daughter.DIM maybe eat.2SG something
\end{verbatim}
come (my) daughter maybe you’ll eat something
\end{transbox}
\end{mdframednoverticalspace}
%
\begin{mdframednoverticalspace}[style=secondfoc]
\begin{transbox}{5}{wie}
\begin{verbatim}
nie:: ja jestem po    śniadaniu.
no    I  am.1SG after breakfast.LOC
\end{verbatim}
no, I have had breakfast
\end{transbox}
\end{mdframednoverticalspace}
%
\begin{transbox}{6}{dor}
\begin{verbatim}
ale nie do ciebie mówię(h) ((laughter))
but not to you    talk.1SG
\end{verbatim}
but I am not talking to you
\end{transbox}\bigskip

More commonly, a rejection is accomplished with just an explanation for not doing as requested.  In \REF{ex:zinken:37}, Ania has sat down with her back straight to the camera, and the participants have just commented on this.  At line 1, Ela directs Ania to sit ‘here’, on the chair next to the one she is sitting on.  Ania does not respond to the initial recruiting move, with her gaze directed at the free chair Ela is indicating.  When Ela redoes the recruiting move in amended format, Ania rejects this with a turn composed of two units, each of which formulates a reason for not taking the “better” chair: \textit{ale ja nie zostanę} `but I am not staying' (line 4) and \textit{ja już jestem po śniadaniu} ‘I have already had breakfast’ (lines 4--5).

\transheader{ex:zinken:37}{PP1-1\_0615520b}\vspace{-1mm}
%
\begin{mdframednoverticalspace}[style=firstfoc]
\begin{transbox}{1}{ela}
\begin{verbatim}
usiądź           tu ((points to vacant chair))
sit.down.PFV.IMP here
\end{verbatim}
sit down here
\end{transbox}
\end{mdframednoverticalspace}
%
\emptytransbox{2}{(0.8)}
%
\begin{mdframednoverticalspace}[style=firstfoc]
\begin{transbox}{3}{ela}
\begin{verbatim}
siada:j
sit.IPFV.IMP
\end{verbatim}
(do) sit (down)
\end{transbox}
\end{mdframednoverticalspace}
%
\begin{mdframednoverticalspace}[style=secondfoc]
\begin{transbox}{4}{ani}
\begin{verbatim}
ale ja nie zostanę      ja już     jestem
but I  not stay.PFV.1SG I  already be.1SG
\end{verbatim}
but I am not staying, I have already
\end{transbox}
\end{mdframednoverticalspace}
%
\begin{mdframednoverticalspace}[style=secondfoc]
\begin{transbox}{5}{~}
\begin{verbatim}
po    śniad(h)a(niu)(h)
after breakfast
\end{verbatim}
had breakfast
\end{transbox}
\end{mdframednoverticalspace}

\section{Acknowledgment in third position}\label{sec:zinken:5}

As in the other languages examined in the comparative project \citep{FloydEtAl2018}, acknowledgment of compliance is rare in the Polish data: only 3 cases were found.  We have seen two of these in \REF{ex:zinken:22} and \REF{ex:zinken:28}, where the recruiter thanks after receiving a requested object.  The third case, shown below, is also an object request.  Kasia asks Georg to pass the horseradish across the table.  Georg does this, accompanying the action with a verbal turn, \textit{prosz}:: `here you are'.  Kasia takes the horseradish and quietly says \textit{dziękuję} `thanks'.

\transheader{ex:zinken:38}{MiBrApr2012\_0552334}\vspace{-1mm}
%
\begin{mdframednoverticalspace}[style=firstfoc]
\begin{transbox}{1}{kas}
\begin{verbatim}
poproszę      chrzanik¿
plead.PFV.1SG horseradish.DIM
\end{verbatim}
I ask for the horseradish / can I have the horseradish
\end{transbox}
\end{mdframednoverticalspace}
%
\begin{mdframednoverticalspace}[style=secondfoc]
\begin{transbox}{2}{geo}
\begin{verbatim}
prosz:: ((takes horseradish, places in front of Kasia))
plead.IPFV.1SG
\end{verbatim}
here you are
\end{transbox}
\end{mdframednoverticalspace}
%
\begin{transbox}{3}{kas}
\begin{verbatim}
°dziękuję°
thank.1SG
\end{verbatim}
thanks
\end{transbox}\bigskip

Thanking is a way of recognizing another’s agency in providing assistance \citep{ZinkenRossiReddy}.  It is also worth noting that in two out of three cases, the recruitee points to his compliance with \textit{proszę} `please, here you are', which might make the provision of an acknowledgment more likely.

\section{Social asymmetries}\label{sec:zinken:6}
\largerpage
The videos in the Polish corpus were recorded by families in their homes. Social asymmetries enter the picture in so far as interactions are sometimes between parents and their adult children.  The interactions mostly take place in the parents’ homes, and both the setting and the social relationship might contribute to some deference on the part of the adult children.  No strong influence was noticed in terms of the ratio of fulfillments to rejections. However, one striking aspect in the formulation of recruiting moves is the common use of vocatives by adult children when attempting to recruit their parents. Out of 24 recruitment sequences in which the recruiter was analyzed as occupying a higher social position than the recruitee, only 3 (12\%) contained a recruiting move with a vocative.  But out of 23 recruitment sequences in which the recruiter was analyzed as occupying a lower social position than the recruitee, 10 (43\%) contained a recruiting move with a vocative (e.g. Extract \ref{ex:zinken:35} above).

\extref{ex:zinken:39} is one of those rare cases where a father uses a vocative in addressing a recruiting move to his adult daughter, Dorota.  Olek has his toddler grandson on his lap, and the toddler wants to get off to walk around. This has been problematic before, because the toddler has a sausage in his hand, and Dorota -- whose home is this is -- does not want the little ones to run around with food in their hands.  Olek addresses a turn to Dorota, in which he raises this problem and thereby recruits her to do something about it (a recruited action that is about to be made more specific in line 3).  Dorota responds in two ways: she rejects the plan to ‘go’, attributed to the toddler (lines 2 and 5), while walking towards him, and taking the sausage from him and putting it on a plate (line 7), thus creating the circumstances in which the toddler can have his wish to walk around granted.

\transheader{ex:zinken:39}{Pa02Apr2012\_1227960}\vspace{-1mm}
%
\begin{mdframednoverticalspace}[style=firstfoc]
\begin{transbox}{1}{ole}
\begin{verbatim}
nie mo-  ciocia on chce     iść    patrz         o
not pos- aunt   he want.3SG go.INF look.IPFV.IMP PTC
\end{verbatim}
you can’t- aunt, he wants to go look \textit{o}
\end{transbox}
\end{mdframednoverticalspace}
%
\begin{mdframednoverticalspace}[style=secondfoc]
\begin{transbox}{2}{dor}
\begin{verbatim}
n[ie ((gaze at toddler, eyebrows raised))
no
\end{verbatim}
\end{transbox}
\end{mdframednoverticalspace}
%
\begin{mdframednoverticalspace}[style=firstfoc]
\begin{transbox}{3}{ole}
\begin{verbatim}
 [trzymaj       go      na (   )
  hold.IPFV.IMP him.ACC on
\end{verbatim}
\hspace{0.2cm} hold him (\hspace{0.6cm})
\end{transbox}
\end{mdframednoverticalspace}
%
\begin{transbox}{4}{kas}
\begin{verbatim}
ale to
but this
\end{verbatim}
\end{transbox}
%
\begin{mdframednoverticalspace}[style=secondfoc]
\begin{transbox}{5}{dor}
\begin{verbatim}
nie ((gaze at toddler, eyebrows raised))
no
\end{verbatim}
\end{transbox}
\end{mdframednoverticalspace}
%
\begin{transbox}{6}{kas}
\begin{verbatim}
poprostu go
simply   him.ACC
\end{verbatim}
\end{transbox}
%
\begin{mdframednoverticalspace}[style=secondfoc]
\begin{transbox}{7}{dor}
\begin{verbatim}
tutaj to   hopsa i   można    iść ((takes sausage from toddler))
here  this hop   and possible go.INF
\end{verbatim}
here (we put) this, hop, and you can go
\end{transbox}
\end{mdframednoverticalspace}\bigskip

Of particular interest is Olek’s use of the category term \textit{ciocia} `aunt' (line 1).  Dorota is in fact Olek’s daughter and the toddler’s aunt.  Olek addresses Dorota in her family relationship role to the toddler, who is the target of the recruitment (see also Extracts \ref{ex:zinken:10}, \ref{ex:zinken:16}, and \ref{ex:zinken:25} above).  For one thing, a vocative addresses the recruiting move to a particular person in a multiparty setting; at the same time, it provides a slot in which the choice of vocative item can be used to mobilize or acknowledge particular social relationships (see also Kendrick, \chapref{sec:kendrick}, \sectref{sec:kendrick:4.3.1}; Baranova, \chapref{sec:baranova}, \sectref{sec:baranova:3.4}).

\section{Discussion}

This chapter provided an overview of practices that speakers of Polish use for the organization of collaboration and assistance in informal family settings.  In many respects, the Polish data are consistent with findings from other languages in the cross-linguistic project, and with expectations based on the extant literature.

For example, the findings show that there seem to be hardly any verbal turn formats that could \textit{not} become part of a recruiting move: imperative, declarative, and interrogative turn shapes are all attested, as are turns without a predicate and interactional moves without any talk.  This supports the contention that drawing on others’ cooperation is a fundamental facet of human sociality that does not make any specific demands on grammatical structures (\citealt{Tomasello2008}).  Imperatives are the most common sentence type in recruiting moves, as we would expect given that imperatives are dedicated to the delivery of directive actions (e.g. \citealt{aikhenvald2010}).  Also, the findings support arguments for a bias towards prosocial orientations at work in human interaction (e.g. \citealt{heritage1984garfinkel}).  Rejections are much less frequent than compliant responses, and are mostly done by providing explanations for non-compliance, rather than by bluntly rejecting the recruitment.  Even cases in which a person does not respond to a recruiting move at all show traces of such a prosocial orientation: recruitees skillfully orient to aspects of the situation that could make their lack of response accountable in terms other than ``ignoring''.

Other findings might be cross-linguistically more restricted.  For example, imperative recruiting moves with imperfective verbal aspect in Polish display that the recruiting move does not convey new information. Speakers can use this resource to indicate that the other person should have acted already -- that they already knew what to do (cf. \citealt{KentKendrick2016}). Conventionally indirect \citep{BrownLevinson1987} recruiting practices, such as questions about the ability or willingness to do something, are very rare in the examined corpus.  Instead, declarative turns and turns without a predicate make up nearly half of recruiting moves with a verbal element that are not imperatives. These turn formats have received little attention in the literature relative to their prominence in (Polish) informal everyday interaction.

\section*{Acknowledgments}

I am grateful to two anonymous reviewers for their detailed and helpful comments on an earlier version of this chapter.

% \section*{Abbreviations}

\sloppy
\printbibliography[heading=subbibliography,notkeyword=this]
\end{document}
