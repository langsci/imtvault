\documentclass[output=paper]{langsci/langscibook}
\ChapterDOI{10.5281/zenodo.4018374}

\author{Simeon Floyd\affiliation{Department of Anthropology, Universidad San Francisco de Quito}}

\title{Getting others to do things in the Cha’palaa language of Ecuador}

\abstract{This chapter describes the resources that speakers of Cha'palaa use when recruiting assistance and collaboration from others in everyday social interaction. The chapter draws on data from video recordings of informal conversation in Cha'palaa, and reports language-specific findings generated within a large-scale comparative project involving eight languages from five continents (see other chapters of this volume). The resources for recruitment described in this chapter include linguistic structures from across the levels of grammatical organization, as well as gestural and other visible and contextual resources of relevance to the interpretation of action in interaction. The presentation of categories of recruitment, and elements of recruitment sequences, follows the coding scheme used in the comparative project (see \chapref{sec:coding} of the volume). The present chapter extends our knowledge of the structure and usage of the Cha'palaa language with detailed attention to the properties of sequential structure in conversational interaction. The chapter is a contribution to an emerging field of pragmatic typology. }
\begin{document}
\maketitle
\label{sec:floyd}

\section{Introduction}\label{sec:floyd:1}

This paper offers a first description of one area of everyday interaction among speakers of the indigenous Cha'palaa language of Ecuador, sequences in which one party “recruits” the behavior of another for some practical action such as transferring an object or physically assisting with or collaborating in an activity. The analysis of these instances is based on a video corpus of informal conversation recorded in the Chachi communities where Cha'palaa is spoken. This area of Cha'palaa interaction is characterized by a tendency toward direct recruiting moves, employing an extensive set of imperative formats, within the contexts of the different rights and responsibilities of individuals in Chachi society.

\subsection{The Cha’palaa Language}\label{sec:floyd:1.1}

The Cha’palaa language is spoken by the Chachi people in small communities and homesteads along the rivers of the Ecuadorian Province of Esmeraldas between the Andean foothills and the Pacific coast. It is one of the modern members of the Barbacoan family, which was once the dominant language family of the region corresponding to northern Ecuador and southern Colombia until it was displaced by Quechuan languages and, later, Spanish, in much of the Andean highlands. The Chachi people avoided the pressure of language shift by migrating to the coastal lowlands where they live today \citep{JijónyCaamaño1914,DeBoer1996,Floyd2010}. Estimates of the number of speakers vary between about 6,000 (\citealt{INEC2010}) and 10,000 in Ethnologue \citep{LewisSimonsFennig2014}.

Until recently, Cha’palaa was a relatively unstudied language, with only a few descriptive sources by missionary linguists produced over the last few decades \citep{Moore1962,LindskoogLindskoog1964,Vittadello1988}; recent work by the author has begun to bring more aspects of the language and the interactive practices of its speakers to light \citep{Floyd2009,Floyd2010,Floyd2014a,Floyd2014b,Floyd2015,Floyd2016, Floyd2018,FloydBruil2011,dingemanse_conversation_2014,FloydNorcliffe2016,DingemanseRossiFloyd2017}. Like many South American languages, Cha’palaa has a basic SOV word order with extensive agglutinating verbal morphology. Some of its grammatical features that are relevant for recruiting practices include its large imperative paradigm, its egophoric system (a distinctive type of epistemic marking), its complex predicate system, its morphological case markers, and other elements that will be described in the sections that follow.

\subsection{Data collection and corpus}\label{sec:floyd:1.2}

The video corpus on which this work is based was constructed in accordance with a set of guidelines developed by and for the members of the comparative project being reported on in this volume (see Chapters 1--2). The corpora studied in this project feature informal conversation among friends and family, and participants received no special instructions other than to talk or go about their daily activities as they wished.

The Cha’palaa corpus was recorded by the author over a period between 2007 and 2015 in household and village settings in Chachi communities of north-west\-ern Ecuador. The majority of the recordings come from the Rio Cayapas area, particularly from its tributary the Rio Zapallo, and a few come from other areas. In most cases the camera was placed in a household or common area during regular daily activities and then retrieved after about an hour. All videos included adult participants (adolescents or older), including dyads and larger groups of family members and friends, sometimes changing configuration during filming, with children often coming and going. Participants were involved in cooking, eating, doing other household tasks, making handcrafts such as woven baskets, or simply relaxing and conversing.

The data considered for analysis consists of a sample of selections from the recordings (see \chapref{sec:coding}, \sectref{sec:coding:4}). The goal was to identify at least 200 recruitment sequences among adults. The nature of the Cha’palaa corpus, made up primarily of recordings in multi-generational households, meant that there were many cases in which one or both of the involved participants was a small child (not yet adolescent, below about 12 years). These were excluded from the present comparison  in order not to introduce complicating issues of language development. Excluding these cases, which were twice as frequent as adult-only cases, necessitated reviewing a large sample of about 9.5 hours from 16 different recordings to reach a total of 205 cases (out of the initial 653, with 448 child-involved cases that were excluded). Excerpts from this sample of cases are presented below to illustrate the range of linguistic forms and practices that make up the Cha’palaa recruitment system.

\section{Basics of recruitment sequences}\label{sec:floyd:2}

As defined in \chapref{sec:intro}, \sectref{sec:intro:4}, a recruitment is a basic cooperative phenomenon in social interaction consisting of a sequence of two moves with the following characteristics:

\begin{description}
\item[Move A:] participant A says or does something to participant B, or that B can see or hear;
\item[Move B:] participant B does a practical action for or with participant A that is fitted to what A has said or done.
\end{description}

In general, “practical” actions in Move B (M-B) were considered to be goal-driven bodily movements or manipulations of the physical environment, and contrast with states of rest and inactivity. Such practical actions can often be thought of as “target” actions of participant A, when they are made explicit in Move A (M-A). Also, participant A may often be a beneficiary of the recruited action, but in other cases both A and B together, and sometimes even mostly B, may benefit from the outcomes of the recruitment sequence. Further details relevant for this definition, including what happens when Move B rejects the recruitment, or when there is no response, will be discussed in \sectref{sec:floyd:4}. The next subsections give some basic examples of recruitment sequences. In the transcripts, ▶ and ▷ designate Move A and Move B, respectively.

\subsection{Minimal recruitment sequence}\label{sec:floyd:2.1}

\extref{ex:floyd:1} provides an example of a minimal recruitment sequence in Cha’palaa. In Move A, participant A turns to look at participant B, his wife, and uses an interrogative format to invoke a specific target action: ‘did you sweep? did you sweep?’. As part of his recruiting move, he also adds a reason for doing this action: ‘the child is in all that trash’. Participant B’s reaction is to walk off camera and to return shortly with a broom, sweeping the spot indicated by participant A.

\transheader{ex:floyd:1}{CHSF2011\_01\_11S2\_1531121}\vspace{-1mm}
%
\begin{mdframednoverticalspace}[style=firstfoc]
\begin{transbox}{1}{a}
\begin{verbatim}
mankashyu mankashyu
ma   -n   -kash -yu   ma  -n   -kash -yu
again-IPFV-sweep-EGO again-IPFV-sweep-EGO
\end{verbatim}
did you sweep? did you sweep?
\end{transbox}
\end{mdframednoverticalspace}
%
\begin{transbox}{2}{~}
\begin{verbatim}
na tsamantsa ujtu'paatala
na    tsamantsa ujtu'-pala -tala
small very.much trash-place-among
\end{verbatim}
the child is in all that trash
\end{transbox}
%
\begin{mdframednoverticalspace}[style=secondfoc]
\xtransbox{3}{b}{((leaves))}
\end{mdframednoverticalspace}
%
\emptytransbox{4}{(15.0)}\vspace{-1mm}
%
\begin{mdframednoverticalspace}[style=secondfoc]
\xtransbox{5}{b}{((returns with broom, sweeps))}
\end{mdframednoverticalspace}

\normalsize
The format selected by A in line 1 illustrates how a distinctive feature of Cha’pa\-laa’s grammar, an “egophoric marker” (\citealt{FloydNorcliffeSanRoque2018}; also referred to as “conjunct-disjunct” markers, see \citealt{Hale1980,DeLancey1992,Bickel2000,Creissels2008,Dickinson2000,Post2013}) is employed for the interactive function of instigating a behavior on the part of B. This type of knowledge-based morphology used in this context treats the addressee as the locus of knowledge (in statements the marker might associate with the speaker’s perspective in a similar way). While this is a distinctive morphological resource of Cha’palaa, its usage for recruitment also fits a more general pattern of question-like formats. Looking at Move B, the uptake by B provides evidence that functionally this question was taken as a request for the provision of a service, namely, sweeping up.

\subsection{Extended recruitment sequence}\label{sec:floyd:2.2}

In the simplest sequences, B takes steps to accomplish the target action in Move B immediately after A produces Move A, but this does not always occur. Recruitment sequences sometimes feature more than one recruiting move; these cases may be “pursuits”, in which A repeats a version of the recruiting move \citep{Pomerantz1984response,BoldenMandelbaumWilkinson2012} or other types of sequences in which the response to Move A comes later, such as in repair sequences like that seen in \REF{ex:floyd:2}. Participant A, a woman who is washing her clothes on the shore, asks her friend participant B for a plastic tub, but before she passes it to A, B requests a clarifying confirmation of the target object, in line 2. After A provides this confirmation, B accomplishes the target action.

\transheader{ex:floyd:2}{CHSF\_2012\_08\_04S4\_1712020}\vspace{-1mm}
%
\begin{mdframednoverticalspace}[style=firstfoc]
\begin{transbox}{1}{a}
\begin{verbatim}
Daira ñaa inu tina ka' eede
Daira ñu -ya   i  -nu  tina ka  -tu ere -de
Daira 2SG-FOC  1SG-ACC tub  grab-SR pass-IMP
\end{verbatim}
Daira you pass me the tub
\end{transbox}
\end{mdframednoverticalspace}
%
\begin{mdframednoverticalspace}[style=secondfoc]
\begin{transbox}{2}{b}
\begin{verbatim}
enstaa? ((pointing at tub))
ensta-a
this-Q
\end{verbatim}
this one?
\end{transbox}
\end{mdframednoverticalspace}
%
\begin{mdframednoverticalspace}[style=firstfoc]
\begin{transbox}{3}{a}
\begin{verbatim}
jee tsadekee
jee tsa-de-ke-e
yes SEM-PL-do-IMP
\end{verbatim}
yeah do that
\end{transbox}
\end{mdframednoverticalspace}
%
\begin{mdframednoverticalspace}[style=secondfoc]
\xtransbox{4}{b}{((throws tub to A))}
\end{mdframednoverticalspace}

In these types of non-minimal sequences it is possible to observe “side sequences” \citep{jefferson_side_1972}, “insert sequences” \citep[chap. 6]{schegloff_sequence_2007}, and other types of intervening interaction that may occur between the original Move A and the fulfillment of the recruitment. When the request is not fulfilled in the first Move B, this can generate further iterations of the M-A/M-B structure until the sequence is completed (or abandoned).

\subsection{Subtypes of recruitment sequence}\label{sec:floyd:2.3}

Despite considerable overlap, the concept of recruitment is intended to capture a broader range of phenomena than terms like “request” or “directive” (see \chapref{sec:intro}, \sectref{sec:intro:4}). Four broad subtypes of sequence are further identified as a way to categorize and analyze cases. These categories distinguish cases in which the target action of M-A is best thought of as
(i) the provision of a service,
(ii) the transfer of an object,
(iii) the alteration of some ongoing trajectory of behavior, or
(iv) if there was no clear M-A and participant B stepped in to provide assistance in response to A’s current or anticipatable trouble. This last category is not a request in that there is no on-record solicitation of a response, but is a recruitment in that practical assistance is instigated by A’s visible trouble. The term “on-record” here refers to identifiable moves in social interaction that ask for or otherwise overtly signal the need for a target action; categories (i), (ii) and (iii) were required to be on-record in this sense, while (iv) was not.

\begin{table}
\begin{tabularx}{0.66\textwidth}{Xrr}
\lsptoprule
Recruitment subtype & Count & Proportion\\
\midrule
Service provision & 152 & 74\%\\
Object transfer & 42 & 20\%\\
Trouble assistance & 7 & 3\%\\
Alteration of trajectory & 4 & 2\%\\
\lspbottomrule
\end{tabularx}
\caption{Relative frequencies of recruitment sequence subtypes in the Cha'palaa sample (\textit{n}=205).}
\label{tab:floyd:1}
\end{table}

Extracts \ref{ex:floyd:1} and \ref{ex:floyd:2} have already provided examples of the two most frequently occurring categories; in \REF{ex:floyd:1} participant A requests the provision of a service, sweeping the floor, and in \REF{ex:floyd:2} participant A asks for an object, a plastic tub. \extref{ex:floyd:3} shows an example of an alteration of a trajectory of action: A notices that B is sitting in such a way that she appears uncomfortable, and tells her to alter the way she is currently sitting to a more restful position, giving the reason that otherwise her back will hurt. This example also helps us illustrate how benefit may be differently distributed in recruitment sequences as here the primary beneficiary is the recruitee herself.

\transheader{ex:floyd:3}{CHSF2011\_06\_25S2\_3916900}\vspace{-1mm}
%
\begin{mdframednoverticalspace}[style=firstfoc]
\begin{transbox}{1}{a}
\begin{verbatim}
leka leka leka beenbushu kiya
rest rest rest back      hurt
\end{verbatim}
rest rest rest, (your) back will hurt
\end{transbox}
\end{mdframednoverticalspace}
%
\begin{mdframednoverticalspace}[style=secondfoc]
\xtransbox{2}{b}{((reclines))}
\end{mdframednoverticalspace}

As illustrated in \tabref{tab:floyd:1}, alterations of trajectory were the least frequent of the sequence types in the Cha'palaa sample. Preliminary analysis of the cases involving children -- excluded from the comparative data set, as mentioned above in \sectref{sec:floyd:1.2} -- show many more attempts to alter and correct behavior in those cases, suggesting that social status may play a role, and that adults may try to avoid such potentially face-threatening interactions among each other, while in similar interactions with children such recruitments may be the norm. Future work with child-involved cases stands to shed more light on these issues.

The last subtype of case that was included in the sample were sequences in which B steps in to assist A with some problem that, while usually evident from the context, has not been explicitly formulated by A. For example, in \REF{ex:floyd:4}, B and several boys were sitting in front of the kitchen door, scraping and eating coconut shavings. When A begins to approach with a heavy load of bananas, B and the others first gaze at her and then proceed to move the bowls, stools, and other objects out of her way, and to lean to the side to allow her to pass into the kitchen more easily (\figref{fig:floyd:1}).

\transheader{ex:floyd:4}{CHSF2012\_08\_05S5\_363190}%\vspace{-1mm}
%
\begin{mdframednoverticalspace}[style=firstfoc]
\xtransbox{1}{a}{((walks towards door with load of bananas))}
\end{mdframednoverticalspace}
%
\begin{mdframednoverticalspace}[style=secondfoc]
\xtransbox{2}{b}{((moves bowl out of way, leans away))}
\end{mdframednoverticalspace}

\begin{figure}
\caption{Frames from \extref{ex:floyd:4}. Family members facilitate a woman's arrival with a large load of plantains that she needs to deposit in the kitchen.}
%Figures are numbered to correspond to extracts, for convenience of reference to specific cases. If a different numbering system is required (e.g. consecutive starting with 1) this can be implemented.
\subfigure[\label{fig:floyd:1a} Participant B and accompanying children collecting coconut shaving in a bowl, while participant A is approaching with a load of bananas (in front of them, off camera) (line 1).]{
  \includegraphics[height=.22\textheight]{figures/chapalaa-img1.jpg}
}

\subfigure[\label{fig:floyd:1b} Participant B and children see participant A approaching and move the bowl and stool out of her way (line 2).]{
  \includegraphics[height=.22\textheight]{figures/chapalaa-img2.jpg}
}

\subfigure[\label{fig:floyd:1c} Participant B and children lean out of the way as participant A passes with her load of bananas (line 2).]{
  \includegraphics[height=.22\textheight]{figures/chapalaa-img3.jpg}
}
\label{fig:floyd:1}
\end{figure}

While in these types of cases there is no on-record M-A by A, the types of services and objects that B provides in such cases are the same types of local practical actions that are explicitly asked for in other instances (e.g. ‘Move over so I can pass’, etc.).

\section{Move A: The recruiting move}\label{sec:floyd:3}

The formats used by participant A in M-A could be fully nonverbal, fully verbal, or a composite of verbal and nonverbal elements. This section describes the composition of M-A in both the visual and the spoken channels.

\subsection{Fully nonverbal recruiting moves}\label{sec:floyd:3.1}

Most of the recruiting formats in the Cha'palaa sample included spoken elements; of the 205 cases sampled, only nine were fully nonverbal in M-A (excluding cases of trouble assistance, see \sectref{sec:floyd:2.3}). An example of a fully nonverbal case is shown in \REF{ex:floyd:5}, in which A and B are taking care of an injured chick together. During a moment when no spoken conversation is ongoing, A holds out the chick for B to hold for a moment so A can free his hands to manipulate some thread. B responds to A holding the chick out by reaching up to take it (\figref{fig:floyd:2}).

\newpage
\transheader{ex:floyd:5}{ CHSF2012\_01\_20S6\_3387180}%\vspace{-1mm}
%
\begin{mdframednoverticalspace}[style=firstfoc]
\xtransbox{1}{a}{((holds out chick))}
\end{mdframednoverticalspace}
%
\begin{mdframednoverticalspace}[style=secondfoc]
\xtransbox{2}{b}{((takes chick))}
\end{mdframednoverticalspace}

\begin{figure}
\includegraphics[height=.25\textheight]{figures/chapalaa-img4.png}
\caption{Participant A holds out his hand and participant B hands the chick to him.}
\label{fig:floyd:2}
\end{figure}

The main formats for nonverbal requests in the sample were holding out objects, as in \REF{ex:floyd:5}, and reaching out to receive objects. These were also two of the major nonverbal formats seen accompanying verbal recruiting formats, the topic of the next section.

\subsection{Nonverbal behavior in composite recruiting moves}\label{sec:floyd:3.2}

In most Cha'palaa recruitment cases M-A includes no nonverbal behavior that is salient or relevant for the sequence; instead, target actions or other elements are expressed verbally. However, in 55 of 205 cases some relevant nonverbal behavior occurred. As mentioned in the previous section, nine of these were independent nonverbal recruiting moves, but the other 46 were composites including verbal and nonverbal elements. While some nonverbal behavior was idiosyncratic and did not lend itself to categorization, several well-defined types of practices made up the majority of the nonverbal elements for Cha'palaa. In Cha'palaa three practices accounted for about 85\% of all nonverbal behavior seen in recruiting moves (see \tabref{tab:floyd:2}). Pointing was the most common of these, accounting for 42\% of nonverbal elements. Pointing gestures usually indicated an object, location, or person that was relevant for the recruitment in some way. The next-most-common practice was holding out an object that is instrumentally involved in the recruitment, typically to be taken by B to do something with, as seen in \REF{ex:floyd:5}, above. The other major practice in the sample was reaching out to receive an object. In addition, there was one instance of iconic-symbolic gesture (beckoning for B to approach A), and about 15\% were heterogeneous practices that did not fit into any of the cross-linguistic coding categories applied in the comparative project.

\begin{table}
\begin{tabularx}{0.66\textwidth}{Xrr}
\lsptoprule
Nonverbal practice & Count & Proportion \\
\midrule
Pointing gesture & 23 & 42\%\\
Holding out object & 15 & 27\%\\
Reach to receive object & 8 & 15\%\\
Iconic gesture & 1 & 2\%\\
Other & 8 & 15\%\\
\lspbottomrule
\end{tabularx}
\caption{Types of nonverbal practices in recruiting moves (\textit{n}=55).}
\label{tab:floyd:2}
\end{table}

\figref{fig:floyd:3} shows a pointing gesture that accompanied the spoken recruiting turn ‘give me the string there’ (full sequence shown in \extref{ex:floyd:10}). Along with his indexical point, participant A also uses lip pointing, a practice observed commonly among Cha'palaa speakers \citep{dingemanse_conversation_2014} and in many other languages \citep{Sherzer1973,Enfield2001}.

\begin{figure}
\caption{\label{fig:floyd:3} Screenshot from \extref{ex:floyd:10}; index finger and lip pointing as part of Move A.
}
\includegraphics[height=.25\textheight]{figures/chapalaa-img5.png}
% \todo[inline]{check labelling}
\end{figure}

In some cases the nonverbal behavior was relatively complex, as in \REF{ex:floyd:6} below, in which A first extends his arm and points at the menthol ointment he is requesting (\figref{fig:floyd:4}), saying ‘give me that also’ and, after a brief pause, providing a reason for the request (‘I will smell a little’) as he turns his palm upwards to receive the object. One interesting element of the nonverbal behavior in this case is that A has already extended his arm by the first part of M-A, suggesting that he has high expectations that the request will be fulfilled, possibly based on it being a relatively “low contingency” request \citep{CurlDrew2008,CravenPotter2010}.

\transheader{ex:floyd:6}{CHSF2011\_01\_11S3\_2692960}\vspace{-1mm}
%
\begin{mdframednoverticalspace}[style=firstfoc]
\begin{transbox}{1}{a}
\begin{verbatim}
kuke inuba aantsa (.) jayu ishkeechi
ku  -ke   i  -nu -ba   aansta jayu   ish  -kera -chi
give-do   1SG-ACC-also that   little smell-see  -INGR
\end{verbatim}
give me also that (.) (I) will smell a little
\end{transbox}
\end{mdframednoverticalspace}
%
\emptytransbox{2}{((reaches out pointing while speaking, turns hand upward))}\vspace{-2mm}
%
\begin{mdframednoverticalspace}[style=secondfoc]
\xtransbox{3}{b}{((hands menthol to A))}
\end{mdframednoverticalspace}

\begin{figure}
\caption{\label{fig:floyd:4} A (center left) reaches out while requesting the menthol (in line 1). }
\includegraphics[height=.25\textheight]{figures/chapalaa-img6.png}
\end{figure}

\normalsize
In Cha'palaa, these three practices of pointing, holding out objects, and reaching to receive object made up more that 80\% of the total nonverbal behavior seen in recruiting moves (see \tabref{tab:floyd:2}).\footnote{ These are practices with deep roots, being among the first to appear developmentally \citep{Masur1983,Cameron_FaulknerEtAl2015}.} However, there is an asymmetry between M-A and M-B in recruitment sequences in that while M-B tends to involve nonverbal elements, especially the accomplishment of the target action, more than two thirds of M-As were in the verbal channel (71\%, \textit{n}=146/205), with only 29\% of M-As including nonverbal elements (\textit{n}=59/205).\footnote{Note, however, that of the 146 cases involving language, in 38 the presence or absence of nonverbal elements could not be ascertained due to the recruiter being off camera or with visual access impeded by another participant.} The next sections describe the verbal recruiting formats.

\subsection{Verbal elements: construction types and subtypes}\label{sec:floyd:3.3}

The spoken elements of M-A are mainly made up of the morphosyntactic resources of the Cha'palaa language (sometimes also with elements of Spanish, as the primary local second language). Verbal elements were classified according to cross-linguistic syntactic categories of declarative, interrogative, imperative (see \citealt{KönigSiemund2007}), as well as cases of “no predicate”, and “other” cases for predicates that do not fit well with any of the main categories (see \chapref{sec:coding}, \sectref{sec:coding:6}). Not all languages distinguish among sentence types in the same way, but in most cases Cha'palaa features very clear and unambiguous morphological distinctions on the verb associated with the three major sentence types (more on this below). As for frequency, imperatives outnumber the others considerably.

\begin{table}
\begin{tabularx}{0.66\textwidth}{Xrr}
\lsptoprule
Construction type & Count & Proportion\\
\midrule
Imperative & 137 & 71\%\\
No predicate & 22 & 11\%\\
Declarative & 20 & 10\%\\
Interrogative & 13 & 7\%\\
\lspbottomrule
\end{tabularx}
\caption{Construction type of recruiting moves including spoken elements (\textit{n}=192).}
\label{tab:floyd:3}
\end{table}

\subsubsection{No predicate}\label{sec:floyd:3.3.1}

Reviewing all of the different morphosyntactic types and their functions, we can start with cases in which there was no predicate, which belong to no sentence type in a strict sense. These cases can be classified by a few simple categories: of the 22 cases without a predicate, 12 name an object to be transferred, 6 name places that were relevant for the target action, 3 are vocatives selecting the recipient, and one was an interjection. Because the last two categories do not specify any element of the recruited action, they generally occur as a second attempt to a previous recruiting move that was not successful (see also Kendrick, \chapref{sec:kendrick}, \sectref{sec:kendrick:4.2.4}). The other formats can generally function as independent recruiting moves as well as subsequent attempts. For example, speakers can name destinations as a way to tell addresses to go to those places or take things to or from those places. Sometimes other grammatical resources come into play, like the locative case marker with the first-person pronoun in \REF{ex:floyd:7} that specifies that the addressee should do something to or for participant A. Object naming usually functions to request the object in question, sometimes with additional material like in M-A of \REF{ex:floyd:7}, which also specifies a recipient, but which leaves the target action of giving or passing up to the recipient’s inference. While unspecified, the requested action is usually obvious from the context, and so this type of recruiting practice can be considered relatively explicit or on-record (on “namings” see also \citealt[chap. 2]{Rossi2015a}).

Also worth noting here is that, in M-B, participant B acts towards the fulfillment of the target action, but she does so in a particular way: by delegating to a third party (see also Enfield, \chapref{sec:enfield}, \sectref{sec:enfield:6}; Blythe, \chapref{sec:blythe}, \sectref{sec:blythe:4.2.2}). This was a strategy sometimes observed when the target action was obviously easier for a third party, for example when they were closer to a target object, or of lower social status, both of which were the case for participant C in \REF{ex:floyd:7}.\footnote{A note on the translation of \textit{apa} in \REF{ex:floyd:7}, line 2: in Cha’palaa it is common to use affectionate vocatives that are the equivalent of “mommy” and “daddy” for small male and female children; the result is difficult to translate to English.}

\transheader{ex:floyd:7}{CHSF2012\_08\_04S4\_1524500}\vspace{-1mm}
%
\begin{mdframednoverticalspace}[style=firstfoc]
\begin{transbox}{1}{a}
\begin{verbatim}
inu  jabon ((in water, points at soap on shore))
i  -nu  jabon
1SG-ACC soap
\end{verbatim}
to me, soap
\end{transbox}
\end{mdframednoverticalspace}
%
\begin{mdframednoverticalspace}[style=secondfoc]
\begin{transbox}{2}{b}
\begin{verbatim}
jabon tya'kide apa ñaa
jabon tyatyu-ki-de  apa     ñu-ya
soap  throw -do-IMP father 2SG-FOC
\end{verbatim}
throw the soap, son, you
\end{transbox}
\end{mdframednoverticalspace}
%
\emptytransbox{3}{((points at soap))}\vspace{-2mm}
%
\xtransbox{4}{c}{((child throws soap to A))}
% \todo{updated numbering. Check crossrefs}

\normalsize
Aside from the 22 cases of M-A without a predicate, all other cases with verbal material in M-A included a predicate of some kind.

\subsubsection{Imperatives}\label{sec:floyd:3.3.2}

In contrast with the more context-dependent cases without predicates, most of the time speakers gave more information about the target action by producing a predicate (89\% of recruitments with spoken M-A). Of these, as noted in \tabref{tab:floyd:3}, imperative forms were by far the most frequent type of predicates seen in recruiting moves. The imperative sentence type in Cha'palaa does not consist of a single construction, but instead features several options (see also Enfield, \chapref{sec:enfield}, \sectref{sec:enfield:4.3.1}). Cross-linguistically, imperative verb forms tend to be relatively short, frequently consisting of just a verb root or a root with a minimal marker \citep{Khrakovskij2001,aikhenvald2010}. Cha'palaa fits this pattern; its two main imperative constructions are a bare verb root or a suffix, \textit{-de}, seen in line 2 of \REF{ex:floyd:7} above. \tabref{tab:floyd:4} summarizes the different imperative formats observed in the sample.

\begin{table}
\begin{tabularx}{\textwidth}{Xllrr}
\lsptoprule
Imperative subtype & Form & Plural & Count & Proportion \\
\midrule
Bare imperative & V & (de-)V & 89 & 65\%\\
Simple imperative & V-de & (de-)V-dei & 37 & 27\%\\
Speaker-directed & V-ka & (de-)V-kai & 7 & 5\%\\
Strong hortative & V-da & (de-)V-dai & 4 & 3\%\\
Weak hortative & V-sa & (de-)V-sai & 0 & 0\%\\
\lspbottomrule
\end{tabularx}
\caption{The Cha’palaa imperative paradigm, singular and plural forms (optionally marked for plural), and percentages of each format within the total of imperatives in the sample (\textit{n}=137).}
\label{tab:floyd:4}
\end{table}

The bare root option is shown in \REF{ex:floyd:8}. In a few limited contexts, declaratives can also occur as bare verb roots, so the comparable imperative format relies to a small degree on context for disambiguation.  Cha'palaa has a system of complex predicates in which multiple roots combine in single predicates, where one of the roots, usually one of a set of verb classifiers, occurs farthest to the right, and takes the finite morphology (\citealt{Floyd2014a}; see also \citealt{Dickinson2000} for a description of a similar system in a related language, Tsafiki). In most cases finite predicates take at least one verbal morpheme, but one of the options for forming imperatives is to use just the verb root. In \REF{ex:floyd:8} participant A takes this option, telling B to look at a magazine she is passing to her.

\newpage
\transheader{ex:floyd:8}{CHSF2011\_06\_24S3\_1304600}\vspace{-1mm}
%
\begin{mdframednoverticalspace}[style=firstfoc]
\begin{transbox}{1}{a}
\begin{verbatim}
lenke' kerake ((hands B magazine))
len -ke-tu kera-ke
read-do-SR see -do
\end{verbatim}
read this
\end{transbox}
\end{mdframednoverticalspace}
%
\begin{mdframednoverticalspace}[style=secondfoc]
\xtransbox{2}{b}{((takes magazine and reads))}
\end{mdframednoverticalspace}

In addition to the two most frequent imperative formats shown above in \REF{ex:floyd:7}, line 2, and \REF{ex:floyd:8}, Cha’palaa has three further imperative markers, plus distinct plural forms of each. \tabref{tab:floyd:4} shows each of the formats’ frequencies relative to the other imperative options in the sample. The bare verb form sometimes occurs with a lengthened vowel and a related shift in primary stress to the right, which might be considered a kind of strengthener (see \sectref{sec:floyd:3.4}). The next most-frequent format is the general imperative \textit{-de}, which conveys an on-record wish that the addressee do the target action, and a speaker-directed imperative \textit{-ka} that conveys that the speaker is the indirect object and beneficiary. Then there are two hortative forms for groups that the speaker includes him or herself in, one for stronger, “command” types of recruiting moves (\textit{-da}), and one for weaker, “suggestion” types of recruiting moves (\textit{sa}). This final polite option did not occur in the sample, perhaps in connection with the maximally informal nature of the recordings, which may lead more formal, delicate types of recruiting practices to be infrequent.

When using an imperative, it is possible to mark a beneficiary of three-place predicates like ‘give’ with a full noun phrase, as in \REF{ex:floyd:9}.

\transheader{ex:floyd:9}{CHSF2011\_02\_14S3\_2673050}\vspace{-1mm}
%
\begin{mdframednoverticalspace}[style=firstfoc]
\begin{transbox}{1}{a}
\begin{verbatim}
inu jayu kude aamama shipijcha ((reaches towards B))
i  -nu  jayu     ku  -de  aamama  shipijcha
1SG-ACC a.little give-IMP grandma 'madroña'
\end{verbatim}
give me a little madroña grandma
\end{transbox}
\end{mdframednoverticalspace}
%
\begin{mdframednoverticalspace}[style=secondfoc]
\xtransbox{2}{b}{((turns towards A, begins passing fruit))}
\end{mdframednoverticalspace}

There is also a special imperative marker that is only compatible with first-person beneficiaries, \textit{{}-ka}, illustrated in \REF{ex:floyd:10}, line 3.

\transheader{ex:floyd:10}{CHSF2012\_01\_20S6\_2952739}\vspace{2mm}
%
\begin{transbox}{1}{a}
\begin{verbatim}
vieja
\end{verbatim}
old lady
\end{transbox}
%
\begin{transbox}{2}{b}
\begin{verbatim}
aa
\end{verbatim}
huh?
\end{transbox}
%
\begin{mdframednoverticalspace}[style=firstfoc]
\begin{transbox}{3}{a}
\begin{verbatim}
inu chuwa manka' kuka junu jee ((finger and lip point))
i  -nu  chuwa ma   -n   -ka  -tu ku  -ka   junu  jee
1SG-ACC vine  again-IMFV-grab-SR give-IMP1 there yes
\end{verbatim}
give me the string there hey
\end{transbox}
\end{mdframednoverticalspace}
%
\begin{mdframednoverticalspace}[style=secondfoc]
\xtransbox{4}{b}{((brings string))}
\end{mdframednoverticalspace}

When the speaker is included as a participant in the target action along with the interlocutor, one of two different hortatives may be used. The first, \textit{{}-da}, was the only one of the two attested in the sample, indicating that it is probably used more frequently in general in informal contexts. \extref{ex:floyd:11} shows a case of this hortative, when one teenager attempts to recruit another to go fishing. The sequence was unsuccessful and was abandoned when A did not respond to B after line 4.

\transheader{ex:floyd:11}{CHSF2011\_02\_14S3\_3479997}\vspace{2mm}
%
\begin{transbox}{1}{a}
\begin{verbatim}
Gringo
\end{verbatim}
Gringo ((nickname))
\end{transbox}
%
\begin{transbox}{2}{b}
\begin{verbatim}
aa
\end{verbatim}
huh?
\end{transbox}
%
\begin{mdframednoverticalspace}[style=firstfoc]
\begin{transbox}{3}{a}
\begin{verbatim}
waaku tyuinsha jidaa laaba
waaku tyui -n   -sha ji-daa  lala-ba
net   press-IPFV-LOC go-HORT 1PL -with
\end{verbatim}
let's go net fishing, with us
\end{transbox}
\end{mdframednoverticalspace}
%
\begin{transbox}{4}{b}
\begin{verbatim}
maa waaku tyuindetsun
mu -ya  waaku tyui -n   -de-tsu -n
who-FOC net   press-IPFV-PL-PROG-Q
\end{verbatim}
who is going net fishing?
\end{transbox}\bigskip

Outside of the sample, looking into the video corpus more broadly, it was possible to find examples of the second hortative, \textit{-sa}. This marker is identical to a dependent clause marker for modal complements, and it is likely that the hortative use developed through processes of “insubordination” (\citealt{Evans2007,Floyd2016,EvansWatanabe2016}) when the dependent clause developed a conventionalized main clause usage, and became incorporated into the imperative paradigm. It alternates with \textit{{}-da} as a more “mitigating” option. At present it is so integrated into the imperative system that it takes the plural marker that only combines with imperative forms, \textit{{}-i}. \extref{ex:floyd:12} shows an example of the plural form of \textit{-sa}.

\transheader{ex:floyd:12}{CHSF2012\_01\_07S1\_137560}\vspace{-1mm}
%
\begin{mdframednoverticalspace}[style=firstfoc]
\begin{transbox}{1}{a}
\begin{verbatim}
jisai lalaa
ji-sai      lala-ya
go-HORT2.PL 1PL -FOC
\end{verbatim}
let's go
\end{transbox}
\end{mdframednoverticalspace}
%
\begin{mdframednoverticalspace}[style=secondfoc]
\xtransbox{2}{b}{((leaves house with A)) }
\end{mdframednoverticalspace}

Note that, at a finer level of categorization, the social actions in \REF{ex:floyd:11} and \REF{ex:floyd:12} may be also analyzed as “proposals” \citep{couper-kuhlen_what_2014,StiversSidnell2016}. These fall within the phenomenon of recruitment, which broadly encompasses sequences in which A obtains B’s assistance or collaboration in doing something for or with them (see \chapref{sec:intro}, \sectref{sec:intro:4}).

The imperative system is flexible with respect to grammatical norms, plural marking is available but optional, and the motivations of speakers for choosing one of the three second-person imperatives or one of the two hortatives will require further research to be fully determined. This first analysis shows that such grammatical flexibility provides diverse options for different interactional contingencies.

\subsubsection{Interrogatives}\label{sec:floyd:3.3.3}

While imperatives usually function as unambiguous on-record recruiting moves, the other two main sentence types, interrogatives and declaratives provide ways for instigating a fitted response to the recruiting move without going explicitly on-record, and have been discussed with respect to indirectness in speech acts \citep{Searle1969}. Interrogatives often inquire about preconditions for the request \citep{Levinson1983}, potentially launching a “pre-sequence” \citep{Schegloff1980,schegloff_sequence_2007,Rossi2015b}, but conventional use of interrogative formats for pre-requesting can result in the “collapse” of the pre-sequence to the point that interrogatives can act as independent request formats (\citealt{Levinson1983}; \citealt{Fox2015}). This is the case with the format seen in line 1 of \REF{ex:floyd:13}, which inquires about the availability of an object (‘is your saw not there?’), but which ends up being taken as a request for the object.

\transheader{ex:floyd:13}{CHSF2012\_01\_21S3\_2615530}\vspace{-1mm}
%
\begin{mdframednoverticalspace}[style=firstfoc]
\begin{transbox}{1}{a}
\begin{verbatim}
ñuchi serruchu tsutyuu ((off camera, outside of house))
ñu -chi  serruchu tsu-tyu-u
2SG-POSS saw      lie-NEG-Q
\end{verbatim}
is your saw not there?
\end{transbox}
\end{mdframednoverticalspace}
%
\begin{transbox}{2}{b}
\begin{verbatim}
aa
\end{verbatim}
huh?
\end{transbox}
%
\begin{transbox}{3}{a}
\begin{verbatim}
serruchu tsutyuu
serruchu tsu-tyu-u
saw      lie-NEG-Q
\end{verbatim}
is (your) saw not there?
\end{transbox}
%
\begin{transbox}{4}{b}
\begin{verbatim}
enku (.) tanami ibain (.)
en  -ku  ta  -na       -mi   i  -bain
here-LOC have-be.in.POS-DECL 1SG-also
\end{verbatim}
also here, I have it (.)
\end{transbox}
%
\begin{transbox}{5}{~}
\begin{verbatim}
jayaa   finberaya
jayu      -ya  fi -n  -bera -ya
little.bit-FOC eat-PFV-still-FOC
\end{verbatim}
it ‘eats’ a little (it saws decently)
\end{transbox}
%
\begin{transbox}{6}{~}
\begin{verbatim}
seruchu  tii ((to C))
serruchu ti -i
saw      say-Q
\end{verbatim}
did (he) say ‘saw’?
\end{transbox}
%
\begin{transbox}{7}{c}
\begin{verbatim}
mm
\end{verbatim}
yeah
\end{transbox}
%
\xtransbox{8}{b}{((goes to get saw, returns))}\vspace{-2mm}
%
\begin{mdframednoverticalspace}[style=secondfoc]
\begin{transbox}{9}{~}
\begin{verbatim}
Ebe jee ((holds out saw))
Ebe hey
\end{verbatim}
here Ebe
\end{transbox}
\end{mdframednoverticalspace}\bigskip

\normalsize
Going through the interaction above line-by-line helps to illustrate how a question about the presence of an object is treated by the participants as a request for the object. In line 1 A inquires about the saw, using the verb ‘lie’, which is the appropriate positional verb for elongated objects on flat surfaces. Possibly because A is standing outside the house and did not have B's full attention, B displays some trouble hearing line 1 and initiates repair in line 2, occasioning a full repetition in line 3 (typical for an “open” repair initiator \citealt{Drew1997}; see \citealt{Floyd2015} for a description of the Cha’palaa repair system). In line 4, B answers the question, confirming that the saw does in fact exist. But this is not all he does; in line 5 he also gives some information about the status of the object with respect to its function (‘it saws decently’), giving evidence that he understands that lines 1 and 3 are geared towards getting the saw. Interestingly, B has chosen to respond to A even though he appears not to be fully certain of the target action, as in line 5 he requests further confirmation from his wife C. After this, B proceeds to fulfill the request in 8, but at no point has A overtly asked to be given the saw.

Interrogative formats like that seen above in \REF{ex:floyd:13} are frequent in cases of requests for transfers of objects: compare 7\% of interrogative recruiting moves in the total sample (\textit{n}=13/192) with 21\% for object transfers (\textit{n}=9/42). Additionally, if the object is not visible, interrogatives are used in 47\% of cases (\textit{n}=8/17); this is to be expected, because these are canonical contexts in which a participant might check the preconditions for a request before making it, thus avoiding rejection on the grounds of a faulty presupposition that the object was available (see also Rossi, \chapref{sec:rossi}, \sectref{sec:rossi:3.3.3}). 

In other situations, other interrogative formats can be used. In cases of requests for the provision of services, for example, a speaker might ask a question about a target action to convey that they would like an addressee to do this action. An example of this was seen in \REF{ex:floyd:1} with the question ‘did/do you sweep?’. The common feature of all the different types of interrogative formats is that, each in different ways, they use what is on the surface a request for information as a way to request an activity.

\subsubsection{Declaratives}\label{sec:floyd:3.3.4}

Declaratives are another format for less direct or off-record requests. These work by introducing a proposition about some state of affairs, but with an implicit understanding that some action should be taken by an addressee. Relative to imperatives, this format allows speakers to avoid overtly selecting an addressee for the recruitment. In some cases the addressee may be obvious from the context -- if A is gazing directly at B, for instance (see \citealt{Lerner2003}), or if the interaction is dyadic -- while in other cases participants might self-select and construe themselves as the addressee. One cross-linguistically common format for declarative recruiting moves is deontic statements about how things should be, or what things need to be done (see also Rossi, \chapref{sec:rossi}, \sectref{sec:rossi:3.3.4}; Zinken, \chapref{sec:zinken}, \sectref{sec:zinken:3.3.2}; Baranova, \chapref{sec:baranova}, \sectref{sec:baranova:3.3.3}). \extref{ex:floyd:14} provides one such example of a deontic construction in Cha'palaa, which is formed with the combination of an infinitive verb and a copula (a very common deontic construction type in South America, \citealt{Müller2013}). In this case a husband A and wife B were working together to nail in some boards to repair a wall, and when A makes a statement about the task that should be done, B responds by altering the way in which she is performing the task.

\transheader{ex:floyd:14}{CHSF201}\vspace{-1mm}
%
\begin{mdframednoverticalspace}[style=firstfoc]
\begin{transbox}{1}{a}
\begin{verbatim}
tu- tu'pushujuntsaa kanu juaa
tu-  tu'pu-shujunsta-ya  ka  -nu  ju-ya
nail nail -REL.CL   -FOC grab-INF be-FOC
\end{verbatim}
(one) must grab the part that was nailed
\end{transbox}
\end{mdframednoverticalspace}
%
\begin{mdframednoverticalspace}[style=secondfoc]
\xtransbox{2}{b}{((grabs and moves board))}
\end{mdframednoverticalspace}

Research on some European languages has shown similar usages of deontic constructions \citep{ZinkenOgiermann2011,Couper-KuhlenEtelämäki2015,RossiZinken2016}, suggesting that this particular strategy may be cross-lin\-guis\-tically recurrent. Apart from the specific deontic constructions seen above, a further wide range of declarative construction types can function as recruiting moves in the right contexts. For example, \REF{ex:floyd:15} gives us a case of a pursuit of a recruitment that was not fulfilled after the first attempt, which was an imperative: ‘look for lice on me’. While a bit taboo in Western cultures, picking parasites of each other is an important social interactive practice among peoples from different parts of the world, including the Chachis of Ecuador, for whom it is considered an affectionate form of behavior most common among family members. In this case, however, when A tells her husband B to groom her in this way, he displays no uptake, and continues a parallel line of conversation, leading to a second attempt by B in line 3, this time in a declarative format.

\transheader{ex:floyd:15}{CHSF2012\_08\_04S4\_1524500}\vspace{-1mm}
%
\begin{mdframednoverticalspace}[style=firstfoc]
\begin{transbox}{1}{a}
\begin{verbatim}
inu mu keraa ((sits with back towards B))
i  -nu  mu   kera-a
1SG-ACC lice look-IMP
\end{verbatim}
look for lice on me
\end{transbox}
\end{mdframednoverticalspace}
%
\xtransbox{2}{b}{((no uptake, 88.0 unrelated conversation))}
%
\begin{mdframednoverticalspace}[style=firstfoc]
\begin{transbox}{3}{a}
\begin{verbatim}
ñaa inu mu keetyunkayu mm mm ((scratches head))
ñu-ya  i  -nu  mu    kee-tyu-nkayu mm mm
2 -FOC 1SG-ACC louse see-NEG-EV    mm mm
\end{verbatim}
you aren't looking for lice on me, hey
\end{transbox}
\end{mdframednoverticalspace}
%
\xtransbox{4}{b}{((no uptake, continues unrelated conversation))}

\normalsize
Participant A had been sitting with her back to her husband, giving him access to her hair for over a minute when she makes a second attempt at recruitment (line 3). This time she uses a declarative format, using a negation construction to call attention to a state of affairs that is not currently the case (‘you aren’t looking for lice’). Similar “negative observations” have been shown to be a format for complaining \citep{Schegloff1988,Rossi2018}. In light of the first, more overt recruiting move in line 1, this statement can be taken as a request that B do the relevant action.

\subsubsection{Other construction types}\label{sec:floyd:3.3.5}

In addition to specific verbal morphemes, there are specialized verbal constructions that can be resources for initiating recruitment. One good example of this is a benefactive construction using the verb ‘give’ as an auxiliary to indicate beneficiaries, a construction which appears in several other unrelated local languages and may be a product of areal convergence (see \citealt{Bruil2008} on Ecuadorian Spanish and Quechua). While this construction literally asks one to ‘give’ the action, when the verb ‘give’ is used with a second verb it means ‘do it for someone’s benefit’. An example can be seen in \REF{ex:floyd:16}.

\transheader{ex:floyd:16}{CHSF2011\_02\_14S3\_1828314}\vspace{-1mm}
%
\begin{mdframednoverticalspace}[style=firstfoc]
\begin{transbox}{1}{a}
\begin{verbatim}
panda tune' kude junka tsai kalarade
panda tune-tu ku  -de  junka tsai kalara     -de
food  cook-SR give-IMP there SEM  take(photo)-IMP
\end{verbatim}
cook plantain for them, take a video like that.
\end{transbox}
\end{mdframednoverticalspace}
%
\begin{transbox}{2}{~}
\begin{verbatim}
jee junka kera' uyudenaa tinkai
jee junka kera-tu uyu  -de-na -ya  ti -nkayu
hey there see -SR stand-PL-POS-FOC say-EV
\end{verbatim}
yeah, standing and looking over there, it was said.
\end{transbox}
%
\begin{transbox}{3}{b}
\begin{verbatim}
panda tsunami   nain
panda tsu-na -mi   na -i     -n
food  lie-POS-DECL how-become-Q
\end{verbatim}
how is there plantain?
\end{transbox}\bigskip

In line 1 A asks B to cook plantain, but uses the ‘give’ construction to mean that a third party, the other family members present, will benefit (the additional comment about filming is a bit of “camera behavior” in which participants in the recording make reference to the recording equipment). This benefactive ‘give’ construction can occur with any of the major sentence types, and is not dedicated solely to recruitment, but when it occurs in M-A of recruitment sequences it has the effect of introducing beneficiaries through a conventionalized use of a ditransitive predicate to modify the argument structure.

\subsection{Additional verbal elements}\label{sec:floyd:3.4}

In addition to the predicate and its core arguments, there are other aspects of turn design that are relevant for the format of the recruiting move. This section reviews several of these non-core elements.

\subsubsection{Strengtheners and mitigators}\label{sec:floyd:3.4.1}

Some non-core elements can be considered strengtheners or mitigators with respect to how they upgrade or downgrade the recruitment in terms of its claimed urgency, importance, appropriateness, ease of accomplishment, and so on. One common strengthener in Cha’palaa is the word \textit{jee}, which shares several functions: it is the main positive polarity token (‘yes’), a vocative often used to secure attention (‘hey’), and a strengthening element in recruitments. Usually these different functions can be easily distinguished from their context of use, but \textit{jee} generally needs to combine with other elements like a verb to be able to specify a target action in a recruitment sequence, as in \REF{ex:floyd:17} where it combines with the imperative verb ‘look.’

\newpage
\transheader{ex:floyd:17}{CHSF2011\_06\_25S2\_1428820}\vspace{-1mm}
%
\begin{mdframednoverticalspace}[style=firstfoc]
\begin{transbox}{1}{a}
\begin{verbatim}
entsa ka' ura urake jee ((passes fiber to B))
entra ka-'      ura  ura -ke jee
this  grasp-SR  good good-do yes
\end{verbatim}
put this away, hey
\end{transbox}
\end{mdframednoverticalspace}
%
\begin{mdframednoverticalspace}[style=secondfoc]
\begin{transbox}{2}{b}
\begin{verbatim}
tse'mitya  lepe pupuki ((moves fiber piece))
tse-'mitya  lepe   pu -pu -ki
SEM-because broken put-put-do
\end{verbatim}
so then put the broken pieces here
\end{transbox}
\end{mdframednoverticalspace}\bigskip

Here A asked B to help remove some broken pieces of fiber during basket weaving, finishing the spoken part of her turn with \textit{jee}; B then takes the proffered fiber and proceeds to fulfill by doing the task. In this position, using \textit{jee} to help secure the attention of B can be seen as a strengthener, although it may occur in other contexts doing different things (for example, in Extract \ref{ex:floyd:9}, line 8, \textit{jee} occurs in the fulfillment of an object transfer request: ‘here take this’).

Another quite different format for strengthening consists of modulating the volume of the spoken elements of the recruiting move. \extref{ex:floyd:18} provides a good example of this strategy in the context of a pursuit sequence. The initial recruiting move in line 1 concerns A telling his wife B to hold onto a string so he can tie it. In line 3 he produces a second recruiting move giving more information about the position he wanted her to hold (‘on the tip’), implying that her first attempt at fulfillment had not been totally acceptable. Then in line 5 he repeats the format from line 3, consisting of a noun \textit{kapa}, meaning the ‘side’ or ‘tip’ of an object, and a locative suffix \textit{{}-nu}, but now produces it at higher volume.

\transheader{ex:floyd:18}{CHSF2012\_01\_20S6\_2509823}\vspace{-1mm}
%
\begin{mdframednoverticalspace}[style=firstfoc]
\begin{transbox}{1}{a}
\begin{verbatim}
tadi
ta  -di
have-come.into.POS
\end{verbatim}
 hold (this)
\end{transbox}
\end{mdframednoverticalspace}
%
\begin{mdframednoverticalspace}[style=secondfoc]
\xtransbox{2}{b}{((begins to hold string))}
\end{mdframednoverticalspace}
%
\begin{mdframednoverticalspace}[style=firstfoc]
\begin{transbox}{3}{a}
\begin{verbatim}
mm kapanu
mm   kapa    -nu
yeah side/tip-LOC
\end{verbatim}
yeah, on the tip
\end{transbox}
\end{mdframednoverticalspace}
%
\begin{mdframednoverticalspace}[style=secondfoc]
\xtransbox{4}{b}{((begins to hold more firmly)) }
\end{mdframednoverticalspace}
%
\begin{mdframednoverticalspace}[style=firstfoc]
\begin{transbox}{5}{a}
\begin{verbatim}
KAPANU
kapa    -nu
side/tip-LOC
\end{verbatim}
ON THE TIP!
\end{transbox}
\end{mdframednoverticalspace}\vspace{1mm}
%
\begin{mdframednoverticalspace}[style=secondfoc]
\xtransbox{6}{b}{((holds firmly so that A can cut string))}
\end{mdframednoverticalspace}\vspace{-2mm}
%
\begin{transbox}{7}{a}
\begin{verbatim}
enu main kake
e   -nu  main ka  -ke
here-LOC one  grab-do
\end{verbatim}
grab one here
\end{transbox}\bigskip

It is easy to see the difference in the two pronunciations of \textit{kapanu} from the waveform of the audio recording (\figref{fig:floyd:5}, in which participant A increased the volume of a repeated M-A to upgrade its format in a pursuit sequence.

\begin{figure}
\caption{Waveform of lines 3--5 of \extref{ex:floyd:18}, comparing the different volume levels of two successive productions of the same word.}
\includegraphics[width=\textwidth]{figures/chapalaa-img7.png}
\label{fig:floyd:5}
\end{figure}

It is interesting to note that the two strengthening strategies discussed here, the particle \textit{jee} and increased volume, both have connections to strategies for securing the attention of an addressee, either through using a vocative like ‘hey!’ or by making the words more perceptually salient by amplifying them. If other languages show a similar link between securing attention and strengthening recruiting moves, this may turn out to be a recurring strategy that combines securing attention to a recruitment and pursuing its fulfillment.

In addition to strengthening recruiting moves, Cha'palaa speakers also used different formats for mitigating them, or downplaying the contingency of the recruitment. One of these is a minimizing strategy that uses the word \textit{jayu}, or ‘a little bit’, in order to frame a recruitment as something small, insignificant, or easily accomplished. Work on politeness strategies has noted that words or morphemes about smallness are a cross-linguistically common strategy for managing face-threatening acts like recruiting moves (\citealt{BrownLevinson1987}). In \REF{ex:floyd:19} A tells his wife to ‘make rice’, but then adds the word \textit{jayu}. The word order is relevant because Cha'palaa is in most cases verb-final and modifiers like \textit{jayu} generally precede their nouns, suggesting that here it was added on as a late-stage calibration of the contingency of the recruitment.

\newpage
\transheader{ex:floyd:19}{CHSF2011\_02\_15S4\_6949122}\vspace{-1mm}
%
\begin{mdframednoverticalspace}[style=firstfoc]
\begin{transbox}{1}{a}
\begin{verbatim}
arosya kee jayu
aros-ya  ke-e   jayu
rice-FOC do-IMP a.little
\end{verbatim}
 make rice, a little.
\end{transbox}
\end{mdframednoverticalspace}
%
\begin{mdframednoverticalspace}[style=firstfoc]
\xtransbox{2}{b}{((gets out of hammock))}
\end{mdframednoverticalspace}

Here it appears that the quantity of rice was not really the issue, and that \textit{jayu} has more to do with minimization of the imposition of the recruitment.

\subsubsection{Explanations}\label{sec:floyd:3.4.2}

Another type of strategy that may be used for mitigation of recruitments is the provision of explanations (including accounts and similar) in the same turn as the core recruiting component, providing background information that presents the target action as more necessary, justified, or reasonable (see \citealt{Parry2013,BaranovaDingemanse2016}). For example, in \REF{ex:floyd:20} speaker A tells speaker B to ‘clean the baby's face’, which is very specific about the target action and its beneficiary, and could be a complete recruiting move on its own. However, A also adds the phrase ‘it is dirty’, which provides motivation for the target action.

\transheader{ex:floyd:20}{CHSF2011\_02\_15S4\_499970}\vspace{-1mm}
%
\begin{mdframednoverticalspace}[style=firstfoc]
\begin{transbox}{1}{a}
\begin{verbatim}
nanu kajuru mankijtikee kuchinuu
na   -nu  kajuru ma   -n   -kijti-ke-e   kuchinu-ju
small-ACC face   again-IPFV-clean-do-IMP dirty  -be
\end{verbatim}
clean the baby's face, it's dirty
\end{transbox}
\end{mdframednoverticalspace}
%
\begin{mdframednoverticalspace}[style=secondfoc]
\xtransbox{2}{b}{((gets up, takes baby and starts washing face))}
\end{mdframednoverticalspace}

Extra elements like mitigators and strengtheners were relatively uncommon. Explanations were present in 14\% of cases that included a verbal element (\textit{n}=27{\slash}192), while just 4\% included mitigators (\textit{n}=7/192), and strengtheners occurred in 10\% of cases (\textit{n}=20/192).

\section{Move B: The responding move}\label{sec:floyd:4}

Like M-A, M-B, the responding move, can be fully nonverbal, fully verbal, or a composite. But in this sequential position, speech and nonverbal behavior are subject to different conditions than in the initial position of most recruitment sequences. Since the cases in the sample all involved practical actions that could be accomplished or begun during the interaction, most cases included some kind of relevant nonverbal behavior in the response. More than half of cases (\textit{n}=105/205) included a clearly identifiable nonverbal response, either as the only response (\textit{n}=81/205) or as part of a composite move with verbal elements (\textit{n}=24/2015). In a number of cases it is impossible to see whether there is a visual element of a response due to participant B moving off camera or behind another person or an object (\textit{n}=79/205). In a smaller number of cases, responses only included verbal elements (\textit{n}=21/205); since these cases include no practical action, they partly correlate with cases of rejection, while cases including nonverbal, practical actions tend to be cases of fulfillment.

\begin{table}
\fittable{
\begin{tabular}{lrrrr}
\lsptoprule
Response type & Total cases (\textit{n}=205) & \% & Sequence-final cases (\textit{n}=125) & \%\\
\midrule
Fulfillment & 97 & 47\% & 69 & 55\%\\
Ignores & 40 & 20\% & 21 & 17\%\\
Other & 20 & 10\% & 10 & 8\%\\
Repair & 19 & 9\% & 2 & 1\%\\
Rejection & 8 & 4\% & 7 & 6\%\\
Not visible & 21 & 10\% & 16 & 13\%\\
\lspbottomrule
\end{tabular}
}
\caption{Fulfillment, rejection, and other response types in the Cha'palaa sample.}
\label{tab:floyd:5}
\end{table}

\tabref{tab:floyd:5} shows the breakdown of different types of responses in M-B of the recruitment sequences of the sample. The data show an overall predominance of fulfillment, but also a number of other options for response, including rejection, which will be discussed below in \sectref{sec:floyd:4.3}. First, response formats will be addressed in \sectref{sec:floyd:4.1} and \sectref{sec:floyd:4.2}.

\subsection{Fully nonverbal responses}\label{sec:floyd:4.1}

Well over half (64\%) of the responses in which B is visible in the video consisted of nonverbal elements only (\textit{n}=81/126); in most cases this corresponded to the accomplishment or beginning of the target action (see also \citealt{RauniomaaKeisanen2012}). Some target recruited actions could be accomplished quickly (e.g. A: ‘to me, the soap’ B: passes the soap), while for others B could only respond by beginning some activity that A can understand as projecting the completion of the target action (e.g. A: ‘cook a little rice’ B: begins cooking rice). \extref{ex:floyd:21} gives an example of the former, a case of a request for the alteration of a trajectory of ongoing activity that was fulfilled immediately after M-A. Participant B had been holding a baby but was not devoting full attention and the baby was beginning to slip out of his grasp. Participant A, noticing this, prompts B to hold the baby more firmly. B's change in behavior is immediate, includes no verbal elements, and is treated as a satisfactory fulfillment, in that it is no longer pursued by A beyond line 2.

\transheader{ex:floyd:21}{CHSF2011\_01\_11S3\_4728040}\vspace{-1mm}
%
\begin{mdframednoverticalspace}[style=firstfoc]
\begin{transbox}{1}{a}
\begin{verbatim}
kake kake kake
ka  -ke ka  -ke ka  -ke
grab-do grab-do grab-do
\end{verbatim}
grab (him) grab (him) grab (him)
\end{transbox}
\end{mdframednoverticalspace}
%
\begin{mdframednoverticalspace}[style=secondfoc]
\xtransbox{2}{b}{((holds baby more carefully, baby stops slipping))}
\end{mdframednoverticalspace}

Also notable is the repetition of the recruitment predicate. \citet{Stivers2004} observes that this type of repetition can be associated with urgency, and a similar connection can be made here: in the sample, several repetitions occur in recruitments dealing with alteration of trajectory in already-ongoing activities, which in this sense are more urgent than requests for services or objects, since the potential negative effects of not fulfilling the recruitment may be mounting while M-A is being produced (see also \extref{ex:floyd:3}, above, for a similar example of repetition in a case of urgency).

\subsection{Verbal elements of responses}\label{sec:floyd:4.2}

In cases in which a spoken element was part of the response, some of these were rejections, especially when only a verbal element was present, as these cases included no practical activity fulfilling the request. However, verbal elements could accomplish other things as well in the sequential position of M-B. For example, in \REF{ex:floyd:9} above, participant B initiated repair with an interjection after the initial M-A (see \citealt{Floyd2015} on other-initiated repair in Cha'palaa). Another function that verbal elements of responses can accomplish is to manage the temporal contingencies of the sequence. For example, in \REF{ex:floyd:22}, participant A asks participant B to take a basket, which A is holding out, but B is unable to immediately comply, so she makes it known that she intends to do the target action soon with the utterance ‘wait a little’. % (see also Kendrick, \chapref{sec:kendrick:5.3}, \sectref{sec:kendrick:} of this volume)

\transheader{ex:floyd:22}{CHSF2011\_06\_25S2\_3468149}\vspace{-1mm}
%
\begin{mdframednoverticalspace}[style=firstfoc]
\begin{transbox}{1}{a}
\begin{verbatim}
aanku ka' tsuude ((holding out basket))
aanku ka  -tu tsure-de
there grab-SR lie  -CAUS-IMP
\end{verbatim}
there get it and set it down
\end{transbox}
\end{mdframednoverticalspace}
%
\begin{mdframednoverticalspace}[style=secondfoc]
\begin{transbox}{2}{b}
\begin{verbatim}
kai keedi ((continues previous activities))
kayi   kera-di
little see -come.into.POS
\end{verbatim}
wait a little
\end{transbox}
\end{mdframednoverticalspace}
%
\emptytransbox{3}{(1.5)}
%
\begin{mdframednoverticalspace}[style=secondfoc]
\xtransbox{4}{b}{((takes and moves basket))}
\end{mdframednoverticalspace}

\normalsize
In \figref{fig:floyd:6} A is holding out the basket for B to take, but B's hands are busy (she appears to be rubbing saliva on a dry area of her arm). Often if there is no immediate nonverbal response, speakers pursue with further recruiting turns. However, because B conveys to A she will address the target action shortly, A simply waits with the basket outstretched for a couple of seconds; in \figref{fig:floyd:6}, B takes the basket and sets it down as A requested.

\begin{figure}
\includegraphics[height=.25\textheight]{figures/chapalaa-img8.png}
\caption{\label{fig:floyd:6} Participant A holds out basket for participant B and tells her to ‘get it and set it down’ (line 1).}
\end{figure}

\begin{figure}
\includegraphics[height=.25\textheight]{figures/chapalaa-img9.png}
\caption{\label{fig:floyd:7} Participant B takes the basket and sets it down (line 4), after having delayed a moment, saying ‘wait a little’ (line 2). }
\end{figure}

\subsection{Rejections and other non-fulfilling responses}\label{sec:floyd:4.3}

Most examples shown in previous sections have been fulfillments. Fulfillments by definition will fall into the nonverbal or composite categories, since they must include a practical nonverbal action or its beginning (although some nonverbal actions did not count as fulfillment). On the other hand, fully verbal responses tended to be rejections, as generally rejections require some on-record statement which, while possible to convey visually, tends to be spoken (\textit{n}=7/8). Outright rejections were rare in Cha'palaa, and only 4\% of total cases featured rejection as the response (\textit{n}=8/205), compared to 47\% which included fulfillment (\textit{n}=97/205). The fulfillment rate is even higher (55\%, \textit{n}=69/125) when considering only sequence-final cases (\tabref{tab:floyd:5}), reflecting how unsuccessful first attempts can be pursued for eventual fulfillment. Another type of spoken response to unsuccessful first attempts is be repair initiation, which accounted for 9\% (\textit{n}=19/205) of total responses (predictably, this rate was much lower in sequence-final cases). Additional options included ignoring M-A, or “other” responses like delegating to a third party (see \extref{ex:floyd:7}), giving information that A can use to resolve the problem him/herself, making a counter-proposal, or pursuing some unrelated sequence. These additional types of responses were generally more frequent than overt rejection, and so it seems that Cha'palaa speakers tend to opt for less explicit ways of avoiding the uptake of recruitments besides overt rejection. In addition, the types of rejections that were seen were not on-record refusals (a flat-out ‘no’) but tended to take other forms. In rare cases rejections could be fully nonverbal, such as in \REF{ex:floyd:23}, where a nonverbal recruiting move -- A reaching out for a slingshot -- is not responded to with a transfer of the object by B, who instead pulls the object away out of reach (\figref{fig:floyd:8}).

\transheader{ex:floyd:23}{CHSF2011\_01\_11S3\_2717590}%\vspace{-1mm}
%
\begin{mdframednoverticalspace}[style=firstfoc]
\xtransbox{1}{a}{((reaches for slingshot))}
\end{mdframednoverticalspace}
%
\begin{mdframednoverticalspace}[style=secondfoc]
\xtransbox{2}{b}{((pulls hand away)) }
\end{mdframednoverticalspace}

\begin{figure}
\includegraphics[height=.25\textheight]{figures/chapalaa-img10.png}
\caption{Participant A (in center in shirt with stripe) reaches for the slingshot (line 1) as participant B (right) pulls it back out of reach (line 2).}
\label{fig:floyd:8}
\end{figure}

When there is spoken material in a rejection, most often it can be classified as an account or explanation for why B is unwilling or unable to comply (87\% of rejections, \textit{n}=7/8). \extref{ex:floyd:24} is a good example of rejection through explanation. Participants A, B, and other friends are washing clothes together, but B is getting ready to leave while A still has more to wash. A suggests that B accompany her by doing some more washing, using a declarative format (‘you'll wash’). However, B has no more clothes to wash, so she offers this state of affairs as an explanation for why she cannot fulfill the request.

\transheader{ex:floyd:24}{CHSF\_2012\_08\_04S4\_1193345}\vspace{-1mm}
%
\begin{mdframednoverticalspace}[style=firstfoc]
\begin{transbox}{1}{a}
\begin{verbatim}
tsaaren manpipunaa manbije
tsaa-ren    man  -pi   -pu -nu -ya  man  -bije
SEM -PRECIS again-water-put-INF-FOC again-time
\end{verbatim}
so (you) will wash one more time
\end{transbox}
\end{mdframednoverticalspace}
%
\begin{mdframednoverticalspace}[style=secondfoc]
\begin{transbox}{2}{b}
\begin{verbatim}
naaketaa manpipunu nejtaa deiñu
naa-ke-tu-ya  man  -pi   -pu -nu  nejtu  -ya  de  -i     -ñu
how-do-SR-FOC again-water-put-INF because-FOC CMPL-become-DR
\end{verbatim}
 how can (I) wash since (it) is already finished?
\end{transbox}
\end{mdframednoverticalspace}
%
\begin{mdframednoverticalspace}[style=secondfoc]
\begin{transbox}{3}{~}
\begin{verbatim}
nejtaa yumaa deiñu
netju  -ya  yumaa de  -i     -ñu
because-FOC now   CMPL-become-DR
\end{verbatim}
since it is now finished
\end{transbox}
\end{mdframednoverticalspace}\bigskip

\begin{figure}
\caption{Participant A (center, foreground) asks participant B (right) if she will continue to accompany her washing clothes (line 1). Participant B rejects (line 2) and offers an explanation (line 3). }
\label{fig:floyd:9}
\includegraphics[height=.25\textheight]{figures/chapalaa-img11.png}
\end{figure}

\normalsize
The rejection turn includes a main clause calling into question B's ability to perform the target action (‘how can I wash?’) as well as a clause providing an explanation, including the word \textit{nejtu} which can be translated as ‘because’ or ‘since’ (‘since it is now finished’). Requesters generally accepted such explanations amicably and did not insist, and such rejections do not seem particularly conflictive or strongly face-threatening. Here participant B is smiling as she rejects the request (\figref{fig:floyd:9}).

One final point about rejection formats is that there are relevant connections between the formats seen in the recruiting move and the formats seen in the responding move. Particular first pair parts make relevant the provision of “type-conforming” responses \citep{Raymond2000,Raymond2003}. For example, the recruiting move in \REF{ex:floyd:25} is in the format of an interrogative clause inquiring about the existence of a target object, which is a common format for requesting objects that are not visible as discussed in \sectref{sec:floyd:3.3.3}. The response in line 2 both answers the question with the appropriate format and at the same time accomplishes rejection by appealing to an explanation citing the lack of the target object.

\transheader{ex:floyd:25}{CHSF2011\_01\_11S2\_249991}\vspace{-1mm}
%
\begin{mdframednoverticalspace}[style=firstfoc]
\begin{transbox}{1}{a}
\begin{verbatim}
lemu tsutyuu, lemu deii ((turns towards B))
lemu tsu-tyu-u lemu de  -i     -i
lime lie-NEG-Q lime CMPL-become-Q
\end{verbatim}
there are no limes, did the limes run out?
\end{transbox}
\end{mdframednoverticalspace}
%
\begin{mdframednoverticalspace}[style=secondfoc]
\begin{transbox}{2}{b}
\begin{verbatim}
lemu jutyu kaa ruku ((reaching into basket))
lemu ju-tyu kaa-ruku
lime be-NEG DIM-man
\end{verbatim}
there are no limes little husband
\end{transbox}
\end{mdframednoverticalspace}\bigskip

Since B has done due diligence here by checking the basket to see if there are any limes, she does not end up being held accountable for non-compliance with the recruitment (see also \citealt{Rossi2015b} and \chapref{sec:rossi}, \sectref{sec:rossi:4.2.2}). In many sequences which qualify as rejections speakers are able to maintain their affiliative stance, suggesting that people avoid the most fraught exchanges altogether when possible. For example, in \REF{ex:floyd:25} B rejects the recruitment with an affective, diminutive term ‘little husband’. In general, the high fulfillment rate and low rejection rate indicate an orientation to affiliation in such sequences in Cha'palaa interaction.

\section{Acknowledgment in third position}

After M-A and M-B, recruitment sequences may optionally include a move in third position by A that acknowledges the fulfillment of the recruitment. While in principle speakers of any language can make a positive assessment in this position, in some languages there are conventionalized resources that function as this type of “sequence-closing third” \citep{schegloff_sequence_2007} like the English \textit{thank you}. Cha'palaa speakers are familiar with such linguistic resources through contact with Spanish, which has the format \textit{gracias}, but when asked if there is a Cha'palaa equivalent, they end up puzzled and unable to think of anything. This illustrates how practices like saying ‘thank you’ can be highly variable across different populations (see \citealt{FloydEtAl2018}). Other research on acknowledgments has reached similar conclusions, like \citegen{apte_thank_1974} observation that while thanking is relatively unmarked in American English in most contexts, in South Asia it is very marked except in a few specific contexts.\footnote{A number of other studies discuss norms of thanking in different languages and cultures \citep{EisensteinBodman1986,Pedersen2010,Ohashi2013}. Many studies are concerned with second language learning (\citealt{Hinkel1994,Intachakra2004,ÖzdemirRezvani2010,Cui2012,FarashaiyanHua2012}, among others).} In Cha'palaa it appears that acknowledgment is not only marked, but that there is no conventionalized format for thanking in the language at all.

In Cha’palaa recruitment sequences, speakers tend to either close the sequence or continue some other conversational trajectory after M-B, where in other languages third-position acknowledgment practices are sometimes observed. The video corpus was collected in highly informal contexts, so acknowledgments might be expected to be infrequent for mundane requests among speakers who are highly familiar to each other \citep{FloydEtAl2018}. However, even in these contexts speakers of other languages showed some evidence of orientation to this kind of “face work” \citep{Goffman1955,BrownLevinson1987}, while speakers of Cha'palaa did not. Along with its preference for direct imperative formats over less direct interrogative and declarative forms, this suggests that some of the typical practices associated with politeness in English and many other languages are quite different among speakers of Cha’palaa.

\section{Social asymmetries}\label{sec:floyd:6}

Differences in social status among people in interaction are highly significant for how recruitment sequences play out, but these are more difficult to characterize analytically than, for example, the morphosyntactic formats seen in those sequences. The most reliable method for assessing how social asymmetry may be relevant for a given society in the mundane, everyday contexts considered in this study is long-term ethnographic observation and participation in the community. Based on my experience in Chachi communities over a period of about 8 years, including one full year spending two weeks per month in the field (2008--2009), it is possible to generalize that the some of the most relevant types of social asymmetries are based on a combination of age, gender, and kinship relations. Grounded on this information, each dyad was classified as symmetrical or asymmetrical. While it is true that, at least to some extent, status is locally negotiated in every interaction, in practice rights and duties around recurrent household activities remain relatively stable from instance to instance (and it seems difficult for a society to function without a relatively stable distribution of rights and duties). The dyad classifications apply only for the comparable village and household settings of the corpus, involving recurrent activities like cooking and cleaning, but this relative stability is partially contingent on context and is not always stable for every dyad in every context. However, they are stable enough in these contexts to see some trends.

The Chachi people have a system of traditional law that governs questions of morality, based around strong gender roles and normative family structure, and punishing transgressions like adultery or marriage outside the ethnicity (\citealt{Barrett1925,Altschuler1964,Floyd2010}). The male and female roles in the family are well-defined, and men and women are responsible for different tasks. Men usually participate in hunting and fishing, some agriculture, logging and canoe-making, while women are in charge of household work like cooking, cleaning and childcare, in addition to some agricultural tasks and handcrafts like basket weaving. Most of the mundane activities that made up the target actions in the sample of recruitment cases from the Cha'palaa corpus were the types of household activities that many Chachis consider to be women's responsibilities. For that reason, in most cases when men directed recruiting moves at women, typically men telling their wives to do things, such cases were classified as high-status recruitments involving lower-status individuals (A>B). Additionally, children are accountable for a number of household responsibilities such as carrying buckets of water from the river and assisting adults in their tasks. While cases involving young children were excluded from the sample as described in \sectref{sec:floyd:1.1}, adolescents usually continue to be accountable for such tasks until marriage, so cases of adult family members like parents, grandparents, or aunts and uncles initiating recruitments with adolescents and young adults were also classified as A>B. Cases with the inverse situation, when adolescents told their elders to do things or wives told their husbands to do things, were classified as low-status individuals initiating recruitment of higher-status individuals (A<B). All other cases among adults with no relevant family relationships were classified as symmetrical (A=B). This qualitative classification should be thought of as a flexible, pragmatic approach that takes into account both more stable aspects of social roles but is also attuned to situational factors for this data set.

\begin{table}
\begin{tabularx}{0.66\textwidth}{Xrr}
\lsptoprule
(A)symmetry & Count & Proportion \\
\midrule
A>B & 77 & 38\%\\
A<B & 18 & 9\%\\
A=B & 106 & 53\%\\
\lspbottomrule
\end{tabularx}
\caption{Relative frequencies of dyads by type of social (a)symmetry (\textit{n}=201/205 cases). In four cases there was not enough information to classify the dyad.}
\label{tab:floyd:6}
\end{table}

The high rate of fulfillment versus rejection in the Cha'palaa recruitments sample may be in part accounted for because individuals who may not be socially entitled to make certain requests can simply avoid them as a way to avoid potential rejection. In their model of politeness, \citet[69--74]{BrownLevinson1987} proposed this option as preferable in cases in which the potential costs are too extreme to attempt the face-threatening act. While more than half of the recruitments in the sample were between individuals classified as socially symmetrical, there were also over four times more A>B recruitments than A<B recruitments. Based on this, it appears that Cha'palaa speakers are more likely to initiate recruitment of individuals with similar or lower social status than of individuals with higher social status (see also Enfield, \chapref{sec:enfield}, \sectref{sec:enfield:6}; Baranova, \chapref{sec:baranova}, \sectref{sec:baranova:6}; Dingemanse, \chapref{sec:dingemanse}, \sectref{sec:dingemanse:5.2}).

There is some evidence that the format of the recruiting move is also sensitive to social asymmetries in that the more direct formats like imperatives and no predicate are more frequent as social entitlement increases (see also \citealt{heinemann_will_2006}, \citealt{CurlDrew2008}, and \citealt{CravenPotter2010} on the concept of entitlement). Imperatives are “direct” in a straightforward sense, but no-predicate recruiting moves can also be considered very direct in that, like imperatives, they are usually on-record and understood as explicitly requesting a target action by way of naming objects, recipients, places and so on. If we compare direct formats with ostensibly off-record formats like interrogatives and declaratives, we can consider this as a measure of directness. \tabref{tab:floyd:7} compares percentages of imperative and no-predicate recruiting moves for the three dyad types (A>B, A<B, A=B).

\begin{table}
\begin{tabularx}{0.75\textwidth}{Xcr}
\lsptoprule
(A)symmetry & \# Imperatives + no predicate & \% \\
\midrule
A>B & (49+16)/77 & 84\%\\
A<B & (12+0)/18 & 67\%\\
A=B & (73+6)/106 & 75\%\\
\lspbottomrule
\end{tabularx}
\caption{Percentage of direct (imperative + no predicate) formats in Move A by dyad type.}
\label{tab:floyd:7}
\end{table}

Across all cases, the rate of direct formats was approximately 78\%. In cases of recruitments initiated by high-status individuals with low-status individuals, this rate rises to 84\%, with a particularly high rate of no-predicate recruiting moves. However, in cases of recruitments initiated by lower-status individuals with higher-status individuals, the rate of direct formats falls to 67\%. Among equal-status individuals, the rate is between these two values, at 75\%. These results illustrate that the relative status of recruiter and recruitee can affect both the base rate of recruitments (\tabref{tab:floyd:6}) and the directness of the format selected (\tabref{tab:floyd:7}). Lower-status individuals are less likely to begin recruitment sequences, and more likely to use less direct strategies when they do. Higher-status individuals are more likely to begin recruitment sequences, and more likely to use more direct strategies. In this social context, this means that male heads of households initiate more recruitments, and women and young people are more often in the position to respond and comply, a finding that resonates with the observed social roles in the community.

\section{Discussion}

This chapter has reviewed the particular ways that speakers of Cha’palaa address the common human problem of coordinating cooperative behaviors and joint actions in light of individual concerns about being imposed on by or imposing on others. Cha’palaa speakers draw on a wide range of spoken and nonverbal resources in order to accomplish this, and calibrate the formats they use in social interaction with respect to different contingencies. In about half (\textit{n}=97/205) of Cha'palaa recruitment sequences the target action was accomplished, while in only a small percentage was there overt rejection. In another considerable portion of sequences the recruitment received no uptake but was abandoned and not pursued. In some cases the risk of overt rejection may be too high a price to pay for pursuing the target action. The overall tendency in the sample appears to be to avoid rejection when possible.

The avoidance of rejection can be interpreted both as prosocial and as a reflection of social asymmetry. Chachi culture has been resilient over the centuries in part due to strict enforcement of traditional laws, but these laws are based on rigid norms concerning social roles, and there are strong expectations about the appropriate responsibilities for daily tasks linked to gender roles and age grades. On the one hand, the high rate of successful recruitments shows that Cha’palaa speakers are highly affiliative and cooperative. On the other hand, cases of disaffiliative rejections may be low in part because people “know their place” and do not initiate recruitment sequences at all when their social rights to do so are questionable (see \citealt{Floyd2017}). Sequences in which lower-status individuals requested actions of higher-status individuals were indeed the least frequent in the sample, while higher-status individuals were not so restrained.

Many of the practices and tendencies described for Cha'palaa resemble those seen in other languages described in the literature cited above, and in this volume. However, in other ways Cha’palaa is distinct, including the grammatical forms employed (e.g. the large imperative paradigm), the types of target action requested (e.g. tasks involved in traditional basket weaving), and the cultural rationales behind the reasons and explanations offered as part of recruiting and responding moves. A lack of acknowledgment practices and a low frequency of indirect formats appears to place Cha’palaa on the low end of a cross-linguistic politeness scale. Even so, perhaps a better interpretation is that Cha’palaa speakers do their face-work by other means, as recruitments are mostly successful and face-threatening conflict is rare. Whether viewed as more prosocial or more hierarchical, the Cha’palaa recruitment system reflects deep social cohesion and interconnectedness that allows for individuals to instigate actions that go beyond their own lack of ability or willingness to act, and as such it plays an important role in Chachi society.

\section*{Abbreviations}

\begin{tabularx}{.43\textwidth}{>{\scshape}lQ}
acc & accusative/direct object\\
caus & causative\\
cmpl & completive\\
decl & declarative\\
dim & diminutive\\
ego & egophoric\\
ev & evidential\\
foc & focus/topic\\
hort & hortative imperative\\
ipfv & imperfective\\
imp & imperative\\
inf & infinitive
\end{tabularx}
\begin{tabularx}{.55\textwidth}{>{\scshape}lQ}
ingr & ingressive/immediate intention\\
loc & locative\\
neg & negation\\
pl & plural\\
pos & positional\\
prog & progressive\\
q & interrogative\\
rel.c & relative clause\\
sg & singular\\
sr & same referent\\
sem & semblative\newline  ‘like this/that'
\end{tabularx}

\section*{Acknowledgments}

This research was supported by European Research Council grant 240853 for the project \textit{Human Sociality and Systems of Language Use} (HSSLU) directed by Nick Enfield, with additional support from the Language and Cognition Department directed by Steve Levinson at the Max Planck Institute for Psycholinguistics. Special thanks goes to members of the Chachi communities of Tsejpi, Zapallo Grande, Santa María, and all of the Chachi people who participated in the recordings and in the analysis, particularly Johnny Pianchiche who transcribed much of the corpus. Thanks also to my collaborators in the Recruitments Project: Julija Baranova, Joe Blythe, Mark Dingemanse, Paul Drew, Nick Enfield, Kobin Kendrick, Giovanni Rossi, and Jörg Zinken, and to the reviewers, for their comments and discussions that improved this work. Any errors are my own. % of course

\sloppy
\printbibliography[heading=subbibliography,notkeyword=this]
\end{document}
