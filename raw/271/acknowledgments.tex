\addchap{\lsAcknowledgementTitle} 

In a recent post on his lab's website, my former advisor Alec Marantz wrote:
\begin{quote}
	However, one shouldn't forget that I wrote a very long dissertation turned book in the early 1980's that concerned the relationship between word formation and syntax. The work is titled, ``On the Nature of Grammatical Relations,'' because it is, in a sense, a paean to Relational Grammar.
\end{quote}
The dissertation-turned-book you are now reading is, in a sense, a paean to Edit Doron and Maya Arad. It is the culmination of a project which first properly started with my 2016 dissertation at NYU, so I have my dissertation committee to thank for their guidance first and foremost: Alec, Maria Gouskova, Stephanie Harves, Michael Becker and the late Edit. Edit's work in particular influenced the way I have approached the intricacies of the Hebrew verbal system, and I was looking forward to many more years of discussing it with her.

Most of this book was written while I was a post-doc at Artemis Alexiadou's \emph{Research Unit on (Experimental) Syntax and Heritage Languages} at the Humboldt-Universit\"at zu Berlin (funded by her well deserved Gottfried Wilhelm Leibniz-Preis AL 554/8-1 from the DFG). I am indebted to Artemis for establishing the ideal working environment where these ideas could be developed, and for providing a place where I could engage her and my colleagues Odelia Ahdout, Fabienne Martin, Florian Sch\"afer and Giorgos Spathas in many informal conversations. Odelia in particular helped sharpen many specific aspects of the analysis and corrected some of the empirical claims. I also thank the local PhD students, post-docs and research assistants for the friendly atmosphere and useful discussions.

Quite a few people helped get the book itself over the finish line. At Language Science Press, I am grateful to Susi Wurmbrand for her support of the project, Elena Anagnostopoulou for her work as handling editor, and an anonymous reviewer for highly constructive comments. On the production side, the proofreaders deserve a special word of thanks for volunteering their time, Felix Kopecky fixed more problems than I realized I'd caused, and Sebastian Nordhoff made sure the whole operation ran remarkably smoothly. My thanks also to Yifei Hu at the University of Edinburgh for editorial assistance preparing the manuscript for production.

The research in this book led to a number of individual papers, some co-au\-thored, and all owing a debt to feedback from different people at different stages. Thanks to Odelia Ahdout, Matthew Tucker and Vera Zu for joint work which has ended up in this book in one way or another. Earlier parts of this work have appeared in \emph{Glossa}, \emph{Morphology}, \emph{Natural Language \& Linguistic Theory} and \emph{The Linguistic Review}; I thank the reviewers and editors there for their work. Thanks especially to participants at the \emph{Parameters Workshop in Honour of Lisa Travis} for providing the impetus to get this thing done.

I was very lucky to benefit from written comments on previous versions of the manuscript. Thanks to Yining Nie for insightful comments on a previous draft; and to Jim Wood and the students in his 2019 Yale seminar on Argument Structure and Morphology for their comments, including important discussion of alternative analyses.

For some suggestions that didn't make it in, thanks to Daniel Harbour for suggesting the titles ``The Valence of Voice'' and ``Hebrew D-Voicing'' and to Dylan Bumford for suggesting the title ``Modern Morphophonemics of Hebrew''. Thanks also to Tricia Irwin, Neil Myler, James Whang and Vera Zu (just because).

This book also symbolizes for me the end of three and a half wonderful years in Berlin. Apart from everyone above who made our time there so enjoyable, my warmest thanks to our adoptive families: Olia \& Luca, and Florian \& Fabienne.

Finally, thanks to A.~and T.~for struggling with Hebrew morphology in their own way.

\bigskip

	\hfill Edinburgh, February 2020
