\documentclass[output=paper]{langsci/langscibook} 
\ChapterDOI{10.5281/zenodo.3402066}

% Chapter 7

\title{Case and agreement in Mehweb}

\author{Dmitry Ganenkov\affiliation{University of Bamberg;
National Research University Higher School of Economics}}

\abstract{The chapter deals with patterns of case marking and agreement in Mehweb. On the basis of morphosyntactic coding and anaphoric binding, a system of five valency classes is described. The chapter covers basic monoclausal structures with verbs of the five valency classes as well as their interaction with several specific constructions, such as reciprocal, causative, and biabsolutive.

\emph{Keywords}: case, personal agreement, gender, transitivity,
experiential verbs, dative verbs, subject, reported speech,
biabsolutive construction.}

\begin{document}
\maketitle

{\def\thefootnote{}
\footnotetext{Financial support from the Basic Research Program of 
the National Research University Higher School of Economics (Moscow) is gratefully acknowledged.}

}


% \exewidth{(123)}

\let\exfont\rm
\let\eachwordone\rm

The present chapter deals with the morphosyntax of argument expression
in Mehweb. In many respects, Mehweb is a fairly typical representative
of the Dargwa branch of Nakh-Daghestanian, and of the Nakh-Daghestanian family in general.
In certain respects, however, the language displays rare features only
attested in a few other languages of the family. Three linguistic
phenomena – argument case marking, gender agreement, and person
agreement – are the focus of this chapter. The three coding properties
are interrelated in many ways and together constitute major surface
evidence about grammatical functions (including subjecthood), supported
by other diagnostics, such as the binding of reflexive and reciprocal
pronouns. They also generally determine how the Mehweb verbal lexicon
can be broken down into verb (valency) classes. The notion of \emph{core
argument} will be key to capturing the system of valency classes. In
this chapter, I define \emph{core argument} as a clausal constituent
expressed by a noun phrase that is able to determine at least one type
of verbal agreement, either gender or person, or both. Depending on the
number of core arguments and their morphosyntactic behavior with respect
to coding properties, the Mehweb verbal lexicon is divided into the
following valency classes:

{\let\raggedright\relax
\ea \label{ex:7:1} % (1)
Mehweb valency classes
\ea  % a.
\emph{Intransitive verbs} have a single core argument in the
absolutive that triggers both person and gender agreement.

\ex % b.
\emph{Transitive verbs} feature two core arguments. One core
argument, the subject, is in the ergative case and triggers person
agreement on the finite verb. The other core argument, the direct
object, is in the absolutive case and determines agreement features in
the gender agreement slot.

\ex % c.
\emph{Locative subject verbs} are also bivalent verbs with two core
arguments. Instead of an ergative argument, as with transitive verbs,
they possess a core experiencer argument in the spatial case called
\emph{inter-lative}, see \citet{chechuro2019} [this volume] for details of the
nominal paradigm. Like the ergative subject of a transitive verb, the
inter-lative (henceforth, locative) subject of a locative-subject verb
also triggers person agreement.

\ex % d.
\emph{Dative subject verbs} have one core argument in the
absolutive, which triggers gender agreement only. No argument of a dative subject
verb is able to determine person agreement on its own.

\ex % e.
The \emph{inter-elative subject verb} \emph{buhes} `manage, be able'
features one core argument in the inter-elative case which optionally
triggers person agreement but cannot control gender agreement.
\z
\z
}

The rest of this chapter provides empirical evidence about the behavior of
various types of verbal arguments that motivates the above
classification. \sectref{Case-marking-and-structural-prominence}
describes patterns of case marking and
provides evidence from reflexive binding about the relative structural
prominence of verbal arguments. \sectref{Verbal-gender-agreement} and \sectref{Verbal-person-agreement} deal with rules of
gender and person agreement. \sectref{Reciprocals} presents an overview of case
marking and agreement in reciprocal constructions. \sectref{Causative-construction} discusses
causative constructions. \sectref{Biabsolutive-construction} describes basic properties of the
biabsolutive construction. The conclusion briefly summarizes the main
issues covered in the paper.

\clearpage

% 1.
\section{Case marking and structural prominence}\label{Case-marking-and-structural-prominence}

Mehweb is a morphologically ergative language where the sole argument
(S) of intransitive verbs is grouped together with the direct object (P)
of transitive verbs with regard to morphological case marking, but
separately from the subject (A) of transitive verbs: S and P arguments
are in the unmarked absolutive case, while A arguments bear the ergative
case.

\ea \label{ex:7:2} % (2)
\gll \emph{ʡali} \emph{w-ak'-ib.}\\
Ali(\textsc{abs}) \textsc{m}-come:\textsc{pf}-\textsc{aor}\\
\glt `Ali came.'

\ex \label{ex:7:3} % (3)
\gll \emph{sinka-ni} \emph{ʡali} \emph{uc-ib}.\\
bear-\textsc{erg} Ali(\textsc{abs}) (\textsc{m})catch:\textsc{pf}-\textsc{aor}\\
\glt `A bear seized Ali.'

\ex \label{ex:7:4} % (4)
\gll \emph{ʡali-ini} \emph{sinka} \emph{b-aˤbʡ-ib.}\\
Ali-\textsc{erg} bear(\textsc{abs}) \textsc{n}-kill:\textsc{pf}-\textsc{aor}\\
\glt `Ali killed a bear.'
\z

In (\ref{ex:7:2}), the DP \emph{ʡali} `Ali (a man's name)' is in its unmarked form
and functions as the core argument of the intransitive verb
\emph{bak'es} `come'. In (\ref{ex:7:3}), the same form is used to express the
direct object (patient) of the transitive verb \emph{buces} `catch,
seize'. In (\ref{ex:7:4}), however, the DP functions as the subject of the
transitive verb \emph{baˤbʡes} `kill' and thus must be in the ergative
case.

An absolutive case DP is present in almost every Mehweb clause. In
intransitive clauses, the absolutive argument is the highest one from
the structural point of view, as seen from the fact that it can bind
reflexive pronouns in any other position, but cannot be bound itself by
any other argument.\footnote{In this paper, to diagnose structural
  prominence, I employ sentences with \emph{wh}-pronouns serving as
  antecedents of reflexive pronouns. This is necessary in order to
  exclude the possibility of the co-reference relation between the
  antecedent and the reflexive \citep{reinhart1981}. Co-reference is normally
  available with referential antecedents and works on pragmatic rather
  than strictly syntactic grounds in Mehweb. In particular, the
  ``antecedent'' can appear in a structurally lower position in
  co-reference, as in (i), which is not a grammatical option under
  semantic binding by non-referential antecedents (quantified,
  \emph{wh}-pronouns), cf.\ (\ref{ex:7:5}b).

  \begin{exe}
    \exi{(i)}
  \gll \emph{sune-la-l} \emph{urši} \emph{madina-če} \emph{ħule} \emph{w-iz-ur.}\\
  self-\textsc{gen}-\textsc{emph} son(\textsc{abs}) Madina-\textsc{super(lat)} look \textsc{m}-\textsc{lv}:\textsc{pf}-\textsc{aor}\\
  \glt  `Her\textsubscript{i} son looked at Madina\textsubscript{i} (a woman's
  name).'
\end{exe}

\removelastskip

\vspace{-\baselineskip}

%   The ungrammaticality of (b) examples in (\ref{ex:7:5}), (\ref{ex:7:7}), (\ref{ex:7:8}) also attests to
%   the relevance of the weak crossover effect in the syntax of Mehweb.
}
Example (\ref{ex:7:5}) shows the intransitive verb \emph{ħule \textup{\textsc{cl}}-izes} `look' with an
oblique  argument which is diagnosed as structurally less
prominent than the clause-mate absolutive argument.

\pagebreak

\ea \label{ex:7:5} % (5)
\emph{ħule \textup{\textsc{cl}}-izes} `look': absolutive \textgreater{} super-lative


\ea % a.
\gll \emph{čija} \emph{ħule} \emph{d-iz-ur-a} \emph{sune-la-l} \emph{urši-li-če?}\\
who(\textsc{abs}) look \textsc{f1-lv}:\textsc{pf}-\textsc{aor}-\textsc{q} self-\textsc{gen}-\textsc{emph} son-\textsc{obl}-\textsc{super(lat)}\\
\glt `Who\textsubscript{i} looked at her\textsubscript{i} son?'

\ex % b.
\gll \emph{*sune-la-l} \emph{urši} \emph{hi-če} \emph{ħule} \emph{w-iz-ur-a?}\\
self-\textsc{gen}-\textsc{emph} son(\textsc{abs}) who-\textsc{super(lat)} look \textsc{m}-\textsc{lv}:\textsc{pf}-\textsc{aor}-\textsc{q}\\
\glt `Who\textsubscript{i} did her\textsubscript{i} son look at?'
\z
\z

The linear order plays no role in acceptability of the ``reversed''
anaphoric binding, thus attesting to the relevance of weak crossover
effects in reflexive binding, as shown in example (\ref{ex:7:5}c).

\addtocounter{equation}{-1}
\ea %  (5)
\emph{ħule \textup{\textsc{cl}}-izes} `look': absolutive \textgreater{} super-lative
\begin{xlist}
\exi{c.}
\gll \emph{*hi-če} \emph{sune-la-l} \emph{urši} \emph{ħule} \emph{w-iz-ur-a?}\\
who-\textsc{super(lat)} self-\textsc{gen}-\textsc{emph} son(\textsc{abs}) look \textsc{m}-\textsc{lv}:\textsc{pf}-\textsc{aor}-\textsc{q}\\
\glt `Who\textsubscript{i} did her\textsubscript{i} son look at?'
\end{xlist}
\z

The absolutive argument is not restricted to expressing any particular
thematic role. It can denote an agentive participant, a patientive
participant, or an experiencer. Unergative and unaccusative verbs in
Mehweb are not distinguished by case marking. Some intransitive
verbs are listed in (\ref{ex:7:6}).

\ea \label{ex:7:6} % (6)
Intransitive verbs

\emph{a꞊izes} `stand up', \emph{arces} `fly', \emph{aqas} `raise,
climb', \emph{꞊alħʷes} `wake up', \emph{꞊ebk'es} `die', \emph{꞊erʔʷes}
`become dry', \emph{꞊ises} `cry', \emph{꞊usaʔʷas} `fall asleep',
\emph{꞊urdes} `become worn', \emph{꞊ušes} `die out (of fire)',
\emph{꞊uzes} `work', \emph{kalʔes} `remain', \emph{uruχ ꞊aˤqes} `get
afraid'
\z

Two-place verbs are those verbs that mark their structurally highest
argument with a morphological case other than the absolutive. As suggested
in (\ref{ex:7:1}) above, depending on the particular case of the highest argument,
two-place verbs fall into three classes: transitive verbs with ergative
subjects, locative subject verbs with inter-lative subjects, and dative
subject verbs with dative subjects.

With transitive verbs, the ergative-marked argument is structurally the
most prominent, as evidenced by its ability to bind a reflexive pronoun
in any other position in the clause, including the absolutive argument,
as in (\ref{ex:7:7}a-\ref{ex:7:8}a). The reverse binding of the ergative reflexive by an
oblique or absolutive argument is impossible, as shown in (\ref{ex:7:7}b) and (\ref{ex:7:8}b).

\ea \label{ex:7:7} % (7)
\emph{haraq'e} \emph{ihʷes} `deceive': ergative \textgreater{} absolutive
\ea % a.
\gll \emph{hinija} \emph{haraq'e} \emph{ihʷ-es-a} \emph{sune-la-l} \emph{urši?}\\
who(\textsc{erg}) forward throw:\textsc{pf}-\textsc{fut}-\textsc{q} self-\textsc{gen}-\textsc{emph} son(\textsc{abs})\\
\glt `Who\textsubscript{i} will deceive his\textsubscript{i} son?'

\ex % b.
\gll \emph{*sune-la-l} \emph{urši-li-ni} \emph{čija} \emph{haraq'e} \emph{ihʷ-es-a?}\\
self-\textsc{gen}-\textsc{emph} son-\textsc{obl}-\textsc{erg} who(\textsc{abs}) forward throw:\textsc{pf}-\textsc{fut}-\textsc{q}\\
\glt `Who\textsubscript{i} will his\textsubscript{i} son deceive?'
\z

\ex \label{ex:7:8} % (8)
\emph{kumak} \emph{baq'es} `help': ergative \textgreater{} dative
\ea % a.
\gll \emph{hinija} \emph{sune-la-l} \emph{urši-li-s} \emph{kumak} \emph{b-aq'-ib-a?}\\
who(\textsc{erg}) self-\textsc{gen}-\textsc{emph} son-\textsc{obl}-\textsc{dat} help(\textsc{abs}) \textsc{n}-do:\textsc{pf}-\textsc{aor}-\textsc{q}\\
\glt `Who\textsubscript{i} helped his\textsubscript{i} son?'

\ex % b.
\gll \emph{*sune-la-l} \emph{urši-li-ni} \emph{hi-sa} \emph{kumak} \emph{b-aq'-ib-a?}\\
self-\textsc{gen}-\textsc{emph} son-\textsc{obl}-\textsc{erg} who-\textsc{dat} help(\textsc{abs}) \textsc{n}-do:\textsc{pf}-\textsc{aor}-\textsc{q}\\
\glt `Who\textsubscript{i} did his\textsubscript{i} son help?'
\z
\z

Apart from agents, the ergative argument of a transitive verb can also
denote a non-agentive causer (see also \citealt{chechuro2019} [this volume] on the
instrumental function of the ergative).

\ea % (9)
\gll \emph{ʒab-li-ni} \emph{mura} \emph{d-aˤʜʷ-aˤq-ib.}\\
rain-\textsc{obl}-\textsc{erg} hay(\textsc{abs}) \textsc{npl}-become~wet:\textsc{pf}-\textsc{caus}-\textsc{aor}\\
\glt `The rain made the hay wet.'

\ex % (10)
\gll \emph{ʁʷaˤl-li-ni} \emph{ʁut'-be} \emph{šiš} \emph{d-uk'-aq-uwe} \emph{le-r.}\\
wind-\textsc{obl}-\textsc{erg} tree-\textsc{pl}(\textsc{abs}) move \textsc{npl}-\textsc{lv}:\textsc{ipf}-\textsc{caus}-\textsc{cvb.ipfv} \textsc{aux}-\textsc{npl}\\
\glt `The wind is shaking the trees.'

\ex % (11)
\gll \emph{c'a-li-ni} \emph{qul-le} \emph{ig-uwe} \emph{le-r.}\\
fire-\textsc{obl}-\textsc{erg} house-\textsc{pl}(\textsc{abs}) burn:\textsc{ipf}-\textsc{cvb.ipfv} \textsc{aux}-\textsc{npl}\\
\glt `Fire is burning the houses.'
\z

Ergative case is thus tightly associated with agentive and causative
semantics and is not employed to express participants with other
thematic roles. Almost every transitive clause contains an absolutive
argument. Exceptions are very few and can be summarized as follows.

With verbs of contact like \textsc{cl}\emph{-aʔaqas} `hit (an inanimate object)' and
\textsc{cl}\emph{-aˤqas} `hit (an animal)', the absolutive argument expresses the
instrument. Generally, instruments are never obligatory and can be
freely omitted from overt expression. The absolutive argument in the
instrumental function thus often does not appear overtly.

\ea % (12)
\gll \emph{it-ini} \emph{q'ʷaˤj-če} \emph{(derxa)} \emph{b-aˤq-ib.}\\
{this}-\textsc{erg} cow+\textsc{obl}-\textsc{super(lat)} stick(\textsc{abs}) \textsc{n}-hit:\textsc{pf}-\textsc{aor}\\
\glt `She hit the cow (with a stick).'

\ex % (13)
\gll \emph{ʡali-ni} \emph{(χunk')} \emph{unza-li-ze} \emph{b-aʔ-aq-ib.}\\
Ali-\textsc{erg} fist(\textsc{abs}) door-\textsc{obl}-\textsc{inter(lat)} \textsc{n}-hit-\textsc{lv}:\textsc{pf}-\textsc{aor}\\
\glt `Ali hit the door with his fist (lit. his fist into the door).'
\z

The absolutive argument, when omitted from overt expression, is arguably
still present in the sentence, as evidenced by the possibility of
non-default (neuter plural) gender agreement.

\ea \label{ex:7:14} % (14)
\gll \emph{ʡali-ni} \emph{unza-li-ze} \emph{d-aʔ-aq-ib.}\\
Ali-\textsc{erg} door-\textsc{obl}-\textsc{inter(lat)} \textsc{npl}-hit-\textsc{lv}:\textsc{pf}-\textsc{aor}\\
\glt `Ali hit the door (with his fists).'
\z

In (\ref{ex:7:14}), the plural gender marking on the verb reflects the plurality of
the instrumental DP in the absolutive.

With some transitive verbs of speech and thought, the absolutive
argument denotes the content of speech/thought.

\ea % (15)
\gll \emph{ħu-ni} \emph{sija} \emph{i-ra?}\\
you.sg-\textsc{erg} what(\textsc{abs}) say:\textsc{pf}+\textsc{aor}-\textsc{ego}+\textsc{q}\\
\glt `What did you say?'

\ex % (16)
\gll \emph{nu-ni} \emph{b-urh-iša} \emph{ca} \emph{χabar.}\\
I-\textsc{erg} \textsc{n}-tell:\textsc{pf}-\textsc{fut}.\textsc{ego} one story(\textsc{abs})\\
\glt `I will tell (you) one story.'
\z

Likewise, many such verbs alternatively subcategorize for either an
absolutive DP argument or a clausal argument. In the latter case, 
no absolutive argument is present in the clause.

\ea % (17)
\gll \emph{rasuj-ni} \emph{abzulaj-ze} \emph{b-urh-ib} \emph{murad-li} \emph{mašina} \emph{as-ib} \emph{ile.}\\
Rasul+\textsc{obl}-\textsc{erg} all+\textsc{obl}-\textsc{inter(lat)} \textsc{n}-tell:\textsc{pf}-\textsc{aor} Murad-\textsc{erg} car(\textsc{abs}) take:\textsc{pf}-\textsc{aor} \textsc{comp}\\
\glt `Rasul told everyone that Murat had bought a car.'
\z

With some complex transitive verbs, a nominal constituent in the
unmarked form functions as a non-verbal component.

\ea % (18)
\gll \emph{mallarasbadij-ni} \emph{žawab} \emph{b-aq'-i-le} \emph{le-b.}\\
Molla~Nasreddin.\textsc{obl}-\textsc{erg} answer \textsc{n}-do:\textsc{pf}-\textsc{aor}-\textsc{cvb} \textsc{aux}-\textsc{n}\\
\glt `Molla Nasreddin answered.' (lit.\ `made an answer')

\ex % (19)
\gll \emph{nu-ni} \emph{di-la꞊l} \emph{urši-li-s} \emph{kumak} \emph{b-aq'-i-ra.}\\
I-\textsc{erg} I.\textsc{obl}-\textsc{gen}꞊\textsc{emph} son-\textsc{obl}-\textsc{dat} help(\textsc{abs}) \textsc{n}-do:\textsc{pf}-\textsc{aor}-\textsc{ego}\\
\glt `I helped my son.'
\z

The morphosyntactic status of such unmarked nominals is not clear. They
can be analyzed either as absolutive-cased DPs or as
(pseudo)-incorporated caseless NPs. More work is needed to decide on
this question.

Some verbs are P-labile, that is, have both a transitive use and an
intransitive use where the subject of the intransitive use corresponds to
(i.e.\ expresses the same participant as) the direct object of the
transitive use. The verbal lexicon has not been systematically studied
for P-lability. The transitive use with an unspecified (omitted) subject
of a P-labile verb and the intransitive use of the same verb are
distinguished by: (i) different imperative marking (see \citealt{daniel2019} [this
volume]), and (ii) the ability of the absolutive argument to trigger
person agreement on the finite verb (see \sectref{intransitive-transitive-and-locative-subject-verbs-in-synthetic-indicative-forms} below).

Two other classes of two-place verbs are locative subject verbs and
dative subject verbs. The locative subject class includes the verbs
\emph{arʁes} `hear, understand', \emph{bahes} `know', \emph{barges}
`find', \emph{gʷes} `see'.

\ea % (20)
\gll \emph{ʡali-ze} \emph{it} \emph{dehʷ} \emph{arʁ-ib.}\\
Ali-\textsc{inter(lat)} {this} word(\textsc{abs}) hear/understand:\textsc{pf}-\textsc{aor}\\
\glt `Ali heard/understood this word.'

\ex\label{ex:7-21}
\gll \emph{rasuj-ze} \emph{ʡali} \emph{w-alh-an.}\\
Rasul+\textsc{obl}-\textsc{inter(lat)} Ali(\textsc{abs}) \textsc{m}-know:\textsc{ipf}-\textsc{hab}\\
\glt `Rasul knows Ali.'

\ex % (22)
\gll \emph{ʡali-ze} \emph{arc} \emph{d-arg-ib.}\\
Ali-\textsc{inter(lat)} money(\textsc{abs}) \textsc{npl}-find:\textsc{pf}-\textsc{aor}\\
\glt `Ali found money.'

\ex % (23)
\gll \emph{rasuj-ze} \emph{ʡali} \emph{g-ub.}\\
Rasul+\textsc{obl}-\textsc{inter(lat)} Ali(\textsc{abs}) see:\textsc{pf}-\textsc{aor}\\
\glt `Rasul saw Ali.'
\z

The dative subject class includes the verbs \emph{biges} `want, love',
\emph{bikes} `happen', \emph{eba buhes} `get bored', \emph{určeb leb}
`remember', \emph{urče bak'as} `recall', \emph{urče bikes} `recall'.

\ea % (24)
\gll \emph{madina-s} \emph{rasul} \emph{w-ig-an.}\\
Madina-\textsc{dat} Rasul(\textsc{abs}) \textsc{m}-love:\textsc{ipf}-\textsc{hab}\\
\glt `Madina loves Rasul.'

\ex % (25)
\gll \emph{ʡali-s} \emph{ʡaˤχ-il} \emph{q'immat} \emph{b-ik-ib.}\\
Ali-\textsc{dat} good-\textsc{atr} grade(\textsc{abs}) \textsc{n}-happen:\textsc{pf}-\textsc{aor}\\
\glt `Ali got a good grade.'

\ex % (26)
\gll \emph{madina-s} \emph{rasul} \emph{eba} \emph{uh-ub.}\\
Madina-\textsc{dat} Rasul(\textsc{abs}) bore (\textsc{m})become:\textsc{pf}-\textsc{aor}\\
\glt `Madina got bored with Rasul.'

\ex % (27)
\gll \emph{madina-s} \emph{ʡali} \emph{urče-w} \emph{le-w.}\\
Madina-\textsc{dat} Ali(\textsc{abs}) in.heart-\textsc{m(ess)} {be}-\textsc{m}\\
\glt `Madina remembers Ali.'

\ex % (28)
\gll \emph{rasuj-s} \emph{hel} \emph{dehʷ} \emph{urče} \emph{b-ak'-ib.}\\
Rasul+\textsc{obl}-\textsc{dat} {this} word(\textsc{abs}) in.heart(\textsc{lat}) \textsc{n}-come:\textsc{pf}-\textsc{aor}\\
\glt `Rasul recalled that word.'
\z

The verb \emph{qumartes} `forget' alternatively allows
for either locative or dative case marking on its subject.

\ea \label{ex:7-29}
\gll \{\emph{ʡali-ze} / \emph{ʡali-s}\} \emph{deč'} \emph{qum-art-ur.}\\
Ali-\textsc{inter(lat)} / Ali-\textsc{dat} song(\textsc{abs}) forget-\textsc{lv}:\textsc{pf}-\textsc{aor}\\
\glt `Ali forgot the song.'
\z

The inter-lative (locative) and dative arguments are the highest
arguments in their respective clauses. Again, this is evidenced by the
ability of the locative/dative argument to bind any other argument
(including the absolutive), while the reverse binding pattern is
ungrammatical.

\ea % (30)
\emph{gʷes} `see': inter-lative \textgreater{} absolutive

\ea % a.
\gll \emph{hi-ze} \emph{g-ub-a} \emph{sune-la-l} \emph{urši?}\\
who-\textsc{inter(lat)} see:\textsc{pf}-\textsc{aor}-\textsc{q} self-\textsc{gen}-\textsc{emph} son(\textsc{abs})\\
\glt `Who\textsubscript{i} saw her\textsubscript{i} son?'

\ex % b.
\gll \emph{*sune-la-l} \emph{urši-li-ze} \emph{čija} \emph{g-ub-a?}\\
self-\textsc{gen}-\textsc{emph} son-\textsc{obl}-\textsc{inter(lat)} who(\textsc{abs}) see:\textsc{pf}-\textsc{aor}-\textsc{q}\\
\glt `Who\textsubscript{i} did her\textsubscript{i} son see?'
\z

\ex % (31)
\emph{biges} `love': dative \textgreater{} absolutive

\ea % a.
\gll \emph{hi-sa} \emph{ħa-d-ig-ul} \emph{sune-la-l} \emph{abaj?}\\
who-\textsc{dat} \textsc{neg}-\textsc{f1}-love:\textsc{ipf}-\textsc{ptcp} self-\textsc{gen}-\textsc{emph} mother(\textsc{abs})\\
\glt `Who\textsubscript{i} does not love his\textsubscript{i} mother?'

\ex % b.
\gll \emph{*sune-la-l} \emph{abaj-s} \emph{čija} \emph{ħa-d-ig-ul?}\\
self-\textsc{gen}-\textsc{emph} mother-\textsc{dat} who(\textsc{abs}) \textsc{neg}-\textsc{f1}-love:\textsc{ipf}-\textsc{ptcp}\\
\glt `Who\textsubscript{i} does his\textsubscript{i} mother not love?'
\z
\z

Again, while the absolutive argument generally must be present in a clause with a
locative or dative subject verb, it may be absent in case the
corresponding semantic argument is expressed by another constituent.
Most locative and dative subject verbs allow a clausal complement
instead of the absolutive argument.

\ea % (32)
\emph{arʁes} `hear' with finite complement

\gll \emph{ʡali-ze} \emph{arʁ-ib} [\emph{abaj} \emph{iz-uwe} \emph{le-r} \emph{ile}].\\
Ali-\textsc{inter(lat)} hear:\textsc{pf}-\textsc{aor} mother(\textsc{abs}) be.sick:\textsc{ipf}-\textsc{cvb.ipfv} \textsc{aux}-\textsc{f} \textsc{comp}\\ 
\glt `Ali heard that mother was sick.'

\ex % (33)
\emph{biges} `want' with infinitival complement

\gll \emph{rasuj-s} \emph{dig-uwe} \emph{le-b} [\emph{anži-li} \emph{uˤq'-es}].\\
Rasul+\textsc{obl}-\textsc{dat} want:\textsc{ipf}-\textsc{cvb.ipfv} \textsc{aux}-\textsc{n} Makhachkala-\textsc{in(lat)} (\textsc{m})go:\textsc{pf}-\textsc{inf}\\
\glt `Rasul wants to go to Makhachkala.'\\

\ex % (34)
\emph{bikes} `happen' with a finite complement

\gll \emph{abzulaj-s} \emph{b-ik-ib} [\emph{ʡali} \emph{w-ebk'-i-le} \emph{ile}].\\
everyone+\textsc{obl}-\textsc{dat} \textsc{n}-happen:\textsc{pf}-\textsc{aor} Ali(\textsc{abs}) \textsc{m}-die:\textsc{pf}-\textsc{aor}-\textsc{cvb} \textsc{comp}\\
\glt `Everyone thought (lit.\ it occurred to everyone) that Ali was dead.'
\z

Finally, the verb \emph{buhes} `manage, be able' is the only verb in
Mehweb that licenses a core argument in the inter-elative case.

\ea \label{ex:7:35} % (35)
\gll \emph{rasuj-ze-la} \emph{ajz-es} \emph{ħa-b-urh-an.}\\
Rasul+\textsc{obl}-\textsc{inter}-\textsc{el} (\textsc{m})rise:\textsc{pf}-\textsc{inf} \textsc{neg}-\textsc{n}-manage:\textsc{ipf}-\textsc{hab}\\
\glt `Rasul cannot stand up.'

\ex \label{ex:7:36} % (36)
\gll \emph{rasuj-ze-la} \emph{ħa-b-uh-ub} \emph{ʁarʁa} \emph{aq} \emph{b-aq'-as.}\\
Rasul+\textsc{obl}-\textsc{inter}-\textsc{el} \textsc{neg}-\textsc{n}-manage:\textsc{pf}-\textsc{aor} stone(\textsc{abs}) up \textsc{n}-do:\textsc{pf}-\textsc{inf}\\
\glt `Rasul did not manage to lift the stone.'
\z

To summarize, Mehweb has five verb classes depending on the case of
the structurally highest argument: (i) intransitive verbs with
absolutive subject, (ii) transitive verbs with ergative subject, (iii)
locative subject verbs with inter-lative subject, and (iv) dative
subject verbs with dative subject, and (v) one inter-elative subject
verb \emph{buhes} `manage, be able'. The argument structure of all verbs
includes an absolutive argument. As will be shown below, the subject and
the absolutive argument (when they are different) play a special role in
gender and person agreement, and thus are called \emph{core} \emph{arguments}.
All other arguments are \emph{oblique}.

% 2.
\section{Verbal gender agreement}\label{Verbal-gender-agreement}

Two morphological slots for gender agreement are potentially available
in the Mehweb clause. One is the prefixal gender agreement marker on
lexical verbs. Every verbal stem is specified for whether it hosts the
prefixal gender agreement slot. Most verbs are specified to host this
agreement marker in their perfective stems. In imperfective stems, the
slot is often absent. For more on agreement morphology and its relation
to stems, see \citet{daniel2019} [this volume].

\ea \label{ex:7:37} % (37)
\ea % a.
\gll \emph{urši-li-ni} \emph{kaʁar-t} \emph{d-elk'-un.}\\
boy-\textsc{obl}-\textsc{erg} letter-\textsc{pl}(\textsc{abs}) \textsc{npl}-write:\textsc{pf}-\textsc{aor}\\
\glt `The boy wrote letters.'

\ex % b.
\gll \emph{urši-li-ni} \emph{kaʁar-t} \emph{luk'-an.}\\
boy-\textsc{obl}-\textsc{erg} letter-\textsc{pl}(\textsc{abs}) write:\textsc{ipf}-\textsc{hab}\\
\glt `The boy writes letters (every day).'
\z
\z

The verb `write' has a prefixal slot for gender agreement in its
perfective stem, as shown in (\ref{ex:7:37}a), but lacks any such slot in its
imperfective stem, as in (\ref{ex:7:37}b). If a stem features gender agreement, it
is obligatory in any verbal form based on this stem, be it finite or
non-finite.

The other morphological slot for gender agreement in the verbal complex
is the suffix on the auxiliary in periphrastic verbal forms.

\ea % (38)
\gll \emph{urši-li-ni} \emph{kaʁar-t} \emph{luk'-uwe} \emph{le-r.}\\
boy-\textsc{obl}-\textsc{erg} letter-\textsc{pl}(\textsc{abs}) write:\textsc{ipf}-\textsc{cvb.ipfv} \textsc{aux}-\textsc{npl}\\
\glt `The boy is writing letters.'
\z
 
The rule of thumb for gender agreement in monoclausal structures is to
agree with the clause-mate absolutive argument. With regard to gender
agreement on lexical verbs, this means that agreement is always with the
absolutive subject of an intransitive verb or with the absolutive direct
object of other verb classes,
% (transitive, locative subject, and dative subject),
as shown below.

\ea % (39)
\ea % a.
\gll \emph{urši} \emph{w-ak'-ib.}\\
boy(\textsc{abs}) \textsc{m}-come:\textsc{pf}-\textsc{aor}\\
\glt `The boy came.'

\ex % b.
\gll \emph{dursi} \emph{d-ak'-ib.}\\
girl(\textsc{abs}) \textsc{f1}-come:\textsc{pf}-\textsc{aor}\\
\glt `The girl came.'
\z

\ex % (40)
\ea % a.
\gll \emph{ʡali-ini} \emph{sinka} \emph{b-aˤbʡ-ib}.\\
Ali-\textsc{erg} bear(\textsc{abs}) \textsc{n}-kill:\textsc{pf}-\textsc{aor}\\
\glt `Ali killed a bear.'

\ex % b.
\gll \emph{sinka-li} \emph{ʡali} \emph{w-aˤbʡ-ib.}\\
bear-\textsc{erg} Ali(\textsc{abs}) \textsc{m}-kill:\textsc{pf}-\textsc{aor}\\
\glt `A bear killed Ali.'
\z

\ex % (41)
\ea % a.
\gll \emph{abaj-ze} \emph{urši} \emph{w-arg-ib.}\\
mother-\textsc{inter(lat)} boy(\textsc{abs}) \textsc{m}-find:\textsc{pf}-\textsc{aor}\\
\glt `Mother found her son.'

\ex % b.
\gll \emph{adaj-ze} \emph{dursi} \emph{d-arg-ib.}\\
father-\textsc{inter(lat)} girl(\textsc{abs}) \textsc{f1}-find:\textsc{pf}-\textsc{aor}\\
\glt `Father found his daughter.'
\z

\ex % (42)
\ea % a.
\gll \emph{madina-s} \emph{ʡali} \emph{w-ig-ib.}\\
Madina-\textsc{dat} Ali(\textsc{abs}) \textsc{m}-love:\textsc{ipf}-\textsc{ipft}\\
\glt `Madina loved Ali.'

\ex % b.
\gll \emph{ʡali-s} \emph{madina} \emph{d-ig-ib.}\\
Ali-\textsc{dat} Madina(\textsc{abs}) \textsc{f1}-love:\textsc{ipf}-\textsc{ipft}\\
\glt `Ali loved Madina.'
\z
\z

If a clause lacks an absolutive argument, as observed with some types of
formally transitive verbs, gender agreement on the lexical verb appears
as the default singular neuter agreement marker \emph{b}-. This is also
observed with intransitive impersonal predicates. See examples in
\sectref{Case-marking-and-structural-prominence} above.

The verb \emph{buhes} `manage, be able' subcategorizes for an
inter-elative subject and an infinitival complement and thus does not
have an absolutive argument. This verb, therefore, invariably appears
with the default (singular neuter) marker \emph{b-}, as in examples (\ref{ex:7:35})
and (\ref{ex:7:36}) above.

The second morphological slot for gender agreement appears on the auxiliary
within periphrastic verbal forms like Present and Past Progressive,
Present and Past Resultative. This slot cross-references the
gender-number features of the highest absolutive argument or shows the
default (neuter singular agreement) in clauses with no absolutive
argument.

%%%% !!!! Shift numbers (00000000

% \setcounter{equation}{35}

\ea % (36)
\ea % a.
\gll \emph{urši} \emph{iz-uwe} \emph{le-w.}\\
boy(\textsc{abs}) be.sick:\textsc{ipf}-\textsc{cvb.ipfv} \textsc{aux}-\textsc{m}\\
\glt `The boy is sick.'

\ex % b.
\gll \emph{dursi} \emph{iz-uwe} \emph{le-r.}\\
girl(\textsc{abs}) be.sick:\textsc{ipf}-\textsc{cvb.ipfv} \textsc{aux}-\textsc{f}\\
\glt `The girl is sick.'
\z

\ex % (37)
\ea % a.
\gll \emph{madina-ze} \emph{rasul} \emph{w-alh-uwe} \emph{le-w.}\\
Madina-\textsc{inter(lat)} Rasul(\textsc{abs}) \textsc{m}-know:\textsc{ipf}-\textsc{cvb.ipfv} \textsc{aux}-\textsc{m}\\
\glt `Madina knows Ali.'

\ex % b.
\gll \emph{rasuj-ze} \emph{madina} \emph{d-alh-uwe} \emph{le-r.}\\
Rasul+\textsc{obl}-\textsc{inter(lat)} Madina(\textsc{abs}) \textsc{f1}-know:\textsc{ipf}-\textsc{cvb.ipfv} \textsc{aux}-\textsc{f}\\
\glt `Rasul knows Madina.'
\z

\ex % (38)
\ea % a.
\gll \emph{madina-s} \emph{rasul} \emph{w-ig-uwe} \emph{le-w.}\\
Madina-\textsc{dat} Rasul(\textsc{abs}) \textsc{m}-love:\textsc{ipf}-\textsc{cvb.ipfv} \textsc{aux}-\textsc{m}\\
\glt `Madina loves Rasul.'

\ex % b.
\gll \emph{rasuj-s} \emph{madina} \emph{d-ig-uwe} \emph{le-r.}\\
Rasul+\textsc{obl}-\textsc{dat} Madina(\textsc{abs}) \textsc{f1}-love:\textsc{ipf}-\textsc{cvb.ipfv} \textsc{aux}-\textsc{f}\\
\glt `Rasul loves Madina.'
\z

\ex % (39)
\gll \emph{urši-li-ni} \emph{i-le} \emph{le-b}~...\\
boy-\textsc{obl}-\textsc{erg} say:\textsc{pf}+\textsc{aor}-\textsc{cvb} \textsc{aux}-\textsc{n}\\
\glt `The boy said that \ldots{}'
\z

In complex verbs that include an adjectival stem specified for prefixal
gender agreement as a non-verbal component, the adjective always agrees
with the absolutive argument.

\ea % (40)
\ea % a.
\gll \emph{adam-ule-ni} \emph{huni} \emph{b-aˤʡu} \emph{b-aq'-ib.}\\
man-\textsc{pl}-\textsc{erg} road(\textsc{abs}) \textsc{n}-wide \textsc{n}-do:\textsc{pf}-\textsc{aor}\\
\glt `Men widened the road.'

\ex % b.
\gll \emph{adam-ule-ni} \emph{hun-be} \emph{d-aˤʡu} \emph{d-aq'-ib.}\\
man-\textsc{pl}-\textsc{erg} road-\textsc{pl}(\textsc{abs}) \textsc{npl}-wide \textsc{npl}-do:\textsc{pf}-\textsc{aor}\\
\glt `Men widened the roads.'
\z
\z

If a sentence contains two absolutive arguments, as attested in
biabsolutive constructions, the auxiliary agrees with the subject (see
\sectref{Biabsolutive-construction}).


% 3.
\section{Verbal person agreement}\label{Verbal-person-agreement}

% 3.1.
\subsection{Intransitive, transitive, and locative subject verbs in synthetic indicative forms}\label{intransitive-transitive-and-locative-subject-verbs-in-synthetic-indicative-forms}

In synthetic indicative tense-aspect forms (aorist, imperfect, habitual,
future), person agreement operates on a nominative-accusative basis and
cross-references the person of the subject: the absolutive argument of
intransitive verbs, the ergative argument of transitive verbs, or the
inter-lative argument of locative subject verbs.

\ea % (41)
\gll \emph{nu} \emph{usaʔ-un-na.}\\
I(\textsc{abs}) (\textsc{m})fall~asleep:\textsc{pf}-\textsc{aor}-\textsc{ego}\\
\glt `I fell asleep.'

\ex % (42)
\gll \emph{nuša-jni} \emph{qali} \emph{b-aq'-i-ra.}\\
we-\textsc{erg} house(\textsc{abs}) \textsc{n}-do:\textsc{pf}-\textsc{aor}-\textsc{ego}\\
\glt `We built a house.'

\ex % (43)
\gll \emph{di-ze} \emph{sinka} \emph{g-ub-ra.}\\
I-\textsc{inter(lat)} bear(\textsc{abs}) see:\textsc{pf}-\textsc{aor}-\textsc{ego}\\
\glt `I saw a bear.'
\z

Morphologically, person inflection only distinguishes two options. One
is a form overtly specified for person (\emph{-iša} in the Future,
\emph{-s} in the Habitual, \emph{-ra} in other indicative
tense-aspect forms), the other is a non-agreeing form. A peculiar
feature of Mehweb is that person agreement is sensitive to the
illocutionary force of the utterance.\footnote{This type of agreement
  system is also referred to as \emph{egophoric},
  \emph{conjunct}/\emph{disjunct}, or \emph{assertive} \emph{agreement},
  see \citet{creissels2008} who discusses assertive agreement in another
  Nakh-Daghestanian language, Akhwakh (Andic branch).} In declarative
sentences, the overt person
marker signals a first person subject,
whereas non-agreeing forms are observed with second and third
person subjects. By contrast, the same person marker indicates second
person subject in interrogative sentences, while first and third person
subjects do not trigger overt person marking on the verb. The following
question-answer pairs illustrate.

\ea\label{ex:7-44} % (44)
\gll Q: \emph{ħu} \emph{dag} \emph{kuda} \{\emph{w-aˤq'-un-na} / *\emph{w-aˤq'-un-a}\}?\\
{} you.sg(\textsc{abs)} yesterday where \textsc{m}-go:\textsc{pf}-\textsc{aor}-\textsc{ego}+\textsc{q} / \textsc{m}-go:\textsc{pf}-\textsc{aor}-\textsc{q}\\
\glt `Where did you go yesterday?'

\gll A: \emph{nu} \emph{anži-li} \{\emph{w-aˤq'-un-na} / *\emph{w-aˤq'-un}\}.\\
{} I(\textsc{abs}) Makhachkala-\textsc{in(lat)} \textsc{m}-go:\textsc{pf}-\textsc{aor}-\textsc{ego} / \textsc{m}-go:\textsc{pf}-\textsc{aor}\\
\glt `I went to Makhachkala.'

\ex\label{ex:7-45} % (45)
\gll Q: \emph{dag} \emph{nu-ni} \emph{sija} \{\emph{b-aq'-ib-a} / *\emph{b-aq'-i-ra}\}?\\
{} yesterday I-\textsc{erg} what(\textsc{abs}) \textsc{n}-do:\textsc{pf}-\textsc{aor}-\textsc{q} / \textsc{n}-do:\textsc{pf}-\textsc{aor}-\textsc{ego}+\textsc{q}\\
\glt `What did I do yesterday?'

\gll A: \emph{ħu-ni} \emph{poˤroˤm} \{\emph{b-uˤrʡ-aq-ib} / *\emph{b-uˤrʡ-aq-i-ra}\}.\\
{} you.sg-\textsc{erg} glass(\textsc{abs}) \textsc{n}-break:\textsc{pf}-\textsc{caus}-\textsc{aor} / \textsc{n}-break:\textsc{pf}-\textsc{caus}-\textsc{aor}-\textsc{ego}\\
\glt `You broke a window.'
\z

Example (\ref{ex:7-44}) shows that second person subjects in interrogatives and
first person subjects in declaratives obligatorily require overt
person marking, whereas subjects in reverse the combinations of person and
illocutionary force – first person subjects in interrogatives and second
person subjects in declaratives~– can never trigger person marking, as
example (\ref{ex:7-45}) demonstrates. (For discussion of one notable exception
see \sectref{matrix-infinitival-questions} below.)

Person marking on synthetic tense-aspect forms is obligatory with
intransitive absolutive subjects and transitive ergative subjects and
cannot be omitted. Locative subject verbs display variation here.
The verb \emph{gʷes} `see' patterns with the transitive and
intransitive verbs in requiring person agreement, whereas with all other
locative subject verbs, person marking is optional.

\ea % (46)
\gll \emph{di-ze} \emph{urx-ne} \{\emph{d-arg-i-ra} / \emph{d-arg-ib}\}.\\
I-\textsc{inter(lat)} key-\textsc{pl}(\textsc{abs}) \textsc{npl}-find:\textsc{pf}-\textsc{aor}-\textsc{ego} / \textsc{npl}-find:\textsc{pf}-\textsc{aor}\\
\glt `I found the keys.'

\ex % (47)
\gll \emph{di-ze} \emph{rasu-wa} \emph{t'ama} \{\emph{arʁ-i-ra} / \emph{arʁ-ib}\}.\\
I-\textsc{inter(lat)} Rasul+\textsc{obl}-\textsc{gen} sound(\textsc{abs}) hear:\textsc{pf}-\textsc{aor}-\textsc{ego} / hear:\textsc{pf}-\textsc{aor}\\
\glt `I heard Rasul's voice.'

\ex % (48)
\gll \emph{di-ze} \emph{rasul} \{\emph{w-alh-as} / \emph{w-alh-an}\}.\\
I-\textsc{inter(lat)} Rasul(\textsc{abs}) \textsc{m}-know:\textsc{ipf}-\textsc{hab}.\textsc{ego} / \textsc{m}-know:\textsc{ipf}-\textsc{hab}\\
\glt `I know Rasul.'
\z

Similar to locative subject verbs, the inter-elative subject of the verb
\emph{buhes} `manage, be able' triggers overt person marking only
optionally.

\ea % (49)
\gll \emph{di-ze-la} \emph{ajz-es} \{\emph{ħa-b-urh-an} / \emph{ħa-b-urh-as}\}.\\
I-\textsc{inter}-\textsc{el} (\textsc{m})rise:\textsc{pf}-\textsc{inf} \textsc{neg}-\textsc{n}-manage:\textsc{ipf}-\textsc{hab} / \textsc{neg}-\textsc{n}-manage:\textsc{ipf}-\textsc{hab}.\textsc{ego}\\
\glt `I cannot stand up.'

\ex % (50)
\gll \emph{di-ze-la} \emph{ħa-b-uh-ub(-ra)} \emph{ʁarʁa} \emph{aq} \emph{b-aq'-as.}\\
I-\textsc{inter}-\textsc{el} \textsc{neg}-\textsc{n}-manage:\textsc{pf}-\textsc{aor}-\textsc{ego} stone(\textsc{abs}) up \textsc{n}-do:\textsc{pf}-\textsc{inf}\\
\glt `I did not manage to lift the stone.'
\z

Non-subjects, including absolutive direct objects, inter-lative indirect
objects (addressee, causee), inter-elative arguments (including
involuntary agents) and other oblique arguments can never trigger person
agreement.

\ea % (51)
\gll \emph{ʡali-ini} \emph{nu} \{\emph{w-it-ib} / *\emph{w-it-i-ra}\}.\\
Ali-\textsc{erg} I(\textsc{abs}) \textsc{m}-beat:\textsc{pf}-\textsc{aor} / \textsc{m}-beat:\textsc{pf}-\textsc{aor}-\textsc{ego}\\
\glt `Ali beat me up.'

\ex % (52)
\gll \emph{madina-ze} \emph{nu} \{\emph{g-ub} / *\emph{g-ub-ra}\}.\\
Madina-\textsc{inter(lat)} I(\textsc{abs}) see:\textsc{pf}-\textsc{aor} / see:\textsc{pf}-\textsc{aor}-\textsc{ego}\\
\glt `Madina saw me.'

\ex % (53)
\gll \emph{rasuj-ni} \emph{di-ze} \emph{ca} \emph{χabar} \{\emph{b-urh-ib} / *\emph{b-urh-i-ra}\}.\\
Rasul-\textsc{erg} I-\textsc{inter(lat)} one story(\textsc{abs}) \textsc{n}-tell:\textsc{pf}-\textsc{aor} / \textsc{n}-tell:\textsc{pf}-\textsc{aor}-\textsc{ego}\\
\glt `Rasul told me a story.'

\ex % (54)
\gll \emph{abaj-ni} \emph{di-ze} \emph{ʁadur-me} \{\emph{d-az-aq-ib} / *\emph{d-az-aq-i-ra}\}.\\
mother-\textsc{erg} I-\textsc{inter(lat)} dish-\textsc{pl}(\textsc{abs}) \textsc{npl}-wash:\textsc{pf}-\textsc{caus}-\textsc{aor} / \textsc{npl}-wash:\textsc{pf}-\textsc{caus}-\textsc{aor}-\textsc{ego}\\
\glt `Mother made me wash the dishes.'

\ex % (55)
\gll \emph{di-ze-la} \emph{guruška} \emph{b-uˤrʡ-uˤb}(-*\emph{ra}).\\
I-\textsc{inter}-\textsc{el} cup(\textsc{abs}) \textsc{n}-break:\textsc{pf}-\textsc{aor}-\textsc{ego}\\
\glt `A cup broke on me.'
\z

This strict subject orientation of agreement allows us to distinguish
between transitive and intransitive uses of P-labile verbs, as shown in
the following examples.

\ea\label{ex:7-56} % (56)
\ea % a.
\gll \emph{nu} \emph{quli-w} \emph{w-aˤld-un-na}.\\
I(\textsc{abs}) house-\textsc{m}(\textsc{ess}) \textsc{m}-hide:\textsc{pf}-\textsc{aor}-\textsc{ego}\\
\glt `I hid in the house.'

\ex % b.
\gll \emph{nu} \emph{quli-w} \emph{w-aˤld-un}.\\
I(\textsc{abs}) house-\textsc{m}(\textsc{ess}) \textsc{m}-hide:\textsc{pf}-\textsc{aor}\\
\glt `They hid me in the house.'
\z
\z

In (\ref{ex:7-56}a), the presence of the agreement marker on the verb indicates
that the first person singular pronoun \emph{nu} is in the subject
position, and that the sentence therefore instantiates the intransitive
use of the labile verb. The absence of agreement in (\ref{ex:7-56}b) can only
indicate that the absolutive pronoun is in the direct object position
and that we are thus dealing with the transitive use of the labile verb.

% 3.2.
\subsection{Dative subject verbs}\label{dative-subject-verbs}

Unlike subjects of intransitive, transitive, and locative subject verbs,
dative subjects do not trigger overt person agreement.

\ea % (57)
\gll \emph{nab} \emph{rasul} \{\emph{w-ig-an} / *\emph{w-ig-as}\}.\\
I(\textsc{dat}) Rasul(\textsc{abs}) \textsc{m}-love:\textsc{ipf}-\textsc{hab} / \textsc{m}-love:\textsc{ipf}-\textsc{hab}.\textsc{ego}\\
\glt `I love Rasul.'

\ex % (58)
\gll \emph{nab} \emph{ʡaˤχ-il} \emph{q'immat} \{\emph{b-ik-ib} / *\emph{b-ik-i-ra}\}.\\
I(\textsc{dat}) good-\textsc{atr} grade(\textsc{abs}) \textsc{n}-happen:\textsc{pf}-\textsc{aor} / \textsc{n}-happen:\textsc{pf}-\textsc{aor}-\textsc{ego}\\
\glt `I got a good grade.'

\ex % (59)
\gll \emph{nab} \emph{rasul} \emph{eba} \{\emph{uh-ub} / *\emph{uh-ub-ra}\}.\\
I(\textsc{dat}) Rasul(\textsc{abs}) bore (\textsc{m})become:\textsc{pf}-\textsc{aor} / (\textsc{m})become:\textsc{pf}-\textsc{aor}-\textsc{ego}\\
\glt `I got bored with Rasul.'

\ex % (60)
\gll \emph{nab} \emph{ʡali} \emph{urče-w} \{\emph{le-w} / *\emph{le-w-ra}\}.\\
I(\textsc{dat}) Ali(\textsc{abs}) in.heart-\textsc{m(ess)} {be}-\textsc{m} / {be}-\textsc{m}-\textsc{ego}\\
\glt `I remember Ali.'

\ex % (61)
\gll \emph{nab} \emph{hel} \emph{dehʷ} \emph{urče} \{\emph{b-ak'-ib} / *\emph{b-ak'-i-ra}\}.\\
I(\textsc{dat}) {this} word(\textsc{abs}) in.heart(\textsc{lat}) \textsc{n}-come:\textsc{pf}-\textsc{aor} / \textsc{n}-come:\textsc{pf}-\textsc{aor}-\textsc{ego}\\
\glt `Rasul recalled that word.'

\ex % (62)
\gll \emph{nab} \{\emph{b-ik-ib} / *\emph{b-ik-i-ra}\} \emph{ʡali} \emph{w-ebk'-i-le} \emph{ile.}\\
I(\textsc{dat}) \textsc{n}-happen:\textsc{pf}-\textsc{aor} / \textsc{n}-happen:\textsc{pf}-\textsc{aor}-\textsc{ego} Ali(\textsc{abs}) \textsc{m}-die:\textsc{pf}-\textsc{aor}-\textsc{cvb} \textsc{comp}\\ 
\glt `I thought (it occurred to me) that Ali was dead.'
\z

The contrast between locative and dative subject verbs is clearly seen
in sentences with the verb \emph{qumartes} `forget'. Recall that this
verb allows both locative and dative subjects. With a first person
locative subject, the verb has optional person agreement, as with other
locative subject verbs. With a first person dative subject, the verb
cannot show overt person marking, as is usual with dative subject verbs.

\ea % (63)
\ea % a.
\gll \emph{di-ze} \emph{ʡali} \emph{qum-art-ur}(\emph{-ra}).\\
I-\textsc{inter(lat)} Ali(\textsc{abs}) forget-\textsc{lv}:\textsc{pf}-\textsc{aor}-\textsc{ego}\\

\ex % b.
\gll \emph{nab} \emph{ʡali} \emph{qum-art-ur}(-*\emph{ra}).\\
I(\textsc{dat}) Ali(\textsc{abs}) forget-\textsc{lv}:\textsc{pf}-\textsc{aor}-\textsc{ego}\\
\glt `I forgot Ali.'
\z
\z

In sentences with dative subjects, absolutive direct objects do not
trigger person agreement either, as shown in examples (\ref{ex:7-64}) to (\ref{ex:7-67}).

\ea\label{ex:7-64}
\gll \emph{madina-s} \emph{nu} \{\emph{w-ig-an} / *\emph{w-ig-as}\}.\\
Madina-\textsc{dat} I(\textsc{abs}) \textsc{m}-love:\textsc{ipf}-\textsc{hab} / \textsc{m}-love:\textsc{ipf}-\textsc{hab}.\textsc{ego}\\
\glt `Madina loves me.'

\ex % (65)
\gll \emph{madina-s} \emph{nu} \emph{eba} \{\emph{uh-ub} / *\emph{uh-ub-ra}\}.\\
Madina-\textsc{dat} I(\textsc{abs}) bore (\textsc{m})become:\textsc{pf}-\textsc{aor} / (\textsc{m})become:\textsc{pf}-\textsc{aor}-\textsc{ego}\\
\glt `Madina got bored with me.'

\ex % (66)
\gll \emph{madina-s} \emph{nu} \emph{urče-w} \{\emph{le-w} / *\emph{le-w-ra}\}.\\
Madina-\textsc{dat} I(\textsc{abs}) in.heart-\textsc{m(ess)} {be}-\textsc{m} / {be}-\textsc{m}-\textsc{ego}\\ \unskip
\glt `Madina remembers me.'

\ex \label{ex:7-67}
\gll \emph{rasuj-s} \emph{nu} \emph{urče} \{\emph{b-ak'-ib} / *\emph{b-ak'-i-ra}\}.\\
Rasul+\textsc{obl}-\textsc{dat} I(\textsc{abs}) in.heart(\textsc{lat}) \textsc{n}-come:\textsc{pf}-\textsc{aor} / \textsc{n}-come:\textsc{pf}-\textsc{aor}-\textsc{ego}\\ 
\glt `Rasul recalled me.'
\z

The absence of agreement with the absolutive argument is 
unexpected given the fact that many of the dative subject verbs clearly
go back to intransitive structures where absolutive arguments
diachronically go back to intransitive subjects, and thus could act as
agreement triggers, contrary to fact.

\ea % (68)
\ea % a.
\gll \emph{X} \emph{Y} \emph{eba} \emph{b-uh-es.}\\
\textsc{dat} \textsc{abs} bore \textsc{n}-become:\textsc{pf}-\textsc{inf}\\
\glt `For X, Y becomes boring.'

\ex % b.
\gll \emph{X} \emph{Y} \emph{urče-b} \emph{le-b.}\\
\textsc{dat} \textsc{abs} in.heart-\textsc{n}(\textsc{ess}) {be}-\textsc{n}\\
\glt `To X, Y is on heart.'

\ex % c.
\gll \emph{X} \emph{Y} \emph{urče} \emph{b-ak'-as.}\\
\textsc{dat} \textsc{abs} in.heart(\textsc{lat}) \textsc{n}-come:\textsc{pf}-\textsc{inf}\\
\glt `To X, Y comes to heart.'
\z
\z

The clear contrast between intransitive and dative subject constructions
with respect to person agreement is observed in a construction with the
verb \emph{haraq'e} \emph{bak'as} (lit.\ `come forward'), which denotes
``illusionary seeing'', as in dreams or hallucinations, as in (\ref{ex:7-69}).

\pagebreak

\ea\label{ex:7-69}
\gll \emph{rasuj-s} \emph{tamaša-l} \emph{si-k'al-t} \emph{haraq'e} \emph{d-ik'-uwe} \emph{le-r.}\\
Rasul+\textsc{obl}-\textsc{dat} surprising-\textsc{atr} what-\textsc{indef}-\textsc{pl} forward \textsc{npl}-come:\textsc{ipf}-\textsc{cvb.ipfv} \textsc{aux}-\textsc{npl}\\
\glt `Rasul sees something bizarre.' (lit.\ `Something bizarre is coming
forward to Rasul.')
\z

As in other dative subject structures, neither of the two arguments,
the dative subject or the absolutive direct object, is able to trigger
person agreement on the verb.

\ea % (70)
\ea % a.
\gll \emph{nab} \emph{tamaša-l} \emph{si-k'al-t} \emph{haraq'e} \{\emph{d-ak'-ib} / *\emph{d-ak'-i-ra}\}.\\
I(\textsc{dat}) surprising-\textsc{atr} what-\textsc{indef}-\textsc{pl} forward \textsc{npl}-come:\textsc{pf}-\textsc{aor} / \textsc{npl}-come:\textsc{pf}-\textsc{aor}-\textsc{ego}\\
\glt `Something bizarre appeared to me.'

\ex % b.
\gll \emph{rasuj-s} \emph{nu} \emph{haraq'e} \{\emph{w-ak'-ib} / *\emph{w-ak'-i-ra}\}.\\
Rasul+\textsc{obl}-\textsc{dat} I(\textsc{abs}) forward \textsc{m}-come:\textsc{pf}-\textsc{aor} / \textsc{m}-come:\textsc{pf}-\textsc{aor}-\textsc{ego}\\
\glt `I appeared to Rasul (in a hallucination).'
\z
\z

Overt person marking on the verb \emph{bak'as} `come' in the latter
example is grammatical only in the literal sense of physical movement.

\ea % (71)
\gll \emph{rasuj-s} \emph{nu} \emph{haraq'e} \{\emph{w-ak'-i-ra} / *\emph{w-ak'-ib}\}.\\
Rasul+\textsc{obl}-\textsc{dat} I(\textsc{abs}) forward \textsc{m}-come:\textsc{pf}-\textsc{aor}-\textsc{ego} / \textsc{m}-come:\textsc{pf}-\textsc{aor}\\\unskip
\glt `I came forward to Rasul.' (not: `I appeared to Rasul (in a
hallucination).')
\z

We therefore have a minimal pair: in the same construction with
\emph{haraq'e} \emph{bak'as} `come forward', person agreement with the first
person absolutive argument is obligatorily required when denoting
physical movement and completely prohibited when referring to imaginary
visions.

To sum up, neither of the two arguments of a dative subject verb – the
dative subject or the absolutive direct object – can control person
agreement on their own. Strikingly enough, overt person marking on a
finite dative subject verb is nevertheless possible in constructions
where both the dative subject and the absolutive direct object are first
person (i.e.\ in reflexive constructions with a first person subject).

\ea % (72)
\gll \emph{nab} \emph{nu꞊wal} \emph{w-ig-as.}\\
I(\textsc{dat}) I(\textsc{abs})꞊\textsc{emph} \textsc{m}-love:\textsc{pf}-\textsc{hab}.\textsc{ego}\\
\glt `I love myself.'
\z

The syntax of dative subject constructions and the way they interact
with person agreement require further syntactic analysis.

% 3.3.
\subsection{Agreement in the Present Progressive}\label{agreement-in-the-present-progressive}

Present Progressive forms exhibit a different pattern of person
agreement in sentences with transitive and locative subject verbs.
Unlike other indicative forms, not only the person feature of the
subject is taken into account here, but also the person feature of the
direct (absolutive) object.

The descriptive generalization is that overt person agreement with the
first person subject is only possible (and obligatory) when the
absolutive direct object is a locutor (first or second person).
Otherwise, with third person direct objects, person agreement is
ungrammatical, and the finite verb is in the unmarked form.\footnote{In
  transitive clauses with third person direct objects, such as (\ref{ex:7-73}a),
  first person marking is marginally accepted by some native speakers.
  It is not clear where such marginal acceptability stems from. One
  option could be that optional person agreement in these configurations
  is actually a part of Mehweb grammar. Another option, however, is that
  it arises from confusion with biabsolutive constructions where person
  agreement with the subject is obligatory in the Present Progressive
  (see \sectref{Biabsolutive-construction}). Indeed, many speakers, when accepting person
  agreement in examples like (\ref{ex:7-73}a), tend to rephrase the ergative
  construction of (\ref{ex:7-73}a) into the corresponding biabsolutive
  construction with the absolutive subject, with subject-controlled
  person and gender agreement on the auxiliary. Note that with locative
  subject verbs, which are not easily allowed in biabsolutive
  constructions, person agreement in the Present Progressive is
  definitely rejected by all speakers, see (\ref{ex:7-74}a).}

\ea\label{ex:7-73}
\ea % a.
\gll \emph{nu-ni} \emph{kung} \emph{luč'-uwe} \emph{le-b}(*\emph{-ra}).\\
I-\textsc{erg} book(\textsc{abs}) read:\textsc{ipf}-\textsc{cvb.ipfv} \textsc{aux}-\textsc{n}-\textsc{ego}\\
\glt `I am reading a book.'

\ex % b.
\gll \emph{nu-ni} \emph{ħu} \emph{ulc-uwe} \emph{le-w-}*(\emph{ra}).\\
I-\textsc{erg} you.sg(\textsc{abs)} (\textsc{m})catch:\textsc{ipf}-\textsc{cvb.ipfv} \textsc{aux}-\textsc{m}-\textsc{ego}\\
\glt `I am catching you (male).'
\z

\ex\label{ex:7-74}
\ea % a.
\gll \emph{di-ze} \emph{sinka} \emph{irg-uwe} \emph{le-b}(*\emph{-ra}).\\
I-\textsc{inter(lat)} bear(\textsc{abs}) see:\textsc{ipf}-\textsc{cvb.ipfv} \textsc{aux}-\textsc{n}-\textsc{ego}\\
\glt `I can see a bear.'

\ex % b.
\gll \emph{di-ze} \emph{ħu} \emph{irg-uwe} \emph{le-w}-*(\emph{ra}).\\
I-\textsc{inter(lat)} you.sg(\textsc{abs}) see:\textsc{ipf}-\textsc{cvb.ipfv} \textsc{aux}-\textsc{m}-\textsc{ego}\\
\glt `I can see you.'
\z
\z

Examples (\ref{ex:7-73}a) and (\ref{ex:7-74}a) show that agreement with first person subjects
is impossible in the presence of a third person absolutive direct
object. By contrast, agreement is obligatory when the direct object is
also a locutor. Relative specification of the subject and the direct
object for number plays no role in the availability of person agreement.

\ea % (75)
\ea % a.
\gll \{\emph{nu-ni} / \emph{nuša-jni}\} \emph{ħuša} \emph{b-ulc-uwe} \emph{le-b}-*(\emph{ra}).\\
I-\textsc{erg} / we-\textsc{erg} you.pl(\textsc{abs)} \textsc{hpl}-catch:\textsc{ipf}-\textsc{cvb.ipfv} \textsc{aux}-\textsc{hpl}-\textsc{ego}\\
\glt `\{I am~/ we are\} catching you all.'

\ex % b.
\gll \emph{nuša-jni} \emph{ħu} \emph{ulc-uwe} \emph{le-w}-*(\emph{ra}).\\
we-\textsc{erg} you.sg(\textsc{abs)} (\textsc{m})catch:\textsc{ipf}-\textsc{cvb.ipfv} \textsc{aux}-\textsc{m}-\textsc{ego}\\
\glt `We are catching you.'
\z

\ex % (76)
\ea % a.
\gll \{\emph{nu-ni} / \emph{nuša-jni}\} \emph{ul-e} \emph{b-ulc-uwe} \emph{le-b}(-*\emph{ra}).\\
I-\textsc{erg} / we-\textsc{erg} child-\textsc{pl}(\textsc{abs}) \textsc{hpl}-catch:\textsc{ipf}-\textsc{cvb.ipfv} \textsc{aux}-\textsc{hpl}-\textsc{ego}\\
\glt `\{I am~/ we are\} catching the kids.'

\ex % b.
\gll \emph{nuša-jni} \emph{qazam} \emph{b-iz-uwe} \emph{le-b}(-*\emph{ra}).\\
we-\textsc{erg} cauldron(\textsc{abs)} \textsc{n}-wash:\textsc{ipf}-\textsc{cvb.ipfv} \textsc{aux}-\textsc{n}-\textsc{ego}\\
\glt `We are washing the cauldron.'
\z
\z

% 3.4.
\subsection{Matrix infinitival questions}\label{matrix-infinitival-questions}

One exception to the generalization that only second, but not first,
person subjects trigger person agreement in interrogative sentences
concerns agreeing Future forms, which may co-occur with first person
subjects in interrogatives, yielding questions with modal semantics.

\ea \label{ex:7-77}
\gll \emph{nu-ni} \emph{ħad} \emph{sija} \emph{g-iša?}\\
I-\textsc{erg} you.sg(\textsc{dat}) what(\textsc{abs}) give:\textsc{pf}-\textsc{fut}.\textsc{ego}+\textsc{q}\\
\glt `What should I give you?' (not: `What will I give you?')

\ex \label{ex:7-78}
\gll \emph{nu} \emph{uˤq'-iša?}\\
I(\textsc{abs}) (\textsc{m})go:\textsc{pf}-\textsc{fut}.\textsc{ego}+\textsc{q}\\
\glt `Should I go?' (not: `Will I go?')
\z

Examples like (\ref{ex:7-77}) and (\ref{ex:7-78}) are remarkable in two respects. First, they
are only available in the Future, and not in other tense-aspect forms.

\ea % (79)
\gll \emph{*nu-ni} \emph{ħad} \emph{sija} \emph{g-i-ra?}\\
I-\textsc{erg} you.sg(\textsc{dat}) what(\textsc{abs}) give:\textsc{pf}-\textsc{aor}-\textsc{ego}+\textsc{q}\\
\glt intended: `What should I have given you?' (or `What did I give you?')
\z

Second, the modal interpretation of the questions in (\ref{ex:7-77}) and (\ref{ex:7-78}) only
arises with first person subjects, but never with second person
subjects, cf.\ the contrast between (\ref{ex:7-80}) and (\ref{ex:7-81}).

\ea\label{ex:7-80}
\gll \emph{nu} \emph{kuda} \emph{uˤq'-iša?}\\
I(\textsc{abs}) where (\textsc{m})go:\textsc{pf}-\textsc{fut}.\textsc{ego}+\textsc{q}\\
\glt `Where should I go?' (not: `Where will I go?')

\ex\label{ex:7-81}
\gll \emph{ħu} \emph{kuda} \emph{uˤq'-iša?}\\
you.sg(\textsc{abs)} where (\textsc{m})go:\textsc{pf}-\textsc{fut}.\textsc{ego}+\textsc{q}\\
\glt `Where will you go?' (not: `Where should you go?')
\z

This contrast raises the question whether the two sentences in (\ref{ex:7-80}) and
(\ref{ex:7-81}) contain the same or two different verb forms. This question is
especially relevant in the light of the fact that the infinitive in
Mehweb is formally identical to non-agreeing future forms, which appear,
for example, in declarative sentences with second/third person subjects,
as shown in (\ref{ex:7-82}).

\ea\label{ex:7-82}
\ea % a.
\gll \emph{ʡali} \emph{šaˤ-baˤʜ} \emph{uˤq'-es.}\\
Ali(\textsc{abs}) village-\textsc{dir} (\textsc{m})go:\textsc{pf}-\textsc{fut}\\
\glt `Ali will go to the village.'

\ex % b.
\gll \emph{ʡali-s\textsubscript{i}} [pro\textsubscript{i} \emph{šaˤ-baˤʜ} \emph{uˤq'-es}] \emph{dig-uwe} \emph{le-b}.\\
Ali-\textsc{dat} \textsc{abs} village-\textsc{dir} (\textsc{m})go:\textsc{pf}-\textsc{inf}
want:\textsc{ipf}-\textsc{cvb.ipfv} \textsc{aux}-\textsc{n}\\
\glt `Ali wants to go to the village.'
\z
\z

The infinitive and the future are normally distinguished in contexts
with overt person marking (e.g.\ declarative sentences with first person
subjects). The Future takes overt person marking, while the
infinitive never does so, as shown in (\ref{ex:7-83}).

\ea \label{ex:7-83}
\ea % a.
\gll \emph{nu} \emph{šaˤ-baˤʜ} \emph{uˤq'-iša}.\\
I(\textsc{abs}) village-\textsc{dir} (\textsc{m})go:\textsc{pf}-\textsc{fut}.\textsc{ego}\\
\glt `I will go to the village.'

\ex % b.
\gll \emph{nab\textsubscript{i}} [pro\textsubscript{i} \emph{šaˤ-baˤʜ} \emph{uˤq'-es}] \emph{dig-uwe} \emph{le-b}.\\
I(\textsc{dat}) \textsc{abs} village-\textsc{dir} (\textsc{m})go:\textsc{pf}-\textsc{inf} want:\textsc{ipf}-\textsc{cvb.ipfv} \textsc{aux}-\textsc{n}\\
\glt `I want to go to the village.'
\z
\z

Now note that across Dargwa languages, the modal semantics
found in the Mehweb examples in (\ref{ex:7-77}), (\ref{ex:7-78}), (\ref{ex:7-80}) is commonly expressed
by a special form with a first person marker added on top of the
infinitive, as seen in (\ref{ex:7-84}) from Chirag Dargwa.

\ea\label{ex:7-84}
Chirag Dargwa

\gll \emph{di-cːe} \emph{χabar-e} \emph{d-urs-i-da-j?}\\
I-\textsc{erg} story-\textsc{pl}(\textsc{abs}) \textsc{npl}-tell:\textsc{pf}-\textsc{inf}-\textsc{ego}-\textsc{q}\\
\glt `Should I tell the stories?'
\z

The same modal semantics is cross-linguistically characteristic of
matrix infinitival questions (cf.\ English \emph{Where} \emph{to} \emph{go?} or German
\emph{Wohin} \emph{gehen?}, \citealt[108, 110]{bhatt2006}).

It is natural to propose that Mehweb modal questions as in (\ref{ex:7-77})
and (\ref{ex:7-78}) actually involve a combination of the infinitive and overt
person marking rather than the formally identical agreeing form of the Future, as suggested by (i) the formal identity between the infinitive
and the future in non-agreeing forms and (ii) the morphological evidence
that the combination of infinitive with first person marking may yield
the modal semantics of `should' in other Dargwa languages.

% 3.5.
\subsection{Indexical shift and agreement shift in embedded
reports}\label{indexical-shift-and-agreement-shift-in-embedded-reports}

Person agreement as described above is only available in finite clauses:
no non-finite clause can feature a person agreement marker. The
following examples show that person agreement is unavailable in
complements headed by nominalizations.

\ea % (85)
\gll \emph{rasuj-ze} \emph{b-alh-an~...}\\
Rasul+\textsc{obl}-\textsc{inter(lat)} \textsc{n}-know:\textsc{ipf}-\textsc{hab}\\
\glt `Rasul knows \ldots{}'

\ea % a.
\gll \emph{nu-ni} \emph{kung} \{\emph{b-elč'-un-deš} / *\emph{b-elč'-un-na-deš}\}.\\
I-\textsc{erg} book(\textsc{abs}) \textsc{n}-read:\textsc{pf}-\textsc{aor}-\textsc{nmlz} / \textsc{n}-read:\textsc{pf}-\textsc{aor}-\textsc{ego}-\textsc{nmlz}\\\unskip
\glt `\ldots{} that I read (past) the book.'

\ex % b.
\gll \emph{nu-ni} \emph{ħu} \emph{ulc-uwe} \{\emph{le-w-deš} / *\emph{le-w-ra-deš}\}.\\
I-\textsc{erg} you.sg(\textsc{abs}) (\textsc{m})catch:\textsc{ipf}-\textsc{cvb.ipfv} \textsc{aux}-\textsc{m}-\textsc{nmlz} / \textsc{aux}-\textsc{m}-\textsc{ego}-\textsc{nmlz}\\
\glt `\ldots{} that I am catching you.'

\ex % c.
\gll \emph{nu-ni} \emph{kung-ane} \{\emph{luč'-an-deš} / *\emph{luč'-as-deš}\}.\\
I-\textsc{erg} book(\textsc{abs}) read:\textsc{ipf}-\textsc{hab}-\textsc{nmlz} / read:\textsc{ipf}-\textsc{hab}.\textsc{ego}-\textsc{nmlz}\\
\glt `\ldots{} that I read (habitual) books.'
\z
\z

Apart from the independent finite clauses described above, Mehweb also
features finite complement clauses with the complementizer \emph{ile}.
Etymologically, the complementizer stems from (and is still
synchronically identical)  the perfective converb of the verb
\emph{es} `say'. It is used with verbs of speech and thought to
introduce reported speech (attitude reports).

\ea % (86)
\ea % a.
\gll \emph{abaj-s} \emph{b-ik-ib} \emph{ca} \emph{insan} \emph{w-ak'-ib} \emph{ile.} \\
mother-\textsc{dat} \textsc{n}-happen:\textsc{pf}-\textsc{aor} one person(\textsc{abs}) \textsc{m}-come:\textsc{pf}-\textsc{aor} \textsc{comp}\\
\glt `Mother thought that someone had come.'

{\glossglue2.5pt
\ex % b.
\gll \emph{abaj-ni} \emph{b-urh-ib} \emph{ca} \emph{insan} \emph{w-ak'-ib} \emph{ile.}\\
mother-\textsc{erg} \textsc{n}-tell:\textsc{pf}-\textsc{aor} one person(\textsc{abs}) \textsc{m}-come:\textsc{pf}-\textsc{aor} \textsc{comp}\\
\glt `Mother said that someone had come.'

}
\ex % c.
\gll \emph{abaj} \emph{uruχ} \emph{d-aˤq-ib} \emph{ca} \emph{insan} \emph{w-ak'-ib} \emph{ile.}\\
mother(\textsc{abs}) be.afraid \textsc{f1}-\textsc{lv}:\textsc{pf}-\textsc{aor} one person(\textsc{abs}) \textsc{m}-come:\textsc{pf}-\textsc{aor} \textsc{comp}\\
\glt `Mother feared that someone had come.'
\z
\z

Personal pronouns and person agreement in embedded reports under the
complementizer \emph{ile} are subject to \emph{\textsc{person shift}}
(\emph{\textsc{indexical shift}} and \emph{\textsc{agreement shift}},
respectively), see \citet{schlenker2003}, \citet{anand-nevins2004}, \citet{nikitina2012}, \citet{shklovsky-sudo2014} on indexical shift in a theoretical
and typological perspective.


Indexical shift affects the interpretation of first and second person
pronouns and is always optional. Personal pronouns in embedded reports
may refer not only to the participants of the actual speech act, as in
independent finite clauses, but also to the participants of the speech
act denoted by the matrix clause. In the latter case, the first person
pronoun refers to the reporter (attitude holder) expressed as the
subject of the matrix clause, while the second person pronoun denotes
the addressee of the matrix reporter.

\ea % (87)
\gll \emph{rasuj-ni} \emph{ib} \emph{di-la} \emph{mašin} \emph{b-uˤrʡ-uˤb} \emph{ile.}\\
Rasul+\textsc{obl}-\textsc{erg} say:\textsc{pf}+\textsc{aor} I-\textsc{gen} car(\textsc{abs}) \textsc{n}-break:\textsc{pf}-\textsc{aor} \textsc{comp}\\

\ea % a.
`Rasul\textsubscript{i} said that my\textsubscript{j} car was broken.' (unshifted reading of the 1\textsuperscript{st} person pronoun)

\ex % b.
`Rasul\textsubscript{i} said that his\textsubscript{i} car was broken.' (shifted reading of the 1\textsuperscript{st} person pronoun)
\z

\ex % (88)
\gll \emph{madina-ini} \emph{rasuj-ze} \emph{ib} \emph{ħa-la} \emph{mašin} \emph{b-uˤrʡ-uˤb} \emph{ile.}\\
Madina-\textsc{erg} Rasul+\textsc{obl}-\textsc{inter(lat)} say:\textsc{pf}+\textsc{aor} you.sg-\textsc{gen} car(\textsc{abs}) \textsc{n}-break:\textsc{pf}-\textsc{aor} \textsc{comp}\\

\ea % a.
`Madina said to Rasul\textsubscript{i} that your\textsubscript{j} car
was broken.' (unshifted reading of the 2\textsuperscript{nd} person
pronoun)

\ex % b.
`Madina said to Rasul\textsubscript{i} that his\textsubscript{i} car
was broken.' (shifted reading of the 2\textsuperscript{nd} person
pronoun)
\z
\z

With matrix verbs selecting for a complement clause with \emph{ile} but
lacking an addressee, such as matrix verbs of thought, only first person
pronouns can be shifted, while second person pronouns only denote the
addressee of the actual speech act.

\ea % (89)
\gll \emph{rasul} \emph{uruχ} \emph{w-aˤq-ib} \emph{di-la} \emph{mašin} \emph{b-uˤrʡ-uˤb} \emph{ile.}\\
Rasul(\textsc{m}) be.afraid \textsc{m}-\textsc{lv}:\textsc{pf}-\textsc{aor} I-\textsc{gen} car(\textsc{abs}) \textsc{n}-break:\textsc{pf}-\textsc{aor} \textsc{comp}\\

\ea % a.
`Rasul\textsubscript{i} fears that my\textsubscript{j} car was
broken.' (unshifted reading of the 1\textsuperscript{st} person pronoun)

\ex % b.
`Rasul\textsubscript{i} fears that his\textsubscript{i} car was broken.' (shifted reading of the 1\textsuperscript{st} person pronoun)
\z

\ex % (90)
\gll \emph{rasul} \emph{uruχ} \emph{w-aˤq-ib} \emph{ħa-la} \emph{mašin} \emph{b-uˤrʡ-uˤb} \emph{ile.}\\
Rasul(\textsc{m}) be.afraid \textsc{m}-\textsc{lv}:\textsc{pf}-\textsc{aor} you.sg-\textsc{gen} car(\textsc{abs}) \textsc{n}-break:\textsc{pf}-\textsc{aor} \textsc{comp}\\
\glt `Rasul\textsubscript{i} fears that your\textsubscript{j} car was
broken.' (only unshifted reading of the 2\textsuperscript{nd} person
pronoun)
\z


Person agreement in finite embedded clauses is subject to obligatory
\emph{\textsc{agreement shift}}. Only arguments denoting the
participants of the reported speech act can control person agreement,
while other arguments including those representing the participants of
the actual speech act can never trigger agreement. In declarative
embedded clauses, only embedded subjects denoting the closest reporter~/
attitude holder trigger overt agreement on the verb. One possibility is
that the embedded subject is expressed by the shifted first person
pronoun.

\ea \label{ex:7-91}
\gll \emph{rasul} \emph{uruχ} \emph{w-aˤq-ib} \emph{nu-ni} \emph{mašin} \emph{b-uˤrʡ-aq-i-ra} \emph{ile.}\\
Rasul(\textsc{abs}) be.afraid \textsc{m}-\textsc{lv}:\textsc{pf}-\textsc{aor} I-\textsc{erg} car(\textsc{abs}) \textsc{n}-break:\textsc{pf}-\textsc{caus}-\textsc{aor}-\textsc{ego} \textsc{comp}\\
\glt `Rasul\textsubscript{i} feared that he\textsubscript{i} had broken the car.'
\z

In (\ref{ex:7-91}), the subject is expressed by the first person pronoun that
undergoes indexical shift; that is, it does not refer to the speaker of
the actual speech act, but rather to the attitude holder (Rasul)
expressed as the subject of the matrix clause. The embedded verb thus
shows obligatory overt agreement for person.

Another possibility is that the embedded subject is expressed by the
long-distance reflexive pronoun bound by the matrix subject representing
the attitude holder. The long-distance reflexive thus ends up being
co-referent with the attitude holder, and the verb obligatorily shows
overt person marking.

\ea\label{ex:7-92}
\gll \emph{rasul} \emph{uruχ} \emph{w-aˤq-ib} \emph{sune-jni} \emph{mašin} \emph{b-uˤrʡ-aq-i-ra} \emph{ile.}\\
Rasul(\textsc{abs}) be.afraid \textsc{m}-\textsc{lv}:\textsc{pf}-\textsc{aor} self-\textsc{erg} car(\textsc{abs}) \textsc{n}-break:\textsc{pf}-\textsc{caus}-\textsc{aor}-\textsc{ego} \textsc{comp}\\
\glt `Rasul\textsubscript{i} feared that he\textsubscript{i} had broken the car.'
\z

No other argument can trigger person agreement on the finite verb in
embedded reports, including unshifted first person pronouns denoting the
speaker of the actual speech act. Example (\ref{ex:7-93}) illustrates.

\ea\label{ex:7-93}
\gll \emph{rasul} \emph{uruχ} \emph{w-aˤq-ib} \emph{nu-ni} \emph{mašina} \{\emph{b-uˤrʡ-aq-ib} / *\emph{b-uˤrʡ-aq-i-ra}\} \emph{ile.}\\
Rasul(\textsc{abs}) be.afraid \textsc{m}-\textsc{lv}:\textsc{pf}-\textsc{aor} I-\textsc{erg} car(\textsc{abs}) \textsc{n}-break:\textsc{pf}-\textsc{caus}-\textsc{aor} / \textsc{n}-break:\textsc{pf}-\textsc{caus}-\textsc{aor}-\textsc{ego} \textsc{comp}\\
\glt `Rasul\textsubscript{i} feared that I\textsubscript{j} had broken the car.'
\z

\leavevmode{\addfontfeature{LetterSpace=-.7}
\citet{kozhukhar2019} [this volume] reports that overt person marking with
an unshifted first person pronoun is also possible in examples like (\ref{ex:7-93}).
Indeed, consultants sometimes judge such sentences to be acceptable. I
maintain, however, that overt person agreement with an unshifted first
person pronoun is ungrammatical, and the judgments must stem from
confusion. First person pronouns strongly tend to shift their reference
in embedded reports, and consultants usually struggle to recognize
that the pronoun could refer to the actual speaker. So, when
presented with a sentence containing a first person pronoun and overt
person marking on the verb, some consultants judge it acceptable due to
the fact that they have a different reading in mind. Instead of the
reference to the speaker of the actual speech act, they interpret the
pronoun as denoting the attitude holder. However, if a suitable example
is constructed where the confusion is not possible because of overt
morphological marking, overt person marking with unshifted first person
pronouns is uniformly judged unacceptable. Consider the following
examples.}

\ea\label{ex:7-94}
\gll \emph{abaj-s} \emph{b-ik-ib} \emph{nu} \emph{usaʔ-uwe} \emph{le-w}(-*\emph{ra}) \emph{ile.}\\
mother-\textsc{dat} \textsc{n}-happen:\textsc{pf}-\textsc{aor} I(\textsc{abs}) (\textsc{m})fall~asleep:\textsc{pf}-\textsc{aor.cvb} \textsc{aux}-\textsc{m}-\textsc{ego} \textsc{comp}\\
\glt `Mother\textsubscript{i} thought that I\textsubscript{j} had fallen asleep.'

\ex\label{ex:7-95}
\gll \emph{abaj} \emph{uruχk'-uwe} \emph{le-r} \emph{nu} \{\emph{arik-es} / *\emph{arik-iša}\} \emph{ile.}\\
mother(\textsc{abs}) be.afraid:\textsc{ipf}-\textsc{cvb.ipfv} \textsc{aux}-\textsc{f} I(\textsc{abs}) (\textsc{m})fall:\textsc{pf}-\textsc{fut} / (\textsc{m})fall:\textsc{pf}-\textsc{fut}.\textsc{ego} \textsc{comp}\\
\glt `Mother\textsubscript{i} is afraid that I\textsubscript{j} am going to
fall down.'
\z

In (\ref{ex:7-94}) and (\ref{ex:7-95}), the first person pronoun in the embedded clause is
unambiguously interpreted as denoting the actual speaker, since
masculine gender marking appears on the embedded verb (both on the
converb of the lexical verb and the auxiliary), indicating that the
referent of the first person pronoun is a man. Since the attitude holder
(`mother') is unambiguously female, the embedded first person pronoun
may only receive a disjoint reference, and thus denote the speaker of
the actual speech act. In this configuration, overt agreement was
unanimously considered grossly ungrammatical.


Agreement shift thus makes possible various mismatches between the
``lexical'' person feature of an argument and verbal person agreement.
On the one hand, third person reflexive pronouns trigger overt person
marking, as in (\ref{ex:7-92}). On the other hand, first person pronouns referring
to the actual speaker can never trigger overt person agreement, as in
(\ref{ex:7-93}) through (\ref{ex:7-95}).

The examples above show that the attitude holder can be lexically
expressed in the embedded clause by either a shifted first person
pronoun or a long-distance reflexive pronoun. However, these two options
cannot co-occur within the same embedded clause. In the presence of a
long-distance reflexive bound by the matrix subject, first person
pronouns are obligatorily interpreted as referring to the speaker of the
actual speech act.

\ea\label{ex:7-96}
\gll \emph{rasul} \emph{uruχ} \emph{w-aˤq-ib} \emph{nu-ni} \emph{sune-la} \emph{mašina} \emph{b-uˤrʡ-aq-i-ra} \emph{ile.}\\
Rasul(\textsc{abs}) be.afraid \textsc{m}-\textsc{lv}:\textsc{pf}-\textsc{aor} I-\textsc{erg} self-\textsc{gen} car(\textsc{abs}) \textsc{n}-break:\textsc{pf}-\textsc{caus}-\textsc{aor}-\textsc{ego} \textsc{comp}\\

\begin{xlist}
\exi{i.} *'Rasul\textsubscript{i} feared that he\textsubscript{i} broke
his\textsubscript{i} car.'

\exi{ii.} `Rasul\textsubscript{i} feared that he\textsubscript{i} broke
his\textsubscript{j} car.'

\exi{iii.} *'Rasul\textsubscript{i} feared that I\textsubscript{j} broke
his\textsubscript{i} car.'
\end{xlist}
\z

In (\ref{ex:7-96}), the embedded clause includes both the first person pronoun in
the ergative subject position and the possessive reflexive pronoun that
modifies the direct object. The two cannot be interpreted as denoting
the same participant, as shown by the ungrammaticality of reading (i).
Two further options are logically possible: either the first person
pronoun or the reflexive is interpreted as denoting the attitude holder.
In the former case, the reflexive must have disjoint reference
(long-distance bound by an even higher subject or a free logophor, see
\citealt{kozhukhar2019} [this volume]), as indicated in interpretation (ii).
In the latter case, the first person pronoun must refer to the actual
speaker, which is not possible in this sentence, since unshifted first
person pronouns do not trigger verbal person marking, hence the
ungrammaticality of reading (iii). Should the finite verb in the
embedded report be in the unmarked form \emph{buˤrʡaqib}, reading (iii)
becomes available.

In interrogative embedded clauses, a similar distribution is observed:
only arguments co-valued with the addressee of the reporter (expressed
as the addressee argument of the matrix verb) show overt person marking
on the embedded verb, whereas unshifted second person pronouns cannot
trigger overt person marking.

\ea % (97)
\gll \emph{rasuj-ni} \emph{madina-ze} \emph{xarba-ib} \emph{ħu} \emph{kuda} \{\emph{d-aš-as-a} / *\emph{d-aš-an-a}\} \emph{har} \emph{barħi} \emph{ile.}\\
Rasul-\textsc{erg} Madina-\textsc{inter(lat)} ask:\textsc{pf}-\textsc{aor} you.sg(\textsc{abs}) where \textsc{f1}-walk:\textsc{ipf}-\textsc{hab}.\textsc{ego}-\textsc{q} / \textsc{f1}-walk:\textsc{ipf}-\textsc{hab}-\textsc{q} every day \textsc{comp}\\
\glt `Rasul asked Madina\textsubscript{i} where she\textsubscript{i} goes
every day.'

\ex \label{ex:7-98}
\gll \emph{rasuj-ni} \emph{madina-ze} \emph{xarba-ib} \emph{ħu} \emph{kuda} \{\emph{w-aš-an-a} / *\emph{w-aš-as-a}\} \emph{har} \emph{barħi} \emph{ile.}\\
Rasul-\textsc{erg} Madina-\textsc{inter(lat)} ask:\textsc{pf}-\textsc{aor} you.sg(\textsc{abs}) where \textsc{m}-walk:\textsc{ipf}-\textsc{hab}-\textsc{q} / \textsc{m}-walk:\textsc{ipf}-\textsc{hab}.\textsc{ego}-\textsc{q} every day \textsc{comp}\\
\glt `Rasul asked Madina where you go every day.'
\z

Again, in examples like (\ref{ex:7-98}), the second person pronoun in the embedded
clause may only be interpreted as disjoint from the matrix addressee
argument, due to a gender mismatch between the feminine gender of the
matrix addressee and the masculine gender agreement on the embedded
verb. When this is the case, overt person agreement is ungrammatical
with a second person pronoun in interrogative embedded clauses.

For the sake of completeness, a few words are in order about the
availability of indexical shift and agreement shift. As mentioned above, both
are only possible in finite complement clauses with the complementizer
\emph{ile} under verbs of speech and thought, but not in other types of
complements. The examples below demonstrate that indexical shift and
agreement shift are possible in the finite complement of the verb
\emph{arʁes} `hear', but not in the factive non-finite (nominalized)
complement with the same verb.

\ea % (99)
\gll \emph{rasuj-ze} \emph{arʁ-ib} \emph{di-la} \emph{mašin} \emph{b-uˤrʡ-uˤb} \emph{ile.}\\
Rasul+\textsc{obl}-\textsc{inter(lat)} understand:\textsc{pf}-\textsc{aor} I-\textsc{gen} car(\textsc{abs}) \textsc{n}-break:\textsc{pf}-\textsc{aor} \textsc{comp}\\

\ea % a.
`Rasul\textsubscript{i} realized that my\textsubscript{j} car was
broken.' (unshifted reading of the 1\textsuperscript{st} person pronoun)

\ex % b.
`Rasul\textsubscript{i} realized that his\textsubscript{i} car was
broken.' (shifted reading of the 1\textsuperscript{st} person pronoun)
\z

\ex % (100)
\gll \emph{rasuj-ze} \emph{arʁ-ib} \emph{di-la} \emph{mašin} \emph{b-uˤrʡ-uˤb-deš} \emph{ile.}\\
Rasul+\textsc{obl}-\textsc{inter(lat)} understand:\textsc{pf}-\textsc{aor} I-\textsc{gen} car(\textsc{abs}) \textsc{n}-break:\textsc{pf}-\textsc{aor}-\textsc{nmlz} \textsc{comp}\\

\ea %  a.
`Rasul\textsubscript{i} realized that my\textsubscript{j} car was
broken.' (unshifted reading of the 1\textsuperscript{st} person pronoun)

\ex % b.
*'Rasul\textsubscript{i} realized that his\textsubscript{i} car was
broken.' (shifted reading of the 1\textsuperscript{st} person pronoun)
\pagebreak[3]
\z
\z

Whether or not a matrix verb combines with \emph{ile}-complements is not
lexically determined, but rather depends on the semantics of the matrix
verb (speech or thought report). This is clearly seen in cases like
those shown in the following examples.

\ea\label{ex:7-101}
\gll \emph{rasuj-ze} \emph{b-ah-ur} \emph{abaj} \emph{iz-uwe} \{\emph{le-r-deš} / *\emph{le-r} \emph{ile}\}.\\
Rasul-\textsc{inter(lat)} \textsc{n}-know:\textsc{pf}-\textsc{aor} mother(\textsc{abs}) be.sick:\textsc{ipf}-\textsc{cvb.ipfv} \textsc{aux}-\textsc{f}-\textsc{nmlz} / \textsc{aux}-\textsc{f} \textsc{comp}\\
\glt `Rasul found out that his mother was sick.'

\ex\label{ex:7-102}
\gll \emph{madina-ini} \emph{rasuj-ze} \emph{b-ah-aq-ib} \emph{abaj} \emph{iz-uwe} \{\emph{le-r-deš} / \emph{le-r} \emph{ile}\}.\\
Madina-\textsc{erg} Rasul-\textsc{inter(lat)} \textsc{n}-know:\textsc{pf}-\textsc{caus}-\textsc{aor} mother(\textsc{abs}) be.sick:\textsc{ipf}-\textsc{cvb.ipfv} \textsc{aux}-\textsc{f}-\textsc{nmlz} / \textsc{aux}-\textsc{f} \textsc{comp}\\
\glt `Madina let Rasul know that their mother was sick.'
\z


Example (\ref{ex:7-101}) shows that the factive matrix verb \emph{bahes} `know'
does not combine with finite \emph{ile}-complements. In (\ref{ex:7-102}), the
causative \emph{bahaqas} of the same verb is normally understood as
denoting a speech act (`let know, inform'), and is therefore compatible
with an \emph{ile}-complement.

% 4.
\section{Reciprocals}\label{Reciprocals}

Reciprocal pronouns consist of two instances of the numeral \emph{ca}
`one' adjacent to one another.

\ea\label{ex:7-103}
\gll \emph{uz-be-ni} \emph{ca-li-ni} \emph{ca-li-če} \emph{b-aʔ-aq-ib.}\\
brother-\textsc{pl}-\textsc{erg} one-\textsc{obl}-\textsc{erg} one-\textsc{obl}-\textsc{super}(\textsc{lat}) \textsc{n}-hit-\textsc{lv}:\textsc{pf}-\textsc{aor}\\
\glt `The brothers hit each other.'
\z

As can be seen in the example above, the two components of the
reciprocal bear independent case marking. One component is always in the
case of the subject, while the other component bears the case of the
second argument of the reciprocal construction. The distribution of case
marking on the two components of the reciprocal pronoun
depends on the particular argument/case combination.

Absolutive case, whether it corresponds to the subject or to the direct
object, is always marked on the second component of the reciprocal. The
first component therefore bears the case of the other argument
participating in the reciprocal construction.


\ea\label{ex:7-104}
\gll \emph{uz-be} \emph{ca-li-če} \emph{ca} \emph{ħule} \emph{b-iz-ur.}\\
brother-\textsc{pl}(\textsc{abs}) one-\textsc{obl}-\textsc{super(lat)} one(\textsc{abs}) look \textsc{hpl}-\textsc{lv}:\textsc{pf}-\textsc{aor}\\
\glt `The brothers looked at each other.'

\ex\label{ex:7-105}
\gll \emph{uz-be-ni} \emph{ca-li-ni} \emph{ca} \emph{b-aˤbʡ-ib.}\\
brother-\textsc{pl}-\textsc{erg} one-\textsc{obl}-\textsc{erg} one(\textsc{abs}) \textsc{hpl}-kill:\textsc{pf}-\textsc{aor}\\
\glt `The brothers killed each other.'
\z

In (\ref{ex:7-104}), the intransitive verb \emph{ħule \textup{\textsc{cl}}-izes} `look' is used in the
reciprocal construction. The absolutive case of the subject is marked on
the second part of the reciprocal, whereas the case of the oblique
argument is marked on the first part. In (\ref{ex:7-105}), the transitive verb
\emph{baˤbʡas} `kill' participates in the reciprocal construction.
Again, the absolutive case, which is the case of the direct object here,
is marked on the second part of the reciprocal pronoun, while the
ergative case of the transitive subject is marked on the first part.

When no absolutive argument participates in a reciprocal construction,
the case marking on the reciprocal pronoun is determined by structural
prominence. The first component is in the case of the higher argument,
while the second component is in the case of the lower argument, as in
(\ref{ex:7-103}) above and in the following examples.

\ea\label{ex:7-106}
\gll \emph{ul-e-jni} \emph{ca-li-ni} \emph{ca-li-s} \emph{kumak} \emph{b-aq'-ib.}\\
child-\textsc{pl}-\textsc{erg} one-\textsc{obl}-\textsc{erg} one-\textsc{obl}-\textsc{dat} help(\textsc{abs}) \textsc{n}-do:\textsc{pf}-\textsc{aor}\\
\glt `The children helped one another.'

\ex \label{ex:7-107}
\gll \emph{ul-e-jni} \emph{ca-li-ni} \emph{ca-li-ze-la} \emph{arc} \emph{ar-is-an.}\\
child-\textsc{pl}-\textsc{erg} one-\textsc{obl}-\textsc{erg} one-\textsc{obl}-\textsc{inter}-\textsc{el} money(\textsc{abs}) away-take:\textsc{ipf}-\textsc{hab}\\
\glt `The children take money from one another.'
\z

The case of the overt antecedent NP also depends on the presence of an
absolutive argument in the construction. As a rule, the overt antecedent
bears the case of a more structurally prominent argument. Examples (\ref{ex:7-103}),
(\ref{ex:7-105}), (\ref{ex:7-106}), and (\ref{ex:7-107}) above show that in the reciprocal construction
with transitive verbs, the overt antecedent is in the ergative case.
Example (\ref{ex:7-104}) shows that the reciprocal construction with intransitive
verbs requires an overt antecedent in the absolutive case. Example
(\ref{ex:7-108}) below illustrates the reciprocal construction with locative
subject verbs.
\pagebreak[3]

\ea \label{ex:7-108}
\gll \emph{uz-be-ze} \emph{ca-li-ze} \emph{ca} \{\emph{g-ub} / \emph{b-ah-ur} / \emph{b-arg-ib} / \emph{qum-art-ur}\}.\\
brother-\textsc{pl}-\textsc{inter(lat)} one-\textsc{obl}-\textsc{inter(lat)} one(\textsc{abs}) see:\textsc{pf}-\textsc{aor} / \textsc{hpl}-know:\textsc{pf}-\textsc{aor} / \textsc{hpl}-find:\textsc{pf}-\textsc{aor} / forget-\textsc{lv}:\textsc{pf}-\textsc{aor}\\
\glt `The brothers \{saw~/ recognized~/ found~/ forgot\} each other.'
\z

The only exception to this rule comes with dative subject verbs, where
absolutive marking of the overt antecedent is preferred over dative
marking.

\ea\label{ex:7-109}
\gll \{\emph{it-ti} / \textsuperscript{??}\emph{it-ti-li-s}\} \emph{ca-li-s} \emph{ca} \emph{b-ig-uwe} \emph{le-b.}\\
{this}-\textsc{pl}(\textsc{abs}) / {this}-\textsc{pl}-\textsc{obl}-\textsc{dat} one-\textsc{obl}-\textsc{dat} one(\textsc{abs}) \textsc{hpl}-love:\textsc{ipf}-\textsc{cvb.ipfv} \textsc{aux}-\textsc{hpl}\\
\glt `They love each other.'

\ex % (110)
\gll \{\emph{it-ti} / \textsuperscript{??}\emph{it-ti-li-s}\} \emph{ca-li-s} \emph{ca} \emph{eba} \emph{b-uh-ub.}\\
{this}-\textsc{pl}(\textsc{abs}) / {this}-\textsc{pl}-\textsc{obl}-\textsc{dat} one-\textsc{obl}-\textsc{dat} one(\textsc{abs}) bored \textsc{hpl}-become:\textsc{pf}-\textsc{aor}\\
\glt `They got bored with each other.'
\z

The absolutive marking of the overt antecedent is also possible in
reciprocal constructions with two core arguments of two-place verbs.

\ea % (111)
\gll \emph{uz-be} \emph{ca-li-ni} \emph{ca} \emph{b-aˤbʡ-ib.}\\
brother-\textsc{pl}(\textsc{abs}) one-\textsc{obl}-\textsc{erg} one(\textsc{abs}) \textsc{hpl}-kill:\textsc{pf}-\textsc{aor}\\
\glt `The brothers killed each other.'

\ex\label{ex:7-112}
\gll \emph{uz-be} \emph{ca-li-ze} \emph{ca} \{\emph{g-ub} / \emph{b-ah-ur} / \emph{b-arg-ib} / \emph{qum-art-ur}\}.\\
brother-\textsc{pl} one-\textsc{obl}-\textsc{inter(lat)} one(\textsc{abs}) see:\textsc{pf}-\textsc{aor} / \textsc{hpl}-know:\textsc{pf}-\textsc{aor} / \textsc{hpl}-find:\textsc{pf}-\textsc{aor} / forget-\textsc{lv}:\textsc{pf}-\textsc{aor}\\
\glt `The brothers \{saw~/ recognized~/ found~/ forgot\} each other.'
\z

Therefore, we have two possibilities for antecedent marking in constructions
featuring the two core arguments of two-place verbs. The
antecedent can be marked for the morphological case of the higher argument
(i.e.\ the subject) or for the absolutive case, even though the
absolutive is the morphological case of the lower argument (i.e.\ the
direct object) in such configurations. With dative subject verbs, the
first\pagebreak[3] option is severely disfavored and the second option is preferred,
while with other two-place verbs (transitive and locative subject), the
two options are equally acceptable.

No other reciprocal construction allows the overt antecedent in the case
of a lower argument. Example (\ref{ex:7-113}) illustrates this for a combination of
the intransitive subject and an oblique argument, cf.\ (\ref{ex:7-104}). Example
(\ref{ex:7-114}) shows a reciprocal construction with a transitive subject and a
dative recipient, cf.\ (\ref{ex:7-106}).

\ea \label{ex:7-113}
\gll *\emph{uz-be-če} \emph{ca-li-če} \emph{ca} \emph{ħule} \emph{b-iz-ur.}\\
brother-\textsc{pl}-\textsc{super(lat)} one-\textsc{obl}-\textsc{super(lat)} one(\textsc{abs}) look \textsc{hpl}-\textsc{lv}:\textsc{pf}-\textsc{aor}\\
\glt `The brothers looked at each other.'

\ex \label{ex:7-114}
\gll *\emph{ul-e-s} \emph{ca-li-ni} \emph{ca-li-s} \emph{kumak} \emph{b-aq'-ib.}\\
child-\textsc{pl}-\textsc{dat} one-\textsc{obl}-\textsc{erg} one-\textsc{obl}-\textsc{dat} help(\textsc{abs}) \textsc{n}-do:\textsc{pf}-\textsc{aor}\\
\glt `The kids helped one another.'
\z

In transitive constructions where the absolutive direct object does not
participate in the reciprocal relation, the absolutive case cannot be
used to mark the overt antecedent either.

\ea % (115)
\gll *\emph{ul-e} \emph{ca-li-ni} \emph{ca-li-s} \emph{kumak} \emph{b-aq'-ib.}\\
child-\textsc{pl}(\textsc{abs}) one-\textsc{obl}-\textsc{erg} one-\textsc{obl}-\textsc{dat} help(\textsc{abs}) \textsc{n}-do:\textsc{pf}-\textsc{aor}\\
\glt `The kids helped one another.'
\z

Gender agreement in reciprocal constructions functions according to the
general rule of agreement with the absolutive argument. In structures
with an overt absolutive NP this is straightforward, as shown in
examples (\ref{ex:7-104}) and (\ref{ex:7-109}) through (\ref{ex:7-112}). In structures with no overt
absolutive NP, as in (\ref{ex:7-105}) and (\ref{ex:7-108}), the verb shows the gender and
number features of the overt antecedent.

Person agreement also works as usual in constructions where the overt
antecedent is in the morphological case of the subject; that is, first
person intransitive absolutive, transitive ergative, and locative
subjects trigger overt person marking on the finite verb.

\ea % (116)
\gll \emph{nuša} \emph{ca-li-če} \emph{ca} \emph{ħule} \emph{b-iz-ur-ra.}\\
we(\textsc{abs}) one-\textsc{obl}-\textsc{super(lat)} one(\textsc{abs}) look \textsc{hpl}-\textsc{lv}:\textsc{pf}-\textsc{aor}-\textsc{ego}\\
\glt `We looked at each other.'

\ex % (117)
\gll \emph{nuša-jni} \emph{ca-li-ni} \emph{ca} \emph{b-iˤbʡ-iša.}\\
we-\textsc{erg} one-\textsc{obl}-\textsc{erg} one(\textsc{abs}) \textsc{hpl}-kill:\textsc{ipf}-\textsc{fut}.\textsc{ego}\\
\glt `We will kill each other.'

\ex % (118)
\gll \emph{nuša-ze} \emph{ca-li-ze} \emph{ca} \{\emph{g-ub-ra} / \emph{b-ah-ur-ra}\}.\\
we-\textsc{pl}-\textsc{inter(lat)} one-\textsc{obl}-\textsc{inter(lat)} one(\textsc{abs}) see:\textsc{pf}-\textsc{aor}-\textsc{ego} / \textsc{hpl}-know:\textsc{pf}-\textsc{aor}-\textsc{ego}\\
\glt `We \{saw~/ recognized\} each other.'
\z

In structures with the overt antecedent in the absolutive case, as in
(\ref{ex:7-109}) through (\ref{ex:7-112}), first person pronouns also trigger obligatory
person marking.

\ea % (119)
\gll \emph{nuša} \emph{ca-li-ni} \emph{ca} \emph{b-iˤbʡ-iša.}\\
we(\textsc{abs}) one-\textsc{obl}-\textsc{erg} one(\textsc{abs}) \textsc{hpl}-kill:\textsc{ipf}-\textsc{fut}.\textsc{ego}\\
\glt `We will kill each other.'

\ex % (120)
\gll \emph{nuša} \emph{ca-li-ze} \emph{ca} \{\emph{g-ub-ra} / \emph{b-ah-ur-ra}\}\\
we(\textsc{abs}) one-\textsc{obl}-\textsc{inter(lat)} one(\textsc{abs}) see:\textsc{pf}-\textsc{aor}-\textsc{ego} / \textsc{hpl}-know:\textsc{pf}-\textsc{aor}-\textsc{ego}\\
\glt `We \{saw~/ recognized\} each other.'
\z

The reciprocal construction with the absolutive marking of the
antecedent thus behaves like an intransitive structure with respect to
person agreement.

% 5.
\section[Causative construction]{Causative construction\protect\footnotemark}\label{Causative-construction}
\footnotetext{The description of case marking in
  causative constructions in this section is based on \citet{ageeva2014}.}

Morphologically, the causative construction is formed by means of the
suffix \mbox{\emph{-aq-}} (\emph{-aχaq-}) attached to an aspectual stem of the
causativized verb, as described by \citet{daniel2019} [this volume]. Syntactically, the
causative morpheme introduces an additional participant which is
interpreted as the causer of the event described by the lexical stem.
The causer is always marked by ergative case. Case marking of the causee
depends on the class of the causativized verb. Absolutive subjects of
intransitive verbs always remain in the absolutive case. The causative
construction based on an intransitive verb thus features two arguments:
the ergative causer and the absolutive causee, as with regular
transitive verbs.

\ea % (121)
\ea % a.
\gll \emph{ʡali} \emph{w-alħ-un.}\\
Ali(\textsc{abs}) \textsc{m}-wake.up:\textsc{pf}-\textsc{aor}\\
\glt `Ali woke up.'

\ex % b.
\gll \emph{pat'imat-ini} \emph{ʡali} \emph{w-alħ-aq-ib.}\\
Patimat-\textsc{erg} Ali(\textsc{abs}) \textsc{m}-wake.up:\textsc{pf}-\textsc{caus}-\textsc{aor}\\
\glt `Patimat woke up Ali.'
\z
\z

Ergative subjects of transitive verbs obligatorily receive locative
(inter-lative) marking in the causative construction. Case marking of
the causee with transitive causativized verbs does not depend on the
degree of agentivity. Both agentive and non-agentive transitive causees
are in the inter-lative.

\ea % (122)
\ea % a.
\gll \emph{ʡali-ni} \emph{ʁarʁa} \emph{b-alc'-un.}\\
Ali-\textsc{erg} stone(\textsc{abs}) \textsc{n}-pick.up:\textsc{pf}-\textsc{aor}\\
\glt `Ali picked up a stone.'

\ex % b.
\gll \emph{pat'imat-ini} \{\emph{ʡali-ze} / *\emph{ʡali-ni}\} \emph{ʁarʁa} \emph{b-alc'-aq-ib.}\\
Patimat-\textsc{erg} Ali-\textsc{inter(lat)} / Ali-\textsc{erg} stone(\textsc{abs}) \textsc{m}-pick.up:\textsc{pf}-\textsc{caus}-\textsc{aor}\\
\glt `Patimat made Ali pick up a stone.'
\z

\ex % (123)
\ea % a.
\gll \emph{ħark'ʷ-i-ni} \emph{urculi} \emph{d-erʁ-ib.}\\
river-\textsc{obl}-\textsc{erg} wood(\textsc{abs}) \textsc{npl}-sweep.away:\textsc{pf}-\textsc{aor}\\
\glt `The river swept away the wood.'

\ex % b.
\gll \emph{rasuj-ni} \{\emph{ħark'ʷi-ze} / \textsuperscript{???}\emph{ħark'ʷ-ini}\} \emph{urculi} \emph{d-erʁ-aq-ib.}\\
Rasul+\textsc{obl}-\textsc{erg} river-\textsc{inter(lat)} / river-\textsc{erg} wood(\textsc{abs}) \textsc{npl}-sweep.away:\textsc{pf}-\textsc{caus}-\textsc{aor}\\
\glt `Rasul floated the wood down the river.' (literally: `Rasul made the river
sweep away the wood.')
\z
\z

Locative subjects of the verbs `see', `hear, understand', `find',
`know', and `forget' are marked with inter-lative case when they occur
as a causee in the causative construction. This is the same marking as in
the baseline construction.

\ea % (124)
\gll \emph{rasuj-ni} \emph{di-ze} \emph{sune-la-l} \emph{qali} \emph{gʷ-aχaq-ib.}\\
Rasul+\textsc{obl}-\textsc{erg} I-\textsc{inter(lat)} self-\textsc{gen}-\textsc{emph} house(\textsc{abs}) see:\textsc{pf}-\textsc{caus}-\textsc{aor}\\
\glt `Rasul showed me his house.'

\ex % (125)
\gll \emph{t'ahil-li} \emph{di-ze} \emph{χabar} \emph{b-ah-aq-ib.}\\
Tahir-\textsc{erg} I-\textsc{inter(lat)} news(\textsc{abs}) \textsc{n}-know:\textsc{pf}-\textsc{caus}-\textsc{aor}\\
\glt `Tahir let me know the news.'

\ex % (126)
\gll \emph{rasuj-ni} \emph{di-ze} \emph{dars} \emph{arʁ-aq-ib.}\\ 
Rasul+\textsc{obl}-\textsc{erg} I-\textsc{inter(lat)} lesson(\textsc{abs}) understand:\textsc{pf}-\textsc{caus}-\textsc{aor}\\
\glt `Rasul explained the lesson to me.'

\ex % (127)
\gll \emph{ʡali-ni} \emph{di-ze} \emph{urx-ne} \emph{d-arg-aq-ib.}\\ 
Ali-\textsc{erg} I-\textsc{inter(lat)} key-\textsc{pl}(\textsc{abs}) \textsc{npl}-find:\textsc{pf}-\textsc{caus}-\textsc{aor}\\
\glt `Ali made me find the keys.'

\ex % (128)
\gll \emph{ʡali-ni} \emph{di-ze} \emph{hel} \emph{dehʷ} \emph{qum-art-aq-ib.}\\
Ali-\textsc{erg} I-\textsc{inter(lat)} {this} word(\textsc{abs}) forget-\textsc{lv}:\textsc{pf}-\textsc{caus}-\textsc{aor}\\
\glt `Ali made me forget that word.'
\z

It is not quite clear whether the locative case of the causee in
causative constructions with locative subject verbs reflects the
inter-lative subject marking assigned by the lexical verb or the
inter-lative causee marking assigned in the causative construction.

Causatives of two locative subject verbs exhibit special behavior as
they can denote a situation with no additional causer of the event.
Instead, the experiencer subject acquires a higher degree of agentivity
and is marked by ergative case, cf.\ examples (\ref{ex:7-21}) and (\ref{ex:7-29}) above.

\ea % (129)
\gll \emph{ʡali-ni} \emph{q'urʔan} \emph{b-alh-aq-uwe} \emph{le-b.}\\
Ali-\textsc{erg} Qur'an(\textsc{abs}) \textsc{n}-know:\textsc{ipf}-\textsc{caus}-\textsc{cvb.ipfv} \textsc{aux}-\textsc{n}\\
\glt `Ali is studying the Qur'an.'

\ex % (130)
\gll \emph{ʡali-ni} \emph{uzi} \emph{qum-art-aq-ib.}\\
Ali-\textsc{erg} brother(\textsc{abs}) forget-\textsc{lv}:\textsc{pf}-\textsc{caus}-\textsc{aor}\\
\glt `Ali forgot his brother (as a result of a conscious intention to do
so).'
\z

When a dative subject verb is causativized, the experiencer participant
can either remain in the dative, as in the original construction, or
bear inter-lative marking assigned to the causee in the causative
construction.

\ea % (131)
\ea % a.
\gll \emph{nab} \emph{it} \emph{dehʷ} \emph{urče} \emph{b-ik-ib.}\\
I(\textsc{dat}) {this} word(\textsc{abs}) in.heart(\textsc{lat}) \textsc{n}-happen:\textsc{pf}-\textsc{aor}\\
\glt `I recalled that word.'

\ex % b.
\gll \emph{abaj-ni} \{\emph{di-ze} / \emph{nab}\} \emph{it} \emph{dehʷ} \emph{urče} \emph{b-ik-aq-ib.}\\
mother-\textsc{erg} I-\textsc{inter(lat)} / I(\textsc{dat}) {this} word(\textsc{abs}) in.heart(\textsc{lat}) \textsc{n}-happen:\textsc{pf}-\textsc{caus}-\textsc{aor}\\
\glt `Mother reminded me of that word.'
\z
\z

The difference in interpretation between the two variants of causee
marking relates to the degree of control exhibited by the causer over
the caused situation. Dative marking implies a lesser degree of
involvement of the causer, while inter-lative marking indicates a
more direct causation on the part of the causer.

The causative form of the verb \emph{biges} `want, love' does not
normally have a causative interpretation. Neither the number of
arguments nor their case marking changes. The semantics is usually
conveyed as `like' rather than `love' (as is the case with the underived
forms of \emph{biges}).

\ea % (132)
\gll \emph{nab} \emph{it} \emph{dursi} \emph{d-ig-aq-uwe} \emph{le-r.}\\
I(\textsc{dat}) {this} girl(\textsc{abs}) \textsc{f1}-love:\textsc{ipf}-\textsc{caus}-\textsc{cvb.ipfv} \textsc{aux}-\textsc{f}\\
\glt `I like this girl.'
\z

The causative reading of the causative form of the verb \emph{biges}
`want, love' is also accepted by many speakers, though not by all of
them, and often not without hesitation. As in causatives of other dative
subject verbs, the causee can be marked by either dative or inter-lative
case (with no sharp interpretational differences between the two
variants).

\ea % (133)
\gll \emph{adaj-ni} \{\emph{di-ze} / \emph{\textsuperscript{?}nab}\} \emph{it} \emph{dursi} \emph{d-ig-aq-uwe} \emph{le-r.}\\
father-\textsc{erg} I-\textsc{inter(lat)} / I(\textsc{dat}) {this} girl(\textsc{abs})
\textsc{f1}-love:\textsc{ipf}-\textsc{caus}-\textsc{cvb.ipfv} \textsc{aux}-\textsc{f}\\
\glt `Father makes me love this girl.'
\z

Gender and person agreement in the causative construction follows the
rules operative in transitive clauses. Gender agreement on the lexical
verb is always with the absolutive argument. Gender agreement on the
auxiliary in progressive verb forms is also with the absolutive argument.

\ea % (134)
\ea % a.
\gll \emph{pat'imat-ini} \emph{ʡali} \emph{w-alħ-aq-ib.}\\
Patimat-\textsc{erg} Ali(\textsc{abs}) \textsc{m}-wake.up:\textsc{pf}-\textsc{caus}-\textsc{aor}\\
\glt `Patimat woke up Ali.'

\ex % b.
\gll \emph{ʡali-ni} \emph{pat'imat} \emph{d-alħ-aq-ib.}\\
Ali-\textsc{erg} Patimat(\textsc{abs}) \textsc{f1}-wake.up:\textsc{pf}-\textsc{caus}-\textsc{aor}\\
\glt `Ali woke up Patimat.'
\z

\ex\label{ex:7-135}
\ea % a.
{\glossglue0pt
\gll \emph{nu-ni} \emph{urši-li-ze} \emph{inc} \emph{b-uk-aq-uwe} \rlap{\emph{le-b.}}\\
I-\textsc{erg} boy-\textsc{obl}-\textsc{inter(lat)} apple(\textsc{abs}) \textsc{n}-eat:\textsc{ipf}-\textsc{caus}-\textsc{cvb.ipfv} \rlap{\textsc{aux}-\textsc{n}}\\
\glt `I am making the boy eat an apple.'

}
\ex % b.
\gll \emph{nu-ni} \emph{urši-li-ze} \emph{inc-be} \emph{d-uk-aq-uwe} \emph{le-r.}\\
I-\textsc{erg} boy-\textsc{obl}-\textsc{inter(lat)} apple-\textsc{pl}(\textsc{abs}) \textsc{n.pl}-eat:\textsc{ipf}-\textsc{caus}-\textsc{cvb.ipfv} \textsc{aux}-\textsc{n.pl}\\
\glt `I am making the boy eat apples.'
\z
\z


Person agreement is controlled by the ergative causer according to the
rules described above in \sectref{intransitive-transitive-and-locative-subject-verbs-in-synthetic-indicative-forms} and \sectref{agreement-in-the-present-progressive}. This includes the
restriction on overt marking in the Present Progressive, as shown in
(\ref{ex:7-135}). The inter-lative causee or the absolutive argument can never
control person agreement.

\ea % (136)
\gll \emph{nu-ni} \emph{c'a} \{\emph{d-uš-aq-i-ra} / *\emph{d-uš-aq-ib}\}.\\
I-\textsc{erg} fire(\textsc{abs}) \textsc{npl}-die.out:\textsc{pf}-\textsc{caus}-\textsc{aor}-\textsc{ego} / \textsc{npl}-die.out:\textsc{pf}-\textsc{caus}-\textsc{aor}\\ 
\glt `I extinguished the fire.'

\ex % (137)
\gll \emph{pat'imat-ini} \emph{nu} \{\emph{w-alħ-aq-ib} / *\emph{w-alħ-aq-i-ra}\}.\\
Patimat-\textsc{erg} I(\textsc{abs}) \textsc{m}-wake.up:\textsc{pf}-\textsc{caus}-\textsc{aor} / \textsc{m}-wake.up:\textsc{pf}-\textsc{caus}-\textsc{aor}-\textsc{ego}\\
\glt `Patimat woke me up.'

\ex % (138)
\gll \emph{pat'imat-ini} \emph{di-ze} \emph{ʁarʁa} \{\emph{b-alc'-aq-ib} / *\emph{b-alc'-aq-i-ra}\}.\\
Patimat-\textsc{erg} I-\textsc{inter(lat)} stone(\textsc{abs}) \textsc{n}-pick.up:\textsc{pf}-\textsc{caus}-\textsc{aor} / \textsc{n}-pick.up:\textsc{pf}-\textsc{caus}-\textsc{aor}-\textsc{ego}\\
\glt `Patimat made me pick up a stone.'
\z

Note, however, that despite the absence of an overt ergative argument in
causative constructions based on transitive verbs, it is possible to
show that they do contain an unexpressed ergative subject of the lexical
verb. This is seen from case marking that appears on reciprocal
pronouns. As explained in \sectref{Reciprocals} above, the two parts of the
reciprocal pronoun always bear two different morphological cases
corresponding to the case marking of the arguments in the reciprocal
relation. When used in a causative construction describing a reciprocal
relationship between the causee and the absolutive direct object, one
part of the reciprocal pronoun shows up in the ergative case, even
though no overt ergative argument appears on the surface.

\pagebreak

\ea \label{ex:7-139}
\gll \emph{madina-jni} \{\emph{ul-e} / \emph{ul-e-ze}\} \emph{ca-li-ni} \emph{ca} \emph{b-az-aq-ib.}\\
Madina-\textsc{erg} child-\textsc{pl}(\textsc{abs}) / child-\textsc{pl}-\textsc{inter(lat)} one-\textsc{obl}-\textsc{erg} one(\textsc{abs}) \textsc{hpl}-wash:\textsc{pf}-\textsc{caus}-\textsc{aor}\\
\glt `Madina made the kids wash one another.'
\z

Note that in example (\ref{ex:7-139}), the causee in the causativized reciprocal
construction of the transitive verb can be expressed by the absolutive
or by the inter-lative. This corresponds to two possibilities observed
in non-causativized reciprocals: (i) the overt subject is marked by the
absolutive, and the whole construction behaves as an intransitive
structure, or (ii) the overt subject is marked by the ergative, and the
whole reciprocal construction is a transitive structure. Under
causativization, the intransitive variant (i) of the reciprocal
construction yields absolutive marking of the causee, whereas the
transitive variant (ii) of the reciprocal construction yields 
inter-lative marking of the causee.

% 6.
\section{The biabsolutive construction}\label{Biabsolutive-construction}

Periphrastic verbal forms with durative semantics (present and past
progressive) allow for an alternative layout of argument case marking
with transitive verbs. Instead of the standard transitive pattern with
an ergative subject and an absolutive object, transitive verbs can
participate in the \emph{biabsolutive construction}, where
both the subject and the direct object are expressed in the absolutive
case.\footnote{See \citet{forker2012} for an overview of the biabsolutive
  across Nakh-Daghestanian. \citet{gagliardi-etal2014} present a minimalist
  analysis of the biabsolutive construction in the Nakh-Daghestanian
  languages Lak and Tsez. Harris and Campbell discuss the diachrony of
  the biabsolutive construction (\citeyear[187–189]{harris-campbell1995}).} Changes in argument
case marking are accompanied by a change in gender agreement on the
auxiliary, which is controlled by the absolutive subject; gender agreement
of the lexical verb is invariably controlled by the absolutive direct
object.

\ea % (140)
\gll Q: \emph{sija} \emph{b-iq'-uwe} \emph{le-w-a} \emph{rasul?}\\
{} what(\textsc{abs}) \textsc{n}-do:\textsc{ipf}-\textsc{cvb.ipfv} \textsc{aux}-\textsc{m}-\textsc{q} Rasul(\textsc{abs})\\
\glt `What is Rasul doing?'

\gll A: \emph{rasul} \emph{kung} \emph{luč'-uwe} \emph{le-w.}\\
{} Rasul(\textsc{abs}) book(\textsc{abs}) read:\textsc{ipf}-\textsc{cvb.ipfv} \textsc{aux}-\textsc{m}\\
\glt `Rasul is reading a book.'
\z

Unlike ergative constructions with periphrastic forms, the biabsolutive
construction shows no restrictions on person agreement of the absolutive
subjects. Overt person marking with the absolutive subject is
obligatory, as shown in (\ref{ex:7-141}a), cf.\ the minimally different example
(\ref{ex:7-141}b), where the ergative subject cannot agree with the finite verb.

\ea\label{ex:7-141}
\ea % a.
\gll \emph{nu} \emph{kung} \emph{luč'-uwe} \emph{le-w-ra}.\\
I(\textsc{abs}) book(\textsc{abs}) read:\textsc{ipf}-\textsc{cvb.ipfv} \textsc{aux}-\textsc{m}-\textsc{ego}\\
\glt `I am reading a book.'

\ex % b.
\gll \emph{nu-ni} \emph{kung} \emph{luč'-uwe} \emph{le-b\(*-ra\)}.\\
I-\textsc{erg} book(\textsc{abs}) read:\textsc{ipf}-\textsc{cvb.ipfv} \textsc{aux}-\textsc{m}-\textsc{ego}\\
\glt `I am reading a book.'
\z
\z

Unlike what is attested in related languages \citep{forker2012}, there seem
to be no observable differences in semantics between the ergative and
biabsolutive alignment of the transitive clause. In fact, the
biabsolutive construction is often resorted to when person
agreement with the subject fails in certain subject-object combinations
in periphrastic forms, see \sectref{agreement-in-the-present-progressive}.

Synthetic verbal forms with imperfective semantics do not allow the
biabsolutive construction.

\ea % (142)
\gll \{\emph{nu-ni} / *\emph{nu}\} \emph{kung-ane} \emph{luč'-as.}\\
I-\textsc{erg} / I(\textsc{abs}) book-\textsc{pl}(\textsc{abs}) read-\textsc{hab}.\textsc{ego}\\
\glt `I read books (every day).'

\ex % (143)
\gll \{\emph{nu-ni} / *\emph{nu}\} \emph{kung-ane} \emph{luč'-iša.}\\
I-\textsc{erg} / I(\textsc{abs}) book-\textsc{pl}(\textsc{abs}) read-\textsc{fut}.\textsc{ego}\\
\glt `I will be reading books.'
\z

Only clauses with agentive subjects normally participate in the
biabsolutive construction, whereas clauses with non-agentive subjects
are either considerably degraded or completely ungrammatical.

\ea % (144)
\gll \emph{\textsuperscript{??}ʁʷaˤr} \emph{ʁut'-be} \emph{šiš} \emph{d-uk'-aq-uwe} \emph{le-b.}\\
wind(\textsc{abs}) tree-\textsc{pl}(\textsc{abs}) move \textsc{npl}-\textsc{lv}:\textsc{ipf}-\textsc{caus}-\textsc{cvb.ipfv} \textsc{aux}-\textsc{n}\\
\glt `The wind is shaking the trees.'

\ex % (145)
\gll *\emph{c'a} \emph{qul-le} \emph{ig-uwe} \emph{le-b.}\\
fire(\textsc{abs}) house-\textsc{pl}(\textsc{abs}) burn:\textsc{ipf}-\textsc{cvb.ipfv} \textsc{aux}-\textsc{n}\\
\glt `A fire is burning the houses.'

\ex % (146)
\gll *\emph{zab} \emph{mura} \emph{d-aˤlʜʷ-aˤq-uwe} \emph{le-r.}\\
rain(\textsc{abs}) hay(\textsc{abs}) \textsc{npl}-become~wet:\textsc{ipf}-\textsc{caus}-\textsc{cvb.ipfv} \textsc{aux}-\textsc{npl}\\
\glt `The rain is making the hay wet.'
\z

Similarly, non-agentive subjects of locative-subject verbs are not
allowed to participate in the biabsolutive construction for many
speakers, though some sentences are judged to be more acceptable than others. The
acceptability of locative-subject verbs in the biabsolutive
construction may depend on semantic and pragmatic factors and requires
further investigation.

\ea % (147)
\gll *\emph{nu} \emph{sinka} \emph{irg-uwe} \emph{le-w-ra.}\\
I(\textsc{abs}) bear(\textsc{abs}) see:\textsc{ipf}-\textsc{cvb.ipfv} \textsc{aux}-\textsc{m}-\textsc{ego}\\
\glt `I see a bear.'

\ex % (148)
\gll \textsuperscript{?}*\emph{urši} \emph{d-aˤld-un-i} \emph{arc} \emph{d-urg-uwe} \emph{le-w.}\\
boy(\textsc{abs}) \textsc{npl}-hide:\textsc{pf}-\textsc{aor}-\textsc{ptcp} money(\textsc{abs}) \textsc{npl}-find:\textsc{ipf}-\textsc{cvb.ipfv} \textsc{aux}-\textsc{m}\\
\glt `The boy is finding the hidded money.'

\ex % (149)
\gll \textsuperscript{??}\emph{rasul} \emph{het} \emph{dehʷ} \emph{b-alh-uwe} \emph{le-b.}\\
Rasul(\textsc{abs}) {this} word(\textsc{abs}) \textsc{n}-know:\textsc{ipf}-\textsc{cvb.ipfv} \textsc{aux}-\textsc{n}\\
\glt `Rasul knows that word.'
\z

The dative subject verb \emph{biges} `love, want' can occasionally
participate in the biabsolutive construction.

\ea % (150)
\gll \emph{nu} \emph{het} \emph{urši} \emph{w-ig-uwe} \emph{le-l-la}.\\
I(\textsc{abs}) {this} boy(\textsc{abs}) \textsc{m}-love:\textsc{ipf}-\textsc{cvb.ipfv} \textsc{aux}-\textsc{f}-\textsc{ego}\\
\glt `I love this boy.'
\z

Despite initial appearances, the biabsolutive construction contains an
unexpressed ergative argument of the lexical verb which can be seen in
reciprocal constructions. Similar to what is found in causative
constructions, one of the two components of the reciprocal pronoun in
the biabsolutive always bears the ergative case licensed by the lexical
verb, despite the phonological absence of an ergative argument, compare
(\ref{ex:7-151}) with (\ref{ex:7-106}) above.

\pagebreak

\ea\label{ex:7-151}
\gll \emph{ul-e} \emph{ca-li-ni} \emph{ca-li-s} \emph{kumak} \emph{b-iq'-uwe} \emph{le-b}.\\
child-\textsc{pl}(\textsc{abs}) one-\textsc{obl}-\textsc{erg} one-\textsc{obl}-\textsc{dat} help(\textsc{abs}) \textsc{n}-do:\textsc{ipf}-\textsc{cvb.ipfv} \textsc{aux}-\textsc{hpl}\\
\glt `The kids help one another.'
\z

Syntactically, the biabsolutive construction may thus be analyzed as
consisting of two layers. The lower layer is headed by the lexical verb
and contains the lexical verb itself and all of its arguments in their
respective cases. The higher layer is headed by the copula and contains
the absolutive subject. The biabsolutive construction thus has two
important properties: (i) it requires the subject to have the agent
theta-role, and (ii) it includes an unexpressed ergative argument which
is obligatorily interpreted as having the same reference as the overt
absolutive subject. These two properties make the biabsolutive
construction look like an obligatory control construction. A schematic
representation of the syntactic structure of the biabsolutive
construction is given in (\ref{ex:7-152}).

\ea\label{ex:7-152}
\ea % a.
\gll [\emph{rasul\textsubscript{i}} [\textsc{pro}\textsubscript{i} \emph{kung} \emph{luč'-uwe}] \emph{le-w}].\\
Rasul(\textsc{abs}) \textsc{erg} book(\textsc{abs}) read:\textsc{ipf}-\textsc{cvb.ipfv} \textsc{cop}-\textsc{m}\\
\glt `Rasul is reading a book.'

\ex % b.
{}[\textsubscript{CopP} \textsc{np\textsubscript{abs}} [\textsubscript{vP} \textsc{pro\textsubscript{erg}} \textsc{np\textsubscript{abs}} V] \textsc{cop}]
\z
\z

The causative construction may also be transformed into a biabsolutive
construction. With causatives of intransitive verbs, the biabsolutive
construction works the same way as with biabsolutives of ordinary
transitive verbs: both the causer and the causee are in the absolutive
case. The former controls gender and person agreement on the copula,
while the latter controls gender agreement on the lexical verb.

\ea % (153)
\gll \emph{rasul} \emph{c'a} \emph{d-uš-aq-uwe} \emph{le-w.}\\ 
Rasul(\textsc{abs}) fire(\textsc{abs}) \textsc{npl}-die.out:\textsc{ipf}-\textsc{caus}-\textsc{cvb.ipfv} \textsc{cop}-\textsc{m}\\
\glt `Rasul is extinguishing the fire.'
\z

With causatives of transitive verbs, there are three case
marking options in the biabsolutive construction. One option is to mark the
causer with absolutive case, as with causatives of intransitive verbs
above. Gender and person agreement on the copula are determined by
features of the higher absolutive, in this case the causer. Example
(\ref{ex:7-154}) shows the baseline causative construction in (a) and the
biabsolutive construction with  absolutive marking of the causer in
(b).

\ea\label{ex:7-154}
\ea % a.
\gll \emph{abaj-ni} \emph{urši-li-ze} \emph{kung} \emph{luč'-aq-uwe} \emph{le-b.}\\
mother-\textsc{erg} boy-\textsc{obl}-\textsc{inter(lat)} book(\textsc{abs}) read:\textsc{ipf}-\textsc{caus}-\textsc{cvb.ipfv} \textsc{aux}-\textsc{n}\\

\ex % b.
\gll \emph{abaj} \emph{urši-li-ze} \emph{kung} \emph{luč'-aq-uwe} \emph{le-r.}\\
mother(\textsc{abs}) boy-\textsc{obl}-\textsc{inter(lat)} book(\textsc{abs}) read:\textsc{ipf}-\textsc{caus}-\textsc{cvb.ipfv} \textsc{cop}-\textsc{f}\\
\glt `Mother makes the boy read the book.'
\z
\z

The second option is to mark the causee with the absolutive case,
whereas the causer bears its usual ergative case. Again, gender and
person agreement on the copula are determined by features of the higher
absolutive, which is the causee in this case.

\addtocounter{equation}{-1}
\ea\label{ex:7-154c} 
\begin{xlist}
\exi{c.}
\gll \emph{abaj-ni} \emph{urši} \emph{kung} \emph{luč'-aq-uwe} \emph{le-w.}\\
mother-\textsc{erg} boy(\textsc{abs}) book(\textsc{abs}) read:\textsc{ipf}-\textsc{caus}-\textsc{cvb.ipfv} \textsc{cop}-\textsc{m}\\
\glt `Mother makes the boy read the book.'
\end{xlist}
\z

Finally, the third option is to mark both the causer and the causee with
absolutive case. We therefore have three absolutive arguments in the
same clause. Again, gender and person agreement on the copula is
determined by the highest absolutive, that is, the subject causer.

\addtocounter{equation}{-1}
\ea\label{ex:7-154d}
\begin{xlist}
\exi{d.}
\gll \emph{abaj} \emph{urši} \emph{kung} \emph{luč'-aq-uwe} \emph{le-r.}\\
mother(\textsc{abs}) boy(\textsc{abs}) book(\textsc{abs}) read:\textsc{ipf}-\textsc{caus}-\textsc{cvb.ipfv} \textsc{cop}-\textsc{f}\\
\glt `Mother makes the boy read the book.'
\end{xlist}
\z

The possibilities of case marking shown in (\ref{ex:7-154c}c–d) require further
investigation. In standard biabsolutive constructions as described in
this section, the absolutive marking of the transitive subject
apparently becomes available due to the presence of a second clausal
layer headed by the copula. It is not quite clear how the copula in the
progressive could license absolutive marking of the transitive
causee in (\ref{ex:7-154c}c) and, especially, the absolutive marking of both the
ergative causer and the transitive causee in (\ref{ex:7-154d}d). Any syntactic
speculation on this question, however, requires more specific
assumptions about the clause structure and mechanisms of case licensing
which lie outside the scope of the present work. I~therefore
leave this issue for another occasion.

% 7.
\section{Summary}\label{Summary}

In this chapter, I have discussed the major morphosyntactic properties of
monoclausal sentences in Mehweb, including case marking, gender and
person agreement. The paper describes the system of Mehweb verbal
(valency) classes on the basis of their arguments' morphosyntactic
behavior and ability to bind reflexive pronouns. I distinguish (i)
intransitive verbs with absolutive subjects, (ii)~transitive verbs with
ergative subjects, (iii) verbs with inter-lative subjects, (iv)~verbs
with dative subjects, and (v) one verb with inter-elative subject.
Gender agreement operates on an ergative–absolutive basis, whereas
person agreement has nominative–accusative syntax.

Mehweb person agreement is unique  within Nakh-Daghestanian in that it is sensitive to the
illocutionary force of the utterance. As in other Daghestanian
languages with person agreement, verbal person marking is also sensitive
to the syntactically introduced logophoric center, as in finite
logophoric clauses with the complementizer \emph{ile}. In such
environments, personal pronouns undergo optional indexical shift,
whereas person marking is obligatorily shifted to the perspective of the
syntactic logophoric center.

Although traditionally Mehweb person agreement is considered to be
purely subject-oriented, this chapter argues that several constructions,
such as agreement in sentences with dative subject verbs and agreement
in the Present Progressive, reveal the sensitivity of person agreement to
the person feature of the absolutive direct object.

I also describe case marking and agreement in causative and
biabsolutive constructions. Despite overall semantic and syntactic
differences between the two, they demonstrate similar behavior with
respect to the ergative subject of the lexical verb, which can still be
diagnosed when it is absent from the phonological expression, by means
of case marking on reciprocal pronouns. Finally, I identify a previously
unattested construction with three absolutive arguments.


\section*{List of abbreviations}

\begin{longtable}[l]{@{}ll@{}}
\textsc{abs}	& absolutive \\
\textsc{dir}	& motion directed towards a spatial domain \\
\textsc{aor}	& aorist \\
\textsc{atr}	& attributivizer \\
\textsc{aux}	& auxiliary \\
\textsc{caus}	& causative \\
\textsc{cl}	& gender (class) agreement slot \\
\textsc{comp}	& complementizer \\
\textsc{cop}	& copula \\
\textsc{cvb}	& converb \\
\textsc{dat}	& dative \\
\textsc{ego}	& egophoric \\
\textsc{el}	& motion from a spatial domain \\
\textsc{emph}	& emphasis (particle) \\
\textsc{erg}	& ergative \\
\textsc{ess}	& static location in a spatial domain \\
\textsc{f}	& feminine (gender agreement) \\
\textsc{f1}	& feminine (unmarried and young women gender prefix) \\
\textsc{fut}	& future \\
\textsc{gen}	& genitive \\
\textsc{hab}	& habitual (durative for verbs denoting states) \\
\textsc{hpl}	& human plural (gender agreement) \\
\textsc{indef}	& indefinite particle \\
\textsc{inf}	& infinitive \\
\textsc{inter}	& spatial domain between multiple landmarks \\
\textsc{ipf}	& imperfective (derivational base) \\
\textsc{ipft}	& imperfect \\
\textsc{lat}	& motion into a spatial domain \\
\textsc{lv}	& light verb \\
\textsc{m}	& masculine (gender agreement) \\
\textsc{n}	& neuter (gender agreement) \\
\textsc{neg}	& negation (verbal prefix) \\
\textsc{nmlz}	& nominalizer \\
\textsc{npl}	& non-human plural (gender agreement) \\
\textsc{obl}	& oblique (nominal stem suffix) \\
\textsc{pf}	& perfective (derivational base) \\
\textsc{pl}	& plural \\
\textsc{ptcp}	& participle \\
\textsc{q}	& question (interrogative particle) \\
\textsc{super}	& spatial domain on the horizontal surface of the landmark \\
\end{longtable}

\clearpage

\printbibliography[heading=subbibliography,notkeyword=this]


\end{document}

%%% Local Variables:
%%% mode: latex
%%% TeX-master: "../main"
%%% End:
