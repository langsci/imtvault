\documentclass[output=paper]{langsci/langscibook} 
\ChapterDOI{10.5281/zenodo.3402068}

% Chapter 8

\title{Specialized converbs in Mehweb}
\author{Maria Sheyanova\affiliation{National Research University Higher School of Economics, Faculty of Humanities, School of Linguistics, masha.shejanova@gmail.com}}

% Maria V. 

\abstract{This paper describes the semantic inventory and morphosyntactic properties of specialized converbs in Mehweb. The data for the description were collected during field trips to the village of Mehweb. Converbs with the following meanings are described:~anteriority, immediacy, inceptivity, simultaneity, posteriority, hypothetical conditionality, counterfactuality, concessivity (and another meaning close to concessivity), causality, purpose and graduality. A participle-based form with the semantics of a locative converb is discussed.

\emph{Keywords}: East Caucasian languages, adverbial subordination, converbs,
specialized converbs, verbal morphology.}

\begin{document}
\maketitle

% 1
\section{Introduction}

Specialized converbs are a subtype of converbs that specify the semantic
relation between the main and dependent clauses (e.g.\ purposive or
causal); for a fuller definition see \sectref{specialized-converbs}.
This paper describes the
inventory and morphosyntactic properties of specialized converbs in
Mehweb. The structure of the paper is as follows: \sectref{defining-the-terms} introduces
the subject of this study and defines the terms. \sectref{specialized-converbs-in-mehweb} describes
specialized converbs in Mehweb. \sectref{examples-of-the-forms} summarizes the morphology of
the forms in two tables. \sectref{conclusion-9} is the conclusion.

% 2.
\section{Defining the terms}\label{defining-the-terms}

% 2.1
\subsection{Converb}

According to \citet[3]{haspelmath1995a}, a converb is ``a non-finite verb
form whose main function is to mark adverbial subordination''. In
other words, one can understand converbs as ``verbal adverbs, just like
participles are verbal adjectives''. The definition of converb in
\citet{nedjalkov1995} is similar: ``as a first approximation, we can define a
converb as a verb form which depends syntactically on another verb form,
but is not its syntactic actant, i.e., it does not realize its semantic
valencies''. In an example of a canonical converb provided by Nedjalkov (\ref{ex:8:1}), the
verb form \emph{uleta-des} `crossing' is dependent on the main verb
\emph{poordus} `turned', but is not its argument.

\ea \label{ex:8:1} % (1)
{\upshape Estonian \citep{nedjalkov1995}}

\gll {uleta-des joge poordus ratsanik ulati paremale.}\\
cross-\textsc{cvb} river turned horseman always right\\
\glt `Crossing the river, the horseman always turned right.'
\z

The two definitions agree that a converb: 1) is a form of a verb, and 2)
marks adverbial subordination (i.e.\ is not a semantic argument of the
main verb).

% 2.2.
\subsection{Specialized converbs}\label{specialized-converbs}

Some languages distinguish specialized and general (contextual)
converbs. As formulated in \citet{haspelmath1995b}, unlike general converbs,
which ``leave the precise nature of the semantic link between the
clauses open'', specialized converbs have ``a quite specific adverbial
meaning'', i.e.\ establish a specific semantic relation between the
matrix and converbal clauses. The Lezgian sentences (\ref{ex:8:2}) and (\ref{ex:8:3}) taken
from \citet{haspelmath1995b} exemplify general (contextual
in \citealt{haspelmath1995b}) and specialized converbs, respectively.

\ea \label{ex:8:2} % (2)
\gll {am ajwandi-k gazet k'el-iz aq̃waz-nawa.}\\
he:\textsc{abs} balcony-\textsc{sbess} [newspaper read-\textsc{cvb}] stand-\textsc{prf}\\
\glt `He is standing on the balcony, reading a newspaper.'

\ex \label{ex:8:3} % (3)
\gll {dide annidi ǧülü-z fi-daldi muallimwil-e k'walax-na.}\\
mother Anni(\textsc{erg}) [husband-\textsc{dat} go-\textsc{pstr}] teachership-\textsc{iness} work-\textsc{aor}\\
\glt `My mother Anni worked as a teacher until she got married.'
\z

While the imperfective general converb \emph{k'el-iz} `reading' simply
describes an additional action, the posterior specialized converb
\emph{fi-daldi} `after going' expresses an action immediately preceding
the action in the main clause.

The relations expressed by specialized converbs can be of a temporal,
locative or logical nature. Converbs of logical relation
normally also have a temporal meaning. Specialized converbs are never
used in clause chaining\footnote{In \citet{haspelmath1995a} clause
  chaining is defined as a sequence in which each converb depends on the
  verb that immediately follows it and which contains only one fully
  finite final verb.} or periphrasis, which are two other common
functions of general converbs. For a discussion of general converbs in
Mehweb, see \citet{kustova2019}.

% 2.3
\subsection{Problems in defining specialized converbs}\label{problems-in-defining-specialized-converbs}

After distinguishing between specialized and general converbs, problems
remain with the definition of the category of specialized converbs. This
includes distinguishing specialized converbs from other non-finite verb
forms which introduce subordinate clauses with adverbial semantics. I
discuss three verb forms that are problematic in this resoect, namely
infinitives, participles and action nominals inflected for case.

Distinguishing an infinitive clause from converbal clauses is
problematic when the infinitive has purposive semantics. Here, the
infinitive formally fits the definition of a converb. The issue is
discussed in \citet{haspelmath1995a}. According to Haspelmath, prototypical
infinitives have a crucial property that converbs lack: they are
primarily used in complement clauses, e.g.\ as arguments of modal or
phasal verbs.

The issue of participles and action nominals is not so easy to solve. In
this paper, I consider inflected participles heading subordinate
clauses, such as \emph{wak'ibičela} in example (\ref{ex:8:4}), to be
specialized converbs.

\ea \label{ex:8:4} % (4)
\gll {ʡaˤχul w-ak'-ib-i-če-la ur-uwe le-r.}\\
guest \textsc{m}-come:\textsc{pfv}-\textsc{aor}-\textsc{ptcp}-\textsc{super}-\textsc{el} rain:\textsc{ipfv}-\textsc{cvb.ipfv} \textsc{aux}-\textsc{npl}\\
\glt `From the moment the guest arrived, it was raining.'
\z

However, I do not count case-marked action nominals, like
\emph{berχʷrilizela} in example (\ref{ex:8:5}), as specialized converbs.

\ea \label{ex:8:5} % (5)
\gll {šahal-li-če b-uħna b-erχʷ-ri-li-ze-la d-iq'-es d-aʔ-ib zab.}\\
town-\textsc{obl}-\textsc{super(lat)} \textsc{hpl}-inside(\textsc{lat}) \textsc{hpl}-enter:\textsc{pfv}-\textsc{nmlz}-\textsc{obl}-\textsc{inter}-\textsc{el} \textsc{npl}-do:\textsc{ipfv}-\textsc{inf} \textsc{npl}-begin:\textsc{pfv}-\textsc{aor} rain\\
\glt `As soon as they entered the town, it started raining.'
\z

There are two main reasons why participles and action nominals are
treated differently. The first is the fact that case-inflected
participles seem to be a more frequent source of specialized converbs
(see e.g.\ \citealt{creissels2010}). Second, in adverbial subordination,
case-inflected action nominals are in their prototypical syntactic
position, similar to nominal adjuncts. Case-inflected participles, on
the other hand, are not. In Mehweb, participles are prototypically used in
adnominal position and are not inflected for case. To be used as heads
of adverbial clauses, they thus need to change category, from an
attribute to a nominal head. Although this is a productive syntactic
process, its use to produce adverbial subordination may be considered
evidence for grammaticalization. Note that there is no systematic data available on
case inflection of headless participles; they appear to be infrequent in the corpus.

% 3.
\section{Specialized converbs in Mehweb}\label{specialized-converbs-in-mehweb}

This section provides a description of specialized converbs in Mehweb.
For each converb, I provide examples showing that the form can be used
both when the subject of the converb is coreferential with the subject of
the main clause and when the two clauses have different subjects. Some
examples also show that the converb clause can be embedded to the main
clause; this is considered to provide evidence for its subordinate status.

The section focuses on temporal converbs and converbs expressing logical
relations. There is one occurrence in the corpus of what seems to be a locative
converb, discussed at the end of the section, but this form needs
further investigation. In \sectref{examples-of-the-forms}, I provide a table showing the
availability of each converb for perfective and imperfective verbal
stems and provide examples of the relevant word forms.

% 3.1
\subsection{Temporal converbs}

% 3.1.1
\subsubsection{Anterior converb}

A converbal clause with an anterior converb expresses an event that
takes place before the event in the main clause, and can be translated as
`when P happened' or `after P happened'. There are a number of variant
markers of this converb – \emph{-arʁle, -aʁle, -aʁe, -arʁ, -aʁ, -ʁale,
-ʁela} – which are added to the participle. Speakers vary in the extent to
which they consider each variant acceptable. Only \emph{-aʁle}\pagebreak[4]
is equally accepted by all speakers. It is possible that there are slight
semantic differences between these markers, but I was unable to
establish any. In general, a speaker accepts several variants,
considering them to be interchangeable without any change in meaning.
The form is derived from participles based on both perfective and
imperfective stems. In perfective forms, a hiatus between the \emph{-i}
of the participle and the \emph{-a} of the marker is eliminated by a more
or less clearly articulated prothetic \emph{j} (not reflected in the
transcription).

\ea \label{ex:8:6} % (6)
\gll {iχi-šu, barħi b-uq-un-i-aʁle, dursi d-ak'-ib.}\\
this-\textsc{ad}(\textsc{lat}) sun \textsc{n}-enter:\textsc{pfv}-\textsc{aor}-\textsc{ptcp}-\textsc{ante} girl \textsc{f1}-come:\textsc{pfv}-\textsc{aor}\\
\glt `When the sun rose, a girl came to him.'

\ex \label{ex:8:7} % (7)
\gll {unna-li-šu b-ak'-ib-i-ʁale iχ-di cenħe b-ik-ib.}\\
neighbour-\textsc{obl}-\textsc{ad}(\textsc{lat}) \textsc{hpl}-come:\textsc{pfv}-\textsc{aor}-\textsc{ptcp}-\textsc{ante} this-\textsc{pl} together(\textsc{lat}) \textsc{hpl}-happen:\textsc{pfv}-\textsc{aor}\\
\glt `They met when they came to their neighbour.'

\ex \label{ex:8:8} % (8)
\gll {il w-ik'-ul-aʁle, ʡaˤχ-le le-b-re.}\\
this \textsc{m}-come:\textsc{ipfv}-\textsc{ptcp}-\textsc{ante}, good-\textsc{advz} {be}-\textsc{n}-\textsc{pst}\\
\glt `Every time he came, it was good.'
\z


% 3.1.2
\subsubsection{Immediate anterior converb}

The immediate anterior converb encodes an event which immediately
precedes the event in the matrix clause. Its meaning is comparable to
that of the English construction `as soon as P happened'. The marker of
the immediate anterior converb is the suffix \emph{-rijal} attached to
the perfective stem followed by the irrealis suffix.

\ea \label{ex:8:9} % (9)
\gll {doˤʜi, iχ unna-li-šu w-ak'-a-rijal, b-iq'-es b-aʔ-ib.}\\
snow this neighbour-\textsc{obl}-\textsc{ad}(\textsc{lat}) \textsc{m}-come:\textsc{pfv}-\textsc{irr}-\textsc{imm} \textsc{n}-do:\textsc{ipfv}-\textsc{inf} \textsc{n}-begin:\textsc{pfv}-\textsc{aor}\\
\glt `Just after he got to his neighbour's, it began to snow.'

\pagebreak


\ex \label{ex:8:10} % (10)
\gll {sudi-če} {w-aʔ-a-rijal,} {šalʔu} {naˤʁ꞊ra} {b-aˤq-i-le,} {sudija-li-s} {haraq'e-r} {ʁarʁ-ube} {χʷarčara} {d-iz-aq-i-le} {le-r.}\\
court-\textsc{super}(\textsc{lat}) \textsc{m}-arrive:\textsc{pfv}-\textsc{irr}-\textsc{imm} in.the.bosom(\textsc{lat}) hand꞊\textsc{add} \textsc{n}-hit:\textsc{pfv}-\textsc{aor}-\textsc{cvb} judge-\textsc{obl}-\textsc{dat} in.front-\textsc{npl}(\textsc{ess}) stone-\textsc{pl} shake \textsc{npl}-\textsc{lv}:\textsc{ipfv}-\textsc{caus}-\textsc{aor}-\textsc{cvb} \textsc{aux}-\textsc{npl}\\
\glt `As soon as he got to the court, he put his hand under his arm and shook
the stones (so they jingled like coins) in front of the judge.' \citep[p.~147, sentence 27]{magometov1982}
\z


This form cannot be derived from the imperfective irrealis stem, cf.\
*\emph{wik'arijal} from imperfective \emph{-ik'es} `come', *\emph{urcarijal} from imperfective \emph{urces} `fly'.

There is another way to express the same meaning using an action
nominal in the inter-elative form:

\ea \label{ex:8:11} % (11)
\gll {šahal-li-če b-uħna b-erχʷ-ri-li-ze-la, d-iq'-es d-aʔ-ib zab}\\
town-\textsc{obl}-\textsc{super(lat)} \textsc{hpl}-inside(\textsc{lat}) \textsc{hpl}-enter:\textsc{pfv}-\textsc{nmlz}-\textsc{obl}-\textsc{inter}-\textsc{el} \textsc{npl}-do:\textsc{ipfv}-\textsc{inf} \textsc{npl}-begin:\textsc{pfv}-\textsc{aor} rain\\
\glt `As soon as they entered the town, it started raining.'
\z

\removelastskip

% \largerpage[.5]

% 3.1.3
\subsubsection{Inceptive converb}

\leavevmode{\addfontfeature{LetterSpace=-.5}
The event encoded by the inceptive converb is the initial boundary of
the event described by the main clause. It can be translated into
English as `from the moment when' or `after'. The marker of the
inceptive converb is \emph{-čela}, which is attached to the perfective
participle. The converb marker originates from a combination of the
nominal suffixes \emph{-če-la} (\textsc{super}-\textsc{el}), which literally
means `from above'.}

\ea \label{ex:8:12} % (12)
\gll {dus, nu-ni kaʁar b-arx-ib-i-čela, ʡaˤr-b-aˤq'-un.}\\
year I-\textsc{erg} letter \textsc{n}-send:\textsc{pfv}-\textsc{aor}-\textsc{ptcp}-\textsc{incp} away-\textsc{n}-go:\textsc{pfv}-\textsc{aor}\\
\glt `A year has passed since I sent the letter.'

\ex \label{ex:8:13} % (13)
\gll {iχ w-ak'-ib-i-čela i‹w›aˤʜaˤd, iχ duči-rk'-uwe le-w.}\\
this \textsc{m}-come:\textsc{pfv}-\textsc{aor}-\textsc{ptcp}-\textsc{incp} ‹\textsc{m}›back this laugh-\textsc{lv}:\textsc{ipfv}-\textsc{cvb.ipfv} \textsc{aux}-\textsc{m}\\
\glt `From the moment he came back was laughing.'
\z

The form is impossible with the imperfective stem, cf.\
\emph{*wik'uličela} (from imperfective \emph{-ik'es} `come'),
\emph{*urculičela} (from imperfective \emph{urces} `fly').

% 3.1.4
\subsubsection{Simultaneous converb}

Simultaneity is expressed by the converb marker \emph{-ijadal} attached
to the imperfective participle or to the aorist in the perfective. The
form is probably related to one of the nominal elative markers
\emph{-adal}, with an \emph{-i} of unclear origin and a prothetic
\emph{-j}-. On the other hand, while the elative appears in the corpus in the forms of \emph{-ad}, 
\emph{-adal} and \emph{-adala},
the attested variants of the simulataneous converb include \emph{-ijadal} and \emph{-ijal}.
In the examples below \emph{-ijal} may be used without any change in meaning.

\ea \label{ex:8:14} % (14)
\gll {nu, di-la uzi luč'-ul-ijadal, čaj b-už-uwe le-l-la.}\\
I(\textsc{nom}) I.\textsc{obl}-\textsc{gen} brother read:\textsc{ipfv}-\textsc{ptcp}-\textsc{smlt} tea \textsc{n}-drink:\textsc{ipfv}-\textsc{cvb.ipfv} \textsc{aux}-\textsc{f}-\textsc{ego}\\
\glt `I'm drinking tea while my brother is reading.'

\ex \label{ex:8:15} % (15)
\gll {nu luč'-ul-ijadal, čaj b-už-uwe le-l-la.}\\
I.\textsc{nom} read:\textsc{ipfv}-\textsc{ptcp}-\textsc{smlt} tea \textsc{n}-drink:\textsc{ipfv}-\textsc{cvb.ipfv} \textsc{aux}-\textsc{f}-\textsc{ego}\\
\glt `I'm drinking tea while reading.'
\z

When/if formed from a perfective participle, the converb is semantically
different from its imperfective counterpart in that it acquires a sense of
immediacy:

\ea \label{ex:8:16} % (16)
\gll {predloženije b-elč'-un-ijadal, perewod b-aq'-ib.}\\
sentence \textsc{n}-read:\textsc{pfv}-\textsc{aor}-\textsc{smlt} translation \textsc{n}-do:\textsc{pfv}-\textsc{aor}\\
\glt `As soon as (s)he read the sentence, (s)he translated it.'
\z

Note however that not all speakers accepted (\ref{ex:8:16}).



% 3.1.5
\subsubsection{Posterior converb}

The posterior converb either marks the final boundary of the event in
the main clause or indicates that the event in the converb clause
takes place after the event in the main clause. It could be translated into
English as `before'. The form can be used with both perfective and
imperfective verb stems. This converb is formed by the affix \emph{-če}
added to the perfective stem followed by a vowel,
either \emph{a}, as in (\ref{ex:8:19})\pagebreak[3] or \emph{e}, as in (\ref{ex:8:17}), (\ref{ex:8:18}) and (\ref{ex:8:20}).
The distribution of the
vowel is phonologically conditioned and is exactly the same as the distribution of the
vowel of the infinitive in the same verbs (see \citealt{daniel2019} [this volume]),
which suggests derivation of the converb from the infinitive stem.
The suffix of the converb can be identified with the nominal marker \emph{-če}
(\textsc{super}).

\ea \label{ex:8:17} % (17)
\gll {zab, iχ-di šahal-li-ħe b-erχʷ-eče, d-aʔ-ib d-iq'-es}\\
rain this-\textsc{pl} town-\textsc{obl}-\textsc{in(lat)} \textsc{hpl}-enter:\textsc{pfv}-\textsc{pstr} \textsc{npl}-begin:\textsc{pfv}-\textsc{aor} \textsc{npl}-do:\textsc{ipfv}-\textsc{inf}\\
\glt `It started raining before they entered the town'

\ex \label{ex:8:18} % (18)
\gll {iχ-di-li-ni karawat b-aq'-ib hil-b-ix-eče}\\
this-\textsc{pl}-\textsc{obl}-\textsc{erg} bed \textsc{hpl}-do:\textsc{pfv}-\textsc{aor} \textsc{pv}-\textsc{hpl}-lie.down:\textsc{pfv}-\textsc{pstr}\\
\glt `They made the bed before going to bed'

\ex \label{ex:8:19} % (19)
\gll {iχ-di šahal-li-ze b-ak'-ače iχ-di-li-ze hun-ħe-di d-aq-il si-k'al gu-b}\\
this-\textsc{pl} town-\textsc{obl}-\textsc{inter(lat)} \textsc{hpl}-come:\textsc{pfv}-\textsc{pstr} this-\textsc{pl}-\textsc{obl}-\textsc{inter}(\textsc{lat}) road-\textsc{in}-\textsc{trans} \textsc{npl}-much-\textsc{atr} what-\textsc{ptcl} see-\textsc{aor}\\
\glt `They saw a lot before they entered the town.'

\ex \label{ex:8:20} % (20)
\gll {luk'-eče, nuša-jni deč' b-aq'-i-ra}\\
write:\textsc{ipfv}-\textsc{pstr} we-\textsc{erg} song \textsc{n}-do:\textsc{pfv}-\textsc{aor}-\textsc{ego}\\
\glt `Before writing, we sang a song.'
\z

% 3.2
\subsection{Conditional and counterfactual converbs}

In this section I present a brief description of the morphosyntactic
properties of the conditional\is{conditional|(} and counterfactual converbs. For more
information on the semantics of the conditional forms see \citet{dobrushina2019} [this volume],
which is the source of most examples in this section.

% 3.2.1
\subsubsection{Hypothetical conditional converb}

The marker of the conditional converb \emph{-k'a} attaches to the
irrealis stem. The form is used with both perfective and imperfective
verb stems.

\pagebreak

\ea \label{ex:8:21} % (21)
\gll {nu-ni ʡat' g-a-k'a, ħu-ni na-b t'ult' b-aq'-iša꞊w?}\\
you.sg-\textsc{erg} flour give:\textsc{pfv}-\textsc{irr}-\textsc{cond} you-\textsc{erg} I.\textsc{obl}-\textsc{dat} bread \textsc{n}-do:\textsc{pfv}-\textsc{fut}.\textsc{ego}꞊\textsc{q}\\
\glt `If I bring the flour, will you make bread for me?'

\ex \label{ex:8:22} % (22)
\gll {nu, di-la urši-li-ni xunul k-a-k'a, iχ-di-li-šu-r d-uʔ-es-i.}\\
I I.\textsc{obl}-\textsc{gen} boy-\textsc{obl}-\textsc{erg} wife  bring:\textsc{pfv}-\textsc{irr}-\textsc{cond} that-\textsc{pl}-\textsc{obl}-\textsc{ad}-\textsc{hpl}(\textsc{ess}) \textsc{f1}-be-\textsc{inf}-\textsc{atr}\\
\glt `If my son marries, I will live at their place.' \citep{dobrushina2019}
\z

In example (\ref{ex:8:23}), the main clause and the converb clause share
the subject:

\ea \label{ex:8:23} % (23)
\gll {anwar} {w-ak'-i-le} {w-arg-a-k'a,} {abaj-šu} {uˤq'-es-i}.\\
Anwar \textsc{m}-come:\textsc{pfv}-\textsc{aor}-\textsc{cvb} \textsc{m}-find:\textsc{pfv}-\textsc{irr}-\textsc{cond} mother-\textsc{ad}(\textsc{lat}) \textsc{m}.go:\textsc{pfv}-\textsc{inf}-\textsc{atr}\\
\glt `If Anwar comes, he will go to his mother.' \citep{dobrushina2019}
\z

This conditional use of `find' in (\ref{ex:8:23}) is discussed in \citet{maisak-daniel2018}.

Example (\ref{ex:8:24}) shows the same converb formed from the
imperfective stem:

\ea \label{ex:8:24} % (24)
\gll {d-aq-il} {kung-ane} {luč'-a-k'a,} {d-aq-il} {si-k'al} {nuša-ze} {d-alh-ul}.\\
\textsc{npl}-much-\textsc{atr} book-\textsc{pl} read:\textsc{ipfv}-\textsc{irr}-\textsc{cond} \textsc{npl}-much-\textsc{atr} what-\textsc{indef} we-\textsc{inter}(\textsc{lat}) \textsc{npl}-know:\textsc{ipfv}-\textsc{ptcp}\\
\glt `If we read many books, we will know many things\textbf{.'}
\z

% 3.2.2
\subsubsection{Counterfactual converb}

The affix \emph{-q'alle} forms the converb of counterfactual condition
(that is, the event in the main clause could have taken place if the
event in the converbal clause had taken place -- but this did not
happen). It attaches to the perfective stem (\ref{ex:8:25}, 26) or to the imperfective
participle (\ref{ex:8:27}).

\ea \label{ex:8:25} % (25)
\gll {ħu anawaje w-aq'-un-q'alle, nuša-jni muħammad ulc-a-re.}\\
you.sg(\textsc{nom}) fast.\textsc{advz} \textsc{m}-do:\textsc{pfv}-\textsc{aor}-\textsc{ctrf}
we-\textsc{erg} Muhammad \textsc{m}.catch:\textsc{ipfv}-\textsc{irr}-\textsc{pst}\\
\glt `If you had driven fast, we would have caught Muhammad.'

\ex \label{ex:8:26} % (26)
\gll {nu꞊ra} {iχ} {w-ebk'-ib-q'alle,} {d-ubk'-a-re.}\\
I(\textsc{nom})꞊\textsc{add} this \textsc{m}-die:\textsc{pfv}-\textsc{aor}-\textsc{ctrf} \textsc{f1}-die:\textsc{ipfv}-\textsc{irr}-\textsc{pst}\\
\glt `If he had died, I would have also died.' \citep{dobrushina2019}
\z

Examples (\ref{ex:8:27}) shows this converb formed from the imperfective
stem:

\ea \label{ex:8:27} % (27)
\gll {iχ} {išbari} {ʡaˤš-w-irk-ul-q'alle} {ʡaˤχ-le} {b-uʔ-a-re.}\\
this today \textsc{pv}-\textsc{m}-come.back:\textsc{ipfv}-\textsc{ptcp}-\textsc{ctrf} good-\textsc{advz} \textsc{n}-be-\textsc{irr}-\textsc{pst}\\
\glt `If he had come today, it would have been good.'
\citep{dobrushina2019} 
\z

% 3.2.3
\subsubsection{Concessive converb}

The concessive converb is formed by the complex suffix \emph{-k'a-ra}
(-\textsc{cond}-\textsc{add}) preceded by the irrealis suffix \emph{-a}-, as exemplified in (\ref{ex:8:28}–\ref{ex:8:30}).

\ea \label{ex:8:28} % (28)
\gll {iχ-ini, iχ-di-li-ni ʡat' ħa-g-a-k'ara, pirog b-aq'-ib.}\\
this-\textsc{erg} this-\textsc{pl}-\textsc{obl}-\textsc{erg} flour \textsc{neg}-give:\textsc{pfv}-\textsc{irr}-\textsc{conc} pie \textsc{n}-do:\textsc{pfv}-\textsc{aor}\\
\glt `Although they didn't given her any flour, she baked a pie.'

\ex \label{ex:8:29} % (29)
\gll {dunijal zab-li ur-a-k'ara nuša quli ʜaˤ-b-aˤq'-un-na.}\\
world rain-\textsc{obl}(\textsc{erg}) rain:\textsc{ipfv}-\textsc{irr}-\textsc{conc} we home(\textsc{lat}) \textsc{neg}-\textsc{hpl}-go:\textsc{pfv}-\textsc{aor}-\textsc{ego}\\
\glt `Although it was raining, we didn't go home.'

\ex \label{ex:8:30} % (30)
\gll {nu-ni b-iq'-a-k'ara, ħu razi ħa-rhʷ-an.}\\
I.\textsc{erg} \textsc{n}-do:\textsc{ipfv}-\textsc{irr}-\textsc{conc} you.\textsc{nom} agree \textsc{neg}-\textsc{m}.become:\textsc{ipfv}-\textsc{hab}\\
\glt `Although I do (this), you are not happy.' \citep{magometov1982}
\z

% 3.2.4
\subsubsection{Second concessive converb}

The marker \emph{-leʡur} conveys a meaning close to concession\is{concession|(} and
causality. It is used when the event described by the converbal clause
was unlikely to happen and probably undesirable; but since it
nevertheless did happen, the action in the main clause takes place. The
structure of the form is unclear. In regular verbs, it may be analysed as 
the participle followed by the suffix \emph{-leʡur}. For the auxiliary \mbox{\emph{-le-}\textsc{cl}}, 
however, the suffix seems to attach directly to the finite form (\mbox{\emph{le-}\textsc{cl}\emph{-leʡur}}). 
This form is attested in Magometov's texts but is, at best, marginal. All my examples are elicited.
Not all speakers accept this form, and even those who find it acceptable with some verbs are 
unable to think of examples with other verbs. 

\ea \label{ex:8:31} % (31)
\gll {ħu w-ak'-i-leʡur, nuša-ni ħa-d ʡoˤχlad-deš d-aq'-iša.}\\
you.sg(\textsc{nom}) \textsc{m}-come:\textsc{pfv}-\textsc{aor}-\textsc{ptcp?}-\textsc{conc2} we-\textsc{erg} you-\textsc{dat} hospitable-\textsc{nmlz} \textsc{npl}-do:\textsc{pfv}-\textsc{fut.ego}\\
\glt `As you have come here, we will show you hospitality.'

\ex \label{ex:8:32} % (32)
\gll {ħu b-ak'-i-leʡur, b-ug-e.}\\
you.sg(\textsc{nom}) \textsc{n}-come:\textsc{pfv}-\textsc{aor}-\textsc{cvb}-\textsc{conc2} \textsc{n}-eat:\textsc{pfv}-\textsc{imp}\\
\glt `Since you have come here, eat.' (addressed to an animal)

\ex \label{ex:8:33} % (33)
\gll {χʷe har-b-ulq-u-leʡur, b-uc-a.}\\
dog \textsc{pv}-\textsc{n}-flee:\textsc{ipfv}-\textsc{cvb.ipfv}-\textsc{conc2} \textsc{n}-catch:\textsc{pfv}-\textsc{imp}.\textsc{tr}\\
\glt `Since the dog is running away, catch it!'
\z

\removelastskip
\is{concession|)}
\is{conditional|)}

% 3.3
\subsection{Other converbs of logical relations}

% 3.3.1
\subsubsection{Causal converb}

The causal converb describes an event which is the cause of the
situation described in the main clause. The causal converb affix
\emph{-na} is attached to the general converb. The converb is formed
from both perfective (cf.\ \ref{ex:8:34} and \ref{ex:8:35}) and imperfective (cf. \ref{ex:8:36}) stems.

\ea \label{ex:8:34} % (34)
\gll {iχ, doˤʜi b-aq'-i-le-na, ruzi-li-šu w-aˤq'-un.}\\
this snow \textsc{n}-do:\textsc{pfv}-\textsc{aor}-\textsc{cvb}-\textsc{causal} sister-\textsc{obl}-\textsc{ad}(\textsc{lat}) \textsc{m}-go:\textsc{pfv}-\textsc{aor}\\
\glt `Because it started to snow, he went to his sister.'

\ex \label{ex:8:35} % (35)
\gll {xunuj-ni sual-t xar d-i-uwe le-l-le iχi-ze, quli ʡaraʁa w-ak'-i-le-na.}\\
wife.\textsc{obl}-\textsc{erg} question-\textsc{pl} ask \textsc{npl}-\textsc{lv}:\textsc{ipfv}-\textsc{cvb.ipfv} \textsc{aux}-\textsc{npl}-\textsc{pst} he-\textsc{inter}(\textsc{lat}) home(\textsc{lat}) late \textsc{m}-come:\textsc{pfv}-\textsc{aor}-\textsc{cvb}-\textsc{causal}\\
\glt `His wife asked him questions because he came home late.'

\ex \label{ex:8:36} % (36)
\gll {murad w-ik'-uwe-na, nuša ħule b-uʔ-i-ra.}\\
Murad \textsc{m}-come:\textsc{ipfv}-\textsc{cvb.ipfv}-\textsc{causal} we eye \textsc{hpl}-be:\textsc{pfv}-\textsc{aor}-\textsc{ego}\\
\glt `We were waiting because Murad was due to come.'
\z

% 3.3.2
\subsubsection{Purposive converb}

The purposive converb expresses an event conceptualized as the purpose
of the action described in the main clause (`in order to'). It is formed
by the affix \mbox{\emph{-alis}} added to the bare verb stem and can be formed
with both perfective and imperfective stems. The marker is likely to
originate from \emph{-a-li-s} (‑\textsc{irr}-\textsc{obl}-\textsc{dat}). Indeed,
cross-linguistically, the dative often expresses a purposive meaning
(see for example, \citealt{haspelmath1995b}), and purposive meaning is
related to the irrealis domain (see for example \citealt[131]{palmer2001}).

\ea \label{ex:8:37} % (37)
\gll {iχ-di-li, dursi hil-d-ix-alis, buruš b-aq'-ib.}\\
this-\textsc{pl}-\textsc{erg} girl \textsc{pv}-\textsc{f1}-lie.down:\textsc{pfv}-\textsc{purp} bed \textsc{n}-do:\textsc{pfv}-\textsc{aor}\\
\glt `They made the bed so that the girl could go to sleep.'

\ex \label{ex:8:38} % (38)
\gll {dursi-li-ni buruš b-aq'-ib, hil-d-ix-alis.}\\
girl-\textsc{obl}-\textsc{erg} bed \textsc{n}-do:\textsc{pfv}-\textsc{aor} \textsc{pv}-\textsc{f1}-lie.down:\textsc{pfv}-\textsc{purp}\\
\glt `The girl made the bed in order to go to sleep.'

\ex \label{ex:8:39} % (39)
\gll {ʡali w-ik'-alis nu-ni igruš-une as-i-ra.}\\
Ali \textsc{m}-come:\textsc{ipfv}-\textsc{purp} I-\textsc{erg} toy-\textsc{pl} take:\textsc{pfv}-\textsc{aor}-\textsc{ego}\\
\glt `I brought the toys so that Ali would come.'
\z

The semantics of purpose can also be expressed by the infinitive, as in (\ref{ex:8:40}) and (\ref{ex:8:41}). 
Just as the construction with \emph{-alis},
the infinitive construction may have the same subject as in the main clause or a different one. 
The difference in meaning between the two constructions, if it exists, has not been studied.

\ea \label{ex:8:40} % (40)
\gll {iχ-di-li buruš b-aq'-ib dursi hil-d-ix-es.}\\
this-\textsc{pl}-\textsc{obl} bed \textsc{n}-do:\textsc{pfv}-\textsc{aor} girl \textsc{pv}-\textsc{f1}-lie.down:\textsc{pfv}-\textsc{inf}\\
\glt `They made the bed for the girl to go to sleep.'

\ex \label{ex:8:41} % (41)
\gll {dursi-li-ni buruš b-aq'-ib, hil-d-ix-es.}\\
girl-\textsc{obl}-\textsc{erg} bed \textsc{n}-do:\textsc{pfv}-\textsc{aor} \textsc{pv}-\textsc{f1}-lie.down:\textsc{pfv}-\textsc{inf}\\
\glt `The girl made the bed in order to go to sleep.'
\z

However, for the reasons discussed above in \sectref{defining-the-terms}, I do not count
the infinitival construction as converbal, though, in this case,
it fits the definition from a functional point of view. For further
detailed discussion, see \citet[28]{haspelmath1995a}.

% 3.3.3
\subsubsection{Gradual converb}

The affix \emph{-cad(i)} attaches to the participle to express
graduality. Clauses with this converb can be translated into English
using the expression `the more\ldots{}, the more \ldots{}'. This affix
also exists in standard Dargwa as a nominal marker expressing the
meaning `as much as, about' (called ``equative'' in \citealt[25]{vandenberg2001}).
It also occurs with verb forms and in this case shows
nearly the same semantics as in Mehweb. Historically, the first part of
this marker (\emph{-ca-}) may derive from the spatial marker meaning
`from the speaker' (translocative), but the origins of the second
part \emph{-di} are not clear. A possible cognate of this affix is
\emph{-cat} in Tanti Dargwa, which conveys the meaning of
approximation and similarity \citep{sumbatova-lander2014:hse}. The form can be 
derived from both the perfective and imperfective participles.

\ea \label{ex:8:42} % (42)
\gll {urši, d-aqnal dursi luč'-ul-cadi, w-aqnal uk-uwe le-w.}\\
boy \textsc{f1}-often girl read:\textsc{ipfv}-\textsc{ptcp}-\textsc{grad} \textsc{m}-often \textsc{m}.eat:\textsc{ipfv}-\textsc{cvb.ipfv} \textsc{aux}-\textsc{m}\\
\glt `The more the girl reads, the more the boy eats.'

\ex \label{ex:8:43} % (43)
\gll {it kung b-elč-un-i-cad, nab b-elč-es dig-an.}\\
this book \textsc{n}-read:\textsc{pfv}-\textsc{aor}-\textsc{ptcp}-\textsc{grad} I.\textsc{dat} \textsc{n}-read:\textsc{pfv}-\textsc{inf} want:\textsc{ipfv}-\textsc{hab}\\
\glt `The more I read this book, the more I want to read.'
\z

Apart from the semantics described above, this form may also have a
temporal interpretation of simultaneity, as shown in (\ref{ex:8:44}):

\ea \label{ex:8:44} % (44)
\gll {dursi šaˤ-baˤʜ q'-uˤwe le-r-cad, iχija pikru-me le-l-le ruzi-li-če-r.}\\ 
girl village-\textsc{dir} go:\textsc{ipfv}-\textsc{cvb.ipfv} \textsc{aux}-\textsc{f}-\textsc{grad} this.\textsc{gen} thought-\textsc{pl} {be}-\textsc{f}-\textsc{pst} sister-\textsc{obl}-\textsc{super}-\textsc{f}(\textsc{ess})\\
\glt `While going to the village, the girl was thinking about her sister.'
(lit. her thoughts were about the sister)
\z


% 3.4
\subsection{Locative converb}

The form that can be interpreted as locative converb was only identified in the corpus after the
field phase of the research. Cf. the second wordform in (\ref{ex:8:45}):

\ea \label{ex:8:45} % (45)
\gll {nu q'oˤ-j-ħe w-arg-a-k'a uk-iša.}\\
you.sg(\textsc{nom}) go:\textsc{ipfv}-\textsc{ptcp}-\textsc{in}(\textsc{lat}) \textsc{m}-find:\textsc{pfv}-\textsc{irr}-\textsc{cond} \textsc{m}.bring:\textsc{pfv}-\textsc{fut}.\textsc{ego}\\
\glt {\addfontfeature{LetterSpace=-1}`If it turns out that (you are going) where I am going, I will give you a \rlap{ride.'}}
\z

Morphologically, it is a participle followed by the suffix of localization inside a hollow container. 
For the reasons discussed above in \sectref{problems-in-defining-specialized-converbs}, I consider this form together with converbs. 
The form only occurs in the corpus two more times (e.g.\ \emph{dig-uj-ħe} `to
where one wants', `wherever one wants (to go)'), also based on the
imperfective participle (\emph{-uj} \textless{} \mbox{\emph{-ul}}). Under
elicitation, the same form has also been produced for the perfective
stem:

\ea \label{ex:8:46} % (46)
\gll {nu-ni uʡaˤ b-ix-ib-i-ħe-w it kajʔ-ib.}\\
I-\textsc{erg} cheese \textsc{m}-put:\textsc{pfv}-\textsc{aor}-\textsc{ptcp}-\textsc{in}-\textsc{m}(\textsc{ess}) this \textsc{m}.sit:\textsc{pfv}-\textsc{aor}\\
\glt `He sat down where I had put the cheese.'
\z

\removelastskip
\largerpage

\begin{table}[b]

  % Tab. 1.

% \begin{longtable}{@{}lp{.15\textwidth}<{\raggedright}llll@{}}
%  \caption{Compatibility of converbal markers with perfective or imperfective verb stems (continued)}\\
% \toprule
% converb & marker & \multicolumn{2}{c}{perfective} & \multicolumn{2}{c@{}}{imperfective}\tabularnewline \cmidrule(lr){3-4} \cmidrule(l){5-6}
% & & {example} & {base} & {example} & {base}\tabularnewline \midrule \endhead

\captionsetup{margin=0pt} 
  \caption{Compatibility of converbal markers with perfective or imperfective verb stems}\label{tab:8:1}

  \small
\advance\tabcolsep-2pt
\begin{tabular}{@{}lp{.2\textwidth}<{\raggedright}@{}llll@{}}
    \toprule
converb & marker & \multicolumn{2}{c}{perfective} & \multicolumn{2}{c@{}}{imperfective}\tabularnewline \cmidrule(r){3-4} \cmidrule(l){5-6}
& & {example} & {base} & {example} & {base}\tabularnewline \midrule % \endfirsthead
\textsc{imm} & \emph{-a-rijal} & (\ref{ex:8:6}), (\ref{ex:8:7}) & irrealis stem & – & – \tabularnewline
\textsc{purp} & \emph{-a(-)lis} & (\ref{ex:8:34}), (\ref{ex:8:35}) & irrealis stem & (\ref{ex:8:36}) &
irrealis stem\tabularnewline
\textsc{cond} & \emph{-k'a} & (\ref{ex:8:18}), (\ref{ex:8:19}), (\ref{ex:8:20}) & irrealis stem & (\ref{ex:8:21}) &
irrealis stem\tabularnewline
\textsc{conc} & \emph{-k'a-ra} & (\ref{ex:8:25}) & irrealis stem & (\ref{ex:8:26}), (\ref{ex:8:27}) &
irrealis stem\tabularnewline
\textsc{pstr} & \emph{-a/e(-)če} & (\ref{ex:8:15}), (\ref{ex:8:16}) & infinitive stem & (\ref{ex:8:17})
& infinitive stem\tabularnewline
\textsc{smlt} & \emph{-ijadal} & (\ref{ex:8:13}) & aorist & (\ref{ex:8:11}), (\ref{ex:8:12}) &
participle\tabularnewline
\textsc{ctrf} & \emph{-q'alle} & (\ref{ex:8:22}), (\ref{ex:8:23}) & aorist & {(\ref{ex:8:24})} &
participle\tabularnewline
\textsc{incp} & \emph{-čela} & (\ref{ex:8:9}), (\ref{ex:8:10}) & participle & – & – \tabularnewline
\textsc{loc} & \emph{-ħe} & (\ref{ex:8:46}) & participle & (\ref{ex:8:45}) &
participle\tabularnewline
\textsc{ante} & \emph{-a(r)ʁle, -aʁe, -a(r)ʁ, -ʁale, -ʁela} & (\ref{ex:8:3}), (\ref{ex:8:4}) &
participle & (\ref{ex:8:5}) & participle\tabularnewline
\textsc{grad} & \emph{-cad(i) } & (\ref{ex:8:39}) & participle & (\ref{ex:8:40}) &
participle\tabularnewline
\textsc{causal} & \emph{-na} & (\ref{ex:8:31}), (\ref{ex:8:32}) & general converb & (\ref{ex:8:33}) &
general converb\tabularnewline
\textsc{conc2} & \emph{-leʡur} & (\ref{ex:8:28}), (\ref{ex:8:29}) & general converb & (\ref{ex:8:30}) &
general converb\tabularnewline
\bottomrule
% \end{longtable}
\end{tabular}
\end{table}

% 4.
\section{Examples of the forms}\label{examples-of-the-forms}

This section contains two tables. \tabref{tab:8:1} shows the compatibility of
converbal markers with different stems and the verb forms serving as the
base for the corresponding converbs, with references to the examples
above. \tabref{tab:8:2} provides an example of each of the specialized converb
forms.


\vspace{-2\baselineskip}
\largerpage


% Tab. 2.

\begin{table}[b]
% \vskip-4\baselineskip
  % \begin{longtable}{@{}lp{.42\textwidth}<{\raggedright}p{.42\textwidth}<{\raggedright}@{}}
%   \caption{Examples of specialized converbs (continued)}\\
% \toprule
% & \textsc{pfv} & \textsc{ipfv}\tabularnewline \midrule \endhead

  \caption{Examples of specialized converbs}\label{tab:8:2}

  \small
  \renewcommand\arraystretch{.9} % advance\baselineskip-.5pt
  \begin{tabular}{@{}lp{.42\textwidth}<{\raggedright}p{.42\textwidth}<{\raggedright}@{}}
\toprule
& \textsc{pfv} & \textsc{ipfv}\tabularnewline  \midrule % \endfirsthead
\textsc{imm} & \emph{w-ak'-a-rijal} (\ref{ex:8:6})

\textsc{m}-come:\textsc{pfv}-\textsc{irr}-\textsc{imm}

`just after he came' & \emph{--}\tabularnewline
\textsc{purp} & \emph{hil-d-ix-alis} (\ref{ex:8:34}), (\ref{ex:8:35})

\textsc{pv}-\textsc{f1}-lie.down:\textsc{pfv}-\textsc{purp}

`in order for her to go to sleep' & \emph{w-ik'-alis} (\ref{ex:8:36})

\textsc{m}-come:\textsc{ipfv}-\textsc{purp}

`in order for him to come'\tabularnewline
\textsc{cond} & \emph{g-a-k'a} (\ref{ex:8:18})

give:\textsc{pfv}-\textsc{irr}-\textsc{cond}

`if (s)he gave' & \emph{luč'-a-k'a} (\ref{ex:8:21})

read:\textsc{ipfv}-\textsc{irr}-\textsc{cond}

`if (s)he gave'\tabularnewline
\textsc{conc} & \emph{ħa-g-a-k'ara} (\ref{ex:8:25})

\textsc{neg}-give:\textsc{pfv}-\textsc{irr}-\textsc{conc}

`though (s)he did not give' & \emph{b-iq'-a-k'ara} (\ref{ex:8:27})

\textsc{n}-do:\textsc{ipfv}-\textsc{irr}-\textsc{conc}

`though (s)he does'\tabularnewline
\textsc{pstr} & \emph{b-ak'-ače} (\ref{ex:8:16})

\textsc{hpl}-come:\textsc{pfv}-\textsc{pstr}

`before they came' & \emph{luk'-eče} (\ref{ex:8:17})

write:\textsc{ipfv}-\textsc{pstr}

`before reading'\tabularnewline
\textsc{smlt} & \emph{b-elč'-un-ijadal} (\ref{ex:8:13})

\textsc{n}-read:\textsc{pfv}-\textsc{aor}-\textsc{smlt}

`as soon as (s)he reads' & \emph{luč'-ul-ijadal} (\ref{ex:8:12})

read:\textsc{ipfv}-\textsc{ptcp}-\textsc{smlt}

`while (s)he reads'\tabularnewline
\textsc{ctrf} & \emph{w-aq'-un-q'alle} (\ref{ex:8:22})

\textsc{m}-do:\textsc{pfv}-\textsc{aor}-\textsc{ctrf}

`if he had done' & \emph{ʡaˤš-w-irk-ul-q'alle} (\ref{ex:8:24})

\textsc{pv}-\textsc{m}-come.back:\textsc{ipfv}-\textsc{ptcp}-\textsc{ctrf}

`if he comes'\tabularnewline
\textsc{incp} & \emph{w-ak'-ib-i-čela} (\ref{ex:8:10})

\textsc{m}-come:\textsc{pfv}-\textsc{aor}-\textsc{ptcp}-\textsc{incp}

`since the moment he came' & \emph{--}\tabularnewline
\textsc{loc} & \emph{b-ix-ib-i-ħe-w} (\ref{ex:8:46})

\textsc{N}-put:\textsc{pfv}-\textsc{aor}-\textsc{ptcp}-\textsc{in}-\textsc{m}(\textsc{ess})

`where I put it' & \emph{q'oˤ-j-ħe} (\ref{ex:8:45})

go:\textsc{pfv}-\textsc{ptcp}-\textsc{in}(\textsc{lat})

`to where (s)he goes'\tabularnewline
\textsc{ante} & \emph{b-ak'-ib-i-ʁale} (\ref{ex:8:4})

\textsc{hpl}-come:\textsc{pfv}-\textsc{aor}-\textsc{ptcp}-\textsc{ante}

`when they came' & \emph{w-ik'-ul-aʁle} (\ref{ex:8:5})

\textsc{m}-come:\textsc{ipfv}-\textsc{ptcp}-\textsc{ante}

`when he came'\tabularnewline
\textsc{grad} & \emph{b-elč-un-i-cad(i)} (\ref{ex:8:40})

\textsc{n}-read:\textsc{pfv}-\textsc{aor}-\textsc{ptcp}-\textsc{grad}

`the more (s)he has read' & \emph{luč'-ul-cad(i)} (\ref{ex:8:39})

read:\textsc{ipfv}-\textsc{ptcp}-\textsc{grad}

`the more (s)he reads'\tabularnewline
\textsc{causal} & \emph{d-ak'-i-le-na} (\ref{ex:8:32})

\textsc{f1}-come:\textsc{pfv}-\textsc{aor}-\textsc{cvb}-\textsc{causal}

`because she came' & \emph{w-ik'-uwe-na} (\ref{ex:8:33})

\textsc{m}-come:\textsc{ipfv}-\textsc{cvb.ipfv}-\textsc{causal}

`because he came'\tabularnewline
\textsc{conc2} & \emph{w-ak'-i-leʡur} (\ref{ex:8:28})

\textsc{m}-come:\textsc{pfv}-\textsc{aor}-\textsc{ptcp?}-\textsc{conc2}

`since he came' & \emph{har-b-ulq-u-leʡur} (\ref{ex:8:30})

\textsc{pv}-\textsc{n}-flee:\textsc{ipfv}-\textsc{ptcp?}-\textsc{conc2}

`since it is running away'\tabularnewline
\bottomrule
% \end{longtable}
\end{tabular}
\end{table}

\normalsize

% 5.
\section{Conclusion}\label{conclusion-9}

Mehweb has a relatively rich inventory of specialized converbs, with
five temporal converbs (anterior, immediate anterior, inceptive,
simultaneous and posterior), seven converbs expressing logical relations
(hypothetical conditional, counterfactual, concessive, converb
expressing another meaning close to concessive, causal, purposive and
gradual) and, probably, a locative converb. The anterior converb marker shows strong phonological
variation. Other variants of converb
markers include two variants for the simultaneous converb marker
(\emph{-jadal}/\emph{-jal}) and two variants for the gradual converb
(\emph{-cad}/\emph{-cadi}).

Specialized converbs are formed in several different ways, with the
converb marker attached to either:
\begin{enumerate}[topsep=\medskipamount,itemsep=0pt,partopsep=0pt,parsep=0pt,label={\arabic*})]
\item % 1)
  irrealis stem: immediate, anterior, concessive converbs and possibly
also the purposive converb;

\item % 2)
  infinitive stem: posterior converb (probably);

\item % 3)
  aorist in the perfective and participle in the imperfective:
simultaneous and counterfactual converbs;

\item % 4)
  participle in both perfective and imperfective: anterior, gradual
and locative converb;

\item % 5)
  general converb: causal and second concessive converbs.
\end{enumerate}

These types may be interpreted as reflecting an increasing degree of
grammaticalization of the forms. The irrealis and the infinitive stems
are bound (cannot appear without further marking). The third pattern includes  
converb suffixes that are selective in terms of the stem they attach to (aorist in the
perfective, participle in the imperfective). The fourth type includes 
converbs derived from stems that also function as free forms (participles).
However, as discussed in the introduction, participles do not typically
function as nominal adjuncts in the clause, and their inflection is
different from the converb markers, so a certain degree of
grammaticalization is still present. Finally, in the fifth type, the forms to which the converb
markers (causal and second concessive) are attached are not only free forms but are already marked for adverbial
subordination. Their only function is to further specify the general converb in terms of its relation 
to the main clause, which makes these markers functionally similar to particles. On the other hand, they are bound 
and must be considered as derivational suffixes.


\clearpage

\section*{List of abbreviations}

\begin{longtable}[l]{@{}ll@{}}
\textsc{abs}	& absolutive \\
\textsc{ad}	& spatial domain near the landmark \\
\textsc{add}	& additive particle \\
\textsc{advz}	& adverbializer \\
\textsc{dir}	& motion directed towards a spatial domain \\
\textsc{ante}	& anterior converb \\
\textsc{aor}	& aorist \\
\textsc{atr}	& attributivizer \\
\textsc{aux}	& auxiliary \\
\textsc{caus}	& causative \\
\textsc{causal}	& causal (case form) \\
\textsc{cl}	& gender (class) agreement slot \\
\textsc{conc}	& concessive \\
\textsc{conc2}	& concessive \\
\textsc{cond}	& conditional \\
\textsc{ctrf}	& counterfactual \\
\textsc{cvb}	& converb \\
\textsc{dat}	& dative \\
\textsc{ego}	& egophoric \\
\textsc{el}	& motion from a spatial domain \\
\textsc{erg}	& ergative \\
\textsc{ess}	& static location in a spatial domain \\
\textsc{f}	& feminine (gender agreement) \\
\textsc{f1}	& feminine (unmarried and young women gender prefix) \\
\textsc{fut}	& future \\
\textsc{gen}	& genitive \\
\textsc{grad}	& gradual converb \\
\textsc{hab}	& habitual (durative for verbs denoting states) \\
\textsc{hpl}	& human plural (gender agreement) \\
\textsc{imm}	& immediate converb \\
\textsc{imp}	& imperative \\
\textsc{in}	& spatial domain inside a (hollow) landmark \\
\textsc{incp}	& inceptive converb \\
\textsc{indef}	& indefinite particle \\
\textsc{iness}	& location inside \\
\textsc{inf}	& infinitive \\
\textsc{inter}	& spatial domain between multiple landmarks \\
\textsc{ipfv}	& imperfective (derivational base) \\
\textsc{irr}	& irrealis (derivational base) \\
\textsc{lat}	& motion into a spatial domain \\
\textsc{loc}	& locative converb \\
\textsc{lv}	& light verb \\
\textsc{m}	& masculine (gender agreement) \\
\textsc{n}	& neuter (gender agreement) \\
\textsc{neg}	& negation (verbal prefix) \\
\textsc{nmlz}	& nominalizer \\
\textsc{nom}	& nominative \\
\textsc{npl}	& non-human plural (gender agreement) \\
\textsc{obl}	& oblique (nominal stem suffix) \\
\textsc{pfv}	& perfective (derivational base) \\
\textsc{pl}	& plural \\
\textsc{prf}	& perfect \\
\textsc{pst}	& past \\
\textsc{pstr}	& posterior converb \\
\textsc{ptcl}	& particle \\
\textsc{ptcp}	& participle \\
\textsc{purp}	& purposive converb \\
\textsc{pv}	& preverb (verbal prefix) \\
\textsc{q}	& question (interrogative particle) \\
\textsc{sbess}	& subessive (location under) \\
\textsc{smlt}	& simultaneous converb \\
\textsc{super}	& spatial domain on the horizontal surface of the landmark \\
\textsc{tr}	& transitive \\
\textsc{trans}	& motion through a spatial domain \\
\end{longtable}


\printbibliography[heading=subbibliography,notkeyword=this]

\iffalse

\section*{References}

Creissels,~D. 2010. `Specialized converbs and adverbial subordination in
Axaxdərə Akhvakh.' In Bril, I. (ed.), \emph{Clause Linking and Clause
Hierarchy: Syntax and pragmatics},\emph{121}, 105.

Dobrushina,~N.~R. 2015. `Moods of Mehweb'. In print.

Haspelmath, M. 1995a. `The converb as a cross-linguistically valid
category'. In Haspelmath,~M. \& E.~König (eds.), \emph{Converbs in
cross-linguistic perspective}. 1–55. Berlin: Mouton de Gruyter.

Haspelmath, M. 1995b. `Contextual and specialized converbs in Lezgian'.
In Haspelmath,~M. \& E.~König (eds.), \emph{Converbs in cross-linguistic
perspective}. 415–40. Berlin: Mouton de Gruyter.

Khalilova,~Z. 2009. `A grammar of Khwarshi'. (Doctoral dissertation,
LOT, Netherlands Graduate School of Linguistics, Utrecht).

Kustova,~M.~A. 2015. `General converbs in Mehweb'. In print.

Nedjalkov,~V. 1995. `Some typological parameters of converbs'. In
Haspelmath,~M. \& E.~König (eds.), \emph{Converbs in cross-linguistic
perspective}. 97–136. Berlin: Mouton de Gruyter.

Palmer F. R. 2001.~\emph{Mood and
Modality.}~2nd~ed.~Cambridge~University~Press.

Sumbatova,~N.~R., Lander U.~A. 2014. `The \emph{Dargwa} Variety of the
\emph{Tanti} Village. A Grammatical Sketch. Issues in Syntax '. Higher
School of Economics Research Paper No. WP BRP,~12.

Van den Berg,~H. 2001. `Dargi folktales: Oral stories from the Caucasus
with an introduction to Dargi grammar' (Vol. 106). Leiden University
Press.

% Maria V. Sheyanova

% National Research University Higher School of Economics (Moscow,
% Russia), Faculty of Humanities, School of Linguistics;

% E-mail: masha.shejanova@gmail.com, Tel. +7-916-922-31-14

\fi


\end{document}

%%% Local Variables:
%%% mode: latex
%%% TeX-master: "../main"
%%% End:
