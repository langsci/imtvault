\documentclass[output=paper]{langsci/langscibook} 
\ChapterDOI{10.5281/zenodo.3402064}

% Chapter 6

\title{Periphrastic causative constructions in Mehweb}

\author{Daria Barylnikova\affiliation{National Research University Higher School of Economics}}

\abstract{In Mehweb, periphrastic causatives are formed by a combination of the infinitive of the lexical verb with another verb, originally a caused motion verb. 
Various tests that Mehweb periphrastic causatives do not qualify as fully grammaticalized. But the constructions are not compositional expressions, either. While a clause usually contains either a morphological or a periphrastic causative marker, there are instances where, in a periphrastic causative construction, the lexical verb itself may carry the causative affix, resulting in only one causative meaning.

\emph{Keywords}: causative, periphrastic causative, double causative,
Mehweb, Dargwa, East Caucasian.}


\begin{document}
\maketitle

\exewidth{(23)}

\let\exfont\rm
\let\eachwordone\rm


% 1.
\section{Introduction}

The causative construction denotes a
complex situation consisting of two component events: (1) the event that
causes another event to happen; and (2) the result of this causation
(\citealt[165–166]{comrie1989}; \citealt{nedjalkov-silnitsky1973}; \citealt{kulikov2001}).
Here, the first event refers to the action of the causer and the
second explicates the effect of the causation on the causee.

Causativization is a valency-increasing derivation which is applied to
the structure of the clause. In the resulting construction, the causer
is the subject and the causee shifts to a non-subject position. The set
of semantic roles does not remain the same. Minimally, a new agent is
added. With a new argument added, we have to redistribute the
grammatical relations taking into account how these participants
semantically relate to each other. The general scheme of the causative
derivation always implies a participant that is treated as a causer
(someone or something that spreads their control over the
situation and ``pulls the trigger''). At the same time, there is someone
who is, willingly or not, involved in the situation induced by the
causer. With two-place predicates there is also another, undergoer
participant who does not interact with the causer directly and does not
play a role in the redistribution of grammatical relations. This
participant retains the marking that it had in the original sentence.
The following English examples illustrate these options:

\begin{enumerate}[label=\alph{enumi}.,topsep=\medskipamount,itemsep=0pt,partopsep=0pt,parsep=0pt]
\item
  \emph{The professor made his student work hard}. (originally
  intransitive)

\item % b.
  \emph{The professor made his student drop a course this semester}.
  (originally transitive)

\item  % c.
  \emph{The professor made his student laugh at his joke}. (originally
  intransitive with an oblique object)
\end{enumerate}

{\addfontfeature{LetterSpace=-.13}
Mehweb has a morphologically productive category of
causative \citep{ageeva2014,daniel2019}. The aim of this study is to
identify and investigate the means of building periphrastic
constructions with causative semantics, with a verb that functions as a
separate cause predicate in the construction (``causative verb'' below). As
noted in \citet[151–194]{harris-campbell1995}, biclausal structures may
undergo simplification over the history of a language and end up as a
fused clause. In this paper I shall briefly discuss the degree of
grammaticalization of periphrastic causative constructions in Mehweb by
considering their clause structure.}

I propose the following research questions:

\begin{enumerate}[topsep=\medskipamount,itemsep=0pt,partopsep=0pt,parsep=0pt]
\item
  Are there any grammaticalization effects in constructions with
  causative verbs?
\item
  What are the meanings these constructions express, in addition to
  causation?
\item
  What is the syntactic structure of periphrastic causatives? Are there
  any syntactic constraints on building such constructions?
\item
  Is there any difference between constructions involving animate or
  inanimate causees?
\end{enumerate}

The paper is divided into five sections. They present the results of
syntactic tests applied in order to detect whether these constructions
are periphrastic causatives or not. \sectref{Synthetic-and-suppletive-causatives} surveys  possible ways
of non-periphrastic expression of the causative meaning, including
synthetic and suppletive causatives. \sectref{Periphrastic-causativization} introduces lexical verbs
participating in periphrastic causative constructions. \sectref{The-syntax-of-causatives}
considers the syntax of such constructions in more detail, in
particular, what types of verbs are allowed to be used with each
causative verb. In \sectref{Negation}, some aspects of forming negative causative
clauses are discussed. Finally, \sectref{Double-causative} provides some evidence on the double
causative construction.

% 2.
\section{Synthetic and suppletive causatives}\label{Synthetic-and-suppletive-causatives}

There are three possible ways of expressing causative meaning in Mehweb:
synthetic (morphological), suppletive (lexical) and analytic
(periphrastic).

Synthetic causatives are formed by adding an affix to the verbal stem.
Synthetic means of expressing causation usually produce monoclausal
structures, with no lexical predicate added to the syntactic
structure. In Mehweb, the causative affix \emph{-aq-} is used. It has an
allomorph \emph{-aχaq-} with a very limited distribution. The affix can
be added to both perfective and imperfective verb bases.

\ea % (1)
\gll \emph{abaj-ni} \emph{urši-li-ze} \emph{kung} \emph{\textbf{b-aˤld-aˤq-ib}.} \\ 
mother-\textsc{erg} boy-\textsc{obl}-\textsc{inter}(\textsc{lat}) book \textbf{\textsc{n}-hide:\textsc{pfv}-\textsc{caus}-\textsc{aor}}\\
\glt `Mother made her son hide a book.'
\z


This way of causative derivation is highly productive in Mehweb. The
causative affix can be added to all kinds of verbs. For further
discussion of morphological causative formation see \citet{ageeva2014} and \citet{daniel2019}.

Suppletive causatives are also called ``covert'' causatives \citep{kulikov2001},
since they share no morphological material with their
non-causative equivalents. The English pair \emph{kill} and \emph{die}
is commonly treated as an example of suppletive
causativization. In Mehweb, the pair \textsc{cl}\footnote{Here and further I will
  use glossing \textsc{cl}- to refer to a gender agreement slot (on verb
  agreement morphology, see \citealt{daniel2019} [this volume]).}\emph{-aˤbʡas} `to
kill' and \textsc{cl}\emph{-ebk'es} `die' is also an example of
lexical causativization.


% 3.
\section{Periphrastic causativization}\label{Periphrastic-causativization}

\is{causative, periphrastic|(}

The constructions considered in this paper (originally) represent
complementation with several matrix verbs\footnote{Further, verbal forms
  from the list will be given with the perfective stem as a quotation
  form.}:

\begin{itemize}[topsep=\medskipamount,itemsep=0pt,partopsep=0pt,parsep=0pt]
\item
    \emph{aʔas} `drive:\textsc{pfv}' – \emph{ʔes} `drive:\textsc{ipfv}' (cause to move, for
  sheep)
\item
    \textsc{cl}\emph{-aqas} `leave:\textsc{pfv}' – \textsc{cl}\emph{-irqes}
  `leave:\textsc{ipfv}' (leave something, let stay)
\item
    \textsc{cl}\emph{-aq'as} `do:\textsc{pfv}' – \textsc{cl}\emph{-iq'es}
  `do:\textsc{ipfv}'
\end{itemize}

Compare the two causative constructions in (\ref{ex:6:2}). Ex.~(\ref{ex:6:2}a)
illustrates the synthetic causative expression. (\ref{ex:6:2}b) conveys the
causative meaning, but involves two verbs. The main predicate is the
verb \emph{aʔib} `drove', and its dependent argument is the verb of
caused action (\textsc{cl}\emph{-aˤldes} `hide').

\ea \label{ex:6:2} % (2)
\ea % a.
\gll \emph{abaj-ni} \emph{urši-li-ze} \emph{kung} \emph{\textbf{b-aˤld-aˤq-ib}.}\\
mother.\textsc{obl}-\textsc{erg} boy-\textsc{obl}-\textsc{inter}(\textsc{lat}) book \textsc{n}-hide:\textsc{pfv}-\textsc{caus}-\textsc{aor}\\

\ex % b.
\gll \emph{abaj-ni} \emph{urši} \emph{kung} \emph{\textbf{b-aˤld-es}} \emph{aʔib.}\\
mother.\textsc{obl}-\textsc{erg} boy book \textsc{n}-hide:\textsc{pfv}-\textsc{inf} drive:\textsc{pfv}-\textsc{aor}\\
\glt `Mother made her son hide a book.'
\z
\z

{\addfontfeature{LetterSpace=-.2}
The lexical meaning of the verb \emph{aʔas} `drive, cause to move'
involves caused motion, describing the action
of driving e.g.\ a herd. The lexical meaning of the verb
\textsc{cl}\emph{-aqas} is `leave', `leave behind', `let stay where it
is' and expresses the permissive caused motion. Consider examples of
non-causative uses of these verbs:}

\ea % (3)
\gll \emph{adaj-ni} \emph{aʔ-ib} \emph{maza} \emph{ʡaˤjne.}\\
father-\textsc{erg} drive:\textsc{pfv}-\textsc{aor} ram yard.\textsc{in}(\textsc{lat})\\
\glt `Father drove ram into the yard.'

\ex %  (4)
\gll \emph{adaj-ni} \emph{b-aq-ib} \emph{inc} \emph{ustuj-če-b.}\\
father-\textsc{erg} \textsc{n}-leave:\textsc{pfv}-\textsc{aor} apple table.\textsc{obl}-\textsc{super}-\textsc{n}(\textsc{ess})\\
\glt `Father left an apple on the table.'
\z

According to \citet{song2001}, analytic causatives include two predicates.
One is the {predicate of cause}, namely a verb that
expresses causative impact. It has two functions: (1) to introduce a new
argument (the causer), and (2) to establish the new position of the
causee. The other predicate which functions as a lexical argument to the
predicate of cause is called the {predicate of effect}. It fills
the slot established by the predicate of cause. For instance, in
\emph{The concierge made the lobby boy carry the bags on his own}
the predicate of cause is the verb \emph{make} and \emph{carry} is the
predicate of effect. Below, I follow this terminology.

I will discuss the causative constructions produced by combining cause
and effect predicates. Note that the verbs used as predicate of cause
continue to be used in their lexical meaning, and this meaning involves
an element of causation. The question is thus whether these verbs should
be considered grammaticalized expressions of causation. Below, I argue
that there is linguistic evidence to conclude that they are, to some
extent, grammaticalized.

% 3.1.
\subsection{The structure of the periphrastic construction}

In Mehweb the syntactic structure of causative constructions requires
using a finite predicate of cause and a non-finite predicate of effect.
The predicate of cause functions as the predicate of a simple
transitive sentence, with its A (the causer) in ergative case and the
causee in the absolutive case. The effect predicates are represented by
infinitives, either perfective or imperfective (see (\ref{ex:6:5}a–b)). Other
verbal forms are ungrammatical, either finite or non-finite; cf.
examples (\ref{ex:6:5}c–e) with the aorist, imperfective past and perfective
converb, respectively.

\ea \label{ex:6:5} % (5)  
\ea % a.
\gll \emph{adaj-ni} \emph{kung} \emph{urši} \emph{\textbf{b-elč'-es}} \emph{aʔ-ib.} \\
father-\textsc{erg} book boy \textbf{\textsc{n}-read:\textsc{pfv}-\textsc{inf}} drive:\textsc{pfv}-\textsc{aor}\\
\glt `Father made his son read the book.'

\ex %  b.
\gll \emph{adaj-ni} \emph{kung} \emph{urši} \emph{\textbf{luč'-es}} \emph{aʔ-ib.} \\
father-\textsc{erg} book boy \textbf{read:\textsc{ipfv}-\textsc{inf}} drive:\textsc{pfv}-\textsc{aor}\\
\glt `Father made his son be reading the book.'

\ex % c.
\gll *\emph{adaj-ni} \emph{kung} \emph{urši} \emph{\textbf{b-elč'-un}} \emph{aʔ-ib.} \\
father-\textsc{erg} book boy \textbf{\textsc{n}-read:\textsc{pfv}-\textsc{aor}} drive:\textsc{pfv}-\textsc{aor}\\
\glt
`Father made his son read the book.'

\ex %  d.
\gll *\emph{adaj-ni} \emph{kung} \emph{urši} \emph{\textbf{luč'-ib}} \emph{aʔ-ib.} \\
father-\textsc{erg} book boy \textbf{read:\textsc{ipfv}-\textsc{ipft}} drive:\textsc{pfv}-\textsc{aor}\\
\glt `Father made his son be reading the book.'

\ex % e.
\gll *\emph{adaj-ni} \emph{kung} \emph{urši} \emph{\textbf{b-elč'-i-le}}  \emph{aʔ-ib.}\\
father-\textsc{erg} book boy \textbf{\textsc{n}-read:\textsc{pfv}-\textsc{aor}-\textsc{cvb}} drive:\textsc{pfv}-\textsc{aor}\\
\glt `Father made his son read the book.'
\z
\z


The word order is not strict, but there is a preference for \textsc{sov}.
Considering the clausality of the whole construction, we may expect
the object `book' of the embedded verb `read' to be adjacent to it, but
it is not. This is, however, not a good criterion for postulating
biclausal structure. Native speakers do not seem to be very sensitive to
changing word order of the direct and indirect object in the examples
above. The finite verb is typically in the final position, and the
infinitive immediately precedes it. These two forms cannot be separated
by an additional phrase, e.g.\ by a temporal adverb (see (\ref{ex:6:6}c);
the rule is only relevant in case if both verbal forms are located at
the end of the phrase).


\ea \label{ex:6:6} % (6)
\ea % a.
\gll \emph{abaj-ni} \emph{rasul} \emph{q'ar} \emph{\textbf{iˤšq-es}}  \emph{\textbf{iʔ-an}}  \emph{har} \emph{barħi.}\\
 mother.\textsc{obl}-\textsc{erg} Rasul grass mow:\textsc{ipfv}-\textsc{inf} drive:\textsc{ipfv}-\textsc{hab} every day\\


\ex % b.
\gll  \emph{har} \emph{barħi} \emph{abaj-ni}  \emph{rasul} \emph{q'ar} \emph{\textbf{iˤšq-es}}  \emph{\textbf{iʔ-an}.} \\
every day mother.\textsc{obl}-\textsc{erg} Rasul grass mow:\textsc{ipfv}-\textsc{inf} drive:\textsc{ipfv}-\textsc{hab}\\

\ex % c.
\gll *\emph{abaj-ni} \emph{rasul} \emph{q'ar} \emph{\textbf{iˤšq-es}}  \emph{har} \emph{barħi} \emph{\textbf{iʔ-an}.} \\
mother.\textsc{obl}-\textsc{erg} Rasul grass mow:\textsc{ipfv}-\textsc{inf} every day drive:\textsc{ipfv}-\textsc{hab}\\
\glt `Mother makes Rasul mow the lawn every day.'
\z
\z 

The scope of the temporal phrase depends on the context. Sometimes
the temporal or adverbial phrase belongs to the main clause, sometimes
it belongs to the subordinate clause. Both readings are available when
the temporal phrase is placed at the border between the two clauses.
Consider the next example:

\ea \label{ex:6:7} % (7)
\ea % a.
\gll \emph{adaj-ni}  \emph{urši}  \emph{aʔ-ib} \emph{har} \emph{barħi} \emph{mašina} \emph{as-es.} \\
father-\textsc{erg} boy  drive:\textsc{pfv}-\textsc{aor} every day car take:\textsc{pfv}-\textsc{inf} \\

\ex % b.
\gll \emph{adaj-ni}  \emph{urši}  \emph{aʔ-ib} \emph{har} \emph{barħi} \emph{mašina} \emph{is-es.}\\
father-\textsc{erg} boy  drive:\textsc{pfv}-\textsc{aor} every day car take:\textsc{ipfv}-\textsc{inf}  \\


\glt `Every day the father made his son buy a car.'

`The father made his son buy a car every day.'
\z
\z

In (\ref{ex:6:7}), even though the cause predicate has perfective aspect,
there are no restrictions on the aspect of the effect predicate. The
same is observed in constructions with the cause predicate in the
imperfective, where either imperfective or perfective effect predicates
are allowed. In other words, aspectual categories of the cause and effect
predicates are mutually independent.

Causative semantics has two major subtypes: (a) something is made/urged
to be done/happen (factitive causative), and (b) something is not
prevented from being done (permissive causative). The first meaning is
associated with the verb \emph{aʔas} `drive'. The second meaning is
associated with the verb \textsc{cl}\emph{-aqas} `leave'.


% 3.2.
\subsection{The use of \emph{aʔas} `drive'}

Factitive causatives (English constructions with \emph{make},
\emph{force}, \emph{get} or \emph{have} someone (to) do something) are
formed by means of the verb \emph{aʔas} `drive'. The causee usually is
an animate object. Inanimate objects are incompatible with the semantics
of coercion. They can be urged to do something, but due to their lack of volition, they cannot comply (see below for
exceptions). The causer is marked with the ergative, while the causee
carries the absolutive. Consider examples (\ref{ex:6:8}–\ref{ex:6:10}):

\ea \label{ex:6:8} % (8)
\gll \emph{pat'imat-ini} \emph{anwar} \emph{uˤq'-es} \emph{aʔ-ib.} \\
 Patimat-\textsc{erg} Anwar \textsc{m}.go:\textsc{pfv}-\textsc{inf} drive:\textsc{pfv}-\textsc{aor} \\
\glt `Patimat made Anwar go away.'

\ex % (9)
\gll \emph{sovet-ini} \emph{direktur} \emph{uškul} \emph{q'-aˤbʡ-es} \emph{aʔ-ib.}\\
 administration-\textsc{erg} principal school \textsc{pv}-close:\textsc{pfv}-\textsc{inf} drive:\textsc{pfv}-\textsc{aor}\\
\glt `Administration made the principal close the school.'

\ex \label{ex:6:10} % (10)
\gll \emph{*anwal-li-ni} \emph{inc} \emph{b-erħ-es} \emph{aʔ-ib.}\\
Anwar-\textsc{obl}-\textsc{erg} apple \textsc{n}-rotten:\textsc{pfv}-\textsc{inf} drive:\textsc{pfv}-\textsc{aor}\\
\glt `Anwar made the apple rot.'
\z

The causer is typically an animate agent. However, it is
also possible to have an inanimate causer. These uses seem to be explained through
personification, attributing control to natural forces.

\ea \label{ex:6:11} % (11)
\gll \emph{izaj-ni} \emph{abaj-la} \emph{beč'} \emph{ulč'-es} \emph{aʔ-ib.}\\
illness.\textsc{obl}-\textsc{erg} mother.\textsc{obl}-\textsc{gen} head be.bald:\textsc{ipfv}-\textsc{inf} drive:\textsc{pfv}-\textsc{aor}\\
\glt 
`The illness made mother grow bald.'

\ex % (12)
\gll \emph{izaj-ni} \emph{anwar} \emph{balnica-le-ħe} \emph{uˤq'-es} \emph{aʔ-ib.}\\
illness.\textsc{obl}-\textsc{erg} Anwar hospital-\textsc{obl}-\textsc{in}(\textsc{lat}) \textsc{m}.go:\textsc{pfv}-\textsc{inf} drive:\textsc{pfv}-\rlap{\textsc{aor}}\\
\glt `The illness caused Anwar to go to hospital.'
\z

In (\ref{ex:6:13}a) the snow appears as a human causer, not a natural force. In a
more realistic situation, for instance after a meltdown in the
mountains, the sentence would be as in (\ref{ex:6:13}b).

\ea \label{ex:6:13} % (13)
\ea % a.
\gll \emph{doˤʜi-li-ni} \emph{ħark'ʷ} \emph{χʷala} \emph{b-aq'-as} \emph{aʔ-ib.}\\
snow-\textsc{obl}-\textsc{erg} river big \textsc{n}-do:\textsc{pfv}-\textsc{inf} drive:\textsc{pfv}-\textsc{aor}\\
\glt `Snow has made a river become [\emph{lit}. to be done] bigger.'

\ex % b.
\gll \emph{doˤʜi-li-ni} \emph{ħark'ʷ} \emph{χʷala} \emph{b-aq'-ib.}\\
snow-\textsc{obl}-\textsc{erg} river big  \textsc{n}-do:\textsc{pfv}-\textsc{aor}\\
\glt `Snow has made the river big.'
\z
\z

Examples with an inanimate causee are not common, but not very difficult to
construct. The consultants produce them freely and do not have troubles
in identifying the participants' roles. More about the third kind of causative with `do' see in \sectref{Adjectival-causative}.

\ea % (14)
\gll \emph{ʡali-ni} \emph{adaj-la} \emph{sune-če-l} \emph{naˤʁ} \emph{aq b-aq'-as} \emph{aʔ-ib.}\\
Ali-\textsc{erg} father-\textsc{gen} self.\textsc{obl}-\textsc{super(lat)}-\textsc{emph}(\textsc{lat}) hand up \textsc{n}-\(do):\textsc{pfv}-\textsc{inf} drive:\textsc{pfv}-\textsc{aor}\\
\glt `Ali made his father raise a hand against him.'
\z

The causative construction with the verb \emph{aʔas} is, thus, flexible.
It allows using an inanimate as well as an animate causer. The same
applies to the causee. In particular, in example (\ref{ex:6:11}), the illness is
presented as something physically real which functions as a living
creature (fairy tale style). While consultants allow such uses, they do not produce them as
first answer in the elicitation task but simply accept a constructed
sentence. In any case, it is important that there are no strict
constraints on animacy of the participants.

% 3.3.
\subsection{Permissive causative with \textsc{cl}\emph{-aqas} `leave'}

\is{causative, permissive|(}

In the permissive construction, the causer permits rather than causes
the causee to bring about the caused event. In Mehweb, it is usually
expressed by means of the verb \textsc{cl}\emph{-aqas} `leave'. The
causer carries ergative marking, while the causee is in the absolutive.
Consider some examples with different effect predicates (\ref{ex:6:15}–\ref{ex:6:17}):

\ea \label{ex:6:15} % (15)
\gll \emph{sovet-ini} \emph{direktur} \emph{uškul} \emph{q'-aˤbʡ-es} \emph{w-aq-ib.}\\
 administration-\textsc{erg} principal school \textsc{pv}-close:\textsc{pfv}-\textsc{inf} \textsc{m}-leave:\textsc{pfv}-\textsc{aor}\\
\glt `Administration let the principal close the school.'

\ex % (16)
\gll \emph{adaj-ni} \emph{dursi} \emph{urši} \emph{qum-art-es} \emph{d-aq-ib.}\\
father-\textsc{erg} girl boy forget-\textsc{lv}:\textsc{pfv}-\textsc{inf} \textsc{f1}-leave:\textsc{pfv}-\textsc{aor}\\
\glt `Father let his daughter forget the boy.'
\z

One of the main contexts for the permissive is a positive response to
request. For instance, in (\ref{ex:6:17}), it is entailed that, before kissing
Patimat, Anwar actually asked permission for this action.

\ea \label{ex:6:17} % (17)
\gll \emph{pat'imat-ini} \emph{anwar} \emph{w-aq-ib} \emph{umma} \emph{d-aq'-as.}\\
Patimat-\textsc{erg} Anwar \textsc{m}-leave:\textsc{pfv}-\textsc{aor} kiss \textsc{npl}-do:\textsc{pfv}-\textsc{inf}\\
\glt `Patimat let Anwar kiss her.'
\z

On the other hand, there may be no inquiries or requests, and the causer
is introduced as an independent agent. Inanimate causees are widespread in such contexts.
Consider some examples:


\ea \label{ex:6:18} % (18)
\gll \emph{rasuj-ni} \emph{šin} \emph{rurq-es} \emph{d-aq-ib.}\\
Rasul.\textsc{obl}-\textsc{erg} water flow:\textsc{ipfv}-\textsc{inf} \textsc{npl}-do:\textsc{pfv}-\textsc{aor}\\
\glt ‘Rasul let the water flow.’ (did not prevent this from happening)

\ex \label{ex:6:19} % (19)
\gll \emph{rasuj-ni} \emph{uq'laha} \emph{abx-es} \emph{b-aq-ib.}\\
Rasul.\textsc{obl}-\textsc{erg} window open:\textsc{pfv}-\textsc{inf} \textsc{n}-leave:\textsc{pfv}-\textsc{aor}\tabularnewline\\
\glt 
`Rasul let the window open.' (did not prevent this from happening)
\z

Examples like (\ref{ex:6:18}) and (\ref{ex:6:19}) can be described in terms of a physical
situation in which the causer does not interfere with what is happening to 
the causee. There are some other effect predicates that denote natural
processes. For instance, verbs like \emph{ulč'es} `become
bald', \emph{miʔ aʔʷas} `freeze', \textsc{cl}\emph{-ic'es} `melt' in
causative constructions usually are found in combination with the cause
predicate \textsc{cl}\emph{-aqas} `leave'. Cf.\ the following examples:

\ea % (20)
\ea % a.
\gll \emph{anwal-li-ni} \emph{diʔ} \emph{miʔ aʔʷ-as} \emph{b-aq-ib.}\\
Anwar-\textsc{obl}-\textsc{erg} meat freeze:\textsc{pfv}-\textsc{inf} \textsc{n}-leave:\textsc{pfv}-\textsc{aor}\\

\ex % b.
\gll \emph{*anwal-li-ni} \emph{diʔ} \emph{miʔ aʔʷ-as} \emph{aʔ-ib.}\\
Anwar-\textsc{obl}-\textsc{erg} meat freeze:\textsc{pfv}-\textsc{inf} drive:\textsc{pfv}-\textsc{aor}\\
\glt  `Anwar froze the meat.'
\z
\z

\ea \label{ex:6:21} % (21)
\ea % a.
\gll \emph{anwal-li-ni} \emph{k'ʷama} \emph{b-ac'-es} \emph{b-aq-ib.} \\
Anwar-\textsc{obl}-\textsc{erg} butter \textsc{n}-melt:\textsc{pfv}-\textsc{inf} \textsc{n}-leave:\textsc{pfv}-\textsc{aor} \\

\ex % b.
\gll \emph{*anwal-li-ni} \emph{k'ʷama} \emph{b-ac'-es} \emph{aʔ-ib.}\\
Anwar-\textsc{obl}-\textsc{erg} butter \textsc{n}-melt:\textsc{pfv}-\textsc{inf} drive:\textsc{pfv}-\textsc{aor}\\
\glt `Anwar melted butter.'
\z
\z


The permissive constructions in Mehweb are closely connected to the
original meaning of the word \textsc{cl}\emph{-aqas} `leave'. The causer
leaves the causee on its own without taking any part in the change of
its state. This is especially visible when the causer is an inanimate
object (\ref{ex:6:18}–\ref{ex:6:21}). In cases where the causee is a person (\ref{ex:6:17}), the
permissive element is evident. The permissive is then understood in a
metaphorical sense of not preventing someone's action. I interpret the
construction with \textsc{cl}\emph{-aqas} `leave' as a permissive
causative.
%
\is{causative, permissive|)}


% 3.4.
\subsection{Agreement in permissive causative construction}\label{Agreement-in-permissive-causative-construction}

The relation between case assignment and gender agreement is relevant
only for the verb \textsc{cl}\emph{-aqas} `leave', because \emph{aʔas}
`drive' does not carry any gender markers. Periphrastic causative
constructions allow two agreement patterns. The first one apparently
prevails, with the causee retaining the absolutive case (\ref{ex:6:22}a). Note that
gender agreement on the verb is controlled by the absolutive participant
(the masculine gender marker appears on the verb `leave'). The
second pattern shows marking of the causee by
inter-lative\footnote{See \citet{chechuro2019} on the use of the
  form.} case; the gender agreement changes (from masculine to neutral). There is no absolutive
participant in the matrix clause to agree with. What we observe is
distant agreement between the matrix predicate and the absolutive
argument of the dependent clause. Consultants translate both (\ref{ex:6:22}a) and
(\ref{ex:6:22}b) in the same way.

\ea \label{ex:6:22} % (22)
\ea % a.
\gll \emph{sovet-ini} \emph{\textbf{direktur}} \emph{uškul} \emph{q'-aˤbʡ-es} \emph{\textbf{w-aq-ib}.}\\
administration-\textsc{erg} principal school \textsc{pv}-close-\textsc{inf} \textsc{m}-leave:\textsc{pfv}-\textsc{aor}\\

\ex % b.
\gll \emph{sovet-ini} \emph{\textbf{direktur-li-ze}} \emph{uškul} \emph{q'-aˤbʡ-es} \emph{\textbf{b-aq-ib}.}\\
administration-\textsc{erg} principal-\textsc{obl}-\textsc{inter}(\textsc{lat}) school \textsc{pv}-close-\textsc{inf} \textsc{n}-leave:\textsc{pfv}-\textsc{aor}\\
\glt `The administration let the principal close the school.'
\z
\z

In (\ref{ex:6:23}), the causative verb shows plural agreement with the absolutive
argument in the dependent clause.

\ea \label{ex:6:23} % (23)
\gll \emph{pat'imat-ini} \emph{urši-li-ze} \emph{\textbf{d-aq-ib}} \emph{d-ix-es} \emph{heš-di} \emph{karawatu-ne} \emph{caj-li} \emph{quli.}\\
Patimat-\textsc{erg} boy-\textsc{obl}-\textsc{inter(lat)} \textsc{npl}-leave:\textsc{pfv}-\textsc{aor} \textsc{npl}-put:\textsc{pfv}-\textsc{inf} \textsc(prox)-\textsc{pl} bed-\textsc{pl} one-\textsc{obl} room.\textsc{in}(\textsc{lat})\\
\glt `Patimat let the boy carry these beds to another room.'
\z


% 3.5.
\subsection{Adjectival causative}\label{Adjectival-causative}

Adjectives form causatives by means of `do'-periphrasis, adding the verb
\textsc{cl}\emph{-aq'as} `do' (\ref{ex:6:24}b). In Mehweb, this is one of the rare
contexts where the adjective cannot be used with the attributive affix
(cf.\ \ref{ex:6:24}b and \ref{ex:6:24}c).

\ea \label{ex:6:24} % (24)
\ea % a.
\gll \emph{musa} \emph{zuba-l.}\\
Musa blind-\textsc{atr}\\
\glt `Musa is blind.'

\ex % b.
\gll \emph{χaj-ni} \emph{musa} \emph{zuba} \emph{w-aq'-ib.}\\
khan.\textsc{obl}-\textsc{erg} Musa blind \textsc{m}-do:\textsc{pfv}-\textsc{aor}\\
\glt `Khan blinded Musa.'   

\ex % c.
\gll *\emph{χaj-ni} \emph{musa} \emph{zuba-l}  \emph{w-aq'-ib.}\\
khan.\textsc{obl}-\textsc{erg} Musa blind-\textsc{atr} \textsc{m}-do:\textsc{pfv}-\textsc{aor}\\
\glt  `Khan blinded Musa.'
\z
\z

\removelastskip
\is{causative, periphrastic|)}


% 4.
\section{The syntax of causatives}\label{The-syntax-of-causatives}

% 4.1.
\subsection{Biclausality}

While morphological causative constructions are monoclausal,
periphrastic causatives are apparently biclausal. This means that they
have a main clause that contains the causative predicate that introduces
the causer and the dependent clause that describes the caused event. The
causee also belongs to the matrix clause. In Mehweb, the dependent
clause is headed by an infinitive (\ref{ex:6:25}).

\ea \label{ex:6:25} % (25)
\gll \emph{anwal-li-ni} \emph{rasul} \emph{abaj-ze} \emph{b-arx-le} \emph{b-urh-es} \emph{{aʔ-ib}.}\\
Anwar-\textsc{obl}-\textsc{erg} Rasul mother.\textsc{obl}-\textsc{inter}(\textsc{lat}) \textsc{n}-be.right-\textsc{cvb} \textsc{n}-tell:\textsc{pfv}-\textsc{inf} drive:\textsc{pfv}-\textsc{aor}    \\
\glt `Anwar made Rasul tell mother the truth.'
\z

In order to prove that there are two syntactic clauses in periphrastic
causative constructions, I use several tests. The first test is based on
the case of the \isi{causee}. In (\ref{ex:6:26}), two agentive participants are present.
It is impossible to have two ergative arguments in one clause. The verb
\textsc{cl}\emph{-erhʷes} `slaughter' also requires an ergative agent,
but only the verb \emph{aʔas} `drive' assigns the ergative to its agent.
The case of the causee is absolutive and is thus assigned by the
predicate of cause.

\ea \label{ex:6:26} % (26)
\ea % a. 
\gll \emph{rasuj-ni} \emph{uzi} \emph{maza} \emph{b-erhʷ-es} \emph{aʔ-ib.}\\
Rasul.\textsc{obl}-\textsc{erg} boy ram \textsc{n}-slaughter:\textsc{pfv}-\textsc{inf} drive:\textsc{pfv}-\textsc{aor}\\

\ex % b.
\gll *\emph{rasuj-ni} \emph{uzi-ni} \emph{maza} \emph{b-erhʷ-es} \emph{aʔ-ib.}\\
Rasul.\textsc{obl}-\textsc{erg} boy-\textsc{erg} ram \textsc{n}-slaughter:\textsc{pfv}-\textsc{inf} drive:\textsc{pfv}-\textsc{aor}\\
\glt  `Rasul made his son slaughter the ram.'
\z 
\z

The second test is based on agreement. The verb agrees in gender
with the absolutive participant of its clause. If the analytic causative
constituted only one clause, it would be possible for a verbal form
which is marked with a gender marker to agree with the sole
absolutive argument. In (\ref{ex:6:27}), the predicate of cause agrees with the
absolutive argument (i.e.\ the causee) in the main clause, whereas the
predicate of effect agrees in gender with the other absolutive argument.
Changing agreement so that the predicate of cause agrees with
\emph{kung} `book' is ungrammatical. Based on \sectref{Agreement-in-permissive-causative-construction}, one could
expect that distant agreement from the embedded clause is available,
because, in principle, the matrix verb may agree with the embedded
absolutive argument.

\ea \label{ex:6:27} % (27)
\ea % a.
\gll \emph{adaj-ni}  \emph{urši} \emph{kung} \emph{b-elč'-es} \emph{iʔ-uwe}  \emph{le-w.} \\
father-\textsc{erg}  boy book \textsc{n}-read:\textsc{pfv}-\textsc{inf} drive:\textsc{ipfv}-\textsc{cvb.ipfv}  \textsc{aux}-\textsc{m}\\

\ex % b.
\gll \emph{*adaj-ni}  \emph{urši} \emph{kung} \emph{b-elč'-es} \emph{iʔ-uwe}  \emph{le-b.}\\
father-\textsc{erg}  boy book \textsc{n}-read:\textsc{pfv}-\textsc{inf} drive:\textsc{ipfv}-\textsc{cvb.ipfv}  \textsc{aux}-\textsc{n}\\
\glt `Father made his son read the book.'
\z
\z

The periphrastic causative construction contains two absolutive
arguments. Only one of them controls the agreement of the causative
verb. The other triggers agreement on the predicate of effect. It is
thus biclausal.

% 4.2.
\subsection{Types of predicates of effect}

The predicate of effect fills the valency of the causative verb. In all
periphrastic causative constructions the causer gets ergative marking,
while the causee appears in the absolutive or inter-lative
case. All other arguments preserve their case marking. Below, different
possible types of effect predicates with the verb \emph{aʔas} `drive'
(factitive causatives) are discussed. The permissive causative verb
\textsc{cl}\emph{-aqas} `leave' behaves in exactly the same way.

% 4.2.1.
\subsubsection{A-intransitive verbs and Р-intransitive verbs}

In general, intransitive verbs are more frequently causativised. An
agentive intransitive verb takes one lexical subject in the absolutive
case and represents an action, as \emph{duc'} \textsc{cl}\emph{-uqes}
`run' in (\ref{ex:6:28}).

\ea \label{ex:6:28} % (28)
\ea % a.
\gll \emph{anwar} \emph{duc'} \emph{uq-un.}  \\
Anwar run \textsc{m}.\textsc{lv}:\textsc{pfv}-\textsc{aor} \\
\glt `Anwar ran.'

\ex % b.
\gll \emph{učitej-ni} \emph{anwar} \emph{duc'} \emph{uq-es} \emph{aʔ-ib.}\\
teacher.\textsc{obl}-\textsc{erg} Anwar run \textsc{m}.\textsc{lv}:\textsc{pfv}-\textsc{inf} drive:\textsc{pfv}-\textsc{aor}\\
\glt `The teacher made Anwar run.'
\z
\z

The difference between A- and P-intransitive verbs is the degree of
control of the subject. While the subject of A-intransitive controls the
situation they are involved in, the subject of P-intransitive does not.
cf.\ (\ref{ex:6:29}):

\ea \label{ex:6:29} % (29)
\ea % a.
\gll \emph{inc} \emph{b-erħ-ib.}\\
apple \textsc{n}-rotten:\textsc{pfv}-\textsc{aor}\\
\glt `The apple has rotten.'

\ex % b.
\gll \emph{anwal-li-ni} \emph{inc} \emph{b-erħ-es} \emph{b-aq-ib.} \\
Anwar-\textsc{obl}-\textsc{erg} apple \textsc{n}-rotten:\textsc{pfv}-\textsc{inf} \textsc{n}-leave:\textsc{pfv}-\textsc{aor}\\
\glt `Anwar let the apple rot.'
\z
\z

% 4.2.2.
\subsubsection{Experiential verbs}

\is{experiential verbs|(}
In East Caucasian, subjects of experiential verbs are non-canonical
subjects and take non-core case marking. In Mehweb, they are coded with
the inter-lative case (\ref{ex:6:30}a), or with a dative with the verb
\textsc{cl}\emph{-iges} `want'. Under causativization, the causee
switches from inter-lative to absolutive, according to the
general scheme causee case marking in analytic causative constructions.

\ea \label{ex:6:30} % (30)
\ea % a.
\gll \emph{dursi-li-ze} \emph{urši} \emph{qum-art-ur.}\\
girl-\textsc{obl}-\textsc{inter}(\textsc{lat}) boy forget-\textsc{lv}:\textsc{pfv}-\textsc{aor}\\
\glt `The girl forgot the boy.'

\ex % b.
\gll \emph{adaj-ni} \emph{dursi} \emph{urši} \emph{qum-art-es} \emph{aʔ-ib.}\\
father-\textsc{erg} girl boy forget-\textsc{lv}:\textsc{pfv}-\textsc{inf} drive:\textsc{pfv}-\textsc{aor}\\
\glt `Father made his daughter forget the boy.'

\ex % c.
\gll \emph{*adaj-ni} \emph{dursi-li-ze} \emph{urši} \emph{qum-art-es} \emph{aʔ-ib.}\\
father-\textsc{erg} girl-\textsc{obl}-\textsc{inter}(\textsc{lat}) boy forget-\textsc{lv}:\textsc{pfv}-\textsc{inf} drive:\textsc{pfv}-\textsc{aor}\\
\glt `Father made his daughter forget the boy.'
\z
\z

With morphological causatives of experiential effect predicates, the
causee retains its inter-lative case. Consider the following
example, quoted from \citet[8]{ageeva2014}:

\ea % (31)
\ea % a.
\gll \emph{ʡali-ze} \emph{χabar} \emph{arʁ-ib.}\\
Ali-\textsc{inter}(\textsc{lat}) tale hear:\textsc{pfv}-\textsc{aor}\\
\glt `Ali heard a tale.'

\ex % b.
\gll \emph{pat'imat-ini} \emph{ʡali-ze}  \emph{χabar} \emph{arʁ-aq-ib.}\\
Patimat-\textsc{erg} Ali-\textsc{inter}(\textsc{lat}) tale hear:\textsc{pfv}-\textsc{caus}-\textsc{aor}\\
\glt `Patimat told Ali a tale.'
\z
\z

Unlike what happens in morphological causatives, in the analytic causative
construction the original marking of the causee as non-canonical subject
is ungrammatical (see \ref{ex:6:30}b).
%
\is{experiential verbs|)}

% 4.2.3.
\subsubsection{Transitive verbs}

\is{transitivity|(}

With originally transitive constructions, case marking of the causee
changes. In analytic causatives, the causer takes the ergative, leaving
the absolutive slot to the causee (\ref{ex:6:32}b). Having two ergative arguments
in one utterance is not allowed (\ref{ex:6:32}c).

\ea \label{ex:6:32} % (32)
\ea % a.
\gll \emph{uzi-li-ni} \emph{maza} \emph{b-erh-un.}\\
boy-\textsc{obl}-\textsc{erg} ram \textsc{n}-slaughter:\textsc{pfv}-\textsc{aor}\\
\glt `The son slaughtered the ram.'

\ex % b.
\gll \emph{rasuj-ni} \emph{uzi} \emph{maza} \emph{b-erhʷ-es} \emph{aʔ-ib.}\\
Rasul.\textsc{obl}-\textsc{erg} boy ram \textsc{n}-slaughter:\textsc{pfv}-\textsc{inf} drive:\textsc{pfv}-\textsc{aor}\\

\ex % c.
\gll *\emph{rasuj-ni} \emph{uzi-ni} \emph{maza} \emph{b-erhʷ-es} \emph{aʔ-ib.}\\
Rasul.\textsc{obl}-\textsc{erg} boy-\textsc{erg} ram \textsc{n}-slaughter:\textsc{pfv}-\textsc{inf} drive:\textsc{pfv}-\textsc{aor}\\
\glt `Rasul made his brother cut the ram.'
\z
\z

\removelastskip
\is{transitivity|)}

% 4.2.4.
\subsubsection{Ditransitive verbs}

\is{ditransitive verbs|(}

Ditransitive verbs take three arguments that correspond to the subject,
the recipient and the theme. As with causativization of transitive
verbs, analytic causativization of ditransitive verbs does not license
two ergative arguments. The causee is coded by inter-lative.

\ea % (33)
\ea % a.
\gll \emph{urši-li-ni} \emph{abaj-ze} \emph{arc} \emph{g-ib.}\\
boy-\textsc{obl}-\textsc{erg} mother-\textsc{inter}(\textsc{lat}) money give:\textsc{pfv}-\textsc{aor}\\
\glt `The boy gave his mother the money'

\ex % b.
\gll \emph{anwal-li-ni} \emph{urši} \emph{abaj-ze} \emph{arc} \emph{g-es} \emph{aʔ-ib.}\\
Anwar-\textsc{obl}-\textsc{erg} boy mother-\textsc{inter}(\textsc{lat}) money give:\textsc{pfv}-\textsc{inf} drive:\textsc{pfv}-\textsc{aor}\\
\glt 
`Anwar made his son give his mother the money.'
\z
\z

Causativization of transitive and ditransitive verbs thus follows the
same scheme, with the causer in ergative and the causee in a peripheral
case.
%
\is{ditransitive verbs|)}

% 5.
\section{Negation}\label{Negation}

\is{polarity|(}

Formation of a negative clause is one of several possible ways for testing the
degree of grammaticalization of causative constructions. The negation in
constructions with \emph{aʔas} `drive' is only allowed on the matrix
predicate, that is, the predicate of cause. The dependent infinitive
cannot take the negation prefix \emph{ħa}-.

\ea % (34)
\ea % a.
\gll \emph{abaj-ni} \emph{rasul} \emph{q'ar} \emph{iˤšq-es}  \emph{aʔ-ib} \emph{har} \emph{barħi.}\\
 mother-\textsc{erg} Rasul grass mow:\textsc{ipfv}-\textsc{inf}  drive:\textsc{pfv}-\textsc{aor} every day\\
\glt `Mother made Rasul mow the lawn every day.'

\ex % b.
\gll \emph{abaj-ni} \emph{rasul} \emph{q'ar} \emph{iˤšq-es} \emph{\textbf{ħa-}ʔ-ib} \emph{har} \emph{barħi.}\\
mother-\textsc{erg} Rasul grass mow:\textsc{ipfv}-\textsc{inf} \textsc{neg}-drive:\textsc{pfv}-\textsc{aor} every day\\

\ex % c.
\gll \emph{*abaj-ni} \emph{rasul} \emph{q'ar} \emph{\textbf{ʜa}ˤ{-}šq-es}  \emph{aʔib} \emph{har} \emph{barħi.}\\
mother-\textsc{erg} Rasul grass \textsc{neg}-mow:\textsc{ipfv}-\textsc{inf} drive:\textsc{pfv}-\textsc{aor} every day\\

\ex % d.
\gll \emph{*abaj-ni} \emph{rasul} \emph{q'ar} \emph{\textbf{ʜa}ˤ-šq-es}  \emph{\textbf{ħa-}ʔ-ib}  \emph{har} \emph{barħi.}\\
mother-\textsc{erg} Rasul grass \textsc{neg}-mow:\textsc{ipfv}-\textsc{inf} \textsc{neg}-drive:\textsc{pfv}-\textsc{aor} every  day\\
\glt `Mother did not make Rasul mow the lawn every day.'
\z
\z

Examples (34c) and (34d) are considered ungrammatical by consultants
no matter what meaning is implied (whether the negation scopes over the
embedded predicate `makes not to mow' or the matrix verb `does not make
mow'). Another example shows the same effect.

\ea % (35)
\ea % a.
\gll \emph{učitel-t-ini} \emph{nuša} \emph{meħʷe-la} \emph{mezi-sum} \emph{b-uʜ-aˤq'-as} \emph{\textbf{ħ}-aʔ-ib.}\\
teacher-\textsc{pl}-\textsc{erg} we in.Mehweb-\textsc{gen} language-\textsc{repl} \textsc{hpl}-talk-\textsc{lv}:\textsc{ipfv}-\textsc{inf} \textsc{neg}-drive:\textsc{pfv}-\textsc{aor}\\
\glt  `Teachers do not make us speak Mehweb [at school].'

\pagebreak

\ex % b.
\gll *\emph{učitel-t-ini} \emph{nuša} \emph{meħʷe-la} \emph{mezi-sum} \emph{\textbf{ħa}-b-uʜ-aˤq'-as} \emph{aʔ-ib.}\\
teacher-\textsc{pl}-\textsc{erg} we in.Mehweb-\textsc{gen} language-\textsc{repl}  \textsc{neg}-hpl-talk-\textsc{lv}:\textsc{ipfv}-\textsc{inf} drive:\textsc{pfv}-\textsc{aor}\\
\glt `The teachers make us not speak Mehweb [at school].'
\z
\z

On the other hand, in constructions with \textsc{cl}\emph{-aqas} `leave'
it is possible to use the negative prefix both on the predicate of effect and on the predicate of cause,
with different resulting meanings.

\ea % (36)
\gll \emph{adaj-ni} \emph{urši} \emph{zul} \emph{kak}  \emph{\textbf{ħa-b-iq'-es}}  \emph{w-aq-ib.}\\
father-\textsc{erg} boy in.the.morning pray  \textsc{neg}-\textsc{n}-do:\textsc{ipfv}-\textsc{inf} \textsc{m}-leave:\textsc{pfv}-\textsc{aor}\\
\glt `Father let his son not to do the morning prayers.'

\ex % (37)
\gll \emph{adaj-ni} \emph{urši} \emph{zul} \emph{kak} \emph{b-iq'-es} \emph{\textbf{ħa-q-ib}.}\\
father-\textsc{erg} boy in.the.morning pray \textsc{n}-do:\textsc{ipfv}-\textsc{inf} \textsc{neg}-\textsc{m}.leave:\textsc{pfv}-\textsc{aor}\\
\glt `Father did not let his son do the morning pray.'
\z

The next pair of examples illustrates the same.

\ea % (38)
\gll \emph{abaj-ni} \emph{urši} \emph{\textbf{ħa-q-ib}} \emph{uškuj-ħe} \emph{w-aš-es.}\\
mother.\textsc{obl}-\textsc{erg} boy \textsc{neg}-\textsc{m}.leave:\textsc{pfv}-\textsc{aor} school.\textsc{obl}-\textsc{in}(\textsc{lat}) \textsc{m}-go:\textsc{ipfv}-\textsc{inf}\\
\glt `Mother did not let her son go to school.'

\ex % (39)
\gll \emph{abaj-ni} \emph{urši} \emph{w-aq-ib} \emph{uškuj-ħe} \emph{\textbf{ħa-š-es}.}\\
mother.\textsc{obl}-\textsc{erg} boy \textsc{m}-leave:\textsc{pfv}-\textsc{aor} school.\textsc{obl}-\textsc{in}(\textsc{lat}) \textsc{neg}-\textsc{m}.go:\textsc{ipfv}-\textsc{inf}\\
\glt `Mother let her son not to go to school.'
\z

The examples above show the possibility of placing the negative prefix
on either the causative or the effect predicate. On the other hand, it is
considered ungrammatical to use the negative form of the infinitive of
the verb dependent on \emph{aʔas} `drive'. The verb
\textsc{cl}\emph{-aqas} `leave' forms a looser connection with its
predicate of effect and, thus, seems to be less grammaticalized than
\emph{aʔas} `drive'.
%
\is{polarity|)}


% 6.
\section{Double causative}\label{Double-causative}

\is{causative, double|(}

Morphological and periphrastic causatives may co-occur. In other words, if a
construction already contains a predicate of cause (i.e.\ \emph{aʔas
`drive'} or \textsc{cl}\emph{-aqas} `leave'), the predicate of effect
can be additionally marked with a causative affix \mbox{\emph{-aq-}}. In (\ref{ex:6:40}a)
and (\ref{ex:6:40}b), the morphological marker is optional and may be dropped,
while the analytic causative predicate remains in the sentence and the
meaning of the whole does not change.

\ea \label{ex:6:40} % (40)
\ea % a.
\gll \emph{adaj-ni} \emph{urši} \emph{kung} \emph{b-elč'-aq-es} \emph{aʔ-ib.}\\
father-\textsc{erg} boy book \textsc{n}-read:\textsc{pfv}-\textsc{caus}-\textsc{inf} drive:\textsc{pfv}-\textsc{aor}\\
\glt `Father made his son read a book'

\ex % b.
\gll \emph{adaj-ni} \emph{urši} \emph{kung} \emph{b-elč'-es} \emph{aʔ-ib.}\\
father-\textsc{erg} boy book \textsc{n}-read:\textsc{pfv}-\textsc{inf} drive:\textsc{pfv}-\textsc{aor}\\
\glt `Father made his son read a book.'
\z
\z

Constructions with inanimate causees show the same effect.

\ea \label{ex:6:41} % (41)
\ea % a.
\gll \emph{anwal-li-ni} \emph{inc} \emph{b-erħ-es} \emph{b-aq-ib.}\\
Anwar-\textsc{obl}-\textsc{erg} apple \textsc{n}-rotten:\textsc{pfv}-\textsc{inf} \textsc{n}-let:\textsc{pfv}-\textsc{aor}\\
\glt `Anwar let an apple rot.'

\ex % b.
\gll \emph{anwal-li-ni} \emph{inc} \emph{b-erħ-aq-as} \emph{b-aq-ib.}\\
Anwar-\textsc{obl}-\textsc{erg} apple \textsc{n}-rotten:\textsc{pfv}-\textsc{caus}-\textsc{inf} \textsc{n}-let:\textsc{pfv}-\textsc{aor}\\
\glt `Anwar let an apple rot.'
\z
\z

Examples (\ref{ex:6:40}) and (\ref{ex:6:41}) illustrate a double causative construction.
\citet[10]{ageeva2014} points out that it is possible to build a double
morphological causative by adding a second causative affix (cf.\
\textsc{cl}\emph{-arʡaˤqaqib} `freeze'). The meaning of the form remains the same,
with no (clear) semantic change as compared to the (simple)
morphological causative. Here, a similar phenomenon is observed in
periphrasis. Constructions with double causative marking sound natural
to native speakers and are produced spontaneously during elicitation.
Consultants easily derive double analytic causatives from all analytic
causatives discussed previously in the paper.
%
\is{causative, double|)}



% 7.
\section{Conclusions}

Periphrastic causative constructions co-exist in Mehweb with synthetic
causa\-tives. There is no difference in meaning between analytic and
morphological markers. It does not matter what syntactic type the
predicate of effect is; verbs\pagebreak[3] of all morphosyntactic classes are
allowed. There are however some structural limitations on periphrastic
causative formation.
{\emergencystretch2em\par}

There is a semantic division of labor between the causative predicates.
Factitive causativization is expressed by means of the verb \emph{aʔas}
`drive'. The permissive meaning is expressed by \textsc{cl}\emph{-aqas}
`leave'. Both predicates introduce an infinitive expressing the
predicate of effect. In adjectival causativization, the
\textsc{cl}\emph{-aq'as} `do' is used.

Cause predicates also show other differences. The verb \emph{aʔas}
`drive' only allows animate causees. The verb \textsc{cl}\emph{-aqas}
`leave' also allows inanimate causees. In both factitive and permissive
constructions, the negation marker may attach to the matrix predicate.
However, the verb \textsc{cl}\emph{-aqas} `leave' also allows negation
on the infinitive.


These differences are summarized in Table 1.

\begin{table}[h]
  % Tab. 1.
  \caption{Summary of the causative predicates.}

  \advance\tabcolsep1.5pt
  \begin{tabular}{@{}m{.15\textwidth}cccc*2{m{.11\textwidth}<{\centering}}@{}}
\toprule
 & \multicolumn{2}{c}{causer} & \multicolumn{2}{c}{causee} & \multicolumn{2}{c@{}}{negation} \tabularnewline \cmidrule(lr){2-3} \cmidrule(lr){4-5} \cmidrule(l){6-7}
& \footnotesize animate & \footnotesize inanimate & \footnotesize animate & \footnotesize inanimate & \footnotesize on the matrix predicate &  \footnotesize on the dependent predicate \tabularnewline \midrule
\emph{aʔas} `drive:\textsc{pfv}' & + & \multicolumn{1}{p{.1\textwidth}<{\centering}}{+\linebreak \footnotesize\makebox[10pt]{(personification)}} & + & \multicolumn{1}{p{.1\textwidth}<{\centering}}{+\linebreak \footnotesize (rare)} & + & \tabularnewline \midrule
\textsc{cl}\emph{-aqas} `leave/let:\rlap{\textsc{pfv}'} & + & & +  & + & + & + \tabularnewline
\bottomrule
  \end{tabular}
\end{table}

In terms of case assignment, arguments other than the causee behave
identically with all morphosyntactic types of predicates. The causer is
always marked with the ergative. Other arguments retain their original
case marking. As to the causee, the intransitive causee keeps its
original absolutive marking and the transitive causee is marked with the
inter-lative. No causative construction seems to allow two
ergative arguments, marking both the causee and the causer with the
ergative. This is similar to what happens under morphological
causativization. Morphological and analytic causativization, however,
become different if one compares what happens to the causee of
experiential predicates with originally non-canonical subjects. Under
morphological causatives, the causee keeps its original peripheral case
marking (dative or inter-lative, depending on the verb). In
analytic causative constructions, however, it obligatorily changes to the
absolutive.


Causative constructions in Mehweb may combine morphological and analytic
causative strategy together. Apparently, the meaning of such
constructions is not different from the usual causative construction
with either a synthetic or an analytic causative alone. Double
causatives seem to be semantically redundant.

The tests discussed in the paper reveal some divergences between the
constructions under consideration. The results of the negation test show
that the factitive causative construction, apparently, is more
grammaticalized than the permissive causative. It is not possible to
apply negation to the dependent verb form in constructions with the verb
\emph{aʔas} `drive', while \textsc{cl}\emph{-aqas} `leave' allows the
negation either on the main verb or on the infinitive.

The negation test and agreement tests diverge. While the negation test
in factitive causatives indicates a monoclausal structure, gender
agreement indicates two separate clauses. Only negative constructions
support grammaticalization of periphrastic causatives in Mehweb.



\section*{Acknowledgements}

I would like to express my gratitude to all Mehweb consultants who
agreed to help in this research. I also thank Nina
Dobrushina, Michael Daniel, Dmitri Ganenkov, Denis Creissels and
Samira Verhees.


\section*{List of abbreviations}

\begin{longtable}[l]{@{}ll@{}}
\textsc{aor}	& aorist \\
\textsc{atr}	& attributivizer \\
\textsc{aux}	& auxiliary \\
\textsc{caus}	& causative \\
\textsc{cl}	& gender (class) agreement slot \\
\textsc{cvb}	& converb \\
\textsc{emph}	& emphasis (particle) \\
\textsc{erg}	& ergative \\
\textsc{ess}	& static location in a spatial domain \\
\textsc{f1}	& feminine (unmarried and young women gender prefix) \\
\textsc{gen}	& genitive \\
\textsc{hab}	& habitual (durative for verbs denoting states) \\
\textsc{hpl}	& human plural (gender agreement) \\
\textsc{in}	& spatial domain inside a (hollow) landmark \\
\textsc{inf}	& infinitive \\
\textsc{inter}	& spatial domain between multiple landmarks \\
\textsc{ipft}	& imperfect \\
\textsc{ipfv}	& imperfective (derivational base) \\
\textsc{lat}	& motion into a spatial domain \\
\textsc{lv}	& light verb \\
\textsc{m}	& masculine (gender agreement) \\
\textsc{n}	& neuter (gender agreement) \\
\textsc{neg}	& negation (verbal prefix) \\
\textsc{npl}	& non-human plural (gender agreement) \\
\textsc{obl}	& oblique (nominal stem suffix) \\
\textsc{pfv}	& perfective (derivational base) \\
\textsc{pl}	& plural \\
\textsc{pv}	& preverb (verbal prefix) \\
\textsc{repl}	& replicative (nominal case) \\
\textsc{super}	& spatial domain on the horizontal surface of the landmark \\
\end{longtable}


\nocite{barylnikova2015,chechuro2015,nedjalkov-litvinov1995}
\printbibliography[heading=subbibliography,notkeyword=this]


% \iffalse
% \section*{References}

% Barylnikova 2015 – Barylnikova, D 2015, ``Quasi-Causative constructions
% in Mehweb'', Working papers by NRU HSE, series WP BRP "Linguistics", no.
% 26.

% Comrie 1989 – Comrie, B 1989, Language universals and linguistic
% typology, Chicago: University of Chicago Press.

% Chechuro 2015 – Chechuro, I 2015, Nominal Morphology of Mehweb Dargwa,
% Working papers by NRU HSE, series WP BRP "Linguistics", no. 32.

% Harris, Campbell 1995 – Harris A.C., Campbell L. Historical Syntax in
% Cross-Linguistic Perspective{74}. Cambridge: Cambridge University Press,
% 1995.

% Kulikov 2001 – Kulikov, LI 2001, Causatives, in M Haspelmath
% et al. (eds), Language typology and language universals. An
% international handbook, vol. 2. Walter de Gruyter, pp. 886–898.

% Magometov 1982 – Magometov, AA 1982, Megebskij dialect darginskogo
% jazyka, Tbilisi. Nedjalkov, Silnitsky 1973 – Nedjalkov, V \& Silnitsky,
% G 1973, The Typology of Morphological

% and Lexical Causatives, in F Kiefer (ed.), Trends in Soviet Theoretical
% Linguistics, Reidel, pp. 1–32.

% Nedjalkov, Litvinov 1995 – Nedjalkov, V \& Litvinov, V 1995, The St
% Petersburg/Leningrad Typology Group, in M Shibatani \& T Bynon
% (eds.), Approaches to language typology, Oxford, pp. 215–271.

% Song 2001 – Song, JJ 2001, Linguistic Typology: Morphology and Syntax,
% Harlow.

% Ageeva 2014 – Ageeva, EV 2014, Kauzativy v megebskom, in Devjatkina, EM
% \& Ganenkov, DS \& Mazurova, JuV i dr. (eds.), Problemy jazyka: Sbornik
% naučnyh statej po materialam
% Tretjej konferencii-školy `Problemy jazyka: vzglad molodyh učenyh'.
% Institut jazykoznanija RAN, pp. 4–11.́


% \fi
\end{document}


%%% Local Variables:
%%% mode: latex
%%% TeX-master: "../main"
%%% End:
