\documentclass[output=paper]{langsci/langscibook}
\ChapterDOI{10.5281/zenodo.573777}

\author{Sonia Cristofaro\affiliation{University of Pavia}}
\title{Implicational universals and dependencies} 
\maketitle
\begin{document}
 
\section{Introduction}
In the typological approach \is{typology}that originated from the work of Joseph
Greenberg, implicational universals \is{implicational universals}of the form X $\rightarrow$ Y
capture recurrent cross-linguistic correlations between different
grammatical phenomena X (the antecedent of the universal) and Y (the
consequent of the universal), such that  X only occurs when Y also
occurs. Y, on the other hand, can also occur in the absence of X.


Classical typological explanations for these  correlations often
invoke functional principles \is{functional principles}that  favor Y and disfavor X. For
example,  a number of implicational universals describe the
distribution of overt marking\is{overt marking} for different grammatical categories. If
overt marking is used for nominal, inanimate or indefinite direct
objects, then it is used for pronominal, animate  or definite ones. If
it is used  for inalienable possession (`John's mother', `John's
hand'), then it is used for alienable possession (`John's book'). If
it is used for singular, then it is used for plural. These universals
have  been accounted for by postulating an economy principle\is{economy principle} whereby
the use of overt marking  is favored for the categories in the consequent
of the universal (pronominal, animate, or definite objects,
alienable possession, plural) and disfavored for those in the
antecedent (nominal, inanimate, or indefinite direct objects, inalienable
possession, singular). This is assumed to be due to  the former categories being less frequent and
therefore more in need of disambiguation (\citealt{Greenberg1966universals}; \citealt{Nichols1988}; \citealt{Comrie2};
  \citealt{Dixon1994};  \citealt{TU2}; \citealt{Martinmarkedness} and
  \citealt{Haspelmath2008}, among others).

This type of explanation  accounts for
the fact that there are cases where Y occurs while X does not, rather
than the implicational correlation between the occurrence of X and that of Y.  To the extent
that they are offered as explanations for the implicational universal \is{implicational universals}
as a whole, however, the relevant functional principles \is{functional principles} are  meant to
account  also for this correlation. In this respect, there is an (often
implicit) assumption that the phenomena disfavored by some
functional principle, for example overt marking for a more frequent category, can
only take place if the phenomena favored by that principle, for
example overt marking for a less frequent category, also occur. This presupposes that the occurrence of the latter phenomena is
a precondition for the occurrence of the former, hence there is a
dependency relationship between the two.\footnote{An alternative
  possibility would be that particular principles that favor Y and
  disfavor X lead to the former being present in most languages an the
  latter being absent in many languages. In this case, the
  languages that have X would most likely also have Y, but there would
  be no dependency between X and Y. This implies, however, that Y
should be found in most of the world's languages, which is often not the
case. For example, while languages
usually do not have overtly marked inanimate direct objects and zero marked
animate ones, they often use zero marking for both. Zero
marking for animate direct objects, then, is not infrequent, so in principle it would be
perfectly possible for a language to have overtly marked inanimate direct
objects and zero marked animate ones.}

These explanations, however, have mainly been proposed based on the
synchronic distribution of the relevant grammatical phenomena, not the actual diachronic \is{diachrony} processes that give  rise to this
distribution in individual languages.  In what follows, it will be argued
that many such processes do not provide evidence for the postulated
dependencies between
grammatical phenomena, and suggest alternative ways to look at
implicational universals in general. 


\section{The diachrony of implicational universals}

\subsection{No  functional principles leading to dependency}
A first problem with assuming a dependency relationship between
different grammatical phenomena X and Y in an implicational universal \is{implicational universals}
is that, in many cases, the actual diachronic processes leading to
configurations where Y occurs while X does not do not appear to be related
to principles  that favor Y as opposed to X. A a result, there is no
evidence that there should be a dependency relationship between X and
Y due to these principles.

This is illustrated precisely by a number of processes leading to the
use of \is{zero vs. overt marking} zero vs. overt marking for different grammatical
categories. Sometimes, the initial situation is one where all of these
categories are marked overtly, and the marker for the less frequent
category is eliminated as a result of regular phonological changes.\is{phonological changes} In \ili{English}, for example, the current
configuration with zero marked singulars and {\em -s} marked plurals
resulted from a series of phonological changes that led to the
elimination of all inflectional endings except genitive singular {\em
  -s} and plural {\em -es} \citep{Mosse}.
As phonological changes are
arguably independent of the categories encoded by the affected forms,
such
cases provide no evidence that the presence  of overt
marking is related to the need to disambiguate the relevant
categories, and hence that this should lead to a  dependency between
these categories in regard to their ability to
receive overt marking.
In fact, cross-linguistically, such processes can also affect the less
frequent category.  In \ili{Sinhala}, for example, some inanimate nouns have
overtly marked singulars and
zero marked plurals
 (e.g. {\em pot-a/ pot} `book-\textsc{sg}/ book.\textsc{pl}'). This was a result of
 phonological changes \is{phonological changes} leading to the loss of the plural ending of a
 specific inflectional class \citep{Nitz-Nordhoff2010}.

In other cases, all of the relevant categories are originally zero
marked, and overt markers for the less frequent category arise as a
result of the  reinterpretation of
pre-existing elements. For example, as illustrated in \refP{kanuri} below for
\ili{Kanuri}, markers for pronominal, animate or definite direct objects are
often structurally identical to, and diachronically \is{diachrony} derived from topic
markers\is{topic}. 


\ea\label{kanuri}
\langinfo{Kanuri}{}{Nilo-Saharan}
\ea
\gll \ili{Músa} shí-\textbf{ga} cúro \\
Musa \textsc{3sg-obj} saw \\
\glt `Musa saw him' \citep[52]{Kanuri}
\ex
\gll wú-\textbf{ga}\\
\textsc{1sg}-as.for\\
\glt `as for me' \citep[52]{Kanuri}
\nomenclature{\textsc{dep.fut}}{dependent future}
\nomenclature{\textsc{impf}}{imperfect}
\nomenclature{\textsc{sg}}{singular}
\nomenclature{\textsc{obj}}{object}
\z
\z

Markers for alienable possession arise  from locative
expressions, e.g. `at the home of' and the like, as illustrated in
\refP{ngiti} for \ili{Ngiti}.

\ea\label{ngiti}
\langinfo{Ngiti}{}{Nilo-Saharan}\\
\ea
\gll ma m-ìngyè àba \textbf{bhà} {\textbari}dzalí-nga  \\
\textsc{1sg} \textsc{sc}-be.in.the.habit.\textsc{pfpr} father \textsc{poss} courtyard-\textsc{nomlzr} \\
\glt `I normally stay at the courtyard of my father' \citep[322]{Ngiti} 
\nomenclature{NOMLZ}{nominalizer}
\ex
\gll \textbf{bhà:} \\
at.home \\
\glt `at home' \citep[154]{Ngiti}
\z
\z
\nomenclature{\textsc{inal}}{inalienable}
\nomenclature{\textsc{pfpr}}{perfective present}

Plural markers can arise from a variety of  sources,
for example distributive expressions, as in \ili{Southern Paiute}, illustrated
in \refP{paiute}. Another source are partitive expressions of the type
`many of us' and the like, in which 
the quantifier is dropped and
the plural meaning associated with it is transferred to a
co-occurring element, for example a genitive case inflection originally
indicating partitivity, as illustrated  in \refP{bengali} for \ili{Bengali},
 or a
verbal form, as illustrated in \refP{assamese} for \ili{Assamese}. In this
language, the plural marker was originally a participial form
of the verb `to be'  used in expressions such as `both of them'
(literally, `(they) being two').

\ea\label{paiute}
\langinfo{Southern Paiute}{}{Uto-Aztecan}\\
\gll qa'n{\textsci} / \textbf{qa{\textipa N}qa'n{\textsci}} \\
house / house.\textsc{distr} \\
\glt `house, houses' \citep[258]{Sapir1930}
% \nocite{Paiute}
\z
\nomenclature{DISTR}{distributive}



\ea\label{bengali}
\langinfo{Bengali}{}{Indo-European}\\
\ea
\gll ch\=el\=e-\textbf{r\=a}  \\
child-\textsc{gen} \\

\glt `children' (15th century: \citealt[736]{Chatterji})
\ex
\gll \=amh\=a-\textbf{r\=a}  \\
we-GEN  \\
\glt `of us' (14th century: \citealt[735]{Chatterji})
\z
\z


\ea\label{assamese}
\langinfo{Assamese}{}{Indo-European}\\
\ea
\gll ch\=atar-\textbf{hãt} \\
student-\textsc{pl} \\
\glt `Students' (Modern \ili{Assamese}: \citealt[295]{Assamese})
\ex
\gll dui-\textbf{hanta} \\
two-be.\textsc{ptcpl} \\
\glt `Both of them' (Early \ili{Assamese}: \citealt[282]{Assamese})
\z
\z
\nomenclature{\textsc{gen}}{genitive}
\nomenclature{\textsc{ptcpl}}{participle}

These processes are plausibly \is{context-driven} context-driven, either in the sense that some element becomes associated
with a meaning that can be inferred \is{inference} from the context or in the sense
that it takes on a meaning originally associated with a co-occurring
element. Any restrictions in the distribution of the resulting markers
are directly related to the properties of the source construction.
For example, topic markers \is{topic} can become direct object markers
when they are used with topicalized direct
objects  (\citealt{Iemmolo2010}, among others). As topics are usually pronominal, animate, and definite, it
is natural that the resulting markers should be restricted to these
types of direct objects, at least initially. Possession can be inferred in many contexts involving
locative expressions (e.g., `the courtyard in my
  father's house' $>$ `my father's courtyard':
  \citealt{Claudi-Heine1986};
  \citealt[chapter 6]{Heineal.1991}), so these expressions can easily
  develop a possessive meaning. As they are not
  usually used to refer to inalienably possessed items (? `The mother
  in John's house', ? `The hand in John's house'), the resulting
  possessive markers will be restricted in the same way. 
Distributives can develop a plural
  meaning because, when applied to individuated items, they
  always involve the notion of plurality
  \fatcit{Mithun1999}{90}. Partitive expressions with plural quantifiers
  also involve the notion of plurality, so this notion is easily
  transferred from one component of the expression to
  another.

This type of process has long been described in classical historical
linguistics and grammaticalization \is{grammaticalization} studies (see, for example, \citealt{Heineal.1991},
\citealt{Bybee-Perkins-Pagliuca1994}, or
\citealt{Traugott-Dasher2005}). In all of the cases just discussed, the use of overt
marking for particular categories is a result of contextually dependent \is{context dependent}
associations that speakers  establish between those categories and
highly specific source elements. The categories  not involved in this
process retain zero marking, \is{zero marking} which was the strategy originally used for
all categories. In such cases too,  then, there is no obvious
evidence that the
distribution of overt marking \is{overt marking} reflects some principle that
favors overt marking for particular categories as opposed to others,
nor that such a principle should determine a dependency between the
use of overt marking for some category and its use for some other
category. This is further confirmed by the fact that, depending on the
source construction, some of these processes can also give rise to
markers for more frequent categories, even if less frequent categories
are zero marked in the language. In \ili{Imonda}, for example, a partitive
case ending took on a meaning component originally associated with a
co-occurring quantifier. As this process took place in expressions
involving singular quantifiers (e.g. `one of the women'), the result was the creation
of a singular marker, leading to a situation where singular is overtly
marked and plural is zero marked. This is illustrated in \refP{imonda} (the
marker is also used to indicate dual, and is therefore called 
``nonplural" in the source)\footnote{Evidence that the distribution of overt
    markers is directly related to the properties of the source
    construction is also provided by the fact that, cross-linguistically,
overt markers derived from
  sources compatible with different categories usually apply to all of
  these categories regardless of their relative 
  frequency. This is discussed in detail in \citet{Otareferential} and (\citeyear{Otacompetingmotivations})
  with regard to the
  development of direct object markers applying to all types of direct
  objects.}.

\ea\label{imonda}
\langinfo{Imonda}{}{Border}\\
\ea 
\gll agõ-{\textbf ian\`ei}-m ainam fa-i-kõhõ \\
women-\textsc{nonpl}-\textsc{gl} quickly \textsc{cl}-\textsc{lnk}-go \\
\glt `He grabbed the woman' \fatcit{Imonda}{194}
\ex
\gll mag-m ad-{\textbf ian\`ei}-m  \\
one-\textsc{gl} boys-\textsc{src}-\textsc{gl}  \\
\glt `To one of the boys' \fatcit{Imonda}{219}
\nomenclature{\textsc{nonpl}}{non-plural}
\nomenclature{\textsc{gl}}{goal}
\z
\z

\subsection{Co-occurrence patterns are not dependency patterns}
\is{Co-occurrence patterns}
Another problem for the idea of a dependency between X and Y in
implicational universals \is{implicational universals}of the form  X $\rightarrow$ Y is that, in
several cases where X and Y co-occur,  the two are not actually
distinct phenomena, hence there is no evidence that one of the two is
a precondition for the other. 

When overt marking for singular co-occurs with overt marking \is{overt marking} for
plural, for example, the relevant markers are actually sometimes
gender markers that evolved from demonstratives or
personal pronouns, as is often
the case with gender markers \is{gender} \citep{Greenberggender}.  As the
source elements had distinct singular and
plural forms, the resulting gender markers end up indicating singular
and plural in addition to gender. This  process,  for instance, has been
reconstructed by \citeN{Heine1982} for \ili{Kxoe}, where a series of gender
markers with distinct singular and  plural forms 
 are structurally similar to the
forms of the third person pronoun, as can be seen from Table
\ref{kxoe}.




\begin{table}
\begin{tabular}{llllll}
\lsptoprule
& & Nouns & & Pronouns &\\
\midrule
\textsc{sg} &\textsc{m} &/õ{\acm{a}}-{\textbf mà} &`boy' &xà-{\textbf má}, á-{\textbf mà}, i-{\textbf mà}&`he'\\
&\textsc{f} &/õ{\acm{a}}-{\textbf h\`{\textepsilon}}&`girl' &xà-{\textbf
  h\`{\textepsilon}}, á--{\textbf h\`{\textepsilon}}, i--{\textbf
  h\`{\textepsilon}} &`she'\\
&\textsc{c} &/õ{\acm{a}}-{\textbf ('à)}, /õ{\acm{a}}-{\textbf djì} &`child'
&(xa-'{\textbf à})& `it'\\
\textsc{pl} &\textsc{m}&/õ{\acm{a}}-{\textbf //u`a}& `boys'&xà-{\textbf
  //{\textsubarch{u}}á}, á-{\textbf //{\textsubarch{u}}á},
í-{\textbf //{\textsubarch{u}}á}& `they'\\
&\textsc{f}&/õ{\acm{a}}-{\textbf djì}&`girls'&xà-{\textbf djí}, á-{\textbf djí}, í-{\textbf djí}& `they'\\
&\textsc{c}&õ{\acm{a}}-{\textbf nà} &`children' &xà-{\textbf nà}, á-{\textbf nà}, í-{\textbf nà}& `they'\\
\lspbottomrule
\end{tabular}
\caption{Gender/number markers and third person pronouns in \ili{Kxoe} (\ili{Khoisan}: Heine 1982: 211)}\label{kxoe}
\nocite{Heine1982}
\end{table}
\nomenclature{C}{common}


As the singular and plural markers are originally 
different
paradigmatic forms of the same source element (one not specifically
used to indicate number), cases like this provide no
evidence that there is a dependency between overt marking for
singular and overt marking for plural in themselves. 
To prove this,
one would need cases where singular and plural markers develop
through distinct processes. It is not clear, however, how many of the cases where
singular and plural markers co-occur synchronically are actually of
this type.

A similar example is provided by a word order universal discussed by
Hawkins (\citeyearNP{Hawkins1983}; \citeyearNP{Hawkins2004}). In prepositional languages, \is{prepositional languages} 
 if the relative clause precedes the
noun, then so does the possessive phrase. Hawkins  accounts for this by
assuming that, since relative clauses are structurally more complex
than possessive phrases, the insertion of the former between the
preposition and the noun creates a configuration more difficult to
process than the insertion of the latter. Thus, a language will permit
the more difficult configuration only if it also permits the easier one. 

\citet{Aristar1991} shows, however, that relative clauses and
possessive phrases sometimes represent an evolution of the same construction,
 one
where an expression involving a demonstrative is in
apposition to a head noun, e.g. `That (who) Verbed, X' or `That (of) Y,
X', which give rise, respectively, to `The X who Verbed' and `The X of
Y', with the demonstrative evolving into a genitive and a relative
marker. Evidence of this process is provided for example by \ili{Amharic}
(one of the languages considered by Hawkins), where the
same element, derived from a demonstrative, is used both as a relative
and as a possessive marker (\citealt{AmharicCohen}; \citealt{Amharic}).


\ea\label{amharic}
\langinfo{Amharic}{}{Semitic}\\
\ea
\gll {\textbf y\"a}-mä{\textsubdot{t}}{\textsubdot{t}}a säw \\
\textsc{rel}-come.\textsc{perf}.\textsc{3sg} person \\
\glt `a person who came' \citep[81]{Amharic}
\ex
\gll {\textbf y\"a}-t\"amari m\"a{\textsubdot{s}}af \\
\textsc{poss}-student book  \\
\glt `a student's book' \citep[81]{Amharic}
\glend
\z
\z
\nomenclature{\textsc{art}}{article}
\nomenclature{\textsc{rel}}{relative}

In such cases too, there is no evidence of a dependency
between preposed relatives and preposed possessive phrases in
themselves, because the reason why both the relative clause and the
possessive phrase precede the noun is that this was the order of the
demonstrative phrase from which they both derive. Evidence for the
correlation could be provided by cases where preposed relative clauses
and preposed possessive phrases develop independently, but, once again,
it is not clear how many of the synchronic cases where the two
co-occur are actually of this type.

\section{Accounting for unattested configurations: goal-oriented
  vs. source-oriented explanations}
\is{goal oriented vs source oriented}

\largerpage
The idea that the configurations described by an implicational
universal \is{implicational universals} X $\rightarrow$ Y reflect the properties of particular
source constructions and developmental processes provides no specific
explanation for why X does not usually occur in the absence of Y. In
theory, this could still be viewed as evidence that there must be
some general functional principle \is{functional principles} that disfavors X as opposed to Y,
leading to a dependency relationship between the two. In this case,
however, it is necessary to explain how such a principle could interact
with the actual, apparently unrelated diachronic processes \is{diachrony} leading to the configurations described by the
universal.


One possibility would be to suppose that the
principle provides the ultimate motivation for individual diachronic
processes. For example, overt markers for less frequent categories
develop through several processes of reinterpretation of different
source elements, but these processes 
could all somehow be triggered by the relative need to give overt
expression to those categories. Likewise, phonological erosion \is{erosion!phonological} of 
markers used for more frequent categories could ultimately be
related to the lower need to give overt expression to those
categories.  \is{overt marking}

These assumptions, however, 
are not part of any standard account of the relevant processes in historical
linguistics\is{historical linguistics}, and they are not supported by any kind of direct evidence
 (see
\citealt{Otareferential} and \citeyear{Otacompetingmotivations} for
further discussion). Rather, some  processes provide
evidence to the contrary. 
For
example, when markers for particular categories develop through the
reinterpretation of pre-existing elements, the language often already
has other markers for those categories. This supports the idea that
such processes are a result of  context-driven inferences, \is{inference}
not the relative need to give overt expression to particular categories.
Also,
some of the processes that give rise to configurations where Y
occurs while X does not can also give rise to the opposite
configuration. For example, as mentioned above, phonological erosion can target both
markers for more frequent categories and markers for less frequent
categories, leading to configurations where more frequent categories are overtly marked \is{overt marking}
and less frequent categories are zero marked.\is{zero marking} Likewise, depending on
the source construction, some processes of context-driven reinterpretation can 
give rise both to markers for less frequent categories and markers for
more frequent categories, leading to configurations where less
frequent categories
are zero marked and more frequent categories are overtly marked.
 This suggests that whether or not X can occur without Y actually
depends on particular processes and source constructions that give
rise to X, rather than any principle specifically pertaining to X or Y in themselves.


Another possibility would be that
particular functional principles\is{functional principles} that favor Y as opposed to X are
responsible for differential transmission rates \is{transmission}for X and Y within a
speech community, ultimately leading to the loss or maintenance \is{loss vs. maintainence} of
different configurations involving X and Y. For example, it could be
the case that, while the development of overt marking for particular
categories is independent of the relative frequency of those
categories, 
overt marking \is{overt marking}for  less frequent categories is more easily
transmitted than overt marking for more frequent categories
because the latter are less in need of disambiguation. This could eventually lead to
the loss of configurations where more frequent categories are overtly marked\footnote{Note, however, that this predicts that configurations
  where more frequent and less frequent categories are both overtly marked should not
  occur, or should be relatively rare, which is not the case.}. 

As suggested
by a referee, this would
be the equivalent of the technical distinction between proximate vs. ultimate \is{proximate vs. ultimate} explanations in
evolutionary biology \is{evolutionary biology} (\citealt{proximateultimate}, among many others):
the development of particular traits is independent of the fact that
those traits confer an evolutionary advantage to the organisms
carrying them, but this provides the ultimate explanation for their
distribution in a population.  In evolutionary biology, \is{evolutionary biology} however, this idea
is based on the fact that particular traits are demonstrably adaptive
to the environment, in the sense that they make it more likely for the organisms
carrying them to survive and pass them on to their
descendants. For languages, there is generally no
evidence
that particular functional properties of grammatical
constructions (e.g. the fact that they conform to a principle of
economy) are adaptive, in the sense of these properties making it
demonstrably more
likely for the construction to be transmitted from one speaker to
another. This is a crucial difference between linguistic
evolution and biological evolution, and there is a long tradition of
linguistic thought in which the transmission of individual
constructions within  a speech community is entirely determined by
social factors independent of particular functional properties of the
construction (see, for example, \citealt{McMahon1994}
and \citealt{BillLC} for reviews of the relevant issues and literature).

\largerpage
In general, diachronic evidence \is{diachrony}
suggests a different way to tackle the
problem of why certain configurations are unattested or rare. Classical explanations of this phenomenon are goal-oriented, in the sense that they
assume that particular configurations arise or do not arise in a
language depending on whether the
properties of the configuration conform to particular principles, for
example economy or
processing ease. To the extent that individual configurations are a
result of specific developmental processes involving pre-existing
constructions, however, the issue of why certain configurations arise
or do not arise should rather be addressed by taking a source-oriented
approach,  that is, by looking at what
source constructions, contexts and developmental processes  could
give rise to those configurations, and how frequent these are. This
need not be related to any principle pertaining to the resulting configurations in
themselves, and should therefore be assessed independently.


\largerpage
\section{Concluding remarks}
Ever since Greenberg's work, implicational universals \is{implicational universals} have been regarded as one of the most important results of typological \is{typology}
research because it is generally assumed that they capture some type of dependency between
logically distinct grammatical phenomena. The fact that diachronic \is{diachrony}
data  often provide no evidence either for the principles assumed to
motivate the dependency or for the dependency in the first place
suggests that this view is at least partly biased by the adoption of
an exclusively synchronic \is{synchrony} perspective. In general, this supports the point raised by some typologists
that explanations for language universals should always be tested
against the
diachronic \is{diachrony} processes that give rise to the relevant grammatical
phenomena in individual languages (\citealt{Bybee1988},
\citeyear{Bybee2006} and \citeyear{Bybee2008}, among others; see
also \citealt{Otareferential} and \citeyearNP{Otacompetingmotivations} for a recent elaboration on this view
and \citealt{Blevins2004} for a
similar approach in phonology). 

There also is, however, a more fundamental
sense in which diachronic evidence challenges current
views of implicational universals \is{implicational universals} \is{diachrony} The use of implicational universals
to describe the attested distributional configurations for two grammatical
phenomena X and Y (that is,  given X $\rightarrow$ Y, X and Y both
present or both absent, or X absent and Y present) is usually associated with
an assumption that these configurations are manifestations of some
overarching pattern captured by the universal. 
This is
apparent from the fact that the various configurations are usually
accounted for in terms of a single principle, for example economy or
processing ease. \is{processing ease} Diachronic \is{diachrony} evidence shows, however, not only that individual principles that
can be postulated on synchronic \is{synchrony} grounds may play no role in the actual
diachronic processes that give rise to the relevant configurations,
but also that 
different configurations described by a universal can be a result of very
different processes. 

For example, the use of overt marking for both
singular and plural and its use just for plural can be a result of
different grammaticalization \is{grammaticalization} processes involving different source
constructions, such as demonstratives or personal pronouns evolving
into gender \is{gender} markers on the one hand and distributives evolving into
plural markers on the other. Different instances of
the same configuration can also be a result of very different processes. For
example, phonological erosion, meaning transfer from a
quantifier to an accompanying element, and the grammaticalization of
distributives into plural markers can all give rise to a configuration with
zero marking for singular and overt marking for plural, yet they do
not obviously have anything in common.  In fact, at least some of
these processes may also sometimes have the opposite outcome (zero marking for a more
frequent category and overt marking for a less frequent one).

These facts suggest that implicational universals \is{implicational universals}might actually 
just be sche\-mas that are general enough to capture the outputs of
several particularized  diachronic \is{diachrony}
processes, rather than theoretically significant generalizations
capturing an
overarching pattern. In domains such as biological evolution, \is{evolutionary biology} the distribution of
some trait in a population is demonstrably  related to
particular properties of that trait that are independent of its
origin. Even if the trait develops through
different mechanisms in different cases, then, its distribution will
reflect some general underlying pattern.  There is no evidence, however, that this is the case in
linguistic evolution.  In order to obtain a full understanding of
implicational universals, then, we should  focus on qualitative and
  quantitative data on different source constructions and developmental
  processes that can give rise to the distributional configurations described by individual
  universals, rather than  the configurations in themselves.

 
 \largerpage
\section*{Acknowledgements}
 I wish to thank Joan Bybee, Bill Croft,
Matthew Dryer, Spike Gildea, Martin Haspelmath, Elena Lieven, and
Seán Roberts for their feedback on previous versions of this
paper. Seán Roberts, in particular, provided extremely detailed and
stimulating comments, not all of which could be addressed here due to
space constraints. The usual disclaimers apply.

\section*{Abbreviations}
\begin{tabularx}{.45\textwidth}{>{\scshape}lQ} 
art & article\\
c & common\\
dep.fut & dependent future\\
distr & distributive\\
gen & genitive\\
gl & goal\\
impf & imperfect\\
inal & inalienable\\
\end{tabularx}
\begin{tabularx}{.45\textwidth}{>{\scshape}lQ} 
nomlz & nominalizer\\
nonpl & non-plural\\
obj & object\\
pfpr & perfective present\\
ptcpl & participle\\
rel & relative\\
sg & singular\\
\\
\end{tabularx}

% \begin{thenomenclature} 

%  \nomgroup{A}

%   \item [{ART}]\begingroup article\nomeqref {7}\nompageref{6}
%   \item [{C}]\begingroup common\nomeqref {6}\nompageref{5}
%   \item [{DEP.FUT}]\begingroup dependent future\nomeqref {1}
% 		\nompageref{3}
%   \item [{DISTR}]\begingroup distributive\nomeqref {3}\nompageref{3}
%   \item [{GEN}]\begingroup genitive\nomeqref {5}\nompageref{4}
%   \item [{GL}]\begingroup goal\nomeqref {6}\nompageref{4}
%   \item [{IMPF}]\begingroup imperfect\nomeqref {1}\nompageref{3}
%   \item [{INAL}]\begingroup inalienable\nomeqref {2}\nompageref{3}
%   \item [{NOMLZ}]\begingroup nominalizer\nomeqref {2}\nompageref{3}
%   \item [{NONPL}]\begingroup non-plural\nomeqref {6}\nompageref{4}
%   \item [{OBJ}]\begingroup object\nomeqref {1}\nompageref{3}
%   \item [{PFPR}]\begingroup perfective present\nomeqref {2}
% 		\nompageref{3}
%   \item [{PTCPL}]\begingroup participle\nomeqref {5}\nompageref{4}
%   \item [{REL}]\begingroup relative\nomeqref {7}\nompageref{6}
%   \item [{SG}]\begingroup singular\nomeqref {1}\nompageref{3}

% \end{thenomenclature} 
 
{\sloppy
\printbibliography[heading=subbibliography,notkeyword=this]
}

\end{document} 