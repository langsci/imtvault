\documentclass[output=paper]{langsci/langscibook} 
\ChapterDOI{10.5281/zenodo.5675855}

% metadata
\author{Jakob Neels\affiliation{Leipzig University} and Stefan Hartmann\affiliation{{University of Düsseldorf}}}
\title[Grammaticalisation, schematisation and paradigmaticisation]{Grammaticalisation, schematisation and paradigmaticisation: How they intersect in the development of German degree modifiers}

\abstract{Grammaticalisation research has identified the tight integration of structures into paradigms as the final state of many grammaticalisation processes. Construction grammar approaches are particularly suitable for modelling such cases of paradigmaticisation since they invite researchers to study constructions not in isolation but in a wider network context. Drawing on this theoretical perspective, this paper investigates the grammaticalisation of a whole family of constructions. Based on synchronic and diachronic corpus studies, it presents quantitative analyses on the interrelated German constructions [\textit{ein wenig} X] (‘a little’), [\textit{ein bissche}n X] (‘a bit’), [\textit{ein Quäntchen} X] (lit. ‘a quantum’), [\textit{ein Tick} X] (lit. ‘a tick’) and [\textit{eine Idee} X] (lit. ‘an idea’). The paper discusses to what extent their individual and shared developments can be understood in terms of paradigm formation, concluding that early paradigmaticisation involves emergent paradigms of multiple orders in the sense of constructions at different levels of schematicity. Change appears to be guided mainly by associations to micro\hyp{}constructions and lower-level meso\hyp{}constructions \citep[e.g.][]{Traugottconceptsconstructionalmismatch2007}. Only in advanced stages of grammaticalisation, when micro\hyp{}constructions become sufficiently homogeneous, do high-level abstractions – i.e. paradigms in the traditional understanding of the term – act as decisive organisational forces. The key forces at the domain-general cognitive level are argued to be efficiency-driven automation and analogy.}

\begin{document}
\maketitle

% PAPER STARTS HERE

\section{Introduction} \label{introduction}
In constructionist approaches to language, the question of how the internal structure of the constructicon can be modelled has recently come to the centre of attention \citep[see e.g.][]{DiesselGrammarNetworkHow2019, SommererSmirnova2020}. The inventory of inheritance relations proposed by \citet[][]{GoldbergConstructionsConstructionGrammar1995} has been complemented by various suggestions that put a particular emphasis on horizontal links, called “lateral links” by \citet{NordeDerivationcategorychange2018}  or “sister” links by \citet{AudringMotherssistersencoding2019}. Links between constructions at the same level of abstraction are particularly interesting from a usage-based point of view in that they can account for phenomena of language variation and change such as constructional alternations \citet{PerekArgumentstructureusagebased2015} and constructional contamination \citep{PijpopsConstructionalcontaminationHow2016, PijpopsConstructionalcontaminationmorphology2018}.

The conceptualisation of linguistic knowledge as a fine-grained taxonomic network of constructions leads us to the classic concept of paradigmatic, or, as \citet{SaussureCoursLinguistiqueGenerale1916} calls them, associative relations between linguistic units. As \citet[174]{BodenParadigmvssyntagm2005} puts it, “[p]aradigmatic relations oppose a unit to others that could replace it in a given sequence”. In diachronic research, grammaticalisation theory has identified the tight integration of structures into paradigms as a target of grammaticalisation processes. For example, \citet{LehmannThoughtsGrammaticalization2015}  posits paradigmaticity as one of his well-known grammaticalisation parameters and paradigmatic integration, or paradigmaticisation, as one of the key processes involved in grammaticalisation. In this process, “grammaticalized elements join preexistent paradigms and assimilate to their other members” \citep[144]{LehmannThoughtsGrammaticalization2015}.

Overall, it is received wisdom that paradigmatic relations are essential to language structure in general, and that paradigms in the more specific sense of indexical, closed units of organisation are central to grammatical structure, most notably in categories such as tense, aspect or case. In usage-based construction grammar, paradigmatic associations are central as well, since they are tightly integrated into the taxonomic architecture of the constructicon. However, some usage\hyp{}based findings also call into question the importance of paradigms as highly abstract generalisations, suggesting that language users rely on more local generalisations and lower-level schemas (e.g. \cites[]{Boas2003}[]{Dabrowska2008}[Ch. 5]{PerekArgumentstructureusagebased2015}[]{Schmiddynamicslinguisticsystem2020}). Thus, the status of paradigms in usage-based construction grammar is ambivalent. 
The present paper therefore explores the question of how relevant paradigms are to grammatical structure from a cognitive-functional usage-based perspective. Theoretically, a range of answers to this question is conceivable, with opposing views such as the following: paradigms are merely an epiphenomenon of other motivations and mechanisms shaping grammar; or they are essential organisational forces, and thus cognitive entities, themselves. The present study approaches this issue from the perspective of language change. It investigates to what extent paradigmatic forces are reflected in the diachronic development of a family of grammaticalising constructions. In particular, the selected test case is quantifier\slash degree\hyp{}modifier constructions in German. This constructional family forms a layered domain of grammar (cf. \citealt{Hopperprinciplesgrammaticalization1991}), with older members such as [\textit{ein bisschen} X] (‘a bit (of) X’) and [\textit{ein wenig} X] (‘a little X’) and less grammaticalised members such as [\textit{ein Quäntchen} X] (lit. ‘a quantum-\textsc{dim} (of) X’) and [\textit{eine Idee} X] (lit. ‘an idea (of) X’). Through quantitative and qualitative corpus analyses, this study examines to what extent the grammaticalisation trajectories of each individual construction constitute independent changes or an interconnected process of paradigmaticisation influenced by network links as well as by overarching mid-level or high-level schemas. Despite some constructional individuality, the case study finds empirical evidence suggesting a scenario in which older constructions serve as attractor sets promoting the analogical recruitment of new members to an increasingly strengthened schema they collectively instantiate \citep[cf., e.g.,][]{Amit1989, TraugottGrammaticalizationNPNP2008, VerveckkenBinominalquantifiersSpanish2015, Aaron2016, DeSmet2017}. The organisational units contributing to this trend of convergence between multiple constructions appear to be partially schematic constructions at the mid-levels of abstraction. We reason that such a scenario is characteristic of early paradigmaticisation in the grammaticalisation of periphrastic constructions, and that high-level schemas – which come closest to the traditional idea of paradigms – exert an influence mainly in advanced stages of grammaticalisation and paradigmaticisation as typical of inflectional morphology.

\begin{sloppypar}
The remainder of this paper is structured as follows. \sectref{paradigms} discusses the theoretical background, focusing on the status of paradigms in (diachronic) usage-based construction grammar. \sectref{sec3} presents the corpus-based case study and interprets its results against this theoretical background. \sectref{sec4} provides a more general discussion of the process of paradigmaticisation from the perspective of the constructionist concept of schematisation. Finally, \sectref{sec5} offers a brief conclusion.
\end{sloppypar}

\section{Background: Paradigms and usage-based construction grammar} \label{paradigms}

The structuralist notion of \textit{paradigmatic} comes in broader and in more narrow senses, and not all of them have a close equivalent in the theoretical apparatus of usage-based construction grammar. In a broad sense, \textit{paradigmatic} denotes a relation of choice between linguistic items that are associated with each other based on functional (e.g. synonymy) or formal (e.g. homonymy) features. As mentioned in \sectref{introduction}, paradigmatic relations in this sense translate into links in constructional networks, such as subpart links and metaphorical extension links \citep[e.g.][]{GoldbergConstructionsConstructionGrammar1995} that connect constructions at the same level of schematicity (e.g. \textit{arm} and \textit{leg}; ditransitive construction and \textit{for}-benefactive construction). In a narrow sense, \emph{paradigmatic} and \emph{paradigm} can be understood as sets with a fixed number of mutually exclusive forms whose meanings are indexical in that they are largely determined by their relations within those closed-class sets. Prototypical instances of this type of paradigm are found in grammatical domains such as tense, person, voice and case. Such grammatical paradigms clearly differ from loose sets of open-class items such as lexemes being paradigmatically related via various sense relations. Construction grammar understates this difference in tightness and closedness since it represents lexical items and grammatical items in a uniform format within a lexicon–grammar continuum \citep[cf.][]{Diewald2020b}. In response to this problem, several construction grammarians have recently proposed theoretical add-ons: consider Booij's (e.g. \citealt{Booijrolesecondorderschemas2015}) concept of second-order schemas, Audring’s (e.g. \citeyear{AudringMotherssistersencoding2019}) sister schemas, Diewald’s (\citeyear{Diewald2020b}) hyper-constructions and Leino`s \parencitetv{chapters/03_leino} metaconstructions. Advances in constructionist theorising are needed to model paradigms more comprehensively. 

\begin{figure}
  \includegraphics[width=\textwidth]{figures/HN-fig1.png}
  \caption{Constructional levels, exemplified with a network of English}\label{fig:networkmodel}
\end{figure}

A classification that is very much in line with such network models is found in Traugott’s (e.g. \citeyear{Traugottconceptsconstructionalmismatch2007}) notion of constructional levels in terms of micro-, meso- and macro-constructions. Using the example of English future constructions \citep[cf.][36]{TrousdaleDegrammaticalizationConstructionalizationTwo2013}, \figref{fig:networkmodel} visualises the basic idea behind these constructional levels. Micro\hyp{}constructions, i.e. individual (largely) substantive construction types, can be subsumed under a single highly abstract schema, the macro-construction, if they share the same basic function. In \figref{fig:networkmodel}, the five formally diverse micro\hyp{}constructions at the bottom can be taken to be weakly linked at the macro-level based on their shared basic function of marking futurity. In-between, meso\hyp{}constructions, i.e. mid-level schemas, unite subsets of similarly-behaving micro\hyp{}constructions with a shared structure. Of the five micro\hyp{}constructions used for illustration, [\textit{will}~\textsc{v}] and [\textit{shall} \textsc{v}] are structurally alike (e.g. \textsc{v} as bare infinitive) and so are [\textsc{be} \textit{going to} \textsc{v}] and non-standard [\textsc{be} \textit{fixing to} \textsc{v}] (e.g. progressive). Unlike [\textsc{be} \textit{about to} \textsc{v}], each of these two pairs of micro\hyp{}constructions can therefore be subsumed under one meso-construction, respectively. The higher the number of micro\hyp{}constructions that collectively constitute one meso-construction, the more entrenched and productive this meso-construction tends to become. Meso\hyp{}constructions are hypothesised to influence neighbouring constructions in the network through the domain-general cognitive process of analogy. Diachronically, constructional clusters headed by well-entrenched meso\hyp{}constructions may attract other/novel micro\hyp{}constructions, leading to the growth of a family of constructions. As detailed in the corpus study further below, Traugott’s constructional levels thus qualify as a valuable descriptive tool for studying the reorganisation of the constructicon. 

However, it has been a point of debate whether all of these constructional levels are psychologically plausible. Usage-based linguists have questioned whether patterns at the highest levels of abstraction/schematicity (e.g. fully schematic argument structure constructions) are represented in the minds of most speakers. Possibly, high-level generalisations are no more than linguists’ constructs emerging from the analysis of aggregated usage data. As hinted at by some psycholinguistic and corpus-linguistic evidence (e.g. \cites[]{Boas2003}[]{Dabrowska2008}[]{Dabrowska2015}[Ch. 5]{PerekArgumentstructureusagebased2015}[Ch. 5]{SchmidMantlik2015}), the actual generalisations of individual speakers may stop at the mid-level of partially schematic constructions. In other words, mid- and low-level constructions appear to be the key material of linguistic knowledge thanks to their greater cognitive accessibility (i.e. entrenchment). From this perspective, high-level schemas, including paradigms, seem to be of relatively little psycholinguistic importance. 

Yet, other findings from psycholinguistically minded grammaticalisation research indirectly stress the importance of paradigms. As argued most convincingly in \citet{LehmannGrammaticalizationautomation2017}, grammaticalisation is essentially a linguistic instance of cognitive automation, i.e. the domain-general process that turns more controlled, intentional activity into efficient, unconscious, rigid behaviour (cf. \citealt{Schneider1977, Loganinstancetheoryautomatization1988, Givon1989}: Ch. 7; \citealt{MoorsDeHouwer2006}). Automation streamlines the execution of recurrent tasks, and grammaticalisation does exactly that in language: it creates efficient solutions for frequently recurring communicative tasks, such as signalling futurity, plurality, possession or negation. Paradigms are an expected outcome of automation in language since they are rigid closed-class structures that can be executed with minimal attention. They allow speakers to maximise the processing capacities available for more complex, discursively primary communicative tasks \citep[cf.][]{HarderBoye2011}. Humans’ efficiency-driven capacity for automation thus seems to be a key factor underlying the trend that grammaticalisation processes are directed towards paradigms as a target state. From this perspective, paradigm formation is essential to language processing. 

In short, \textit{paradigm} and \textit{paradigmaticisation} can be translated into usage-based constructionist concepts as follows. The synchronic notion of paradigms roughly corresponds to higher-level schemas and the “horizontal” associations between the constructions sanctioned by these schemas. It must be noted, however, that there is currently no single conventional solution in construction grammar regarding the exact types of schemas and links that fully capture grammatical paradigms in the narrow sense (see \citealt{SmirnovaSommererIntroduction2020} for a problem-oriented discussion of different conceptions of nodes and vertical as well as horizontal links in constructional networks). The diachronic process of paradigmatisation is then a process of schema formation, i.e. schematisation, and of the gradual convergence of subordinate constructions into an increasingly tight set \citep[cf.][]{DiewaldParadigmaticintegration2012}. Two basic cognitive mechanisms of paradigmaticisation appear to be analogy and automation; a basic motivation is processing efficiency. 

Based on the theoretical considerations presented above, we derive the following working hypotheses for the present case study on paradigmaticisation: 

\begin{enumerate}[label=(\roman*)]\sloppy
    \item Paradigmaticisation involves emergent paradigms of multiple orders in the sense of constructions at multiple levels of schematicity. 
    \item Micro\hyp{}constructions united by a meso-construction are likely to converge formally and/or functionally over time, thus producing more homogeneous paradigms. 
    \item New micro\hyp{}constructions will be attracted to an emergent paradigm via analogy, especially when extant micro\hyp{}constructions and their overarching schema(s) are strongly entrenched, with determinants of entrenchment being, among others, usage frequency and coherence of schema members. 
\end{enumerate}

In the case study presented below, our goal is to link these theoretical assumptions with observations based on the diachronic development and synchronic behaviour of one particular family of constructions. On a terminological note, we employ the term \textit{constructional family} as an exploratory notion: unlike \textit{paradigm} and constructional levels like \textit{meso-construction}, which are meant to capture mental associations and representations, \textit{constructional family} more loosely refers to a group of linguistic expressions with functional and/or formal commonalities at a pre-theoretical level.

\section{Case study: The development of German quantifier\slash degree\hyp{}modifier constructions} \label{sec3}

Parts of the present investigation draw on findings from a previous case study {\citep{NeelsReductionexpansionbit2018}} focusing on the diachrony of the frequent German degree modifiers [\textit{ein wenig} X] (‘ a little X’) and [\textit{ein bisschen} X] (‘a bit (of) X’). In that study, we extended our account of the diachronic development to other structurally similar quantifier\slash degree\hyp{}modifier constructions with low usage frequencies, but we did not provide systematic corpus data on these less established modifiers. The present follow-up study adds these data, providing quantitative corpus analyses on three more members of the constructional family: [\textit{ein Quäntchen} X] (lit. ‘a quantum (of) X’), [\textit{ein Tick} X] (lit. ‘a tick (of) X’) and [\textit{eine Idee} X] (lit. ‘an idea (of) X’). \sectref{sec3-1} first introduces the data and methods used in this investigation. \sectref{sec3-2} sketches out three possible grammaticalisation scenarios leading to degree\hyp{}modifier functions, before moving on to the results. \sectref{sec3-3} reviews the main results on \textit{ein wenig} and \textit{ein bisschen}. \sectref{sec3-4} details the results on the less grammaticalised expressions \textit{ein Quäntchen}, \textit{ein Tick} and \textit{eine Idee}, and provides a synthesis of our earlier findings and the present analyses. 

\subsection{Data and methods} \label{sec3-1}

Five members of the expanding family of German quantifier\slash degree\hyp{}modifier constructions are under scrutiny in the present investigation.\footnote{Cross-linguistically, structurally similar constructions include English \textit{a bit} (\textit{of}), \textit{a shred of} and \textit{a bunch of} \citep[e.g.][]{BremsGrammaticalizationSmallSize2007, TraugottGrammaticalizationNPNP2008, Shao2019}, Spanish \textit{un montón de} 
`a heap of' and \textit{un hatajo de} `a herd of' \citep[e.g.][]{VerveckkenBinominalquantifiersSpanish2015}, and Dutch \textit{massa}(\textit{'s}) `mass(es)' \citep[e.g.][]{DeClercknounintensifiermassa2013} among others.}  These five micro\hyp{}constructions are exemplified in (\ref{ex:microcxn1}) to (\ref{ex:microcxn5}) below. 

\begin{exe}
	\ex \label{ex:microcxn1}   
	\begin{xlist}
		\ex \label{ex:microcxn1a} Warte doch, du mußt \textit{ein bißchen} Eigenlob hören. \\
		`Wait, you have to hear \textit{a bit} of self-praise.' \\
		(1896, DeReKo-HIST) 
		
		\ex \label{ex:microcxn1b} 	Ich tränke gern ein Glas, die Freiheit hoch zu ehren, Wenn eure 	Weine nur \textit{ein Bißchen} besser wären. \\ 	
		`I’d love to drink a glass, in freedom’s honour, if only the wine 	were \textit{a bit} better.' \\ (1790, DTA)
		\ex \label{ex:microcxn1c} Rück doch mal \textit{'n bischen} den Tisch!  \\
	`Move the table \textit{a bit}!' \\ (1890, DTA)
		
		\end{xlist}

    \ex \label{ex:microcxn2}aber seine Aussprache war \textit{ein wenig} bäuerisch, und sein Auge 	blickte nicht fein \\
	`but his pronunciation was \textit{a little} rural, and his eye didn't look fine' \\ (1805, DeReKo-HIST: Jean Paul)

    \ex \label{ex:microcxn3} Doch er hat auch \textit{ein Quäntchen} Humor im Hinterkopf \\
	`But he also has \textit{a bit} (lit.: \textit{quantum}) of humour in the back of 	his mind' \\ (2009, DeReKo-Tagged-C) 

    \ex \label{ex:microcxn4} Da waren die Gäste aus Wien \textit{einen Tick} effektiver. \\
    `The guests from Vienna were \textit{a bit} (lit.: \textit{tick}) more effective.' \\ (2008, DeReKo-Tagged-C) 

    \ex \label{ex:microcxn5} ihre Darstellung ist \textit{eine Idee} zu ernst. \\
	`her portrayal is \textit{a bit} (lit.: \textit{idea}) too serious' \\ (1999, DeReKo-Tagged-C)
\end{exe}

In present-day German, [\textit{ein bisschen} X] (‘a bit (of) X’) can be considered the “ideal” representative of the constructional family, being highly frequent and productive and exhibiting all prototypical features. Typical family members are made up of (i) the indefinite article \textit{ein}, (ii) a noun denoting a small unit, and (iii) an open slot that can be filled by items from various word classes. Different word classes hosted in the constructions are associated with different constructional functions. Partitive and quantifier uses are associated with noun modification, typically with concrete nouns and mass nouns (cf. Examples 
\ref{ex:microcxn1a}, \ref{ex:microcxn3}), 
respectively. Constructs modifying adjectives 
(\ref{ex:microcxn1b}, \ref{ex:microcxn2}, 
\ref{ex:microcxn4}, \ref{ex:microcxn5}), verbs 
(\ref{ex:microcxn1c}) or other parts of speech 
generally fulfil degree\hyp{}modifier functions. As 
detailed further below, individual members of this 
family of quantifier\slash degree\hyp{}modifier constructions 
differ in their constraints and preferences regarding the syntactic categories sanctioned in their 
productive slot.

Usage data reflecting the diachronic trajectories and potential mutual influences of the five micro\hyp{}constructions were extracted from several historical and contemporary German corpora. In particular, historical data on all five constructions was collected from the \textit{Deutsches Textarchiv} (DTA; German Text Archive; \citealt{Geykenlivingtextarchive2015}); additional diachronic data on [\textit{ein bisschen} X] and [\textit{ein wenig} X] (‘a little X’) were extracted from the historical component of the \textit{Deutsches Referenzkorpus} (DeReKo-HIST; German Reference Corpus; \citealt{KupietzKeibel2009}); and we searched a tagged synchronic component of that corpus, DeReKo-Tagged-C, for present-day uses of the low-frequency constructions [\textit{ein Quäntchen} X] (lit. ‘a quantum (of) X’), [\textit{ein Tick} X] (lit. ‘a tick (of) X’) and [\textit{eine Idee} X] (lit. ‘an idea (of) X’). For extracting these three younger constructions, the following queries were used:

\begin{verbatim}
REG(^Idee$|^Tick$|^Quäntchen$) /+w1 MORPH(A)
REG(^Tick$|^Quäntchen$) /+w1 MORPH(N)
REG(^Idee$|^Tick$|^Quäntchen$) /+w1 MORPH(ADV)
\end{verbatim}

False hits as well as duplicates were removed manually, and the remaining data were coded for the part of speech of the modified element. In total, this procedure yielded 44 instances of \textit{eine Idee}, 1,777 of \textit{ein Tick}, and 1,694 of \textit{ein Quäntchen} from the contemporary DeReKo-Tagged-C data. Historical data of these three constructions are fairly rare, as described in \sectref{sec3-4} further below. For (\textit{ein}) \textit{bisschen} and \textit{ein wenig}, the dataset extracted from DTA amd DeReKo-HIST comprised 3,226 and 15,783 historical tokens, respectively,(see \citet{NeelsReductionexpansionbit2018} for details).

\subsection{Diachronic paths leading to degree\hyp{}modifier functions} \label{sec3-2}

Despite the structural similarity of the German degree modifiers, their individual origins and developmental paths seem to be fairly diverse. We consider three possible grammaticalisation scenarios giving rise to such micro\hyp{}constructions. The plausibility of each scenario for the German constructions at hand will be discussed in the subsequent sections in the light of our corpus data.

\begin{sloppypar}
In the first scenario, a lexical source construction grammaticalises “under its own steam” along a cross-linguistically attested grammaticalisation path. This scenario theoretically entails a highly gradual development with discernable chronological stages and with little to no analogical influence by extant grammatical constructions. It roughly corresponds to what Lehmann (\citeyear{Lehmann2004a}, \citeyear{LehmannThoughtsGrammaticalization2015})  calls “pure” grammaticalisation or “innovation”, contrasting with his notions of “analogically-oriented” grammaticalisation and “renovation”. According to \citet[e.g.][161]{Lehmann2004a}, only the former type, i.e. pure grammaticalisation, may give rise to a genuinely new grammatical category (but cf. \citealt[Section 4.1]{DeSmetDoesinnovationneed2014}). In such cases of category emergence, meso\hyp{}constructions can be expected to play no significant role. In the case and time frame investiaged here, they potentially do since categories like quantifier and degree modifier are in place and instantiated by multiple expressions of German. Under these conditions, applying the theoretical concept of pure grammaticalisation is not expedient. We therefore define the first scenario by a feature that is empirically more accessible, namely the incremental progression of grammaticalisation along a cline. In line with previous research on the diachrony of similar degree modifiers in English \citep[esp.][]{TraugottGrammaticalizationConstructionsIncremental2008, TraugottGrammaticalizationNPNP2008}, the grammaticalisation path applying to the German constructions is likely to be as follows: pre-partitive > partitive > quantifier > degree modifier (e.g. \textit{the bite of an apple} > \textit{bits of bread} > \textit{a bit of work} > \textit{a bit tired}).
\end{sloppypar} 

In the second scenario, the micro-construction starts as a fixed idiomatic expression but expands and develops a productive slot. For example, it is conceivable that the degree modifier [\textit{ein Tick} X] started out as a fairly fixed expression combining with a particular temporal adjective, as in \textit{einen Tick schneller} ‘a tick/bit faster’ or \textit{einen Tick zu spät} ‘a tick/bit too late’, which are conventional collocations and in which the noun \textit{Tick} ‘tick (of a clock)’ retains a concrete time-related meaning. Possibly, the expression has been gradually increasing its productivity through item-based expansion from one temporal adjective to others, to non-temporal adjectives and eventually to other word classes. This scenario differs from the first one in that it does not presuppose earlier stages with partitive and quantifier uses. 

In the third hypothesised scenario, analogically driven grammaticalisation, a micro-construction directly joins an existing degree\hyp{}modifier meso-construction, with firmly entrenched schema members serving as role models. As introduced in \sectref{paradigms}, it is this scenario that demonstrates the organisational force of paradigms-as-schemas most clearly. Whereas in the first scenario the process of paradigmatic integration is likely to become evident only relatively late in the grammaticalisation process (cf. stage IV in the model by \citealt{DiewaldParadigmaticintegration2012}), paradigmatic integration is at the heart of the beginning of grammaticalisation scenarios that are chiefly determined by analogy. 

Although the three scenarios are theoretically distinct, they are not entirely mutually exclusive. Mixed trajectories and influences are possible, and distinguishing the three scenarios on the basis of usage data is not straightforward. Still, there are some fairly reliable variables in usage that help disentangle the hypothesised scenarios. The following analyses focus on four parameters: 
 
 
\begin{enumerate}[label=(\roman*)]
    \item The chronology of functional expansion leading up to degree\hyp{}modifier uses; 
    \item patterns of behavioural convergence between constructions; 
    \item patterns of collocational range; and 
    \item diachronic token-frequency levels. 
\end{enumerate}

\begin{sloppypar}
More than in the analogically driven grammaticalisation scenario or the scenario from fixed to productive expression, the isolated grammaticalisation of a micro-construction along a grammaticalisation path correlates with, and even depends on, increasingly high frequencies of use. The rationale behind this correlation is that high frequency – token frequency especially, but also co-occurrence and type frequency – appears to be a crucial causal force in independent/“pure” grammaticalisation, fuelling several underlying cognitive processes such as chunking, habituation, neuromotor automation and schema entrenchment (\citealt{BybeeMechanismsChangeGrammaticization2003, KrugFrequencyDeterminantGrammatical2003, DisselHilpert2016, NeelsRefiningfrequencyeffectexplanations2020}). Thus, the scenario of independent grammaticalisation along a cline can be expected to differ from the other two scenarios with respect to diachronic token-frequency levels and, above all, with respect to the chronology of functional expansion. What makes the scenario of grammaticalisation by analogy empirically distinguishable from the scenario of a fixed expression turning productive is patterns of behavioural convergence between constructions and patterns of collocational range. 
\end{sloppypar}

\subsection{\textit{Ein wenig} and \textit{ein bisschen}} \label{sec3-3}

The quantifier\slash degree\hyp{}modifier constructions \textit{ein wenig} and \textit{ein bisschen} can arguably be used largely interchangeably in present-day German, although the former variant – which is also the older one – may be seen as slightly more formal.\textit{ Ein wenig} has largely ousted the yet older variant \textit{ein lützel}, which used to be fairly frequent in the Middle High German period\footnote{This study adopts the traditional periodisation of German language history: c. 750–1050 Old High German; 1050–1350 Middle High German; 1350–1650 Early New High German; 1650–today New High German.}  but had fallen out of use in most German dialects by the 17\textsuperscript{th} century {\citep[see][143]{NeelsReductionexpansionbit2018}}.

\textit{Ein bisschen} undergoes a development that bears striking similarities to the evolution of English \textit{a bit} as outlined by Traugott \citep[][]{Traugottconceptsconstructionalmismatch2007, TraugottGrammaticalizationConstructionsIncremental2008, TraugottGrammaticalizationNPNP2008}. Like its English equivalent, \textit{bisschen} derives from the (diminutivised version of the) noun \textit{Bissen} ‘bite’, and just like English \textit{a bit}, it tends to combine with concrete nouns as in (\ref{ex:ex6}) at first before it comes to modify abstract nouns to an ever larger extent (\ref{ex:ex7}), as {\citet[][]{NeelsReductionexpansionbit2018}} show, taking data from the mid-17th to the mid-20th century into account. 

\begin{exe}
	\ex \label{ex:ex6}   Der diebische Schösser wird mir nach meinem \textit{bisschen} Brot 	trachten  \\
	‘The thievisch tax collector will strive for my \textit{bit} of bread’  \\	(1713, DeReKo-HIST: HK5)

	\ex \label{ex:ex7}   So gilt \textit{ein bißchen} Witz mehr als ein gutes Herz! \\
	‘So \textit{a bit} of wit is considered to be more important than a good 	heart!’ \\ (1746, DeReKo-HIST: HK3) 
\end{exe}

While the construction tends to combine with nouns in the early stages of its development, the proportion of verbs modified by \textit{ein bisschen} increases steadily {\citep[150]{NeelsReductionexpansionbit2018}}, as in (\ref{ex:ex8}). A slight upward trend can also be seen for adjectives, as in (\ref{ex:ex9}), although the pattern is less clear here, especially due to data sparsity in the early time slices analysed.

\begin{exe}
	\ex \label{ex:ex8}   Wollten Sie nicht \textit{ein bisschen} ruhen? \\
	‘Did you not want to rest \textit{a bit}?’ \\ (1776, DeReKo-HIST: HK3)

	\ex \label{ex:ex9}   Wär ich doch so hold, wie jener Freund der Liebeskönigin!	Oder nur \textit{ein bißchen} schöner, Als ich Armer izo bin! \\
	‘If I was as fair as that friend of the love queen! Or just \textit{a bit }	more beautiful, than I am, poor me, right now!’ \\ (1778, DTA)
\end{exe}

One striking result of our previous corpus study is that the distributional characteristics of \textit{ein bisschen} seem to align over time with those of the older \textit{ein wenig} construction, especially with regard to relative frequencies of nouns, verbs and adjectives modified by the two constructions. Furthermore, we observed a diachronic shift in the variability of determiners in [\textsc{det} \textit{bisschen} \textsc{x}] towards the increasingly fixed string \textit{ein bisschen}. These changes rendered the \textit{ein bisschen} construction structurally very similar to \textit{ein wenig}, suggesting some analogical influence by the latter construction. 

Other aspects of the diachrony of the \textit{ein bisschen} construction, in contrast, are more indicative of a fairly independent grammaticalisation process. For one thing, the attested shift from concrete to abstract nouns and from noun modification to adjective and verb modification closely matches the steps on the cline from partitive to quantifier to degree modifier. That is, the historical usage data examined in {\citet[][]{NeelsReductionexpansionbit2018}} support the conclusion that \textit{ein bisschen} passed through the stages of the grammaticalisation path in successive order. Another relevant piece of evidence is the long-term token-frequency profile of \textit{ein bisschen}. As plotted in \figref{fig:dwdsfreqs}, the grammaticalisation of the \textit{ein bisschen} construction is accompanied by a pronounced increase in absolute token frequency.\footnote{The DWDS corpora are used here to visualise the increase in frequency as it is much larger than the database we used for {\citet[][]{NeelsReductionexpansionbit2018}}; however, the findings based on our own data, covering the time span from c. 1650 to 1900, are very much in line with what \figref{fig:dwdsfreqs} shows (see \citealt[148]{NeelsReductionexpansionbit2018}).}  The written records are, however, likely to depict this increase with a considerable temporal delay owing to the fact that, until the mid-19\textsuperscript{th} century at least, \textit{ein bisschen} was evaluated as colloquial \citep[see][6]{Tiefenbachchenundlein1987}. 

\begin{sloppypar}
Thus, for \textit{ein bisschen}, the usage-based parameters introduced in \sectref{sec3-2} tentatively suggest a fairly typical case of grammaticalisation whereby one particular micro-construction incrementally emancipated from its lexical source through frequency effects – however, possibly with additional support from extant constructions acting as analogical models. 
Interestingly, the picture emerging for the younger German quantifier\slash degree\hyp{}modifier constructions examined next is notably different. 
\end{sloppypar}

\subsection{\textit{Ein Quäntchen, ein Tick, eine Idee}} \label{sec3-4}

The modifiers [\textit{ein Quäntchen} X] (lit. ‘a quantum (of) X’), [\textit{ein Tick} X] (lit. ‘a tick (of) X’) and [\textit{eine Idee} X] (lit. ‘an idea (of) X’) are clearly more constrained and much less frequent in contemporary usage than the \textit{ein bisschen} and \textit{ein wenig} constructions. \tabref{tab:1:frequencies} (page \pageref{tab:1:frequencies}) provides a first overview of the usage patterns of \textit{ein Quäntchen}, \textit{ein Tick} and \textit{eine Idee}. In the 1.5-billion-word corpus DeReKo-Tagged-C, none of the three constructions is attested much more frequently than about 1,500 times, which yields normalised token frequencies of merely about one occurrence per million words (cf. \textit{ein bisschen} in \figref{fig:dwdsfreqs}). The modifier [\textit{eine Idee} X] is particularly infrequent, with 0.03 tokens per million words. Degree\hyp{}modifier uses of the three expressions do exist, but their developments constitute cases of low-frequency grammaticalisation, a phenomenon that is somewhat problematic for frequency-effect explanations in the Bybeean tradition (\citealt{HoffmannArelowfrequencycomplex2004, BremsGrammaticalizationSmallSize2007, NeelsRefiningfrequencyeffectexplanations2020}).

\begin{figure}[p]
  \includegraphics[width=\textwidth]{figures/HN-fig2.png}
  \caption{The normalised absolute token frequency of (\textit{ein}) \textit{bisschen} for the period from 1600 to the present; reproduced from the DWDS; based on aggregated data from the reference corpora and the newspaper corpora used by the DWDS (retrieved on 3 September 2020)}\label{fig:dwdsfreqs}
\end{figure}

\begin{table}[p]
\caption{Overview of the corpus data for the three younger constructions}
\label{tab:1:frequencies}
\begin{tabular}{l *{3}{r@{ }r} }
\lsptoprule
    & \multicolumn{2}{c}{\textit{eine Idee}} & \multicolumn{2}{c}{\textit{ein Tick}} & \multicolumn{2}{c}{\textit{ein Quäntchen}}\\\midrule
Adjective (positive)             & -  &           & 11    & (0.62\%)  & 4  & (0.24\%)\\
Adjective (comparative)          & 20 & (45.5\%)  & 1,280 & (72.03\%)  & 71 & (4.19\%)\\
Adjective with \textit{zu}       & 17 & (38.6\%)  & 312   & (17.56\%)  & -  & \\
Adverb                           & 5  & (11.4\%)  & 54    & (3.04\%)  & 26    & (1.53\%)\\
Noun                             & 2  & (4.5\%)   & 59    & (3.32\%)  & 1,588 & (93.74\%)\\
Preposition phrase               & -  &           & 17    & (0.96\%)  & -  & \\
Verb                             & -  &           & 44    & (2.48\%)  & 5     & (2.95\%)\\
\midrule
Total                            & 44 &            & 1,777 &          & 1,694  & \\
\lspbottomrule
\end{tabular}
\end{table}


Accordingly, only few historical tokens of the three younger quantifier\slash degree\hyp{}modifier constructions can be found in the smaller DTA database. The corpus searches returned no more than 7 tokens of [\textit{eine Idee} X] and no quantifier or degree\hyp{}modifier uses of \textit{ein Tick} at all. As for \textit{ein Quäntchen}, the historical picture is slightly different. The expression originated as a commercial weight, \textit{Quentchen}, i.e. the fourth or fifth part of one lot.\footnote{Compare Latin \textit{quintus} ‘the fifth’.} About 90\% of the 158 DTA tokens attested between 1756 and 1910 are clear instances of \textit{Quäntchen/Quentchen} being used as a technical measure, as exemplified in (\ref{ex:ex10}); a few tokens are ambiguous, leaning towards more colloquial uses, as in (\ref{ex:ex11}). 

\begin{exe}
	\ex \label{ex:ex10}   Die Kuhmilch enthält endlich an Ram zwanzig \textit{Quentchen}, an 	fester Butter sechs \textit{Quentchen}, an dichtem Käse drei Unzen, an 	eingedickter Wadikke zehn \textit{Quentchens} \\
	‘Cow’s milk ultimately contains 20 \textit{Quäntchen} of cream, 6 	\textit{Quäntchen} of solid butter, 3 ounces of thick cheese, 10 	\textit{Quäntchen} of concentrated whey’ \\ (1756, DTA: Haller) 

	\ex \label{ex:ex11}  Ein \textit{Quentchen} Mutterwitz ist besser, als ein Zentner 	Schulwitz. \\
	‘One \textit{Quäntchen}/A \textit{bit} of mother wit is better than one centner 	of school wit.’ \\
	(1762, DTA: Rabener)
\end{exe}

Through folk-etymological reanalysis, as the commercial weight of \textit{Quentchen} became uncommon in the 20th century, the expression was interpreted as a diminutive form related to ‘quantum’.\footnote{Note that the spelling was changed from \textit{Quentchen} to \textit{Quäntchen} in the German spelling reform in 1996. This is why most of the cited examples are spelled with an 〈e〉 instead of an 〈ä〉.} It survives as a lower-frequency quantifier and, to a certain extent, as degree modifier, as quantified below. 

In addition to the total frequencies of \textit{ein Quäntchen}, \textit{ein Tick} and \textit{eine Idee} in DeReKo-Tagged-C, Table \ref{tab:1:frequencies} shows the distribution of parts of speech across the three constructions. This provides a first glimpse into their functional commonalities and relative differences in contemporary usage. Part of the information of Table \ref{tab:1:frequencies} is condensed into Figure \ref{fig:mosaic}, which visualises the distribution of nouns, verbs and adjectives in the three constructions. Together, Table \ref{tab:1:frequencies} and Figure \ref{fig:mosaic} reveal that all three constructions exhibit clear preferences with regard to the parts of speech with which they combine. \textit{Eine Idee} and \textit{ein Tick} combine mainly with adjectives, more precisely with adjectives in the comparative or with the degree particle \textit{zu} ‘too’. \textit{Ein Quäntchen}, on the other hand, shows a strong preference for nouns, while also combining with graded adjectives and other parts of speech from time to time. Thus, \textit{ein Quäntchen} serves primarily as a quantifier whereas \textit{ein Tick} and \textit{eine Idee} are used mostly for degree-modifying purposes.

\begin{figure}[p]
  \includegraphics[width=\textwidth]{figures/HN-fig3.png}
  \caption{Distribution of nouns, adjectives, and verbs in the three younger constructions}\label{fig:mosaic}
\end{figure}

\begin{sloppypar}
At a more fine-grained level of analysis, \tabref{tab:2:top20} (page \pageref{tab:2:top20}) lists the most frequent lexical items occurring in each of the constructions. Two things become immediately clear in these lists of collocates. Firstly, there is a considerable amount of overlap between the lexical items that preferentially occur in the three constructions. For example, \textit{mehr} is a top-three collexeme in each construction, including the \textit{ein Quäntchen} construction despite the fact that this construction is dispreferred as a degree modifier. Secondly, their use seems to be largely constrained to a relatively small set of lexemes, which sets them apart from the \textit{ein bisschen} and \textit{ein wenig} constructions discussed above.
\end{sloppypar}\largerpage

\begin{table}
\caption{Top 20 modified items for each construction (without hapax legomena).\label{tab:2:top20}}
\small
\begin{tabularx}{\textwidth}{lrQrQr}
\lsptoprule
\multicolumn{2}{c}{\textit{eine Idee}} & \multicolumn{2}{c}{\textit{ein Tick}} & \multicolumn{2}{c}{\textit{ein Quäntchen}} \\
\cmidrule(lr){1-2}\cmidrule(lr){3-4}\cmidrule(lr){5-6}
{Lemma} & {Freq} & {Lemma} & {Freq} & {Lemma} & {Freq}\\
\midrule
gut `good' & 393 & gut `good' & 8 & Glück `luck' & 1,303\\
schnell `fast' & 174 & schnell `fast' & 4 & viel `much' & 51\\
viel `much' & 111 & viel `much' & 4 & gut `good' & 21\\
stark `strong' & 102 & schnell `fast' & 3 & Humor `humour' & 13\\
spät `late' & 91 & voraus `ahead' & 3 & Trost `solace' & 12\\
aggressiv `aggressive' & 60 & hoch `high' & 2 & Wahrheit `truth' & 7\\
clever & 36 & lang `long' & 2 & Energie `energy' & 6\\
voraus `ahead' & 29 & Algenfood `seaweed food' & 1 & Fortune `luck' & 5\\
viel `much' & 25 & bunter `more colorful' & 1 & Ironie `irony' & 5\\
warm `warm' & 18 & darunter `below' & 1 & Kraft `strength' & 5\\
hoch `high' & 17 & eitel `vein' & 1 & Mut `courage' & 5\\
offensiv `offensive' & 15 & ernst `serious' & 1 & Konzentration `concentration' & 4\\
eher `earlier' & 14 & flach `flat' & 1 & Stolz `pride' & 4\\
lang `long' & 13 & gleich `equal' & 1 & Begeisterung `enthusiasm' & 3\\
schlecht `bad' & 13 & herzig `cute' & 1 & davon `thereof' & 3\\
stark `strong' & 12 & Jazz & 1 & Disziplin `discipline' & 3\\
vor `before' & 12 & klangverliebt `sound-loving' & 1 & Entschlossenheit `determination' & 3\\
langsam `slow' & 11 & kurz `short' & 1 & Leistung `performance' & 3\\
weit `far/broad' & 11 & leise `quiet/silent' & 1 & Präzision `precision' & 3\\
früh `early' & 10 & schwach `weak' & 1 & Qualität `quality' & 3\\
\lspbottomrule
\end{tabularx}
\end{table}

Some collocations are particularly idiomatic and dominant in current usage. Most notably, combinations with the noun \textit{Glück} ‘luck’ make up about 80\% of all \textit{ein Quäntchen} tokens found in DeReKo-Tagged-C, and \textit{Glück} is more than 25 times more frequent than the second-ranked collexeme. Also note that, in line with the prototypical phrase \textit{ein Quäntchen Glück}, the other top collexemes of \textit{ein Quäntchen} listed in \tabref{tab:2:top20} are nouns that have positive connotations as well. This specific collocational pattern is not to be expected solely on the basis of the construction’s past as a commercial weight. Based on this origin, one might expect an early grammaticalisation trajectory that involves high relative frequencies of (mass) nouns in general, but not necessarily the dominance of one particular noun. However, it seems that, to a large extent, the construction owes its present-day existence as a low- to mid-productive quantifier to the strongly entrenched collocation \textit{ein Quäntchen Glück}. Conceivably, the low-frequency [\textit{ein Quäntchen} X] construction also receives some analogical support through the firmly established \textit{ein bisschen} construction, with which \textit{ein Quäntchen} shares associative links thanks to similarities such as the diminutive suffix \textit{-chen}. Overall, however, the most informative parameter for the diachrony of \textit{ein Quäntchen} appears to be the parameter of collocational range. It strongly points to the second grammaticalisation scenario introduced in \sectref{sec3-2}, i.e. a process whereby a fixed idiomatic expression develops an open slot with a certain degree of productivity. 

Contemporary corpus data demonstrates that the \textit{ein Quäntchen} construction is barely entering more advanced stages of grammaticalisation in which it expands from quantifier (95\%) to degree\hyp{}modifier uses (5\%), whereas \textit{ein Tick} and \textit{eine Idee} are first and foremost degree modifiers and only rarely used as quantifiers. Their usage profiles thus yield the somewhat strange picture that \textit{ein Tick} and \textit{eine Idee}, as it were, skipped the presumably less grammaticalised quantifier stage on the grammaticalisation path presented in \sectref{sec3-2} (pre-partitive > partitive > quantifier > degree modifier). Interestingly, \citet[][]{DeClerckSizenounsmatter2016}, studying similar English constructions with size nouns (e.g. \textit{mass}(\textit{es}), \textit{heap}(\textit{s}), \textit{bunch}), also observe trajectories with unattested (supposedly) intermediate grammaticalisation stages. The parameter we labelled “chronology of functional expansion” suggests that the degree modifiers \textit{ein Tick} and \textit{eine Idee} have not evolved through independent grammaticalisation processes. Given the remarkably low absolute token frequencies of the two constructions, also the frequency parameter speaks against the scenario of independent grammaticalisation. Moreover, unlike in the case of \textit{ein Quäntchen}, the analysis on collocational range has not revealed any highly dominant collexemes for [\textit{ein Tick} X] or [\textit{eine Idee}~X]. This makes the grammaticalisation scenario from fixed to productive expression seem less likely for these two constructions. Both constructions, we argue below, owe their existence or emergence largely to pre-existing templates in the modern German network of quantifier\slash degree\hyp{}modifier constructions. 

Diachronic corpus data suggest that \textit{ein Tick} and \textit{eine Idee} gained some ground only around 1900. [\textit{eine Idee} X] presumably emerged in the 19th century. The 7 tokens extracted from the DTA stem from that time; one of them is reproduced in (\ref{ex:ex12}). 

\begin{exe}
	\ex \label{ex:ex12}  Ceara: Eine der Maranham sehr ähnliche Baumwolle, vielleicht 	sogar \textit{eine Idee} besser. \\
	‘Ceará: a type of cotton very similar to Maranhão cotton, 	possibly even \textit{a bit} (lit.: \textit{idea}) better.' \\ (1889, DTA: Justi) 
\end{exe}

With not a single token in the DTA data, the modifier [\textit{ein Tick} X] seems to be even younger than \textit{eine Idee}. 

The development of \textit{eine Idee}, \textit{ein Tick} or any other German quantifier\slash degree\hyp{}modifier construction grammaticalising during the 20\textsuperscript{th} century must be understood in the context of a variety of co-existing near-synonymous constructions. Not only are there the three other constructions analysed in the present study, i.e. \textit{ein wenig}, \textit{ein bisschen} and \textit{ein Quäntchen}, but the constructional family comprises several more lower-frequency members, two of which are exemplified in (\ref{ex:ex13}) and (\ref{ex:ex14}). 

\begin{exe}
	\ex \label{ex:ex13} \textit{Eine Prise} Liberalismus wird dem Land guttun \\
	‘\textit{a pinch} of liberalism will do the country good’ \\ (2012, DeReKo: Die Zeit) 
	
	\ex \label{ex:ex14} Wir fanden Woody immer noch wunderbar -- und schon \textit{eine Spur} langweilig.\\
	‘We still found Woody wonderful -- and yet \textit{a bit} (lit.: \textit{trace}) boring.’ \\ (1993, DeReKo: Die Zeit)
\end{exe}

By the end of the 19\textsuperscript{th} century at the very latest (but probably much earlier; recall the observations on the frequency profile and colloquial status of \textit{ein bisschen} in \sectref{sec3-3}), constructional network constellations had developed that assist the grammaticalisation of novel quantifier\slash degree\hyp{}modifier constructions. Multiple micro\hyp{}constructions, above all the highly grammaticalised \textit{ein bisschen} and \textit{ein wenig}, had entered into a paradigmatic relation. As discussed in \sectref{paradigms}, the coexistence of multiple formally and functionally similar expressions, some of which being highly frequent, assist the recognition and entrenchment of an overarching mid-level schema. In the present case, this schema or meso-construction features an emergent slot with the feature ‘small\slash minor unit’: [\textit{ein} \textsc{small/minor-unit x}]. Such a meso-construction, together with some frequent members, can be conceived of as an attractor set for the development of novel micro\hyp{}constructions. According to this line of reasoning, \textit{ein Tick} and \textit{eine Idee} were attracted to the existing, increasingly strengthened quantifier\slash degree\hyp{}modifier meso-construction in a scenario of analogically driven grammaticalisation. This account is empirically supported above all by the observation that the diachronic context expansion of \textit{ein Tick} and \textit{eine Idee} did not proceed along a multi-stage grammaticalisation path but rather resembles an instant recruitment as degree modifiers. 

The notion of a German quantifier\slash degree\hyp{}modifier meso-construction, however, needs some refinement to do justice to the usage profiles of its hypothesised members. This meso-construction should not be conceived of as a single homogeneous mid-level schema but rather as an assembly with multiple subschemas. For one thing, not all associated micro\hyp{}constructions are equally productive as both quantifiers and degree modifiers. As shown in the corpus analysis above, degree\hyp{}modifier uses of \textit{ein Quäntchen} are uncommon, while quantifier uses have low relative frequencies in the usage profiles of \textit{ein Tick }and \textit{eine Idee}. What is more, even within one of the two functional domains, individual micro\hyp{}constructions may pattern more locally. For instance, the degree\hyp{}modifier constructs of \textit{ein Tick} and \textit{eine Idee} in the corpus have been found to combine overwhelmingly with graded adjectives. Such construction-specific differences make a unified account seem somewhat problematic. Clearly, each micro-construction has its own unique properties and history. However, it is important to stress that there are certain dominant micro\hyp{}constructions that freely participate in most or all quantifier and degree\hyp{}modifier subschemas. The degree\hyp{}modifier subschema with graded adjectives, which is central to \textit{ein Tick} and \textit{eine Idee}, is also served by \textit{ein bisschen} (\ref{ex:ex15}) and \textit{ein wenig} (\ref{ex:ex16}) with a substantial (absolute and relative) token frequencies.

\begin{exe}
	\ex \label{ex:ex15}\textit{Ein bißchen} amüsanter ist es hier doch. \\
`It is \textit{a bit} more amusing here.' \\ (1897, DeReKo-HIST: HK3) 

	\ex \label{ex:ex16} Der Diskurs dauert mir \textit{ein wenig} zu lang. \\ 
`The discourse takes \textit{a little} too long for my taste.' \\ (1823, DeReKo-HIST: HK3)
\end{exe}

Highly productive members such as \textit{ein bisschen} and \textit{ein wenig} can be expected to serve as prototypes of the constructional family. Other, less productive members with more specific constraints are associated with the prototype, and they are associated with each other through family resemblance (see \citealt{TraugottGrammaticalizationNPNP2008}: 33 for a similar account of English [\textsc{n} \textit{of} \textsc{np}] patterns). 

There is not only a great formal similarity between the family members but also a considerable functional overlap regarding the quantifier and degree\hyp{}modifier subschemas served by them. This hypothesised network constellation with interrelated quantifier and degree\hyp{}modifier subschemas at the meso-level can be argued to promote the analogical recruitment of novel micro\hyp{}constructions as well as the analogical realignment of existing ones, turning some quantifiers into degree modifiers and vice versa. Without the support of a quantifier\slash degree\hyp{}modifier meso-construction and subschemas linking extant micro\hyp{}constructions, the observed emergence of a variety of lower-frequency degree modifiers and quantifiers would be hard to account for. 


\section{Schematisation and paradigmaticisation in morphosyntax} \label{sec4}
\begin{sloppypar}
The grammaticalisation of individual micro\hyp{}constructions and constructional families involves schematisation in multiple ways. In micro\hyp{}constructions, shifts towards grammatical functions translate into increasing schematicity not only at the level of constructional semantics but also at the level of the constructional slots as reflected by the types of collexemes entering these slots \citep[cf.][]{Perek2020}. In other words, the formerly more contentful, referential meaning of a grammaticalising micro-construction becomes more abstract and more schematic in this regard, and the slots can be understood as schematic categories constraining the host-class \citep{Himmelmann2004} of the construction. In the present case study, it has been illustrated that the slots of less grammaticalised quantifier\slash degree\hyp{}modifier constructions are less schematic and more constrained. The [\textit{ein Quäntchen}~X] construction, for example, seems to be structured around one prototypical exemplar, \textit{ein Quäntchen Glück}. The slot of the highly grammaticalised [\textit{ein bisschen}~X], in contrast, has hardly any semantic constraints and hosts virtually all parts of speech. Another dimension of schematisation is the formation and increasing entrenchment of higher-level schemas, which we refer to as meso\hyp{}constructions (and macro-constructions) and which emerge as generalisations over multiple structurally similar micro\hyp{}constructions. At high levels of abstraction, such as a potential macro-construction for German quantifiers and/or degree\hyp{}modifiers, the micro\hyp{}constructions studied here might be linked to formally dissimilar, but functionally related constructions such as the single-word modifiers \textit{etwas} ‘somewhat’, \textit{leicht} ‘slightly’, \textit{sehr} ‘very’ and \textit{viel} ‘much’. However, there appears to be little empirical ground for the postulation of such a highly abstract link as a potential source of mutual influence in the development of such formally dissimilar micro\hyp{}constructions. There is more reason to argue that change in the early grammaticalisation and paradigmaticisation of syntactic constructions is guided mainly by associations to micro\hyp{}constructions and lower-level meso\hyp{}constructions. 
\end{sloppypar}

There are signs of analogical and thus paradigmatic forces, but this influence is limited in that the micro\hyp{}constructions in our case study each have their unique diachronic trajectories and retain idiosyncratic properties. For instance, although the degree modifiers \textit{ein Tick} and \textit{eine Idee}, skipping the quantifier stage expected in a scenario of independent grammaticalisation, appear to have evolved through analogy, their use has been shown to be largely restricted to one specific subtype of degree modification, namely combinations with graded adjectives. Possibly, such construction-specific properties are effects of persistence \citep{Hopperprinciplesgrammaticalization1991} based on features of the original lexical source construction. They can, however, also be thought of as subschemas at the meso-level if there is a subset of two or more micro\hyp{}constructions within the family that share a particular distribution. In our case study, this applies to \textit{ein Tick} and \textit{eine Idee}, and this also applies to \textit{ein bisschen} and \textit{ein wenig} to the extent that a considerable proportion of their uses match the hypothesised degree\hyp{}modifier subschema for graded adjectives. From this perspective, the observation that some grammaticalising micro\hyp{}constructions pattern more locally in usage than might be expected need not always result from micro-level persistence (or competition) but may be linked to meso-constructional subschemas \citep[cf.][29]{LangackerDynamicUsageBasedModel2000}. The lower the degree of schematicity, the more accessible a template tends to be – that is what a lot of usage-based research suggests (cf. \sectref{paradigms}). Accordingly, novel micro\hyp{}constructions in analogically driven grammaticalisation are likely to be most strongly attracted to the lowest-level schemas entrenched at the meso-level.

Families of grammatical(ising) constructions form a layered \citep{Hopperprinciplesgrammaticalization1991} domain of grammar shaped by diverse, partly opposing forces. Older and younger micro\hyp{}constructions both support and compete with each other \citep[cf.][]{DelorgeCompetingtransferconstructions2014, DeSmet2017, DeSmetchangingfunctionscompeting2018}. The present case study has focused on the role of support, but the aspect of competition in the German quantifier\slash degree\hyp{}modifier constructions is also evident, for example in the aforementioned disappearance of the old \textit{ein lützel} construction accompanying the rise of \textit{ein wenig}. Another possible outcome of competition is division of labour and thereby the specialisation of micro\hyp{}constructions, which might be the fate of some of the lower-frequency expressions studied here (e.g. \textit{ein Quäntchen}, \textit{ein Tick}). At the same time, analogical forces cause micro\hyp{}constructions to converge with respect to certain formal and/or functional features, creating more homogenous constructional families. It is remarkable that this convergence and growth has been occurring in the German constructional family under scrutiny, when considering that some of the micro\hyp{}constructions originated from fairly diverse lexical sources, including concrete nouns like \textit{Biss}(\textit{chen}) ‘bite’, abstract nouns like \textit{Idee} ‘idea’, a commercial weight in the case of \textit{Quentchen/Quäntchen}, and even an adjective in the case of \textit{wenig} ‘minor, small’. The observed convergence and growth seem to be caused, above all, by the pervasive process of analogical thinking and pattern matching, i.e. the establishment of form–function associations across constructions (e.g. \citealt{FischerGrammaticalizationanalogicallydriven2011, TraugottConstructionalizationConstructionalChanges2013}: esp. Section 1.6.4.2). Collectively, the diachronic trajectories of the micro\hyp{}constructions examined here amount to a macro-process that can be thought of as (early) paradigmaticisation. 

The macro-process of paradigmaticisation is a long-term effect of numerous (low-level) constructional changes whose directionality is constrained above all by the domain-general cognitive processes of analogy and automation. Early and advanced paradigmaticisation seem to involve different kinds of paradigms-as-schemas and different degrees of automation typically linked to the transition from syntax to morphology. In early paradigmaticisation, as typical of sets of fairly heterogeneous periphrastic constructions in syntax, the parts of the network seized by this process are micro\hyp{}constructions as well as lower-level schemas at the meso-level. As long as the members of the constructional family exhibit rather heterogeneous properties, high-level generalisations such as macro-constructions can be expected to be only weakly present (if at all) in speakers’ minds. This stage of paradigmaticisation involves emergent paradigms that are still relatively open; micro\hyp{}constructions that are part of such emergent paradigms may be quasi-synonymous, and their use is still optional \citep[cf.][]{DiewaldParadigmaticintegration2012}. This phase is witnessed in the case of the quantifier\slash degree\hyp{}modifier constructions in present-day German. In more advanced stages of paradigmaticisation, the relations between the family members become more rigid. The members tend to be formally more homogeneous, but quasi-synonymous uses have given way to contrastive meanings; and the paradigms are essentially closed. This stage of paradigmaticisation corresponds to degrees of grammaticalisation that are prototypically associated with (inflectional) morphology rather than periphrastic syntactic constructions. It seems that only as a constructional family approaches such late stages in its life cycle do the relations of its members become defined by one organisational unit at a very high level of abstraction. 

Grammatical(ising) micro\hyp{}constructions are not teleologically determined to become more homogeneous and meet in paradigms; still, this direction of change is highly likely. Closing this section, we summarise what we argue to be the reasons for this long-term trend. One causal factor is how the domain-general cognitive process of analogy shapes language structure \citep[e.g.][]{AntillaAnalogywarpwoof2003}. The present paper has discussed the constructional network constellations – “attractor sets” involving meso\hyp{}constructions – that promote the analogical emergence and/or change of micro\hyp{}constructions. These analogical forces are certainly influential, but even beyond analogy as a cognitive basis of paradigmatic relations there are forces that lead grammaticalising micro-construction in similar directions, making them more alike as well. These directions are the well-known tendencies of semantic generalisation and morphosyntactic fixing and reduction. Cognitive mechanisms and motivations underlying these strongly directional changes in grammaticalisation include chunking, increasing ease of retrievability, habituation, neuromotor practice and frequency-induced predictability among others. Many of the underlying processes can be understood as concomitants of the even more general cognitive process of automation (\citealt{BybeeLanguageUsageCognition2010, NeelsRefiningfrequencyeffectexplanations2020}). The asymmetry inherent in automated versus more controlled activity should be considered a cognitive key factor accounting for the unidirectionality of grammaticalisation: since highly automated grammatical operators are withdrawn from conscious control, they are very unlikely to be manipulated for more contentful, referential – i.e. more “lexical” – purposes \citep[cf.][]{LehmannGrammaticalizationautomation2017, HaspelmathWhygrammaticalizationirreversible1999}. Automation serves efficiency; and, as pointed out in \sectref{paradigms}, grammatical paradigms represent efficient solutions in processing. In short, at the general level of language\hyp{}related cognition and performance, paradigm formation is governed by efficiency, automation and analogy; at the level of linguistic representations, the locus of change can be modelled as clusters of interrelated schemas, above all micro- and meso\hyp{}constructions.

\section{Conclusion} \label{sec5}
This paper has investigated a family of German quantifier\slash degree\hyp{}modifier constructions against the theoretical background of construction grammar and in the light of research on grammaticalisation. Regarding the theoretical background, construction grammar has been shown to have a somewhat ambivalent relationship to paradigms: some proponents of constructionist approaches to language have rejected the concept, others have suggested that it might be theoretically useful for capturing horizontal relations in the constructicon, especially in the domain of mid-level constructions. \largerpage{}
Following up on a previous study showing that the development of \textit{ein bisschen} and \textit{ein wenig} bears many similarities to the development of English \textit{a bit} in the sense of a gradual extension from more concrete to more abstract readings and from quantifier to degree\hyp{}modifier uses, we have investigated the low-frequency constructions [\textit{eine Idee} X], [\textit{ein Tick} X], and [\textit{ein Quäntchen} X] in contemporary German. The overall picture emerging from the corpus analyses is that the older quantifier\slash degree\hyp{}modifier constructions served as attractor sets for an increasingly strengthened mid-level schema, i.e. meso-construction, thereby promoting the analogically driven grammaticalisation of younger micro\hyp{}constructions. This scenario is supported by the finding that younger micro\hyp{}constructions (i) approximate the usage patterns of more established ones, and (ii) “skip” stages on the cross-linguistically attested grammaticalisation path pre-partitive > partitive > quantifier > degree modifier. For modelling such linguistic developments, the constructionist idea of grammaticalising expressions being linked via meso\hyp{}constructions has proved fruitful. Recognising types of grammaticalisation that are enabled by meso\hyp{}constructions and analogy – as opposed to independent/“pure” grammaticalisation \citep[]{Lehmann2004a} – moreover helps reconcile frequency-effect approaches with the phenomenon of low-frequency grammaticalisation. The infrequent modifiers \textit{ein Tick} and \textit{eine Idee} seem to owe their existence to an established template, but they also exhibit constraints that cannot be derived solely from a single overarching mid-level schema. Such properties specific to individual micro\hyp{}constructions or subsets of constructions may be interpreted not just as persistence but as entrenched subschemas or local emergent paradigms. Early paradigmaticisation has been argued to involve emergent paradigms of multiple orders in the sense of constructions at multiple levels of schematicity. Change is guided mainly by associations to micro\hyp{}constructions and lower-level meso\hyp{}constructions. Only in advanced stages of grammaticalisation and paradigmaticisation, when micro\hyp{}constructions become sufficiently homogeneous, do higher-level meso\hyp{}constructions and macro-constructions – which come closest to paradigms in the traditional understanding of the term – act as decisive organisational forces. 

\section*{Data availability}
\begin{sloppypar}
The annotated raw data are available at
\url{https://osf.io/t9j7p/}.
\end{sloppypar}

\section*{Acknowledgements}

The authors would like to thank the editors of the present volume as well as the anonymous reviewers for helpful comments and suggestions. The usual disclaimers apply.

{\sloppy\printbibliography[heading=subbibliography]}

\end{document}
