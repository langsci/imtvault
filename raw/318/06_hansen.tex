\documentclass[output=paper]{langsci/langscibook} 
\ChapterDOI{10.5281/zenodo.5675849}
\author{Bjarne Simmelkjær Sandgaard Hansen\affiliation{University of Copenhagen}}
\title[Redundant indexicality and paradigmatic reorganisations in Middle Danish]{Redundant indexicality and paradigmatic reorganisations in the Middle Danish case system}
\abstract{The Danish case system changed profoundly throughout the Middle Danish era. Based on examples from mainly three texts written in East Danish (Scanian dialect), I describe the steps of and stages in these changes and claim that they were caused neither by an unstressed-vowel-neutralising sound law nor by language contact as often assumed, but by various interrelated processes of grammaticalisation. I focus on one of these processes, viz., that the fixed topology of the Middle Danish noun phrase simply made noun-phrase internal agreement by means of case marking redundant and caused the loss of the indexical relations signalling this agreement, which, in turn, contributed to the gradual phase-out of case marking. Moreover, I relate this phase-out to two general linguistic principles, viz. those of markedness agreement \citep[27--37]{Andersen2001} and single encoding \citep[258–261]{Norde2001}. Finally, based on \citet[5--6]{Nørgård-Sørensen2011} and \citegen[262--263]{Nørgård-Sørensen2015} five criteria for what constitutes a grammatical paradigm, I also demonstrate that, irrespective of the existence of some level of free variation, the Middle Danish case system may be described paradigmatically and, correspondingly, that the changes it undergoes constitutes an instance of paradigmatic and thus grammatical change.}

\begin{document}
\maketitle 

\section{Introduction} \label{hansen:1}

Grammaticalisation as defined by \citet[xi, 71–72]{Nørgård-Sørensen2011} and \citet[261--262]{Nørgård-Sørensen2015} equals paradigmatisation. This implies that, in order to count as an instance of grammar, any linguistic phenomenon must me paradigmatic, and any grammatical change must be describable as a paradigmatic change.

In the present paper, I will account for one type of grammatical and thus paradigmatic change, viz., the change of the case system in Middle Danish of the Scanian dialect from having four fully functional cases to, virtually, none in the nominal paradigm. I shall focus mainly on the changes that happened to the original genitive and dative and the change in relations between these and the accusative, since \citet{Jensen2011} has already effectively accounted for the changes happening to the relations between the original nominative and accusative.

More specifically, I will shed light on one among several factors causing this change, viz., redundancy in the indexical relations of noun-phrase internal agreement, and present the paradigmatic consequences of the changes caused by this and other factors, including in particular the paradigmatic consequence of seemingly free variation in the case system during the period of change. Finally, in tandem with that, I will discuss the consequence of free variation to the aforementioned scholars’ understanding of paradigmatic essence.

In order to fulfil these tasks, I will first provide synchronic descriptions of the Middle Danish use of case with examples from Middle Danish texts in \sectref{hansen:2}, after which I will seek to explain the developments of the case system by means of processes of grammaticalisation in \sectref{hansen:3}. These two sections, which build on – and constitute a concise version of – my previous outline of these matters in \citet{Hansen2021}, serve as the starting point for my discussion of the paradigmatic consequences, which will occupy \sectref{hansen:4}, while \sectref{hansen:5} will constitute the conclusion of the article.

\section{Data from Middle Danish} \label{hansen:2}
\subsection{Relevant texts} \label{hansen:2.1}

As I have described in detail in \citet[282--289]{Hansen2021} and will now recapitulate here, a comparison of three texts written in the East Danish (Scanian) dialect in the first half of the 15th century, viz., \textit{Skånske Lov} (SkL) after Cod. E don. var. 136, 4\textsuperscript{o}, \textit{Sjælens Trøst} (SjT), and \textit{Søndagsevangelier} (SdE), reveals the existence of multiple simultaneous systems of case application in this period. What follows is therefore a brief outline of the systems found in these texts. For a full description of the details in the systems, see \citet[282--289]{Hansen2021}.

\subsection{Case marking on both nouns and on typical modifiers} \label{hansen:2.2}

In one system, genitive and dative case is marked on all members of a noun phrase, i.e., both on nouns and on typical modifiers (adjectives, articles, pronouns, numerals and other determiners), in situations with a potential for genitive or dative government.

Marking for dative thus appears both on the possessive pronoun \textit{sinum} ‘their’ and the noun \textit{thiænarum} ‘servants’ in \REF{ex:hansen:1}, where the function of the noun phrase as the indirect object of \textit{budho} ‘they commanded’ triggers the use of dative, and similarly on both the indefinite pronoun \textit{ene} ‘a’, the adjective \textit{longe} ‘long’ and the noun \textit{iærnlenkio} ‘iron chain’ in \REF{ex:hansen:2}, where the preposition \textit{mæth} ‘with’ governs the dative. In the 15th century, this system prevails with the feminine singular and with the (genitive/dative) plural for all genders, but especially in SkL, this system is also found with other case forms, as evidenced by \REF{ex:hansen:3} which represents a case of a preposition governing the dative.

\ea \label{ex:hansen:1}
{SjT: 70\textsuperscript{25}} \\ 
\gll buth-o sin-um thiænar-um\\
     command-\textsc{pst.3pl} their-\textsc{m.dat.pl} servant\textsc{(m)-dat.pl}{\footnotemark}\\
\glt ‘they commanded their servants [that …]’
\footnotetext{All category labels used in interlinear glossing follow the Leipzig Glossing Rules, the only additions being \textsc{obl} and \textsc{c}, which signify “oblique” and “commune” (common gender), respectively.}
\ex \label{ex:hansen:2}
{SjT: 128\textsuperscript{27}}  \\ 
\gll mæth en-e long-e iærnlænki-o\\
     with a\textsc{{}-}\textsc{f.dat.sg} long\textsc{{}-}\textsc{f.dat.sg} iron.chain\textsc{(f)-dat.sg}\\
\glt ‘with a long iron chain’
\ex \label{ex:hansen:3}
{SkL after E don. var. 136, 4\textsuperscript{o}: 36v} \\ 
\gll a thredi-e thing-i\\
     on third\textsc{{}-}\textsc{n.obl.sg} moot\textsc{(n)-dat.sg}\\
\glt ‘on the third moot’
\z

\subsection{Case marking on typical modifiers only} \label{hansen:2.3}

A second system is the one found in examples like (\ref{ex:hansen:4}--\ref{ex:hansen:6}) where all members of a noun phrase but the typical noun-phrase head, i.e., the noun itself, are marked for case.\footnote{Although nouns do not inflect for case in this system, they still inflect for number; hence, I still need to mark them for inflectional endings as per, e.g., the ∅-ending in (\ref{ex:hansen:4}--\ref{ex:hansen:6}).}

\ea \label{ex:hansen:4}
{SkL after E don. var. 136, 4\textsuperscript{o}: 44v} \\ 
\gll at andr-u thing-∅\\
     at other\textsc{{}-}\textsc{n.dat.sg} moot\textsc{(n)-sg}\\
\glt ‘at the next/second moot’
\ex \label{ex:hansen:5}
{SjT: 29\textsuperscript{32}}\\ 
\gll gifv-in th-øm fatig-o folk-∅\\
     give\textsc{{}-}\textsc{imp.2pl} it\textsc{{}-}\textsc{acc.pl} poor\textsc{{}-}\textsc{n.dat.sg} people\textsc{(n)-sg}\\
\glt ‘give them to poor people’
\ex \label{ex:hansen:6} 
{SjT: 122\textsuperscript{9}}\\ 
\gll mæth en-um stor-um hær-∅\\
     with a\textsc{{}-}\textsc{m.dat.sg} great\textsc{{}-}\textsc{m.dat.sg} army\textsc{(m)-dat.sg}\\
\glt ‘with a great army’
\z

As revealed by a comparison of \REF{ex:hansen:3} and the structurally almost identical example \REF{ex:hansen:4}, at least SkL after E don. var. 136, 4\textsuperscript{o} had some level of apparently free variation between the former system of case marking on all members of the noun phrase and this system of case marking on all members but nouns. Because of this paradigmatic choice, it may also be, however, that the seemingly case-uninflected noun \textit{thing} ‘moot’ actually does inflect for case, but merely expresses the traditional dative content with the endingless accusative instead of the historically expected dative. Only in a linguistic system with no option left for case marking on nouns may one establish with certainty that an endingless noun is indeed also uninflected.

Examples (\ref{ex:hansen:5}--\ref{ex:hansen:6}) stem from SjT, and both display the same system of no case marking on nouns. Here, it seems more certain that the nouns \textit{folk} ‘people’ and \textit{hær} ‘army’ are, indeed, uninflected for case, since in this text, singular forms of masculine and neuter nouns ending in a consonant never enter into a paradigmatic opposition with variants that are inflected for case.

On the surface, example \REF{ex:hansen:7}, which stems from SdE, resembles (\ref{ex:hansen:4}--\ref{ex:hansen:6}) by marking for case on typical modifiers only, not on nouns. Here, however, the traditionally dative-governing preposition \textit{met} ‘with’ suddenly governs the accusative instead. For that reason, the endingless form \textit{renlik} ‘cleanliness’ in \REF{ex:hansen:7} might actually, at least theoretically, have represented the accusative of the noun rather than the noun stripped for case marking, just as with the noun \textit{thing} ‘moot’ in \REF{ex:hansen:4}. The second member of this prepositional phrase, i.e., \textit{fasta} ‘fasting’, clearly represents the form uninflected for case, however, since the accusative would have been expressed by the oblique form *\textit{fasto} or *\textit{fastæ}/\textit{faste} (with unstressed-vowel neutralisation).

\ea \label{ex:hansen:7}
{SdE: 42\textsuperscript{27}} \\ 
\gll met lang-en renlik-∅ ok fasta\\
     with long\textsc{{}-}\textsc{m.acc.sg} cleanliness\textsc{(m)-(acc.?).sg} and fasting\textsc{(f).sg}\\
\glt ‘with long cleanliness and fasting’
\z

The most remarkable feature of \REF{ex:hansen:7} is, however, not the status of \textit{renlik} and \textit{fasta} as either accusatives or bare nouns uninflected for case, but the unambiguous use of the accusative in the adjective \textit{langen} ‘long’. As I have already mentioned, the preposition \textit{met} ‘with’ traditionally governed the dative, not the accusative. If this were a standalone example of failure of choosing the historically expected case, I might have dismissed it as a simple error, but seeing that one may come across several structurally similar examples – e.g., (\ref{ex:hansen:8}--\ref{ex:hansen:9}) with the preposition \textit{til} ‘to’, which traditionally governed the genitive – this is hardly the case.

\ea \label{ex:hansen:8}
{SjT: 32\textsuperscript{27–28}} \\ 
\gll af hørdh-a-na\\
     of herdsman(\textsc{m)-pl-}the\textsc{.m.acc.pl}\\
\glt ‘from the herdsmen’
\ex \label{ex:hansen:9} 
{SjT: 96\textsuperscript{18}} \\ 
\gll til en hælgh-an abod-a\\
     to a\textsc{{}-}\textsc{m.nom/acc.sg} holy\textsc{{}-m.acc.sg} abbot\textsc{(m)-obl.sg}\\
\glt ‘to a holy abbot’
\z 

Rather, the explanation for the use of the accusative in (\ref{ex:hansen:7}--\ref{ex:hansen:9}) lies either with prepositions simply allowing accusative government in addition to their traditional government, i.e., \textit{met} ‘with’ may govern either the accusative or the traditional dative, or with the accusative simply taking optionally over for the dative and genitive in all regards, i.e., so-called participation (\cites{Hjelmslev1935}[87]{Hjelmslev1970}[8–10, 38–40]{Bjerrum1966}[46]{Andersen2001}[16–18]{Heltoft2010}{Jensen2012}, among others). The concept of participation constitutes a typical example of markedness relations. In an opposition between two members of a paradigm, one member is restricted in its functions (in this case: the genitive and the dative), whereas the other member both covers its own restricted functions and may participate in the functions of the first member (in this case: the accusative). Needless to say, accusative participation may also constitute an important trigger for the loss of case marking on Middle Danish nouns seen in (\ref{ex:hansen:4}--\ref{ex:hansen:6}), since in most nominal classes, the accusative ended in \textit{{}-∅} and was thus formally identical to the bare stem; see also \citet[250--251]{Norde2001} on a similar situation in Old Swedish. 

\subsection{No case marking} \label{hansen:2.4}

Finally, I have recorded a third – or is it a fourth? – system in Middle Danish in which case marking is neither present on nouns, nor on the typical noun-phrase modifiers, as evidenced by (\ref{ex:hansen:10}--\ref{ex:hansen:11}). This is largely reminiscent of the nominal system recorded in modern Danish, which (\ref{ex:hansen:12}--\ref{ex:hansen:13}) serve to illustrate.

\ea \label{ex:hansen:10}
{SjT: 122\textsuperscript{9}} \\ 
\gll mæth en-∅ stor-∅ hær-∅\\
     with a\textsc{{}-}\textsc{m.sg} great\textsc{{}-}\textsc{m.sg} army\textsc{(m)-sg}\\
\glt ‘with a great army’
\ex \label{ex:hansen:11}
{SdE: 17\textsuperscript{8–9}} \\ 
\gll fran all-∅ køtlik-∅ lust-∅\\
     from all\textsc{{}-}\textsc{f.sg} corporeal\textsc{{}-f.sg} lust\textsc{(f)-sg}\\
\glt ‘from all pleasures of the flesh’
\ex \label{ex:hansen:12}
{Modern Danish} \\ 
\gll jeg giv-er e-t klog-t barn bog-∅-en\\
     I give-\textsc{prs} a{}-\textsc{n.sg} wise\textsc{{}-n.sg} child\textsc{(n){\textbackslash}sg} book-\textsc{sg-}the\textsc{.c.sg}\\
\glt ‘I give the book to a wise child’
\ex \label{ex:hansen:13}
{Modern Danish} \\ 
\gll til e-n ung-∅ dreng-∅\\
     to a-\textsc{c.sg} young\textsc{{}-c.sg} boy\textsc{(c)-sg}\\
\glt ‘to a young boy’
\z

I cannot rule out completely, though, that \REF{ex:hansen:10} represents not the lack of inflection, but instead a traditional nominative outside of its original domain and with the novel function of marking foreground information \citep[264]{Jensen2011}, but \REF{ex:hansen:11} is an unequivocal example of true caseless forms in both the modifier and the noun, as are (\ref{ex:hansen:12}--\ref{ex:hansen:13}) from modern Danish.

\subsection{Older and younger stages of Danish} \label{hansen:2.5}

When compared to the situation in an older manuscript of SkL (such as Cod. Holm. B 74, 4\textsuperscript{o} from the first half of the 13th century) and to later linguistic stages such as modern Danish, it is evident that what the nominal system undergoes is a development from case marking with the historically expected case on all members of a noun phrase, as in (\ref{ex:hansen:1}--\ref{ex:hansen:3}), to no case marking at all, as in (\ref{ex:hansen:10}--\ref{ex:hansen:11}).

For instance, as witnessed by (\ref{ex:hansen:14}--\ref{ex:hansen:16}), SkL after Cod. Holm. B 74, 4\textsuperscript{o} displays many instances of traditional case marking, even in contexts where traditional case marking would normally not occur on nouns in the 15th century, viz., in the masculine and neuter singular. I have also found examples of accusative participation on the noun as in \REF{ex:hansen:17}, where the accusative form \textit{brofial} ‘plank’ replaces the historically expected dative form *\textit{brofialu} or *\textit{brofialo} \citep[39]{Bjerrum1966}. This form, to which the participating accusative form \textit{brofial} would stand in paradigmatic opposition, is not attested in Cod. Holm. B 74, 4\textsuperscript{o}, however, but cf. the dative form \textit{brofiæle} in \REF{ex:hansen:18} from an addendum to \textit{Skånske Lov} (Add.SkL), printed in a manuscript (Cod. Holm. B 73) that exhibits neutralisation of some unstressed vowels.

\ea \label{ex:hansen:14}
{SkL after Cod Holm. B 74, 4\textsuperscript{o}: Ch. 145} \\ 
\gll til annar-s thing-s\\
     to other-\textsc{n.gen.sg} moot\textsc{(n)-gen.sg}\\
\glt ‘to the next/second moot’
\ex \label{ex:hansen:15}
{SkL after Cod Holm. B 74, 4\textsuperscript{o}: Ch. 145} \\ 
\gll at andr-u thing-i\\
     at other-\textsc{n.dat.sg} moot\textsc{(n)-dat.sg}\\
\glt ‘at the next/second moot’
\ex \label{ex:hansen:16}
{SkL after Cod Holm. B 74, 4\textsuperscript{o}: Ch. 158} \\
\gll gifu-ær andr-um mann-j thiuf sac-∅\\
     give-\textsc{prs.3sg} other-\textsc{m.dat.sg} man\textsc{(m)-dat.sg} thief charge\textsc{(f)-acc.sg}\\
\glt ‘[if he] accuses another man’
\ex \label{ex:hansen:17}
{SkL after Cod Holm. B 74, 4\textsuperscript{o}: Ch. 142} \\
\gll ofna brofial-∅ sinn-j\\
     on plank\textsc{(f)-acc.sg} his\textsc{{}-f.dat.sg}\\
\glt ‘in his house’
\ex \label{ex:hansen:18}
{Add.SkL: Ch. 1}\\
\gll a brofiæl-e sin-æ\\
     on plank\textsc{(f)-dat.sg} his\textsc{{}-f.dat.sg}\\
\glt ‘in his house’
\z

Conversely, the frequency of case marking decreases significantly in a 16th-century post-reformation text such as the first full Danish bible translation, \textit{Christian 3.s danske Bibel} (Chr.3.B) from 1550. Examples (\ref{ex:hansen:19}--\ref{ex:hansen:20}) from Chr.3.B illustrate the absence of case marking even in such contexts where case distinctions were kept for the longest time, viz., in the feminine singular and in the plural, respectively.

\ea\label{ex:hansen:19}
{Chr.3.B: E919 (Luk.II)} \\
\gll ligg-endis i e-n krubbe-∅\\
     lying-\textsc{prs.ptcp} in a\textsc{{}-f/c.sg} manger\textsc{(f/c)-sg}\\
\glt ‘lying in a manger’
\ex \label{ex:hansen:20}
{Chr.3.B: E21 (Gen.XII)} \\
\gll met stor-e plaffue-r\\
     with great-\textsc{f/c.pl} plague\textsc{(f/c)-pl}\\
\glt ‘with great plagues’
\z

\subsection{Stages in and developments of the Middle Danish case systems} \label{hansen:2.6}

When taking into account both my outline of the situation prior and posterior to the 15th century and the possible steps of and pivots for reanalysis, I may sum up the potential stages in and developments of the Middle Danish case systems as follows.

\begin{enumerate}
\item  Starting point: Historically expected case on both nouns and typical modifiers
\item  Historically expected case on typical modifiers and optional accusative participation on nouns
\item  Historically expected case on typical modifiers and no case marking on nouns (excluding the genitival clitic \textit{{}-s})
\item  Optional accusative participation on typical modifiers and no case marking on nouns (excluding the genitival clitic \textit{{}-s})
\item  End point: Neither case marking on nouns nor on typical modifiers (excluding the genitival clitic \textit{{}-s})
\end{enumerate}

As I have already shown, these systems and subsystems existed side by side within one and the same text. A simple comparison of, e.g., \REF{ex:hansen:1}, \REF{ex:hansen:6}, \REF{ex:hansen:8} and \REF{ex:hansen:10}, which all stem from SjT, serves to illustrate this.

\section{Explaining the developments of the Middle Danish case system} \label{hansen:3}
\subsection{Traditional views} \label{hansen:3.1}\largerpage

Traditionally, two views have prevailed on how the changes in the Middle Danish case system came about, the former generally more accepted – or at least more frequently mentioned – than the latter. 

First, viewing the reductions in the Danish case system as a result of sound laws, above all the Danish unstressed-vowel-neutralising sound law \textit{e}/\textit{a}/\textit{o} > [ə], has long been the prevalent position among historical linguists (\citealt[XIV–XV, 16–17]{FalkFalk1900}, \citealt[71, 95–100, 113]{Meillet1922}, \citealt[266]{Skautrup1944}, etc.). This reductionist or “phonology-first” view holds that the coalescence of unstressed vowels (i.e., typically vowels in non-first syllables) resulted in homophony and syncretism of many inflectional endings. Such an explanation may theoretically work not only for Danish, but also for other languages with phonological reductions in non-first syllables; see, e.g., \citet[167--168]{BarberBarber2009} on English.

Second, \citet[27]{Wessén1954} regards Middle Low German influence as (one) reason for the Scandinavian case-system reductions:

\begin{quote}
Vi har stor anledning att tro, att det främmande inflytandet har sträckt sig jämväl till ordens böjning och till uttalet. Då fornspråkets rika formsystem mot medeltidens slut upplöses och förenklas, har man med skäl sökt en av orsakerna därtill i att de inflyttade tyskarna aldrig kunde lära sig att rätt bruka de gamla kasusformerna och ändelserna; deras förenklade ordböjning smittade efterhand av på landets egna barn.\footnote{My translation: ``We have great reason to believe that the foreign influence has encompassed even the inflection of words and the pronunciation. Since the rich morphological system of the ancient language is dissolved and simplified around the end of the mediaeval period, scholars have reasonably regarded the following circumstance as one of the reasons for that, viz., that the immigrating Germans could never learn to use the old case forms and endings correctly; their simplified inflection gradually rubbed off on the country’s own children.”}
\end{quote}

Although admitting that it is difficult to establish the exact extent of the Middle Low German influence, \citet[65]{Haugen1976} agrees with Wessén by noting that both English and Scandinavian underwent case-system reductions while dominated by other languages and that Low German has a structure similar to that which the mainland Scandinavian languages adopted. Several scholars still include this specific influence or similar types of language contact in their list of causes for the mainland Scandinavian case-system reductions; see, e.g., \citet[243]{Norde2001} on Swedish.

\subsection{Challenging the traditional views} \label{hansen:3.2}

Appealing as these two traditional explanations may seem, they suffer from some major deficiencies.

Taking the reductionist view first, the neutralisation of unstressed vowels and the reductions in the case system simply do not seem to be connected. One would expect this sound law to be operational in texts where the case system is in the process of change, but as all the examples from SjT reveal, this is certainly not the case. For instance, \REF{ex:hansen:9} shows an instance of accusative where the traditional system would have dictated a genitive, even though both \textit{hælghan} ‘holy’ and \textit{aboda} ‘abbot’ preserve the unstressed \textit{{}-a}, and in \REF{ex:hansen:5}, \textit{fatigo} ‘poor’ keeps its unstressed \textit{{}-o} in spite of the absence of the historical dative singular ending *\textit{{}-i}/*\textit{{}-e} on \textit{folk} ‘people’. The reductionist view meets the exact same challenge when attempting to explain the case-system reductions in Swedish. As \citet[18]{Jensen2011} points out, the Swedish case system has been reduced and changed to the same extent and more or less with the same result as the Danish one, but Swedish does not display any substantial weakening of unstressed vowels.

SdE also demonstrates the mismatch between the unstressed-vowel-neu\-tral\-ising sound law and the case-system reductions, but in the opposite manner. In SdE, unstressed vowels have been neutralised in many positions, as revealed by, e.g., \textit{langen} (< \textit{langan}) ‘long’ in \REF{ex:hansen:7}, but this neutralisation has not prevented a by-and-large retention of the Danish case system, illustrated again by the accusative form \textit{langen} in \REF{ex:hansen:7}. Following \citet[4--5]{BaechlerBaechler2018}, I will stress this point even further by drawing attention to a language like standard German, in which the process of unstressed-vowel neutralisation is just as advanced as in modern Danish despite the preservation of a functional distinction between all four historical cases: nominative, accusative, genitive and dative.

Finally, leaving the possibility of a causal correlation between the unstressed-vowel-neu\-tral\-ising sound law and the case-system reductions aside, this sound law is not capable of explaining the loss of consonantal endings. Neither the functional changes and subsequent loss of the historical nominative singular ending \textit{{}-Vr} in the mainland Scandinavian area \citep[18]{Jensen2011} nor the Faroese and dialectal Norwegian and Swedish loss of the genitive in \textit{{}-s} \citep[6--7]{Enger2013} finds any catalyst in reductionist sound laws. To sum up my objections against the reductionist view, I will cite \citet[42]{Loporcaro2018} who concludes on some instances of gender agreement in Italian dialects that “there is no deterministic impact of sound change on morphosyntax”.

Turning now to the second traditional explanation of the mainland Scandinavian case-system reductions, i.e., language contact, I will call attention to \citegen[177--182]{Ringgaard1986} highly valid objection of a mere chronological mismatch. Followed by \citet[2--3]{Askedal2005} and \citet[13--14]{Enger2013}, Ringgaard claims that Middle Low German would exert its allegedly system-changing influence too late on Danish and the remaining mainland Scandinavian languages for it to constitute a factor. The Middle Low German influence was most pervasive in the 14th century, but one may register case-system reductions already in the earliest manuscripts of the Danish regional laws, the language of which may have been settled as early as the end of the 12th century. In all fairness, while Ringgaard’s criticism is indeed relevant for Danish, it may be slightly less so for Swedish, for which \citet[27]{Wessén1954} postulated his claim originally.

\subsection{Processes of grammaticalisation} \label{hansen:3.3}

Scholars like Andersen (e.g. \citeyear[143--144]{Andersen2010}), Heltoft (e.g. \citeyear{Heltoft2010}, \citeyear{chapters/07_heltoft} [this volume]), and \citet{Petersen2018} offer an alternative to the traditional explanations of the Danish case-system changes. They all attribute certain instances of language change, including those of the Danish case system, to processes of grammaticalisation, which they define as processes of change in the function and contents of the grammatical signs and in the paradigmatic oppositions between them (\cites{Andersen2006b}[123]{Andersen2010}[xi, 7–8, 11–17]{Nørgård-Sørensen2011}[261–262]{Nørgård-Sørensen2015}, etc.).\footnote{By entailing both grammation, regrammation and degrammation (i.e., the rise, change and dissolution of grammar, respectively) and also insisting on syntax (topology and constructions) and syntactic changes forming part of grammaticalisation \citep[43–45]{Nørgård-Sørensen2011}, this definition of grammaticalisation goes beyond the mainstream unidirectionality hypothesis advanced by, e.g., \citet[7]{HopperTraugott2003} and \citet[12, 121–123]{Lehmann1995}. According to the mainstream definition, grammaticalisation equals the rise of grammar by means of the movement of a linguistic element down the cline of grammaticality, i.e., a unidirectional development of grammar from syntax to morphology going through the stages from content item via grammatical word and clitic to inflectional affix, revealed by accompanying features such as phonetic reduction, increased syntactic bonding, desemanticisation, use in new contexts and increasing frequency. As a consequence of the definition by \citet{Nørgård-Sørensen2011} etc. of grammaticalisation as processes of change in the function and contents of the grammatical signs and in the paradigmatic oppositions between them, their type of grammaticalisation cannot limit itself to such a change from syntax to morphology, but must comprise also, e.g., the rise of morphologically and/or syntactically expressed grammatical sign oppositions from a reanalysis of formerly lexical items (grammation) and restructurings of existing morphologically and/or syntactically expressed grammatical oppositions (regrammation). Such regrammations may also comprise changes from morphological to syntactic expression of grammatical content oppositions (as in the grammaticalisation processes suggested in the present article) and not only changes from syntactic to morphological expression as per the limitations of the mainstream unidirectionality hypothesis.}

For instance, \citet[13--22]{Heltoft2010} describes the changes in the Middle Danish case system as part of a larger process that turns the Old Scandinavian noun phrase into a determiner phrase in the modern mainland Scandinavian languages. \citet[201–232, 283–311]{Jensen2011}, in turn, focuses specifically on a reanalysis of the relationship between the nominative and the accusative, resulting in nominatives marking only such subjects and subjective complements that also provide foreground information. Finally, \citet[e.g. 63–89]{Petersen2018} connects all the changes mentioned above to the rise of definite and indefinite articles and the subsequent shift in markedness between inflected nouns with and bare nouns without an article as well as to the rise of unity stress. According to Petersen, all these changes form part of the process that gives rise to the concept of incorporation in Danish.

\subsection{Redundancy in the indexical relations of noun-phrase-internal agreement} \label{hansen:3.4}

The three grammaticalisation-based approaches to explaining the Danish case-system changes mentioned in \sectref{hansen:3.3} by no means contradict each other. On the contrary, they complement each other, each contributing one important factor to the complex overall explanation of the systemic changes. Neither do they provide the full explanation, however. As I have advocated for in \citet{Hansen2021}, one additional factor must be added, viz., an apparent desire among the language users for eliminating redundancy in noun-phrase internal agreement as evinced by, e.g., the mere existence of the Middle Danish case-system stages 3–4 in \sectref{hansen:2.6}.\footnote{See also \sectref{hansen:3.5} on the application of \citegen[258–261]{Norde2001} principle of single encoding.}

Most grammaticalisation-based explanations of the case-system changes focus on noun-phrase-external relations and functions of case marking. For instance, \citet{Jensen2011} regards the functional change of the historical nominative from marking subjects and subjective complements to signalling foreground information as a decisive factor. However, although the primary function of case marking is, indeed, noun-phrase external in terms of 1) indexical reference to the valency of a predicate (i.e., revealing the argument status of the noun phrase in question) or to the government of an adposition and 2) symbolic reference to location, direction, means, etc. (\citealt[1080–1086]{Blake2004}, \citealt[2]{AndersenConf}, \citealt[154–155]{Heltoft2019}), case may also point indexically within the noun phrase. This is what creates noun-phrase-internal or endophoric agreement, i.e., that multiple members of a noun phrase inflect identically (on the functional level; the formal expression of the endings may differ) and point indexically to each other (\citealt[2]{AndersenConf}, \citealt[52]{Haspelmath1996}, \citealt[82]{Nielsen2010}, see also \citet[86–89]{Nielsen2010} on the general principle of what he labels \textit{conditioned agreement}).\footnote{For an outline of the general distinctions between symbolic and indexical sign relations, see \textcites[4–5, 27–30]{Andersen1980}{Andersen2010}{AndersenConf} with further reference to Peircean sign theory.}

Example \REF{ex:hansen:21} serves to illustrate this double function of case marking. Please note both the noun-phrase-external relations, because the noun phrase \textit{ondom quinnom} ‘evil women’ in the dative points indexically to the dative-governing preposition \textit{for} ‘against’, and the noun-phrase-internal or endophoric agreement, because the dative ending \textit{{}-om} of \textit{ondom} ‘evil’ and \textit{quinnom} ‘women’ point indexically to each other, signalling that they belong together as members of the same noun phrase.


\ea \label{ex:hansen:21}
{SjT: 52\textsuperscript{28}}\\
\gll for ond-om quinn-om\\
     for evil-\textsc{f.dat.pl} woman\textsc{(f)-dat.pl}\\
\glt ‘against evil women’
\z

This type of noun-phrase internal agreement may be instrumental for language users when they attempt to group individual words of a sentence together in phrases, but this would hold true mostly for languages where a fixed topology, including juxtaposition of constituents belonging to the same phrase, is not the general rule. For instance, in \REF{ex:hansen:22}, the noun phrases \textit{contiguās domōs} ‘neighbouring houses’ and \textit{altam urbem} ‘upper city’ are separated by other constituents (\citealt[2]{AndersenConf}, \citealt[89–93]{Nielsen2010}), and readers of that sentence must therefore rely heavily on case marking in order to group the words together correctly.

\ea\label{ex:hansen:22}
{Latin}\\
\gll contigu-ās tenuere dom-ōs, ubi dīcitur alt-am    coctil-ibus mūr-is cinxisse Semīram-is urb-em\\
     neigbouring\textsc{{}-f.acc.pl}  they.lived house\textsc{(f)-acc.pl}  where  is.said high\textsc{{}-f.acc.sg} bricked\textsc{{}-m.abl.pl} wall\textsc{(m)-abl.pl} to.have.surrounded Semiramis\textsc{(f?)-nom.sg} city\textsc{(f)-acc.sg}\\
\glt ‘they lived in neighbouring houses where Semiramis is said to have surrounded the upper city with brick walls’
\z

In languages with a fixed topology and with juxtaposition of constituents that belong together, this type of noun-phrase internal agreement becomes redundant. As \citet[93–107]{Diderichsen1941} has demonstrated, Middle Danish is such a language, at least when it comes to noun-phrase-internal topology. Juxtaposition is standard in the Middle Danish noun phrase, and the position of the modifiers (pre-head or post-head) follows from a set of fixed rules that I may sum up as follows.\footnote{For an elaborate presentation of these rules, see \citet[93–107]{Diderichsen1941} and \citet{Hansen2021}.}

\begin{enumerate}
\item Modifier in pre-head position: Determiners (quantitative adjectives, indefinite pronouns etc.) and numerals as well as characterising or emphatically used adjectives and possessive pronouns
\item Modifier in post-head position: Simple descriptive adjectives, adjectival appositions, participles equivalent of subordinate clauses, “superfluous”\footnote{This term and the examples given in this footnote stem from \citet[100, 206]{Diderichsen1941}. Superfluous possessive pronouns comprise cases such as \textit{faþær sin} ‘his/her father’, \textit{kuna sin} ‘his wife’, \textit{barn sit} ‘his/her wife’ where the possessor is self-evident from the context and can easily be left out as in \textit{at barn uar føt æftir faþur}, lit. ‘that child was born after father’ with omission of \textit{sin} ‘his’.} possessive pronouns, and partitive genitives
\end{enumerate}

Hence follows that the Middle Danish case-marking system is redundant in one of its two functions, viz., noun-phrase-internal agreement, leaving noun-phrase-external reference as its sole non-redundant function. In actual fact, noun-phrase-internal agreement is marked in three ways: (1) by juxtaposition and fixed rules of noun-phrase-internal topology, (2) by morphological case marking, and (3) by morphological marking of gender and number. One could reasonably argue, therefore, that Middle Danish displays double redundancy in noun-phrase-internal agreement. In light of this triple marking or double redundancy, it is hardly surprising that one of the ways of marking this agreement, viz., case marking, would become prone to loss, as evidenced by stages 2–5 in \sectref{hansen:2.6}.

Figures \ref{fig:hansen:1}--\ref{fig:hansen:6} express the loss that happened in terms of grammatical sign relations and changes in these relations, referring concretely to the prepositional phrase \textit{for ondom quinnom} ‘against evil women’ in \REF{ex:hansen:21}. \figref{fig:hansen:1}\footnote{All figures and tables: CC-BY 4.0 Bjarne Simmelkjær Sandgaard Hansen. In Figures \ref{fig:hansen:1}--\ref{fig:hansen:6}, the label “not coinc.” stands for “not coincidental”, referring to the noun-phrase-internal position of the head and the modifier not being coincidental, but governed by a set of fixed rules (\citealt{Diderichsen1941}: 93–107).} illustrates the traditional representation of the original situation with morphologically marked case on every member of the noun phrase.


\begin{figure}
	\caption{Symbolic sign relations morphologically expressed (case focus only)\label{fig:hansen:1}}
	\includegraphics[width=.75\textwidth]{figures/Hansen-fig1.PNG}
\end{figure}


As I have already shown, however, topology also plays a decisive role in the marking of noun-phrase-internal agreement, to which point I will add that the order of constituents within a prepositional phrase like \textit{for ondom quinnom} ‘against evil women’ is also fixed: preposition first, noun phrase second \citep[109]{Diderichsen1941}. In order to illustrate the grammatical relations in further detail, I will therefore need to add an additional topological layer to the presentation of \figref{fig:hansen:1}, for which see \figref{fig:hansen:2}, where thicker arrows represent topologically marked relations, and thinner arrows those that are morphologically marked.


\begin{figure}
	\caption{Symbolic sign relations morphologically and topologically expressed (case focus only)\label{fig:hansen:2}}
	\includegraphics[width=.75\textwidth]{figures/Hansen-fig2.PNG}
\end{figure}


So far, the model does not reveal much about the claimed redundancy in noun-phrase internal agreement. In order for that to be illustrated as well, I will need to extent the model even further. \figref{fig:hansen:3} therefore adds morphologically marked indexical relations, including such that represent noun-phrase-internal agreement, and \figref{fig:hansen:4} is even further augmented with those indexical relations that are topologically marked. Full lines in blue colour represent symbolic relations (as in Figures \ref{fig:hansen:1}--\ref{fig:hansen:2}), whereas dotted lines in red represent indexical relations. As illustrated especially in \figref{fig:hansen:4}, both the noun-phrase-internal agreement and the noun-phrase external relations to the preposition are doubly marked, viz., both morphologically and topologically. One of these layers, the morphological one, is therefore dispensable and subject to gradual phase-out by the language users.


\figref{fig:hansen:5} illustrates the grammatical relations in my model with the morphological level (i.e., marking by means of case) phased out. It has now become evident that the topological level alone is fully capable of marking both the noun-phrase-internal agreement and the noun-phrase-external relations to the preposition. In other types of situations, e.g., when a noun phrase originally marked for dative did not form part of a prepositional phrase with a dative-governing preposition but functioned as an indirect object, the topological level would be capable of marking noun-phrase-internal agreement only, seeing that the Middle Danish topology is not fixed on the level of sentential constituents.


\begin{figure}
	\caption{Symbolic and indexical sign relations morphologically expressed (case focus only)\label{fig:hansen:3}}
	\includegraphics[width=.75\textwidth]{figures/Hansen-fig3.png}
\end{figure}


\begin{figure}
	\caption{Symbolic and indexical sign relations morphologically and topologically expressed (case focus only)\label{fig:hansen:4}}
	\includegraphics[width=.75\textwidth]{figures/Hansen-fig4.png}
\end{figure}

\begin{figure}
	\caption{Symbolic and indexical sign relations morphologically and topologically expressed after the removal of case inflection (case focus only)\label{fig:hansen:5}}
	\includegraphics[width=.75\textwidth]{figures/Hansen-fig5.png}
\end{figure}


Despite what one may have deduced and thus been led to believe from the model in Figures \ref{fig:hansen:1}--\ref{fig:hansen:5} so far, one must not forget that, even after the loss of case marking, noun-phrase-internal agreement is still marked in two ways, viz., by juxtaposition and fixed rules of noun-phrase-internal topology and by morphological marking of gender and number. Redundancy thus remains, but only singly, not doubly as in the original system prior to the loss of case marking. \figref{fig:hansen:6} serves to illustrate that, besides expressing the number and gender of the elements of a noun phrase symbolically, number and gender marking also express indexical relations within the noun phrase, i.e., noun-phrase-internal agreement.

\begin{figure}
	\caption{Symbolic and indexical sign relations morphologically and topologically expressed (gender and number focus)\label{fig:hansen:6}}
	\includegraphics[width=.75\textwidth]{figures/Hansen-fig6.png}
\end{figure}


\subsection{Gradual abandonment of case marking} \label{hansen:3.5}

In \sectref{hansen:3.4}, I presented one of the factors that motivated the Middle Danish case-system changes, viz., redundancy, but it remains unexplained so far why the language users simply did not remove such redundancy at once, but did it gradually instead. In what follows, I will therefore outline why case marking remained longer in the feminine singular and in the plural than in the masculine and neuter singular, and longer, too, on typical modifiers than on nouns.

\citegen[27--37]{Andersen2001} principle of markedness agreement may account for some of this asymmetry. According to this principle, elements that are marked similarly behave identically. Marked forms behave in the same way as other marked forms, and unmarked forms in the same way as other unmarked forms. When it comes to linguistic change, \citet[36]{Andersen2001} claims that one would expect

\begin{quote}\sloppy
[…] the innovation to occur earliest in environments with equivalent markedness value and to subsequently gain ascendancy first in such contexts and then, as it loses its novelty, in the complementary contexts with opposite markedness value.
\end{quote}

This corresponds well to the Middle Danish situation, where the case-system innovations occur first in the unmarked environment, i.e., in nouns \citep[44]{Andersen1980} and in the masculine and neuter singular. The marked environments, i.e., the non-substantival nominal parts of speech, the feminine singular and the plural, keep case marking longer.

One additional factor that may account for the marking of case on only one (type of) noun-phrase member, viz., the typical modifiers, at the intermediate steps of the Middle Danish development (represented by stages 3–4 in \sectref{hansen:2.6}) may be the application of a principle of single encoding (\citealt[258--261]{Norde2001}). This principle, which entails that noun phrases, for instance, only inflect for case once and not on every single (type of) noun-phrase member, also operates in the well-known way of marking case in German noun phrases. Here, only one noun-phrase member distinguishes fully for case, while the remaining case markers remain underspecified. In \REF{ex:hansen:23}, for instance, case is expressed explicitly only on the adjective, leaving the indefinite article underspecified, whereas in \REF{ex:hansen:24}, case is expressed explicitly only on the definite article, leaving the adjective underspecified.

\ea \label{ex:hansen:23}
{German}\\
\gll ein-∅ gut-\textit{er} Mann\\
     a-\textsc{m.nom.sg} good-\textsc{m.nom.sg} man\textsc{(m).sg}\\
\glt ‘a good man’

\ex\label{ex:hansen:24}
{German}\\
\gll de-\textit{r} gut-e Mann\\
     the-\textsc{m.nom.sg} good-\textsc{m.nom.sg} man\textsc{(m).sg}\\
\glt ‘the good man’

\z

\begin{sloppypar}
Returning to the application of this principle in the Middle Danish noun phrase, one may attribute the preference for inflection on typical modifiers over inflection on nouns to the circumstance that the adjectival and pronominal paradigms historically contain more and clearer distinctions than the nominal paradigm.
\end{sloppypar}

\section{Paradigmatisation} \label{hansen:4}\largerpage
\subsection{Theoretical viewpoint} \label{hansen:4.1}

Now that I have described and explained some of the possible reasons for the Middle Danish case-system changes, it is not only interesting, but also necessary to witness their paradigmatic consequences. As mentioned briefly in \sectref{hansen:1}, I follow the theoretical viewpoint of \citet[xi, 71–72]{Nørgård-Sørensen2011} and \citet[261–262]{Nørgård-Sørensen2015} that grammaticalisation equals paradigmatisation, meaning that one cannot have grammar and changes in grammar without also having paradigms and changes in paradigms; see also \citet[2--4]{DiewaldSmirnova2010}. Consequently, in accordance with this theoretical viewpoint, the changes of the Middle Danish case system described in \sectref{hansen:2} and explained in Sections~\ref{hansen:3.3} and~\ref{hansen:3.4} cannot be grammatical changes unless they can be formalised paradigmatically.

In \citet[5--6]{Nørgård-Sørensen2011} and \citet[262--263]{Nørgård-Sørensen2015} understanding of what constitutes a grammatical paradigm, any grammatical paradigm must meet five criteria. First, they state, it must be closed, in principle, thus containing only a fixed number of members. Second, it must be possible to specify the domain of the paradigm, i.e., the syntagmatic context to which the paradigm applies. Third, in close correspondence with the domain introduced by the second criterion, any paradigm must have a semantic frame within which the content of the specific members of the paradigm is defined. In other words, the semantic frame reveals what type of oppositions the members of a paradigm serve to express. Fourth, the choice between the members of the paradigm is obligatory, meaning that, when producing an utterance that activates the domain of a grammatical paradigm, the language users cannot avoid choosing one of its members. Fifth and finally, grammatical paradigms tend to be asymmetric and thus to distinguish automatically between marked and unmarked members, the latter being the one without a specific semantic load. For that reason, the unmarked paradigm member may sometimes participate in the functions of the other, i.e., the marked, members; see also \sectref{hansen:2.3} for a further discussion of and references to the concept of participation.

\subsection{Paradigmatic consequences of the Middle Danish case system changes} \label{hansen:4.2}\largerpage

The question to be answered now is how the notion of a paradigm outlined in \sectref{hansen:4.1} fits the data presented in \sectref{hansen:2} and analysed in \sectref{hansen:3.3} and \sectref{hansen:3.4}. In order to answer this question, I will first attempt to set up a paradigm of the original Middle Danish case system with historically expected case on both nouns and typical modifiers (represented by stage 1 in \sectref{hansen:2.6}) in \tabref{tab:hansen:1}, the focus of which lies on the paradigmatic opposition between the accusative and the dative. Similar and more extensive tables may be set up for the inclusion of the nominative and the genitive, but for the sake of clarity, the accusative-dative opposition will suffice.

This paradigm contains a syntagmatic domain (Middle Danish noun phrases consisting of a modifier and a noun), a semantic frame (“case”, i.e., indexes of noun-phrase external government as well as noun-phrase-internal agreement) and a closed set of case endings as its members.\footnote{Please note that the expressional label \textsc{acc} covers a wide array of endings from different inflectional classes such as \textit{{}-an} (adjectival \textsc{m.acc.sg}), \textit{{}-a} (masculine \textit{n}{}-stem noun \textsc{acc.sg}) and \textit{{}-∅} (vowel-stem noun \textsc{acc.sg}), while \textsc{dat} covers endings such as \textit{{}-u} (feminine \textit{n}{}-stem noun \textsc{dat.sg} or adjectival \textsc{n.dat.sg}), \textit{{}-i} (adjectival \textsc{f.dat.sg} or masculine \textit{a}{}-stem noun \textsc{dat.sg}) and \textit{{}-um}/\textit{{}-om} (\textsc{dat.pl}).} In addition, the language users cannot avoid choosing between an accusative and a dative ending in utterances relevant to this opposition.

\tabref{tab:hansen:1} reveals that the accusative consistently marks historical accusative contexts, and dative the historical dative contexts. Consequently, no overlapping occurs between the accusative and the dative in this system.

\begin{table}
\caption{Paradigmatic visualisation of the system of historically expected case marking both on nouns and on typical modifiers\label{tab:hansen:1}}
\begin{tabularx}{\textwidth}{lQQ}
\lsptoprule
{Domain} & \multicolumn{2}{l}{Noun phrases (examples here: modifier + noun)}\\\midrule
Frame & \multicolumn{2}{p{10cm}}{“Case”, i.e., indexes of government (noun-phrase-externally) + agreement (noun-phrase-internally)}\\
Content & Indexes of historical accusative contexts (direct-object/object-complement government, prepositional government, etc.) & Indexes of historical dative contexts (indirect-object government, prepositional government, etc.)\\
Expression & Modifier\textsc{{}-acc} + noun-\textsc{acc}\newline
Ex.: \textit{(vith) swa ohørlig-a synd-∅} ‘against a sin this unheard-of’ & Modifier\textsc{{}-dat} + noun-\textsc{dat}\newline
Ex.: \textit{(at) andr-u thing-i} ‘at the next/second moot’\\
\lspbottomrule
\end{tabularx}
\end{table}

As I mentioned in \sectref{hansen:3.4}, the Middle Danish noun-phrase-internal word order was fixed even at the earliest attested stage of Middle Danish when the application of this historical case system was most widespread. In order to illustrate the grammatical potential of this topological system, I have entered it into a paradigm, as well, as represented by \tabref{tab:hansen:2}.


\begin{table}
\caption{Paradigmatic visualisation of the Middle Danish predictability of noun-phrase-internal topology}
\label{tab:hansen:2}
\begin{tabularx}{\textwidth}{lQQ}
\lsptoprule
Domain & \multicolumn{2}{l}{Noun phrases (examples here: modifier\,+\,noun)}\\\midrule
Frame & \multicolumn{2}{p{10cm}}{Type and function of modifier\,+\,agreement/mutual connection (noun-phrase-internally)} \\
Content & Determiners (quantitative adjectives, indefinite pronouns etc.) and numerals + characterising or emphatically used adjectives and possessive pronouns & Simple descriptive adjectives, adjectival appositions, participles equivalent of subordinate clauses, “superfluous” possessive pronouns and partitive genitives\\
Expression & Position X = Modifier \newline Position Y = Head \newline Ex.: \textit{andru thingi} ‘the next/second moot’ & Position X = Head \newline Position Y = Modifier\newline Ex.: \textit{børnum sinum} ‘his children’ \\
\lspbottomrule
\end{tabularx}
\end{table}

So far, both tables have represented clear-cut grammatical paradigms with no vacillation between the members. However, the paradigmatic representation attempted in \tabref{tab:hansen:1} of each stage in a separate paradigm does not depict the actual situation, since all the case-system stages of \sectref{hansen:2.6} actually do occur within one and the same text; cf. again, e.g., \REF{ex:hansen:1}, \REF{ex:hansen:6}, \REF{ex:hansen:8} and \REF{ex:hansen:10}, which all stem from SjT and may reveal up to four competing systems. Focusing on the many different ways to express indexes of historical dative contexts, one may therefore be tempted to produce a paradigm like that of \tabref{tab:hansen:3} to check if it would constitute a more precise rendition of the actual situation.


\begin{table}
\caption{Attempt at a paradigmatic visualisation of four competing case-system stages in noun phrases based on the content of indexing historical dative contexts\label{tab:hansen:3}}
\begin{tabularx}{\textwidth}{lQQQQ}
\lsptoprule
{Domain} & \multicolumn{4}{l}{Noun phrases (examples here: modifier + noun)}\\\midrule
Frame & \multicolumn{4}{p{10cm}}{“Case”, i.e., indexes of government (noun-phrase-externally) + agreement (noun-phrase-internally)}\\
Content & \multicolumn{4}{p{10cm}}{Indexes of historical dative contexts (indirect-object government, prepositional government, etc.)}\\
Expression & Modifier\textsc{{}-dat} + noun-\textsc{dat}\newline Ex.: \textit{sin-um thiænar-um} ‘their servants’ & Modifier\textsc{{}-dat} + noun\newline Ex.: \textit{(mæth) en-um stor-um hær} ‘with a great army’ & Modifier\textsc{{}-acc} + noun\newline Ex.: \textit{(af) hørdh-a-na} ‘from the herdsmen’ & Modifier + noun\newline Ex.: \textit{(mæth) en stor hær} ‘with a great army’\\
\lspbottomrule
\end{tabularx}
\end{table}


This representation creates an entirely novel issue, viz., the introduction of free choice within the paradigm. Admittedly, the existence of a free choice need not necessarily violate \citet[5--6]{Nørgård-Sørensen2011} and \citegen[262--263]{Nørgård-Sørensen2015} fourth criterion that the choice between the members of a paradigm is obligatory, for, as they further state (\citet[5]{Nørgård-Sørensen2011}, “[t]his choice may be free or bound, but will ultimately be determined by the content of the forms constituting the paradigm.”\largerpage

In this statement lies the real problem with the paradigm of \tabref{tab:hansen:3}. Even though it seems to contain a domain, a frame, a closed set of members and an obligatory, yet free choice, it does not entail an opposition of content, unless one may assume that the content opposition is one of indexing variation within the language users’ personal register, i.e., within the language users’ range of varieties between which they may choose at different times \citep[77]{Halliday1994}.\footnote{I am greatly indebted to Henning Andersen for pointing this possibility out to me during the discussion round at the SLE workshop “Paradigms regained” where I first gave the presentation upon which this article is based.} If such a register-focused content opposition is not present, this paradigm is invalid, for a grammatical paradigm must oppose grammatical signs, i.e., linguistic units consisting of both a content side and an expression side. In other words, the absence of a content opposition equals the absence of a grammatical opposition, which, in turn, equals the absence of a paradigmatic opposition. In that sense, this may rather be a case of allomorphy in its broadest sense than a case of paradigmatic\largerpage{} opposition.\footnote{Whether one may really label an apparently free choice between ways of expressing indexes of historical dative contexts a case of allomorphy depends on one's definition of this term. As \citet[17, 113–114]{Bauer2003} points out, the prototypical allomorph is a phonologically, grammatically or lexically conditioned variant of the same morpheme. Consequently, one could regard the different realisations of the historical Middle Danish dative singular ending in nouns as grammatically (gender) or lexically (inflectional class) conditioned allomorphs of a morpheme that expresses indexes of historical dative contexts; see fn. 10 for examples of these different realisations. Since the choice presented in \tabref{tab:hansen:3} is free rather than phonologically, grammatically or lexically conditioned, I should be able to rule out allomorphy here, at least in its prototypical sense. \citet[113--114]{Bauer2003} adds, however, that in a broader context, allomorphs may be conditioned by the choice of register as in the choice between the English plural forms \textit{tempos} and \textit{tempi}. If one accepts this expansion of the definition of allomorphy, I would be able to regard it as a case of allomorphy.}

Notwithstanding any considerations on allomorphic variation, a more effective way of presenting the paradigm in question would be that of \tabref{tab:hansen:4}, which reintroduces the content opposition between expressing indexes of historical accusative and historical dative contexts. Here, it becomes evident that what seems to be (and is indeed) free variation between different ways of expressing indexes of historical dative contexts also represents an instance of difference in markedness relations, i.e., the fifth criterion of \citet[6]{Nørgård-Sørensen2011} and \citet[263]{Nørgård-Sørensen2015}. The option for using historical accusative forms or uninflected forms in historical dative contexts simply follows from the general unmarkedness of these forms; see again \sectref{hansen:2.3} on accusative participation. In contrast to that, the historical dative forms remain marked and are applicable only in their original contexts. What \tabref{tab:hansen:4} does not reveal, however, is the relative markedness of the accusative forms and the uninflected forms, since the rendition of such a distinction would require the inclusion of a larger set of data and forms (the nominative and the genitive), which lies outside the scope of the present article.


\begin{table}
\caption{Paradigmatic visualisation of four competing case-system stages in noun phrases with the inclusion of markedness differences. For the sake of visual clarity, the focus of this table lies specifically on the marking of case on typical noun-phrase modifiers.\label{tab:hansen:4}}
\begin{tabularx}{\textwidth}{lQQQ}
\lsptoprule
{Domain} & \multicolumn{3}{l}{Noun phrases (examples here: modifier + noun)}\\\midrule
Frame & \multicolumn{3}{p{10cm}}{“Case”, i.e., indexes of government (noun-phrase-ex\-ter\-nally)\,+\,agreement (noun-phrase-internally)}\\
Content &  {Indexes of historical accusative contexts (direct-object/object-complement gov\-ern\-ment, prepositional government, etc.)} & \multicolumn{2}{p{5cm}}{Indexes of historical dative contexts (indirect-object gov\-ern\-ment, prepositional gov\-ern\-ment, etc.)}\\
Expression & \multicolumn{2}{p{5cm}}{Modifier\textsc{{}-acc} + noun\newline Exx.: \textit{(vith) swa ohørlig-a synd-∅} ‘against a sin this unheard-of’, \textit{(af) hørdh-a-na} ‘from the herdsmen’} & Modifier\textsc{{}-dat} + noun(-\textsc{dat)}\newline Exx.: \textit{sin-um thiænar-um} ‘their servants’, \textit{(mæth) en-um stor-um hær} ‘with a great army’\\
& \multicolumn{3}{p{5cm}}{Modifier + noun\newline Ex.: \textit{(mæth) en stor hær} ‘with a great army’}\\
\lspbottomrule
\end{tabularx}
\end{table}

\section{Conclusion} \label{hansen:5}

In this article, I have analysed linguistic data from mainly three Middle Danish texts with respect to case use and presented a development of the Middle Danish case system divided into possibly five stages:

\begin{enumerate}
\item  Starting point: Historically expected case on both nouns and typical modifiers
\item  Historically expected case on typical modifiers and optional accusative participation on nouns
\item Historically expected case on typical modifiers and no case marking on nouns (excluding the genitival clitic \textit{{}-s})
\item Optional accusative participation on typical modifiers and no case marking on nouns (excluding the genitival clitic \textit{{}-s})
\item End point: Neither case marking on nouns nor on typical modifiers (excluding the genitival clitic \textit{{}-s})
\end{enumerate}

I have demonstrated that the Middle Danish case-system changes result neither from reductionist sound laws, nor from linguistic simplification due to language contact. Rather, various processes of grammaticalisation, i.e., processes of change in the function and contents of the grammatical signs and in the paradigmatic oppositions between them, are responsible for the changes. One of these processes is the change from double to single redundancy in noun-phrase-internal agreement. After the fixation of noun-phrase-internal topology had rendered the use of case for expressing noun-phrase-internal agreement superfluous, this type of indexical reference was phased out gradually in general accordance with both \citegen[27--37]{Andersen2001} principle of markedness agreement and \citegen[258--261]{Norde2001} principle of single encoding.

Based on \citet[5--6]{Nørgård-Sørensen2011} and \citet[262--263]{Nørgård-Sørensen2015} five criteria for what constitutes a grammatical paradigm, I have also shown that the Middle Danish case system may be described paradigmatically and, correspondingly, that the changes it undergoes constitute an instance of paradigmatic change. The existence of an intermediate transitional period with competition and seemingly free variation between different Middle Danish case-system stages does not challenge this claim, since these stages do not only represent free variation, but also an instance of difference in markedness relations, i.e., the fifth criterion of \citet[6]{Nørgård-Sørensen2011} and \citet[263]{Nørgård-Sørensen2015}.

After subjecting the Middle Danish case-system changes to this paradigmatic test, I dare now claim that the changes in the Middle Danish case system are indeed grammatical changes – and thus represent a process of both grammaticalisation and paradigmatisation – in \citet[xi, 71–72]{Nørgård-Sørensen2011} and \citegen[261--262]{Nørgård-Sørensen2015} sense.

{\sloppy\printbibliography[heading=subbibliography,notkeyword=this]}
\end{document} 
