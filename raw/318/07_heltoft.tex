\documentclass[output=paper]{langscibook} 
\ChapterDOI{10.5281/zenodo.5675851}
\author{Lars Heltoft\affiliation{University of Copenhagen}}
\title[Semantic reorganisation of case paradigms]{The semantic reorganisation of case paradigms and word order paradigms in the history of Danish}
\abstract{This article is a study of the relation between the paradigmatic organisation of case and the paradigmatic organisation of word order in late Middle Danish (1300--1500) and in Modern Danish. A content analysis of these paradigms shows a typological difference, even if the older pronominal case system looks exactly like the modern system. Middle Danish preserves inactive (impersonal, traditionally) constructions with an inactive argument 1, and at this stage, the dimensions of case, position, argument hierarchy and subjecthood (still) combine freely. The case system is still indexical. The alignment of subjecthood, status as argument 1, position and nominative case with symbolic meaning is a development of post-Reformation Danish.}


\begin{document}
\maketitle 
\largerpage[-2]

\section{Introduction} \label{heltoft:1}

It is no surprise that a morphological category like case should be organised in paradigms, since this usage of ``paradigm'' has been current since antiquity. It may come as more of a surprise that a syntagmatic aspect of language like word order can have paradigmatic organisation; yet, this refers to contrasts of meaning between word order patterns and is therefore not a new idea either, but rather a neglected aspect of language. What I will try to add, is the possibility of co-organisation between morphological content and alternating constructional organisation. Case meaning can interact with constructional hierarchic organisation of arguments, forming what could be called second-order paradigms (or hyperparadigms), see \citet{Christensen2007}, \citet{Nørgård-Sørensen2011}, and \citet{Nielsen2016}.

After Sections \ref{heltoft:1.1} and \ref{heltoft:1.2}, I present in \sectref{heltoft:2} the system of Middle Danish inactive constructions in some detail, including  important differences from the parallel West Norse (Old Icelandic) system. \sectref{heltoft:3} deals with the paradigmatic organisation of case in its interplay with transitive and inactive. The function of case is to point to the governing verb and thus to determine the semantic role value of the arguments. \sectref{heltoft:4} documents that topology (word order) is not included in these constructional paradigms, since there are no positions reserved for subjects and objects. This is very different from the system of present-day Danish, with specific positions for subjects and objects (see \sectref{heltoft:5}) and a case system that has lost its coding of semantic roles.

Some readers may find it easier to skip initially Sections \ref{heltoft:2.2}--\ref{heltoft:2.5} and go directly to the overview in \sectref{heltoft:2.6} and then on to \sectref{heltoft:3}, to return later to the details of the data in \sectref{heltoft:2}.

\subsection{Inactive constructions} \label{heltoft:1.1}\largerpage[2]

A central problem in the analysis of case morphology in constructional contexts lies in its function in so-called impersonal constructions, in my terminology inactive\footnote{\textit{Inactive construction} bears resemblance to the contrast between active features and inactive features in so-called active languages, cf. \citealt{Lehmann1989, Lehmann1993}. When the \textit{active voice} is meant, I will be explicit about this and simply write ``active voice''.} constructions, as in (\ref{ex:heltoft:1}--\ref{ex:heltoft:2}).\footnote{Line two is rendered in the normalised orthography of the collections of the Old Danish Dictionary, see \sectref{heltoft:sources}.}

\ea \label{ex:heltoft:1} 
    \glll om thek wærkær i howæth oc i thinninge, Tha tac thæn \\
         um   thik\textsubscript{A1} værk-er     i hoveth ok i thinninge,     tha tak thæn \\
         if    2\textsc{sg.obl}   ache-\textsc{prs.sg}   in head and in temples,     then take the \\
    \glt `If your head and temples are aching, then take the …' (AM 187, 3, 3--4)
\ex \label{ex:heltoft:2} 
    \glll Hwy   angher tik ey nw, at thu haffwer illde giorth  \\
         hvi   angrer   thik\textsubscript{A1}     æj    nu     [at thu haver ille gjorth]\textsubscript{A2}\\
         why  repent  \textsc{2sg.obl}  not   now   that you have wrong done \\
    \glt `Why do you not repent now that you have done wrong?' (ML 57,\,16--58,\,1)
\z

Example \REF{ex:heltoft:1} has a one-place predicate \textit{thik} `thee', in the oblique case, manifesting the semantic role Inactive. Since this role is basic to this constructional paradigm, I name the Inactive argument A1, the primary argument. 

Argument hierarchy reflects basic semantic choices. The Inactive role applied to the A1 of this construction excludes agentive meaning, as also emphasised by \citet{Faarlund2001, Faarlund2004} for Old Icelandic. Agentive meaning is characteristic of transitivity, and neither \REF{ex:heltoft:1} nor \REF{ex:heltoft:2} can convey agentive meaning. 

Examples (\ref{ex:heltoft:1}--\ref{ex:heltoft:2}) show a difference in valence within inactive constructions. Example \REF{ex:heltoft:2} is a two-place predicate; the A1 is again \textit{thik}, likewise in the oblique case, manifesting a variant of the inactive role, what other case grammarians have called the Experiencer role. Here, the A2 is an embedded clause: \textit{at thu haver ille gjorth}, denoting the content of the mental impact on the referent of the A1. \citet{Halliday1994} speaks of this semantic role as the Phenomenon role. The term A2 reflects an extension of the construction, possible with certain verbs.

The paradigmatic organisation of inactive constructions will be laid out in \sectref{heltoft:3}, esp. \sectref{heltoft:3.3}. I will not go into details here, since a clear exposition will call for a comparison with especially transitivity and other organisation principles. But it must be pointed out now that  argument hierarchy reflects the semantic valence of predicates, and this differs from the level of sentence members and thus from the grammatical functions subject and direct object. In principle, the subject function can apply to either A1s or A2s, and similarly, direct objects to either A1s or A2s. Again, this means that one should not just assume that one of these levels can reduce to one of the other levels, for instance of subjects to A1s. On the contrary, a claim that these levels are or have been aligned must be the outcome of the application of empirical criteria and cannot be taken for granted a priori.\largerpage

The A1s of \REF{ex:heltoft:1} and \REF{ex:heltoft:2} have the oblique form, and this is of course a case of what other traditions call oblique subjects (among many others \citealt{Allen1995, BarðdalEyþórssson2003, BarðdalEyþórsson2018, EyþórssonBarðdal2005, Kiparsky1997}). The analysis of oblique case subjecthood has been advocated for many older Indo-European languages by esp. Jóhanna Barðdal and Þórhallur Eyþórsson. I shall not in this context discuss their views in detail, nor will I refer to the sometimes-polemic discussions between different positions. Barðdal and Eyþórsson have a specific definition of subject as a starting point, namely the identification of subject and A1. As they see  it, the A1 is the subject, or rather, the universal definition of a subject is taken to be the status as an argument 1.

\begin{quote}
    One potential subject definition that we have used as a working definition since \citet{EyþórssonBarðdal2005}, is to view the first argument of the argument structure as being the syntactic subject.\footnote{The following quotation will illustrate their view: ``The reason that we have proposed such a subject definition is that when generalizing across the subject tests, we have found that it is always the first argument of the argument structure that is targeted by the subject tests. In that sense, our approach is bottom-up; we have arrived at a subject definition on the basis of the subject tests, a definition which can then be applied independently of the individual tests.” \citep[263--264]{BarðdalEyþórsson2018}}  \citep[263]{BarðdalEyþórsson2018}
\end{quote}

A discussion of their subject criteria and of similar approaches (e.g. \cite{Sigurðsson1989}) and the way they are operationalised\footnote{The only separate treatment of Danish known to me is a brief article by \citet{Hrafnbjargarson2003}, using the criteria of \citet{Sigurðsson1989}.} must be the topic of another article \citep{Heltoft2021b}.

In the hierarchical-linear configurations of generative grammar, such reductional notions of a subject will always come out as a Specifier of something, normally generated as a Vp-specifier, next upgraded to I-Spec, or for V2-languages, all the way to C-Spec. This presupposition, that the subject holds the upmost position, inherently linear and hierarchical at the same time, is shared also by linguists (esp. \citealt{Kiparsky1997}) who try to combine and reconcile syntax and morphological case by ascribing syntactic features to the arguments and case features to morphology, and thereafter, working out unification procedures for the respective feature clusters. Kiparsky refers to Cynthia Allen for the insight that Old English had IP available as a category since ``it had dative subjects, in the sense that oblique experiencers were structurally parallel with nominative subjects”, interpreting this as ``at least a \textit{prima facie} indication of Spec-IP positioning” \citep[12]{Kiparsky1997}. Behind this, we also find the identification of subject and A1, meaning that A1 is the hierarchically upmost argument. Instead of this assumption, as mentioned in \sectref{heltoft:1.1}, I hold that argument hierarchy should be seen as organised by valence; see further \sectref{heltoft:2}. 

Of course, the sign-oriented approach adopted here determines part of what is possible. Sign-oriented grammars such as \citet{Croft2001, TraugottTrousdale2013} or Danish Functional Grammar \citep{Engberg-PedersenEtAl1996} must respect the sign limits delineated by the expression side and must therefore seek for models that do not presuppose reduction attempts of Inactive A1s to underlying subjects across the sign boundaries.\largerpage

I will return to the importance of linearity in \sectref{heltoft:4}, but the road to there will go via an analysis of the structure and function of simple traditional case paradigms like the ones behind examples \REF{ex:heltoft:1} and \REF{ex:heltoft:2}, consisting of oppositions of number, deixis and a case distinction of just two: nominative and oblique.\footnote{The genitive is only used in possessive constructions and is therefore not part of this paradigm.} 

\begin{table}
\caption{Pronominal case in 14\textsuperscript{th} century western Middle Danish\label{tab:heltoft:1}}
\begin{tabularx}{\textwidth}{lXXXXXXX} 
\lsptoprule
& \textsc{1p.sg} & \textsc{2p.sg} & \multicolumn{2}{c}{\textsc{3p.sg}} & \textsc{1p.pl} & \textsc{2p.pl} & \textsc{3p.pl}\\\midrule
\textsc{nom}. & \textit{jæk/jak} ‘I' & \textit{thu} ‘thou' & \textit{han} ‘he' & \textit{hun} ‘she' & \textit{vi} ‘we' & \textit{i} ‘you' & \textit{the} ‘they'\\
\textsc{obl.} & \textit{mik} ‘me' & \textit{thik} ‘thee' & \textit{hanum} ‘him' & \textit{hænne} ‘her' & \textit{os} ‘us' & \textit{ither} ‘you' & \textit{thæm} ‘them'\\
\lspbottomrule
\end{tabularx}
\end{table}

Apart from orthography and sound change, and a few later shortenings (of \textit{hanum} to \textit{ham}, \textit{ither} to \textit{jer}), the expression system of \tabref{tab:heltoft:1} is exactly the same as that of the modern language. The \textsc{3p.sg/pl} forms are attested in the Jutish Law (of 1241, oldest manuscript from 1284), and this has led to the traditional assumption that Middle Western Danish\footnote{The written tradition of Middle Danish falls in two main dialects, Western Middle Danish in Jutland and the central islands and the more archaic Eastern Middle Danish (Scanian) in the provinces east of the Sound, in present-day Sweden.} had already introduced roughly the modern pronominal case system (e.g. \citealt[129]{Karker1991}, \citealt[198]{Karker1993}). As we shall see, however, when properly analysed at the level of content, the Western Middle Danish two-case pronominal system turns out to be typologically different from the modern system. Some generative grammarians \citep{Sigurðsson2006, Sigurðsson2012, Sigurðsson2012b, Parrott2012} use the term ‘case impoverished' for such modern Germanic languages that have reduced their case inventory to pronouns and there to a minimum of two cases, and insofar as they speak of case as an expression system, this term might apply to Middle Danish as well. However, what matters is not quantity, but the quality of the content organised in such minimal case paradigms. I will claim that a content analysis of the Middle Danish case paradigm will show that it is clearly typologically different from the modern Danish case paradigm, and secondly, that this analysis demands a thorough analysis of the way Middle Danish case paradigms are integrated in more complex paradigms interlocking morphology and constructional alternations. 

To conduct this analysis, we must take the semiotic function and content of even a reduced case system seriously. We cannot simply assume that case has no meaning potential and relegate it to a status as part of the expression system, or, in the generative terminology, to phonological form.\footnote{Not all generative grammarians buy the reduction of morphology to PF, of course. Among them especially \citet{Kiparsky1997}, but also \citet{Sigurðsson2006, Sigurðsson2012, Sigurðsson2012b} realises this is a weak point.}  Nor can we assume that its content is simply the positions defined by an abstract, a priori given syntactic configuration.  One part of the exercise will consist in determining the content system of the Middle Danish case paradigm, and contrary to most other present-day approaches, I will not accept any a priori distinction between syntactic and lexical case. Given convincing arguments, the discussion is open to the possible conclusion that even a reduced system like the one under scrutiny can manifest a semantic role system, and just that. This is the topic to be investigated in \sectref{heltoft:2}.

\subsection{The word order systems: Why include them?} \label{heltoft:1.2}

The word order systems of  late Middle Danish and Modern Danish will be investigated and compared, too, as a way to determine whether the oblique A1s share properties with nominative subjects. The model to be used is the so-called \textit{sentence frame model}, a descriptive model with Scandinavian and German roots (\citealt{Diderichsen1946}, \citealt{Faarlund1989, Faarlund1990, Faarlund1997}; \citealt{Heltoft1992}). This positional model does not intertwine syntactic hierarchy and linearity. It presupposes a nonlinear dependency model for syntax but consists of concatenated positions in itself. Some are characteristic or even definitional \citep{Melčuk2014} of their syntactic category, others are open positions for a set of syntactic categories. Such open positions can express a separate content system independently of the categories that may fill them. One relevant example for the present agenda is illocutionary force, or better: \textit{illocutionary frame}, the speech act potential coded in word order; another example is background-focus structure. 

 In Modern Danish, subjects – in the nominative form, if possible – are confined to a limited number of  positions, namely two: the open initial position, the so-called fundamental field of Danish topological tradition (the P1 of Simon C. Dik, see \citealt[408--416]{Dik1997}), and the third position immediately after the V2 position.

\ea \label{ex:heltoft:3} 
\ea \label{ex:heltoft:3a} 
    \gll \textit{Han}.(1pos.)    beundrer.(2pos.)   (3pos.empty)  hende.(post-subject-pos.)    \\
                 he-\textsc{nom} admires            ${\emptyset}$    her-\textsc{obl}\\
    \glt ‘He admires her'

\ex \label{ex:heltoft:3b}
     \gll Hende.(1pos.)     beundrer.(2pos.)    \textit{han}.(3pos.)    \\
          her-\textsc{obl}  admires              he-\textsc{nom}\\
    \glt ‘Her he admires'

 \ex \label{ex:heltoft:3c}
     \gll (1pos.empty)    Beundrer.(2pos.)  \textit{han}.(3pos.)    hende?.(post-subject-pos.)    \\
          ${\emptyset}$        admires              he-\textsc{nom}  her-\textsc{obl}\\
    \glt  ‘Does he admire her?'
\z\z

The Modern Danish system is clearly an XVSO-system, and in traditional terms, the contrast (\ref{ex:heltoft:3a}--\ref{ex:heltoft:3b}) vs. \REF{ex:heltoft:3c} codes declarative function vs. interrogative function. The basic structure of the paradigm can be laid out in terms from Peircean semiotics, namely symbolic and indexical meaning. The main expression contrast is between a filled-in position 1 (see \ref{ex:heltoft:3a}--\ref{ex:heltoft:3b}) and its zero opponent \REF{ex:heltoft:3c}. The symbolic contrast is between constative (pos. 1 filled-in) and interrogative meaning (pos.1 zero), and there is an indexical function to notice as well, namely the position 2 filled by the finite verb.

So in \tabref{tab:heltoft:2}, position 1 is the locus of the contrast zero vs. X, position 2 the indexical identification of this locus. Position 1 holds the symbolic, illocutionary frame contrast of the paradigm and is thus the locus of the frame of the paradigm; position 2 indicates the locus for this frame and defines the domain of the paradigm.\footnote{For the terminology of this paragraph, see \citet{Nørgård-Sørensen2011, Nørgård-Sørensen2015, Heltoft2019}.} 
  
\begin{table} 
\caption{The indexical function of position 2 in Modern Danish word order\label{tab:heltoft:2}}
\begin{tabular}{lclll}
\lsptoprule
    1.pos. & & 2.pos. & 3.pos.& \\\midrule
    X  & ${\Leftarrow}$   &   V & & \\
    Zero & ${\Leftarrow}$  &   V & &  \\
    hende & ${\Leftarrow}$ &   beundrer  &  han \\
    Zero & ${\Leftarrow}$  &   beundrer  &  han & hende?\\
\lspbottomrule
\end{tabular}
\end{table}


Notice that the subject's unique position is position 3, and that this position must be filled in to form the yes-no question \REF{ex:heltoft:3c}. And since the subject in active clauses can in the modern language readily be identified with the argument 1 (A1), the case system and the positional system are clearly related.

There is every reason to ask whether the medieval language had a characteristic, let alone definitional subject position in the way the modern language has it, that is, whether position plays a role for the identification of subjects and objects, and furthermore, of the arguments A1 and A2. Thus, after an analysis of the role of case in the inactive construction, I will suggest in section 4 an analysis of the word order paradigm for late Middle Danish.

\section{The inactive construction in late Middle Danish} \label{heltoft:2}

The inactive construction of late Middle Danish falls in a number of subtypes, of which I shall deal with three. It is a continuation of a common Norse (and Germanic, further back Indo-European) set of constructions that deviate in important ways from transitive constructions. Late Middle Danish differs from Icelandic as well, but a detailed comparison is not available, so I will restrict myself to dealing with one basic difference, see \sectref{heltoft:2.6}.

The constructional set comprises 1) verbs that are inherently semantically inactive, that is: their stems will construe with an oblique A1; 2) verbs with transitive stems, needing an inflectional modification, namely the middle voice form, to form an inactive construction, 3) verbs with neutral stems, construing either with a nominative A1 or an oblique A1, that is, verbs with semantically different case constructional potential, but no morphological change of the stem to mark this difference, and 4) a type with no obvious difference between active voice and middle voice. The subtypes have been selected from a list of lemmas \citep{Bom1954} for the Old Danish Dictionary (not yet completed), and from a next to complete collection of quotations (card copies in electronic form, GldO). 

Two basic issues: 1) The inactive system's interplay with the voice system must be clarified. Some stems allow the inactive construction with the middle voice only, see \sectref{heltoft:2.2}; others allow it with the active voice, see \sectref{heltoft:2.3}; again, some apparent mergers of voice allow inactive construction both with the active and the middle voice, but at least in some cases, this distinction expresses a semantic contrast between two subtypes of inactive constructions. 2) Like many other older Indo-European languages, Old Scandinavian, including Old and Middle Danish, allows zero arguments, meaning that NPs at all levels can be let out, or better, replaced by zero. This leads to a methodological problem of how to determine whether an argument is a valence-governed actant of a verb stem that has been optionally replaced by zero, or whether it could instead be considered a free syntagmatic extension of the semantic nucleus of the clause (cf. \citealt{NielsenHeltoft2020}); see \sectref{heltoft:2.2} and Sections \ref{heltoft:2.4}--\ref{heltoft:2.5} for details.

\subsection{Verbs that are inherently semantically inactive} \label{heltoft:2.1}

The verbs belonging to this subcategory take an argument 1 (A1) denoting an animate referent that is causally affected, be it by bodily demands, by mental or social impression or by incidents of fate. I call this semantic role \textit{Inactive}, and constructions comprising it \textit{inactive constructions}.  

Some are one-place verbs, excluding the possibility of an argument 2 (A2), for instance: \textit{hungre} ‘starve', \textit{thyrste} ‘thirst', \textit{værke} ‘feel pain' (see \ref{ex:heltoft:5}); \textit{lithe} ‘do, fare'; \textit{fare ille/væl} ‘have a misfortune/have good fortune'.

\begin{exe}
\ex \label{ex:heltoft:5}
    \glll then timæ mek hungrudæ tha gauæ i megh at ædæ \\
         thæn time  mik    hungrethe  tha   gave  i    mik  at  æte   \\
         the time   me.\textsc{obl}   starved  then   gave  \textsc{2pl}   me   to eat\\
    \glt  `When I was hungry, then you gave me something to eat' (Luc 69v 8--10)

\ex \label{ex:heltoft:6} 
    \glll Dønær munnæn af thi, at [maghen] ær saar, Tha mat thu mærkæ athættæ: hanum thyrstær, oc thæn næthræ læbæ thyrckæs\\
    døner 	munen 	af thi at 		maghen	ær sar, tha  mat 	thu 		mærke 	{a thætte}: ha-num  thyrster, ok thæn næthre læpe thyrkes\\
    stinks {the mouth} from this that {the stomach} is sore, then can you {pay attention} {to this}: he-\textsc{obl}  thirsts,   and   the lower lip       {dries out}\\
    \glt `If the mouth stinks from a wound in the stomach, then you can pay attention to this (symptom): He is thirsty, and his lower lip is drying out' (AM 187, 30, 2--4)

\ex \label{ex:heltoft:7}
    \glll muæ i vidhæ, ath jegh ær karsk, ock megh lidher vell  \\
         mughe   i     vite at     jæk ær karsk   ok    mik    lith-er     væl\\
         may   \textsc{2pl} know that   I   am sound    and   me.\textsc{obl}  do-\textsc{prs.sg} well\\
    \glt `I can let you know that I am sound and I am doing well' (Miss II 389, Roskilde app. 1510?)

\ex \label{ex:heltoft:8}
    \glll een stundh for hannum   vell ath \\
         en    stund    for          han-um    væl    at  \\
         an     hour    fare.\textsc{prt.act}    he-\textsc{obl}    well  along\\
    \glt `At one time he (a rich king) fared well (i.e. he succeeded)' (RD II, 249, 3957--3958)
\end{exe}

Two-place: \textit{æve} (forms with breaking: \textit{jave}, \textit{jæve}) `doubt, be in doubt'; \textit{tvivle}\footnote{\textit{Tvivle} is a 15\textsuperscript{th} century Low German replacement loan for \textit{æve}. Sources show both inactive and transitive construction and thus, the continuous productivity of the inactive pattern. A handful of later manuscripts have \textit{tvivle} for \textit{æve} in example \REF{ex:heltoft:8}; of these, 4 retain an inactive construction, 3 are transitives, according to the edition's critical apparatus.} `doubt',  \textit{skilje} `disagree'. The A2 of these three verbs must have predicational value, either through clausal form as in \REF{ex:heltoft:9a} or through a predicational noun \REF{ex:heltoft:9b}.

\ea \label{ex:heltoft:9a} \ea
    \glll iafuær them um oc skil them um hwat hældær hun ær mæth ællær ey \\
         jav-er        thæm    um   ok    skil          thæm   um   hvat  hælder   hun   ær    mæth  æller  æj\\
         doubt-\textsc{prs.act}   they.\textsc{obl}   about   and   disagree-\textsc{prs.3sg}    they.\textsc{obl} about  what   either    she    is    with  or    not\\
    \glt `If they (appointed good women) doubt and disagree whether she is with (a child) or not' (DgL V. 5,3) 

\ex \label{ex:heltoft:9b}
    \glll hwaræ sum mæn æuær um sannænd. thær skal logh lethæ hwilt ræt ær \\
         hvare  sum  mæn    æv-er    um    sannende  thær   skal  logh  lethe hwilt     ræt   ær\\
         where   \textsc{rel}  man.\textsc{pl}  doubt-\textsc{sg}   about  truth    there   must  law  guide which    right  is\\
    \glt  `Where people are in doubt about truth, there the law must guide which is right' (CCD X, 3v)       
\z\z 

Example \REF{ex:heltoft:9b} is included because it shows a secondary morphological effect of the construction's semantics. There are hosts of medieval manuscripts of this text, the prologue of the Jutish Law, but not a single variant of this reading showing a plural form \textit{æv-e} to agree with \textit{mæn} `men'. The inactive construction does not allow concord between A1 and verbal number, only transitive constructions with a nominative A1 allow this, and even though nouns no longer inflect for the nominative vs. oblique distinction, the concord rules are still maintained\footnote{\citet[166]{Bjerrum1949} writes: In ``impersonal constructions” into which it is impossible to interpolate any subject (…) the verb is invariably in the singular, e.g. \textit{skil børn with mothær} (51\textsuperscript{5}) si mater et pueri discordant … ", that is: `if the children disagree with their mother'}, banning concord with inactive constructions.

\subsection{Middle voice inactive verbs} \label{heltoft:2.2}\largerpage[-1]

Some verbs need a middle voice form in order to construe inactively. The verb \textit{te} is from te-a, Icelandic \textit{tjá}, and its active voice forms are transitive only \REF{ex:heltoft:10}, the \textit{s}{}-form has a clearly passive variant \REF{ex:heltoft:11}.

\begin{exe}
\ex \label{ex:heltoft:10}
    \glll Ok ther thu hanum thitt wredhe anledhe, Tha ær thet  ey taknemælight, hwat     got thu gør hanum \\
         ok    te-r           thu    han-um    thit        vrethe  andlete  tha    ær    thæt   æj    taknemlikt    hvat got    thu  gør  han-um\\
         and  show-\textsc{prs.act.sg}    \textsc{2sg}  he-\textsc{obl}    \textsc{poss.2sg}  angry  face then  is     it     not  evident      what good    you do he-\textsc{obl}\\
    \glt `And if you show him your angry face, then it is not evident what good you are doing to him' (Sydr 161, 18--19)

\ex \label{ex:heltoft:11}
    \glll oc ænglæ føræ foræ hanum korss tegn, ath thet skal theræ thees foræ al mankøn. \\
         ok  engle føre fore hanum    kors tekn,  at    thæt    skal  thære    te-s      fore  al  mankyn\\
         and angles carry {in front of} him   cross sign, that  it      will  there    show-\textsc{pass}  for    all  mankind\\
    \glt `And angels will carry the sign of the cross in front of him, so that it be there shown to all mankind' (Luc 69r 7--10)
\end{exe}

The middle form in East Norse \textit{{}-s} (West Norse \textit{{}-sk/-st}) has four semantic variants (\citealt{Dyvik1980}; \citealt{Heltoft2006}), of which the passive is but one. The middle voice functional varieties are the reflexive function, the reciprocal function, and the detransitive function. The reflexive and the reciprocal functions are transitive variants, so the relevant function for the discussion of the inactive construction is the latter, detransitive one\footnote{The reflexive function is demonstrated in (i) \textit{Gudh alsommæctigste teedes henne} ‘the almighty God showed himself for her' (Bønneb II, 133, 15); the reciprocal function in (ii) \textit{the tordæ æy tees førræ æn the brudæ kostæ oc skyuldæ tøm met} (Luc 76v 7--10) ‘they (Adam and Eve) dared not show themselves to each other until they had broken off twigs to hide themselves with'.}. Examples are (\ref{ex:heltoft:12}--\ref{ex:heltoft:14}):

\begin{exe}
\ex \label{ex:heltoft:12} 
    \glll Tees thic thet, thic wel liger, tha ladh sighe messe de trinitate (…) \\
         te\textsubscript{a}{}-s     thik\textsubscript{A1a}  [thæt,     [thik\textsubscript{A1b}    væl liker\textsubscript{b}\_\_\textsubscript{A2b}]]\textsubscript{A2a}, tha lat sighje   misse de trinitate\\
         show-\textsc{middle}  \textsc{2sg.obl}   that.\textsc{nom}  \textsc{2sg.obl}    well likes, then let say  {the masses} of Trinity \\
    \glt `If you behold that which pleases you, then let say the masses of Trinity' (Bønneb III, 122, 17)
\end{exe}

In \REF{ex:heltoft:12}, both verbs are inactive. The verb form \textit{tes} governs the arguments subscribed with an \textit{a}, the verb \textit{liker} those with a subscribed \textit{b}. In both cases, the A2 is an embedded clause. In \REF{ex:heltoft:13} and \REF{ex:heltoft:14}, the A2's cannot be read as agents and hence they are not transitive, but inactive.

\begin{exe} 
\ex \label{ex:heltoft:13} 
    \glll Meg\textsubscript{} thee-s twæne honde folck\textsubscript{} \\
         Mik\textsubscript{A1}    te-s          [tvænne hande  folk]\textsubscript{A2}\\
         \textsc{1sg.obl}   appear-\textsc{prs.middle}    two {kinds of} people\\
    \glt `I see two kinds of people before me' (JBB kap.7, b5v)

\ex \label{ex:heltoft:14} 
    \glll ogh ther thedhes them stiærnen {i geen,} efter ad hun borthe war \\
         ok thær   te-th-es       thæm\textsubscript{A1}   stiarne-n\textsubscript{A2}   igen,   æfter at hun borte   var \\
         and there  appear-\textsc{prt-middle}  \textsc{3pl.obl}  star-\textsc{def}     again,  after that she gone   was \\  
    \glt `And there the star appeared to them again, after it had been gone' (Vejl Pilgr 220, 12)
\end{exe}

In \REF{ex:heltoft:15} and \REF{ex:heltoft:16}, I address the problem of zero arguments. In Old and Middle Scandinavian, NPs at all levels can be replaced by zero, and as premises for assuming a zero, I posit either conceptual necessity or linguistically well-defined ellipsis, and \REF{ex:heltoft:15} will show conceptual necessity.  In \REF{ex:heltoft:15}, the A1 is represented by zero, since it is referentially unspecified. The A2 is specified: ‘then some sign (A2) would appear (to whoever might be the perceiver, A1)', a conceptually necessary A1 referent, in the present case generic and therefore also textually omissible.

\ea \label{ex:heltoft:15} 
    \glll vare han saan at saken, tha tedess e noget teken i hans andlade, (...) æn vare han vsan, tha tediss icke. \\
        var-e     han san at saken,     tha   {te-th-es}           e noket tekn\textsubscript{A2}  i hans   andlete, (...) æn vare       han usan,     tha  {te-th-es}            ække\textsubscript{A2}\\
        be-\textsc{subj}  he guilty as charged  then   {appear-\textsc{prt-middle}}    always  some   sign    in his face  (...)    but be-\textsc{subj}    he {not guilty}  then {appear-\textsc{prt-middle}}  nothing\\
    \glt `If he should be guilty as charged, then some sign would appear in his face (…) but should he be not guilty, then nothing would show' (HellKv 8, 1)
\z

Apart from the omissibility of A1 (a zero argument, again of the verb \textit{tethes}), example \REF{ex:heltoft:16} is included to document the existence of actantless predicates (here: \textit{ræghne} `rain') in Middle Danish, in the sense that they have \textit{zero valence}, that is: no actant at all. This proves that Middle Danish, like so many other old Indo-European languages, does not have categorical NP-VP structure as a necessary structural principle. The context is: … \textit{that from Adam's time and until the day of Noah …}

\begin{exe}
\ex \label{ex:heltoft:16}
    \glll Tha regnedhe aldrigh, Ok teddes ekke {regn bwæ} pa hemmelen\\
         Tha   ræghnethe    aldrigh, ok te-th-es ække  ræghnbughe\textsubscript{A2} pa hemelen\\
         then  rained never,  and appear-\textsc{prt-middle}   not    rainbow in {the sky}\\
    \glt `then it never rained, and no rainbow appeared in the sky' (Sydr 51, 11--12)
\end{exe}

I have interpreted \REF{ex:heltoft:15} and \REF{ex:heltoft:16} according to the classical rules of zero arguments in Old Scandinavian; see \citet{Heltoft2012} and \citet{Faarlund2004}. Theoretically, they could be seen as bridging examples allowing also  the modern intransitive reading with a subject A1. In both cases, they would show subjects in a position later than the third structural position, cf. \sectref{heltoft:4.2}. 

\subsection{Neutral stems} \label{heltoft:2.3}

Some stems are neutral with respect to the transitive-inactive contrast, examples being: \textit{thrængje} ‘put a strain on, bother' • ‘need, be in jeopardy'; \textit{varthe} ‘be responsible for, guard' • ‘concern, be somebody's task or obligation'. Such verbs allow inactive construction with the active voice, and the opposition between transitive and inactive is manifested by the syntagmatic argument hierarchy only. Notice that (\ref{ex:heltoft:17a}--\ref{ex:heltoft:17b}) are transitive constructions, so the A1s are subjects, the oblique case arguments are A2s and direct objects.

\ea \label{ex:heltoft:17} 
\ea \label{ex:heltoft:17a} 
    \glll Mæn vndher haffde swa trængth hannum, at han wisthe ey, hwat han skulle sighæ. \\
         Men   under\textsubscript{A1}  havthe sva   thrængth       han-um\textsubscript{A2}   at han viste æj hvat han skulle sighje\\
         but   miracle  had   such  overwhelmed    he-\textsc{obl}    that he knew not what he should say\\
    \glt `But the miracle had overwhelmed him so that he knew not what to say' (ML 152, 19--153, 2)

\ex \label{ex:heltoft:17b}
    \glll Nar ikten trængher tegh tha strygh tegh wel om medh salffuen   \\        
        nar    ikten\textsubscript{A1}  trængher  thik\textsubscript{A2}   tha strygh thik væl um mæth salven\\
         when   {the gout} bothers    \textsc{2sg.obl}   then smear yourself well around with {the balm}\\
    \glt `When the gout bothers you, then smear yourself well with the balm' (Lægeb Thott 47, 30)

\ex \label{ex:heltoft:17c}
    \glll oc skal han bevi[se] them ydermer vinskap om them   threnger eller vetherthorvæ\\
         ok skal han bevise thæm ythermer vinskap   um    thæm\textsubscript{A1}  thrænger æller   vitherthurv-e\\
         and must he show them more friendship,   if     3\textsc{pl.obl}   {are in distress} or      need-\textsc{prs.pl}\\
    \glt `And he must show more friendship to them, if they are in distress or they need this' (3/8 1442 Varberg)
\z\z

In \REF{ex:heltoft:17c}, however, the oblique case argument \textit{thæm} is the A1 (the A2 is probably zero = \textit{ythermer vinskap} ‘more friendship').\footnote{The verb \textit{vitherthurve} ‘be in need of something' and its simplex \textit{thurve} ‘need' are not inactive verbs, and the GldO has no examples. The conjunction between \textit{thrænger} and \textit{vitherthurve} does not prove anything about subject status for the A1, since oblique A1s cannot agree with verbal number. \textit{Vitherthurve} can easily be read as a zero-argument transitive: (they) are in need (of this) (i.e. friendship). There is nothing in Old Scandinavian like Modern English or Modern Danish gapping rules.}

The verb \textit{varthe} is transitive in \REF{ex:heltoft:19a}. It has number agreement between the nominative subject and the finite verb, and the A2 in in the accusative, as in unmarked transitive patterns. Example \REF{ex:heltoft:19c}, however, is an inactive construction on the basis of the same verb stem in the active voice.\footnote{Similarly in Old Icelandic, with an accusative A1: (\textit{at segja þ}\textit{ér þat}) \textit{er þik} (acc) \textit{varðar} ‘to tell you what concerns you'.} The use of the cataphoric nominative pronoun \textit{thæt} is not obligatory, it is not a formal subject marker, and this construction therefore consists of an A1 in the oblique case, and a predicational A2 (\textit{at the hava æj vin}). The A1 has inactive semantic role meaning (in this case as the Obliged in a relation of duty or relevance coming from the outside).

\ea \label{ex:heltoft:19a} 
<the owner of a pond may bar his fellow-villagers' access to the pond>
\ea
    \glll utæn   the warthæ   han æm wæl sum han. \\
         uten  the\textsubscript{A1}      varth-e    han\textsubscript{A2}    æm væl    sum han\\
         unless  they.\textsc{nom.pl}  guard-\textsc{pl}  it.\textsc{acc}  {just as} well   as he.\\
    \glt `Unless they guard it just as well as he' (DgL V 192, 3)\footnote{In the Scanian Law, the transitive interpretation of the verb \textit{vartha} governs a dative object: \textit{Eld-e} (\textsc{d}) \textit{sin-um} (\textsc{d)} \textit{scal man vartha} (CCD III 93r) ‘a man must safeguard (or ‘be responsible for') his fire'. The West Danish example could either match the Old Icelandic situation where \textit{varða} in the sense of ‘guard, watch' governs the accusative, or it could be an instance of the general loss of verbal government of the dative case. I retain \textsc{acc} here, since the form indicates that this source preserves the accusative (\textit{han}) vs. dative (\textit{hanum}). There are no examples known to me of inactive constructions in Western Middle Danish that preserve a distinction between the accusative and the dative.}

\ex \label{ex:heltoft:19b}
    \glll Hwat waardher thet miik eller tik, at the hawa ey wiin, (...)\\
          hvat  varthar    thæt\textsubscript{A2}   [mik    æller  thik]\textsubscript{A1}  [at     the       hav-a    æj     vin]\textsubscript{A2}\\
        what  concerns  it.\textsc{nom}    me.\textsc{obl}   or    \textsc{2sg.obl} that   they.\textsc{nom.pl}   have-\textsc{pl}   not   wine, (...)\\
    \glt ‘How does it concern me or you that they have no wine'

\ex \label{ex:heltoft:19c} 
    \glll Thet wordhar them som os hawa budhit, oc ey os, thet at the hawa ey wiin.\\
         thæt\textsubscript{A2} varthar    thæm\textsubscript{A1}  sum os     hava  buthit  ok æj    os\textsubscript{A1} [thæt     at     the   hava   æj     vin.]\textsubscript{A2}\\
         it-\textsc{nom}   concerns  \textsc{3pl.obl}  \textsc{rel} us.\textsc{obl}  have   asked     and not   us.\textsc{obl} this     that  they   have   not    wine\\
    \glt ‘It concerns those who have invited us, and not us, that they have no wine' (Post 46, 9--13)
\z\z

A fourth example of a neutral stem would be the verb \textit{skilje}, meaning (transitive) ‘divide', (reflexive) ‘part, divorce' and (inactive) ‘disagree'. The inactive function is exemplified in \REF{ex:heltoft:9a}.

\subsection{An apparent voice merger} \label{heltoft:2.4}
Some inactive verbs construe inactively as such irrespective of voice, that is, both the active voice and the middle voice can be used. I will discuss the verb \textit{thækje} ‘learn, find reasonable' • ‘like, please', which allows an A2 of either type: non-predicational or predicational. In the active voice, the inactive construction of \textit{thækje} means that ‘somebody knows or learns something', or that ‘somebody finds something reasonable', as in (\ref{ex:heltoft:19}--\ref{ex:heltoft:20}).


\ea \label{ex:heltoft:19}
    \glll vthæn standæ moth høymot oc bældæ met mywgdom, {tho uær men} hanum tekker thet at han vorthær forsmoth ther aff fore værdæn.\\
           uten stande mot høghmot ok bælde mæth mjukdom tho-at-hvarem han-um\textsubscript{A1} thækk-er  thæt\textsubscript{A2} [at han varther forsmath thær af fore værden.]\textsubscript{A2}\\
         but    stand against haughtiness and arrogance with meekness {even if} he-\textsc{obl}    learn-\textsc{3prs.sg}   that   that he becomes despised  there from for {the world}.\\
    \glt ‘But he must resist haughtiness and arrogance with meekness, even if he learns he is despised for this by the world' (Luc 65r 14--17)

\ex \label{ex:heltoft:20}
    \glll æn ther forudhen   ma man delæ hannom fore hærwirke sagh, oc   æn   ydermere vm hannom thecker \\
     æn    thær  foruthen    ma  man  dele  hanum    fore hærwirke sak ok    æn    ythermere  um    han-um\textsubscript{A1}  thækker  \\
     even   there   {in addition}  may one  charge him     for armed robbery and  even   more    if    he-\textsc{obl}    {seems reasonable}\\
    \glt `And in addition to this, one may charge him with armed robbery, and even more if he finds this reasonable' (Thord Degn text 2, 122, 20)
\z
  
The middle form of this verb is \textit{thækkjes} ‘to please, to satisfy', in religious texts a most frequently discussed relation to God and Jesus, and therefore one of the best documentations of the distribution of case forms, including word order. 

\begin{exe}
\ex \label{ex:heltoft:21}
    \glll Oc {æy thes mynne} gøre the {ther æffter} alt thet them thækkes\\
        ok   {æj thæs minne} gøre the    {thær æfter} alt thæt   thæm\textsubscript{A1} {thækk-es \_\_}\textsubscript{A2}\\
        and nevertheless   do they  thereafter  all that    they.\textsc{obl}  please-\textsc{middle}\\
    \glt ‘And nevertheless they do thereafter [after the Holy Communion] anything they please.' (Fragm 107, 15--16)
\end{exe}


Examples (\ref{ex:heltoft:22}--\ref{ex:heltoft:23}) have 2\textsc{sg} nominative A2s.

\ea    \label{ex:heltoft:22}
    \glll {i gardagh} thæckthes thu mik mæsth\\
         {i gardagh}  thæk-t-es          thu\textsubscript{A2}   mik\textsubscript{A1}    mæst\\
         yesterday  please-\textsc{prt-middle.3sg}  \textsc{2sg}\textsc{.nom}  \textsc{1sg.obl} most\\
    \glt ‘Yesterday I loved you the most.' (ML 424, 21)   

\ex \label{ex:heltoft:23}
    \glll hwn leffdhæ fulkommelighæ i ræthfærdughet, oc {ther fore} thæktes hwn gudh\\
        hun    livde  fulkommelike   i rætfærthughhet     ok {thær fore} thæk-t-es            hun\textsubscript{A2}    guth\textsubscript{A1}\\
        she.\textsc{nom}   lived  completely     in righteousness    and therefore please-\textsc{prt-middle.3sg}    she-\textsc{nom}  God\\
    \glt ‘She (Anna) lived completely in righteousness, and therefore she pleased God.' (Bønneb III, 61, 8--10)
\z

Notice that (\ref{ex:heltoft:22}--\ref{ex:heltoft:23}) cannot have the transitive reading ‘do something to please'. They mean ‘A1 finds pleasure in A2'.

In the case of \textit{thækkje} there was a clear semantic difference between the lexical meanings realised, in the active and the middle voices, respectively. In all probability, some instances of genuine mergers are also found. In addition to example \REF{ex:heltoft:8}, there is also the following version of a poetic formula:

\begin{exe}
\ex    \label{ex:heltoft:24}
    \glll {jen stwndh} fors hanum fwld wæl adh\\
       {en  stund}  for-s          han-um    ful    væl   at \\ 
       {at a time}   fare.\textsc{prt-middle.3sg}  he-\textsc{obl}    full    well  along \\
    \glt ‘For some time, he (a rich king) fared very well (i.e. he had great luck).' (RD I, 147, 4620 – 4621)
\end{exe} 

Other candidates would be \textit{thykje(s)} ‘think, seem' and \textit{hope(s)} ‘hope, wish'.  However, space will forbid a more thorough investigation, and the full overview is – as I see it – no precondition for the present line of thought: to lay out the paradigmatic organisation of case and constructions in Middle Western Danish, in order to relate this to some basic differences between the paradigmatic organisation of word order in Middle and Modern Danish.

\subsection{The loss of the accusative-dative distinction} \label{heltoft:2.5}

Even a brief comparison of Middle Danish with Old Icelandic will show a major difference, namely the loss of a clear accusative vs. dative distinction in Danish. The archaic Scanian dialect preserves clear datives in cases like \REF{ex:heltoft:25} and \REF{ex:heltoft:26}.

\ea    \label{ex:heltoft:25}
    \glll æn   brista   brythianum\textsubscript{} the logh\\
          æn   brist-a       brytia-num\textsubscript{} the logh\\
          but  fail-\textsc{prs.pl}   tenant-*-\textsc{def.dat.sg}     [\textsc{dem.nom.pl} proof\nobreakdash-through\nobreakdash-oath.\textsc{nom.pl]}\\
    \glt ‘But should the tenant fail in performing his proofs.' (CCD III, B74 95v)
\z

Notice the number concord in \REF{ex:heltoft:25} between the subject \textit{the logh} and the finite verb \textit{brista}. Whether this construction belongs to the inactives, will be discussed below.\footnote{In (\ref{ex:heltoft:25}--\ref{ex:heltoft:26}), * = a syncretism of \textsc{acc/dat/gen}, characteristic of the an-stems and ōn-stems.}


\ea \label{ex:heltoft:26} 
    \glll Sama nattena tha hon var dødh tha tedhis abodanum en andelik syn.\\
        sama nattena tha hon var døth tha  te-th-is abod-a-num\textsubscript{A1} en andelik syn\\
        same night when she was dead then   show-\textsc{prt-middle}   abbot-*-\textsc{def.dat.sg} {}[a.\textsc{nom}   spiritual-\textsc{nom}     vision]\textsubscript{A2}\\
    \glt ‘On the same night when she had died, then the abbot had a spiritual vision.' (SjT 34, 5–7)
\z

Clearly, \REF{ex:heltoft:26} documents the distribution of case with this type of inactives in the archaic Scanian dialect, but what is hard to document in Danish is not the use of explicit datives for the A1 of inactive constructions, it is the accusative. The earlier, presumably Common Norse system was preserved in Old Icelandic (and to a large extent even in Modern Icelandic), and here the A1s can appear in the accusative. I shall compare the situation with two verbs, in \sectref{heltoft:2.5.1}. the verb OIcel. \textit{reka}, Middle Danish \textit{vreke}, \textit{vrake}; in \sectref{heltoft:2.5.2}. the verb OIcel. \textit{bresta}, Middle Danish \textit{briste}.

\subsubsection{A difference from Old Icelandic} \label{heltoft:2.5.1}

Old Icelandic has the inactive construction type \REF{ex:heltoft:27} (cf. \citealt{Sigurðsson1989, Sigurðsson2006}):

\begin{exe}
\ex \label{ex:heltoft:27}
    \gll bát-a-na           rak         {til lands}\\
         boat-\textsc{acc.pl-acc.pl.m}   drift.\textsc{prt.sg}    ashore\\
    \glt ‘The boats drifted ashore.'
\end{exe}

The archive of the Dictionary of Old Danish lists as comparable verb forms transitive \textit{vreke} ‘drive out, expel' • ‘open a lawsuit', from *wrekan (Ablaut type~5), and a parallel (mainly East Danish) form \textit{vrake}, corresponding to Germanic *wra\-kan, but possibly a relatively recent remodeling to Ablaut type 6.\footnote{East Norse preserves Germanic *w- in front of r-, compare Old Danish \textit{vrēth} ‘angry' to Old Icelandic \textit{reiðr} ‘angry'. There is even a -jan-formation \textit{vrekje} ‘expel', from *wrak-jan, to be disregarded here.}  The intransitive meaning ‘drift' and the inactive construction is not found in the data in the active voice but has apparently been replaced by a mediopassive intransitive. Such intransitives as (\ref{ex:heltoft:28a}--\ref{ex:heltoft:28b}) can have nominative subjects.

\ea \label{ex:heltoft:28} 
\ea 
    \glll  oc han scal castæ af sit timbær (…) oc thet wrax {in til lands}\\
    ok han skal kaste af sit timber (…)     ok     thæt       vrak-s {in til lands}  \\
    and he must throw off his timber (…)   and   this-\textsc{nom/obl}   drift-\textsc{prs.middle}  ashore\\
    \glt ‘And he must throw overboard his timber or other valuables, and this drifts ashore.' (DgL V 352, 4) \label{ex:heltoft:28a}
\ex \label{ex:heltoft:28b} 
    \glll um wrac af haf wræcs {in til landz} \\
    um vrak        af hav     vræk-s         {in til lands}\\
    about wreckage  from sea   drift-\textsc{prs.middle}     ashore\\
    \glt ‘About wreckage that drifts ashore from the sea.' (DgL V 349, 8)  
\z
\z


True, the pronominal form \textit{thæt} does not distinguish the nominative from the accusative but judging from Old Icelandic this distinction is clear-cut. Example \REF{ex:heltoft:29a} is the inactive construction, while \REF{ex:heltoft:29b} is a reflexive construction with a nominative subject.

\ea \label{ex:heltoft:29} 
\ea
    \gll <hann> skilr svá við hana          at     hana     rek-r  dauð-a     eptir ánni\\
        (he) departs {in such a way} from her   that  she.\textsc{acc}   drift-\textsc{prs.act} dead-\textsc{acc.f}  along {the river}\\
    \glt ‘He gets rid of her in such a way that she drifts dead down along the river.' (\textit{HeiðrR} 53\textsuperscript{15}, Normalised by author) \label{ex:heltoft:29a}
\ex \label{ex:heltoft:29b}
    \glll segir þat osynniu   ath hon rekiz j suo dyrum klædum\\
          segir þat ósynju    at     hon    rek-i-z  í svá dýrum klæðum\\
          {(he) calls} it unwise   that   she.\textsc{nom}  {go around-\textsc{prs.subj-middle}} in such costly garments\\
    \glt `He says it is unwise for her to walk around in such costly garments.' (ClarB 19\textsuperscript{30})  
\z\z

The correct strategy here is to postulate only inactive constructions where inevitable. The data are scarce, but it seems likely that this type of inactive construction has been replaced in Danish, in this case by an intransitive middle form.

\subsubsection{The verb \textit{briste/bresta}} \label{heltoft:2.5.2}

The polysemous verb \textit{briste} ‘burst, split' • ‘fail' • ‘miss, lack, be short of' (Old Icelandic \textit{bresta}) is yet another illustration of the way the Danish construction has been reshaped. In the sense of ‘lack, miss', \textit{bresta} is documented with an accusative A1 \textit{mik} (the dative is \textit{mér}):

\begin{exe}
\ex  \label{ex:heltoft:30}
    \gll eigi   brest-r     mik       áræði\\
         not   lack-\textsc{prs.3sg}  \textsc{1sg.acc}    courage\\
    \glt ‘I do not lack fighting spirit.' (ONP 750 Vatnsdæla saga)
\end{exe}

Even in the most archaic Danish data, I have found nothing similar with any type of NP, so the accusative type has been merged with the dative type, as typical of almost all other occurrences, as in \REF{ex:heltoft:31}:

\begin{exe}
\ex    \label{ex:heltoft:31}
    \glll førstæ them brøster wobn i strid tha holdæ the met æn hand oc slaa met then annen\\
        fyrste   thæm      brist-er     vapn     i strith      tha   halde   the  mæth en hand   ok     sla     mæth thæn annen\\
        first  them.\textsc{obl}   lack-\textsc{prs.sg}  weapon in combat  then   hold  they  with one hand    and   punch  with the other\\
    \glt ‘As soon as they lack a weapon in combat, they grip (the enemy) with one hand and punch with the other one.' (Luc 60r 21--23)
\end{exe}

This means we can ask whether constructions with the other senses of \textit{briste} should also be analysed as inactive constructions. Consider \REF{ex:heltoft:32} (sense ‘burst') and \REF{ex:heltoft:33} (sense ‘fail'):

\begin{exe}
\ex  \label{ex:heltoft:32}
    \glll Æn cumær thet swa at hin ær akær at hanum bristær tømæ. ællær hin  er rithær at hanum bristær tyghlæthær. oc wagn løpær ællær  hæst rænnær mæth hanum. oc man fár thæræ døth af. tha …\\
        en kumer thæt sva at hin ær aker   at     han-um   brister  tøme   æller hin  ær rither at     han-um   brister       tyghlæther.  ok vagn løper æller hæst rænner   mæth hanum ok man far thære døth af. tha …\\
        but comes it so that he who drives  that   he-\textsc{obl}    bursts  rein  or he  who rides   that   he-\textsc{obl}    burst-\textsc{prs.sg}  bridle      and cart runs or horse runs       with him and man becomes there death from, then …\\
    \glt ‘But if it happens that he who drives  that the rein bursts for him, or he who rides that the bridle should split for him, and the cart or horse run with him and (this) man meets his death from this, then ...' (DgL V 202, 9) 
\end{exe}

Where the sense of ‘burst, split' is concerned, there is no conceptual necessity that an oblique actant should be part of the valence schema. We can have \textit{tyghlæther brister} ‘the bridle splits' and \textit{bughe brast} ‘the bow burst', Old Norwegian \textit{Jorðin oll brestr oc rifnar} (ONP 2: 750) ‘the whole earth is bursting and quaking', without implying an extra Afficiary\footnote{The terms Afficiary and Maleficiary are from \citet{Zúñiga2011}.} actant. The Norwegian example has a nominative subject and documents that the verb is intransitive in this sense. An Afficiary actant may of course be added, but then freely, as a free oblique argument with the Afficiary Role as the A2, in the present case the Maleficiary variant. In \REF{ex:heltoft:33}, the meaning of \textit{briste} is ‘fail, not succeed', clearly implying an argument ascribing the notion of a Maleficiary to its referent.    

\begin{exe}
\ex \label{ex:heltoft:33}
    \glll en  brister hannum takk. eth skotæ. tha gøme  bondæ sialf sin thiuf\\
         æn   brist-er    han-um    tak      æth skote  tha   gøme  bonde sialf  sin     thjuv\\ 
         but fail-\textsc{prs.3sg}  he-\textsc{obl}    guarantee  or proof  then   guard   landowner self  \textsc{refl}  thief\\
    \glt `But if guarantee or proof fail him (a suspected thief), then the landowner may himself alone take his thief into custody.' (JL CCD X, C 37, 45r)
\end{exe}

On the basis of this line of argument, I group the types \REF{ex:heltoft:25} and \REF{ex:heltoft:31} together with \REF{ex:heltoft:33} as synchronically belonging to the inactive constructions. The A1 is in the oblique case, and neither subordinate sense is compatible with any notion of agenthood where the semantic roles are concerned. What we obtain, is a new variant of the A1 Inactive role, namely the Afficiary role, in addition to the Experient role. Notice that in transitive constructions, Afficiary meaning can only be ascribed to A3s, since the dative with verbs like \textit{thakke} ‘thank', \textit{skathe} ‘do harm to', \textit{møte} ‘meet', \textit{varthe} ‘be responsible for' has been lost in Middle Western Danish.

\subsection{Case roles of the inactive constructional system} \label{heltoft:2.6}

In this survey of inactive Middle Danish predicates, the categorisation below seems to cover most of the occurrences. No Agentive meanings are coded, and the Inactive semantic roles apply to animate referents that could in a different constructional context very well carry Agentive meaning. The inactive role differs from the patient role in that the latter applies freely to animate referents and inanimate referents alike, the former only to potential agents.

\begin{enumerate}
    \item Unspecified inactive one-place verbs, for instance: \textit{hungre} ‘starve', \textit{lithe} ‘go, pass' (of time and fate), \textit{thyrste} ‘thirst, be thirsty', \textit{værkje} ‘feel pain, be in agony'.
    \item Three subtypes of two-place verbs, each displaying a bound variant (a variety) of the Inactive role, depending on the type of relation denotated.
    \begin{enumerate}
        \item A1 (Experient), A2 (External factor), such as: \textit{angre}  ‘repent', \textit{drøme} ‘dream', \textit{hope(s)} ‘hope', \textit{minnes} ‘remember', \textit{sjunes} ‘seem', \textit{tes} ‘appear', \textit{thryte} ‘regret', \textit{thækkje(s)} ‘know, learn'; ‘please'.
        \item A1 (Afficiary), A2 (External factor), such as: \textit{briste} ‘fail', \textit{rækkje} ‘be enough, suffice', \textit{vanskes} ‘lack', \textit{vante} ‘lack'.
        \item A1 (Obliged), A2 (External factor), such as: \textit{byrje} ‘ought', \textit{høre} ‘ought', \textit{sta/stande} ‘befit', \textit{varthe} ‘be responsible for, have as one's duty'.
    \end{enumerate}
\end{enumerate}

\section{The paradigmatic organisation of pronomial case in Western Middle Danish} \label{heltoft:3}\largerpage

So far the analysis has shown that we cannot know the actual content of the case forms without checking their valence bearer, i.e. the verb stem governing them. Actants in the oblique case are polysemous as far as the content of the case form is concerned. Case forms with one-place verbs are simple, since a nominative actant will be checked against an intransitive verb and the abstract, open semantic role (the classical, general function of the nominative) will be selected, for instance, \textit{the} \textit{gape} ‘they gape, open their mouths wide' has the nominative \textit{the}, and since both agentive and non-agentive readings are possible, the stem \textit{gap-} confirms that this nominative must be read in the open, unmarked sense. In \textit{mik thyrster} ‘I am thirsty' the oblique form \textit{mik} will point to the stem \textit{thyrst-} to acquire the inactive role reading, excluding the patient reading. In the case of two-place stems, let alone the polysemous ones like \textit{thrængje}, \textit{varthe} and \textit{skilje}, the argument hierarchy helps to determine which variety (bound variant) of the case meaning is the relevant one, and it is therefore part of the paradigmatic organisation. Case meaning and constructional meaning must both be included in the paradigmatic analysis. Say that the semantic roles relevant for transitive constructions are Unmarked role (very often Agent), calling for the nominative case, and Patient, calling for the oblique case. This pair of roles will not apply as case meanings for the arguments of the inactive constructions such as \textit{hanum thækker thæt} ‘he learns this', cf. \REF{ex:heltoft:19}. The oblique form \textit{hanum} must manifest an A1 and hence this case form must denote the inactive role, an animate referent, with two-place predicates, influenced by some external factor, for instance: a phenomenon perceived, a norm to be complied with, or some state-of-affairs related to what is in one's interest or need. Notice again: It is excluded from any meaning of agenthood or intentional action. 

The form \textit{thæt} ‘that/it' denotes the external factor leading to the state of satisfaction on behalf of A1's referent, that is, it is a nominative A2 with a very specific meaning. Syntagmatic hierarchy and case oppositions go together, and such combinations of morphological contrasts and syntagmatic systems were called connecting paradigms by \citet{Nørgård-Sørensen2011}, since they consist of both morphologically determined meaning potential and constructional determination of the choice between options given by the polysemous case system. Thus, the structurally determined meaning of the members of the case paradigm is the result of an intersection between morphology and construction, and case meaning has both a morphological expression system and a syntagmatic, constructional one.

To see this in uncomplicated practice, take the German dative case form. This will receive different semantic interpretations from different predicates. In (\ref{ex:heltoft:34}--\ref{ex:heltoft:36}), a well-known type of example, case meaning differs along with argument hierarchy.

\begin{exe}
\ex \label{ex:heltoft:34}
    \gll \textit{Mir} (dative A1) ist kalt.\\
        Me.\textsc{dat} (dative A1) is cold.\\
    \glt ‘I am cold'.

\ex \label{ex:heltoft:35}
    \gll Sie hat \textit{mir} (dative A2) gedankt.\\
         She has me.\textsc{dat} (dative A2) thanked.\\
    \glt ‘She thanked me'.

\ex \label{ex:heltoft:36}
    \gll Wer hat \textit{mir} (dative A3) das Hemd schenken wollen?\\ 
        Who has me.\textsc{dat} (dative A3) the shirt {give as a present} want?\\
    \glt ‘Who wanted to give me this shirt?'
\end{exe}

Schematically, cf. \tabref{tab:heltoft:3}.\footnote{It is not important to discuss here whether the dative A2 means Afficiary or Patient, or whether dative verbs like \textit{begegnen}, \textit{begleiten and} \textit{folgen} take a Comitative A2.} The present analysis of Middle Danish can be represented as \tabref{tab:heltoft:4}.

\begin{table}
\caption{Hierarchy of German dative\label{tab:heltoft:3}}
\begin{tabular}{lccc}
\lsptoprule
         & \multicolumn{3}{c}{Hierarchy}\\\cmidrule(lr){2-4}
Case     & {A1} & {A2} & {A3} \\\midrule
         & & & {Afficiary}\\
{Dative} & & {Patient/Comitative}  \\
         & {Inactive} &  & \\
\lspbottomrule
\end{tabular}
\end{table}


\begin{table}
\caption{Case paradigm and argument hierarchy in Middle Western Danish\label{tab:heltoft:4}}
\begin{tabularx}{.66\textwidth}{Qll}
\lsptoprule
 & \multicolumn{2}{c}{Hierarchy}\\\cmidrule(lr){2-3}
{Case} & \multicolumn{1}{c}{A1} & \multicolumn{1}{c}{A2}\\\midrule
{Nominative} & {Unmarked} & {External Factor}\\
{Oblique} & {Inactive} & {Patient}\\
\lspbottomrule
\end{tabularx}
\end{table}

The status of the nominative A1 as unmarked is of course fully compatible with the expectation that the majority of lexically, not grammatically, determined roles will be Agents, but the nominative in transitive constructions does not insist on this. 

\subsection{The indexicality of case} \label{heltoft:3.1}

The semiotic function of pointing between signs is well-known from C.S. Peirce's semiotics as a subtype of the indexical function. This notion has been applied especially to morphology by \citet{Andersen1980} and \citet{Antilla1975}, with a clear indication that it will apply to syntactic and topological issues as well. 

When indexical, case forms point to their governing predicate as the locus where their exact semantic function is determined. The predicate determines the relevant argument status and the relevant variety of semantic role. With the oblique form, the choice is between Inactive and Patient; with the nominative, it is between Unmarked role (very often lexically filled in as Agent) and External factor (Experiencer, Afficiary or Obliged). Thus, the nominative of the two-place inactive construction points to an inactive verb and receives A2 status, with a very specialised semantic role meaning potential.  

In conclusion, indexical case means the case form depends for its actualised meaning on its predicate. Importantly, indexical case structure is but one typological organisation of case. To some extent, Old Indo-European languages have symbolic case structure\footnote{Where symbolic case is concerned, the case form alone bears case meaning. A well-known remnant of simple symbolic case in Latin is found in \textit{cave canem} ‘beware of the dog' vs. \textit{cave cani} ‘take care of the dog'. The case opposition specifies the meaning potential of the verb stem cave-, in itself neutral to this opposition. Case is normally indexical in Latin. In \textit{signa … detracta lucis} ‘emblems carried out from the groves' (Tacitus Germ. 140, 3), the case ending -is indicates the stem detract- from where the ablative sense of the case ending is determined.}, and as we shall see, Modern Danish has in fact abolished indexical case to replace it by a simple symbolic opposition, see \sectref{heltoft:5}.

\subsection{Subjects and objects} \label{heltoft:3.2}

Up to this point, I have by and large avoided the issue of grammatical relations in the sense of subjects and objects. The argument hierarchy is laid out as projections of valence structure, and a priori assumptions of a connection between A1 and subject, A2 and direct object has been deliberately shunned. 

\citet{Melčuk2014} suggests a set of universal syntactic criteria for (\textit{not} features of) the universality of subjects, applied by me in \citet{Heltoft2021, Heltoft2021b}. To the criteria of \citet{Keenan1976}, he adds a distinction between definitional criteria (necessary for a given language) and characterising criteria (frequent, but not necessary). Very briefly, his definitional criteria are laid out in 1--7. The \textit{subject candidate} (SC) must be checked against the following parameters: 

\begin{enumerate}
    \item Is SC an immediate actant of the main verb? (it must be)
    \item Is SC omissible or not?
    \item Does SC hold a particular linear position?
    \item Morphological impact on the main verb (personal-numeral agreement)
    \item The main verb's morphological impact on SC (Does the main verb govern SC's case marking?)
    \item The main verb's inflection affecting morphological links to the SC (refers to voice, antipassive construction)
    \item SC's pronominalisation if this affects morphological links between the MV and SC.
\end{enumerate}

On the basis of \sectref{heltoft:2}, we can now determine the subject criteria relevant for Middle Danish and compare them to the criteria relevant for Modern Danish. 

  Criterion 1 applies to all instances of A1 and A2, both in transitive and in inactive constructions\footnote{I omit here a discussion about the status of Predicative complements as Main Predicates; see \citet{Heltoft2017}, in general \citet{HansenHeltoft2011}.}, and where criterion 2 is concerned, all arguments are omissible. Thus, neither of these parameters are relevant for Middle Danish. 

  It must be an open empirical question whether the subject candidates hold a particular linear position, and I will deal with this in \sectref{heltoft:4}. To reveal the conclusion already here, Middle Danish does not have a subject position, whereas the modern language certainly has developed one, cf. \sectref{heltoft:1.2}. This means that we are referred to morphological criteria, namely to numeral concord and to case rection (government). In transitive constructions, the A1 must be in the nominative case; inactive constructions, by contrast, take the A2 in the nominative. As a general principle, nominative DPs agree with the finite verb in number,\footnote{Some details omitted, especially about the singular substituting for the plural, never vice versa.} cf. \sectref{heltoft:2.1}. These criteria point to the nominative DPs as the subjects of Western Middle Danish. Parameter 7 is relevant as far as it determines the application range of nominative government. Voice cannot count as a defining feature, since inactive constructions do not have an active vs. passive voice contrast.

\subsection{A constructional typology: Case, grammatical relations and argument structure} \label{heltoft:3.3}

The outcome of the analysis is that the overall distribution of case defines the subject in Middle Danish, whereas the argument status is responsible for the ascription of semantic role variety within the case system. There is no traditional term for a grammatical relation corresponding to the A1 inactive, since the idea of a direct object is intimately connected with the transitive pattern. We can illustrate the two types in \tabref{tab:heltoft:5}.

\begin{table}\small
\caption{Transitive and inactive constructional typology}
\label{tab:heltoft:5}
    \begin{subtable}[b]{.5\linewidth}\centering
        \caption{Transitive structure}
    \begin{tabular}{ll}
        \lsptoprule
        {A1} & {A2}\\\midrule
        {S} {(nominative)} & --- \\
        {S} {(nominative)} & {DO} {(oblique)}\\
        \lspbottomrule
    \end{tabular}
    \end{subtable}\begin{subtable}[b]{.5\linewidth}\centering
        \caption{Inactive structure}
    \begin{tabular}{ll}
        \lsptoprule
        {A1} & {A2}\\\midrule
        {inactive} {(oblique)} & --- \\
        {inactive} {(oblique)} & {S} {(nominative)}\\
        \lspbottomrule
    \end{tabular}
    \end{subtable}
\end{table}

To add to the relevance of the distinction between arguments and grammatical relations, I include two further possible interaction types between morphology, grammatical relations and argument structure, namely the constructional option found in both English and Modern Mainland Scandinavian, somewhat confusingly named ``ergative'' by \citet{Halliday1968, Halliday1994}. \citet{HansenHeltoft2011} call this pattern the incausative pattern, and Danish verbs construing in this way are: \textit{brænde} ‘burn', \textit{dreje} ‘turn', \textit{standse} ‘stop', \textit{vælte} ‘turn over'\textit{, øge} ‘increase', etc., the translations immediately offering English parallels.

\ea \label{ex:heltoft:37} 
\ea 
    \gll De            brændte \\
    They.\textsc{nom}      burned\\
    \glt `They burned.' \label{ex:heltoft:37a}
\ex \label{ex:heltoft:37b}
    \gll Hun        brændte    dem\\
    She.\textsc{nom} burned      them.\textsc{obl}  \\
    \glt `She burned them.'
\z\z

The incausative structure is shown in \tabref{tab:heltoft:6}. It is a combination of ergative argument structure with transitive grammatical relations and transitive case morphology. The modern case morphology  involved is different from that of Middle Danish, see below \sectref{heltoft:5.2}, in that it no longer marks semantic role. It is an example of ergative argument articulation in combination with what looks like transitive morphology. To make this point stand out, I add classical ergative structure, as represented by Greenlandic in \tabref{tab:heltoft:7}, examples (\ref{ex:heltoft:38}--\ref{ex:heltoft:39}). 

\begin{table}\small\captionsetup{margin=.025\linewidth}
\begin{floatrow}
\ttabbox{\begin{tabular}{ll}
\lsptoprule
{A1} & {A2}\\\midrule
{S} {(nominative)} & --- \\
{DO} {(oblique)}   & {S} {(nominative)}\\
\lspbottomrule
\end{tabular}}
{\caption{Incausative-causative structure\label{tab:heltoft:6}}}

\ttabbox{\begin{tabular}{ll}
\lsptoprule
{A1} & {A2}\\\midrule
{S} {(absolutive/nominative)} & --- \\
{O} {(absolutive/nominative)} & {S} {(relative)}\\
\lspbottomrule
\end{tabular}}
{\caption{Ergative constructional typology\label{tab:heltoft:7}}}
\end{floatrow}
\end{table}

Greenlandic has always number and person concord between subject and finite verb, and in transitive clauses even between direct object and finite verb. In transitive clauses, the intransitive concord is maintained and yet another layer of concord is added. In elementary Greenlandic:  

\ea \label{ex:heltoft:38} \ea 
    \gll piniarto-q       piniar-poq \label{ex:heltoft:38a}\\
        sealer-\textsc{abs.3sg}  hunt-\textsc{indic} [3\textsc{sg(subj)]}\\
    \glt ‘The sealer is/was hunting.'

\ex \label{ex:heltoft:38b}
    \gll piniartu-t     piniar-put\\
         sealer-\textsc{abs.3pl}  hunt-\textsc{indic} [3\textsc{pl(subj)]}\\
    \glt ‘The sealers are/were hunting.'
\z\ex \label{ex:heltoft:39} \ea
    \gll puisi        siku-mi        sinip-poq  \label{ex:heltoft:39a} \\
         seal-\textsc{abs.3sg}    ice\textsc{{}-loc.sg} sleep\textsc{{}-indic [}3\textsc{sg(subj)]}\\
    \glt ‘The seal is/was asleep on the ice.'

\ex \label{ex:heltoft:39b}
    \gll piniartu-p     puisi           pisar-aa   \\
         sealer-\textsc{rel.3sg}  seal-\textsc{abs.3sg}       catch-\textsc{indic} [3\textsc{sg(subj).}3\textsc{sg(obj)]}\\
    \glt ‘The sealer catches/caught the seal.'

\ex \label{ex:heltoft:39c}
    \gll piniartu-p     puisi-t        pisar-ai\\
        sealer-\textsc{rel.3sg}  seal-\textsc{abs.3pl}     catch-\textsc{indic} [3\textsc{sg(subj).}3\textsc{pl(obj)]}\\
    \glt ‘The sealer catches/caught the seals.'
    
\ex \label{ex:heltoft:39d}
    \gll piniartu-t     puisi-t        pisar-aat\\
        sealer-\textsc{rel.3pl}  seal-\textsc{abs.3pl}     catch-\textsc{indic} \textsc{[3pl(subj).}3\textsc{pl(obj)]}\\
    \glt ‘The sealers catch/caught the seals.'
\z\z

\subsection{Summary} \label{heltoft:3.4}

Summarising \sectref{heltoft:3}, the main point is that inactive constructions cannot be reduced to transitive constructions, and the semantic role ascription to their A1s cannot be reduced to that of transitive subjects. The polysemy of the members of the case category is resolved by indexical pointing to the predicate as the valence bearer. In \sectref{heltoft:5}, we shall see that this system was replaced by a symbolic, non-valence governed case system, mirroring at first syntactic relations alone, later also phoric distinctions.

\section{Inactive constructions and the topology of Middle Danish} \label{heltoft:4}

In this section, we return to Melčuk's criterion 3 \citep{Melčuk2014} and the question whether Middle Danish subjects can be positionally identified. One point here is Melčuk's distinction between definitional criteria, which are necessary for a given empirical language, and characterising criteria, for instance prototypically relevant features, and thus also standard identifications of subject positions as the position held in unmarked clauses (the more marked positions then being transformationally derived). What we are asking, then, along with Melčuk, is whether some positional criterion is unique for the Middle Danish subject. For instance, Modern French subject topology is unique, in that this language has a position reserved for subjects, and furthermore, an obligatory one. 

I shall add the question whether Middle Danish had a particular linear position for subjects, and next, whether subjects contribute to the content side of the word order paradigm for Middle Danish. We have already seen in \sectref{heltoft:1.2} that Modern Danish certainly has a semantically coded subject position. 

I have claimed elsewhere (\citealt{Heltoft2003, Heltoft2003, Heltoft2011, Nørgård-Sørensen2015}) that word order can be paradigmatically organised. Just like with morphological paradigms, we must distinguish between the \textit{frame} of word order paradigms: the semantic content zone coded in the paradigm, and the \textit{domain} of a paradigm: the syntagmatic context where the paradigmatic contrast applies. For the old Scandinavian languages, the semantic frame of word order was not argument status, nor syntactic relations, but information structure. 

\subsection{The iconic focus pattern of Middle Danish} \label{heltoft:4.1}

Initially, all Old Scandinavian languages are verb second, but in relation to the non-finite verbs, they retain the possibility of OV-order, or more generally, XV-order, X being all types of NPs, predicatives and adverbials. The finite and the non-finite verb define three topological zones, a prefield preceding the finite verb, a middle field between the verbs and a postfield, following the non-finite verb. I illustrate this through examples of transitive constructions, namely \REF{ex:heltoft:40} showing pronominal object\,+\,V, \REF{ex:heltoft:41} showing full NP object\,+\,V and \REF{ex:heltoft:42}, pronominal object\,+\,full NP subject\,+\,V. In \REF{ex:heltoft:43}, I add an example of V\,+\,negation\,+\,subject\,+\,V, in which the object holds the initial position. 

\begin{exe}
\ex \label{ex:heltoft:40} 
    \glll Herræ … , giff thætte barn toll, at iach motte ok henne see ændæ sith liff i fulkomen troo, som iech soa myn førmer dotter.\\
         Herre … , giv thætte barn thol, at jak matte    ok   hænne se ænde sit liv i fulkomen tro   sum   jæk sa     min førmer dotter.\\
         Lord … , give this child endurance  that I could also her see    end \textsc{refl} life  in perfect faith   as     I saw     my former daughter.\\
    \glt ‘Lord …, give this child endurance so that I could see her, too, ending her life in perfect faith, like I saw my now late daughter.'  (HellKv 85, 23--25)

\ex  \label{ex:heltoft:41} 
    \glll viste thu huilc myn hug ær, thu hafde tesse ord icke melth\\
         viste thu   hvilk min hugh ær,   thu     havthe thæsse orth   ække mælt\\
         knew \textsc{2sg}  what my mind is,   \textsc{2sg}   had   these words   not uttered\\
    \glt ‘If you knew what I have in mind, you would not have uttered these words.' (HellKv. 76, 22--23)

\ex \label{ex:heltoft:42} 
    \glll thæræ ma han hwærkin kunung nøthæ til oc ængin landz ræt\\
         thære   ma   han    hværken  kunung   nøthe   til     oc ængen   lands ræt\\
         there  may  he.\textsc{acc}   neither  king    coerce  to    and no    land's law\\
    \glt ‘To do this neither the king nor any law of the land may coerce him.' (DgL V 75, 6--8)

\ex \label{ex:heltoft:43}
    \glll Thænnæ steen ma æi eld skathæ\\
         Thænne sten   ma   æj     eld   skathe \\
         this gem    can    not    fire    harm \\
    \glt ‘Not even fire can harm this gem.' (Harpestreng 191,13--14)
\end{exe}

These examples document two points: 1) Focus operators such as \textit{ække} ‘not', \textit{æj} ‘not', \textit{ok} ‘also' and \textit{hværken} ‘neither' define information structural subzones, a background zone preceding the operator and a focus zone following it. 2) There is no specific subject position, and like objects, a subject can be in focus position. If there is no operator, the system predicts that an object or adverbial will precede a focused subject. The relevant portions of text can be laid out topologically as in \tabref{tab:heltoft:8}.


\begin{table}
    \caption{Information structure and word order in Middle Danish\label{tab:heltoft:8}}
    \fittable{\begin{tabular}{lllllll}
        \lsptoprule
        {Prefield} & {V}  &  \multicolumn{3}{c}{Middle field}  & {V}  &  {Postfield}   \\\cmidrule(lr){3-5}
        & {V}  &  {Background}  &  {Operator} & {Focus} & {V}    & \\\midrule
        jak  & matte   &    & ok & \textit{hænne} & se & ænde sit liv\\
        thu & havthe & \textit{thæsse orth}\textsubscript{A2} & ække & & mælt & \\
       thære  & ma &  \textit{han}\textsubscript{A2} &  hværken &  \textit{kunung}\textsubscript{A1}  &  nøthe   &  til …\\
       thænne sten\textsubscript{A2}& ma & &  æj & \textit{eld}\textsubscript{A1} & skathe & \\
        \lspbottomrule
    \end{tabular}}
\end{table}



In Peircean terms, the finite and the nonfinite verb indicate the middle field, the zone for word order to manifest symbolic information structural meaning, the opposition of \textit{background} versus \textit{focus}. In analogy with morphological paradigms, a given member cannot manifest both meanings; however, this symbolic paradigmatic contrast must be mapped onto a linear sequence, and this iconic sequence (\citealt{Heltoft2019, Heltoft2003}) is then indicated again by the position of focus operators. The indexical function of verbs and focus operators define the domain of the paradigm.

\begin{figure}
\centering
  {V} ${\Rightarrow}$  [\hspace{3.1cm} {Middle field} \hspace{3.1cm}] ${\Leftarrow}$  {V}\\
  {V}  ${\Rightarrow}$ {[Background positions}  ${\Leftarrow}$  {Operator} ${\Rightarrow}$   {Focus position]}  ${\Leftarrow}$    {V}
  \caption{Topological analysis of inactive clauses\label{fig:heltoft:2}}
\end{figure}    


Notice that there is no coded subject position. Subjects can occur anywhere in a clause, depending on the textual organisation. Again, what is structurally possible -- not what is frequent -- defines what is grammaticalised. No doubt, subjects in the 3\textsuperscript{rd} position, immediately after the finite verb, have an overwhelmingly high frequency, but this fact can in all probability be derived from the fact that the A1s of the transitive system are very often lexically coded as Agents. At any rate, there is obviously no interlock between A1, 3\textsuperscript{rd} position and subject, so the Middle Danish subject is clearly not topologically coded.

\subsection{Inactive clauses follow the general pattern} \label{heltoft:4.2}

In this subsection, I consider a number of examples illustrating the positional range of A1s and A2s. Since the domain of the paradigm is the middle field, special attention will be given to examples where both A1 and A2 are in this field. Example \REF{ex:heltoft:44} documents that A1 can hold the third position, A2 holding the open initial position; and vice versa, \REF{ex:heltoft:19}, partly repeated here as \REF{ex:heltoft:45}, documents initial A1 and 3\textsuperscript{rd} positional A2.   

\begin{exe}
\ex \label{ex:heltoft:44}
    \glll hon thæktis honum migit væl \\
    hun\textsubscript{A2}    thæk-t-es        hon-um\textsubscript{A1}  miket væl\\
    she.\textsc{nom}  please-\textsc{prt-middle}  he-\textsc{obl}    very much  \\
    \glt ‘She pleased him very much.' (SjT 53, 17)

\ex \label{ex:heltoft:45}
    \gll han-um\textsubscript{A1}   thækk-er     thæt\textsubscript{A2}   [at han varther forsmath   thær af fore værden.]\textsubscript{A2}\\
    he-\textsc{obl}    learn-\textsc{3prs.sg}  that    that he becomes despised there from for {the world}\\
    \glt ‘He learns he is despised for this by the world.'
\end{exe}

\REF{ex:heltoft:46} documents A2 preceding A1 in the middle field, (\ref{ex:heltoft:22}--\ref{ex:heltoft:23}) likewise, see above. For A1 preceding A2 in this context, see \REF{ex:heltoft:14}.

\ea \label{ex:heltoft:46}
\ea 
    \gll… ganghe vthen kiortell, giærne wille iek giffwe tik myn enesthe kiortell.\\
     … go without tunic, gladly would I give you my only tunic.\\
    \glt ‘… to be without a tunic. I would be glad to give you my only tunic.'
\ex 
    \glll Nw  sømmer thet mik icke oc jek {kan ey} fanghe noghet andhet klædhe \\
        Nu    søm-er      thæt\textsubscript{A2}   mik\textsubscript{A1}     ække  ok    jæk {kan æj}  fange noket    annet  klæthe.\\
        now  befit-\textsc{prs.sg}  it.\textsc{nom}    me.\textsc{obl}  not    and   I   cannot   get any     other   garment.\\
    \glt ‘Now this (anaphor = ‘wearing no tunic') does not befit me and I cannot get any other garment.' (ML 407, 7--10)
\z\z

\begin{exe}
\ex  \label{ex:heltoft:47}
    \glll Theth ær æy megheth ath wel omgonges meth sakthmodugh ok gode meniske; nattrulige tha tækkes theth alle\\
    Thæt ær æj miket    at væl umganges mæth   saktmodugh   ok     gothe mænneske; naturlike tha  thækk-es        thæt\textsubscript{A2} alle\textsubscript{A1}\\
    it is not much      to well {get along} with    meek       and   good {human beings}; {in a natural way}  then please-\textsc{prs.sg.middle}   this    all\\
    \glt ‘Getting along well with meek and good human beings is not much;... in a natural way, this is what all people like.' (Kempis 58, 14)

  \ex  \label{ex:heltoft:48}
    \gll hvat varthar    thæt\textsubscript{A2}   [mik     æller  thik]\textsubscript{A1}   [at  the       hav-a æj vin]\textsubscript{A2} thæt\textsubscript{A2} varthar    thæm\textsubscript{A1}  sum os     hava  buthit ok æj    os [at the hava æj vin.]\textsubscript{A2}\\
    what concerns it.\textsc{nom}   me.\textsc{obl} or    \textsc{2sg.obl}  that they.\textsc{nom.pl} have-\textsc{pl}  not wine it.\textsc{nom} concerns  \textsc{3pl.obl}  \textsc{rel}  us.\textsc{obl}  have   asked and not   us.\textsc{obl} [that they have not wine]\\
    \glt ‘How does it concern me or you that they have no wine? It concerns those who have invited us, and not us, that they have no wine.'

\ex   \label{ex:heltoft:49}
    \glll  Teckes ether naadhe, att theres egett budth skall føræ breffwet fræm tiill thee Lubskæ, thaa staar thet i ether naades hendher; tæckes ether naade ickæ thet, tha haffwe wii …\\
    thækk-es   [ither nathe]\textsubscript{A1} [at theres eghet buth* skal føre brevet frem til the lybske]\textsubscript{A2} tha star thæt i ither nathes hænder; thækkes    [ither  nathe]\textsubscript{A1} ække    thæt\textsubscript{A2}   tha    have   vi …\\
    please-\textsc{prs.sg.middle}  \textsc{poss.2pl} grace that their own messenger should bring {the letter} forward to the {people of Lübeck}, then is that in Your Grace's hands; pleases     your grace     not     this,    then   have  we …\\
    \glt ‘If it pleases Your Grace that their own messenger should bring the letter forward to the people of Lübeck, then this is in Your Grace's hands; if Your Grace is not satisfied with this, then we have (…)' (29/2 1512 (Halmstad; AarsberGeh VI, Till. 13)) [*The Swedes' own royal courier, whether he should be granted transfer through Denmark.]\\
\end{exe}


A template including these examples is \tabref{tab:heltoft:9}. Examples \REF{ex:heltoft:22} and \REF{ex:heltoft:46} have both arguments in background position, the focus being on the adverb \textit{mæst} ‘most' in \REF{ex:heltoft:22} and on the verb \textit{sømer} ‘is decent' in \REF{ex:heltoft:46}; examples \REF{ex:heltoft:23}, \REF{ex:heltoft:47}, and \REF{ex:heltoft:48} have their A1s in focus position, but \REF{ex:heltoft:49}, by contrast, has the A2 in focus position.

  The logic behind this does not include argument hierarchy or grammatical relations, but the middle field contains a purely topological grammaticalised system, consisting of focus and non-focus (background) positions, indexically identifiable through the position of the focus operators, esp. negation. Examples \REF{ex:heltoft:46} and \REF{ex:heltoft:49} both contain the pronoun \textit{thæt} ‘that', in \REF{ex:heltoft:46} in background position, in \REF{ex:heltoft:49} in focus position. The paradigm's coded contrast is between background and focus, since a linguistic element cannot have both of these information structural values at a time. In this type of paradigm, the contrast is mapped onto the syntagmatic axis, that is: onto word order, see further \citet{Heltoft2019}. The system works without any assumptions of grammaticalised connections between topology (word order) on one hand and case morphology, argument hierarchy and subject-object articulation on the other. One could say this type of topological system is neutral with respect to transitivity and inactivity.

For a final argument, notice that in example \REF{ex:heltoft:2}, included in \tabref{tab:heltoft:9}, the constituent in focus position is the deictic adverbial \textit{nu} ‘now'. The examples have mainly been of objects and subjects, but this position is also open to adverbials, should they be intended as the focused constituent.

\begin{table}
    \caption{The topological frame for Middle Danish\label{tab:heltoft:9}}
    \fittable{\begin{tabular}{llllllll}
        \lsptoprule
        {Prefield} & {V}  &  \multicolumn{3}{c}{{Middle field}}  &  {Postfield}   \\\cmidrule(lr){3-5}
       {1.Pos} & {V}  &  {Background}  &  {Focus Op.} & {Focus} &  \\\midrule
         hvi   &  angrer   &  thik\textsubscript{A1} &  æj  & nu  &  [at… & \REF{ex:heltoft:2}\\
         Nu  &  sømer  &  thæt\textsubscript{A2} mik\textsubscript{A1} & ække & &  & \REF{ex:heltoft:46}\\
            & thækkes  & ether nathe\textsubscript{A1}  &  ække  &  thæt\textsubscript{A2} &  & \REF{ex:heltoft:49}\\
        Hvat  &  varther &  thæt\textsubscript{A2}   & & [mik æller tik]\textsubscript{A1} & at… & \REF{ex:heltoft:48}\\
        Thæt\textsubscript{A2} & varthar & & &  thæm\textsubscript{A1} sum (…) & & \REF{ex:heltoft:48}\\
        (thæt\textsubscript{A2} & varthar)  & & æj   &  os\textsubscript{A1} &   &   \REF{ex:heltoft:48}\\
        hun\textsubscript{A2 nom} & thæktes & honum\textsubscript{A1 obl} &  & miket væl & & \REF{ex:heltoft:44}\\
        i gardagh & thæktes & thu\textsubscript{A2 nom} mik\textsubscript{A1 obl} & &    mæst & & \REF{ex:heltoft:22}\\
        thær fore & thæktes &  hun\textsubscript{A2 nom}  & & gudh\textsubscript{A1} &   &  \REF{ex:heltoft:23} \\
        tha  &   thækkes &  thæt\textsubscript{A2}  & & alle\textsubscript{A1} & &   \REF{ex:heltoft:47}\\ 
         & tethes & &  ække  &  ræghnbughe\textsubscript{A2} … &  & \REF{ex:heltoft:15}\\
        tha  & tethes  & &  e & noket tekn\textsubscript{A2} & & \REF{ex:heltoft:16}\\ 
        \lspbottomrule
    \end{tabular}}
\end{table}    

I have added examples (\ref{ex:heltoft:15}--\ref{ex:heltoft:16}), in order to add to the number of subject A2s definitely not in the 3\textsuperscript{rd} position.

\section{Categorical clause structure and the loss of indexical case} \label{heltoft:5}

During the period app. 1400--1750 the inactive construction was reinterpreted as transitive constructions, including a shift in case marking aligning the relationship between arguments, grammatical relations and case selection. This actualisation process must be the topic of another study, and I will just give two examples by the same author, the lutheran bishop Palladius:

\begin{exe}   
\ex   \label{ex:heltoft:51}
    \glll derfor bør i at haffue denne sted och kirke kierist offuer alle andre steder i verden\\
          derfor    bør i  at have  denne sted og kirke kjærest over alle andre steder i verden \\
          therefore  ought.\textsc{prs.sg}  \textsc{2pl.nom}  to hold   this place and church dearest beyond all other places in {the world}\\
    \glt ‘Therefore, you ought to hold this place and church dearer than anywhere else in this world.' (Palladius 38, 18--19)
\end{exe}

In the very same text, we find the older construction in a sentence otherwise identical:  

\begin{exe}
\ex \label{ex:heltoft:52}
    \glll derfor bør eder att haffue denne sted kierist\\
          derfor bør eder at have denne sted kjærest \\
          therefore  ought.\textsc{prs.sg}  \textsc{2pl.obl}   to love this place {the most}\\
    \glt ‘Therefore, you ought to love this place the most.' (Palladius 39, 14)
\end{exe}

\begin{sloppypar}
The use of the nominative as the marker of the subject-predicate was abolished during the 16\textsuperscript{th} century. For the details of the distribution of case in this post-medieval period, see especially \citet{Jensen2017, Jensen2018}, with supplementary overviews and details in \citet{Heltoft2019} and \citet{Jørgensen2000}.
\end{sloppypar}

\subsection{Modern subject topology} \label{heltoft:5.1}
Returning to Melčuk's subject criteria, the difference between the medieval system and the modern one is striking. The modern language has a subject definable along parameters 1--3, and no longer by morphological binding by the finite verb. The subject is the only obligatory DP-constituent, in the sense that its positions must be filled in, if not by a referential DP, then by a formal marker (\textit{det} ‘it' or a deictic marker \textit{der} ‘there' or \textit{her} ‘here'), to facilitate the illocutionary system. A feature not mentioned by Melčuk is the interdependence (\textit{catataxis}, in Hjelmslevian terms, \textit{exocentrism} in Bloomfield's) between finite verb and subject, a relational type and criterion normally disregarded in modern grammatical theories and schools. In contrast to the predicate valence system of the medieval language, these relations are solely between grammatical categories, thus defining clausal structure as subject vs. predicate (in the wider sense), so-called categorical sense structure, the presumedly universal DP-VP dichotomy. This structure is again mirrored in the modern sentence frame, in which the middle field has lost its positions for objects and valence bound PPs. These go into the postfield, mirroring the VP, subject positions illustrated by (\ref{ex:heltoft:53}--\ref{ex:heltoft:54b}), a next to translation of \REF{ex:heltoft:43}. The focus operator \textit{selv} is inserted to match better the meaning of \REF{ex:heltoft:43}, but it may well be let out.

\begin{exe}
\ex[]{\label{ex:heltoft:53}
    \gll Denne ædelsten kan \textit{selv} \textit{ild} jo ikke skade \hfill{(subject in 3\textsuperscript{rd} pos.)}\\
    This gem can even fire \textsc{part} not harm  \\
    \glt ‘This gem even fire -- for sure -- cannot harm.'}

\ex[*]{Denne ædelsten kan jo ikke selv ild skade \hfill{(no focus position)} \label{ex:heltoft:54a}}
\ex[]{Selv ild kan jo ikke skade denne ædelsten \hfill{(subject in initial pos.)} \label{ex:heltoft:54b}}
\end{exe}  

The topological frame of Modern Danish \citep{Diderichsen1946,Heltoft1992} mirrors categorical clause structure, in that the middle field contains the definitional subject position and the postfield the non-finite verb and the rest of the valence bound constituents.

\begin{table}
    \caption{The topological frame for Modern Danish\label{tab:heltoft:10}}
    \begin{tabular}{llllllll}
        \lsptoprule
        {Prefield} & {V}  &  \multicolumn{3}{c}{Middle field}  & {V}  &  \multicolumn{2}{c}{Postfield}   \\\cmidrule(lr){3-5}\cmidrule(lr){7-8}
        {1.pos} & {V} & {Subject} & {S-advb.} & {Focus Op.} & {V} & {IO DO ...}\\\midrule
        Denne  &  kan & selv ild & jo  & ikke  &  skade  & \\
        \hspace{1ex} ædelsten & \\
        Selv ild   &   kan   & &     jo   &   ikke   &  skade  &  denne\\
                   &         & &          &          &         &  \hspace{1ex}ædelsten\\
        \lspbottomrule
    \end{tabular}
\end{table}


The modern word order paradigm and the role of the subject in this paradigm was mentioned in \sectref{heltoft:1.2}. Given the present preconditions of the analysis, the Middle Danish has nothing similar, and there is no cogent reason to assume any underlying categorical structure.

\subsection{From indexical case to symbolic case} \label{heltoft:5.2}

In symbolic case systems, case forms are self-dependent in the sense that their meaning can be identified on the basis of the case sign itself. When the indexical case system was lost with the inactive constructional alternation to transitivity, the nominative form (still only in the same handful of pronouns as before, see \tabref{tab:heltoft:2}), lost its polysemy and could no longer carry semantic role meaning. It was left with the sole content of manifesting the subject function, in the sense of the argument that the VP is predicated about, the categorical subject. The Modern Danish nominative case has the content ‘subject' in all contexts\footnote{‘All contexts' refer to all uses as 1\textsuperscript{st} rank constituents as heads. I take examples such as the following to be 2\textsuperscript{nd} rank constituents : Ham (obl) \textit{og Peter kommer forbi i dag} ‘him and Peter will pop by today'; \textit{det er svært for mor og} jeg (nom) (lit. ‘this is difficult for mummy and I').}. Indexical case systems call for reference to the governing verb and the constructional level of argument hierarchy in order to resolve the polysemy of the case forms and identify the referents. The paradigmatic \tabref{tab:heltoft:11} shows modern case meaning, see \citet{Heltoft2021b} for more detail.

\begin{table}
\caption{Danish symbolic case paradigm in categorical sentence structure\label{tab:heltoft:11}}
\begin{tabular}{lll}
\lsptoprule
{Expression} & {Nominative} & {Oblique} \\
\midrule
{Content}  & {Subject}  & {Non-subject} \\
& {(marked)} & {(unmarked)}\\
\lspbottomrule
\end{tabular}
\end{table}

Notice that where the medieval transitive pattern had an unmarked nominative in relation to the relevant zones of semantic roles, the shift of content function leads to the reverse relation of a marked subject meaning in contrast to the non-subject function of the oblique case. In modern times, from app. 1900 forwards, the meaning of the nominative specialises even more, so that except for some formal registers, the nominative now means ‘anaphoric subject' \citep{Hansen1967}. The oblique form is used in subjects with all kinds of restrictive modifiers contributing to the identifiability of the subject referent, such as (\ref{ex:heltoft:55}--\ref{ex:heltoft:56}). 

\begin{exe}
\ex \label{ex:heltoft:55}
    \gll ham der       er pusher\\
         he.\textsc{obl} there  is {a pusher}\\
    \glt ‘The guy there is a pusher.'

\ex \label{ex:heltoft:56}
    \gll hende     Marie  er sød, ikke?\\
         she.\textsc{obl}  Marie  er sød, ikke\\
    \glt ‘This Marie is sweet, isn't she?'
\end{exe}

Thus, within the frame of the symbolic case paradigm the nominative has again specialised, the oblique form ‘bleached', see \tabref{tab:heltoft:12}. 


\begin{table}
\caption{Danish symbolic case paradigm adding phoricity to its content\label{tab:heltoft:12}}
\begin{tabular}{lll}
\lsptoprule
{Expression} & {Nominative}  & {Oblique}\\
\midrule
{Content} & {Anaphoric} & {Non-anaphoric subjects} \\
& {subjects} &  {All non-subjects}\\
\lspbottomrule
\end{tabular}
\end{table}


\subsection{Positions indicate roles and arguments} \label{heltoft:5.3}

In the modern language, arguments and grammatical relations have been aligned, so that all A1s are subjects and all A2s are direct objects. The indexicality of the case category in relation to the predicate is gone, the general rule being that all A1s are subjects and prototypical  subjects – whatever the predicate's semantics – are in the nominative. The topological system has changed from a more open and free information structural system to a case-like system with specific positions for the subject, the direct object (and in fact, for the indirect object as well). In this system, the subject position indicates the predicate as the category and stem determining the A1 and its meaning, the direct object position indicates the A2, and the indirect object position the A3,\footnote{For a detailed analysis of the shift from symbolic to indexical function in the topology of the indirect object, see \citet{NielsenHeltoft2020}.} with its more specific semantic role (Recipient). 

The dimensions of linear position, case meaning and syntactic hierarchy have been aligned as definitional criteria for the identification of subject and A1 in the modern language. 

\section{Conclusion} \label{heltoft:6}

Middle Danish with its very reduced case system still retains the indexical character of Germanic case. In spite of the case system's simple morphological expression side, its content side is very complex. Both cases, nominative and oblique, differ in meaning, depending on their constructional context: inactive and transitive constructions, and these constructions and their case differences are distinguished indexically. The predicate's stem must be checked in order to identify the relevant contextual variety (bound variant) of the case forms. 

Grammatical relations (subjects and objects) were not aligned in Middle Danish (or in the Norse languages in general). The core actant A1 is the subject of transitive and intransitive constructions, but the object of inactive constructions, one-place or two-place. Both types are found with an additional A2, the oblique direct object of transitives, but a nominative subject in the inactive construction. Case assigns semantic roles according to the semantics of the predicate and the constructional pattern.

When the inactive construction was finally lost during the 18\textsuperscript{th} century, the case paradigm also lost semantic roles as its content frame. In present-day Danish, the case system has turned symbolic, in that they code directly the relevant grammatical roles and argument status. Now, the nominative case in itself marks its status as subject and A1, the oblique form – now the unmarked form – has roughly the content non-subject and non-A1. 

Topology (word order) has taken over the indexical function the case system had, but in a simpler version with no systematic polysemy. Positions, not case forms, point to the predicate stem. 


\section*{Acknowledgements}

This work is also a part of the project \textit{The Middle Danish language in the light of a modern theory of grammaticalisation}, grant DFF-7013-00045 from The Danish Council for Independent Research.

\section*{Sources} \label{heltoft:sources}

\begin{description}[font=\normalfont]\sloppy
\item[AarsberGeh:]   \textit{Aarsberetninger fra Det kgl. Geheimearchiv} I-VII, ed. by Wegener. 1852--1882.
\item[AM 187, 8\textsuperscript{o}:] Det arnamagnæanske håndskrift nr. 187 i oktav, indeholdende en dansk lægebog. Ed. by Viggo Såby. Universitets-Jubilæets danske Samfund. Copenhagen: Thieles Bogtrykkeri. 1883.
\item[AM 28, 8\textsuperscript{o}:]  Det arnamagnæske Haandskriftet Nr. 28, 8vo, \textit{Codex Runicus}, udgivet i fotolitografisk Aftryk af Kommissionen for det Arnamægnæanske Legat. Copenhagen: Gyldendalske boghandel. 1877.
\item[Bønneb:] \textit{Middelalderens danske Bønnebøger} I-V, ed. by Karl Martin Nielsen. Copenhagen: Gyldendalske Boghandel/Nordisk Forlag. 1945--1982.
\item[CCD:]   \textit{Corpus Codicum Danicorum Medii Aevi}. CCD III. \textit{Lex Scaniae}, codd. B74 and GkS3121, ed. by Johannes Brøndum-Nielsen; CCD X. \textit{Lex Jutiae}, codd. C37 and C39, ed. by Peter Skautrup. Copenhagen: Munksgaard. 1961 and 1973.
\item[DgL V:]  \textit{Danmarks gamle Landskabslove} V, \textit{Eriks Sjællandske Lov}, ed. by Peter Skautrup. Copenhagen 1936.
\item[Fragm:]  \textit{Fragmenter af gammeldanske Haandskrifter}, ed. by Paul Diderichsen. Copenhagen: Schulz. 1931--1937.
\item[GldO:]  An electronically accessible, non-digitalised collection of excerpts for Gammeldansk Ordbog [Old Danish Dictionary]. \url{https://gammeldanskseddelsamling.dk/}
\item[Harpestreng:]  \textit{Harpestræng. Gamle danske Urtebøger, Stenbøger og Kogebøger}. Ed. by Marius Kristensen. Copenhagen 1908--1920.
\item[\textit{HeiðrR}:]  \textit{Heiðreks saga}, ed. by Jón Helgason. Selskab til Udgivelse af gammel nordisk Litteratur 48. Copenhagen 1924.
\item[HellKv:]  \textit{De hellige Kvinder}. \textit{En Legende-Samling}. Ed. by C.J. Brandt. Copenhagen 1859.
\item[JBB:]  \textit{Jesu Barndoms Bog}. Ghemen ca. 1508. \textit{Danske Folkebøger} 1, Apokryfe Bibelhistorier, ed. by J.P. Jacobsen, 27--105. Copenhagen: Gyldendalske Boghandel Nordisk Forlag. 1915. Electronic edition: \url{https://tekstnet.dk/jesu-barndomsbog/1}.  Det Danske Sprog- og Litteraturselskab.
\item[Kempis:]  \textit{Thomas a Kempis}. \textit{Fire bøger om Kristi Efterfølgelse}, i dansk oversættelse fra 15. Århundrede, ed. by F. Rønning. Copenhagen 1884--1885.
\item[ClarB:]    \textit{Clarus Saga}\slash\textit{Clari fabella}, ed. and transl. by G. Cederschiöld (ed. \& transl.). Festskrift till Kgl. Universitetet i Köpenhamn vid dess fyrahundra års jubileum i juni 1879 från Kgl. Carolinska Universitetet i Lund 1, Lund 1879.
\item[Luc:]  Lucidarius. In \textit{A Danish Teacher's Manual from the Mid-Fifteenth Century} (COD. AM 76, 8°). Ed. by S. Kroon et al. Lund: Lund University Press. 1993.
\item[Lægeb Thott:]  \textit{En middelalderlig dansk Lægebog}. Ed. by Poul Hauberg. Copenhagen 1927.
\item[Mandevilles Rejse i gammeldansk Oversættelse,] ed. by M. Lorenzen. Copenhagen 1882. 
\item[Miss:]  \textit{Missiver fra Chr. I's og Hans' Tid I-II}, ved W. Christensen. 1912--1914.
\item[ML:]  \textit{Mariager Legendehåndskrift}, ed. by Gunnar Knudsen, for Samfund til Udgivelse af gammel nordisk Litteratur. Copenhagen 1917--1930.
\item[ONP:]  \textit{Ordbog over det Norrøne Prosasprog} / \textit{Old Norse Prose Dictionary}. Institut for Nordiske Studier og Sprogvidenskab. \url{https://nors.ku.dk/forskning/arnamagnaeanske-samling/onp/}. Copenhagen.
\item[Palladius:]  \textit{Visitatsbog}. Peder Palladius: \textit{Danske Skrifter} V, ed. by Lis Jacobsen. Universitets-Jubilæets danske Samfund. Copenhagen 1925.
\item[Post:]  Postil i Uppsala. In \textit{Svenska Medeltids-postillor} III, ed. by G.E. Klemming. Stockholm: P.A. Norstedts \& söner. 1873.
\item[RD:]  \textit{Romantisk Digtning fra Middelalderen} I-III, ed. by C. J. Brandt. Copenhagen 1869--1877. 
\item[SjT:]  \textit{Sjælens Trøst}. Consolatio animae, ed. by Niels Nielsen. Universitets-Jubilæets danske Samfund. Copenhagen 1937--1952.
\item[Sydr:]  \textit{Sydrak}. Efter haandskriftet Ny kgl. Samling 236, 4to, ed. by Gunnar Knudsen. Universitets-Jubilæets danske Samfund. Copenhagen 1921--1932.
\item[Tacitus Germ.:]   \textit{Tacitus} I. \textit{Germania}. Translated by M. Hutton, rev. by E.H. Warmington. Loeb Classical Library 35. 1970[1914].
\item[Thord Degn text 2:]  \textit{Danmarks gamle Landskabslove} IV. \textit{Tillæg til Bind IV}. \textit{Thord Degns Artikler}, Text 2.
\item[VejlPilgr:]   [A Pilgrim's Manual], from \textit{Mandevilles Rejser}, ed. by M. Lorenzen, 207--225.
\end{description}


{\sloppy\printbibliography[heading=subbibliography,notkeyword=this]}
\end{document} 
