\documentclass[output=paper]{langsci/langscibook} 
\ChapterDOI{10.5281/zenodo.5675847}
\author{Tabea Reiner\affiliation{LMU München}}
\title{Recursion and paradigms}
\abstract{This paper sketches the current status of morphology and paradigms in linguistic theorising. In particular, it is shown that from a constructionist as well as from a compositional perspective, morphology including paradigms tends to dissolve. The former might be less obvious; however the paper argues that a constructional deconstruction of paradigms and morphology follows directly from \citegen{Haspelmath2011} take on \citet{Booij2010} and related approaches in the realm of \textup{Construction Morphology (CxM)}. The latter is more obvious; in particular, proponents of \textup{Distributed Morphology (DM)} regularly emphasise that morphology is but an interface and paradigms are epiphenomenal. Throughout the paper I assume some familiarity with Construction Morphology and Construction Grammar more generally whereas I introduce DM specifically. However, the paper is not intended as a thorough discussion of the approaches presented (nor do I take sides); rather it is their shared detachment from paradigms that is at stake here. Consequently, also what is sometimes called \textup{Autonomous Morphology} is addressed in the paper: a rather recent approach that advocates morphology as an irreducible level of description and upholds the paradigm as a format of description in its own right. The balance of the paper is rather pessimistic for morphology and paradigms but eventually I come up with a presumably new argument in favour of regarding paradigms as fundamental: restrictions on inflectional recursion fall out naturally from them.}

\begin{document}
\multicolsep=.25\baselineskip
\maketitle

\section{Introduction} \label{reiner:1}

In the last decades, morphology as an independent level of description has come under pressure from two sides. In construction-oriented approaches it runs the risk to dissolve somewhere in the middle of the lexicon-syntax continuum: “it’s constructions all the way down” \citep[18]{Goldberg2006}. In certain composition-oriented approaches, in turn, its traditional tasks have been transferred to syntax: it’s “syntactic hierarchical structure all the way down” \citep[3]{HarleyHarley1999}. The last quotation is from Distributed Morphology, which does have \textit{Morphology} in its name but refers to a mere interface between syntax and phonology by this (\citealt[114]{HalleHalle1993}, \citealt[1]{Bobaljik2017}).

Together with morphology, a certain format of description appears to be discarded, which has been central to at least what is traditionally termed inflectional morphology:\footnote{This is the kind of morphology on which the present paper focuses; however, most points before \sectref{reiner:4} carry over, mutatis mutandis, to derivational morphology: word formation can be modelled using paradigms \citep{HathoutNamer2019} but it can probably be modelled just as well using constructions or rules, including special rules. On a related note, let me add a word on terminology: in the present paper, the root \textit{deriv}{}- is sometimes used meaning ʻrelated to the formation of new wordsʼ and sometimes used meaning ʻrelated to compositionʼ (especially when presenting DM); I assume that in each case the context suffices for disambiguation.} the paradigm. Throughout this paper, \textit{paradigm} is to be understood in the following sense, unless noted otherwise:

\begin{quote}
[…] a set of \textsc{cells}; each such cell is the pairing of a word form with the set of morphosyntactic properties which that word form realizes. \citep[147]{Stump2002}
\end{quote}

I consider this definition quite general, since, in principle, it does not exclude an incremental instead of a realisational approach (for an overview of morphological theories cf. \citealt{Stewart2016}): after completion, the string can be regarded as a whole. Please note, however, that the definition given here differs from \citegen{Politt2019} contemporary concept of paradigm in two respects. First, it is narrower in that it refers to merely word forms (the notion of word is a point to be discussed below); second it is broader in that it does not take a stance on whether paradigms are part of native speaker knowledge (another point to be discussed below).

To give an impression of how the discussion in the present paper is going to proceed, I pick out two quotations from the literature, which coincide in rendering paradigms superfluous eventually. First, consider \citet{Haspelmath2011} on constructions:

\begin{quote}
Clearly, the form–meaning relationship is often straightforward and compositional, but it is also often more complex. For the latter cases, morphologists have used paradigms and realisation-based rules, and syntacticians have used constructional idioms. The similarity between realisation-based morphology and construction-based syntax has recently been emphasized especially by \citet{Gurevich2006} and \citet{Booij2010}. As far as I have been able to determine, the differences between them mostly derive from different traditions, not from any substantive differences. \citep[59]{Haspelmath2011}
\end{quote}

I conclude from this quotation that paradigms can be rewritten as constructions. Incidentally, the reverse is also true, provided that the definition of paradigm is relaxed to include all sorts of paradigmatic relations in a structuralist sense: imagine a micro-paradigm like \textit{be all ears} vs. \textit{be all eyes}. However, from a constructionist perspective, constructions are needed anyway, so it is the paradigm that one can do without. In practice, the rewriting of paradigms as constructions can be thought of as conceptualising every single cell from a paradigm as the small semiotic entity that it is. For example, \tabref{tab:reiner:1} from German can be rewritten in the manner of \figref{fig:reiner:1}.

\begin{table}
\caption{Present indicative of German \textit{sein} ʻbeʼ\label{tab:reiner:1}}
\begin{tabular}{lll}
\lsptoprule
 & Singular & Plural\\\midrule
1\textsuperscript{st} person & bin & sind\\
2\textsuperscript{nd} person & bist & seid\\
3\textsuperscript{rd} person & ist & sind\\
\lspbottomrule
\end{tabular}
\end{table}

  
\begin{figure}
  \includegraphics[width=\textwidth]{figures/Reiner1.pdf}
  \caption{Constructions instead of a paradigm\label{fig:reiner:1}}
\end{figure}


The rewriting can also be accomplished in a less space-consuming way (cf. \sectref{reiner:2.1}.); however I use a partly Saussurean way of presentation here in order to highlight the fact that every cell is a sign. Needless to say, these signs (=\,constructions) may form networks among themselves, which in turn might equal paradigms. Crucially however, paradigms are no more than emergent from this perspective.

This coincides surprisingly well with the view on paradigms following from an otherwise very different theoretical perspective, i.e. from the perspective of Distributed Morphology (henceforth referred to as \textit{DM}). Consider the following footnote in a recent paper.

\begin{quote}
[…], we use paradigms only for representational issues. As is well-known, in DM, paradigms are epiphenomenal. […] \citep[473]{PominoPomino2019}
\end{quote}

Thus, paradigms are not rejected as such but they are rejected as parts of mental grammar (which is the only area of interest in DM). The most detailed argument to that effect might be found in \citet{Bobaljik2002}. More recently, David \citet{Embick2015} even managed to write a book-long introduction to DM without even using the term \textit{paradigm}(\textit{atic}) beyond the bibliography and one footnote \citep[232]{Embick2015}. These researchers’ view on paradigms derives directly from the architecture of grammar assumed in DM, which will be laid out in \sectref{reiner:2.2} of the present paper.

Summarizing for now, the theoretical significance of paradigms may be seriously called into question from otherwise very different theoretical perspectives. The present contribution aims to portray these positions in some depth (\sectref{reiner:2}), explore a common defence strategy applied by Autonomous Morphology (\sectref{reiner:3}), and eventually come up with one task in linguistic theorising that only paradigms appear to fulfil directly, i.e. delimiting recursion (\sectref{reiner:4}). Hence, the key term \textit{recursion} will not reappear until the last section.

\section{No morphology, no paradigms} \label{reiner:2}
\subsection{From a construction-oriented perspective} \label{reiner:2.1}

If the introductory example had not been from German but from Turkish or another language with considerably less fusion and suppletion than German (\citealt[4]{AikhenvaldDixon2017}, \citealt[334]{Dressler1985} but also cf. \citealt{Bacanlı2011}), both the paradigmatic as well as the constructional notation would have displayed forms that are systematically segmentable to a large extent. So the question arises whether such forms should be described by rules rather than by paradigms or constructions. The present section aims at answering this question from a constructionist perspective. To anticipate the answer, compositional word forms are to be described by constructions like compositional syntax is to be described by constructions.

\begin{sloppypar}
For a start, consider the Turkish example to which \citet[59]{Haspelmath2011} alludes, stemming from \citet[403]{Hankamer1989}: one Turkish verb root yields 1,830,248 different forms when counting (what would traditionally be called) inflection and (what would traditionally be called) derivation, not even allowing for iterations. Obviously, this is too much to write down in paradigms. And things get worse when iterations are eventually taken into account. \REF{ex:reiner:1} provides an example.
\end{sloppypar}\largerpage[-1]

\ea \label{ex:reiner:1}
{Turkish (\citealt[396, emphasis added]{Hankamer1989})}\\
\gll daya    {}-n  {}-ış  \textit{{}-tır}  \textit{{}-t}  {}-ıl  {}-a {}-mı  {}-yabil  {}-ecek  {}-ti  {}-k\\
     prop\_up    \textsc{rfl}  \textsc{rcp}  \textit{\textsc{caus}}  \textit{\textsc{caus}}  \textsc{pass}  \textsc{pot}1 \textsc{neg}  \textsc{pot}2  \textsc{asp}  \textsc{tns}  \textsc{agr}{\footnotemark}\\
\glt ‘we might not have been able to be made to make someone else practice mutual aid’
\z
\footnotetext{Abbreviations (in order of occurrence): \textsc{rfl} – reflexive, \textsc{rcp} – reciprocal, \textsc{caus} – causative, \textsc{pass} – passive, \textsc{pot1} – first potential, \textsc{neg} – negation, \textsc{pot2} – second potential, \textsc{asp} – aspect, \textsc{tns} – tense, \textsc{agr} – agreement.}

When, starting from \REF{ex:reiner:1}, one tries to imagine all other possible combinations of affixes, this example gives an impression of the enormous size and systematicity any Turkish verbal paradigm would have. So using paradigms just does not seem to make sense for Turkish verbs.\footnote{The same holds for other categories in the language. For example, when \citet[311--320]{Kornfilt1997} presents demonstrative and interrogative pronouns, she does spell out quite a few paradigms but only in order to illustrate the high degree of systematicity with which the pronominal items are followed by separate suffixes for number and case.} Are things different when counting exclusively inflection? Yes and no. Yes, when only counting those forms that are usually considered inflectional in the language (i.e. those expressing aspect, tense, passive, mood, agreement, and negation) then the total number amounts to 576 (\citealt{Kornfilt1997}, my count), which does exceed the number for, e.g., Latin easily (up to 120, \citealt[396, my count]{Matthews1972}) but still seems to be manageable. Now, writing down the (reduced) paradigm still appears to be pointless as the realisation of the cells is largely predictable, quite different from the situation in Latin. As an example, consider the future and past forms of Turkish \textit{yapmak} ʻdoʼ in \tabref{tab:reiner:2}.\largerpage[-2]

\begin{table}
    \caption{Future and past forms of Turkish \textit{yapmak} ʻdoʼ (\citealt{Kornfilt1997}, Ch. 2.1.3)\label{tab:reiner:2}}
\begin{tabular}{lll} 
\lsptoprule
               &   {\textsc{fut}}    & {\textsc{pst}}\\\midrule
{\textsc{1sg}} &  yap-acağ-ım &    yap-tı-m\\
{\textsc{2sg}} &  yap-acak-sın &   yap-tı-n\\
{\textsc{3sg}} &  yap-acak &          yap-tı\\
{\textsc{1pl}} &  yap-acağ-ız &    yap-tı-k\\
{\textsc{2pl}} &  yap-acak-sınız &    yap-tı-nız\\
{\textsc{3pl}} &  yap-acak-lar &      yap-tı-lar\\ 
\lspbottomrule
\end{tabular}
\end{table}


The root (\textit{yap}) stays the same throughout, the tense suffixes (-\textit{AcAK} and -\textit{DI} respectively) merely undergo phonological alternations,\footnote{In particular, note that the change from <k> to <ğ>, i.e. from [k] to \textrm{∅} (with a lengthening of the previous vowel), appears to be purely phonologically conditioned \citep[13–14, 91]{Ketrez2012}.} and the agreement suffixes are clearly separable. In fact, the paradigm above may be replaced by two simple instructions, cf. \REF{ex:reiner:2}.

\ea \label{ex:reiner:2}
 {Turkish}\\
    \ea 
    \{root; \textsc{tns}:\textsc{fut}\} ${\Rightarrow}$ root-\textit{AcAK}{}-\textsc{agr}1 \\
    \ex \{root; \textsc{tns}:\textsc{pst}\} ${\Rightarrow}$ root-\textit{DI}{}-\textsc{agr}2
\z \z

Read: for expressing (an instantiation of) the meaning on the left hand side in Turkish, use (an instantiation of) the form on the right hand side. The manner of notation is equivalent to the graphical way of representation in \figref{fig:reiner:1}. Note that the instructions can be read as either rules or constructions, which is, in fact, not a contradiction \citep[123--124]{Rostila2011}. However, since I am portraying constructionist thinking here, the natural choice is for constructions.

Admittedly, the existence of more than one set of agreement forms in Turkish introduces some irregularity into the picture, especially since, in total, four such sets are posited \citep[382]{Kornfilt1997}. Kornfilt refers to these sets as paradigms; in fact, this is one of the few places in the book where she uses the term at all. Although this usage is not in line with the definition adopted in the present paper – where paradigms are not about morphemes but about whole word forms – it already indicates that the purview of what we call paradigms is the moderately irregular. This becomes more tangible when we shift from Turkish to Latin. Consider \tabref{tab:reiner:3}, which is an (approximate) translation of \tabref{tab:reiner:2}.

\begin{table}
\caption{future and past (“perfect”) forms of Latin \textit{facere} ʻdoʼ \citep[Ch. 7]{Panhuis2009}\label{tab:reiner:3}}
\begin{tabular}{lllllll}
\lsptoprule
& {\textsc{1sg}} & {\textsc{2sg}} & {\textsc{3sg}} & {\textsc{1pl}} & {\textsc{2pl}} & {\textsc{3pl}}\\
\midrule
{\textsc{fut}} & faciam & facies & faciet & faciemus & facietis & facient\\
{\textsc{pst}} & feci & fecisti & fecit & fecimus & fecistis & fecerunt\\
\lspbottomrule
\end{tabular}
\end{table}

Instead of two clearly distinguished tense suffixes we find merely one (-\textit{e} in the future) plus a root vowel change between the two tenses (\textit{fac}{}- > \textit{fec}{}-). Moreover, the future marker -\textit{e} becomes -\textit{a} in the 1\textsuperscript{st} person singular and in all future forms an -\textit{i} slips in between the root and tense. Now, this appears to be exactly the kind of situation for which paradigms have been invented in the first place: they provide an economic way to write down the unpredictable (e.g., the -\textit{i}) coupled directly with the predictable (e.g., 1\textsc{pl} -\textit{mus} but also \textit{f}V\textit{c}{}-). To this extent, paradigms are a convenient analytical tool and at the same time a concise format for instructed L2-acquisition.

However, paradigms are not the only option. In order to capture the unpredictable as well as the predictable in a very general fashion, Construction Grammarians have long developed other means, i.e. constructions. Crucially, these are not only meant for syntax but for the whole syntax-lexicon continuum, including words and even morphemes \citep[5]{Goldberg2006}. For example, the rules/constructions chosen in \REF{ex:reiner:2} above for the Turkish data from \tabref{tab:reiner:2} can be transferred to the Latin data from \tabref{tab:reiner:3}, cf. \REF{ex:reiner:3}.


\ea \label{ex:reiner:3} 
{Latin}\\ 
\ea \{ʻdoʼ; \textsc{ps}:1, \textsc{num}:\textsc{sg}, \textsc{tns}:\textsc{fut}\} ${\Rightarrow}$ \textit{fac}{}-\textit{i}{}-\textit{a}{}-\textsc{agr}1
\ex \{ʻdoʼ; \textit{not} (\textsc{ps}:1 \& \textsc{num}:\textsc{sg}), \textsc{tns}:\textsc{fut}\} ${\Rightarrow}$ \textit{fac}{}-\textit{i}{}-\textit{e}{}-\textsc{agr}1
\ex \{ʻdoʼ; \textsc{tns}:\textsc{pst}\} ${\Rightarrow}$ \textit{fec}{}-\textsc{agr}2
\z\z

Admittedly, these are three lines for one verb instead of two lines for a whole range of verbs like in \REF{ex:reiner:2}. We cannot even generalise from \textit{facere} to all fifth-conjugation verbs, since not all of them show the root-vowel change, e.g., \textit{cupere} ʻdesireʼ, \textit{fugere} ʻfleeʼ, \textit{rapere} ʻplunder, seizeʼ \citep{BennettEdwin1918, GreenoughEtAl}.\footnote{Note that the fifth conjugation is also called third-\textit{io} conjugation.} However, in principle, the notation is possible. It is like listing idioms, complete with their schematic parts. In contrast, the same kind of notation for Turkish was more like stating syntactic rules or writing down highly abstract constructions. The crucial point is that the same kind of notation is apt for both types of data.

In this sense, constructions can replace paradigms: paradigms cells \textit{are} constructions. In order to elaborate on this idea, I am going to discuss additional examples in the following paragraphs. Most of the examples are adopted from the literature referenced in \citet[58--59]{Haspelmath2011}, i.e. from \citet{Spencer2001}, \citet{Gurevich2006}, and \citet{Booij2010}. Importantly, the first one of these authors, i.e. Andrew Spencer, does uphold the paradigm as a theoretically relevant notion. He belongs to a school of thought which not only holds that there are genuinely morphological phenomena (not reducible to something else, especially syntax) but also maintains that these phenomena can be described best by using paradigms. This school of thought seems to thrive especially within the Surrey Morphology Group and is called \textit{Autonomous Morphology} here.\footnote{Accordingly, I will call the practitioners \textit{Autonomous Morphologists}, accepting the bracketing paradox. The classic reference is \citet{Aronoff1994}; later publications include \citet{MaidenEtAl2011}. Also Stump’s Paradigm Function Morphology belongs here \citep{Stump2016}; however I will not treat this theory in any detail in the present paper since this would require another introduction (in addition to the one to DM).} In any case, what \citet{Spencer2001} presents are, at the same time, realisations of paradigm cells \textit{and} constructions. One of my main aims will be to demonstrate in detail how his examples can indeed be written down as constructions; a fact, whose further theoretical consequences will be explored in \sectref{reiner:3}. By contrast, Olga Gurevich as well as Geert Booij explicitly opt for a purely constructionist approach to morphology with paradigms being merely emergent. As a whole, the following paragraphs may be read as a fleshing out of Haspelmath’s rather brief remarks on the equivalence of realisational morphology and constructionist syntax, partially quoted above in the introduction (\sectref{reiner:1}). Having said this, Haspelmath himself does not explicitly state that his observations render the traditional paradigm superfluous; rather this is the conclusion that I have drawn above (in particular with respect to the examples from Turkish and Latin) and that I will substantiate in the course of the following discussion.

First, consider some examples based on \citet{Spencer2001}, starting with an extended version of his example for cumulative exponence.

\ea \label{ex:reiner:4} 
{Spanish (based on \citealt[285]{Spencer2001} and \citealt[170--172]{ButtEtAl2019})}\\
\ea \{ʻsingʼ; \textsc{ps}:1, \textsc{num}:\textsc{sg}, \textsc{tns}:\textsc{prs}, \textsc{mood}:\textsc{indic}\} ${\Rightarrow}$ \textit{canto} \label{ex:reiner:4a}
\ex  \{ʻsingʼ; \textsc{ps}:1, \textsc{num}:\textsc{sg}, \textsc{tns}:\textsc{pst}, \textsc{mood}:\textsc{indic}\} ${\Rightarrow}$ \textit{canté} \label{ex:reiner:4b}
\ex  \{ʻsingʼ; \textsc{ps}:1, \textsc{num}:\textsc{sg}, \textsc{tns}:\textsc{prs}, \textsc{mood}:\textsc{sbjv}\} ${\Rightarrow}$ \textit{cante} \label{ex:reiner:4c}
\ex  \{ʻsingʼ; \textsc{ps}:1, \textsc{num}:\textsc{sg}, \textsc{tns}:\textsc{impf}, \textsc{mood}:\textsc{indic}\} ${\Rightarrow}$ \textit{cantaba} \label{ex:reiner:4d}
\ex  \{ʻsingʼ; \textsc{ps}:2, \textsc{num}:\textsc{sg}, \textsc{tns}:\textsc{impf}, \textsc{mood}:\textsc{indic}\} ${\Rightarrow}$ \textit{cantabas} \label{ex:reiner:4e}
\z \z

Here, the predictable part is the structure [\textit{cant}-\,+\,X] and the unpredictable part is whether and how the respective feature values are expressed cumulatively, i.e. together in one morph. For example, the -\textit{o} in context \REF{ex:reiner:4a} \textit{canto} appears to realise 1\textsuperscript{st} person, singular, present, indicative (and active) all at once, while the -\textit{abas} in \REF{ex:reiner:4e} \textit{cantabas} might be split into a thematic vowel (\nobreakdash-\textit{a}\nobreakdash-), an imperfective past tense marker for the relevant inflectional class (\nobreakdash-\textit{ba}\nobreakdash-), and an exponent of agreement (\nobreakdash-\textit{s}). So while the latter form seems to be compositional and apt for a non-constructionist morpheme-by-morpheme description, the former escapes such a description (provided that we try to avoid null elements). Here, the holistic pairing of form and meaning, i.e. the conception as a construction, presents itself as the only option as opposed to an incremental approach. Accordingly, the format of presentation chosen in \REF{ex:reiner:4} is simply the same one as in \REF{ex:reiner:2} and \REF{ex:reiner:3} above: \{\textsc{meaning}\} ${\Rightarrow}$ \textsc{form}.

Since this format is equally apt for forms with a higher degree of compositionality (cf. the discussion above), it has been chosen in \REF{ex:reiner:4} throughout, even for \REF{ex:reiner:4e} \textit{cantabas}. However, please note that on closer inspection not even this form meets the agglutinating (Turkish-style) ideal: in contrast to the other feature values, indicative mood is not signalled by a dedicated suffix but has to be inferred from the fact that \textit{cant}{}- belongs to the -\textit{ar} inflectional class, which would have -\textit{e} as its thematic vowel in the present subjunctive. So this, like \REF{ex:reiner:4a} above, is a situation we would usually describe by putting the form as a whole into a paradigm cell – while it can be captured equally well by setting up a construction.

Next is \citegen{Spencer2001} example for extended exponence, again written down as a construction here.

\ea \label{ex:reiner:5} 
{Spanish \citep[286]{Spencer2001}}\\ 
\glt \{ʻeatʼ; \textsc{ps}:1, \textsc{num}:\textsc{sg}, \textsc{tns}:\textsc{impf}, \textsc{mood}:\textsc{indic} \} ${\Rightarrow}$ \textit{comía}
\z

According to Spencer, the feature value imperfective is signalled twice within this word form: by the imperfective marker for verbs of the \nobreakdash-\textit{er} inflectional class, i.e. -\textit{í}, as well as by the \nobreakdash-\textit{a}, since the latter is a first person singular marker only in the imperfective (provided that the \nobreakdash-\textit{a} in the present subjunctive form \textit{coma} is a thematic vowel, not a person\slash number suffix). This is extended exponence: the marking of one meaning extends over more than one morph. Again, this is a situation that is a) hard to capture by an incremental approach, b) traditionally captured by drawing a paradigm, and c) equally well captured by writing down a construction.

The same is true for Spencer’s examples of zero exponence, e.g. \REF{ex:reiner:6}.

\ea \label{ex:reiner:6} 
{Latvian (\citealt{Spencer2001}:286, \citealt{FennellGelsen1980}:542)}\\ \ea
\label{ex:reiner:6a}
   \textup{\{ʻtravel, driveʼ;} \textsc{ps}\textup{:2,} \textsc{num}\textup{:}\textsc{sg}, \textsc{tns}\textup{:}\textsc{prs}\textup{\} ${\Rightarrow}$} \textit{\textbf{brauc}}\\
\ex \label{ex:reiner:6b} \textup{\{ʻtravel, driveʼ;} \textsc{ps}\textup{:2,} \textsc{num}\textup{:}\textsc{sg}, \textsc{tns}\textup{:}\textsc{pst}\textup{\} ${\Rightarrow}$} \textit{brauci}\\
\ex \label{ex:reiner:6c} \{ʻtravel, driveʼ; \textsc{ps}:2, \textsc{num}:\textsc{pl}, \textsc{tns}:\textsc{pst}\} ${\Rightarrow}$ \textit{braucat}\\
\ex \label{ex:reiner:6d}  \textup{\{ʻtravel, driveʼ;} \textsc{ps}\textup{:3,} \textsc{num}\textup{:}\textsc{sg}, \textsc{tns}\textup{:}\textsc{prs}\textup{\} ${\Rightarrow}$} \textit{\textbf{brauc}}\\
\ex \label{ex:reiner:6e} \{ʻtravel, driveʼ; \textsc{ps}:3, \textsc{num}:\textsc{sg}, \textsc{tns}:\textsc{pst}\} ${\Rightarrow}$ \textit{brauca}\\
\ex \label{ex:reiner:6f}  \textup{\{ʻtravel, driveʼ;} \textsc{ps}\textup{:3,} \textsc{num}\textup{:}\textsc{pl}, \textsc{tns}\textup{:}\textsc{prs}\textup{\} ${\Rightarrow}$} \textit{\textbf{brauc}}\\
\ex \label{ex:reiner:6g} \{ʻtravel, driveʼ; \textsc{imp}\} ${\Rightarrow}$ \textbf{\textit{brauc}}\\
\z\z


What can be seen in \textit{brauc} is essentially what you have to expect for any construction due to its nature as a linguistic sign: polysemy. Compare \REF{ex:reiner:7} to \REF{ex:reiner:8}.

\begin{exe}
   \ex \label{ex:reiner:7} \textit{brauc} ${\Rightarrow}$ \{ʻtravelʼ; (\textsc{ps}:2 \textit{or} 3, \textsc{num}:\textsc{sg}, \textsc{tns}:\textsc{prs)} \textit{or} (\textsc{ps}:3, \textsc{num}:\textsc{pl}, \textsc{tns}:\textsc{prs)} \textit{or} \textsc{imp}\}
    \ex \label{ex:reiner:8} \textit{drive} ${\Rightarrow}$ \{ʻoperate a vehicle \textit{or} motivate the processʼ\}\\ \citep[38]{RumshiskyBatiukova2008}
\end{exe}

To be sure, ʻ[…] 2\textsuperscript{nd} \textit{or} 3\textsuperscript{rd} person […]ʼ and ʻoperate a vehicle \textit{or} motivate the processʼ represent very different kinds of meanings (in constructionist terms: a finite verb represents a partly schematic construction while a lexical entry represents a substantive construction, \citealt[2]{HoffmannHoffmann2013}). However, the underspecification in both examples can be neatly captured by writing them down as constructions. So, ironically, a constructionist analysis can do something that also DM strives for (albeit in a different way, cf. \sectref{reiner:2.2}): treating syncretisms\footnote{Syncretism = in a given context two feature values are not overtly distinguished although they are overtly distinguished in another context in the same language (adopted from \citealt[96]{Kramer2016}).\label{fn:reiner:7}} as cases of underspecification.

Please note that, strictly speaking, the constructions presented so far are only halves. For being full constructions, they would need a double arrow, signalling that not only the respective meaning triggers the respective form but also the other way round. However, against the background of syncretisms as treated above it is clear that the back arrow would require a more complete picture of the languages at hand than can be given here. For example, after ensuring that through the entire verbal paradigm of Spanish the form \textit{canto} (without stress on the final vowel) is really only 1\textsuperscript{st} person, singular, present, indicative (and active), \REF{ex:reiner:4a} could be rewritten as \REF{ex:reiner:9}.

\ea \label{ex:reiner:9} 
{Spanish (\citealt[170--712]{ButtEtAl2019})}\\
\textup{\{ʻsingʼ;} \textsc{ps}\textup{:1,} \textsc{num}\textup{:}\textsc{sg}, \textsc{tns}\textup{:}\textsc{prs}, \textsc{mood}\textup{:}\textsc{indic}\textup{\} $\leftrightarrow $} canto
\z 

Let me add another word on modes of presentation. When I refrain from adding morpheme boundaries and associating the resulting units with individual meanings I take the following passage from \citet{Spencer2001} seriously.


\begin{quote}
[…], in a sense it’s a mistake to speak of meanings being concentrated in one morph or spread across several morphs or realized by zero morphs. \citep[287]{Spencer2001}
\end{quote}

So in my running text wordings like “the \nobreakdash-\textit{o} in context \REF{ex:reiner:4} \textit{canto} appears to realise […]” should be taken with a grain of salt: from the constructionist perspective, which I am portraying here, it is not necessarily the \nobreakdash-\textit{o}, not even the \nobreakdash-\textit{o} in a certain context, but \textit{canto} as a whole that has some meaning to begin with. Similarly, if the instructions for Turkish in \REF{ex:reiner:2} above are read as constructions, the hyphens indicate what have turned out to be internal semantic entities (probably aligning with internal distributional entities); however neither Autonomous Morphology (based on paradigms) nor Construction Grammar (with emergent paradigms) depends on the presence of any semantic parts below the level of the word form.\footnote{This does not only hold for inflection but also for word formation, cf. \citet[428]{Booij2016}.}

Before turning to his main example – auxiliary structures in Slavic – \citet[287]{Spencer2001} mentions a last general group of examples in favour of the realisational approach to morphology: meaningless morphemes. More specifically, he judges Spanish \nobreakdash-\textit{ar} to be a case in point: in one class of verbs this element appears after the root in all three, the infinitive, the future, and the conditional. I am not sure whether we are rather dealing with a case of polysemy rather than meaning\textit{less}ness here. However, a more obvious example may be found in the diachrony of German vs. Dutch:

\ea \label{ex:reiner:10} 
{From Middle High German to New High German \citep[200–201]{Roberge1985}}\\
leb\textit{e}te > lebte \textup{ʻlivedʼ}\\
  rett\textit{e}te  \textup{>  *}rett\sout{t}e \textup{but} rett\textit{e}te \textup{ʻrescuedʼ}\\
\ex\label{ex:reiner:11}
{From Middle Dutch to Present-Day Dutch \citep[200–201]{Roberge1985}}\\
redd\textit{e}de > redd\sout{d}e \textup{ʻrescuedʼ}
\z

That is, German retains a thematic vowel where Dutch does not – and this vowel does not appear to have any meaning (anymore), not even indicating inflectional class. Admittedly, the \nobreakdash-\textit{e}\nobreakdash- does distinguish the 3\textsuperscript{rd} person singular past (\textit{rettete}, /retətə/) from the 1\textsuperscript{st} person singular present (\textit{rette}, /ʁɛtətə/ = /ʁɛttə/ = /ʁɛttə/ = /ʁɛtə/). However, distinguishing between meanings is not the same thing as having a meaning (recall the classical definition of phonemes vs. morphemes). The only “meaning” that could be assigned to the \nobreakdash-\textit{e}\nobreakdash- in \textit{rettete} would be ʻif you have a choice between a 3\textsuperscript{rd}+past reading and a 1\textsuperscript{st}+present reading, choose the formerʼ. This piece of information does not count as a meaning since, as far as I can see, it is not directly evoked by the \nobreakdash-\textit{e}\nobreakdash- in native speakers (let alone vice versa). As a result, we get a meaningless element in an otherwise more or less segmentable string, cf. \REF{ex:reiner:12}.

\ea \label{ex:reiner:12} 
{German}\\ 
\gll rett-e-te\\
     rescue-?-1/3\textsc{sg}.\textsc{pst}\\
\glt ʻrescuedʼ
\z

Thus, again a realisational notation appears to be practical. My point is that such a notation does not intrinsically need paradigms but solely constructions, cf. the construction in \REF{ex:reiner:13}.

\ea \label{ex:reiner:13}
{German}\\ 
 \textup{\{ʻrescueʼ;} \textsc{ps}\textup{:1 \textit{or} 3,} \textsc{num}\textup{:}\textsc{sg}\textup{, ((}\textsc{tns}\textup{:}\textsc{pst}, \textsc{mood}\textup{:}\textsc{indic)} \textit{or} \textsc{mood}\textup{:}\textsc{irr}\textup{)\} $\leftrightarrow $} rettete
 \z


As an aside, from example \REF{ex:reiner:7} onwards in this paper, one-to-many relations between form and meaning have been treated as cases of polysemy; however, a constructionist account is even able to distinguish between polysemy and homonymy.\footnote{Both of which can be special cases of syncretism as defined in the present paper, cf. fn. \ref{fn:reiner:7}.} Consider \tabref{tab:reiner:4}, of which the light shaded cells are presented as a case of polysemy in \REF{ex:reiner:14} and the dark shaded cells are presented as a case of homonymy in \REF{ex:reiner:15}. The motivation for drawing the distinction is that 1\textsuperscript{st} and 3\textsuperscript{rd} plural share a positive feature value (plural), while 3\textsuperscript{rd} singular and 2\textsuperscript{nd} plural do not; moreover, the former syncretism runs through all verbal forms in German (and extends to the present infinitive) while the latter dissolves in the past tense as well as in the present tense of umlaut verbs.

\begin{table}
\caption{Present indicative of German \textit{kaufen} ʻbuyʼ (\citealt[23]{HelbigHelbig2001})\label{tab:reiner:4}}
\begin{tabular}{lll}
\lsptoprule
 & Singular & Plural\\\midrule
1\textsuperscript{st} person & kaufe & \cellcolor{lsLightGray}kaufen\\
2\textsuperscript{nd} person & kaufst & \cellcolor{lsGuidelinesGray}kauft\\
3\textsuperscript{rd} person & \cellcolor{lsGuidelinesGray}kauft & \cellcolor{lsLightGray}kaufen\\
\lspbottomrule
\end{tabular}
\end{table}

\ea \label{ex:reiner:14} 
{German}\\
\{ʻbuyʼ; \textsc{ps}:1 \textit{or} 3, \textsc{num}:\textsc{pl}, \textsc{tns}:\textsc{prs}, \textsc{mood}:\textsc{indic}\} $\leftrightarrow $ kaufen\\
\ex \label{ex:reiner:15}
 {German}\\
\ea \textup{\{ʻbuyʼ;} \textsc{ps}\textup{:3,} \textsc{num}\textup{:}\textsc{sg}, \textsc{tns}\textup{:}\textsc{prs}, \textsc{mood}\textup{:}\textsc{indic}\textup{\} $\leftrightarrow $} kauft \textit{and}\\
\ex \textup{\{ʻbuyʼ;} \textsc{ps}\textup{:2,} \textsc{num}\textup{:}\textsc{pl}, \textsc{tns}\textup{:}\textsc{prs}, \textsc{mood}\textup{:}\textsc{indic}\textup{\} $\leftrightarrow $} kauft
\z\z


To be sure, technically speaking, also the dark shaded cells of \tabref{tab:reiner:4} could be rendered as a case of polysemy, cf. \REF{ex:reiner:16}.

\ea \label{ex:reiner:16}
 {German}\\ 
  \textup{\{ʻbuyʼ; ((}\textsc{ps}\textup{:3,} \textsc{num}\textup{:}\textsc{sg}\textup{) \textit{or} (}\textsc{ps}\textup{:2,} \textsc{num}\textup{:}\textsc{pl})), \textsc{tns}\textup{:}\textsc{prs}, \textsc{mood}\textup{:}\textsc{indic}\textup{\} $\leftrightarrow $} kauft
  \z

However, as argued above, in the specific case at hand, other facts from the language cast doubt on this latter analysis. So here I would opt for the former analysis in terms of homonymy rather than for the latter in terms of polysemy. In sum, while both analyses make use of underspecification they are apt for different kinds of syncretisms.

\begin{sloppypar}
Turning now to Spencer’s main example, auxiliary structures in Slavic, it seems natural that his realisational treatment of not fully compositional word forms is transferred to not fully compositional strings of words that appear to realise similar meanings (e.g., tenses or aspects). It is especially this generalisation that is picked up by \citet[59]{Haspelmath2011} and, among other issues, contributes to Haspelmath’s overall thesis that any universal morphology/syntax boundary is elusive for the time being. Against this background, it will not come as a surprise that also auxiliary structures (or periphrases) may be written down as constructions, so I will not go through this here. However, please note that \sectref{reiner:4} of the present paper will provide a fundamentally different look on such structures.

Summarising my fleshing out of \citegen{Haspelmath2011} reference to \citet{Spencer2001} for the moment, there are reasons to use realisational rules rather than incremental procedures; however it does not seem to matter whether the feature specifications are conceived of as a grid, establishing paradigm cells to be realised, or simply as meanings of constructions (accordingly, the sounds/characters materialise a cell or provide the signifiant of a construction). The notation used in \citet{Spencer2001} is, in fact, similar to the one used above. Going one step further, there is a choice between one representational format that is needed anyway from a constructionist perspective, i.e. the construction, and another representational format that is not needed anyway (though might emerge from applications of the former). Theoretical parsimony requires that we stick to the format needed anyway.
\end{sloppypar}

As a last reflection on \citet{Spencer2001} let me note that the author is perfectly aware of the connection between realisational and constructionist approaches. For example, consider the following quotation on auxiliary structures.

\begin{quote}
We are dealing here with constructional idioms much like phrasal verbs. \citep[283]{Spencer2001}
\end{quote}

In the same vein, consider the following paragraph, where he compares (in\-fer\-en\-tial-realisational) Word-and-Paradigm models (which, needless to say, he favours) to Item-and-Arrangement models.\footnote{Item-and-Arrangement (i.e. incremental), Item-and-Process as well as Word-and-Paradigm are families of morphological theories, for their characteristics cf. \citet{Stewart2016}.}

\begin{quote}
Inferential models can be contrasted with lexical models in which the mapping between form and meaning is specified in the lexical entry of a morpheme. On such a theory, the \nobreakdash-\textit{z}{}-suffix contributes its own [Num:Pl] ʻmeaningʼ to the word form by being combined with the (numberless) form \textit{dog}. In this respect, the plural suffix is a Saussurean sign, a pre-compiled pairing of form and meaning. It is precisely that conception that is denied in paradigmatic and inferential-realisational approaches to morphology. Inferential-realisational models by their nature cannot involve lexically listed mor\-phemes-as-signs. As a consequence, inflectional formatives turn out to be simply morphophonological ʻmarkersʼ on stems, signalling (realising) some subset of the feature set to be expressed. In the simplest cases there is a one:one mapping observable between form and function and the formative then has the appearance of a classical morpheme, but it’s important to realise that this is just one extreme of the form-function mapping (and in inflection a rather rare occurrence at that). (\citealt[280--281]{Spencer2001})
\end{quote}

\begin{sloppypar}
Note the affinity of this approach to morphology with constructionist approaches to syntax: the primary signifiant is the expression as a whole; its subparts have their function within the expression but they do not necessarily have the same function in other environments as well (cf. the reflection on Spanish theme vowels above). A remaining difference seems to be that inferential-realisational morphology still refers to superordinate rules for realisation, e.g.: if there is no dedicated expression for a given combination of feature values then use the default form (\citealt[280, 289]{Spencer2001}). As an alternative to such rules, in Construction Grammar there are, however, relationships between constructions \citep[109]{Goldberg1995}. In fact, also realisational rules can be captured by such relationships, e.g.: given a set of meaningful expressions (i.e. constructions) organised in an inheritance hierarchy, use the one that maximally specifies the meaning you want to express (and the maximally specific construction available might turn out to be rather unspecific). Such a conception, including further developments, is run-of-the-mill in Sign-Based Construction Grammar (\citealt[9--14]{SagEtAl2012}).
\end{sloppypar}\largerpage

In sum, the inspection of \citet{Spencer2001} shows that constructionist thinking is influential in morphology and, from an outside perspective, that constructions might even fully replace paradigms. I can be briefer about the two other works mentioned in \citet[59]{Haspelmath2011} and pursued here, i.e. \citet{Gurevich2006} and \citet{Booij2010}. For what had to be worked out carefully with regard to \citet{Spencer2001} is utterly explicit in in these works: Construction Grammar applies to morphology just as it applies to syntax.\footnote{It does become explicit in \citet[84]{Spencer2004}: “Essentially, we need to think of inflected words as akin to constructional idioms”.} So I confine myself here to a central figure from \citet[170]{Gurevich2006} and an exemplary formula from \citet[241]{Booij2010}.

\citet{Gurevich2006} is a book-long treatment of the so called version in Georgian. This is a pre-radical vowel with a whole range of functions, crucially depending on context (\citealt[6--13]{Gurevich2006}). For example, \nobreakdash-\textit{i}\nobreakdash- as a pre-radical vowel indicates that the indirect object is affected (in the presence of an overt 1\textsuperscript{st} or 2\textsuperscript{nd} person indirect object) or that the subject is affected as a beneficiary (in the absence of an overt indirect object) or that there is no affected participant (in the passive) or that the subject is 1\textsuperscript{st}/2\textsuperscript{nd} person (in the evidential perfect). In a way, this is a scaling-up of what was demonstrated above for thematic vowels in Spanish: a given element in a word form does make a specific contribution to the overall meaning of the word and clause but in order to identify that contribution the hearer has to have information far beyond the element as such.

Now, let’s see how \citet[170]{Gurevich2006} combines all these functions of version vowels (of which the example above was but a snippet) into a constructional network. Her presentation is reproduced here as \figref{fig:reiner:2}.

  
\begin{figure}
\includegraphics[width=.75\textwidth]{figures/Reiner2.pdf}
 \caption{Constructional network for Georgian version vowels\linebreak\citep[170]{Gurevich2006}}
 \label{fig:reiner:2}
\end{figure}

Syncretisms\footnote{I assume that all of the distinctions are marked overtly elsewhere in the language.} like the one involving \nobreakdash-\textit{i}\nobreakdash- can easily be read off the network.

Having addressed syncretisms (as well as other kinds of non-bijective form-meaning relations) several times in the present paper, some standard problems for classical Item-and-Arrangement approaches to morphology are covered. Another standard problem from the wealth of non-bijective form-meaning relations, however, has not been stated explicitly so far, i.e. allomorphy. This is tackled by \citegen{Booij2010} last chapter. For example, he gives the following formula for capturing the ablaut in \textit{sing} – \textit{sang}:\largerpage

\begin{center}
[X i Y]\textsubscript{V, [–past]} ${\approx}$ [X a Y]\textsubscript{V, [+past]}  \citep[241]{Booij2010}
\end{center}

\noindent Crucially, none of the forms is derived from the other or from some underlying form but they are equally stored in the mental lexicon according to \citet[Ch. 10]{Booij2010}.\footnote{I concentrate on this last chapter since the other chapters are rather on word formation than on inflection whereas the present paper focusses on the latter, to the extent that the difference can be upheld, cf. \citet{Haspelmath2011}.}

In principle, true suppletion \citep{Mel’čuk1994} may be captured the same way as far as I can see, e.g. for the partial paradigm given in \tabref{tab:reiner:1} (viewed synchronically):

\begin{center}
[bi Y]\textsubscript{V, [1/2}\textsc{\textsubscript{sg}}\textsubscript{]} ${\approx}$ [is Z]\textsubscript{V, [3}\textsc{\textsubscript{sg]}} ${\approx}$ [sind]\textsubscript{V, [1/3}\textsc{\textsubscript{pl}}\textsubscript{]} ${\approx}$ [sei \textsc{Z}]\textsubscript{V, [2}\textsc{\textsubscript{pl}}\textsubscript{]},
\end{center}

\noindent where Z is the regular agreement suffix (phonetically) and Y is either the regular agreement suffix or something else (\citealt[6--7]{Plank2016}). Taxonomically speaking, the formula involves four sister constructions (\citealt[111]{JackendoffAudring2020}).\footnote{Also note the relevance of networks in another recent constructionist approach, i.e. \citet{CrysmannBonami2017}.} The important point about these interconnected constructions is that they can, and do, serve to restate paradigms (\citealt[151--154]{JackendoffAudring2020}). In fact, \citet{JackendoffAudring2020} provide a very nice example of what is going on with paradigms in Construction Morphology (CxM): in a first step, they are acknowledged as central to inflection \citep[133--134]{JackendoffAudring2020} but in a second step, they are restated, one by one, as interconnected constructions (e.g., \citealt[151--154]{JackendoffAudring2020}). This is reminiscent of my argumentation above that cells are in fact constructions and paradigms are (specific) relations between constructions.

The extension of \citet{Booij2010} concludes my fleshing-out of \citegen[58--59]{Haspelmath2011} remark that, essentially, realisational morphology and constructionist syntax do the same thing. To summarise: against the background of strictly agglutinating languages, certain phenomena familiar from fusional languages (fusion as such, syncretism, allomorphy, …) suddenly appear fairly idiosyncratic – just like against the background of compositional syntax idioms appear quite idiosyncratic. For the latter case Construction Grammarians concluded that we simply had the cart before the horse, i.e. that we have to start conceiving of the whole as primary to its parts – and now their solution is transferred to morphology. For \citet{Haspelmath2011}, this expansion of the constructionist idea raises, together with a range of other considerations, the question where to draw the line between the two levels of description generally. In any case, traditional inflectional morphology’s favourite format, i.e. the paradigm, seems to be replaceable by constructions, integrated in networks.

We will see in the next section that DM tries to capture exactly the same problematic groups of phenomena but opts for a radically different solution, i.e. the strict separation of meaning and form (at the outset).


\subsection{From a composition-oriented perspective} \label{reiner:2.2}

\subsubsection{A short introduction to DM} \label{reiner:2.2.1}


Distributed Morphology (DM) is a strictly compositional approach to morphology, presupposing GB or minimalist syntax. Its origins lie in \citet{HalleHalle1993}, fairly recent overviews are \citet{Embick2015} and \citet{Bobaljik2017}, for a critical review of the former cf. \citet{Spencer2019}. The following summary is mainly based on these works; however, occasionally other publications will be referred to as well.

The key feature of DM is not so much the distribution of the morphology as the distribution of the lexicon, which is divided into three lists \citep[28]{Bobaljik2017}. These lists are called A, B, and C below:

\begin{description}
\item[List A:] containing roots, i.e. abstractions over forms, e.g. ${\surd}$\textsc{buy}, as well as feature values, e.g. plural.
\item[List B:] containing vocabulary items, i.e. pairings of phonological form and information about where that form may be inserted (\citealt[4]{HarleyHarley1999}), e.g. /s/ $\leftrightarrow $ [3\textsc{sg}] or /ba\textsc{i/} $\leftrightarrow $ V.
\item[List C:] containing non-grammatical concepts connected to the roots in list A (\citealt[209--210]{Embick2015}), e.g. there is a concept ʻget for moneyʼ in list C linked to the root ${\surd}$\textsc{buy} in list A. So the purview of list C is handling the idiosyncratic – even up to idioms. For example, if the aforementioned link does not fit the context, another link can be established like ʻaccept as trueʼ~–~${\surd}$\textsc{buy} (cf. \textit{I don’t buy your conclusion}).
\end{description}

These lists are the cornerstones of a derivational model for inflection and word formation. The derivation starts by selecting items from list A that attach to the terminal nodes of an appropriate syntactic representation, yielding an output like ${\surd}$\textsc{buy}-3\textsc{sg}. This output is then handed over to list B where matching phonological strings are inserted into the positions, e.g. /ba\textsc{i}/-/s/. For less straightforward derivations, certain processes directly before or after insertion may be appealed to in addition. One of these options, i.e. impoverishment, will be presented in more detail below. First, let’s visualise the basic architecture and say a few more words about the individual steps of the derivation, in particular about vocabulary insertion. For the latter objective it is suitable to pick a language with more agreement than English, so I chose German and the verb \textit{kaufen} ʻbuyʼ whose past indicative forms are given in \tabref{tab:reiner:5}. Needless to say, the paradigmatic presentation is only for convenience, the forms being represented differently in the visualisation to follow.\largerpage[1.5]

\begin{table}
\caption{Past indicative of German \textit{kaufen} ʻbuyʼ \citep[23]{HelbigHelbig2001}\label{tab:reiner:5}}
\begin{tabular}{lll}
\lsptoprule
 & Singular & Plural\\\midrule
1\textsuperscript{st} person & kaufte & kauften\\
2\textsuperscript{nd} person & kauftest & kauftet\\
3\textsuperscript{rd} person & kaufte & kauften\\
\lspbottomrule
\end{tabular}
\end{table}

Now consider \figref{fig:reiner:3}, where \textit{kauften} \{ʻkaufenʼ, \textsc{past}, 3\textsc{pl}\} is derived.\footnote{Note that the corresponding present tense example would be more difficult to handle \citep[231]{Spencer2019}.}


\begin{figure}
% \begin{forest}
%  [IP,name=IP
%     [~]
%     [I$'$
%         [VP
%             [~]
%             [V$^0$
%                 [o, name=o1]
%             ]
%         ]
%         [I$^0$
%             [T$^0$
%                 [o, name=o2]
%             ]
%             [AGR$^0$
%                 [o, name=o3]
%             ]
%         ]
%     ]
%  ]
% \node(listA)[above=2cm of IP,xshift=-2cm]{List A: };
% \node(kauf)[right=0mm of listA,draw,dashed,rectangle,rounded corners,inner sep=1mm]{$\sqrt{\text{KAUF}}$};
% \node(rest)[right=0mm of kauf]{, $\sqrt{\text{LAUF}}$,$\sqrt{\text{HUND}}$,$\sqrt{\text{GUT}}$, \ldots\\};
% \node(tensepast)[below=2mm of kauf,draw,dashed,rectangle,rounded corners,inner sep=1mm]{\footnotesize [\textsc{tense}:\textsc{past}]};
% \node(person12)[right=0mm of tensepast]{\footnotesize , [\textsc{person}:1], [\textsc{person}:2], };
% \node(person3numpl)[right=0mm of person12,draw,dashed,rectangle,rounded corners,inner sep=1mm]{\footnotesize [\textsc{person}:3], [\textsc{num}:\textsc{pl}]};
% \node(genf)[right=0mm of person3numpl]{\footnotesize , [\textsc{gen}:\textsc{f}]};
% %
% \node(vkaufpast3pl)[below = 9cm of kauf]{$\sqrt{\text{KAUF}}$-\textsc{past}-3-\textsc{pl}};
% \node(listB)[below = 3cm of vkaufpast3pl,xshift=-2cm]{List B:};
% \node(kaOf)[right=0mm of listB]{/kaOf/,};
% \node(past-t@)[right=0mm of kaOf]{[\textsc{past}] $\leftrightarrow$ /t@/,};
% \node(2pl-t)[right=0mm of past-t@,text width=2.5cm,text align=center]{[2\textsc{pl}] $\leftrightarrow$ /t/, \dots};
% \node(listC)[right=0mm of 2pl-t]{List C: img img img};
% \node(remainder)[below=0mm of 2pl-t,text width=2cm]{[\textsc{pl}] $\leftrightarrow$ /n/\\
%     {}[\textsc{2}] $\leftrightarrow$ /st/\\
%     {}[] $\leftrightarrow$ /$\empty$/};
% \draw[->](o2)--(vkaufpast3pl);
% \draw(kaOf) -- node[above,sloped] {\footnotesize match} (vkaufpast3pl);
% \draw(past-t@)--node[above,sloped] {\footnotesize match} (vkaufpast3pl);
% \draw(2pl-t.north)--node[above,sloped] {\footnotesize 1st trial no match}(vkaufpast3pl);
% \draw(remainder.north)--node[below,sloped] {\footnotesize  2nd trial match}(vkaufpast3pl);
% %
% \path[dashed](o1)edge [bend left=70,in=40] (kauf.north);
% \path[dashed](o2)edge [bend left=70] (tensepast);
% \draw[dashed](o3.east)edge [bend right=40](person3numpl);
% \path[gray](listC.east) edge [out=30,in=30] (kauf.north);
% \end{forest}



\includegraphics[width=\textwidth]{figures/Reiner3.pdf}
\caption{example \textit{kauften} \{ʻkaufenʼ, \textsc{past}, 3\textsc{pl}\}, showing the basic DM-architecture \citet[6]{Bobaljik2017}, \citet[248--250]{AlbrightAlbright2012}}
\label{fig:reiner:3}
\end{figure}

In list A it can be seen that there are several roots, including ${\surd}$\textsc{kauf}, as well as several feature values, including past, 3\textsuperscript{rd} person, and plural. Crucially, the roots are neither concepts (list C) nor forms (list B). As to the feature values, I have taken the liberty to write them down as such, i.e. [\textsc{feature}:\textsc{value}], whereas it seems more common in DM to write the values only and refer to them as \textit{features}. In any case, this is the sole kind of structure (apart from the distinction between roots and feature values) that enters list A. Otherwise this list is meant to be an unstructured cloud. In particular, the individual items are not pre-arranged into paradigms.

The next step in \figref{fig:reiner:3} is the selection of a root and several features, all of which attach to the terminal nodes of the syntactic tree. From my perspective as an outsider to the theory, a mystery is connected with this step: what kind of mechanism, apart from communicative intent, determines which feature values are selected? For example, somewhere in the model there must be information on which contexts require which tense-aspect categories to be realised in the language at hand (cf. English \textit{He is eating} vs. German \textit{Er isst}\textsubscript{perfective-or-imperfective}). In any case, the root and the feature values are arranged by the syntax, independently from any phonological information. So the intermediate result is an abstract string like ${\surd}$\textsc{kauf}-\textsc{past}.3\textsc{pl}.

Now, vocabulary insertion can take place. The phonological form /kaɔf/ is one of the forms over which ${\surd}$\textsc{kauf} is an abstraction, so they match (at least this is what I have been able to conjecture; insertion for roots in simple cases is not extensively discussed in the DM-literature). Also for the second part of the abstract string, i.e. \textsc{past}, there is just one matching vocabulary item, i.e. [\textsc{past}] $\leftrightarrow $ /tə/, so this item is inserted. Things get more interesting and more difficult with respect to [3\textsc{pl}]. Several vocabulary items are available for realising AGR and, crucially, these are ordered from most to least specific. Now the mechanism runs through this ordered list and first tries the highest (= most specific) vocabulary item, i.e. [2\textsc{pl}] $\leftrightarrow $ /t/. However, 2\textsuperscript{nd} person does not match 3\textsc{pl}, so this item is discarded. The second highest item is [\textsc{pl}] $\leftrightarrow $ /n/, which does match 3\textsuperscript{rd} person plural (i.e.: there is no contradiction). So this item is inserted. Crucially, the same item matches 1\textsc{pl}, hence it would likewise be inserted in the derivation of \textit{kauften} \{ʻkaufenʼ, \textsc{past}, 1\textsc{pl}\}.

What can be seen here are two uses of underspecification at once. First, it serves to order the list in such a way that, by browsing the list downward, insertion may operate on the Paninian Principle (=~Elsewhere Condition~=~Subset Principle). That is: given two rules, where one is more specific than the other, application of the more specific one – here: insertion of the more specific item – blocks application of the less specific one (for an overview on the principle cf. \citealt[132]{Anderson1992}). Put differently: the most general rule – here: insertion of the least specific item – is the default and whenever there are more specific rules/items available, the most specific one of them will overwrite the default. The “only” job of the data-tackling linguist is to find the most economic list for the data set at hand. With respect to the German past indicative forms, I adopted the list from \citet[6]{Bobaljik2017}, although from a German Linguistics point of view it might be debatable whether the schwa belongs to the tense or agreement suffix (pro Bobaljik’s solution: the verb \textit{tu\sout{e}n} ʻdoʼ). Moreover, as far as I can see, the order of [\textsc{pl}] $\leftrightarrow $ /n/ and [2] $\leftrightarrow $ /st/ is arbitrary since none of them is more specific than the other.

The second, ensuing, use of underspecification here is modelling syncretism. As mentioned above, the vocabulary item [\textsc{pl}] $\leftrightarrow $ /n/, matching 3\textsc{pl} as well as 1\textsc{pl,} may be inserted into both potential strings, ${\surd}$\textsc{kauf}-\textsc{past}.\textit{3}\textsc{pl} (cf. \figref{fig:reiner:3}) as well as ${\surd}$\textsc{kauf}-\textsc{past}.\textit{1}\textsc{pl}. Likewise, when we derive the realisation of ${\surd}$\textsc{kauf}-\textsc{past}.3\textit{\textsc{sg}} or ${\surd}$\textsc{kauf}-\textsc{past}.1\textit{\textsc{sg}}, the mechanism runs through the list, discards [2\textsc{pl}] $\leftrightarrow $ /t/, [\textsc{pl}] $\leftrightarrow $ /n/, and [2] $\leftrightarrow $ /st/ as not matching and ultimately reaches the least specific pairing [] $\leftrightarrow $ ∅. So for both persons, AGR is realised as null. Please note that this way systematic syncretisms may be captured without referring to paradigms.\footnote{The 1/3 conformity really is a syncretism in the sense of fn. \ref{fn:reiner:7} since the two person values are distinguished overtly elsewhere in the language, viz. in the singular of the present tense (\textit{kaufe} vs. \textit{kauft}).}

However, underspecification is not the only way to model syncretism in DM (although the most desirable one according to \citealt[253]{Harley2008}). In particular, syncretisms that appear to be even more systematic than the ones treated above may be modelled by a process called impoverishment. This is the deletion of features/feature values in the abstract string that has been issued from the syntax and awaits vocabulary insertion \citep[139]{Embick2015}. For example, German determiners distinguish three gender values in the nominative singular (nominative: \textit{der}, \textit{die}, \textit{das}) but neutralise this distinction in the nominative plural in favour of the feminine form (nominative: \textit{die}),\footnote{Cf. \citet[214]{HelbigHelbig2001}. Note that the following analysis works only if \textit{die} is viewed as feminine per se (\citealt{Meinunger2017}, but cf. \citealt[291--292]{Leiss1994} on why this feature value has little to do with femininity as such). For relevant considerations cf. also \citet[184]{Kramer2019}.} in accordance with Greenberg’s well-known universal no. 37. So all three output strings, ${\surd}$\textsc{d}-\textsc{nom}.\textit{\textsc{masc}}.\textsc{pl}, ${\surd}$\textsc{d}-\textsc{nom}.\textit{\textsc{fem}}.\textsc{pl}, and ${\surd}$\textsc{d}-\textsc{nom}.\textit{\textsc{neutr}}.\textsc{pl} have to yield \textit{die} in the end; likewise for the other case values (\textit{derer}/\textit{n}, \textit{den}, \textit{die}). In this situation it seems reasonable to delete the gender value from the string altogether. A rule expressing this is shown in \REF{ex:reiner:17} for the example at hand.

\ea \label{ex:reiner:17} 
[+\textsc{masc}/\textsc{fem}/\textsc{neutr}] ${\Rightarrow}$ \textsc{fem}/\_\textsc{pl}\\
read: any gender value is set to the feminine under the condition of a plural environment
\z

Having addressed syncretism, which is handled by underspecification or impoverishment in DM, I have to touch upon allomorphy, too, since these two types of phenomena – syncretism and allomorphy – appear to be the standard testing ground for any morphological theory. Allomorphy then can be handled by adding conditions on vocabulary items \citep[169]{Embick2015}. For example, the German vocabulary item [\textsc{past}] $\leftrightarrow $ /tə/ works well in a derivation like the one in \figref{fig:reiner:3} above but would yield ungrammatical forms for strong verbs, e.g. the past 3\textsuperscript{rd} person plural of \textit{gehen} ʻwalkʼ would be wrongly predicted as */gehtən/. In order to derive the correct form /g\textsc{i}ŋən/ we may use the adopted vocabulary item in \REF{ex:reiner:18}.

\ea \label{ex:reiner:18}
/g\textsc{i}ŋ/ $\leftrightarrow $ ${\surd}$\textsc{geh}/\_\textsc{past}\\
  read: /g\textsc{i}ŋ/ is the form to be inserted for ${\surd}$\textsc{geh} under the condition of a past environment
\z

Additionally, the insertion of [\textsc{past}] $\leftrightarrow $ /tə/ has to be suspended in some way or other. A remaining problem of this analysis is how to supply the schwa of /g\textsc{i}ŋən/ when assuming \citegen{Bobaljik2017} segmentation.

At this point I have presented a snapshot of DM in action, not even mentioning further processes like fusion under locality conditions, fission, or readjustment (also note that linearisation is not standardly predictable from syntax).\footnote{Fusion = combination of two sister nodes into one, which retains the features of both input nodes but has no internal structure \citep[15]{Bobaljik2017}; mnemonic: fusional morphology in the typological sense.  Fission = splitting of a single node in the syntax into two nodes in the morphological representation \citep[19]{Bobaljik2017}; mnemonic: multiple exponence. Readjustment = phonological alternation after vocabulary insertion \citep[7]{Bobaljik2017}; mnemonic: remedy for everything else that leaves derivation ill-formed.} To conclude for the moment, DM’s slogan, it’s “syntactic hierarchical structure all the way down” \citep[3]{HarleyHarley1999} is to be taken with a grain of salt: it’s syntactic hierarchical structure all the way down until vocabulary insertion and its satellite processes. That is, many things can happen on the way from syntax to final spell-out and it is this many-faceted interface that deserves the name \textit{morphology} in DM. Crucially, however, DM strives to make even these processes as predictable as possible.

In any case, the architecture as a whole permits reading paradigms off individual derivations (e.g., \tabref{tab:reiner:5} may be read off the derivation depicted in \figref{fig:reiner:3}) but there are no pre-designed paradigms in the model. That is: to the extent that DM provides a model of language representation and processing, paradigms do not have any psychological reality in the theory (cf. also \citealt[53]{Bobaljik2002}). This negative attitude towards paradigms will be summarised and put into perspective in the following section (\sectref{reiner:2.2.2}).


\subsubsection{The status of paradigms in DM} \label{reiner:2.2.2}

This section is not a real section but a convenient synopsis for those readers who skipped the introduction to DM. So the main point of the section is summarising what was laid out in \sectref{reiner:2.2.1}. with respect to paradigms in DM. There are at least two places in the architecture at which paradigmatic structures would be expected by the average linguist but in fact do not play a role.

First, feature values (list A) are not organised into paradigms. Obviously, they could be (though, with a placeholder for roots/stems); however, according to the theory this is just not necessary for deriving correct forms. And unnecessary pre-syntactic structure is to be avoided in a strictly compositional approach \citep[17]{Embick2015}.\largerpage[1.5]

Second, vocabulary items (list B) are indeed ordered but not according to intersecting feature values (as in paradigms) but according to specificity. It is striking that DM is all the same able to capture syncretism and allomorphy – phenomena that are otherwise thought to be inextricably linked with paradigms (cf. also \citealt[54]{Bobaljik2002}).

All of this does not mean, however, that practitioners of DM do not use paradigms for presentation. To pick a random example, \citet{Harley2008} is full of paradigms. The important thing is that they do not have a primary status in the architecture. To use the food metaphor from \citegen{Trommer2016} title: from a DM perspective, compiling paradigms is like pre-sorting the ingredients in your kitchen cupboard according to nutrients – useful for some purposes but not necessary (and hence not desirable) for cooking tasty meals.

I conclude this micro-section by a synoptic quotation, even more explicit than the corresponding one in \sectref{reiner:1}:

\begin{quote}
Importantly, paradigms are epiphenomenal in DM. They have no theoretical status and they are never referred to by morphological operations. \citep[97]{Kramer2016}\footnote{One reviewer remarks: “Interestingly, they [= paradigms] seem to organise those morphological operations into meaningfully related sets, which would place them above those operations. In this sense, they are not epiphenomenal but even more abstract than the abstract operations”. From a DM-perspective, this is not a contradiction: the meaningfully related sets might be constructed by the linguist or the L2-teacher; however they are not part of any L1-speaker’s mental grammar. This position gains some plausibility from anecdotal evidence: it appears hard to write down L1-paradigms if one is asked to do so for the very first time.}
\end{quote}

\subsection{Comparison} \label{reiner:2.3}\largerpage[1.5]

This section serves to compare the two approaches regarding their shared detachment from paradigms and their shared tendency to reduce inflectional morphology to something else (constructions or the syntax/phonology interface). To this end I will first address their respective standard data and then delve into a more theoretical discussion. For completeness, also Autonomous Morphology is occasionally integrated into the picture before it takes centre stage in \sectref{reiner:3}.

While \citet[58--59]{Haspelmath2011} refers to Turkish for showing that morphology can look like syntax, proponents of DM adduce Swahili for the same reason. However, Swahili inflection does not only display a high degree of compositionality but it also displays purely positional contrasts – something that we tend to expect from syntax exclusively. For example, consider the pair of examples in \REF{ex:reiner:19a} and \REF{ex:reiner:19b}, taken from \citet[18]{Trommer2001}.\footnote{In a similar vein cf. \citet{Crippen2019}.}

\ea \label{ex:reiner:19} 
Swahili (Atlantic-Congo, Tanzania et al.)\footnote{\url{https://glottolog.org/resource/languoid/id/swah1253}}
\begin{multicols}{2}\raggedcolumns
\ea
\gll \label{ex:reiner:19a}ni-wa-penda\\
     1\textsc{sg}{}-3\textsc{pl}{}-like\\
\glt ʻI like them.ʼ\columnbreak
\ex \label{ex:reiner:19b}
\gll wa-ni-penda\\
     3\textsc{pl}{}-1\textsc{sg}{}-like\\
\glt ʻThey like me.ʼ
\z
\end{multicols}
\z

To be sure, this does not work equally well for all forms, cf. \tabref{tab:reiner:6}.

\begin{table}
\caption{Swahili person forms \citep[15, 102; only M-/WA-class]{AlmasiAlmasi2014}\label{tab:reiner:6}}
\begin{tabular}{lllll}
\lsptoprule
& \multicolumn{2}{c}{subject} & \multicolumn{2}{c}{object}\\\cmidrule(lr){2-3}\cmidrule(lr){4-5}
& \textsc{sg} & \textsc{pl} & \textsc{sg} & \textsc{pl}\\\midrule
{1\textsuperscript{st}} {person} & ni & tu & ni & tu\\
{2\textsuperscript{nd}} {person} & u & m & ku & wa\\
{3\textsuperscript{rd}} {person} & a & wa & m(w) & wa\\
\lspbottomrule
\end{tabular}
\end{table}

\begin{table}
\begin{floatrow}
\captionsetup{margin=.05\linewidth}
\ttabbox{\begin{tabular}{lll}
\lsptoprule
& {Subject} & {Object}\\\midrule
{Singular} & ni & ni\\
{Plural} & tu & tu\\
\lspbottomrule
\end{tabular}}
        {\caption{Swahili person forms extracted from \tabref{tab:reiner:6}\label{tab:reiner:7}}}
\ttabbox{\begin{tabular}{lll}
\lsptoprule
& {Subject} & {Object}\\
\midrule
{Singular} & car & car\\
{Plural} & cars & cars\\
\lspbottomrule
\end{tabular}}
        {\caption{English nominal case forms\label{tab:reiner:8}}}
\end{floatrow}
\end{table}

Still, writing down the first line of the paradigm appears to make as much sense as writing down nominal case forms for English, compare \tabref{tab:reiner:7} to \tabref{tab:reiner:8}.

Obviously, here it is not the paradigm that tells the language user which form is the subject or object – it is the combinatorics (here: surface linear order). Switching to a constructionist perspective taken to the extreme, even the second and third line of the Swahili paradigm may be treated very much like the first line: it is still the relative position (slot) of a given form that determines its syntactic function – and makes it vary in idiosyncratic ways (e.g. \textit{u} ${\Rightarrow}$ \textit{ku}).

So to some extent both kinds of data – Turkish-style and Swahili-style – support the idea that paradigms might be quite parochial a format: apt for Latin type languages but hardly beyond.

To repeat, with regard to the constructionist perspective, the above conclusion is my own one, neither \citet[58--59]{Haspelmath2011} nor related works plainly oppose against paradigms. To the contrary, \citet{Haspelmath2000} even allows periphrases as paradigm cells and \citet{Booij2016} reconceptualises paradigms as “second order schemas”. However, let’s have a closer look at these two conceptions.\largerpage[2]

\citet{Haspelmath2000} argues against a gap-filling account of periphrasis and in favour of a grammaticalisation-based account.\footnote{Some of \citegen{Haspelmath2000} arguments against a gap-filling account may be countered by the criterion of feature intersection (\citealt[250--252]{BrownBrown2012}, \citealt{Reiner2020}); however this is beyond the scope of the present paper.} Crucially, he does not only allow periphrases as paradigm cells but, by extension, also entire clauses:

\begin{quote}
However, it is not difficult to find syntactic phenomena that provide a striking analog of inflectional paradigms, gaps, and periphrasis in morphology. Again, a good example comes from English, where only a small subclass of verbs can occur without complications in interrogative and negative clauses. In (16), this well-known pattern is represented in such a way that the similarities with morphological suppletive periphrasis become apparent. […] Clearly, “periphrastic \textit{do}” is periphrastic in much the same way as the cases of morphological periphrasis, but the filled gaps in \REF{ex:reiner:16} are not morphological monolectic forms. \textit{Did you see} is a syntactic phrase which replaces the impossible syntactic phrase *\textit{saw you}. \citep[662]{Haspelmath2000}
\end{quote}

Here is Haspelmath’s example number (16), reproduced as \tabref{tab:reiner:e20}.

% \ea \label{ex:reiner:20}
% {English \citep[662]{Haspelmath2000}}\\
% \glll decl., affirm. interrogative \hphantom{text} \hphantom{text} \hphantom{text} \hphantom{text} negative \\
% \textit{You are here} \textit{Are you here?} \hphantom{text} \textit{You are not here}\\
% \textit{You saw her} [\textit{Did you see her?}] [\textit{You did not see her}]\\
% (*\textit{Saw you her?}) (*\textit{You saw not her})
% \z 


\begin{table}
\caption{English \citep[662]{Haspelmath2000}\label{tab:reiner:e20}}
\begin{tabular}{lll}
\lsptoprule
decl., affirm. & interrogative & negative\\
\midrule
\textit{You are here} & \textit{Are you here?} & \textit{You are not here}\\
\textit{You saw her} & [\textit{Did you see her?}] & [\textit{You did not see her}]\\
& (*\textit{Saw you her?}) & (*\textit{You saw not her})\\
\lspbottomrule
\end{tabular}
\end{table}


This extension seems simply logical; however, it raises the question whether there is anything at all that can\textit{not} be described by paradigms in this sense, i.e. by oppositions. In fact, a strictly constructionist perspective mandates that any set of clauses (transparent or not) is viewed as a set of constructions between which the language user may chose, hence as a paradigm in the above sense. This notion of paradigm, then, is so abstract that it becomes vacuous: if everything is paradigmatic it is pointless to state that such and such linguistic phenomenon (e.g., inflection) is, indeed, organised paradigmatically.

Turning now to Booij’s CxM, as summarised in \citet{Booij2016}, the first thing to note is that inflection as well as word formation and phrasal idioms are captured by constructional schemas. \REF{ex:reiner:21} is an example for inflection.

\ea \label{ex:reiner:21}
{English \citep[440, number (37) there]{Booij2016}}\\
  〈[(x\textsubscript{i})\textsubscript{ω-j} $\leftrightarrow $ [N\textsubscript{i}, +sg]\textsubscript{j} $\leftrightarrow $ [SG [SEM\textsubscript{i}]]\textsubscript{j}〉
\z

Read: x\textsubscript{i} constitutes a phonological word (ω), which is associated with a certain morphosyntactic structure, which is associated with a certain semantic structure; the indices show identity relations.\footnote{Inflectional class information can be integrated into the morphosyntactic structure or modelled by second order schemas (to be introduced below).}

Schemas that relate to each other (i.e.: share at least one element) constitute second order schemas, e.g. \REF{ex:reiner:22}.

\ea
\label{ex:reiner:22} 
{English \citep[440, number (39) there]{Booij2016}}\\

  \hphantom{${\approx}$} 〈(x\textsubscript{i})\textsubscript{ω{}-j} $\leftrightarrow $ [N\textsubscript{i}, +sg]\textsubscript{j} $\leftrightarrow $ [SG [SEM\textsubscript{i}]]\textsubscript{j}〉 \\
  ${\approx}$ 〈(x\textsubscript{i}{}-z)\textsubscript{ω{}-j} $\leftrightarrow $ [N\textsubscript{i}, +pl]\textsubscript{j} $\leftrightarrow $ [PL [SEM\textsubscript{i}]]\textsubscript{j}〉
\z

Such schemas correspond to traditional paradigms in an obvious way and, crucially, they are said to organise language in their own special manner (for a lucid example from Saami cf. \citealt[442]{Booij2016}). However, paradigms in this sense, central as they are, do not constitute the fundamental building blocks of morphological theory as Stump or Spencer would have it. Rather, they emerge from a more general organisational principle, which is the constructional schema: as soon as two or more schemas share one or more elements, they constitute a second order schema and in this sense a paradigm. To put it in a nutshell: paradigms are relevant but not basic (also cf. \citealt[239–240, 257]{MarziEtAL2020}).

To conclude on \citet{Haspelmath2000} and \citet{Booij2016}, while they do use paradigms in some sense, I argued that, implicitly, they deprive paradigms of any fundamental theoretical status: the former extends the notion to such a degree that it becomes void and the latter’s approach allows reducing paradigms to a mere consequence of the fact that there are shared elements between constructional schemas.

\begin{sloppypar}
Note that both takes on paradigms just presented differ from the position of Autonomous Morphology. There, periphrases are allowed into paradigms, too; however only under a very restricted notion of periphrases, which ensures that they are mere surrogates for true word forms \citep{BrownBrown2012}. When, moreover, \citet[147--148]{Stump2002} speaks of “syntactic paradigms”, he means something different: word forms seen as instantiations of a lexeme, among which a given syntactic context may chose the appropriate one.
\end{sloppypar}

Coming back to CxM, with regard to the non-fundamental status of paradigms it is surprisingly close to DM, except that the demotion is explicit in DM. However, in another regard, it is DM that implements considerations otherwise basic to CxM as well as to Autonomous Morphology. Prima facie non-compositional phenomena like syncretism, allomorphy (including suppletion) or polyfunctionality seem to call for a constructionist account and/or might constitute the irreducibly morphological in language (more on this relation below). Yet, also DM has developed means to deal with such phenomena: some are handled directly by underspecification, others may require processes like impoverishment, fusion, fission, or readjustment (cf. \sectref{reiner:2.2.1}). Although even these processes are designed to be as predictable as possible, one cannot deny that this is much more than just syntactic derivation plus vocabulary insertion. Put differently, if morphology is only an interface between syntax and phonology (as DM says) then it is a quite rich one. To be fair, not all processes (or rules) mentioned above are embraced equally by all proponents of DM. For example, Trommer takes a very critical stance on the accumulation of rule types in DM, summarised in \citet[61]{Trommer2016}. In the same volume, \citet{Haugen2016} argues explicitly against the use of readjustment rules, in particular when modelling stem allomorphy. Also other practitioners of DM try to restrict the purview of readjustment rules (e.g. \citealt[475]{PominoPomino2019}). In fact, my example of /g\textsc{i}ŋ/ above has been designed in this spirit.

As an interim summary, I compared CxM and DM with respect to their take on paradigms (implicit vs. explicit demotion) as well as with respect to their conception of morphology vis-à-vis other levels of description (explicit vs. implicit acknowledgment, but see below). In the rest of this section, I will extend the comparison to three further dimensions, ordered in growing distance from the actual topic of paradigms and independent morphology: the notion of word, the question of psychological reality, and the kind of restrictiveness found in the respective approach.

\textit{The notion of word} is discussed at length in \citet{Haspelmath2011}, from which I have cited but a snippet up to now. In brief, his overall discussion concludes that criteria for a universal definition of word remain elusive\footnote{But cf. \citet{Gil2020} for a recent proposal.} – and so does a definite border between morphology and syntax. This fits well with the general constructionist idea of a lexicon-syntax continuum. However, from my perspective as an outsider to the theory, any commitment to this idea is in conflict with acknowledging morphology as a level of description qualitatively different from both, syntax and the lexicon: how can we draw solid lines in the middle of a continuum? Thus, reflecting the notion of word casts some doubt on the extent to which constructionist accounts are able to acknowledge “morphology by itself” (to allude to \citeauthor{Aronoff1994}'s \citeyear{Aronoff1994} title). This may be a dividing line between CxM (Cx“M”?) and Autonomous Morphology.

\begin{sloppypar}
Concerning their own definition of word, Autonomous Morphologists are hard to pin down, though. Working in the tradition of Word-and-Paradigm approaches, they seem to accept that for any given string we can tell whether it constitutes a word or not – no matter if we are looking at data from Latin, Turkish, Swahili or West Greenlandic. The only explicit pertinent discussion I am aware of is in a footnote in \citet[127–128]{SpencerLuis2013}, where they elegantly delegate the task to Canonical Typology (more on this relationship in Sections~\ref{reiner:3} and~\ref{reiner:5}).
\end{sloppypar}

Turning now to DM, the theory is not particularly obvious with respect to its attitude towards a universal notion of word. Certainly, it is universalist in spirit and its hallmark is the (far-reaching) structural isomorphism between word structure and phrase/clause structure, which seems to leave no room for words as different from phrases/clauses. However, even in DM there is an endpoint of the derivation, e.g. /ba\textsc{i}z/; and other terminals in the same clause structure tree host their own derivations, e.g. /hi:/. In this sense, there are words as opposed to larger phrases. So, latently, DM does recognise words.

\textit{The question of psychological reality} can be phrased as: do formats of description (possibly including paradigms) have a role to play in our mental representation of language? This question is the least obviously answered in Autonomous Morphology. Again, the topic does not appear to receive much attention in the literature. My impression is that Autonomous Morphologists silently follow a weak version of a Language-as-an-abstract-object approach \citep{Katz1981}, i.e. it is simply not their intention to describe any mental representation of language as a system (Chomsky’s competence or Saussure’s langue), rather they focus on abstractions over the parole. For example, this position is suggested by \citet[197--198]{Aronoff2016}, where the high usefulness of paradigms in language description is enough to justify their role as a central tool for the linguist. Note that being useful for the linguist and being represented in the language user’s mind are not necessarily the same thing \citep[92, fn. 7]{Haspelmath2018}.\footnote{This is one of the few aspects of Haspelmath’s comparative concepts that I embrace, cf. \citet{ReinerInPress}.}

This is in sharp contrast to DM, which inherits the demand for providing a psychologically real model from its background in GB/Minimalist syntax. The demand concerns every aspect of the model, including the irrelevance of paradigms. However, pertinent psycholinguistic evidence seems to be scarce. Even \citet{BarnerBarner2002}, who do collect many pieces of psycholinguistic evidence from the literature, only cover one aspect of the model (viz. roots lack syntactic category information). There is a need for more comprehensive, custom-tailored experiments, in particular for ones checking whether or not speakers draw on ready-made paradigms in language production and/or comprehension.\largerpage

{\interfootnotelinepenalty=10000 In comparison to DM, the psycholinguistic evidence adduced in favour of CxM seems to be much more encompassing. In particular, it has been shown that word forms can be both, computable as well as holistically stored, without contradiction (\citealt{Zwitserlood2018}, \citealt[7–8]{MasiniAudring2019}, \citealt[Ch. 7]{JackendoffAudring2020}).\footnote{Typological coverage might be better; however this is a problem of current psycholinguistic research more generally, which may be overcome in the future (cf., e.g., \url{http://www.llf.cnrs.fr/labex-efl} (as of 24.03.2020)).} The crucial point from the perspective of CxM, though, is not that word forms \textit{can} be stored but that they \textit{are} in fact stored (different from DM, cf. \citealt[392, fn. 3]{McGinnis-Archibald2016}). If stored forms relate to each other, they may constitute paradigms in \citegen{Booij2016} sense.}

\textit{The attitude towards restrictiveness} seems to differ vastly between the theories discussed here. The difference is often said to lie in striving for maximal empirical coverage (CxM\,+\,Autonomous Morphology) vs. striving for testable predictions (DM). For example, \citet{Kramer2016} draws this line. However, I will argue that the difference is rather about \textit{where} to formulate restrictions.\largerpage

Surveying work in Autonomous Morphology, it would be utterly wrong to say that these researchers do not formulate restrictions: every generalisation over data is a falsifiable prediction to the effect that new data are hypothesised to comply with the generalisation. For example, \citegen[220–221]{BaermanBaerman2005} types of syncretism restrict the range of syncretisms we expect to find in the languages of the world. In short, there \textit{are} restrictions and they reside in generalisations over (descriptions of) data, with paradigms being used as a central tool for description. Another – more obvious and more recent – example of restrictiveness is \citet{Herce2019}.

Similarly, practitioners of CxM or Construction Grammar more generally use the construction as a maximally flexible tool of description \textit{before} they start looking for cross-linguistic tendencies in the data thus described (e.g., \citealt[Ch. 7–9]{Goldberg2006}). The tendencies are then to be explained by general cognitive principles \citep{Goldberg2006}. Crucially, cognitive principles restrict the range of what we expect to find in human (linguistic) behaviour but there is no need to build the restrictions into one’s descriptive tools.

This is different from DM, which, as a generative theory, intends to model language as a specific competence, identified by specific restrictions, which, consequently, have to be part of the model. Thus, anything that cannot be derived by the model is predicted not to be accepted by native speakers and vice versa (derivable $\leftrightarrow $ accepted).

So both kinds of approach acknowledge restrictions (and I am agnostic as to which way of doing so is the better one). Incidentally, both also embrace the liberties of language, at least to a certain extent. For DM, this might be not so obvious; however recall that they have list C at their disposal, although this part of the architecture seems to be the one that is worked out least of all.

Taking stock of this section, it appears that, surprisingly, Autonomous Morphology never patterns with CxM, cf. \tabref{tab:reiner:9}.

\begin{table}
\caption{Summary of theory comparison\label{tab:reiner:9}}
\begin{tabularx}{\textwidth}{QQQQ}
\lsptoprule
& {CxM} & {Autonomous Morphology} & {DM}\\
\midrule
{universal} {notion} {of} {word?} & rather no & latently yes & latently yes\\\tablevspace
{paradigms} {psychologically} {real} {and} {useful?} & maybe psychologically real, not fundamentally useful & useful & not psychologically real, occasionally useful\\\tablevspace
{morphology} {qualitatively} {distinct} {from} {syntax/lexicon?} & ? & yes & the little that there is: yes\\
\lspbottomrule
\end{tabularx}
\end{table}


\section{Defending morphology and paradigms} \label{reiner:3}

\begin{sloppypar}
Recalling \tabref{tab:reiner:9}, one might expect that Autonomous Morphologists defend their approach against CxM sense much more forcefully than they defend it against the various incarnations of DM. However, the opposite is true. The present section will elaborate on this remarkable situation and evaluate tentatively whether the Autonomists’ defence of morphology by itself – including paradigms – succeeds.
\end{sloppypar}

Virtually everyone who advocates the idea of Autonomous Morphology includes in their overview publications some words on why DM fails. For example, \citet[19–29]{BrownBrown2012} present DM as an alternative but inferior approach. Other examples include \citet[73–89]{Spencer2004}, \citet[194–195]{Aronoff2016} and, avant la lettre, \citet[82–85]{Aronoff1994}. Additionally, \citegen{Spencer2019} review of \citet{Embick2015} represents a recent argument against DM from an Autonomist’s perspective.

One of the central arguments is that there are phenomena that cannot be described in purely syntactic or lexical terms, i.e. so called morphomes (for a recent overview of the notion cf. \citealt[160–166]{Enger2019}). Crucially, the most prominent examples of morphomes directly refer to paradigms: inflectional classes and patterns of stem allomorphy \citep{Maiden2009}. For substantiation, consider an instance of the latter case, viz. \citegen{Maiden2009} L-pattern in Romance verb morphology. \tabref{tab:reiner:10} represents the pattern abstractly, \tabref{tab:reiner:11} gives an example.

\begin{table}
\captionsetup{margin=.05\linewidth}
\begin{floatrow}
\ttabbox{\begin{tabular}{lcc}
\lsptoprule
 & \textsc{prs.ind} & \textsc{prs.sbjv}\\\midrule
{1\textsc{sg}} & A- & A-\\
{2\textsc{sg}} & B- & A-\\
{3\textsc{sg}} & B- & A-\\
{1\textsc{pl}} & B- & A-\\
{2\textsc{pl}} & B- & A-\\
{3\textsc{pl}} & B- & A-\\
\lspbottomrule
\end{tabular}}
    {\caption{L-pattern \citep{Maiden2009}\label{tab:reiner:10}}}
\ttabbox{\begin{tabular}{lll}
\lsptoprule
 & \textsc{prs.ind} & \textsc{prs.sbjv}\\\midrule
{1\textsc{sg}} & digo    & diga\\  
{2\textsc{sg}} & dices   & digas\\
{3\textsc{sg}} & dice    & diga\\
{1\textsc{pl}} & decimos & digamos\\
{2\textsc{pl}} & decís   & digáis\\
{3\textsc{pl}} & dicen   & digan\\
\lspbottomrule
\end{tabular}}
    {\caption{Spanish example for L-pattern \citep{Maiden2009}\label{tab:reiner:11}}}
\end{floatrow}
\end{table}


Crucially, this pattern, among two additional ones, pervades irregular verbal morphology in Romance languages and comes without any obvious lexical or syntactic core: why should certain verbs show the same stem for exactly these feature value combinations and not for others? In particular, it is hard to imagine a semantic or syntactic property that is shared by, e.g. \textit{digo} and \textit{digan} but not \textit{digo} and \textit{dicen}. What the pattern does refer to, however, are the pairings of word forms and feature values they realise, i.e. the cells of a paradigm: it is always the same set of cells that shares a stem. So while the stem as such is not predictable, its distribution is and, crucially, this distribution refers to paradigm set-up. In short, morphomes seem to provide a very clear indication that, after all, there is something genuinely morphological about language, more precisely something that must be captured by paradigms in Stump’s sense (cf. \sectref{reiner:1}).\footnote{For a typological survey of morphomic structures cf. \citet{Herce2020}.}

In a way, this line of research is continued by \citet{AckermanAckerman2013}, who adduce evidence that it is the complexity of paradigm structure rather than the complexity of individual distinctions and realisations that predicts learnability (so here also the question of psychological reality is touched upon). However, their results have recently been called into question (\citealt{JohnsonEtAlPreprint}).

Moreover, recall that also in DM there is more between syntax and spell-out than just vocabulary insertion (\sectref{reiner:2.2}, \sectref{reiner:2.3}). Against this background, it does not come as a surprise that by now also proponents of DM have started to embrace morphomes. \citet{Trommer2016} provides a “postsyntactic morphome cookbook”, not even using the full DM-machinery. To be prudent, Trommer’s paper is the only DM-treatment of morphomes that I am currently aware of. Future research will show whether Autonomous Morphologists will be beaten at their own game.

A more urgent, but largely unrecognised threat comes from the “friendly take\-over” by constructionist accounts (\sectref{reiner:2.1} and \sectref{reiner:2.3}), especially when considering that even morphomic patterns might be conceived of as second order schemas involving homonymy. While, e.g., \textcites[]{Spencer2001}[84--86]{Spencer2004} welcomes constructionist thinking, he does not seem to recognise that this family of theories tears down the very boundaries he tries to defend: as indicated in \sectref{reiner:2.3} above, Construction Grammar does not recognise any qualitative difference between syntax and the lexicon, let alone a discrete level in between to be justly called morphology (cf. also \cites[5]{Goldberg2006}[17]{Goldberg2013}[1]{HoffmannHoffmann2013}).

As a consequence, it does not appear to be a coincidence that precisely those morphologists who adopt a constructionist perspective most consistently and most explicitly (e.g., \citealt{Booij2010}) are those who are \textit{not} at the same time Autonomists. Moreover, while (Autonomist) Network Morphology (\citealt{BrownBrown2012}) stands side-by-side with Canonical Typology in the Surrey Morphology Group,\footnote{Even literally so: \url{https://www.smg.surrey.ac.uk/approaches/} (as of 24.03.2020).} the latter theory is free to develop an integrative perspective on autonomy. That is, canonical (ideal) morphology might be regarded as \textit{non}{}-autonomous, reducible to either syntax or the lexicon or both, whereas deviations from this ideal represent autonomous morphology – with the deviations often presenting themselves as patterns in paradigms. To the extent that it is the deviations rather than the canonical ideal that we expect to find in real languages, Canonical Typology and Network Morphology fit well together indeed.\footnote{These deliberations are what I understand from \citet{GagliaGaglia2016}, \citet{Hippisley2017} and \citet{Herce2020b}.} This relationship, however, does not prevent Autonomous Morphology, including its commitment to paradigms, from being largely absorbed by constructionist approaches, as described above.

In sum, so far the attempts at defending morphology (as a distinct level of description) and defending paradigms (as defined in the introduction of this paper) have not ultimately succeeded – or the attempts are missing altogether. Does this mean that paradigms have nothing to offer for any linguistic theory? To my mind, one use of paradigms remains in any case. This is the topic of the next section.

\section{A remaining use for paradigms: Restricting recursion} \label{reiner:4}

The marginalisation of paradigms as fundamental organisational units has been noted and criticised recently by \citet{Diewald2020a}, focussing on constructionist approaches. Arguing from a diachronic perspective, she reminds us that [+\,par\-a\-dig\-mat\-i\-city] and [\textminus\,paradigmatic variability] are two of \citegen[132]{Lehmann2015} six famous criteria for grammaticalisation (and that speaking of grammaticalisation is pointless without an analytical distinction between grammar and non-grammar). Thus, in Diachronic Construction Grammar paradigms are both: marginalised as well as fundamentally needed. To escape from this dilemma, she suggests the following concept of paradigms: as constructions-of-constructions they are complex signs, which may constitute nodes in a constructional network. She notes that this is similar to \citet{Booij2016}; however, as far as I can see, \citet[297--301]{Diewald2020a} goes one step further in that she explicitly excludes open class paradigmatic relations from the notion of paradigm. This makes, I would argue, a huge difference, in particular from a diachronic perspective. For example, consider German \textit{R}/\textit{richtung} ʻdirection/toʼ in \REF{ex:reiner:23} vs. German \textit{R}/\textit{riesen} ʻgiantʼ in \REF{ex:reiner:24}.

\ea \label{ex:reiner:23} 
{German (constructed)}\\
\gll Ich  geh     Richtung        Bahnhof.\\
     I    walk    direction/to    station\\
\glt ʻI am walking towards the station.ʼ
\ex \label{ex:reiner:24} 
{German (constructed)}\\
\ea Dort  ist  eine  Riesen-Statue.\\
     there  is  a  giant-statue\\
\glt ʻThere is a giant statue.ʼ
\ex  Dort  ist  eine riesen  Statue.\\
     there  is  a giant  statue\\
\glt ʻThere is a giant statue.ʼ
\z \z

The former would qualify as grammaticalised on \citegen{Diewald2020a} account since it enters the closed class of prepositions. Note that I am assuming here that this class may be described by pairs of feature values. The latter, by contrast, would not qualify as grammaticalised on \citegen{Diewald2020a} account since there is no closed class of either “grading initial parts of compounds” or adjectives.

Thus, \citegen{Diewald2020a} conception of paradigms might reintroduce them into Construction Grammar as fundamental organisational units. Crucially, however, as nodes in constructional networks they have to be psychologically real by themselves (not merely being a consequence of some other psychologically real entities and processes). Here, \citet[306--310]{Diewald2020a} provides a rough design for a psycholinguistic study as well as a core linguistic argumentation to the effect that certain grammaticalisation processes are hardly conceivable without mental representations of paradigms in her sense.

So again, the question whether paradigms (in whichever sense) are psychologically real turns out to be pivotal. To the extent that this remains a tricky question, I suggest that anyone arguing in favour of a fundamental status of paradigms should look for arguments independent of this question, i.e. for a use of paradigms that is just that: fundamental to the linguist but not necessarily reflected in the language user’s mind (very much in the spirit of \citealt[197--198]{Aronoff2016}, quoted above). In the rest of this section, I will present such a use.

There is one kind of data that neither CxM nor DM is good at handling, i.e. limits on recursion. The following paragraphs provide a snapshot of such data, explains why they are problematic for the two approaches and, finally, presents the paradigm in Stump’s sense as a solution.

Basically, CxM and Construction Grammar more generally share with Autonomous Morphology the strategy to seek restrictions in generalisations over data rather than building the restrictions directly into their descriptive tools (cf. \sectref{reiner:2.3}). This strategy adheres to \citegen{Haspelmath2004} postulate of a sharp distinction between description and explanation. At the same time, this strategy requires the descriptive tools to be as flexible as possible: they must be able to capture whatever may be found in a language. So the often heard accusation that “everything is a construction” (e.g.,  \citealt[236]{VanValin2007}) misses the point since here lack of restrictiveness in descriptive tools is a virtue, not a weakness.

Against this background, it is clear that CxM can handle every kind of data well: all we need is an association between a form and a meaning, i.e. a construction. This means that individual morphs like, e.g., thematic vowels are not forced to have any meaning in isolation. It is the verb form as a whole that has a meaning to begin with and this association of meaning and form constitutes a construction. Likewise it is the clause as a whole that has a meaning and again, this association constitutes a construction. Incidentally, a construction may have constituent constructions and in this sense there is internal structure; however the internal structure is not generated bottom-up via valency or subcategorisation.

So verb forms are constructions, clauses are constructions, and the rare case of a monofunctional affix (that is, rare in SAE) is a construction, too. Ideally, we obtain a comprehensive network of constructions for each language. These networks, I hold, are again organised in a network to the extent that they share certain properties.\footnote{For a pertinent but rather non-constructionist proposal, cf. \citealt{ReinerInPress}.} Crucially, these properties are abstractions over individual minds rather than having any psychological reality in their pure form. In principle, the most encompassing of networks might be read as set of generalisations about language. As argued above (\sectref{reiner:2.3}.), generalisations \textit{are} predictions, including predictions about what there is \textit{not}. To take an example from Autonomous Morphology, if there is no language to be found with syncretism of 1\textsc{sg}, 2\textsc{du}, and 3\textsc{pl} subject agreement,\footnote{Surrey Person Syncretism Database, \url{https://www.smg.surrey.ac.uk/personsyncretism/}.} we may predict that this kind of syncretism is impossible. The next step is gathering more data (from the same as well as other languages) in order to see whether the prediction of absence holds. However, most practitioners of CxM and Construction Grammar more generally do not seem to be concerned too much with that “dark matter” (I am borrowing the metaphor from \citealt{Werner2018} on word formation here). To pick a recent example, the ongoing project “FrameNet \& Konstruktion des Deutschen” focuses, according to its self-description, on the wealth of what there is, not on finding systematic gaps.\footnote{\url{https://gsw.phil.hhu.de/}{;} the project relates to parallel projects for other languages.} This might seem to be a mere matter of emphasis – but if we do not even care for the potential absence of certain phenomena we will not be able to explain (in any sense of the word) that absence.

Beside a certain type of syncretism, another example for a potentially non-occurring kind of phenomenon is the one I am concerned with in the present section: meaningful iteration of affixes (presuming here that we know in each case what counts as an affix as opposed to a clitic or function word and that we can always decide whether the affix is inflectional or derivational). Consider \REF{ex:reiner:25} based on \REF{ex:reiner:26}.

\ea \label{ex:reiner:25}
{Turkish (p.c., Seda Yilmaz Wörfel)}\\
\gll  yap-tı-m\\
      do-\textsc{pst}{}-1\textsc{sg}\\
\glt ʻI didʼ
\ex \label{ex:reiner:26} 
{Turkish (p.c., Seda Yilmaz Wörfel)}\\ 
\gll \textup{*} yap-tı-tı-m\\
     {} do-\textsc{pst}{}-\textsc{pst}{}-1\textsc{sg}\\
\glt intended: ʻI had doneʼ
\z

Judging from the “syntax-like” systematicity of Turkish verbal forms and from the possibility to iterate derivational (causative) affixes (cf. \sectref{reiner:2.1}), the unavailability of a systematic iteration in \REF{ex:reiner:26} is surprising. Why not have a past form (\textit{yaptım}) as the input to another past formation (-\textit{tı}), yielding a past-of-past meaning? In other words: if we define recursion as the application of a given structural-semantic operation to an output of some former application of the same operation, example \REF{ex:reiner:26} represents the limits of recursion. In fact, I am not aware of any language that allows for recursion of inflection (or what is usually described under the rubric of inflection).\footnote{But cf. \citet{Voort2016} for potential examples. Importantly, it has to be the whole structural-semantic operation that is applied twice. Hence, for instance reduplicative plurals do not count, since here we are dealing with one structural operation having one semantic effect.} So the limitation seen in \REF{ex:reiner:26} might be quite general. For present purposes, however, the important point is that there is a limitation at all.

Thus, this is a kind of language fact that CxM will capture implicitly but not care for. What about DM? Striving for a restrictive model, this approach is expected to predict the ungrammaticality of \REF{ex:reiner:26} explicitly. At first sight, this is true: as long as there is only one T-node in the syntax, the pattern in \REF{ex:reiner:26} is indeed excluded, since there is simply no terminal for the second -\textit{tı} to attach to.

However, in DM the problem resurrects at a later stage of the derivation: after vocabulary insertion has selected the most specific candidate available, how can we stop the mechanism from starting over again, yielding for example *\textit{play}{}-\textit{ed}{}-\textit{ed} \citep[97]{Embick2015}? Note that this is not just an empty iteration of forms, since every vocabulary item, including [\textsc{past}] $\leftrightarrow $ /əd/, comes with a function. More precisely, a vocabulary item is the pairing of a phonological form and information on where this form may be inserted (cf. \sectref{reiner:2.2.1}.), with this information consisting of morphosyntactic feature values. So when [\textsc{past}] $\leftrightarrow $ /əd/ attaches to the single T-node multiple times (in principle, ad infinitum), it adds [\textsc{past}] every time. Embick’s solution to this problem is a stipulation called \textit{uniqueness}: “In a derivation, only one Vocabulary Item may apply to a morpheme” \citep[98]{Embick2015}. So DM can handle data like \REF{ex:reiner:26} but only by means of a stipulation.

According to \citegen{Spencer2019} review of \citet{Embick2015}, the need for the uniqueness stipulation arises from the “attempt to derive word structure (directly) from syntactic structure” \citep[218]{Spencer2019}. I tried to show above that syntax is not the problem here. However, I agree with \citet{Spencer2019} that paradigm-based theories are not affected by the problem of recursion. The reason is that the paradigm as such restricts recursion: since rows and columns represent different pieces of information, iteration is excluded automatically. For example, since \textsc{past} will not appear in both, a row and a column, no cell can contain a realisation of \textsc{past}{}-\textsc{past}. Consider the partial paradigm for Turkish in \tabref{tab:reiner:12}.

\begin{table}
\caption{Partial paradigm for Turkish, gathered from \citet[Ch. 2.1.3]{Kornfilt1997}.\label{tab:reiner:12}}
\begin{tabular}{llll}
\lsptoprule
               & {\textsc{fut}} & {\textsc{rep pst}} & {\textsc{pst}}\\\midrule
{\textsc{1sg}} & {}-(y)AcAK{}-Im &     {}-mIs-Im &    {}-DI-m\\
{\textsc{2sg}} & {}-(y)AcAK{}-sIn &    {}-mIs-sIn &   {}-DI-n \\
{\textsc{3sg}} & {}-(y)AcAK{}-∅ &      {}-mIs-∅   &   {}-DI-∅\\
{\textsc{1pl}} & {}-(y)AcAK{}-Iz &     {}-mIs-Iz &    {}-DI-k\\
{\textsc{2pl}} & {}-(y)AcAK{}-sInIz &  {}-mIs-sInIz & {}-DI-nIz\\
{\textsc{3pl}} & {}-(y)AcAK{}-lAr &    {}-mIs-lAr &   {}-DI-lAr\\
\lspbottomrule
\end{tabular}
\end{table}

Thus, the paradigm can do what Construction Grammar fails to do and what DM needs a stipulation for: modelling the limits of recursion. Another advantage of paradigms is that no one needs to postulate them. They do have a merely secondary status in CxM as well as in DM (cf. \sectref{reiner:2}); however, they are there.

Before concluding the paper, let me draw your attention to a limitation of my reflections: I focussed solely on phenomena that are usually considered to be inflectional in a narrow sense. Though presenting theories that can as well cope with periphrasis and word formation, I did not have much to say about these phenomena. Indeed, these appear to be inherently different from inflection when it comes to recursion: they do permit it, to a certain extent. For recursion in periphrasis cf. \citet{Rothstein2012, Rothstein2013, Rothstein2013b} on double futures and the references therein to the wealth of works on double perfects. For recursion in word formation cf., e.g., \citet{BratticoEtAl2007}.

\section{Conclusion} \label{reiner:5}\largerpage

The “Morphome Debate” (\citealt{LuísEtAl2016}) is far from settled and it remains to be seen whether morphomes provide an ultimate argument for morphology as a distinct level of description, organised by paradigms in Stump’s sense. In the present paper I only presented an outline of this debate, not even elaborating on inflectional classes as morphomes. The main point of the paper was showing that even if paradigms are becoming secondary in current theorising, the traditional paradigm in Stump’s sense still serves an apparently unique function: it provides an economic way of modelling restrictions on recursion in inflection. More precisely, the habit of having the rows and columns host different features (and a fortiori different feature values) prevents any given feature value from operating on some former application of itself like in *[\textit{play}{}-\textit{ed}]-\textit{ed}.

However, recall that paradigms in Stump’s sense are pairings of word forms and the morphosyntactic properties they realise. Thus, an old problem raises its ugly head again, i.e. the very problem with which \citet{Haspelmath2011} is concerned: how can we tell what constitutes a word (form) in the first place? I doubt that the delegation of this task to Canonical Typology (cf. \sectref{reiner:2.3}) is sufficient. The only thing that, in this respect, Canonical Typology can tell us about a given string in a given language is this: it is a canonical word (form) to such and such a degree. However, in order to decide whether the string can realise a paradigm cell, we need to know whether it \textit{is} a word or not, categorically.

\section*{Abbreviations}
\begin{sloppypar}
Please note that abbreviations in examples adopted from other authors are spelled out in footnotes throughout.
\end{sloppypar}

\begin{multicols}{2}
\begin{tabbing}
\textsc{neutr} \hspace{1ex} \= neuter \kill
1 \> first person \\
2 \> second person \\
3 \> third person \\
\textsc{f} \> feminine \\
\textsc{du} \> dual \\
\textsc{gb} \> Government and Binding \\ \> Theory (Principles and \\ \> Parameters) \\
\textsc{imp} \> imperative \\
\textsc{impf} \> imperfective past tense \\
\textsc{ind} \> indicative \\
\textsc{irr} \> irrealis \\
\textsc{l2} \> second language \\
\textsc{masc} \> masculine \\
\textsc{neutr} \> neuter \\
\textsc{nom} \> nominative \\
\textsc{num} \> number \\
\textsc{pl} \> plural \\
\textsc{prs} \> present \\
\textsc{ps} \> person \\
\textsc{pst} \> past \\
\textsc{ptcp} \> participle \\
\textsc{sae} \> Standard Average European \\ \> \citep{Haspelmath2001} \\
\textsc{sbjv} \> subjunctive \\
\textsc{sg} \> singular \\
\textsc{tns} \> tense \\
\textsc{v} \> verb; vowel
\end{tabbing}
\end{multicols}

{\sloppy\printbibliography[heading=subbibliography,notkeyword=this]}
\end{document} 
