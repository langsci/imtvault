\documentclass[output=paper, colorlinks,citecolor=brown]{langsci/langscibook} 
\ChapterDOI{10.5281/zenodo.5675843}
\author{Jaakko Leino \affiliation{University of Helsinki}}
\title{Formalizing paradigms in Construction Grammar}
\abstract{Construction Grammar sees the language system as consisting solely of conventionalized pairings of form and meaning, i.e. constructions. Constructions may be of any size and complexity, and they may be abstract (or schematic) to any degree. They may be templates for sentences, lexical items, inflectional morphemes, discourse patterns \citep{Östman2005} that organize whole texts or even genres, etc. However, the notion of constructions seems incapable of capturing patterns found within the grammar: systematic similarities between constructions and, notably, paradigms of different sorts. For instance, inflection paradigms consist of sets of constructions, but nothing in common varieties of Construction Grammar explains how those constructions join together to form a paradigm. The paper argues that in addition to constructions, the language system must also include specifiable relations which hold between the constructions of a language and which organize them into a functional system. Crucially, such relations are necessary for the organization of paradigms, be they of morphological, syntactic or other nature. Relations between constructions within the grammar can be -- and have previously been -- described in terms of inheritance (e.g. \citealt{Goldberg1995}), taxonomic and meronomic links \citep{Croft2001}, and the like. However, such very abstract links can only capture simple relations between constructions. Yet, more complex relations, notably of an analogical nature, exist widely within the grammar of apparently all human languages. To capture such analogical relations, the paper uses the notion of \textit{metaconstruction}, briefly introduced in \citet{LeinoÖstman2005}. Metaconstructions may be thought of as generalizations of constructions, partly in the same sense as constructions may be seen as generalizations of actual expressions. It will be argued that such analogical relations, formalizable as metaconstructions, hold paradigms together and also facilitate both producing and interpreting complex expressions.}

\begin{document}
\maketitle 

\section{Introduction} \label{leino_sec1}

Construction Grammar sees the language system as consisting solely of conventional pairings of form and meaning, i.e. constructions. Constructions may be of any size and complexity, and they may be abstract (or schematic) to any degree. However, the notion of construction seems incapable to capture patterns found within the grammar: systematic similarities between constructions and, notably, paradigms of different sorts.

Besides holding paradigms together, analogical relations also facilitate both the production and interpretation of complex expressions and serve as a notable source of linguistic creativity and innovation. Systematic analogical structures often both show existing gaps in the language system and provide means of coining novel but instantly comprehensible ways of filling such gaps. The mechanism is ubiquitous in language, but it seems to lead to (conceived) paradigms only in certain parts of grammar. This, in turn, may be revelatory of the nature of paradigms.

If one takes as a starting point the claim, often made in Construction Grammar (e.g. \citealt{FillmoreKay1995}: 1·15–16, \citealt{Goldberg1995}: 1–5, \citealt{Croft2005}: 273–275), that grammar consists of constructions, an obvious question arises: How are the constructions of a given language organized? Is the grammar of a given language merely a “warehouse” or an inventory of constructions, from which a language user picks out whatever is necessary to produce an utterance?

Cognitive Grammar, a close relative of Construction Grammar,\footnote{Some, e.g. \citet{Goldberg2006}, go as far as to consider Cognitive Grammar a variant of Construction Grammar.} conceptualizes grammar as a \textit{structured inventory of linguistic units} \citep[73]{Langacker1987}. In this view, grammar is organized mainly in terms of \textit{categorization:} the inventory of linguistic units is structured into schematic networks. The relation of \textit{symbolization} also structures the inventory by establishing correspondencies between particular semantic structures and phonological structures: a symbolic relation is necessarily present in every linguistic unit of a language, as these units are taken to be inherently bipolar, i.e. to represent a conventionalized correspondence of form and meaning. (For a more detailed discussion, see \citealt{Langacker1987}: 73–76.)

Cognitive Grammar greatly resembles Construction Grammar in many important aspects (cf. \citealt{Leino2005a}, \citealt{Croft2001}: 6–7), and Langacker’s characterization of grammar could therefore conceivably be rephrased as a \textit{structured inventory of grammatical constructions.} And, indeed, \citet[258]{Tomasello2006} does so (with regard to “Cognitive-Functional Linguistics” in general, but explicitly including Construction Grammar in this notion): “In this approach, mature linguistic competence is conceived as a structured inventory of meaningful linguistic constructions -- including both the more regular and the more idiomatic structures in a given language (and all structures in between).”

\begin{sloppypar}
Thus, besides constructions, the language system must include specifiable relations which hold between the constructions of a language and which organize them into a functional system. Such relations have been described in terms of inheritance (e.g. \citealt{Goldberg1995}), or taxonomic and meronomic links (e.g. \citealt{Croft2001}). However, such very abstract links can only capture simple relations between constructions. More complex relations, notably analogical ones, exist widely within the grammar of apparently all human languages. Analogical relations may be captured by the notion of metaconstruction (briefly introduced in \citealt{LeinoÖstman2005}).
\end{sloppypar}

It is not clear, however, how either Cognitive Grammar or Construction Grammar would express relations between expression types which show obvious similarities but which cannot be said to be in a schematic relation to each other. For example, English active and passive sentences are obviously related expression types, but neither is schematic with regard to the other. Similarly, English assertive sentences and questions are related in a very similar manner, but this relationship cannot be captured in terms of categorization or schematicity either:

\ea \label{ex:leino_1} 
{English}\\
\ea
     \label{ex:leino_1a}  {John built the house.}
\ex \label{ex:leino_1b} 
      {The house was built by John.}
\z
\ex \label{ex:leino_2} English\\
 \ea
     \label{ex:leino_2a}  {Lisa has met my wife}
    \ex \label{ex:leino_2b} 
      {Has Lisa met my wife?}
\z
\z

In both cases, it is clear that the (a) and (b) sentences are related to each other. This relatedness is not incidental but systematic: it is not only the sentences that are related but the sentence \textit{types,} i.e. the constructions -- or, in Cognitive Grammar terms, constructional schemas -- behind the sentences that are related. However, this relatedness cannot be captured as an organizing feature of the grammatical system with the tools provided by Cognitive Grammar.\footnote{This is, of course, not to say that Cognitive Grammar is unable to analyze these sentences or even address the essential similarities between them. The problem that I wish to point out concerns the internal organization of the grammar and the lack of tools in both Cognitive Grammar and Construction Grammar to describe this internal organization in sufficient detail and systematicity.}

Essentially the same situation holds for Construction Grammar as well. Relations between constructions within the grammar can be -- and have been -- described in terms of inheritance links (e.g. \citealt{Goldberg1995}: 73–81, \citealt{MichaelisLambrecht1996}: 235–245, \citealt{Croft2001}: 53–57). However, such very abstract links only capture certain rather simple relations between constructions in terms of what is made of what and what is a part of what. In addition to the aforementioned inheritance links, a complementary mechanism should be taken into use to capture such relations which cannot be described by means of simple networks of schematicity and part–whole relations.

Recently \citet{Diessel2019} has presented a greatly improved way of representing inter-constructional relations within the grammar. His book specifically devotes to the analysis of grammar as a complex network of interconnected constructions, and greatly improve our understanding of that challenging topic. He also describes (e.g. pp. 18–19 \& chap. 11) paradigms in terms of emerging networks of constructions, a view which I shall also adopt in the following. Yet, I feel, there is still more to be said about different types of relations between constructions, and the complex architecture of grammar.

As we saw above, there are other kinds of similarities between the constructions of a language as well. Notably, relationships of an \textit{analogical} nature exist widely within the grammar of apparently all human languages. If the theoretical machinery that we use isn’t sufficient for capturing these similarities, then the description of the language in question will miss possible generalizations, and will thereby not conform to the requirement of full coverage spelled out by \citet{Kay1995}.

To capture such generalizations, I shall make use of the notion of \textit{metaconstruction,} briefly introduced in \citet{Leino2003} and \citet{LeinoÖstman2005}. Metaconstructions may be thought of as generalizations of constructions, partly in the same sense as constructions may be seen as generalizations of actual expressions. Any given construction may be related to other constructions in the language by means of such metaconstructions. Ultimately, the language system will not appear as an unstructured list of constructions, but rather as a structured system in which a certain kind of order prevails.

One notable point of relevance for metaconstructions are paradigms of different sorts within the language system. For the purposes of this paper, a paradigm is seen as a set of constructions which has the following properties:

\begin{enumerate}
\item The members of the paradigm, i.e. the constructions which make up the paradigm, are alternatives to one another in a given linguistic context.
\item The set of constructions together make up a structured inventory of ways to express variations of a given meaning in that context.
\item The set of constructions forms a meaningful whole which “makes sense” to native speakers of the language.
\end{enumerate}

Metaconstructions have a notable role not only in the internal organization of grammar but also in the production of novel types of utterances. In other words, they do more than merely organize the system in a synchronic sense statically: they also serve as dynamic and diachronically relevant instructions of how to form new constructions. More specifically, metaconstructions capture analogical relationships, which have been shown to be of great importance for the internal organization of the language system and the creative use of constructions (e.g. \citealt{Leino2003}: 260–284), as well as syntactic creativity in general (e.g. \citealt{Tabor1994}: 202–205) -- not to mention the importance of analogy in language and linguistic description more generally (cf. e.g. \citealt{Bloomfield1933} for a linguistic classic which emphasizes the importance of analogy, and \citealt{Anttila1977} and \citealt{Itkonen2005} for more general accounts of analogy in language).

\section{Some cases in point} \label{leino_sec2}

In what follows, I shall discuss two cases where analogical relations between constructions have a crucial role in the functioning of the language in question. The first case (also presented in \citealt{LeinoÖstman2005}), in \sectref{sec:leino_2.1}, concerns an arising variation in Finnish subject and object case marking. The second one, in \sectref{sec:leino_2.2}, is the relationship between assertions and questions referred to above, but in the context of the Swedish language. In addition, these sentence types will be discussed with regard to active and passive voice.

\subsection{Metaconstructions and Finnish case marking}\label{sec:leino_2.1}

Finnish subject and object case marking provides a clear-cut example of how analogical relations between constructions affect the functioning of the language system. Briefly stated, Finnish is an accusative language, but it also has a peculiar ergative-like subsystem marked with the partitive case and related to the boundedness of the object, as well as the resultativity -- or, more precisely, telicity -- of the activity denoted by the predicate (for details, see e.g. P. \citealt{Leino1991}, \citealt{Karlsson1999}, \citealt{Heinämäki1984}).

For the purposes of this paper, we may state the following simplified rule of thumb: an object in Finnish is marked with the accusative case if the sentence is not negated and the activity denoted by the verb is telic, i.e. if the activity is carried out completely. If the sentence is negated or the activity is atelic, the object is marked with the partitive case.

To further complicate the Finnish case marking system, the subject -- which is normally in the nominative case -- may also be marked with the partitive case. However, this is traditionally said to require that the following conditions be met:

\begin{itemize}
\item the sentence must be intransitive \textit{and}
\item the sentence must be negated \textit{or}
\item the subject must be a mass noun (or an abstract noun, or a plural) and unbounded.
\end{itemize}

In other words, transitive sentences in Finnish cannot, traditionally speaking, have a partitive subject. However, transitive sentences with partitive subjects do in fact show up sporadically, although normative grammars do not allow them and the vast majority of Finnish speakers find them ungrammatical, or awkward at best.

In terms of Construction Grammar, the Finnish language may be said to have a transitive sentence construction which licenses sentences like those in \REF{ex:leino_3}:\footnote{Finnish has a rich case inflection morphology. A list of abbreviations which indicate morphological case is printed at the end of this chapter. Ablative, allative, essive and comitative do not occur in the examples.}


\ea \label{ex:leino_3}
{Finnish}\\
\ea
\gll Lapset              rikkoivat           ikkunan. \label{ex:leino_3a}\\
     child-\textsc{pl.nom}  break-\textsc{pst.3pl}  window-\textsc{acc}\\
\glt `The children broke the/a window.'%\z

\ex
\gll Miehet            kaatoivat                puut. \label{ex:leino_3b}\\
     man-\textsc{pl.nom}  cut-down-\textsc{pst.3pl}  tree-\textsc{pl.acc}\\
\glt `The men cut down the trees.'
\z \z

This construction may be characterized schematically as in \figref{fig:leino_fig1} or, in the traditional boxes-within-boxes notation as in \figref{fig:leino_fig2}.

\begin{figure}{}
     [S\textsc{\textsubscript{nom}} V O\textsc{\textsubscript{acc}}]
    \caption{Transitive sentence\label{fig:leino_fig1}}
\end{figure}

\begin{figure}
\includegraphics[width=.75\textwidth]{figures/Leino-fig2.png}
\caption{Transitive sentence, box notation\label{fig:leino_fig2}}
\end{figure}

In other words, this construction licenses transitive sentences with a nominative case subject and an accusative case object.

The Finnish language also has an intransitive sentence construction which licenses sentences like the following:

\ea\label{ex:leino_4}
{Finnish}\\
\ea
\label{ex:leino_4a}
\gll Lapset              leikkivät  pihalla.\\
     child-\textsc{pl.nom}  play\textsc{{}-3pl}  yard-\textsc{ade}\\
\glt `The children are playing in the yard.'%\z

\ex\label{ex:leino_4b}
\gll Puut               kaatuivat                  myrskyssä.\\
     tree-\textsc{pl.nom}  fall-down-\textsc{past.3pl}  storm-\textsc{ine}\\
\glt `The trees fell down in the storm.'
\z
\z

Example \REF{ex:leino_5a} is a basic intransitive sentence with a nominative subject, where\-as  \REF{ex:leino_5b} is an intransitive sentence with a partitive subject:

\ea\label{ex:leino_5}
{Finnish}\\
\ea
\gll Pihalla      juoksee  poikia.\\
     yard-\textsc{ade}  run\textsc{{}-3sg}  boy-\textsc{pl.nom}\\
\glt `There are boys running on the yard.'\label{ex:leino_5a}


\ex \label{ex:leino_5b}
\gll Myrskyssä  kaatui                        puita.\\
     storm-\textsc{ine}   fall-down-\textsc{past-3sg}  tree-\textsc{pl.par}\\
\glt `(Some) trees fell down in the storm.'
\z
\z


The similarity between these two sentence types is all the greater due to the fact that, while Finnish allegedly has “free” word order (i.e. one that mostly expresses information structure rather than grammatical relations, see e.g. \citealt{Vilkuna1989}), in a neutral context the partitive subject of the intransitive sentence follows the verb like the partitive object of the transitive sentence.

The constructions which license examples \REF{ex:leino_4a} and \REF{ex:leino_4b} are connected together by a metaconstruction which may be characterized by \figref{fig:leino_fig3}.

\begin{figure}{}
     [[S\textsc{\textsubscript{nom}} V X] $\leftrightarrow $ [X V S\textsc{\textsubscript{par}}]]
    \caption{Metaconstruction between nominative and partitive subject}
    \label{fig:leino_fig3}
\end{figure}

In \figref{fig:leino_fig3}, \textit{S} stands for the subject, \textsc{nom} and \textsc{par} for the nominative and partitive case, \textit{V} for the predicate verb, and \textit{X} for a potential other argument. The same information may be expressed in the boxes-within-boxes notation (greatly simplified) as in \figref{fig:leino_fig4}. 

\begin{figure}
    %\includegraphics[width=\textwidth]{figures/Leino-fig4.png}
    \fbox{
        \fbox{S\textsubscript{NOM}} 
        \fbox{
            \fbox{V} 
            \fbox{X}
        }
    } \hspace{1em} \fbox{~\strut{↔}~} \hspace{1em} 
    \fbox{
        \fbox{
            \fbox{X} 
            \fbox{V}
        } 
        \fbox{S\textsubscript{PAR}}
    }
    \caption{Metaconstruction between nominative and partitive subject, box notation}
    \label{fig:leino_fig4}
\end{figure}

However, the Finnish language does not have a construction that is connected with this metaconstruction to the construction exemplified by the sentences (\ref{ex:leino_3}a--b); i.e., as pointed out above, transitive sentences in Finnish cannot have partitive subjects. Yet, it is very easy to note, on the basis of that construction and this metaconstruction, that such a construction would have the form expressed in Figures \ref{fig:leino_fig5} and \ref{fig:leino_fig6}.

\begin{figure}
\begin{floatrow}
\captionsetup{margin=.05\linewidth}
\ffigbox{ * [O\textsc{\textsubscript{acc}} V S\textsc{\textsubscript{par}}]}
        {\caption{Transitive sentence with partitive subject\label{fig:leino_fig5}}}%
\ffigbox{\fbox{
        \fbox{
            \fbox{O\textsubscript{ACC}}
            \fbox{V}
        }
        \fbox{S\textsubscript{PAR}}
    }}
    {\caption{Transitive sentence with partitive subject, box notation\label{fig:leino_fig6}}}
\end{floatrow}
\end{figure}

This construction would license such sentences as those in (\ref{ex:leino_6a}a–b):

\ea\label{ex:leino_6a}
{Finnish}\\
\ea[*]{%
\gll Pizzan        söi                   poikia.\\
       pizza-\textsc{acc}  eat-\textsc{past-3sg}  boy-\textsc{pl.par}\\
\glt `(Some) boys ate the pizza' or: `the pizza was eaten by boys'}

\ex[*]{\label{ex:leino_6b}%
\gll Puut              kaatoi                       miehiä.\\
      tree-\textsc{pl.acc}  cut-down-\textsc{past.3sg}  man-\textsc{pl.par}\\
\glt `(Some) men cut down the trees' or: `the trees were cut down by men'}\z\z


In \REF{ex:leino_6a}, the sentences have been marked as ungrammatical. However, a language user might well wish to express a transitive event with an unbounded subject -- which would rather naturally be coded as a transitive sentence with a partitive subject, if only this were grammatical in Finnish. In other words, the construction sketched out in Figures~\ref{fig:leino_fig5} and~\ref{fig:leino_fig6} would be a natural tool for expressing such a meaning.

Since the language already has, so to speak, all the “ingredients” for such a construction, it would not be difficult to coin such a construction and start using it. And, in fact, this is happening in the Finnish language at the moment. Sentences like in \REF{ex:leino_6a} do in fact show up not only in colloquial language but also in newspaper headlines and practically all registers of the Finnish language, though only very sporadically:

\ea\label{ex:leino_7a}
{Finnish}\\
\ea[*]{
\gll Tuhansia              Soneran       piensijoittajia                jätti                   käyttämättä  merkintäoikeutensa                                       Soneran       annissa.\\
        thousand-\textsc{pl.par}  Sonera-\textsc{gen}  minor.investor-\textsc{pl.par}  leave-\textsc{past-3sg}  use-\textsc{inf3-abe}  right.to.subscribe.for.shares-\textsc{acc.ps3sg/pl}  Sonera-\textsc{gen} rights.offering-\textsc{ine}\\
\glt `Thousands of Sonera’s minor investors left their share subscription right unused in the Sonera stock rights offering.' {(\textit{Helsingin Sanomat,} 11/24/2001)}}

\ex[]{\label{ex:leino_7b}
\gll Minkä        maan             jalkapalloilijoita   haki                    viime  viikolla  turvapaikkaa  Suomesta?\\
     what-\textsc{gen}  country-\textsc{gen}  footboller-\textsc{pl.par}  apply-\textsc{past.3sg}  last     week     asylum-\textsc{par}    Finland-\textsc{ela}\\
\glt `What country where the football players from who sought asylum in Finland last week?' {(\textit{Uutislehti 100} 8/25/2003)}}
\z\z

In other words, what is happening in the Finnish language in this respect is essentially that the existing constructions, and generalizations based on them, are coupled in such a way that a new construction is taken into use. Actually, this is a rather ordinary case of \textit{analogy}, and the metaconstruction I sketched out serves as an \textit{analogy model} here.\largerpage

Metaconstructions may thus have a role in diachronic change in that they motivate new constructions through several existing constructions and their systematic similarities and differences. However, two points of clarification are in place with regard to the role of metaconstructions in this process. First, they are not the direct cause of the change: the emergence of a new grammatical construction stems primarily either from the need to express a new kind of meaning or from the tendency towards systematicity and simplicity in grammar. Secondly, metaconstructions do not serve as the goal, or the target structure, in such a change. Rather, they provide a structured analogy which serves to motivate (perhaps initially single ad hoc utterances which may then give rise to) the target construction.

\subsection{Assertions, questions, voice, and metaconstructions}\label{sec:leino_2.2}

As we saw with example \REF{ex:leino_2} at the beginning of this paper, there is an obvious similarity between assertive sentences and questions in English. Swedish shows a very similar relationship between assertions and questions. This relationship is systematic rather than incidental; that is, the same similarity holds for each question and a corresponding assertion. Therefore, it is plausible to say that assertions and questions as sentence types, i.e. constructions, are related in some manner.

More generally, not only different kinds of interrogative sentences but sentence types in general form a wide network of different but interrelated constructions. This network includes assertive sentences, several types of question sentences, and a number of other sentence types as well. In fact, at a yet more general level, the entire grammar of a language may be represented as a network of interrelated constructions (perhaps much in the same manner as suggested in \citealt{Diessel2019}) made up of individual constructions and relationships between them which organize the network. Often, as in the case of sentence types, there are parts of the network which may be seen as “subsystems” or, indeed, paradigms.

For the sake of clarity, I shall only refer to yes/no questions here. Of course, the discussion here holds (mutatis mutandis) for other types of questions as well, provided that we take each question type to be a separate construction.

Let us consider the following pairs of sentences:\largerpage[1.75]

\ea\label{ex:leino_8a}
{Swedish}\\\multicolsep=.125\baselineskip
\begin{multicols}{2}
\ea
\gll Du  läste        boken.\\
     you read-\textsc{pst}  book-\textsc{def}\\
\glt `You read the book.'
\ex\label{ex:leino_8b}
\gll Läste       du    boken?\\
     read-\textsc{pst}  you  book-\textsc{def} \\
\glt `Did you read the book?'
\z
\end{multicols}
\ex\label{ex:leino_9b}
{Swedish}\\
\ea
\gll Kalle     har    ätit         soppan.\\
     Charlie  have  eat-\textsc{pcp}  soup-\textsc{def}\\
\glt `Charlie has eaten the soup.'
\ex\label{ex:leino_9c}
\gll Har    Kalle     ätit         soppan?\\
     Have  Charlie  eat-\textsc{pcp}  soup-\textsc{def}\\
\glt `Has Charlie eaten the soup?'
\z
\z


In ordinary terms, forming a yes/no question involves subject and object inversion (or, differently stated, verb-initial word order). This may be stated very simply with the following metaconstruction.

\begin{figure}{}
         [[S V X] $\leftrightarrow $ [V S X]]
        \caption{Metaconstruction between assertive sentence and yes/no question in Swedish}
        \label{fig:leino_fig7}
\end{figure}


\figref{fig:leino_fig7} only shows a very schematic structural association of the associated constructions (i.e. the assertive sentence construction and the yes/no question construction). A more detailed description of this metaconstruction would, of course, include information on the discourse functions of these constructions, on more specific structural properties of the constructions, etc.\footnote{There are no a priori limits to what, and how much, information a metaconstruction may include, just as there are no such limits for the information content of a construction. Constructions, as usage-based generalizations of expressions, may include any amount of observed and generalized linguistic information. Similarly, metaconstructions may, at least in principle, include any amount of information relevant to the constructions that they relate to one another and the relationship between those constructions. In practice, however, the core of a metaconstruction is a relatively simple analogical relation, and the rest of the information included in it is a selection of features of the constructions involved. Even so, of course, metaconstructions tend to be rather complex knowledge structures.}

To take a broader perspective, and to further illustrate the role of metaconstructions in organizing the grammatical system, let us add to this the active vs. passive alternation. In Swedish, the regular way of forming a passive is adding the suffix \textit{{}-s} to the main verb and marking its object argument as the grammatical subject:\largerpage

\ea\label{ex:leino_10}
{Swedish}\\\multicolsep=.125\baselineskip
\begin{multicols}{2}
\ea
\gll Boken       lästes.\\
     book-\textsc{def}  read-\textsc{pst.pass}\\
\glt `The book was read.'
\ex\label{ex:leino_10b}
\gll Lästes              boken?\\
     read-\textsc{pst.pass}  book-\textsc{def}\\
\glt `Was the book read?'
\z\end{multicols}
\ex\label{ex:leino_11}
{Swedish}\\\multicolsep=.125\baselineskip
\begin{multicols}{2}
\ea
\gll Soppan     har    ätits.\\
     soup-\textsc{def}  have  eat-\textsc{pst.pass}\\
\glt `The soup has been eaten.'

\ex\label{ex:leino_11b}
\gll Har    soppan     ätits?\\
     Have  soup-\textsc{def}  eat-\textsc{pst.pass}\\
\glt `Has the soup been eaten?'
\z
\end{multicols}
\z

Thus, the relationship between active and passive in Swedish may be characterized as the following metaconstruction.\footnote{In \figref{fig:leino_fig8}, the notion S\textsubscript{S} stands for a grammatical subject which expresses the subject argument of the verb, S\textsubscript{O} for a grammatical subject which expresses the object argument, etc.}

\begin{figure}{}
         [[S\textsubscript{S} V O\textsubscript{O}] $\leftrightarrow $ [S\textsubscript{O} V-\textit{s}]]
        \caption{Metaconstruction between active and passive sentences in Swedish\label{fig:leino_fig8}}
\end{figure}

We may relate the metaconstructions shown in Figures~\ref{fig:leino_fig7} and~\ref{fig:leino_fig8}, and thereby present a limited subsystem of the Swedish grammar organized by these metaconstructions, in the following manner.

\begin{figure}
    \includegraphics[width=.75\textwidth]{figures/Leino9.pdf}
    \caption{Assertive sentences, yes/no questions and active and passive voice
as a subsystem in Swedish\label{fig:leino_fig9}}
\end{figure}

In \figref{fig:leino_fig9}, there are two instances of both of the metaconstructions shown in Figures~\ref{fig:leino_fig7} and~\ref{fig:leino_fig8}. The one shown in \figref{fig:leino_fig7}, i.e. [[S V X] $\leftrightarrow $ [V S X]], connects the sentence types [S\textsubscript{S} V O\textsubscript{O}] (i.e. active assertive sentence) and [V S\textsubscript{S} O\textsubscript{O}] (active yes/no question) on the one hand, and the sentence types [S\textsubscript{O} V-\textit{s}] (passive assertive sentence) and [V-\textit{s} S\textsubscript{O}] (passive yes/no question) on the other. Correspondingly, the metaconstruction shown in \figref{fig:leino_fig8}, i.e. [[S\textsubscript{S} V O\textsubscript{O}] $\leftrightarrow $ [S\textsubscript{O} V-\textit{s}]], connects the sentence types [S\textsubscript{S} V O\textsubscript{O}] (active assertive sentence) and [S\textsubscript{O} V-\textit{s}] (passive assertive sentence) on the one hand, and [V S\textsubscript{S} O\textsubscript{O}] (active yes/no question) and [V-\textit{s} S\textsubscript{O}] (passive yes/no question) on the other.

This example shows a notably different aspect of metaconstructions than that discussed in \sectref{leino_sec2}. While the example of Finnish subject and object case marking was a case of diachronic change taking place, and metaconstructions serving as a vehicle of such change, the case of the Swedish sentence types is purely synchronical. The synchronic role of metaconstructions may be argued to include such tasks as finding the right construction for a given discourse function, indicating correspondencies between different expression types, and the like. The synchronic and diachronic aspects of metaconstructions will be further discussed in Sections~\ref{leino_sec4} and~\ref{leino_sec5}. It should be noted, however, that metaconstructions do have both aspects to them: they serve both as synchronic devices which organize the grammar of a language and as diachronic analogy models for re-organizing the grammar.

\section{Some related theoretical notions}\label{leino_sec3}

The notion of metaconstruction is quite obviously related to other previously suggested notions. In the following, I shall point out some similarities between metaconstructions and Kay’s \textit{patterns of coining,} on the one hand, and Chomskyan transformations, on the other.

\subsection{Metaconstructions and patterns of coining}\label{sec:leino_3.1}

The concept of metaconstruction, in particular in its use as a basis for novel expressions and expression types, shows great resemblace to Kay’s \textit{patterns of coining} \citep{Kay2013}.\footnote{Kay, in fact, attributes the notion to Charles Fillmore. Apparently, Fillmore has presented the notion in a lecture, the text of which is available online \citep{Fillmore1997}. However, since Fillmore does not elaborate on the notion, whereas Kay does, it seems justified to refer to the notion as \textit{Kay’s} patterns of coining.} Kay’s first example is the word \textit{underwhelm,} which is the result of analogy along the following lines:

\ea\label{ex:leino_12} English \citep[33]{Kay2013}\\
\glt \textit{over : overwhelm :: under : \textbf{underwhelm}}\z

Above, I have not extended the concept of metaconstruction to morphological phenomena, simply because Construction Grammar does not yet have conventionalized ways of representing morphology and morphological phenomena. However, we may rather comprehensibly -- albeit pre-theoretically -- represent \REF{ex:leino_12} in terms of metaconstructions as follows.

\begin{figure}{}
         [[P] $\leftrightarrow $ [P\textit{whelm}]]
        \caption{A metaconstructional account of underwhelm}
        \label{fig:leino_fig10}
\end{figure}

What the metaconstruction in \figref{fig:leino_fig10} states is essentially that there is a relationship between the combination of a preposition and another word which consists of that preposition and the affix (or affix-like element) \textit{{}-whelm.} The pair of words related to one another by this metaconstruction share semantic features in a systematic way. In other words, for the two pairs of words, \textit{over \& overwhelm} and \textit{under \& underwhelm,} the metaconstruction essentially expresses the same information as Kay’s proportional analogy shown in example \REF{ex:leino_12}.

\begin{sloppypar}
The metaconstruction in \figref{fig:leino_fig10} obviously allows for such hypothetical words as \textit{upwhelm, downwhelm, throughwhelm, inwhelm, outwhelm} etc. This could be used as an argument against this formulation, claiming that the metaconstruction in \figref{fig:leino_fig10} \textit{overgenerates} such expression. However, as I shall discuss below, metaconstructions are not intended as generative entities. Rather, they express observed analogies. Thus, the metaconstruction in \figref{fig:leino_fig10} does not state that we should expect such words as \textit{upwhelm} and \textit{inwhelm}. What it does state is that if we were to encounter such words, then \textit{upwhelm} would be to \textit{overwhelm} what \textit{up} is to \textit{over}; i.e. that the relation between \textit{upwhelm} and \textit{overwhelm} is analogous to that between \textit{up} and \textit{over}.
\end{sloppypar}

The same goes, mutatis mutandis, for other patterns of coining discussed by Kay as well. While metaconstructions merely capture similarities between observed expressions and, notably, \textit{types} of expressions, patterns of coining are used (as the name implies) to coin new expressions. In other words, patterns of coining are \textit{productive} to some extent, whereas metaconstructions are not (except for some rare cases such as the one discussed in \sectref{sec:leino_2.1}).

This point leads us to an interesting continuum from metaconstructions via patterns of coining to constructions. According to \citet[38]{Kay2013}, the formula \textit{A as NP} ‘extremely A’ is not a construction but, rather, a pattern of coining. He states two reasons for this (ibid.):

\begin{quote}
First, knowledge of formula \REF{ex:leino_12} [A as NP ‘extremely A’] plus knowledge of the constituent words is not sufficient [to license examples of this formula]. If a young, foreign or sheltered speaker of English knew what \textit{easy} meant, and knew what \textit{pie, duck,} and \textit{soup} meant and knew all the expressions in [Kay’s examples] plus many more built on the same pattern, they would still not know that \textit{easy as pie} and \textit{easy as a duck soup} are ways of saying \textit{very easy}. Secondly, one can’t freely use the pattern to coin new expressions.
\end{quote}

The central point of Kay’s argument -- and, indeed, his whole paper -- is that patterns of coining are less productive than grammatical constructions. As noted above, metaconstructions are less productive than patterns of coining. Thus, we may think of these three as a cline from less to more productive generalizations.

\begin{figure}
    % \includegraphics[width=\textwidth]{figures/Leino-fig11.PNG}
    \begin{tikzpicture}
        \node(oben){metaconstructions \hspace{1cm} patterns of coining~~~~~ \hspace{1cm} constructions};
        \node[below =1mm of oben]{productivity};
        \draw[-latex,thick](oben.south west) -- (oben.south east);
    \end{tikzpicture}
    \caption{Constructions, patterns of coining, and metaconstructions on a productivity scale}    
    \label{fig:leino_fig11}
\end{figure}

While this is by no means the only difference between these three theoretical notions (notably, metaconstructions are generalizations of expression \textit{types,} or more accurately of relations which hold between them, while the other two are generalizations of actual expressions), this aspect of the notions is useful in pinning down the essential nature of each of these notions. And, as we saw in the case of \textit{underwhelm,} metaconstructions can actually be postulated as generalizations of actual words and utterances as well.

\subsection{Metaconstructions and transformations}\label{sec:leino_3.2}

A rather different way to conceive of the notion of metaconstruction has to do with a somewhat different branch of linguistics. As the observant reader may well have noticed, such metaconstructions resemble, to a great degree, the \textit{grammatical transformations} used in the tradition started by \citet{Chomsky1957}. And indeed, such a tool might well be used to revitalize the transformational school of thought. For example, \citegen[43]{Chomsky1957} example:

\begin{quote}
If S1 is a grammatical sentence of the form\\
 \textit{NP1 – Aux – V – NP2,}\\
then the corresponding string of the form\\
 \textit{NP2 – Aux + be + en – V – by + NP1}\\
is also a grammatical sentence
\end{quote}

can be expressed, with a notation which greatly resembles the characterizations of metaconstructions, in the following form:


\begin{quote}{}
[[NP1 – Aux – V – NP2] $\leftrightarrow $ [NP2 – Aux + be + en – V – by + NP1]]
\end{quote}

However, this is by no means what metaconstructions are intended to do. The nature of metaconstructions is deeply different from that of transformations. Although the differences between metaconstructions and transformations may not appear to be as obvious as the similarities, they are all the more noteworthy from a theoretical perspective.

First of all, it is clear that metaconstructions are not nearly as productive as transformations. Metaconstructions are generalizations which a language user may or may not make, and their central function is to keep up analogical relationships among different sets of constructions.

\begin{sloppypar}
Secondly, metaconstructions are not used to turn linguistic material into some other linguistic material, or deep structure into surface structure, the way that transformations are: Metaconstructions are not generative in nature. Where transformations may be said to describe alternations, metaconstructions describe correspondencies. In the construction grammar world view, no material \textit{changes} into other material; rather, everything is described in terms of correspondencies and compatibility.\footnote{Cf. \citet{Kay1995} and the \textit{non-derivational} and \textit{usage-based} properties of Construction Grammar. While metaconstructions may sometimes be used to coin new expressions or expression types, as shown in \sectref{sec:leino_2.1}, they are not part of the standard mechanism of generating sentences in the same manner as transformations are (or were) in transformational grammar.} In accordance with this tradition, metaconstructions do not involve transforming an expression into some more suitable expression.
\end{sloppypar}

And thirdly, metaconstructions have, as we saw above, a more or less diachronic nature. They are generalizations over \textit{types} of expressions, not over actual expressions. If they do have a generative nature, that nature must be diachronic in the sense that metaconstructions are used to create new types of expressions, new constructions, rather than just new expressions.

The synchronic vs. diachronic nature of new constructions being conventionalized, and the role of metaconstructions in that, naturally is a broad topic and falls outside of the scope of this paper. As a general rule, however, I see the conventionalization of new constructions as a diachronic process (cf. e.g. \citealt{Rostila2006}), and certainly more so than the process of generating utterances based on already existing constructions.

\section{Metaconstructions and paradigms}\label{leino_sec4}

The examples in \sectref{leino_sec2} show that metaconstructions capture both static and dynamic relations between groups of constructions. They both show systematic groups of constructions within the grammar and may even serve as patterns for coining new constructions. Crucially with regard to the theme of the present volume, they may also be used to capture paradigmatic relations, unlike any other device yet postulated for Construction Grammar that I am aware of. In what follows, I present an analysis of the person inflection paradigm in Finnish as exemplified in different contexts, or groups of constructions.

\subsection{Finnish personal pronouns and the verb inflection paradigm}\label{sec:leino_4.1}

The Finnish person paradigm rather straightforwardly consists of the set of 1\textsuperscript{st}, 2\textsuperscript{nd}, and 3\textsuperscript{rd} person in singular vs. plural. This paradigm is found both in verb inflection and in personal pronouns (\tabref{tab:leino:1}).

\begin{table}
    \caption{Finnish person paradigm\label{tab:leino:1}}
    \begin{tabular}{ *{5}{l} }
        \lsptoprule
            & \multicolumn{2}{c}{singular} & \multicolumn{2}{c}{plural}\\\cmidrule(lr){2-3}\cmidrule(lr){4-5}
            {1.} &  minä & tee-n  &  me &  tee-mme\\
                 & I & do-\textsc{1sg}  &  we & do-\textsc{1pl}\\
            {2.} &  sinä & tee-t &  te & tee-tte\\     
                 & you.\textsc{sg} & do-\textsc{2sg}  & you.\textsc{pl} & do-\textsc{2pl}\\
            {3.} & hän & teke-e  &  he & teke-vät\\
                 & (s)he & do-\textsc{3sg} & they & do-\textsc{3pl}\\\midrule
            \multicolumn{5}{l}{Passive \hfill teh-dä-än \hfill}  \\
            \multicolumn{5}{l}{\hphantom{Passive} \hfill  do-\textsc{pass}{}-4 \hfill}\\
        \lspbottomrule
    \end{tabular}
\end{table}

A local peculiarity of Finnish, so to speak, is the form conventionally known as the passive, perhaps more accurately an impersonal form. It is sometimes referred to as “the 4\textsuperscript{th} person” (originally by \citealt{Tuomikoski1971}) to reflect the fact that it is effectively a part of the person inflection paradigm, despite the fact that it is used, to some extent, in constructions distinct from the ones used for the active forms. It does, however, resemble the active forms in usage significantly more than e.g. most Indo-European passives. For instance, it can be formed of practically all Finnish verbs, including intransitive verbs and even the copula.

While the passive morphologically has, in addition to the voice marker, a person suffix of its own, there is no separate personal pronoun for the passive. This is naturally in accordance with its primary use for agent demotion or impersonalization. Therefore, it might be argued that the passive is not a full-fledged “4\textsuperscript{th} person” but rather a set of constructions which are used in contexts where the person inflection paradigm is not relevant. As will become apparent in the following sections, personal pronouns are not the only context to speak in favor of such a view. On the other hand, as will also become apparent, the finite verb inflection paradigm is by no means the only context to speak in favor of the opposing view, i.e. of interpreting the passive as a part of the person inflection paradigm.

The precise nature of the Finnish passive need not concern us here. The interested reader is referred to sources like \citet{Shore1988}, \citet{ManninenNelson2004} and \citet{Helasvuo2006} for further details. For the purposes of this paper, the crucial thing is that the passive is a conventional part of the person paradigm but not present in all of its manifestations.

\subsection{Possessive suffixes}\label{sec:leino_4.2}

Possessive relations in Finnish are typically marked, redundantly, with two devices simultaneously: the genitive form of a personal pronoun on the one hand, and a possessive suffix on the other.

\begin{table}
    \caption{Finnish possessive paradigm\label{tab:leino:2}}
    \begin{tabular}{l ll ll}
        \lsptoprule
            & \multicolumn{2}{c}{Singular} & \multicolumn{2}{c}{Plural}\\\cmidrule(lr){2-3}\cmidrule(lr){4-5}
            {1.} &  minu-\textit{n} & puhee-\textit{ni}  &  meidä-\textit{n} &  puhee-\textit{mme}\\
                 & I-\textsc{gen} & speech-\textsc{nom.1sg}  &   we-\textsc{gen} & speech-\textsc{nom.1pl}\\
            {2.} &  sinu-\textit{n} & puhee-\textit{si}  &  teidä-\textit{n} & puhee-\textit{nne}\\     
                 & you.\textsc{sg.gen} & speech-\textsc{nom}  & you.pl & speech-\textsc{nom.2pl}\\
            {3.} & häne-\textit{n} & puhee-\textit{nsa}  &  heidä-\textit{n} & puhee-\textit{nsa}\\
                 & (s)he-\textsc{gen} &  speech-\textsc{nom.3sg/pl} & they & speech-\textsc{nom.3sg/pl}\\\midrule
            \multicolumn{5}{l}{Passive \hfill puhe\hfill}\\
            \multicolumn{5}{l}{\hphantom{Passive} \hfill speech-\textsc{nom} \hfill}\\
        \lspbottomrule
    \end{tabular}
\end{table}

        

As shown above, the passive is left out of the paradigm again. Not only has it no personal pronoun of its own, but it also has no corresponding possessive suffix. What exactly a “passive ownership” would mean is left an open question here. For the sake of the argument, we may assume that a bare noun with no possessive marker is the closest match to “impersonal ownership”. As will become apparent, however, there are other contexts in which the possessive suffixes are used where a passive possessive suffix would be useful to complete the paradigm. As is also shown above, the possessive suffix paradigm does not distinguish between 3\textsuperscript{rd} person singular and plural. While this is an interesting observation per se, it need not concern us here.\largerpage

It seems perfectly natural that the same distinctions among  1\textsuperscript{st}, 2\textsuperscript{nd}, and 3\textsuperscript{rd} person and singular vs. plural are found both in verb inflection and in nominal possessive marking. It seems equally natural that the same personal pronouns occur with verbs inflected for person and nouns marked for possession. And yet, there is no a priori reason why this should be the case, and, more importantly for present purposes, there is nothing in the architecture of Construction Grammar that could capture, let alone explain, the fact that it is the case.

While there has, so far, been no notable attempt at capturing Finnish morphology in a construction-based formalism, it is obviously possible to represent the morphological structures as constructions, e.g. with the formalism proposed in \citet{Booij2010}. Such a formalization is beyond the scope of this paper, but for the present case the crucial point is that the paradigms shown above can be represented as organized groups of constructions. Furthermore, it is important to observe that the organization of the group of constructions discussed in this section (possessive suffixes) is essentially identical to the organization of the groups of constructions discussed in the previous section (personal pronouns and verb inflection).

Given the discussion in \sectref{leino_sec2} of this paper, it seems justified to claim that it is possible to capture the systematic, and, in fact, analogous organization of these groups of constructions in terms of a (somewhat complex) metaconstruction. And, as will become apparent in following sections, the same metaconstruction is also necessary for capturing and explaining other phenomena in the Finnish grammar.

\subsection{The Finnish infinitive system}\label{sec:leino_4.3}\largerpage

In Finnish, as in languages more generally, person inflection is essentially a property of finite verb forms. However, as will be shown in the following sections, Finnish infinitives also show features at least reminiscent of person inflection in some contexts. In order to properly understand that phenomenon, a brief introduction of the complex of Finnish infinitive forms is in place.

According to the traditional view, predominant since the 19\textsuperscript{th} Century, Finnish is said, on morphological grounds, to have either four or five distinct infinitives, each of which has a different morphological marker. Each of the infinitives shows some case inflection, but none of them has a full case inflection paradigm. The forms are traditionally referred to with numbers, but since \citet{HakulinenEtAl2004}, they are more commonly referred to by their morphological marker. Hakulinen et al. only treat the first three as true infinitives for reasons that need not concern us here. The forms are briefly introduced in the following.

\ea
  \ea 1\textsuperscript{st} infinitive (\citealt{HakulinenEtAl2004}: A infinitive):
   \ea  morphological marker -\textit{TA}\footnote{The vowel quality in Finnish affixes is dependent on vowel harmony. In front vowel contexts, the archephoneme /A/ is realized as the frontal vowel \textit{ä} [a], and in back vowel contexts as the back vowel [ɑ]. Similarly, the archephoneme /O/ is realized either as the front vowel ö [ø] or the back vowel o [o] depending on the phonemic context. To further confuse the uninitiated reader, the verb stem varies according to consonant gradation (which also differentiates between the realization of the /T/ of the infinitive marker as either [t] or [d]) and other sound changes triggered by the following affix. Fortunately, for the purposes of this paper, these peculiarities are beside the point.} (i.e. \textit{{}-a, -tä, -da, -dä,} and assimilated variants \textit{{}-lA, -rA}, etc.)
   \ex “short form” (nominative/accusative): \textit{teh-dä} (do-\textsc{inf1)}
   \ex translative: \textit{teh-dä-kse-en} (do-\textsc{inf1.tra.3sg/pl)}
   \z


  \ex 2\textsuperscript{nd} infinitive (E infinitive):
    \ea  morphological marker \textit{{}-den, -ten}
    \ex inessive: teh-de-ssä (do-\textsc{inf2.ine)}
    \ex instructive: teh-de-n (do-\textsc{inf2.ins)}
    \z
    
  \ex 3\textsuperscript{rd} infinitive (MA infinitive):
    \ea morphological marker -\textit{mA} (-ma, -mä)
    \ex inessive: \textit{teke-mä-ssä} (do-\textsc{inf3.ine)}
    \ex elative: \textit{teke-mä-stä} (do-\textsc{inf3.ela)}
    \ex illative: \textit{teke-mä-än} (do-\textsc{inf3.ill)}
    \ex adessive: \textit{teke-mä-llä} (do-\textsc{inf3.ade)}
    \ex instructive: \textit{teke-mä-n} (do-\textsc{inf3.ins)}
    \ex abessive: \textit{teke-mä-ttä} (do-\textsc{inf3.abe)}
    \z

  \ex 4\textsuperscript{th} infinitive:
    \ea morphological marker \textit{{}-minen, -mis-}
    \ex nominative: \textit{teke-minen} (do-\textsc{inf4)}
    \ex partitive: \textit{teke-mis-tä} (do-\textsc{inf4.par)}
    \z

  \ex 5\textsuperscript{th} infinitive:
    \ea morphological marker –-maisi-, -mäisi-
    \ex adessive: \textit{tekemäisillään} (do-\textsc{inf5.ade.3sg/pl)}
    \z
  \z
\z

In other words, according to the traditional view, Finnish has several separate infinitives, each of which has a defective case inflection paradigm. A radically different view, originally presented already by \citet[44]{Lönnrot1841} and rediscovered by \citet{Siro1964}, treats the different forms as variants of the same infinitive, with the infinitive marker varying in different case forms. As Siro points out, the different infinitive markers are in fact nearly in a complementary distribution with regard to case inflection.

\begin{table}
    \caption{Finnish infinitive forms sorted by morphological case\label{tab:leino:3}}
    \begin{tabular}{lllll}
        \lsptoprule
            Nominative\footnote{(or basic form)} & \textit{sano} & \textit{-a} & & 1.\\
                                     & \textit{sano} & \textit{-minen} & & 4.\\
            Partitive & \textit{sano} & \textit{-mis} & \textit{-ta} & 4.\\
            Translative & \textit{sano} & \textit{-a} & \textit{-kseni} & 1.\\
            Inessive & \textit{sano} & \textit{-e} & \textit{-ssa} & 2.\\
                     & \textit{sano} & \textit{-ma} & \textit{-ssa} & 3.\\
            Elative & \textit{sano} & \textit{-ma} & \textit{-sta} & 3.\\
            Illative & \textit{sano} & \textit{-ma} & \textit{-an} & 3.\\
            Adessive & \textit{sano} & \textit{-ma} & \textit{-lla} & 3.\\
                     & \textit{sano} & \textit{-maisi} & \textit{-llani} & 5.\\
            Abessive & \textit{sano} & \textit{-ma} & \textit{-tta} & 3.\\
            Instructive & \textit{sano} & \textit{-e} & \textit{-n} & 2.\\
                    & \textit{sano} & \textit{-ma} & \textit{-n} & 3.\\
        \lspbottomrule 
    \end{tabular}
\end{table}

To complete the complementary distribution, each of the forms is only used in a limited set of non-finite expression types, or constructions, and those few instances where there are two different infinitive forms corresponding to the same case form (nominative, inessive, adessive, and instructive), the two forms are never mutually interchangeable in any of the constructions in which they are used.

\subsection{Finnish infinitives with possessive suffixes}\label{sec:leino_4.4}

Infinitives are often described as verb forms which resemble nouns. This characterization typically refers to their syntactic behavior, but in Finnish it is true also of their morphology. As was shown in \sectref{sec:leino_4.3}, Finnish infinitives are inflected for case like nouns. In addition, some of them also take the possessive suffix in some contexts -- the translative form of the 1\textsuperscript{st} infinitive and the adessive form of the 5\textsuperscript{th} infinitive in fact never occur without a possessive suffix.

Not all Finnish infinitive forms occur with possessive suffixes however. Of the thirteen different infinitive forms (combinations of an infinitive marker and a case ending) listed in \tabref{tab:leino:3}, only five occur with possessive suffixes.

\begin{table}
    \caption{Finnish infinitive forms with possessive suffixes}
    \label{tab:leino:4}
    \begin{tabularx}{\textwidth}{llQ}
        \lsptoprule
          Infinitive & Case & Form \\
\midrule
          1. & nominative/accusative &  \\
             & translative & \textit{lähte-ä-\textbf{ni}} (leave-\textsc{inf1.sg}) ‘me to leave’ (not in contemporary standard language)\\
          2. & inessive & \textit{teh-de-ssä-\textit{mme}} (do-\textsc{inf2.ine.1pl}) ‘while we do’\\
             & instructive & \textit{kuul-te-\textbf{nsa}} (hear-\textsc{inf2.ins.3sg/pl}) ‘with him/her overhearing’\\
          3. & instructive & \textit{piti teke-mä-\textbf{ni}} (must-\textsc{pst.3sg-} do-\textsc{inf3.ins.1sg}) ‘I had to do’ (only used with the necessive modal verb \textit{pitää}, not in contemporary standard language)\\
          4. & adessive & \textit{olin tekemäisillä\textbf{ni}} (be-\textsc{pst.1sg} do-\textsc{inf5.ade.1sg}) ‘I was just about to do’ (always with a possessive suffix)\\
        \lspbottomrule
    \end{tabularx}
\end{table}

In all of the occurrences of these forms, the possessive suffix corresponds to the actor of the event denoted by the infinitive. In other words, for all intents and purposes, the possessive suffix in these forms semantically corresponds to person inflection. Yet, the morphemes used for that purpose in the infinitives are not those used as person affixes in finite verb forms but rather those used for marking possession with nouns as shown in \sectref{sec:leino_4.2}.

\subsection{Passive infinitive forms}\label{sec:leino_4.5}

As was pointed out in \sectref{sec:leino_4.1}, the Finnish passive is effectively a part of the person inflection paradigm. Given that, and the observation made in \sectref{sec:leino_4.4} that possessive suffixes in infinitives resemble person inflection, the question arises whether there are passive forms to complete the person paradigm in those infinitive forms which do take the possessive suffix. And, indeed, there are passive infinitive forms in Finnish, even though they are few in terms of both types and tokens. In fact, there are only three passive infinitive forms in Finnish.

\begin{table}
    \caption{Finnish passive infinitive forms}
    \label{tab:leino:5}
    \begin{tabularx}{\textwidth}{llQ}
        \lsptoprule
          Infinitive & Case & Form \\
\midrule
          1. & nominative/accusative & \textit{teh-tä-ä} (do-\textsc{pass.inf1}) ‘to be done’ (not in contemporary standard language)\\
          2. & inessive & \textit{teh-tä-e-ssä} (do-\textsc{pass.inf2.ine}) ‘while being done’\\
          3. & instructive & \textit{piti teh-tä-mä-n} (must-\textsc{pst.3sg} do-\textsc{pass.inf3.ins}) ‘had to be done’ (only used with the necessive modal verb pitää, not in contemporary standard language)\\
        \lspbottomrule
    \end{tabularx}
\end{table}

As has been pointed out by \citet{Leino2005b}, the following observation holds in the Finnish infinitive system: if a given infinitive (i.e. a combination of one of the four or five infinitive markers and a specific case suffix) has a passive variant, then it also has a variant with a possessive suffix. In other words, there are no passive infinitive forms with no corresponding possessive suffixed forms. This seems to suggest that the few existing passive infinitive forms are motivated by the forms with the possessive suffixes, and have arisen in order to complete what looks like their person inflection paradigm.

\subsection{Emerging non-finite person inflection?}\label{sec:leino_4.6}

As an interim summary, Finnish shows the same paradigm of 1\textsuperscript{st}, 2\textsuperscript{nd}, and 3\textsuperscript{rd} person in singular vs. plural in a number of different contexts, including personal pronouns, finite verb inflection, possession marking on nouns, and the usage of possessive suffixes on infinitives, closely resembling person inflection. Furthermore, the paradigm is supplemented in several, but not all, instances by the passive. Thus, what we observe is a paradigm consisting of seven parts, organized in terms of voice, person, and number.

It seems rather obvious that some of the instances of this pattern, or paradigm, are more fundamental than others. Notably, the apparent person inflection of infinitives seems secondary in comparison to both its clear model, finite verb inflection, and the much more common use of the possessive suffixes as markers of possession. This impression is backed up by the fact that other Baltic Finnic languages make even less, if any, use of passive and possessive suffixed infinitives than Finnish does, suggesting that they are the result of a relatively late development.

\begin{sloppypar}
While the apparently emerging non-finite person inflection is a relatively young development, and one that has not been fully developed (at least yet), it is strongly motivated by related phenomena in the Finnish language. Notably, of course, person marking on finite verbs serves as the single most important semantic model. Possessive marking, on the other hand, provides suitable morphological means to realize the marking, in the form of affixes that naturally attach to noun-like forms such as infinitives.\footnote{A further motivation is the genitive subject which occurs in many non-finite constructions in Finnish and which motivates the use of the possessive suffix. For the sake of simplicity, I will omit this part of the complex here. For details, see \citet{Leino2015}.} 
\end{sloppypar}

To illustrate the similarieties, consider the following:

\ea\label{ex:leino_13}
{Finnish}\\
\ea
\gll minä    tee-n\\
     I-\textsc{nom}  do-\textsc{1sg} \\ 
\glt `I do' \label{ex:leino:13a}
\ex \label{ex:leino:13b}
\gll minu-n  teh-de-ssä-ni\\
     I-\textsc{gen}    do-\textsc{inf2.ine.1sg}\\
\glt `while I do'

\ex \label{ex:leino:13c}
\gll minu-n  auto-ssa-ni\\
     I-\textsc{gen}    car-\textsc{ine.1sg}\\
\glt `in my car'
\z
\z

The resemblance of the possessive suffixed infinitival expression in \REF{ex:leino:13b} to the finite verb expression in \REF{ex:leino:13a}, on the one hand, and the NP with possessive marking and the same case inflection in \REF{ex:leino:13c}, on the other, is clear.

More importantly with regard to the topic of the present paper, however, all of the expressions in \REF{ex:leino:13a} may be altered to represent the 1\textsuperscript{st}, 2\textsuperscript{nd}, and 3\textsuperscript{rd} person in singular vs. plural, and both of the verbal expression, i.e. \REF{ex:leino:13a} and \REF{ex:leino:13b}, may also take the passive morphology. Even \REF{ex:leino:13c} may be claimed to have a counterpart for the passive in the form of a simple noun with no possessive marking. Thus, the correspondence of these expression types is systematic with regard to the person paradigm.

The following figure shows the set of phenomena described above, with their interconnections, and the person paradigm which is present in all of them.

\begin{figure}
   % \includegraphics[width=\textwidth]{figures/Leino-fig12.png}
    \small
\begin{tikzpicture}
  \node(pronouns)[rectangle,black,draw,text width=3cm]{\textit{Personal pronouns}\\
  \begin{tabular}{l@{~:~}l}
    minä & me\\
    sinä & te\\
    hän & he\\
    \multicolumn{2}{c}{}\\
  \end{tabular}
  };
  \node(possessives)[rectangle,black,draw,text width=3cm,below=of pronouns]{\textit{Possesive suffixes}\\
    \begin{tabular}{l@{~:~}l}
    autoni & automme\\
    autoni & autone\\
    autoni & autonsa\\
    \multicolumn{2}{c}{auto}
  \end{tabular}
  };
  \node(finite)[rectangle,black,draw,text width=4cm,right=1cm of pronouns]{\textit{Finite inflection}\\
    \begin{tabular}{l@{~:~}l}
    teen & teemme\\
    teet & teette\\
    tekee & tekevät\\
    \multicolumn{2}{c}{tehdään}
  \end{tabular}
  };
  \node(infinitivessuffixes)[rectangle,black,draw,text width=4cm,right=1cm of possessives]{\textit{Infinitives with possessive suffixes}\\
    \begin{tabular}{l@{~:~}l}
    tehdessäni & tehdessämme\\
    tehdessäsi & tehdessänne\\
    tehdessän & tehdessän\\
    \multicolumn{2}{c}{tehtäessä}
  \end{tabular}
  };

  \node(paradigm)[rectangle,black,draw,text width=3cm,right=5mm of finite,yshift=-1.75cm]{\textit{Person paradigm}\\
    \begin{tabular}{ll}
    singular & plural \\
    1\textsuperscript{st} & 1\textsuperscript{st}\\
    2\textsuperscript{nd} & 2\textsuperscript{nd}\\
    3\textsuperscript{rd} & 3\textsuperscript{rd}\\
    \multicolumn{2}{c}{passive}
  \end{tabular}
  };

  \draw[dashed,overlay] (paradigm.176) -- ++ (-4.95cm,0);
  \draw[dashed] (pronouns) -- (finite);
  \draw[dashed] (pronouns) -- (possessives);
  \draw[dashed] (pronouns) -- (infinitivessuffixes);

  \draw[dashed] (possessives) -- (finite);
  \draw[dashed] (possessives) -- (infinitivessuffixes);

  \draw[dashed] (finite) -- (infinitivessuffixes);

\end{tikzpicture}


    \caption{Motivational structure of the possibly emerging non-finite person inflection}
    \label{fig:leino_fig12}
\end{figure}

Unfortunately, a proper formalization of the various (sets of) constructions and interconnections shown in \figref{fig:leino_fig12} is beyond the scope of this paper -- not least because it would not be possible without a lot of basic groundwork for describing Finnish morphology in a constructionist framework. Importantly, however, it is possible to represent all the relevant expression types as constructions and to formalize their analogical interconnections in the form of metaconstructions, in the spirit of the analyses and formalizations in \sectref{leino_sec2} of this paper.

Understanding the complex \figref{fig:leino_fig12} as an analogically interconnected group of constructions is revelatory to the nature of paradigms. It serves as an example of what was already observed in \sectref{leino_sec2}: that language promotes systematicity and recurring patterns by extending observed analogical relations to other contexts, in other words by “copying” ways to organize grammatical information from one part of the grammar to others. This naturally leads to paradigms which hold to more than one single set of constructions.

\section{Implications and conclusions} \label{leino_sec5}

Based on the phenomena discussed above, and the observations made, we may conclude that analogy plays a central role in the internal organization of grammar. In order to capture analogical relations within grammar, a concept like metaconstruction, or at least something similar capable of capturing those relations, is necessary. Acknowledging, and also formalizing, systematic analogical relations between sets of constructions lets us see paradigms as emergent categories of grammar, based on generalizations over repeatedly observed analogies. Paradigms often involve very complex sets of relations between constructions, but they may nonetheless be described in terms of systematic interconnectedness of rather simple constructions. This becomes particularly clear in a case like the one described in \sectref{leino_sec4} where a new instantiation of a paradigm is emerging.

I hope to have shown that metaconstructions have a role in both the synchronic organization and the diachronic reorganization of grammatical constructions. They participate both in statically structuring the inventory of constructions in a given language and in dynamically restructuring that inventory. This restructuring may occur by reanalysis of existing constructions and relationships between them or by coining new constructions through analogy based on existing ones.

\begin{sloppypar}
Metaconstructions are \textit{generalizations} of constructions, and their central function is to keep up analogical relationships among different sets of constructions. Thus, metaconstructions have to do with networking certain structures and meanings into a coherent system, and with choosing a construction that fits into a given communication setting. This has to do with the topic of synchronic organization of constructions.
\end{sloppypar}

Metaconstructions also have, as we have seen, a diachronic nature. They are generalizations of \textit{types} of expressions, not of actual expressions. If they are to be seen as \textit{generative} parts of the grammar, then they must be generative only in the sense that they are not used to create new expressions but, rather, new \textit{types} of expressions, new \textit{constructions}. Consequently, they also have a major role in the diachronic re-organization of constructions.

Furthermore, metaconstructions are a formalization of a phenomenon which is of great importance with regard to language acquisition. Metaconstructions may, in a number of cases, be thought of as \textit{proto}constructions: they are observed analogies, and when they reach a sufficient level of generality, the language learner may use them to abstract a new construction. In other words, we may assume that a child observes an analogical relation between two linguistic expressions and the situations that they represent, and abstracts a construction based on these expressions which is associated to an abstraction of these situations.\footnote{For a more detailed account of language acquisition based on this type of reasoning, see \citet{Kauppinen1998, Kauppinen1999}.} Similarly, a language learner may observe a similarity of an analogical nature among a group constructions, and abstract a more general construction.

\citet[75]{Goldberg1995} states that inheritance links are “objects in our system”, they are an essential part of the language and the grammar, to the extent that a grammar consists not only of constructions but also of different kinds of links which express different kinds of relations between those constructions. As I have argued in this paper, grammar also includes metaconstructions, which further structure and organize the inventory of constructions and which also express relations between constructions -- albeit more complicated ones than those expressed with inheritance links -- and also provide a dynamic, re-organizational aspect to the grammatical system.\largerpage

\section*{Abbreviations}
\begin{multicols}{4}
\begin{tabbing}
\textsc{ill} \hspace{1ex} \= illative\kill
\textsc{abe} \> abessive\\
\textsc{acc} \> accusative\\
\textsc{ade} \> adessive\\
\textsc{ela} \> elative\\
\textsc{gen} \> genitive\\
\textsc{ill} \> illative\\
\textsc{ine} \> inessive\\
\textsc{ins} \> instructive\\
\textsc{nom} \> nominative\\
\textsc{par} \> partitive\\
\textsc{tra} \> translative
\end{tabbing}
\end{multicols}


{\sloppy\printbibliography[heading=subbibliography,notkeyword=this]}
\end{document} 
