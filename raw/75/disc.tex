\chapter{Toward a cognitive model of stylistic variation in identity construction}\label{ch:disc}

\date{}

%\maketitle

\noindent In this chapter I discuss the results presented in this book within the context of two probabilistic linguistic models of language use: one that uses Bayesian statistics to calculate the probability of the different interpretations of an ambiguous sentence \citep{jurafsky1996,narayananjurafsky2002} and one where utterances are stored as separate exemplars complete with phonetic detail \citep{johnson1997,pierrehumbert2001}.  I then present a usage-based model of speech production and perception that has multidimensional representations of stylistic features abstracted over detailed episodic memories.  First, I briefly summarise the results presented in this book.

%the social aspects of language with the purely linguistic.\footnote{I have used the phrase ``purely linguistic'' to refer to linguistic variation that is not socially conditioned.  However, I view all aspects of language as ultimately requiring a social base.}  %

\section{Summary of results}
 \subsection{Maintaining and rejecting norms}

At SGH, there were a number of norms that were established and maintained by the girls.  Based on whether a group ate lunch in the common room (CR) or not (NCR), I have used the terms CR and NCR to differentiate between the girls who created and conformed to the school's norms, thereby perpetuating the norms themselves (CR), and the girls who rejected the norms and did not conform to them (NCR).  The CR and NCR groups form constellations of stance: the CR girls viewed themselves as ``normal'' whereas the NCR girls viewed themselves as ``weird'' or ``different''.  These stances were reflected in the girls' styles: there were commonalities in the linguistic and non-linguistic stylistic components observed among the girls in different CR groups, and while NCR girls varied across groups in terms of the stylistic features they adopted, they shared a common trend in that their identities were constructed in opposition to the styles of the CR girls.

 \subsection{Patterns in production}
 
%Quotative \textit{like} was the most frequent quotative used by all of the girls, though CR girls were slightly more likely to use it than NCR girls.  Some individual NCR girls

As presented in Chapter \ref{ch:prod}, there was phonetic variation across the different functions of the word \textit{like} and some of this variation depended on whether the speaker ate lunch in the CR or not.  Tokens of quotative \textit{like} were more likely to be monophthongal, have a higher mean pitch, and have a shorter /l/ to vowel duration ratio than tokens of discourse particle \textit{like} and traditionally grammatical functions.  Tokens of grammatical functions were more likely to have a lower F2 value in the nucleus than the discourse particle.  There were also two interactions involving the realisation of /k/: 
\begin{enumerate}
	\item[(1)] When comparing the two discursive functions, there was an interaction between /k/ realisation and where the speaker ate lunch: CR girls were more likely to realise the /k/ in the discourse particle than in the quotative, whereas NCR girls were less likely to realise the /k/ in the discourse particle than in the quotative.  

	\item[(2)] In the model comparing the discourse particle with the traditionally grammatical functions, there was an interaction between the /l/ to vowel duration ratio and the speaker's social group: CR girls were more likely to produce the discourse particle with a long /l/ whereas NCR girls were more likely to produce the grammatical functions of \textit{like} with a long /l/.

\end{enumerate}
  

\noindent \sectref{sec:idconstruction} discussed how the individual girls' use of phonetic variants in the word \textit{like} was related to the degree to which they accepted or rejected norms.  Because a girl's eating place reflected her stance as ``normal'' or ``different'', this finding provides evidence that linguistic variables are correlated with a speaker's stance and that speakers actively adopt and reject linguistic variants as part of the construction of their identity.


\subsection{Patterns in perception}
As discussed in Chapter \ref{ch:perc}, the girls were sensitive during perception to some of these lemma-based phonetic differences from production.  For questions compa\-ring the quotative with the discourse particle, participants were more likely to identify a token as the quotative if it had the shorter /l/ to vowel duration ratio, a tendency that was consistent with trends observed in production.  For questions in Experiment 2 that compared a traditionally grammatical function with either of the discursive functions, participants were more likely to identify tokens as the grammatical function if they had a lower F2 target in the nucleus.  This trend was also consistent with production.  Although not all trends from production were observed in perception, all phonetic-based trends manifested in the perception data reflected the trends in the production data.  A summary of these results is shown in \tabref{tab:sumpercresults}.   




\begin{table}[ht]
\begin{center}
\begin{tabular}{lrrr}
  \lsptoprule
 factor & gram/quote & quote/dp & gram/dp  \\
 \midrule
  nucF2        &   X  & &  X  \\
 EucD   			&     &   	&   \\
 pitch        &     &     &  \\
  duration ratio &    &  X  & \\
 glott        &      &    &  \\
 k present  &    &  &  \\

  \lspbottomrule

\end{tabular}
\caption{Summary of perception results from Experiments 1 and 2}
\label{tab:sumpercresults}
\end{center}
\end{table}

In addition to the perception results outlined above, there was an effect of syntactic information on lemma identification.  In Experiment 1 there was contextual information preceding the token of \textit{like} and it was not matched at the lexical level for all questions (e.g., \textit{He was like} vs. \textit{He is like}).  Participants were more likely to identify the token as the quotative when preceded by the contexts that \citet{buchstallerdarcy2009} found to be the most frequent for the quotative in New Zealand English.  This provides evidence that individuals can use probabilisitic contextual information during speech processing.

In Experiment 3, participants were more likely to identify a voice as belonging to someone who ate lunch in the CR if they indicated that they recognised the voice.  Due to CR girls' high visibility at the school, girls from all groups had extensive exposure to the speech of CR girls, and when a perceiver indicated recognition of a voice (even if they incorrectly identified that voice), they were more likely to indicate that the voice belonged to a CR girl.   Perceivers were also more likely to identify the speaker as a CR girl if the stimulus contained a token of quotative \textit{like} that had a monophthongal vowel.  Because CR girls were more likely to produce monophthongal vowels in all of the functions of \textit{like} analysed, this provides evidence that perceivers were sensitive to sociophonetic trends from production when identifying the eating place of the speaker during perception.

In the following sections, I discuss the theoretical implications of the quantitative results from production and perception, and I present an experience-based model in which both linguistic and non-linguistic stylistic components are indexed to a speaker's style.


\section{Social theory}

\subsection{Phonetic information and identity construction}

In constructing their personae, individuals sometimes make conscious decisions about what symbols to adopt based on the meanings indexed to the symbols.  For example, \citet{eckert2008} describes how particular girls at Palo Alto High School adopted components of other groups' styles that indexed only those characteristics with which they identified \citep[457]{eckert2008}.  These components were then recombined as part of a process of \textit{bricolage}, a term coined by \citet{levistrauss} that refers to the disassembly of an existing whole into parts that can be recombined in the creation of a new whole. However, the meanings indexed to the stylistic components can be different for different individuals.  For example, Supr\'{e} is a chain of stores in Australia and New Zealand that sells clothing, much of which is revealing and inexpensive.  They give free canvas bags of different colours to people who purchase clothes from their shop.  At SGH, the use of Supr\'{e} bags carried a particular social meaning that was usually described as ``skanky''.  While some girls felt that all Supr\'{e} bags indexed ``skanky'', others felt that only the hot pink ones carried this meaning and that bags in other colours, like black, were an indication that the user liked a bargain.  This example helps to portray how individuals segment styles into meaningful elements but the meanings are not necessarily the same for all individuals in a community.  

As with clothing, linguistic variables can be manipulated depending on their indexation to socially-constructed meanings.   \citet{zwicky1997} explains how spea\-kers can adopt variants associated with individuals with whom they identify (identification).  Alternatively, they can avoid using variants that are associated with individuals who they do not want to be similar to or do not believe themselves to be similar to (avoidance).  A speaker can also identify with (or avoid) a particular style shared across numerous individuals, as opposed to associated with a single individual.

As discussed in \S \ref{sec:statusofdisc}, both quotative and discourse particle \textit{like} are highly frequent words that are themselves imbued with social meaning.  I argue that this makes them likely loci of socially-meaningful phonetic variation.  At SGH, girls in the different constellations of stance adopted and rejected linguistic variables to construct their social personae.  I argue that CR girls conformed to each other in terms of their realisation of /k/ in quotative and discourse particle \textit{like}, whereas NCR girls did not conform to one another; the similar trends in terms of /k/ realisation resulted from a common divergence from the speech of the CR girls.  This similarity among girls in NCR groups may be due to chance or, as discussed in \S \ref{sec:proddisc}, it may be due to the exploitation of trends already present in the distribution of /k/ realisation that arose as a function of how likely a speaker was to use quotative \textit{like}.  

In the following section, I discuss how stylistic variation as part of identity construction can be incorporated into a hybrid model that uses both episodic and abstracted representations:  it is possible that both acoustically-rich exemplars and abstract representations of multi-dimensional social information are stored and accessed during the production and perception of speech.

\section{Probabilistic linguistics}

The results provide evidence that mental representations of phonetic information are acoustically-detailed and that their distributions are stochastic: patterns involving gradient phonetic information such as duration and diphthongisation can be observed in both production and perception.  There are several ways that this probabilistic and acoustically-detailed information could be represented.  It is possible that the probabilities are abstracted from the signal such that exposure to new utterances updates the previously stored probabilities \citep{norrismcqueen2008}.  Another possibility is that the utterances themselves are stored and frequency distributions arise as a function of this storage \citep{pierrehumbert2001}.  There is also the possibility that stored representations are made up of some combination of episodic memories, abstracted categories, and distributional probabilities.  For example, exemplars of utterances could be stored complete with acoustic detail and used while accessing phonetic information, and probabilities and categories could be abstracted and stored (rather than computed online) for processing of higher-level (e.g., syntactic) information.  Different levels of the grammar may rely on different levels of representations but, as evidenced by the link between phonetic, contextual, social and lemma-based information observed in the SGH data, these stored representations must be indexed to one another.  

%The models presented here appeared to be promising candidates for accounting for the SGH data.  Covering all types of probabilistic models is beyond the scope of this book.  For extensive discussion of probabilistic models, please see \citet{bodetal2003}.  


\subsection{Bayesian model of syntactic parsing}
  
In a Bayesian model of speech processing, probability distributions over a set of encountered variables are stored.  They are then used during speech processing to determine the most likely candidate given the specific context \citep{norrismcqueen2008} and social information associated with a linguistic form can influence what is identified as the most likely candidate \citep{staumcasasanto2009}.  \citet{norrismcqueen2008} implemented a Bayesian model of speech processing based on phonemes (called Shortlist B) and they state that the model could just as easily be implemented using other units at a prelexical level of processing, including bundles of features and position-specific allophones \citep[362]{norrismcqueen2008}.  They assume that ``word recognition necessarily involves a comparison of the evidence in the current acoustic input with stored knowledge about the phonological form of words'' \citep[379]{norrismcqueen2008}. It is unclear from their description whether this ``knowledge about the phonological form'' could include detailed phonetic information such as the probability of a segment being observed with a particular duration.  In their Bayesian model of phonetic imitation, \citet{nielsenwilson2008} use feature representations ($+$spread glottis) to encode phonetic detail (VOT), and something similar could be implemented in Shortlist B. Results from the perception experiments presented in Chapter \ref{ch:perc} indicate that fine phonetic detail such as the duration of the /l/ relative to the vowel duration affects a perceiver's identification of a word and therefore such information needs to be stored in a form that maintains the multidimensional and gradient nature of the phonetic signal.

Although Bayesian models have yet to be applied to the production and comprehension of patterns involving fine phonetic detail, they are successful at predicting trends in human parsing of syntactic structure during reading tasks.   \citet{narayananjurafsky1998,narayananjurafsky2002} implemented a Bayesian-based model in which probabilities of preceding contextual information, such as tense, contribute to the overall probabi\-lity of different interpretations of thematically ambiguous structures; the model prunes parses that have a low probability.  As described by \citet[59]{narayananjurafsky2002}, \textit{the cop arrested} is ambiguous: \textit{the cop} could be the agent, as in \textit{the cop arrested the crook}, or the theme, as in \textit{the cop arrested by the detective was guilty of taking bribes.}  Because \textit{cop} is most likely to be the agent when followed by \textit{arrested}, reading times are slower when it is the theme. Their model predicts this because it incorporates probability distributions specifying the most likely tense and argument structure for every verb. 
  
The perception results presented in Chapter \ref{ch:perc} provide evidence that perceivers were influenced by the preceding context of a lemma and that the effect of the preceding context was consistent with previously observed trends from production.  In ambiguous contexts such as \textit{I was like}, where \textit{like} could be either the quotative or the discourse particle, participants were more likely to identify tokens as the quotative if they were preceded by the first person pronoun.  \citet{buchstallerdarcy2009} found that in New Zealand English, quotative \textit{like} is more frequently found with the first person pronoun than with the third person pronoun.  Therefore, an interpretation where a token of \textit{like} is the quotative has a higher probability when the preceding context is in the first person than the third person.  In the task, perceivers were simply asked to identify which of the two tokens of \textit{like} was the quotative and which was the discourse particle; the task did not require a comparison of probabilities across all words in that context.  In Narayanan and Jurafsky's model, each interpretation of an utterance receives a probability based on previous experience.  The tense and person most often encountered with quotative \textit{like} would contribute to the overall probability of an interpretation of the utterance where \textit{like} is the quotative.  This would bias responses toward identifying a token as quotative \textit{like} when the preceding contextual information is that which is most frequently observed.

If context-dependent probabilities of lemmas are stored, they may also be accessed during speech production.  \citet{jurafskyetal2002} found that the probability of a word given its context was linked to the duration of that word; the more predictable the word, the more likely it was to have a shorter duration.  The interaction between /k/ realisation and relative frequency of use that was observed in the SGH data supports \citegen{jurafskyetal2002} finding and provides evidence that the effect is speaker-specific; if, when producing a quotative, a speaker's probability of producing quotative \textit{like} is high, a token of quotative \textit{like} that is produced is less likely to have the /k/ present.  

%This provides evidence that language use is probabilistic and that stored probabilities are based on an individual's experience.

%If such a model were also applied to speech production, it could account for the results presented in \sectref{sec:proddisc} indicating a link between /k/ realisation and the speaker-specific probability of a word in a given context.  The results provide evidence that, when producing a quotative, a speaker was less likely to realise the /k/ if they were more likely to use quotative \textit{like} as opposed to one of the alternative quotatives (e.g., \textit{say}, \textit{go}, \textit{scream}).   if the observed difference in phonetic realisations were based entirely on the predictive power of the preceding context, we would not expect to observe sensitivity to lemma-based duration differences in perception


Another model that could account for the observed bias in lemma identification is one that computes the probabilities online during speech processing.  This online computation could occur in a model where each lemma from the context is stored as a cloud of exemplars, and lemma-level exemplars from a given utterance are either indexed to each other or stored as an utterance-complete exemplar.  An exemplar model of speech production and perception is discussed in the next section.


\subsection{Exemplar Theory}\label{exemplar}

A model of speech production and perception that relies on the storage and retrieval of acoustically-rich detail is based in Exemplar Theory.  In an exemplar model of speech production and perception, utterances are stored in the mind as episodic memories (exemplars) \citep{pisoni1997} complete with acoustically detailed information \citep{goldinger1997,pierrehumbert2001,pierrehumbert2002}.  For example, if a listener hears a speaker produce an utterance such as \textit{look at the cat}, the theo\-ry assumes that the listener stores the word \textit{cat} with all of the acoustic detail inherent in the signal.  This includes the quality of the \textipa{/a/}, speaker-specific quali\-ties such as nasality and, if the \textipa{/t/} was released, the exact quality of the release.  The stored exemplar for \textit{cat} is indexed to exemplars from the rest of the utte\-rance.  Phrases that are encountered at a very high frequency can be stored as a single representation \citep{bybee2006}. Attention paid to speech influences the activation of the exemplars through attention weights on the exemplars \citep{nosofsky1986}.  Greater attention results in greater weight, which in turn results in greater activation.  

Phonetic information in the signal is indexed to a separate cloud of phoneme-level exemplars.  For the utterance \textit{look at the cat}, the attributes of [k] in \textit{cat} are indexed to a label \textipa{/k/}.  This phoneme-level exemplar is in the same cloud, or is even the same label, as the phoneme-level exemplar \textipa{/k/} that is indexed to the attributes of [k] from \textit{look}.  \citet{pierrehumbert2006} refers to this type of model that combines acoustically-rich exemplars with abstract labels as a hybrid model.  She argues that it can account for results which provide evidence that speaker-specific information is stored (e.g., Goldinger 1997) as well as results that suggest abstractions are required, such as the opposite effects on word recognition of two highly correlated factors: likely phonotactics and neighborhood density \citep{vitevitchluce1999}.  

\begin{figure}
  \includegraphics[width=.95\textwidth]{images/s90cr.pdf} %cropped version
	\caption{Sketch of exemplar model based on results from \citet{trudgill1972} with distributions of remembered exemplars of the word \textit{fishing}.  Each exemplar is indexed to a label for phonemic category ({/{\ng}/}) and speaker sex (male and female).} 	\label{fig:with-phoneme-social-labels}
	
\end{figure} 

In addition to linguistic contextual information, exemplars are indexed to a myriad of other factors that are stored at the time of the utterance.  These include the formality of the situation and the social characteristics of the person who produced the utterance \citep{johnson1997,foulkesanddocherty2006}.  Again, salience plays a role. Non-linguistic information is only stored if it was available at the time of the utterance and if it may be important to the perceiver \citep[147]{johnson1997}.  A sketch of this indexation is shown in \figref{fig:with-phoneme-social-labels} for an exemplar cloud of the word \textit{fishing} within the context of \citegen{trudgill1972} finding that females were more likely to realise word final {/{\ng}/} as the velar nasal {[{\ng}]} whereas males were more likely to realise it as the alveolar nasal {[n]}.  Exemplars representing encountered utterances produced by males and females are indexed both to the phoneme level (e.g., {/{\ng}/}) and to characteristics of the speaker who produced the utterance (e.g., male).  


During production, the final realisation is a result of averaging over an entire region of an exemplar cloud. There is not a one-to-one mapping of activated exemplar to the token that is ultimately perceived or produced \citep{pierrehumbert2001}.  This is in contrast to some presentations of exemplar models where a one-to-one mapping is assumed \citep{griffithsetal2007}.  Exemplars which have been activated recently and those which are activated frequently carry the highest weight values, resulting in a bias in production toward variants resembling these exemplars.  As with perception, non-linguistic information indexed to the exemplars can bias which variants are produced.  After storage, exemplars immediately begin to decay and frequent activation slows decay.  This activation can occur through encountering an acoustically similar utterance. 

The region of an exemplar cloud that is activated during production may be selected as a result of its indexation to social factors with which the speaker identifies.  Additionally, social characteristics of an addressee activate social exemplars, thereby biasing production toward the speech of that addressee in ways that depend on the speaker's attitudes toward the interlocutor \citep{drageretal2010,babel2012}.  This prediction is consistent with the well-known effects of audience design and speech accommodation \citep{bell1984,gilesetal1991,oprah1999}.

During perception, exemplars are activated to varying levels depending on their similarity to the incoming utterance.  If incoming social information closely matches a previously stored social exemplar, the linguistic exemplar indexed to the social information will receive partial activation.  These partially-activated exemplars reach full activation faster than acoustically similar exemplars that are not indexed to a relevant social exemplar, resulting in a bias in perception depending on the perceived social characteristics of the speaker \citep{strandjohnson1996,niedzielski1999,haywarrendrager2006}.   
 
Thus, exemplar-based models such as these predict a bias in both production and perception toward socially relevant exemplars. Additionally, because diffe\-rent words with the same wordform are indexed to different lemma-specific phonetic information, exemplar models predict that (1) there can be lemma-con\-di\-tioned phonetic variation in production that patterns according to exposure (a speaker's realisations will resemble those of other speakers with whom they regularly interact) and (2) individuals can use phonetic information based on trends in production to identify a lemma during speech perception.  And because exemplars are stored complete with acoustically-detailed information, it predicts that trends in production can be phonetically gradient and that individuals will be sensitive to acoustically-detailed information during the perception of speech.

It is important to keep in mind that the mental representations reflect what was perceived, not what was produced or, even, what could potentially have been perceived.  This means that a number of factors, including attention, influence the form of the representation \citep{foulkeshay2015}.  

In many cases, exemplar-based models and Bayesian-based models behave si\-milarly.	 \citet{pierrehumbert2002} states that an exemplar model of speech production and perception should be viewed as

\begin{quote}

a logical schema rather than taking it as a literal picture of activity in the brain.  Any model which stores implicit and incrementally updatable frequency distributions over a cognitive map will show similar behaviour; it is not important that all percepts are individuated as separate memories in the long term. \citep[113]{pierrehumbert2002}

\end{quote}

\noindent Even if episodic traces of acoustically-rich utterances are not stored, each utte\-rance could update the system in such a way that probabilities (with their base in frequency distributions) could be stored.  However, as evidenced by the work in the previous two chapters, this needs to include probabilities of very detailed, acoustically-rich information as well as very rich social information.

	
In conceptualising this logical schema, it may help to view different modes of representation for different levels of the grammar.  Rich phonetic detail of specific episodes may be stored and may influence both production and perception, and probabilities may also be abstracted and stored, influencing the production and perception of higher-level processes such as those involving syntactic information.  For example, a Bayesian-based model could account for the effect of surrounding contextual information on lemma identification in Experiment 1 and an exemplar-based model could account for the lemma-conditioned phone\-tically gradient trends observed in production and the sensitivity to these trends observed in perception. 


\section{Indexation of social information}\label{sec:richmodel}

In current exemplar-based models, such as those described by \citet{johnson1997} and \citet{haywarrendrager2010}, the representation of social information that is indexed to acoustically-rich exemplars is consistent with variationist work from the First and Second Waves of variation studies, where phonetic variables are treated as indexed directly to social categories. These categories can be broad, as in the sketch in \figref{fig:with-phoneme-social-labels}, or can be locally constructed.  

However, the representation of social information is much richer than this indexation would suggest.  In this section, I step through how, in the construction of social personae, the adoption and rejection of linguistic and non-linguistic features might be understood within an exemplar-based hybrid model.

Work in the Third Wave treats linguistic variables as directly indexed to style.  Style is complex; it is comprised of socially-meaningful components that can shift in meaning depending on other components indexed to the style.  From situation to situation, the style of a single speaker can shift, sometimes subtly, sometimes dramatically.  

A speaker's stance can serve to create that speaker's style.  For example, if a speaker views themselves as ``different'' from the norm, they create their indivi\-dual style through the expression of this stance.  At SGH, goths and geeks created different styles from one another but because their stance was in opposition to a third group (e.g., The PCs or the CR girls as a whole), some of their styles could have components that resembled each other (e.g., patterns of /k/ realisation in quotative and discourse particle \textit{like}).

\citet{mendozadentonetal2003} state that a usage-based probabilistic model is

\begin{quote}
entirely compatible with a view of the social world that relies on gradually built up social categories that emerge from experiences that surround individuals as social actors. \citep[136]{mendozadentonetal2003}
\end{quote}

\noindent Yet, no one has spelled out how stylistic variation occurs within the context of a probabilistic linguistic model.  One challenge that arises when trying to do so is that potential stylistic components not only come to be imbued with social meaning based on the presence of an item, activity, or characteristic but also from the absence of wearing certain items or from not taking part in certain activities.  For example, Santra (The Goths, NCR) wore black clothes and the colour of the clothes was meaningful in that it helped to construct her social persona.  But also meaningful was that Santra did not wear mini skirts or bright colours.  One day when she wore a green shirt, someone commented that they had never seen her wear a bright colour before.  Santra confessed that she didn't feel like herself and was looking forward to going home so that she could change clothes.  Refraining from participating in certain activities (e.g., wearing bright colours), both linguistic and non-linguistic, can itself be socially meaningful and helps to contribute to a speaker's style.  But if exemplar clouds are based on previously encountered occurrences, how does the lack of a characteristic or item of clothing become a stylistic component?  The model presented here addresses this through the indexation of a speaker's style to different parts of multidimensional stylistic features: the part of the distribution to which a speaker's style is indexed indicates the degree to which that component is adopted in the construction of her style.  Both identification and avoidance can occur through comparing how different styles index different parts of multidimensional representations of stylistic components.

In \figref{fig:SketchTwoStyles2}, I present a sketch of Santra's (The Goths, NCR) style and Betty's (The Sporty Girls', CR) style within the context of an exemplar model of speech production and perception.  Of course, a speaker's style is multidimensional and shifts depending on the situation.  A speaker's shifting style may not be the overt abstraction implied by the sketch in \figref{fig:SketchTwoStyles2}.  Instead, components may be indexed to a representation of the speaker and that speaker's style could arise with particular patterns of activation over the components.  This could account for how styles shift in ways that are sometimes subtle and sometimes dramatic.  For simplicity, the styles modelled here represent the general styles that were consistently observed for the girls within the context of the school.  Each of the components is based on their own cloud of exemplars, where the stored exemplars are representations of previous encounters with each of the girls.  

\begin{figure}
	\centering
		\includegraphics[width = \textwidth]{images/s2scr.pdf} %replaced by cropped img
	\caption{Sketch of two speakers' styles and their linguistic and non-linguistic components.  The portion of the shaded box that is indexed reflects the degree to which an individual adopts or rejects that characteristic when constructing a given style.}
	\label{fig:SketchTwoStyles2}
\end{figure}



In order for the lack of an item to become socially meaningful, comparisons must be made between potential stylistic components observed in a social arena.  Different individuals will vary probabilistically in how they are indexed to these components and the components themselves become more socially meaningful the further apart the sections are that the different styles index.  In \figref{fig:SketchTwoStyles2}, this is displayed through indexing different regions of multidimensional components; different parts of the multidimensional representations are indexed depending on the likelihood of a girl possessing that item or characteristic.  Here, I have treated the horizontal plane within each shaded box as frequency across time (e.g., some girls were more likely to wear a skirt than others) and the vertical plane as another dimension at a given point in time, such as the number of items worn in that colour or the length of a skirt when worn.  For example, `wears black' is labelled as a stylistic component and the styles of different girls are indexed to different parts of this abstract representation depending both on how often they wear black and, when wearing black, how much of their clothing is black.  The style represented for Betty is indexed to the whiter region of the box, indicating that she wears black less than half of the time and, when even wearing black, does not wear many items that are black.  This indexation reflects the probability that a single speaker will adopt one of these stylistic components, thereby constructing their personal style.  Indexation can occur not only through the storage of exemplars based on experience with an individual, but through the comparison of that individual with others.  

Not all items or characteristics that could potentially be components of an individual's style become imbued with social meaning.  For example, both Santra and Betty wore tight-fitting t-shirts.  Donning this type of shirt was not particularly meaningful in differentiating their different styles; this is reflected in their inde\-xation to a similar space within this potential component.  However, the colour of the top was potentially meaningful: Betty's might be blue or black depending on the day whereas Santra's was almost certainly black.  

That indexation of stylistic components relies on comparisons also ensures that the lack of an item or characteristic is meaningful only within the context of a given social arena.  This is desirable because traits that are completely absent from the reality of the social arena do not meaningfully affect an individual's style.  For example, that Santra did not wear a tiara was not socially meaningful because none of the girls wore tiaras to school.  

Linguistic variation, like the variation found among other stylistic components, can be converged upon or diverged from depending on both the speaker's attitudes toward an individual and how similar they believe they are to the indivi\-dual.  As with other stylistic components, indexation may occur not only between a speaker's style and a phonetically-rich exemplar representing an utterance produced by that speaker, but also through the absence of producing a particular variant.  A sketch of this relationship is shown in \figref{fig:SketchStylesLing}. While the sketch shows only two dimensions, the model is not limited in this way. Furthermore, it is important to note that speakers can shift their indexations between different parts of the mutlidimensional space and that this shift can occur throughout an interaction.

\begin{figure}
	\centering
		\includegraphics[width=\textwidth]{images/sslcr.pdf} % replaced SketchStylesLing.pdf with cropped version
% 	\input{images/plots/indexation.tex}
	\caption{Sketch of indexation between multidimensional linguistic components and speakers' styles. For simplicity, only two dimensions are shown: frequency of using quoative \textit{like} is shown on the x-axis and the likelihood of dropping the /k/ is on the y-axis.}
	\label{fig:SketchStylesLing}
\end{figure}


As with non-linguistic elements, linguistic components of style are multidimensional, with different styles indexed to different parts of the distribution.  The dimensions represented in \figref{fig:SketchStylesLing} are frequency of use (along the horizontal plane) and the likelihood of realising the /k/ in any given token (along the vertical plane). For example, Marama's General School-Style is indexed to the portion of the distribution of quotative \textit{like} where there is a relatively low likelihood of dropping the /k/ in addition to a lower frequency of use of the quotative when compared with many of her peers.  In contrast, her School-Style is indexed to the portion of the distribution of discourse particle \textit{like} where there is a high likelihood of dropping the /k/ and a low frequency of using the discourse particle relative to other girls at the school.  Of course, other dimensions would include other information, such as the duration of a segment and the frequency bands of the formants.  Such indexation to multidimensional representations of linguistic variables predicts that identity construction will result in convergence among speakers constructing similar styles to one another.  Because distributions of these indices can be compared across different speakers, it also predicts divergence by individuals wishing to differentiate themselves from a particular style; indexation to the multidimensional space can be manipulated through comparison of stored indices of other speakers' styles to the space.  A speaker can index a space that is void of exemplars and can do so in relation to the observed behaviour of the other speakers.

  
These patterns of activation over the multitude of components that comprise a style can explain how vowel shifts can occur within an exemplar-based hybrid model.  In \citegen{pierrehumbert2001} model, vowel shifts could only be driven by a speaker producing variants outside the realm of previously stored exemplars as a result of random noise; there is no socially-driven motivation for vowel shifts built in to the model.  However, it is highly unlikely that vowel shifts result entirely from random noise: individuals who lead vowel shifts are the same individuals who lead other stylistic changes \citep{labov2001}.  For example, elementary school girls who produce the most extreme phonetic variants are the same individuals who begin wearing nail polish or lacy underwear \citep{eckert1996nailpolish}; they are the individuals who first adopt the most extreme components of styles from the heterosexual marketplace in the construction of their identity within an emerging peer social order.  The model presented here, which combines acoustically-detailed episodic memories with multi-dimensional abstractions of the acoustic space, a\-llows speakers to index these spaces that are potentially void of acoustically-rich exemplars.  This indexation can have a direct effect on the variants produced; speakers constructing personae that are extreme within the context of a social arena can produce variants that are extreme in comparison to other speakers in that arena.  Linguistic variation is a stylistic resource and the manner in which it is stored must allow for the construction of a speaker's identity.
%see page 360-364 for Labov's work


\section{Conclusion}

In this chapter, I discussed two probabilistic models of language use and I described how they can account for the results presented in the preceding two chapters.  The most comprehensive model may be some combination of these, incorporating both stored exemplars of utterances complete with acoustic detail and abstracted probabilities of phrase structures.  In a model where clouds of phonetically-rich exemplars contribute to abstractions of multidimensional stylistic components, it is possible to account for phonetic variation that patterns according to stylistic choices made by the speaker.  



%\newpage
%\thispagestyle{empty}
%\mbox{}