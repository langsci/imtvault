\section{Basic syntax}\label{sec:10}



\subsection{Noun phrases}\label{sec:10.1}


The Ik \textsc{noun phrase} consists first and foremost of a noun ‘head’, either a lexical noun or a nominalized lexical verb. As a head-initial language, Ik places its \isi{noun phrase} head first in the phrase. Any subordinate, supporting elements follow the head. These optional elements may include \isi{anaphoric} demonstratives, possessive markers, relative pronouns/temporal demonstratives, number markers, and spatial demonstratives. The Ik \isi{noun phrase} structure can be formalized as follows, where elements in parentheses are optional:


\newpage 
\ea\label{ex:syn:1}
Ik NP structure: \\
\textsc{head (anaph)(poss)(num)(rel/temp) (dem)}
\z


The syntactical structure of noun phrases formalized in \REF{ex:syn:1} is fleshed out among the real Ik noun phrases presented below in examples \REF{ex:syn:2}-\REF{ex:syn:10}:




\ea\label{ex:syn:2}
\gll \textsc{head} \\
 wikᵃ    \\
\glt ‘children’ 
\z





\ea\label{ex:syn:3}
\gll \textsc{head} \textsc{anaph} \\
wika díí    \\
\glt ‘those (specific) children’ 
\z





\ea\label{ex:syn:4}
\gll \textsc{head} \textsc{poss} \\
wika \'{ɲ}cì    \\
\glt ‘my children’ 
\z





\ea\label{ex:syn:5}
\gll \textsc{head} \textsc{anaph} \textsc{poss} \\
wika díí \'{ɲ}cì    \\
\glt ‘those (specific) children of mine’ 
\z





\ea\label{ex:syn:6}
\gll \textsc{head} \textsc{anaph} \textsc{poss} \textsc{num} \\
wika díí \'{ɲ}cìè lèɓètsè    \\
\glt ‘those two (specific) children of mine’ 
\z





\ea\label{ex:syn:7}
\gll \textsc{head} \textsc{anaph} \textsc{poss} \textsc{rel}  \\
wika díí \'{ɲ}cie [ni leɓetse]\textsc{\textsubscript{rel}}    \\
\glt ‘those (specific) children of mine, two in number’ 
\z





\ea\label{ex:syn:8}
\gll \textsc{head} \textsc{anaph} \textsc{poss} \textsc{num} \textsc{rel} \\
wika díí \'{ɲ}cie leɓetse [ní dà]\textsc{\textsubscript{ rel}}    \\
\glt ‘those two nice (specific) children of mine’ 
\z





\ea\label{ex:syn:9}
\gll \textsc{head} \textsc{anaph} \textsc{poss} \textsc{num} \textsc{rel} \textsc{dem} \\
wika díí \'{ɲ}cie leɓetse [{ní daa}]\textsc{\textsubscript{rel}} ni    \\
\glt ‘those two nice (specific) children of mine, these’ 
\z





\ea\label{ex:syn:10}
\gll \textsc{head} \textsc{anaph} \textsc{poss} \textsc{num} \textsc{temp} \textsc{dem} \\
wika díí \'{ɲ}cie leɓetse níi ni    \\
\glt ‘those two (specific) children of mine from earlier, these’ 
\z






\subsection{Clause structure}\label{sec:10.2}
\subsubsection{Intransitive}\label{sec:10.2.1}

Ik \textsc{intransitive} clauses consist minimally of a verb (\textsc{v}) and a subject (\textsc{s}) in a \textsc{vs} \isi{constituent order}. The subject may be explicit, in which case it follows the verb, or it may be implicit, in which case it is merely marked on the verb. Basic \isi{intransitive} \isi{clause structure} is illustrated in example \REF{ex:syn:11}:




\ea\label{ex:syn:11}
\gll Epa\textsc{\textsubscript{v}}\textsc{}    ŋókᵃ\textsc{\textsubscript{s}}. \\
sleep:\textsc{3sg}   dog:\textsc{nom}    \\
\glt ‘The dog sleeps.’ 
\z


When a \isi{tense} \isi{adverb} is needed, it comes directly after the verb and before any explicit subject. And any other adverbial elements like extended objects (\textsc{e}) or adverbs, in that order, come after the subject. This word order is shown in \REF{ex:syn:12}:



\ea\label{ex:syn:12}
\gll Epá\textsc{\textsubscript{v}}=bee\textsc{\textsubscript{tense}}   ŋóká\textsc{\textsubscript{s}}     kurú\textsc{\textsubscript{e}}. \\
sleep:\textsc{3sg}=yester-   dog:\textsc{nom}   shade:\textsc{abl}    \\
\glt ‘The dog slept in the shade yesterday.’ 
\z




\subsubsection{Transitive}\label{sec:10.2.2}

Ik \textsc{transitive} clauses consist minimally of a transitive verb (\textsc{v}), an agent (\textsc{a}), and an object (\textsc{o}) in a \textsc{vao} \isi{constituent order}. The subject may be explicit, in which case it comes between the verb and object, or it may merely be marked on the verb with a suffix. The object may also be dropped, in which case it is inferred from the context. Example \REF{ex:syn:13} illustrates basic transitive \isi{clause structure}:




\ea\label{ex:syn:13}
\gll Átsʼá\textsc{\textsubscript{v}}    ŋóká\textsc{\textsubscript{a}}    ɔkákᵃ\textsc{\textsubscript{o}}\textsc{.} \\
gnaw:\textsc{3sg}  dog:\textsc{nom}  bone:\textsc{acc}    \\
\glt ‘The dog gnaws the bone.’ 
\z


When a \isi{tense} \isi{adverb} is needed, it comes directly after the verb and before any explicit subject. And any other adverbial elements like extended objects (\textsc{e}) or adverbs, in that order, come after the subject. This syntax is shown in \REF{ex:syn:14}:



\ea\label{ex:syn:14}
\gll Átsʼá\textsc{\textsubscript{v}}=bee\textsc{\textsubscript{tense}}   ŋóká\textsc{\textsubscript{a}}     ɔkáá\textsc{\textsubscript{o}}         ódàtù\textsc{\textsubscript{e}} \\
gnaw:\textsc{3sg}=yester-  dog:\textsc{nom}  bone:\textsc{acc}  day:\textsc{ins}    \\
\glt ‘The dog gnawed the bone all day yesterday.’ 
\z




\subsubsection{Ditransitive}\label{sec:10.2.3}

Ik \textsc{ditransitive} clauses consist minimally of a ditransitive verb (\textsc{v}), an agent (\textsc{a}), an object (\textsc{o}), and an extended object (\textsc{e}) in a \textsc{vaoe} \isi{constituent order}. If the agent is not mentioned explicitly, then it will still be marked with a suffix on the verb. The object and extended object may be left implicit but will be understood from context. The basic ditransitive \isi{clause structure} is illustrated in \REF{ex:syn:15}:




\ea\label{ex:syn:15}
\gll Maa\textsc{\textsubscript{v}}     ƙaƙaama\textsc{\textsubscript{a}}   ɔkáá\textsc{\textsubscript{o}}     ŋókíkᵉ\textsc{\textsubscript{e}}. \\
give:\textsc{3sg}   hunter:\textsc{nom}   bone:\textsc{acc}   dog:\textsc{dat}    \\
\glt ‘The hunter gives a bone to the dog.’ 
\z




\subsubsection{Causative}\label{sec:10.2.4}

By adding an extra element in the form of a causing agent, Ik \textsc{causative} verbs change the structure of a clause. If the original clause was a \textsc{vs} \isi{intransitive} one, then the \isi{causative} changes it to a transitive \textsc{vao}. If the original clause was a transitive \textsc{vao}, then the \isi{causative} changes it to a ditransitive \textsc{vaoe}. The following two examples, \REF{ex:syn:16}-\REF{ex:syn:19}, show \isi{causative} verbs making these structural changes:\\




Intransitive \textsc{vs} → Causative \textsc{vao}
\ea\label{ex:syn:16}
\gll Fekíà\textsc{\textsubscript{v}}     \`{ŋ}kᵃ\textsc{\textsubscript{v}}. \\
laugh:\textsc{1sg}   I:\textsc{nom}    \\
\glt ‘I laugh’. 
\z




\ea\label{ex:syn:17}
\gll Fekitéídà\textsc{\textsubscript{va}}   {\`{ŋ}kᵃ}\textsc{\textsubscript{o}}. \\
laugh:\textsc{caus:2sg} I:\textsc{nom}    \\
\glt ‘You make me laugh.’ 
\z





Transitive \textsc{vao} → Causative \textsc{vaoe}
\ea\label{ex:syn:18}
\gll Wetía\textsc{\textsubscript{ v}}     ŋka\textsc{\textsubscript{a}}     cue\textsc{\textsubscript{o}}. \\
drink:\textsc{1sg}   I:\textsc{nom}   water:\textsc{nom}    \\
\glt ‘I drink water.’ 
\z




\ea\label{ex:syn:19}
\gll Wetitéída\textsc{\textsubscript{va}}   ŋka\textsc{\textsubscript{o}}     cuékᵉ\textsc{\textsubscript{E}}. \\
drink:\textsc{caus:2sg} I:\textsc{nom}    water:\textsc{dat}    \\
\glt ‘You make me drink water.’ 
\z




\subsubsection{Auxiliary}\label{sec:10.2.5} 

Ik has both true \textsc{auxiliary} verbs and \textsc{pseudo-auxiliary} verbs. Both types modify sentence syntax. The true auxiliaries, shown in \tabref{tab:syn:aux1}, function as the syntactic main verb in a clause, while the \textit{semantic} main verb follows the subject (\textsc{s/a}) in a morphologically defective form that consists of the bare verb stem plus a suffix \{-a\} (which may be the realis marker from \sectref{sec:8.9.2}). This means the \isi{constituent order} of clauses with true auxiliary verbs is \textsc{auxSV} for intransitives, \textsc{auxAVO} for transitives, and \textsc{auxAVOE} with extended objects. Again, in all these constructions, the \textsc{aux} acts as the main verb from a syntactic perspective, while the defective verb carries the main meaning of the verbal schema. Another way to analyze this construction would be to say that the \isi{auxiliary verb} and the defective verb \textit{together} fill the single verb slot of the clausal syntax.

The true auxiliaries have both lexical and aspectual meanings, which are nevertheless practically identical in their semantics. However, in their lexical function, the verbs in \tabref{tab:syn:aux1} do not require a second, morphologically defective verb to augment them; in their strictly lexical usage, they stand alone:


\begin{table}
\caption{Ik true auxiliary verbs}
\label{tab:syn:aux1}


\begin{tabularx}{.66\textwidth}{lXX}
\lsptoprule

Root & Lexical & Aspectual\\
\midrule
erúts- & ‘be fresh, new’ & \textsc{recentive}\\
ŋ\'{ɔ}r- & ‘do already/early’ & \textsc{anticipative}\\
sár- & ‘be still/not yet’ & \textsc{durative}\\
\lspbottomrule
\end{tabularx}
\end{table}
Example \REF{ex:syn:20} illustrates the use of the recentive aspectual \isi{auxiliary verb} \textit{erúts-} in an \isi{intransitive} clause with the structure \textsc{auxSVE:}




\ea\label{ex:syn:20}
\gll Erúts{íma}\textsc{\textsubscript{AuxS}}   atsa\textsc{\textsubscript{v}}     sédàᵒ\textsc{\textsubscript{e}}. \\
\textsc{recent:1pl.exc}   come     garden:\textsc{abl}    \\
\glt ‘We just came from the garden.’ 
\z


Example \REF{ex:syn:21}, on the other hand, shows the use of the anticipative verb \textit{ŋ\'{ɔ}r-} in a transitive clause with the structure \textsc{auxAVOE}:



\ea\label{ex:syn:21}
\gll Ŋ{\'{ɔ}rá}\textsc{\textsubscript{AuxA}}=naa   cɛa\textsc{\textsubscript{v}}   riáá\textsc{\textsubscript{o}}        baratso\textsc{\textsubscript{e}}=nákᵃ. \\
\textsc{anticip:3sg=pst1}   kill   goat:\textsc{acc} morn:\textsc{ins=dem.pst1}    \\
\glt ‘He already killed the goat earlier this morning.’ 
\z


Lastly, sentence \REF{ex:syn:22} exemplifies the durative aspectual verb \textit{sár-} in a simple transitive clause working with the defective verb \textit{tsʼágwa-}:



\ea\label{ex:syn:22}
\gll Sárá\textsc{\textsubscript{Aux}}  séda\textsc{\textsubscript{s}}     tsʼágwà\textsc{\textsubscript{v}}. \\
\textsc{dur}:\textsc{3sg}   garden:\textsc{nom}   unripe    \\
\glt ‘The garden is still unripe.’ 
\z


In contrast to the above examples, the pseudo-auxiliary verbs only mimic true auxiliaries in that they are fully lexical verbs yet ones with potentially aspectual meanings, including the completive, \isi{inchoative}, and occupative. However, because they are not \textit{syntactically} auxiliary, they take complements as any lexical verb would (direct objects for the transitive ones and extended objects for the \isi{intransitive} one). The pseudo-auxiliaries are presented in \tabref{tab:syn:aux2} with their lexical and aspectual meanings and the cases required in their complements:


\begin{table}
\caption{Ik pseudo-auxiliary verbs}
\label{tab:syn:aux2}


\begin{tabularx}{\textwidth}{XXXX}
\lsptoprule

Stem & Lexical & Aspectual & Case required\\
\midrule
náb-ʉƙɔt- & ‘end, finish’ & \textsc{completive} & \textsc{nom/acc}\\
itsyák-ét- & ‘begin, start’ & \textsc{inchoative} & \textsc{nom/acc}\\
toɗó- & ‘alight, land’ & \textsc{inchoative} & \textsc{nom/acc}\\
isé-ét- & ‘begin, start’ & \textsc{inchoative} & \textsc{nom/acc}\\
c\`{ɛ}m- & ‘fight, struggle’ & \textsc{occupative} & \textsc{ins}\\
\lspbottomrule
\end{tabularx}
\end{table}
Each of the aspectual meanings listed in \tabref{tab:syn:aux2} are given one example in the following sentences. The brackets in example \REF{ex:syn:23} signify that the bracketed \isi{noun phrase} as a whole is the object of the verb:\\



Completive
\ea\label{ex:syn:23}
\gll Nábʉƙɔt{\Í}áa\textsc{\textsubscript{va}}    [isóméésá   ɲáɓúkwi]\textsc{\textsubscript{o}}. \\
finish:\textsc{1sg:prf}   to.read:\textsc{nom}   book:\textsc{gen}    \\
\glt ‘I have finished reading the book.’ 
\z




Inchoative
\ea\label{ex:syn:24}
\gll Itsyaketátaa\textsc{\textsubscript{va}}  wáánàkᵃ\textsc{\textsubscript{o}}. \\
begin:\textsc{3pl:prf}   praying:\textsc{acc}    \\
\glt ‘They have begun praying.’ 
\z




Occupative
\ea\label{ex:syn:25}
\gll Cɛma\textsc{\textsubscript{v}}    wika\textsc{\textsubscript{s}}       wáákᵒ\textsc{\textsubscript{e}}. \\
fight:3   children:\textsc{nom}   playing:\textsc{ins}    \\
\glt ‘The children are busy playing.’ 
\z




\subsubsection{Copular}\label{sec:10.2.6}

Ik \textsc{copular} clauses have relational rather than referential meanings. They link a \textsc{\isi{copular} subject} (\textsc{cs}) to a \textsc{copular} \textsc{complement} (\textsc{cc}) which represents an entity or attribute, depending on the specific \isi{copular} verb involved. The \isi{constituent order} of \isi{copular} clauses is therefore \textsc{v-cs-cc}. Ik has three distinct \isi{copular} or ‘be’ verbs that can express five \isi{copular} relationships between them. These \isi{copular} verbs are presented in \tabref{tab:syn:cop} below, along with the case markings their subjects and complements are obligated to have:


\begin{table}
\caption{Ik \isi{copular} verbs}
\label{tab:syn:cop}


\begin{tabularx}{\textwidth}{XXXX}
\lsptoprule

Verb & Meaning & \textsc{cs} case & \textsc{cc} case\\
\midrule
ì- & Existence & \textsc{nom} & \textsc{–}\\
& Location & \textsc{nom} & \textsc{dat}\\
ìr- & Attribution & \textsc{nom} & (\isi{adverb} only)\\
m{\Ì}t- & Identity & \textsc{nom} & \textsc{obl}\\
& Possession & \textsc{nom} & \textsc{gen}\\
\lspbottomrule
\end{tabularx}
\end{table}
The three \isi{copular} verbs in \tabref{tab:syn:cop} and their five potential meaning are each exemplified briefly in the example sentences \REF{ex:syn:26}-\REF{ex:syn:30}:\\




Existence
\ea\label{ex:syn:26}
\gll Ia\textsc{\textsubscript{v}}     didigwarí\textsc{\textsubscript{cs}}. \\
be:\textsc{3sg}   rain.top:\textsc{nom}    \\
\glt ‘Heaven [i.e. God] is (there).’ 
\z




Location
\ea\label{ex:syn:27}
\gll Ia\textsc{\textsubscript{v}}   lɔŋ\'{ɔ}tá\textsc{\textsubscript{cs}}     muceékᵉ\textsc{\textsubscript{cc}}. \\
be:3   enemies:\textsc{nom}   way:\textsc{dat}    \\
\glt ‘Enemies are on the way.’ 
\z




Attribution
\ea\label{ex:syn:28}
\gll Ira\textsc{\textsubscript{vcs}}     tíyé\textsc{\textsubscript{adv}}. \\
be:\textsc{3sg}   like.this    \\

\glt ‘It is like this.’ 
\z




Identity
\ea\label{ex:syn:29}
\gll Mɨt{\Í}á\textsc{\textsubscript{v}}   ŋka\textsc{\textsubscript{cs}}    bábò\textsc{\textsubscript{cc}}.\\
be:\textsc{1sg}   I:\textsc{nom}    father.your:\textsc{obl}\\
\glt ‘I am your father.’ 
\z




Possession
\ea\label{ex:syn:30}
\gll Mɨta\textsc{\textsubscript{v}}     [awa=na]\textsc{\textsubscript{cs}}   ŋgóᵉ\textsc{\textsubscript{cc}}.\\
be:\textsc{3sg}   home:\textsc{nom}=this   we:\textsc{gen}\\
\glt ‘This house is ours.’ 
\z




\subsubsection{Fronted}\label{sec:10.2.7}

Ik can put special emphasis on any core nominal element by moving it to the front of the clause, before the verb, subject, and other constituents. Doing so obviously disrupts the usual syntactic structure of main clauses. Two kinds of fronting are observed in the language: 1) a \textsc{cleft} construction and 2) \textsc{left-dislocation}. In a \isi{cleft construction}, the emphasized noun is moved to the front and given the \isi{copulative case}. This puts it in an identifying relationship with the original clause out of which it just came. As a result, the newly arranged clause can be viewed as a kind of \isi{copular} clause where the fronted element is the \isi{copular} subject and the original clause the \isi{copular} \isi{complement}. This can in turn be formulized as: [NP:\textsc{cop}]\textsc{\textsubscript{cs}}\textsc{ [clause]}\textsc{\textsubscript{cc}}. To make this more concrete, the next examples show the \isi{cleft construction} with a simple transitive clause in \REF{ex:syn:31} whose object (\textit{m\`{ɛ}s}) gets fronted and marked with the \isi{copulative case} in \REF{ex:syn:32}:\\ 




Cleft construction
\ea\label{ex:syn:31}
\gll B\'{ɛ}ɗ{\Í}mà\textsc{\textsubscript{v}}    {\`{ŋ}gwà}\textsc{\textsubscript{a}}    m\`{ɛ}s\textsc{\textsubscript{o}}. \\
want:\textsc{1pl.exc}   we:\textsc{nom}  beer:\textsc{nom}    \\
\glt ‘We want beer.’ 
\z




\ea\label{ex:syn:32}
\gll Mɛsɔɔ\textsc{\textsubscript{cc}}     [ŋgóá    b\'{ɛ}ɗ{\Í}m.]\textsc{\textsubscript{cs}} \\
beer:\textsc{cop}    we:\textsc{acc}   want:\textsc{1pl.exc}    \\   
\glt ‘It is beer (that) we want.’ 
\z


Whereas the \isi{cleft construction} involves removing a clausal element from a clause and building a new clause, \isi{left-dislocation} simply relocates the element to the front of the clause, but still within the same clause. In this fronted position it is given the \isi{nominative case}. This type of fronting can be formulized as: [NP:\textsc{nom} \textsc{‖}\textsc{ clause]}\textsc{\textsubscript{clause}}, where the double vertical line symbolize a short pause. This type of \isi{left-dislocation} is illustrated between example sentences \REF{ex:syn:33}-\REF{ex:syn:34}:\\




Left-dislocation
\ea\label{ex:syn:33}
\gll Mée   eníí     kaúdza=díí. \\
not:\textsc{prf}   see:\textsc{1sg}   money:\textsc{nom}=\textsc{anaph}    \\
\glt ‘I haven’t seen that money.’ 
\z




\ea\label{ex:syn:34}
\gll Kaúdza=díí,     mée     ení. \\
money:\textsc{nom}=\textsc{anaph}   not:\textsc{prf}   see:\textsc{1sg}    \\
\glt ‘That money, I haven’t seen (it).’ 
\z






\subsection{Subordinate clauses}\label{sec:10.3}
\subsubsection{Overview}\label{sec:10.3.1}

The \isi{constituent order} of Ik \textsc{subordinate} clauses differs from that of \textsc{main} clauses. Ik subordinate clauses exhibit an \textsc{sv} order with \isi{intransitive} verbs, an \textsc{av} order with transitives, and an \textsc{ave} order with ditransitives – in short ‘\textsc{sv}’ instead of the usual ‘\textsc{vs}’. Case marking in subordinate clauses is also different: The fronted subject/agent and \textit{every} direct object take the \isi{accusative case}. 

The next two subsections deal with two key kinds of Ik \isi{subordinate clause}, the \isi{relative clause} (\sectref{sec:10.3.2}) and the \isi{adverbial clause} (\sectref{sec:10.3.3}).


\subsubsection{Relative clauses}\label{sec:10.3.2}

\textsc{relative clauses} are subordinate clauses that modify a noun within a \isi{main clause}. Ik relative clauses are restrictive, meaning they can only narrow the reference of their head noun rather than merely adding extra details about it. Relative clauses are introduced by the tensed relative pronouns discussed back in (\sectref{sec:5.7}), which, within the \isi{relative clause}, stand in for a noun in the \isi{main clause} called the \textsc{common argument} (\textsc{ca}). As such, the \isi{common argument} is a full verbal argument in the \isi{main clause}, while in the \isi{relative clause}, the \isi{relative pronoun} fills its syntactic slot.

As a \isi{subordinate clause}, an Ik \isi{relative clause} exhibits a different \isi{constituent order} than typical main clauses. Specifically, an \isi{intransitive} \isi{relative clause} has the order \textsc{sv} (instead of \textsc{vs}), and a transitive \isi{relative clause} has the order \textsc{oav} (instead of \textsc{vao}). In the former (\isi{intransitive}), the subject slot (\textsc{s}) is filled by the \isi{relative pronoun}, and in the latter (transitive), it is the object (\textsc{o}) that is represented by the \isi{relative pronoun}. Furthermore, apart for the relative pronouns themselves, all subjects and direct objects in relative clauses are marked with the \isi{accusative case} – another sign of grammatical subordination in Ik.

These attributes of Ik relative clauses are illustrated in examples \REF{ex:syn:35}-\REF{ex:syn:36}. In \REF{ex:syn:35}, the \isi{common argument} in the \isi{main clause} is \textit{emuta} ‘story’, which is modified by the \isi{relative clause} \textit{nɛ \'{ɛ}f} ‘that is funny’. Note how the subject slot of the \isi{relative clause} is filled by the \isi{relative pronoun} \textit{nɛ} (\textit{na} with its vowel assimilated). Then, in \REF{ex:syn:36}, the \isi{common argument} of the \isi{main clause} is \textit{ima} ‘child’, modified by the \isi{relative clause} \textit{náa ɲcia tákí} ‘that I mentioned’. Since the verb of the \isi{relative clause} is transitive (\textit{tákés} ‘to mean, mention’), it requires an object, which in this case is fulfilled by the \isi{relative pronoun} \textit{náa} representing the noun \textit{ima}:\\




Intransitive (\textsc{sv})

\ea\label{ex:syn:35}
\gll Nesíbimaa     emuta\textsc{\textsubscript{ca}}=[nɛ\textsc{\textsubscript{s}}   \'{ɛ}f\textsc{\textsubscript{v}}{]}\textsc{\textsubscript{rel}}. \\
hear:\textsc{1pl.exc:prf}  story:\textsc{nom}=\textsc{rel}   sweet:\textsc{3sg}    \\
\glt ‘We’ve heard a story that is funny.’ 
\z


Transitive (\textsc{oav})
\ea\label{ex:syn:36}
\gll Atsáá       ima\textsc{\textsubscript{ca}}=[náa\textsc{\textsubscript{o}}   ɲcia\textsc{\textsubscript{a}}   tákí\textsc{\textsubscript{v}}{]}\textsc{\textsubscript{rel}}. \\
come:\textsc{3sg}:\textsc{prf}   child=\textsc{rel}   I:\textsc{acc}   mention:\textsc{1sg}    \\
\glt ‘The child I mentioned earlier has come.’ 
\z




\subsubsection{Adverbial clauses}\label{sec:10.3.3}

The category of \textsc{adverbial clauses} is rather broad as it includes any \isi{subordinate clause} that modifies a \isi{main clause} adverbially. Adverbial clause are subordinate or ‘dependent’ precisely because they cannot stand alone but must be linked to an independent \isi{main clause}. As subordinate clauses, adverbial clauses exhibit a \isi{constituent order} that differs from both main clauses and relative clauses. Specifically, \isi{intransitive} adverbial clauses have the order \textsc{sv}, while transitive adverbial clauses have the order \textsc{avo}. Another correlate of subordination seen in most adverbial clauses – except for the conditional and hypothetical ones – is accusative case-marking on all core constituents (\textsc{s/a/o}) if they are explicitly mentioned. 

Among the main kinds of \isi{adverbial clause} in Ik are the following: \textsc{temporal}, \textsc{simultaneous}, \textsc{conditional}, \textsc{hypothetical}, \textsc{manner}, \textsc{reason}/\textsc{cause}, and \textsc{concessive}. Most types of \isi{adverbial clause} – except for \textsc{manner} – have their own dedicated connective (or ‘conjunction’) or set of connectives, many of which are listed back in \tabref{tab:morph:subordconn} under \sectref{sec:3.14}. Without exception, the subordinating connectives come first in the \isi{adverbial clause}. Lastly, in terms of position, Ik adverbial clauses may come before or after the \isi{main clause} they modify. Each of these types of \isi{adverbial clause} is given one example apiece in \REF{ex:syn:37}-\REF{ex:syn:43}:\\



Temporal
\ea\label{ex:syn:37}
\gll [Noo   ntsíá     baduƙotâdᵉ]\textsc{\textsubscript{temp}},   ƙ\'{ɔ}ɗɨakᵒ. \\
when   he:\textsc{3sg}   die:\textsc{3sg:dp}     cry:\textsc{1sg:seq}    \\
\glt ‘When he died, I cried.’ 
\z



Simultaneous
\ea\label{ex:syn:38}
\gll [Náa   ntsíá     badúƙótìkᵉ]\textsc{\textsubscript{simul}},   ƙ\'{ɔ}ɗ\'{ɛ}sɨakᵒ. \\
as   he:\textsc{3sg}   die:\textsc{3sg:sim}    cry:\textsc{ipfv}:\textsc{1sg:seq}    \\
\glt ‘As he was dying, I was crying.’ 
\z



Conditional
\ea\label{ex:syn:39}
\gll [Na   ntsa     badúƙótùkᵒ]\textsc{\textsubscript{cond}},   ƙ\'{ɔ}ɗɨakᵒ. \\
if   he:\textsc{nom}   die:\textsc{3sg:seq}     cry:\textsc{1sg:seq}    \\
\glt ‘If he dies, I’ll cry.’ 
\z



Hypothetical
\ea\label{ex:syn:40}
  \ea
  \gll [Na   ƙánoo   ntsa    badúƙótùkᵒ]\textsc{\textsubscript{hypo}},  \\
  if   would’ve   he:\textsc{3sg}  die:\textsc{3sg:seq}    \\ 
  \glt ‘If he would’ve died,
  \ex
  \gll ƙ\'{ɔ}ɗɨaa   ƙánòkᵒ. \\
  cry:\textsc{1sg:seq}  would’ve    \\
  \glt  I would’ve cried.’ 
  \z
\z



Manner
\ea\label{ex:syn:41}
\gll Badúƙótuo   [(ntsíá)   tisílíkᵉ]\textsc{\textsubscript{manner}}. \\
die:\textsc{3sg:seq}   (he:\textsc{acc})  peaceful:3\textsc{sg:sim}    \\
\glt ‘And he died peacefully (lit. ‘he being peaceful’).’ 
\z



Reason/cause
\ea\label{ex:syn:42}
\gll Baduƙotáá   [ɗúó     ídzanâdᵉ]\textsc{\textsubscript{reason}}. \\
die:\textsc{3sg:prf}   because   shoot:\textsc{ips:3sg:dp}    \\
\glt ‘He has died because he was shot.’ 
\z



Concessive
\ea\label{ex:syn:43}
\gll [Áta   ntsíá     badúƙótìkᵉ]\textsc{\textsubscript{concess}},   ńtá   ƙ\'{ɔ}ɗ{\Í}. \\
even   he:\textsc{acc}   die:\textsc{3sg:sim}    not   cry:\textsc{1sg}    \\
\glt ‘Even if he dies, I will not cry.’ 
\z






\subsection{Questions}\label{sec:10.4}
\subsubsection{Overview}\label{sec:10.4.1}

Questions in Ik can be formed in two mutually exclusive ways: 1) by leaving the final word in the question in its non-final form (along with a questioning \isi{intonation}) or 2) by using \isi{interrogative} pronouns and often rearranging the syntax of the sentence. The first method is employed with what is called \textsc{polar} or yes/no questions: those whose answer is either ‘yes’ or ‘no’. The second method is used for \textsc{content} or wh-questions: those whose answer is a substantive response to such \isi{interrogative} pronouns as \textit{who?}, \textit{what?}, \textit{when?}, \textit{where?}, etc. These two types of question are briefly described in the following two subsections.


\subsubsection{Polar questions}\label{sec:10.4.2}

Polar questions are those that elicit a ‘yes’ or ‘no’ in response. In Ik, they are formed by leaving the last word or \isi{particle} of the question in its non-final form (revisit \sectref{sec:2.3} and \sectref{sec:2.4.3} for a review). This open-endedness of form is a fascinating way the grammar reflects the open-endedness of a question – open to a response. Besides the non-final form of the last word, polar questions are identified by a change in \isi{intonation}. This \isi{interrogative} \isi{intonation} is enacted by what is called a \textsc{boundary} low tone: a low tone that attaches to the final \isi{syllable}. If the final \isi{syllable} already has a low tone, then the \isi{boundary tone} is not audible. But if the final \isi{syllable} has a high tone, the \isi{boundary tone} manifests as a high-low glide. 

Examples \REF{ex:syn:44}-\REF{ex:syn:45} illustrate these features of polar questions. Note in the first part of \REF{ex:syn:44} how the present perfect suffix \{-\'{}ka\} shows up in its non-final form (\textit{{}-\'{}à}), while in the second part, the final form is used (\textit{{}-\'{}k\ᵃ}). Then, \REF{ex:syn:45} shows the \isi{interrogative} boundary low tone attaching to the high tone on the final \isi{syllable} of \textit{cekúó} ‘is a woman’, creating a high-low down-glide (\textit{cekúô}):




\ea\label{ex:syn:44}
  \ea
  \gll Nábʉƙɔtáà?\\
finish:\textsc{comp:3sg:prf}[\textsc{nf}]  \\
  \glt ‘Is it finished?’    
  \ex
  \gll Ee, nábʉƙɔtákᵃ. \\
     yes finish:\textsc{comp:3sg:prf[ff]}   \\
  \glt ‘Yes, it is finished.’
  \z
\z





\ea\label{ex:syn:45}
  \ea
  \gll Cekú{ô}? \\
woman:\textsc{cop[nf]}    \\
  \glt ‘Is it a woman?’    
  \ex
  \gll Ee, cekúó     ntsaᵃ. \\
     yes woman:\textsc{cop}   she:\textsc{nom}    \\
  \glt ‘Yes, it’s a woman.’
  \z
\z





\subsubsection{Content questions}\label{sec:10.4.3}

In contrast to polar questions, content questions cannot logically take ‘yes’ or ‘no’ for an answer. Rather, answers to content questions – as their name implies – must contain content relevant to the specific \isi{interrogative} pronoun used to make the inquiry (Ik \isi{interrogative} pronouns are listed in \tabref{tab:pro:inter}). So if the question contains the pronoun \textit{ǹdò-} ‘who?’, the answer must include a person. Or if the question contains the pronoun \textit{ndaí-} ‘where?’, the response must refer to a specific location, and so on. Ik forms content questions by placing an \isi{interrogative} pronoun in the syntactic slot of the unknown entity being queried (i.e. a person, place, time, manner, etc.). For example, in \REF{ex:syn:46}, the \isi{interrogative} pronoun \textit{ndaí-} ‘where?’ is filling the normal place where an object encoding the destination of \textit{ƙà-} ‘go’ would go. A similar thing occurs in \REF{ex:syn:47}, where the pronoun \textit{ìsì-} ‘what?’ fills the direct object slot required by the verb \textit{b\'{ɛ}ɗ-} ‘want’:




\ea\label{ex:syn:46}
\gll Ƙeesída     ndaíkᵉ? \\
go:\textsc{int:2sg:real}   where:\textsc{dat}    \\
\glt ‘You are going where? 
\z




\ea\label{ex:syn:47}
\gll B\'{ɛ}ɗá       ìsìkᵃ? \\
want:\textsc{3sg:real}   what:\textsc{nom}    \\
\glt ‘He wants what?’ 
\z


However, what is more common is for the \isi{interrogative} pronoun to be fronted for emphasis. As in other instances of fronting in Ik (see \sectref{sec:10.2.7}), the fronted element takes the \isi{copulative case} marker \{-ko\}. In \REF{ex:syn:48}-\REF{ex:syn:49}, examples \REF{ex:syn:46}-\REF{ex:syn:47} are repeated in their fronted (focused) forms, and two other \isi{interrogative} pronouns are used in \REF{ex:syn:50}-\REF{ex:syn:51} to illustrate content questions:



\ea\label{ex:syn:48}
\gll Ndaíó   ƙeesídàdᵉ? \\
where:\textsc{cop}   go:\textsc{2sg:real:dp}    \\
\glt ‘Where are you going?’ 
\z




\ea\label{ex:syn:49}
\gll Isio     b\'{ɛ}ɗᵃ? \\
what:\textsc{cop}   want:\textsc{3sg:real}    \\
\glt ‘What does he want?’ 
\z




\ea\label{ex:syn:50}
\gll Ndoo     óá       \'{ɲ}cìkᵃ? \\
who:\textsc{cop}   call:\textsc{3sg:real}   I:\textsc{acc}    \\
\glt ‘Who calls me?’ 
\z




\ea\label{ex:syn:51}
\gll {\'{N}t\'{ɛ}\'{ɛ}n\'{ɔ}\'{ɔ}   tákîdᵃ?} \\
which:\textsc{cop}   mean:\textsc{2sg:real}    \\
\glt ‘Which (one) do you mean?’ 
\z






\subsection{Quotations}\label{sec:10.5}


Quotations involve reporting someone’s speech (or thought) – the speaker’s own or someone else’s – directly or indirectly. Ik fulfills this communicative need through the use of the verb \textit{k\`{ʉ}t-} ‘say’ followed by the actual quotation treated as an add-on clause. That is, unlike complements described below in \sectref{sec:10.6}, a quoted sentence in Ik is technically \textit{not} an object of the verb \textit{k\`{ʉ}t-}. Instead, it is tacked on ‘extra-syntactically’ and given the \isi{oblique case} (the ‘leftover’ case). This is proven by the fact that when the pronoun \textit{ìsì-} ‘what?’ appears to be the object of \textit{k\`{ʉ}t-} with a \textsc{3sg} or \textsc{3pl} subject, \textit{ìsì-} takes the \isi{oblique case} instead of the \isi{accusative case} as one would expect otherwise from case grammar (\sectref{sec:7.3}).

Many languages, English included, distinguish between direct and indirect \isi{quotative} formulas, for example the direct “I said, ‘I will come’” versus the indirect “I said I will come”. By contrast, Ik does not distinguish the two grammatically. Instead, the proper sense has to be discerned from the context (and possibly from \isi{intonation}). So the statement \textit{Kʉt{\Í}á naa atsésí} could mean either “I said, ‘I will come’” or “I said I will come”, depending on factors other than syntax. 

In Ik \isi{quotative} sentences, if there is an addressee of the quotation, they will appear in the \isi{dative} case. And the \isi{quotative} \isi{particle} \textit{tàà} ‘that’ is often inserted just before the quotation, though by all appearances it is optional. The example sentences \REF{ex:syn:52}-\REF{ex:syn:53} provide a demonstration of the \isi{quotative} construction:



\ea\label{ex:syn:52}
\gll Kʉt{\Í}á     bie   [Pakóícéo=noo   dzígwì]\textsc{\textsubscript{quotation}} \\
say:\textsc{1sg}   you:\textsc{dat} Turkana:\textsc{cop=pst4} buy:\textsc{plur}    \\
\glt ‘I’m telling you it was the Turkana who used to buy.’ 
\z




\ea\label{ex:syn:53}
\gll Kʉtana ŋgóé  taa   [atsúó   ɗ\`{ɛ}m\`{ʉ}s]\textsc{\textsubscript{quotation}}  \\
say:\textsc{ips} we.\textsc{exc}:\textsc{dat}  that   come:\textsc{imp}   quickly    \\
\glt ‘They are saying to us, ‘Come quickly!’.” 
\z






\subsection{Complements}\label{sec:10.6}


\textsc{Complements} are individual clauses that function as an ‘argument’\textsc{} of the verb – as either subject or object. In other words, they are clauses within clauses. Unlike subordinate clauses which are added \textit{onto} main clauses, \isi{complement} clauses are added \textit{into} other clauses. The main type of Ik \isi{complement} clause is introduced by the \textsc{complementizer} \textit{tòìm\`{ɛ}nà-} ‘that’, which is combination of a form of the verb \textit{tód-} ‘speak’ and the noun \textit{mɛná-} ‘issues, words’. This compound word gives some evidence that Ik \isi{complement} clauses (of this particular type) evolved from \isi{quotative} clauses like those described above in \sectref{sec:10.5}.

Because a \isi{complement} clause fits within the clausal grammar, it must somehow be declined for case (because all arguments of a verb in Ik take case, without exception). To meet this requirement, the \isi{complementizer} \textit{tòìm\`{ɛ}nà-} bears the burden of case on behalf of the whole \isi{complement} clause it is introducing. So technically, it is the \isi{complementizer} – not the \isi{complement} clause alone – that is the verbal argument. But because \textit{tòìm\`{ɛ}nà-} plus the \isi{complement} is a frozen \isi{quotative} formula, the whole construction can be analyzed as an argument.

To illustrate this, \REF{ex:syn:54} presents a simple \isi{complement} clause governed by the cognitive verb \textit{èn-} ‘see’. The \{curly brackets\} indicate the boundaries of the \isi{main clause} from the point of view of the syntax, in which the verb \textit{èn-} ‘see’ selects its object \textit{tòìm\`{ɛ}nà-} ‘that’ for the \isi{accusative case}. The [square brackets] mark the boundary of the \isi{complement} clause seen from the point of view of semantics, for the actual content of ‘seeing’ is the clause \textit{that we have become very rich}:




\ea\label{ex:syn:54}
\gll \{Enáta  [toimɛnaa\}\textsc{\textsubscript{obj}} barʉƙɔt{\Í}máà   zùkᵘ]\textsc{\textsubscript{compl}} \\
see:\textsc{3pl}   that:\textsc{acc}    rich:\textsc{comp:1pl.exc:prf}   very    \\
\glt ‘They see that we have become very rich.’ 
\z


In addition to a direct object, an Ik \isi{complement} clause can also function as an indirect object or even the ‘\isi{complement}’ of a \isi{copular} clause. For instance, in \REF{ex:syn:55} below, \textit{tòìm\`{ɛ}nà-} and by extension the whole \isi{complement} clause is acting as the indirect object of the verb \textit{x\`{ɛ}ɓ-} ‘be afraid of, fear’, which requires the \isi{ablative case}. Then, in \REF{ex:syn:56}, the verb is the \isi{copular} verb \textit{m{\Ì}t-} ‘be’, which requires its nominal compliment to be in the \isi{oblique case}, as is seen with \textit{tòìm\`{ɛ}nà-}:



\ea\label{ex:syn:55}
\gll Xɛɓ{\Í}á     [toimɛnɔɔ   maíá     sílím]\textsc{\textsubscript{compl}} \\
fear:\textsc{1sg}   that:\textsc{abl}   ill:\textsc{1sg}   AIDS:\textsc{nom}    \\
\glt ‘I am afraid that I’m ill with AIDS.’ 
\z




\ea\label{ex:syn:56}
\gll Mɨta ʝa   [toimɛna   ńtá   nesíbi       mɛnákᵃ]\textsc{\textsubscript{compl}} \\
be:\textsc{3sg} just   that[\textsc{obl}]   not   hear:\textsc{3sg} words:\textsc{acc}    \\
\glt ‘It is just that she doesn’t understand instructions.’ 
\z






\subsection{Comparatives}\label{sec:10.7}


\textsc{Comparatives} are grammatical constructions that allow the comparison of two entities on the basis of some shared characteristic. Ik has two strategies for doing this: 1) the mono-clausal, which involves one simple clause, and 2) the bi-clausal, which involves a complex clause. Mono-clausal comparatives place the \textsc{comparee} (entity being compared) in the \isi{nominative case} and the \textsc{standard} (entity the comparee is being compared to) in the \isi{ablative case}. Since most comparable attributes are expressed as \isi{intransitive} verbs in Ik, the \textsc{parameter} (attribute) of the comparison is also an \isi{adjectival} verb in such constructions. For example, in \REF{ex:syn:57}-\REF{ex:syn:58} below, the \isi{intransitive} verbs \textit{zè-} ‘big’ and \textit{dà-} ‘nice’ are acting as the parameters, while their subjects are the comparees in the \isi{nominative case} and their extended objects the standards in the \isi{ablative case}:




\ea\label{ex:syn:57}
\gll Zeíá     \'{ŋ}kà     bù. \\
big:\textsc{1sg}   I:\textsc{nom}   you:\textsc{abl}    \\
\glt ‘I am bigger than you.’ 
\z




\ea\label{ex:syn:58}
\gll Daa     ɗa=na       kɨɗ\'{ɔ}\'{ɔ} \\
nice:\textsc{3sg}   this.one:\textsc{nom}=this   that.one:\textsc{abl}    \\
\glt ‘This one is nicer than that one.’ 
\z


Bi-clausal comparatives, on the other hand, combine a \isi{main clause} with a subordinate or ‘co-subordinate’ clause (\sectref{sec:10.8.2}). Both types are introduced by the verb \textit{ɨl\'{ɔ}-} ‘exceed, surpass’, which acts as the \textsc{index} of the comparison (the gauge of the degree of difference between compared entities). If the indexical verb introduces a \isi{subordinate clause}, it takes the \isi{simultaneous aspect}, while if it introduces a co-\isi{subordinate clause}, it takes the \isi{sequential aspect}. In such bi-clausal comparatives, the comparee is still the subject of the \isi{main clause}, while the standard is the object of the dependent clause. The parameter remains with the \isi{main clause} verb (as in mono-clausal comparatives). But unlike mono-clausals, bi-clausal comparatives can have \isi{intransitive} or transitive parametric verbs. In other words, actions as well as attributes can be compared in this type of construction.

In \REF{ex:syn:59}, the parameter lies with the verb \textit{tɔk\'{ɔ}b-} ‘cultivate’, and ‘he’ (marked as 3\textsc{sg} on the verb) is being compared with ‘us’ (\textit{ŋgó-}). The index of the comparison is the verb \textit{ɨl\'{ɔ}{\Í}ɛ} ‘he surpassing’, which reveals the inequality of the compared actions of the two entities. Example \REF{ex:syn:60} follows the exact same logic, only that the indexical verb \textit{ɨl\'{ɔ}ɨnɨ} is in the \isi{sequential aspect} instead of the simultaneous: 



\ea\label{ex:syn:59}
\gll Tokóbia     eɗíá        [ɪl\'{ɔ}{\Í}ɛ     ŋgókᵃ]\textsc{\textsubscript{sim}} \\
cultivate:\textsc{plur:3sg} grain:\textsc{acc} surpass:\textsc{3sg:sim} we:\textsc{acc}    \\
\glt ‘He cultivates grain more than us.’ 
\z




\ea\label{ex:syn:60}
\gll Sáɓúmósáta     [ɨl\'{ɔ}ɨnɨ          toni  ɲeryaŋ]\textsc{\textsubscript{seq}} \\
kill:\textsc{recip:3pl} exceed:\textsc{3pl:seq} even government[\textsc{obl}]    \\
\glt ‘They’re killing each other even more than the government.’ 
\z






\subsection{Clause combining}\label{sec:10.8}
\subsubsection{Clause coordination}\label{sec:10.8.1}

Two or more clauses can be linked in Ik through clause \textsc{coordination}. This can result in clause \textsc{addition} (‘and’), which joins two independent clauses of equal status. It can result in \textsc{contrast} (‘but’), which joins clauses of equal syntactic status, the second of which is a counterexpectation to the first. And thirdly, clause coordination can result in \textsc{disjunction} (‘or’), in which two clauses of equal status are presented as different possible options.

Clause addition is achieved in two ways: 1) simply adjoining the clauses with a pause in between (represented by a period or comma in writing) or 2) linking the clauses with a coordinating connective like \textit{kòtò} ‘and, but, then’ or \textit{ńdà} ‘and’. These first two methods are illustrated in \REF{ex:syn:61}-\REF{ex:syn:62}. A third way to add one clause to another is to nominalize it – change all its main parts to nouns, put them in a \isi{noun phrase}, and link it up to the other clause with \textit{ńdà}. Note from \REF{ex:syn:63} that with this third method, because the word \textit{ńdà} ‘and’ is acting as a sort \isi{preposition}, it requires its head noun(s) to be in the \isi{oblique case}. Its head nouns in \REF{ex:syn:63} are the subject (\textit{ŋgo}) and \isi{infinitive} (\textit{ŋƙ\'{ɛ}s{\Í}}) – both in the \isi{oblique case}:




\ea\label{ex:syn:61}
\gll M{\Í}n{\Í}a     ɲécáyᵃ.   M{\Í}ná       ntsa   m\'{ɛ}s\`{ɛ}kᵃ. \\
love:\textsc{1sg}   tea:\textsc{nom}   love:\textsc{3sg} she:\textsc{nom}   beer:\textsc{acc}    \\
\glt ‘I love tea. She loves beer.’ 
\z




\ea\label{ex:syn:62}
  \ea
  \gll Ƙaƙiésána=noo       ńtí, \\
hunt:\textsc{plur:ipfv:ips:real=pst}   how \\
  \glt ‘How did people used to go hunting,
  \medskip
  \ex
  \gll ńda   ƙaíána=noo         waa   waicíkée     ńtí?  \\
     and   go:\textsc{plur:ips:real=pst} pick:\textsc{nom} greens:\textsc{gen} how  \\
  \glt and how did they used to go picking greens?’
  \z
\z




\ea\label{ex:syn:63}
  \ea
  \gll Itétimaa awákᵉ, \\
return:\textsc{1sg:seq} home:\textsc{dat}     \\
  \glt ‘We returned home, 
  \medskip
  \ex
  \gll ńda  ŋgo     ŋƙ\'{ɛ}s{\Í}     tɔbɔŋ\'{ɔ}ᵉ. \\
and   we:\textsc{obl}   to.eat:\textsc{obl}   mush:\textsc{gen}    \\
  \glt and we ate mealmush.’
  \z
\z

Contrast between two clauses in Ik can be expressed in two primary ways: 1) by simply adjoining the two clauses with a brief pause in between (marked with by a comma or period in writing) or 2) by linking the two clauses with the contrastive connective \textit{kòtò}, which can mean ‘but’ as well as ‘and, then, therefore, etc.’. These two types are demonstrated in examples \REF{ex:syn:64}-\REF{ex:syn:65}, respectively:



\ea\label{ex:syn:64}
\gll Bɛna     \'{ɲ}cùkᵒ.     Bùkᵒ. \\
not:\textsc{3sg}   I:\textsc{cop}     you:\textsc{cop}    \\
\glt ‘It’s not me. It’s you. 
\z




\ea\label{ex:syn:65}
  \ea
  \gll Bɛɗʉƙɔt{\Í}a=naa     ɲ\'{ɛ}mɛlɛk\'{ʉ}, \\
search:\textsc{comp:1sg=pst1}   hoe:\textsc{nom}    \\
  \glt ‘I went and looked for the hoe, 
  \medskip
  \ex
  \gll koto   máa=naa   ŋunetí. \\
but   not=\textsc{pst1}   find:\textsc{1sg}    \\
  \glt but I did not find (it).’
  \z 
\z

Lastly, the idea of \isi{disjunction} is expressed in Ik through the use of the connectives \textit{kèɗè} ‘or’ or \textit{kòrì} ‘or’, as illustrated in example sentences \REF{ex:syn:66}-\REF{ex:syn:67}:



\ea\label{ex:syn:66}
  \ea
  \gll Tɔk\'{ɔ}bɛs{\Í}da       eɗa,  \\
farm:\textsc{ipfv:2sg}   grain:\textsc{nom}    \\
  \glt ‘Are you farming grain, 
  \medskip
  \ex
  \gll keɗe   ńtá   tɔk\'{ɔ}bɛs{\Î}d\ᶤ? \\
or   not   farm:\textsc{ipfv:2sg}    \\
  \glt or are you not farming (it)?’
  \z  
\z


\ea\label{ex:syn:67}
  \ea
  \gll Enída       mɛna     gaanaakátìkᵉ, \\
see:\textsc{2sg}   things:\textsc{nom}   bad:\textsc{distr:3pl:sim}    \\
  \glt ‘Do you see things being bad all around, 
  \medskip
  \ex
  \gll kori   maráŋaakátìkᵉ? \\
or   good:\textsc{distr:3pl:sim}    \\
  \glt or as being good all around?
  \z  
\z


\subsubsection{Clause chaining}\label{sec:10.8.2}

But in fact, the most common way Ik links independent clauses is through clause ‘co-subordination’ or \textsc{clause chaining}. To create a chain of clauses, the grammar starts with an anchoring phrase or clause to set the stage modally or temporally, and then it puts all the following mainline verbs in the \isi{sequential aspect} (see \sectref{sec:8.10.7}), creating a chain of two or more clauses. When \isi{clause chaining} is used in a story, the temporal ‘anchor’ can be a simple \isi{time expression} like \textit{ka{\Í}n{\Í}kò nùk\ᵘ} ‘in those years’ or a tensed statement like \textit{Atsa noo ámá ntanée taa Apáálɔr\'{ɛ}ŋ} ‘There came a man named Apaaloreng’. In \REF{ex:syn:68}, the clause chain is anchored by the initial adverbial phrase \textit{Na kónít}\textit{ó ódoue baratsoó} ‘One day, in the morning’, which puts the whole sentence in a temporal frame. Thenceforth, the clause chain proceeds clause by clause, each marked as \textsc{seq1}, \textsc{seq2}, etc:




\ea\label{ex:syn:68}
  \ea
  \gll [Na     kónító      ódoue   baratsoó]\textsc{\textsubscript{adv}} \\
when    one    day:\textsc{gen}   morning:\textsc{ins}     \\  
  \glt ‘One day, in the morning,  
  \medskip
  \ex
  \gll [ipu\textbf{{o}}            taƙá{\Í}kakᵃ]  \textsc{\textsubscript{seq1}} \\
cast:\textsc{3sg:seq}   shoes:\textsc{acc}    \\
  \glt he cast (his) shoes (in divination),
  \medskip
  \ex  
  \gll [eɡu\textbf{{o}}           taƙá{\Í}ka         \'{ɛ}bakᵃ]\textsc{\textsubscript{seq2}} \\
put:\textsc{3sg:seq}   shoes\textsc{:nom}     gun:\textsc{acc}    \\
  \glt and the shoes made (the shape of) a gun,
  \medskip
  \ex
  \gll [ipu\textbf{{o}}            naɓó]\textsc{\textsubscript{seq3}} \\
cast:\textsc{3sg:seq}   again    \\
  \glt and he cast (them) again,
  \medskip
  \ex
  \gll [eɡ\textbf{{ini}}      \'{ɛ}bakᵃ]\textsc{\textsubscript{seq4}} \\
put:\textsc{3pl}:\textsc{seq}     gun:\textsc{acc}    \\
  \glt and they made a gun.’
  \z  
\z

Although the \isi{sequential aspect} and clause chains are common in narratives, they are also used extensively for other types of discourse, for example, exposition and instruction. The following expository clause chain in \REF{ex:syn:69} details some of the steps taken in the process of grinding tobacco leaves. Note that there are two anchoring adverbial clauses, one at the beginning and one in the third line. After each one, there is a string of one or more verbs set in the \isi{sequential aspect}:



\ea\label{ex:syn:69}
  \ea
  \gll [Náa   iryámétan{\Í}\'{ɛ}   gwasákᵉ]\textsc{\textsubscript{adv1}} \\
when   get:\textsc{ips:sim}   stone:\textsc{dat}    \\
  \glt ‘When a stone is acquired, 
  \medskip
  \ex
  \gll [ŋɔ\'{ɛ}\textbf{ɛsɛ}     ɲaɓáláŋɨtᵃ]\textsc{\textsubscript{seq1}} \\
grind:\textsc{inch:sps}   soda.ash:\textsc{nom}    \\
  \glt soda ash is ground up.
  \medskip
  \ex
  \gll [náa   ɲaɓáláŋɨt{\Í}á     iwíɗímètìkᵉ]\textsc{\textsubscript{adv2}} \\
when   soda.ash:\textsc{acc}   pulverize:\textsc{mid:sim}    \\
  \glt When the soda ash is ground to powder,
  \medskip
  \ex
  \gll [páka ɲapúɗúmùƙòtù\textbf{k\ᵒ}{]}\textsc{\textsubscript{seq4}} \\
until powdery:\textsc{comp:seq}    \\
  \glt until it becomes fine powder.’
  \z  
\z 

 
 
Finally, the \isi{sequential aspect} and \isi{clause chaining} is often found operating in a set of commands or instructions. Such a clause chain may begin with one or more \isi{imperative} verbs, followed by the sequential verbs in a chain of further commands or instructions. This type of clause chain is shown in \REF{exa}:



\ea\label{exa}
  \ea
  \gll [Na   b\'{ɛ}ɗɨdɔ\textbf{ɔ}     bɛr\'{ɛ}sá   hoe]\textsc{\textsubscript{adv}} \\
if   want:\textsc{2sg:seq}   to.build:\textsc{nom}  house:\textsc{gen}    \\
  \glt ‘If you want to build a house, 
  \medskip
  \ex
  \gll [kawete   titíríkᵃ,   kɛɗɨt{\Í}n,   ńda   sim]\textsc{\textsubscript{imp2}} \\
cut:\textsc{imp}   pole:\textsc{pl}   reed:\textsc{pl}    and   fiber    \\
  \glt Cut poles, reeds, and fiber,
  \medskip
  \ex
  \gll [iréɲuƙoidu\textbf{o}     bác{\Í}kᵃ]\textsc{\textsubscript{seq1}} \\
clear:\textsc{comp:2sg:seq}   area:\textsc{nom}    \\
  \glt clear away the area,
  \medskip
  \ex
  \gll [úgidu\textbf{o}   ripitín]\textsc{\textsubscript{seq2}} \\
dig:\textsc{2sg:seq}   hole:\textsc{pl:nom}    \\
  \glt dig holes,
  \medskip
  \ex
  \gll [otídu\textbf{kó}é     titíríkᵃ]\textsc{\textsubscript{seq3}} \\
pour:\textsc{2sg:seq:dp}   pole:\textsc{pl:nom}    \\
  \glt and put the poles into them.’
  \z  
\z 
\newpage  %needed for example ref to resolve due to redefinition of clearpage 
