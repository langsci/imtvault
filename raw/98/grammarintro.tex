% \addchap{Grammar sketch}
\setcounter{section}{0}
\section{Introduction}\label{sec:1}


Although the bulk of this book is devoted to the dictionary and reversal index, the following section offers an overview sketch of Ik grammar that covers most important features of the total grammatical system. Those who wish to dig deeper are encouraged to consult the fuller treatment published as \textit{A grammar of Ik (Icé-tód): Northeast Uganda’s last thriving Kuliak language} \citep{Schrock2014}, which is available for free downloading from several websites on the internet. 

Linguistic concepts are most easily defined with linguistic terminology. Thus, due to limitations of time and space, this sketch of Ik grammar is geared in style toward the general linguist. And yet a primary aim has been to clearly define some of the key terms used and to describe the grammatical structures in simple, straightforward language. Unfortunately, some of the discussion may still remain opaque to non-linguist readers. If such persons wish to know more, I am very willing to clarify or explain in layman’s terms any point raised in this grammar sketch. Feel free to contact me any time at: \href{mailto:betsoniik@gmail.com}{betsoniik@gmail.com}.

The grammar sketch begins with the language’s sound system (phonology) and then proceeds to words and word-building strategies (morphology). It ends with a shallow dip into syntax. Because of its length and technical nature, the grammar sketch is probably most useful as a reference tool. However, should the reader have the opportunity, it may prove beneficial to read the sketch from front to back in order to gain a bird’s-eye view of the whole system.

Learning any language from printed sources alone is rarely ideal. Rather, every learner would ideally have the chance to soak up language naturally as children do. Sadly, most adult learners do not have that luxury. Because of that, I recommend creatively mixing language-learning approaches to suit one’s personality, learning style, schedule, and responsibilities. Studying grammar from a book like this one will not appeal to everyone, yet all learners will occasionally get stuck on points of grammar during the course of their learning. Just as the foregoing dictionary can help you fill in gaps where specific words need to be, this grammar sketch can help fill in holes in your understanding of how Ik works.