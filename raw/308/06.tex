\chapter{Complex sentences}

\section{Coordination}\label{sec:6.1}
\isi{Coordinating conjunctions} are \textit{a} (before consonant) / \textit{ad} (before vowel) `and' (\ref{ex:coord1}), \textit{api} `and, and then' (\ref{ex:coord1}), \textit{u tga ... né tga} `either ... or' (\ref{ex:coord6}),  \textit{né tga} `or', \textit{pi} `then' (\ref{ex:coord2}), \textit{ábar} `but' (\ref{ex:coord3}), and \textit{dantaun} `however, but' (\ref{ex:coord4}).

\ea
\label{ex:coord1}
\gll    In’ jèda tschò, in’ jèda lò, \textbf{ad} in’ jèda saragòrd’ ju aun … tga nuṣ èran \isi{í} ajn ustria òl Mírar, \textbf{api} sa ju bégja sén tgé nuṣ èran vagní [...].\\
one.\textsc{f.sg} time here one.\textsc{f.sg} time there and one.\textsc{f.sg} time \textsc{refl}.remember.\textsc{prs.1sg} \textsc{prs.1sg} still {}  \textsc{comp} \textsc{1pl} be.\textsc{impf.1pl} go.\textsc{ptcp.m.pl} in restaurant.\textsc{f.sg} out.\textsc{def.m.sg} \textsc{pn} and  know.\textsc{prs.1sg} \textsc{prs.1sg} \textsc{neg} upon what \textsc{1pl} be.\textsc{impf.1pl} come.\textsc{ptcp.m.pl}\\
\glt `Once here, once there, and once I still remember ... that we had gone to Mirer's restaurant, and I don't know what we had come up with [...].' (Zarcúns, m2, \sectref{sec:8.13}
\z

\ea
\label{ex:coord6}
\gll \textbf{U} \textbf{tga} té fas ùrdan ajn tia tgombra \textbf{né} \textbf{tga} té lavas gjù.\\
either \textsc{comp} \textsc{2sg} do.\textsc{prs.2sg} order.\textsc{m.sg} in \textsc{poss.2sg.f.sg} room or \textsc{comp} \textsc{2sg} wash.\textsc{prs.2sg} down\\
\glt`Either you tidy up your room or you do the dishes.' (Sadrún, m5)
\z

\ea
\label{ex:coord2}
\gll    Mintgataun mavan nuṣ èra … plas pitgògnas a cavá cristalas anstagl mirá dlas tgauras, \textbf{pi} vignévan nus halt in téc tart. \\
sometimes go.\textsc{impf.1pl} \textsc{1pl} also {} through.\textsc{def.f.pl} steep\_slope.\textsc{pl} \textsc{subord} dig.\textsc{inf} crystal.\textsc{f.pl} instead look\_for.\textsc{inf} of.\textsc{def.f.pl} goat.\textsc{pl} and come.\textsc{impf.1pl} \textsc{1pl} simply \textsc{indef.m.sg} bit late \\
\glt `From time to time we would also … go farther up to extract crystals instead of looking after the goats, and then we would come back a bit late.' (Surajn, f5, \sectref{sec:8.10})
\z

\ea
\label{ex:coord3}
\gll Ju èr’ in gjuvanò̱tar, sùn maj staus fumègl \textbf{ábar} ins mav’ a gidá òra [...].\\
textsc{1sg} \textsc{cop.impf.1sg} \textsc{indef.m.sg} youngster be.\textsc{prs.1sg} never \textsc{cop.ptcp.m.sg} farmhand but \textsc{gnr} go.\textsc{impf.3sg} \textsc{comp} help.\textsc{inf} out \\
\glt `I was a youngster, I never was a farmhand but we would go and help out [...]' (Ruèras, m3, \sectref{sec:8.16})
\z

\ea
\label{ex:coord4}
\gll Basta, da fá fùssi ùssa gl atún la pjal plajna da tùtas uísas. \textbf{Dantaun} scha l' aura tégn ansjaman vòi schòn.\\
enough \textsc{comp} do.\textsc{inf} be.\textsc{cond.3sg.expl} now \textsc{def.m.sg} autumn \textsc{def.f.sg} skin full of all.\textsc{f.pl} way.\textsc{pl} however if \textsc{def.f.sg} weather hold.\textsc{prs.3sg} together go.\textsc{prs.3sg.expl} indeed\\
\glt `Well, now during autumn there would be a lot of things to do. However, if the weather holds, it will do indeed.' (Ruèras, \DRGoK{9}{452})
\z

In the case of coordination with \textit{né} `or', the complementiser \textit{tga} must be used with clauses as in (\ref{ex:coord5}).

\ea
\label{ex:coord5}
\gll A né grad è gl unviarn cunzún mavanṣ bjè a bagagjavan sprungṣ a dèvan cun skis né cun ajssa \textbf{né} \textbf{tg}’ ins mava cun bòb da vias gjù [...].\\
and right precisely also \textsc{def.m.sg} winter especially go.\textsc{impf.1pl.1pl} often and build.\textsc{impf.1pl} jump.\textsc{m.pl} and give.\textsc{impf.1pl} with ski.\textsc{m.pl} or with board.\textsc{f.sg} right \textsc{comp} \textsc{gnr} go.\textsc{impf.3sg} with bob.\textsc{m.sg} from street.\textsc{f.pl} down\\
\glt `And, right?, especially during winter we often went and built ski jumps and would go skiing or snowboarding, or we would go down the streets on bobsleigh [...].' (Sadrún, m9, \sectref{sec:8.15})
\z


\section{Subordination}\label{sec:6.2}
The general \isi{subordinating conjunction} for \isi{finite subordinate clauses} is \textit{tga} (realised as \textit{tgu} in combination with the \isi{first person singular pronoun} \textit{ju}). It is used with\isi{relative clauses}, with \isi{argument clauses}, and some \isi{adjunct clauses}; in the latter case, compound conjunctions may also be used, like e.g.  \textit{durònt tga}, where the first part, \textit{durònt} `during', refers to the semantics of the subordinate clause and the second part, \textit{tga} `that' indicates \isi{subordination}.

With non-finite subordinate clauses, it is usually \textit{da} which is used.

\subsection{Argument clauses}\label{sec:6.2.1}

\subsubsection{Subject clauses}\label{sec:6.2.1.1}
Subject clauses are either finite or non-finite. If the \isi{non-finite subject clause} is located at the beginning of the sentence, the \isi{infinitive} may be modified by the definite masculine \isi{article} (\ref{ex:subjclause1} and \ref{ex:subjclause2}) or may lack it (\ref{ex:subjclause3} and \ref{ex:subjclause4}), without any functional change.

\ea
\label{ex:subjclause1}
	\gll    Òh \textbf{gl} \textbf{ampréndar} \textbf{tudèstg} è stau, l’ antschata ṣè quaj schòn stau in téc curjù̱s.\\
oh \textsc{def.m.sg} learn.\textsc{inf} German.\textsc{m.sg} be.\textsc{prs.3sg}  \textsc{cop.ptcp.unm} \textsc{def.f.sg} beginning be.\textsc{prs.3sg} \textsc{dem.unm} indeed \textsc{cop.ptcp.unm} \textsc{indef.m.sg} bit strange.\textsc{adj.unm}\\
\glt `Oh, to learn German was, at the beginning this was indeed a little bit strange.' (Zarcúns, m2, \sectref{sec:8.13})
\z

\ea
\label{ex:subjclause2}
\gll [...] ábar \textbf{al} \textbf{trá} \textbf{ajn} \textbf{èla} èra … da maz.   \\
{} but \textsc{def.m.sg} pull.\textsc{inf} in \textsc{3sg.f} \textsc{cop.impf.3sg} {} of killing.\textsc{m.sg}\\
\glt `[...] but bringing it in was ... terrible.' (Ruèras, m3, \sectref{sec:8.16})
\z

\ea
\label{ex:subjclause3}
\gll \textbf{Fugí} è bian, a \textbf{mitschè} è aun mé̱gljar. \\
   flee.\textsc{inf} \textsc{cop.prs.3sg} good and escape.\textsc{inf} \textsc{cop.prs.3sg} still better  \\
\glt `To flee is good, and to escape is even better.' (\DRGoK{6}{671})
\z

\ea
\label{ex:subjclause4}
\gll    A \textbf{dumagnè} als amprèmṣ dis quèls pòrs gjù da da quaj trùtg ajnagjù a sjantar atrás l’ aua dal Drun tga vagnéva mù pauc vi da tschèla vart, \textbf{qu}’ è stau álṣò in martéri.\\
and cope.\textsc{inf} \textsc{def.m.sg} first.\textsc{pl}  day.\textsc{pl} \textsc{dem.m.pl}  pig.\textsc{pl} down from from \textsc{dem.m.sg} path into\_and\_down and after through \textsc{def.f.sg} water of.\textsc{def.m.sg}  \textsc{pn} \textsc{rel} come.\textsc{impf.3sg} only little over of \textsc{dem.f.sg} side \textsc{dem.unm.} be.\textsc{prs.3sg} \textsc{cop.ptcp.unm} well \textsc{indef.m.sg} ordeal \\
\glt `And the first days, to cope with these pigs [going] down this path and then through the water of the Drun, of which only few would go over to the other side, well, this was an ordeal [...].' (Sadrún, m6, l\sectref{sec:8.11})
\z

It is also possible to use the complementiser \textit{da} (\ref{ex:subjclause7}), but in this case the \isi{infinitive} is highlighted.

\ea
\label{ex:subjclause7}
\gll \textbf{Da} \textbf{còschar} fùs stau mé̱gljar.\\
\textsc{comp} keep\_silent.\textsc{inf} be.\textsc{cond.3sg} \textsc{cop.ptcp.unm} better \\
\glt `To keep silent would have been better.' (Ruèras, m10)
\z

If the \isi{subject clause} is located after the verb, the \isi{expletive pronoun} \textit{i} is required in subject position and the \isi{infinitive} may be introduced by the complementiser \textit{da} (\ref{ex:subjclause5}) or may lack it (\ref{ex:subjclause6}).

\ea
\label{ex:subjclause5}
\gll I è stau mé̱gljar \textbf{da} \textbf{còschar}.\\
\textsc{expl} be.\textsc{prs.3sg} \textsc{cop.ptcp.unm} better \textsc{comp} keep\_silent.\textsc{inf}\\
\glt `It was better to keep silent.' (Sadrún, m5)
\z

\ea
\label{ex:subjclause6}
\gll [...] \textbf{i} è gréjv \longrule {} \textbf{dí} [...].\\
 {} \textsc{expl} \textsc{cop.prs.3sg} difficult.\textsc{adj.unm} {} say.\textsc{inf}\\
\glt `[...] it is difficult to say [...].' (Sadrún, m9, \sectref{sec:8.15})
\z

\ea
\label{ex:subjclause9}
\gll    A sjantar c’ ins mava, sch’ mav’ ins sél Albṣú cul trèn, a \textbf{quaj} custav’ in franc ... \textbf{dad} \textbf{í} \textbf{sédòra}.\\
and after \textsc{subord} \textsc{gnr}  go.\textsc{impf.3sg} \textsc{corr} go.\textsc{impf.3sg} \textsc{gnr} on.\textsc{def.m.sg} \textsc{pn} with.\textsc{def.m.sg} train and \textsc{dem.unm} cost.\textsc{impf.3sg} one.\textsc{m.sg}  franc {} \textsc{comp} go.\textsc{inf} up\\
\glt `And after this, if one went, one would go up to the Alpsu [pass] by train and this cost one franc ... to go up there.' (m1, Ruèras, \sectref{sec:8.2})
\z


\isi{Finite subject clauses} are introduced by the complementiser \textit{tga}; the \isi{expletive} \isi{pronoun} \textit{i} must occur in subject position (\ref{ex:subjclause8} and \ref{ex:subjclaus10}).

\ea
\label{ex:subjclause8}
\gll    Ad \textbf{i} è aun schabagjau plé … \textbf{tga} nuṣ èssan i … tiar duas, álṣò ajn duaṣ jèdas … tiar ina api sjantar aun í ad í vi tiar tschèla.\\
and \textsc{expl} be.\textsc{prs.3sg} in\_addition happen.\textsc{ptcp.unm} more {} \textsc{comp}  \textsc{1pl} be.\textsc{prs.1pl} go.\textsc{ptcp.m.pl} {} to two.\textsc{f.pl} well in two.\textsc{f} time.\textsc{pl} {} to one.\textsc{f.sg} and afterwars in\_addition go.\textsc{ptcp.m.pl} and  go.\textsc{ptcp.m.pl} over to \textsc{dem.f.sg}\\
\glt `And it also happened more … that we went … to two [girls], well at two different moments … to the one and went also, and went over to the other.' (Zarcúns, m2, \sectref{sec:8.13})
\z

\ea
\label{ex:subjclaus10}
\gll Ad \textbf{i} èr’ è bitga úsit \textbf{tg}’ ins \textbf{mava} \textbf{a} \textbf{scùlèta}.\\
and  \textsc{expl} \textsc{cop.impf.3sg} also \textsc{neg} usage.\textsc{m.sg}  \textsc{comp} \textsc{gnr} go.\textsc{impf.3sg} to nursery\_school.\textsc{f.sg}\\
\glt `And it was not usual that one attended nursery school.' (Sadrún, m4,  \sectref{sec:8.3})
\z 


\subsubsection{Object clauses}\label{sec:6.2.1.2}
Object clauses can be finite or non-finite.\isi{Non-finite object clauses} are introduced by the complementiser \textit{da} (\ref{ex:objcl1} and \ref{ex:objcl2}).

\ea
\label{ex:objcl1}
\gll Al pástar gròn ò \textbf{atmonju} \textbf{da} bétga \textbf{fá} tupadats.\\
\textsc{def.m.sg} herdsman big have.\textsc{prs.3sg} admonish.\textsc{ptcp.unm} \textsc{comp} \textsc{neg} do.\textsc{inf} stupidity.\textsc{f.pl}\\
\glt `The main herdsman warned [people] not to commit stupidities.' (Surajn, \citealt[128]{Büchli1966})
\z

\ea
\label{ex:objcl2}
\gll  [...] api vagi èl \textbf{tartgau} ... \textbf{dad} \textbf{í} \textbf{vi} ajn via ... a tanaj sé èls.  \\
{} and have.\textsc{prs.sbjv.3sg} \textsc{3sg.m} think.\textsc{ptcp.unm} {} \textsc{comp} go.\textsc{inf} over on road.\textsc{f.sg} {} and hold.\textsc{inf} up \textsc{3pl.m}\\
\glt `[...] and he thought... that he would go on the road ... and stop them.' (Ruèras, m10, \sectref{sec:8.7})
\z

\isi{Finite object clauses} are introduced by the complementiser \textit{tga} `that' (\ref{ex:objcl13}) or \textit{scha} `whether' (\ref{ex:objcscha1}). \textit{Tga} is not always adjacent to the main verb; in (\ref{ex:sacutga}) a preposed \isi{temporal clause} belonging to the \isi{object clause} separates the complementiser from the main verb, \textit{savaj} in this case.

\ea
\label{ex:objcl13}
\gll    [...] in’ jèda saragòrd’ ju aun … \textbf{tga} nuṣ èran í ajn ustria òl Mírar [...].\\
one.\textsc{f.sg} time \textsc{refl}.remember.\textsc{prs.1sg} \textsc{prs.1sg} still {} \textsc{comp} \textsc{1pl} be.\textsc{impf.1pl} go.\textsc{ptcp.m.pl} in restaurant.\textsc{f.sg} out.\textsc{def.m.sg} \textsc{pn}\\
\glt `[...] once I still remember ... that we had gone to Mirer's restaurant [...].' (Zarcúns, \sectref{sec:8.13})
\z

\ea
\label{ex:sacutga}
\gll    Ju \textbf{sa} cu [...] ju sùn maridaus \textbf{tga} … èri ajn Camischùlas circa quindiṣch ufauns … tga mavan a scùla da Camischùlas.\\
\textsc{1sg} know.\textsc{prs.1sg} when {}  \textsc{1sg} be.\textsc{prs.1sg}  marry.\textsc{ptcp.m.sg} \textsc{comp} {} \textsc{exist.impf.3sg.expl} in \textsc{pn} about fifteen child.\textsc{m.pl} {} \textsc{rel} go.\textsc{impf.3pl} to school.\textsc{f.sg} of  \textsc{pn}\\
\glt `I know that when when … when I got married … there were in Camischolas about fifteen … children who attended the school of Camischolas.' (Zarcúns, m2, \sectref{sec:8.13})
\z

\ea
\label{ex:objcscha1}
\gll   [...] a … lu vajn nuṣ, va ju dumandau \textbf{sch}’ èl prandèssi mè tòcan … a Ruèras. \\
{} and {} then have.\textsc{prs.1pl} \textsc{1pl} have.\textsc{prs.1sg}  \textsc{1sg} ask.\textsc{ptcp.unm} if \textsc{3sg.m} take.\textsc{cond.indir.3sg} \textsc{1sg} until {} to  \textsc{pn}\\
\glt `[...] and … then we, I asked whether he could take me down to Rueras.' (Ruèras, m10, \sectref{sec:8.7})
\z

\subsubsection{Indirect interrogative clauses}\label{sec:6.2.1.3}
\isi{Indirect interrogative clauses}, which are a subcategory of \isi{object clauses}, are either finite (\ref{ex:indinterr1}--\ref{ex:indinterr8}) or non-finite (\ref{ex:indinterr9}). They do not trigger \isi{subject inversion}, in contrast to direct interrogative sentences (see \sectref{sec:5.2}). An exception occurs in (\ref{ex:indinterr9}), where the \isi{non-finite interrogative clause} is focused on and precedes the verb, which triggers \isi{subject inversion}.

\ea
\label{ex:indinterr9}
\gll [...] \textbf{cù} \textbf{barsá} stù ju bigja \textbf{dí} [...].»\\
{} how roast.\textsc{inf} must.\textsc{prs.1sg} \textsc{1sg} \textsc{neg} say.\textsc{inf}\\
\glt `[...] how to roast [it] I don't have to tell [you] [...].' (Cavòrgja, m7, \sectref{sec:8.17})
\z

\ea
\label{ex:indinterr1}
\gll    [...] a \textbf{damònda} \textbf{cù} i vòndi.\\
{} and ask.\textsc{prs.3sg} how \textsc{expl} go.\textsc{prs.sbjv.3sg}\\
\glt `[…] and he asked how he was.' (Sadrún, \citealt[105]{Büchli1966})
\z

\ea
\label{ex:indinterr2}
\gll   Ad èr' è zatgé bi da \textbf{mirá} \textbf{cù} quèls tiars luvravan, cù quèls … mavan ad èran ruassajvalṣ a ... pazjènts.\\
and \textsc{cop.impf.3sg} also something beautiful.\textsc{adj.unm} \textsc{mod} look.\textsc{inf} how \textsc{dem.m.pl} animal.\textsc{pl} work.\textsc{impf.3pl} how \textsc{dem.m.pl} {} go.\textsc{impf.3sg} and \textsc{cop.impf.3pl} calm.\textsc{m.pl} {} and patient.\textsc{m.pl}  \\
\glt `And it was also something nice to look at, how these animals worked, how they … used to go and keep calm and patient.' (Ruèras, m10, \sectref{sec:8.7})
\z

\ea
\label{ex:indinterr3}
\gll    Las nòtízjas \textbf{sa} ju bétg \textbf{danù̱ndar} als gjaniturs, als duṣ baps prandèvan aj, i dèva ajnta Ruèras, dèv’ aj in ca vèva rádjò. \\
\textsc{def.f.pl} news.\textsc{pl} know.\textsc{prs.1sg} \textsc{1sg} \textsc{neg} from\_where \textsc{def.m.pl} parents.\textsc{pl} \textsc{def.m.pl} two.\textsc{m.pl} father.\textsc{pl} take.\textsc{impf.3pl} \textsc{3sg} \textsc{expl} \textsc{exist.impf.3sg} in \textsc{pn} \textsc{exist.impf.3sg} \textsc{expl}  one.\textsc{m.sg} \textsc{rel} have.\textsc{impf.3sg} radio.\textsc{m.sg}\\
\glt `I don’t know where my parents had the news from, the two fathers took them, there was in Rueras, there was [only] one who had a radio.' (Ruèras, m1, \sectref{sec:8.2})
\z

\ea
\label{ex:indinterr4}
\gll [...] `l \textsc{pn} ò è fatg lò in pèr placats tga \textbf{mùssan} ajn via \textbf{nùc}’ ins sa è mirá quaj.\\
{} \textsc{def.m.sg} \textsc{pn} have.\textsc{prs.3sg} also make.\textsc{ptcp.unm} there \textsc{indef.m.sg} pair poster.\textsc{m.pl} \textsc{rel} show.\textsc{prs.3pl} in way where \textsc{gnr} can.\textsc{prs.3sg} also see.\textsc{inf} \textsc{dem.unm} \\
\glt `[...] \textsc{pn} also put there some posters which show where on the way one can have a look at this.' (Sadrún, m4, \sectref{sec:8.3})
\z

\ea
\label{ex:indinterr5}
\gll A lu vajn nus, quaj èra tùt fatg a racògnòszau avaun tg’ ins \textbf{savèva} \textbf{núa} inṣ vèva da durmí, \textbf{nu} i èra … da métar ah ṣur nòtg als als méls, \textbf{nu} i dèva pával pl̩s méls [...]. \\
and then have.\textsc{prs.1pl} \textsc{1pl} \textsc{dem.unm} \textsc{pass.impf.3sg} all do.\textsc{ptcp.unm} and  reconnoitre.\textsc{ptcp.unm} before \textsc{subord} \textsc{gnr}  know.\textsc{impf.3sg} where \textsc{gnr} have.\textsc{impf.3sg} to sleep.\textsc{inf} where \textsc{expl} be.\textsc{impf.3sg} {} \textsc{comp} put.\textsc{inf} ah over night.\textsc{f.sg} \textsc{def.m.pl} \textsc{def.m.pl} mule.\textsc{pl} where \textsc{expl} \textsc{exist.impf.3sg} food.\textsc{m.sg} for.\textsc{def.m.pl} mule.\textsc{pl}\\
\glt `And then we have, this had all been done and reconnoitred before, so that one knew where to sleep, where to put the mules over night, where there was food for the mules [...].' (Ruèras, m10, \sectref{sec:8.7})
\z

\ea\label{ex:indinterr6}
\gll    El ò \textbf{dumandau} èlas \textbf{partgéj} èlas sé̱gian bétg idas a mèssa.\\
\textsc{3sg.m} have.\textsc{prs.3sg} ask.\textsc{ptcp.unm} \textsc{3pl.f} why  \textsc{3pl.f} be.\textsc{prs.sbjv.3pl} \textsc{neg} go.\textsc{ptcp.3pl.f} to mass\\
\glt `He asked them why they didn’t go to mass.' (Sadrún, \citealt[103]{Büchli1966})
\z

\ea
\label{ex:indinterr7}
\gll [...] a lu \textbf{sa} ju schòn in téc \textbf{sc}’ \textbf{i} \textbf{funczjanava}.   \\
{} and then know.\textsc{prs.1sg} \textsc{1sg} indeed \textsc{indef.m.sg} bit how \textsc{expl} function.\textsc{impf.3sg}\\
\glt `[...] and therefore I know a bit how it used to function.' (Sadrún, m9, \sectref{sec:8.15})
\z

\ea
\label{ex:indinterr8}
\gll [...] a lu ò `l \textbf{grju} li gljut [...] tga ségi trajs rùsnas; ajn \textbf{tgénina} èl dégi métar ajn la crusch.\\
{} and then have.\textsc{prs.3sg} \textsc{3sg.m} shout.\textsc{ptcp.unm} \textsc{def.dat.sg} people.\textsc{f.sg} {} \textsc{comp} \textsc{exist.prs.sbjv.3sg.expl} three hole.\textsc{f.pl} into which.\textsc{f.sg} \textsc{3sg.m} must.\textsc{prs.sbjv.3sg} put into \textsc{def.f.sg} cross\\
\glt `[...] and then he shouted to the people [...] [saying that] there were three holes; [asking] into which he should put the cross.' (Bugnaj, \citealt[134]{Büchli1966})
\z

A special case is (\ref{ex:indinterr10}) in which the \isi{manner clause} introduced by \textit{scù} `like, as' modifies the manner adverb \textit{usché} `so'.

\ea
\label{ex:indinterr10}
\gll [...] in téc da la gjuvantétgna sa ju schòn \textbf{raquintá} … \textbf{usché} \textbf{scù} i mava da nòs tjams a \textbf{tgé} \textbf{ca} va ùsa [...].\\
{} \textsc{indef.m.sg} bit of \textsc{def.f.sg} youth can.\textsc{prs.1sg} \textsc{1sg} indeed tell.\textsc{inf} {} so as \textsc{expl} go.\textsc{impf.3sg} of \textsc{poss.1pl.m.sg} time.\textsc{pl} and what \textsc{rel} go.\textsc{prs.3sg} now \\
\glt `[...] a bit of my youth I can indeed tell [you about] ... the way it was when we were young and the way it is nowadays [...].' (Sadrún, m9, \sectref{sec:8.15})
\z

\subsubsection{Prepositional argument clauses}\label{sec:6.2.1.4}
\isi{Prepositional argument clauses} are non-finite. In the corpus, the following verbs occur: \textit{amprèndar d'ancanùschar} `get to know' (\ref{ex:preparg1}), \textit{antschajvar a} `begin to' (\ref{ex:preparg2}), \textit{calá da} `stop doing' (\ref{ex:preparg3}), \textit{prèndar avaun da} `plan' (\ref{ex:preparg4}), and \textit{sadacídar da} `decide to' (\ref{ex:preparg5}).
 
\ea
\label{ex:preparg1}
\gll Api, ah, quaj ah fascinava pròpi mè, ju vèṣ ah gè ju vèṣ è ugèn \textbf{ampríu} \textbf{d’} \textbf{ancanùschar} quaj mél [...]. \\
and eh \textsc{dem.unm} eh fascinate.\textsc{impf.3sg} really \textsc{1sg}  \textsc{1sg} have.\textsc{cond.1sg} ah yes \textsc{1sg} have.\textsc{cond.1sg} also with\_pleasure learn.\textsc{ptcp.unm} \textsc{comp} know.\textsc{inf} \textsc{dem.m.sg} mule\\
\glt `And, eh, this really fascinated me, I would have eh yes I would have very much liked to get to know this mule [...].' (Ruèras, m 10, \sectref{sec:8.7})
\z

\ea
\label{ex:preparg2}
\gll  Sé Nalps vèvani è grad \textbf{antschiat} \textbf{a} bagagè cantinaṣ ètcè̱tara.\\
up \textsc{pn} have.\textsc{impf.3pl.3pl} also just start.\textsc{ptcp.unm} \textsc{comp} build.\textsc{inf} canteen.\textsc{f.pl} et\_cetera\\
\glt `In Nalps they had just begun to build canteens and so on.' (Sadrún, m4, \sectref{sec:8.3})
\z

\ea
\label{ex:preparg3}
\gll  Ju \textbf{cala} \textbf{dad} \textbf{í} a scùlèta, ju pùs bitg' í plé.\\
\textsc{1sg} stop.\textsc{prs.1sg} \textsc{comp} go.\textsc{inf} to nursery\_school.\textsc{f.sg} \textsc{1sg} can.\textsc{prs.1sg} \textsc{neg} go.\textsc{inf} any\_more  \\
\glt `I’ll stop going to nursery school, I can’t stand it any longer.' (Sadrún, m4, \sectref{sec:8.3})
\z

\ea
\label{ex:preparg4}
\gll Gè, ju a atgnamajn príu, \textbf{príu} \textbf{avaun} \textbf{da} \textbf{préndar} ah, zatgéj històrjas cò da quaj ah détgas a praulaṣ [...].\\
yes \textsc{1sg} have.\textsc{prs.1sg} actually take.\textsc{ptcp.unm} take.\textsc{ptcp.unm} before \textsc{comp} take.\textsc{inf} eh something story.\textsc{f.pl} here of \textsc{dem.unm} eh legend.\textsc{f.pl} and fairy\_tale.\textsc{f.pl}\\
\glt `Actually, I had planned to take, eh, some stories here from these legends and fairy tales, [...].' (Sadrún, m4, \sectref{sec:8.3})
\z

\largerpage
\ea
\label{ex:preparg5}
\gll Lu sùnd ju \textbf{sadacidjus} \textbf{da} … \textbf{raṣdá} in pau ṣur da la ... da mi’ ufaunza [...].\\
then be.\textsc{prs.1sg}  \textsc{1sg}  \textsc{refl}.decide.\textsc{ptcp.m.sg}  \textsc{comp} {} talk.\textsc{inf} \textsc{indef.m.sg} little over of  \textsc{def.f.sg} {} of  \textsc{poss.1sg.f.sg} childhood\\
\glt `Then I decided to ... talk a bit about ... my childhood [...].' (Sadrún, m4, \sectref{sec:8.3})
\z

\subsection{Adjunct clauses}\label{sec:6.2.2}

\subsubsection{Temporal clauses}\label{sec:6.2.2.1}
Temporal clauses are either finite or non-finite. \isi{Non-finite temporal clauses} are introduced by \textit{da} (\ref{ex:tempcltgu1} -- \ref{ex:nonfintempcl4}),  \textit{avaun ca} (\ref{ex:nonfintempcl3}), or \textit{sjantar} (\ref{ex:sjantarnonf1}).

\ea
\label{ex:tempcltgu1}
\gll    […] faː ajn […] ʦaˈkɔnʣ diants aʎ rǝʃˈtiː dɐ lɐ ˈdʊːnɐ ʨɐ l ɔ rut ɔːr ejr \textbf{dɐ} \textbf{mɐˈʦaː} in ruʃp\\
{} make.\textsc{inf} in {} some.\textsc{m.pl} tooth.\textsc{pl} \textsc{dat} rake.\textsc{m.sg} of \textsc{def.f.sg} woman \textsc{rel} \textsc{3sg.m} have.\textsc{prs.3sg} break.\textsc{ptcp.unm} out also \textsc{subord} kill.\textsc{inf} \textsc{indef.m.sg} toad\\
\glt `[…] put in some teeth to the woman’s rake he had broken when he killed a toad.' (Ruèras, \citealt[8]{Valär2013b})
\z
\ea
\label{ex:nonfintempcl1}
\gll    \textbf{Da} \textbf{vagní} \textbf{anavùs} vònd ju plaunsju!\\
\textsc{subord} come.\textsc{inf} back go.\textsc{prs.1sg} \textsc{1sg} slowly \\
\glt `When I come back, I’ll walk slowly!' (Cavòrgja, \citealt[120]{Büchli1966})
\z

\ea
\label{ex:nonfintempcl4}
\gll    A \textbf{da} \textbf{gjantá} … sch’ ina sòra … tudèstga èra … vida majṣa, scha stuèvan tùt quèla- nuṣ ròmòntschas raṣdá tudèstg.\\
and \textsc{subord} lunch.\textsc{inf} {} if \textsc{indef.f.sg} nun {} German \textsc{cop.impf.3sg} {} at\_of table.\textsc{f.sg} \textsc{corr} must.\textsc{impf.3pl} all \textsc{dem.f.sg} \textsc{1pl} Romansh.\textsc{f.pl} speak.\textsc{inf} German.\textsc{m.sg}\\
\glt `And during lunch … if a German … nun was … at table, all these – we, the Romansh speaking people, had to speak German.' (Camischùlas, f6, \sectref{sec:8.4})
\z


\ea
\label{ex:nonfintempcl3}
\gll  [...] \textbf{avaun} \textbf{c}’ \textbf{í} \textbf{vidajn}, ah, staus lu tial miadi [...].\\
{} before \textsc{subord} go.\textsc{inf} uphill eh \textsc{cop.ptcp.m.sg} then  at.\textsc{def.m.sg} doctor\\
\glt `[...] before going uphill, eh, [he] went to the doctor [...].' (Sadrún, m4, \sectref{sec:8.3})
\z

\ea
\label{ex:sjantarnonf1}
\gll \textbf{Sjantar} \textbf{vaj} \textbf{fatg} als pènṣumṣ, ṣ' als scùlarṣ i a spaz.\\
after have.\textsc{inf} do.\textsc{ptcp.unm} \textsc{def.art.m.pl} homework.\textsc{pl} be.\textsc{prs.3sg} \textsc{def.art.m.pl} pupil.\textsc{pl} go.\textsc{ptcp.m.pl} to walk\\
\glt `After doing their homework, the pupils went for a walk.' (Sadrún, m5)
\z

\isi{Finite temporal clauses} are introduced by \textit{avaun ca tga} `before', \textit{cu} `when', \textit{dafartáuntiar tga} `whilst', \textit{durònt tga} `when', \textit{schi glajti scù} `as soon as', \textit{sjantar tga} `after' (\ref{ex:sjantarf1}), and \textit{tòca} `until'.

With the subordinator \textit{cu} (derived from the interrogative pronoun \textit{cura} `when'), correlative \textit{scha} `if, then' is usually used (\ref{ex:cuscha1} and \ref{ex:cuscha2}).

\ea
\label{ex:cuscha1}
\gll  […] \textbf{cu} i dat la banadiczjun, \textbf{scha} fò la sòntga crusch.\\
    {} when \textsc{expl} give.\textsc{prs.3sg} \textsc{def.f.sg} blessing \textsc{corr} make.\textsc{imp.2sg} \textsc{def.f.sg} holy cross\\
\glt `[…] when it comes to the blessing, make the sign of the cross.' (\citealt[23]{Berther2007})
\z

\ea
\label{ex:cuscha2}
	\gll  A pér \textbf{cu} quaj èra fatg, \textbf{scha} èra la scòtga mèmi tgauda [...].  \\
and only when \textsc{dem.unm} \textsc{pass.aux.impf.3sg} do.\textsc{ptcp.unm} \textsc{corr} \textsc{cop.impf.3sg} \textsc{def.f.sg} whey too hot\\
\glt `And only when this was done was the whey too hot [...].' (Ruèras, m3, \sectref{sec:8.16})
\z

Instead of \textit{cu} `when' one also finds \textit{tga}, which in combination with \textit{ju} `I' is realised as \textit{tgu} (\ref{ex:tempcltgu2} and\ref{ex:tempcltgu3}).

\ea
\label{ex:tempcltgu2}
\gll    A sjantar [...] ṣè `l bap lu mòrts \textbf{tgu} vèva … mù vèntgadúṣ òns.\\
and after {} be.\textsc{prs.3sg} \textsc{def.m.sg} father then die.\textsc{ptcp.m.sg} \textsc{subord.1sg} have.\textsc{impf.3sg} {} only twenty-two year.\textsc{m.pl}\\
\glt `And after that [...] my father then died when I was … only 22 years old.' (Ruèras, m1, \sectref{sec:8.2})
\z

\ea
\label{ex:tempcltgu3}
\gll  Ò lò vòu fòrza schòn è survagnú in téc quajda d' í par crapa, \textbf{tgu} a vju difarènts lògans tg’ i vèvan sitau gjù ad èra … vagnú ò cristalaṣ [...]. \\
down there  have.\textsc{prs.1sg.1sg} maybe really also get.\textsc{ptcp.m.unm} \textsc{indef.m.sg} bit desire.\textsc{f.sg} \textsc{attr} go.\textsc{inf} for stone.\textsc{coll} \textsc{subord.1sg} have.\textsc{prs.1sg} see.\textsc{ptcp.m.unm} different.\textsc{m.pl} place.\textsc{pl} \textsc{subord} \textsc{3pl} have.\textsc{impf.3pl} blast.\textsc{ptcp.m.unm} down and  be.\textsc{impf.3sg} {} come.\textsc{ptcp.m.unm} out crystal.\textsc{f.pl}\\
\glt `Out there I might have started enjoying looking for stones a bit, when I saw different places where they had blasted [the rocks], and crystals [...] had come out ... .' (Sadrún, m4, \sectref{sec:8.3})
\z

Examples (\ref{ex:tempcl1}—\ref{ex:tempcl4}) illustrate the other subordinators that introduce a \isi{temporal clause}.

\ea
\label{ex:tempcl1}
\gll \textbf{Dafartáuntiar} \textbf{tga} la mùma fò quaj, miras té dal pòp.\\
whilst \textsc{subord} \textsc{def.f.sg} mother do.\textsc{prs.3sg} \textsc{dem.unm} look.\textsc{prs.2sg} \textsc{2sg} of.\textsc{def.m.sg} baby\\
\glt `Whilst mother is doing this, you look after the baby.' (Sadrún, m5)
\z

\ea
\label{ex:tempcl2}
\gll Ju a gju tèma [...] tg' als tiars ré̱jṣdian \textbf{durònt} \textbf{tga} la sòntga mèssa da Nadal végni lagida.\\
      \textsc{1sg} have.\textsc{prs.1sg} have.\textsc{ptcp.unm} fear {} \textsc{comp} \textsc{def.m.pl} animal.\textsc{pl} talk.\textsc{prs.sbjv.3pl} during \textsc{subord} \textsc{def.f.sg} holy mass of Christmas \textsc{pass.prs.sbjv.3sg} read.\textsc{ptcp.f.sg}\\
\glt `I feared […] that the animals could talk when Christmas Mass would be read.' (Tschamùt, \citealt[19]{Büchli1966})
\z

\ea
\label{ex:tempcl3}
\gll  A schi … \textbf{schi} \textbf{glajti} \textbf{scù} nuṣ èssan staj sén la via cantunala òni antschiat a galòpá.  \\
and so {} so soon as \textsc{1pl} be.\textsc{prs.1pl} \textsc{cop.ptcp.m.pl} on \textsc{def.f.sg} way cantonal have.\textsc{prs.3pl.3pl} begin.\textsc{ptcp.unm} \textsc{comp} gallop.\textsc{inf}\\
\glt `And as … as soon as we were on the cantonal way they started to gallop.' (Ruèras, m10, \sectref{sec:8.7})
\z

\ea
\label{ex:sjantarf1}
\gll \textbf{Sjantar} \textbf{tga} nuṣ vajn vju al film, èssan nuṣ i a durmí.\\
after \textsc{subord} \textsc{1pl} have.\textsc{prs.1pl} see.\textsc{ptcp.m.pl} \textsc{def.art.m.sg} film be.\textsc{prs.1pl} \textsc{1pl} go.\textsc{ptcp.unm} \textsc{subord} sleep.\textsc{inf}\\
\glt `After we saw the film, we went to sleep.' (Sadrún, m5)
\z

\ea
\label{ex:tempcl4}
\gll    Api èra la sòra òra uschéja … avaun niaṣ ésch ad ò spatgau a spatgau \textbf{tòca} \textbf{la} \textbf{audi} \textbf{anzatgéj} [...].\\
and \textsc{cop.impf.3sg} \textsc{def.f.sg} nun out so {} in\_front\_of \textsc{poss.1pl.m.sg} door and have.\textsc{prs.3sg} wait.\textsc{ptcp.unm} and wait.\textsc{ptcp.unm} until \textsc{3sg.f} hear.\textsc{prs.sbjv.3sg} something\\
\glt `And then the nun was out [on the corridor] like this ... in front of our door, waiting and waiting until she would hear something [...].' (Camischùlas, f6, \sectref{sec:8.4})
\z


\subsubsection{Manner clauses}\label{sec:6.2.2.2}
Non-finite manner clauses are introduced by \textit{cun} `with' (\ref{ex:manncl1}) and finite manner clauses either by \textit{scù} (\ref{ex:manncl2}), \textit{scù tga} (\ref{ex:manncl3}), or only \textit{tga} (\ref{ex:manncl4}).

\ea
\label{ex:manncl1}
\gll   Gè, i fùs schòn flòt, ábar ju sùn è vagnúṣ atrás \textbf{cun} \textbf{fá} `\textbf{l} \textbf{pur}. \\
yes \textsc{expl} \textsc{cop.cond.3sg} really great.\textsc{adj.unm} but \textsc{1sg}  be.\textsc{prs.1sg} also  come.\textsc{ptcp.m.sg} through with  do.\textsc{inf} \textsc{def.m.sg} farmer\\
\glt `Yes, this would be really great, but I also could earn a living by being a farmer.' (Ruèras, m1, \sectref{sec:8.2})
\z

\ea
\label{ex:manncl2}
\gll    Òz ṣaj al al Furka né al … la ban\footnotemark{} né \textbf{scù} i végn raṣdau òz sén tudèstg.\\
today \textsc{cop.prs.3sg.expl} \textsc{def.m.sg} \textsc{def.m.sg} \textsc{pn} or \textsc{def.m.sg} {} \textsc{def.f.sg} train right as \textsc{expl} \textsc{pass.aux.prs.3sg} speak.\textsc{ptcp.unm} today on German.\textsc{m.sg}\\
\glt `Today it is the Furka or the … the \textit{ban} `train', as nowadays it is called in German.' (i.e the Matterhorn-Gotthard-Bahn)\footnotetext{\textit{Ban} is another German word (\textit{Bahn}) for `train'.} (Ruèras, m1, \sectref{sec:8.2})
\z

\ea
\label{ex:manncl3}
\gll Nus mavan bjè, gè, a fá gjucs, \textbf{scù} \textbf{tg}’ ins fagèva plé baut [...].   \\
\textsc{1pl} go.\textsc{impf.1pl} often yes \textsc{subord} do.\textsc{inf} game.\textsc{m.pl} as \textsc{comp} \textsc{gnr} do.i\textsc{mpf.3sg} more early\\
\glt `We would often go and play, as one would do it in the past [...].' (Sadrún, m9, \sectref{sec:8.15})
\z

\ea
\label{ex:manncl4}
\gll   La bùca stuèv' èssar \textbf{tga} la pudèv' ajn, la lata.\\
\textsc{def.f.sg} mouth should.\textsc{impf.3sg} \textsc{cop.inf} \textsc{subord} \textsc{3sg.f} can.\textsc{impf.3sg} into \textsc{def.f.sg} slat\\
\glt `The cutting should be such that the slat could fit into it.' (Camischùlas, \DRGoK{3}{583})
\z

\subsubsection{Purposive clauses}\label{sec:6.2.2.3}
\isi{Non-finite purposive clauses} are introduced by \textit{a} after a \isi{verb of movement} (\ref{ex:purp:inf:1} and \ref{ex:purp:inf:2}) and by \textit{par/pr̩} (\ref{ex:purp:inf:3}) or \textit{da} (\ref{ex:purp:inf:4} and \ref{ex:purp:inf:5}) in all other cases; \textit{da}, however, is very rare. Example (\ref{ex:purp:inf:2}) contains both \textit{par} and \textit{a}.

\ea\label{ex:purp:inf:1}
\gll  Api ṣchèvan aj èba in tg’ è mòrts tga \textbf{végn} \textbf{a} \textbf{métar} als, als tjarms ... la nòtg [...].\\
and say.\textsc{impf.3pl} \textsc{3pl} precisely one \textsc{rel} \textsc{cop.prs.3sg} dead.\textsc{m.sg} \textsc{rel} come.\textsc{prs.3sg} \textsc{subord} put.\textsc{inf} \textsc{def.m.pl} \textsc{def.m.pl} boundary\_stone.\textsc{pl} {} \textsc{def.f.sg} night \\
\glt `And, precisely, they also used to say [that] somebody who was dead came and put the, the boundary stones ... at night [...].' (Sèlva, f2, \sectref{sec:8.6})
\z

\ea
\label{ex:purp:inf:2}
\gll [...] ju vèva … siṣ òns tga èran vida prapará la via \textbf{par} \textbf{í} sé Nalps \textbf{a} \textbf{bagagè} al mir da farmada.\\
{} \textsc{1sg} have.\textsc{impf.1sg} {} six year.\textsc{m.pl} \textsc{comp} \textsc{cop.impf.3pl} \textsc{prog} prepare.\textsc{inf} \textsc{def.f.sg} road \textsc{subord} go.\textsc{inf} up \textsc{pn} \textsc{subord} build.\textsc{inf} \textsc{def.m.sg} wall.\textsc{m.sg} of reservoir.\textsc{f.sg} \\ 
\glt `[...] I was six years old when they were preparing the road in order to go to Nalps to build the wall of the reservoir.' (Sadrún, m4, \sectref{sec:8.3})
\z

\ea
\label{ex:purp:inf:3}
\gll Als méls èran, qu’ èra … fèrm … fèrms tiarṣ ad èl duvrava quaj mél pr̩ trá, \textbf{par} \textbf{trá} lèna sé da Cavòrgja.\\
\textsc{def.m.pl} mule \textsc{cop.impf.3sg} \textsc{dem.unm} \textsc{cop.impf.3sg} {} strong.\textsc{adj.unm} {} strong.\textsc{m.pl} animal.\textsc{pl} and \textsc{3sg.m} use.\textsc{impf.3sg} \textsc{dem.m.sg} mule \textsc{subord} pull.\textsc{inf} \textsc{subord} pull.\textsc{inf} wood.\textsc{coll} up from \textsc{pn}\\
\glt `The mules were, these were in fact …. strong … strong animals and he used that mule for transporting wood up from Cavorgia.' (Ruèras, m10, \sectref{sec:8.7})
\z

\ea
\label{ex:purp:inf:4}
\gll  [...] méz in pétg sc’ inṣ drùva par tùt las lavurṣ da parmavèra a gl atún \textbf{da} \textbf{cavá} \textbf{trúfals}.\\
{} put.\textsc{ptcp.unm} \textsc{indef.m.sg} hoe like \textsc{gnr} need.\textsc{prs.3sg} for all \textsc{def.f.pl} work.\textsc{pl} of spring.\textsc{f.sg} and \textsc{def.m.sg} autumn \textsc{subord} dig.\textsc{inf} potato.\textsc{m.pl}\\
\glt `[…] put a hoe like [the one] one needs for all the work that must be done in spring, and in autumn in order to dig out potatoes.' (Bugnaj, \citealt[143]{Büchli1966})
\z

\ea
\label{ex:purp:inf:5}
\gll    Api savèv’ ins bigja cù, cù fá \textbf{da} \textbf{purtá} \textbf{las} \textbf{nèglas} [...].\\
and know.\textsc{impf.3sg} \textsc{gnr} \textsc{neg} how how do.\textsc{inf} \textsc{comp} carry.\textsc{inf} \textsc{def.f.pl} carnation.\textsc{pl}\\
\glt `And one would not know how to put the carnations [...] (literally `how to do in order to put the carnations').' (Zarcúns, m2, \sectref{sec:8.13})
\z

There are several examples where the subordinator \textit{a} is absent (\ref{ex:purp:inf:6}). In (\ref{ex:purp:inf:7}) both constructions occur — with and without the subordinator \textit{a}.

\ea
\label{ex:purp:inf:6}
	\gll [...] api èri dad \textbf{í} \longrule {} \textbf{métar} trúfals [...] .\\
{} and be.\textsc{impf.3sg.expl} \textsc{comp} go.\textsc{inf} {}  put.\textsc{inf} potato.\textsc{m.pl}\\
\glt `[...] and then one had to sow potatoes [...].' (Ruèras, f4, \sectref{sec:8.16})
\z

\ea
\label{ex:purp:inf:7}
\gll [...] api èri dad \textbf{í} \longrule {} \textbf{zarclá} trúfals api èri dad \textbf{í} \textbf{a} \textbf{zarclá} al graun.   \\
{} and be.\textsc{impf.3sg.expl} to go.\textsc{inf} {} weed.\textsc{inf} potato.\textsc{m.pl} and be.\textsc{impf.3sg.expl} \textsc{comp} go.\textsc{inf} \textsc{subord} weed.\textsc{inf} \textsc{def.m.sg} cereal\\
\glt `[...] and then one had to weed the potato fields and then one had to go and weed the cereal fields.' (Ruèras, f4, \sectref{sec:8.16})
\z

\isi{Finite purposive clauses} are introduced by par \textit{par tga} (\ref{ex:purppartga}) or only by \textit{tga} (\ref{ex:purptga}).

\ea
\label{ex:purppartga}
\gll  A la sèra \textbf{par} \textbf{tga} \textbf{briṣchi} bétg … vagnéva quaj, quaj mava `l ajnagjù cul maun èra sènza … [vòns] a trèva vidò̱ còtgla gjù sé sél plantschju.\\
and \textsc{def.f.sg} evening \textsc{subord} \textsc{comp} burn.\textsc{prs.sbjv.3sg} \textsc{neg} {} \textsc{pass.aux.impf.3sg} \textsc{dem.unm} \textsc{dem.unm} go.\textsc{impf.3sg} \textsc{3sg.m} in\_down with.\textsc{def.m.sg} hand also without {} [glove.\textsc{m.pl}] and pull.\textsc{impf.3sg} out charcoal.\textsc{coll} down up on.\textsc{def.m.sg} floor  \\
\glt `And in the evening, to avoid it burning … was that, there he went into [the fire] with one hand, also without [gloves], and pulled out charcoal from down there up to the floor.' (Sadrún, m4, \sectref{sec:8.3})
\z

\ea
\label{ex:purptga}
	\gll    A quaj stèvnṣ èssar … pulits-pulits l’ jamna … {\longrule} \textbf{tg}’ al bap \textbf{dètschi} in frang a miaz.\\
	and \textsc{dem.unm} must.\textsc{impf.1pl.1pl} \textsc{cop.inf} {} \textsc{red}\textasciitilde{well\_behaved}.\textsc{m.pl} \textsc{def.f.sg} week {} {}  \textsc{comp} \textsc{def.m.sg} father  give.\textsc{prs.sbjv.3sg} one.\textsc{m.sg} franc and half.\textsc{m.sg}\\
\glt `And we had to be … very well-behaved during the week … so that my father would give [us] one and a half francs.' (Ruèras, m1, \sectref{sec:8.2})
\z



In the DRG materials, there is one occurrence of the negator \textit{bétg} preceding the subordinator \textit{tga}, i.e. outside the subordinate clause (\ref{ex:betgtga1}).

\ea
\label{ex:betgtga1}
\gll Vèndardé sògn stù in muantá la tgarn a las ljòngjas \textbf{bétga} \textbf{tga} vignan ajn als baus.\\
Friday.\textsc{m.sg} holy must.\textsc{prs.3sg} \textsc{gnr} move.\textsc{inf} \textsc{def.f.sg} meat and \textsc{def.f.pl} sausage.\textsc{pl} \textsc{neg} \textsc{subord} come.\textsc{prs.3pl} in \textsc{def.m.pl} beetle.\textsc{pl}\\
\glt `On Good Friday one must move the meat and the sausages so that the beetles cannot get into the meat.' (\DRGoK{3}{385})
\z

This construction has not been accepted by my consultants; however, a similar construction which, in contrast to the Tuatschin example (\ref{ex:betgtga1}), includes the subordinator \textit{par}, can be found in other Romansh varieties like in the \ili{Sutsilvan} dialect of Dalin (\ref{ex:betgtga2}).

\ea\label{ex:betgtga2}
\gll  \textbf{Par} \textbf{bétg} \textbf{tg}' in schleschi dat il calger eign in pêr guspas els calzers.\\
     for \textsc{neg} \textsc{subord} \textsc{gnr} slip.\textsc{prs.sbjv.3sg} give.\textsc{prs.3sg} \textsc{def.m.sg} shoemaker in \textsc{indef.m.sg} couple nail.\textsc{pl} in.\textsc{def.m.pl} shoe.\textsc{pl}\\
\glt `In order not to slip, the shoemaker hits some nails into the shoes.' (Ṣutsilván, Dagljégn, \DRGoK{4}{607})
\z

Therefore it is possible that the Tuatschin construction in (\ref{ex:betgtga1}) belonged to an older variety of the language.


\subsubsection{Causal clauses}\label{sec:6.2.2.4}
\isi{Non-finite causal clauses} are introduced by \textit{da} (\ref{ex:caus1}), and \isi{finite causal clauses} are introduced by \textit{parquaj/pr̩quaj tga} (\ref{ex:caus3}), \textit{partgé} (\ref{ex:caus4}), \textit{má tga} (\ref{ex:caus5}), \textit{tga} (\ref{ex:caus6}), or \textit{cunquaj tga} `since' (\ref{ex:caus7}).
 
 \ea
 \label{ex:caus1}
  \gll ˈʥeːvjǝ zɛ l ˈtɔni dɐ lɐ mɐtˈlajnɐ sǝʃfǝrdɐnˈtaws ʃi feʨ \textbf{dɐ} ˈ\textbf{bajbǝr} \textbf{trajs} \textbf{ˈmjɔːlas} \textbf{pɛn} \textbf{frajt}\\
 Thursday \textsc{cop.prs.3sg} \textsc{def.m.sg} \textsc{pn} of \textsc{def.f.sg} \textsc{pn} \textsc{refl}.catch.cold.\textsc{ptcp.m.sg} so much \textsc{subord} drink.\textsc{inf} three cup.\textsc{pl} buttermilk cold\\
 \glt `On Thursday Matlaina’s Toni caught a very strong cold because he drank three cups of cold buttermilk.' (Ruèras, \citealt[9]{Valär2013b})
 \z

\ea
\label{ex:caus3}
\gll  [...] api lura va ju in’ jèda talafònau dad èl \textbf{pr̩quaj} \textbf{tg}' èl vèva tarmèz in’ anunzja da mòrt [...].\\
{} and then have.\textsc{1sg}  \textsc{1sg} one.\textsc{f.sg} time call.\textsc{ptcp.unm} \textsc{dat} \textsc{3sg.m} because \textsc{subord} \textsc{3sg.m} have.\textsc{impf.3sg} send.\textsc{ptcp.unm} \textsc{indef.f.sg} announcement of death.\textsc{f.sg}\\ 
\glt `[...] and then I phoned him once, because he had sent a death notice [...].' (Sadrún, f3, \sectref{sec:8.1})
\z

\ea
\label{ex:caus4}
\gll    Avaun ina fjasta mavan aj … tialas … gjufnas … par nègla, \textbf{partgé} matévan sé sé la tgapjala … ina nègla.\\
before \textsc{indef.f.sg} celebration go.\textsc{impf.3pl} \textsc{3pl} {} to.\textsc{def.f.pl} {}  young\_woman.\textsc{pl} {} for carnation.\textsc{f.pl} because put.\textsc{impf.3pl} up up  \textsc{def.f.sg} hat {} \textsc{indef.f.sg} carnation \\
\glt `Before a celebration they would go … to the … girls for carnations, because they would put … a carnation on their hat.' (Zarcúns, m2, \sectref{sec:8.13})
\z

\ea
\label{ex:caus5}
\gll Cò ò la mùma dau ina sèrvla, \textbf{má} \textbf{tg}’ i è damaun dumèngja.\\
here have.\textsc{prs.3sg} \textsc{def.f.sg} mother give.\textsc{ptcp.unm} \textsc{indef.f.sg} cervelat because \textsc{subord} \textsc{expl} \textsc{cop.prs.3sg} tomorrow Sunday\\
\glt `Here, mother provided a cervelat, because tomorrow is Sunday.' (Cavòrgja, m7, \sectref{sec:8.17})
\z

\ea
\label{ex:caus6}
\gll Ùssa léjva \textbf{tg'} i è clar dé.\\
  now get\_up.\textsc{imp.2sg} \textsc{subord} \textsc{expl} \textsc{cop.prs.3sg} clear day\\
\glt `Get up now since day has already broken.' (\DRGoK{3}{719})
\z

\ea
\label{ex:caus7}
\gll [...] \textbf{cunquaj} \textbf{tg}' èl èra … fòrsa staus da malitèr tials tgavals, né gju da fá cun tgavals, scha … vèv’ èl … cumprau in asan, álṣò in mél, bitg in asan, in mél.\\
{} since \textsc{subord} \textsc{3sg.m} be.\textsc{impf.3sg} {} maybe \textsc{cop.ptcp.m.sg} of army.\textsc{m.sg} at.\textsc{def.m.pl} horse.\textsc{pl} or have.\textsc{ptcp.unm}  to do.\textsc{inf} with horse.\textsc{m.pl} \textsc{corr} {} have.\textsc{impf.3sg} \textsc{3sg.m} {}  buy.\textsc{ptcp.unm}  \textsc{indef.m.sg} donkey this\_is \textsc{indef.m.sg} mule \textsc{neg} \textsc{indef.m.sg} donkey \textsc{indef.m.sg} mule \\
\glt `[...] since in the army he had ... maybe been, with the horses, or had to do with horses, then ... he had ... bought a donkey, that is to say a mule, not a donkey, a mule.' (Ruèras, m10, \sectref{sec:8.7})
\z

\subsubsection{Conditional clauses}\label{sec:6.2.2.5}
\isi{Conditional clauses} are formed in three different ways:

\begin{itemize}
	\item a correlative construction with \textit{scha} ìf'' in both the protasis and the apodosis (\ref{ex:cond1} and \ref{ex:cond2}), 
	\item only the protasis is introduced by \textit{scha} 'if' (\ref{ex:cond3}--\ref{ex:cond5}),
	\item without subordinator in the protasis but with \isi{subject inversion} and \textit{scha} `if' or \textit{lura} `when' in the apodosis (\ref{ex:cond6}).
\end{itemize}

In all these cases there is \isi{subject inversion} in the apodosis, with some rare exceptions. Furthermore, correlative \textit{lura} is very rare in the corpus; it only occurs in \citet{Büchli1966}.

\ea
\label{ex:cond1}
\gll \textbf{Scha} té as lu mèmja bjè da raclamá a grí, \textbf{scha} matajn nus té ajnagjù ‘l Run.\\
if \textsc{2sg} have.\textsc{prs.2sg} then too much \textsc{attr} complain.\textsc{inf} and shout.\textsc{inf} \textsc{corr} put.\textsc{prs.1pl} \textsc{1pl} \textsc{2sg} in\_down \textsc{def.m.sg} \textsc{pn}\\
\glt `If you really have so much to complain and shout about, we will throw you down into the Run [creek].' (\citealt[120]{Berther1998})
\z

\ea
\label{ex:cond2}
\gll    A da gjantá … \textbf{sch}’ ina sòra … tudèstga èra … vida majṣa, \textbf{scha} stuèvan tùt quèla- nuṣ ròmòntschas raṣdá tudèstg.\\
and of lunch.\textsc{inf} {} if \textsc{indef.f.sg} nun {} German \textsc{cop.impf.3sg} {} at\_of table.\textsc{f.sg} \textsc{corr} must.\textsc{impf.3pl} all \textsc{dem.f.sg} \textsc{1pl} Romansh.\textsc{f.pl} speak.\textsc{inf} German.\textsc{m.sg}\\
\glt `And during lunch … if a German … nun was … at table, all these – we, the Romansh speaking people, had to speak German.' (Camischùlas, f6, \sectref{sec:8.4})
\z

\ea
\label{ex:cond3}
\gll \textbf{Scha} ju antupás quèla gljut, \textbf{sabatès} \textbf{ju} gjù ajn ganugljaṣ a bitscháṣ als cazès.\\
if \textsc{1sg} meet.\textsc{cond.1sg} \textsc{dem.f.sg} people \textsc{refl}.throw.\textsc{cond.1sg} \textsc{1sg} down in knee.\textsc{f.pl} and kiss.\textsc{cond.1sg} \textsc{def.m.pl} shoe.\textsc{pl}\\
\glt `If I met these people, I’d kneel down and kiss their shoes.' (\citealt[60]{Berther1998})
\z

\ea
\label{ex:cond4}
\gll Da mintga pur stèvanṣ \isi{í} a \textbf{scha} `l vèva gju dus, \textbf{stèva} \textbf{`l} pijè …  dus pòrs né in né uschéa [...].\\
of every farmer.\textsc{m.sg} must.\textsc{impf.1pl.1pl} go.\textsc{inf} and if \textsc{3sg.m} have.\textsc{impf.3sg} have.\textsc{ptcp.unm} two.\textsc{m} must.\textsc{impf.3sg}  \textsc{3sg.m} pay.\textsc{inf} {} two.\textsc{m} pig.\textsc{pl} or one.\textsc{m} or so\\
\glt `We had to go to every farmer and if he had given two [pigs], he should [pay] more … two pigs or one or so [...].' (Sadrún, m6, \sectref{sec:8.11})
\z

\ea
\label{ex:cond5}
\gll  [...] \textbf{scha} `l vèva bian, \textbf{vagnév}’ \textbf{ins} schòn séssúra inqual tgaussas.\\
{} if \textsc{3sg.m}  have.\textsc{impf.3sg} good.\textsc{adj.unm} come.\textsc{impf.3sg} \textsc{gnr} indeed upon some thing.\textsc{f.pl} \\
\glt `[...] if he was in a good mood, one could get to know some things.' (Sadrún, m4, l\sectref{sec:8.3})
\z

\ea
\label{ex:cond6}
\gll    \textbf{Vasèvan} \textbf{ins} ina signura […] cun schuba cuérta, còtschna, […] \textbf{lura} spitgavan als purs ina grònda malaura […].\\
     see.\textsc{impf.3sg.euph} \textsc{gnr} \textsc{indef.f.sg} woman {} with shirt.\textsc{f.sg} short red {} \textsc{corr} expect.\textsc{impf.3pl} \textsc{def.m.pl} farmer.\textsc{pl} \textsc{indef.f.sg} big storm\\
\glt `If one saw a woman with a short shirt, a red one, the farmers would expect a heavy storm.' (Sèlva, \citealt[34]{Büchli1966})
\z

Tense and mood is used in the following way. \isi{Present factual conditional clauses} take the present tense in the protasis as well as in the apodosis (\ref{ex:cond1}), and \isi{past factual} {conditional clauses} are the \isi{imperfect} in both the protasis and the apodosis (\ref{ex:cond2}, \ref{ex:cond4}, \ref{ex:cond5}, and \ref{ex:cond6})

\isi{Present counterfactual conditional clauses} have the \isi{direct conditional } in the protasis as well as in the apodosis (\ref{ex:cond3}), and \isi{past counterfactual conditional clauses} take the of the \isi{indirect conditional} in both the protasis and apodosis. For examples, see \sectref{sec:4.1.2.2.10}.

\subsubsection{Consecutive clauses}\label{sec:6.2.2.6}
\isi{Consecutive clauses} are introduced by \textit{tga} (\ref{ex:consec1}--\ref{ex:consec3}) or by \textit{usché tga} (\ref{ex:consec5}). If an \isi{adjective} triggers a \isi{consecutive clause}, the adjective is modified by \textit{schi} `so' (\ref{ex:consec4}).

\ea
\label{ex:consec1}
\gll  Èla dètgi in' jèda ina curnada li èl \textbf{tg’} \textbf{èl} \textbf{stètschi} \textbf{sél} \textbf{plaz}.\\
\textsc{3sg.f} give.\textsc{prs.sbjv.3sg} one.\textsc{f} time \textsc{def.f.sg} push\_with\_horn.\textsc{ptcp.f.sg} \textsc{dat} \textsc{3sg.m} \textsc{subord} \textsc{3sg.m} stay.\textsc{prs.sbjv.3sg} on.\textsc{def.m.sg} place \\
\glt `She [the cow] would give him a push with her horns so that he would remain on the spot.' (Tschamùt, \citealt[20]{Büchli1966})
\z

\ea
\label{ex:consec2}
	\gll    [...] quaj piartg èra juṣ atráṣ a vèva rùt gjù al matg \textbf{tga} `l vèva mù la còrda plé antù̱rn.\\
{} \textsc{dem.m.sg} pig be.\textsc{impf.3sg} go.\textsc{ptcp.m.sg} through and have.\textsc{impf.3sg} break.\textsc{ptcp.unm} down \textsc{def.m.sg} bunch  \textsc{subord} \textsc{3sg.m} have.\textsc{impf.3sg} only \textsc{def.f.sg} rope more around\\
\glt `[...] this pig had gone through and had broken the bunch of flowers so that he only had the rope around [his belly].' (Sadrún, m6, \sectref{sec:8.11})
\z

\ea
\label{ex:consec3}
\gll  Bjè jèdaṣ ṣèni scapaj \textbf{tga} nuṣ vajn ah pròpi gju ah gròndas misé̱rjas [...].\\
many time.\textsc{pl} be.\textsc{prs.3pl.3pl} escape.\textsc{ptcp.m.pl} \textsc{subord} \textsc{1pl}   have.\textsc{prs.1pl} eh really have.\textsc{ptcp.unm} ah big.\textsc{f.pl} trouble.\textsc{pl}\\
\glt `They escaped many times so that we had big troubles [...].' (Ruèras, m10, \sectref{sec:8.7})
\z

\ea
\label{ex:consec5}
\gll    [...] a gl aucségnar è pròpi vagnuṣ ála lavina, ad è … staus mòrts, \textbf{usché} \textbf{tga} las stréaṣ òn … fatg vandètga cugl aucségnar da Sadrún.\\
{} and \textsc{def.m.sg} priest be.\textsc{prs.3sg} really come.\textsc{ptcp.m.sg} into.\textsc{def.f.sg} avalanche and be.\textsc{prs.3sg} {} \textsc{cop.ptcp.m.sg} dead.\textsc{m.sg} so \textsc{comp} \textsc{def.f.pl} witch.\textsc{pl} have.\textsc{prs.3pl} {} do.\textsc{ptcp.unm} revenge.\textsc{f.sg} with.\textsc{def.m.sg} priest of \textsc{pn}\\
\glt `[...] and the priest really was engulfed by the avalanche and was … dead, so that the witches got their revenge on the priest of Sedrun.' (Sadrún, m6, \sectref{sec:8.5})
\z



\ea
\label{ex:consec4}
\gll  Èla è saprida ajn quaj \textbf{schi} \textbf{starmantús} \textbf{tg'} èla è curdada ajn ina rùsna nùndétga.\\
\textsc{3sg.f} be.\textsc{prs.3sg} \textsc{refl.}take.\textsc{ptcp.f.sg} in \textsc{dem.unm} so terrible.\textsc{adj.unm} \textsc{subord} \textsc{3sg.f} be.\textsc{prs.3sg} fall.\textsc{ptcp.f.sg} into \textsc{indef.f.sg} hole awful\\
\glt `She took this so seriously that she fell in an awful hole.' (Cavòrgja, f1)
\z

\subsubsection{Comparative clauses}\label{sec:6.2.2.7}
\isi{Non-finite comparative clauses} are introduced by \textit{scù da}, literally `how of' or `like of' (\ref{ex:comparative1}), and \isi{finite comparative clauses} are introduced by \textit{scù} (\ref{ex:comparative2}) or \textit{scù tga} (\ref{ex:comparative3}).


\ea
\label{ex:comparative1}
\gll  I vò fil a fil \textbf{scù} \textbf{da} \textbf{caná} in anṣégjl.\\
    \textsc{expl} go.\textsc{prs.3sg} jet to jet like \textsc{subord} stab.\textsc{inf} \textsc{indef.m.sg} kid \\
\glt `[Blood] flows like when one stabs a kid.' (\DRGoK{6}{300})
\z

\ea
\label{ex:comparative2}
\gll  [...] lu va ju … tlafònau dad èl a détg, éba, mi' ùm ségi èba mòrts \textbf{scù} \textbf{i} \textbf{sápian} [...].\\
{} then have.\textsc{prs.1sg} \textsc{1sg} {} call.\textsc{ptcp.unm} \textsc{dat} \textsc{3sg.m} and say.\textsc{ptcp.unm} exactly \textsc{poss.1sg.m.sg} man be.\textsc{prs.sbjv.3sg} precisely die.\textsc{ptcp.m.sg} as \textsc{3pl} know.\textsc{prs.sbjv.3pl}\\ 
\glt `[...] then I … phoned him and said that my husband had died as they knew [...].' (Sadrún, f3, \sectref{sec:8.1})
\z

\ea
\label{ex:comparative3}
\gll Nus mavan bjè, gè, a fá gjucs, \textbf{scù} \textbf{tg}’ ins fagèva plé baut [...].   \\
\textsc{1pl} go.\textsc{impf.1pl} often yes \textsc{subord} do.\textsc{inf} game.\textsc{m.pl} as \textsc{comp} \textsc{gnr} do.\textsc{impf.3sg} more early \\
\glt `We would often go and play, as one would do it earlier [...].' (Sadrún, m9, \sectref{sec:8.15})
\z

\subsubsection{Concessive clauses}\label{sec:6.2.2.8}
\isi{Concessive clauses} are introduced by \textit{schabi tga} (\ref{ex:schabi1}).

\ea
\label{ex:schabi1}
\gll \textbf{Schabi} \textbf{tga} lu, cun siṣ òns capév’ ins lu hald aun mèmja pauc a vèva bigja la... fòrsa da fá zatgéj.   \\
although \textsc{comp} then with six year.\textsc{m.pl} understand.\textsc{impf.3sg} \textsc{gnr} then just still too little and have.\textsc{impf.3sg} \textsc{neg} \textsc{def.f.sg} strength \textsc{attr} do.\textsc{inf} something\\
\glt `Although then, at the age of six, one would understand too little and wouldn’t have the ... strength to do something.' (Sadrún, m4, \sectref{sec:8.3})
\z

\subsubsection{Instead of}\label{sec:6.2.2.9}
`Instead of' only occurs in non-finite clauses in the corpus. They are introduced either by \textit{anstagl} (\ref{ex:anstagl1}) or \textit{anstagl da} (\ref{ex:anstagl2}).

\ea
\label{ex:anstagl1}
\gll    Api \textbf{anstagl} \textbf{bájbar} \textbf{al} \textbf{vin} … èran nus lu i sé, vèvani fatg ina bòla.\\
and instead drink.\textsc{inf} \textsc{def.m.sg} wine {} be.\textsc{impf.1pl} \textsc{1sg} then go.\textsc{ptcp.m.pl} up have.\textsc{impf.3pl.3pl} do.\textsc{ptcp.unm} \textsc{indef.m.sg} punch \\
\glt `And instead of drinking the wine … we went up, they had prepared a punch.' (Zarcúns, m2, \sectref{sec:8.13})
\z

\ea
\label{ex:anstagl2}
	\gll [...] \textbf{anstagl} \textbf{da} \textbf{mùngjar} òtgònta vacas èri fòrsa mù tschuncònta [...].\\
 {} instead\_of milk.\textsc{inf} eighty cow.\textsc{f.pl} \textsc{exist.impf.3sg.expl} maybe only fifty\\
\glt `[...]  instead of milking eighty cows there were maybe only fifty [...].' (Ruèras, m3, \sectref{sec:8.16})
\z



\section{Focus}\label{sec:6.3}
If the \isi{focus} is on verbs, participles, particles belonging to verbs, certain adverbs, or clauses which give new information, they are moved to the beginning of the sentence or clause.

In case a \isi{verb} is focused, it is moved in its \isi{infinitive} form and the finite form of the same verb is left behind in the background clause (\ref{ex:focverb1} and \ref{ex:focverb2}).


\ea
\label{ex:focverb1}
\gll  Ju a cò in bagljèt tòcan gjù Turitg, ábar \textbf{ira} \textbf{vònd} \textbf{ju} mù gjù Sumvitg.\\
\textsc{1sg} have.\textsc{3sg} here \textsc{indef.m.sg} ticket until down \textsc{pn} but go.\textsc{inf} go.\textsc{prs}.\textsc{1sg} \textsc{1sg} only down \textsc{pn}\\
\glt `I have here a ticket to Zurich, but I only go as far as Sumvitg.' (Sadrún, \citealt[106]{Büchli1966}) (literally '... but GOING I only go till Sumvitg')
\z

\ea
\label{ex:focverb2}
\gll Na na, a \textbf{durmí} \textbf{durmévan} nus cò.\\
no no and sleep.\textsc{inf} sleep.\textsc{impf.1pl} \textsc{1pl} here \\
\glt `No, no, and as for sleeping, we would sleep here.' (Surajn, f5, \sectref{sec:8.10})
\z

In the \isi{perfect tense}, participles are moved without their \isi{auxiliary verb}, which remains in the background clause (\ref{ex:focptcp1} and \ref{ex:focptcp2}). Example (\ref{ex:focptcp3}) shows that the participle can be moved with its complements.

\ea
\label{ex:focptcp1}
\gll Ju sùn dada gjù séla fatscha, mù \textbf{fatg} \textbf{òi} nuét.\\
\textsc{1sg} be.\textsc{prs.1sg} give.\textsc{ptcp.f.sg} down on.\textsc{def.f.sg} face but do.\textsc{ptcp.unm} have.\textsc{prs.3sg.expl} nothing\\
\glt `I fell down on my face but it didn't do anything.' (Cavòrgja, \citealt[106]{Büchli1966})
\z

\ea
\label{ex:focptcp2}
\gll  [...] quaj è vagnú da bètòn’ ajn, a \textbf{tanju} ò laṣ aun adina. \\
{} \textsc{dem.unm} be.\textsc{prs.3sg} come.\textsc{ptcp.unm} \textsc{comp} concrete.\textsc{inf} in and hold.\textsc{ptcp.unm} have.\textsc{prs.3sg} \textsc{3pl.f} still always \\
\glt `[...] this has been concreted, and they still hold.' (Sadrún, f3, \sectref{sec:8.1})
\z

\ea
\label{ex:focptcp3}
\gll [...] ábar \textbf{stju} \textbf{luvrá} \textbf{còrpòrálmajn} vajn nus schi fétg scù quèls.\\
{}	but must.\textsc{ptcp.unm} work.\textsc{inf} physical.\textsc{adj.m.adv} have.\textsc{prs.1pl} \textsc{1pl} so much as \textsc{dem.m.pl}\\
\glt `[...] but physically we had to work as hard as those [children].' (Ruèras, f4, \sectref{sec:8.16})
\z

\isi{Infinitives} modified by a \isi{modal verb} are left-dislocated without the modal verb, which remains in the background clause (\ref{ex:focmod1}).

\ea
\label{ex:focmod1}
\gll Als tiars vèzan al barlòt a tèman, mù \textbf{dí} \textbf{sòn} i nuét.\\
     \textsc{def.m.pl} animal.\textsc{pl} see.\textsc{prs.3pl} \textsc{def.m.sg} sorcery and be\_afraid.\textsc{prs.3pl} but say.\textsc{inf} can.\textsc{prs.3pl} \textsc{3pl} nothing\\
\glt `The animals see the sorcery and are afraid, but they cannot say anything.' (Cavòrgja, \citealt[125]{Büchli1966})
\z

Example (\ref{ex:focpcl1}) shows the movement of the \isi{particle} \textit{cùntar} `towards' out of the particle verb \textit{prèndar ancùntar} `receive' (calqued on German \textit{entgegennehmen}).
\ea
\label{ex:focpcl1}
\gll Alṣò \textbf{ancùntar} prandès `l tùt ùsa?\\
well towards take.\textsc{cond.3sg} \textsc{3sg} all	now\\
\glt `Well, would he receive everything now?' (Sadrún, m5)
\z

If an \isi{indirect interrogative clause} is focused, it may (\ref{ex:focclause1}) or may not be resumed by a \isi{demonstrative pronoun} (\ref{ex:focclause2}). In the case of (\ref{ex:focclause1}), the \isi{demonstrative pronoun} used is \textit{gljèz}.

\ea
\label{ex:focclause1}
\gll Ins vèz’ aun tg’ èra dau vidajn pùntgas né trádals; \textbf{sch}’ \textbf{i} \textbf{sitavan} \textbf{gljèz} sau bétg.\\
\textsc{gnr} see.\textsc{prs.3sg} still \textsc{comp} \textsc{pass.aux.impf.3sg} give.\textsc{ptcp.unm} into chisel.\textsc{f.pl} or power\_drill.\textsc{m.pl} whether \textsc{3pl} blow\_up.\textsc{impf.3pl} \textsc{dem.unm} know.\textsc{prs.1sg.1sg}  \textsc{neg}\\
\glt `One still can see that chisels or power drills had been used; whether they would blow up I don’t know.' (Sadrún, m4, l. \sectref{sec:8.3})
\z

\ea
\label{ex:focclause2}
\gll   Aah, \textbf{tgé} ... \textbf{prandévan} \textbf{pròpi} \textbf{òra} sa ins bégj éxáct [...]. \\
ah what {} take.\textsc{impf.3pl} exactly out know.\textsc{prs.3sg} \textsc{gnr} \textsc{neg} exactly\\
\glt `Ah, what … they really mined, one does not know exactly [...].' (Sadrún, m4, \sectref{sec:8.3})
\z

If the focus is on a \isi{noun}, a \isi{pronoun} referring to it must be left in the background clause (\ref{ex:focnoun1}).

\ea
\label{ex:focnoun1}
\gll  \textbf{Las} \textbf{nòtízjas} sa ju bétg danù̱ndar als gjaniturs, als dus baps prandèvan \textbf{aj} [...]. \\
\textsc{def.f.pl} news.\textsc{pl} know.\textsc{prs.1sg} \textsc{1sg} \textsc{neg} from\_where \textsc{def.m.pl} parents.\textsc{pl} \textsc{def.m.pl} two.\textsc{m.pl} father.\textsc{pl} take.\textsc{impf.3pl} \textsc{3sg}\\
\glt `I don’t know where my parents had the news from, the two fathers took them, there was in Rueras, there was [only] one who had a radio.' (Sadrún, m1, \sectref{sec:8.2})
\z

Contrastive focus is done by intonation; the focused constituents remain in their place according to their syntactic function (\ref{ex:foccontrnoun1} and \ref{ex:foccontrnoun2}).

\ea
\label{ex:foccontrnoun1}
\gll \textbf{Gjòn} ò angulau la gaglina, bigja Maria.\\
\textsc{pn} have.\textsc{prs.3sg} steal.\textsc{ptcp.unm} \textsc{def.f.sg} hen \textsc{neg} \textsc{pn}\\
\glt `It is Gion who stole the hen, not Maria.' (Sadrún, m5)
\z

\ea
\label{ex:foccontrnoun2}
\gll Èl vut dá in cùdisch \textbf{da} \textbf{Gjòn}, bigja da Maria.\\
\textsc{3sg.m} want.\textsc{prs.3sg} give.\textsc{inf} \textsc{indef.m.sg} book \textsc{dat} \textsc{pn} \textsc{neg} \textsc{dat} \textsc{pn}\\
\glt `It is to Gion that he wants to give a book, not to Maria.' (Sadrún, m4)
\z







