\chapter{Typological overview}
\label{chap:2}

This chapter is a~brief overview covering the most central features of Palula. For more in"=depth coverage of each topic, and for information on those not explicitly covered here, the later chapters will need to be consulted.


\section{Phonology}
\label{sec:2-1}


With its 32--37 members, the Palula consonant inventory (\tabref{tab:2-cons}) is moderately large to large \citep{maddieson_consonant_2013}. There are five basic places of articulation (labial, dental, retroflex, palatal and velar), with a~voicing contrast in the plosive and fricative sets, and an~aspiration contrast in the plosive and affricate sets. 


\begin{table}
\caption{Palula consonants}
\begin{tabularx}{\textwidth}{ Q Q Q Q Q Q Q }
\lsptoprule
p &
t &
ʈ &
&
k &
(q) &
\\
pʰ &
tʰ &
ʈʰ &
&
kʰ &
&
\\
b &
d &
ɖ &
&
ɡ &
&
\\
&
ʦ &
ʈʂ &
ʨ &
&
&
\\
&
(ʦʰ) &
(ʈʂʰ) &
ʨʰ &
&
&
\\
(f) &
s &
ʂ &
ɕ &
x &
&
h\\
&
z &
(ʐ) &
ʑ &
ɣ &
&
\\
m &
n &
ɳ &
&
&
&
\\
&
r &
ɽ &
&
&
&
\\
&
l &
&
&
&
&
\\
w &
&
&
j &
&
&
\\\lspbottomrule
\end{tabularx}
\label{tab:2-cons}
\end{table}


Palula has ten phonemic vowels, comprising five basic qualities, each having a~long and a~short counterpart. This inventory (\tabref{tab:2-voc}) forms a~symmetrical and typologically common system. Vowel nasalisation is a~marginal, possibly emerging, feature in the language, but not so far fully contrastive. 


\begin{table}[t]
\caption{Palula vowels}
\begin{tabularx}{.4\textwidth}{ Q Q Q Q Q Q Q Q Q Q }
\lsptoprule
iː &
i &
&
&
&
&
&
&
uː &
u\\
&
&
eː &
e &
&
&
oː &
o &
&
\\
&
&
&
&
aː &
a &
&
&
&
\\\lspbottomrule
\end{tabularx}
\label{tab:2-voc}
\end{table}

\largerpage
The language has a~complex syllable structure \citep{maddieson_syllable_2013}, permitting three consonants in the onset position and two in the coda position (although a limited number of consonant combinations are permitted before or after the vowel nucleus, as shown in \tabref{tab:2-clusters}). There is a~tendency to drop the final consonant in word final clusters.


\begin{table}[b]
\caption{Word boundary syllable clusters (vd = voiced; vl = voiceless)}
\begin{tabularx}{\textwidth}{ l l l l}
\lsptoprule
Types &
Combinations &
Examples \\\midrule
CCC- &
vd plosive + /r/ + /h/&
/ɡrheːɳɖ/ &
`knot'\\
&
vd plosive/nasal + approximant + /h/&
/ˈdjhuːɽi/ &
`granddaughter'\\\\
CC- &
plosive/nasal + /r/ &
/kraːm/ &
`work'\\
&
consonant + approximant &
/ˈswaːnu/ &
`is sleeping'\\
&
vd consonant + /h/ &
/lhoːɳ/ &
`salt'\\\\
-CC &
nasal + consonant &
/ɡrhoːnk/ &
`worm'\\
&
vl fricative + vl plosive (/s + t/, /ʂ + ʈ/) &
/ɡhoːʂʈ/ &
`house'\\
&
/t/ + /r/ &
/putr/ &
`son'\\\lspbottomrule
\end{tabularx}
\label{tab:2-clusters}
\end{table}


Main stress falls on the final or the penultimate syllable of the lexical root. One of the vocalic moras of the stressed
syllable receives pitch accent, phonetically realised as: a) high level or high falling on a~short vowel
[\'{}], represented in this work as \textit{á} (in polysyllabic words, elsewhere no marking); b)
low rising on a~long vowel [\v{}], represented as \textit{aá}; or c) high falling on a~long vowel [\^{}],
represented as \textit{áa}. Pitch accent is contrastive, as illustrated in (\ref{ex:2-a}).


\begin{exe}
\extab
\label{ex:2-a}
\begin{tabularx}{116mm}{ l l l l l }
/seːtí/ (\textit{seetí})  &
vs &
/sêːti/ (\textit{séeti}) \\
`looked after' &
&
`thigh' \\
/děːdi/ (\textit{deédi})  &
vs &
/dêːdi/ (\textit{déedi})  \\
`burnt \textsc{f}' &
&
`grandmother' \\
/hár/ (\textit{har})  &
vs &
/hǎːr/ (\textit{haár})  &
vs &
/hâːr/ (\textit{háar}) \\
`every' &
&
`defeat' &
&
`take away!'\\
\end{tabularx}
\end{exe}


\section{Morphology}
\label{sec:2-2}

Palula morphology is suffixing, and formatives are almost exclusively concatenative \citep{bickel_fusion_2013}, with a~moderately high degreee of synthesis \citep{bickel_inflectional_2013}. 


Nouns are inflected for number (singular, plural) and case (nominative, oblique, genitive). In most of the declensional classes, nominative plural and oblique singular are cumulated into a~single formative \citep{bickel_exponence_2013}. The genitive (at least in the plural) can be analysed as suffixed to the oblique rather than to the nominative stem. The noun exemplified in \tabref{tab:2-nouns} is \textit{ṣinɡ} `horn'.

\begin{table}[ht]
\caption{Inflection of nouns}
\begin{tabularx}{.5\textwidth}{ Q Q Q }
\lsptoprule
&
Singular &
Plural
\\\midrule
Nominative &
\textit{ṣinɡ} &
\textit{ṣínɡ-a}\\
Oblique &
\textit{ṣínɡ-a} &
\textit{ṣínɡ-am}\\ 
Genitive &
\textit{ṣínɡ-ii} &
\textit{ṣínɡ-am"=ii}
\\\lspbottomrule
\end{tabularx}
\label{tab:2-nouns}
\end{table}


There are three main functions of the oblique case of nouns: a) as the transitive subject in the perfective (i.e., as an ergative case marker); b) as the form to which postpositions are added; and c) as a~locative. A number of other case"=like functions (such as recipients) and more peripheral arguments appear as postpositional phrases.


Palula displays core"=case asymmetry \citep{iggesen_asymmetrical_2013}, within the category of nouns as well as for NPs in general (more on pronouns below). While most nouns (those belonging to the two major \textit{a}- and \textit{i}-declensions) make a~nominative"=oblique distinction, one declensional class (the \textit{m}-declension) in particular does not make this distinction at all, whereas some of the pronouns make an~even more fine"=tuned nominative"=accusative"=oblique distinction, as seen in \tabref{tab:2-case}.


\begin{table}[ht]
\caption{Core case distinctions}
\begin{tabularx}{\textwidth}{ l@{\hspace{25pt}} Q Q Q Q }
\lsptoprule
&
`man' &
`sister' &
`woman' &
\textsc{3sg}
\\\midrule
Nominative &
\textit{míiš} &
\textit{bheéṇ} &
\textit{kúṛi} &
\textit{so} \\
Accusative &
\textit{míiš} &
\textit{bheéṇ} &
\textit{kúṛi} &
\textit{tas} \\
Oblique (=ergative) &
\textit{míiš-a} &
\textit{bheeṇ-í} &
\textit{kúṛi} &
\textit{tíi}
\\\lspbottomrule
\end{tabularx}
\label{tab:2-case}
\end{table}


Palula has a~fairly typical Indo"=European two"=gender system, which is primarily sex"=based \citep{corbett_sex-based_2013}. A noun is either masculine or feminine, a~property established through morphological agreement. The basis for gender assignment is semantic as well as formal \citep{corbett_systems_2013}.


The pronoun system proper (i.e., 1st and 2nd person) is interesting in that it makes more distinctions in the plural than in the singular (\tabref{tab:2-pron}), as there are dedicated ergative case forms only in the plural.\footnote{The forms \textit{ma} and \textit{tu} are glossed throughout this work as nominative and \textit{míi} and \textit{thíi} as genitive, regardless of their functions in the clause.}

\begin{table}[t]
\caption{Pronominal case distinctions}
\begin{tabularx}{\textwidth}{ l Q Q Q Q }
\lsptoprule
&
Nominative &
Accusative &
Genitive &
Ergative
\\\midrule
\textsc{1sg} &
\textit{ma} &
\textit{ma} &
\textit{míi} &
\textit{míi} \\
\textsc{2sg} &
\textit{tu} &
\textit{tu} &
\textit{thíi} &
\textit{thíi} \\
\textsc{1pl} &
\textit{be} &
\textit{asaám} &
\textit{asíi} &
\textit{asím} \\
\textsc{2pl} &
\textit{tus} &
\textit{tusaám} &
\textit{tusíi} &
\textit{tusím} 
\\\lspbottomrule
\end{tabularx}
\label{tab:2-pron}
\end{table}


The demonstratives, which are used as third"=person pronouns, essentially make the same case distinctions as the plural personal pronouns (although there are other uses of the oblique apart from its ergative function). Additionally, they display gender distinctions (in the nominative singular) as well as a~three"=way deictic contrast (\tabref{tab:2-dem}).


\begin{table}[htb]
\caption{Demonstrative distinctions}
\begin{tabularx}{\textwidth}{ l l l Q Q Q Q }
\lsptoprule
&
&
&
Nominative &
Accusative &
Genitive &
Oblique \\\midrule
Proximal &
\textsc{sg} &
\textsc{m} &
\textit{nu} &
\textit{nis} &
\textit{nisíi} &
\textit{níi} \\
&
&
\textsc{f} &
\textit{ni} &
\textit{nis} &
\textit{nisíi} &
\textit{níi} \\
&
\textsc{pl} &
&
\textit{ni} &
\textit{ninaám} &
\textit{niníi} &
\textit{niním} \\
Distal &
\textsc{sg} &
\textsc{m} &
\textit{lo} &
\textit{las} &
\textit{lasíi} &
\textit{líi} \\
&
&
\textsc{f} &
\textit{le} &
\textit{las} &
\textit{lasíi} &
\textit{líi} \\
&
\textsc{pl} &
&
\textit{le} &
\textit{lanaám} &
\textit{laníi} &
\textit{laním} \\
Remote &
\textsc{sg} &
\textsc{m} &
\textit{so} &
\textit{tas} &
\textit{tasíi} &
\textit{tíi} \\
&
&
\textsc{f} &
\textit{se} &
\textit{tas} &
\textit{tasíi} &
\textit{tíi} \\
&
\textsc{pl} &
&
\textit{se} &
\textit{tanaám} &
\textit{taníi} &
\textit{taním} \\\lspbottomrule
\end{tabularx}
\label{tab:2-dem}
\end{table}


Adjectives are inflected for agreement in gender (masculine, feminine), number (singular, plural) and case (nominative, non"=nominative). The adjective in \tabref{tab:2-adj} is \textit{paṇáaru} `white'.


\begin{table}[t]
\caption{Inflection of adjectives}
\begin{tabularx}{\textwidth}{ l l l Q }
\lsptoprule
&
Masculine singular &
Masculine plural &
Feminine\\\midrule
Nominative &
\textit{paṇáaru} &
\textit{paṇáara} &
\textit{paṇéeri} \\
Non"=nominative &
\textit{paṇáara} &
\textit{paṇáara} &
\textit{paṇéeri} \\\lspbottomrule
\end{tabularx}
\label{tab:2-adj}
\end{table}


Finite verbs are inflected for tense("=aspect), mood (in a~limited sense) and agreement in a)
gender/number, \textit{or} b) person (the type of agreement expressed depending on tense-aspect, see \sectref{sec:2-3}
below). There are also some non"=finite forms. For the sake of a~more economical presentation that
takes verbs of different inflectional classes into account, all verbs are analysed as having
a~perfective and a~non"=perfective stem. The verb in \tabref{tab:2-verb} is \textit{til-} `walk'.


\begin{table}[ht]
\caption{Inflection of verbs}
\begin{tabularx}{\textwidth}{ l l Q Q }
\lsptoprule
&
&
Singular &
Plural \\\midrule
\textbf{Non"=perfective stem} &
&
&
\\
Present &
\textsc{m} &
\textit{til-áan-u} &
\textit{til-áan-a} \\
&
\textsc{f} &
\textit{til-éen-i} &
\textit{til-éen"=im} \\
Future &
1 &
\textit{tíl-um} &
\textit{til-íia} \\
&
2 &
\textit{tíl-aṛ} &
\textit{tíl-at} \\
&
3 &
\textit{tíl-a} &
\textit{tíl-an} \\
Imperative &
&
\textit{tíl} &
\textit{tíl-ooi} \\
Infinitive &
&
\textit{til-áai} &
\\
Converb &
&
\textit{til-í} &
\\
Obligative &
&
\textit{til"=eeṇḍeéu} &
\\
Copredicative participle &
&
\textit{til-íim} &
\\
Verbal noun &
&
\textit{til"=ainií} &
\\
Agentive verbal noun &
\textsc{m} &
\textit{til-áaṭ-u} &
\textit{til-áaṭ-a} \\
&
\textsc{f} &
\textit{til-éeṭ-i} &
\textit{til-éeṭ-im} \\
\textbf{Perfective stem} &
&
&
\\
Perfective &
\textsc{m} &
\textit{tilíl-u} &
\textit{tilíl-a} \\
&
\textsc{f} &
\textit{tilíl-i} &
\textit{tilíl-im} \\\lspbottomrule
\end{tabularx}
\label{tab:2-verb}
\end{table}


\section{Syntax}
\label{sec:2-3}
Four of the most frequently occurring TMA"=categories (Future, Present, Simple Past (=perfective) and Imperative) make use of inflectional morphology only (as displayed in \tabref{tab:2-verb}).\footnote{Language"=specific, and functionally defined, verbal categories are capitalised to set them apart from grammatical terms applied cross"=linguistically. See \chapref{chap:9} for details.} Another three basic TMA"=categories (Past Imperfective, Perfect and Pluperfect) are expressed periphrastically, by adding auxiliaries to inflected verb forms (as shown in \tabref{tab:2-peri}).


\begin{table}[ht]
\caption{Periphrastically formed TMA"=categories}
\begin{tabularx}{\textwidth}{ l l Q Q }
\lsptoprule
TMA"=category &
Inflectional category &
Auxiliary &
Example \\\midrule
Past Imperfective &
future &
\textsc{pst} &
\textit{tíl-um de} \\
Perfect &
perfective &
`be.\textsc{prs-agr}' &
\textit{tilíl-u hín-u} \\
Pluperfect &
perfective &
\textsc{pst} &
\textit{tilíl-u de} \\\lspbottomrule
\end{tabularx}
\label{tab:2-peri}
\end{table}


There is relatively little synchronically productive derivational morphology in the language, but a~productive process for deriving verbs from other categories from within the language as well as from entirely novel or non"=native elements is the use of verbalisers such as \textit{the-} `do' and \textit{bhe-} `become'. Some examples are shown in (\ref{ex:2-b}).


\begin{exe}
\extab
\label{ex:2-b}
\begin{tabularx}{116mm}{ l l l l l l l}
\textit{madád} &
`help' &
+ &
\textit{the-} &
{\textgreater} &
\textit{madád thíili} &
`helped'\\
\textit{tanɡ} &
`narrow' &
+ &
\textit{the-} &
{\textgreater} &
\textit{tanɡ thíilu} &
`troubled'\\
\textit{milaáu} &
`joined' &
+ &
\textit{bhe-} &
{\textgreater} &
\textit{milaáu bhílu} &
`met'\\
\textit{ašáq} &
`loving' &
+ &
\textit{bhe-} &
{\textgreater} &
\textit{ašáq bhílu} &
`fell in love with'\\
\end{tabularx}
\end{exe}


Word order in Palula is typically head"=final (\tabref{tab:2-worder}). This is seen in the word order in noun phrases (determiner--noun, adjective--noun, numeral--noun, genitive--noun), adjective phrases (adjunct--adjective) and in adpositional phrases (noun phrase--postposition). As far as entire clauses are concerned, the word order (or rather constituent order) is more flexible, but the most frequent and pragmatically unmarked order is intransitive subject--verb, transitive subject--verb, and direct object--verb.


\begin{table}[ht]
\caption{Word order features}
\begin{tabularx}{\textwidth}{ l l Q }
\lsptoprule
Order &
Example\\\midrule
\textbf{Determiner}--noun &
\textit{\textbf{eesó} ḍhínɡar} &
`that wood' \\
\textbf{Adjective}--noun &
\textit{\textbf{paṇéeri} déeṛi} &
`white beard' \\
\textbf{Numeral}--noun &
\textit{\textbf{páanǰ} toobakí} &
`five rifles'\\
\textbf{Genitive}--noun &
\textit{\textbf{íṇc̣ii} rhaíi} &
`bear's footprints'\\
\textbf{Adjunct}--adjective &
\textit{\textbf{bíiḍi} dhríɡi} &
`very long'\\
\textbf{NP}--adposition &
\textit{\textbf{míi putrá} sanɡí} &
`with my son'\\
\textbf{S}--V &
\textit{\textbf{raaǰaá} múṛu} &
`The king died.'\\
\textbf{A}--V &
\textit{\textbf{tíi} áa ḍáaɡ mheerílu} &
`He killed a markhor.' \\
\textbf{O}--V &
\textit{tíi \textbf{áa ḍáaɡ} mheerílu} &
`He killed a markhor.' 
\\\lspbottomrule
\end{tabularx}
\label{tab:2-worder}
\end{table}


As far as alignment is concerned, Palula displays an~intricate split system. In the perfective categories (Simple Past, Perfect and Pluperfect), the pattern is essentially ergative, as seen in example (\ref{ex:2-1}), with a~non"=nominatively marked agent"=subject and verbal agreement with the feminine direct object. In the non"=perfective categories (Future, Present and Past Imperfective), in contrast, it is essentially accusative, which can be observed from the nominatively marked agent"=subject in (\ref{ex:2-2}), which is also the NP that the transitive verb agrees with in gender and number.

\begin{exe}
\ex
\label{ex:2-1}
\glll íṇc̣-a čhéeli khéel-i \\
bear[\textsc{msg}]-\textsc{obl} she.goat[\textsc{fsg}] eat.\textsc{pfv-}\textsc{f}\\
\textbf{A} \textbf{O} \textbf{V}\\
\glt `The bear ate the goat.' (A:PAS056)
\end{exe}


\begin{exe}
\ex
\label{ex:2-2}
\glll iṇc̣ áaṇc̣-a kha-áan-u\\
bear[\textsc{msg}] raspberry[\textsc{m}]-\textsc{pl} eat-\textsc{prs-msg} \\
\textbf{A} \textbf{O} \textbf{V}\\
\glt `The bear is eating raspberries.' (A:KAT145)
\end{exe} 


Agreement is part of all finite verb forms, but the particular agreement features realised are related to tense"=aspect. In Future and Past Imperfective, the verb agrees with its target in person (and number), as in (\ref{ex:2-3}), whereas in Present and the categories based on the perfective, the verb agrees in gender and number, as can be seen in (\ref{ex:2-4}).

\begin{exe}
\ex
\label{ex:2-3}
\glll so múṛee ǰand-óo de \\
\textsc{3msg.nom} dead.person.\textsc{pl} make.alive-\textsc{3sg} \textsc{pst} \\
\textbf{A} \textbf{O} \textbf{V} \\
\glt `He was resurrecting the dead.' (A:ABO034)
\end{exe}

\begin{exe}
\ex
\label{ex:2-4}
\glll táapaṛ-a túuri íṇc̣-a čhéeli ɡhašíl-i hín-i \\ 
hill-\textsc{obl} below bear[\textsc{msg}]-\textsc{obl} she.goat[\textsc{f}] catch.\textsc{pfv-f} be.\textsc{prs-f} \\
{} {} \textbf{A} \textbf{O} \textbf{V} \\
\glt `Below the hill, the bear has captured the goat.' (A:PAS054)
\end{exe}


Alignment in the realm of verbal alignment is summarized in \tabref{tab:2-verbagr}. 


\begin{table}[t]
\caption{Alignment: Verbal agreement}
\begin{tabularx}{\textwidth}{ l l Q Q }
\lsptoprule
Aspect &
TMA"=category &
Agreement features &
Controller \\\midrule
Non"=perfective &
Future, Past Imperfective &
Person/number &
S, A \\
&
Present &
Gender/number &
S, A \\
&
Imperative &
Number &
S, A \\
Perfective &
&
Gender/number &
S, O \\\lspbottomrule
\end{tabularx}
\label{tab:2-verbagr}
\end{table}

\largerpage[-1]
Several NP splits further complicate the picture. Apart from the singling out of the transitive subject (A) in the perfective (\textit{asím} in (\ref{ex:2-2a})), we also have pronominal forms particular to the direct object (O) (\textit{asaám} in (\ref{ex:2-2b})), both of them different from the form used as the subject (S) of an intransitive clause (\textit{be} in (\ref{ex:2-2c})).

\begin{exe}
\ex
\label{ex:2-2a}
\gll \textbf{asím} ǰinaazá khaṣeel-í wheelíl-u de \\
\textsc{1pl.erg} corpse drag-\textsc{cv} take.down.\textsc{pfv"=msg} \textsc{pst}  \\
\glt `We dragged the corpse down.' (A:GHA044)

\ex
\label{ex:2-2b}
\gll nu ba \textbf{asaám} mhaaranií the ukháat-u de \\
\textsc{3sg.prox.nom} \textsc{top} \textsc{1pl.acc} kill.\textsc{vn} to come.up.\textsc{pfv"=msg} \textsc{pst}  \\
\glt `He has come up here to kill us.' (A:HUA071)

\ex
\label{ex:2-2c}
\gll rhootašíi-a \textbf{be} ɡíia  \\
morning-\textsc{obl} \textsc{1pl.nom} go.\textsc{pfv.pl}  \\
\glt `In the morning we left.' (A:GHA006)
\end{exe}


Alignment in the realm of case marking is summarized in \tabref{tab:2-casealign}. This is a somewhat simplified overview in that some minor noun declensions (in which some case distinctions are upheld only in the plural) are not included.


\begin{table}[t]
\caption{Alignment: Case marking}
\begin{tabularx}{\textwidth}{ l l Q }
\lsptoprule
NP"=type &
Aspect &
Case differentiation\\\midrule
Nouns: \textit{a}- and \textit{i}-declensions &
Non"=perfective &
A=S=O \\
&
Perfective &
A≠S=O \\
Nouns: \textit{m}-declension &
&
A=S=O \\
Pronouns: \textsc{3sg, 1pl, 2pl, 3pl} &
Non"=perfective &
A=S≠O \\
&
Perfective &
A≠S≠O \\  
Pronouns: \textsc{1sg, 2sg} &
Non"=perfective &
A=S=O \\
&
Perfective &
A≠S=O
\\\lspbottomrule
\end{tabularx}
\label{tab:2-casealign}
\end{table}


Sentences lacking an~overt copula are allowed, and for predicate nominals in the Present tense, as the one shown in (\ref{ex:2-5}), they are typical.

\begin{exe}
\ex
\label{ex:2-5}
\gll míi báabu áak zamindaár míiš \\ 
\textsc{1sg.gen} father \textsc{idef} farmer man \\
\glt `My father is a~farmer.' (A:OUR002)
\end{exe}
Although it is possible to conjoin clauses with a~conjunctive suffix (also used for conjoining noun phrases), other strategies are preferred, such as juxtaposition for symmetrical clauses or the overwhelmingly favoured Converb construction, exemplified in (\ref{ex:2-6}), which is used for a~great variety of same"=subject clause combinations.

\begin{exe}
\ex
\label{ex:2-6}
\gll tíi ba {\ob}bhun wha-í ba{\cb} {\ob}so mhaás muṭ-í bhun wheel-í ba{\cb} {\ob}teeṇíi ɡhooṣṭ-á the ɡhin-í{\cb} ɡáu \\
\textsc{3sg.obl} \textsc{top} down come.down-\textsc{cv} \textsc{top} \textsc{def.msg.nom} meat tree-\textsc{gen} down take.down-\textsc{cv} \textsc{top} \textsc{refl} house-\textsc{obl} to take-\textsc{cv} go.\textsc{pfv.msg} \\
\glt `He came down [having come down], took down the meat from the tree [having taken down the meat from the tree], and brought it to his house.' (B:SHB762)
\end{exe}

In complex constructions, the unmarked order is a~complement clause followed by (or embedded in) the main clause, as in (\ref{ex:2-7}), and similarly an~adverbial clause followed by (or, again, embedded in) the main clause, as in (\ref{ex:2-8}). 

\begin{exe}
\ex
\label{ex:2-7}
\gll \textbf{neečíir} \textbf{theníi-e} díiš-e xalk-íim xwaaíš thíil-i \\
	hunt do.\textsc{vn"=gen} village\textsc{-gen} people\textsc{-pl.obl} desire do.\textsc{pfv-f} \\
\glt `People in the village wanted to go hunting.' (B:AVA200)
\end{exe}

\begin{exe}
\ex
\label{ex:2-8}
\gll \textbf{raaǰaá} \textbf{múṛ-u} \textbf{ta} putr-óom tasíi hukumát bulooṣṭéel-i \\
	king die.\textsc{pfv"=msg} \textsc{sub} son\textsc{-pl.obl} \textsc{3sg.gen} government snatch.\textsc{pfv-f} \\
\glt `When the king died, the sons seized the power.' (A:MAB003)
\end{exe}

However, a~post"=posed construction with the complementiser \textit{ki} is also commonly used (\ref{ex:2-9}), especially for utterance complements.

\begin{exe}
\ex
\label{ex:2-9}
\gll ɡhueeṇíi-am maníit-u ki \textbf{ni} \textbf{bíiḍ-a} \textbf{zinaawúr} \textbf{xálaka} \textbf{hín-a}\\
	Pashtun-\textsc{pl.obl} say.\textsc{pfv-msg} \textsc{comp} \textsc{3pl.prox.nom} much-\textsc{mpl} wild people be.\textsc{prs"=mpl} \\
\glt `The Pashtuns said, ``These are very wild people.''' (A:CHA008)
\end{exe}

Polar interrogatives are formed with a~clitical sentence"=final question particle \textit{ee} (B \textit{aa}), as in (\ref{ex:2-10}), whereas an~indefinite"=interrogative pronoun (or other proform), such as \textit{kasée} (B) `whose' in (\ref{ex:2-11}), is used in content interrogatives. 

\begin{exe}
\ex
\label{ex:2-10}
\gll ux-á díi khooǰóol-u ki tu insaán\textbf{=ee}\\
	camel-\textsc{obl} from ask.\textsc{pfv"=msg} \textsc{comp} \textsc{2sg.nom} human.being=\textsc{q} \\
\glt `He asked the camel, ``Are you a~man?''' (A:KIN007)
\end{exe}


\begin{exe}
\ex
\label{ex:2-11}
\gll aní \textbf{kasée} ziaarat-í thaní \\
	\textsc{prox.3pl.nom} whose shrine-\textsc{pl} \textsc{quot} \\
\glt `Whose shrines are these?' (B:FOR026)
\end{exe}

Negation is formed with a~separate and invariable negative particle \textit{na}, preceding the predicate (\ref{ex:2-12}).

\begin{exe}
\ex
\label{ex:2-12}
\gll muṣṭúk-a xálak-a dhii-á díi \textbf{na} khooǰ-óon de \\
	of.past-\textsc{mpl} people-\textsc{pl} daughter-\textsc{obl} from \textsc{neg} ask-\textsc{3pl} \textsc{pst} \\
\glt `People in the old days were not asking their daughter [who she wanted to marry].' (A:MAR018)
\end{exe}