\chapter{Phonology}
\label{chap:3}

\section{Consonants}
\label{sec:3-1}

\subsection{Consonant inventory}
\label{subsec:3-1-0}

\begin{table}[b]
\caption{Inventory of consonants (IPA). Marginal or doubtful phonemes within parentheses}
\begin{tabularx}{\textwidth}{ P{20mm} Q Q Q Q Q Q Q Q }
\lsptoprule
&
\rotatehead{Labial} &
\rotatehead{Dental} &
\rotatehead{Retroflex} &
\rotatehead{Palatal} &
\rotatehead{Velar} &
\rotatehead{Postvelar} &
\rotatehead[1.2cm]{Glottal} \\\midrule
Plo\-sive
&
p &
t &
ʈ &
&
k &
(q) &
\\
&
pʰ &
tʰ &
ʈʰ &
&
kʰ &
&
\\
&
b &
d &
ɖ &
&
ɡ &
&
\\
Affricate
&
&
ʦ &
ʈʂ &
ʨ &
&
&
\\
&
&
(ʦʰ) &
(ʈʂʰ) &
ʨʰ &
&
&
\\
Fricative
&
(f) &
s &
ʂ &
ɕ &
x &
&
h\\
&
&
z &
(ʐ) &
ʑ &
ɣ &
&
\\
Nasal
&
m &
n &
ɳ &
&
&
&
\\
Flap
&
&
r &
ɽ &
&
&
&
\\
\parbox[t]{2cm}{Lateral\\[-.3em]approximant}
&
&
l &
&
&
&
&
\\
\mbox{Approximant\hspace*{-2mm}}
&
w &
&
&
j &
&
&
\\\lspbottomrule
\end{tabularx}
\label{tab:3-1}
\end{table}

\largerpage[-1]
The consonant inventory is rather symmetrical, with the dental and retroflex places of articulation displaying the most well"=developed system of manner contrasts. The ancient (OIA) contrast between the three sibilants /s ʂ ɕ/ is preserved \citep[375]{cardonaluraghi2009}, with the present voicing contrast probably not evolving until quite recently, partly through lenition of voiced affricates, partly through foreign loans. 

While the plosive and fricative sets show a~contrast in voicing except for in the (marginal) labial and glottal places of articulation, voiced counterparts are missing in the small affricate set. Mirroring that is a~general allophonic variation (see below) between voiced fricative and affricate pronunciations. The voiced palatal fricative could equally well be treated as an~affricate, as that is the more common allophone (especially in the A dialect), but to provide more symmetry to the system, I have chosen to include it among the voiced fricatives,\footnote{As pointed out by \citet[34]{zoller2005}, this particular asymmetry within the affricate/fricative sets is a~feature shared by a~number of languages of northern Pakistan, due to a ``lenition process which is more advanced in case of the voiced phonemes than in case of the voiceless phonemes''.} while in the common transcription it is represented as \textit{ǰ}.


The post"=velar (or uvular) place of articulation is represented by a~voiceless post"=velar or uvular plosive /q/ alone. This marginal phoneme is only pronounced distinctly post"=velar by some educated speakers~-- and even then rather inconsistently~-- when occurring in loanwords of mainly Perso"=Arabic origin. In many speakers' pronunciation, however, it normally tends to approximate a~velar fricative pronunciation [x], thus not contrasting with the phoneme /x/. The fricatives /z, x, ɣ/ are rather frequent in present"=day Palula, and many of the words probably have a~long history in the language, although they to a~large extent are found in vocabulary borrowed from languages in the immediate region, and to a~much lesser extent are found in inherited vocabulary affected by phonological processes. A labio"=dental [f] is sometimes heard in more recent loans, primarily from Urdu and English, but with many speakers it alternates freely with or is entirely replaced by the native voiceless aspirated bilabial plosive /pʰ/, hence /f/ is considered a~marginal phoneme. 


The voiced retroflex fricative /ʐ/ is also a~marginal phoneme, but it is included for comparative reasons; an~even more rarely occurring voiced retroflex affricate sound [ɖʐ] is tentatively analysed as an~allophone of the same phoneme. 


There is insufficient proof to regard a~velar nasal [ŋ] as a~phoneme independent from /n/, as it only occurs before /k/ and /ɡ/, or as a~variant pronunciation of /nɡ/: [ŋɡ]$\sim$[ŋ]. 


Although initial findings identified several voiced aspirated consonants, later observations favoured a~cluster analysis, e.g., /b/ followed by /h/ rather than a~phoneme /bʰ/. However, it should be noted that voiceless aspirated sounds do share some characteristics with (frequently occurring) clusters of voiced consonants and /h/, as will be further discussed in \sectref{subsec:3-4-1}.

\subsection{Distribution and variation}
\label{subsec:3-1-1}

Examples of the distribution of consonants are shown in \tabref{tab:3-2}. 


The retroflex consonants are in some descriptions called ``retracted'' \citep[16]{schmidtkohistani2008} or ``cerebrals'' \citep{morgenstierne1941}; it has been questioned whether these consonants in HKIA languages are retroflex in the same sense or to the same extent as in the main NIA languages or in Dravidian languages. 


\begin{sidewaystable}[p!]
\caption{The distribution of consonants: word"=initial, medial, and final.{\protect\footnotemark}
  The occurrences within parentheses are matters of interpretation (see \sectref{subsec:3-2-3},
  \sectref{subsec:3-4-1})}
\begin{tabularx}{\textwidth}{ l@{\hspace{20pt}} l@{\hspace{20pt}} Q l@{\hspace{20pt}} Q Q Q Q }
\lsptoprule
/p/ &
&
/piːli/ &
`drank (\textsc{f})' &
/ɕopu/ &
`navel' &
/ʈip/ &
`drop'\\
/pʰ/ &
\textit{(ph)} &
/pʰoː/ &
`boy' &
/aɖapʰaːr/ &
`halfways' &
-- &
\\
/b/ &
&
/biːɖi/ &
`many (\textsc{f})' &
/ʑabal/ &
`iron bar' &
/ɖaːb/ &
`plain'\\
/f/ &
&
/fasil/ &
`crop' &
/xafa/ &
`upset' &
/muaːf/ &
`excuse'\\
/m/ &
&
/miːɕa/ &
`men' &
/hiːmaːl/ &
`glacier' &
/braːm/ &
`joint'\\
/w/ &
&
/wiːwaj/ &
`wife's brother' &
/heːwaːnd/ &
`winter' &
(/ɡhaːw/ &
`cow')\\
/r/ &
&
/reːti/ &
`nights' &
/beːriʂ/ &
`summer' &
/anɡoːr/ &
`fire'\\
/l/ &
&
/leːdi/ &
`found (\textsc{f})' &
/baliː/ &
`roof end' &
/ʨʰaːl/ &
`goat kid'\\
/t/ &
&
/teːti/ &
`hot (\textsc{f})' &
/pʰutu/ &
`fly' &
/baːt/ &
`word'\\
/tʰ/ &
\textit{(th)} &
/tʰuːɳi/ &
`pillar' &
/sutʰaːn/ &
`trousers' &
-- &
\\
/d/ &
&
/deːdi/ &
`father's mother' &
/leːdi/ &
`found (\textsc{f})' &
/ɕid/ &
`coldness'\\
/n/ &
&
/neːɽi/ &
`stream bed' &
/ʑaːnu/ &
`person' &
/soːn/ &
`pasture'\\
/s/ &
&
/seːti/ &
`looked after (\textsc{cv})' &
/buːsi/ &
`kiss' &
/deːs/ &
`day'\\
/z/ &
&
/zeːri/ &
`supplication' &
/baːzoːr/ &
`bazaar' &
/anɡreːz/ &
`Brit'\\
/ʦ/ &
\textit{(ts)} &
/ʦiːpi/ &
`squeezed (\textsc{cv})' &
/buʦu/ &
`stick' &
/uʦ/ &
`spring'\\
/ʦʰ/ &
\textit{(tsh)} &
-- &
&
/baʦʰaːr/ &
`calf' &
-- &
\\
/ɕ/ &
\textit{(š)} &
/ɕeːmi/ &
`spleen' &
/huːɕi/ &
`wind' &
/diːɕ/ &
`village'\\
/ʨ/ &
\textit{(č)} &
/ʨeːri/ &
`spouted jug' &
/kuʨuru/ &
`dog' &
/baːlbaʨ/ &
`child'\\
/ʨʰ/ &
\textit{(čh)} &
/ʨʰeːli/ &
`she"=goat' &
/piʨʰaː/ &
`swept  (\textsc{cv})' &
-- &
\\\lspbottomrule
\end{tabularx}
\end{sidewaystable}

\addtocounter{table}{-1}
\begin{sidewaystable}[p!]
\caption{The distribution of consonants: word"=initial, medial, and final. The occurrences within parentheses are matters of interpretation (see \sectref{subsec:3-2-3}, \sectref{subsec:3-4-1}), (continued)}
\begin{tabularx}{\textwidth}{ l@{\hspace{20pt}} l@{\hspace{20pt}} Q l@{\hspace{20pt}} Q Q Q Q }
\lsptoprule
/ʑ/ &
\textit{(ǰ)} &
/ʑeːli/ &
`bore (\textsc{f)}' &
/beːʑi/ &
`heifer' &
/raːʑ/ &
`rope'\\
/j/ &
\textit{(y)} &
/jiːɽi/ &
`sheep' &
/lhaːja/ &
`will find' &
(/babaːj/ &
`apple')\\
/ʈ/ &
\textit{(ṭ)} &
/ʈaːka/ &
`call!' &
/beːʈi/ &
`lamb' &
/baːʈ/ &
`stone'\\
/ʈʰ/ &
\textit{(ṭh)} &
/ʈʰonɡi/ &
`axe' &
/buʈʰeː/ &
`all' &
-- &
\\
/ɖ/ &
\textit{(ḍ)} &
/ɖaːka/ &
`robbery' &
/ɡeːɖi/ &
`big (\textsc{f)}' &
/haːɖ/ &
`bone'\\
/ɳ/ &
\textit{(ṇ)} &
-- &
&
/deːɳi/ &
`calf (of leg)' &
/bheːɳ/ &
`sister'\\
/ɽ/ &
\textit{(ṛ)} &
-- &
&
/deːɽi/ &
`beard' &
/kiroːɽ/ &
`chest'\\
/ʂ/ &
\textit{(ṣ)} &
/ʂeːti/ &
`disputed (\textsc{f)}' &
/kʰaʂiː/ &
`hoe' &
/baːʂ/ &
`rain'\\
/ʐ/ &
\textit{(ẓ)} &
/ʐami/ &
`sister's husband' &
/ʈʂaɳʐa/ &
`torch' &
/riːʐ/ &
`track'\\
/ʈʂ/ &
\textit{(c̣)} &
/ʈʂiːnki/ &
`twittered (\textsc{cv)}' &
/teːʈʂi/ &
`wood chisel' &
/drhaːʈʂ/ &
`grape'\\
/ʈʂʰ/ &
\textit{(c̣h)} &
/ʈʂʰiːr/ &
`milk' &
/aʈʂʰiː/ &
`eye' &
/buʈʂʰ/ &
`hunger'\\
/k/ &
&
/kati/ &
`how many?' &
/bakaːra/ &
`flock' &
/ɖoːk/ &
`back'\\
/kʰ/ &
\textit{(kh)} &
/kʰur/ &
`foot' &
/nikʰai/ &
`appeared (\textsc{cv)}'  &
-- &
\\
/ɡ/ &
&
/ɡaɖi/ &
`taken out (\textsc{cv)}' &
/siɡal/ &
`sand' &
/pʰaːɡ/ &
`fig'\\
/x/ &
&
/xati/ &
`letters' &
/maːxaːm/ &
`evening' &
/mux/ &
`face'\\
/q/ &
&
/qisa/ &
`story' &
/alaːqa/ &
`area' &
/aɕaq/ &
`love'\\
/ɣ/ &
&
/ɣeːri/ &
`caves' &
/kaːɣaːz/ &
`paper' &
/baːɣ/ &
`garden'\\
/h/ &
&
/hari/ &
`removed \textsc{(cv)}' &
(/kuhiː/ &
`well') &
-- &
\\\lspbottomrule
\end{tabularx}
\label{tab:3-2}
\end{sidewaystable}

\footnotetext{In Palula common transcription these would be: \textit{píili, šópu, ṭip, phoó, aḍaphaár, bíiḍi, ǰabál,
    ḍáab, fásil, xafá, muaáf, míiša, hiimaál, bráam, wíiwai,
    heewaánd, ɡhaáu, reetí, béeriṣ, anɡóor, léedi, balíi, čhaál,
    téeti, phútu, baát, thúuṇi, suthaán, déedi, léedi, šid, néeṛi, ǰáanu, sóon,
    seetí, búusi, deés, zeerí, baazóor, anɡreéz, tsiipí, bútsu, uts,
    batsaár, šéemi, húuši, díiš, čéeri, kučúru, baalbáč, čhéeli, pičhaá, ǰéeli,
    béeǰi, ráaǰ, yíiṛi, lháaya, babaái, ṭaaká, beeṭí, báaṭ, ṭhónɡi, buṭheé, 
    ḍaaká, ɡéeḍi, haáḍ, déeṇi, bheéṇ, déeṛi, kiroóṛ, ṣéeti,
    khaṣíi, báaṣ, ẓamí, ẓaṇẓá, ríiẓ, c̣iinkí, téec̣i, dhraác̣,
    c̣hiír, ac̣híi, buc̣h, katí, bakáara, ḍóok, khur, nikhaí, ɡaḍí, síɡal, phaáɡ, xatí, maaxaám, mux,
    qisá, alaaqá, ašáq, ɣeerí, kaaɣaáz, baáɣ, harí, kuhíi}.}



I am presently in no position to determine the exact nature of retroflexion in Palula, but I prefer, nevertheless, to retain the term, as the most prominent feature in the pronunciation of these consonants is the articulation with the tip of the tongue against a~place at the rear end of the alveolar ridge and usually with the tongue slightly curled back. The dental consonants on the other hand are indeed dental, often articulated against the lower as well as the upper teeth. Generally, the area of contact between the tongue and the place of articulation is larger than in the case of the retroflex consonants.


The palatal consonants can also be described as alveolo"=palatal, with the blade of the tongue against the area covering the rear part of the alveolar and the front part of the palate, and with the tip of the tongue behind the lower teeth. 

\subsubsection*{Plosives}
The set of plosives includes: /p/: [p], /pʰ/: [pʰ]$\sim$[f], /b/: [b], /t/: [t̪], /tʰ/: [tʰ], /d/: [d̪], /ʈ/: [ʈ]$\sim$[ṯ], /ʈʰ/: [ʈʰ], /ɖ/: [ɖ]$\sim$[ḏ], /k/: [k], /kʰ/: [kʰ], /ɡ/: [ɡ], (/q/: [q]$\sim$[x]).

With respect to frequency, the voiceless plosives can be considered the unmarked subset, occurring almost twice as often as their voiced counterparts. The voiced plosives do not commonly occur word"=finally, and when they do they tend to be devoiced, as in /ɕid/ \textit{(šid)} `coldness': [ɕid̪̥]. Voiceless aspirated plosives occur in the majority of cases word"=initially, only seldom word"=medially, and never (as far as has been determined) in word"=final position. 


Intervocalically, the voiced plosives are often slightly fricativised, and frequently occur in clusters with /h/ (see \sectref{subsec:3-4-1}). Some of the voiceless aspirated plosives show lenition, for example /pʰ/ with an~alternating pronunciation [f]$\sim$[ɸ]$\sim$[pʰ], as in /pʰeːrimaː/ \textit{(pheerimaá)} `Ferima (\textsc{place name)}'. 


The phonemic status of [q] was already commented on above (see \sectref{subsec:3-1-0}).


\subsubsection*{Affricates}

The set of affricates includes: /ʦ/: [ts], /ʦʰ/: [tsʰ]$\sim$[s], /ʈʂ/: [ʈʂ], /ʈʂʰ/: [ʈʂʰ]$\sim$[ʂ], /ʨ/: [tɕ].


Affricates occur at three places of articulation, dental, retroflex and palatal, but with respect to frequency the dentals are quite limited as compared to the other two. The explanation of the missing voicing contrast is partly explainable (as already commented on above) by the overlap or neutralisation between the affricate and fricative sets. 


There is also a~less consistent neutralisation of the contrast between aspirated dental (\ref{ex:3-1}) and retroflex voiceless (\ref{ex:3-2}) affricates and their fricative counterparts (but as far as I have been able to observe, never between aspirated voiceless palatal affricates and fricatives), apparently limited to certain lexical items. 

\begin{exe}
\ex
\label{ex:3-1}
\gll atshareét \\
 /aʦʰareːt/: [aʦʰaɾěːt]$\sim$[asaɾěːt] \\
\glt `Ashret'

\ex
\label{ex:3-2}
\gll aaṣaáṛ \\
/aːʂaːṛ/: [aːʂǎːɽ]$\sim$[aːʈʂʰǎːɽ] \\
\glt `apricot'
\end{exe}

\subsubsection*{Fricatives}

The set of fricatives includes: (/f/: [f]$\sim$[pʰ]), /s/: [s], /z/: [z], /ʂ/: [ʂ]$\sim$[ʈʂʰ], /ʐ/: [ʐ]$\sim$[ɖʐ], /ɕ/: [ɕ], /ʑ/: [dʑ]$\sim$[ʑ], /x/: [x], /ɣ/: [ɣ], /h/: [ɦ]$\sim$[h].


As already pointed out in connection with the affricates, there is a~close link between the affricate set and the fricative set, with some overlaps and neutralisations taking place between them. The voiced palatal fricative is alternately realised as [ʑ] and [dʑ] (more often with an~affricate pronunciation in A) and the voiced retroflex as [ʐ] and [ɖʐ], whereas /z/ seems to occur consistently as [z] and never with an~affricate pronunciation.


The marginal phoneme /f/ is often realised as [pʰ], thus neutralised with /pʰ/, as in /faːjda/ \textit{(faaidá)} `benefit': [pʰaːjdá]$\sim$[faːjdá].


The voiced retroflex fricative is extremely rare, occurring only in a~few words. [ɖʐ] is most likely an~allophone of it, as in /ʐʰaɳʐiːr/ \textit{(ẓhaṇẓíir)} `chain': [ɖʐʰaɳɖʐ îːɾ]. 


There is a~strong affinity between /h/ and historical aspiration (\sectref{subsec:3-4-1}), especially when occurring in clusters of voiced consonants and /h/, in which case it is mostly realised as [ɦ]. Historical occurrences of word medial /h/ through movement to syllable onsets have most likely been reinterpreted as voiced aspiration. In the present language, a single /h/ only rarely occurs intervocalically, and even then often with an~interpretational ambivalence: \textit{(rhayíi)} `footprints': /rhajiː/ or /rahiː/.


\subsubsection*{Nasals}

The set of nasals includes: /m/: [m], /n/: [n̪]$\sim${}[ɲ]$\sim$[ŋ], /ɳ/: [ɳ].


Phonetically there are at least five places of articulation attested for nasals: labial, dental, retroflex, palatal and velar. The palatal nasal, however, is analysed as deriving from a~sequence of /n/ + a~palatal consonant, as it never occurs in any other environment. The same analysis may be applied to the velar nasal, where the sequence /n/ + a~velar stop usually is the likely source.


The case is a~little more complicated with the retroflex nasal, /ɳ/. Although it is clear in some cases that retroflexion is the result of assimilation with an~adjacent retroflex consonant, this cannot always be concluded. Whereas a~retroflex nasal normally does not occur word initially (although the word /ɳiɳeː/ \textit{(ṇiṇeé)} `popcorn' can be cited as an~isolated counterexample), it contrasts intervocalically with dental /n/, compare /deːni/ and /deːɳi/ in (\ref{ex:3-3}), and word"=finally, /kan/ and /kaɳ/ in (\ref{ex:3-4}). 

\begin{exe}
\ex
\label{ex:3-3}
\gll déeni -- déeṇi \\
/deːni/ -- /deːɳi/ \\
\glt `is giving'~-- `calf (of the leg)' 

\ex
\label{ex:3-4}
\gll kan -- kaṇ \\
/kan/ -- /kaɳ/ \\
\glt `shoulder'~-- `ear' (B)
\end{exe}

The labial and the dental nasals are very frequent in the language, as these two segments are part of some of the most productive inflectional forms in the language.

\subsubsection*{Flaps}

The set of flaps includes: /r/: [ɾ], /ɽ/: [ɽ].


While /r/ commonly occurs word"=initially, intervocalically, and word"=finally, the occurrence of /ɽ/
is more restricted. In B it does not occur word"=initially at all, whereas in A it occurs in free
variation with /l/ in weak forms of a~series of demonstratives (\ref{ex:3-5}), for example \textit{(lo)} or
\textit{(ṛo)} `he, that', related to strong forms of the same series with an~intervocalic
/ɽ/, \textit{(eeṛó)} `he, that', but otherwise not.

\begin{exe}
\ex
\label{ex:3-5}
\gll lo$\sim${}ṛo \\
/lo/$\sim${}/ɽo/ \\
\glt `he, that' 
\end{exe}

\subsubsection*{Lateral approximant}

There is one lateral approximant: /l/: [l]($\sim$[ɫ] B).


Preceded by a~back vowel /a aː o oː u uː/, /l/ is velarised, but only markedly so in the B variety: compare non"=velarised \textit{khéeli} and velarised \textit{khúulu} in (\ref{ex:3-6}).

\begin{exe}
\ex
\label{ex:3-6}
\gll khéeli~--~khúulu \\
/kʰeːli/~[kʰêːli]~-- /kʰuːlu/~[kʰûːɫu] \\ 
\glt `ate \textsc{fsg}'~-- `ate \textsc{msg}' (B)
\end{exe}

\subsubsection*{Approximants}

The set of approximants includes: /w/: [β̞]$\sim$[ʋ], /y/: [j].


In the speech of my main A consultant, the front"=most approximant /w/ is usually pronounced bilabially [β̞], but with many speakers this phoneme seems to alternate between a~bilabial and something close to a~labiodental [ʋ] pronunciation.


The two approximants are sometimes challenging to interpret, whether they should be regarded as consonants or vowels, and they are in various ways susceptible to articulatory fluctuation or variation, especially when occurring intervocalically, an~issue that will be further discussed in connection with vowels (see \sectref{subsec:3-2-3}).


\section{Vowels}
\label{sec:3-2}

\subsection{Vowel inventory}

For the vowels, there are five contrasting places of articulation, as can be seen in \tabref{tab:3-3}: a) close front, b) close back, c) open front, d) open back rounded, and e) open back unrounded. Together with phonemic length contrasts, there is a~ten"=vowel system. A convincing and consistent contrast (as the one shown by \citet[19]{radloff1999} for Gilgiti Shina) between oral and nasalised vowels has not been found. Instead, nasalisation seems to be a~marginal suprasegmental feature of a~limited number of lexemes. Apart from those, nasalisation is a~non"=contrastive phonetic property of vowels occurring adjacent to a~nasal consonant. 



\begin{table}[ht]
\caption{Inventory of vowels (IPA)}
\begin{tabularx}{\textwidth}{ Q Q Q Q Q }
\lsptoprule
&
&
Front
&
Back unrounded &
Back rounded \\\midrule
Close &
short &
i &
&
u\\
&
long &
iː &
&
uː\\
Open &
short &
e &
a &
o\\
&
long &
eː &
aː &
oː\\\lspbottomrule
\end{tabularx}
\label{tab:3-3}
\end{table}

\subsection{Distribution and variation}
\label{subsec:3-2-1}

\tabref{tab:3-4} exemplifies target articulations of the vowels, all of which take on more centralised qualities in natural and connected speech. Generally, the short vowels /i/, /a/, and /u/ tend to be pronounced as less peripheral than their long counterparts. The short /i/ is not necessarily more open than the long /iː/, but it has a~rather more central pronunciation; the short /u/, on the other hand, is both more open and slightly more central than the long /uː/; the short /a/ is also slightly less open and more fronted than the long /aː/. 


\footnotetext{In Palula common transcription, the words are: \textit{ɡir, ɡiír, ṭíki, tíiṇi, preṣ, keéṇ, ṭéka, teeká, šak, káaṇ, ṭáka, ṭaaká, sum, kúuṇ, thúki, thúuṇi, khoṇḍ, kóoṇ, tróki, ṭooká}.}


\begin{table}[ht]
\caption{Vowel contrasts exemplified (see \sectref{subsec:3-4-3} for details on pitch accent){\protect\footnotemark}}
\begin{tabularx}{\textwidth}{ l l l@{\hspace{40pt}} l l l Q }
 \lsptoprule
/i/ &
/ɡir/ &
`turn around!' &
/iː/ &
\textit{(ii)} &
/ɡǐːr/ &
`saw'\\
&
/ʈíki/ &
`bread' &
&
&
/tî:ɳi/ &
`sharp'\\
/e/ &
/preʂ/ &
`mother"=in"=law' &
/eː/ &
\textit{(ee)} &
/kěːɳ/ &
`cave'\\
&
/ʈéka/ &
`peaks' &
&
&
/ʈeːká/ &
`labour'\\
/a/ &
/ɕak/ &
`doubt' &
/aː/ &
\textit{(aa)} &
/kâːɳ/ &
`ear'\\
&
/ʈáka/ &
`insult' &
&
&
/ʈaaká/ &
`call!' \\
/u/ &
/sum/ &
`dry mud' &
/uː/ &
\textit{(uu)} &
/kûːɳ/ &
`corner'\\
&
/tʰúki/ &
`spittle' &
&
&
/tʰûːɳi/ &
`pillar'\\
/o/ &
/kʰoṇḍ/ &
`speak!' &
/oː/ &
\textit{(oo)} &
/kôːɳ/ &
`arrow'\\
&
/tróki/ &
`worn, thin' &
&
&
/ʈoːká/ &
`push!' \\\lspbottomrule
\end{tabularx}
\label{tab:3-4}
\end{table}


Phonetically, there is a~significant difference between short and long vowels. The duration of a~long vowel like /aː/ as compared to its short counterpart /a/ is not just slightly longer, but usually of at least twice the duration.


Environment as well as accent (see \sectref{subsec:3-4-3}) further influences the exact pronunciation of each of the ten vowels. Under certain conditions, some neutralisations take place (see \sectref{subsec:3-2-2}). 


As pointed out already by \citet[58]{morgenstierne1932}, the most important~-- if not all~-- phonological dialect differences between A and B concern the vowels rather than the consonants. 

\subsection{Vowel neutralisation}
\label{subsec:3-2-2}

While there is a~consistent contrast between all the five vowel qualities as well as a~contrast in length, when the vowels are accented (see \sectref{subsec:3-4-3} for details on accent), these contrasts are fewer and less convincing when the vowels are unaccented. The two main dialects also show some differences in this regard. 


Whereas B maintains a~word"=final /a/ vs. /e/ contrast~-- as is clearly evidenced in the morphological contrast between the general oblique inflection \textit{-a}, as in /ˈdiːɕa/ `in the village', and the genitive singular \textit{-e}, as in /ˈdiːɕe/ `of the village', of many masculine nouns~-- there is no evidence of contrast between these two unaccented short vowels in A (where the unaccented genitive ending instead is /iː/). In contrast, a~non"=variable masculine ending [u] in B, corresponds to two different (but grammatically identical) masculine endings [o] and [u] in A. Curiously, the realisations of these two variants are in complementary distribution, although there is no obvious phonological motivation behind it. When preceded by /aː/ (in the previous syllable), the allophone is [u], while it is [o] when preceded by any other vowel, as can be seen in (\ref{ex:3-7}).

\begin{exe}
\ex
\label{ex:3-7}
\gll paṇáaru~-- tóoru~-- bhíiru~-- bhúuru~--  léku \\
[paɳâːɾu]~-- [tôːɾo]~-- [bɦîːɾo]~-- [bɦûːɾo]~-- [léko] \\ 
\glt `white \textsc{msg}'~-- `star [\textsc{m}]'~-- `he-goat [\textsc{m}]'~-- `deaf \textsc{msg}'~-- `small \textsc{msg}'
\end{exe}

As a consequence of these observations, all instances of unaccented word"=final [u] and [o] are consistently transcribed as \textit{u} in the Palula common transcription, whether A or B, and only word"=final unaccented \textit{a} occurs in A examples, not \textit{e}.


\subsection{The status of diphthongs}
\label{subsec:3-2-3}

A complex issue still needing more careful study concerns the interpretation and representation of ambiguous vowel sequences. However, for the time being there is no strong evidence for stipulating any phonemic diphthongs with a~status comparable to that of the ten vowels already introduced.

The sequences of vowels in lexical stems all consist of at least one close vowel (also interpretable as an~approximant), such as [ai], [aːi], [ui], [oːi], [oi], [eːi], [ueː], [uaː], [iaː], [ioː], [aːu], [au]. Probably most, if not all, combinations of a~short close vowel and another long or short vowel are possible. Some examples are given in (\ref{ex:3-8}).


\begin{exe}
\extab
\label{ex:3-8}
\begin{tabularx}{\textwidth}{ l l l l }
&
[bɾɦaːdʑai] &
\textit{(bhraaǰái)} &
`sister"=in"=law'\\
&
[baba:i] &
\textit{(babaái)} &
`apple'\\
&
[dʑabui] &
\textit{(ǰabúi)} &
`velum'\\
&
[bɦoːi] &
\textit{(bhoói)} &
`daughter"=in"=law'\\
&
[lɦoilo] &
\textit{(lhóilu)} &
`red'\\
&
[jeːi] &
\textit{(yéei)} &
`mother'\\
&
[kakueːki] &
\textit{(kakuéeki)} &
`hen'\\
&
[suaːl] &
\textit{(suaál)} &
`question'\\
&
[taːpiaːl] &
\textit{(taapiáal)} &
`near'\\
&
[pʰioːɽ] &
\textit{(phióoṛ)} &
`side (of an~animal)' \\
&
[ɡɦa:u] &
\textit{(ɡhaáu)} &
`cow'\\
&
[maɳɖau] &
\textit{(maṇḍáu)} &
`veranda'\\
\end{tabularx}
\end{exe}


Taking a~number of factors into account, such as mother"=tongue speakers' counting (``knocking'') syllables, the apparent absence of sequences not including any of the two close vowels [i] and [u], and the evidence for approximants occurring word"=initially as well as intervocalically (and therefore if not being interpreted consonantally leaving a~gap word"=finally), would favour an~approximant interpretation, which would render the following phonemic output: /brhaːˈʑaj/, /baˈbaːj/, /ʑaˈbuj/, /bhoːj/, /ˈlhojlu/, /jeːj/, /kakˈweːki/, /ˈswaːl/, /taːˈpjaːɽ/, /pʰjoːɽ/, /ɡhaːw/, /maɳˈɖaw/.


However, in a~number of words with a~vowel + [i] sequence, the final [i] can be considered a~feminine gender suffix (in some cases derived by that suffix, at least diachronically, from a~masculine stem), and in the morphological behaviour of monomorphemic stems, such as those exemplified above, it is an~advantage to show that there is an~underlying vowel /i/ or /u/ (rather than a~consonant) involved. Therefore, I have chosen to represent them as \textit{bhraaǰái}, \textit{yéei}, etc., in Palula common transcription, to signal precisely the connection between a~stem and its derivations or inflected forms.\footnote{Although not attempted here, an~alternative analysis of [ái] would be to consider it an~allophone of first"=mora accented /ée/.} 


The latter representation makes even more sense for sequences in polymorphemic words, such as \textit{dhióomii} `of the daughters' from \textit{dhií} + \textit{-óom} (\textsc{obl.pl)} + \textit{-ii} (\textsc{gen)}, although the surface phonemic representation would be /djhoːmiː/, the latter taking de facto syllabification into account at the expense of morphemic transparency. This holds for inflected forms of verbs as well: A purely phonemic representation such as /swâːnu/ `is sleeping \textsc{msg}' obscures the fact that we have the verbal stem \textit{só-} `sleep' inflected for present tense with \textit{-áan}, and therefore a~Palula common transcription \textit{suáanu} has been chosen for it.


When, on the other hand, there is a~need to show that there indeed is a~syllable break between two successive vowels, whether the word is mono- or polymorphemic, an~approximant, \textit{y} or \textit{w}, is inserted: \textit{bhooyóomii} /bhoːˈjoːmiː/ `of the daughters"=in"=law', and \textit{bharíiwa} /bhaˈriːwa/ `husbands'.


\section{Phonotactics}
\label{sec:3-3}

\subsection{Syllable structure}
\label{subsec:3-3-1}


A typical syllable in Palula is an~open syllable consisting of a~consonant and a~vowel. This is the most common type when the syllable is unaccented. Long, as well as short vowels (\ref{ex:3-9}) could constitute the nucleus of such a~syllable: CV or CVV. There are monosyllabic words (/be/, /wiː/) which conform to this basic CV pattern, but most words are polysyllabic, consisting of two or more CV (or CVV) syllables (such as /ɡuː.li/ and /ku.ɳaː.koː.miː/).

\protectedex{
\begin{exe}
\extab
\label{ex:3-9}
\begin{tabular}{ l l l }
/be/ &
\textit{(be)} &
`we'\\
/wiː/ &
\textit{(wíi)} &
`water'\\
/ɡuː.li/ &
\textit{(ɡúuli)} &
`bread'\\
/ku.ɳaː.koː.miː/ &
\textit{(kuṇaakóomii)} &
`of the children'\\
\end{tabular}
\end{exe}
}


The closed"=syllable pattern, CV(V)C, is also a~very common syllable, see examples in (\ref{ex:3-10}), and the most common one in accented syllables. This type occurs in monosyllabic as well as in polysyllabic words. Commonly, however, a~word is made up of a~combination of open and closed syllables. 


\begin{exe}
\extab
\label{ex:3-10} 
\begin{tabular}{ l l l }
/pil/ &
\textit{(pil)} &
`drink!' \\
/ɕiːn/ &
\textit{(\v{s}íin)} &
`bed'\\
/ʈiːn.ʨuk/ &
\textit{(ṭíinčuk)} &
`scorpion'\\
/lan.ɡuːm/ &
\textit{(lanɡúum)} &
`I will take across'\\
/ʈom.bu/ &
\textit{(ṭómbu)} &
`stem'\\
/piɳ.ɖuː.ru/ &
\textit{(piṇḍúuru)} &
`round'\\
/heː.wan.da/ &
\textit{(heewandá)} &
`winter (\textsc{obl)}'\\
\end{tabular}
\end{exe}


Even onsetless syllables, V(C) or VV(C), occur in Palula (\ref{ex:3-11}), though less frequently. That means that both the onset and the coda is optional, i.e., a~vowel nucleus can occur alone or at least word"=initially. Whether this is also possible word"=medially or word"=finally is an~interpretational issue, but in any case, there are no single phonological words consisting of only a~vowel nucleus.\footnote{An alternative analysis not attempted in this work would be to regard a~glottal stop as a~consonant phoneme preceding all vowels that are here considered word"=initial, thus doing away with onsetless syllables altogether.}


\begin{exe}
\extab
\label{ex:3-11}
\begin{tabular}{ l l l }
/u.ɽi/ &
\textit{(uṛí)} &
`pour!' \\
/ux/ &
\textit{(ux)} &
`camel'\\
/oː.ɖhoːl/\ \ &
\textit{(ooḍhóol)} &
`flood'\\
\end{tabular}
\end{exe}


The minimal word can therefore be defined as consisting of a~vowel nucleus plus either an~onset or a~coda consonant. There seems also to be further constraints on words belonging to the major open classes as opposed to words from closed classes when it comes to minimal words. Nouns, adjectives and verbs (except for imperative forms and a~few participle forms) must consist of at least a~short vowel plus a~coda, or an~onset plus a~long vowel. Pronouns, on the other hand, may very well consist of only a~short vowel with an~onset: /ma/ `I', /be/ `we', etc.


\subsection{Consonant clusters}
\label{subsec:3-3-2}


The preservation of a~number of clusters, especially some that occur word finally, sets Palula off as more conservative than most other Shina varieties. In addition, a~set of changes, at least partly related to, on the one hand, vowel metathesis and re-syllabification, and on the other hand, laryngeal metathesis and the subsequent reinterpretation of what was earlier voiced aspirates (see \sectref{subsec:3-4-1}), have produced a few new, primarily word"=initial, clusters.


There is a~maximum onset of three consonants in Palula words, as can be seen in (\ref{ex:3-12}). These are clusters of voiced consonants only, whose third member always is /h/ (phonetically realised as [ɦ]).  One type, whose middle member is /r/ preceded by a plosive, go back to old (or secondarily formed) voiced aspirates followed by /r/. The aspirates have been reinterpreted as plosive + /h/ clusters, and a subsequent realignment has taken place, whereby the more sonorant /h/ has changed to the position closest to the syllable nucleus: /brhoː/ `brother' < */bhraː/ < \textit{bhr\'{\={a}}tr̥-}. Another type, whose middle member is one of the two approximants /j/ or /w/ (if we go with the analysis presented above, \sectref{subsec:3-2-3}) preceded by a plosive or a nasal, have arisen through de"=syllabification of a short unaccented closed vowel, in some cases subsequent to vowel metathesis, such as is the case (in A) with /ɡwheːɳiː/ `Pashtun' (< /ɡhweːɳiː/ < /ɡhueːɳiː/ < /uɡheːɳiː/, the latter which is still the form heard in the conservative B dialect). That other clusters similar to the last-mentioned type seem to be in the process of evolving is evidenced by co-existing forms: /ukʰaːndu/$\sim$/kʰwaːndu/ \textit{(ukháandu)} `is coming/going up \textsc{msg}', perhaps pointing to a~preference for Cw and Cj clusters vis-à-vis word initial V-syllables. 


\begin{exe}
\extab
\label{ex:3-12}
\begin{tabular}{ l l l l }
/brh-/ &
/brhoː/ &
\textit{(bhróo)} &
`brother'\\
/drh-/ &
/drhuːk/ &
\textit{(dhrúuk)} &
`gorge, stream' \\
/ɡrh-/ &
/ɡrheːɳɖ/ &
\textit{(ɡhreéṇḍ)} &
`knot' \\
/njh-/ &
/ˈnjhaːɽa/ &
\textit{(nhiáaṛa)} &
`near'\\
/djh-/ &
/ˈdjhuːɽi/ &
\textit{(dhiúuṛi)} &
`granddaughter'\\
/ɡwh-/ &
/ɡwheːˈɳiː/ &
\textit{(ɡhueeṇíi)} &
`Pashtun'\\
/dwh-/ &
/ˈdwheːli/ &
\textit{(dhuéeli)} &
`washed (\textsc{f)}'\\
\end{tabular}
\end{exe}


Initial two"=consonant clusters, see (\ref{ex:3-13}), share many of the features already mentioned for three"=consonant onsets. The second member of such a~cluster is either /r/ (most of them of considerable age), an~approximant (with a recent history, derived along the same lines as was presented above for /ɡwheːɳiː/), or /h/ (which is either historical voiced aspiration reinterpreted as a cluster, or a new initial cluster arisen through laryngeal metathesis). Usually, but not exclusively, /r/ is preceded by a~plosive (in the majority of cases a voiceless one). Voiceless aspirated plosives in clusters with /r/ are rare indeed, the verb /pʰrajaːnu/ `send' is the only example found so far in the data with a following /r/. Approximant may be preceded by plosive, fricatives or nasals. Nearly any voiced consonant may precede /h/ in initial clusters (the only exceptions in the data being the "new" phoneme /ɣ/, and the distributionally limited consonants /ɳ/ and /ɽ/).


\begin{exe}
\extab
\label{ex:3-13}
\begin{tabular}{ l l l l }
/pr-/ &
/ˈpraːʨu/ &
\textit{(práaču)} &
`guest'\\
/pʰr-/ &
/pʰraˈjaːnu/ &
\textit{(phrayáanu)} &
`is sending (\textsc{msg)}'\\
/br-/ &
/braːm/ &
\textit{(bráam)} &
`joint'\\
/tr-/ &
/ˈtroki/ &
\textit{(tróki)} &
`thin, weak (\textsc{f)}'\\
/kr-/ &
/kraːm/ &
\textit{(kráam)} &
`work' \\
/mr-/ &
/ˈmrinɡa/ &
\textit{(mrínɡa)} &
`deer'\\
/nj-/ &
/njaːˈʈa/ &
\textit{(niaaṭá)} &
`shave!, shear!' \\
/pj-/ &
/pjaːˈla/ &
\textit{(piaalá)} &
`cup'\\
/pʰj-/ &
/pʰjoːɽ/ &
\textit{(phióoṛ)} &
`side (of animal)' \\
/sw-/ &
/ˈsweːni/ &
\textit{(suéeni) } &
`is sleeping (\textsc{f)}'\\
/dh-/ &
/dhut/ &
\textit{(dhut)} &
`mouth'\\
/ʑh-/ &
/ʑhaːʈ/ &
\textit{(ǰhaáṭ)} &
`goat's hair'\\
/lh-/ &
/lhoːɳ/ &
\textit{(lhoóṇ) } &
`salt'\\
/mh-/ &
/mhaːs/ &
\textit{(mhaás)} &
`meat'\\
/jh-/ &
/ˈjhuɳɖi/ &
\textit{(yhúṇḍi)} &
`stick'\\
\end{tabular}
\end{exe}


Two"=consonant clusters in coda position, see (\ref{ex:3-14}), seem to be subject to a~much higher degree of variability, although the position also seems to be slightly more permissive than the onset. The more frequently occurring type observed at word boundaries consists of nasal + plosive/affricate/fricative, the other types being more marginal in occurrence. 


\begin{exe}
\extab
\label{ex:3-14}
\begin{tabular}{ l l l l }
/-nd/ &
/daːnd/ &
\textit{(dáand)} &
`tooth'\\
/-ɳɖ/ &
/ɡrheːɳɖ/ &
\textit{(ɡhreéṇḍ)} &
`knot'\\
/-nk/ &
/ɡrhoːnk/ &
\textit{(ɡhroónk)} &
`worm'\\
/-nɡ/ &
/ɕoːnɡ/ &
\textit{(šóonɡ) } &
`branch'\\
/-nʑ/ &
/paːnʑ/ &
\textit{(páanǰ) } &
`five' \\
/-nɕ/ &
/bheːnɕ/ &
\textit{(bheénš) } &
`beam (of wood)' \\
/-ɳʈʂ/ &
/iɳʈʂ/ &
\textit{(iṇc̣) } &
`bear' \\
/-tr/ &
/suːtr/ &
\textit{(súutr)} &
`thread' \\
/-st/ &
/ɡrhaːst/ &
\textit{(ɡhraást)} &
`wolf' \\
/-ʂʈ/ &
/ɡhoːʂʈ/ &
\textit{(ɡhoóṣṭ) } &
`house'\\
\end{tabular}
\end{exe}


The final affricate or fricative is always articulated, even if sometimes only weakly, whereas the
nasal (homorganic with the affricate or fricative) is sometimes~-- more with some speakers than
others and depending on word emphasis~-- phonetically absent but leaves a~trace of nasalisation on
the preceding vowel. Even the nasal + plosive sequences are subject to much variability. With some
speakers and dialects, one of the phonemes in the sequence is altogether absent, sometimes the nasal
(then leaving the preceding vowel nasalised, for example in /naːnɡ/
(náanɡ) `finger or toe"=nail' B: [n̪\^{ã}ːɡ]), sometimes the stop (for example in /ɕaːnɡ/ (\v{s}áanɡ)
`branch' B: [ɕâːŋ]), whereas in the corresponding inflected forms the stop would never be
omitted: [n̪ûːŋɡa] (núunɡa), [ɕûːŋɡa]
(\v{s}úunɡa). This especially pertains to the /n +
d/ sequences in singular nouns, where it seems to be more rule than exception that the final stop is dropped, especially in B, whereas these are clearly articulated when occurring non"=finally in the inflected forms: /dan/ `tooth', but /daːnda/ `teeth'; /pan/ `path', but /paːnda/ `paths'.


In the final /tr/-cluster, the /r/ is present as a~segment in the speech of all my informants, but its articulation is not exactly identical to its non"=final allophones (as in the inflected forms of the same lexical items). There is a~strong tendency for it to be pronounced with very little energy, almost always being devoiced and sometimes also followed by an~optional very short schwa"=like sound: [putɾ̥(ə)] in /putr/ `son'. 


As for the realisation of the final /st/ and /ʂʈ/-clusters, there are differences between different speakers, and possibly between different dialects as well. My B informants tend to articulate both members of the cluster, even in final position, though the plosive is somewhat softened, whereas my A informants seem to prefer to omit the plosive altogether in final position, e.g [ɡɦ\v{o}ːʂ] `house', [n̪aːs] \textit{(náas)} `nose'. However, in all speech varieties both the fricative and the plosive are clearly present when occurring medially, i.e., in the corresponding inflected forms: [ɡɦoːʂʈá], [nastí].


A special case is the final cluster /ndr/ in /jaːndr/, see (\ref{ex:3-15}). This is the only three"=consonant cluster at a~word boundary discovered so far, but its exact phonetic realisation is not entirely easy to define in terms of segments. With some speakers, the /n/ is clearly articulated, whereas the /dr/ part is only faintly present, and in other pronunciations the final /r/ gets a~schwa"=like sound attached to it, in practice making /dr/ the onset of an~additional syllable. As with the above"=mentioned clusters occurring at the end of singular nouns, the same cluster stretching over a~syllable boundary in an~inflected form of the same noun is clearly and unambiguously articulated: /jaːn.dra/ `mills'.


\begin{exe}
\extab
\label{ex:3-15}
\begin{tabular}{ l l l l }
/-ndr/ &
/jaːndr/ &
\textit{(yáandr)} &
`mill'\\
\end{tabular}
\end{exe}


Clusters occurring in syllable onsets and syllable codas intervocalically, see (\ref{ex:3-16}), are subject to the same restrictions as the clusters at word boundaries described above, but a few of them, particularly those containing /h/, are extremely rare in that position. 


\begin{exe}
\extab
\label{ex:3-16}
\begin{tabular}{ l l l l }
/-drh-/ &
/beːdrhiː/ &
\textit{(beedhríi)} &
`it [the sky] will clear up'\\
/-br-/ &
/ˈubru/ &
\textit{(úbru)} &
`a kind of bird'\\
/-tr-/ &
/ˈbaːtru/ &
\textit{(báatru)} &
`irrigation lock'\\
/-kr-/ &
/ˈʨukru/ &
\textit{(čúkru)} &
`sour'\\
/-dh-/ &
/badhoːˈɽaːnu/ &
\textit{(badhooṛáanu)} &
`is butting its horns (\textsc{msg)}'\\
/-nj-/ &
/dunˈjaː/ &
\textit{(duniaá)} &
`dowry'\\
/-mb-/ &
/oːmˈbaːr/ &
\textit{(oombaár)} &
`canal inlet'\\
/-nʨ-/ &
/ˈʈiːnʨuk/ &
\textit{(ṭíinčuk)} &
`scorpion'\\
\end{tabular}
\end{exe}


Apart from those, a number of other consonant clusters not permitted at word"=boundaries are found word"=internally. However, in all of those cases the clusters are analyzable as occurring across a syllable boundary (and not seldom across a morpheme"=boundary as well). Only a few examples are shown in (\ref{ex:3-17}).


\begin{exe}
\extab
\label{ex:3-17}
\begin{tabular}{ l l l l }
/-m.br-/ &
/ʑham.ˈbreːɽi/ &
\textit{(ǰhambréeṛi)} &
`bride'\\
/-ɕ.tr-/ &
/poːɕ.ˈtra/ &
\textit{(pooštrá)} &
`fattened'\\

\end{tabular}
\end{exe}


\section{Suprasegmentals}
\label{sec:3-4}

\subsection{Aspiration and breathiness}
\label{subsec:3-4-1}

In the present description, seven consonants, all voiceless, with aspiration as a~secondary articulation have been postulated: /pʰ/, /tʰ/, /ʈʰ/, /kʰ/, /ʦʰ/, /ʈʂʰ/, and /ʨʰ/. The two affricates /ʦʰ/ and /ʈʂʰ/ are treated with a~higher degree of tentativeness, as their contrastiveness vis-à-vis their non-aspirated counterparts is far less convincing than is the case with the other five aspirated segments, and would deserve further, more detailed, investigation.\footnote{At least as far as /ʈʂ/ is concerned, it seems, it is almost by ``default'' more or less clearly aspirated. A similar hesitation has been expressed on the /ʈʂ/-/ʈʂʰ/ contrast in Khowar \citep[239]{endresenkristiansen1981}.} There are, however, certain characteristics (synchronic as well as diachronic) that the voiceless aspirated sounds share with clusters of voiced consonants and /h/ (as described in \ref{subsec:3-3-2}), suggesting an alternative (or perhaps complementary), unitary treatment of aspiration as a~feature of a~word (or more correctly the lexical stem), as is already reflected in the consistent use of \textit{h} in the Palula common transcription.\footnote{It is also possible that a~more careful analysis will result in all instances of /h/ (here considered a~consonant in its own right when occurring alone in syllable onset) being treated in the same way.} 


This feature, realised as [ʰ] with a~voiceless consonant and [ɦ] (or [ʱ]) with a~voiced consonant, occurs only once in a~word\footnote{\label{fnt:ftn32} The process of dissimilation of aspirates in two successive syllables is known as \textit{Grassman's Law} within Indo"=European historical linguistics (\citealt[19, 56]{szemerenyi1996}; \citealt[153--154, 162--163]{lehmann1992}) and has been applied to the development of OIA as well as Greek. A synchronic process or rule restricting the occurrence of aspirated sounds to one per word has also been stipulated for other NIA languages (see for example \citet[32]{losey2002} for Gojri, and \citet[34--35]{shackle1976} for Siraiki). This process, however, is not confined to Indo"=European languages: compare Tibeto"=Burman Manipuri \citep[13--14]{bhatningomba1997}.} (in a~majority of the cases word"=intially). Some minimal pairs in (\ref{ex:3-18}) illustrate the contrastiveness of this feature.


\begin{exe}
\extab
\label{ex:3-18}
\begin{tabularx}{116mm}{ l l l }
/ˈbhoːla/ \textit{(bhóola)} &
vs &
/ˈboːla/ \textit{(bóola)}\\
`were able to' &
&
`hair'\\
/kʰaˈreːɽi/ \textit{(kharéeṛi)} &
&
/kaˈreːɽi/ \textit{(karéeṛi)}\\
`bolt' &
&
`leopardess'\\
/ˈʨʰeːli/ \textit{(čhéeli)} &
&
/ˈʨeːli/ \textit{(čéeli)}\\
`she"=goat' &
&
`wide (\textsc{f})'\\
/whiː/ \textit{(whíi)} &
&
/wiː/ \textit{(wíi)}\\
`will come down (\textsc{3sg})' &
&
`water'\\
\end{tabularx}
\end{exe}


The vowel immediately following /h/ occurring in clusters with voiced consonants is normally phonetically realised with (at least partially) breathy voice: [bʱo̤ːla]. 


Most voiced consonants can occur in clusters with /h/, from plosives to approximants (as described in \ref{subsec:3-3-2}). This generous occurrence of aspiration (in the wider sense) is not a~feature of most other languages in the immediate region, possibly with the exception of Indus Kohistani, where OIA aspiration, like in Palula, has been preserved and where aspiration is concomitant with most of its consonants \citep[19--25]{hallberghallberg1999}. 


Whereas the phonetic realisation of the aspiration with the voiceless consonants is more or less equal to a~secondary pronunciation of the voiceless segment, the ``breathiness'' affecting the pronunciation of the following vowel is somewhat mobile within the syllable, and for some words even beyond the realm of that syllable. Especially in B, there is a~fluctuation in some words, as seen in (\ref{ex:3-20}) and (\ref{ex:3-21}), between a~realisation as a single intervocalic /h/"=segment and the occurrence of /h/ as part of a word"=initial cluster, as described above, the intervocalic /h/ probably representing an~older pattern.\footnote{Possibly this is preserved to a~larger extent in the conservative variety spoken in Puri (mainly agreeing with Biori Palula), where the following nominative"=oblique alternation was recorded: /brhuː/ \textit{(bhruú)} `brother' vs. /brahu/ `brothers'. The unstable character of the phoneme /h/ and voiced aspiration in Kalasha is also commented on by \citet[50]{morchheegaard1997}.}

\begin{exe}
\extab
\label{ex:3-20}
/lhójlo/$\sim$/lohílo/ \\
\textit{(lhóilu)} \\
`red' (B)

\extab
\label{ex:3-21}
/bjhûːɽi/$\sim$/bihûːɽi/ \\
\textit{(bhiúuṛi)} \\
`Biori' (B)
\end{exe}

In A, too, there are words, as in example (\ref{ex:3-22}), for which the location of aspriation is alternating (between speakers and possibly even with one and the same speaker).

\begin{exe}
\extab
\label{ex:3-22}
/ɡhaɖeːró/$\sim$/ɡaɖheːró/ \\
\textit{(ɡhaḍeeró)} \\
`elder'
\end{exe}

The aspiration feature (whether synchronically a cluster with /h/ or a voiceless aspirated consonant) has multiple diachronic sources: One is the OIA aspiration, thus preserved in Palula to an~extent not evidenced in the major Shina varieties,\footnote{On the contrary, \citet[30]{schmidtkohistani2008} state that the voiced aspirates in modern Kohistani Shina have come into the language through borrowing, while they have been lost in OIA voiced aspirate cognates.} such as in /ɡhuːɽu/ \textit{(ɡhúuṛu)} `horse' {\textless} OIA \textit{ɡhōṭa} \citep[4516]{turner1966}. Another is the above"=mentioned intervocalic /h/ advanced to a~more word"=initial position. Finally, an~old point of aspiration can be advanced, such as in /ɡhoːʂʈ/ \textit{(ɡhoóṣṭ)} `house' {\textless} OIA \textit{ɡōṣṭhá} \citep[4336]{turner1966}. Other words may have followed other routes, possibly further reinforced by the rising pitch of a~second mora accent (see \sectref{subsec:3-4-3} below). In any case, not all occurrences of aspiration, even when concomitant with plosives, are justified or explained solely by etymology (as pointed out by \citealt[57]{morgenstierne1932}).


Further study is needed to determine to what extent aspiration is preserved in two aspirated words that are compounded. There is an~indication that the primary stressed part of the compound keeps its aspiration, while the other point of aspiration is entirely or partly deaspirated (compare with the comment in footnote \ref{fnt:ftn32} on \textit{Grassman's Law}): /dhut/ \textit{(dhut)} `mouth' + /ɡhaːnu/ \textit{(ɡháanu)} `large' {\textgreater} /dutaɡhaːnu/ \textit{(dutaɡháanu)} `talkative'.


The interaction between accent and aspiration is another topic for further research. As breathiness (or a voiced cluster with /h/) quite often precedes a~second"=mora accented long vowel, the feature may have influenced or reinforced the rising pitch of that accent. However, it should be pointed out that breathiness also occurs before unaccented, as in /ʑhamatroː/ \textit{(ǰhamatroó)} `son"=in"=law', as well as first"=mora accented vowels, as in /oːɖhoːl/ \textit{(ooḍhóol)} `flood'; hence the two suprasegmental features are, in essence, independent.\footnote{This contrasts with the situation in Gawri and Torwali (both languages belonging to the Kohistani group of HKIA languages), where aspiration is only contrastive with voiceless consonants, whereas breathiness is a~feature only optionally concomitant with low tone (\citealt[92]{baart1999b}; \citealt[36--37]{lunsford2001}). } 


\subsection{Nasalisation}
\label{subsec:3-4-2}


As mentioned earlier, obligatory nasalisation seems to be a~marginal suprasegmental feature associated with an~extremely limited number of lexemes, among which figure those in (\ref{ex:3-23}). It is unclear whether in all those lexemes there is a~historical loss of a~nasal segment. In any case, there are too few instances to postulate any individual nasalised vowel phonemes, or even less a whole set of nasalised vowels. Neither are there in the data any examples of a lexical distinction being made solely by contrasting an~oral vowel with a~corresponding nasalised vowel. 


\begin{exe}
\extab
\label{ex:3-23}
\begin{tabularx}{\textwidth}{ l l l }
/ʑh\~{i}ː/ &
\textit{(ǰhií$\sim$) } &
`louse'\\
/kũj/ &
\textit{(kúi$\sim$)} &
`valley'\\
/bãːˈjilu/ &
\textit{(baa$\sim$ílu)} &
`made of oak wood'\\
/hẽː/ &
\textit{(hée$\sim$)} &
`yes'\\
/õː/ &
\textit{(óo$\sim$)} &
`mouth'\\
/drhũːʂ/ &
\textit{(dhrúu$\sim$ṣ)} &
`Drosh (a place)'\\
\end{tabularx}
\end{exe}

\subsection{Pitch accent}
\label{subsec:3-4-3}

A phonological word in Palula may carry one, and only one, accent. Phonetically the accent is primarily realised as relatively higher pitch, accompanied to some extent by higher amplitude. Generally speaking, in a~single word, accent is associated with high pitch, and the corresponding lack of accent is associated with low (or default) pitch. 

The accent"=bearing unit is the vocalic mora, which means that accent can be associated with a~short vowel (as in (\ref{ex:3-24})), or with the first mora of a~long vowel (as in (\ref{ex:3-25})), or with the second mora of a~long vowel (as in (\ref{ex:3-26})).


\begin{exe}
\extab
\label{ex:3-24}
\begin{tabularx}{\textwidth}{ l l l }
\multicolumn{3}{l}{Accent on short vowel:}\\
/ʂíʂ/ &
\textit{(ṣiṣ)} &
`head'\\
/híɽu/ &
\textit{(híṛu)} &
`heart'\\
/kilí/ &
\textit{(kilí)} &
`key'\\
\end{tabularx}
\end{exe}

\begin{exe}
\extab
\label{ex:3-25}
\begin{tabularx}{\textwidth}{ l l l }
\multicolumn{3}{ l}{First"=mora accent on long vowel:}\\
/ɖôːk/ &
\textit{(ḍóok)} &
`back'\\
/pûːtri/ &
(\textit{púutri)} &
`granddaughter'\\
/aʈʂîː/ &
\textit{(ac̣híi)} &
`eye'\\
\end{tabularx}
\end{exe}


\begin{exe}
\extab
\label{ex:3-26}
\begin{tabularx}{\textwidth}{ l l l }
\multicolumn{3}{ l}{Second"=mora accent on long vowel:}\\
/bǎːt/ &
\textit{(baát)} &
`talk, word, issue'\\
/kuɳǎːk/ &
\textit{(kuṇaák)} &
`child'\\
/dǎːdu/ &
\textit{(daádu)} &
`burnt (\textsc{m})'\\
\end{tabularx}
\end{exe}

This means that the pitch accent\footnote{Pitch accent here is of the kind also observed in e.g., Lithuanian \citep[73--82]{szemerenyi1996}, a~mora accent which is ``free within limits''.} (henceforth only ``accent'') has one of the following phonetic manifestations:
\begin{enumerate}
\item[a)] high level or falling on a~short vowel [\'{}], represented in this work with an~acute accent mark: \textit{á} (only in polysyllabic words, elsewhere without marking), 
\item[b)] rising on a~long vowel [\v{}], represented with an~acute accent mark on the second vowel symbol: \textit{aá}, or 
\item[c)] falling on a~long vowel [\^{}], represented with an~acute accent mark on the first vowel symbol: \textit{áa}.
\end{enumerate}
A word, as referred to here, is either a~bare stem, such as /děːs/ \textit{(deés)} `day', /paːɽú/ \textit{(paaṛú)} `magician', or a~stem with one or more suffixes added to it, such as /deːs-ôːm-iː/ \textit{(dees-óom"=ii)} `of the days', /ʑhôːn-um/ \textit{(ǰhóon"=um)} `I will know'. Even though some combinations of syllables and accents are more common than others and there are restrictions on accent placement (see below), the location of the accent within a~given word is not entirely predictable. Therefore, accent in Palula must be defined as lexical. 


Difference in accent placement is in a~few cases, as in (\ref{ex:3-27}), the only phonemic contrast between two lexical items.


\begin{exe}
\extab
\label{ex:3-27}
\begin{tabularx}{116mm}{ l l l l l }
/ʨûːr/ \textit{(čúur)} &
`four' &
vs &
/ʨǔːr/ \textit{(čuúr)} &
`hot fire'\\
/káti/ \textit{(káti)} &
`saddle' &
&
/katí/ \textit{(katí)} &
`how many?'\\
/dêːdi/ \textit{(déedi)} &
`grandmother' &
&
/děːdi/ \textit{(deédi)} &
`burnt (\textsc{f})'\\
/dhûːra/ \textit{(dhúura)} &
`distant' &
&
/dhuːrá/ \textit{(dhuurá)} &
`separate'\\
\end{tabularx}
\end{exe}

Although this kind of minimal pair is not particularly common in the language, the system definitely allows them to occur.

\subsubsection*{The position of accent on stems}

Accent falls either on the final or the penultimate syllable in a~non"=verbal lexical stem, as seen
in (\ref{ex:3-28}) and (\ref{ex:3-29}); whereas in verbal stems, as in (\ref{ex:3-30}), the accent
is always on the final syllable. In terms of vocalic moras, only one of the three last moras, of any lexical stem in Palula, may be the locus of the pitch accent.


\begin{exe}
\extab
\label{ex:3-28}
\begin{tabularx}{\textwidth}{ l l l }
\multicolumn{3}{l}{Accent on noun stems:}\\
/tôːruɳ/ &
\textit{(tóoruṇ)} &
`forehead'\\
/paːlǎː/ &
\textit{(paalaá)} &
`leaf'\\
/kakarîː/ &
\textit{(kakaríi)} &
`skull'\\
/kuɳôːku/ &
\textit{(kuṇóoku)} &
`puppy'\\
/ɡhaɖeːró/ &
\textit{(ɡhaḍeeró)} &
`elder'\\
\end{tabularx}

\extab
\label{ex:3-29}
\begin{tabularx}{\textwidth}{ l l l }
\multicolumn{3}{ l }{Accent on some other non"=verbal stems:}\\
/típa/ &
\textit{(típa)} &
`now'\\
/pʰaré/ &
\textit{(pharé)} &
`along, toward'\\
/eːtríli/ &
\textit{(eetríli)} &
`the day before yesterday'\\
/taːqatwár/ &
\textit{(taaqatwár)} &
`powerful'\\
/ǒːra/ &
\textit{(oóra)} &
`over here'\\
\end{tabularx}

\extab
\label{ex:3-30}
\begin{tabularx}{\textwidth}{ l l l }
\multicolumn{3}{ l}{Accent on verb stems:}\\
/krín-/ &
\textit{(krín-)} &
`sell'\\
/piʨʰíl-/ &
\textit{(pičhíl-)} &
`slip'\\
/karoːɽé-/ &
\textit{(karooṛé-)} &
`dig, scratch'\\
/nûːʈ-/ &
\textit{(núuṭ-)} &
`return'\\
\end{tabularx}
\end{exe}

\subsubsection*{Accent properties of suffixes}

There are two types of suffixes: those that carry their own accent, which will be referred to as
accent"=bearing suffixes, and those that are accent"=neutral. When a~suffix of the first type is added
to a~stem, the accent of the stem is eliminated, and the word accent falls on the suffix. When
a~suffix of the second type is added to a~stem, the stem accent may be retained. However, under
certain conditions, even in the latter case, the accent (defined by the lexical stem) shifts from
the stem to the suffix, a~matter I will return to in \sectref{subsec:3-5-1}.

\begin{table} 
\caption{Accent"=bearing suffixes}
\begin{tabularx}{\textwidth}{ l l l Q }
\lsptoprule
Suffix &
Function &
Example &
\\\midrule
\textit{-í} &
plural &
\textit{kuḍ-í} &
`walls'\\
\textit{-í} &
oblique, locative &
\textit{dukeen-í} &
`in the shop'\\
\textit{-íim} /îːm/ &
plural oblique &
\textit{dukeen-íim} &
`in the shops'\\
\textit{-í} &
converb &
\textit{ɡhin-í} &
`having taken'\\
\textit{-áan} /âːn/ &
present &
\textit{ɡhin-áan-u} &
`(he) is taking'\\
\textit{-íia} /îjːa/ &
1 plural &
\textit{ɡhin-íia} &
`we will take'\\
\textit{-íl} &
perfective (stem) &
\textit{čhin-íl-i} &
`(was) cut'\\
\textit{-eeṇḍeéu} /eːɳɖěːw/ &
obligative &
\textit{ɡhin"=eeṇḍeéu} &
`has to be taken'\\
\textit{-áaṭ} /âːʈ/ &
agentive &
\textit{čhin-áaṭ-u} &
`the person cutting'\\
\textit{-áai} /âːj/ &
infinitive &
\textit{ɡhin-áai} &
`to take'\\
\textit{-ainií} /ajnǐː/ &
verbal noun &
\textit{ɡhin"=ainií} &
`taking'\\
\textit{-íim} /îːm/ &
copredicative &
\textit{khaṣeel-íim} &
`dragging'\\
\textit{-íǰ} /íʑ/ &
passive (stem) &
\textit{paš-íǰ-aṛ} &
`it will be seen\newline (by you)'\\
\textit{-á} &
causative (stem) &
\textit{pal-á} &
`hide (it)!'\\\lspbottomrule
\end{tabularx}
\label{tab:3-5}
\end{table}


The most productive accent"=bearing suffixes (mainly verbal) are presented in \tabref{tab:3-5}. Some of them also have other allomorphs, such as \textit{-áan}: \textit{-éen}, \textit{-áand}, and \textit{-éend}.


Some of these suffixes may be added cumulatively, in which case the accent is carried by the last accent"=bearing suffix.


As with the accent"=bearing suffixes, only the most productive accent"=neutral suffixes are presented in \tabref{tab:3-6}, excluding possible allomorphs.



\begin{table} 
\caption{Accent"=neutral suffixes}
\begin{tabularx}{\textwidth}{ l@{\hspace{20pt}} l@{\hspace{20pt}} l@{\hspace{20pt}} Q }
\lsptoprule
Suffix &
Function &
Example &
\\\midrule
\textit{-um} &
\textsc{1sg} &
\textit{ɡhín-um} &
`I will take'\\
\textit{-aṛ} /aɽ/ &
\textsc{2sg} &
\textit{ɡhín-aṛ} &
`you (\textsc{sg}) will take'\\
\textit{-a} &
\textsc{3sg} &
\textit{ɡhín-a} &
`he/she/it will take'\\
\textit{-at} &
\textsc{2pl} &
\textit{ɡhín-at} &
`you (\textsc{pl}) will take'\\
\textit{-an} &
\textsc{3pl} &
\textit{ɡhín-an} &
`they will take'\\
\textit{-u} &
\textsc{msg} &
\textit{ɡhináan-u} &
`(he) is taking'\\
\textit{-a} &
\textsc{mpl} &
\textit{ɡhináan-a} &
`(they) are taking'\\
\textit{-i} &
\textsc{f} &
\textit{ɡhinéen-i} &
`(she) is taking'\\
\textit{-a} &
\textsc{pl} &
\textit{díiš-a} &
`villages'\\
\textit{-a} &
\textsc{obl} &
\textit{díiš-a} &
`in the village'\\
\textit{-am} &
\textsc{pl.obl} &
\textit{díiš-am} &
`in the villages'\\
\textit{-ii} /iː/ (B \textit{-e}) &
\textsc{gen} &
\textit{díiš-ii} (B \textit{díiš-e}) &
`of the village'\\\lspbottomrule
\end{tabularx}
\label{tab:3-6}
\end{table}


Some quantitative and qualitative morphophonemic alternations relating to the position of the accent in the word will be dealt with in \sectref{subsec:3-5-1}.


\section{Morphophonology}
\label{sec:3-5}

\subsection{Morphophonemic alternations relating to accent}
\label{subsec:3-5-1}

A number of segmental modifications (primarily in the nominal paradigm) are related to accent, or more precisely to the position of the accent within the word; whether they are described as synchronically productive processes or the result of a~diachronic process, the latter also offering some explanations to the more regular dialectal variation observed. 



\tabref{tab:3-7} illustrates the main types of alternations: \textit{ḍheér--ḍheerí} has an~accent alternating between the stem and an~accent"=bearing suffix without any concomitant segmental modification (see \sectref{subsec:3-4-3} and \tabref{tab:3-5}), \textit{kuṇaák--kuṇaaká} has an~accent shifting from the stem to an~accent"=neutral suffix without any concomitant segmental modification (see \sectref{subsec:3-4-3} and \tabref{tab:3-6}), \textit{ṣiṣ--ṣiṣóom} has an~accent shifting from the stem to an~affix"=neutral suffix with accompanying suffix modification (the unaccented allomorph being \textit{-am}), and \textit{haál--halá} has an~accent shifting from the stem to an~affix"=neutral suffix with accompanying stem vowel modification. These four and some of their more salient subtypes will be further discussed and illustrated in the rest of this subsection.


\begin{table}
\caption{ Accent"=related alternations in the nominal paradigm}
\begin{tabularx}{\textwidth}{ l@{\hspace{30pt}} Q Q Q }
\lsptoprule
Stem &
&
Inflected form &
\\\midrule
\textit{ḍheér}\footnote{In this section, only the Palula common transcription is being used (as in the rest of this work) without any accompanying IPA transcription.} &
`stomach' &
\textit{ḍheer-í} &
`stomachs'\\
\textit{kuṇaák} &
`child' &
\textit{kuṇaak-á} &
`children'\\
\textit{ṣiṣ} &
`head' &
\textit{ṣiṣ-óom} &
`heads (\textsc{obl)}'\\
\textit{haál} &
`plough' &
\textit{hal-á} &
`ploughs'\\\lspbottomrule
\end{tabularx}
\label{tab:3-7}
\end{table}


\subsubsection*{Accent alternation (accent"=bearing suffix) without modification}

A stem inflected with one of the accent"=bearing suffixes presented in \tabref{tab:3-5} will without
exception carry its accent on the accent"=bearing unit of the suffix. Some further examples are given in \tabref{tab:3-8}.


\begin{table}[h]
\caption{ Accent alternating between stem and accent"=bearing suffix}
\begin{tabularx}{\textwidth}{ l@{\hspace{25pt}} Q l@{\hspace{25pt}} Q }
\lsptoprule
Stem &
&
Inflected form &
\\\midrule
\textit{preṣ} &
`mother"=in"=law' &
\textit{preṣ-í} &
`mothers"=in"=law'\\
\textit{keéṇ} &
`cave' &
\textit{keeṇ-í} &
`in the cave'\\
\textit{khoṇḍ} &
`speak!' &
\textit{khoṇḍ-íia} &
`we will speak'\\
\textit{til} &
`walk!' &
\textit{til-áan-a} &
`(they) are walking'\\\lspbottomrule
\end{tabularx}
\label{tab:3-8}
\end{table}

\subsubsection*{Accent alternation (accent"=neutral suffix) without modification}

In some cases, the accent shifts to the suffix in spite of it being an~accent"=neutral suffix. While
the previously mentioned process is fully predictable from the type of suffix itself, the reason for
the current process happening is instead to be found in the lexical stem itself.


For the most part (as far as the nominal paradigm is concerned), this shift takes place when the
lexical stem accent is on the last mora, as in \tabref{tab:3-9}.



\begin{table}[p] 
\caption{ Accent shift from final"=moraic accented stems to accent"=neutral suffix}
\begin{tabularx}{\textwidth}{XXXX}
\lsptoprule
Stem &
&
Inflected form &
\\\midrule
\textit{putr} &
`son' &
\textit{putr-á} &
`sons'\\
\textit{oóṛ} &
`chicken' &
\textit{ooṛ-á} &
`chickens'\\
\textit{dheeṛúm} &
`pomegranate' &
\textit{dheeṛum-á} &
`pomegranates'\\
\textit{atshareét} &
`Ashret' &
\textit{atshareet-á} &
`in Ashret'\\\lspbottomrule
\end{tabularx}
\label{tab:3-9}
\end{table}


While this is true for all polysyllabic stems, there are quite many monosyllabic stems for which accent shift remains non"=predictable from a~purely synchronic perspective. On the one hand, there are those final"=mora accented stems that do not produce an~accent shift (\tabref{tab:3-10}), and on the other hand, there are those non"=final mora accented stems that contrary to expectation do undergo an~accent shift (the latter will be discussed further below).\footnote{An alternative, and possibly more economic, way of describing accent patterns in Palula, would be to regard final"=mora accent as the default accent on any phonological word (whether inflected or not), and any from that deviating placement as specified in the lexicon.}



\begin{table}[p]
\caption{ Stems with final"=mora accent not displaying accent shift}
\begin{tabularx}{\textwidth}{XXXX}
\lsptoprule
Stem &
&
Inflected form &
\\\midrule
\textit{dhut} &
`mouth' &
\textit{dhút-a} &
`mouths'\\
\textit{iṇc̣} &
`bear' &
\textit{íṇc̣-a} &
`bears'\\
\textit{bhruk} &
`kidney' &
\textit{bhrúk-a} &
`kidneys'\\
\textit{haát} &
`hand' &
\textit{háat-a} &
`hands'\\\lspbottomrule
\end{tabularx}
\label{tab:3-10}
\end{table}

\subsubsection*{Accent alternation with suffix modification}

In some cases, an~accented suffix vowel is, qualitatively or quantitatively, modified as compared with an~unaccented allophone. This can be seen in \tabref{tab:3-11}. This particularly concerns the
accent"=neutral plural oblique suffix \textit{-am} and the genitive suffix \textit{-e} (in B only).

The alternation in the paradigms to some extent reflects general vowel shifts in the language
\textit{(a {\textgreater} aa, aa {\textgreater} oo/uu, ee {\textgreater} ii)} conditioned by accent,
and, in the B dialect, also by syllable structure, to which I will have reason to return to in the
discussion below on stem modification.

In the A dialect, \textit{-am} regularly has the form \textit{-óom} when accented.

\begin{table}[p]
\caption{Accent shift with suffix modification (A dialect)}
\begin{tabularx}{\textwidth}{ l@{\hspace{20pt}} l@{\hspace{20pt}} Q Q Q }
\lsptoprule
Stem &
&
Plural nom &
Plural obl &
Plural gen \\\midrule
\textit{deés} &
`day' &
\textit{dees-á} &
\textit{dees-óom} &
\textit{dees-óom"=ii} \\
\textit{ɡhoóṣṭ} &
`house' &
\textit{ɡhooṣṭ-á} &
\textit{ɡhooṣṭ-óom} &
\textit{ɡhooṣṭ-óom"=ii} \\
\textit{kuṇaák} &
`child' &
\textit{kuṇaak-á} &
\textit{kuṇaak-óom} &
\textit{kuṇaak-óom"=ii} \\\lspbottomrule
\end{tabularx}
\label{tab:3-11}
\end{table}


In the B dialect, \textit{\--am} regularly takes the form \textit{-áam} in closed syllables, and \textit{-úum} in open syllables (as, for example, when followed by a~genitive suffix), as illustrated in \tabref{tab:3-12}. The unaccented genitive suffix \textit{-e} in B corresponds regularly to an~accented form \textit{-í}.



\begin{table}[p]
\caption{Accent shift with suffix modification (B dialect)}
\begin{tabularx}{\textwidth}{ Q l l l l l }
\lsptoprule
Stem &
&
Singular gen &
Plural nom &
Plural obl &
Plural gen \\\midrule
\textit{deés} &
`day' &
\textit{dees-í} &
\textit{dees-á} &
\textit{dees-áam} &
\textit{dees}-\textit{úum-e}\\
\textit{ɡhoóṣṭ} &
`house' &
\textit{ɡhooṣṭ-í} &
\textit{ɡhooṣṭ-á} &
\textit{ɡhooṣṭ-áam} &
\textit{ɡhooṣṭ-úum-e} \\
\textit{kuṇaák} &
`child' &
\textit{kuṇaak-í} &
\textit{kuṇaak-á} &
\textit{kuṇaak-áam} &
\textit{kuṇaak-úum-e} \\\lspbottomrule
\end{tabularx}
\label{tab:3-12}
\end{table}

\subsubsection*{Accent alternation with stem modification}

In the nominal paradigm of the A dialect, some vowel modifications affecting the nominative have been blocked by accent shift in the inflected forms, resulting in alternations between \textit{aa} and \textit{a}, which is obvious with the nouns in \tabref{tab:3-13}. The lengthening of the accented vowels has produced a~second"=mora accent in polysyllabic stems and vowels preceded by aspiration (in the wider sense, see \ref{subsec:3-4-1}). 

\begin{table}[p]
\caption{Alternations between \textit{a} and \textit{aa} (A dialect)}
\begin{tabularx}{\textwidth}{ Q Q Q }
\lsptoprule
Stem &
&
Inflected form\\\midrule
\textit{báaṭ} &
`stone' &
\textit{baṭ-á} \\
\textit{heewaán(d)} &
`winter' &
\textit{heewand-á} \\
\textit{dhaán} &
`goat' &
\textit{dhan-á}\\
\textit{sáar} &
`lake' &
\textit{sar-í}\\
\textit{aaṣaáṛ} &
`apricot' &
\textit{aaṣaṛ-í} \\
\textit{c̣haár} &
`waterfall' &
\textit{c̣har-í} \\\lspbottomrule
\end{tabularx}
\label{tab:3-13}
\end{table}


Along the same lines, there are regular alternations (\tabref{tab:3-14}) between accented stems with \textit{ii, oo} (B \textit{uu}) and \textit{aa} vs. unaccented stems with \textit{ee, aa} and \textit{a}.



\begin{table}[p]
\caption{Alternations in the verbal paradigm: \textit{a--áa}, \textit{aa--óo} and \textit{ee--íi}}
\begin{tabularx}{\textwidth}{ Q l Q l }
\lsptoprule
Form with\newline stem accent &
&
Form with\newline suffix accent &
\\\midrule
\textit{páaš-um} &
`I will see' &
\textit{paš-áan-u} &
`(he) is seeing'\\
\textit{ǰhóon"=um} (B~\textit{ǰhúun"=um}) &
`I will know' &
\textit{ǰhaan-áan-u} &
`(he) is knowing'\\
\textit{uḍhíiw"=um} &
`I will escape' &
\textit{uḍheew-áan-u} &
`(he) is escaping'\\\lspbottomrule
\end{tabularx}
\label{tab:3-14}
\end{table}

\subsubsection*{Other alternations}

Some other alternations having to do with the interaction between stem vowels and suffix vowels will
be discussed at length in the chapters on noun and verb morphology, see \chapref{chap:4} and
\chapref{chap:8}, respectively.


\begin{table}[p]
\caption{Alternations between \textit{a--áa}, \textit{áa--úu} and \textit{ée--íi}, respectively (B dialect)}
\begin{tabularx}{\textwidth}{ Q Q Q }
\lsptoprule
Stem &
&
Inflected form\\\midrule
\textit{kaṇ} &
`ear' &
\textit{káaṇa} \\
\textit{kram} &
`work' &
\textit{kráama} \\
\textit{dan} &
`tooth' &
\textit{dáanda} \\
\textit{ooḍháal} &
`flood' &
\textit{ooḍhúula} \\
\textit{sáan} &
`pasture' &
\textit{súuna} \\
\textit{baazáar} &
`bazaar' &
\textit{baazúura} \\
\textit{méeš} &
`man' &
\textit{míiša} \\
\textit{šéen} &
`bed' &
\textit{šíina} \\\lspbottomrule
\end{tabularx}
\label{tab:3-15}
\end{table}
\clearpage
\subsection{Morphophonemic alternations relating to syllable structure}
\label{subsec:3-5-2}

Morphophonemic alternations relating to syllable structure, briefly touched upon above (see \tabref{tab:3-12}), is exclusively a~feature of the B dialect. The alternations are between closed syllables with \textit{ee, aa}, and \textit{a} and open syllables with \textit{ii, uu}, and \textit{aa}. It is most clearly displayed in the nominal paradigm, as shown in \tabref{tab:3-15}.





\subsection{Umlaut}
\label{subsec:3-5-3}

There are numerous examples, 
in the nominal (\tabref{tab:3-16}) and verbal paradigms (\tabref{tab:3-17}) of anticipatory fronting (``umlaut'') of \textit{aa} to \textit{ee} when preceding an~\textit{i} in the following syllable. The anticipation normally does not occur if the \textit{a} is short. The nouns `glass' and `book' in \tabref{tab:3-16} show that this process has been productively extended even to relatively recent loans. 


\begin{table}[b]
\caption{Alternations in the nominal paradigm between \textit{aá} and umlaut"=\textit{ee} }
\begin{tabularx}{\textwidth}{ Q Q Q Q }
\lsptoprule
Form without umlaut &
&
Form with umlaut &
\\\midrule
\textit{baát} &
`word, issue' &
\textit{beetí} &
`words, issues'\\
\textit{ɡilaás} &
`glass' &
\textit{ɡileesí} &
`glasses'\\
\textit{hiimaál} &
`glacier' &
\textit{hiimeelí} &
`glaciers'\\
\textit{kitaáb} &
`book' &
\textit{kiteebí} &
`books'\\\lspbottomrule
\end{tabularx}
\label{tab:3-16}
\end{table}






\begin{table}[b]
\caption{Alternations in the verbal paradigm between \textit{áa} and umlaut"=\textit{ee} }
\begin{tabularx}{\textwidth}{ l Q l Q l l }
\lsptoprule
\multicolumn{2}{l}{Form without umlaut} &
\multicolumn{2}{l}{Form with umlaut} &
\multicolumn{2}{l}{Form with umlaut}\\\midrule
\textit{mhaaráanu} &
`(he) is killing' &
\textit{mheerí} &
`having killed' &
\textit{mheerílu} &
`killed'\\
\textit{phaaláanu} &
`(he) is splitting' &
\textit{pheelí} &
`having split' &
\textit{pheelílu} &
`split'\\
\textit{ǰhaanáanu} &
`(he) is knowing' &
\textit{ǰheení} &
`having known' &
\textit{ǰheenílu} &
`knew'\\\lspbottomrule
\end{tabularx}
\label{tab:3-17}
\end{table}

\clearpage
In addition to that, umlaut is applied to: a) verbal suffixes (or the final part of the perfective stem) in anticipation of a~following adjectival feminine agreement suffix -\textit{i} (\tabref{tab:3-18}), b) the adjectival stem in anticipation of a~feminine agreement suffix (\tabref{tab:3-19}), and c) to derivations of various kinds in which the derivational suffix contains \textit{i} (\tabref{tab:3-20}).



\begin{table}[t]
\caption{Umlaut in verbal suffixes anticipating feminine agreement suffixes}
\begin{tabularx}{\textwidth}{ l@{\hspace{30pt}} Q l@{\hspace{30pt}} Q }
\lsptoprule
\multicolumn{2}{l}{Form without umlaut} &
\multicolumn{2}{l}{Form with umlaut}\\\midrule
\textit{mhaaráanu} &
`(he) is killing' &
\textit{mhaaréeni} &
`having killed'\\
\textit{phaaláanu} &
`(he) is splitting' &
\textit{phaaléeni} &
`having split'\\
\textit{phooṭóolu} &
`broke (\textsc{msg)}' &
\textit{phooṭéeli} &
`broke (\textsc{fsg)}'\\
\textit{mučóolu} &
`opened (\textsc{msg)}' &
\textit{mučéeli} &
`opened (\textsc{fsg)}'\\
\textit{láadu} &
`found (\textsc{msg)}' &
\textit{léedi} &
`found (\textsc{fsg)}'\\
\textit{nikháatu} &
`(he) appeared' &
\textit{nikhéeti} &
`(she) appeared'\\\lspbottomrule
\end{tabularx}
\label{tab:3-18}
\end{table}




 


\begin{table}[b]
\caption{Umlaut in adjectival stems anticipating feminine agreement suffixes}
\begin{tabularx}{\textwidth}{ l@{\hspace{20pt}} Q l@{\hspace{20pt}} Q }
\lsptoprule
\multicolumn{2}{l}{Form without umlaut} &
\multicolumn{2}{l}{Form with umlaut}\\\midrule
\textit{paṇáaru} &
`white \textsc{(msg)}' &
\textit{paṇéeri} &
`white \textsc{(fsg)}'\\
\textit{táatu} &
`hot \textsc{(msg)}' &
\textit{téeti} &
`hot \textsc{(fsg)}'\\
\textit{sóoru} &
`fine, whole (\textsc{msg)}' &
\textit{séeri} &
`fine, whole (\textsc{fsg)}'\\\lspbottomrule
\end{tabularx}
\label{tab:3-19}
\end{table}






\begin{table}[b]
\caption{Umlaut in derivations}
\begin{tabularx}{\textwidth}{ l@{\hspace{25pt}} Q l@{\hspace{25pt}} Q }
\lsptoprule
\multicolumn{2}{l}{Corresponding form without umlaut} &
\multicolumn{2}{l}{Derived form with umlaut}\\\midrule
\textit{káaku} &
`older brother' &
\textit{kéeki} &
`older sister'\\
\textit{kuṇóoku} &
`puppy' &
\textit{kuṇéeki} &
`female dog'\\
\textit{ɡhuaaṇaá} &
`Pashto (language)' &
\textit{ɡhueeṇíi} &
`Pashtun (person)'\\
\textit{bakaraál} &
`shepherd' &
\textit{bakareelí} &
`shepherding'\\\lspbottomrule
\end{tabularx}
\label{tab:3-20}
\end{table}