\documentclass[output=paper]{langsci/langscibook}
\ChapterDOI{10.5281/zenodo.4680294}

\author{András Bárány\affiliation{Bielefeld University} and
        Theresa Biberauer\affiliation{University of Cambridge, Stellenbosch
        University \& University of the West Cape} and
        Jamie Douglas\affiliation{University of Cambridge} and
        Sten Vikner\affiliation{Aarhus University}}

\title{Introduction}

\abstract{}

\begin{document}
\maketitle

\noindent The three volumes of \emph{Syntactic architecture and its
consequences} present contributions to comparative generative linguistics that
\enquote{rethink} existing approaches to an extensive range of phenomena,
domains, and architectural questions in linguistic theory. At the heart of the
contributions is the tension between descriptive and explanatory adequacy which
has long animated generative linguistics and which continues to grow thanks to
the increasing amount and diversity of data available to us. As the three
volumes show, such data from a large number of understudied languages as well
as diatopic and diachronic varieties of well-known languages are being used to
test previously stated hypotheses, develop novel ideas and expand on our
understanding of linguistic theory.

The volumes feature a combination of squib- and regular-length discussions
addressing research questions with foci which range from micro to macro in
scale. We hope that together, they provide a valuable overview of issues that
are currently being addressed in generative linguistics, broadly defined,
allowing readers to make novel analogies and connections across a range of
different research strands. The chapters in Volume 1, \emph{Syntax inside the
grammar}, and Volume 2, \emph{Between syntax and morphology}, address issues at
the syntactic interfaces and in morphosyntax, such as language change,
complexity, and variation, as well as syntactic categories, constituent orders,
and demonstrative systems.

The contributions to the present, third volume, \emph{Inside syntax}, develop
novel insights into a number of core syntactic phenomena, such as the structure
and properties of relative clauses, constituent orders, demonstrative systems,
case and agreement splits, and the syntax of null elements. The volume is
divided into two parts, \emph{Case and agreement} (Part I), and \emph{Null
syntax} (Part II).

The chapters in Part I, \emph{Case and agreement}, address case and agreement
in different domains across languages, from both conceptual and empirical
perspectives. A novel approach to alignment typology is proposed by Zwart and
Lindenbergh, Alexiadou and Anagnostopoulou provide a fresh take on the nature
of nominative case, and Baker and Vinokurova address the nature of structural
case by considering the properties of partitive case in Sakha. Kallulli’s
contribution, in turn, deals with deponents and how they affect properties of
verbal paradigms. Several chapters deal with the interaction of case and
agreement: Den Dikken and Dékány reconsider \citegen{Roberts2010} notion of
“defective goals” and how it applies to clitics and noun incorporation, whilst
clitics in French ditransitive constructions are the topic of Sportiche’s
chapter. Holmberg discusses case and agreement in the nominal domain, namely in
possessive noun phrases, while Galves and Avelar compare case and agreement in
Romance and Bantu. A further domain of agreement is discussed by Corver, who
focuses on inflected modifiers in the Dutch noun phrase. The chapters by van
Riemsdijk and Taraldsen deal with mismatches in case and agreement
respectively, namely mismatches found in relative clauses and nominal phrases
(\enquote{unagreement}).

Part II, \emph{Null syntax}, deals with different types of null elements that
have been assumed in syntactic theory in recent decades. The chapters by
Wurmbrand, Sevdali and Sheehan, as well as Bobaljik, respectively address
implicit, partial and absolutive control in a number of different languages.
Michelioudakis’ contribution takes a new look at the syntax of implicit agents,
while Rizzi revisits the empty category principle or what is left of it in the
21st century. The final chapters deal with null categories in Brazilian
Portuguese: Kato and Duarte explore parametric variation in null subjects,
while Cyrino relates Brazilian Portuguese null objects to differential object
marking in Spanish.

The contributions to this volume, many of which have been influenced and
inspired by \textcite{Roberts2010,Roberts2012},~\textcite{RobRou2003},
\textcite{RobHol2010}, and \textcite{BibRob2015}, thus provide varied
perspectives on syntactic variation in diathesis and agreement, the interaction
of case and agreement, and the nature of null elements.

{\sloppy
\printbibliography[heading=subbibliography,notkeyword=this]
}

\end{document}
