\documentclass[output=paper]{langsci/langscibook} 
\ChapterDOI{10.5281/zenodo.1181801}
\author{Manfred Krug\lastand
Lukas Sönning\affiliation{University of Bamberg}}
\title{Language change in Maltese English: The influence of age and parental languages}
\shorttitlerunninghead{Language change in ME: The influence of age and parental languages}
%\ChapterDOI{} %will be filled in at production
\abstract{In this study, which is based on questionnaire data collected in 2013 from 430 Maltese informants, we investigate ongoing language change in Maltese English. We concentrate on 63 pairs of lexical variants that are known to differ in usage between British English and American English (e.g. \textit{vacation} vs. \textit{holiday}). Overall, informants clearly tend towards BrE usage. Regardless of the statistical approach we adopt, our studies show consistently apparent-time trends towards a less exclusively British English usage in Malta, converging on a more globalized usage of lexical items, in particular among the youngest cohorts. This confirms trends reported for older Maltese English data (collected in 2008; see \citealt{Krug2015}). While Age emerges as the most important factor in our data, lexical choices are also sensitive to the native languages of the informants’ parents. When the mother’s native language(s) includes English, the informants’ lexical choices are biased in the expected direction, figuring in an increase in Britishness of the informants. Informants whose parents’ L1 is neither English nor \ili{Maltese} show the highest degree of linguistic globalization. Overall, the native language(s) of the mother appeared to be more influential than that of the father. 
}
\maketitle

\begin{document}



\section{Introduction}

\ili{Maltese} and English are the two official languages in the Republic of \isi{Malta}. Not surprisingly, therefore, bilingualism is widespread: In the \citet{census} (the latest census which collected such information), 88\% of the population aged 10 and older, i.e. some 300,000 people living on the archipelago, reported to speak at least some English. With 93\% of the population speaking \ili{Maltese} as a first language (\textit{ibid.}), English is a second language for the vast majority of speakers. Frequency of use and exposure to the English language vary considerably, however. About 9\% of the population use English as the – or a – main language at home. As is often the case in places with a colonial history involving British rule, the varieties of English that are spoken in \isi{Malta} represent in actual fact a continuum between an acrolectal variety (a near-RP \isi{pronunciation} with a grammar and lexicon that is very similar to standard \ili{British English}) on the one hand, and basilectal varieties on the other. The latter are characterized by typical EFL learner features and more structural parallels with \ili{Maltese}, i.e. contact features, plus extensive code-switching.\footnote{Compare the continuum described in \citet{Vella1994}, \citet{Bonnici2010} and such notions as \textit{mixed Maltese English}. Compare also the discussion of \ili{Maltese English} and its relation to \citegen{Schneider2007} model of postcolonial Englishes in \citet{Thusatetal2009}, \citet{Bonnicietal2012} and \citet{Grech2015}.} In this contribution on lexical items usage, we will use the term “\ili{Maltese English}” (or “\ili{MaltE}”, for short) to cover the entire continuum of varieties of English spoken and written in \isi{Malta}.

In this study, which is based on questionnaire data from 430 Maltese informants collected in 2013, we investigate ongoing change in English language usage in \isi{Malta}. In 2007 and 2008, pilot web-based studies and first questionnaire-based studies were carried out to empirically investigate the varieties of English in \isi{Malta} and their relation to the major reference varieties of standard British and American English. In the present study, we will concentrate on data from 2013, which have not been subjected to statistical analysis to date. After an outline of the methodologies employed, §2 will present descriptive statistics, §3 inferential statistics. The first focus of our analysis will be apparent-time studies, i.e. the factor Informant Age. We shall also investigate closely the influence of participants’ gender, the \isi{native language}(s) of the mothers and fathers of the informants as well as interactions between these factors.



\section{Methodology}

For the statistical analyses presented in this paper, informants with
more than one third missing answers in the questionnaire were excluded
(\textit{n} = 6), leaving a total of 424 informants for analysis. We
concentrate on 68 pairs of lexical variants using the \textit{Bamberg
  questionnaire for lexical and morphosyntactic variation in English}
(see Appendix for exemplification; \citealt{KrugSell2013} for
methodological detail; \citealt{KrugetalF}
  for the full questionnaire). Given a choice between two
  referentially synonymous items that are known to have differed in
  usage between \ili{British English} (\ili{BrE}) and American English (AmE) in
  the late 20\textsuperscript{th} century (cf.,
  e.g. \citealt{Algeo2006}), informants select whether they always use
  one of the two variants, prefer one over the other, have no
  preference, or do not use any of them (see Figure~\ref{fig:krug:app1} in the Appendix%1
  ). For
  expository clarity, we will use expressions such as \textit{pairs,
    binaries} and \textit{British} vs. \textit{American English
    usage}, although these are clearly simplifications. Some items
  have more than two alternatives, e.g. \textit{dummy – pacifier –
    soother} (or \textit{compare X with/to/and Y}).\footnote{In the
    questionnaire, raters can add comments regarding their own
    preference and alternative terms in each case (see Figure~\ref{fig:krug:app1} in the Appendix%1
    ).}
  Similarly, we simplistically use \textit{BrE} (or \textit{AmE}, as
  the case may be) when we refer to ‘more British’
  (e.g. \textit{backwards} vs. \textit{backward}), ‘exclusively
  British’ (e.g. \textit{-isation} spellings) or ‘traditionally
  British’ items (e.g. \textit{lorry} vs. \textit{truck}).

Items exceeding 20\% missing cases (i.e. informants ticking that they use neither of the two variants offered) were excluded from the analysis (\textit{n} = 5). These were \textit{bicentenary/bicentennial, glocalis/zation, storm in a teacup/tempest in a tea\-pot}, \textit{laund(e)rette/laundromat,} and \textit{a drop in the ocean/bucket}. Consequently, 63 items remain in the ensuing analysis.

The questionnaire data were converted from ordinal ratings into numerical values (cf. \citealt{Rohrmann2007}; \citealt{AgrestiFinlay2009}: 40 for translating ordinal into interval scales), with usage preferences ranging from --2 for exclusively AmE usage to +2 for exclusively \ili{BrE} usage. For the statistical analysis of the questionnaire, the following values are assigned to the five possible answers:

\begin{itemize}
\item +2 if the \isi{informant} reports consistent use of the (more) British variant;
\item +1 if the \isi{informant} reports more frequent use of the (more) British variant;
\item 0 if the \isi{informant} has no preference;
\item --1 if the \isi{informant} reports more frequent use of the (more) American variant;
\item --2 if the \isi{informant} reports consistent use of the (more) American variant;
\item no entry if the \isi{informant} claims to use neither of the two variants.
\end{itemize}

Automatic digitization was combined with extensive manual post-editing. The following analyses are based on the overall mean questionnaire score for each of our 424 informants, which may range from --2 to +2. To safeguard against distorted averages, we imputed missing item values based on the full set of informants’ ratings. More specifically, we applied mean imputation adjusted by subject and item effects. That is, the imputed value for each cell reflected the overall tendency for the \isi{informant} (with subjects showing an overall trend towards British usage receiving higher fill-in values) and item (with items showing bias towards the British variant receiving higher fill-ins). To this end, we ran a mixed-effects model (using the lme4 package in R, \citealt{BatesEtAl2015}) with subject and item as random factors and then derived fill-in values by adding to the intercept the random effects for the particular cells. This procedure takes into account differences between items and informants. A reanalysis of the data shows that our conclusions are not affected by this imputation (compared to an analysis simply excluding missing cases). 

\section{Results}
\subsection{Descriptive statistics}
\subsubsection{Methodological caveats and major trends}

As for the reliability of the findings presented here, the lower the proportion of respondents opting for “I never use either expression” in our questionnaire, the more confident we can be about the results. %Appendix~2
Figure~\ref{fig:krug:app2} in the Appendix gives the proportion of respondents reporting to use none of the two binaries. Out of our 68 lexical binaries, 63 have over 85\% of informants responding that they use one or both of the variants. The remaining five binaries have been excluded from the analysis, since they have between 20\% and 35\% missing cases. We would have to exert greater caution in interpreting the results because of a significant gap (greater than 10\%) between them and the remaining items (see \figref{fig:krug:1}). The set of excluded items is interesting, nevertheless: Two learned words known primarily from formal and academic discourse (\textit{bicentenary/bicentennial} and \textit{glocalis/zation}) score lowest of all 68 items. In addition, there are two phraseological units (\textit{a drop in the ocean/bucket} and \textit{a storm in a teacup/a tempest in a teapot}), which seem infrequent in \ili{MaltE}. Notice that another phraseological unit – with only 10\% missing cases, however – ranks sixth lowest in terms of usage rate: \textit{touch wood/knock on wood}. This suggests that idioms may be relatively rare in current, mainstream \ili{MaltE}. The fifth binary with many missing cases is \textit{launderette} vs. \textit{laundromat}, which appears to be an uncommon concept in the 21\textsuperscript{st} century when most members of Western societies have access to washing machines in their homes (or dorms, condominiums etc). 

  
\begin{figure}
\includegraphics[width=0.8\textwidth]{figures/a10KrugSoenning-img1.png}
\caption{Proportion of respondents reporting to use none of the two given lexical alternatives: The 10 least frequently used lexical items (traditionally BrE/AmE terms)}
\label{fig:krug:1}
\end{figure}

Let us turn to the overall results. Unsurprisingly, given the history of English in \isi{Malta}, half a century after independence, the Maltese informants still clearly tend towards \ili{BrE} usage. The overall mean in 2013 is +0.85 (the standard deviation being 0.35; the overall median +0.87). This compares with an arithmetic mean of +1.0 in 2008 (see \citealt{KrugRosen2012} for detail). On average, therefore, Maltese informants reported more frequent use of the \ili{British English} terms, but the preference appears to have somewhat weakened over the five years between 2008 and 2013. 

\begin{figure}
\includegraphics[width=\textwidth]{figures/a10KrugSoenning-img2.png}
\caption{Distribution of the 63 lexical binaries}
\label{fig:krug:2}
\end{figure}

\figref{fig:krug:2} shows a dot plot (cf. \citealt{Soenning2016}) of the mean rating and standard deviation for each individual pair. Clearly, the preference for (more) British terms comes in degrees:

\begin{itemize}
\item British usage dominates for the vast majority of our lexical items (52 of the 63 binaries have values greater than 0). 
\item About half of the binaries display strong \ili{BrE} preferences and have means greater than 1, with some items being used almost exclusively in the \ili{BrE} variant (e.g. \textit{postman, roundabout, pushchair, petrol, football}).
\item Hovering around an arithmetic mean of about 0, eight items display a fairly neutral usage. This is to say that they are used~– on average – interchangeably by individual informants or that their \ili{BrE} and AmE variants are preferred by comparable numbers of informants (or strongly preferred by some and slightly dispreferred by twice as many). In descending order from slightly more British to slightly more American usage, these are: \textit{to let} vs. AmE \textit{for rent; (potato) crisps} vs. AmE (\textit{potato}) \textit{chips; jacket potato} vs. AmE \textit{baked potato; compare X to/with Y; package}, which is slightly preferred over \ili{BrE} \textit{parcel.} A special case is the \isi{verb} \textit{to license} vs. \textit{to licence}. Here we may assume that people freely choose, possibly due to a potentially unknown minimal and non-systematic spelling difference between an allegedly \ili{BrE} and AmE variant. Interestingly, this item has the highest standard deviation in our data set (of nearly 2). Most informants therefore reported using only one of the two options rather than having genuinely free variation as individuals. It would be interesting to compare actual corpus citations in \isi{Malta} and also intuition- as well as corpus-based data from other varieties worldwide, including the reference varieties of British and American English. 
\item Few items are preferred in their (traditionally) more American
  form, the most striking one being
  \textit{truck}\footnote{Cf. \citealt{oed} \textit{Online}
    \textit{s.v.}  \textit{truck} n.2 (meaning 3g), whose earliest
    citations for ‘a motor vehicle for carrying goods and troops etc.’
    are Canadian (from 1916) and U.S. American (1930), the latter
    actually pointing to British-American differences: “American
    English has universally chosen \emph{motor truck} and \emph{truck}
    rather than \emph{auto-truck} or the British \emph{lorry}”
    (\textit{AmericanSpeech 5, 274}). Despite a first British
    attestation from 1932, even a quotation from 1950 points to
    transatlantic differences: “Many soldiers in the last war will
    remember that ‘gas’ might or might not be petrol and a ‘truck’
    might or might not be a lorry” (\textit{Times}, 27 Apr. 1950,
    6/7). We are grateful to an anonymous reviewer for pointing out
    that \textit{truck} is a relatively old English loanword in
    \ili{Maltese}, as is indicated by a \ili{Semitic} \isi{plural} form (\textit{trakk-ijiet}),
    which differs from the \textit{-s} plurals of more recent loanwords like
    \textit{film-s}, \textit{printer-s}, \textit{kompjuter-s}. In the absence of \isi{phonological}
    factors in the \isi{plural} formation of \ili{Maltese} loanwords (like
    \isi{sonority} of the stem-final \isi{phoneme}), we can only reconcile such
    observations with the textual evidence presented from the \textit{OED} by
    assuming that, while there was early variation (from at least 1932
    onwards) in \ili{British English} between \textit{truck} and \textit{lorry}, the
    predominant \ili{BrE} lexical choice remained \textit{lorry} until at least the
    1950s. Further factors may be meaning specification (for instance
    in military domains) and the co-existence of different meanings of
    \textit{truck} (especially ‘large motor vehicle for carrying goods’ and
    ‘smaller motor vehicle with an open, load-carrying surface’),
    although in our questionnaire we refer explicitly to a ‘large
    motor vehicle for carrying goods by road’ (see Appendix, Figure~\ref{fig:krug:app1}).},
  which is strongly favoured over (erstwhile) \ili{BrE} \textit{lorry} with
  a mean value of about --1.5. \textit{Sick}, too, is preferred to
  \textit{ill}. A special case is \textit{while}, which is clearly
  also the unmarked choice in modern British usage when compared to
  \textit{whilst}. \textit{Forward}, too, is special because the \isi{noun}
  (known from football) may have played a role in the informants’
  ratings. This is suggested by a strong preference for
  \textit{backwards} (to AmE \textit{backward}) among the Maltese
  informants, which is not commonly used nominally. The preference for
  \textit{sports} (over traditionally \ili{BrE} \textit{sport}) seems to be
  an analogy to other school subjects and disciplines like
  \textit{physics} or \textit{linguistics.}
\end{itemize}




Prior to conducting the questionnaire study on a large scale and in different regions of the world, internet-based data were collected on the Maltese domain .mt for the lexical binaries, first in order to test feasibility and hypotheses, e.g. relating to colonial lag (by comparing the British and US-American domains .uk and .us) and later to check the reliability and external validity of intuition-based data. The internet data for items that eventually figured in the questionnaires are provided in Table~\ref{tab:krug:app3} %Appendix 3
in the Appendix. Three items from the questionnaire were excluded from the internet ranking: 

\begin{itemize}
\item 
the American alternative to \textit{biscuit}, i.e. \textit{cookie} (when used generically for something sweet and crispy, not necessarily containing chocolate chunks) occurs commonly as a digital cookie (‘authentication method’, ‘trace of visited websites’) on the internet; 
\item 
a digital \textit{shopping trolley} is virtually always a \textit{shopping cart};
\item 
an internet \textit{chemist’s} is virtually always a \textit{drugstore} or figures under a certain brand name; also the online occurrences of \textit{chemist’s} would have to be disambiguated because the term routinely refers to a profession not targeted in the questionnaire proper.
\end{itemize}

Rank-based correlations for the remaining items show a highly significant association between web frequencies and pilot questionnaire data from 2008, with Spearman’s rank correlation at \textit{r} = 0.63, \textit{p} < .0001, 95\% CI [0.44; 0.76] (for comparison: Kendall’s tau = 0.43, \textit{p} < .0001; 95\% CI [0.30; 0.58]). Needless to say, such strongly correlated ranks enhance considerably the reliability of both independently collected data sets (and conclusions drawn therefrom) as the likelihood that two rankings consisting of 65 items spuriously produce highly significant correlations approximates zero.

\subsubsection{Apparent-time distributions and diachronic trends}


\begin{figure}  
\includegraphics[width=0.8\textwidth]{figures/a10KrugSoenning-img3.png}
\caption{Mean score by Age (all informants): Least-squares regression line and lowess smoother}
\label{fig:krug:3}
\end{figure}

Informant Age is the single-most influential factor in our data, explaining the largest share of the variation found (cf. §3.2 below for inferential statistics). \figref{fig:krug:3} displays a clear trend: The younger the informants become, the more likely they become to use American forms. Whichever method we apply, the mean difference between our oldest and youngest informants is about 0.5. The right panel in Figure~3 overlays a lowess smoother \citep{Cleveland1979}, an exploratory tool that detects non-linearity in the data. The latter throws into relief that roughly between the ages 40 and 15, there is a pronounced increase in Americanization. In fact, \textit{globalization} (on which see \citealt{Krugetal2016}) would be the more adequate label here because almost all Maltese informants still have positive, i.e. British, arithmetic means; they are merely becoming more neutral, as it were, as they approach a mean value of 0. Owing to the high number of respondents in the relevant age cohorts of the present study, we are quite confident that this is not a spurious finding and would attribute this strengthening of an already existing trend to increased language contact with non-\ili{British English} for speakers under 40. Conceivable is also the weakening of prescriptive \ili{BrE} pressures in education. It seems noteworthy, therefore, that our data suggest a rough temporal correlation between speeded-up globalization for informants under 40 and a change in teacher education: Until the 1970s, teachers were mostly trained by British personnel, but more recently teachers have been trained by \isi{bilingual} \ili{Maltese} native-\isi{speaker} scholars at the University of \isi{Malta} (\citealt[192]{Calleja1994}, \citealt[2]{MartinelliRaykov2014}). 



\subsubsection{Influence of parental languages}

For the analysis of the effect of parental native languages, three groups were compared: (i) \ili{Maltese} as L1, (ii) English or both English and \ili{Maltese} as L1, and (iii) other L1s. \tabref{tab:krug:1} shows the cross-tabulated distribution of Mother’s and Father’s \isi{native language} for the 424 respondents in the analysis. Rather unexpectedly, overall the parents’ native languages are very similarly distributed: around 85\% of both fathers (\textit{n} = 342) and mothers (\textit{n} = 348) speak exclusively \ili{Maltese} as a \isi{native language}; around 10\% have fathers (\textit{n} = 43) or mothers (\textit{n} = 42) who speak English (plus possibly \ili{Maltese}) as a \isi{native language}. And around 6\% to 7\% have mothers (\textit{n} = 24) or fathers (\textit{n} = 29) whose L1 is neither English nor \ili{Maltese}. Furthermore, there seems to be an association between Mother’s and Father’s L1. This is especially noticeable for the groups “E(+M)” and “Other”. Thus, informants with one parent who speaks English as \isi{native language} have a disproportionate likelihood of the second parent also speaking English as an L1 (that share is about 50\%; compared to an overall share of 10\% in the sample). The same is true for parents with an L1 different from English and \ili{Maltese}.

\begin{table}
\caption{Distribution of parents’ native language(s)}
\label{tab:krug:1}

\begin{tabular}{lrrrr@{~~~~~~~~}r}
\lsptoprule
&  & \multicolumn{3}{c}{\bfseries Father} & \bfseries Total of \\
&  & \bfseries M & \bfseries E(+M) & \bfseries Other & \bfseries mothers \\
\midrule 
& M & 320 & 21 & 7 & \textbf{348}\\
\textbf{Mother} & E(+M) & 15 & 20 & 7 & \textbf{42}\\
% \hhline{~-----} 
& Other & 7 & 2 & 15 & \textbf{24}\\
\midrule
\multicolumn{2}{c}{\textbf{Total of fathers}} & \textbf{342} & \textbf{43} & \textbf{29} & \textbf{414}\\
\lspbottomrule
\end{tabular}
\end{table}

\figref{fig:krug:4} shows the distribution of questionnaire scores by Mother’s and Father’s \isi{native language}, respectively. As is evident from the left panel, informants with a mother whose L1 background includes English are more likely to tend towards British lexical choices than informants whose mother’s L1 is exclusively \ili{Maltese} or a language other than \ili{Maltese} or English. If the mother’s \isi{native language} is neither English nor \ili{Maltese}, then both mean and median values are lower, i.e. more American or globalized. The right panel shows that in our data exactly the same tendencies obtain for the influence of the \isi{native language}(s) of the informants’ fathers as for their mothers. Such results are not counterintuitive: For one, native speakers of English in \isi{Malta} (especially of the older, parental generations) tend to be British or oriented towards \ili{BrE} usage. For another, those speakers in \isi{Malta} whose \isi{native language} is neither English nor \ili{Maltese} are statistically more likely to have been in contact with or exposed to (more) American English or to non-native (and thus potentially more globalized) varieties than native English speakers in \isi{Malta}.


\begin{figure}[t]
\includegraphics[width=0.8\textwidth]{figures/a10KrugSoenning-img4.png}
\caption{Mean questionnaire score by Mother’s and Father’s native language(s)}
\label{fig:krug:4}
\end{figure}

\tabref{tab:krug:2} shows that both means and medians are strikingly similar in each row. The closest value for every mean and median of the Father’s \isi{native language} is that in the respective cell of the Mother’s \isi{native language}. In other words:

\largerpage[1.5]
\begin{itemize}
\item When an \isi{informant}’s mother’s L1 is (only) \ili{Maltese}, he/she has almost the same mean and median as an \isi{informant} whose father’s L1 is (only) \ili{Maltese}.
\item When an \isi{informant}’s mother’s L1 is English (plus possibly \ili{Maltese}), he/she has almost the same mean and median as an \isi{informant} whose father’s L1 is English (plus possibly \ili{Maltese}).
\item When an \isi{informant}’s mother’s L1 is neither English nor \ili{Maltese}, he/she has almost the same mean and median as an \isi{informant} whose father’s L1 is neither English nor \ili{Maltese}.
\end{itemize}
\newpage 

\begin{table}
\caption{Father’s and Mother’s native language: Frequency (n) and mean (M) and median (Mdn) questionnaire score}
\label{tab:krug:2}
\begin{tabularx}{\textwidth}{lr rrr rrr} 
\lsptoprule
& \multicolumn{3}{c}{ \bfseries Father} &  & \multicolumn{3}{c}{ \bfseries Mother}\\
\bfseries Native language & \bfseries n & \bfseries M & \bfseries Mdn &  & \bfseries n & \bfseries M & \bfseries Mdn\\
\midrule 
\ili{Maltese} & 342 & 0.85 & 0.87 &  & 349 & 0.84 & 0.86\\
English (or English and \ili{Maltese}) & 43 & 0.92 & 1.00 &  & 42 & 1.00 & 1.04\\
Other & 29 & 0.65 & 0.66 &  & 24 & 0.62 & 0.68\\
\lspbottomrule
\end{tabularx}
\end{table}

It appears reasonable to assume that the relative influence of mother’s and father’s language use may differ between male and female informants. Specifically, boys may be more likely to identify with their fathers and thus more likely to adopt the linguistic behaviour of the father (see \citealt{Hurdetal2009} on role models). While the same may hold for girls and mothers, we may also speculate that, on average, language contact between mothers and children is generally higher, which would suggest that boys and girls are influenced by their mother in similar ways. To explore possible role model effects of fathers (on sons) as well as language contact-induced levelling of influence of mothers on children in general, we carried out subgroup analyses. \figref{fig:krug:5} shows the influence of Mother’s and Father’s L1 separately for male and female informants. Indeed, there appears to be an interaction between Gender and parental L1. For Mother’s L1, male and female informants show the same pattern of influence. The effect is more pronounced for male informants, however. The effect of Father’s \isi{native language}, on the other hand, in fact appears to differ for male and female informants. The rightmost panel in \figref{fig:krug:5} suggests that for female informants, there is no effect of their father’s \isi{native language} on the use of lexical binaries; however, there emerges an interesting pattern for men: If male informants have a father whose \isi{native language} is (or includes) English, these informants’ use of lexical items receives a boost towards traditionally British terms. The mean values closest to the neutral zero, and thus the highest degrees of linguistic globalization, are found for male informants whose fathers’ \isi{native language} is neither English nor \ili{Maltese}. Our data therefore suggest that male \ili{Maltese} are more strongly influenced by their father’s L1 than female language users of English in \isi{Malta}, at least as far as lexical usage is concerned. 

\begin{figure}
\includegraphics[width=\textwidth]{figures/a10KrugSoenning-img5.png}
\caption{Interaction of Mother’s and Father's native language with Gender}
\label{fig:krug:5}
\end{figure}

The descriptive analyses in the present section have identified various potential factors for the choice between British and American lexical binaries in our data set. The following section will elaborate on these aspects with the help of inferential statistical analyses. The descriptive trends will be subject to statistical significance tests in order to determine whether the patterns in our sample can be generalized to the population of speakers of \ili{Maltese English}. 

\subsection{Inferential analysis: The interaction of informants’ age and their parents’ native languages}


In essence, the aim of the inferential analysis is to determine the degree of (un-)\-certainty associated with each of the potential factors and trends outlined above. The pool of explanatory factors is reduced to a set of predictors whose effect is generalizable to the population, i.e. whose pattern of influence is relatively robust in the sense that it can be relied on with a sufficient degree of confidence. We use multiple linear regression to model questionnaire score as a function of potential explanatory factors. The selection of variables for the final model was based on the criterion of statistical significance. As can be seen in \ref{fig:krug:3},  the distribution of Age in our sample is skewed, with an overrepresentation of younger informants. The variable was therefore log-transformed before modelling. 

In the process of model selection, \isi{informant} Gender was removed since – in the presence of the other factors – this variable did not contribute significantly to the explanation of lexical choices, i.e. the usage of more or less British and American variants. Nor did the interactions between either one of the parental native languages and \isi{informant} Gender pass the critical threshold. On the basis of the present data, therefore, the different patterns identified in \figref{fig:krug:5} for male and female participants cannot be generalized to the entire population of \ili{MaltE} speakers. The final model includes Age (log-transformed), Mother’s \isi{native language} and Father’s \isi{native language}. Tables \ref{tab:krug:3} and \ref{tab:krug:4} provide a technical summary of the model, listing the coefficients and the type II analysis of variance with F-tests for each term in the model, respectively. While Age and Mother’s \isi{native language} both reached the \textit{p}-value criterion of 0.05, Father’s \isi{native language} may be described as trending towards the established benchmark. We decided to retain it in the model for two reasons: First, a chi-squared test showed a statistically significant association between Father’s and Mother’s \isi{native language}, $\chi $\textsuperscript{2} (4) = 209.16, \textit{p} < 0.0001 (Cramer’s V = 0.50). As \tabref{tab:krug:1} shows, parents were likely to have the same \isi{native language}(s). Knowing the mother’s L1 thus allows us to guess the father’s L1 at above-chance level. From a statistical perspective, these variables thus contribute very similar information to the model. This introduces collinearity and reduces the precision of (and confidence in) model parameters. As a result, the \textit{p}-values for both Father’s and Mother’s \isi{native language} are inflated. Second, and more importantly, including both parental L1s as predictors in the model allows us to judge their relative importance, i.e. to determine whether (the \isi{native language} of) the mother or father is more influential in shaping informants’ lexical preferences. 

\begin{table}[p]
\caption{Type II analysis-of-variance table for the terms in the model}
\label{tab:krug:3}
\begin{tabularx}{\textwidth}{X rrrrr}
\lsptoprule
\bfseries Source & \bfseries SS & \bfseries df & \bfseries F & \bfseries p & \\
\midrule
Age (log-transformed) &  5.52 & 1 &  49.59 &  8.1 e\textsuperscript{-12} &  ***\\
Mother’s \isi{native language} &  1.55 & 2 &  6.98 &  .001 &  **\\
Father’s \isi{native language} &  .53 & 2 &  2.36 &  .096 & \\
\lspbottomrule
\end{tabularx}
\end{table}

\begin{table}[p]
\caption{Coefficients for the model}
\label{tab:krug:4}

\begin{tabularx}{\textwidth}{Qrr}
\lsptoprule
\bfseries Coefficient & \bfseries Estimate & \bfseries SE\\
\midrule
Intercept &  .02 &  .12\\
Age (log-transformed) &  .25 &  .04\\
\tablevspace
Mother’s \isi{native language} (reference: \ili{Maltese}) &  & \\

~~~~English (and \ili{Maltese}) &  .18 &  .06\\
~~~~Other &  $-$.14 &  .09\\
\tablevspace
Father’s \isi{native language} (reference: \ili{Maltese}) &  & \\
~~~~English (and \ili{Maltese}) &  .00 &  .06\\
~~~~Other &  $-$.17 &  .08\\
\multicolumn{3}{c}{\textit{Note.} N = 414, k = 6, residual SD = .33, adjusted R\textsuperscript{2} = .15}\\
\lspbottomrule
\end{tabularx}
\end{table}

The relative importance of Age and parental \isi{native language} on the preference for \ili{BrE} vs. AmE variants will be illustrated and discussed using (i) the proportion of variance explained by each variable and (ii) effect displays (see \figref{fig:krug:6} below). Overall, the model accounts for 15\% of the variation in questionnaire scores. \tabref{tab:krug:5} shows the proportion of variance explained by each factor, a useful measure of the relative importance of the three variables in the model. The metrics were calculated with the package relaimpo \citep{Groemping2013} in R, using the lmg metric \citep[119 ff]{Lindemannetal1980}. Age clearly emerges as the most important factor, explaining more than 10\% of the variance. This corresponds to a correlational effect size measure of around 0.32, which, according to the benchmarks suggested by \citet{Cohen1988}, may be considered a medium-sized effect. Parental native languages contribute less to the overall model, with Mother’s \isi{native language} at 3.5\% (r = 0.19) and that of the father just below 2\% (r = 0.13). The fact that Age outranks parents’ native languages is partly due to the uneven distribution of parental native languages: With more than 80\% of mothers and fathers having \ili{Maltese} as a \isi{native language}, there is not much variation between informants (cf. \tabref{tab:krug:1}). Importantly, however, Mother’s \isi{native language} accounts for more variation than that of the father.

\begin{table}[p]
\caption{Comparison of the predictors: Proportion of the variance explained}
\label{tab:krug:5}
\begin{tabularx}{\textwidth}{Qr}
\lsptoprule
\bfseries Predictor & \bfseries Variance explained\\
\midrule
Age (log-transformed) &  10.2 \%\\
Mother’s \isi{native language} &  3.5 \%\\
Father’s \isi{native language} &  1.8 \%\\
\lspbottomrule
\end{tabularx}
\end{table}

\figref{fig:krug:6} shows effect displays \citep{Fox1987} for the three factors in the model. These were constructed with the effects package \citep{Fox2016b} in R. Such displays allow for closer inspection of each factor in a similar fashion to the descriptive charts above. Importantly, the patterns in these displays are usually more trustworthy, since they take into account (i.e. control for) the influence of the other factors in the model. In other words, they show the effect of a specific predictor while holding constant the effect of the remaining factors. The effect displays show fitted mean values and 95\% confidence intervals, which indicate the precision of the estimates (for more information see \citealt[172-177]{FoxWeisberg2011}, \citealt{Fox2016a}). Such displays greatly facilitate the interpretation of multivariate models and make it possible to directly compare effect magnitudes and patterns across different factors.

  

 
\begin{figure}
\includegraphics[width=\textwidth]{figures/a10KrugSoenning-img6.jpg}
\caption{Effect display for the model: Estimates with 95\% confidence intervals}
\label{fig:krug:6}
\end{figure}

The first panel in \figref{fig:krug:6} shows the main effect of Age, which was discussed above. The non-linearity of the trendline that was apparent in \figref{fig:krug:4} also emerges here, with younger cohorts showing an increasingly stronger trend towards more AmE or globalized language use. While informants aged 60 and older score above 1.0, informants younger than 20 typically score below 0.7. The confidence interval for the linear trend indicates some uncertainty for the older cohorts due to the (relatively) small number of older informants (see \figref{fig:krug:3}). The second panel in \figref{fig:krug:6} exhibits the same pattern as \figref{fig:krug:4}, with English or \isi{bilingual} (English and \ili{Maltese}) mothers triggering a stronger tendency towards British variants. While \ili{Maltese} is intermediate, the group of informants whose parents have native languages other than \ili{Maltese} or English appears to be more globalized in terms of their lexical preferences. The rightmost panel shows similar trends for Father’s \isi{native language} as regards “\ili{Maltese}” and “Other”. In contrast, however, fathers with English (and \ili{Maltese}) as their \isi{native language} do not bias their offspring towards a preference for British items. A comparison of the factors in \figref{fig:krug:6} further underlines the role of Age as the most important factor: the fitted values range from 0.6 to around 1, thus covering a range of 0.4 in respondents’ overall mean values. The levels of Mother’s \isi{native language}, on the other hand, cover a range of 0.3, while Father’s \isi{native language} only accounts for differential effects in the range of around 0.15.

\section{Discussion}

On the basis of a hitherto unanalysed data set from 2013, we essentially confirm, but also significantly qualify, properties and trends reported for older \ili{Maltese English} data and their relation to aspects of globalization (cf. \citealt{KrugRosen2012}; \citealt{Krug2015}). In the descriptive part of this paper, we showed that while the vast majority of items in \ili{MaltE} are preferred in their \ili{BrE} form, there are exceptions indicating that (erstwhile) AmE items such as \textit{truck, sports, package} are preferred over their (traditional) British counterparts \textit{lorry, sport, parcel,} respectively.\footnote{Notice that the standard deviation for \textit{parcel} vs. \textit{package} is among the highest in our data set (similarly to the \isi{verb} \textit{to licence} and spellings ending in \textit{-is/zation}; see § 3.1 and \figref{fig:krug:2} above). The apparent-time trend suggests that, rather than having free variation, informants over 35 prefer \textit{parcel}, whereas younger cohorts show a pronounced shift towards \textit{package}.} 

In apparent-time studies, we visualized and isolated as statistically significant the influence of informants’ age, thus establishing for \ili{Maltese English} novel patterns of ongoing language change (cf. \citealt{Labov1990}). Regardless of the statistical approach and regression model we adopted, all our studies yielded strong apparent-time trends towards a less exclusively \ili{British English} usage in \isi{Malta}, converging on a more globalized usage of lexical items. This trend was seen to be nonlinear and gained pace among the younger cohorts (cf. \figref{fig:krug:3} and the curvilinear shape of the leftmost panel in \figref{fig:krug:6}). 

Since the comparison of apparent-time and real-time studies of the same phenomena in a clear majority of cases reflects actual diachronic change (cf. \citealt{Cukor-AvilaBailey2013}), we conclude that we are witnessing ongoing language change in \isi{Malta} with regard to the choice of the lexical binaries under investigation. Previous apparent-time studies in linguistics have focused on phonetic, \isi{inflectional} and \isi{syntactic} features, whereas we tested lexical binaries. We see, however, no fundamental differences that would forbid applying the same principles to our data. Since we did not elicit information on stigmatized lexical items, there is, we believe, no reason to assume age-graded behaviour for our data (cf. \citealt[98-112]{Labov1994}). Abrupt reversal to older usage preferences seems unlikely as long as no major changes in language and education policies occur.

While Informant Age is the single most important factor for explaining the variation found in our dataset, the native languages of the informants’ parents also play a role. More specifically, the \isi{native language}(s) of the mother appeared to be more influential than that of the father, at least as far as their offspring’s lexical choices are concerned. It was evident (and statistically significant) that when the mother’s \isi{native language}(s) included – in \isi{Malta}, a typically British-oriented form of – English, the informants’ lexical choices were biased in the expected direction, figuring in higher mean values, i.e. increased Britishness of both male and female informants’ choices (cf. \figref{fig:krug:4}). The influence of fathers was less obvious, and this is, prima facie at least, an intuitively plausible result: Children in \isi{Malta} in all likelihood still have, on average, more language contact with their mothers than fathers and thus are more likely to be influenced by their mother’s than their father’s lexical choices.

Furthermore, our data suggest tentatively (cf. \figref{fig:krug:5}) that male informants are more strongly influenced than female informants by their fathers’ \isi{native language}(s). While this trend needs to be confirmed by additional research, it is in line with non-linguistic studies (the transferability of which needs to be treated with even greater caution); such studies, at least, have occasionally shown gender-matched role models to be more influential than non-matched role models (cf. \citealt{Hurdetal2009}). One question emerging from this contribution therefore seems to offer particularly interesting avenues for future research: Does the linguistic behaviour of fathers have a greater impact on their son(s) than on their daughter(s) language beyond the narrow confines of lexical choices and in regions other than \isi{Malta}? If the answer to both parts of the question is yes (or probably yes), it would be fascinating to investigate whether similar tendencies can be found for other cognitive-behavioural domains in studies of human psychology or evolutionary anthropology.

\section*{Acknowledgements}
We are grateful to two anonymous reviewers for their valuable comments. The usual disclaimers apply.


\newpage
\section*{Appendix}

\begin{figure}[h]
%\includegraphics[width=0.80\textwidth]{figures/a10KrugSoenning-img7.png}
  \includegraphics[width=\textwidth]{figures/a10KrugSoenning-img7.png}
  \caption{
% Appendix 1: 
Questionnaire excerpt}
 \label{fig:krug:app1}
\end{figure}
  
\begin{figure}[h]
  \includegraphics[angle=90,origin=c, width=\textwidth]{figures/a10KrugSoenning-img8.png}
\caption{
% Appendix 2: 
Percentage of respondents choosing “I never use either expression” (more/traditionally {BrE} terms given)}
 \label{fig:krug:app2}
\end{figure}

\begin{table}[h]
\caption{ 
% Appendix 3: 
Web data from 2007 (percentage of {BrE} variants among hits) and questionnaire data from 2008 (average across all subjects)
}
\fittable{
\begin{tabular}{lrr p{1cm} lrr}
\lsptoprule

\bfseries Item & \bfseries Web 2007 & {\bfseries Malta} 

 \bfseries 2008 &  & \bfseries Item & \bfseries {Web 2007} & {\bfseries Malta} 

 \bfseries 2008\\
 \midrule
\textit{a drop in the ocean} & 100.0 & 1.44 &  & \textit{torch} & 88.0 & 1.52\\
\textit{laundrette} & 100.0 & 1.08 &  & \textit{pushchair} & 87.8 & 1.86\\
\textit{roundabout} & 99.9 & 1.93 &  & \textit{fish fingers} & 86.7 & 1.85\\
\textit{rubber} & 99.5 & 1.51 &  & \textit{chips} & 84.4 & .91\\
\textit{driving licence} & 99.2 & 1.35 &  & \textit{bicentenary} & 83.3 & .46\\
\textit{cinema} & 99.1 & 1.77 &  & \textit{rubbish} & 82.9 & 1.57\\
\textit{mobile phone} & 98.6 & 1.74 &  & \textit{subway} & 79.2 & 1.91\\
\textit{petrol} & 98.2 & 1.91 &  & \textit{jacket potato} & 77.8 & .02\\
\textit{colour} & 98.0 & 1.81 &  & \textit{liberalisation} & 75.6 & .98\\
\textit{aluminium} & 97.8 & 1.62 &  & \textit{crisps} & 72.5 & .72\\
\textit{petrol station} & 97.4 & 1.80 &  & \textit{modernisation} & 71.3 & 1.01\\
\textit{holiday} & 97.4 & 1.35 &  & \textit{boot} & 68.3 & .80\\
\textit{trainers} & 97.3 & 1.14 &  & \textit{lorry} & 66.8 & -1.71\\
\textit{a tap} & 97.1 & 1.84 &  & \textit{licence} & 64.0 & .18\\
\textit{dummy}  & 95.3 & 1.55 &  & \textit{globalisation} & 62.7 & .83\\
\textit{bookings} & 94.7 & .13 &  & \textit{to let} & 58.8 & .37\\
\textit{postman} & 94.3 & 1.93 &  & \textit{backwards} & 55.8 & 1.06\\
\textit{football} & 94.2 & 1.67 &  & \textit{cupboard} & 53.1 & 1.59\\
\textit{maths} & 93.9 & 1.80 &  & \textit{nappies} & 50.9 & 1.70\\
\textit{railway} & 93.6 & 1.41 &  & \textit{glocalisation} & 50.0 & .81\\
\textit{car park} & 93.5 & 1.73 &  & \textit{ill} & 48.8 & -1.02\\
\textit{dustbin} & 93.1 & 1.69 &  & \textit{sport} & 45.8 & -1.33\\
\textit{centre} & 92.8 & 1.17 &  & \textit{aubergine} & 41.7 & .45\\
\textit{in autumn} & 92.1 & 1.59 &  & \textit{realisation} & 40.4 & .78\\
\textit{pavement} & 92.0 & 1.64 &  & \textit{whilst} & 28.3 & -1.02\\
\textit{storm in a teacup} & 90.9 & 1.64 &  & \textit{potato chips} & 22.3 & .02\\
\textit{localisation} & 90.1 & .94 &  & \textit{organisation} & 21.9 & .70\\
\textit{touch wood} & 90.0 & 1.10 &  & \textit{potato crisps} & 21.4 & .57\\
\textit{anticlockwise} & 89.5 & 1.79 &  & \textit{parcel} & 1.7 & .01\\
\textit{windscreen} & 88.8 & 1.69 &  & \textit{forwards} & 0.6 & -1.09\\
\lspbottomrule
\end{tabular}
}
 \label{tab:krug:app3}
\end{table}
 
% \section*{Abbreviations}

\sloppy
\clearpage
\printbibliography[heading=subbibliography,notkeyword=this] 
\end{document}
