\documentclass[output=paper]{langsci/langscibook} 
\ChapterDOI{10.5281/zenodo.1181799}
\author{Natalie Schembri\affiliation{University of Malta}}
\title{On the characterisation of Maltese English: 
An error-analysis perspective based on nominal structures in Maltese university student texts
}
\shorttitlerunninghead{On the characterisation of Maltese English}
 

\abstract{ The characterisation of varieties of English is an ongoing
  process that has focused on speech communities around the world for
  whom English is the mother tongue or is one of two main languages competing for dominance in a bilingual setting, 
  as is the case for Maltese English. This paper aims to contribute to the growing body of research on
Maltese English as a variety in its own right
(e.g. \citealt{Vella1995}; \citealt{Schembri2005};
\citealt{HilbertKrug2012}; \citealt{KrugRosen2012}). It reflects on
the theoretical assumptions that underpin its characterisation as a
dialect in a bilingual setting distinct from Standard English and
therefore identifiable on grammatical and lexical as well as
phonological levels (see \citealt{Trudgill2002}, for example, for a
characterisation of Standard English along these lines). It analyses
nominal phrase structure data from university student texts produced
by 30 undergraduate Commerce students at the University of Malta. The
study focuses on affixation, compounding and prepositional usage and
examines the contention that not all deviations from Standard English
can be given the status of characteristics of Maltese
English. Applying an error analysis approach to the analysis of the
data, it distinguishes between developmental errors that are
untraceable to Maltese as the background language, and transfer errors
that have this origin by definition. It further contends that only
those transfer errors that fossilize over time are capable of
achieving the status of core characteristics of Maltese English.
Following studies such as \citet{Hyltenstam1988}, the analysis works
on the assumption that fossilized transfer errors are identifiable in
the current data by virtue of the fact that they are still present in
the output of participants who have achieved advanced learner status. From
a varieties-of-language point of view, once fossilized transfer errors
have been identified, their status ceases to be considered as
erroneous and is construable instead as characteristic of the variety
\citep{Selinker1974}. The study concludes that the overuse of the
preposition \textit{of} was the most likely error type to fossilize
and gain status as a stable nominal feature of Maltese English.
As a Maltese bilingual, I use my first-hand understanding 
of the Maltese English linguistic scenario to provide
some insights into what is by linguistic standards still a young and
developing linguistic variety.  }
% \textbf{Keywords}: Varieties of English; Maltese English; fossilization; noun phrase structures; error analysis

\hyphenation{prepo-si-tional}
\maketitle

\begin{document}

\section{Introduction}

The characterisation of varieties of language centres around the
identification and description of linguistic features that make
specific varieties distinct from others of the same language. From a
language description perspective, the central theoretical issue is the
distinction between common core features and stylistically significant
features \citep{CrystalDavy1969}, with the descriptive emphasis
falling on the latter. Common core features occur across varieties and
fulfil the fundamental cohesive function of basic building blocks of
the language that allow it to operate as a unified system. In
contrast, stylistically significant features are distinctive by virtue
of their lack of common occurrence, as is the case with specialised
terminology, or by virtue of some aspect of their use that makes them
variationally distinct, such as the relatively high frequency of
passive forms in academic writing \citep{Swales1990}.  A comprehensive
description of a regional variety would therefore comprise a
description of significant features at all linguistic levels that are
regionally distinct in both the variety’s spoken and written forms.

This ongoing process is in its initial stages in the case of Maltese
English, a variety of English spoken on the Maltese islands where
\ili{Maltese} is the national language and English has official status (as stated in the Constitution of \isi{Malta}, \nocite{Con1964} Articles 5(1) and 5(2)). This paper
aims to contribute to the growing body of research describing regional
features of \ili{Maltese English} in its written form
(e.g. \citealt{Schembri2005}; \citealt{HilbertKrug2012}; 
\citealt{KrugRosen2012}). It will employ an error-analysis approach in an
initial attempt to identify and characterise regional nominal features
of \ili{Maltese English} in a corpus of academic commerce texts, with a
focus on distinguishing these from other nominal features that are
present in the variety but do not necessarily identify it as Maltese
English. The study will start by exploring the theoretical assumptions
that underpin the role \isi{error analysis} can play in identifying
regional characteristics. These will be followed by the methodological
procedures undertaken to collect and analyse the \isi{noun} \isi{phrase} data in
\sectref{sec:key:3}; \sectref{sec:key:4} will then provide the error
analysis and some conclusions as to possible candidates that
characterise the variety are drawn in the last section.

\section{Theoretical background}

This section will outline the theoretical framework underlying the
identification of features of \ili{Maltese English} in this study. It will
explain how an error-analysis approach and its characteristic
distinction between developmental and transfer errors can be employed
to advantage in the identification of regional features. The pivotal
argument will be the role \isi{fossilization} of transfer errors plays in
the development of a set of features that become common across
speakers of a variety to the extent that they configure as its
identifying characteristics.


\subsection{Deviation from standard varieties: The role of developmental errors and transfer errors}
\label{sec:key:2.1}

Error analysis has provided a basic distinction in the identification and classification
of errors, or output that varies from standard usage
\citep{Corder1974}. Although the distinction is normally applied in
the analysis of \isi{learner output}, it will be argued here that it can
also be used to advantage in the identification of regional features
as specified below. The distinction is based on the application of a
systematic comparison of deviant structures in learner varieties to
corresponding target-language structures in the \isi{standard variety}, a
principle first introduced by \citet{Lado1957} as the \isi{contrastive}
analysis hypothesis. Systematic deviant structures are first defined 
%in the first instance 
as erroneous and 
%subsequent to comparative analysis
characterised as developmental or transfer errors after comparative analysis (see, for example,
\citealt{DulayBurt1974} for an application
%of this 
in the
analysis of \isi{bilingual} children’s speech). Developmental errors are
target-language generated \citep[173]{Richards1974} and are generally
understood to be the result of simplification of target-language
structures, for example, when target-language rules are overapplied
\citep{Jain1974}. Developmental errors are therefore \isi{intralingual} in
nature, and are in fact also evident in native-\isi{speaker} output in
children \citep{Jain1974}.


In contrast, transfer errors are interlingual in nature and are seen
to be present when \isi{intralingual} explanations are ruled out and an
examination of corresponding background language structures indicates that
\isi{negative transfer} of linguistic knowledge has taken place
(\citealt{Lado1957}; \citealt{Wardaugh1975}). By virtue of their
provenance therefore, transfer errors establish a contextual link with
the \isi{speaker}’s background language that developmental errors do not, an
important point to bear in mind for the purposes of this study.

As a first premise on the application of this distinction in
variational contexts, it is important to foreground the fact that it
is a distinction endemic to linguistic output in \isi{bilingual}
situations. Transfer only comes into play when a background language
provides pre-existing linguistic knowledge the \isi{speaker} perceives as
transferable to the \isi{target language}. Furthermore, a \isi{speaker}’s
perception of what is and is not transferable changes and becomes more
accurate as knowledge of target-language structures improves
\citep{Taylor1975}. This to the extent that the majority of errors in
advanced \isi{learner output} are expected to be developmental
\citep{McLaughlin1987}. Data from \citet{Thewissen2013} in fact
shows a trend towards plateauing in various errors occurring across
learners from three different language backgrounds from upper
intermediate level\footnote{As defined by the Common European
  Framework of Reference} onwards. This is identified as one of the
three main error developmental profiles in her data taken from the
International Corpus of Learner English. The phenomenon that is of
interest to this study, however, is the fact that in spite of a
general trend to the contrary, some systematic transfer errors will
remain in advanced \isi{learner output}, providing traces of the background
language that eventually establish themselves as stable features in
the \isi{target language} \citep{Selinker1974}.

\subsection{Regional varieties and fossilization}

In the context of regional variation, the notion that some deviant
linguistic phenomena are resistant to the kind of change that results
in the achievement of target-language norms and consequent native-like
competence has particular significance. It can in fact be argued that
a widespread systematic failure by speakers of a specific speech
community to adapt to target-language norms is crucial to the
development of regional varieties. \citet{Selinker1974} has described
the tendency of certain deviant structures to remain in the output of
speakers over time and stabilize themselves in the output of learners
even in \isi{advanced learner} competences as \isi{fossilization}. He defines
fossilized linguistic material as “linguistic items, rules, and
subsystems which speakers of a particular NL will tend to keep in
their IL relative to a particular TL, no matter what the age of the
learner or amount of explanation and instruction he receives in the
TL” \citep[36]{Selinker1974}. There is currently some sense of
dissatisfaction with lack of clarity relating in particular as to
whether fossilisation is a product- or a process-oriented concept and
whether it is global (relating to general linguistic competence) or
local (relating to specific areas of language use; see
\citealt{Fidler2006} for a review of relevant literature). However,
this basic definition has prevailed and will be used in this study,
which is a product-oriented analysis focusing on specific areas of
language use and therefore has a clear orientation in terms of these
two issues.  To foreground issues more central to this study, the
focus on contrast in the definition of \citet{Selinker1974}
%Selinker’s (1974) definition 
makes it clear that
\isi{fossilization} has particular relevance in \isi{bilingual}
contexts. Furthermore, fossilized competences are competences that
have reached a mature stage of development at which a depletion of
errors has taken place to leave a reduced set of errors characterised
by resistance to target-language norms. As pointed out in
\sectref{sec:key:2.1}, an indication that a \isi{speaker} has reached this
level of competence is the presence of a high proportion of
developmental errors in relation to transfer errors.



To put this into a variationist perspective, an important point that
needs to be considered is the perception of \citet{Selinker1974}
%Selinker’s (1974) perception 
that “not only
can entire IL competences be fossilized in individual learners
performing in their own interlingual situation, but also in whole
groups of individuals, resulting in the emergence of a new \isi{dialect} [\textellipsis]
where fossilized IL competences may be the normal situation”
(p. 38). Selinker’s argument indicates that fossilized competences
include both developmental and transfer errors. However, it is the
contention of this paper that only those fossilized errors in advanced
learner competences that are capable of contextualising the linguistic
output, that is fossilized \textit{transfer} errors, are capable of
gaining the status of core characteristics of a regional variety. Any
co-existing fossilized developmental errors will serve to identify the
variety as deviant from the 
%original 
\isi{standard variety}, but will not
have the ability to mark it as regional. Previous work in the
characterisation of \ili{Maltese English} has tended to ignore this
distinction. It is therefore the purpose of this study to examine the
linguistic output of Maltese university students in its capacity as
advanced \isi{learner output}, with a view to identifying initial possible
candidates characterising \ili{Maltese English} as a regional variant. Initial work in this
direction was carried out in an earlier study (Schembri, under review)
that focused on article usage, singular and \isi{plural} forms and noncount
nouns and identified overuse of the \isi{definite article} as a likely
\isi{nominal feature} of \ili{Maltese English}. The current study will consider
\isi{affixation}, \isi{compounding} and prepositional errors to provide a more
comprehensive picture of nominal characteristics of the variety. It
will apply an error-analysis approach to identify both developmental
and transfer errors in so doing and will also provide some insight
into fossilized developmental errors that play a role in
characterising the variety but have secondary status as non-context
bound features.


\section{Methodology}
\label{sec:key:3}
This section will outline the details of the methodological procedures
undertaken for the purposes of data collection. Apart from the \isi{noun}
phrases that constituted the primary data, secondary data was
collected through a questionnaire as a source of information about the
subjects’ language background. The Faculty of Economics, Management
and Accountancy provided scripts from the May/June 1997 session and
\isi{noun} phrases were collected from them as specified below. All
potential participants were sent a consent form and sampling was
carried out on the pool of consenting candidates.


\subsection{Participants}
\largerpage
All participants had satisfied the University of \isi{Malta} entry
requirements and had a pass at Grade 5 or better in English in the
Secondary Education Certificate (SEC) \citep{GER2010}. It should be noted that this allows for a fairly broad spectrum
of linguistic competence levels. Participants were following one of
the four degree courses run by the Faculty of Economics, Management
and Accountancy. These were the Bachelor of Commerce, the Bachelor of
Commerce (Honours), the Bachelor of Arts (Honours) in Accountancy and
the Bachelor of Arts (Honours) in Tourism. Examinations were held at
the end of the first year, which is common to the first three courses, and
subsequently during the third year and the fourth or fifth year
depending on the length of the course. Three sets of examination
scripts from three different student cohorts were therefore available
at the point of data collection.



In order for sampling to take place, potential participants were asked
to fill in a questionnaire with their demographic details and details
relating to their language background. This information was used to
filter out participants whose \isi{native language} was not Maltese and
whose language competence might have been influenced by atypical
language exposure. Two hundred and thirty seven candidates from the
May/June 1997 session answered the questionnaire, 30 of whom were
considered atypical because they were foreign, had dual nationality,
had a foreign parent or had lived in an English-speaking country for a
significant period of time. A random sample of 10 candidates from
each year group was chosen after this filtering had taken place and
\isi{noun} phrases were collected from the scripts produced by the 
%thirty
30 candidates chosen. The subjects were sixteen males and fourteen
females who had been educated in \isi{Malta}. They had received formal
instruction in English 
%at least from the ages of five to sixteen. 
between the ages of 5 and 16, at least.
Twenty-seven of them were between eighteen and twenty-four
years old when they sat for the examinations in question, and the
remaining three were in their early thirties. As
\citet{Hyltenstam1988} has pointed out, adult learners have been
considered the “natural population” (p. 69) for studies of
fossilisation, on the basis that higher levels of mastery act as an
automomatic filter for deviances that are not likely to be
fossilizable (p.70).


\subsection{Collection and analysis of noun phrase data}

Two hundred and fifty \isi{noun} phrases were collected manually from the
scripts of each of the thirty subjects to make up a corpus of 7,500
\isi{noun} phrases. The definition of a \isi{noun} \isi{phrase} used was that given by
\citet{Quirketal1985} and considers phrases functioning “as \isi{subject},
object, and complement of clauses and as complement of prepositional
phrases” (p. 245) to be nominal. Pronouns were not included.



Once the corpus was complete, each \isi{noun} \isi{phrase} was judged erroneous or
error-free by the researcher and doubts as to errors in specialised
terminology and border-line cases were double checked by a specialist
in the field and a second rater respectively. The researcher satisfied
%\citet{Etherton1977}'s 
the criteria of \citet[72]{Etherton1977} regarding the qualities needed for
satisfactory error judgement. Apart from being a linguist by
profession and therefore in possession of “an understanding of how the
English language works or genuine curiosity on this point”, %(p. 72),
she had the required level of competence in the
language,\footnote{\citet[72]{Etherton1977}'s criteria specify a “high standard
  of English”. The researcher’s standard of English was
  considered high enough to warrant her inclusion as a member of the
  Academic English Team at the Institute of Linguistics of the
  University of \isi{Malta} whose role was to ensure and maintain standards
  of English at the university. At the time the study was undertaken
  she had been fulfilling this role for five years.} was a native
\isi{speaker} of \ili{Maltese} and had taught for a number of years at the level
concerned.



Data was collected as evenly as possible from the first, middle and
last parts of essay-type questions answered by the
subjects. First-year students had answered 12 essay-type questions on
average and third-year and honours students 25-26 questions. The ratio
of erroneous to error-free structures was then computed for each
\isi{subject}. Noun phrases classified as erroneous were given target forms
as close to the original structures as possible. Care was taken to
disregard infelicities of style and concentrate on instances of
incorrect usage. All structures were listed as output from a specific
\isi{subject} (i.e. candidate) and numbered for ease of identification. In
the \isi{error analysis} that follows, these details are given in a bracket
at the end of each example. For example, Subject 2: 184
indicates the error in the example preceding it occurred in the
184\textsuperscript{th} \isi{noun} \isi{phrase} collected from the output of the
second candidate. Errors are given in enough context to identify them
as such and italicised for ease of identification. Corresponding
target forms are given immediately below. Erroneous structures were
subsequently categorised on the basis of structure and error-type to
facilitate the identification of any existing patterns and enable
errors of a similar type to be discussed in tandem. In the discussion,
an attempt was made to distinguish between developmental and transfer
errors on the basis of comparative analysis carried out in line with
the theoretical assumptions underlying the study.


\section{Error analysis}
\label{sec:key:4}
This section will provide a systematic \isi{error analysis} of \isi{affixation},
\isi{compounding} and prepositional errors in that order. It will
characterise the errors in the data falling under these three
categories as developmental or transfer errors and will subsequently
consider their possible status as fossilized features and potential
candidates as nominal characteristics of \ili{Maltese English}. A general
picture of erroneous versus error-free \isi{noun} phrases broken down by
student group is given in \tabref{tab:key:1}.

\begin{table}
\begin{tabularx}{\textwidth}{lSS}
\lsptoprule
\textbf{Student group} & \textbf{Average of erroneous \isi{noun} phrases} & \textbf{Average of error-free \isi{noun} phrases}\\
\midrule 
First-year students & 45 (18\%) & 205 (82\%)\\
Third-year students & 35 (14\%) & 215 (86\%)\\
Honours students & 32 (13\%) & 218 (87\%)\\
Total & 112 (15\%) & 638 (85\%)\\
\lspbottomrule
\end{tabularx}
\caption{Average of erroneous versus error-free noun phrases across year groups}
\label{tab:key:1}
\end{table}

The results in \tabref{tab:key:1} indicate an overall 15\% error rate
average located in \isi{noun} phrases in the data. These include the three
types of error examined in this paper, as well as other error types
such as faulty article usage, proform errors and the misuse of
singular and \isi{plural} forms. The subset of errors falling under the
three categories examined in this paper are given as raw scores in
\tabref{tab:key:2}.

\begin{table}
\begin{tabular}{lr}
\lsptoprule
\textbf{Error type} & \textbf{Frequency (n)}\\
\midrule 
Prepositional errors & 29\\
Affixation errors & 21\\
Compounding errors & 7\\
\lspbottomrule
\end{tabular}
\caption{Frequency of propositional affixation and compounding errors}
\label{tab:key:2}
\end{table}

\subsection{Affixation errors}
The errors discussed in this section concern faulty nominal word-formation 
processes involving \isi{affixation} located in the head of the NP.
%\isi{noun} head. 
These are of two types: the first type involves word class
changes that result in lack of correspondence between form and
function. One example is “insurance \textit{brokering}” (Subject 11:
34), where the \isi{verb} form is being incorrectly used as a \isi{noun} instead
of \textit{brokerage}. The second type involves \isi{affixation} processes
resulting in the formation of non-words, as in the use of the
\textit{un-} prefix in “\textit{unadmissible} assets” (Subject 22:
227). The next two sections will consider the two different types of
error in turn.


\subsubsection{Lack of correspondence between form and function}
Different factors were seen to come into play in errors of \isi{affixation}
that resulted in lack of correspondence between form and
function. “The reduction of the \isi{target language} to a simpler system”
\citep[191]{Jain1974}, or simplification, was a likely motivator in
cases of non-suffixed forms that were in need of a \isi{suffix}, as in the
three errors in \REF{ex1} and \REF{ex2} below:


\begin{exe}  
  \ex{}
  {Subject 3: 154, 155}\\
    \textit{Physiological needs include the very basic ones for survival – \textbf{drink} and \textbf{eat} here come to mind.}\\
    {‘Physiological needs include the very basic ones for survival – drinking and eating here come to mind.’}
  \label{ex1}
\end{exe}

\begin{exe}  
  \ex{}
  {Subject 2: 184}\\
    \textit{or dies for force reasons, that is during arrest or to the \textbf{safeguard} of others or during state emergencies}\\
    {‘or dies for force reasons, that is during arrest or the safeguarding of others or during state emergencies’}
  \label{ex2}
\end{exe}


The base forms \textit{drink}, \textit{eat} and \textit{safeguard} all
require the \isi{suffix} -\textit{ing} to change their word class into the
nouns appropriate for use in their current contexts. All three are
base forms of verbs, and \textit{drink} and \textit{safeguard} can
also function as nouns, albeit with different meanings to the ones
intended here. Although \textit{eat} cannot function as a \isi{noun}, the
tendency to use base forms as opposed to more complex affixed forms
requiring a choice of \isi{suffix} is present in all three.



The opposite is however the case in \REF{ex3}, where the \isi{suffix} -\textit{ing} has erroneously been added to the \isi{base form} \textit{search}, which 
%can also double as 
is the \isi{noun} needed in this context:


\begin{exe}  
  \ex{}
  {Subject 9: 139}\\
    \textit{needs such as exploration and the \textbf{searching} for meaning and knowledge}\\
    {‘needs such as exploration and the search for meaning and knowledge’}
  \label{ex3}
\end{exe}


The unnecessary addition of the \isi{suffix} %is probably 
could be indicative of a
more advanced type of error occurring at a stage in the learning
process when \isi{affixation} is being used rather than avoided, but is
overapplied in some cases.



The data also provided a case of \isi{affixation} involving an incorrect choice of \isi{suffix}:

\begin{exe}  
  \ex{}
  {Subject 11: 34}\\
    \textit{insurance \textbf{brokering}}\\
    {‘insurance brokerage’}
  \label{ex4}
\end{exe}

This error may be conditioned by the fact that -\textit{ing} is much
more productive than -\textit{age} and might therefore be functioning
as a default \isi{suffix} until a more detailed understanding of \isi{affixation}
is in place. It should also be noted that, as opposed to \textit{eat},
\textit{drink}, \textit{search} and to some extent also
\textit{safeguard}, \textit{brokerage} has a more specialised usage
and the \isi{subject} who made the error may still have been developing some
familiarity with the \isi{lexeme} and its different forms in the process of
acquiring new vocabulary in this field.

The above analysis suggests these types of problems are developmental
in nature insofar as they deal with simplification of target-language
structures or the use of \isi{intralingual} processes that are valid in
themselves but incorrectly applied.

\subsubsection{Non-words}

Affixation processes sometimes resulted in the formation of
non-words. The following three examples show non-existent words
resulting from an incorrect choice of the negative prefix:


\begin{exe}  
  \ex{}
  {Subject 22: 226}\\
    \textit{loans that are \textbf{unadequately} secured}\\
    {‘loans that are inadequately secured’}
  \label{ex5}
\end{exe}


\begin{exe}  
  \ex{}
  {Subject 22: 227}\\
    \textit{\textbf{unadmissible} assets}\\
    {‘inadmissible assets’}
  \label{ex6}
\end{exe}


\begin{exe}  
  \ex{}
  {Subject 17, 146}\\
    \textit{the disorders created when rule of law is weak or \textbf{inexistent}}\\
    {‘the disorders created when rule of law is weak or nonexistent’}
  \label{ex7}
\end{exe}

The first two errors were made by the same \isi{subject} and, as in the case
of -\textit{ing} earlier, it is possible that \textit{un-} is being
used as a kind of default negative \isi{suffix}, particularly since it is
more productive than \textit{in-}. The use of \textit{inexistent} in
the third example is not likely to be such a case, however, since
\textit{in-} is not as productive as \textit{un-}.



Other cases of non-words resulting from NP \isi{affixation} 
%in the \isi{noun} head
were the following:


\begin{exe}  
  \ex{}
  {Subject 1: 191}\\
    \textit{illegitimation}\\
    {‘illegitimacy’}
  \label{ex8}
\end{exe}

\begin{exe}  
  \ex{}
  {Subject 7: 180}\\
    \textit{incapacitance}\\
    {‘incapacitation’}
  \label{ex9}
\end{exe}

\begin{exe}  
  \ex{}
  {Subject 5: 200}\\
    \textit{enbreechment}\\
    {‘breech’}
  \label{ex10}
\end{exe}

The first two examples, taken from the output of different subjects,
show the \isi{suffix} -\textit{ation} being overapplied in \REF{ex8} and replaced
by -\textit{ance} in \REF{ex9}. More evidence would be needed to deduce
whether the use of a default \isi{suffix} is in operation. It is possible
that the \ili{Maltese} cognate \textit{ille\.gittimazzjoni}, where
-\textit{azzjoni} corresponds formally to -\textit{ation}, influenced
the choice in the first case. In the second case, the \ili{Maltese} cognate
\textit{kapacità} is not suffixed;
however, my intuition as a \ili{Maltese} \isi{speaker} would indicate that the final accented \textit{à} is more in harmony
% however, the final accented \textit{à} is clearly more in harmony 
with -\textit{ance} than with
-\textit{ation}. These two examples indicate some possible traces of
transfer in the use of \isi{affixation} and suggest that not all word-formation
processing errors are necessarily developmental in nature,
particularly where cognates are concerned. The third example, on the
other hand, cannot be attributed to transfer, and the unnecessary
addition of a \isi{suffix} to \textit{breech} is probably a result of lack
of familiarity with its double function, which would make the error
\isi{intralingual} and therefore developmental.

Although there is clearly different patterning at work, what is
interesting in the above three examples is the fact that they are all
specialised terms from Law, which is a \isi{subject} area Commerce students
are tested on, but not one they are particularly familiar with. On the
basis that studies such as \citet{Nation1993} show that specialised
vocabulary increases in tandem with one’s understanding of the \isi{subject}
matter, these errors can be taken as an indication that learners need
some time to familiarise themselves with the different forms of new
lexemes and, particularly in the case of cognates, may fall back on
their knowledge of the L1 to fill in any existing gaps.


\subsection{Compounding errors}
\label{sec:key:4.2}
Other types of non-forms in the data concerned the use of non-existent
phrases, most of which were unacceptable \isi{noun} compounds. Compound
nouns have been found to be problematic in
\citet{AlaminAhmed2012} who explained that students studying Science at Taif
University in Saudi Arabia who had previously studied English as a
foreign language for five to ten years failed to use compound nouns
correctly in spite of having been taught their use in scientific
English.

Since \isi{compounding} is not possible in \ili{Maltese}, \isi{direct transfer} cannot
be considered as possible motivation for errors of
\isi{compounding}. Contrastive analysis suggests it is more likely that the
motivation is avoidance. As is evident from the following examples,
the correct \isi{target form} for most unacceptable \isi{noun} compounds in the
data is a \isi{postmodifying} \isi{prepositional phrase}:


\begin{exe}  
  \ex{}
  {Subject 3: 192}\\
    \textit{Ombudsman decisions}\\
    {‘decisions taken by the ombudsman’}
  \label{ex11}
\end{exe}

\begin{exe}  
  \ex{}
  {Subject 13: 127}\\
    \textit{a new \textbf{management line of though}} [sic]\\
    {‘a new line of thought in management’}
  \label{ex12}
\end{exe}

\begin{exe}  
  \ex{}
  {Subject 21: 142}\\
    \textit{the Dividend Article of the Treaties number 10}\\
    {‘point Number 10 of the Dividend Article of the Treaties’}
  \label{ex13}
\end{exe}

%\begin{exe}  
%  \ex{}
%  {Subject 2: 14}\\
%    \textit{the bigger relative value of \textbf{assets prices}}\\
%    {‘the bigger relative value of the prices of assets’}
%  \label{ex14}
%\end{exe}

\begin{exe}  
  \ex{}
  {Subject 4: 171}\\
    \textit{the \textbf{human personality} and his behaviour}\\
    {‘the personality and behaviour of human beings’}
  \label{ex15}
\end{exe}

It is possible that the construction of erroneous compound nouns in
the above cases is an attempt to avoid prepositional phrases, which
involve the notoriously difficult area of prepositional usage
\citep{Jain1974} and it is interesting to consider to what extent such
strategies may be influenced by the background language.

In a \isi{contrastive} study, \citet{Schachter1974} considered avoidance in
the light of the acquisition of English relative clauses by native
speakers of \ili{Persian}, \ili{Arabic}, \ili{Chinese}, and Japanese. She found that the
\ili{Persian} and \ili{Arabic} learners produced significantly more, albeit at
times erroneous relative clauses in English than the \ili{Chinese} and
Japanese learners. One of the insights that came out of the study was
the fact that avoidance possibly occurred as a result of the
perception of language distance resulting from the postnominal
position of relative clauses in English, as contrasted with their
prenominal position in \ili{Chinese} and Japanese.


In cases such as those in \citegen{Schachter1974} study, knowledge of
corresponding background language structures can be seen to influence
target-language output, and some element of transfer therefore
understood to be present, if indirectly. With respect to prepositional
usage in English, however, avoidance can much more readily be
interpreted as a result of the degree of arbitrariness in
prepositional usage present in the \isi{target language} itself
\citep{Jain1974}, and less so as a background language related
issue. It is however interesting to note that in the current data, the
prepositions in all the examples would have been correct had they been
directly translated from \ili{Maltese}, and that with the possible exception
of \ref{ex11}, which has a relatively simple structure, it is not
unlikely that avoidance was significantly conditioned by the level of
complexity of the corresponding target-language structures.



With respect to whether or not any specific erroneous \isi{noun} compounds
are likely to \isi{fossilize} as stable developmental features of the
variety, it is difficult to come to any definitive conclusion. As in
the case of the production of errors occurring as a result of valid
\isi{affixation} processes, it is doubtful whether the production of
erroneous \isi{noun} compounds is productive and widespread enough to be
fossilizable, and a large-scale study would need to be conducted to
determine whether this is the case.


\subsection{Prepositional errors}

Errors in prepositional usage were found in twenty-seven out of the
thirty subjects. This is in line with what one would expect since
prepositions are considered one of the areas of the surface structure
of English that are “more facilitative of indeterminacy than others”
\citep[205]{Jain1974} and are therefore highly problematic for
learners, including those at an advanced stage of their language
learning. A number of recent studies on adult learners with semitic
background languages show correspondingly high frequencies of errors
involving prepostions. \citet{GholamiZeinolabedini2015}, for
example, found prepositions to be one of the four \isi{grammatical} areas
with the most frequent errors in a corpus of sixty \ili{Iranian} medical
research articles published in international English journals
(p. 64). In this study, the published versions of the articles were
compared to their first drafts to identify which areas had required
\isi{grammatical} improvement in the process of publication. Although 
the data is largely comparable, it included instances such as the replacement of 
'to' instead of a dash in phrases such as '8-10', which 
would not be considered error types in the current study. In another
recent study on the written production of sixty \ili{Iranian} adult
students’ performance in a mock IELTS test, prepositions accounted for
10.9\% of \isi{grammatical} errors \citep{NosratiNafisi2015}. Similarly, an
earlier study involving 50 male and 50 female advanced \ili{Iranian} EFL
learners found misuse of prepositions accounted for 13.5\% of
\isi{syntactic} errors in female writers and 15.5\% of \isi{syntactic} errors in
males \citep{BoroomandRostami2013}; and \citet{Al-HarafshehPandian2012} listed
adjectives with prepositions as the second most frequent type of
adjectival error in a test on the use of adjectives administered on
150 twenty-two-year-old \ili{Jordanian} students at Al-Albeyt
University. Similar indications of the problematic nature of
prepositions were present in an \isi{error analysis} carried out on
forty-nine third-year university students majoring in English in
Northeast Normal University Changchun with \ili{Chinese} as their first
language. In this study, prepositional errors accounted for 11.6\% of
errors produced in a narrative essay.



In the current study, errors involving prepositional usage were in
most cases located in %\isi{postmodifying} prepositional 
phrases that
involved an incorrect choice of \isi{preposition} whose \isi{target form} needed a
simple substitution of \isi{preposition}, or else more complex modification
of the \isi{prepositional phrase} or its substitution with some form of
premodification. What shall be considered here is whether any specific
prepositional error type is a likely candidate for \isi{fossilization} and
if that is the case, if it can be considered a possible characteristic
of \ili{Maltese English} on the basis that it is the result of negative
transfer from \ili{Maltese}.



The most common errors needing a simple substitution of \isi{preposition}
were cases of \textit{for} being replaced incorrectly by \textit{of},
as in the following:


\begin{exe}  
  \ex{}
  {Subject 9: 148}\\
    \textit{the need \textbf{of} beauty, order and symmetry}\\
    {‘the need for beauty, order and symmetry’}
  \label{ex16}
\end{exe}

\begin{exe}  
  \ex{}
  {Subject 7: 166}\\
    \textit{the need \textbf{of} esteem}\\
    {‘the need for esteem’}
  \label{ex17}
\end{exe}

\begin{exe}  
  \ex{}
  {Subject 6: 200}\\
    \textit{reason \textbf{of} arrest}\\
    {‘reason for arrest’}
  \label{ex18}
\end{exe}

\begin{exe}  
  \ex{}
  {Subject 27: 181}\\
    {the best price \textbf{of} assets}\\
    {‘the best price for assets’}
  \label{ex19}
\end{exe}


These errors can be considered transfer errors since the phrases are directly translated from \ili{Maltese}, with the \isi{preposition} \textit{of} translating \textit{ta’}. \textit{Of} is also seen to incorrectly replace other prepositions, such as \textit{to} and \textit{about} respectively in the following:


\begin{exe}  
  \ex{}
  {Subject 10: 198}\\
    \textit{the right \textbf{of} life}\\
    {‘the right to life’}
  \label{ex20}
\end{exe}

\begin{exe}  
  \ex{}
  {Subject 9: 180}\\
    \textit{the film \textbf{of} Nature and Nurture}\\
    {‘the film about Nature and Nurture’}
  \label{ex21}
\end{exe}


In both examples, \textit{ta’} again gives a valid version in \ili{Maltese},
although \textit{għal} is also possible in \REF{ex20}. Direct transfer where
\textit{of} translates \textit{ta’} is again therefore likely.


As in \REF{ex20}, \textit{ta’} is also seen to replace \textit{to} in the
following examples, however with a different \isi{target form} that requires
the \isi{base form} of the \isi{verb} to follow it:


\begin{exe}  
  \ex{}
  {Subject 2: 225}\\
    \textit{his need \textbf{of} feeling loved}\\
    {‘the need to be loved’}
  \label{ex22}
\end{exe}

\begin{exe}  
  \ex{}
  {Subject 9: 146}\\
    \textit{the need \textbf{of} being seen at his best}\\
    {‘the need to be seen at his best’}
  \label{ex23}
\end{exe}

\begin{exe}  
  \ex{}
  {Subject 18: 56}\\
    \textit{the failure \textbf{of} paying attention}\\
    {‘the failure to pay attention’}
  \label{ex24}
\end{exe}

In \REF{ex24}, \isi{direct transfer} is likely since the \ili{Maltese} translation would
be \textit{in-nuqqas ta’} \textit{attenzjoni}. Example \REF{ex22} could
follow the same structure translated as \textit{il-bżonn ta’
  l-imħabba}; however, \textit{li} is also possible if a \isi{verb} follows
the \isi{preposition} instead of a \isi{noun} to produce \textit{il-bżonn li tkun
  maħbub}. The latter structure would also be needed for a translation
of \REF{ex23} in \textit{il-bżonn li jidher fl-aħjar tiegħu}. In the latter
two cases transfer through direct translation of the \isi{preposition} can
be ruled out since \textit{li} does not translate \textit{of.}



Examples \REF{ex22}, \REF{ex23} and \REF{ex24} introduce a set of examples where
V\textit{ing} follows \textit{of} to produce NP + \textit{of} +
V\textit{ing}. This structure is perfectly acceptable in English in
certain cases but is used erroneously as indicated above and in the
following examples:

\begin{exe}  
  \ex{}
  {Subject 16: 97}\\
    \textit{the prevention \textbf{of letting} hardware get damaged by humidity or mishandling}\\
    {‘the prevention of harware damage caused by humidity or mishandling’}
  \label{ex25}
\end{exe}

\begin{exe}  
  \ex{}
  {Subject 16: 94}\\
    \textit{The prevention \textbf{of losing} information}\\
    {‘The prevention of information loss’}
  \label{ex26}
\end{exe}

\begin{exe}  
  \ex{}
  {Subject 27: 226}\\
    \textit{the pursuit \textbf{of making} higher profits}\\
    {‘the pursuit of higher profits’}
  \label{ex27}
\end{exe}


In these three cases, \textit{of} needs to be followed by NP to
produce the NP + \textit{of} + NP structure that was used erroneously
in (\ref{ex16}--\ref{ex21}) above. \ili{Maltese} would tend to have an NP following
\textit{of} so that \isi{direct transfer} would have favoured the correct
choice of structure. However, it would also have favoured
\textit{damage of hardware} and \textit{loss of information}
respectively in the first two examples since \isi{compounding} is not
possible. It is difficult to determine whether the resulting double
use of \textit{of} may have created some perception of awkwardness the
writers wanted to avoid, but this still rules out \isi{direct transfer}. It
is more likely that the NP + \textit{of} + V\textit{ing} structure is
being overapplied until further familiarization limits its usage to acceptable environments.



The last set of errors concerned \isi{postmodifying} prepositional phrases
whose target structures required some form of
premodification. Occasionally, as in \REF{ex28}, an adjective was needed
instead:


\begin{exe}  
  \ex{}
  {Subject 1: 216}\\
    \textit{various roles \textbf{of} managers}\\
    {‘various managerial roles’}
  \label{ex28}
\end{exe}

It is likely that the \ili{Maltese} equivalent would favour a stucture with
\textit{ta’} which would indicate \isi{direct transfer}.



Postmodifying prepositional phrases more commonly needed to be
replaced by an \textit{’s} genitive:


\begin{exe}  
  \ex{}
  {Subject 22: 169}\\
    \textit{the consent \textbf{of} the partners}\\
    {‘the partners’ consent’}
  \label{ex29}
\end{exe}


\begin{exe}  
  \ex{}
  {Subject 24: 85}\\
    \textit{an opinion \textbf{of} the auditor on the truth and fairness of the financial statements}\\
    {‘the auditor’s opinion on the the truth and fairness of the financial statements’}
  \label{ex30}
\end{exe}


\begin{exe}  
  \ex{}
  {Subject 27: 54}\\
    \textit{the advantages associated with the use \textit{by} the company of debt capital}\\
    {‘the advantages associated with the company’s use of debt capital’}
  \label{ex31}
\end{exe}

It is possible to interpret the above errors as the result of \isi{negative transfer} since in all three cases \ili{Maltese} would have a \isi{postmodifying} \isi{prepositional phrase} with \textit{ta’.} It is, however, also possible to use the same structure in English in other cases, and therefore the errors can also be interpreted as the overapplication of a TL rule. 
%However, the \textit{of} construction to indicate possession is the exception rather than the rule in English, and as has been discussed previously, it is usually the more common structure that is overapplied when more than one realisation is possible. 
However, the \textit{of} construction to indicate possession is less common in English, and it is usually the more common structure that is overapplied when more than one realisation is possible. A crosslinguistic motivation is therefore more likely.

Lastly, \isi{postmodifying} prepositional phrases with \textit{of} also
replaced compound nouns, as can be seen in the following:



\begin{exe}  
  \ex{}
  {Subject 7: 50}\\
    \textit{a fixed rate \textbf{of} tax}\\
    {‘a fixed tax rate’}
  \label{ex32}
\end{exe}


\begin{exe}  
  \ex{}
  {Subject 28: 50}\\
    \textit{the confidence \textbf{of} investors}\\
    {‘investor confidence’}
  \label{ex33}
\end{exe}

\begin{exe}  
  \ex{}
  {Subject 7: 84}\\
    \textit{an accountant \textbf{for} the government}\\
    {‘a government accountant’}
  \label{ex34}
\end{exe}

\begin{exe}  
  \ex{}
  {Subject 16: 136}\\
    \textit{the Brandt Commission \textbf{of} the 1990}\\
    {‘the 1990 Brandt Commission’}
  \label{ex35}
\end{exe}

These errors are clearly crosslinguistic in nature, since in \ili{Maltese} a
\isi{postmodifying} \isi{prepositional phrase} would be used in such cases and its
substitution with some form of premodification would not be
possible. It is interesting to note that the subjects who had
erroneous compound nouns in their production (see
\sectref{sec:key:4.2}) did not make these kinds of errors. This
suggests the two error types are indicative of different stages of
development, with erroneous postmodification of the type shown in
Examples (\ref{ex32}--\ref{ex35}) above preceding the production of erroneous compound
nouns. The fact that different stages of development are characterised
by the quality of the errors and not simply by error rates has
recently been shown in a study on error rates and error types in three
different IELTS bands by \citet{Muller2015}. \citet{NezamiNajafi2012}
also found significant differences across low, mid and high
proficiency groups on error types made in a TOEFL-based written
English test taken by 103 \ili{Iranian} students of English at two
universities in Iran.



As is clear from the above, the majority of prepositional errors
concern \textit{of}, which shows a wider application of its use than
its target-language usage would allow, and suggests a tendency for its
application as a preferred option when in doubt. The above analysis
indicates it is likely to replace \textit{for} in NP + \textit{of} +
NP structures, but may also replace other prepositions such as
\textit{to} and \textit{about}. As the analysis of previous errors
related to the overuse of affixes suggests, it is not unusual for
learners to overuse the most common realisation of a \isi{grammatical form},
particularly if there is overlap in meaning and some degree of
arbitrariness in their application, as is the case here. More
importantly in relation to issues of characterisation of the variety,
the data suggest that the preference for \textit{of} is triggered by
the usage of \textit{ta’} in \ili{Maltese} in such cases.



Other relatively frequent errors related to \textit{of}  are \isi{postmodifying}
prepositional phrases replacing \textit{’s} genitives or compound
nouns, which add to the frequency of problematic structures starting
with NP + \textit{of} likely to be the result of transfer and
strengthen the possibility that a preference for such structures is a
possible feature of \ili{Maltese English}. Less clear cases of transfer with
initial NP + \textit{of} structures that contribute to this general
picture are NP + \textit{of} + V\textit{ing} structures. The
possibility of prepositional usage being affected by the L1 follows
findings by  \citet{KooshaJafarpour2006} indicating that errors in
the collocation of prepositions in a test administered on 200 \ili{Iranian}
university English majors were more than twice as likely to be
interlingual in nature (68.4\% as opposed to 31.6\% \isi{intralingual}
errors).


\section{Conclusion}

This study applied an error-analysis approach to errors of \isi{affixation},
\isi{compounding} and prepositional usage in Maltese university students’
commerce texts in an attempt to identify nominal characteristics of
\ili{Maltese English}. With respect to \isi{affixation}, it concluded that with
the exception of some evidence of transfer in cognates, \isi{affixation}
errors were mostly developmental in nature and unlikely candidates for
\isi{fossilization}. Transfer was even less evident in \isi{noun} compounds, where
avoidance of complex prepositional phrases was seen to be the most
probable cause of error. Prepositional errors were mostly
\textit{of}{}-related, and were the errors that showed the clearest
evidence of transfer. There was some indication that \textit{of} might
be functioning as a preferred \isi{preposition} in cases of doubt as to the
correct choice of \isi{preposition}. It was concluded that the overuse of
\textit{of} was the most likely type of error to \isi{fossilize} and gain
status as a stable \isi{nominal feature} of \ili{Maltese English}.


More evidence is needed to determine whether any of the developmental
error types found in the data are likely candidates for
\isi{fossilization}. One of the issues that needs to be addressed with
respect to \isi{affixation} and \isi{compounding} and other errors of this type is
to what extent \isi{fossilization} is likely in such cases. The point is
that it is not the \isi{affixation} process in itself that is erroneous, but
its overapplication in specific cases. What needs to be determined is
therefore whether any specific usage of an incorrect form has
fossilized, and this is difficult to do unless the \isi{noun} happens to be
commonly used. Unlike highly productive linguistic forms, such as the
definite or indefinite article, \isi{fossilization} of linguistic items that
are not highly productive would need substantial amounts of data for
enough instances of their usage to give clear indications as to
whether \isi{fossilization} is taking place. Furthermore, such usage would
need to be found across subjects to determine whether \isi{fossilization} is
ideolectal or else more widespread and therefore possibly
variational. A further 
%implication 
question is whether actual lack of
productiveness of linguistic items may deter \isi{fossilization} in any
case, particularly if repeated usage is found to be a determining
factor.


This study has extended the examination of nominal features of Maltese
English initially examined in Schembri (under review). Further
evidence is however necessary to consolidate the findings from both
these studies and to create a more comprehensive picture of regional
features characterising the variety, particularly since the dataset is
not extensive. It is also important at this point to consider more
recent data to see whether current linguistic trends match those
evident in the data collected in 1997 for this study. Although
variational change takes time to establish itself, the timespan at
this point is probably large enough for any significant linguistic
development to become evident.



On a more general note, the discussion of the theoretical
underpinnings of the identification of such features, in particular
the role of \isi{fossilization} and the distinction between the status of
developmental and transfer features in such a context needs to be
further developed. A clearer understanding is also needed of what
determines which features in a given variety are likely to \isi{fossilize}
and which are more likely to develop to native \isi{speaker} competence
levels. A detailed examination of such issues will shed light on the
development of regional varieties in \isi{bilingual} contexts.

 
 
 
 
 
%\section*{Abbreviations}
%\section*{Acknowledgements}

\sloppy
\printbibliography[heading=subbibliography,notkeyword=this] 
\end{document}
