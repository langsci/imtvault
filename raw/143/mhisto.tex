\begin{exo}\label{exo:mHisto}
On interprète $\Xlo\doit$ comme historique. 
\pagesolution{crg:mHisto}
Démontrez, en utilisant par exemple le modèle de la figure~\ref{futurs4}, que $\Xlo\doit\mF\phi$ et $\Xlo\mF\doit\phi$ ne sont pas équivalentes.
\\
Et démontrez que $\Xlo[\mP\phi\implq\doit\mP\phi]$ est vraie pour tout $w$ et tout $i$.
\begin{solu}(p.~\pageref{exo:mHisto})\label{crg:mHisto}
$\Xlo\doit\mF\phi$ dit que pour tout  monde actuellement identique au monde d'évaluation courant, il y a un instant du futur où \vrb\phi\ est vraie. $\Xlo\mF\doit\phi$ dit qu'il existe un instant du futur durant lequel, dans tous les mondes qui seront alors identiques au monde d'évaluation courant, \vrb\phi\ est vraie.  Reprenons la figure  \ref{futurs4} et supposons que $\vrb\phi$ est vraie dans les états \tuple{\w_3,i_5} et \tuple{\w_4,i_5} et que $\vrb\phi$ est fausse dans tous les états de $\w_5$ et $\w_6$.  Plaçons nous dans l'état \tuple{\w_4,i_4}.  De là, \(\denote{\Xlo\mF\doit\phi}^{\Modele,\w_4,i_4}=1\) car $i_5$ est postérieur à $i_4$ et dans tous les mondes en relation avec $\w_4$ à l'instant $i_5$ (c'est-à-dire $\w_3$ et $\w_4$), \vrb\phi\ est vraie. Mais \(\denote{\Xlo\doit\mF\phi}^{\Modele,\w_4,i_4}=0\) car tous les mondes reliés à $\w_4$ à l'instant $i_4$ sont $\w_3$, $\w_4$, $\w_5$ et $\w_6$, et dans $\w_5$ et $\w_6$ il n'y a pas d'instant postérieur à $i_4$ où \vrb\phi\ est vraie.

Soit \tuple{w,i} un état du monde quelconque.  Et supposons que $\Xlo\mP\phi$ est vraie dans \tuple{w,i}.  Donc il existe un instant $i'$ antérieur à $i$ tel que \vrb\phi\ est vraie dans \tuple{w,i'}.  Par définition des relations d'accessibilité historiques, tous les mondes $w'$ reliés à $w$ par $\RK_i$ sont aussi reliés à $w$ par $\RK_{i'}$ (parce que $i'\tprec i$), et tous les états \tuple{w',i'} sont identiques à \tuple{w,i'}.  Donc comme $\vrb\phi$ est vraie dans \tuple{w,i'}, nous savons que $\vrb\phi$ est vraie dans tous les états \tuple{w',i'} ; cela permet de conclure que $\Xlo\mP\phi$ est vraie dans tous les états \tuple{w',i} et donc que $\Xlo\doit\mP\phi$ est vraie dans \tuple{w,i}. Ainsi, si $\Xlo\mP\phi$ est vraie dans \tuple{w,i}, alors $\Xlo\doit\mP\phi$ est vraie dans \tuple{w,i}, c'est-à-dire que $\Xlo[\mP\phi\implq\doit\mP\phi]$ est vraie dans \tuple{w,i}.

\end{solu}
\end{exo}
