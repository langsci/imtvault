\chapter{Clause linkage}
\label{sec:10}

\is{valency adjustments}
Relations between clauses may be established in various ways in complex clauses consisting of more than one verb. A relation between clauses can be expressed by using linking adverbials and anaphoric pronouns (\sectref{sec:10.1}). Adjacent clauses may also be linked by continuative intonation\is{continuative intonation} alone, or in combination with the other means available (\sectref{sec:10.8}). Clause linkers may be employed in order to form complex coordinate (\sectref{sec:10.3}), complement (\sectref{sec:10.5}), relative (\sectref{sec:10.6}), and adverbial clauses (\sectref{sec:10.7}). 


The resulting constructions are syntactically integrated to varying degrees. For instance, subjunctive clauses introduced by \textit{mék} ‘\textsc{sbjv}’ (\sectref{sec:10.5.5}), purpose clauses introduced by \textit{fɔ} ‘\textsc{prep’}, and complement clauses introduced by \textit{fɔ} ‘\textsc{prep’} or \textit{de} ‘\textsc{ipfv’} are less finite and arguably syntactically subordinate to their main clauses. At the same time, it is not very useful to posit a relation of syntactic subordination\is{subordination} between clauses in many (other) adverbial relations. In these structures, the linked clauses retain their full potential for the expression of person, tense\is{tense}, \isi{aspect}, and \isi{modality} (e.g. the various types of adverbial clauses introduced by \textit{wé} ‘\textsc{sub}’, cf. \sectref{sec:10.7.1}). Clauses may also be linked in multiverb constructions, which are covered separately in section \sectref{sec:10.8}.


\section{Linking adverbs and anaphor}\label{sec:10.1}

Linking adverbs occur at the beginning of a clause and ensure referential continuity with a preceding clause, often in combination with continuative intonation. Recurrent linking elements are \textit{áfta} ‘then, afterwards’, \textit{bɔt} ‘but’, the phrasal adverbial \textit{dán tɛ́n} ‘(at) that time’, as well as the anaphoric phrase \textit{na ín} ‘\textsc{foc 3sg.indp}’. 


The adverb \textit{áfta} ‘then, afterwards’ relates a situation with a previous one. It can be employed in ways very similar to that of certain clause linkers in prosodically more integrated constructions involving the clause linker \textit{wé} ‘\textsc{sub}’ (cf. \sectref{sec:10.7.1}). In the following sentence, \textit{áfta} and \textit{wé} both establish a link of temporal succession with the preceding clause. Both elements are preceded by continuative intonation\is{continuative intonation} (indicated by a comma): 



\ea%1355
    \label{ex:key:1355}
    \gll Yu  gó  yu  pé,  siete  mil      yu  baja,  \textbf{áfta}    yu  fínis    yu  sube,
\textbf{wé}  yu  de  pák    mɔ́    siete  mil,      \textbf{wé}  yu  sube.\\
\textsc{2sg}  go  \textsc{2sg}  pay  seven  thousand  \textsc{2sg}  go.down  then  \textsc{2sg}  finish  \textsc{2sg}  go.up
\textsc{sub}  \textsc{2sg}  \textsc{ipfv}  pack  more  seven  thousand  \textsc{sub}  \textsc{2sg}  go.up\\

\glt ‘You go, you pay, seven thousand, you go down, then you finish, you 
go up and take seven thousand again and go up.’ [f203fp 012]
\z

Example \REF{ex:key:1356} shows how the sequential meaning of \textit{áfta} can be read as a result relation in combination with continuative intonation\is{continuative intonation}:


\ea%1356
    \label{ex:key:1356}
    \gll \MakeUppercase{A}   nó  sabí    ús=tín  bin  kán  pás,    \textbf{áfta}    e    gó  
na  hospital.\\
\textsc{1sg.sbj}  \textsc{neg}  know  \textsc{q}=thing  \textsc{pst}  \textsc{pfv}  pass    then  \textsc{3sg.sbj}  go 
\textsc{loc}  hospital\\
\glt ‘I don’t know what happened that he went to (the) hospital.’ [ye03cd 074]
\z

The adverb \textit{áfta} may also introduce the then-clause of reality conditionals in which the if-clause is introduced by \textit{lɛk} ‘like’ \REF{ex:key:1357}:{\fff}


\ea%1357
    \label{ex:key:1357}
    \gll \textbf{Lɛk}  náw,  \textbf{lɛk}  Boyé  só    na  mi    mán,  \textbf{áfta}    mi    sísta  
go  kɔ́l=an    sé,    wé  e    go  kán,    “ús=sáy  mi
brɔda-lɔ́      dé?”\\
like  now    like  \textsc{name}  like.that  \textsc{foc}  \textsc{1sg.poss}  man    then  \textsc{1sg.poss}  sister
\textsc{pot}  call=\textsc{3sg.obj}  \textsc{quot}    \textsc{sub}  \textsc{3sg.sbj}  \textsc{pot}  come   \textsc{q}=side  \textsc{1sg.poss}
brother.\textsc{cpd}{}-law  \textsc{be.loc}\\

\glt ‘Suppose now, suppose Boyé here were my husband, then my sister would 
call him, if she came, “where’s my brother-in-law?”’ [ro05de 005]
\z

Example \REF{ex:key:1358} shows how the sequential meaning of \textit{áfta} can be read as a reason relation: 


\ea%1358
    \label{ex:key:1358}
    \gll Ɛf  yu  sí  sé,    sɔn    sáy  di  plés    klín,    \textbf{áfta}    dɛn  de  dú  
di  tín    dɛn  fáyn,  yu  nó  go  bísin  ɛf  yu  gasta  mɔní.\\
if  \textsc{2sg}  see  \textsc{quot}    some  side  \textsc{def}  \textsc{place}  be.clean  then  \textsc{3pl}  \textsc{ipfv}  do  
\textsc{def}  thing  \textsc{pl}  fine    \textsc{2sg}  \textsc{neg}  \textsc{pot}  be.busy  if  \textsc{2sg}  spend  money\\

\glt ‘If you see that, somewhere the place is clean and/ because things are done well, 
you don’t bother if you spend money.’ [ma03hm 066]
\z

The phrasal adverbial \textit{dán tɛ́n} ‘at that time’ also relates a situation to a preceding one. In \REF{ex:key:1359b}, \textit{dán tɛ́n} indicates a temporal relation of simultaneity with the preceding clause (a):


\ea%1359
    \label{ex:key:1359}
\ea{
\gll
E    mít    mi    antí.\\
  \textsc{3sg.sbj}  meet  \textsc{1sg.poss}  aunt\\
\glt   ‘He met my aunt.’ [fr03ft 086]
}\ex{\label{ex:key:1359b}
\gll
\textbf{Dán}    \textbf{tɛ́n}    mi    antí    gɛ́t  bɛlɛ́.\\
  that    time    \textsc{1sg.poss}  aunt    get  belly\\
\glt   ‘At that time my aunt was pregnant.’ [fr03ft 087]
}\z\z

The phrase \textit{na ín}, consisting of the focus marker \textit{na} and the emphatic \textsc{3sg} pronoun \textit{ín} establishes various types of anaphoric relationships (cf. also \sectref{sec:7.4.3.3}). In \REF{ex:key:1360}, a temporal interpretation is favoured due to the presence of the adverbial \textit{las doce} ‘twelve (o’clock)’:


\ea%1360
    \label{ex:key:1360}
    \gll Bikɔs  ín    de  sé,    ɛ́ni    \textbf{las}    \textbf{doce}  \textbf{na}  \textbf{ín}    in
abuela    kin  kán    kɔ́l=an.\\
because  \textsc{3sg.indp}  \textsc{ipfv}  \textsc{quot}    every  the.\textsc{pl}  twelve  \textsc{foc}  \textsc{3sg.indp}  \textsc{3sg.poss}
grandmother  \textsc{hab}  come  call=\textsc{3sg.obj}\\

\glt ‘Because she would say, always at twelve o’clock, that’s when her 
grandmother used to come and call her.’ [ed03sb 150]\is{anaphora}
\z

\section{Clause linkers}\label{sec:10.2}

Next to the use of anaphors, intonation, and SVCs, Pichi employs a large array of clause linkers to express relations between clauses. Linkers that serve to introduce adverbial clauses more specialised in their meanings are dealt with in \sectref{sec:10.7}. At the same time, most types of relations, including adverbial ones, can be expressed by one, or a combination of, the multifunctional elements \textit{wé} \textsc{‘sub’,} \textit{sé} \textsc{‘quot’,} \textit{mék} \textsc{‘sbjv’,} and \textit{fɔ} \textsc{‘prep’.} 


These four linkers have multiple, partially overlapping functions, which are mapped in \figref{fig:key:10.1}. The ways in which these four linkers introduce different types of clauses are covered in the following sections of this chapter.

% \todo{needs shading and boxes, unclear, where}
% TODO: Make fig
\begin{figure}
\caption{Functions of \textit{fɔ}, \textit{mék}, \textit{wé}, and \textit{sé} by clause type}
\label{fig:key:10.1}
\begin{tikzpicture}[x=1.4cm,y=0.6cm]
\node[fill=black!20,minimum height=1.8cm,minimum width=3.4cm] at (0,10.7) {};
\node at (0,14.2)   {Serial verb construction};
\node at (0,11.4)   {Purpose};
\node at (0,10)     {Complement};
\node at (0,8)      {Time};
\node at (0,7)      {Reality conditional};
\node at (0,6)      {Manner};
\node at (0,5)      {Circumstance};
\node at (0,4)      {Concessive};
\node at (0,2.7)    {Adversative};
\node at (0,1.7)    {Coordinate};
\node at (0,0)      {Independent clause};
\node at (3,5.5)    {Cause};
\node at (-3.2,7.5) {Relative};
\node at (3,10)     {Quotative};
\node at (4.3,10)   {Naming};

\draw[dashed] (-2,10.7) -- (2,10.7) -- (2,0.9) -- (-2,0.9) -- (-2,6.5) -- (-4,6.5) -- (-4,8.5) -- (-2,8.5) -- (-2,10.7);
\draw (-2.4,13.4) -- (2.4,13.4) -- (2.4,11) -- (5,11) -- (5,9.1) -- (2.4,9.1) -- (2.4,6.3) -- (3.7,6.3) -- (3.7,4.6) -- (2.4,4.6) -- (2.4,3.3) -- (-2.4,3.3) -- (-2.4,13.4);
\draw[loosely dotted] (-1.6,12.8) -- (1.6,12.8) -- (1.6,8.6) -- (-1.6,8.6) -- (-1.6,12.8);

\node[minimum width=1.2cm] at (3.5,1.5) (fɔ) {\textit{fɔ}}; \draw[loosely dotted] (fɔ.south west) -- (fɔ.south east);
\node[minimum width=1.2cm,fill=black!20] at (4.5,1.5) {\textit{mék}};
\node[minimum width=1.2cm] at (3.5,0.5) (we) {\textit{wé}}; \draw[dashed] (we.south west) -- (we.south east);
\node[minimum width=1.2cm] at (4.5,0.5) (se) {\textit{sé}}; \draw (se.south west) -- (se.south east);
\end{tikzpicture} 
\end{figure}


\section{Coordination}\label{sec:10.3}

Coordinate clauses may be linked by way of intonation as well as the linkers \textit{wé} ‘\textsc{sub}’ and \textit{an} ‘and’. In \REF{ex:key:1361}, \textit{bús} ‘forest’ bears a continuative boundary tone, which links the clause to the following one after the comma.


\ea%1361
    \label{ex:key:1361}
    \gll Só  e    gó  na  \textbf{bús,}    e    sé    e    de
gó  kíl  bíf.\\
so  \textsc{3sg.sbj}  \textsc{pot}  \textsc{loc}  forest  \textsc{3sg.sbj}  \textsc{quot}    \textsc{3sg.sbj}  \textsc{ipfv}
go  kill  wild.animal\\

\glt ‘So he went to the forest, (and) he said he was going to kill
wild game.’ [ma03sh 004]
\z

The clause linker \textit{wé} ‘\textsc{sub}’ can, amongst its other uses, link coordinate clauses. The preposition \textit{wet} ‘with’ may only conjoin \textsc{NPs} (cf. \sectref{sec:5.5}), hence an important function of \textit{wé} is to serve as a clausal connective that can be translated as ‘and (then)’. The formal differentiation between \textsc{NP} and clausal coordination in Pichi corresponds to an areal (West) African pattern (\citealt[349–353]{Mithun1988}). 


In the following excerpt from a personal narrative, the first \textit{wé} ‘\textsc{sub}’ in (b) establishes a link (b) to the preceding clause (a) after a clause-final declarative intonation\is{declarative intonation} (indicated by the full stop). At the same time, context suggests a more temporal meaning of ‘when’ of the second \textit{wé} in (b). Clause (c) resumes the narrative after declarative intonation at the end of (b): 



\ea%1362
    \label{ex:key:1362}
\ea{
    \gll  \'{A}fta    na  mi    gran-má    a    bin  de  kɔ́l  mamá. \\
  then  \textsc{foc}  \textsc{1sg.poss}  grand-ma  \textsc{1sg.sbj}  \textsc{pst}  \textsc{ipfv}  call  mother\\
\glt   ‘So it’s my grandmother that I used to call mother.’ [fr03ft 016]
}\ex{\label{ex:key:1362b}
\gll
\textbf{Wé}\textstylePichiexamplespaceZchn{} wi  kán  kán    na  tɔ́n,    \textbf{wé}  a    bigín  gó  skúl,  
  wé  a    bin  gɛ́t,  a    tínk    sé    seis  años.\\
  \textsc{sub}  \textsc{1pl}  \textsc{pfv}  come  \textsc{loc}  town  \textsc{sub}  \textsc{1sg.sbj}  begin  go  school
  \textsc{sub}  \textsc{1sg.sbj}  \textsc{pst}  get  \textsc{1sg.sbj}  think  \textsc{quot}    six  years\\
\glt 
  And then we came to town, and then I began to go to school, when I was, 
I think six years old.’ [fr03ft 017]
}\ex{
\gll
A    bigín  gó  skúl\\
  \textsc{1sg.sbj}  begin  go  school\\
\glt   ‘I began going to school.’ [fr03ft 018]
}\z\z

The sequential and temporal meanings of \textit{wé} ‘\textsc{sub}’ in clauses like \REF{ex:key:1362b} above may extend into contiguous meanings such as adversative \REF{ex:key:1363}. The various related meanings of \textit{wé} in these contexts may blur beyond recognition the demarcation between the coordinate clauses described in this section and the adverbial clauses covered in \sectref{sec:10.7.1}.


\ea%1363
    \label{ex:key:1363}
\ea{
    \gll  Frijoles  yɛ́s  frijoles.  \\
  bean.\textsc{pl}  yes  bean.\textsc{pl}  \\
\glt   ‘[The Cubans call them] frijoles, yes frijoles.’ [ed03sp 119]
}\ex{
\gll
\textbf{Wé}  yá    só,    frijoles  na  haricots  na  yá.\\
  \textsc{sub}  here    like.that  bean.\textsc{pl}  \textsc{foc}  beans  \textsc{loc}  here\\
\glt   ‘While here, frijoles is haricot here.’ [ed03sp 120]
}\z\z

The quotative marker \textit{sé} ‘\textsc{quot}’ also functions as a sequential connective and clause coordinator in ways very similar to \textit{wé} ‘\textsc{sub}’ when it signals inner speech or “internal awareness” (\citealt[422]{Güldemann2008}) and thereby often occurs without an overt subject\is{subject omission} as in \REF{ex:key:1364}:


\ea%1364
    \label{ex:key:1364}
\ea{
    \gll Dɛn  de  kɔ́l  dís  tín    fɔ  cacahuete,
  dɛn  de  kɔ́l=an    maní.\\
  \textsc{3pl}  \textsc{ipfv}  call  this  thing  \textsc{prep}  groundnut  
  \textsc{3pl}  \textsc{ipfv}  call=\textsc{3sg.obj}  ground.nut\\
\glt
  ‘They call this peanut thing, they call it “maní”.’ [ed03sp 082]\\
}\ex{
\gll
\textbf{Sé}    mɔ́nin  tɛ́n    a    go  gó,  a    báy,  a    ték  tú\\
\textsc{quot}  morning  time  \textsc{1sg.sbj}  \textsc{pot}  go  \textsc{1sg.sbj}  buy  \textsc{1sg.sbj}  take  two\\
  
\gll
peso    \op...\cp{}\\
peso\\
\glt 
‘So in the morning, I would go and buy (it), I would take two pesos (...)’ [ed03sp 083]
}\z\z

The element \textit{an} ‘and’ may link \textsc{NPs} as well as coordinate clauses. Its use is, however, exceedingly rare, and speakers overwhelmingly favour coordinate structures linked by means of \textit{wé} ‘\textsc{sub}’ or reduced clauses involving secondary predication (cf. \sectref{sec:11.3}):


\ea%1365
    \label{ex:key:1365}
    \gll E    nák  di  tébul  \textbf{an}  di  stáyl  wé  e    nák  di  tébul  strɔ́n,
e    kán  sék    di  plét    \textbf{an}  di  plét    kán  brók.\\
\textsc{3sg.sbj}  hit  \textsc{def}  table  and  \textsc{def}  style  \textsc{sub}  \textsc{3sg.sbj}  hit  \textsc{def}  table  be.strong
\textsc{3sg.sbj}  \textsc{pfv}  shake  \textsc{def}  plate  and  \textsc{def}  plate  \textsc{pfv}  break\\

\glt ‘He hit the table and the way that he hit the table in a strong way, 
he shook the plate, and the plate broke.’ [au07se 014]\is{coordination!clauses}
\z

The disjunctive coordinator \textit{ɔ} ‘or’ may also link coordinate clauses, cf. \REF{ex:key:1408} for an example.

\section{Quotation}\label{sec:10.4}

The element \textit{sé} ‘\textsc{quot}’ is characterised by an exceptional polyfunctionality that includes use as a lexical verb ‘say’ and use as quotation marker\is{quotative clauses} for direct speech\is{direct speech} and naming, renders inner speech and internal awareness, introduces adverbial clauses of manner, circumstance, and purpose\is{purpose clauses}, and reaches into the domain of clausal complementation. Following \citet{Güldemann2008}, I assume that the function as an index of direct reported speech lies at the heart of the functional versatility of \textit{sé} ‘\textsc{quot}’. 


The element \textit{sé} occurs with a more lexical meaning of ‘say’. It may take \textsc{TMA} marking and at the same time predicate a quotative construction. In the following example, \textit{sé} is employed as a speech verb. It is marked for potential mood by means of \textit{go} ‘\textsc{pot}’ and introduces a direct quote: 



\ea%1366
    \label{ex:key:1366}
    \gll Di  dé  wé  yu  go  níd=an,    yu  go  \textbf{sé}    “a    nó  gɛ́t  pamáyn”,
yu  go  kɔ́t  gadinɛ́ks.\\
\textsc{def}  day  \textsc{sub}  \textsc{2sg}  \textsc{pot}  need=\textsc{3sg.obj}  \textsc{2sg}  \textsc{pot}  \textsc{quot}    \textsc{1sg.sbj}  \textsc{neg}  get  oil
\textsc{2sg}  \textsc{pot}  cut  egg-plant\\

\glt ‘The day that you will need it, you are going to say “I don’t have oil,” (and) 
you will cut egg-plants.’ [ab03ay 015]
\z

In the example below, the use of \textit{sé} as a lexical verb ‘say’ coincides with the presence of habitual\is{habitual aspect} marking (i.e. \textit{kin} ‘\textsc{hab}’). However, in the overwhelming majority of instances, \textit{sé} remains bare, and hence marked for factative TMA\is{factative TMA}, since quotative constructions by their very nature occur in reported, past-time discourse: 


\ea%1367
    \label{ex:key:1367}
    \gll E    \textbf{kin}  \textbf{sé}    “kán    wi  gó  na  Barca  wi  gó  dríng.”\\
\textsc{3sg.sbj}  \textsc{hab}  \textsc{quot}    come  \textsc{1pl}  go  \textsc{loc}  \textsc{place}  \textsc{1pl}  go  drink\\

\glt ‘He usually says “come let’s go to Barca and drink”.’ [ro05rt 029]
\z

The transition from a more lexical reading of \textit{sé} to a more functional one is far from clear-cut (which is why I have opted for a unitary gloss of ‘\textsc{quot}’ in all contexts). First, distributional restrictions set \textit{sé} apart from the true speech verbs \textit{tɔ́k} ‘talk, say’ and \textit{tɛ́l} ‘tell’. For instance\textit{, }sé does not normally take a nominal object, as does \textit{tɔ́k}. Compare \REF{ex:key:1368a} and \REF{ex:key:1368b}.


\ea%1368
    \label{ex:key:1368}
\ea[*]{\label{ex:key:1368a}
\gll
Mék  a    \textbf{sé}    \textbf{wán}    \textbf{wɔ́d}.\\
   \textsc{sbjv}    \textsc{1sg.sbj}  \textsc{quot}    one    word\\
\glt   Intended: ‘Let me say one word.’ [to07fn 219]
}\ex[]{\label{ex:key:1368b}
\gll
A    \textbf{tɔ́k}    \textbf{wán}    \textbf{wɔ́d}.\\
  \textsc{1sg.sbj}  talk    one    word\\
\glt   ‘I said one word.’ [to07fn 220]
}\z\z

Beyond that, adverbials do not usually modify \textit{sé} ‘\textsc{quot}’ \REF{ex:key:1369a}. Adverbials only appear as quoted complements indexed by \textit{sé} (b). Again, there is no restriction on adverbial modification of the speech verb \textit{tɔ́k} ‘talk, say’ (c): 


\ea%1369
    \label{ex:key:1369}
\ea[*]{\label{ex:key:1369a}
\gll
A    sé=an    \textbf{kwík}.\\
    \textsc{1sg.sbj}  \textsc{quot}=\textsc{3sg.obj}  quickly\\
\glt   Intended: ‘I said it quickly.’ [to07fn 221]
}\ex[]{
\gll
A    sé    “\textbf{kwík}”.\\
  \textsc{1sg.sbj}  \textsc{quot}    quickly\\
\glt   ‘I said “quickly”.’ [to07fn 222]
}\ex[]{
\gll
A    tɔ́k=an    \textbf{kwík}\\
  \textsc{1sg.sbj}  talk=\textsc{3sg.obj}  quickly\\
\glt   ‘I said it quickly.’ [to07fn 223]
}\z\z

Secondly, \textit{sé} ‘\textsc{quot}’ is not normally encountered as a verbal complement. Hence below, the speech verb \textit{tɔ́k} ‘talk, say’ appears as verbal complement to the modal verb \textit{fít} ‘can’. The appearance of \textit{sé} in this position is not attested. 


\ea%1370
    \label{ex:key:1370}
    \gll Yu  \textbf{fít}  \textbf{tɔ́k}  “a    de  fíl  di  sɛ́nt    fɔ  lɛk  háw 
e    de  kúk    di  plantí”  ɔ  “a    de  siente  di  sɛ́nt
sé   pɔ́sin  de  kúk    plantí  dé”.\\
\textsc{2sg}  can  talk  \textsc{1sg.sbj}  \textsc{ipfv}  feel  \textsc{def}  scent  \textsc{prep}  like  how
\textsc{3sg.sbj}  \textsc{ipfv}  cook  \textsc{def}  plantain  \textsc{sp}  \textsc{1sg.sbj}  \textsc{ipfv}  feel    \textsc{def}  scent
\textsc{quot}    person  \textsc{ipfv}  cook  plantain  there\\

\glt ‘You can say “I smell the scent of him cooking the plantain”, or “I smell 
the scent that somebody is cooking plantain there”.’ [dj05ae 026]
\z

Note that I do not analyse \textit{sé} as a V2 of a complementation SVC when it functions as a complementiser to a verb like \textit{siente} ‘feel’ above (cf. also \sectref{sec:10.5.6}). The peculiar distribution of \textit{sé} as a speech “verb” and its broad functional domain, which extends far beyond complementation, may point to the fact that \textit{sé} ‘\textsc{quot’} did not start out as a speech verb in the first place. Instead, it is conceivable that the use of \textit{sé} as a speech “verb” is derived from quotation just like its many other functions (cf. \citealt[272–275]{Güldemann2008}). In this view, the resemblance of \textit{sé} with a purported English etymon \textit{sáy} may be due either to chance or to the convergence of diverse etymologies and functions in one form.


The recurrent use of quotative clauses introduced by \textit{sé} ‘\textsc{quot}’ with or without a preceding subject in order to render direct and inner speech is a conspicuous feature of longer stretches of narrative discourse. Direct speech\is{direct speech} in Pichi rarely serves the sole aim of giving neutral reports of utterances. One of its crucial functions is the creation of an atmosphere of vivacity and authenticity that builds up tension and draws listeners into the narrative. Compare (\ref{ex:key:1371a}–\ref{ex:key:1371e}), in which speaker (ed) recalls his difficulty in distinguishing a transsexual man from a woman: 



\ea%1371
    \label{ex:key:1371}
\ea{\label{ex:key:1371a}
    \gll A    \textbf{sé}    “na  mán    dís?”\\
  \textsc{1sg.sbj}  \textsc{quot}     \textsc{foc}  man    this\\
\glt   ‘I said “this is a man?”’ [ed03sb 222]
}\ex{\gll
E    \textbf{sé}    “na  mán.”\\
  \textsc{3sg.sbj}  \textsc{quot}    \textsc{foc}  man\\

\glt   ‘He said “it’s a man”.’ [ed03sb 223]
}\ex{\gll
A    \textbf{sé}    “yu  de  krés    mán.”\\
  \textsc{1sg.sbj}  \textsc{quot}    \textsc{2sg}  \textsc{ipfv}  be.crazy  man\\
\glt   ‘I said “you’re crazy, man”.’ [ed03sb 224]
}\ex{\gll
E    \textbf{sé}    “na  mán    dís.”\\
  \textsc{3sg.sbj}  \textsc{quot}    \textsc{foc}  man    this\\
\glt   ‘He said “this is a man”.’ [ed03sb 225]
}\ex{\label{ex:key:1371e}
\gll \textbf{Sé}    na  mán?\\
  \textsc{quot}    \textsc{foc}  man\\
\glt   ‘(You) say it’s a man?’ [ed03sb 226]
}\z\z

Example \REF{ex:key:1372} shows that the absence of overt subjects in this type of discourse opens up a grey area in which there is ample room for both a more functional and a more lexical reading of a subject-less, clause-initial \textit{sé}. Compare the unambiguous use of \textit{sé} as a speech verb in \REF{ex:key:1372a} with the alternative translations of the subject-less \textit{sé} in \REF{ex:key:1372b}:


\ea%1372
    \label{ex:key:1372}
\ea{ \label{ex:key:1372a}
    \gll
\textbf{E}   \textbf{go}  \textbf{sé}    e    de  fíɛ,  e    nó  go  gí
  mí    di  tín    wé  a    de  sɛ́n=an.\\
  \textsc{3sg.sbj}  \textsc{pot}  \textsc{quot}    \textsc{3sg.sbj}  \textsc{ipfv}  fear  \textsc{3sg.sbj}  \textsc{neg}  \textsc{pot}  give
  \textsc{1sg.indp}  \textsc{def}  thing  \textsc{sub}  \textsc{1sg.sbj}  \textsc{ipfv}  send=\textsc{3sg.obj}\\
\glt 
  ‘He would say, he was afraid (and) he wouldn’t give me the thing that 
  I was sending him for.’ [ab03ab 041]
}\ex{ \label{ex:key:1372b}
\gll
\textbf{Sé}    ín    nó  wánt  in    abuelo    skrách=an.\\
  \textsc{quot}    \textsc{3sg.indp}  \textsc{neg}  want  \textsc{3sg.poss}  grandfather  scratch=\textsc{3sg.obj}\\
\glt 
  ‘(He’d) say he [\textsc{emp}] doesn’t want his grandfather to scratch him.’ or 
‘Because he doesn’t want his grandfather to scratch him.’\is{subject omission}
}\z\z

Reported discourse also renders inner speech at important narrative junctures. In such a context, reported discourse may serve to express the intention of referents as in the sentences below:


\ea%1373
    \label{ex:key:1373}
    \gll In    brɔ́da  dɛn  ɔ́l  kɔmɔ́t  na  tɔ́n    yá    só  
\textbf{dɛn}  \textbf{sé} dɛn  de  kán    ték=an.\\
\textsc{3sg.poss}  brother  \textsc{pl}  all  go.out  \textsc{loc}  town  here    like.that
\textsc{3pl}  \textsc{quot} \textsc{3pl}  \textsc{ipfv}  come  take=\textsc{3sg.obj}\\

\glt ‘His brothers all left town, (so) they said they came to take her.’ [ab03ay 142]
\z


\ea%1374
    \label{ex:key:1374}
    \gll E    nó  sabí    tɔ́k  ni    Panyá,  \textbf{e}    \textbf{sé} e    wánt
muchachita  de  diecisiete  años.\\
\textsc{3sg.sbj}  \textsc{neg}  know  talk  even  Spanish  \textsc{3sg.sbj}  \textsc{quot}    \textsc{3sg.sbj}  want
young.girl  of  seventeen  year.\textsc{pl}\\

\glt ‘He doesn’t even know how to talk Spanish (and) he says he wants a young girl 
of seventeen years.’ [ye03cd 053]
\z

Speakers may use 3\textsuperscript{rd} person pronouns in reported speech as in \REF{ex:key:1374} above or insert direct quotations as in \REF{ex:key:1375} below. These elements together constitute some of the conspicuous characteristics of Pichi narrative discourse, in which the already weak boundary between direct and indirect speech\is{indirect speech} in Pichi is often deliberately blurred as part of a performance-oriented narrative technique:


\ea%1375
    \label{ex:key:1375}
    \gll Tidé    e    kán    \textbf{e}    sé,    “\textbf{a}    tínk    sé
a    go  fínis    ɔ́l  di  resto”.\\
today  \textsc{3sg.sbj}  come  \textsc{3sg.sbj}  \textsc{quot}    \textsc{1sg.sbj}  think  \textsc{quot}  
\textsc{1sg.sbj}  \textsc{pot} finish  all  \textsc{def}  rest\\

\glt ‘Today he came, he said “I think I am going to finish all the rest”.’ [ye03cd 147]
\z

A further facet of the quotative function is the use of \textit{sé} in a naming construction which serves to identify a nominal element by name and introduce members of a list (cf. \citealt[398]{Güldemann2008}). The named or listed items appear as nominal objects\is{objects} of \textit{sé}. 


\ea%1376
    \label{ex:key:1376}
    \gll Krío  mamá  dɛn  wé  dɛn  de  tɔ́k  Píchi  dɛn  kin  \textbf{tɔ́k}  \textbf{sé}    \textbf{grín}.\\
Krio  mother  \textsc{pl}  \textsc{sub}  \textsc{3pl}  \textsc{ipfv}  talk  Pichi  \textsc{3pl}  \textsc{hab}  talk \textsc{quot}    green\\

\glt ‘The elderly Krio women, when they talk Pichi, they usually say green.’ \newline [as opposed to ‘verd’ like younger people] [dj05ce 257]
\z

In combination with the verb \textit{kɔ́l} ‘call’, the naming construction translates as ‘be in a kinship\is{kinship terminology} relation with X’:


\ea%1377
    \label{ex:key:1377}
    \gll Na  fada-lɔ́,      na  di  papá  wé  e    bɔ́n    mí,    na  ín  
mi    mán    go  \textbf{kɔ́l}  \textbf{sé}    \textbf{suegro}.\\
\textsc{foc}  father.\textsc{cpd}{}-law  \textsc{foc}  \textsc{def}  father  \textsc{sub}  \textsc{3sg.sbj}  beget  \textsc{1sg.indp}  \textsc{foc}  \textsc{3sg.indp}  
\textsc{1sg.poss}  man    \textsc{pot}  call  \textsc{quot}    father-in-law\\

\glt ‘That is the father-in-law, that is the father who begat me, it is him that my 
husband would call father-in-law.’ [ro05de 007]
\z

Sentence \REF{ex:key:1378} exemplifies the use of \textit{sé} in listing. In these examples, the name or members of the list appear as nominal complements of \textit{sé}: 


\ea%1378
    \label{ex:key:1378}
    \gll A    fít  tɛ́l  yú    \textbf{sé}    morera,    teca,  kalabo.\\
\textsc{1sg.sbj}  can  tell  \textsc{2sg.indp}  \textsc{quot}    mulberry  teak    kalabo\\

\glt ‘I can tell you mulberry, teak, kalabo [listing types of wood].’ [\textstylePichiexamplenumberZchnZchn{ro05de 051]}
\z

The use of \textit{sé} to identify a nominal element represents the only context in which the quotative marker does not introduce a clause. Through this characteristic, the naming construction may be structurally identical to a copula construction involving the focus marker and identity copula \textit{na} ‘\textsc{foc}’. Compare the two consecutive sentences in (a) and (b) below: 


\ea%1379
    \label{ex:key:1379}
\ea{\gll
Na  mi    mamá.\\
  \textsc{foc}  \textsc{1sg.poss}  mother\\
\glt   ‘That’s my mother.’ [dj05ce 036]
}\ex{\gll
\textstylePichiexamplenumberZchnZchn{\textbf{Sé}}    mi    móm.\\
  \textsc{quot}    \textsc{1sg.poss}  mother\\
\glt   ‘Namely my mum.’ [dj05ce 037]
}\z\z

The data also contains examples in which the use of \textit{sé} as a deictic identifier of a nominal entity has been taken to its logical conclusion. In \REF{ex:key:1380}, \textit{sé} expresses identity in combination with the copula and focus marker \textit{na:}


\ea%1380
    \label{ex:key:1380}
    \gll Di  pikín  ɔ́l  \textbf{sé}    \textbf{na} mi    yón   
bikɔs  a    dɔ́n  pé  mɔní.\\
    \textsc{def}  child  all  \textsc{quot}    \textsc{foc}  \textsc{1sg.poss}  own  
because  \textsc{1sg.sbj}  \textsc{prf}  pay  money\\

\glt ‘The children are all mine because I have paid money 
[the dowry].’ [hi03cb 196]
\z

Aside from the functions covered in this section, the element \textit{sé} ‘\textsc{quot}’ is employed as a general clausal complementiser (cf. \sectref{sec:10.5}).\is{quotative clauses} 

\section{Complementation}\label{sec:10.5}

This section covers complex clauses featuring subordinate clauses with the syntactic function of complements. In the following, such clausal participants are referred to as complement clauses. Five strategies of integration of main and subordinate verbs are used next to each other, and sometimes they overlap (cf. \tabref{tab:key:10.1}). These strategies are covered in the following sections. 

\subsection{Finiteness}\label{sec:10.5.1}

Finiteness is an indicator of the degree of integration of Pichi complement clauses with main clauses. Main verbs vary with respect to how syntactically independent their complement predicates may be. Main verbs differ with respect to the complementiser they occur with, the time reference they project over their complement predicates, the person and TMA marking potential they accord their complement verbs, and the potential they confer on their complement verbs to be negated. In this vein, complement clauses consisting of a verb alone constitute the non-finite pole and complement clauses, in which the verb retains its full syntactic potential and constitutes the finite pole of complement clauses. \tabref{tab:key:10.1} checks the four principal complementation strategies in Pichi against five diagnostics of finiteness. “Complement clause” is abbreviated as “CC” in the table, “main clause” as “MC”.

%%please move \begin{table} just above \begin{tabular
\begin{sidewaystable}
\caption{Complementation and finiteness\is{complementisers}}
\label{tab:key:10.1}

\begin{tabularx}{\textwidth}{QlQQQ}
\lsptoprule

Feature/strategy & \textstyleTablePichiZchn{fɔ} ‘\textsc{prep}’ & \textsc{${\emptyset}$} /\textit{de} \textsc{‘ipfv’} & \textstyleTablePichiZchn{mék} ‘\textsc{sbjv}’ & \textstyleTablePichiZchn{sé} ‘\textsc{quot}’\\
\midrule
TMA reference of CC verb? & depends on MC verb & depends on MC verb & depends on MC verb & independent from MC verb\\

\tablevspace
Same or different subject CC? & same & same & same or different & same or different\\

\tablevspace
Is person marking with the CC verb obligatory, optional, or illicit? & illicit & illicit\par & obligatory & obligatory\\

\tablevspace
Is independent negation of the CC verb obligatory, optional, or illicit? & illicit & illicit & obligatory & obligatory\\

\tablevspace
Is TMA marking on the CC verb obligatory, optional, or illicit? & illicit & optional, with some verbs & optional, but restricted & obligatory\\
\lspbottomrule
\end{tabularx}
\end{sidewaystable}
The complementation strategies in \tabref{tab:key:10.1} form part of a continuum of complement clauses. The cline from non-finiteness to finiteness encompasses four complementation strategies, featuring the three overt complementisers \textit{fɔ} ‘\textsc{prep}’, \textit{mék} ‘\textsc{sbjv}’, and \textit{sé} ‘\textsc{quot}’, and a “zero” strategy. At the left end of the continuum, we find the highest number of syntactic restrictions in CCs linked to main verbs via the associative preposition \textit{fɔ} \textsc{‘prep’}. These are aspectual and modal auxiliary constructions. The subject of the CC verb must be co-referential with that of the main verb, is dependent on the temporal specification provided by the main verb, and may not be marked independently for person, negative polarity, or TMA. 


A significant number of modal and aspectual auxiliary verbs take clausal complements without an intervening complementiser, indicated by the column headed by “${\emptyset}$ (none)” in \tabref{tab:key:10.1}. A small sub-group of these verbs may, however, optionally be followed by the imperfective marker \textit{de}, which may then be seen to function as complementiser. However, the presence of \textit{de} \textsc{‘ipfv’} also adds an aspectual nuance by emphasising the continuous nature of the situation denoted by the CC verb. Such structures are therefore slightly more finite. On the one hand, the CC verb may be marked for aspect. On the other hand, the time reference of  the CC verb is determined by the taxis relation projected by the MC verb over the complex clause; for example the CC verb \textit{chɔ́p} ‘eat’ is necessarily in a relation of simultaneous taxis with the MC verb \textit{bigín} ‘begin to’ in a complement construction like \textit{a} \textit{bigín} \textit{de} \textit{chɔ́p} ‘I began to eat.’



Subjunctive complement clauses are, again, more finite. They may be same or different subject, always feature person marking, and must be negated independently of the main verb to signal negative polarity. They are, however, restriced in their TMA marking potential and depend on the main verb in their time reference (they are invariably future-projecting). At the right end of the continuum we find fully-fledged biclausal structures introduced by the quotative marker\is{quotative marker} \textit{sé} ‘\textsc{quot}’, which therefore functions as a typical finite complementiser. Not included in \tabref{tab:key:10.1} are the clause linkers \textit{ɛf(ɛ)} and\textit{ íf} ‘if’, which may function as complementisers in indirect question clauses (cf. \sectref{sec:10.6.5}).\is{finiteness}


\subsection{Complement-taking verbs and complementisers}

\tabref{tab:key:10.2} lists approximately sixty frequent Pichi main verbs that may take different types of complement clauses. The table sorts these verbs according to the type of complement clause linkage these verbs are attested with. The feature “semantic class” correlates strongly with the complementiser provided in the “linkage type” column. Beginning from the top of the table, the clause “linkage types” increase in finiteness as they descend towards the bottom. Verbs that may take complements introduced by \textit{sé} ‘\textsc{quot}’ are not fully listed, since that would make the list unduly long. Equally, some of the verbs listed with complementisers other than \textit{sé} ‘\textsc{quot’} may nevertheless take complements introduced by \textit{sé} when these are statements of fact and have independent time reference, e.g. \textit{a} \textit{de} \textit{sɔ́ri} \textbf{\textit{sé}} \textit{e} \textit{dɔ́n} \textit{kán} ‘I’m sorry that he has come.’ Conversely, speech verbs take quotative complements introduced by \textit{sé} but subjunctive complements when these are indirect commands, e.g. \textit{a} \textit{hála \textbf{sé}} \textit{“kán”} ‘I hollered “come”’ vs. \textit{a} \textit{hála} \textit{\textbf{sé}} \textbf{\textit{mék}} \textit{e} \textit{kán} ‘I hollered for him to come.’


Some verbs are listed twice under two types of clause linkage where the functions of complement clauses differ correspondingly. For example, \textit{wánt} usually appears without an overt complementiser (${\emptyset}$) in prospective aspect constructions. However, \textit{wánt} takes ${\emptyset}$ and \textit{de} \textsc{‘ipfv’} complements in same-subject (desire) modal auxiliary constructions, and must take \textit{mék} \textsc{‘sbjv’} with different-subject complements. Likewise, the general subordinator \textit{wé} \textsc{‘sub’} is not listed in \tabref{tab:key:10.2}, since its function as a complementiser is marginal. \tabref{tab:key:10.2} does not capture many other distributional complexities of complementisers and idiosyncracies of complementation, including negation in complement constructions. Details are provided in the corresponding sections of this chapter. 


% TODO: table needs to be go across two pgs
\begin{table}
\caption{Complement-taking verbs, semantic class, and type of clause linkage}
\label{tab:key:10.2}

\begin{tabularx}{\textwidth}{lll XXXXX}
\lsptoprule

Semantic class & Verb & Gloss & \textsc{${\emptyset}$} & \textit{de} \textsc{‘ipfv’} & \textstyleTablePichiZchn{fɔ} ‘\textsc{prep}’ & \textstyleTablePichiZchn{mék} ‘\textsc{sbjv}’ & \textstyleTablePichiZchn{sé} ‘\textsc{quot}’\\
\midrule
\itshape \textup{Aspectual}  & \itshape kɔmɔ́t & Egressive\is{egressive aspect} & $\times$ &  &  &  & \\
\itshape \textup{\& modal} & \itshape fínis & Completive\is{completive aspect} & $\times$ &  &  &  & \\
& \itshape sigue & Continuative\is{continuative aspect} & $\times$ &  &  &  & \\
& \itshape wánt & Prospective & $\times$ &  &  &  & \\
& \itshape bigín & Ingressive & $\times$ & $\times$ &  &  & \\
& \itshape fít & ‘can’ & $\times$ & $\times$ &  &  & \\
& \itshape gɛ́fɔ & ‘have to’ & $\times$ & $\times$ &  &  & \\
& \itshape hébul & ‘be capable of’ & $\times$ &  &  &  & \\
& \itshape mánech & ‘manage to’ & $\times$ &  &  &  & \\
& \itshape sabí & ‘know how to’ & $\times$ &  &  &  & \\
& \itshape lɛ́k & ‘like to’ & $\times$ &  &  &  & \\
& \itshape kɔ́stɔn & ‘be used to’ & $\times$ &  &  &  & \\
& \itshape lɛ́f & ‘stop (doing)’ &  &  & $\times$ &  & \\
& \itshape lán & ‘learn to’ &  &  & $\times$ &  & \\
& \itshape fɔgɛ́t & ‘forget to’ &  &  & $\times$ &  & \\
\itshape \textup{Experiential} \is{body states} & \itshape bísin & ‘be busy (with)’ &  &  & $\times$ &  & \\
\itshape \textup{\& body state} & \itshape táya & ‘be tired of’ &  &  & $\times$ &  & \\
& \itshape gládin & ‘be happy to’ &  &  & $\times$ &  & \\
& \itshape sɔ́ri & ‘be sorry to’ &  &  & $\times$ &  & \\
& \itshape sém & ‘be ashamed of’ &  &  & $\times$ &  & \\
\itshape \textup{Weak} \is{deontic modality} & \itshape fáyn & ‘be fine to’ &  &  &  $\times$ & $\times$ & $\times$\\
\itshape \textup{deontic} & \itshape bád & ‘be bad to’ &  &  & $\times$ & $\times$ & $\times$\\
& \itshape gúd & ‘be good to’ &  &  & $\times$ & $\times$ & $\times$\\
& \itshape hád & ‘be difficult to’ &  &  & $\times$ & $\times$ & $\times$\\
& \itshape ísi & ‘be easy to’ &  &  & $\times$ & $\times$ & $\times$\\
& \itshape fía & ‘be afraid to’ &  &  & $\times$ & $\times$ & $\times$\\
& \itshape mɛ́mba & ‘remember to’ &  &  & $\times$ & $\times$ & $\times$\\
& \itshape fíl & ‘feel like’ &  &  & $\times$ & $\times$ & $\times$\\
& \itshape tráy & ‘try to’ & $\times$ &  & $\times$ & $\times$ & $\times$\\
& \itshape níd & ‘need to’ &  &  & $\times$ & $\times$ & $\times$\\
& \itshape grí & ‘agree to’ &  &  & $\times$ & $\times$ & $\times$\\
& \itshape hɛ́lp & ‘help to’ &  &  & $\times$ & $\times$ & $\times$\\
\midrule 
\end{tabularx}
\end{table}

\begin{table}
\begin{tabularx}{\textwidth}{lll XXXXX}
\midrule 
Semantic class & Verb & Gloss & \textsc{${\emptyset}$} & \textit{de} \textsc{‘ipfv’} & \textstyleTablePichiZchn{fɔ} ‘\textsc{prep}’ & \textstyleTablePichiZchn{mék} ‘\textsc{sbjv}’ & \textstyleTablePichiZchn{sé} ‘\textsc{quot}’\\
\midrule
\itshape \textup{Strong}  & \itshape wánt & ‘want to’ & $\times$ & $\times$ &  & $\times$ & $\times$\\
\itshape \textup{deontic} & \itshape mék & ‘cause to’ &  &  &  & $\times$ & $\times$\\
& \itshape lɛ́f & ‘allow to’ &  &  &  & $\times$ & $\times$\\
& \itshape fɔ́s & ‘force to’ &  &  &  & $\times$ & $\times$\\
& \itshape tún & ‘persuade to’ &  &  &  & $\times$ & $\times$\\
& \itshape tɛ́l & ‘tell to’ &  &  &  & $\times$ & $\times$\\
& \itshape áks & ‘ask to’ &  &  &  & $\times$ & $\times$\\
& \itshape bɛ́g & ‘ask to’ &  &  &  & $\times$ & $\times$\\
\itshape \textup{Speech} & \itshape tɔ́k & ‘talk, say’ &  &  &  &  & $\times$\\
& \itshape tɛ́l & ‘tell that’ &  &  &  &  & $\times$\\
& \itshape hála & ‘shout that’ &  &  &  &  & $\times$\\
& \itshape ánsa & ‘answer that’ &  &  &  &  & $\times$\\
\itshape \textup{Perception} & \itshape chɛ́k & ‘think that’ &  &  &  &  & $\times$\\
\itshape \textup{\& cognition} & \itshape tínk & ‘think that’ &  &  &  &  & $\times$\\
& \itshape nó\textup{/}sabi & ‘know that’ &  &  &  &  & $\times$\\
& \itshape bilíf & ‘believe that’ &  &  &  &  & $\times$\\
& \itshape kechɔ́p & ‘realise that’ &  &  &  &  & $\times$\\
& \itshape sí & ‘see that’ &  &  &  &  & $\times$\\
& \itshape hía & ‘hear that’ &  &  &  &  & $\times$\\
& \itshape smɛ́l & ‘smell that’ &  &  &  &  & $\times$\\
& \itshape fíl & ‘feel that’ &  &  &  &  & $\times$\\
\itshape \textup{(Other)} & \itshape e dé & ‘it is that’ &  &  &  &  & $\times$\\
\itshape \textup{factives} & \itshape na (nóto) & ‘it is (not) that’ &  &  &  &  & $\times$\\
& \itshape di tín dé & ‘the thing is that’ &  &  &  &  & $\times$\\
& \itshape di kés dé & ‘the thing is that’ &  &  &  &  & $\times$\\
\lspbottomrule
\end{tabularx}
\end{table}


\subsection{\textit{De} ‘\textsc{ipfv}’}\label{sec:10.5.3}

The aspectual and modal verbs \textit{bigín} ‘begin’, \textit{wánt/wɔ́nt} ‘want, be about to’, \textit{fít} ‘can’, and \textit{gɛ́fɔ} ‘have to’ feature complements introduced by the zero strategy or complement verbs preceded by the imperfective marker \textit{de} ‘\textsc{ipfv}’. \textit{Bigín} is particularly likely to occur with \textit{de} ‘\textsc{ipfv}’ when used as an ingressive auxiliary (cf. \sectref{sec:6.4.1} for examples). The use of the imperfective marker de emphasises the continuous nature of the situation dennoted by the verb. Compare the following constructions.{\fff}


\ea%1381
    \label{ex:key:1381}
\gll
        \textstylePichiglossZchn{Yú}   \textbf{wɔ́nt}  \textbf{de} gó?\\
\textsc{2sg}  want  \textsc{ipfv}  go\\

\glt ‘You want to (get) go(ing)?’ [nn07fn 202]
\z


\ea%1382
    \label{ex:key:1382}
\gll
Yu  \textbf{fít}  \textbf{de} bɔ́n\textbf{}     yu  pikín  dɛn   \op...\cp{}\\
\textsc{2sg}  can  \textsc{ipfv}  give.birth  \textsc{2sg}  child  \textsc{pl}\\

\glt ‘You can be (continuously) having your children (...)’ [ab03ab 197]
\z


\ea%1383
    \label{ex:key:1383}
    \gll Yu  \textbf{gɛ́fɔ}    \textbf{de}  tɔ́n=an.\\
\textsc{2sg}  have.to  \textsc{ipfv}  turn=\textsc{3sg.obj}\\

\glt ‘You have to (continuously) be stirring it.’ [dj03do 057]
\z

\largerpage
Note that both verbs in the constructions above are always co-referential; they have a subject\is{subjects} in common. Aspect-marking for simultaneous taxis via imperfective aspect is also found with depictive secondary predicates (cf. \sectref{sec:11.3}).\is{imperfective aspect}

\subsection{\textit{Fɔ} ‘\textsc{prep}’} 

The multifunctional element \textit{fɔ} ‘\textsc{prep}’ is, amongst its many other uses, employed to mark the citation form of verbs (e.g. \textit{fɔ} \textit{rós} ‘to burn’, \textit{fɔ} \textit{espia} ‘to spy on’). As a clause linker, \textit{fɔ} introduces nominal, hence non-finite complements. Hence, when \textit{fɔ} is used as a complementiser, the complement verb may not take an overtly expressed subject\is{subjects} and the main and complement verbs have the same subject by default.


Some aspectual and modal verbs are characterised by variation in their occurrence with \textit{fɔ}{}-complements. For instance, \textit{grí} ‘agree’ and \textit{tráy} ‘try’ are attested with the zero strategy of complementation and with complements introduced by \textit{fɔ}. The modal verb \textit{tráy} ‘try’ appears without the element \textit{fɔ} in \REF{ex:key:1384} and with it in \REF{ex:key:1385}:



\ea%1384
    \label{ex:key:1384}
    \gll E    wánt  \textbf{tráy}    mɛ́n    fɔ́s.\\
\textsc{3sg.sbj}  want  try    cure    first\\

\glt ‘She wanted to try to get better first.’ [ed03sb 044]
\z


\ea%1385
    \label{ex:key:1385}
    \gll E    de  tínap,  smɔ́l  pikín  wé  e    de
\textbf{tráy}    \textbf{fɔ} \textstylePichiexamplenumberZchnZchn{tínap}    yet.\\
\textsc{3sg.sbj}  \textsc{ipfv}  stand up  small  child  \textsc{sub}  \textsc{3sg.sbj}  \textsc{ipfv}
try    \textsc{prep}  stand up    yet\\

\glt ‘She’s beginning to stand, a small child that is still trying
to stand.’ [dj05be 219]
\z

As a complementiser, \textit{fɔ} introduces the complements of aspectual and modal verbs that may not occur without an overt complementiser. One of these verbs is \textit{lɛ́f} ‘leave, stop to’ \REF{ex:key:1386}, a verb that expresses the aspectual notion of cessation: 


\ea%1386
    \label{ex:key:1386}
    \gll Mék    e    \textbf{lɛ́f}    \textbf{fɔ}  dríng.\\
\textsc{sbjv}    \textsc{3sg.sbj}  leave  \textsc{prep}  drink\\

\glt ‘She should leave drinking.’ [ra07fn 033]
\z

The element\textit{ fɔ} ‘\textsc{prep}’ also introduces the complements of a number of experiential and body state \is{body states}verbs, which are also not attested in any other type of construction. These verbs predetermine a simultaneous time reference of their complements. An example follows, in which \textit{fɔ} introduces the complement of the experiential verb \textit{sém} ‘be ashamed’ \REF{ex:key:1387}:


\ea%1387
    \label{ex:key:1387}
    \gll Náw    a    dɔ́n    de  fínis    \textbf{sém}      \textbf{fɔ}  wɛ́r    dán    sús,
ɛf  a    bin  nó    a    fɔ  kɛ́r    ɔ́da    sús.\\
now    \textsc{1sg.sbj}  \textsc{prf}    \textsc{ipfv}  finish  be.ashamed  \textsc{prep}  wear  that    shoe
if  \textsc{1sg.sbj}  \textsc{pst}  know  \textsc{1sg.sbj}  \textsc{cond}  carry  other  shoe\\

\glt ‘Now I am completely ashamed to be wearing those shoes, if I had known
I would have brought another (pair of) shoes.’ [ma03hm 021]
\z

Furthermore, \textit{fɔ} introduces complements of a number of verbs whose meaning contains an element of proposal, desire, evaluation, and similar affective nuances compatible with deontic modality. I regroup these verbs under the label “weak deontic”. The deontic meaning of these verbs is also compatible with the modal meanings of \textit{fɔ} itself (cf. \sectref{sec:6.7.3.2}). When main and complement verbs have the same subject, the complement clause may be introduced by \textit{fɔ}. Compare the verbs \textit{ísi} ‘be easy’ \REF{ex:key:1388} and \textit{grí} ‘agree’ \REF{ex:key:1389}: 


\ea%1388
    \label{ex:key:1388}
    \gll Di  chɔ́p  \textbf{ísi}    \textbf{fɔ} chɔ́p.\\
\textsc{def}  food    be.easy  \textsc{prep}  eat\\

\glt ‘The food is easy to eat.’ [ye07je 095]
\z


\ea%1389
    \label{ex:key:1389}
    \gll Di  gál    nó  \textbf{grí}    \textbf{fɔ} fála    mí.\\
\textsc{def}  girl    \textsc{neg}  agree  \textsc{prep}  follow  \textsc{1sg.indp}\\

\glt ‘The girl didn’t agree to come with me.’ [au07ec 060]
\z

Any weak deontic verb may alternatively take a subjunctive clause complement introduced by the subjunctive marker and modal complementiser \textit{mék} ‘\textsc{sbjv}’ if the main verb is understood to induce a posterior time reference over the complement verb. For example, the complements of the weak deontic verb \textit{mɛ́mba} ‘remember to’ may be introduced by \textit{fɔ} ‘\textsc{prep}’ \REF{ex:key:1390} or by \textit{mék} ‘\textsc{sbjv}’ \REF{ex:key:1391}. In both sentences below, the main and complement clauses share the same subject{\fff}. However, the subjunctive clauses is more finite – it requires an overt subject. In contrast, the use of a \textit{fɔ}-complement does not permit the occurrence of an overt subject.


\ea%1390
    \label{ex:key:1390}
    \gll A    \textbf{mɛ́mba}    \textbf{fɔ}  kɔ́l=an.\\
\textsc{1sg.sbj}  remember  \textsc{prep}  call=\textsc{3sg.obj}\\

\glt ‘I remembered to call her.’ [au07ec 067]
\z


\ea%1391
    \label{ex:key:1391}
    \gll A    \textbf{mɛ́mba}    \textbf{mék}    \textbf{a}    kɔ́l=an.\\
\textsc{1sg.sbj}  remember  \textsc{sbjv}    \textsc{1sg.sbj}  call=\textsc{3sg.obj}\\

\glt ‘I remembered to call her.’ [au07ec 065]
\z

With weak deontic verbs, the subjunctive marker \textit{mék} ‘\textsc{sbjv}’ may not only be employed instead of \textit{fɔ} ‘\textsc{prep}’. A subjunctive clause may also immediately follow \textit{fɔ}. Hence all weak deontic verbs may feature the complementiser series \textit{fɔ} \textit{mék} ‘\textsc{prep} \textsc{sbjv}’ as in \REF{ex:key:1392} below:


\ea%1392
    \label{ex:key:1392}
    \gll So  wé  yu  dɔ́n    lán    yu  lángwech  ɛ́n,  e    dɔ́n  \textbf{hád}
\textbf{fɔ}  \textbf{mék}   yu  lán    Panyá.\\
so  \textsc{sub}  \textsc{2sg}  \textsc{prf}    learn  \textsc{2sg}  language    \textsc{intj}  \textsc{3sg.sbj}  \textsc{prf}  hard
\textsc{prep}  \textsc{sbjv}    \textsc{2sg}  learn  Spanish\\

\glt ‘So when you’ve learned your (home) language, it is hard for you to learn 
Spanish.’ [to03gm 020]
\z

The use of subjunctive complement clauses is, however, required with weak deontic verbs whenever the main and complement clauses do not have the subject in common. Compare \REF{ex:key:1385} above with \REF{ex:key:1393} below. Both sentences feature the main verb \textit{tráy} ‘try’:


\ea%1393
    \label{ex:key:1393}
    \gll \op...\cp{}  \textbf{a}   go  tráy    \textbf{mék}   \textbf{e}    báy mí    dán    káyn
gafas  por  dios.\\
{}  \textsc{1sg.sbj}  \textsc{pot}  try    \textsc{sbjv}    \textsc{3sg.sbj}  buy  \textsc{1sg.indp}  that    kind
glasses  by  God\\

\glt ‘(...) I will try that she buys me that kind of glasses, by God.’ [ye07ga 003]
\z

A subjunctive complement is also necessary if the complement verb is negated. This is so because non-finite verbs – including those that appear in \textit{fɔ}-complements – are not normally negated in Pichi. Compare the negated complement clause introduced by \textit{mék} ‘\textsc{sbjv}’ in \REF{ex:key:1394} with the affirmative complement clause introduced by \textit{fɔ} ‘\textsc{prep}’ in \REF{ex:key:1390} above. Both sentences involve the main verb \textit{mɛ́mba} ‘remember’:


\ea%1394
    \label{ex:key:1394}
    \gll Na  ín    a    \textbf{mɛ́mba}    \textbf{mék}    a    \textbf{nó}  gó  dé.\\
\textsc{foc}  \textsc{3sg.indp}  \textsc{1sg.sbj}  remember  \textsc{sbjv}    \textsc{1sg.sbj}  \textsc{neg}  go  there\\

\glt ‘That’s when I remembered not to go there.’ [bo05fn 021]
\z

The evaluative verbs \textit{fáyn} ‘be fine’, \textit{hád} ‘be hard’, \textit{ísi} ‘be easy’, \textit{bád} ‘be bad’, and \textit{gúd} ‘be good’ may be followed by a \textit{fɔ}{}-complement when the subject of the main clause is expletive, i.e. refers to no specific person or entity as in \REF{ex:key:1395}. Complements of evaluative main verbs with expletive subjects function as the notional subject of the main clause:


\ea%1395
    \label{ex:key:1395}
    \gll E    \textbf{fáyn}  \textbf{fɔ} dríng  smɔ́l-wán.\\
\textsc{3sg.sbj}  fine    \textsc{prep}  drink  small\textsc{{}-adv}\\

\glt ‘It’s good to drink little.’ [ma03hm 071]
\z

Once the complement situation has a fully referential subject (which is necessarily not co-referential with the expletive subject\is{expletive} of the main clause), a subjunctive complement clause is required \REF{ex:key:1396}:


\ea%1396
    \label{ex:key:1396}
    \gll Wé  yu  de  dríng,  e    dé \textbf{fáyn}  \textbf{sé}    \textbf{mék}  yu  nó  chák.\\
\textsc{sub}  \textsc{2sg}  \textsc{ipfv}  drink  \textsc{3sg.sbj}  \textsc{be.loc}  fine    \textsc{quot}    \textsc{sbjv}    \textsc{2sg}  \textsc{neg}  get.drunk\\

\glt ‘When you drink, it’s good not to get drunk.’ [ur07fn 288]\is{complementisers!non-finite}
\z

Note the presence of the quotative marker\is{quotative marker} and general complementiser \textit{sé} ‘\textsc{quot}’ in \REF{ex:key:1396} above. Any subjunctive clause may additionally be preceded by \textit{sé} (cf. \sectref{sec:10.5.5}):\is{associative preposition}

\subsection{\textit{Mék} ‘\textsc{sbjv}’} \label{sec:10.5.5}

Verbs expressing the strong deontic notion of manipulation are only attested with subjunctive complements introduced by \textit{mék} ‘\textsc{sbjv}’. These complements have a dependent time reference; the complement situation is always posterior to that of the main verb. Strong deontic verbs invariably express a strong degree of manipulation, a notion that is compatible with the use of subjunctive mood in directives\is{directives} (cf. \sectref{sec:6.7.3.3}). Below follow subjunctive complements of the manipulative verbs \textit{fɔ́s} ‘force’ \REF{ex:key:1397} and \textit{tún} ‘tune, persuade’ \REF{ex:key:1398}:


\ea%1397
    \label{ex:key:1397}
    \gll \MakeUppercase{A}   \textbf{fɔ́s}=an    \textbf{mék}    e    lúk    mí.\\
\textsc{1sg.sbj}  force=\textsc{3sg.obj}  \textsc{sbjv}    \textsc{3sg.sbj}  look    \textsc{1sg.indp}\\

\glt ‘I forced him to look at me.’ [dj05ae 034]
\z


\ea%1398
    \label{ex:key:1398}
    \gll \MakeUppercase{A}   \textbf{tún}=an    \textbf{sé}    \textbf{mék}   e    báy    mí    motó.\\
\textsc{1sg.sbj}  tune=\textsc{3sg.obj}  \textsc{quot}    \textsc{sbjv}    \textsc{3sg.sbj}  buy    \textsc{1sg.indp}  car\\

\glt ‘I coaxed her into buying me a car.’ [ye05fn 044]
\z

The class of manipulative complement-taking verbs also includes the verbs \textit{mék} ‘make, cause to’ \REF{ex:key:1399} and \textit{lɛ́f} ‘leave, permit’ in their respective functions as causative and permis\-sive-causative verbs (cf. \sectref{sec:9.4.4}):{\fff}


\ea%1399
    \label{ex:key:1399}
    \gll E    \textbf{mék}    \textbf{mék}    in    húman  dríng  di  cerveza.\\
\textsc{3sg.sbj}  make  \textsc{sbjv}    \textsc{3sg.poss}  woman  drink  \textsc{def}  beer\\

\glt ‘He made his woman drink the beer.’ [dj05be 001]
\z

The manipulative verb of desire \textit{wánt} ‘want’ is very versatile. It may occur in same subject\is{subjects} complement clauses (cf. e.g. \ref{ex:key:412}) or may take complements featuring the imperfective marker \textit{de} ‘\textsc{ipfv}’ (cf. e.g. \ref{ex:key:1381}). When the subjects of the main and complement clauses are not the same, a subjunctive \textit{mék}{}-complement is required: 


\ea%1400
    \label{ex:key:1400}
    \gll E    nák  di  plét    pan  di  tébul  bikɔs  e    \textbf{wánt}
\textbf{mék}    di  plét    brók.\\
\textsc{3sg.sbj}  hit  \textsc{def}  plate  pan  \textsc{def}  table  because  \textsc{3sg.sbj}  want
\textsc{sbjv}    \textsc{def}  plate  break\\

\glt ‘He hit the plate on the table because he wanted the plate to break.’ [au07se 194]
\z

Speech verbs employed as verbs of ordering and manipulation always take subjunctive complements. Examples of such indirect imperatives{\fff} are provided in the following two sentences involving the verbs \textit{tɛ́l} ‘tell (to)’ \REF{ex:key:1401} and \textit{bɛ́g} ‘ask to’ \REF{ex:key:1402}: {\fff}


\ea%1401
    \label{ex:key:1401}
    \gll \'{A}fta,  bueno  \textbf{tɛ́l}=an    sé    \textbf{mék}    e    bák
yú    di  mɔní.\\
then  good  tell=\textsc{3sg.obj}  \textsc{quot}    \textsc{sbjv}    \textsc{3sg.sbj}  give.back
\textsc{2sg.indp}  \textsc{def}  money\\

\glt ‘Then, ok, tell him that he should give you back the money.’ [ye03cd 032]
\z


\ea%1402
    \label{ex:key:1402}
    \gll E    bin  \textbf{bɛ́g}=an    sé    \textbf{mék}    e    kíl  di  fɔ́l.\\
\textsc{3sg.sbj}  \textsc{pst}  beg=\textsc{3sg.obj}  \textsc{quot}    \textsc{sbjv}    \textsc{3sg.sbj}  kill  \textsc{def}  fowl\\

\glt ‘She asked him to kill the fowl.’ [dj05ae 043]
\z

In a few instances in the data, the complements of strong deontic verbs are not introduced by \textit{mék} ‘\textsc{sbjv}’; the subjunctive marker is absent. I give two examples featuring the main verbs \textit{lɛ́f} ‘leave, permit’ \REF{ex:key:1403} and \textit{wánt} ‘want’ \REF{ex:key:1404}: 


\ea%1403
    \label{ex:key:1403}
    \gll Na  ín    mi    gran-má    bin  kán    tɔ́k  sé
in    nó  go  \textbf{lɛ́f}    mi    a    \textbf{gó}.\\
\textsc{foc}  \textsc{3sg.indp}  \textsc{1sg.poss}  grand-ma  \textsc{pst}  come  talk  \textsc{quot}
\textsc{3sg.indp}  \textsc{neg}  \textsc{pot}  leave  \textsc{1sg.indp}  \textsc{1sg.sbj}  go\\

\glt ‘That’s when my grandma said that she [\textsc{emp}] wouldn’t let me go.’ [fr03ft 078]
\z


\ea%1404
    \label{ex:key:1404}
    \gll Sé    ín    nó  \textbf{wánt}  {in}    {abuelo}    \textbf{skrách}\textbf{\textmd{=an}}.\\
\textsc{quot}    \textsc{3sg.indp}  \textsc{neg}  want  \textsc{3sg.poss}  grandfather  scratch=\textsc{3sg.obj}\\

\glt ‘(He) said, he [\textsc{emp}] didn’t want his grandfather to scratch him.’ [ab03ab 042]
\z

Notwithstanding the absence of the subjunctive marker, I analyse the clauses in bold in \REF{ex:key:1403} and \REF{ex:key:1404} above as subjunctive clauses. Evidence comes from the reduced TMA marking that characterises these clauses. Although both subordinate clauses are future-referring, they are not marked by \textit{go} ‘\textsc{pot}’ as they would if they occurred in main clauses or clauses with independent time reference (e.g. in quotative clause\is{quotative clauses}s introduced by \textit{sé} ‘\textsc{quot}’). Instead, the subordinate verbs \textit{gó} ‘go’ and \textit{skrách} ‘scratch’ appear stripped of any TMA marking as do subjunctive complements introduced by \textit{mék} ‘\textsc{sbjv}’.\is{deontic modality} 


This shows that the reduction of TMA marking, or “deranking” (\citealt[76–86]{Stassen1985}; cf. also \citealt{Cristofaro2003}) of the subjunctive subordinate clause is just as much a diagnostic of subjunctive mood as is the presence of the modal complementiser \textit{mék} ‘\textsc{sbjv}’. \is{subjunctive mood}


\subsection{\textit{Sé} ‘\textsc{quot}’}\label{sec:10.5.6}

We saw in the preceding two sections that the quotative marker \textit{sé} ‘\textsc{quot}’ can optionally introduce any subjunctive complement featuring the modal complementiser \textit{mék} ‘\textsc{sbjv}’. This distribution is in line with the function of the quotative marker as a general complementiser. 


The quotative marker \textit{sé} ‘\textsc{quot}’ introduces the finite complement clauses of speech \REF{ex:key:1405}, cognition \REF{ex:key:1406}, and perception verbs \REF{ex:key:1407}. Complement clauses introduced by \textit{sé} have independent time reference and are not reduced; they are finite and may occur with the full range of \textsc{TMA} marking as in the following examples: 



\ea%1405
    \label{ex:key:1405}
    \gll Yɛ,    a    kán  \textbf{tɛ́l}=an    \textbf{sé}    ‘chica,  mí    nó  lɛ́k  yú
bɔt  wi  fít  dé    lɛk  kɔ́mpin’.\\
yeah  \textsc{1sg.sbj}  \textsc{pfv}  tell=\textsc{3sg.obj}  \textsc{quot}    girl    \textsc{1sg.indp}  \textsc{neg}  like  \textsc{2sg.indp}
but  \textsc{1pl}  can  \textsc{be.loc}  like  friend\\

\glt ‘Yeah, I told her “girl, I don’t love you but we can be like friends”.’ [ru03wt 029]
\z


\ea%1406
    \label{ex:key:1406}
    \gll Nɔ́  a    \textbf{tínk}    \textbf{sé}    realmente  yu  níd    pikín.\\
\textsc{intj}  \textsc{1sg.sbj}  think  \textsc{quot}    really    \textsc{2sg}  need  child\\

\glt ‘Actually, I think that one really needs children.’ [fr03ft 163]
\z


\ea%1407
    \label{ex:key:1407}
    \gll Yu  jɔ́s  \textbf{hía}    \textbf{sé}    pɔ́sin  dɛn  bin  de  tɔ́k,  bɔt  yu  nó  listin.\\
\textsc{2sg}  just  hear    \textsc{quot}    person  \textsc{pl}  \textsc{pst}  \textsc{ipfv}  talk  but  \textsc{2sg}  \textsc{neg}  listen\\

\glt ‘You just heard that people were talking but you didn’t listen.’ [au07se 109]
\z

When \textit{sé} ‘quot’ introduces the complements of speech verbs, the difference between direct and indirect speech{\fff} hinges on pronominal reference. For instance, the sentence in quotes in \REF{ex:key:1405} above is a direct speech {\fff}complement of \textit{tɛ́l} ‘tell’, because reference to \textit{chica} ‘girl’ switches from \textit{=an} ‘\textsc{3sg.obj}’ in the main clause to the object pronoun \textit{yú} ‘\textsc{2sg.indp}’ in the complement clause.


With cognition and perception main verbs, the perceived situation can also be expressed as an adverbial time clause introduced by \textit{sé} ‘\textsc{quot}’ \REF{ex:key:1408} (cf. also \ref{ex:key:1469} further below) or \textit{wé} ‘\textsc{sub}’ (cf. \ref{ex:key:1463}), and an adverbial time clause introduced by \textit{lɛk háw} ‘the way that’ \REF{ex:key:1408}. The adverbial clause is marked for imperfective aspect, since it is simultaneous with the main clause situation: 



\ea%1408
    \label{ex:key:1408}
    \gll \MakeUppercase{A}   de  \textbf{hía}    ín    \textbf{sé}    e    de  nák  di  gitá    ɔ 
a    de  \textbf{hía}    ín    \textbf{lɛk}  \textbf{háw}    e    de  nák  di  gita. \\
\textsc{1sg.sbj}  \textsc{ipfv}  hear    \textsc{3sg.indp}  \textsc{quot}    \textsc{3sg.sbj}  \textsc{ipfv} hit  \textsc{def}  guitar  or
\textsc{1sg.sbj}  \textsc{ipfv}  hear    \textsc{3sg.indp}  like  how    \textsc{3sg.sbj}  \textsc{ipfv} hit  \textsc{def}  guitar\\

\glt 
\textit{Lit}. ‘I hear him that he’s playing the guitar.’ or ‘I hear him how he’s playing 
the guitar.’ [dj05ae 053]
\z

The quotative marker also introduces the complements of copula verbs in statements of facts. In such factive clauses, the copula verb takes a dummy noun\is{dummy nouns} like \textit{tín} ‘thing’, \textit{kés} ‘matter’, or the expletive subject\is{expletive} pronoun \textit{e} ‘\textsc{3sg.sbj}’. Factive main clauses like the one in \REF{ex:key:1409} are very common as introductory formulas in narrative discourse (cf. also \ref{ex:key:1136}): 


\ea%1409
    \label{ex:key:1409}
    \gll \textbf{E}    \textbf{dé}    \textbf{sé}    dán    gál    e    bin  de  kán    yá.\\
\textsc{3sg.sbj}  \textsc{be.loc}  \textsc{quot}    that    girl    \textsc{3sg.sbj}  \textsc{pst}  \textsc{ipfv}  come  here\\

\glt ‘It’s that/it came to pass that that girl used to come here.’ [ru03wt 019]
\z

Evaluative verbs like \textit{fáyn} ‘be fine’, \textit{gúd} ‘be good’, or \textit{bád} ‘be bad’ can induce either an indicative or a subjunctive mood over their complements. Evaluative verbs are followed by indicative complements when these are intended to convey factual information about present or past situations \REF{ex:key:1410}: 


\ea%1410
    \label{ex:key:1410}
    \gll E    \textbf{fáyn}  \textbf{sé}    e    kán    \textbf{yɛ́stadé}.\\
\textsc{3sg.sbj}  fine    \textsc{quot}    \textsc{3sg.sbj}  come  yesterday\\

\glt ‘It’s good that he came yesterday.’ [dj07ae 260]
\z

A subjunctive complement (albeit with the usual optional \textit{sé} ‘\textsc{quot}’) is required when the evaluative main verb refers to a potential situation \REF{ex:key:1411}. By expressing a preference, it harmonises with the deontic sense associated with the subjunctive mood in Pichi:


\ea%1411
    \label{ex:key:1411}
    \gll E    \textbf{fáyn}  sé    \textbf{mék}   e    kán    \textbf{tumɔ́ro}.\\
\textsc{3sg.sbj}  fine    \textsc{quot}    \textsc{sbjv}    \textsc{3sg.sbj}  come  tomorrow\\

\glt ‘It’s good for him to come tomorrow.’ [dj07ae 257]
\z

Interrogative complements of speech, cognition, and perception verbs are no different from headless, free relative clauses and are covered in \sectref{sec:10.6.5}.\is{quotative marker} 

\subsection{\textit{Wé} ‘\textsc{sub}’}

The multifunctional linker \textit{wé} ‘\textsc{sub’} is employed as a subordinator in relative clauses, an adverbial clause linker and a clausal coordinator. In a small minority of complement relations in the corpus, it is also used as a complementiser. 


The \textit{wé-}clause in \REF{ex:key:1412} is a borderline case that may either be analysed as an adverbial clause, i.e. a modifying time clause, or a subject complement clause:



\ea%1412
    \label{ex:key:1412}
    \gll E    dɔ́n  sté,    a    tínk    sé    e    dɔ́n  \textbf{sté}
\textbf{wé}  una  bin  gɛ́t  insecticida  yá.\\
\textsc{3sg.sbj}  \textsc{prf}  be.long  \textsc{1sg.sbj}  think  \textsc{quot}    \textsc{3sg.sbj}  \textsc{prf}  be.long
\textsc{sub}  \textsc{2pl}  \textsc{pst}  get  insecticide  here\\

\glt ‘It’s long ago, I think that it’s long ago that you people had insecticide here/
when you people last had insecticide here.’ [fr03wt 060]
\z

The same holds for the \textit{wé-}clause in \REF{ex:key:1413}, which can be interpreted as the complement clause of \textit{hía} ‘hear’ or an indirect interrogative clause, although the presence of a subsequent \textit{sé-}complement clause favours the latter interpretation: 


\ea%1413
    \label{ex:key:1413}
    \gll Yu  nó  \textbf{hía}  \textbf{wé}  a    tɛ́l  Mario  sé    quiero  cocinar?\\
\textsc{2sg}  \textsc{neg}  hear  \textsc{sub}  \textsc{1sg.sbj}  tell  \textsc{name}  \textsc{quot}    I.want  cook\\

\glt ‘You didn’t hear that I told Mario that I want to cook?’ or 
‘You didn’t hear when I told Mario that I want to cook?’ [ye03cd 124]
\z

In turn, the complement status of the \textit{wé-}clause in \REF{ex:key:1414} featuring the experiential main predicate \textit{sɔ́ri} ‘feel sorry’ is unequivocal. The functional equivalence of \textit{wé} and the general complementiser \textit{sé} in such complement clauses is illustrated by way of the analogous example in \REF{ex:key:1415}. However, the data contains no examples of \textit{wé}{}-complement clauses to speech verbs:


\ea%1414
    \label{ex:key:1414}
    \gll Mék    yú    nó  \textbf{fíl}  \textbf{sɔ́ri}    \textbf{wé}  a    nó  gí  yú    nó  nátín.\\
\textsc{sbjv}    \textsc{2sg.indp}  \textsc{neg}  feel  sorry  \textsc{sub}  \textsc{1sg.sbj}  \textsc{neg}  give  \textsc{2sg.indp}  \textsc{neg}  nothing\\

\glt ‘Don’t be disappointed that I didn’t give you anything.’ [to03gm 046]
\z


\ea%1415
    \label{ex:key:1415}
    \gll A    de  \textbf{fíl}  \textbf{sɔ́ri}    \textbf{sé}    e    de  kíl  di  fɔl.\\
\textsc{1sg.sbj}  \textsc{ipfv}  feel  sorry  \textsc{quot}    \textsc{3sg.sbj}  \textsc{ipfv}  kill  \textsc{def}  fowl\\

\glt ‘I feel sorry that she’s killing the fowl.’ [dj05ae 014]
\z

The following example involving \textit{wé} ‘\textsc{sub}’ is also a straightforward case of complementation involving an experiential main predicate:\is{complementisers!finite} 


\ea%1416
    \label{ex:key:1416}
    \gll \textbf{Tɛnk}  \textbf{gɔ́d}  \textbf{wé}  yu  dɔ́n  kán!\\
thank  God  \textsc{sub}  \textsc{2sg}  \textsc{prf}  come\\

\glt ‘Thank God that you have come!’\is{subordinator}
\z

\subsection{Complements of nouns}\label{sec:10.5.8}

The elements \textit{fɔ} ‘\textsc{prep}’ and \textit{sé} ‘\textsc{quot}’ may also introduce purposive complements of nouns and modify a head noun in a way very similar to a relative clause. Below, \textit{fɔ} introduces the non-finite (hence nominal) complement \textit{pás} ‘pass’ of the head noun \textit{sáy} ‘place’. The same function may be fulfilled by \textit{sé} ‘\textsc{quot}’. In the second half of the, the \textit{sé}{}-clause attributes a finite complement clause to the head noun \textit{sáy} ‘place’, and thereby, introduces a \textit{quasi} relative clause:


\ea%1417
    \label{ex:key:1417}
    \gll E    gɛ́t  ɔ́da    \textbf{sáy}    \textbf{fɔ}  pás,    bɔt  a    de  fɛ́n
di  \textbf{sáy}    \textbf{sé}    yu    nó  go  gɛ́t  hambɔ́g    fɔ  pípul  dɛn.\\
\textsc{3sg.sbj}  get  other  side    \textsc{prep}  pass    but  \textsc{1sg.sbj}  \textsc{ipfv}  {look for}  
\textsc{def}  side    \textsc{quot}   \textsc{2sg}    \textsc{neg}  \textsc{pot}  get  irritation  \textsc{prep}  people  \textsc{pl}\\

\glt ‘There is another place to pass (through), but I am looking for the place where you
wouldn’t be bothered by people.’ [ma03ni 009]
\z

In the first example below, a \textit{sé-}clause specifies the matter of the abstract noun \textit{fúlis} ‘foolishness’. In the second example, the anaphoric demonstrative pronominal \textit{dá wán} ‘that one’ is modified by a subjunctive\is{subjunctive mood} marked purpose\is{purpose clauses} clause introduced by \textit{sé} ‘\textsc{quot}’:


\ea%1418
    \label{ex:key:1418}
    \gll A    sé    bikɔs  una  Camerún,  una  gɛ́t  \textbf{di}  \textbf{fúlis}      \textbf{sé},
wé  náw    wé  yu  ték=an,    yu  go  sɛ́l=an.\\
\textsc{1sg.sbj}  \textsc{quot}    because  \textsc{2pl}  \textsc{place}    \textsc{2pl}  get  \textsc{def}  foolishness  \textsc{quot}
\textsc{sub}  now    \textsc{sub}  \textsc{2sg}  take=\textsc{3sg.obj}  \textsc{2sg}  \textsc{pot}  sell=\textsc{3sg.obj}\\

\glt ‘I say because you Cameroonians, you have the foolish habit that, when now, 
when you take it, you will sell it.’ [ab03ay 151]
\z


\ea%1419
    \label{ex:key:1419}
    \gll Yu  trowé=an,  yu  pút  ɔ́da    nyú  wán    ínsay,
\textbf{dá}  \textbf{wán}    \textbf{sé}    \textbf{mék}    e    nó  simɛ́l.\\
\textsc{2sg}  pour=\textsc{3sg.obj}  \textsc{2sg}  put  other  new  one    inside
that  one    \textsc{quot}    \textsc{sbjv}    \textsc{3sg.sbj}  \textsc{neg}  smell\is{relative clauses}\\

\glt ‘You pour it away, (then) you put another new one [water] inside, 
that (is) so that it does not smell.’ [dj03do 048]\is{complement clauses}
\z

\section{Relativisation}\label{sec:10.6}

In Pichi, subjects, objects, and PPs,\is{prepositional phrases} as well as possessor and possessed nouns may be relativised. The most common means of forming relative clauses involves the use of the morphologically invariant subordinator \textit{wé} ‘\textsc{sub}’ as a relative clause linker. Next to \textit{wé} ‘\textsc{sub}’, the linkers \textit{sé} ‘\textsc{quot}’ and \textit{fɔ} ‘\textsc{prep}’ marginally fulfil the function of relative clause linkers when they introduce noun complements (cf. \sectref{sec:10.5.8}).


In the second strategy of relative clause formation, no relative clause linker is employed and the relative clause simply follows the main clause. Hence, there is a “gap” between the two clauses. However, resumptive pronoun\is{resumptive pronouns}s may optionally refer back to the relativised head noun in most types of relative clauses. Aside from that, restrictive and non-restrictive relative clauses are not systematically distinguished on formal grounds. 


The use of resumptive pronouns is nearly general in subject relative clauses with [+specific] head nouns, fairly common in object relative clauses, and rare in the relativisation of PPs\is{prepositional phrases}. The frequency of resumptive pronouns with subject relative clauses runs counter to the predictions of the relativisation accessibility\is{relativisation accessibility} hierarchy (cf. \citealt{KeenanComrie1977}), and it should be worthwhile investigating whether it constitutes an areal West African phenomenon (see, however, a similar distribution of resumptive pronouns in Tok Pisin (\citealt{SankoffBrown1976}) and popular Brazilian Portuguese \citep{Tarallo1983}.


In the example sentences in this section, relative clauses are set in squared brackets. \tabref{tab:key:10.3} summarises important features of the different types of relative clauses that Pichi has (RC = relative clause).


%%please move \begin{table} just above \begin{tabular
\begin{table}
\caption{Features of relative clauses}
\label{tab:key:10.3}

\begin{tabularx}{\textwidth}{Qllll}
\lsptoprule
Feature & Subject RC & Object RC & PP RC & Possessor RC\\
\midrule 
Are “gap” RCs attested? & No & Yes & Yes & No\\
\tablevspace 
Are resumptive pronouns found in relativised position? & Yes & Yes & Yes & n.a.\\
\tablevspace 
Are free relative clauses attested? & Yes & Yes & Yes & No\\
\tablevspace 
Is stranding\is{stranding} of prepositions attested? & n.a. & n.a. & Yes & n.a.\\
\tablevspace 
Is pied-piping of prepositions attested? & n.a. & n.a. & No & n.a.\\
\lspbottomrule
\end{tabularx}
\end{table}
\subsection{General characteristics}\label{sec:10.6.1}

The linker \textit{wé} ‘\textsc{sub}’ introduces relative clauses as well as adverbial and coordinate clauses. Since the use of resumptive subject pronouns is very common (but still optional) in subject relative clauses with [+specific] head nouns (cf. \tabref{tab:key:10.3} above), some subject relative clauses may therefore have the same constituent order as an adverbial clause introduced by \textit{wé} ‘\textsc{sub’}. Consider the alternative relative and adverbial translations I provide for sentence \REF{ex:key:1420}:


\ea%1420
    \label{ex:key:1420}
    \gll Ɔ́l  dí  mán    dɛn  [\textbf{wé}  \textbf{dɛn}  gɛ́t  mɔní],  na  di  tín
wé  dɛn  de  mék.\\
all  this  man    \textsc{pl}   \phantom{[}\textsc{sub}  \textsc{3pl}  get  money    \textsc{foc}  \textsc{def}  thing
\textsc{sub}  \textsc{3pl}  \textsc{ipfv}  make\\

\glt ‘All these men who have money, that’s what they do.’ or 
‘All these men, when they have money, that’s what they do.’ [ed03sb 133]
\z

However, the meaning of the sentence above is not as ambiguous as it may appear. Relative clauses are never separated from their main clauses by a prosodic break; relative constructions form single prosodic units. In contrast, adverbial clauses are very often separated from their main clauses by a prosodic break: The main clause bears continuative intonation,\is{continuative intonation} and the subordinate clause is separated from the main clause by a pause. The adverbial clause then begins with the high pitch onset that is characteristic for independent utterances (cf. also \sectref{sec:3.4.4}). An adverbial interpretation of the clause introduced by \textit{wé} ‘\textsc{sub’} in \REF{ex:key:1420} above would therefore only be possible if a comma were inserted between \textit{mán dɛn} ‘men’ and \textit{wé} ‘\textsc{sub’}. 


In contrast, pronoun resumption, even if possible, is not very often seen in object relative clauses, even if the head noun is [+specific]. In the object relative clause below, \textit{gɛ́t} ‘get’ is not followed by an object pronoun co-referential with the head noun \textit{mɔní}:



\ea%1421
    \label{ex:key:1421}
    \gll Mék    e    bák      yú    di  mɔní  [\textbf{wé}  e    gɛ́t].\\
\textsc{sbjv}    \textsc{3sg.sbj}  give.back  \textsc{2sg.indp}  \textsc{def}  money  \phantom{[}\textsc{sub}  \textsc{3sg.sbj}  get\\

\glt ‘Let him give you back the money that he got.’ [fr03cd 027]
\z

The possibility of abstaining from pronoun resumption in Pichi relative clauses, such as \REF{ex:key:1421} (for a subject relative clause without a resumptive pronoun\is{resumptive pronouns}, cf. \ref{ex:key:1432} below) and the prosodic unity of relative constructions are good arguments for viewing relative clauses as embedded clauses.


Relative clauses always follow the head \textsc{NP} that they refer to. The head \textsc{NP} and its relative clause can be separated by quantifier\is{quantifiers}s \REF{ex:key:1422}, as well as topic and focus particles \REF{ex:key:1423}. The examples in this section and the following ones also show that TMA and person marking in relative clauses is “balanced” \citep{Stassen1985}; hence it is not reduced in comparison with that of declarative clauses:



\ea%1422
    \label{ex:key:1422}
    \gll Somos  \textbf{tú}  \textbf{dásɔl}  [wé  wi  dé    láyf]    \op...\cp{}\\
we.are  two  only     \textsc{sub}  \textsc{1pl}  \textsc{be.loc}  life\\

\glt ‘We are, (it’s) only two of us that are alive (...)’ [ab03ay 133]
\z


\ea%1423
    \label{ex:key:1423}
    \gll Sɔn    dé    yét  \textbf{sɛ́f}  [wé  a    nó  mék].\\
some  \textsc{be.loc}  yet  \textsc{emp}   \textsc{sub}  \textsc{1sg.sbj}  \textsc{neg}  make\\

\glt ‘Some is actually still left that I haven’t made.’ [dj03do 009]
\z

Headed restrictive and non-restrictive relative clauses cannot be distinguished on formal grounds. In \REF{ex:key:1424}, the commas in squared brackets in the translation indicate the non-restrictive alternative interpretation of the sentence. Note the presence of the English loan \textit{apart from} in this example: 


\ea%1424
    \label{ex:key:1424}
    \gll Apart  from  mi    antí    [wé  e    dé    yá],    ɔ  di  pikín
dɛn  fɔ  mi    gran-má    wet    mi    gran-pá    [wé  dɛn  stíl  dé
láyf],  dɛn-ɔ́l      dɛn  dé    na  Panyá.\\
apart  from  \textsc{1sg.poss}  aunt     \textsc{sub}  \textsc{3sg.sbj}  \textsc{be.loc}  here    or  \textsc{def}  child
\textsc{pl}  \textsc{prep}  \textsc{1sg.poss}  grand-ma  with    \textsc{1sg.poss}  grand-pa     \textsc{sub}  \textsc{3pl}  still  \textsc{be.loc}
life     \textsc{3pl.indp.cpd-}all  \textsc{3pl}  \textsc{be.loc}  \textsc{foc}  Spain\\

\glt ‘Apart from my aunt [,] who is here, or the children of my grandmother and grandfather
[,] who are still alive, they are all in Spain.’ [fr03ft 038]
\z

If the head noun has plural reference, the pluraliser\is{pluraliser} \textit{dɛn} ‘\textsc{pl}’ appears immediately after the head noun and before the subordinator \textit{wé} ‘\textsc{sub}’ \REF{ex:key:1425}. Note the presence of the resumptive subject pronoun \textit{dɛn} ‘\textsc{3pl}’ in the relative clause, which is co-referential with the head noun \textit{tín dɛn} ‘things’: 


\ea%1425
    \label{ex:key:1425}
    \gll Porque  dán    tín    na  tín    \textbf{dɛn}    [wé  \textbf{dɛn}  dɔ́n  sté  
dán  tɛ́n    dɛn  wé  esclavitud  dé].\\
because  that    thing  \textsc{foc}  thing  \textsc{pl}     \textsc{sub}  \textsc{3pl}  \textsc{prf}  stay  
that  time    \textsc{pl}  \textsc{sub}  slavery    \textsc{be.loc}\\
\glt ‘Because those are things that have stayed (from) those times when there
was slavery.’ [hi03cb 228]
\z

Pichi exhibits generous possibilities of relative clause formation (cf. \citealt[148]{Keenan1985}). For example, the relativisation of a nominal that is part of a coordinate structure is permitted \REF{ex:key:1426}. Equally, a relative clause may contain a focused resumptive pronoun\is{resumptive pronouns} \REF{ex:key:1427}:


\ea%1426
    \label{ex:key:1426}
    \gll Bikɔs  mí    dé    sɔn    stáyl,  layk  \textbf{dán}  \textbf{gɛ́l}  [wé  mí    \textbf{wet=an}
bin  gó  dé],    a    tɛ́l=an  sé  a    wɔ́nt  sí  háw  dɛn  de  mék.\\
because  \textsc{1sg.indp}  \textsc{be.loc}  some  style  like  that  girl   \textsc{sub}  \textsc{1sg.indp}  with\textsc{=3sg.obj}
\textsc{pst}  go  there  \textsc{1sg.sbj}  tell=\textsc{3sg.obj}  {} \textsc{1sg.sbj}  want  see  how  \textsc{3pl}  \textsc{ipfv}  make\\
\glt ‘Because I was (feeling) a way, like that girl with whom I went there, I told her that
I wanted to see how it is done.’ [ed03sb 149]
\z


\ea%1427
    \label{ex:key:1427}
    \gll Bɔt  di  pé  wé  yu  gɛ́fɔ    pé,  if  yu  nó  de    gí  mí    yu
fɔ́s    \textbf{mán}  [wé  \textbf{na}  \textbf{in}    gí    yú    dí  bɛlɛ́],  yu  de  gí
mi    di  pikín  [wé  de  kɔmɔ́t].\\
but  \textsc{def}  pay  \textsc{sub}  \textsc{2sg}  have.to  pay  if  \textsc{2sg}  \textsc{neg}  \textsc{ipfv}    give  \textsc{1sg.indp}  \textsc{2sg}
first    man     \textsc{sub}  \textsc{foc}  \textsc{3sg.indp}  give    \textsc{2sg.indp}  this  belly  \textsc{2sg}  \textsc{ipfv}  give
\textsc{1sg.indp}  \textsc{def}  child   \textsc{sub}  \textsc{ipfv}  come.out\\
\glt 
\textit{Lit}. ‘But the price that you have to pay (is), if you don’t give me your first man,
who it is him who gave you the first pregnancy, you will give me the child that
will come out.’ [ed03sb 020]
\z

Likewise, there is no restriction on the relativisation of the subject or object of a complement clause \REF{ex:key:1428} or of an indirect question clause \REF{ex:key:1429}:


\ea%1428
    \label{ex:key:1428}
    \gll Na  dán  \textbf{bɔ́y}  [wé  a    tɛ́l  yú    \textbf{sé}    in    mamá  dɔ́n
gó  na  Panyá].\\
\textsc{foc}  that  boy   \textsc{sub}  \textsc{1sg.sbj}  tell  \textsc{2sg.indp}  \textsc{quot}    \textsc{3sg.poss}  mother  \textsc{prf}
go  \textsc{loc}  Spain\\

\glt ‘It is that boy (of) who I told that his mother has gone to Spain.’ [he07fn 253]
\z


\ea%1429
    \label{ex:key:1429}
    \gll A    gɛ́t  sɔn    kɔ́mpin,  sɔn    \textbf{Ghana-mán}    [wé  a    nó  sabí
\textbf{ús=sáy}  dán  mán    dé].\\
\textsc{1sg.sbj}  get  some  friend  some  Ghana\textsc{.cpd}{}-man   \textsc{sub}  \textsc{1sg.sbj}  \textsc{neg}  know
\textsc{q}=side  that  man    \textsc{be.loc}\\

\glt \textit{Lit}. ‘I have a friend, a Ghanaian who I don’t know where that man is.’ [ed03sb 188]\is{relativisation accessibility}
\z

Relative constructions are also made use of to express adverbial relations of time, location, and manner through the relativisation of generic nouns{\fff} like \textit{áwa} ‘time, hour’, \textit{tɛ́n} ‘time’ \REF{ex:key:1473}, \textit{dé} ‘day’ \REF{ex:key:1474} and \textit{stáyl} ‘manner, style (\sectref{sec:10.7.4}).

\subsection{Subjects and objects} \label{sec:10.6.2}

Subject relative clauses normally feature a resumptive subject pronoun that is co-referential with the [+specific] relativised noun. Relative clauses featuring a relativised subject pronoun also usually contain a resumptive pronoun if the head \textsc{NP} is not a 3\textsuperscript{rd} person pronoun \REF{ex:key:1431}:


\ea%1430
    \label{ex:key:1430}
    \gll Ɛf  yu  chɔ́p  ɔ́l  \textbf{dís}  \textbf{chɔ́p}  [wé  \textbf{e}    nó  dɔ́n],  tumɔ́ro
yu  go  sík.\\
if  \textsc{2sg}  eat    all  this  food    \textsc{sub}  \textsc{3sg.sbj}  \textsc{neg}  done  tomorrow
\textsc{2sg}  \textsc{pot}  be.sick\\

\glt ‘If you eat all this food that is not done you’ll be sick tomorrow.’ [ro05ee 045]
\z


\ea%1431
    \label{ex:key:1431}
\gll
\textbf{Mí}    na  wán    húman  [wé  \textbf{a}   síryɔs].\\
\textsc{1sg.indp}  \textsc{foc}  one    woman   \textsc{sub}  \textsc{1sg.sbj}  be.serious\\
\glt ‘I \textsc{[emp]} am a woman who is serious.’  [ro05ee 017]
\z

Sentence \REF{ex:key:1431} exemplifies the relativisation of subject \textsc{NPs} without resumptive pronominal marking. Although the head nouns \textit{gabonés} and \textit{guineano} are preceded by the definite article \textit{di} ‘\textsc{def’}, these nouns have [-specific], generic reference, hence they are not reiterated by a resumptive subject pronoun in the relative clause (cf. also \sectref{sec:5.1.4}): 


\ea%1432
    \label{ex:key:1432}
    \gll Pero    \textbf{di}  \textbf{gabonés}    [wé  de  tɔ́k  Bata]  wet    \textbf{di}  \textbf{guineano}  
[wé  de  tɔ́k   Bata],  di  sonido  nó  dé    di  sén.\\
but    \textsc{def}  Gabonese   \textsc{sub}  \textsc{ipfv}  talk  Fang  with    \textsc{def}  Guinean    
 \textsc{sub}  \textsc{ipfv}  talk Fang    \textsc{def}  sound  \textsc{neg}  \textsc{be.loc}  \textsc{def}  same\\

\glt ‘But the Gabonese who talks Fang and the Guinean who talks Fang, 
the sound is not the same.’ [ma03hm 048]
\z

“Gap” subject relative constructions without the subordinator are not attested. However, object relative clauses formed by means of the gap strategy are sometimes heard. The relativised cognate object\is{cognate objects} in \REF{ex:key:1433} is a patient\is{patient} object. Note the absence of the subordinator \textit{wé} ‘\textsc{sub}’ as well as that of a resumptive object pronoun in the relative clause after the verb \textit{wánt} ‘want’:


\ea%1433
    \label{ex:key:1433}
    \gll Mék    e    bít    yú,    mék    e    dú  yú  di  \textbf{dú}
[e    \textbf{wánt}]  \op...\cp{}\\
\textsc{sbjv}    \textsc{3sg.sbj}  beat    \textsc{2sg.indp}  \textsc{sbjv}    \textsc{3sg.sbj}  do  \textsc{2sg}  \textsc{def}  do
\phantom{[}\textsc{3sg.sbj}  want \\

\glt ‘Let him beat you, let him do to you [what he wants] (...)’ [bo03cb 135]\is{subjects}
\z

Object relative clauses involve the use of the subordinator \textit{wé} ‘\textsc{sub}’ in the vast majority of cases. Take note of the absence of a resumptive object pronoun with reference to the non-specific head noun \textit{bloques dɛn} ‘blocks’: 


\ea%1434
    \label{ex:key:1434}
    \gll Sɔn    bloques  dɛn  lɛ́f    [\textbf{wé}  dɛn  gɛ́fɔ    monta]  nɔ́?\\
some  block.\textsc{pl}   \textsc{pl}  remain   \phantom{[}\textsc{sub}  \textsc{3pl}  have.to  mount  \textsc{intj}\\

\glt ‘Some blocks remain that have to be mounted, right?’ [ye03cd 114]
\z

A resumptive pronoun may also refer to a recipient\is{recipient} head noun in a double-object construction \REF{ex:key:1435}. Recipient resumptive pronouns are optional and may therefore be omitted as in \REF{ex:key:1436}:


\ea%1435
    \label{ex:key:1435}
    \gll Yu  sí  dán  \textbf{pikín}  dé    [wé  in    mamá  de  gí=\textbf{an}    chɔ́p]?\\
\textsc{2sg}  see  that  child  there   \phantom{[}\textsc{sub}  \textsc{3sg.poss}  mother  \textsc{ipfv}  give=\textsc{3sg.obj}  food\\

\glt ‘Have you seen that child whose mother is giving her food?’ [li07fn 455]
\z


\ea%1436
    \label{ex:key:1436}
    \gll \MakeUppercase{A}   bin  sí  di  \textbf{pikín}  [wé  di  húman  bin  gí  chɔ́p  na  strít].\\
\textsc{1sg.sbj}  \textsc{pst}  see  \textsc{def}  child   \phantom{[}\textsc{sub}  \textsc{def}  woman  \textsc{pst}  give  food    \textsc{loc}  street\\

\glt ‘I saw the child that the woman gave food to in the street.’ [dj05ae 065]\is{resumptive pronouns}
\z

\subsection{Prepositional phrases}

There are no formal constraints on the relativisation of PPs\is{prepositional phrases}. However, this type of relativisation is rather rare compared to that of subjects and objects. The following relative constructions involve relativised prepositional phrases introduced by the prepositions \textit{fɔ} ‘\textsc{prep}’ and \textit{pan} ‘on’. These two prepositions, as well as the preposition \textit{wet} ‘with’, can also be stranded\is{stranding}, in other words they may remain in their original position, while the relativised NP appears at the beginning of the sentence. Pied-piping of prepositions, i.e. the appearance of the preposition at the beginning of the relative clause, is not attested:


\ea%1437
    \label{ex:key:1437}
    \gll Di  \textbf{béd}  [\textbf{wé}  e    de  slíp    \textbf{pan}],  e    dé    na  di  rúm.\\
\textsc{def} bed \phantom{[}\textsc{sub} \textsc{3sg.sbj}  \textsc{ipfv} sleep  on  \textsc{3sg.sbj}  \textsc{be.loc}  \textsc{loc}  \textsc{def}  room\\

\glt ‘The bed that she sleeps on, it’s in the room.’ [tr05fn 047]
\z

In the more common alternative to stranding, a resumptive pronoun\is{resumptive pronouns} fills the original position of the relativised noun. Compare \textit{wet=an} ‘with her’ in \REF{ex:key:1426} above. Alternatively, a resumptive pronoun need not be used at all. The exact meaning of the sentence is then provided by pragmatic context. In such instances of “prepositional phrase chopping” (\citealt{Tarallo1983,Tarallo1985}) disambiguation is left to pragmatic context. 


In \REF{ex:key:1438}, there is no \textit{wet} ‘with’ in the relative clause to point to the semantic role of instrument\is{instrument} of the relativised head noun \textit{gɔ́n} ‘gun’:



\ea%1438
    \label{ex:key:1438}
    \gll Dɛn  de  gó  wet    dán    \textbf{gɔ́n}  [\textbf{wé}  dɛn  de  kíl  bíf]      ɔ  pistola.\\
\textsc{3pl}  \textsc{ipfv}  go  with    that    gun   \textsc{sub}  \textsc{3pl}  \textsc{ipfv}  kill  wild.animal  or  pistol\\

\glt ‘They go with that gun which they kill wild animals (with) or a pistol.’ [ed03sb 114]
\z

Similarly, the \textit{wé-}clause in \REF{ex:key:1439} induces a locative, that in \REF{ex:key:1440b} an instrumental interpretation. It is also of interest that \REF{ex:key:1440b} is an example for the use of \textit{fɔ} as an introducer of a noun complement that is very similar in function to the preceding relative clause (cf. \sectref{sec:10.5.8}): 


\ea%1439
    \label{ex:key:1439}
    \gll A    kán  kɔmɔ́t  na  dán  \textbf{hós}    [\textbf{wé}    a    bin  dé].\\
\textsc{1sg.sbj}  \textsc{pfv}  go.out  \textsc{loc}  that  house   \textsc{sub}    \textsc{1sg.sbj}  \textsc{pst}  \textsc{be.loc}\\

\glt ‘I left that house which I had been (in).’ [ab03ay 097]
\z


\ea%1440
    \label{ex:key:1440}
\ea{\gll
Yu  nó  nó    na  ús=tín,  matapenso?\\
  \textsc{2sg}  \textsc{neg}  know  \textsc{foc}  \textsc{q}=thing  pestle\\

\glt   ‘You don’t know what it is, a pestle?’ [ye05ce 098]
}\ex{\label{ex:key:1440b}
\gll 
Dán  tín    [\textbf{wé}  dɛn  de  mék    súp],  \textbf{fɔ}  mék    fufú.\\
  that  thing   \textsc{sub}  \textsc{3pl}  \textsc{ipfv}  make  soup  \textsc{prep}  make  fufu\\
\glt   ‘That thing they make soup (with), in order to make fufu (with).’ [dj05ce 099]
}\z\z

In a similar vein, the \textit{wé-}clauses in \REF{ex:key:1441} and \REF{ex:key:1442} allow that a causal meaning is inferred:


\ea%1441
    \label{ex:key:1441}
    \gll So  na  \textbf{di}  \textbf{tín}    [\textbf{wé}  e    rɔ́n],  e  kɔmɔ́t.\\
so  \textsc{foc}  \textsc{def}  thing   \phantom{[}\textsc{sub}  \textsc{3sg.sbj}  run  \textsc{3sg.sbj}  go.out\\

\glt ‘So that is why [\textit{lit}. the thing that] she fled, (and) she left.’ [ed03sb 041]
\z


\ea%1442
    \label{ex:key:1442}
    \gll \op...\cp{}  e    go  sé    e    de  fíɛ  e    nó  go  gí  mí
\textbf{di}  \textbf{tín}    [\textbf{wé}    a    de  sɛ́n=an].\\
{}  \textsc{3sg.sbj}  \textsc{pot}  \textsc{quot}    \textsc{3sg.sbj}  \textsc{ipfv}  fear  \textsc{3sg.sbj}  \textsc{neg}  \textsc{pot}  give  \textsc{1sg.indp}
\textsc{def}  thing   \phantom{[}\textsc{sub}    \textsc{1sg.sbj}  \textsc{ipfv}  send=\textsc{3sg.obj}\\

\glt ‘(...) he would say he is afraid, he would not give me the thing that I had sent 
him (for).’ [ab03ab 041]
\z

Such constructions are structurally no different from those involving objects, and, like the latter, they may involve “gap” constructions. Note the absence of the subordinator \textit{wé} ‘\textsc{sub}’ in the following example. The head noun of the relative clause \textit{sáy} ‘side, place’ is the syntactic object of \textit{sidɔ́n} ‘sit (down), stay’: 


\ea%1443
    \label{ex:key:1443}
    \gll A    de  gó  nía    \textbf{di}  \textbf{sáy}  [Paquita  \textbf{sidɔ́n}].\\
\textsc{1sg.sbj}  \textsc{ipfv}  go  near    \textsc{def}  side   \phantom{[}\textsc{name}  stay\\

\glt ‘I am going near where Paquita stays.’ [dj05be 147]
\z

Prepositional phrase chopping should be differentiated from instances in which the goal\is{goal} of a verb may be expressed as an object, as is the case in double-object constructions involving \textit{pút} ‘put’ in \REF{ex:key:1444} (cf. \sectref{sec:9.3.4} for more details). Once more, note the occurrence of a “gap” relative clause in this example: 


\ea%1444
    \label{ex:key:1444}
    \gll A    ték    tú  peso    a    báy  dán    dís-tín,  
\textbf{sɔn}    \textbf{smɔ́l}  \textbf{pépa}  [\textstylePichiexamplenumberZchnZchn{dɛn}    de  \textbf{pút}=\textbf{an}    cacahuete].\\
\textsc{1sg.sbj}  take    two  peso    \textsc{1sg.sbj}  buy  that    this-thing
some  small  paper   \phantom{[}\textsc{3pl}    \textsc{ipfv}  put=\textsc{3sg.obj}  groundnut\\

\glt ‘I took two pesos (and) I bought this whatsit, a small paper 
(into which) groundnuts are put.’ [ed03sp 083]
\z

Example \REF{ex:key:1445} shows how the resumption of the entire relativised noun in the position of relativisation can be an alternative to stranding\is{stranding} or chopping. Anaphoric \textsc{NP} reiteration is accompanied by a deictic element, the demonstrative \textit{dís} ‘this’ in \REF{ex:key:1445}:


\ea%1445
    \label{ex:key:1445}
    \gll Bikɔs  \textbf{wán}    \textbf{isla} dé    [wé  e    fíba    sé
petroleo  dé    \textbf{na}  \textbf{dís}  \textbf{isla}].\\
because  one    island  \textsc{be.loc}   \phantom{[}\textsc{sub}  \textsc{3sg.sbj}  seem  \textsc{quot}
oil    \textsc{be.loc}  \textsc{loc}  this  island\\
\glt ‘Because there is an island of which it seems that there is oil 
on this island.’ [fr03ft 109]
\z

In sentence \REF{ex:key:1446} below, the direct object \textit{sɔn fáyn} ‘a beauty’ is resumed through another full NP, namely the demonstrative pronominal \textit{dá wán} ‘that (one)’:


\ea%1446
    \label{ex:key:1446}
    \gll A    sé    blák    gɛ́l  dɛn  gɛ́t  \textbf{sɔn}    \textbf{fáyn}
[wé  wáyt  húman  dɛn nó  gɛ́t  \textbf{dá}  \textbf{wán}].\\
\textsc{1sg.sbj}  \textsc{quot}    black  girl  \textsc{pl}  get  some  fine  
 \phantom{[}\textsc{sub}  white  woman  \textsc{pl}   \textsc{neg}  get  that  one\\

\glt ‘I say black girls have a beauty which white women do not 
have (that one).’ [ed03sp 046]
\z

Full \textsc{NP} anaphora can also be observed in the complex relative construction in \REF{ex:key:1429} above, where \textit{dán mán} ‘that man’ in the relative clause refers to the head noun \textit{Ghana-mán} ‘Ghanaian’. All these structures are reminiscent of correlative constructions found in other languages and demonstrate the diversity of relativisation strategies in Pichi.

\subsection{Possessors}

When a possessor noun is relativised, a co-referential possessive pronoun and the possessed noun immediately follow the subordinator \textit{wé} ‘\textsc{sub}’ \REF{ex:key:1447}:


\ea%1447
    \label{ex:key:1447}
\gll
\op...\cp{}  dɛn  de  kɔmɔ́t    na  wán    \textbf{pueblo}  [wé    \textbf{in}    \textbf{ném}  na
{Basakato   dé la Sagrada Familia}].\\
{} \textsc{3pl}  \textsc{ipfv}  hail.from  \textsc{loc}  one    village   \phantom{[}\textsc{sub}    \textsc{3sg.poss}  name  \textsc{foc}
\textsc{place}\\

\glt ‘(...) they come from a village whose name is Basakato dé la Sagrada 
Familia.’\textstylePichiglossZchn{ [fr03ft 042]}
\z

The preceding example features a possessor head noun that functions as the subject of the relative clause. When the possessor head noun functions as the object of the relative clause, it is relativised by way of a structure in which the head noun and the relative clause function as the topic. The remainder of the main clause functions as the comment, and is set off from the topic by a pause, while a possessive pronoun anaphorically refers to the head noun \REF{ex:key:1448}:


\ea%1448
    \label{ex:key:1448}
    \gll Dán  \textbf{húman}  [wé  a    só    yú],    \textbf{in}      \textbf{motó}  dé    na  strít.\\
that  woman   \phantom{[}\textsc{sub}  \textsc{1sg.sbj}  show  \textsc{2sg.indp}  \textsc{3sg.poss}    car    \textsc{be.loc}  \textsc{loc}  street\\

\glt ‘That woman which I showed you, her car is in the street.’ [dj05ae 068]
\z

Possessed nouns are relativised like core participant\is{core participants}s. Reference is upheld due to the juxtaposition of the possessed noun and the relative clause \REF{ex:key:1449}:


\ea%1449
    \label{ex:key:1449}
    \gll \MakeUppercase{A}   ték    di  stík    \textbf{in}    \textbf{kandá}  [wé  a    sí  dé],
a    rós=an.\\
\textsc{1sg.sbj} take \textsc{def} tree    \textsc{3sg.poss} bark  \phantom{[}\textsc{sub}  \textsc{1sg.sbj} see there 
\textsc{1sg.sbj}  burn=\textsc{3sg.obj}\\

\glt ‘I took the bark\textsubscript{i} of the tree\textsubscript{j} that\textsubscript{i} saw there, I burnt it\textsubscript{i}.’ [bo05n 001]\is{relative clauses}
\z

\subsection{Free relatives and indirect questions}\label{sec:10.6.5}

Free relative clauses\is{headless relative clauses} do not feature an overt head noun and are introduced by a question word\is{question words}. In free relative constructions featuring question words, the relative clause is formally identical with the corresponding content question\is{indirect questions} (cf. \sectref{sec:7.3.2}). The subordinator \textit{wé} ‘\textsc{sub’} is not employed to introduce free relative clauses. Free relative clauses often function as objects\is{objects} of verbs of cognition, perception, asking, or speaking. 


Below, we find a free subject relative clause, which is introduced by the question word \textit{wétin} ‘what’:



\ea%1450
    \label{ex:key:1450}
    \gll A    dɔ́n    tɛ́l  yú    [\textbf{wétin}  pás    na  nɛ́t],    dán    nɛ́t.\\
\textsc{1sg.sbj}  \textsc{prf}    tell  \textsc{2sg.indp}   \phantom{[}what  pass    \textsc{loc}  night  that    night\\

\glt ‘I’ve already told you what happened in the night, that night.’ [ab03ab 043]
\z

Free relatives introduced by the question words \textit{údat} ‘who’, \textit{ús=mán} ‘who’, and \textit{ús=pɔ́sin} ‘who’ question human referents. The following two examples are free object relative clauses:


\ea%1451
    \label{ex:key:1451}
    \gll Dɛn  nó  nó    [\textbf{údat}  hambɔ́g=an].\\
\textsc{3pl}  \textsc{neg}  know   \phantom{[}who  bother=\textsc{3sg.obj}\\

\glt ‘They don’t know who disturbed her.’ [dj05ce 127]
\z


\ea%1452
    \label{ex:key:1452}
    \gll Mí    nó  sabí    [\textbf{ús=mán}  dɛn  kíl],    a    nɔ́ba  hía
dán    torí    sɛ́f.\\
\textsc{1sg.indp}  \textsc{neg}  know   \phantom{[}\textsc{q}=man  \textsc{3pl}  kill    \textsc{1sg.sbj}  \textsc{neg}.\textsc{prf}  hear
that    story  \textsc{emp}\\

\glt ‘I don’t know which man they killed, I haven’t even heard that 
story.’ [ro05de 049]
\z

The corresponding question words also introduce the free variants of relative clauses with generic head nouns like \textit{tɛ́n} ‘time’ and \textit{sáy} ‘side’ which function as adverbial clauses of time and place. Compare \REF{ex:key:1453}. 


\ea%1453
    \label{ex:key:1453}
    \gll E    nɛ́a    tɛ́l  mí    [\textbf{ús=tɛ́n}  e    go  rích    dé].\\
\textsc{3sg.sbj}  \textsc{neg}.\textsc{prf}  tell  \textsc{1sg.indp}   \phantom{[}\textsc{q}=time  \textsc{3sg.sbj}  \textsc{pot}  arrive  there\\

\glt ‘He hasn’t told me when he is going to arrive there.’ [eb07fn 582]
\z

The question word \textit{háw} ‘how’ introduces free relatives and indirect questions that question a property \REF{ex:key:1454}, quantity, or degree; the latter two in the collocation \textit{háw mɔ́ch} ‘how much’ \REF{ex:key:1455}:


\ea%1454
    \label{ex:key:1454}
    \gll Bɔt  mí    wánt  sabí    [\textbf{háw}  dán    tín    dé].\\
but  \textsc{1sg.indp}  want  know   \phantom{[}how  that    thing  \textsc{be.loc}\\

\glt ‘But I wanted to know how that thing is.’ [ed03sb 147]
\z


\ea%1455
    \label{ex:key:1455}
    \gll Mí    nó  áks=an    [\textbf{háw}  \textbf{mɔ́ch}  e    wɔ́nt].\\
\textsc{1sg.indp}  \textsc{neg}  ask=\textsc{3sg.obj}   \phantom{[}how  much  \textsc{3sg.sbj}  want\\

\glt ‘I [\textsc{emp}] didn’t ask him how much he wants.’ [lo07fn 068]
\z

Indirect yes-no question clauses may be introduced by the clause linker \textit{ɛf(ɛ)} or \textit{if} ‘if’ which then functions as a complementiser in combination with sentence-final question intonation. Alternatively, such question clauses may be introduced by \textit{sé} ‘\textsc{quot}’ if phrased as a question in the type of direct speech\is{direct speech} that characterises the use of quotative \textit{sé} ‘\textsc{quot}’ in many contexts:


\ea%1456
    \label{ex:key:1456}
    \gll Sé    yu  wánt  \textbf{sabí}    \textbf{ɛf}  rén    de  fɔ́l,  nɔ́?\\
\textsc{quot}    \textsc{2sg}  want  know  if  rain    \textsc{ipfv}  fall  \textsc{intj}\\

\glt ‘(You) say you want to know if the rain is falling, right?’ [dj07ae 236]
\z


\ea%1457
    \label{ex:key:1457}
    \gll Mí    sɛ́f,  ɔ́l  pɔ́sin  dɛn  kin  \textbf{áks}  mí    \textbf{sé}
yu  dɔ́n  bɔ́n?\\
\textsc{1sg.indp}  \textsc{emp}  all  person  \textsc{3pl}  \textsc{hab}  ask  \textsc{1sg.indp}  \textsc{quot}
\textsc{2sg}  \textsc{prf}  give.birth\\

\glt ‘Even me, everybody usually asks me “have you given 
birth”?’ [fr03ft 144]\is{free relative clauses}
\z

\section{Adverbial relations}\label{sec:10.7}

The clause linkers \textit{wé} ‘\textsc{sub}’ and \textit{sé} \textsc{‘quot’} together have the potential to participate in the expression of most types of adverbial relations that we find in Pichi. Additionally, Pichi features an array of adverbial clause linkers with more specific meanings. These are summarised in \tabref{tab:key:10.4} below. The following sections provide an overview of adverbial clause formation in Pichi. Purpose clauses are covered in \sectref{sec:10.7.6}.


The first column in \tabref{tab:key:10.4} below provides an overview of the types of adverbial clauses attested. The second column contains the linkers that introduce these types of clauses in Pichi. Alternative means of formation are given in the remaining three columns: The third column indicates whether a clause introduced by \textit{wé} ‘\textsc{sub}’ or \textit{sé} ‘\textsc{quot}’ can be used instead of the linker in the second column in order to express the same adverbial relation. 



The fourth column provides other alternatives for expressing the corresponding adverbial relation. Independent sentences may also be linked through adverbials. These are contained in the last column on the right. A blank space indicates that the corresponding means is not available. 


%%please move \begin{table} just above \begin{tabular
\begin{sidewaystable}
\caption{Adverbial relations}
\label{tab:key:10.4}

\begin{tabularx}{\textwidth}{lQQQQ}
\lsptoprule

Clause type & Clause linkers & Linkage with \textstyleTablePichiZchn{wé} or \textstyleTablePichiZchn{sé} alone? & Other means of linkage? & Linkage by adverbial?\\
\midrule 

Time & \textstyleTablePichiZchn{bifó} ‘before’ 

 \textstyleTablePichiZchn{lɛk háw} ‘as soon as’ & \textstyleTablePichiZchn{wé} ‘\textsc{sub}’ & \textstyleTablePichiZchn{di tɛ́n wé} 

 ‘the time that’ & \textstyleTablePichiZchn{áfta} ‘then’, \textstyleTablePichiZchn{dasɔ́l} ‘then’, \textstyleTablePichiZchn{dán tɛ́n} ‘that time’, \textstyleTablePichiZchn{na ín/na dé} ‘then’\\
Manner & \textstyleTablePichiZchn{lɛk háw} ‘the way that’ &  & \textit{di stáyl wé} ‘the manner that’ & \textstyleTablePichiZchn{na só} ‘that’s how’\\
Locative\is{locative clauses} &  &  & \textstyleTablePichiZchn{di sáy/plés wé} ‘the place that’ & \\
Cause & \textstyleTablePichiZchn{bikɔs}\textstyleTablePichiZchn{\textup{/}}\textstyleTablePichiZchn{porque} ‘because’, \textstyleTablePichiZchn{as}\textstyleTablePichiZchn{\textup{/}}\textstyleTablePichiZchn{como} ‘since’,

\textstyleTablePichiZchn{fɔséka} ‘due to’ & \textstyleTablePichiZchn{sé} ‘\textsc{quot}’ &  & \textstyleTablePichiZchn{na ín (mék)}

 \textstyleTablePichiZchn{} ‘that’s why’, \textit{so} ‘so’\\
Purpose & \textstyleTablePichiZchn{mék} ‘\textsc{sbjv’}, \textstyleTablePichiZchn{fɔ} ‘\textsc{prep}’ & \textstyleTablePichiZchn{sé} ‘\textsc{quot}’ &  & \\
Extent & \textstyleTablePichiZchn{sóté} ‘until’ &  &  & \\
Limit & \textstyleTablePichiZchn{dásɔl sé}/\textstyleTablePichiZchn{ónli sé}

 ‘only that’ &  &  & \\
Source & \textstyleTablePichiZchn{frɔn wé}\textstyleTablePichiZchn{\textup{/}}\textstyleTablePichiZchn{síns (wé)} ‘since’ &  &  & \\
Conditional & \textstyleTablePichiZchn{ɛf/ɛfɛ/if} ‘if’,

 \textstyleTablePichiZchn{lɛk (sé)} ‘like’ & \textstyleTablePichiZchn{wé} ‘\textsc{sub}’, 

 \textstyleTablePichiZchn{sé} ‘\textsc{quot}’ & Juxtaposition & \\
Concessive & \textstyleTablePichiZchn{ɛf/ɛfɛ/if} — \textstyleTablePichiZchn{sɛ́f} ‘even if’, \textstyleTablePichiZchn{aunque} ‘although’, \textstyleTablePichiZchn{adɔnkɛ́} — \textstyleTablePichiZchn{wáns} ‘even if’ & \textstyleTablePichiZchn{wé} ‘\textsc{sub}’,

 \textstyleTablePichiZchn{sé} ‘\textsc{quot}’ &  & \textstyleTablePichiZchn{bɔt} ‘but’\\
\lspbottomrule
\end{tabularx}
\end{sidewaystable}


\subsection{\textit{Wé} ‘\textsc{sub’}}\label{sec:10.7.1}

The subordinator \textit{wé} ‘\textsc{sub}’ may introduce adverbial clauses of time, condition, and concession. Although \textit{wé} is most commonly used to express temporal relations the other uses are frequent as well. A \textit{wé}{}-clause may precede \REF{ex:key:1458} or follow (cf. \ref{ex:key:1464} below) its main clause and is often set off from preceding and following material by a prosodic break (cf. also \sectref{sec:10.6.1}). In this function, \textit{wé} is best translated as ‘when’:


\ea%1458
    \label{ex:key:1458}
    \gll \textbf{Wé}  a    go  fínis    skúl,  a    go  tɔ́n    dɔ́kta.\\
\textsc{sub}  \textsc{1sg.sbj}  \textsc{pot}  finish  school  \textsc{1sg.sbj}  \textsc{pot}  turn    doctor\\

\glt ‘When I finish school, I’ll become a doctor.’ [ro05ee 023]
\z

The expression of time relations by means of \textit{wé}{}-clauses cannot be divorced from the function of \textit{wé} ‘\textsc{sub’} of introducing sequences of coordinate clauses. Compare the time clause in \REF{ex:key:1458} with the multiple occurrences of \textit{wé} \textit{\textup{here}}:


\ea%1459
    \label{ex:key:1459}
    \gll Pero    \textbf{wé}  a    kán  mít    dís  mán,  \textbf{wé}  wi  bigín  bɔ́n
in    yón     pikín  dɛn.\\
but    \textsc{sub}  \textsc{1sg.sbj}  \textsc{pfv}  meet  this  man    \textsc{sub}  \textsc{1pl}  begin  beget
\textsc{3sg.poss}  own    child  \textsc{pl}\\

\glt ‘But then/when I met this man, and then we began to have his
own children.’ [ab03ab 214]
\z

Time clauses introduced by \textit{wé} are interpreted as being in a relation of temporal overlap with the main clause if both clauses contain imperfective\is{imperfective aspect} readings \REF{ex:key:1460} or are unspecified with respect to aspect like the two clauses in \REF{ex:key:1458} above containing the potential mood marker \textit{go} ‘\textsc{pot}’:


\ea%1460
    \label{ex:key:1460}
    \gll \textbf{Wé}  e    \textbf{kin}  kɔmɔ́t    wók    a    \textbf{kin}  mék=an    só,
lɛk  háw    mún    fínis.\\
\textsc{sub}  \textsc{3sg.sbj}  \textsc{hab}  come.out  work  \textsc{1sg.sbj}  \textsc{hab}  make=\textsc{3sg.obj}  like.that
like  how    month  finish\\

\glt ‘When he leaves work, I do to him like this [stretches out hand in a gesture
that indicates that her husband’s salary should be handed over to her], 
as soon as the month is over.’ [ro05rt 042]
\z

The relation between a main clause and a dependent clause introduced by \textit{wé} can also be one of temporal succession rather than overlap. The interpretation of the temporal relation between the clauses depends on the lexical aspect class of the verbs involved as well as on aspect-marking. For example, in \REF{ex:key:1461} perfective\is{perfective aspect} marking with the dynamic verbs \textit{rích} ‘reach’ and \textit{sé} ‘say, \textsc{quot’} implies succession, however brief the interval:


\ea%1461
    \label{ex:key:1461}
    \gll \textbf{Wé}  a    \textbf{rích}    na  hós    dé,    a    \textbf{sé}  
‘yu  go  tɛ́l  mi    di  sáy  wé  unu  kin  gó  mítɔp.’\\
\textsc{sub}  \textsc{1sg.sbj}  reach  \textsc{loc}  house  there  \textsc{1sg.sbj}  \textsc{quot}  
\textsc{2sg}  \textsc{pot}  tell  \textsc{1sg.indp}  \textsc{def}  side  \textsc{sub}  \textsc{2pl}  \textsc{hab}  go  meet\\

\glt ‘When I reached the house, I said “you’re going to tell me where 
you usually meet.’ [ro05rt 018]
\z

Temporal succession can be rendered more explicit through the use of the perfect marker \textit{dɔ́n} ‘\textsc{prf’} in the main or dependent clause. Hence, the main clause in \REF{ex:key:1462} is posterior to the time clause introduced by \textit{wé} ‘\textsc{sub}’:


\ea%1462
    \label{ex:key:1462}
    \gll A    go  firma  \textbf{wé}  a  go  \textbf{dɔ́n}  chɔ́p.\\
\textsc{1sg.sbj}  \textsc{pot}  sign  \textsc{sub}  \textsc{1sg.sbj}  \textsc{pot}  \textsc{prf}  eat\\

\glt ‘I will sign when I have finished eating.’ [ye03cd 038]
\z

The boundary is fuzzy between temporal and other adverbial meanings of clauses introduced by \textit{wé}. In \REF{ex:key:1463}, the temporal sense of the \textit{wé}{}-clause shades off into a manner or circumstance sense. Context may also give rise to a concessive\is{concessive clauses} meaning of the subordinate clause \REF{ex:key:1464}: 


\ea%1463
    \label{ex:key:1463}
    \gll Dɛn  púl    di  motó  na  garaje    \textbf{wé}  \textbf{dɛn}  \textbf{de}  \textbf{pús}=\textbf{an}.\\
\textsc{3pl}  remove  \textsc{def}  car    \textsc{loc}  workshop  \textsc{sub}  \textsc{3pl}  \textsc{ipfv}  push=\textsc{3sg.obj}\\

\glt ‘They removed the car from the workshop by pushing it.’ [ro05ee 052]
\z


\ea%1464
    \label{ex:key:1464}
    \gll Náw    fɔ  mék    dɛn  fít  gɛ́t  wán    amiga    nadó  \textbf{wé}  \textbf{yu}
\textbf{sísta}  \textbf{dɛn}  \textbf{sabí},  in    go  had.\\
now    \textsc{prep}  \textsc{sbjv}    \textsc{3pl}  can  get  one    girlfriend  outside  \textsc{sub}  \textsc{2sg}
sister  \textsc{3pl}  know  \textsc{3sg.indp}  \textsc{pot}  be.hard\\

\glt ‘Now for them to be able to have a girl-friend outside while/although 
your sisters know, that will be difficult.’ [ro05rt 034]
\z

The relation between the first clause in \REF{ex:key:1465a} and the clause introduced by \textit{wé} is best interpreted as adversative. This is illustrated by the follow-up clause in \REF{ex:key:1465b}:


\ea%1465
    \label{ex:key:1465}
\ea{\label{ex:key:1465a}
    \gll Yu  nó  bin  dé    na  mákit,  \textbf{wé}  a    tɛ́l  yú    sé  
  mék    yu  bríng  mi    watá? \\
  \textsc{2sg}  \textsc{neg}  \textsc{pst}  \textsc{be.loc}  \textsc{loc}  market  \textsc{sub}  \textsc{1sg.sbj}  tell  \textsc{2sg.indp}  \textsc{quot}  
  \textsc{sbjv}    \textsc{2sg}  bring  \textsc{1sg.indp}  water\\

\glt   ‘Weren’t you at the market although I had told you to bring me water?’ [ye0503e? 069]
}\ex{\label{ex:key:1465b}
\gll
Wétin  yu  kán    sin    watá?\\
  what  \textsc{2sg}  come  without  water\\
\glt   ‘Why did you come without water?’ [ye0503e? 070]\is{aspect}
}\z\z

Finally, in \REF{ex:key:1466b}, we find two wholly independent clauses separated by an intonation break, with the second one being introduced by \textit{wé}. The \textit{wé-}clause is contrasted with the implicitly understood concessive proposition in squared brackets. Clause \REF{ex:key:1466b} may be interpreted as being in a causal relationship to clause (a):


\ea%1466
    \label{ex:key:1466}
\ea{\label{ex:key:1466a}
    \gll Sɔn    mamá  dɛn,    dɛn  bád.  \\
  some  mother  \textsc{pl}    \textsc{3pl}  bad\\
\glt   ‘Some mothers, they are bad.’ [ab03ay 109]
}\ex{\label{ex:key:1466b}
\gll
\textbf{Wé}  yu  pikín,  yu  nó  aconseja    ín    frɔn    doce  años.\\
  \textsc{sub}  \textsc{2sg}  child  \textsc{2sg}  \textsc{neg}  advise    \textsc{3sg.indp}  from  twelve  years\\
\glt 
  ‘Because as for your child, you didn’t advise her from twelve years on.’
[although you know about the dangers of early pregnancy].’ [ab03ay 109]
}\z\z

The linker \textit{wé} ‘\textsc{sub}’ is also encountered in the temporal source clause introducers \textit{frɔn wé} and \textit{síns wé,} both of which mean ‘since’ (cf \sectref{sec:10.7.10}).\is{subordinator}

\subsection{\textit{Sé} ‘\textsc{quot’}} \label{sec:10.7.2}

The quotative marker \textit{sé} ‘\textsc{quot}’ may provide adverbial modifications of purpose\is{purpose clauses} and result, cause, manner and circumstance, time and condition. The answer to (a) in (b) below can be interpreted as a cause clause\is{cause clauses}. The \textit{sé-}clause in this example once more vividly illustrates the diversity of meanings of \textit{sé}, particularly in contexts like this one, where it straddles the boundary between quotation proper and other, related functions:


\ea%1467
    \label{ex:key:1467}
\ea{
    \gll Wétin  yu  de  wét?\\
  what  \textsc{2sg}  \textsc{ipfv}  wait\\
\glt   ‘What [why] are you waiting?’ [fr03wt 048]
}\ex{\gll
\textbf{Sé}    in    mamá  go  dráyb=an  fɔ́s.\\
  \textsc{quot}    \textsc{3sg.poss}  mother  \textsc{pot}  drive=\textsc{3sg.obj}  first\\
\glt   ‘(He) says/because his mother will chase him away at first.’ [dj03wt 049]
}\z\z

The codemixed example \REF{ex:key:1468} features a \textit{sé}{}-clause that permits a temporal or conditional interpretation. These interpretations are favoured due to the sentence-initial position of the \textit{sé}{}-clause. The sentence is also instructive because the speaker uses the Spanish temporal conjunction \textit{cuando} ‘when’ in order to render Pichi \textit{sé} ‘\textsc{quot}’ when reiterating the clause in Spanish:


\ea%1468
    \label{ex:key:1468}
    \gll “Yu  hól    wán  motó”,  yu  de  dráyb=an,  pero  \textbf{sé}    yu  gɛ́t,
\textbf{cuando}  tienes,  “a    gɛ́t  wán    motó”.\\
\phantom{“}\textsc{2sg}  hold    one  car    \textsc{2sg}  \textsc{ipfv}  drive=\textsc{3sg.obj}  but    \textsc{quot}    \textsc{2sg}  get
when  you.get  \textsc{1sg.sbj}  get  one    car\\
\glt ` “Yú hól wán motó” (means) you’re driving it, but if you possess it, 
when you have it “a gɛ́t wán motó”.’ [dj05ae 223]
\z

A \textit{sé}{}-clause that follows a main clause and is marked for temporal overlap with the main clause by means of imperfective\is{imperfective aspect} aspect may function as a modification of manner or circumstance \is{circumstantial clauses}in the same way as a \textit{wé}{}-clause. Compare \REF{ex:key:1469} with \REF{ex:key:1463} above: \is{aspect}


\ea%1469
    \label{ex:key:1469}
    \gll Dɛn  púl    di  motó  na  garaje    \textbf{sé}    \textbf{dɛn}  \textbf{de}  \textbf{pús}=\textbf{an}.\\
\textsc{3pl}  remove  \textsc{def}  car    \textsc{loc}  workshop  \textsc{quot}    \textsc{3pl}  \textsc{ipfv}  push=\textsc{3sg.obj}\\

\glt ‘They removed the car from the workshop by pushing it.’ [pa05fn 024]
\z

Such clauses also lend themselves to a concessive interpretation if suggested so by pragmatic context. Compare the concessive \textit{wé}{}-clause in \REF{ex:key:1464} with the following \textit{sé}{}-clause in \REF{ex:key:1470}:


\ea%1470
    \label{ex:key:1470}
    \gll E    dú  di  ejercicio    \textbf{sé}    \textbf{e}    \textbf{táya}.\\
\textsc{3sg.sbj}  do  \textsc{def}  exercise    \textsc{quot}    \textsc{3sg.sbj}  be.tired\\

\glt ‘She did the exercise while/although she was tired.’ [ra07ve 021]
\z

Finally, \textit{sé} is optionally attested with many adverbial clause linkers, among them \textit{bikɔs} (\textit{sé}) ‘because’. \textit{Sé} is obligatory when prepositions take clausal, rather than nominal complements, e.g. \textit{fɔséko} \textit{sé} ‘due to, because’, and \textit{lɛk sé} ‘as if’ \REF{ex:key:1471}:


\ea%1471
    \label{ex:key:1471}
    \gll “A    hól    wán    motó”  na  \textbf{lɛk}  \textbf{sé}    yu  de  dráyb
wé  yu  de  wók.\\
\phantom{“}\textsc{1sg.sbj}  hold    one    car    \textsc{foc}  like  \textsc{quot}    \textsc{2sg}  \textsc{ipfv}  drive
\textsc{sub}  \textsc{2sg}  \textsc{ipfv}  work\\

\glt ‘“A hól wán motó” is like you drive (a car temporarily) while you work.’ [dj05ae 225]
\z

\subsection{Time clauses}\label{sec:10.7.3}

I have shown that temporal relations between clauses may be established in various ways through the polyfunctional linker \textit{wé} ‘\textsc{sub’}. The following clause linkers express adverbial relations of time with more specific meanings. 


Relative clause{\fff}s featuring the generic head nouns \textit{áwa} ‘time’, \textit{tɛ́n} ‘time’, and \textit{dé} ‘day’ function as time clauses. The nature of the temporal relation between the main and the relative clause situations is determined by lexical and clausal aspect marking:



\ea%1472
    \label{ex:key:1472}
    \gll Di  húman  kán    na  hós    \textbf{di}  \textbf{áwa}    [wé  a    de  kúk].\\
\textsc{def}  woman  come  \textsc{loc}  house  \textsc{def}  hour   \phantom{[}\textsc{sub}  \textsc{1sg.sbj}  \textsc{ipfv}  cook\\

\glt ‘The woman came to the house when I was cooking.’ [ro05de 022]
\z


\ea%1473
    \label{ex:key:1473}
    \gll \textbf{Di}  \textbf{tɛ́n}    [wé  dɛn  bin  de  kán    hía    wet    kenú],  \op...\cp{}\\
\textsc{def}  time    \phantom{[}\textsc{sub}  \textsc{3pl}  \textsc{pst}  \textsc{ipfv}  come  here    with    canoe\\

\glt ‘(The time) when they were coming here by canoe (...)’ [ed03sb 189]
\z


\ea%1474
    \label{ex:key:1474}
    \gll \textbf{Di}  \textbf{dé}  [wé  a    nó  wánt  gí  yú    quinientos]
a    de  gí  yú    trescientos    para    tu    cigarillo.\\
\textsc{def}  day   \phantom{[}\textsc{sub}  \textsc{1sg.sbj}  \textsc{neg}  want  give  \textsc{2sg.indp}  five.hundred
\textsc{1sg.sbj}  \textsc{ipfv}  give  \textsc{2sg.indp}  three.hundred  for    your  cigarette\\

\glt ‘(The day) when I don’t want to give you five hundred, I give you 
three hundred for your cigarette.’ [ro05rt 045]
\z

The clause-linker and collocation\textit{ lɛk háw} ‘as soon as’ introduces time clauses. Time clauses introduced by \textit{lɛk háw} precede their main clauses and establish a relation of anteriority with the main clause. This linker may also introduce adverbial manner clause\is{manner clauses}s (cf. \sectref{sec:10.7.4} below): 


\ea%1475
    \label{ex:key:1475}
    \gll Tumɔ́ro,    \textbf{lɛk}  \textbf{háw}    yu  tɔ́k  wet    Buehú,  yu  kɔ́l  mí,    \op...\cp{}\\
tomorrow  like  how    \textsc{2sg}  talk  with    \textsc{name}  \textsc{2sg}  call  \textsc{1sg.indp}\\

\glt ‘Tomorrow, as soon as you’ve talked to Buehu, you call me, (...)’ [fr03cd 111]
\z

The linker \textit{bifó} ‘before’ introduces time clauses that are in a relation of posteriority to the main clause. \textit{Bifó}{}-clauses are preferably sentence-initial, though they are also found in sentence-final position in after-thought apposition, as in \REF{ex:key:1477}: 


\ea%1476
    \label{ex:key:1476}
    \gll \textbf{Bifó}    a    kin  gráp,  a    de  sí  bíg  bíg  fáya.\\
before  \textsc{1sg.sbj}  \textsc{hab}  get.up  \textsc{1sg.sbj}  \textsc{ipfv}  see  big  \textsc{rep}  fire\\

\glt ‘Before I could get up, I saw a huge fire.’ [ab03ay 067]
\z


\ea%1477
    \label{ex:key:1477}
    \gll \op...\cp{}  wé  dɛn  sáyn  yu  bigín  baja    mɔ́,    \textbf{bifó}    yu  ɛ́nta.\\
  \textsc{sub}  \textsc{3pl}  sign    \textsc{2sg}  begin  go.down  more  before  \textsc{2sg}  enter\\

\glt ‘(...) when they have signed, you begin to go down once more before 


\glt you enter.’ [f203fp 004]
\z

It is interesting that the corpus contains no instance of an after-relation expressed by \textit{áfta} ‘after’ in analogy with \textit{bifó} in \REF{ex:key:1477} above. Apparently, \textit{áfta} may only serve as an ‘and then’ clausal connective and does not mean relational ‘after’. Hence, after-relations must be constructed as iconical ‘and then’ relations with the proadverbial \textit{áfta} as in \REF{ex:key:1478}:


\ea%1478
    \label{ex:key:1478}
    \gll Lɛ́f=an,    a    go  chɔ́p,  \textbf{áfta}    a    go  dríng.\\
leave=\textsc{3sg.obj}  \textsc{1sg.sbj}  \textsc{pot}  eat    then  \textsc{1sg.sbj}  \textsc{pot}  drink\\

\glt ‘Leave it, I will eat, then I will drink.’ [ye03cd 079]
\z

Alternatively, the after-relation can be expressed by an initial \textit{wé-}clause accompanied by perfect marking, as in \REF{ex:key:1479}:


\ea%1479
    \label{ex:key:1479}
    \gll Sifta,  \textbf{wé}  a    \textbf{dɔ́n}  sifta    ín,    e    de  lɛ́f    wet    di  watá.\\
sift    \textsc{sub}  \textsc{1sg.sbj}  \textsc{prf}  sift    \textsc{3sg.indp}  \textsc{3sg.sbj}  \textsc{ipfv}  leave  with    \textsc{def}  water\\

\glt ‘Sift (it), when I have sifted it, it’ll be left with the water.’ [dj03do 007]
\z

\subsection{Manner clauses}\label{sec:10.7.4}

Manner clauses may be expressed through a relative construction featuring the generic head noun \textit{stáyl} ‘style, manner’:


\ea%1480
    \label{ex:key:1480}
    \gll \MakeUppercase{A}   bin  chɔ́p  di  plantí  di  \textbf{stáyl}  [wé  pɔ́sin  dɛn
fɔ  Malábo  dɛn  de  chɔ́p=an]\\
\textsc{1sg.sbj} \textsc{pst} \textstylePichitranslationZchn{eat} \textsc{def} \textstylePichitranslationZchn{plantain} \textsc{def} \textstylePichitranslationZchn{ style} \phantom{[}\textsc{sub} \textstylePichitranslationZchn{person} \textsc{pl} 
\textsc{prep}  Malabo  \textsc{3pl}  \textsc{ipfv}  food=\textsc{3sg.obj}\\

\glt ‘I ate the plantain the way Malabo people eat it.’ [dj05ae 069]
\z

Manner clauses may also be formed by way of adverbial clauses introduced by the collocation \textit{lɛk háw} ‘like how’ = ‘the way that’. Compare the near-identical sentence above with the two following ones:


\ea%1481
    \label{ex:key:1481}
    \gll Mí    chɔ́p  di  plantí  \textbf{lɛk}  \textbf{háw}    Malabo-pípul      dɛn  de  chɔ́p=an.\\
\textsc{1sg.indp}   eat \textsc{def}   plantain  like  how    Malabo.\textsc{cpd}{}-people \textsc{pl}  \textsc{ipfv}  eat=\textsc{3sg.ob}j\\

\glt ‘I \textsc{[emp]} ate the plantain the way Malabo people eat it.’ [ro05de 019]
\z


\ea%1482
    \label{ex:key:1482}
    \gll A    nó  sabí    ús=tín  dɛn  nó  go  restaura    ín
\textbf{lɛk}  \textbf{háw}   e    bin  dé    jamás.\\
\textsc{1sg.sbj}  \textsc{neg}  know  \textsc{q}=thing  \textsc{3pl}  \textsc{neg}  \textsc{pot}  restorate    \textsc{3sg.indp}
like  how    \textsc{3sg.sbj}  \textsc{pst}  \textsc{be.loc}  ever\\

\glt ‘I don’t know why they won’t restore it the way it was back then.’ [hi03cb 038]
\z

Manner clauses introduced by \textit{lɛk háw} ‘like how’ are also often employed to denote the perceived situation of a main clause verb of sensory perception like \textit{hía} ‘hear’ \REF{ex:key:1483}, \textit{sí} ‘see’, \textit{lúk} ‘look’, \textit{smɛ́l} ‘smell’ \REF{ex:key:1484}, and \textit{fíl} ‘feel’. Such clauses vacillate between readings denoting manner and temporal overlap:


\ea%1483
    \label{ex:key:1483}
    \gll A    de  \textbf{hía}  ín    \textbf{lɛk}  \textbf{háw}    e    de  nák  di  gita.\\
\textsc{1sg.sbj}  \textsc{ipfv}  hear  \textsc{3sg.indp}  like  how    \textsc{3sg.sbj}  \textsc{ipfv}  hit  \textsc{def}  guitar\\

\glt ‘I hear him playing the guitar.’ or ‘I hear (him) how he’s playing the guitar.’ [dj05ae 053]
\z


\ea%1484
    \label{ex:key:1484}
    \gll A    de  \textbf{smɛ́l}    di  sɛ́nt    fɔ  \textbf{lɛk}  \textbf{háw}    e    de  kúk    plantí.\\
\textsc{1sg.sbj}  \textsc{ipfv}  smell  \textsc{def}  scent  \textsc{prep}  like  how    \textsc{3sg.sbj}  \textsc{ipfv}  cook  plantain\\

\glt ‘I smell the scent of him cooking plantain.’ [dj05ae 025]
\z

The collocation \textit{lɛk háw} also forms part of the idiomatic phrase \textit{lɛk háw yu (de) sí X} (X referring a person) which means something along the lines of ‘when looking at X you should also know’. Compare the following example: 


\ea%1485
    \label{ex:key:1485}
    \gll Mí,    \textbf{lɛk}  \textbf{háw}    \textbf{yu}  \textbf{de}  \textbf{sí}  \textbf{mí}    a    dɔ́n  sí  plɛ́nte  tín.\\
\textsc{1sg.indp}  like  how    \textsc{2sg}  \textsc{ipfv}  see  \textsc{1sg.indp}  \textsc{1sg.sbj}  \textsc{prf}  see  plenty  thing\\

\glt ‘(As for) me, when you looking at me you should also know that I have seen many 
 things [in life].’ [ab03ab 023]
\z

Manner clauses introduced by \textit{lɛk háw} may shade off into a temporal reading and vice-versa. Manner clauses generally follow their main clauses as in the preceding examples. In contrast, time clauses introduced by\textit{ lɛk háw} normally precede their main clauses (cf. \ref{ex:key:1475} above in the previous section).


However, we also sometimes find manner clauses introducecd by \textit{lɛk háw} in a sentence-initial, topical position. When such a clause is marked for an imperfective\is{imperfective aspect} reading, it is likely to be interpreted as a manner clause. \textit{Lɛk háw} then means ‘the way that’ \REF{ex:key:1486}: 



\ea%1486
    \label{ex:key:1486}
    \gll \textbf{Lɛk}  \textbf{háw}    e    \textbf{de}  \textbf{wáka},  e    butú,  e    nó  bɛ́n.\\
like  how    \textsc{3sg.sbj}  \textsc{ipfv}  walk  \textsc{3sg.sbj}  stoop  \textsc{3sg.sbj}  \textsc{neg}  bend\\

\glt ‘The way he’s walking (now), he’s stooped over, he’s not bent over.’ [au07se 082]
\z

On the other hand, if a sentence-initial clause introduced by \textit{lɛk háw} is marked for a perfective\is{perfective aspect} reading, it is very likely to be interpreted as a time clause. \textit{Lɛk háw} then translates as ‘as soon as’. In \REF{ex:key:1487}, the subordinate clause contains the factative\is{factative TMA} marked (hence perfective) dynamic verb \textit{pút} ‘put’. Compare the temporal interpretation of this sentence with the manner reading of \REF{ex:key:1486} above. Also compare the temporal interpretation of the factative-marked verb \textit{pút} ‘put’ in the previous section in \REF{ex:key:1475} above:


\ea%1487
    \label{ex:key:1487}
    \gll \textbf{Lɛk}  \textbf{háw}   e    \textbf{pút}  dán  mɔní  na  mi    hán,    nó  wét    mɔ́!\\
like  how    \textsc{3sg.sbj}  put  that  money  \textsc{loc}  \textsc{1sg.poss}  hand  \textsc{neg}  wait    more\\

\glt ‘As soon as he has put that money into my hand, no time to waste!’ [ro05rt 043]
\z

If a manner interpretation is nevertheless desired for a clause featuring a situation marked for a perfective reading, a relative construction featuring the head noun \textit{stáyl} ‘style, manner’ is chosen. In \REF{ex:key:1488}, the manner relation is expressed via a relative construction. This option is chosen because the subordinate dynamic verb \textit{nák} ‘hit’ is marked for factative TMA, hence it is perfective and bounded: 


\ea%1488
    \label{ex:key:1488}
    \gll E    nák  di  tébul  an  di  \textbf{stáyl} [wé\textbf{}   e    \textbf{nák}  di  tébul
strɔ́n],    e    kán  sék    di  plét,    an  di  plét    kán  brók.\\
\textsc{3sg.sbj}  hit  \textsc{def}  table  and   \textsc{def}  style   \phantom{[}\textsc{sub}    \textsc{3sg.sbj}  hit  \textsc{def}  table
be.strong   \textsc{3sg.sbj}  \textsc{pfv}  shake  \textsc{def}  plate  and  \textsc{def}  plate  \textsc{pfv}  break\\

\glt ‘He hit the table and the way that he hit the table in a strong way, he shook the plate,
and the plate broke.’ [au07se 014]
\z

Other means of providing manner modification by clauses are adverbial SVCs and the use of adverbial clauses introduced by \textit{wé} ‘\textsc{sub}’ and \textit{sé} ‘\textsc{quot}’. Note that equative clauses\is{equative clauses} – manner clauses which serve as the standard in a comparison – are also introduced by the collocation \textit{lɛk háw} (cf. \ref{ex:key:508}–\ref{ex:key:509}).\is{manner clauses}

\subsection{Locative clauses}

The formation of locative clauses involves the relativisation of the generic head nouns \textit{sáy} ‘side’ and less frequently \textit{plés} ‘place’. Locative adverbial relations can only be expressed via such relative constructions, because the linker \textit{wé} ‘\textsc{sub}’ does not introduce headless locative relative clause\is{relative clauses}s: 


\ea%1489
    \label{ex:key:1489}
    \gll Náw    e    dɔ́n  wánt  bigín  de  fɛ́t    wet
\textbf{di}  chía,  di  \textbf{sáy}  [wé    dɛn  sidɔ́n]. \\
now    \textsc{3sg.sbj}  \textsc{prf}  want  begin  \textsc{ipfv}  fight  with
\textsc{def}  chair  \textsc{def}  side   \textsc{sub}    \textsc{3pl}  sit\\

\glt ‘Now he already wanted to begin fighting with the chair, 
where they were sitting.’ [ab03ab 132]
\z


\ea%1490
    \label{ex:key:1490}
    \gll Yu  nó  nó    \textbf{di}  \textbf{plés}    [wé    a    sidɔ́n]?\\
\textsc{2sg}  \textsc{neg}  know  \textsc{def}  \textsc{place}   \textsc{sub}    \textsc{1sg.sbj}  stay\\

\glt ‘You don’t know where I stay?’ [he07fn 307]\is{locative clauses}
\z

\subsection{Purpose and result clauses}\label{sec:10.7.6}

The clause linkers \textit{fɔ} ‘\textsc{prep}’ and \textit{sé} ‘\textsc{quot}’, as well as the subjunctive marker \textit{mék} are employed to introduce purpose clauses. A purpose relation typically involves a willful and animate subject\is{subjects} that intentionally performs a main clause action aimed at the completion of the situation in the subordinate clause. There are no semantic restrictions on the type of main verb that purpose clauses may modify in Pichi. Neither is there any formal difference between “realised” (i.e. that the purpose is achieved) and “unrealised” purpose clauses (cf. \citealt[59]{Bickerton1981}).


Thus below, we find purpose clauses modifying main clauses with verbs as diverse as \textit{ol} ‘be old’ or \textit{wét} ‘wait’:



\ea%1491
    \label{ex:key:1491}
    \gll \MakeUppercase{A}   dɔ́n    tú  \textbf{ól}  \textbf{fɔ} máred.\\
\textsc{1sg.sbj}  \textsc{prf}    too  old  \textsc{prep}  marry\\

\glt ‘I’m too old to marry.’ [fr03ab 206]
\z


\ea%1492
    \label{ex:key:1492}
    \gll A    go  firma,   \textbf{wét}     \textbf{fɔ}  \textbf{mék}    a    chɔ́p,  a    bɛ́g.\\
\textsc{1sg.sbj}  \textsc{pot}  sign    wait    \textsc{prep}  \textsc{sbjv}    \textsc{1sg.sbj}  eat    \textsc{1sg.sbj}  beg\\

\glt ‘I’ll sign, wait for me to eat/have eaten, please.’ [ye03cd 043]
\z

The motion verbs\is{motion verbs} \textit{gó} ‘go’ \REF{ex:key:1493} and \textit{kán} ‘come’ (\ref{ex:key:1494} below) may optionally reinforce the purposive sense of the subordinate clause: 


\ea%1493
    \label{ex:key:1493}
    \gll Dɛn  kán  kɛ́r    mí    na  Madrid  \textbf{fɔ}  \textbf{mék}    dɛn  \textbf{gó}  opera  mí.\\
\textsc{3pl}  \textsc{pfv}  carry  \textsc{1sg.indp}  \textsc{loc}  \textsc{place}  \textsc{prep}  \textsc{sbjv}    \textsc{3pl}  go  operate  \textsc{1sg.indp}\\

\glt ‘They took me to Madrid in order to operate on me.’ [fr03ft 026]
\z

When the subjects of the main and subordinate clauses are identical, the purpose clause may be introduced by the non-finite clause linker \textit{fɔ} ‘\textsc{prep}’ alone \REF{ex:key:1494}:


\ea%1494
    \label{ex:key:1494}
    \gll Mi    papá  bin  {kán}    yá    \textbf{fɔ}  {kán}    wók.\\
\textsc{1sg.poss}  father  \textsc{pst}  come  here    \textsc{prep}  come  work\\

\glt ‘My father came here in order to work.’ [fr03ft 063]
\z

When the main and subordinate clauses have different subject\is{subjects}s, the purpose clause is expressed as a more finite subjunctive clause. Such purpose clauses are marked in the same way as other types of different-subject subordinate clauses that involve a form of deontic modality. The subjunctive marker may optionally be preceded by \textit{fɔ} ‘\textsc{prep}’ as in \REF{ex:key:1495}:


\ea%1495
    \label{ex:key:1495}
    \gll Layk  háw    dɛn    go  pút  yú    na  tébul  {yu}  dɔ́n  de  rɔ́tin,
\textbf{fɔ}  \textbf{mék}    dɛn  gó  bɛ́r    yú    kwík.\\
like    how    \textsc{3pl}    \textsc{pot}  put  \textsc{2sg.indp}  \textsc{loc}  table  \textsc{2sg}  \textsc{prf}  \textsc{ipfv}  rot
\textsc{prep}  \textsc{sbjv}    \textsc{3pl}  go  bury  \textsc{2sg.indp}  quickly\\

\glt ‘As soon as they put you on the table you are already rotting away 
for you to be buried quickly.’ [ed03sb 101]
\z

However, a very frequent alternative is for both different- \REF{ex:key:1496} and same-subject \REF{ex:key:1497} purpose clauses to be introduced by the subjunctive marker alone:


\ea%1496
    \label{ex:key:1496}
    \gll Na  ín    \textbf{dɛn}  táy=an    \textbf{mék}    \textbf{e}   nó  kɔmɔ́t.\\
\textsc{foc}  \textsc{3sg.indp}  \textsc{3pl}  tie=\textsc{3sg.obj}  \textsc{sbjv}    \textsc{3sg.sbj}  \textsc{neg}  go.out\\

\glt ‘That’s why they tied it [the dog] so that it wouldn’t leave.’ [ma03hm 005]
\z


\ea%1497
    \label{ex:key:1497}
    \gll \textbf{A}   go  gó  lúk=an    fɔ  wán    vecino    \textbf{mék}    \textbf{a}   
lúk    las    damas.\\
\textsc{1sg.sbj}  \textsc{pot}  go  look=\textsc{3sg.obj}  \textsc{prep}  one    neighbour  \textsc{sbjv}    \textsc{1sg.sbj}  
look    the.\textsc{pl}  lady.\textsc{pl}\\
\glt ‘I’ll watch it at a neighbour’s in order to look at the (first) ladies.’ [ma03hm 074]
\z

Negation\is{negation} of the subordinate situation obligatorily entails the use of subjunctive purpose clauses, even where the subject\is{subjects}s of the main and subordinate clauses are identical, as in \REF{ex:key:1498}: 


\ea%1498
    \label{ex:key:1498}
    \gll \textbf{A}    dríng  di  mɛ́rɛsin    fɔ  \textbf{mék}    \textbf{a}    \textbf{nó}  sík.\\
\textsc{1sg.sbj}  drink  \textsc{def}  medicine  \textsc{prep}  \textsc{sbjv}    \textsc{1sg.sbj}  \textsc{neg}  be.sick\\

\glt ‘I drank the medicine in order not to fall sick.’ [ro05de 021]
\z

When the purpose clause is fronted for emphasis, it is not usually introduced by \textit{mék} alone. Instead, the purpose clause is normally introduced by \textit{fɔ} ‘\textsc{prep}’ or \textit{sé} ‘\textsc{quot}’ and then followed by \textit{mék} ‘\textsc{sbjv}’. This is probably so because a sentence-initial \textit{mék} ‘\textsc{sbjv}’ signals the presence of a subjunctive-marked directive main clause: 


\ea%1499
    \label{ex:key:1499}
    \gll \textbf{Fɔ}  \textbf{mék}    yu  fít  ɛ́nta    yu  gɛ́fɔ    bísin  na  wán  pɔ́sin
fɔ  di  fámbul  \op...\cp{}\\
\textsc{prep}  \textsc{sbjv}    \textsc{2sg}  can  enter  \textsc{2sg}  have.to  be.busy  \textsc{loc}  one  person
\textsc{prep}  \textsc{def}  family\\

\glt ‘In order to be able to enter, you have to be involved with a person 
of the family (...)’ [ed03sb 077]
\z

Different subject purpose clauses may also additionally feature the quotative marker\is{quotative marker} \textit{sé} ‘\textsc{quot}’ like any other subjunctive subordinate clause. In such cases, the purpose clause is also usually marked for subjunctive mood. Compare the sentence below; it contains a purpose clause introduced by \textit{sé mék} as well as one introduced by \textit{mék} alone: 


\ea%1500
    \label{ex:key:1500}
    \gll A    bin  lás    gó  a    de  fɛ́n    bíg  bíg  mamá  dɛn, \textbf{sé}    \textbf{mék}  
dɛn  bí  mi    gɛ́l  frɛ́n,  \textbf{mék}    dɛn  de  gí  mi    chɔ́p.\\
\textsc{1sg.sbj}  \textsc{pst}  end.up  go  \textsc{1sg.sbj}  \textsc{ipfv}  look.for  big  \textsc{rep}  mother  \textsc{pl}  \textsc{quot}    \textsc{sbjv}
\textsc{3pl}  \textsc{be}  \textsc{1sg.poss}  girl  friend   \textsc{sbjv}    \textsc{3pl}  \textsc{ipfv}  give  \textsc{1sg.indp}  food\\
\glt ‘I finally went to look for mature/established women for them to be my girlfriends, 
for them to give me food.’ [ed03sp 079]\is{subjunctive mood}
\z

The following two sentences featuring clauses introduced by the quotative\is{quotative clauses} marker \textit{sé} can be interpreted as purposive although they are not followed by subjunctive clauses. These sentences are further evidence for the polyfunctionality of the quotative marker. Here, the expression of speaker intention through inner speech rendered in a quotative construction acquires a purposive reading. 


This is the case in the 1\textsuperscript{st} person statement of intention in direct speech\is{direct speech} in \REF{ex:key:1501}, in which \textit{sé} functions more like a clause linker as well as in the 3\textsuperscript{rd} person indirect speech\is{indirect speech}, in which \textit{sé} behaves like a lexical verb \REF{ex:key:1502}:



\ea%1501
    \label{ex:key:1501}
    \gll A    wáka  wet=an    \textbf{sé}    ‘tidé    a    go  gó  vive    ín.’\\
\textsc{1sg.sbj}  walk  with=\textsc{3sg.obj}  \textsc{quot}    today  \textsc{1sg.sbj}  \textsc{pot}  go  live    \textsc{3sg.indp}\\

\glt ‘I went with him so that today I would witness it.’ [ed03sb 007]
\z


\ea%1502
    \label{ex:key:1502}
    \gll So  e    go  na  bús    \textbf{e}    \textbf{sé}    e    de  gó  kíl  bíf.\\
so  \textsc{3sg.sbj}  \textsc{pot}  \textsc{loc}  forest  \textsc{3sg.sbj}  \textsc{quot}    \textsc{3sg.sbj}  \textsc{ipfv}  go  kill  wild.animal\\

\glt ‘So he went to the forest in order to/he said he’d go kill a wild animal.’ [ma03sh 004]
\z

Finally, a \textit{sé}{}-clause may acquire a result\is{result clauses} reading when it features non-modal TMA marking or when a modal complementiser is absent. Compare the following example:


\ea%1503
    \label{ex:key:1503}
    \gll \op...\cp{}  e    \textbf{sút}=an    \textbf{sé}    e    \textbf{dɔ́n}  \textbf{wɔ́nt}  gó  ték=an,
{} e    sí  di  tín    dɔ́n  de  tɔ́n    pɔ́sin.\\
  \textsc{3sg.sbj}  shoot=\textsc{3sg.obj}  \textsc{quot}    \textsc{3sg.sbj}  \textsc{prf}  want  go  take=\textsc{3sg.obj}
\textsc{3sg.sbj}  see  \textsc{def}  thing  \textsc{prf}  \textsc{ipfv}  turn    person\\

\glt ‘(...) he shot it [the animal] and was about to go take it, (when) he saw the thing 
turning into a human-being.’ [ma03sh 005]
\z

\subsection{Cause clauses}\label{sec:10.7.7}

Cause relations may be expressed through \textit{sé}{}-clauses and adverbial clauses introduced by the linkers \textit{bikɔs (sé)} ‘because’, \textit{foséka} ‘due to, for the sake of’, and \textit{ás} ‘as’. Speakers also employ the Spanish-origin linkers \textit{porque} ‘because’ and \textit{como} ‘as’, \is{loan words}which have been borrowed into Pichi and form an integral part of the Pichi system of clause linkage (cf. \sectref{sec:13.2.3} for a more detailed treatment in the context of codemixing). Compare \textit{bikɔs} \textit{(sé)} below:


\ea%1504
    \label{ex:key:1504}
    \gll A    drɛ́b    mi    mán    \textbf{bikɔs}  \textbf{sé}    a    nó  wánt=an    mɔ́.\\
\textsc{1sg.sbj}  drive  \textsc{1sg.poss}  man    because  \textsc{quot}    \textsc{1sg.sbj}  \textsc{neg}  want=\textsc{3sg.obj}  more\\

\glt  ‘I drove my husband away because I didn’t want him anymore.’ [ro05de 015]
\z

Cause clauses introduced by \textit{bikɔs} may appear at the beginning of the sentence \REF{ex:key:1505}. When this is so, the cause clause is focused with \textit{na} ‘\textsc{foc}’ and reiterated by means of one of the resumptive expressions \textit{na ín} ‘\textsc{foc} \textsc{3sg.indp}’ and \textit{na di tín} ‘\textsc{foc} \textsc{def} thing’, both of which mean ‘that’s why’ in this particular context:


\ea%1505
    \label{ex:key:1505}
    \gll \textbf{Na}  \textbf{bikɔs}  e    bɔ́n      pikín,  \textbf{na}  \textbf{di}  \textbf{tín} 
mék    e    dáy.\\
\textsc{foc} because \textsc{3sg.sbj} give.birth  child  \textsc{foc} thing make  \textsc{3sg.sbj}  die\\

\glt ‘It is because she gave birth (to a child), that’s why she died.’ [dj05be 051]
\z

Cause clauses introduced by \textit{ás} ‘as’ \REF{ex:key:1506} and \textit{como} ‘since’ \REF{ex:key:1507} precede their main clauses:\is{borrowing}


\ea%1506
    \label{ex:key:1506}
    \gll \textbf{\'{A}s}  dɛn  nɔ́ba  bin  sí  plantí,  dɛn  bin  chɔ́p=an    rɔ́n-wán.\\
as  \textsc{3pl}  \textsc{neg}.\textsc{prf}  \textsc{pst}  see  plantain  \textsc{3pl}  \textsc{pst}  eat=\textsc{3sg.obj}  wrong\textsc{{}-adv}\\

\glt ‘As they hadn’t yet seen plaintain, they ate it in the wrong way.’ [ro05ee 062]
\z


\ea%1507
    \label{ex:key:1507}
    \gll \textbf{Como}  e    sabí    sé    dán  tín    dé    na  mi    hát  \op...\cp{}\\
since  \textsc{3sg.sbj}  know  \textsc{quot}    that  thing  \textsc{be.loc}  \textsc{loc}  \textsc{1sg.poss}  heart\\

\glt ‘Since she knows that that thing [matter] is in my heart (...)’ [ro07fn 673]
\z

The linkers\textit{ bikɔs} ‘because’ and \textit{porque} ‘because’ may be found in the initial position in sentences (i.e. in prosodically independent utterances) with a weak causal link with preceding sentences. In such instances, these linkers function as discourse markers that introduce elaborations to preceding material. Compare the use of \textit{porque} in \REF{ex:key:1508}:


\ea%1508
    \label{ex:key:1508}
\ea{
\gll
E    bin  fɔ    dé    fáyn.\\
  \textsc{3sg.sbj}  \textsc{pst}  \textsc{cond}    \textsc{be.loc}  fine\\

\glt   ‘That would have been fine.’ [fr03ft 172]
}\ex{\gll
\textbf{Porque}  mi    sɛ́f,  fɔ́s  tɛ́n    a    bin  de  sidɔ́n
  dásɔl  wet    húman  dɛn.\\
because  \textsc{1sg.indp}  \textsc{emp}  first  time    \textsc{1sg.sbj}  \textsc{pst}  \textsc{ipfv}  stay
  only    with    woman  \textsc{pl}\\

\glt   ‘Because me, formerly I was staying only with women.’ [fr03ft 173]
}\z\z

The preposition \textit{fɔséka} \textit{\textup{(and its free variant} }\textit{fɔséko}\textit{)} ‘due to, for the sake of’ takes nominal, not clausal, complements (cf. e.g. \ref{ex:key:1063}). However, when \textit{fɔséka} is followed by the quotative marker\is{quotative marker} and complementiser \textit{sé} ‘\textsc{quot}’, the resulting collocation may introduce a cause clause like the other linkers treated in this section \REF{ex:key:1509}:


\ea%1509
    \label{ex:key:1509}
    \gll Mí    dú=an    \textbf{fɔséko}  \textbf{sé}    a    bin  wánt  hɛ́lp=an.\\
\textsc{1sg.indp}  do=\textsc{3sg.obj}  due.to  \textsc{quot}    \textsc{1sg.sbj}  \textsc{pst}  want  help=\textsc{3sg.obj}\\

\glt ‘I [\textsc{emp}] did it because I wanted to help her.’ [ro05ee 069]\is{cause clauses}
\z

\subsection{Extent and result clauses}

Speakers make use of the linker \textit{sóté} ‘until’ in order to express a relation of temporal extent, as in the first example below. Such clauses may also be interpreted as result clauses\is{result clauses} in the appropriate context \REF{ex:key:1511}. \textit{Sóté} ‘until’ is a multifunctional word that is also used as a preposition (cf. \sectref{sec:9.1.3}), as a degree adverbial (cf. \sectref{sec:7.7.3}), and in the expression of spatial extent (cf. e.g. \ref{ex:key:903}):


\ea%1510
    \label{ex:key:1510}
    \gll Mék    e    wét    \textbf{sóté}    mún    dɔ́n,    wé  wi  gɛ́t  di  mɔní,
gó  báy  di  chɔ́p.\\
\textsc{sbjv}    \textsc{3sg.sbj}  wait    until  month  finish  \textsc{sub}  \textsc{1pl}  get  \textsc{def}  money
go  buy  \textsc{def}  food\\

\glt ‘Let him wait until the month is over, when we have the money, (then we) 
go buy the food.’ [hi03cb 214]
\z


\ea%1511
    \label{ex:key:1511}
    \gll A    chɔ́p  frijoles  \textbf{sóté}    a    táya.\\
\textsc{1sg.sbj}  eat    bean.\textsc{pl}  until  \textsc{1sg.sbj}  be.tired\\

\glt ‘I ate beans until I was tired (of it).’ [ed03sp 121]
\z

Extent clauses introduced by \textit{sóté} are marked for subjunctive\is{subjunctive mood} mood when the speaker expresses an anticipated outcome as in \REF{ex:key:1512}. This usage may be due to transfer from \ili{Spanish}. The equivalent Spanish conjunction \textit{hasta que} ‘until (that)’ is also used with the subjunctive mood. Compare the subjunctive-marked \textit{llegue} ‘arrive’ in \REF{ex:key:1513}.


\ea%1512
    \label{ex:key:1512}
    \gll Tɔ́n=an    tɔ́n=an,    mék  yu  nó  para    \textbf{sóté}    \textbf{mék}   e    tík
lɛk  háw    e    bin  dé    só.\\
turn=\textsc{3sg.obj}  turn=\textsc{3sg.obj}  \textsc{sbjv}  \textsc{2sg}  \textsc{neg}  stop    until  \textsc{sbjv}    \textsc{3sg.sbj}  be.thick
like  how    \textsc{3sg.sbj}  \textsc{pst}  \textsc{be.loc}  like.that\\

\glt ‘Stir it, stir it, don’t stop until it is as thick as it was right now!’ [dj03do 058]
\z


\ea%1513
    \label{ex:key:1513}
    \gll ¡Haga  cola    \textbf{hasta} \textbf{que}  \textbf{llegue}    el  cajero!\\
do    line    until that  arrive:\textsc{sbjv}  the  teller\\

\glt ‘Make a line until the teller arrives!’ (Own knowledge)
\z

However, the appearance of subjunctive marking in a clause like \REF{ex:key:1512} above also harmonises with deontic notions like preference and desire that also underlie the use of subjunctive in similar clause types, e.g. purpose\is{purpose clauses} clauses:

\subsection{Limit clauses}\label{sec:10.7.9}

Limit clauses are formed by using the quantifying adverb \textit{dásɔl} ‘only’ before the appropriate adverbial clause linker. Below, \textit{dásɔl} ‘only’ collocates with \textit{fɔ} ‘\textsc{prep}’, which in turn, introduces a non-finite purpose\is{purpose clauses} clause: 


\ea%1514
    \label{ex:key:1514}
    \gll A    bin  mék=an    \textbf{dásɔl}  \textbf{fɔ}  hɛ́lp.\\
\textsc{1sg.sbj}  \textsc{pst}  make=\textsc{3sg.obj}  only    \textsc{prep}  help\\

\glt ‘I did it only in order to help.’ [dj05be 129]
\z

The following example illustrates the use of \textit{dásɔl} followed by \textit{sé} ‘\textsc{quot}’, which introduces a finite complement clause: 


\ea%1515
    \label{ex:key:1515}
    \gll Wi  de  sí  \textbf{dásɔl}  \textbf{sé}    di  písis        dɔ́n  hɛ́ng.\\
\textsc{1pl}  \textsc{ipfv}  see  only    \textsc{quot}    \textsc{def}  piece.of.cloth    \textsc{prf}  hang\\

\glt ‘We only see that the piece of cloth is already hanging.’ [li07pe 059]
\z

The quantifying adverb \textit{ónli} ‘only’ may be employed in the same way as \textit{dásɔl} and occurs equally often in limit clauses. In this sentence, \textit{ónli} ‘only’ precedes a cause clause introduced by \textit{bikɔs} ‘because’: \is{cause clauses}


\ea%1516
    \label{ex:key:1516}
    \gll \textbf{\'{O}nli}    \textbf{bikɔs}  yu  de  tɔ́k  só,    yu  de  salút  só,
yu  de  ánsa    só.\\
only    because  \textsc{2sg}  \textsc{ipfv}  talk  like.that  \textsc{2sg}  \textsc{ipfv}  greet  like.that
\textsc{2sg}  \textsc{ipfv}  answer  like.that \\

\glt ‘Only because you talk like that, you greet like that, you respond like that.’ [au07se 158]
\z

\subsection{Source clauses}\label{sec:10.7.10}

Temporal source clauses may be introduced by the collocations \textit{frɔn wé} \{from \textsc{sub}\} ‘since’ and \textit{síns} \textit{wé} ‘since \textsc{sub’} = ‘since’. Both collocations require the subordinator\is{subordinator} because they involve prepositions that take nominal complements. Compare the following examples:


\ea%1517
    \label{ex:key:1517}
    \gll \textbf{Frɔn}  \textbf{wé}  dán  bɛ́lps  de  wók,  chico,  e    dɔ́n  chénch.\\
from  \textsc{sub}  that  babe  \textsc{ipfv}  work  boy    \textsc{3sg.sbj}  \textsc{prf}  change\\

\glt ‘(Ever) since that babe has been working, man, she has changed.’ [dj07ae 173]
\z


\ea%1518
    \label{ex:key:1518}
    \gll \textbf{Frɔn}    \textbf{wé}  a  bí  pikín  a    bin  wánt  kɔmɔ́t
na  dís  kɔ́ntri.\\
from  \textsc{sub}  \textsc{1sg.sbj}  \textsc{be}  child  \textsc{1sg.sbj}  \textsc{pst}  want  go.away
\textsc{loc}  this  country\\

\glt ‘(Ever) since I was a child, I wanted to leave this country.’ [ro05ee 027]
\z

The preposition \textit{síns} ‘since’ is one of two dedicated temporal prepositions of Pichi (the other one being \textit{ápás} ‘after’, cf. \sectref{sec:8.2.2}) and may introduce source clauses in combination with the subordinator \textit{wé} ‘\textsc{sub’,} cf. \REF{ex:key:1519} below:


\ea%1519
    \label{ex:key:1519}
    \gll \textbf{Síns}    \textbf{wé}  a    bí  pikín,  a    de  mɛ́mba  fɔ
kɔmɔ́t  na  dí  kɔ́ntri.\\
since  \textsc{sub}  \textsc{1sg.sbj}  \textsc{be}  child  \textsc{1sg.sbj}  \textsc{ipfv}  think.of  \textsc{prep}
go.out  \textsc{loc}  this  country\\

\glt ‘Since I was a child, I think about leaving this country.’ [li07fn 303]]
\z

\subsection{Conditional clauses}\label{sec:10.7.11}
\tabref{tab:key:10.5} summarises the most common ways of expressing conditional relations in Pichi. It features the three functionally identical \textsc{if-}clause introducers \textit{ɛf}, \textit{ɛfɛ,} and \textit{if}, all of which mean ‘if’ as well as the various types of TMA marking attested in the \textsc{if-} and \textsc{then-}clauses. I comment on the relative frequency of the different constructions below:

%%please move \begin{table} just above \begin{tabular
\begin{table}
\caption{Conditional relations}
\label{tab:key:10.5}

\begin{tabularx}{\textwidth}{llQQ}
\lsptoprule

Type & Introducer & \textsc{if-}clause & \textsc{then-}clause\\
\midrule
Reality & \textstyleTablePichiZchn{ɛf, if} ‘if’ & Non-modal tense \& aspect & Non-modal tense \& aspect\\
Potential & \textstyleTablePichiZchn{ɛf, if} ‘if’ & \textstyleTableEnglishZchn{Factative} \textstyleTableEnglishZchn{TMA},\newline  \textstyleTablePichiZchn{go} \textstyleTableEnglishZchn{\textsc{‘pot’,}} \textstyleTablePichiZchn{de} \textstyleTableEnglishZchn{\textsc{‘ipfv’}} & \textstyleTablePichiZchn{go} ‘\textstyleTableEnglishZchn{\textsc{pot’,} }\textstyleTablePichiZchn{de} \textstyleTableEnglishZchn{\textsc{‘ipfv’}}\\
Counterfactual & \textstyleTablePichiZchn{ɛf, if} ‘if’ & \textstyleTablePichiZchn{bin} \textstyleTableEnglishZchn{\textsc{‘pst’}} & \textstyleTablePichiZchn{(bin) fɔ} ‘\textstyleTableEnglishZchn{\textsc{(pst)} \textsc{prep’}}\\
\lspbottomrule
\end{tabularx}
\end{table}
For one part, a conditional relation can be expressed by the juxtaposition of clauses and a prosodic break at the margin of the first clause (indicated by commas). In such sentences, the order of clauses is iconical; the \textsc{if-}clause(s) come(s) first, as in \REF{ex:key:1520}:


\ea%1520
    \label{ex:key:1520}
    \gll Yu  kɔmɔ́t  dɔ́n,    yu  wánt  ɛ́nta    mɔ́,    yu  gɛ́fɔ    gó  pé
ɔ́da    quinientos.\\
\textsc{2sg}  go.out  down  \textsc{2sg}  want  enter  more  \textsc{2sg}  have.to  go  pay
other  five.hundred\\

\glt ‘(If) you come out from below and you want to enter again, you have to 
go pay five hundred again.’ [f203fp 005]
\z

Secondly, a conditional relation may be signalled overtly through the use of the equative preposition and clause linker \textit{lɛk (sé)} \{like \textsc{quot}\} ‘as if, supposing that’ \REF{ex:key:1521}. The use of \textit{lɛk (sé)} is not attested with counterfactuals:


\ea%1521
    \label{ex:key:1521}
    \gll \textbf{Lɛk}  \textbf{sé}    yu  de    dríng  nɔ́,  dán    pɔ́sin  wé   dé    yandá,  
e    de  kán    sube    wí    wet    glás,    na  di  tín  
wé    mék    mék    yu  nó  dríng  nó  nátin  wet    glas.\\
like  \textsc{quot}    \textsc{2sg}  \textsc{ipfv}    drink  \textsc{intj}  that    person  \textsc{sub}  \textsc{be.loc}  yonder
\textsc{3sg.sbj}  \textsc{ipfv}  come  go.up  \textsc{1pl.indp}  with    glass   \textsc{foc}  \textsc{def}  thing
\textsc{sub}    make  \textsc{sbjv}    \textsc{2sg}  \textsc{neg}  drink   \textsc{neg}  nothing  with    glass\\

\glt ‘Supposing that you were (out) drinking, right, (and) that person who
is over there comes up to us with a glass, that’s what would make you
not drink anything from a glass.’ [ed03sb 097]
\z

The linker \textit{lɛk} may also introduce the \textsc{then-}clauses of conditional sentences. In the few cases attested, the \textsc{if-}clause is then always explicitly marked by the conditional clause introducer \textit{ɛf} or \textit{if}.\textit{} This constellation renders a form of bipartite and discontinuous conditional clause marking. Compare the following sentence:


\ea%1522
    \label{ex:key:1522}
    \gll \textbf{Ɛf}  yu  bin  bigín  lás  wík,    \textbf{lɛk}  yu  dɔ́n  fínis    di  wók.\\
if  \textsc{2sg}  \textsc{pst}  begin  last  week  like  \textsc{2sg}  \textsc{prf}  finish  \textsc{def}  work\\

\glt ‘If you had begun last week, you would have finished the work.’ [ro05de 029]
\z

The third way of expressing a conditional relation is the most frequent one in the data and involves one of the conditional clause linkers \textit{ɛf, ɛfɛ}\textit{,} or\textit{ if} ‘if’. These forms are equivalent in meaning and occur in free variation. However, \textit{ɛf} is the most frequent form. Any of these linkers may introduce the \textsc{if-}clause of reality, potential, and counterfactual conditionals. Sentence \REF{ex:key:1523} is a reality conditional: 


\ea%1523
    \label{ex:key:1523}
    \gll Pero    \textbf{ɛf}  na  húman  na  bíg  húman  yu  mán  
nó  de  tɔ́n  bíg  mán.\\
but    if  \textsc{foc}  woman  \textsc{foc}  big  woman  \textsc{2sg}  man  
\textsc{neg}  \textsc{ipfv}  turn  big  man\\

\glt ‘But if it’s the wife who’s an influential woman, your [her] husband
doesn’t [automatically] turn into an influential man.’ [ma03hm 079]
\z

Sentence \REF{ex:key:1524} features a potential conditional relation. The most common type of potential conditional features factative\is{factative TMA} TMA in the \textsc{if-}clause, while the \textsc{then-}clause features the potential marker \textit{go}. Sometimes, the imperfective marker\is{imperfective aspect} \textit{de} ‘\textsc{ipfv’} comes to mark conditional modality in the \textsc{then-}clause instead of \textit{go} \textsc{‘pot’} (cf. e.g. \ref{ex:key:1528})


\ea%1524
    \label{ex:key:1524}
    \gll \textbf{Ɛf}  yu  \textbf{chɔ́p}  ɔ́l  dís  chɔ́p  wé  e    nó  dɔ́n,
tumɔ́ro    yu  \textbf{go}  sík.\\
if  \textsc{2sg}  eat    all  this  food    \textsc{sub}  \textsc{3sg.sbj}  \textsc{neg}  done
tomorrow  \textsc{2sg}  \textsc{pot}  sick\\

\glt ‘If you eat/ate all this food that is not done, you’ll/’d be
sick tomorrow.’ [ro05ee 045]
\z

The markers \textit{go} ‘\textsc{pot}’ \REF{ex:key:1525} and \textit{de} ‘\textsc{ipfv’} \REF{ex:key:1526} are also found to mark conditional modality in hypothetical statements contingent upon inferred conditions. The two following sentences are not preceded by an overt \textsc{if-}clause. The “condition” is deduced from context: \is{potential mood}


\ea%1525
    \label{ex:key:1525}
    \gll Mí    nó  \textbf{go}  tɛ́l=an    nó  nátín.\\
\textsc{1sg.indp}  \textsc{neg}  \textsc{pot}  tell=\textsc{3sg.obj}  \textsc{neg}  nothing\\

\glt ‘I \textsc{[emp]} wouldn’t tell him anything.’ [bo03cb 138]
\z


\ea%1526
    \label{ex:key:1526}
    \gll Nóto  mí    a    \textbf{de}  ɛ́nta    ínsay  dán  hós    ó.\\
\textsc{neg}.\textsc{foc}  \textsc{1sg.indp}  \textsc{1sg.sbj}  \textsc{ipfv}  enter  inside  that  house  \textsc{sp}\\

\glt ‘It’s not me who would enter that [haunted] house.’ [ne05fn 031]
\z

Although the verb in the \textsc{if-}clause of potential conditionals usually apears with factative TMA, a minority of conditionals also feature \textit{go} ‘\textsc{pot}’ or \textit{de} ‘\textsc{ipfv}’ in the \textsc{if-}clause and in the \textsc{then-}clause, as in \REF{ex:key:1527} and \REF{ex:key:1528}. I interpret this use as instances of modal harmony between the two hypothetical situations:


\ea%1527
    \label{ex:key:1527}
    \gll \textbf{Ɛf}  dɛn  \textbf{go}  gó  bɛ́r    yú,    dɛn  sén\textbf{  }go  gó  na  dán  bɛ́rin.\\
if  \textsc{3pl}  \textsc{pot}  go  bury  \textsc{2sg.indp}  \textsc{3pl}  \textsc{emp}  \textsc{pot}  go  \textsc{loc}  that  burial\\

\glt ‘If they go to bury you, they themselves will go to that burial.’ [ed03sb 102]
\z


\ea%1528
    \label{ex:key:1528}
    \gll \textbf{If}  yu  nó  \textbf{de}    gí  mí    yu  fɔ́s  mán
wé  na  in    gí  yú    dí  bɛlɛ́,    yu  de  gí
mi    di  pikín  wé  de  kɔmɔ́t.\\
If  \textsc{2sg}  \textsc{neg}  \textsc{ipfv}    give  \textsc{1sg.indp}  \textsc{2sg}  first  man
\textsc{sub}  \textsc{foc}  \textsc{3sg.indp}  give  \textsc{2sg.indp}  this  belly  \textsc{2sg}  \textsc{ipfv}  give
\textsc{1sg.indp}  \textsc{def}  child  \textsc{sub}  \textsc{ipfv}  come.out\\

\glt 
\textit{Lit}. ‘But the price that you have to pay (is), if you don’t give me your first man
who it is him who gave you the first pregnancy, you will give me the child that
will come out.’ [ed03sb 020]
\z

Counterfactual conditionals feature the past marker\is{past tense} \textit{bin} in the \textsc{if-}clause. In the \textsc{then-}clause, we either find the marker sequence \textit{bin fɔ} ‘\textsc{pst} \textsc{cond’}\textit{} \REF{ex:key:1529} or the conditional mood marker \textit{fɔ} ‘\textsc{cond’} alone \REF{ex:key:1530} irrespective of past or present tense reference of the situation. Also note the occurrence of potential mood marking in the complement clause introduced by \textit{sé} ‘\textsc{quot}’ in \REF{ex:key:1530}:


\ea%1529
    \label{ex:key:1529}
    \gll \textbf{Ɛf}  a    \textbf{bin}  sí=an    yɛ́stadé    a    \textbf{bin}
\textbf{fɔ} gí=an    di  mɔní.\\
if  \textsc{1sg.sbj}  \textsc{pst}  see=\textsc{3sg.obj}  yesterday  \textsc{1sg.sbj}  \textsc{pst}
\textsc{cond}    give=\textsc{3sg.obj}  \textsc{def}  money\\

\glt ‘If I had seen her yesterday, I would have given her the money.’ [ro05de 028]
\z


\ea%1530
    \label{ex:key:1530}
    \gll \textbf{Ɛf}  a    \textbf{bin}  nó    sé    e    nó  \textbf{go} fɔ́l 
a    \textbf{fɔ} bríng  ɔ́da    sús.\\
if  \textsc{1sg.sbj}  \textsc{pst}  know  \textsc{quot}    \textsc{3sg.sbj}  \textsc{neg}  \textsc{pot}  rain  
\textsc{1sg.sbj}  \textsc{cond}    bring  other  shoe\\

\glt ‘If I had known that it wouldn’t rain, I would have worn
other shoes.’ [ma03hm 025]
\z

The marker(s) \textit{(bin) fɔ} are also encountered in counterfactual statements contingent upon inferred conditions \REF{ex:key:1531}. Sentence \REF{ex:key:1532} illustrates that \textit{fɔ} may fulfil the latter function by itself, without explicit tense marking by \textit{bin}, if a past tense temporal frame has been set by prior discourse: 


\ea%1531
    \label{ex:key:1531}
    \gll E    \textbf{bin}    \textbf{fɔ}   dé    fáyn.\\
\textsc{3sg.sbj}  \textsc{pst}    \textsc{cond}    \textsc{be.loc}  fine\\

\glt ‘It would have been nice.’ [fr03ft 172]
\z


\ea%1532
    \label{ex:key:1532}
    \gll Yu  \textbf{fɔ}    gɛ́t  hemorragia  sóté    blɔ́d    fínis    náw.\\
\textsc{2sg}  \textsc{cond}    get  hemorrhage  until  blood  finish  now\\

\glt ‘You would have hemorrhaged until your blood would have finished.’ [ab03ay 094]
\z

In the vast majority of cases, the \textsc{if-}clause precedes the \textsc{then-}clause in Pichi conditionals. Nevertheless, the corpus contains a few instances of initial \textsc{then-}clauses \REF{ex:key:1533}. These types of conditionals are pragmatically marked and usually involve focus of the preposed \textsc{then-}clause. This example is also of interest, because it reflects some of the residual obligation meaning that the preposition \textit{cum} modal particle \textit{fɔ} may have in counterfactual conditionals (cf. also \sectref{sec:6.7.3.2}): 


\ea%1533
    \label{ex:key:1533}
    \gll A    \textbf{bin}  \textbf{fɔ} máred  a  los    veinti-uno  \textbf{ɛf}  Maura
in    papa  nó  \textbf{bin}  dáy.\\
\textsc{1sg.sbj}  \textsc{pst}  \textsc{cond}    marry  at  \textsc{def.pl}  twenty-one  if  \textsc{name}
\textsc{3sg.poss}  father  \textsc{neg}  \textsc{pst}  die\\

\glt ‘I should/would have married at twenty-one if Maura’s father hadn’t died.’ [ab03ab 210]\is{conditional clauses}
\z

\subsection{Concessive clauses}\label{sec:10.7.12}

Concessive meaning may be expressed by clauses introduced by \textit{wé} ‘\textsc{sub}’ (cf. \ref{ex:key:1464}) and \textit{sé} ‘\textsc{quot}’ (cf. \ref{ex:key:1470}). Alternatively, concessive meaning may be expressed through conditional clauses in conjunction with clausal focus by means of the focus particle \textit{sɛ́f} ‘\textsc{emp}’, or by way of the \ili{Spanish}-derived clause linker \textit{aunque} ‘although’. In \REF{ex:key:1534}, the conditional relation is not signalled overtly. The presence of the focus particle \textit{sɛ́f} ‘\textsc{emp}’ alone is sufficient to signal concession: 


\ea%1534
    \label{ex:key:1534}
    \gll Yu  nó    \textbf{sɛ́f},  yu  jɔ́s  kán    yu  nó  go  sabí,
yu  nó  go  tɔ́k  lɛkɛ    dɛ́n.\\
\textsc{2sg}  know  \textsc{emp}  \textsc{2sg}  just  come  \textsc{2sg}  \textsc{neg}  \textsc{pot}  know
\textsc{2sg}  \textsc{neg}  \textsc{pot}  talk  like    \textsc{3pl.indp}\\

\glt ‘Even if you know, if you have just come, you wouldn’t know, 
you wouldn’t talk like them.’ [ma03hm 044]
\z

A concessive clause may also be introduced by the linkers \textit{ɛf(ɛ)} and \textit{if}, just like a conditional clause. TMA marking is also the same as in conditional clauses:


\ea%1535
    \label{ex:key:1535}
    \gll \textbf{Ɛf}  yú    na  smɔ́l  húman  \textbf{sɛ́f},   dɛn  go  kɔ́l  yú    dama.\\
if  \textsc{2sg.indp}  \textsc{foc}  small  woman  \textsc{emp}  \textsc{3pl}  \textsc{pot}  call  \textsc{2sg.indp}  lady\\

\glt ‘Even if you [\textsc{emp}] are an insignificant woman, they’ll call you lady.’ [ma03hm 076]
\z

Concessive clauses are sometimes also introduced by the Spanish clause linker \textit{aunque} ‘although’ \REF{ex:key:1536}:


\ea%1536
    \label{ex:key:1536}
    \gll \textbf{Aunque}    nóto  paludismo,  if  dɛn  gív  tratamiento  yu  nó  go  dáy.\\
although    \textsc{neg}.\textsc{foc}  malaria    if  \textsc{3pl}  give  treatment  \textsc{2sg}  \textsc{neg}  \textsc{pot}  die\\

\glt ‘Even if it is not malaria, if they give you a treatment, you won’t die [of the treatment].’ [fr03wt 061]
\z

The linker \textit{adɔnkɛ́} ‘no matter if’ also introduces concessive clauses. \textit{Adɔnkɛ́} is often part of a disjoint structure, namely \textit{adɔnkɛ́} — \textit{wáns}, ‘even if — once’. The concessive clause is introduced by the first, and the main clause by the second element \REF{ex:key:1537}: 


\ea%1537
    \label{ex:key:1537}
    \gll \textbf{Adɔnkɛ́}  e    nó  sí  yú    wán  hól    dé,  e    nó  bísin,
\textbf{wáns}  yu  bríng  di  pamáyn.\\
even.if  \textsc{3sg.sbj}  \textsc{neg}  see  \textsc{2sg.indp}  one  whole  day  \textsc{3sg.sbj}  \textsc{neg}  be.busy
once  \textsc{2sg}  bring  \textsc{def}  oil\\

\glt ‘Even if she didn’t see you the whole day, she didn’t care, if only you 
brought the oil.’ [ab03ab 036]
\z

\section{Intonation}\label{sec:10.8}

Continuative intonation accompanies various types of clause linkage (cf. also \sectref{sec:3.4.4}). For example, it may be found at the boundary between coordinate clauses and the main and subordinate clauses in conditionals. Continuative intonation also occurs on its own without any other linker to signal a relation between adjacent clauses. By definition, serial verb construction do not, however, involve continuative intonation. They form single prosodic units. The main and subordinate clauses of relative constructions are not normally linked by continuative intonation either.


The deictic manner adverb \textit{só} ‘like that’ in the example below bears a continuative boundary tone. Such a non-final intonation at the boundary of the first clause signals that it is linked with the subsequent one. The nature of the relation between the clauses is determined by context. In this case, a cause relation \is{cause clauses}reading is favoured:



\ea%1538
    \label{ex:key:1538}
    \gll Bɔkú  motó  dɛn  dé    yá    \textbf{só,}    a    nó  nó    sé
Pancho  mék    lɛkɛ    sé    e    de  sube    bihɛ́n
wí    e    baja      mɔ́.\\
much  car    \textsc{pl}  \textsc{be.loc}  here    like.that  \textsc{1sg.sbj}  \textsc{neg}  know  \textsc{quot}
\textsc{name}  make  like    \textsc{quot}    \textsc{3sg.sbj}  \textsc{ipfv}  go.up  behind
\textsc{1pl.indp}  \textsc{3sg.sbj}  go.down    more\\

\glt ‘(Because) a lot of cars were just there, I didn’t know that Pancho pretended to 
go up behind us and went down again.’ [ye03cd 176]
\z

Conditional relations are also frequently signalled by means of continuative intonation alone instead of clause linkers \REF{ex:key:1539}:


\ea%1539
    \label{ex:key:1539}
    \gll Yu  mék=an    in    \textbf{fray-rɛ́s},    in    banána  \textbf{dé}, 
e    go  chɔ́p=an.\\
\textsc{2sg}  make=\textsc{3sg.obj}  \textsc{3sg.poss}  fry.\textsc{cpd}{}-rice  \textsc{3sg.poss}  banana  there
\textsc{3sg.sbj}  \textsc{pot}  eat=\textsc{3sg.obj}\\

\glt ‘(if/when) you make him his fried rice (and) his banana, he will eat it.’ [ro05rt 059]
\z

