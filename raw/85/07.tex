\chapter{The clause}

There are four types of basic, non-complex clause structures in Pichi. Pragmatically marked structures that cut across these four types include negative constructions, questions, as well as focus and topic constructions. The expression of \textsc{being} and \textsc{having} involves a network of functionally overlapping copula and existential verbs, and verbs of possession\is{possession verbs}. Pichi adverbs modify verbs and clauses. The majority of adverbs occupy a clause-initial or a clause-final position, but a small set of time and degree adverbs are also found in preverbal position in the company of TMA markers.

\section{Clause structure}\label{sec:7.1}

Four types of clauses can be distinguished by their basic order, as well as the presence and type of the core constituents verb, subject and object: verbal clauses, serial verb clauses, copula clauses and directive clauses. 

\subsection{Verbal clauses}\label{sec:7.1.1}

The order of constituents in verbal clauses corresponds to the pattern presented in \figref{fig:key:7.1}. Details on the structure of the noun phrase and the predicate are provided in \figref{fig:key:5.1} and \figref{fig:key:6.1} respectively. A few observations on \figref{fig:key:7.1} follow: Subject NPs (\textsc{sbj} \textsc{np}) may be picked up by a resumptive personal pronoun (\textsc{pro}). They may hence co-occur in the same clause, but such structures involve topicalisation and are therefore pragmatically marked (hence the separation of \textsc{sbj} \textsc{np} and \textsc{pro} with a slash). There are several adverbial slots in a clause, details on the positions of adverb(ials) are covered in detail in \sectref{sec:7.7}. 


Pichi has double object constructions marked by constituent order. The first object NP slot (\textsc{obj} \textsc{np}) is reserved for recipient or beneficiary objects, the second for theme or patient objects (for details, see \tabref{tab:key:9.10}). There are a clause-initial and a clause-final slot for interjections. The latter may be filled, among other elements, by the sentence-final modal particle and interjection \textit{ó} \textsc{‘sp’} (cf. \sectref{sec:12.2.5}).


\begin{figure}
\caption{Constituent order in verbal clauses}
\label{fig:key:7.1}

\begin{tabularx}{\textwidth}{XXXXXXXXXXX}
\toprule
\textsc{intj} & \textsc{adv} & \textsc{sbj} \textsc{np/} \textsc{pro} & \textsc{neg} & \textsc{tma} & \textsc{adv} & \textsc{verb} & \textsc{obj} \textsc{np} & \textsc{adv} & \textsc{obj} \textsc{np} & \textsc{intj}\\
\bottomrule
\end{tabularx}
\end{figure}
Pichi has a subject-verb word order\is{word order} in intransitive clauses (cf. \ref{ex:key:515} below), and a subject-verb-object order in transitive clauses \REF{ex:key:513}: 


\ea%513
    \label{ex:key:513}
    \gll E    sɛ́n    di  bɔ́l.\\
\textsc{3sg.sbj}  send  \textsc{def}  ball\\

\glt ‘She threw the ball.’ [ra07se 203]
\z

Objects\is{objects} follow the verb. In most double-object constructions\is{double-object constructions}, the primary object with the semantic role of recipient\is{recipient} or beneficiary\is{beneficiary} is found immediately to the right of the verb. The secondary object encodes the theme\is{theme} or patient\is{patient} and follows the primary object:


\ea%514
    \label{ex:key:514}
    \gll A    sé    “nó  gí=\textbf{an}    \textbf{leche},  gí=an    wɔtá”.\\
\textsc{1sg.sbj}  \textsc{quot}    \phantom{“}\textsc{neg}  give=\textsc{3sg.obj}  milk    give=\textsc{3sg.obj}  water\\

\glt ‘I said “don’t give him milk, give him water”.' [ab03ab 099]
\z

Full nouns occur on their own as subjects. But a coreferential dependent pronoun may additionally occur in the clause which picks up the definite subject. Such structures may be seen to involve topicalisation by dislocation (cf. \sectref{sec:7.5.1}). Example \REF{ex:key:515} features both alternatives:


\ea%515
    \label{ex:key:515}
    \gll Di  chía    blák,  di  \textbf{chía},  \textbf{e}    blák.\\
\textsc{def}  chair  be.black  \textsc{def}  chair  \textsc{3sg.sbj}  be.black\\

\glt ‘The chair is black, the chair (it) is black.’ [dj05ae 121]
\z

Pronoun resumption is also found with objects{\fff}. The following two examples illustrate the use of pronominal copying with fronted and topical object NPs. In \REF{ex:key:516}, the full NP \textit{dán mán} ‘that man’ and in \REF{ex:key:517} the emphatic \textsc{3pl} pronoun \textit{dɛ́n} are set off from the rest of the clause by an intonation break and resumed by object pronouns: 


\ea%516
    \label{ex:key:516}
    \gll \textbf{Dán}    \textbf{mán},  a    dɔ́n  sí=\textbf{an}    sɛ́f.\\
that    man    \textsc{1sg.sbj}  \textsc{prf}  see=\textsc{3sg.obj}  \textsc{emp}\\

\glt ‘That man, I have even seen him.’ [ch07fn 236]
\z


\ea%517
    \label{ex:key:517}
    \gll \textbf{Dɛ́n},    a    nó  de  pút  \textbf{dɛ́n}    ínsay.\\
\textsc{3pl.indp}  \textsc{1sg.sbj}  \textsc{neg}  \textsc{ipfv}  put  \textsc{3pl.indp}  inside\\

\glt ‘As for them, I don’t put them inside.’ [dj03do 006]
\z

An indication that subject pronoun copying may also involve a topic-comment structure comes from examples such as \REF{ex:key:518}. This sentence features the independent, emphatic personal pronoun \textit{dɛ́n} at the beginning of the clause, followed by a coreferential dependent pronoun:


\ea%518
    \label{ex:key:518}
    \gll \textbf{Dɛ́n},    \textbf{dɛn}  bin  de,  dɛn  bin  dɔ́n  sabí    \op...\cp{}\\
\textsc{3pl.indp}  \textsc{3pl}  \textsc{pst}  \textsc{ipfv}  \textsc{3pl}  \textsc{pst}  \textsc{prf}  know\\

\glt ‘As for them, they were, they had already found out (...)’ [ma03hm 037]
\z

Constructions like \REF{ex:key:518}, in which a personal pronoun is fronted for focus or emphasis\is{emphasis} and immediately followed by a resumptive dependent personal pronoun, are, however, rare. Instead, emphatic personal pronouns appear more often on their own. This pattern suggests that subject pronoun copying is pragmatically less marked than object pronoun copying as encountered in \REF{ex:key:516} and \REF{ex:key:517}. This observation fits with the high frequency of resumptive pronoun\is{resumptive pronouns} usage in the relativised position of subject relative clause\is{relative clauses}s as compared to the lower frequency in object relative clauses (cf. \sectref{sec:10.6.2}):


\ea%519
    \label{ex:key:519}
    \gll \textbf{Mí}    dɔ́n  sɔ́fa.\\
\textsc{1sg.indp}  \textsc{prf}  suffer\\

\glt ‘I [\textsc{emp}] have suffered.’ [ab03ab 037]
\z

Quotative clauses\is{quotative clauses} introduced by the quotative marker \textit{sé} ‘\textsc{quot}’ can be found in the syntactic position of the subject or object. A clause introduced by \textit{sé} may also occupy the clause-initial or clause-final adverbial position. Consider the two alternative translations of the following sentence. The first translation renders the function of a quotative complement clause, the second that of an adverbial cause clause: \is{cause clauses}


\ea%520
    \label{ex:key:520}
    \gll A    dɔ́n  de  gládin  \textbf{sé}    \textbf{a}     \textbf{dɔ́n}   \textbf{gó}.\\
\textsc{1sg.sbj}  \textsc{prf}  \textsc{ipfv}  be.glad  \textsc{quot}    \textsc{1sg.sbj}  \textsc{prf}  go\\

\glt ‘I was already glad that I was gone.’ or 
‘I was already glad because I was gone.’ [ab03ay 091]
\z

In the predicate, the negator \textit{nó}, \textsc{TMA} markers, and preverbal adverbs\is{preverbal adverbs} occur before the verb, in this order. The clitic \textsc{3sg.obj} pronoun \textit{=an} immediately follows the verb. Apart from the negator \textit{nó} ‘\textsc{neg}’ and \textsc{TMA} markers, the adverbs of degree \textit{tú} ‘too (much)’, \textit{tú (mɔ́ch)} ‘too (much)', \textit{só} ‘so (much)’, as well as the temporal adverbs\textit{ jís}\textit{\textup{/}}\textit{jɔ́s} ‘just’ and \textit{stíl} ‘still’ are the only elements that may appear between a subject pronoun or NP and the verb. 


In \REF{ex:key:521}, \textit{tú} ‘too (much)’ occurs before the stative verb \textit{évi} ‘be heavy’. In \REF{ex:key:522}, \textit{tú} appears before the locative-existential copula \textit{dé}: 



\ea%521
    \label{ex:key:521}
    \gll Di  bɔ́ks    e    \textbf{tú}  \textbf{évi}.\\
\textsc{def}  box    \textsc{3sg.sbj}  too  be.heavy\\

\glt ‘The box (it) is too heavy.’ [dj05ae 143]
\z


\ea%522
    \label{ex:key:522}
    \gll \textsc{D}i  strít    \textbf{tú}  \textbf{dé}    wɔwɔ́.\\
\textsc{def}  street  too  \textsc{be.loc}  ugly\\

\glt ‘The street is too messed up.’ [dj05ae 135]
\z

Other adverbs and adverbials are usually found at the clause margins. Compare the clause-final degree adverb \textit{smɔ́l} ‘a bit’ (< ‘(be) small’) in \REF{ex:key:523}:


\ea%523
    \label{ex:key:523}
    \gll Djunais  dɔ́n  dríng  \textbf{smɔ́l}.\\
\textsc{name}  \textsc{prf}  drink  a.bit\\

\glt ‘Djunais has drunk a bit [of alcohol].’ [fr03wt 182]
\z

\subsection{Copula clauses}

Two types of copula clauses should be distinguished. Equative clauses feature the focus markers \textit{na} \textsc{‘foc’} and \textit{nóto} \textsc{‘neg.foc’} in a copula function. I analyse \textit{na}{}-copula clauses as grammaticalised topic-comment structures, in which the notional subject is topicalised, and the nominal functioning as the copula complement is under focus. These clauses differ from verbal clauses and predicate adjective clauses involving the copula \textit{dé} ‘\textsc{be.loc}’ in two ways: Pronominal subjects are always from the emphatic series \REF{ex:key:524}, and more often than not, the \textsc{3sg} and \textsc{3pl} pronouns remain unexpressed \REF{ex:key:525} because \textit{na} and \textit{nóto} incorporate \textsc{3sg} reference by default:


\ea%524
    \label{ex:key:524}
    \gll Mí    \textbf{na}  di chíf    nɔ́.\\
\textsc{1sg.indp}  \textsc{foc}  \textsc{def}  chief  \textsc{intj}\\

\glt ‘I’m the boss, right.’ [dj05ce 176]
\z


\ea%525
    \label{ex:key:525}
    \gll \textbf{Nóto}  mecánico.\\
\textsc{neg}.\textsc{foc}  mechanic\\

\glt ‘(He’s) not a mecanic.’ [dj0502e1 214]
\z

Predicate adjective clauses constitute the second type of copula clause. A small set of property-denoting verbs may also function as predicate adjectives and appear as complements \is{complements}to the locative-existential copula \textit{dé} \textsc{‘be.loc’} \REF{ex:key:526}. Unlike other property items, these adjectives may therefore appear in the same syntactic position as adverbials in this type of copula clause \REF{ex:key:527}: 


\ea%526
    \label{ex:key:526}
    \gll Tidé    di  húman  \textbf{dé}   \textbf{fáyn}.\\
today  \textsc{def}  woman  \textsc{be.loc}  fine\\

\glt ‘Today the woman is fine.’ [dj05ae 153]
\z


\ea%527
    \label{ex:key:527}
    \gll E    \textbf{dé}    \textbf{na} \textbf{grɔ́n}.\\
\textsc{3sg.sbj}  \textsc{be.loc}  \textsc{loc}  ground\\

\glt ‘He is [lying] on the ground.’ [ab03ab 063]
\z

\subsection{Directive clauses}

The syntax of \textsc{2sg} directive (imperative\is{imperatives}) clauses is distinct from other clause types and other directive clauses in that the \textsc{2sg} subject remains unexpressed \REF{ex:key:528}. However, a \textsc{2pl} subject must be overtly expressed \REF{ex:key:529}:


\ea%528
    \label{ex:key:528}
    \gll Nó  \textbf{láf}!\\
\textsc{neg}  laugh\\

\glt ‘Don’t laugh!’ [ru03wt 022]
\z


\ea%529
    \label{ex:key:529}
    \gll \textbf{Una}    \textbf{púl} di  torí!\\
\textsc{2pl}    pull  \textsc{def}  story\\

\glt ‘Tell \textsc{[pl]} the story!’ [fr03wt 018]
\z

Moreover, directives are the only type of main clause that feature a TMA marker in the prenominal rather than the preverbal slot; compare the subjunctive\is{subjunctive mood} marker \textit{mék} ‘\textsc{sbjv}’ \REF{ex:key:530}:\is{deontic modality}


\ea%530
    \label{ex:key:530}
    \gll \textbf{Mék}   a    púl    wán    smɔ́l  torí?\\
\textsc{sbjv}    \textsc{1sg.sbj}  pull    one    small  story\\

\glt ‘Should I tell a little story?’ [au07se 059]
\z

At the same time, directive subjunctive clauses are structurally no different from other clauses that feature a clause linker at their very left. Compare \REF{ex:key:530} with the sequential clause introduced by \textit{wé} ‘\textsc{sub}’ in \REF{ex:key:531}:


\ea%531
    \label{ex:key:531}
    \gll \textbf{Wé}  e    bin  dáy  só.\\
\textsc{sub}  \textsc{3sg.sbj}  \textsc{pst}  die  like.that\\

\glt ‘And he died just like that.’ [ed03sb 126]\is{directives}\is{clause structure}
\z

\section{Negation}\label{sec:7.2}

Pichi negation revolves around the general negator \textit{nó} \textsc{‘neg’,} which functions as a negative particle in verb negation and as a negative quantifier\is{quantifiers} in \textsc{NP} negation. Besides \textit{nó}, Pichi features the negative indefinite pronoun \textit{nátin} ‘nothing’, which is specialised for use in negative clauses. Other than that, Pichi makes use of negative phrases consisting of \textit{nó} and generic nouns that function as negative indefinites and adverbials. Furthermore, clause negation is characterised by negative concord; when the verb is negated, non-specific \textsc{NPs} may also be preceded by \textit{nó} ‘\textsc{neg}’.


Finally, negation of the perfect aspect as well as equative clauses and focus constructions is not achieved by the addition of the negator \textit{nó}. Instead, negation in these environments is suppletive or “asymmetrical” (\citealt[72]{Miestamo2005}). It relies on the use of morphologically distinct elements that incorporate negative polarity as well as the relevant grammatical category.


\subsection{Verb negation}

\tabref{tab:key:7.1} below provides an overview of the forms and structures employed to express verb negation. “Standard negation”, the negation of declarative clauses \citep{Miestamo2005} revolves around the general negator \textit{nó} ‘\textsc{neg’,} see entry 1a in \tabref{tab:key:7.1}. Verb negation involves “symmetric” \citep{Miestamo2005} or “additive” negation \citep{Jungraithmayr1988} with all TMA categories except for perfect tense-aspect and imperative mood. Symmetric negation involves adding the standard negator \textit{nó} without further adjustments to the clause. The negation of perfect tense-aspect is “asymmetric” \citep{Miestamo2005} or “substitutive” \citep{Jungraithmayr1988}, see entry 1b in \tabref{tab:key:7.1}. Negation relies on the use of morphologically distinct elements that incorporate negative polarity as well as the relevant grammatical category. The negation of imperatives is also optionally achieved by means of negative subjunctives and is therefore also asymmetric, see entry 1c in \tabref{tab:key:7.1}). Further, Pichi makes use of bipolar adverbs to express negative quantification and emphasis, see entry 2 in \tabref{tab:key:7.1}. The negation of identity-equative copulas is covered in \sectref{sec:7.6} and constituent negation is treated in \sectref{sec:7.2.4}.

%%please move \begin{table} just above \begin{tabular
\begin{table}
\caption{Overview of verb negation}
\label{tab:key:7.1}

\begin{tabularx}{\textwidth}{lllQ}
\lsptoprule

Type/polarity & Affirmative & Negative & Function/meaning of negative\\
\midrule
1. Verb & \itshape \textup{a. TMA} & \itshape nó \textup{+ TMA} & General negator\\
& \textup{b.} \textit{dɔ́n} \textsc{‘pfv’} & \itshape nɛ́a/nɔ́ba; nó {}-- yét & Negative perfect;\newline ‘not yet’\\
& \itshape \textup{c. Imperative} & \itshape mék {}-- nó & Negative subjunctive\\

\tablevspace 
2. Verb + adverb & \itshape yét \textup{‘yet, still’} & \itshape nó {}-- yét & ‘not yet’\\
& \itshape mɔ́ \textup{‘more’} & \itshape nó {}-- mɔ́ & ‘no more, not again’\\
&  \textit{sɛ́f} ‘even’ & \itshape nó -- sɛ́f & ‘not even’\\
\lspbottomrule
\end{tabularx}
\end{table}
The negation of declarative clauses is symmetrical. They acquire negative polarity when the general negator \textit{nó} is placed before the bare verb or the relevant TMA marker. The position of the negator is canonical. The imperfective-marked verb \textit{gí} ‘give’ in \REF{ex:key:532} is negated in \REF{ex:key:533}. A negative existential clause is presented in \REF{ex:key:534}. Note the appearance of negative concord in the latter example:


\ea%532
    \label{ex:key:532}
    \gll Dɛn  de  \textbf{gí}  dɛ́n    skúl    fɔ  training  centre.\\
\textsc{3pl}  \textsc{ipfv}  give  \textsc{3pl.indp}  school  \textsc{prep}  training  centre\\

\glt ‘They give them classes at a training centre.’ [to03gm 010]
\z


\ea%533
    \label{ex:key:533}
    \gll Dɛn \textbf{nó} de \textbf{  gí}    \textbf{nó}  nátín.\\
\textsc{3pl} \textsc{neg} \textsc{ipfv} give \textsc{neg} nothing\\
\glt ‘They don’t give anything.’ [ed03sp 075]
\z


\ea%534
    \label{ex:key:534}
    \gll Láyf    \textbf{nó}  \textbf{dé}    náw,  wɔ́l    \textbf{nó}  \textbf{dé}.\\
life    \textsc{neg}  \textsc{be.loc}  now    world  \textsc{neg}  \textsc{be.loc}\\

\glt ‘[Nowadays] there is no life, there is no (proper) world.’ [ab03ay 130]
\z

Sentence \REF{ex:key:535} contains both an affirmative and a negative clause in the potential mood. The two subsequent examples present an affirmative clause marked for past tense\is{past tense} and its negative counterpart (\ref{ex:key:536}-\ref{ex:key:537}).


\ea%535
    \label{ex:key:535}
    \gll Ho,  dán  mán    \textbf{go}  dú  vɔ́mit    tidé,    e    \textbf{nó}  \textbf{go}  slíp.\\
\textsc{intj}  that  man    \textsc{pot}  do  vomit    today  \textsc{3sg.sbj}  \textsc{neg}  \textsc{pot}  sleep\\

\glt ‘That man is going to vomit today, he won’t sleep.’ [ye03cd 143]
\z


\ea%536
    \label{ex:key:536}
    \gll E    \textbf{bin} dé  \textbf{} na  jél.\\
\textsc{3sg.sbj}  \textsc{pst}  \textsc{be.loc}  \textsc{loc}  jail\\

\glt ‘He was in jail.’ [ma03sh 017]
\z


\ea%537
    \label{ex:key:537}
    \gll A    \textbf{nó}  \textbf{bin} fít  ték    motó.\\
\textsc{1sg.sbj}  \textsc{neg}  \textsc{pst}  can  take    car\\

\glt ‘I wasn’t able to take a car.’ [ed03sp 077]
\z

Imperatives \REF{ex:key:538} are negated either with a symmetrical structure \REF{ex:key:539} or with an asymmetrical structure involving a negative subjunctive\is{subjunctive mood} clause \REF{ex:key:540}: 


\ea%538
    \label{ex:key:538}
    \gll \textbf{Pás}    na  mákit  mɔ́!\\
pass    \textsc{loc}  market  again\\

\glt ‘Pass by the market again!’ [dj05ce 071]
\z


\ea%539
    \label{ex:key:539}
    \gll Nó,  wi  de  conversa,  \textbf{nó}  \textbf{vɛ́ks}    Djunais!\\
\textsc{neg}  \textsc{1pl}  \textsc{ipfv}  converse    \textsc{neg}  be.angry  \textsc{name}\\

\glt ‘No, we’re (just) conversing, don’t be angry Djunais!’ [ye03cd 094]
\z


\ea%540
    \label{ex:key:540}
    \gll \textbf{Mék}    yu  \textbf{nó}  kán    a  las    cinco.\\
{\textsc{sbjv}  \textsc{1pl}}  \textsc{2sg}  neg  come  at  the.\textsc{pl}  five  \\

\glt ‘Don’t come at five (o’clock).’ [he07fn 276]
\z

The negation of the perfect tense-aspect is asymmetrical. While the affirmative features the marker \textit{dɔ́n} ‘\textsc{prf}’ \REF{ex:key:541}, the negative perfect is formed with a suppletive allomorph, i.e. either of the free variants \textit{nɛ́a} and \textit{nɔ́ba} ‘\textsc{neg.prf}’ \REF{ex:key:542}: 


\ea%541
    \label{ex:key:541}
    \gll Yu  \textbf{dɔ́n}  bɔ́n      fo    pikín,  \op...\cp{}\\
\textsc{2sg}  \textsc{prf}  give.birth  four    child\\

\glt ‘You have given birth to four children, (...)’ [hi03cb 187]
\z


\ea%542
    \label{ex:key:542}
    \gll E    \textbf{nɛ́a}    \textbf{bɔ́n}      pikín.\\
\textsc{3sg.sbj}  \textsc{neg}.\textsc{prf}  give.birth  child\\

\glt ‘She hasn’t given birth to a child yet.’ [fr03ft 139]
\z

The adverbial \textit{yét} ‘still, yet’ may appear with the negative perfect without providing an additional meaning besides stressing the nuance of current relevance inherent to the perfect \REF{ex:key:543}. However, the combination \textit{nó -- yét} ‘not yet’ can also express this nuance of the perfect by itself and thereby function as a \textit{de} \textit{facto} negative perfect marker \REF{ex:key:544}. In an affirmative clause, the adverbial \textit{yét} means ‘yet, still’, as in \REF{ex:key:247} in \sectref{sec:5.2.3}.


\ea%543
    \label{ex:key:543}
    \gll Yu  sísta    e    \textbf{nɔ́ba}  máred  \textbf{yét}?\\
\textsc{2sg}  sister  \textsc{3sg.sbj}  \textsc{neg}.\textsc{prf}  marry  yet\\

\glt ‘Your sister isn’t married yet?’
\z


\ea%544
    \label{ex:key:544}
    \gll E    \textbf{nó}  máred  \textbf{yét}?\\
\textsc{3sg.sbj}  \textsc{neg}  marry  yet\\

\glt ‘She isn’t married yet?’
\z

The two other combinations of verb negation and a clause-final adverbial are \textit{nó -- mɔ́} ‘no more, not again’ and \textit{nó -- sɛ́f} ‘not even’. Compare the affirmative use of \textit{mɔ́} ‘more’ in \REF{ex:key:538} with \REF{ex:key:545} below. 


\ea%545
    \label{ex:key:545}
    \gll Dɛn  \textbf{nó}  go  fláy  na  Bata    \textbf{mɔ́}.\\
\textsc{3pl}  \textsc{neg}  \textsc{pot}  fly  \textsc{loc}  \textsc{place}  more\\

\glt ‘They’re not going to fly to Bata anymore/again.’ [eb07fn 237]
\z

Examples \REF{ex:key:546} and \REF{ex:key:547} present the use of \textit{sɛ́f} ‘self, \textsc{emp}’ in an affirmative and a negative clause, respectively. The negated clause acquires an emphatic negative meaning: 


\ea%546
    \label{ex:key:546}
    \gll Náw    e    dɔ́n  dáy  \textbf{sɛ́f}.\\
now    \textsc{3sg.sbj}  \textsc{prf}  die  \textsc{emp}\\

\glt ‘Now he is even dead.’ [ma03sh 016]
\z


\ea%547
    \label{ex:key:547}
    \gll Ɛ́n,  dɛn  \textbf{nó}  nó    \textbf{sɛ́f}.\\
\textsc{intj}  \textsc{3pl}  \textsc{neg}  know  \textsc{emp}\\

\glt ‘Yes, they don’t even know (at all).’ [hi03cb 119]
\z

\subsection{Negative concord}\label{sec:7.2.2}

Pichi makes use of negative concord. Verbal and constituent negation co-occur in clauses with negative polarity. Negative concord is pragmatically determined, hence non-strict with lexical nouns, where it only renders emphatic meanings. Negative concord is, grammatically determined, hence strict, with negative indefinite pronouns and phrases. In either case, the negated constituent in constructions featuring negative concord is best interpreted as non-specific. 


Pragmatically neutral lexical nouns in subject position are not normally preceded by the general negator \textit{nó} ‘\textsc{neg}’ in negative clauses:


\ea%548
    \label{ex:key:548}
    \gll Fíba    \textbf{nó}  sube    ín.\\
fever  \textsc{neg}  go.up  \textsc{3sg.indp}\\

\glt ‘(The) fever hasn’t risen on him.’ [eb07fn 171]
\z

In \REF{ex:key:549}, the plural subject \textit{mán dɛn} ‘people’ and the singular subject \textit{chɔ́p} ‘food’ are both not preceded by \textit{nó} ‘\textsc{neg}’. The noun \textit{chɔ́p} is the subject of a negative existential clause. Such clauses usually only feature negative concord when extra emphasis\is{emphasis} is desired \REF{ex:key:551}: 


\ea%549
    \label{ex:key:549}
    \gll \textbf{Mán}    \textbf{dɛn}  \textbf{nó}  de  bísin  fɔ  mék    fám    mɔ́,
yu  gó  fɔ  mákit,  \textbf{chɔ́p}  \textbf{nó}  dé.\\
man    \textsc{pl}  \textsc{neg}  \textsc{ipfv}  be.busy  \textsc{prep}  make  farm  more
\textsc{2sg}  go  \textsc{prep}  market  food    \textsc{neg}  \textsc{be.loc}\\

\glt 
‘People don’t care about farming anymore, (if) you go to the market 
there’s no food.’ [ed03sp 053]
\z

Subject \textsc{NPs} may nevertheless be preceded by \textit{nó}. Such negative clauses featuring negative concord have a single negation reading. Negative concord provides a means of adding an emphatic sense to the negative clause. Compare \textit{dɔ́kta} ‘doctor’ in \REF{ex:key:550} and \textit{motó} ‘car’ in \REF{ex:key:551}:


\ea%550
    \label{ex:key:550}
    \gll E    sé    bueno  ás  \textbf{nó}  \textbf{dɔ́kta}  \textbf{nó}  de  kán    sí    \op...\cp{}\\
\textsc{3sg.sbj}  \textsc{quot}    good  as  \textsc{neg}  doctor  \textsc{neg}  \textsc{ipfv}  come  see\\

\glt ‘She said, ok, since no doctor is at all coming to see (...)’ [hi03cb 091]
\z


\ea%551
    \label{ex:key:551}
    \gll \textbf{Nó}  \textbf{motó}  \textbf{nó}  dé    wé  e    smát  lɛk  mi    yón.\\
\textsc{neg}  car    \textsc{neg}  \textsc{be.loc}  \textsc{sub}  \textsc{3sg.sbj}  be.fast  like  \textsc{1sg.poss}  own\\

\glt ‘There is not a single car that is as fast as mine.’ [ro05ee 140]
\z

Object NPs also only feature negative concord when emphasis is intended. Compare the non-emphatic negative clause in \REF{ex:key:552} with \REF{ex:key:553}. The use of  negative concord in \REF{ex:key:553} gives an emphatic meaning to the object \textit{problema} `problem'. Also note the presence of the independent emphatic pronoun \textit{ín} ‘\textsc{3sg.indp}’ \REF{ex:key:553}:


\ea%552
    \label{ex:key:552}
    \gll A    \textbf{nó}  gɛ́t  pamáyn.\\
\textsc{1sg.sbj}  \textsc{neg}  get  oil\\

\glt ‘I don’t have (any) oil.’ [ab03ay 015]
\z


\ea%553
    \label{ex:key:553}
    \gll \'{I}n    go  chɔ́p=an,    e    \textbf{nó}  gɛ́t  \textbf{nó}  problema.\\
\textsc{3sg.indp}  \textsc{pot}  eat=\textsc{3sg.obj}  \textsc{3sg.sbj}  \textsc{neg}  get  \textsc{neg}  problem\\

\glt ‘He [\textsc{emp}] will eat it, he has no problem whatsoever [with this kind of food].’ [ro05rt 066]
\z


Often, emphasis\is{emphasis} comes in combination with other emphatic features, i.e. suprasegmental cues such as increased volume, higher pitch, or reduced speed in the pronunciation of the negator and the negated \textsc{NP,} or the use of emphatic elements. NPs preceded by \textit{nó} in negative clauses can receive an even higher degree of emphasis if the negative quantifier \textit{nó} is followed by the cardinal numeral and indefinite determiner \textit{wán}, as in \REF{ex:key:554} with the object \textit{wɔ́d} ‘word’:


\ea%554
    \label{ex:key:554}
    \gll Sóté    a    \textbf{nó}  tɔ́k  \textbf{nó}  \textbf{wán}    wɔ́d.\\
until  \textsc{1sg.sbj}  \textsc{neg}  talk  \textsc{neg}  one    word\\

\glt ‘Until I didn’t say a single word (anymore).’ [ab03ay 088]
\z

Negative concord is also found in coordinate \textsc{NPs} featuring the negative coordinator pair \textit{ni -- ni}, which is borrowed\is{borrowing} from \ili{Spanish} \REF{ex:key:555}. Spanish employs no negative concord in this particular construction \REF{ex:key:556}: 


\ea%555
    \label{ex:key:555}
    \gll \textbf{Ni}  ín    \textbf{ni}  in    brɔ́da  dɛn  \textbf{nó}  lán.\\
\textsc{neg}  \textsc{3sg.indp}  \textsc{neg}  \textsc{3sg.poss}  brother  \textsc{3pl}  \textsc{neg}  learn\\

\glt ‘Neither he nor his brothers (have) studied.’ [ro05ee 145]
\z


\ea%556
    \label{ex:key:556}
    \gll \textbf{Ni}  él  \textbf{ni}  su  hermano    han    estudiado.\\
\textsc{neg}  he  \textsc{neg}  his  brother    have  studied\\

\glt ‘Neither he nor his brother has studied.’
\z

\subsection{Negative indefinite pronouns and phrases}\label{sec:7.2.3}

While negative concord is exploited for pragmatic purposes with lexical nouns, negative concord is strict, and hence grammatically conditioned with negative indefinite pronouns and negative indefinite phrases. Pichi has a single item that can unequivocally be qualified as a polarity sensitive, monomorphemic negative indefinite pronoun, namely \textit{nátin} ‘nothing’. The expression \textit{nó} \textit{bɔ́di} (\textit{<} ‘\textsc{neg} body’) ‘nobody’ is partly opaque and may therefore be seen as intermediate between negative indefinite pronoun and negative indefinite phrase: Although \textit{nó} \textit{bɔ́di} is segmentable, the noun \textit{bɔ́di} is not used as a generic noun with the meaning ‘person’. The noun \textit{bɔ́di} also only seldom occurs with the meaning ‘body’, the regular term for ‘body’ being \textit{skín.} 


Concepts other than ‘nobody’ and ‘nothing’ are expressed via segmentable and semantically transparent syntactic phrases featuring the negative quantifier \textit{nó} ‘\textsc{neg’} and a generic noun. This mirrors the formation of indefinite phrases, for which there are, however, no non-segmentable exceptions (i.e. \textit{sɔn} \textit{tín} ‘something’, \textit{sɔn} \textit{pɔ́sin} ‘somebody’, see \sectref{sec:5.4.3}. \tabref{tab:key:7.2} lists Pichi negative indefinite pronouns and negative indefinite phrases:


%%please move \begin{table} just above \begin{tabular
\begin{table}
\caption{Negative indefinite pronouns and negative indefinite phrases}
\label{tab:key:7.2}

\begin{tabularx}{\textwidth}{Xlll}
\lsptoprule
Type & Pronoun/phrase & Gloss & Translation\\
\midrule
Thing & \itshape nátin & nothing & ‘nothing’\\
Person & \itshape nó bɔ́di & \textsc{neg} body & ‘nobody’\\
& \itshape nó mán & \textsc{neg} man & \\
& \itshape nó pɔ́sin & \textsc{neg} person & \\
Place & \itshape nó sáy & \textsc{neg} side & ‘no where’\\
& \itshape nó plés & \textsc{neg} place & \\
& \itshape nó pát & \textsc{neg} part & \\
Manner & \itshape nó (káyn) stáyl & \textsc{neg} (kind) manner & ‘no way’\\
& \itshape nó wé & \textsc{neg} way & \\
Time & \itshape nó wán dé & \textsc{neg} one day & ‘never’\\
Kind & \itshape nó káyn & \textsc{neg} kind & ‘no kind’\\
Pronominal & \itshape nó wán & \textsc{neg} one & ‘none, any’\\
\lspbottomrule
\end{tabularx}
\end{table}
In verbal clauses, the negative indefinite pronoun \textit{nátin} must be used with a preceding negative quantifier \textit{nó} ‘\textsc{neg’} as well as with support from verb negation. This holds for both the subject and object position. Since \textit{nátin} is inherently negative, its use in verbal clauses therefore invariably involves the use of double negative concord. Compare the indefinite \textsc{NP} \textit{sɔn tín} ‘something’ \REF{ex:key:557} with the subject and object negative indefinite pronoun \textit{nátin} ‘nothing’ in \REF{ex:key:558} and \REF{ex:key:559} respectively:


\ea%557
    \label{ex:key:557}
    \gll Mí    wánt  aks yú    \textbf{sɔn}    \textbf{tín}.\\
\textsc{1sg.indp}  want  ask  \textsc{2sg.indp}  some  thing\\

\glt ‘I want to ask you something.’ [fr03ab 191]
\z


\ea[*]{%558
    \label{ex:key:558}
    \gll (\textbf{Nó})  \textbf{nátín}  \textbf{nó}  dé    dé.\\
 \textsc{neg}  nothing  \textsc{neg}  \textsc{be.loc}  there\\
\glt Intended: ‘Nothing is there.’
}
\z


\ea%559
    \label{ex:key:559}
    \gll Mí    \textbf{nó}  go  tɛ́l=an    \textbf{nó}  \textbf{nátín}.\\
\textsc{1sg.indp}  \textsc{neg}  \textsc{pot}  tell=\textsc{3sg.obj}  \textsc{neg}  nothing\\

\glt ‘I \textsc{[emp]} wouldn’t tell him anything.’ [bo03cb 138]
\z

In the same vein, the co-occurrence of the negative quantifier \textit{nó} and the negative indefinite pronoun without the simultaneous use of verbal negation is ungrammatical. 


\ea%560
    \label{ex:key:560}
    \gll \textbf{Nó}   \textbf{nátín}  *(\textbf{nó})  dé    dé.\\
\textsc{neg}  nothing  \textsc{neg}    \textsc{be.loc}  there\\

\glt ‘Nothing is there.’
\z

Strict negative concord also applies to all negative indefinite phrases involving generic nouns including \textit{nó} \textit{bɔ́di} ‘nobody’. Since generic nouns are not inherently negative, verbal clauses featuring negative indefinite phrases involve single negative concord: The generic noun is preceded by the negative quantifier \textit{nó}, and the verb is negated. 


\ea%561
    \label{ex:key:561}
    \gll Dís  sɔ́nde  *(\textbf{nó}) \textbf{  bɔ́di}   *(\textbf{nó})  dé    na  strít.\\
this  Sunday  \textsc{neg}    body  \textsc{neg}    \textsc{be.loc}  \textsc{loc}  street\\

\glt ‘This Sunday, nobody is in the streets.’ [ro05ee 136]
\z

The negative indefinite phrase \textit{nó mán} ‘\textsc{neg} man’ = ‘nobody’ is equally common as \textit{nó bɔ́di} ‘nobody’ (\ref{ex:key:561}–\ref{ex:key:562}). The third logical alternative, \textit{nó pɔ́sin} ‘\textsc{neg} person’ = ‘nobody’, is rare in the data: 


\ea%562
    \label{ex:key:562}
    \gll \textbf{Nó}  \textbf{mán}    \textbf{nó}  blánt  yá    mɔ́    sɛ́f.\\
\textsc{neg}  man    \textsc{neg}  reside  here    more  \textsc{emp}\\

\glt ‘Nobody even lives here anymore.’ [ra07fn 064]
\z

The affirmative counterparts of the negative indefinite phrases in (\ref{ex:key:561}–\ref{ex:key:562}) are indefinite (quantifier) phrases involving \textit{pɔ́sin} ‘person’ and \textit{mán} ‘man’, which function as indefinite pronouns:


\ea%563
    \label{ex:key:563}
    \gll \textbf{Pɔ́sin}  go  entiende    bɔt  e    nó  dé    bien.\\
person  \textsc{pot}  understand  but  \textsc{3sg.sbj}  \textsc{neg}  \textsc{be.loc}  good\\

\glt ‘One would understand, but it doesn’t sound good.’ [dj05be 043]
\z


\ea%564
    \label{ex:key:564}
    \gll \textbf{Ɔ́l}  \textbf{mán} kin  lúk=an,    yu  go  sí  wi  nó  go
mít    nó  bɔ́di    na  hós.\\
all  man    \textsc{hab}  look=\textsc{3sg.obj}  \textsc{2sg}  \textsc{pot}  see  \textsc{1pl}  \textsc{neg}  \textsc{pot}
meet  \textsc{neg}  body  \textsc{loc}  house\\

\glt ‘Everybody watches it [the series], you’ll see we won’t 
meet anybody at home.’ [ma03ni 038]
\z

Negative indefinite adverbials are also formed by means of phrasal syntax. The phrase \textit{nó sáy} ‘\textsc{neg} place’ = ‘nowhere’ is the most commonly employed expression to negate existence in a place. Compare the affirmative and negative sentences involving \textit{sáy} ‘side, place’: 


\ea%565
    \label{ex:key:565}
    \gll \textbf{Ɛ́ni}    \textbf{sáy}  wé  pɔ́sin  wánt  sidɔ́n,  dɛn  de  sidɔ́n.\\
every  side  \textsc{sub}  person  want  stay    \textsc{3pl}  \textsc{ipfv}  stay\\

\glt ‘Everywhere/anywhere people want to stay, they stay.’ [ma03hm 042]
\z


\ea%566
    \label{ex:key:566}
    \gll \MakeUppercase{A}   \textbf{nó}  de  gó  \textbf{nó}  \textbf{sáy}.\\
\textsc{1sg.sbj}  \textsc{neg}  \textsc{ipfv}  go  \textsc{neg}  side\\

\glt ‘I’m\textstylePichiglossZchn{ not going anywhere.’} [pa0502e1 209]
\z

The generic noun \textit{sáy} ‘side, place’ can also be used in a more literal sense to denote ‘space, place’. In that case, it is not usually additionally preceded by \textit{nó} in negative clauses unless extra emphasis is intended. Compare the following two examples: 


\ea%567
    \label{ex:key:567}
    \gll \textbf{Sáy}  \textbf{nó}  dé.\\
side  \textsc{neg}  \textsc{be.loc}\\

\glt ‘There is no space [to sit].’ [ra07fn 029]
\z


\ea%568
    \label{ex:key:568}
    \gll \textbf{Sáy}  \textbf{nó}  dé    fɔ  wás    hán?\\
side  \textsc{neg}  \textsc{be.loc}  \textsc{prep}  wash  hand\\

\glt ‘Is there no place to wash (one’s) hands? [ra07fn 138]
\z

The adverbial concept ‘never’ is expressed via the phrase \textit{nó wán dé} ‘\textsc{neg} one day’ \REF{ex:key:570}. Example \REF{ex:key:569} features the equivalent affirmative phrase \textit{ɔ́l tɛ́n} ‘all time’ = ‘always’: 


\ea%569
    \label{ex:key:569}
    \gll Di  húman  \textbf{ɔ́l}  \textbf{tɛ́n}  e    dé    fáyn.\\
\textsc{def}  woman  all  time  \textsc{3sg.sbj}  \textsc{be.loc}  fine\\

\glt ‘The woman is always looking fine.’ [dj05ae 155]
\z


\ea%570
    \label{ex:key:570}
    \gll E    sé    \textbf{nó}  \textbf{wán}  \textbf{dé}  e    \textbf{nó}  go  dú=an    mɔ́.\\
\textsc{3sg.sbj}  \textsc{quot}    \textsc{neg}  one  day  \textsc{3sg.sbj}  \textsc{neg}  \textsc{pot}  do=\textsc{3sg.obj}  more\\

\glt ‘He said he would never do it again.’ [ro05ee 134]\is{generic nouns}
\z

The negative pronominal meaning of ‘none, any’ may be expressed through verb negation and use of the quantifier\is{quantifiers} and indefinite determiner \textit{sɔn} ‘some, a’, which may refer to count and mass nouns alike. The affirmative clause in \REF{ex:key:571} features \textit{sɔn} used as pronominal (cf. also \ref{ex:key:186}–\ref{ex:key:187}). The negative counterpart of \REF{ex:key:571} may simply be a negative clause \REF{ex:key:572}:


\ea%571
    \label{ex:key:571}
    \gll Dán  banána,  a    gí=an    \textbf{sɔn}.\\
that  banana  \textsc{1sg.sbj}  give=\textsc{3sg.obj}  some\\

\glt ‘That banana, I gave him one.’ [ab03ab 096]
\z


\ea%572
    \label{ex:key:572}
    \gll A    \textbf{nó}  gɛ́t  \textbf{sɔn}.\\
\textsc{1sg.sbj}  \textsc{neg}  get  some\\

\glt ‘I don’t have some/any.’ [eb07fn 303]
\z

Alternatively, the negative indefinite phrase \textit{nó wán}, which features the noun substitute \textit{wán} ‘one’ may be employed when the referent is a count noun or an individuated entity \REF{ex:key:573}:


\ea%573
    \label{ex:key:573}
    \gll \textbf{Nó}  \textbf{wán}    \textbf{nó}  lɛ́f    wet    mí.\\
\textsc{neg}  one    \textsc{neg}  remain  with    \textsc{1sg.indp}\\

\glt ‘None (at all) remains with me.’ [ye07fn 018]
\z

The use of \textit{nó wán} in such contexts often has emphatic connotations. Accordingly, the cardinal numeral \textit{wán} also appears between the negator \textit{nó} and a noun in emphatic negative phrases like \REF{ex:key:574} and \REF{ex:key:575}. This usage also corresponds to the use of \textit{wán} as an emphatic indefinite determiner in other contexts (e.g. with nouns under cleft focus in presentatives (cf. \ref{ex:key:179}):


\ea%574
    \label{ex:key:574}
    \gll A    go  tɛ́l=an    sé    a    nó  de  sɛ́l  
\textbf{nó}  teléfono,    \textbf{nó}  \textbf{wán}.\\
\textsc{1sg.sbj}  \textsc{pot}  tell=\textsc{3sg.obj}  \textsc{quot}    \textsc{1sg.sbj}  \textsc{neg}  \textsc{ipfv}  sell  
\textsc{neg}  telephone  \textsc{neg}  one\\

\glt ‘I’ll tell her that I’m not going to sell any telephone, none (at all).’ [lo07he 049]
\z


\ea%575
    \label{ex:key:575}
    \gll \textbf{Nó}  tɔ́k  \textbf{nó}  \textbf{wán}   wɔ́d!\\
\textsc{neg}  talk  \textsc{neg}  one    word\\

\glt ‘Don’t say a single word!’ [ro05ee 142]
\z

The fixed expression \textit{nó wán dé} ‘never’ in \REF{ex:key:570} above is also such an emphatic negative phrase, even if lexicalised.\is{negative phrases} 

\subsection{Constituent negation}\label{sec:7.2.4}

Sections \sectref{sec:7.2.2} and \sectref{sec:7.2.3} have shown that one means of negating nominal constituents is by placing the negator \textit{nó} ‘\textsc{neg}’ before them. However, this kind of constituent negation does not appear independently of verb negation. A second means available for negating a larger range of constituents is the negative cleft focus construction. An overview of constituent negation is given in \tabref{tab:key:7.3}. 

%%please move \begin{table} just above \begin{tabular
\begin{table}
\caption{Constituent negation}
\label{tab:key:7.3}

\begin{tabularx}{\textwidth}{lXXX}
\lsptoprule
Type & Negator & Gloss & Translation\\
\midrule 
Negative concord & \itshape nó & \textsc{neg} & ‘no’\\
Constitutent negation & \itshape nóto & \textsc{neg}.\textsc{foc} & ‘it’s not’\\
\lspbottomrule
\end{tabularx}
\end{table}

Cleft focus provides a means of negating single constituents and is possible with any constituent that may be focused (cf. \sectref{sec:7.4.3.2}). In cleft focus constructions, the focused element is fronted to the sentence-initial position and preceded by the negative focus marker \textit{nóto} ‘\textsc{neg}.\textsc{foc}’. Compare \REF{ex:key:576}, where the subject \textsc{NP} \textit{ɔ́l húman} ‘all women’ is singled out for constituent negation:


\ea%576
    \label{ex:key:576}
    \gll \textbf{Nóto}  \textbf{ɔ́l}  \textbf{húman}  fít  máred.\\
\textsc{neg}.\textsc{foc}  all  woman  can  marry\\

\glt ‘Not all women can get married.’ [ab03ab 196]
\z

Adverbials are negated in the same way as core \textsc{NPs}. Example \REF{ex:key:577} features the negated time adverbial \textit{tidé} ‘today’, \REF{ex:key:578} the reason adverbial \textit{fɔ dán tín}: 


\ea%577
    \label{ex:key:577}
    \gll Ɛ́n,  \textbf{na}  \textbf{tidé}    mí    híɛ.\\
\textsc{intj}  \textsc{foc}  today  \textsc{1sg.indp}  hear\\

\glt ‘Yes, it’s today that I \textsc{[emp]} heard (it).’ [bo03cb 084]
\z


\ea%578
    \label{ex:key:578}
    \gll \textbf{Nóto}  \textbf{fɔ}  \textbf{dán}    \textbf{tín}    yu  de  kráy?\\
\textsc{foc}    \textsc{prep}  that    thing  \textsc{2sg}  \textsc{ipfv}  cry\\

\glt ‘Is it not because of that that you are crying?’ [ne05fn 004]
\z

In \REF{ex:key:579}, the speaker abbreviated as (hi) complains about the discrimination of women in wedlock, a condition she likens to slavery. In the example, speaker (hi) first negates the direct quote\textit{ e fíba} ‘it resembles’, the second \textit{nóto} negates the verbal constituent as such: 


\ea%579
    \label{ex:key:579}
    \gll Ɛhɛ́,    \textbf{nóto}    “e    fíba,”  na  esclavitud,  \textbf{nóto}  “fíba”.\\
\textsc{intj}    \textsc{neg}.\textsc{foc}  \textsc{3sg.sbj}  resemble    \textsc{foc}  slavery    \textsc{neg}.\textsc{foc}  seem\\

\glt ‘Yes, not “it resembles (slavery)”, it’s slavery, not “resemble”.’ [hi03cb 227]
\z

Sentences \REF{ex:key:580} and \REF{ex:key:581} illustrate how yet larger sentence constituents can be singled out for negation. Both examples are negative factive clauses, in which the existence of the situation of the reference clause is negated:


\ea%580
    \label{ex:key:580}
    \gll Ɛf  \textbf{nóto}    \textbf{yu}  \textbf{bay},    dán  húman  go  bít  yú
sóté    yu  go  gó  lɛ́f=an.\\
if  \textsc{neg}.\textsc{foc}  \textsc{2sg}  buy    that  woman  \textsc{pot}  beat  \textsc{2sg.indp}
until  \textsc{2sg}  \textsc{pot}  go  leave=\textsc{3sg.obj}\\

\glt ‘If it wasn’t the case that you had bought (it), that woman would 
beat you until you’d go and leave it there.’ [ab03ab 033]\is{negation}
\z


\ea%581
    \label{ex:key:581}
    \gll \textbf{Nóto}  \textbf{sé}    na  hɔ́s    dɛn  fɔ  fɔ́s  tɛ́n    wé  dɛn  strɔ́n,
e    fɔ  dɔ́n  fɔdɔ́n.\\
\textsc{neg}.\textsc{foc}  \textsc{quot}    \textsc{foc}  house  \textsc{pl}  \textsc{prep}  first time  \textsc{sub}  \textsc{3pl}  be.strong
\textsc{3sg.sbj}  \textsc{prep}  \textsc{prf}  fall\\

\glt ‘(If) it wasn’t the case that they were houses of the past that 
are strong, it would have already collapsed.’ [hi03cb 045]\is{negative cleft constructions}
\z

\section{Questions}\label{sec:7.3}

This section covers yes-no questions, alternative questions, and content questions, as well as answers to questions. It is useful to refer to \sectref{sec:3.4.5} for details on the intonational characteristics of questions. 

\subsection{Yes-no and alternative questions}

Yes-no questions have the syntax of declarative clauses and do not involve obligatory question particles. Yes-no questions are therefore distinguished from declarative clauses by intonation (cf. \sectref{sec:3.4.5}):


\ea%582
    \label{ex:key:582}
    \gll Yu  wánt  de  \textbf{gó}?\\
\textsc{2sg}  want  \textsc{ipfv}  go\\

\glt ‘Do you want to go?’ [eb07fn 202]
\z

However, speakers often employ the interjections \textit{ɛ́n} and \textit{nɔ́} sentence-finally in biased questions in order to channel-check: 


\ea%583
    \label{ex:key:583}
    \gll Yu  nó=an    \textbf{ɛ́n}?\\
\textsc{2sg}  know=\textsc{3sg.obj}  \textsc{intj}\\

\glt ‘You know her, right?’ [li07pe 032]
\z

In alternative questions, the first alternative bears question intonation\is{question intonation}, while the second alternative carries the intonation of a declarative clause: 


\ea%584
    \label{ex:key:584}
\gll
Yu  sísta    stíl  \textbf{máred}  ɔ  e    nó  máred  \textbf{mɔ́}?\\
\textsc{2sg}  sister  still  marry  or  \textsc{3sg.sbj}  \textsc{neg}  marry  more\\

\glt ‘Is your sister still married or is she no more married?’ [ro05ee 050]\is{alternative questions}
\z

\subsection{Content questions}\label{sec:7.3.2}

Content questions are formed by way of a mixed question-word system summarised in \tabref{tab:key:7.4}. Note that I classify the question element \textit{wétin} ‘what’ as monomorphemic although it could alternatively be analysed as bimorphemic (i.e. \textit{wé.tín} = *\textit{wé}.thing). However, *\textit{wé}= does not function as a question particle with any other generic noun, and an etymological relation with \textit{wé} ‘\textsc{sub}’ remains to be proven.\is{generic nouns}

%%please move \begin{table} just above \begin{tabular
\begin{table}
\caption{Question elements (x = questioned noun)\is{question words}}
\label{tab:key:7.4}

\begin{tabularx}{\textwidth}{lXXX}
\lsptoprule

Concept & Monomorphemic & Bimorphemic & Question phrase\\
\midrule
\textsc{who} & \itshape údat & \itshape ús=pɔ́sin; ús=mán & \\
\textsc{what} & \itshape wétin & \itshape ús=tín & \itshape ús=káyn tín \\
\textsc{which} \textsc{x} & \itshape ús=x; wích x &  & \itshape ús=káyn x\\
\textsc{which} \textsc{one} &  & \itshape ús=wán & \\
\textsc{when} &  & \itshape ús=tɛ́n & \itshape fɔ ús=tɛ́n\\
\textsc{where} &  & \itshape ús=sáy; ús=pát & \itshape fɔ ús=sáy\\
\textsc{why} & \itshape fɔséka; háw;\newline wétin & \itshape ús=tín & \itshape fɔ wétin;\newline fɔ ús=tín \\
&  &  & \itshape fɔséka wétin;\newline fɔséka ús=tín \\
&  &  & \itshape wétin mék;\newline ús=tín mék; \\
&  &  & \itshape wet ús=tín \\
\textsc{how} & \itshape háw & \itshape ús=stáyl & \itshape ús=káyn stáyl \\
\textsc{how} \textsc{much/many} &  & \itshape háw mɔ́ch & \\
\textsc{how} \textsc{much/many} \textsc{x} &  &  & \itshape háw mɔ́ch x\\
\lspbottomrule
\end{tabularx}
\end{table}

The question word system of Pichi involves three types of both “transparent” and “opaque question elements” (\citealt{MuyskenSmith1990}): (1) Monomorphemic elements function as question elements or words in their own right. Amongst these, we find the clitic \textit{ús=} ‘\textsc{q}’, which forms (2) bimorphemic question words with generic nouns in order to render basic concepts like \textsc{who,} \textsc{what,} and \textsc{when.}\is{cliticisation}

Question phrases (3) may consist of a prepositional phrase introduced by \textit{fɔ} ‘\textsc{prep}’, \textit{fɔséka} ‘due to’, and \textit{wet} ‘with, due to’ and contain a mono- or bimorphemic question word (e.g. \textit{fɔ wétin} ‘\textsc{prep} what’ = ‘\textsc{why’}). Alternatively, question phrases may consist of idiomatic clauses featuring the verbs \textit{mék} ‘make’ or \textit{dú} ‘do’ and \textit{wétin} or \textit{ús=tín} ‘\textsc{what’} in subject position. A second type of question phrase involves constructions featuring the bimorphemic question word \textit{ús=káyn} and a generic or other noun (e.g. \textit{ús=káyn pɔ́sin} ‘\textsc{q}=kind person’ = ‘\textsc{who’}, \textit{ús=káyn motó} ‘\textsc{q}=kind car’ = ‘\textsc{which} car’). \is{associative constructions}


In \tabref{tab:key:7.4}, \textsc{x} stands for any noun. W\textsc{hich} \textsc{x} and \textsc{how} \textsc{much} \textsc{x} are therefore question noun modifiers and quantifier\is{quantifiers}s, respectively. The table contains all unequivocally accepted question elements and excludes other logically possible but unattested options (e.g. \textit{?ús=plés} ‘\textsc{q}=place’ = ‘\textsc{where’}; \textit{?ús=káyn mán} ‘\textsc{q}=kind man’ = ‘\textsc{who’}).


\subsubsection{Structural issues}

In content questions, any constituent other than the definite article\is{definite article} \textit{di}, focus and topic particles, or TMA markers can be questioned through replacement by a question element. Question words show some distributional restrictions when compared to regular nouns. 


For instance, question elements are not usually modified by demonstratives\is{demonstratives} and deictic adverbials, or modifier nouns and adjectives. Similarly, only \textit{údat} and \textit{ús=pɔ́sin} ‘\textsc{who’} may optionally take the pluraliser\is{pluraliser} \textit{dɛn} (i.e. \textit{*ús=tín dɛn} ‘what \textsc{pl}’): 



\ea%585
    \label{ex:key:585}
    \gll Yu  sí  \textbf{údat}  \textbf{dɛn}?\\
\textsc{2sg}  see  who    \textsc{pl}\\

\glt ‘Who [plural] did you see?’ [sa07fn 267]
\z


\ea%586
    \label{ex:key:586}
    \gll Yu  sí  \textbf{ús=pɔ́sin}    \textbf{dɛn}?\\
\textsc{2sg}  see  \textsc{q}=person    \textsc{pl}\\

\glt ‘Who [plural] did you see?’ [nn07fn 277]
\z

The pluralisation of ‘\textsc{who’} is likely to be a structural borrowing\is{borrowing} from \ili{Spanish}, or is at least reinforced by the equivalent Spanish structure. Compare the equivalent Spanish question:


\ea%587
    \label{ex:key:587}
    \gll Quién-\textbf{es}    son?\\
\textsc{who}{}-\textsc{pl}    are\\

\glt ‘Who are they?’
\z

Question elements also have other distributional characteristics of regular \textsc{NPs}. For example, in the following sentence, \textit{údat} ‘\textsc{who’} is found in the possessor position of a dislocated possessive construction\is{possessive constructions}, which in turn participates in a presentative clause: 


\ea%588
    \label{ex:key:588}
    \gll Na  \textbf{údat}  \textbf{in}    \textbf{búk}    dís?\\
\textsc{foc}  who    \textsc{3sg.poss}  book  this\\

\glt ‘Whose book (is) this?’ [ro05de 055]
\z

Multiple core \REF{ex:key:589} and adverbial \REF{ex:key:590} \textsc{NP}s forming part of coordinate structures may also be questioned. These two examples also show that in principle, a sentence may contain several question elements, even if this is rare in natural speech:


\ea%589
    \label{ex:key:589}
    \gll \textbf{\'{U}dat}  \textbf{wet}    \textbf{wétin}  de  hambɔ́g  yú?\\
who    with    what  \textsc{ipfv}  bother  \textsc{2sg.indp}\\

\glt ‘Who and what is bothering you?’ [ge07fn 299]
\z


\ea%590
    \label{ex:key:590}
    \gll \textbf{\'{U}s=sáy}  \textbf{wet}    \textbf{háw}    yu  de  wás?\\
\textsc{q}=side  with    how    \textsc{2sg}  \textsc{ipfv}  wash\\

\glt ‘Where and how are you washing?’ [dj05ce 182]
\z

Question elements may occur in situ in the original position of the questioned element, or they may be fronted. Questioned subjects naturally occur at the beginning of the clause as shown in \REF{ex:key:589}. They may also optionally be focused in cleft constructions \REF{ex:key:591}\is{cleft constructions}: 


\ea%591
    \label{ex:key:591}
    \gll \textbf{Na}  \textbf{údat}  hambɔ́g  dɛ́n?\\
\textsc{foc}  who    bother  \textsc{3pl.indp}\\

\glt ‘Who bothered them?’ [ro05de 041]
\z

Objects\is{objects} can be questioned in situ (e.g. \textit{údat dɛn} and \textit{ús=pɔ́sin dɛn} in \ref{ex:key:585} and \ref{ex:key:586} above) or be fronted \REF{ex:key:592}. Fronted objects may also optionally be cleft-focused \REF{ex:key:593}:


\ea%592
    \label{ex:key:592}
    \gll \textbf{\'{U}s=tín}    yu  tɔ́k  mɔ́    sɛ́f?\\
{\textsc{q}=thing}  \textsc{2sg}  talk  again  \textsc{emp}\\

\glt ‘What did you say again?’ [dj07ae 344]
\z


\ea%593
    \label{ex:key:593}
    \gll \textbf{Na}  \textbf{ús=káyn}  \textbf{tín}    dɛn  \textbf{ték}    mék    dís,  digamos  dí  bɔ́tul?\\
\textsc{foc}  \textsc{q}=kind  thing  \textsc{3pl}  take    make  this  let’s.say  this  bottle\\

\glt ‘What’s, let’s say this bottle, made of?’ [ye05ce 113]
\z

The objects of prepositions may also be questioned in situ or be fronted. When fronted, either the entire prepositional phrase appears at the beginning of the clause, or the preposition is stranded. However, stranding\is{stranding} in questions is only attested with \textit{fɔ} ‘\textsc{prep}’ \REF{ex:key:594}, \textit{wet} ‘with’ \REF{ex:key:595}, and \textit{pan} ‘on’ \REF{ex:key:596}:


\ea%594
    \label{ex:key:594}
    \gll \textbf{Wétin}  yu  wánt  sabí    \textbf{fɔ}?\\
what  \textsc{2sg}  want  know  \textsc{prep}\\

\glt ‘What do you want to know for?’ [\textstylePichitranslationZchn{ro05de 045}]
\z


\ea%595
    \label{ex:key:595}
    \gll \textbf{\'{U}s=mán}  yu  bin  de  tɔ́k  \textbf{wet}    yɛ́stadé?\\
\textsc{q}=man  \textsc{2sg}  \textsc{pst}  \textsc{ipfv}  talk  with    yesterday\\

\glt ‘Who were you talking with yesterday?’ [ro07fn 215]
\z


\ea%596
    \label{ex:key:596}
    \gll \textbf{\'{U}s=béd}  yu  kin  slíp    \textbf{pan}?\\
\textsc{q}=bed  \textsc{2sg}  \textsc{hab}  sleep  on\\

\glt ‘Which bed do you usually sleep on?’ [ur07fn 238]
\z

All constituents that may be questioned in main clauses can also be replaced by question elements in subordinate clauses. Non-subject constituents of subordinate clauses can be questioned in situ \REF{ex:key:597} or be fronted \REF{ex:key:598}: 


\ea%597
    \label{ex:key:597}
    \gll Yu  tɔ́k  sé    Pancho  de  yús  \textbf{údat}  \textbf{in}    \textbf{motó}?\\
\textsc{2sg}  talk  \textsc{quot}    \textsc{name}  \textsc{ipfv}  use  who    \textsc{3sg.poss}  car\\

\glt ‘You said that Pancho uses whose car?’ [dj05ce 146]
\z


\ea%598
    \label{ex:key:598}
    \gll \textbf{\'{U}s=tín}  yu  tɔ́k  sé    yu  wánt  \textbf{sabí}?\\
\textsc{q}=thing  \textsc{2sg}  talk  \textsc{quot}    \textsc{2sg}  want  know\\

\glt ‘What did you say you wanted to know?’ [dj05ce 132]
\z

Complement or adverbial clauses introduced by \textit{sé} ‘\textsc{quot}’ are questioned like nominal constituents. The question word is, however, always found in situ as in the rhetorical question in \REF{ex:key:599}. Here a cause clause \is{cause clauses}is questioned by means of the phrase \textit{sé wétin} ‘\textsc{quot} what’ = ‘because of what’:


\ea%599
    \label{ex:key:599}
    \gll A    go  púl=an      na  mi    yáy  \textbf{sé}    \textbf{wétin}?\\
\textsc{1sg.sbj}  \textsc{pot}  remove=\textsc{3sg.obj}  \textsc{loc}  \textsc{1sg.poss}  eye  \textsc{quot}    what\\

\glt ‘I would remove it [the pair of sunglasses] from my eyes for what?’ [ye07ga 011]
\z

\subsubsection{Questioning subjects and objects}

Questioned subjects\is{subjects} naturally occur at the beginning of the question clause, as in \REF{ex:key:591} above. Questioned objects\is{objects} appear at the beginning of the sentence \REF{ex:key:600}, or in their original position \REF{ex:key:601}. These two examples feature the question word \textit{wétin} ‘\textsc{what’}, which is used for questioning inanimate entities: 


\ea%600
    \label{ex:key:600}
    \gll \textbf{Wétin}  yu  wánt  \textbf{nó}?\\
what  \textsc{2sg}  want  know\\

\glt ‘What do you want to know?’ [dj05ce 086]
\z


\ea%601
    \label{ex:key:601}
    \gll Yu  wánt  \textbf{nó}    \textbf{wétin}?\\
\textsc{2sg}  want  know  what\\

\glt ‘You want to know what?’ [dj05ce 087]
\z

Example \REF{ex:key:602} illustrates the questioning of a complex object NP. The dislocated possessive construction{\fff} \textit{údat} in \textit{motó} ‘whose car’ is the object of \textit{yús} ‘use’ and under focus with the focus particle \textit{na}. The questioning of a possessor NP is also achieved by circumlocution with the verb \textit{gɛ́t} ‘get, have’ \REF{ex:key:603}. 


Both examples involve the question word \textit{údat} ‘\textsc{who’}, which is used for questioning human referents. In a minority of cases, the concept ‘\textsc{who’} is also expressed by the bimorphemic question words \textit{ús=pɔ́sin} ‘\textsc{q}=person’ (cf. \ref{ex:key:586} above) and \textit{ús=mán} ‘\textsc{q}=man’ (cf. \ref{ex:key:595} above) in all relevant syntactic positions: 



\ea%602
    \label{ex:key:602}
    \gll \textbf{Na}  \textbf{údat}  \textbf{in}    \textbf{motó}  Pancho  de  yús?\\
\textsc{foc}  who    \textsc{3sg.poss}  car    \textsc{name}  \textsc{ipfv}  use\\

\glt ‘It’s whose car Pancho is using?’ [dj05ce 118]
\z


\ea%603
    \label{ex:key:603}
    \gll \textbf{Na}  \textbf{údat}  gɛ́t  dís  búk?\\
\textsc{foc}  who    get  this  book\\

\glt ‘Who possesses this book?’ [ro05de 054]
\z

The clitic question element \textit{ús=}\textsc{q}\textit{} may combine with the pronominal and noun substitute \textit{wán} ‘one’ in order to render the concept ‘\textsc{which} \textsc{one’.} The collocation may be used to selectively question any noun \REF{ex:key:604}. \textit{\'{U}s=wán} is also employed in an idiomatic question clause in order to ask for a person’s name \REF{ex:key:605}. The latter usage is conventionalised and very likely to be a calque\is{borrowing} from the equivalent Spanish phrase \textit{¿cuál es tú nombre?} ‘which (one) is your name’ = ‘what’s your name?’:


\ea%604
    \label{ex:key:604}
    \gll Ɛhɛ́,    dán  wán  min    sé    \textbf{ús=wán}  na  di  escala?\\
exactly  that  one  mean  \textsc{quot}    \textsc{q}=one  \textsc{loc}  \textsc{def}  scale\\

\glt ‘Exactly, that means which one [of the two] is the scale?’ [fr03cd.092]
\z


\ea%605
    \label{ex:key:605}
    \gll \textbf{\'{U}s=wán}  na  in    ném?\\
\textsc{q}=one  \textsc{foc}  \textsc{3sg.poss}  \textsc{name}\\

\glt ‘What’s his name?’ [ko03sp 061]
\z

A similar syntactic flexibility is characteristic of the objects of V2 minor (i.e. closed class) verbs in SVCs. The questioned object of \textit{pás} ‘(sur)pass’ in the comparative SVC{\fff} in \REF{ex:key:606} and the object of \textit{kɛ́r} ‘carry, take’ in the motion-direction SVC in \REF{ex:key:607} may be found in the original syntactic position:


\ea%606
    \label{ex:key:606}
    \gll E    bíg  \textbf{pás}    \textbf{údat}?\\
\textsc{3sg.sbj}  big  pass    who\\

\glt ‘He is bigger than who?’ [ye05ce 119]
\z


\ea%607
    \label{ex:key:607}
    \gll Dɛn  kɛ́r    di  motó  \textbf{gó}  \textbf{ús=sáy}?\\
\textsc{3pl}  carry  \textsc{def}  car    go  where\\

\glt ‘Where did they take the car to?’ [au07fn 239]
\z

Alternatively, the objects of V2 minor verbs may occur in the sentence-initial, fronted position with or without additional cleft focus marking, with the same liberty as other objects. These constructions leave the V2 of the SVC “stranded\is{stranding}” in the sentence-final position. Compare the following two sentences with the two preceding ones above: 


\ea%608
    \label{ex:key:608}
    \gll \textbf{Na}  \textbf{údat}  dí  bɔ́y  bíg  \textbf{pás}?\\
\textsc{foc}  who    this  boy  big  pass\\

\glt ‘Who is this boy bigger than?’ [lo07he 016]
\z


\ea%609
    \label{ex:key:609}
    \gll \textbf{\'{U}s=sáy}  yu  de  kɛ́r    di  motó  \textbf{gó}?\\
{\textsc{q}=side}  \textsc{2sg}  \textsc{ipfv}  carry  \textsc{def}  car    go\\

\glt ‘Where are you taking the car to?’ [lo07he 018]
\z

At the same time, the questioning of the instrument or material objects of \textit{ték} ‘take’ in participant-introducing SVCs is characterised by some idiosyncracies. Firstly, speakers seem to prefer to front the questioned object rather than leave it in the original syntactic position between \textit{ték} ‘take’ and the following major verb (i.e. \textit{bíl} ‘build’ in the following example). Compare \REF{ex:key:610}:


\ea%610
    \label{ex:key:610}
    \gll \textbf{\'{U}s=káyn}  plɛ́nk  dɛn  \textbf{ték}    bíl    di  hós?\\
\textsc{q}=kind  board  \textsc{3pl}  take    build  \textsc{def}  house\\

\glt ‘What (kind of) board did they build the house with?’ [dj05ce 104]
\z

Secondly, we find double marking of the instrument objects of \textit{ték} ‘take’ as a rather regular way of questioning these objects. In \REF{ex:key:611}, the object of \textit{ték} (i.e. \textit{ús=tín} ‘what’) is fronted and focused. The question word and object \textit{ús=tín} is additionally preceded by the instrumental/comitative preposition \textit{wet} ‘with’ as if the corresponding declarative clause had been something ungrammatical like *\textit{dɛn ték wet plɛ́nk bíl di hós} ‘\textsc{3pl} take with board build \textsc{def} house’ = *‘they took with board to build the house’ (cf. also \ref{ex:key:1571}–\ref{ex:key:1572}): 


\ea%611
    \label{ex:key:611}
    \gll Na  \textbf{wet}    \textbf{ús=tín}  dɛn  \textbf{ték}  bíl    di  hós?\\
\textsc{foc}  with    \textsc{q}=thing  \textsc{3pl}  take  build  \textsc{def}  house\\

\glt ‘With what did they build the house?’ [dj07ae 479]
\z

However, fronting\is{fronting} of the patient object of the major (open class) verb in \textit{ték} SVCs is not accepted \REF{ex:key:612}. Patients are usually questioned in situ in their original syntactic position following the major verb \REF{ex:key:613}:


\ea[*]{%612
    \label{ex:key:612}
    \gll \textbf{\'{U}s=káyn}  \textbf{hós}    dɛn  ték  plɛ́nk  \textbf{bíl}?\\
  \textsc{q}=kind    house  \textsc{3pl}  take  board  build\\
\glt Intended: ‘Which (kind of) house did they take board to build?’ [dj07ae 482]
}\z


\ea%613
    \label{ex:key:613}
    \gll Dɛn  \textbf{ték}  stón    \textbf{bíl}   \textbf{ús=káyn}  \textbf{hós}?\\
\textsc{3pl}  take  stone  build  \textsc{q}=kind  house\\

\glt ‘Which house did they build of stone?’\is{objects}
\z

\subsubsection{Questioning modifiers}

Modifiers and demonstratives\is{demonstratives} in \textsc{NPs} are questioned via three question elements: the clitic \textit{ús=} ‘\textsc{q}, \textsc{which’}; the (marginally employed) phonologically independent question word \textit{wích} ‘w\textsc{hich’,} and the bimorphemic question word \textit{ús=káyn} ‘\textsc{q}=kind’. Quantifier\is{quantifiers}s are questioned by means of \textit{háw mɔ́ch} ‘how much’ (cf. \ref{ex:key:636}–\ref{ex:key:638} below). The element \textit{ús=} straddles the boundary of a more functional and a more lexical meaning. Consider the translations of the following two examples, which contrast the rarely used and more lexical \textit{wích} ‘\textsc{which}’ with the high-frequency question particle \textit{ús=} ‘\textsc{q’}:


\ea%614
    \label{ex:key:614}
    \gll \textbf{Wích}  \textbf{mán}  dɛn  bin  kíl  na  kwáta?\\
which  man    \textsc{3pl}  \textsc{pst}  kill  \textsc{loc}  quarter\\

\glt ‘Which man was killed in (our) quarter?’ [ro05de 047]
\z


\ea%615
    \label{ex:key:615}
    \gll \textbf{\'{U}s=mán}  dɛn  kíl  na  kwáta?\\
\textsc{q}=man  \textsc{3pl}  kill  \textsc{loc}  quarter\\

\glt ‘Which man/who was killed in our quarter?’ [ro05de 048]
\z

One indication of the more functional status of \textit{ús=} is its cliticisation in the first place (cf. \sectref{sec:2.6.3}). Secondly, in the majority of instances in the corpus, \textit{ús=} combines with a limited number of generic nouns (e.g. \textit{pɔ́sin} ‘person’, \textit{mán} ‘man, person’, \textit{tín} ‘thing’, \textit{sáy} ‘side, place’, \textit{tɛ́n} ‘time’) in order to form general, basic question words with meanings like who, what, where, and when. Yet, \textit{ús=} is nevertheless also used with the meaning ‘which’ in order to form specific question words questioning modifiers as in the following two examples: 


\ea%616
    \label{ex:key:616}
    \gll \textbf{\'{U}s=nɔ́mba}    yu  gɛ́t  fɔ  dán  móvil?\\
which=number  \textsc{2sg}  get  \textsc{prep}  that  mobile\\

\glt ‘Which number do you have in that mobile phone?’ [ye03cd.129]
\z


\ea%617
    \label{ex:key:617}
    \gll \textbf{\'{U}s=nésɔn}?\\
which=nation\\

\glt ‘Which people [does he belong to]? [eb07fn 090]
\z

However, questions like \REF{ex:key:616} and \REF{ex:key:617} are equally often formed by employing the question word \textit{ús=káyn} ‘\textsc{q}=kind’ instead of \textit{ús=} alone. The meaning of \textit{ús=káyn} therefore also vacillates between a more literal sense, in which the pronominal and generic noun \textit{káyn} ‘kind’ retains its lexical meaning of ‘kind’, and a more functional one, in which the entire question word \textit{ús=káyn} is equivalent to \textit{ús=}, ‘\textsc{q,} \textsc{which’.} This ambiguity in the meaning of \textit{káyn} ‘kind’ is reflected in the translations of the following two examples:\is{cliticisation}


\ea%618
    \label{ex:key:618}
\ea{\label{ex:key:618a}
\gll
E    kin  kúk    súp.\\
  \textsc{3sg.sbj}  \textsc{hab}  cook  soup\\
\glt   ‘He usually cooks soup.’ [dj03cd 086]
}\ex{\label{ex:key:618b}
\gll
\textbf{\'{U}s=káyn}  súp?\\
  \textsc{q}=kind  soup\\
\glt   ‘Which (kind of) soup?’ [fr03cd 087]
}
\z
\z


\ea%619
    \label{ex:key:619}
    \gll Sé    papá  gɔ́d  \textbf{ús=káyn}  trɔ́bul  dís?\\
\textsc{quot}    father  God  \textsc{q}=kind  trouble  this\\

\glt ‘(I said) God, what (kind of) trouble (is) this?’ [ab03ab 082]
\z

The more functional use of \textit{ús=káyn} is more obvious when it precedes a generic noun as in the following two examples. Here, the phrase \textit{ús=káyn} \textit{tín} ‘\textsc{q}=kind thing’ has the same meanings as \textit{wétin} or \textit{ús=tín} ‘\textsc{what’}. Note that \REF{ex:key:620} is a free relative clause\is{relative clauses} and sentence \REF{ex:key:621} an indirect question. The long forms featuring \textit{káyn} ‘kind’ are equally common in this position as are the shorter forms \textit{wétin} and \textit{ús=tín:} 


\ea%620
    \label{ex:key:620}
    \gll \'{A}fta    a    nó  sabí    \textbf{ús=káyn}  tín    kán  pás.\\
then  \textsc{1sg.sbj}  \textsc{neg}  know  \textsc{q}=kind  thing  \textsc{pfv}  pass\\

\glt ‘Then, I don’t know what happened.’ [fr03ft 110]
\z


\ea%621
    \label{ex:key:621}
    \gll Yu  nó    wet    \textbf{ús=káyn}  tín    dɛn  mék    dís  tín?\\
\textsc{2sg}  know  with    \textsc{q}=kind  thing  \textsc{3pl}  make  this  thing\\

\glt ‘Do you know with what this is made?’ [ye05ce 142]
\z

The same, more functional use can be observed when \textit{ús=káyn} precedes the generic noun \textit{stáyl} ‘style, manner’ in order to question an adverbial of manner (cf. \ref{ex:key:629}–\ref{ex:key:630} below). However, \textit{ús=káyn} is not found in conjunction with human-denoting generic nouns like \textit{mán} ‘man’ or \textit{pɔ́sin} ‘person’ with the meaning of ‘\textsc{who’.}

\subsubsection{Questioning adverbials}

Adverbials are questioned through mono- and bimorphemic question words as well as through question phrases. Adverbials of time may be questioned with the question word \textit{ús=tɛ́n} ‘\textsc{q}=time’. This question word is general in its meaning and may question any time unit:


\ea%622
    \label{ex:key:622}
    \gll \textbf{\'{U}s=tɛ́n}  yu  rích?\\
\textsc{q}=time  \textsc{2sg}  arrive\\

\glt ‘When [which time/day/month/year] did you arrive?' [dj05ce 154]
\z

Nevertheless, speakers prefer to question time units specifically by using the logically most likely option as in the following questions involving the time units \textit{dé} ‘day’, \textit{mún} ‘month’, and \textit{hía} ‘year’, respectively: 


\ea%623
    \label{ex:key:623}
    \gll \textbf{\'{U}s=dé}  yu  kán    yá?\\
\textsc{q}=day  \textsc{2sg}  come  here\\

\glt ‘When [on which day] did you come here?’ [ro05ee 009] 
\z


\ea%624
    \label{ex:key:624}
    \gll \textbf{\'{U}s=mún}    yu  de  gó?\\
\textsc{q}=month    \textsc{2sg}  \textsc{ipfv}  go\\

\glt ‘When [in which month] are you going?’ [ro05ee 010]
\z


\ea%625
    \label{ex:key:625}
    \gll \textbf{\'{U}s=hía}  yu  bɔ́n?\\
\textsc{q}=year  \textsc{2sg}  be.born\\

\glt ‘When [in which year] were you born?’ [ro05ee 011]
\z

In the same vein, time units of the day are often questioned by the more specific bimorphemic question word \textit{ús=áwa} ‘\textsc{q}=hour’ \REF{ex:key:626}, which may refer to units of the clock as well as periods of the day (e.g. \textit{mɔ́nin tɛ́n} ‘morning’, \textit{sán tɛ́n} ‘noon’, \textit{nɛ́t} ‘night’):


\ea%626
    \label{ex:key:626}
    \gll \textbf{\'{U}s=áwa}  yu  rích?\\
\textsc{q}=hour  \textsc{2sg}  reach\\

\glt ‘When [at what period of the day, at what time] did you arrive?’ [dj05ce 153]
\z

The generic nouns \textit{sáy} ‘side, place’ (pervasive) and \textit{pát} \textit{\textup{‘part, place’} }(marginal) combine with \textit{ús=} ‘\textsc{q}’ in order to render ‘\textsc{where’} and question locative adverbials\is{locative adverbials}. The question word \textit{ús=sáy} tends to have a more general meaning than \textit{ús=pát} ‘\textsc{q}=part, place’. The logical option \textit{ús=plés} ‘\textsc{q}=place’ is accepted in elicitation but not attested in natural speech. Compare \REF{ex:key:627} and \REF{ex:key:628}:


\ea%627
    \label{ex:key:627}
    \gll \textbf{\'{U}s=sáy}  yu  kɔmɔ́t?\\
\textsc{q}=side  \textsc{2sg}  come.out\\

\glt ‘Where do you come from?’ [dj05ce 167]
\z


\ea%628
    \label{ex:key:628}
    \gll \textbf{\'{U}s=pát}  yu  kɔmɔ́t?\\
\textsc{q}=part  \textsc{2sg}  come.out\\

\glt ‘Where do you come from?’ or  ‘Which place do you come from?’ [ro05ee 086]
\z

The bimorphemic question word \textit{ús=káyn} ‘\textsc{which’} is also employed as a modifier of the generic noun \textit{stáyl} ‘style’ in order to question manner adverbials (\ref{ex:key:629}–\ref{ex:key:630}). Note the subtle difference in meaning between \textit{ús=káyn stáyl} ‘by which means’ in the following examples and \textit{háw} ‘how’ further below:


\ea%629
    \label{ex:key:629}
    \gll Na  \textbf{ús=káyn}  \textbf{stáyl}  yu  ték    kán    na  yá?\\
\textsc{foc}  \textsc{q}=kind  style  \textsc{2sg}  take    come  \textsc{loc}  here\\

\glt ‘By which means did you come here?’ [ro05ee 005]\\
\z

\ea%630
    \label{ex:key:630}
    \gll \textbf{\'{U}s=káyn} \textbf{stáyl} yu  rích    yá?\\
\textsc{q}=kind    style  \textsc{2sg}  reach  here\\
\glt ‘By which means did you get here?’ [dj05ce 151]
\z

A second and equally common means of questioning manner adverbials is provided by the monomorphemic question word \textit{háw} ‘how’. Sentence \REF{ex:key:631} involves a main clause, example \REF{ex:key:632} a main and a subordinate clause: 


\ea%631
    \label{ex:key:631}
    \gll \textbf{Háw}  e    bin  só,    \textbf{háw}    e    bigín,
\textbf{háw}   e    salút  yú?\\
how    \textsc{3sg.sbj}  \textsc{pst}  show  how    \textsc{3sg.sbj}  begin
how    \textsc{3sg.sbj}  greet  \textsc{2sg.indp}\\

\glt ‘How did he show [respect], how did he begin, how did he greet you?’ [au07se 134]
\z


\ea%632
    \label{ex:key:632}
    \gll \textbf{Háw}  yu  sabí    sé    na  rubio?\\
how    \textsc{2sg}  know  \textsc{quot}    \textsc{foc}  blond\\

\glt ‘How do you know it’s light?’ [ab03ab 182]
\z

In addition, \textit{háw} may precede the quantifier\is{quantifiers} \textit{mɔ́ch} ‘much’ and form an independent question word in order to question a quantity \REF{ex:key:633} as well as the degree to which the property denoted by the property item applies (\ref{ex:key:634}–\ref{ex:key:635}):


\ea%633
    \label{ex:key:633}
    \gll \textbf{Háw}  \textbf{mɔ́ch}  dís  sɔ́t    \textbf{kɔ́s}?\\
how    much  this  shirt  cost\\

\glt ‘How much did this shirt cost?’ [ro05de 061]
\z


\ea%634
    \label{ex:key:634}
    \gll \textbf{Háw}  \textbf{mɔ́ch}  \textbf{lɔ́n}?\\
how    much  be.long\\

\glt ‘How long?’ [ye 07fn 066]
\z


\ea%635
    \label{ex:key:635}
    \gll \textbf{Háw}  \textbf{mɔ́ch}  dí  tín    \textbf{évi}?\\
how    much  this  thing  be.heavy\\

\glt ‘How heavy is this thing?’ [lo07he 047]
\z

The collocation \textit{háw mɔ́ch} is also used to question quantifiers of count and mass nouns alike. Compare \REF{ex:key:636} in which a time quantity (hence duration) is questioned, \REF{ex:key:637} in which a mass nouns is questioned, and \REF{ex:key:638} in which the count noun \textit{pikín} ‘child’ is questioned: 


\ea%636
    \label{ex:key:636}
    \gll Yu  bin  sté    \textbf{háw}    \textbf{mɔ́ch}  \textbf{dé}?\\
\textsc{2sg}  \textsc{pst}  stay    how    much  day\\

\glt ‘How many days did you stay?’ [kw03sp 066]
\z


\ea%637
    \label{ex:key:637}
    \gll \textbf{Háw}  \textbf{mɔ́ch}  \textbf{wɔtá} yu  wánt?\\
how    much  water  \textsc{2sg}  want\\

\glt ‘How much water do you want?’ [lo07he 046]
\z


\ea%638
    \label{ex:key:638}
    \gll \textbf{Háw}  \textbf{mɔ́ch}  \textbf{pikín}  de  gó  na  dán  skúl?\\
how    much  child  \textsc{ipfv}  go  \textsc{loc}  that  school\\
\glt ‘How many children go to that school.’ [ro05de 062]
\z

Adverbials of cause can be questioned in a number of ways. Firstly \textit{wétin} and \textit{ús=tín} ‘\textsc{what’} regularly occur with the meaning of ‘\textsc{why}’ \REF{ex:key:639}. The use of these two question words may colour the question with reproach if the subject of the clause is human:


\ea%639
    \label{ex:key:639}
    \gll \textbf{Wétin}  yu  nó  de  wók    tidé?\\
what  \textsc{2sg}  \textsc{neg}  \textsc{ipfv}  work  today\\

\glt ‘Why [how come] come you’re not working today?’ [ye05ce 171]
\z

The question word \textit{háw} ‘how’ is used in a similar way in rhetorical questions that call the legitimacy of an addressee’s statement, potential answer or behaviour into question. This type of question clause therefore involves the use of the potential mood: 


\ea%640
    \label{ex:key:640}
    \gll \textbf{Háw}  mosquito  nó  \textbf{go}  bɛ́t=an?\\
how    mosquito  \textsc{neg}  \textsc{pot}  bite=\textsc{3sg.obj}\\

\glt ‘Why wouldn’t mosquitos bite him [since you have 
removed the mosquito net]?’ [ab03ab 141]
\z


\ea%641
    \label{ex:key:641}
    \gll \textbf{Háw}  yu  \textbf{go}  dé    yu  nó  gɛ́t  pikín?\\
how    \textsc{2sg}  \textsc{pot}  \textsc{be.loc}  \textsc{2sg}  \textsc{neg}  get  child\\

\glt ‘How would you be [live like] without having children 
[what a ridiculous thing to demand]?’ [kw03sb 203]
\z

\textit{Wétin} and \textit{ús=tín} ’what’ also occur in question phrases as the objects\is{objects} of prepositions that may mark NPs for a cause semantic role (cf. \sectref{sec:9.1.3} for details). In \REF{ex:key:642}, \textit{wétin} is the object of \textit{fɔ} \textsc{‘prep’}, in \REF{ex:key:643}, \textit{wet} ‘with’ is followed by \textit{ús=tín} , and in \REF{ex:key:644}, the preposition \textit{fɔséka} ‘due to’ takes \textit{ús=tín} as its object. The resulting phrases all serve to question adverbials of cause. Note that these phrases can optionally appear under cleft-focus like any other question element: \is{cleft constructions}


\ea%642
    \label{ex:key:642}
    \gll \textbf{Fɔ}  \textbf{wétin}  yu  nó  de  wók    tidé?\\
\textsc{prep} what \textsc{2sg} \textsc{neg} \textsc{ipfv} work  today\\

\glt ‘Why aren’t you working today?’ [dj05ce 172]
\z


\ea%643
    \label{ex:key:643}
    \gll \textbf{Na}  \textbf{wet}    \textbf{ús=tín}  in    pikín  dáy,    ús=sík?\\
\textsc{foc}  with    \textsc{q}=thing  \textsc{3sg.poss}  child  die    \textsc{q}=sickness\\

\glt ‘Due to what did his child die, which sickness?’ [lo07he 055]
\z


\ea%644
    \label{ex:key:644}
    \gll \textbf{Na}  \textbf{fɔséka}  \textbf{ús=tín}  in   pikín  dáy?\\
\textsc{foc}  due.to  \textsc{q}=thing  \textsc{3sg.poss}  child  die\\

\glt ‘Why did his child die?’ [lo07he 053]
\z

The preposition \textit{fɔséka} ‘due to’ may also be employed on its own as a question word in a truncated question phrase of the type presented in \REF{ex:key:645}:


\ea%645
    \label{ex:key:645}
    \gll \textbf{Fɔséka}  in    pikín  dáy?\\
due.to  \textsc{3sg.poss}  child  die\\

\glt ‘Due to (what) did his child die?’ [lo07he 056]
\z

The third way of questioning adverbials of cause is via the idiomatic clauses \textit{wétin mék} ‘\textsc{what} make’ = ‘why, how come’ and \textit{ús=tín mék} \textsc{‘what} make’ \textsc{=} \textsc{‘}why, how come’ \REF{ex:key:646}. \textit{Mék} also occurs with the meaning ‘(to) cause’ as a full verb in questions such as \REF{ex:key:647}:


\ea%646
    \label{ex:key:646}
    \gll \textbf{Wétin}  \textbf{mék}    yu  nó  de  wók    tidé?\\
what  make  \textsc{2sg}  \textsc{neg}  \textsc{ipfv}  work  today\\

\glt ‘How come you aren’t working today?’\textstylePichiglossZchn{ [ro05ee 016]}
\z


\ea%647
    \label{ex:key:647}
    \gll \textbf{Wétin}  \textbf{mék}    dá  wán,  ɛ́n?\\
what  make  that  one    \textsc{intj}\\

\glt ‘What causes that?’ [ma03hm 080]
\z

There is some variation in the degree of idiomaticity of \textit{wétin/ús=tín mék} ‘what make’, which is reflected in the degree of “verbiness” of \textit{mék} ‘make’. Example \REF{ex:key:646} above pres\-ents the most common way of employing \textit{wétin mék}. The element \textit{mék} is neither modified for a TMA category nor is it accompanied by other characteristics that would point to its status as a verb. 


In contrast, the question in \REF{ex:key:648} is indicative of a more “verby” status of \textit{mék} than in \REF{ex:key:646}. Here, the questioned situation denoted by \textit{wók} ‘work’ is the predicate of a quotative clause\is{quotative clauses} to the main verb \textit{mék}. The quotative marker and complementiser \textit{sé} ‘\textsc{quot}’ links the main and subordinate clauses: 



\ea%648
    \label{ex:key:648}
    \gll \textbf{Wétin}  \textbf{mék}    \textbf{sé}    yu  nó  wók    tidé?\\
what  make  \textsc{quot}    \textsc{2sg}  \textsc{neg}  work  today\\

\glt ‘How come you didn’t work today?’ [dj05ce 174]
\z

Sentence \REF{ex:key:649} below contains the most verb-like instance of \textit{mék}. Here, \textit{mék} ‘make’ not only functions as a main verb to the complement verb \textit{wók} ‘work’. It also induces a subjunctive\is{subjunctive mood} mood over the complement clause, because it is employed with its lexical meaning as a deontic causative verb (cf. \sectref{sec:9.4.4})\is{causative constructions}. Equally, the main verb \textit{mék} is fully finite as can be seen by the presence of the imperfective marker \textit{de} ‘\textsc{ipfv}’:


\ea%649
    \label{ex:key:649}
    \gll \textbf{\'{U}s=tín}  \textbf{de}  \textbf{mék}    \textbf{sé}   \textbf{mék}    yu  nó  wók    tidé?\\
\textsc{q}=thing  \textsc{ipfv}  make  \textsc{quot}    \textsc{sbjv}    \textsc{2sg}  \textsc{neg}  work  today\\

\glt ‘What is causing you not to work today?’ [ye05ce 173]\is{content questions}
\z

\subsection{Answers}\label{sec:7.3.3}

In Pichi, \textit{yɛ́(s}) ‘yes’ is the central agreement interjection. Both \textit{yɛ́} and \textit{yɛ́s} are employed in formal and informal registers alike. Compare the answer in \REF{ex:key:650b}:


\ea%650
    \label{ex:key:650}
\ea{\label{ex:key:650a}
\gll
Náw    yu  fít  dríng=an    nɔ́?\\
  now    \textsc{2sg}  can  drink=\textsc{3sg.obj}  \textsc{intj}\\

\glt   ‘Now, you’re able to drink it, right?’\textstylePichiglossZchn{ [kw03sp 115]}
}
\ex{\label{ex:key:650b}
\gll
Náw    so,    \textbf{yɛ́s},  a    fít  dríng=an    fáyn.\\
  now    like.that  yes  \textsc{1sg.sbj}  can  drink=\textsc{3sg.obj}  fine\\

\glt   ‘Now, I’m able to drink it [milk] well.’ [\textstylePichiglossZchn{ed03sp 116]}
}
\z
\z

Stronger degrees of agreement can be signalled by other elements. The interjection ɛhɛ́ signals emphatic ‘yes’. The focus constructions \textit{na só} ‘\textsc{foc} so’ = ‘that’s how it is’, \textit{na ín} ‘\textsc{foc 3sg.indp}’ = ‘that’s it’, and \textit{na di tín} ‘\textsc{foc def} thing’ = ‘that’s it’ also signal strong agreement. 


The elements \textit{nó} and \textit{nɔ́} are used as free variants in order to signal disagreement. The former element is identical in form to the general negator \textit{nó}. Many Pichi speakers agree or disagree with the polarity of the question. Hence agreement with the negative polarity of the question in \REF{ex:key:651} evokes the use of the agreement marker \textit{yɛ́s}:



\ea%651
    \label{ex:key:651}
\ea{\label{ex:key:651a}
    \gll
So  yu  \textbf{nó}  go  chɔ́p?\\
  so  \textsc{2sg} \textsc{neg} \textsc{pot} eat\\

\glt   ‘So you won’t eat?’ [chfn05 001]
}
\ex{\label{ex:key:651b}
\gll
\textbf{Yɛ́s}.\\
yes\\
\glt   ‘No (I won’t eat).’ [lifn05 004]
}
\z
\z

In the same way, disagreement with the positive polarity of the question requires the use of the disagreement marker: 


\ea%652
    \label{ex:key:652}
\ea{\label{ex:key:652a}
    \gll 
Yu  go  chɔ́p?\\
  \textsc{2sg}  \textsc{pot}  eat\\

\glt   ‘Will you eat?’ 
}
\ex{\label{ex:key:652b}
\gll
\textbf{Nó}.\\
  \textsc{neg}\\
\glt   ‘No (I won’t eat).’ [lifn05 005]
}
\z
\z

However, \textit{yɛ́s} and \textit{nó} are also used to agree or disagree with the proposition of the utterance, possibly through \ili{Spanish} influence:


\ea%653
    \label{ex:key:653}
\ea{
    \gll 
So  yu  \textbf{nó}  go  kán?\\
  so  \textsc{2sg}\textstylePichitranslationZchn{} \textsc{neg}\textstylePichitranslationZchn{} \textsc{pot} \textstylePichitranslationZchn{  eat}\\

\glt   ‘So you won’t come?’
}\ex{
\gll
\textbf{Nó},  a    \textbf{nó}  go  kán.\\
\textstylePichitranslationZchn{no} \textsc{1sg.sbj} \textsc{neg} \textsc{pot} \textstylePichitranslationZchn{  eat}\\

\glt   ‘No, I won’t come.’ [lifn05 002]
}
\z
\z

In sentence-final position, \textit{nɔ́} functions as a question-tag, i.e. a conative interjection. In this function, \textit{nɔ́} is used in rhetorical questions as well as in biased questions, in which the speaker expresses the expectation that the answer will correspond to the polarity of the question \REF{ex:key:654}: 


\ea%654
    \label{ex:key:654}
    \gll Yu  de  fíl  hɔ́t  \textbf{nɔ́}?\\
\textsc{2sg}  \textsc{ipfv}  feel  hot  \textsc{intj}\\

\glt ‘You’re feeling hot, aren’t you?' [ma03hm 007]
\z

Aside from that, \textit{nɔ́} also serves as a phatic interjection in order to solicit attention (cf. \sectref{sec:12.2.3}). For example, the clause in \REF{ex:key:655} underlines the speaker’s commitment to the truth of a story that he has just narrated:


\ea%655
    \label{ex:key:655}
    \gll \textbf{Nɔ́},  nó  tɔ́k  ɛ́n!\\
\textsc{intj}  \textsc{neg}  talk  \textsc{intj}\\

\glt ‘No, don’t talk [and call into question the truth of my story].’ [ed03sb 177]
\z

Strong disagreement can be expressed by the focus construction \textit{nóto só} ‘\textsc{neg}.\textsc{foc} so’ = ‘that’s not how it is’. The following two sentences succeed each other in a narrative. The disagreement expressed in \REF{ex:key:656a} is underlined by sentence (b): 


\ea%656
    \label{ex:key:656}
\ea{\label{ex:key:656a}
    \gll Dɛn  tɛ́l=an    sé    “\textbf{nóto}  só.”\\
  \textsc{3pl}  tell=\textsc{3sg.obj}  \textsc{quot}    \phantom{“}\textsc{neg}.\textsc{foc}  like.that\\

\glt   ‘They said to her “that’s not how it was”.‘ [ed03sb 045]
}\ex{
\gll
Tɛ́l  wí    trú!\\
  tell  \textsc{1pl.indp}  true\\

\glt   ‘Tell us (the) truth!’ [ed03sb 046]
}
\z
\z

Even stronger disagreement is expressed through the negative phrases \textit{nó wán dé} ‘never’ \REF{ex:key:657b} and \textit{nó wé} ‘no way’ \REF{ex:key:658b}. The following two sentence pairs illustrate their use in signalling disagreement in response to a question:


\ea%657
    \label{ex:key:657}
\ea{
    \gll
Na  yú  chɔ́p  dí  tín?\\
  \textsc{foc}  \textsc{2sg.indp}  eat  this  thing\\

\glt   ‘Did you eat this (thing)?’ [ur07he 061]
}\ex{\label{ex:key:657b}
\gll
\textbf{Nó}  \textbf{wán}  \textbf{dé},  nóto  mi.\\
  \textsc{neg}  one  day  \textsc{neg}.\textsc{foc}  \textsc{1sg.indp}\\

\glt   ‘Never, it’s not me.’ [lo07he 062]
}
\z
\z


\ea%658
    \label{ex:key:658}
\ea{
    \gll
Yu  go  kán    wet    mí?\\
  \textsc{2sg}  \textsc{pot}  come  with    \textsc{1sg.indp}\\

\glt   ‘Will you come with me?’ [ur07he 063]
}\ex{\label{ex:key:658b}
\gll
\textbf{Nó}  \textbf{wé},  a    nó  go  kán.\\
  \textsc{neg}  way  \textsc{1sg.sbj}  \textsc{neg}  \textsc{pot}  come\\

\glt   ‘No way, I won’t come.’ [lo07he 064]
}
\z
\z

Given the right pragmatic context, the question word\is{question words} \textit{ús=sáy} ‘\textsc{where’} may signal strong disagreement as well \REF{ex:key:659b}. The imperative\is{imperatives} clause \textit{kɔmɔ́t dé} ‘get lost’ can be employed to express strong and abusive disagreement \REF{ex:key:660b}:


\ea%659
    \label{ex:key:659}
\ea{
    \gll Dɛn  dɔ́n  gí  yu  di  mɔní?\\
  \textsc{3pl}  \textsc{prf}  give  \textsc{2sg}  \textsc{def}  money\\

\glt   ‘Have they given you the money?’ [pa07fn 478]
}\ex{\label{ex:key:659b}
\gll
\textbf{\'{U}s=sáy}?\\
  \textsc{q}=side\\

\glt   ‘Where? [not at all]’ [ye07fn 479]
}
\z
\z


\ea%660
    \label{ex:key:660}
\ea{
    \gll Yu  nó  go  dú=an    fɔ  mí?\\
  \textsc{2sg}  \textsc{neg}  \textsc{pot}  do=\textsc{3sg.obj}  \textsc{prep}  \textsc{1sg.indp}\\

\glt   ‘Won’t you do it for me?’ [ne07fn 578]
}\ex{\label{ex:key:660b}
\gll
\textbf{Kɔmɔ́t}  \textbf{dé}!\\
  go.out  there\\
\glt   ‘Get lost!’ [la07fn 579]
}
\z
\z

The answer to a content questions\is{content questions} may be given in full or truncated sentences consisting of the questioned constituent(s) as in \REF{ex:key:661b}: 


\ea%661
    \label{ex:key:661}
\ea{
    \gll \'{U}s=wán  na  in    ném?\\
  \textsc{q}=one  \textsc{foc}  \textsc{3sg.poss}  name\\

\glt   ‘What’s his name?’ [ko03sp 061]
}\ex{\label{ex:key:661b}
\gll
Nguema  Mba.\\
  \textsc{name}  \textsc{name} \\

\glt   ‘Nguema Mba’ [ed03sp 062]\is{answers}
}
\z
\z

\section{Focus}\label{sec:7.4}

The extensive use of focus structures in sentence formation is a distinctive mark of Pichi. Focus constructions have two principal pragmatic functions in the language. Firstly, they serve to present new information. For this function, I employ the term “presentational focus\is{presentational focus}” \citep{Drubig2003}. Secondly, focus constructions serve to assert previously introduced information that runs counter to the presupposition of an addressee. This function is here referred to as “contrastive focus\is{contrastive focus}” (\citealt[35]{Chafe1976}). Focus is realised through three distinct strategies: suprasegmental focus, particle focus, and cleft focus. Cleft focus may also be applied to verbs in so-called predicate cleft constructions (cf. \sectref{sec:7.4.5}). The language also employs various other means for emphasis\is{emphasis}, including presentatives (c.f \sectref{sec:7.4.4}). The syntactic operation of clefting renders elements under cleft focus pragmatically salient. But it is difficult to determine the semantic differences between cleft focus and other types of focus on the basis of the available data. 

\subsection{Suprasegmental focus}

The use of focus constructions is intimately tied to suprasegmental phonology. Firstly, focus at the sentence or clause level may be signalled by emphatic intonation (cf. \sectref{sec:3.4.2}). Extra-high tone may also be employed to focus individual constituents or groups of constituents (cf\textstyleannotationreference{.} \sectref{sec:3.2.5}). These forms of suprasegmental focus may be freely combined with the different types of focus constructions presented in the following. \is{prosodic focus}

\subsection{Particle focus}\label{sec:7.4.2}

Particle focus involves the elements \textit{sɛ́f} ‘self, \textsc{emp}’, \textit{sénwe} ‘\textsc{emp}’ and the sentence particles/interjections \textit{ɛ́n} ‘\textsc{intj}’ and \textit{ó} ‘\textsc{sp}’ (cf. \sectref{sec:12.2.4} for a detailed treatmen of these elements). These elements may signal focus of constituents of varying complexity including entire clauses and sentences. \tabref{tab:key:7.5} provides an overview. {\fff}

%%please move \begin{table} just above \begin{tabular
\begin{table}
\caption{Focus particles}
\label{tab:key:7.5}

\begin{tabularx}{\textwidth}{lQQQQ}
\lsptoprule
Form & Translations & Focus type & Scope & Other uses\\
\midrule
\itshape sɛ́f & ‘-self, too, even, actually, really’ & Presentational; contrastive & Sentence; constituent & Reflexive\is{reflexivity} anaphor\\
\tablevspace
\itshape sénwe & ‘-self, too, exactly’ & Presentational\is{presentational focus}; contrastive & constituent & —\\
\tablevspace
\itshape ó & ‘really, actually, even, at all’ & Presentational; contrastive\is{contrastive focus} & Sentence; constituent & Vocative\is{vocatives}; assertion; encouragement\\
\tablevspace
\itshape ɛ́n & ‘really’ & Presentational & Sentence; constituent & Channel check\\
\lspbottomrule
\end{tabularx}
\end{table}
\subsubsection{Forms and functions}

The reflexive anaphor and emphatic particle \textit{sɛ́f} ‘self, \textsc{emp}’ is the most frequently used form in particle focus. The following sentence presents the use of \textit{sɛ́f} as a reflexive{\fff} anaphor (cf. \sectref{sec:9.3.5} for a detailed treatment): {\fff}


\ea%662
    \label{ex:key:662}
    \gll Dán    gál  e    kin  fíks  in      \textbf{sɛ́f},
pént    in    \textbf{sɛ́f}.\\
that    girl  \textsc{3sg.sbj}  \textsc{hab}  fix  \textsc{3sg.poss}    self
paint  \textsc{3sg.poss}  self\\

\glt 
`That girl, she usually does herself up, paints herself [puts on make-up].’ [dj07ae 114]
\z

The two successively uttered sentences \REF{ex:key:663a} and \REF{ex:key:663b} exemplify the use of \textit{sɛ́f} ‘self, \textsc{emp}’ in signalling presentational focus{\fff}. In \REF{ex:key:663a}, the speaker provides information on the topic \textit{dán mán} ‘that man’. In (b), the same speaker fills in the information gap in combination with presentational focus of the entire sentence: 


\ea%663
    \label{ex:key:663}
\ea{\label{ex:key:663a}
    \gll Dán  mán    e    bin  kán  gó  na  jél  lɔ́n  tɛ́n.\\
  that  man    \textsc{3sg.sbj}  \textsc{pst}  \textsc{pfv}  go  \textsc{loc}  jail  long  time\\

\glt   ‘That man, he went to jail long ago.’ [ma03sh 015]
}\ex{\label{ex:key:663b}
\gll
Náw    e    dɔ́n  dáy  \textbf{sɛ́f}.\\
  now    \textsc{3sg.sbj}  \textsc{prf}  die  \textsc{emp}\\

\glt   ‘Now he is even dead.’ [ma03sh 016]
}\z\z

The corpus contains a single occurrence of \textit{sɛ́f} ‘self, \textsc{emp}’ preceded by a \textsc{3sg.poss} pronoun which is coreferential with the head noun of the focused NP \REF{ex:key:664}. This structure is a dislocated possessive construction{\fff} in which \textit{sɛ́f} functions as a nominal in the possessed noun position. In the construction, the low-toned \textsc{3sg} possessive pronoun and \textit{sɛ́f} together signal emphasis or focus of the preceding noun \textit{di bɔ́y} ‘the boy’:


\ea%664
    \label{ex:key:664}
    \gll Wé  di  bɔ́y  in    \textbf{sɛ́f},  wé  e    sí  mí,
estaba  contento.\\
\textsc{sub}  \textsc{def}  boy  \textsc{3sg.poss}  self  \textsc{sub}  \textsc{3sg.sbj}  see  \textsc{1sg.indp}
he.was  content\\

\glt ‘And the boy himself, when he saw me, he was content.’ [ab03ay 046]
\z

The construction in \REF{ex:key:664} is, however, marginal. Note the difference between \REF{ex:key:664} and the following \REF{ex:key:665}. In the latter example, \textit{sɛ́f} ‘self, \textsc{emp}’ is used as a regular focus particle, postposed to the high-toned \textsc{3sg} emphatic personal pronoun: 


\ea%665
    \label{ex:key:665}
    \gll \'{I}n    \textbf{sɛ́f}  gó  na  baf-rúm    e    wás.\\
\textsc{3sg.indp}  \textsc{emp}  go  \textsc{loc}  bath-room  \textsc{3sg.sbj}  wash\\

\glt ‘He (by) himself went to the bathroom (and) washed.’ [ab03ab 148]
\z

Contrastive use of \textit{sɛ́f} ‘self, \textsc{emp}’ is illustrated in \REF{ex:key:666}. In its function as a focus marker, sɛ́f often assumes a reading of inclusive or exhaustive listing; hence the translation of \textit{sɛ́f} as ‘too, also, even’ and ‘alone, without help’. In fact, a postposed \textit{sɛ́f} most appropriately renders the notion ‘too, also’ in a sentence like \REF{ex:key:667}. {\fff}


The following two examples also show that \textit{sɛ́f} has the most flexible scope of all particles. It may signal focus of sentences \REF{ex:key:666} as well as smaller constituents, such as a personal pronoun \REF{ex:key:667}: 



\ea%666
    \label{ex:key:666}
    \gll Yu  nó    \textbf{sɛ́f},  yu  jɔ́s  kán,    yu  nó  go  sabí.\\
\textsc{2sg}  know  \textsc{emp}  \textsc{2sg}  just  come  \textsc{2sg}  \textsc{neg}  \textsc{pot}  know  \\

\glt ‘Even (if) you know, if you’ve just come, you won’t know.’ [ma03hm 044]
\z


\ea%667
    \label{ex:key:667}
    \gll Mí    \textbf{sɛ́f}  dɔ́n  rích    Cotonou.\\
\textsc{1sg.indp}  \textsc{emp}  \textsc{prf}  arrive  \textsc{place}\\

\glt ‘I myself have been to Cotonou (too).’ [nn05fn 005]
\z

The particle \textit{sénwe} ‘\textsc{emp}’ is presumably a lexicalised collocation (i.e. \textit{sén.wé} ‘same.way’). It is employed in the same way as \textit{sɛ́f} in order to signal presentational and contrastive focus{\fff} \REF{ex:key:668}. The use of \textit{sénwe} as a clausal focus particle is not attested. In general, sénwe occurs less frequently than \textit{sɛ́f} and is found more often to focus personal pronouns than full nouns. Consider the following example, in which \textit{sénwe} signals presentational focus of the personal pronoun \textit{yú} ‘\textsc{2sg.indp}’: 


\ea%668
    \label{ex:key:668}
    \gll Dí  wán,  \textbf{yú}   \textbf{sénwe}  yu  de  gó.\\
this  one    \textsc{2sg.indp}  \textsc{emp}    \textsc{2sg}  \textsc{ipfv}  go\\

\glt ‘This time, you yourself \textstylePichiglossZchn{are going [to die].’} [ed03sb 040]
\z

The element \textit{ó} ‘\textsc{intj}’ may signal presentational or contrastive focus of entire clauses as in \REF{ex:key:670} below. The particle is a sentence-final element which has scope over all preceding material, which may be a predicate-less sentence \REF{ex:key:669} or a clause \REF{ex:key:670}. However, modification by means of \textit{ó} also colours the sentence with meanings like warning, assertion, empathy, or emphasis (cf. \sectref{sec:12.2.4} for more details): 


\ea%669
    \label{ex:key:669}
    \gll Bata    tɔ́ng    \textbf{ó}.\\
\textsc{place}  tongue  \textsc{sp}\\

\glt ‘That’s the Fang language for you [see how peculiar it is].’ [to03gm 014]
\z


\ea%670
    \label{ex:key:670}
    \gll \MakeUppercase{A}   bin  dɔ́n,    a    bin  dɔ́n    blánt  fɔ  Gabón  \textbf{ó}.\\
\textsc{1sg.sbj}  \textsc{pst}  \textsc{prf}    \textsc{1sg.sbj}  \textsc{pst}  \textsc{prf}    reside  \textsc{prep}  Gabon  \textsc{sp}\\

\glt ‘I’ve already, I’ve already lived in Gabon [contrary to what you think].’ 


\glt [ma03hm 035]
\z

The interjection \textit{ɛ́n} ‘\textsc{intj}’ is principally employed in sentence-final position as a chan\-nel-checking device in order to solicit the attention of an addressee (cf. also \sectref{sec:12.2.2}). Channel-checking automatically lends prominence to a preceding utterance, hence ɛ́n may function very much like other sentential focus particles \REF{ex:key:671}. 


\ea%671
    \label{ex:key:671}
    \gll Djunais,  yu  badhát  \textbf{ɛ́n}.\\
\textsc{name}  \textsc{2sg}  be.mean  \textsc{intj}\\

\glt ‘Djunais you‘re really mean.’ [fr03wt 032]
\z

Beyond that, \textit{ɛ́n} may also occur in mid-sentence followed by a pause, in order to focus a single constituent. In \REF{ex:key:672}, the \ili{Spanish} depictive adjective \textit{fresco} ‘fresh’ is fronted and singled out for focus by \textit{ɛ́n}:


\ea%672
    \label{ex:key:672}
    \gll Fresco  \textbf{ɛ́n},  dɛn  de  gí  wí.\\
fresh  \textsc{intj}  \textsc{3pl}  \textsc{ipfv}  give  \textsc{1pl.indp}\\

\glt ‘Fresh, (that’s how) they would give (it) to us.’ [ed03sp 103] 
\z

\subsubsection{Eligible constituents}

Any sentence constituent may be subjected to particle focus save dependent personal pronouns, determiners, and \textsc{TMA} particles. Equally, the individual elements of multi-constituent \textsc{NPs} cannot be focused, since an \textsc{NP} must be focused in its entirety. Other than that, constituents of varying degrees of complexity may be focused. Sentence \REF{ex:key:673} features a prepositional phrase with a single noun under focus, and \REF{ex:key:674} the complex prepositional phrase and reflexive construction \textit{na yu skín} ‘on you(r body)’:


\ea%673
    \label{ex:key:673}
    \gll Na  Trinidad  \textbf{sɛ́f}  nɔ́?\\
\textsc{loc}  \textsc{place}    \textsc{emp}  \textsc{intj}\\

\glt ‘Even in Trinidad, right?’ [au07se 226]
\z


\ea%674
    \label{ex:key:674}
    \gll Ɛf  e    de  gó  yu  se  supone  que  e  de  fáyn
na  yu  skin    \textbf{sɛ́f}.\\
if  \textsc{3sg.sbj}  \textsc{ipfv}  go  \textsc{2sg}  \textsc{refl}  assume  that  \textsc{3sg.sbj}  \textsc{ipfv}  fine
\textsc{loc}  \textsc{2sg}  body  \textsc{emp}\\

\glt ‘If it goes well with you, it’s assumed that it looks nice on you(r body).’ [dj07ae 175]
\z

In dialogue, verbless, prosodically independent sentences can be found which consist of a focused constituent alone. By singling out particular elements in such a way, a speaker may convey strong emphatic force. Compare the discourse excerpt in the two following examples. In \REF{ex:key:675a} speaker (hi) emphasises the lack of responsibility of certain mothers by utilising focus with \textit{sɛ́f} ‘self, \textsc{emp}’. Her statement is confirmed by speaker (bo) in (b):


\ea%675
    \label{ex:key:675}
\ea{\label{ex:key:675a}
\gll
Bɔt  dán  káyn  mamá  dɛn  \textbf{sɛ́f}.\\
  but  that  kind    mother  \textsc{pl}  \textsc{emp}\\

\glt   ‘But these kinds of mother, really.’ [hi03cb 113]
}\ex{
\gll
De  verdad.\\
  of  truth\\

\glt   ‘Really.’ [bo03cb 114]
}\z\z

The corpus contains many examples of focused adverbial phrases, in particular time adverbials, such as \textit{tumɔ́ro} ‘tomorrow’ in \REF{ex:key:676}:


\ea%676
    \label{ex:key:676}
    \gll Wé  a    dɔ́n  jɔ́ch    dɛ́n,    a    sé    “\textbf{tumɔ́ro}    \textbf{sénwe}
a    de  gó  mít    in    mán”.\\
\textsc{sub}  \textsc{1sg.sbj}  \textsc{prf}  judge  \textsc{3pl.indp}  \textsc{1sg.sbj}  \textsc{quot}    \phantom{“}tomorrow  \textsc{emp}
\textsc{1sg.sbj}  \textsc{ipfv}  go  meet  \textsc{3sg.poss}  man\\

\glt ‘When I had talked them down, I said “tomorrow, I’m going to meet 
her husband”.’ [ro05rt 023]
\z

Subordinate clauses may be focused by the same means as other, smaller sentence constituents. The relative clause{\fff} in \REF{ex:key:677} is under the scope of the particle \textit{sɛ́f} ‘self, \textsc{emp}’. In \REF{ex:key:678}, the clause introduced by \textit{sé} ‘\textsc{quot}’ is under focus by means of the sentence-final particle \textit{ó}:


\ea%677
    \label{ex:key:677}
\gll
E    lúk    di  análisis,  tiene  paludismo  de  una  cruz
\textbf{wé}  kin  kíl  pikín  \textbf{sɛ́f}.\\
\textsc{3sg.sbj}  look    \textsc{def}  analysis  he.has  malaria    of  one  cross
\textsc{sub}  \textsc{hab}  kill  child  \textsc{emp}\\

\glt ‘She [the doctor] looked at the analysis “he has malaria of one cross 
which even kills children”.’ [ab03ab 120]
\z


\ea%678
    \label{ex:key:678}
    \gll Bikɔs  dɛn  tɔ́k  \textbf{sé}  na  paludismo  \textbf{ó}.\\
because  \textsc{3pl}  talk  \textsc{quot}  \textsc{foc}  malaria    \textsc{sp}\\

\glt ‘Because they said that it’s malaria.’ [hi03cb 124]
\z

Elements which are part of a coordinate structure can be focused separately \REF{ex:key:679}, and there is no restriction save intelligibility on the number of elements that can be focused in one sentence. Compare \REF{ex:key:679} which features constituent focus by means of the particle \textit{sɛ́f} and clausal focus by means of a sentence-final \textit{ó}:


\ea%679
    \label{ex:key:679}
    \gll Tú  pípul  \textbf{sɛ́f}  wet  wán  pikín  dɔ́n  kán    \textbf{ó}.\\
two  people  \textsc{emp}  with  one  child  \textsc{prf}  come  \textsc{sp}\\

\glt ‘Even two people and one child have come.’ 
\z

Example \REF{ex:key:680} presents clausal focus (or alternatively focus of the object NP \textit{dán} ‘convence’ \textit{dé}) through \textit{sɛ́f}, as well as focus of the ensuing adverbial phrase \textit{na Pichi} by means of \textit{sénwe}: 


\ea%680
    \label{ex:key:680}
    \gll A    bin  wánt  tɔ́k  dán  “convence”  dé    sɛ́f  
na  Pichi  \textbf{sénwe},  a    nó  de  mɛ́mba.\\
\textsc{1sg.sbj}  \textsc{pst}  want  talk  that  convince    there  \textsc{emp}  
\textsc{loc}  Pichi  \textsc{emp}     \textsc{1sg.sbj}  \textsc{neg}  \textsc{ipfv}  remember\\

\glt ‘I had actually wanted to say that “convence” there in Pichi itself (but) 
I don’t remember [how to say it].’ [dj05ae 040]
\z

Constituent and verb negation\is{negation} are compatible with particle focus. When used in combination with negation, particle focus produces emphatic negative readings like ‘not at all, not even’:


\ea%681
    \label{ex:key:681}
    \gll Nó  mán    \textbf{nó}  blánt  yá    mɔ́    \textbf{sɛ́f}.\\
\textsc{neg}  man    \textsc{neg}  reside  here    more  \textsc{emp}\\

\glt ‘Nobody even lives here anymore.’ [ra07fn 064]
\z

Personal pronouns can be focused through the use of the corresponding emphatic, independent form alone instead of resorting to \textit{sɛ́f} or \textit{sénwe} (cf. \ref{ex:key:667}–\ref{ex:key:668} above). Compare subject focus in the rhetorical question in \REF{ex:key:682}:


\ea%682
    \label{ex:key:682}
    \gll \textbf{Mí}    wánt  dán    mán?\\
\textsc{1sg.indp}  want  that    man\\

\glt ‘Do I [\textsc{emp}] want that man?’ [ro05rt 026]
\z

Clausal focus by means of \textit{sɛ́f} is also regularly made use of in combination with the conditional clause linker \textit{ɛf/if} in order to render concessive meaning (cf. \sectref{sec:10.7.12}).

\subsubsection{Word order and scope}

Focused constituents may appear in situ, i.e. in the same syntactic position assigned to them in focus-neutral clauses. When this is the case, focus is signalled by the presence of a particle. In \REF{ex:key:683}, the subject NP \textit{in papá} ‘her father’ is highlighted via presentational focus{\fff} only by means of the post-posed emphatic particle \textit{sɛ́f}: 


\ea%683
    \label{ex:key:683}
    \gll \'{A}fta    in    \textbf{papá}  \textbf{sɛ́f}  kán    ték=an.\\
then  \textsc{3sg.poss}  father  \textsc{emp}  come  take=\textsc{3sg.obj}\\

\glt ‘Then her father came to take her.’ [ab03ab 021]
\z

Focused non-subject \textsc{NPs} may also be found in situ together with a focus particle. Compare the focused PP \textit{fɔ} \textit{di pikín} in \REF{ex:key:684}: 


\ea%684
    \label{ex:key:684}
    \gll \'{A}fta    e    nóto,  e    nó  fáyn    fɔ  di  pikín  \textbf{sɛ́f}.\\
then  \textsc{3sg.sbj}  \textsc{neg}.\textsc{foc}  \textsc{3sg.sbj}  \textsc{neg}  fine    \textsc{prep}  \textsc{def}  child  \textsc{emp}\\

\glt ‘Then it’s not, it’s not good for the child itself.’ [fr03ft 199]
\z

When an object NP retains its usual syntactic position after the verb and is followed by a focus particle, discourse context and the presence of suprasegmental focus will usually disambiguate the resulting structure as involving clausal or phrasal focus. In \REF{ex:key:685} the particle \textit{sɛ́f} ‘self, \textsc{emp}’ may be construed as having narrow scope over the object NP \textit{dán torí} ‘that story’, or alternatively, broad scope over the entire sentence: 


\ea%685
    \label{ex:key:685}
    \gll Mí    nó  sabí    us    mán    dɛn  kíl,  a    nɔ́ba
hía    dán    torí    \textbf{sɛ́f}.\\
\textsc{1sg.indp}  \textsc{neg}  know  which  man    \textsc{3pl}  kill  \textsc{1sg.sbj}  \textsc{neg}.\textsc{prf}
hear     that  story  \textsc{emp}\\

\glt ‘I don’t know who was killed, I haven’t even heard that story yet.’ 
or ‘ I don’t know who was killed, I haven’t heard that particular 
story yet.’ [ro05de 049]
\z

Adverbials may be be focused by exploiting their syntactic flexibility and placing them at the head of the sentence in combination with a focus particle \REF{ex:key:686}. The corpus contains no instance of an object that has been fronted for focus. We only find focused, sentence-initial non-subjects occuring in cleft constructions (cf. e.g. \ref{ex:key:702}):


\ea%686
    \label{ex:key:686}
    \gll Lagos  \textbf{sɛ́f},  e    gɛ́t  di  sáy  wé  na  di  húman  dɛn 
de  máred  di  mán.\\
\textsc{place}  \textsc{emp}  \textsc{3sg.sbj}  get  \textsc{def}  side  \textsc{sub}  \textsc{foc}  \textsc{def}  woman  \textsc{pl} 
\textsc{ipfv}  marry  \textsc{def}  man\\

\glt ‘Even in Lagos, there is a place where it’s the women (who) 
marry the men.’ [hi03cb 177]
\z

In contrast, examples abound, in which we find dislocated, focused core participants\is{core participants} other than subjects simultaneously functioning as clausal topics (cf. \sectref{sec:7.5} for more details). The overlayering of focus and topic structures in a single sentence, and the identity of topical and focused constituents in Pichi is only natural, since “given”, topical elements often also constitute the most important information in a sentence. 


For example, sentence \REF{ex:key:687} features the dislocated and topical object NP \textit{di róp} ‘the rope’, followed by the focus particle \textit{sɛ́f}. In contrast to fronting (i.e. in question formation), the use of dislocation{\fff} comes along with the use of a resumptive pronoun{\fff} (here the \textsc{3sg.obj} pronoun \textit{=an}) in the original object position of the left-dislocated constituent:



\ea%687
    \label{ex:key:687}
    \gll \'{A}fta    di  róp    \textbf{sɛ́f},  wi  nó  sí  nó  mán    wé  e    híb=\textbf{an}.\\
then  \textsc{def}  rope  \textsc{emp}  \textsc{1pl}  \textsc{neg}  see  \textsc{neg}  man    \textsc{sub}  \textsc{3sg.sbj}  throw=\textsc{3sg.obj}\\

\glt ‘And the rope, we didn’t see anybody who threw it.’ [li07pe 005]
\z

Sentence \REF{ex:key:688} contains a left-dislocated object NP, the emphatic pronoun \textit{mí} ‘\textsc{1sg.indp}’, which is reiterated by the coreferential object pronoun \textit{mí} ‘\textsc{1sg.indp}’. In this example, too, focus of the dislocated topic is overtly signalled by means of the particle \textit{sɛ́f}:

\ea%688
    \label{ex:key:688}
    \gll \textbf{Mi}    \textbf{sɛ́f},  ɔ́l  pɔ́sin  dɛn  kin  áks  mi    sé
‘yu  dɔ́n  bɔ́n?’\\
\textsc{1sg.indp}  \textsc{emp}  all  person  \textsc{3pl}  \textsc{hab}  ask  \textsc{1sg.indp}  \textsc{quot}
\textsc{2sg}  \textsc{prf}  give.birth\\

\glt ‘Even me, everybody asks me “have you given birth 
[do you have a child]?”' [fr03ft 152]
\z

Constructions involving personal pronouns are as well the only ones in which “afterthought” apposition is frequently employed in order to signal focus of personal pronouns. Example \REF{ex:key:689} contains an appositive \textit{mí} ‘\textsc{1sg.indp’} within the scope of the focus particle \textit{sénwe} and\textit{} coreferential with the preceding dependent personal pronoun \textit{a} ‘\textsc{1sg.sbj’}:


\ea%689
    \label{ex:key:689}
    \gll A    go  wás=an    wet  mi    hán    \textbf{mí} \textbf{sénwe}.\\
\textsc{1sg.sbj}  \textsc{pot}  wash=\textsc{3sg.obj}  with  \textsc{1sg.poss}  hand  \textsc{1sg.indp}  \textsc{emp}\\

\glt ‘\textsc{I} myself would wash it with my hand.’ [dj07re 049]
\z

\subsection{Cleft focus} \label{sec:7.4.3}

The two elements \textit{na} (affirmative) and \textit{nóto} (negative) are employed in cleft constructions to signal focus of constituents of all degrees of complexity. The focus phrase \textit{es que} ‘it is that’ is of \ili{Spanish} origin and forms an integral part of the Pichi focus system. It is employed to cleft focus entire clauses. Some relevant charateristics of these three elements are given in \tabref{tab:key:7.6}.

%%please move \begin{table} just above \begin{tabular
\begin{table}
\caption{Cleft focus particles}
\label{tab:key:7.6}

\begin{tabularx}{\textwidth}{llQQQ}
\lsptoprule

Form & Gloss & Focus type & Scope & Other uses\\
\midrule
\itshape na & ‘it’s (that)’ & Presentational; contrastive\is{contrastive focus} & Sentence; constituent & Identity copula\\
\tablevspace
\itshape nóto & ‘it’s not (that)’ & Contrastive & Sentence; constituent & Negative identity copula\\
\tablevspace
  \itshape es que & ‘it’s that’ & Presentational; contrastive & Sentence & Borrowed\is{borrowing} from Spanish\\
\lspbottomrule
\end{tabularx}
\end{table}
\subsubsection{Forms and functions}

The form \textit{na} ‘\textsc{foc}’ signals presentational and contrastive focus, \textit{nóto} ‘\textsc{neg}.\textsc{foc}’ contrastive focus. It is noteworthy that in the vast majority of instances in the corpus, cleft constructions do not exhibit any overt sign of relativisation. Hence in the following sentence, the subordinator\is{subordinator} \textit{wé} ‘\textsc{sub}’ is not present in its potential position (indicated by ${\emptyset}$):


\ea%690
    \label{ex:key:690}
    \gll \'{A}fta    \textbf{na}  dán    tɛ́n    ${\emptyset}$  a    kán  gó  na  Alemania.\\
then  \textsc{foc}  that    time    \textsc{sub}  \textsc{1sg.sbj}  \textsc{pfv}  go  \textsc{loc}  \textsc{place}\\

\glt ‘So it’s \textsc{that} \textsc{time} that I went to Germany.’ [fr03ft 030]
\z

The negative focus marker \textit{nóto} is employed instead of \textit{na} to signal negative, contrastive focus. In example \REF{ex:key:691}, \textit{nóto} signals contrastive focus of the object pronoun \textit{ín} ‘\textsc{3sg.indp}’. Note the use of the emphatic form of the personal pronoun as well as the occurrence of a resumptive \textit{=an} ‘\textsc{3sg.obj}’ at the end of the clause: 


\ea%691
    \label{ex:key:691}
    \gll Sé    pero  mán  mi    brɔ́da  dát,    ús=tín  e    de  dú  na  yá,
\textbf{nóto}  \textbf{ín}    wi  bɛ́r=\textbf{an}?\\
\textsc{quot}    but    man  \textsc{1sg.poss}  brother  that    \textsc{q}=thing  \textsc{3sg.sbj}  \textsc{ipfv}  do  \textsc{loc}  here
\textsc{neg}.\textsc{foc}  \textsc{3sg.indp}  \textsc{1pl}  bury=\textsc{3sg.obj}\\

\glt ‘But man, that’s my brother, what’s he doing here, isn’t it him that we 
buried?’ [ed03sb 139]
\z

The \ili{Spanish}-origin focusing device \textit{es que} ‘it is that’ is regularly employed to signal presentational focus with clauses and sentences \REF{ex:key:692}: 


\ea%692
    \label{ex:key:692}
    \gll \textbf{Es}  \textbf{que}   e    fáyn    wé  yu  nó  sabí    sé    e    kɔmɔ́t
fɔ  di  animal.  \\
it.is  that    \textsc{3sg.sbj}  fine    \textsc{sub}  \textsc{2sg}  \textsc{neg}  know  \textsc{quot}    \textsc{3sg.sbj}  come.out
\textsc{prep}  \textsc{def}  animal\\

\glt ‘It’s that it is fine when you don’t know that it [the milk] has just come out of the 
animal.\textstylePichiglossZchn{’} [ed03sp 105]
\z

Cleft constructions may be employed for signalling presentational and contrastive focus alike. In the following three sentences, speaker (ma) talks about a dog that has been tied to a tree by the neighbours downstairs. After providing circumstantial information in (\ref{ex:key:693a}–b), new information is introduced by presentational focus in (c): 


\ea%693
    \label{ex:key:693}
\ea{\label{ex:key:693a}
\gll
Dɛn  táy=an.\\
  \textsc{3pl}  tie=\textsc{3sg.obj}\\
\glt   ‘They’ve tied it [that’s why it’s barking].’ [ma03hm 001]
}\ex{
\gll
Dɛn  gɛ́fɔ    mín    sé    e    de  hambɔ́g  wí.\\
  \textsc{3pl}  have.to  mean  \textsc{quot}    \textsc{3sg.sbj}  \textsc{ipfv}  irritate  \textsc{1pl.indp}\\
\glt   ‘They must mean to make it irritate us.’ [ma03hm 002]
}\ex{
\gll
\textbf{Na}  fɔ  mék    nó  gó  na  dɔ́n.\\
  \textsc{foc}  \textsc{prep}  make  \textsc{neg}  go  \textsc{loc}  down\\

\glt   ‘That’s in order for (us) not to go down.’ [ma03hm 003]
}\z\z

The use of contrastive\is{contrastive focus} focus is exemplified in the discourse excerpt below. In \REF{ex:key:694a}, speaker (dj) jokingly denies any involvement in the spell that has been cast on speaker (dj). Speaker (ru) retorts by contrastively focusing the \textsc{2sg} pronoun used in addressing his interlocutor in (b): 


\ea%694
    \label{ex:key:694}
\ea{\label{ex:key:694a}
\gll
Nó  mete  mí    ínsay  dí  tɔ́k  a    bɛ́g!\\
  \textsc{neg}  put    \textsc{1sg.indp}  inside  this  talk  \textsc{1sg.sbj}  ask.for\\
\glt   ‘Don’t involve me in this matter, please!’ [dj03wt 012]
}\ex{
\gll
\textbf{Na}  \textbf{yú}    mék=an.\\
  \textsc{foc}  \textsc{2sg.indp}  make=\textsc{3sg.obj}\\
\glt   ‘It’s you who made it.’ [ru03wt 013]
}\z\z

Both \textit{na} ‘\textsc{foc}’ and \textit{nóto} ‘\textsc{neg}.\textsc{foc}’ also function as copula-like elements in clauses like \REF{ex:key:695b}, in which a concrete entity is identified in discourse (cf. \sectref{sec:7.6.1} for an extensive treatment of the copula functions of \textit{na}\textit{/}\textit{nóto}). Likewise, \textit{na}\textit{/}\textit{nóto} occur as identity copulas in equative constructions like \REF{ex:key:696}, where we find nominal constituents on both sides of the copula: 

\ea%695
    \label{ex:key:695}
\ea{
\gll
\'{U}dat  de  hala-hála  só?\\
  who    \textsc{ipfv}  \textsc{red.cpd-}shout  like.that\\

\glt   ‘Who is shouting around like that?’
}\ex{\label{ex:key:695b}
\gll
\textbf{Na} \textstylePichiexamplenumberZchnZchn{chak-mán.}\\
  \textsc{foc}  drunk.\textsc{cpd}{}-man\\
\glt   ‘It’s a drunkard.’
}\z\z


\ea%696
    \label{ex:key:696}
    \gll Di  húman  \textbf{na} strɔ́n  húman.\\
\textsc{def}  woman  \textsc{foc}  strong  woman\\

\glt ‘The woman is a strong woman.’ [dj05ae 200]
\z

Presumably, the identificational function of \textit{na/nóto} in pragmatic contexts like \REF{ex:key:695b} is the point of departure for the focus-marking and identity (i.e. equative) functions of \textit{na/nóto} (\citealt[96]{HeineKuteva2002}). The difference between copula clauses and cleft focus has a structural correlate. In focus constructions, the out-of-focus part of the sentence is not normally expressed as a relative clause\is{relative clauses}. Compare the pragmatically neutral clause in \REF{ex:key:697a} and the corresponding focus construction (b), in which the relativiser \textit{wé} is absent (indicated by ${\emptyset}$): 


\ea%697
    \label{ex:key:697}
\ea{\label{ex:key:697a}
    \gll
Dɛn  sɛ́n    di  bɔ́l.\\
  \textsc{3pl}  send  \textsc{def}  ball\\

\glt   ‘The ball was thrown.’ [au07se 169]
}\ex{\label{ex:key:697b}
\gll
\textbf{Na}  pɔ́sin  ${\emptyset}$  sɛ́n    di  bɔ́l.\\
  \textsc{foc}  person  \textsc{sub}  send  \textsc{def}  ball\\
\glt   ‘It’s a person/somebody who threw the ball.’ [au07se 169]
}\z\z

In copula clauses, however, the use of an overt relative clause introduced by \textit{wé} ‘\textsc{sub}’ is obligatory if the identified entity is to be modified by a clause. In \REF{ex:key:698a}, new information is introduced. This given information is implicitly referred to by sentence (b), which is therefore best seen to constitute an equative clause rather than a focus construction: 


\ea%698
    \label{ex:key:698}
\ea{\label{ex:key:698a}
    \gll
Háw    yu  kin  kɔ́l=an    wé  pɔ́sin  de  siente  vergüenza?\\
  how    \textsc{2sg}  \textsc{hab}  call=\textsc{3sg.obj}  \textsc{sub}  person  \textsc{ipfv}  feel    shame\\

\glt   ‘How do you call it, when a person feels ashamed?’ [ko0505e3]
}\ex{
\gll
\textbf{Na} \textstylePichiexamplenumberZchnZchn{pɔ́sin} \textbf{wé}  de  fíl  sém.\\
  \textsc{foc}  person  \textsc{sub}  \textsc{ipfv}  feel  shame\\

\glt   ‘That’s a person who feels ashamed.’ [ro05fe 028]
}\z\z

The difference between copula predication and a focus structure can also be seen in the use of personal pronouns. In a copula construction, a \textsc{3sg} independent pronoun may be inserted before \textit{na}\textit{\textup{/}}\textit{nóto}:


\ea%699
    \label{ex:key:699}
    \gll \textup{(}\textbf{\'{I}n}\textup{)}    \textbf{na}  wán   mán  \textbf{wé}  de  plé  wet    di  bɔ́l.\\
\textsc{3sg.indp}  \textsc{foc}  one    man    \textsc{sub}  \textsc{ipfv}  play  with    \textsc{def}  ball\\

\glt ‘(He/that’s) a man who is playing with the ball.’ [ra07se 038]
\z

By comparison, the insertion of a \textsc{3sg} peronal pronoun is ungrammatical in the focus construction in \REF{ex:key:700}, since \textit{na/nóto} is non-referential in these constructions. Likewise, a cleft focus construction cannot be rephrased as a presentative clause (cf. \sectref{sec:7.4.4}):


\ea%700
    \label{ex:key:700}
    \gll \textup{(}*\'{I}n\textup{)}  \textbf{na}  wán    Annobón  gɛ́l    wích  yú?\\
\textsc{3sg.indp}  \textsc{foc}  one    \textsc{place}    girl    bewitch  \textsc{2sg.indp}\\

\glt ‘(*She) a girl from Annobón bewitched you?’ [fr03wt 002]
\z

\subsubsection{Eligible constituents and word order}\label{sec:7.4.3.2}
\is{word order}
Cleft constructions allow the focusing of constituents belonging to most word class\is{word classes}es. In cleft constructions, the focused constituents invariably appear sentence-initially, irrespective of their syntactic category.


In the overwhelming majority of cases, focused subjects are neither followed by an out-of-focus relative clause\is{relative clauses}, nor are they anaphorically referred to by a resumptive dependent subject pronoun (the latter is usually the case in subject relative clauses). Cleft focus and particle focus may occur together in the same clause as in this example: 



\ea%701
    \label{ex:key:701}
    \gll \textbf{Na}  Nguema  Mba    bin  gí  mí    dán    beca      \textbf{sɛ́f}.\\
\textsc{foc}  \textsc{name}  \textsc{name}  \textsc{pst}  give  \textsc{1sg.indp}  that    scholarship  \textsc{emp}\\

\glt \textsc{‘}It’s Nguema Mba (who) actually gave me that scholarship.’ [ed03sp 058]
\z

Cleft-focused non-subjects appear at the beginning of the sentence \REF{ex:key:702}. The use of resumptive pronoun\is{resumptive pronouns}s is not attested and the expression of the out-of-focus part of the sentence as a relative clause like in \REF{ex:key:703} is rare: 


\ea%702
    \label{ex:key:702}
    \gll \textbf{Na}  \textbf{wán}    \textbf{smɔ́l}  \textbf{híl}  e    klém.\\
\textsc{foc}  one    small  hill  \textsc{3sg.sbj}  climb\\

\glt ‘It’s a small hill that he climbed.’ [au07se 041]
\z


\ea%703
    \label{ex:key:703}
    \gll Wé  wi  smɔ́l,  \textbf{na}  \textbf{sósó}    \textbf{Píchi}  wé  wi  de  tɔ́k.\\
\textsc{sub}  \textsc{1pl}  be.small  \textsc{foc}  only    Pichi  \textsc{sub}  \textsc{1pl}  \textsc{ipfv}  talk\\

\glt ‘When we were small, it’s only Pichi that we would talk.’ [au07se 213]
\z

There are also numerous instances of focused adverbs. Compare the adverb \textit{só} ‘so, like this’ in \REF{ex:key:704}, which is often encountered in a cleft construction \textit{na só} ‘it’s like that, that’s how it is’, as well as focused \textit{dé} ‘there’ \REF{ex:key:705}: 


\ea%704
    \label{ex:key:704}
    \gll Sí,  \textbf{na}  \textbf{só}    mí    sɛ́f  kin  dé.\\
see  \textsc{foc}  like.that  \textsc{1sg.indp}  \textsc{emp}  \textsc{hab}  \textsc{be.loc}\\

\glt ‘See, it’s like that that I’m also usually like.’ [dj03cd 170]
\z


\ea%705
    \label{ex:key:705}
    \gll \textbf{Na}  \textbf{dé}    e    de  gó,  yu  nó    dé?\\
\textsc{foc}  there  \textsc{3sg.sbj}  \textsc{ipfv}  go  \textsc{2sg}  know  there\\

\glt ‘It’s there that she’s going, you know there [that place]?’ [ma03hm 029]
\z

The following two examples are of interest because they each present a focus-neutral clause and constituent focus in one sentence. In \REF{ex:key:706}, the manner adverbial \textit{rɔn-sáy} ‘backwards’ is first encountered in the clause-final adverbial position, then fronted for presentational focus in a \textit{na}{}-focus construction. The same applies to \textit{fá} ‘be far’, which is employed as a locative adverbial\is{locative adverbials} in \REF{ex:key:707}: 


\ea%706
    \label{ex:key:706}
    \gll E    de  wáka  rɔn-sáy,    \textbf{na}  \textbf{rɔn-sáy} e    wáka.\\
\textsc{3sg.sbj}  \textsc{ipfv}  walk  wrong-side  \textsc{foc}  wrong.\textsc{cpd}{}-side  \textsc{3sg.sbj}  walk\\

\glt  ‘He is walking backwards, it’s backwards that he walked.’ [au07se 047]
\z


\ea%707
    \label{ex:key:707}
    \gll E    sé    e    kɔmɔ́t    fá,  \textbf{na}  \textbf{fá} e    kɔmɔ́t.\\
\textsc{3sg.sbj}  \textsc{quot}    \textsc{3sg.sbj}  come.out  far  \textsc{foc}  far  \textsc{3sg.sbj}  come.out\\

\glt  ‘He said he came from far away, it’s far away that he was from.’ [ed03sb 186]
\z

Example \REF{ex:key:708} contains an instrumental prepostional phrase featuring the preposition \textit{wet} ‘with’:


\ea%708
    \label{ex:key:708}
    \gll \textbf{Na}  \textbf{wet}    \textbf{ús=tín}  dɛn  bíl=an?\\
\textsc{foc}  with    \textsc{q}=thing  \textsc{3pl}  build=\textsc{3sg.obj}\\

\glt ‘It’s with what\textstylePichiglossZchn{ that it was built?’} [dj07ae 480]
\z

Sequences of the homophones \textit{na} ‘\textsc{foc}’ and \textit{na} ‘\textsc{loc}’ are not attested. Hence, the use of a focused locative prepositional phrase featuring \textit{fɔ} ‘\textsc{prep}’ as a locative\is{general locative preposition} preposition serves as an alternative in \REF{ex:key:709}:


\ea%709
    \label{ex:key:709}
    \gll \textbf{Na}  \textbf{fɔ}  \textbf{dán}  \textbf{área}  wi  sté.\\
\textsc{foc}  \textsc{prep}  that  area    \textsc{1pl}  stay\\

\glt ‘It’s in that area that we stay.’ [hi03cb 071]
\z

Entire sentences may also be focused by means of the cleft construction. For one part, sentence clefting may be achieved by means of \textit{na/nóto} optionally followed by the quotative marker and complementiser \textit{sé}.


In \REF{ex:key:710}, we witness the use of \textit{na} \textit{sé} ‘it is that’ in order to focus a sentence containing the verb \textit{wánt} ‘want’ together with its subjunctive\is{subjunctive mood} complements. Besides cleft focus, this sentence exemplifies other features that characterise emphatic speech in Pichi: The TMA marker sequence \textit{dɔ́n} \textit{de} ‘\textsc{prf} \textsc{ipfv’} is employed instead of \textit{de} \textsc{‘ipfv’} alone, and the repetitive use of verbs with similar meanings serves as a means of emphatic reinforcement: 



\ea%710
    \label{ex:key:710}
    \gll Wé  yu  dɔ́n  de  nák,    \textbf{na}  \textbf{sé}    yu  wánt  sɔn    tín    
e    \textbf{brók},  \textbf{mék}    e    krás,  \textbf{mék}    e    \textbf{destroza}.\\
\textsc{sub}  \textsc{2sg}  \textsc{prf}  \textsc{ipfv}  hit    \textsc{foc}  \textsc{quot}    \textsc{2sg}  want  some  thing  
\textsc{3sg.sbj}  break   \textsc{sbjv}    \textsc{3sg.sbj}  crash  \textsc{sbjv}    \textsc{3sg.sbj}  destroy\\

\glt ‘When you’re hitting, it’s that you want a thing to break, to crash, to be 
destroyed.’ [au07se 245]
\z

\textit{Nóto sé} ‘it’s not that’ always signals contrastive focus of a clause or sentence \REF{ex:key:711}. In \REF{ex:key:712}, a conditional clause is singled out for focus. Hence, the negative focus marker \textit{nóto} appears after \textit{ɛf} ‘if’:


\ea%711
    \label{ex:key:711}
    \gll E    de  kráy    pero  \textbf{nóto}  \textbf{sé}    e    wánt  chɔ́p.\\
\textsc{3sg.sbj}  \textsc{ipfv}  cry    but    \textsc{neg}.\textsc{foc}  \textsc{quot}    \textsc{3sg.sbj}  want  eat\\

\glt ‘He is crying but it’s not the he wants to eat.’ [dj07ae 520]
\z


\ea%712
    \label{ex:key:712}
    \gll \textbf{Ɛf}  \textbf{nóto}  yu  báy  dán  húman  go  bít    yú    sóté  
yu  go  gó  lɛ́f=an.\\
if  \textsc{neg}.\textsc{foc}  \textsc{2sg}  buy  that  woman  \textsc{pot}  beat    \textsc{2sg.indp}  until
\textsc{2sg}  \textsc{pot}  go  leave=\textsc{3sg.obj}\\

\glt ‘If it’s not that [the correct type] you’ve bought, that woman 
would beat you until you would go return it.’ [ab03ab 033]\is{contrastive focus}
\z

The \ili{Spanish}-derived focus phrase \textit{es que} ‘it’s that’ consists of the \textsc{3sg} present tense form of the Spanish copula \textit{ser} and the complementiser \textit{que} ‘that’. The phrase is firmly entrenched in the Pichi lexicon \is{loan words}and signals affirmative focus of entire sentences. The phrase has an equivalent function in Spanish:


\ea%713
    \label{ex:key:713}
    \gll \textbf{Es}  \textbf{que}   está    bien    usar    el  subjuntivo.\\
It’s  that    it.is    good  use    the  subjunctive\\

\glt ‘It’s that it’s good to use the subjunctive (mood).’
\z

In \REF{ex:key:714}, the topical \textsc{NP} \textit{dí káyn pikín} ‘this kind of child’ is set off from the rest of the sentence by continuative intonation\is{continuative intonation} and a pause. The subsequent clause is under presentational focus with\textit{ es que} ‘it’s that’, and the topical \textsc{NP} is picked up by the resumptive pronoun \textit{e} ‘\textsc{3sg.sbj}’:


\ea%714
    \label{ex:key:714}
    \gll Entonces    \textbf{dí}  \textbf{káyn}  \textbf{pikín},  \textbf{es}  \textbf{que}    normalmente
e    go  tɛ́l  yú    dán  tín,  \op...\cp{}\\
so      this  kind    child  it.is  that    normally    
\textsc{3sg.sbj}  \textsc{pot} tell  \textsc{2sg.indp}  that  thing\\

\glt So this kind of child, it’s that usually it will tell you exactly that (...)’ [to03gm 052]
\z

In the example below, \textit{es que} is immediately followed by a locative adverbial, namely the prepositional phrase introduced by \textit{na} ‘\textsc{loc’}:


\ea%715
    \label{ex:key:715}
    \gll \textbf{Es}  \textbf{que}  \textbf{na}  \textbf{dán}  \textbf{klém}  \textbf{wé} e    de  klém,
e    de  gó e      de  klém.\\
it.is  that  \textsc{foc}  that  climb  \textsc{sub}  \textsc{3sg.sbj}  \textsc{ipfv}  climb
\textsc{3sg.sbj}  \textsc{ipfv}  go \textsc{3sg.sbj}    \textsc{ipfv}  climb\\

\glt ‘It’s that in that climb that she’s climbing, she’s just 
climbing along.’ [au07se 070]
\z

Cleft focus is characterised by a large degree of syntactic flexibility. For example, focusing into a relative clause\is{relative clauses} is permitted. Example \REF{ex:key:716} presents a subject relative clause featuring focus of a \textsc{3sg} person (i.e. \textit{na ín} ‘it’s him’), anaphoric to the preceding head nominal \textit{wán} ‘one (person)’: 


\ea%716
    \label{ex:key:716}
    \gll Bɛt  e    fíba      sé    \textbf{wán}    dé    wé  \textbf{na}  \textbf{ín}
de  púl  di  ɔ́da    wán    di  torí.\\
but  \textsc{3sg.sbj}  resemble    \textsc{quot}    one    \textsc{be.loc}  \textsc{sub}  \textsc{foc}  \textsc{3sg.indp}
\textsc{ipfv}  pull  \textsc{def}  other  one    \textsc{def}  story\\

\glt 
\textit{Lit.} ‘But it seems that one is there that it’s him\textstylePichiglossZchn{ who is telling the other}  one a story.’ [au07se 100]\is{cleft constructions}
\z

\subsubsection{Focus of resumptive elements}\label{sec:7.4.3.3}

Cleft constructions of the type in \REF{ex:key:716} above, where a resumptive element is focused, serve an important function in discourse{\fff}. They serve as anaphors that establish reference to preceding topical material in the sentence or the paragraph. The relevant collocations involve the focus particle \textit{na} ‘\textsc{foc}’ followed by the adverbs \textit{yá} ‘here’, \textit{dé} ‘there’, \textit{só} ‘so, like that’, the personal pronoun \textit{ín} ‘\textsc{3sg.indp}’, as well as complex NPs like \textit{dán tɛ́n} ‘that time’ and \textit{di tín} ‘the thing’. Mostly, these collocations function as resumptive adverbials of location, time, or cause, but ín ‘\textsc{3sg.indp}’ may also refer to preceding subjects and objects{\fff}. 


In \REF{ex:key:717}, the topical, clefted adverbial phrase \textit{frɔn in hós} ‘from her house’ is anaphorically referred to by another clefted adverbial, namely \textit{dé} ‘there’: 



\ea%717
    \label{ex:key:717}
    \gll Na  frɔn    in    hós,    \textbf{na}  \textbf{dé}    yu  go  ték    máred.\\
\textsc{foc}  from  \textsc{3sg.poss}  house  \textsc{foc}  there  \textsc{2sg}  \textsc{pot}  take    marry\\

\glt ‘It’s from her house\textstylePichiglossZchn{, it’s} there\textstylePichiglossZchn{ that you’d enter marriage.’} [ab03ay 033]
\z

A similar anaphoric relation holds between \textit{di sáy} ‘the place’ and \textit{na dé} ‘it’s there’ in \REF{ex:key:718}. In fact, the deictic locative adverbs\is{locative adverbials} \textit{dé} ‘there’ and \textit{yá} ‘here’, as well as the deictic manner adverbial \textit{só} ‘like that’ need to be clefted in this way, if they are to appear in the clause-initial, rather than their usual clause-final position:


\ea%718
    \label{ex:key:718}
    \gll Di  sáy  wé  mɔní  dé,    \textbf{na}  \textbf{dé}    yu  gɛ́fɔ    gó.\\
\textsc{def}  side   \textsc{sub}  money  \textsc{be.loc}  \textsc{foc}  there  \textsc{2sg}  have.to  go\\

\glt ‘The place where there’s money, that’s where you have to go.’
\z

An anaphoric temporal relation may also be established by means of the locative adverbs \textit{yá} ‘here’ and \textit{dé} ‘there’. In \REF{ex:key:719}, the left-dislocated and topical \ili{Spanish} adverbial \textit{a los quince años completamente} is picked up by the resumptive focus construction \textit{na yá} ‘\textsc{foc} here’ = ‘that’s when’. The same principle is at work in \REF{ex:key:720}, where \textit{na dé} refers to a preceding time clause earlier in the paragraph: 


\ea%719
    \label{ex:key:719}
    \gll A  los    quince  años,  \textbf{na}  \textbf{yá}   e    kán.\\
at  \textsc{def.pl}  fifteen  years  \textsc{foc}  here    \textsc{3sg.sbj}  come\\

\glt ‘With exactly fifteen years, that’s when she came.’ [ab03ay 156]
\z


\ea%720
    \label{ex:key:720}
    \gll \textbf{Na}  \textbf{dé}   dán,    dán  kandá,  dán  tín,    dán  membrano,
na  dé    e    kɔmɔ́t.\\
\textsc{foc}  there  that    that  skin    that  thing  that  membrane
\textsc{foc}  there  \textsc{3sg.sbj}  go.out\\

\glt ‘That’s when that, that skin, that thing, that membrane, that’s when 
it came out.’ [ab03ay 093]
\z

The collocation \textit{na ín} features the emphatic \textsc{3sg} pronoun \textit{ín}, which functions as a “catch-all” anaphora. Hence, it may refer to a preceding subject, object, or time or cause adverbial. The exact nature of the anaphoric relation that holds between \textit{na ín} and its antecedent is therefore determined by context.


In \REF{ex:key:721}, \textit{na} \textit{ín} refers to the antecedent subject under focus \textit{na} \textit{di} \textit{fáyn} \textit{chɔ́p} ‘it’s the good food’. Example \REF{ex:key:722} features a resumptive \textit{na} \textit{ín} anaphorical to the dislocated, topical object \textit{dís} \textit{traje} \textit{fɔ} \textit{mono} ‘this overall-like suit’:



\ea%721
    \label{ex:key:721}
    \gll Na  di  fáyn    chɔ́p,  \textbf{na}  \textbf{ín}   de  stáwt=an.\\
\textsc{foc}  \textsc{def}  fine    food    \textsc{foc}  \textsc{3sg.indp}  \textsc{ipfv}  make.corpulent=\textsc{3sg.obj}\\

\glt ‘It’s the good food, that’s what’s making her corpulent.’ [dj07ae 170]
\z


\ea%722
    \label{ex:key:722}
    \gll Tɛ́l=an    sé,    nɔ́,  dís  traje  fɔ  mono,
\textbf{na}  \textbf{ín}    e    wánt.\\
tell=\textsc{3sg.obj}  \textsc{quot}    \textsc{intj}  this  suit    \textsc{prep}  overall
\textsc{foc}  \textsc{3sg.indp}  \textsc{3sg.sbj}  want\\

\glt ‘(He) told him, no, this overall-like suit, that’s what 
he wants.’ [to03gm 004]
\z

In \REF{ex:key:723}, \textit{na ín} refers to an antecedent time clause introduced by \textit{wé} ‘\textsc{sub}’. When there is a relation of temporal succession like in this example, it is only natural that the \textit{wé}{}-clause precedes the main clause:


\ea%723
    \label{ex:key:723}
    \gll Wé  e    dɔ́n  dé    pan  di  chía,  \textbf{na}  \textbf{ín}    e    strét.\\
\textsc{sub}  \textsc{3sg.sbj}  \textsc{prf}  \textsc{be.loc}  on  \textsc{def}  chair  \textsc{foc}  \textsc{3sg.indp}  \textsc{3sg.sbj}  be.straight\\

\glt ‘When she was completely on the chair, that’s when\textstylePichiglossZchn{ she straightened up.’} 
[au07se 089]
\z

In turn, cause clauses\is{cause clauses} are more likely to follow their main clauses. As a consequence, sentence-initial cause clauses are in-focus by default, and are therefore quite often additionally marked for focus in a cleft construction. 


Whenever this the case, the phrasal expressions \textit{na ín} \textit{(mék)} ‘\textsc{foc} \textsc{3sg.indp} (make)’ = ‘that’s why’ \REF{ex:key:724} or alternatively, \textit{na di tín (mék)} ‘\textsc{foc} \textsc{def} thing (make)’ = ‘that’s why’ \REF{ex:key:725} may refer anaphorically to the preceding cause clause (cf. \ref{ex:key:646} for the analoguous content question):



\ea%724
    \label{ex:key:724}
    \gll \textbf{Na}  \textbf{bikɔs}  in    abuelo    dɔ́n  dáy,  
\textbf{na}  \textbf{ín}    e    de  kráy.\\
\textsc{foc}  because  \textsc{3sg.poss}  grandfather  \textsc{prf}  die 
\textsc{foc}  \textsc{3sg.indp}  \textsc{3sg.sbj}  \textsc{ipfv} cry\\

\glt ‘It’s because his grandfather has died, that’s why he’s crying like that.’ [dj05be 046]
\z


\ea%725
    \label{ex:key:725}
    \gll \textbf{Na}  \textbf{bikɔs}  dɛn  púl    di  motó,  \textbf{na}  \textbf{di}  \textbf{tín}
\textbf{mék}    e    chakrá.\\
\textsc{foc}  because  \textsc{3pl}  remove  \textsc{def}  car    \textsc{foc}  \textsc{def}  thing
make  \textsc{3sg.sbj}  destroy\\

\glt ‘It’s because the car was removed, that’s why it got broken.’ [dj05be 047]\is{focus}
\z

Amongst the sentences involving focus of resumptive elements presented so far, we also find focused constituents appearing in the initial position which are not preceded by the focus marker \textit{na} (eg. \ref{ex:key:719} and \ref{ex:key:722}). There is no reason to see these structures as being fundamentally different from cleft constructions involving the focus marker \textit{na}. The only thing “missing” in these constructions is the focus particle.

\subsection{Presentatives}\label{sec:7.4.4}

Pichi features a presentative construction involving \textit{na}\textit{\textup{/}}\textit{nóto} as well as the proximal and distal demonstrative forms \textit{dís} ‘this’ \REF{ex:key:726} and \textit{dat} ‘that’ \REF{ex:key:727} in sentence-final position. Presentatives may be seen as inverted copula clauses with particular deictic force, which direct an addressee’s attention to, and identify, an entity. By highlighting an entity in this way, presentatives manifest a functional overlap with (presentational) cleft constructions: \is{cleft constructions}


\ea%726
    \label{ex:key:726}
    \gll E    sé    \textbf{na} mán    \textbf{dís}.\\
\textsc{3sg.sbj}  \textsc{quot}    \textsc{foc}  man    this\\

\glt ‘He said “this is a man”.’ [ed03sb 224]
\z


\ea%727
    \label{ex:key:727}
    \gll \textbf{Na} róp    \textbf{dát}.\\
\textsc{foc}  rope  that\\

\glt ‘That’s a rope.’ [li07pe 002]
\z

Examples (\ref{ex:key:726}–\ref{ex:key:727}) may also be expressed with less deictic force as regular copula clauses. The following two equative clauses feature the demonstratives \textit{dí} ‘this’ and \textit{dán} ‘that’ in the ordinary prenominal position. When employed in an NP in this way, demonstratives may be realised as the short forms \textit{dí} and \textit{dá} respectively. However, these apocopated forms do not occur in sentence-final position in presentatives like (\ref{ex:key:726}–\ref{ex:key:727}) above: 


\ea%728
    \label{ex:key:728}
    \gll \textbf{Dí}  wán    \textbf{na} bíf.\\
this  one    \textsc{foc}  wild.animal\\

\glt ‘This (one) is a wild animal.’ [ma03sh 011]
\z


\ea%729
    \label{ex:key:729}
    \gll \textbf{Dá}  wán    \textbf{na} bɔbí.\\
that  one    \textsc{foc}  breast\\

\glt ‘That (one) is the breast.’ [dj05ce 209]
\z

The highlighted NP of a presentative construction may be modified by further constituents in the same way as a nominal participant in an equative clause. In \REF{ex:key:730}, the NP \textit{chɔ́p} ‘food’ has been modified prenominally by \textit{bɛ́ta} ‘very good’ and post-nominally by a relative clause{\fff} introduced by \textit{wé} ‘\textsc{sub}’: 


\ea%730
    \label{ex:key:730}
    \gll \textbf{Na}  di  bɛ́ta      chɔ́p  \textbf{wé}  mán    de  chɔ́p  dát.\\
\textsc{foc}  \textsc{def}  very.good  food    \textsc{sub}  man    \textsc{ipfv}  food    that\\

\glt ‘That was the best food that one [I] was eating.’ [ed03sp 123]
\z

Content questions\is{content questions} may also be formulated as presentatives \REF{ex:key:731}, in which case they may occur without a preposed \textit{na.} This distribution may be linked to the fact that questioned constituents are focused by default, and may optionally co-occur with \textit{na}{}-focus anyway (cf. \sectref{sec:7.3.2}):


\ea%731
    \label{ex:key:731}
    \gll Sé  papá  gɔ́d  \textbf{ús}=\textbf{káyn}  trɔ́bul  \textbf{dís}?\\
\textsc{quot}  father  God  \textsc{q}=kind  trouble  this\\

\glt ‘(I) said God, what (kind of) trouble (is) this?’ [ab03ab 082]\is{demonstratives}
\z

\subsection{Predicate cleft}\label{sec:7.4.5}

Besides focus of verbs by means of clausal focus, verbs may be singled out for focus individually in a construction termed “predicate clefting” (e.g. \citealt{Koopman1984}; \citealt{LarsonLefebvre1991}) or “verb fronting\is{fronting}” \citep{Muysken1978}.


In Pichi predicate cleft constructions, the focused verb appears twice in the sentence: fronted in the initial focus position directly after the focus marker \textit{na}, and at the same time in its original syntactic position in the out-of-focus part of the sentence. Compare the following example featuring the clefted dynamic verb \textit{gó} ‘go’. It is noteworthy that a negative predicate cleft by means of \textit{nóto} ‘\textsc{neg.foc}’ is not attested: 



\ea%732
    \label{ex:key:732}
    \gll \textbf{Na}  \textbf{gó}  a    de  \textbf{gó}  ó.\\
\textsc{foc}  go  \textsc{1sg.sbj}  \textsc{ipfv}  go  \textsc{sp}\\

\glt ‘[Mind you] I’m going.’ [ch07fn 151]
\z

Predicate cleft signals presentational or contrastive focus of the predicate and produces intensifying, emphatic meanings. It should therefore be seen as part of the range of emphatic structures that involve iteration in Pichi (i.e. reduplication\is{reduplication} and repetition\is{repetition}, cf. \sectref{sec:4.5} and the use of cognate objects, cf. \sectref{sec:9.3.3}). Neither temporal or causal adverbial meanings, nor factive clauses are expressed through predicate cleft. In natural speech, predicate cleft almost exclusively occurs with dynamic verbs, as in the example above. In fact, the natural speech data in my corpus reveals relatively few instances of predicate cleft constructions in general.


The predicate cleft construction in \REF{ex:key:733} features a stative verb, the property item \textit{bíg} ‘be big’. Like other cleft constructions, predicate cleft does not require marking of the out-of-focus part of the sentence as a relative clause\is{relative clauses}. This is, in fact rejected in unison by all speakers who were asked about this possibility \REF{ex:key:734}: 



\ea%733
    \label{ex:key:733}
    \gll Chico,  \textbf{na}  \textbf{bíg}   e    \textbf{bíg}.\\
boy    \textsc{foc}  big    \textsc{3sg.sbj}  big\\

\glt ‘Oh boy, it’s really big.’ [ye07fn 070]
\z


\ea[*]{%734
    \label{ex:key:734}
    \gll \textbf{Na}  bíg  \textbf{wé}  e  bíg.\\
 \textsc{foc}  big  \textsc{sub}  \textsc{3sg.sbj}  big\\
\glt Intended: ‘It’s really big.’ [ne07fn]
}\z

Sometimes verbs are clefted together with a pronominal object \REF{ex:key:735}. If this is the case, the pronominal object is not repeated with the second verb. The fronting\is{fronting} of a subject or adverbial modifier together with the verb is not accepted \REF{ex:key:736}:


\ea%735
    \label{ex:key:735}
    \gll Na  \textbf{krách}=an    yu  de  \textbf{skrách}.\\
\textsc{foc}  scratch=\textsc{3sg.obj}  \textsc{2sg}  \textsc{ipfv}  scratch \\

\glt ‘You’re actually scratching it.’ [dj07ae 386]
\z


\ea[*]{%736
    \label{ex:key:736}
    \gll Na  \textbf{lúk}  \textbf{fáyn}  yu  \textbf{lúk}.\\
 \textsc{foc}  look  fine    \textsc{2sg}  look\\
\glt Intended: You looked really well. [ne07fn]
}\z

However, verbs are not clefted together with TMA markers \REF{ex:key:737}. These always remain in their “original” position with the second verb. The two following examples are of interest because they involve clefting of the major verb of a motion-direction SVC. As these examples show, the minor verb \textit{gó} ‘go’ remains in its original syntactic position \REF{ex:key:738}: \is{motion verbs}


\ea%737
    \label{ex:key:737}
    \gll Na  wáka  wi  \textbf{bin}  \textbf{de} wáka  \textbf{gó}  dé.\\
\textsc{foc}  walk  \textsc{1pl}  \textsc{pst}  \textsc{ipfv}  walk  go  there\\

\glt ‘We actually walked there.’ [pa07me 002]
\z


\ea%738
    \label{ex:key:738}
    \gll Na  wáka  wi  \textbf{wáka}  \textbf{gó}  dé.\\
\textsc{foc}  walk  \textsc{1pl}  walk  go  there\\

\glt ‘We walked there.’ [pa07me 003]
\z

The same holds for complements of auxiliaries. In \REF{ex:key:739}, it is once again only the major verb \textit{wáka} ‘walk’ that gets fronted, while the modal auxiliary verb \textit{wánt} ‘want’ stays behind: 


\ea%739
    \label{ex:key:739}
    \gll Na  \textbf{wáka}  e    \textbf{wánt}  \textbf{wáka}  só.\\
\textsc{foc}  walk  \textsc{3sg.sbj}  want  walk  like.this\\

\glt ‘He really wants to walk right now.’ [pa07me 008]
\z

A few Pichi verbs have homophonous nominal counterparts which are not merely action nominalisations. One of these is \textit{chɔ́p}, which means ‘eat’ as a verb and ‘food’ (rather than only ‘eating’) as a noun. While \REF{ex:key:740} may be interpreted as involving either predicate or nominal cleft, the cleft construction in \REF{ex:key:741} is unlikely to be anything else than a nominal cleft construction, since the focused noun \textit{chɔ́p} ‘food’ is modified by \textit{bɔkú} ‘be much’:


\ea%740
    \label{ex:key:740}
    \gll \textbf{Na}  \textbf{chɔ́p}    e    \textbf{chɔ́p}  yɛ́stadé    ó.\\
\textsc{foc}  eat/food    \textsc{3sg.sbj}  eat    yesterday  \textsc{sp}\\

\glt ‘He really ate yesterday.’ or ‘It’s (really good) food that he ate yesterday.’ [dj07ae 463]
\z


\ea%741
    \label{ex:key:741}
    \gll \textbf{Na}  \textbf{bɔkú}  \textbf{chɔ́p}  e    kin  \textbf{chɔ́p}.\\
\textsc{foc}  much  food    \textsc{3sg.sbj}  \textsc{hab}  eat\\

\glt ‘It’s a hell of a lot of food that he usually eats.’ [dj07ae 462]
\z

\subsection{Other means of expressing emphasis}

Focus constructions frequently come along with a variety of other emphatic elements and structures which breathe life into discourse and signal speaker involvement. 


For example, the TMA marker sequences \textit{dɔ́n} \textit{de} ‘\textsc{pfv} \textsc{ipfv}’ and \textit{dɔ́n} \textit{de} \textit{fínis} ‘\textsc{pfv} \textsc{ipfv} finish’, rather than the imperfective\is{imperfective aspect} marker \textit{de} alone, may be recruited in order to emphasise that the situation designated by the verb is in full course. 



In \REF{ex:key:742}, NP focus (i.e. \textit{dís wán sɛ́f} ‘this one \textsc{emp}’ co-occurs with a predicate featuring the perfect marker \textit{dɔ́n} and the imperfective marker \textit{de}. Sentence \REF{ex:key:743} additionally features the completive aspect{\fff} auxiliary verb \textit{fínis} ‘finish (doing something)’, which adds even more emphatic force: 



\ea%742
    \label{ex:key:742}
    \gll Dís  wán    \textbf{sɛ́f},  yu  \textbf{dɔ́n}  \textbf{de}  tráy.\\
this  one    \textsc{emp}  \textsc{2sg}  \textsc{prf}  \textsc{ipfv}  try\\

\glt ‘Even this [little Bube that you speak], you’re really making an effort.’ [ab03ab 014]
\z


\ea%743
    \label{ex:key:743}
    \gll Náw    a    \textbf{dɔ́n}  \textbf{de}  \textbf{fínis}  sém  
fɔ  wɛ́r    dán    sús.\\
now    \textsc{1sg.sbj}  \textsc{prf}  \textsc{ipfv}  finish  be.ashamed
\textsc{prep}  wear  that    shoe\\

\glt ‘Now I’m really ashamed to be wearing that (pair of) shoes.’ [ma03hm 021]
\z

Adverbial modification, for example via the value property items \textit{fáyn} ‘be fine’ and \textit{bád} ‘be bad’ or the quantity property item \textit{bɔkú} ‘be much’ \REF{ex:key:744}, may also express emphasis by itself or in conjunction with other elements and/or focus constructions. The use of the demonstrative determiner \textit{dán} ‘that’ together with the possessive construction\is{possessive constructions} \textit{in yáy} ‘his eye’ builds up additional emphatic force in \REF{ex:key:744}:


\ea%744
    \label{ex:key:744}
    \gll E    de  para    na  dán  in    yáy  \textbf{bɔkú}  \textbf{bád.}\\
\textsc{3sg.sbj}  \textsc{ipfv}  stand  \textsc{loc}  that  \textsc{3sg.poss}  eye  much  bad\\

\glt ‘It [the white spot in his eye] just sits there in that his eye really bad.’ [ye03cd 109]
\z

Other means of expressing emphasis and by extension various nuances of sentential focus are the segmental and suprasegmental means outlined in \sectref{sec:3.2.5}, \sectref{sec:3.4.2} and \sectref{sec:7.7.3}, the various forms of iteration, i.e. repetition\is{repetition} \REF{ex:key:745} and reduplication, predicate cleft, and cognate objects \is{cognate objects}– the latter in combination with the particle \textit{ó} in \REF{ex:key:746} as well as ideophones\is{ideophones} \REF{ex:key:747}:


\ea%745
    \label{ex:key:745}
    \gll Ɛ́n,  bɔt  ín    sidɔ́n  \textbf{dɔ́n}    \textbf{dɔ́n}     \textbf{dɔ́n}    yandá.\\
yes  but  \textsc{3sg.indp}  stay    down  \textsc{rep}    \textsc{rep}    yonder\\

\glt ‘Yes, but he stays far down there.’ [ma03ni 020]
\z


\ea%746
    \label{ex:key:746}
    \gll Dɛn  bin  \textbf{fáyn}  wán    \textbf{fáyn}  \textbf{ó}.\\
\textsc{3pl}  \textsc{pst}  fine    one    fine    \textsc{sp}\\

\glt ‘They were really beautiful.’ [mi07fn 120]
\z


\ea%747
    \label{ex:key:747}
    \gll Dɛn  nák=an    na  in    chɛ́s    \textbf{kip}.\\
\textsc{3pl}  hit=\textsc{3sg.obj}  \textsc{loc}  \textsc{3sg.poss}  chest  \textsc{ideo}\\

\glt ‘They hit him (hard) in the chest with a thumping sound.’ [dj05ce 100]\is{emphasis}
\z

\section{Topic}\label{sec:7.5}

Topicalisation involves dislocation: The topic appears at the beginning of the sentence and is reiterated in the original syntactic position by a resumptive pronoun. A topic is often set off from the remainder of the sentence by a short pause and a continuative\is{continuative intonation} boundary tone. The element \textit{náw} ‘now’ may optionally function as a post-posed topic marker.

\subsection{Dislocation}\label{sec:7.5.1}

There is a strong tendency for definite subject \textsc{NPs} to be marked as topical by an intonation break, i.e. a short pause and/or continuative intonation\is{continuative intonation}, and a resumptive subject pronoun (cf. also \sectref{sec:7.1.1}). The definite subject in \REF{ex:key:748} is set off from the rest of the clause by an intonation break, indicated by a comma. At the same time, the following coreferential resumptive pronoun\is{resumptive pronouns} \textit{e} ‘\textsc{3sg.sbj}’ reiterates the topical subject \textsc{NP} \textit{dán skúl} ‘that school’:


\ea%748
    \label{ex:key:748}
    \gll \textbf{Dán}    \textbf{skúl},  \textbf{e}    dé    nía    bɛrin-grɔ́n,    nɔ́?\\
that    school  \textsc{3sg.sbj}  \textsc{be.loc}  near    burial.\textsc{cpd}{}-ground  \textsc{intj}\\

\glt ‘That school is near the cemetery, right?’ [ma03hm 018]
\z

In contrast, the data does not contain a single instance of a resumptive subject pronoun in a clause featuring an indefinite subject. Such clauses are formed in the way of \REF{ex:key:749} without a resumptive pronoun: 


\ea%749
    \label{ex:key:749}
    \gll Wán    dé  wán  pikín  bin  de  sík.\\
one    day  one  child  \textsc{pst}  \textsc{ipfv}  sick\\

\glt ‘One day, a child was sick.’ [ye03cd 071]
\z

Non-subject topical \textsc{NPs} also appear at the beginning of the sentence, are normally separated from the rest of the clause by an intonation break, and are referred to by a resumptive element in the clause. The dislocated object \textit{di cartón} ‘the carton’ in \REF{ex:key:750} is resumed by the coreferential object pronoun \textit{àn} ‘\textsc{3sg.obj}’:


\ea%750
    \label{ex:key:750}
    \gll \textbf{Dí}  \textbf{cartón},  \textbf{e} mít=an    yá?\\
this  carton  \textsc{3sg.sbj}  meet=\textsc{3sg.obj}  here\\

\glt ‘This cardboard box, did she find it here?’ [li07pe 070]
\z

In \REF{ex:key:751}, the topical object NP \textit{ɛ́ni tín} ‘everything’ is reiterated by the resumptive, coreferential object prounoun \textit{=an} ‘\textsc{3sg.obj}’ after the verb \textit{púl} ‘remove’:


\ea%751
    \label{ex:key:751}
    \gll \textbf{Ɛ́ni}    \textbf{tín},    yu  wɔ́nt  púl=\textbf{an} 
na  puerto  yu  de  pé. \\
every  thing  \textsc{2sg}  want  remove=\textsc{3sg.obj} 
\textsc{loc}  harbour  \textsc{2sg}  \textsc{ipfv}  pay\\

\glt ‘Everything, you want to remove it from the port, you pay [tax].’ [f103fp 002]
\z

Sentence \REF{ex:key:752} involves the initial, dislocated topical object pronoun \textit{mí} ‘\textsc{1sg.indp}’, which is reiterated in the object position after \textit{sí} ‘see’ and anaphorically referred to by \textit{a} ‘\textsc{1sg.sbj}’: \is{anaphora}


\ea%752
    \label{ex:key:752}
    \gll \textbf{Mí},    lɛk  háw    yu  de  sí  \textbf{mí},    \textbf{a}    dɔ́n  
sí  plɛ́nte  tín.\\
\textsc{1sg.indp}  like  how    \textsc{2sg}  \textsc{ipfv}  see  \textsc{1sg.indp}  \textsc{1sg.sbj}  \textsc{prf}   
see  plenty  thing \\

\glt ‘As for me, as you see me (now), I’ve seen many things (in life).’ [ab03ab 023]
\z

The resumptive pronoun of an antecedent, dislocated topic may also be focused in a cleft construction. Such cross-cutting topic-focus structures are very common in Pichi. In the following sentence, the topical subject \textsc{NP} \textit{Panyá} ‘Spain’ is picked up by the coreferential \textsc{3sg.indp} pronoun \textit{ín}, which is, in turn, focused in a cleft construction (cf. also \sectref{sec:7.4.3.3}):


\ea%753
    \label{ex:key:753}
    \gll \textbf{Panyá},  \textbf{na}  \textbf{ín}    wɔ́s      mɔ́.\\
Spain  \textsc{foc}  \textsc{3sg.indp}  be.very.bad  more\\

\glt ‘As for Spain, that’s what’s really terrible [as a place to live in].’ [07fn 040]
\z

Certain types of adverbial clauses are more likely to precede their main clauses than to follow them. When such adverbial clauses do precede their main clauses they usually are topical, and may be set off from the following part of the sentence by an intonation break as well. Compare the purpose clause\is{purpose clauses} beginning with \textit{fɔ} ‘\textsc{prep’} in \REF{ex:key:754}:


\ea%754
    \label{ex:key:754}
    \gll \textbf{Fɔ}  \textbf{tɔ́k}  \textbf{Píchi},  yu  nó  níd    fɔ  gó  skúl.\\
\textsc{prep}  talk  Pichi  \textsc{2sg}  \textsc{neg}  need  \textsc{prep}  go  school\\

\glt ‘In order to talk Pichi, you don’t need to go to school.’ [au07se 267]
\z

Sentence \REF{ex:key:755} involves the rather rare case of a right-dislocated, topical (and nominalised hence non-finite) clause namely \textit{fɔ pút nivel} ‘to level the ground’. This last example also shows once more that the transition is smooth to focus marking, since \REF{ex:key:755} may also be seen as an example of pseudo-clefting\is{pseudo-cleft constructions}:\is{resumptive pronouns} 


\ea%755
    \label{ex:key:755}
    \gll Di  tín    wé  bin  dé    difícil  mɔ́    na  dí  hós,  
\textbf{fɔ}  \textbf{pút}  \textbf{nivel}. \\
\textsc{def}  thing  \textsc{sub}  \textsc{pst}  \textsc{be.loc}  difficult  more  \textsc{loc}  this  house
\textsc{prep}  put  level\\
\glt ‘The thing that was most difficult in [building] this house, [was]
to level the ground.’ [07fn 065]\is{dislocation}
\z

\subsection{Topic particle}\label{sec:7.5.2}

It has been shown that dislocation and intonation are by themselves sufficient means of indicating the topicality of a constituent. In addition to dislocation, the adverbial \textit{náw} ‘now’ may optionally indicate the topicality of a constituent. A particle is, however, not obligatory, often accompanied by an intonation break, and in most cases, by a resumptive element in the clause.


Apart from being used to signal topicality, the particle \textit{náw} ‘now’ is a time adverbial \REF{ex:key:756}, which may occur in presentational sentences like the following:



\ea%756
    \label{ex:key:756}
    \gll \textbf{Náw}    e    tínap  na  grɔ́n.\\
now    \textsc{3sg.sbj}  stand  \textsc{loc}  ground\\

\glt ‘Now it’s standing on the ground.’ [li07pe 093]
\z

Sentence \REF{ex:key:757} below is a metacomment in which speaker (dj) classifies the term \textit{mɔnt} ‘month’ as an English word (a more current Pichi term is \textit{mún} ‘moon, month’). In this example, the post-posed particle \textit{náw} signals the topicality of \textit{mɔ́nt}. 


\ea%757
    \label{ex:key:757}
    \gll “\textbf{Mɔ́nt}”  \textbf{náw},  \textbf{e}    dɔ́n  bí  inglés.\\
\phantom{“}month  now    \textsc{3sg.sbj}  \textsc{prf}  \textsc{be}  English\\

\glt ‘As for “mɔnt”, it’s already English.’ [dj05ce 030]
\z

Sometimes we encounter sentences in which the topic is not reiterated in a syntagmatic relation within the clause. In such cases, the topic functions like in many topic-prominent languages: It is adjoined to the clause and provides a referential frame, within which the precise relation between topic and comment is recovered by pragmatic context (cf. \citealt{LiThompson1976}). For example, in \REF{ex:key:758}, the topicality of \textit{pikín} ‘child(ren)’ is signalled by \textit{náw} ‘now’ and an intonation break. However, the “resumptive” pronoun \textit{e} ‘\textsc{3sg.sbj}’ does not refer to the topical syntactic subject \textit{pikín}. Instead, \textit{e} ‘\textsc{3sg.sbj}’ refers to a concept as a whole, namely procreation, which is loosely referred to by the topic \textit{pikín}:


\ea%758
    \label{ex:key:758}
    \gll \textbf{Pikín}  \textbf{náw},  \textbf{e}  nó  hád.\\
child  now  \textsc{3sg.sbj}  \textsc{neg}  hard\\

\glt ‘As for [having] kids, that’s not difficult.’ [hi03cb 162]
\z

Example \REF{ex:key:759} presents the topical and focused \textsc{NP} \textit{sósó Píchi} ‘only Pichi’, however without the focus marker \textit{na} ‘\textsc{foc}’. The topic is followed by \textit{náw} now’ and fronted\is{fronting}. The out-of-focus part of the sentence is exceptionally expressed in a relative clause\is{relative clauses}: 


\ea%759
    \label{ex:key:759}
    \gll \textbf{Sósó}  \textbf{Píchi}  \textbf{náw}    \textbf{wé}  wi  de  tɔ́k.\\
only    Pichi  now    \textsc{sub}  \textsc{1pl}  \textsc{ipfv}  talk\\

\glt ‘(It was) only Pichi that we used to talk.’ [au07se 214]
\z

Example \REF{ex:key:760} below features the \textsc{3sg} personal pronoun \textit{ín} ‘\textsc{3sg.indp}’ under assertive focus by means of clefting and additional topic marking by means of a post-posed \textit{náw} ‘now’:


\ea%760
    \label{ex:key:760}
    \gll \textbf{Na}  \textbf{ín}    \textbf{náw}    a    bin  de  chɛ́k  sé    e    bin  fɔ 
dé    fáyn    if  a    mít    wán    pɔ́sin  \op...\cp{}\\
\textsc{foc}  \textsc{3sg.indp}  now    \textsc{1sg.sbj}  \textsc{pst}  \textsc{ipfv}  check  \textsc{quot}    \textsc{3sg.sbj}  \textsc{pst}  \textsc{cond} 
\textsc{be.loc}  fine    if  \textsc{1sg.sbj}  meet  one    person\\

\glt ‘That’s why I was thinking it would be fine if I met somebody (...)’ [fr03ft 176]\is{topic marker}
\z

\section{\textsc{being} and \textsc{having}}\label{sec:7.6}

The forms employed to express \textsc{being} and \textsc{having} in Pichi form part of a web of interlinked and overlapping functions which extends from the formation of focus structures and copula clauses to the expression of possession and the formation of predicate adjective clauses. An important feature of the expression of both \textsc{being} and \textsc{having} is the notion of time-stability.

\begin{figure}
%\todo[inline]{needs vernacular and solid/dashed}
\begin{tikzpicture}
\node[rectangle,draw,align=center] (location) {\textsc{location}\\\textit{dé}};
\node[rectangle,draw,align=center] (cos)  [left=of location]{\textsc{change-of-state}\\\textit{tɔ́n, lɛ́f, kɔmɔ́t}};
\node[rectangle,draw,align=center] (identity) [above=of cos]{\textsc{identity}\\\textit{na, nóto, bí}};
\node[rectangle,draw,align=center] (focus) [left=of identity] {\textsc{focus}\\\textit{na, nóto}};
\node[rectangle,draw,align=center] (tempposs)  [above=of location,xshift=34mm,yshift=3mm]{\textsc{temporary possession}\\\textit{hól, dé na hán}};
\node[rectangle,draw,align=center] (existence)  [above=of tempposs,xshift=-34mm]{\textsc{existence}\\\textit{dé, gɛ́t}};
\node[rectangle,draw,align=center] (permposs)  [left=of existence]{\textsc{permanent possession}\\\textit{gɛ́t, na fɔ/dé fɔ}};
\node[rectangle,draw,align=center] (pac)  [right=of location]{\textsc{predicate}\\\textsc{adjective clause}\\\textit{dé}};
\node[rectangle,draw,align=center] (xpt)  [below=of location]{\textsc{existence in place \& time}\\ \textit{lɛ́f, sté, pás, rích, dú}};

\draw (focus) -- (identity);
\draw (identity) -- (cos);
\draw (identity) -- (permposs);
\draw[dashed] (permposs) -- (existence);
\draw[dashed] (existence) -- (location);
\draw[dashed] (existence) -- (tempposs);
\draw[dashed] (location) -- (tempposs);
\draw[dashed] (location) -- (xpt);
\draw[dashed] (location) -- (pac);
\end{tikzpicture}


%%[Warning: Draw object ignored]
%%[Warning: Draw object ignored]
\caption{Expression of \textsc{being} and \textsc{having}}
\label{fig:key:7.2}
\end{figure}

\figref{fig:key:7.2} maps the linkages between the different elements that participate in the expression of \textsc{being} and \textsc{having}. Time-stable situations are connected with an unbroken, non-time-stable states with a broken line. Glosses for the elements contained in the figure can be culled from the following sections and \tabref{tab:key:7.8}.

\subsection{Core copulas}\label{sec:7.6.1}

The expression of identity-equation is provided by the elements \textit{na} ‘\textsc{foc}’, \textit{nóto} ‘\textsc{neg}.\textsc{foc}’, and \textit{bí} ‘\textsc{be’}. The element \textit{dé} \textsc{‘be.loc}\textsc{’} serves as the locative-existential copula. Pichi employs overt copulas in all relevant contexts. The expression of \textsc{being} is characterised by several asymmetries. Firstly, there is a functional and formal differentiation between the expression of identity (via \textit{na}\textit{\textup{/}}\textit{nóto}) and location-existence (via \textit{dé}). Secondly the expression of identity is taken care of by the three suppletive forms \textit{na} ‘\textsc{foc}’, \textit{nóto} ‘\textsc{neg}.\textsc{foc}’, and \textit{bí} \textsc{‘be’} which are in complementary distribution with each other. Some relevant characteristics of the distribution of the Pichi core copulas are summarised in \tabref{tab:key:7.7}.

%%please move \begin{table} just above \begin{tabular
\begin{table}
\caption{Core copulas}
\label{tab:key:7.7}

\begin{tabularx}{\textwidth}{Qlllp{1.75cm}}
\lsptoprule
 & \multicolumn{3}{c}{ Identity} & Location \& existence\\ \cmidrule(lr){2-4}
 & \textstyleTablePichiZchn{na} ‘\textsc{foc}’ & \textstyleTablePichiZchn{nóto} ‘\textsc{neg}.\textsc{foc}’ & \textstyleTablePichiZchn{bí} \textsc{‘be’} & \textstyleTablePichiZchn{dé} \textsc{‘be.loc’}\\
\midrule
Can co-occur with \textsc{TMA} markers? & No & No & Yes & Yes\\
Can occur in clauses with factative TMA? & Yes & Yes & No & Yes\\
Has a suppletive counterpart? & Yes & Yes & Yes & No\\
Can occur in non-finite contexts? & No & No & Yes & Yes\\
\lspbottomrule
\end{tabularx}
\end{table}
Clauses involving the three core copulas \textit{na} ‘\textsc{foc}’, \textit{nóto} ‘\textsc{neg}.\textsc{foc}’ and \textit{bí} \textit{\textsc{‘be’}} feature a subject, the copula and a nominal complement\is{complements}. The functions of the copula include expression of the identity of two participants \REF{ex:key:761}, and classification as member of a group \REF{ex:key:762}:


\ea%761
    \label{ex:key:761}
    \gll Dán    tín    \textbf{na}  di pasta.\\
that    thing  \textsc{foc}  \textsc{def}  paste\\

\glt ‘That thing is the paste.’ [fr03do 036]
\z


\ea%762
    \label{ex:key:762}
    \gll \'{I}n    \textbf{na} kres-húman.\\
\textsc{3sg.indp}  \textsc{foc}  be.crazy.\textsc{cpd}{}-woman\\

\glt ‘She’s a crazy woman.’ [ro05ee 037]
\z

Further functions are the attribution of a role \REF{ex:key:763}, a name \REF{ex:key:764}, and the expression of a family relationship \REF{ex:key:765}. Note the presence of the verb \textit{tɔ́n} ‘turn’ which denotes a change of state when used as a copula verb \REF{ex:key:763}: 


\ea%763
    \label{ex:key:763}
    \gll Mi    papá  \textbf{na} dɔ́kta  bɔt  mí    nó  go  \textbf{tɔ́n}  dɔ́kta.\\
\textsc{1sg.poss}  father  \textsc{foc}  doctor  but  \textsc{1sg.indp}  \textsc{neg}  \textsc{pot}  turn  doctor\\

\glt ‘My father is a doctor but I won’t become a doctor.’ [ro05ee 024]
\z


\ea%764
    \label{ex:key:764}
    \gll Yɛ́s,  mi    ném    \textbf{na}  Djunais.\\
yes  \textsc{1sg.poss}  name  \textsc{foc}  \textsc{name}\\

\glt ‘Yes, my name is Djunais.’ [dj05ce 188]
\z


\ea%765
    \label{ex:key:765}
    \gll Na  dán  tɛ́n  a    kán  sabí    sé    mi    mamá 
\textbf{na} mi    mamá.\\
\textsc{foc}  that  time  \textsc{1sg.sbj}  \textsc{pfv}  know  \textsc{quot}    \textsc{1sg.poss}  mother 
\textsc{foc}  \textsc{1sg.poss}  mother\\

\glt ‘It’s then that I came to know that my mother was my mother.’ [fr03ft 019]
\z

Equative clauses\is{equative clauses} are characterised by asymmetries and suppletion in the use of personal pronouns, polarity,\is{negation} and \textsc{TMA} marking. These asymmetries derive from the core function of \textit{na}\textit{\textup{/}}\textit{nóto} to express identification in presentational sentences like \REF{ex:key:766} and \REF{ex:key:767}. In these clauses, the identified elements (i.e. \textit{kasára} ‘cassava’ and \textit{wi Píchi} ‘our (kind of) Pichi’) are in focus by default. Therefore, I consistently gloss \textit{na}\textit{\textup{/}}\textit{nóto} as \textsc{foc} and \textsc{neg.foc,} respectively, in order to render the chiefly pragmatic function of these elements: 


\ea%766
    \label{ex:key:766}
    \gll \textbf{Na} kasára.\\
\textsc{foc}  cassava\\

\glt ‘That’s (a) cassava.’ [li07pe 028]
\z


\ea%767
    \label{ex:key:767}
    \gll \textbf{N}\textbf{óto}  wi  Píchi.\\
\textsc{neg.foc}  \textsc{1pl}  Pichi\\

\glt ‘That’s not our (kind of) Pichi.’ [ra07ve 009]
\z

In sentences like the two above, \textit{na} has expletive reference and is therefore non-referential. The core pragmatic function of identification of \textit{na}\textit{\textup{/}}\textit{nóto} can be extended to express identity between two full NPs (hence with default \textsc{3sg} reference) in equative clauses: 


\ea%768
    \label{ex:key:768}
    \gll In    papá  \textbf{na} chino.\\
\textsc{3sg.poss}  father  \textsc{foc}  Chinese\\

\glt ‘Her father is Chinese.’ [ed03sp 028]
\z

However, when identity between a personal pronoun with reference other than \textsc{3sg} and another NP is expressed, the deeply pragmatic nature of the copula-like element in sentences like \REF{ex:key:768} above is revealed. Since \textit{na}\textit{\textup{/}}\textit{nóto} is not a copula “verb”, the subject pronoun cannot come from the dependent series of the pronominal paradigm. Instead, an independent emphatic pronoun must be used: 


\ea
	\label{ex:key:769}
	\gll
\textbf{Mí}    \textbf{na}  di wan-grén  pikín.\\
\textsc{1sg.indp}  \textsc{foc}  \textsc{def}  one.\textsc{cpd}{}-grain  child.\\

\glt ‘I am the only child.’ [lo07he 060]
\z

Therefore even equative clauses are best analysed as identificational. These clauses are grammaticalised topic-comment structures, in which the topical subject is followed by an entity identified by \textit{na}\textit{\textup{/}}\textit{nóto}. The copula-like element \textit{na}\textit{\textup{/}}\textit{nóto} therefore retains its pragmatic, identificational, and focus-marking function even in such “copula clauses”. 


The two asymmetries in the formation of copula clauses next to negative suppletion (i.e. \ref{ex:key:767} and \textsc{3sg} default reference, i.e. \ref{ex:key:766} and \ref{ex:key:768}) are complemented by a third asymmetry: Whenever overt \textsc{TMA} marking is required or the copula is employed in a context suggesting reduced finiteness\is{finiteness}, the copula verb \textit{bí} \textit{\textsc{‘be’}} is made use of. This complementary distribution is strict. Therefore, a clause like the following one is ungrammatical, since \textit{bí} may not appear in basic identity clauses without overt TMA marking. Compare \REF{ex:key:768} above and \REF{ex:key:770} below: 



\ea[*]{%770
    \label{ex:key:770}
    \gll In    mamá  \textbf{bí}  rusa.\\
 \textsc{3sg.poss}  mother  \textsc{foc}  Russian\\
\glt Intended: ‘Her mother is Russian.’ [dj07ae 532]
}\z

In the following two equative clauses, the presence of the \textsc{TMA} markers \textit{dɔ́n} ‘\textsc{prf}’ \REF{ex:key:771} and \textit{go} ‘\textsc{pot}’ \REF{ex:key:772} motivates the appearance of the suppletive identity copula \textit{bí} ‘\textsc{be’.} In spite of its defective distribution (cf. \ref{ex:key:770}), the copula \textit{bí} behaves much more like a copula verb than \textit{na}\textit{\textup{/}}\textit{nóto:} It may take dependent personal pronouns (e.g. in \ref{ex:key:771}) and appear with TMA marking (e.g. \ref{ex:key:771} and \ref{ex:key:772}). 


\ea%771
    \label{ex:key:771}
    \gll E   \textbf{dɔ́n}  \textbf{bí}  wán    señorita.\\
\textsc{3sg.sbj}  \textsc{prf}  \textsc{be}  one    little.lady\\

\glt ‘She has already become a real young lady.’ [fr03ft 117]
\z


\ea%772
    \label{ex:key:772}
    \gll Mí    \textbf{go}  \textbf{bí}  dɔ́kta.\\
\textsc{1sg.indp}  \textsc{pot}  \textsc{be}  doctor\\

\glt ‘I’ll be doctor.’ [ro05ee 025]
\z

Sentence \REF{ex:key:773} below contains two copula clauses. The first one features the copula \textit{bí} marked for past tense\is{past tense} by \textit{bin} ‘\textsc{pst}’. In contrast, the second clause is not overtly marked for tense, hence the copula cum focus marker \textit{na} is employed. Recall that Pichi employs relational tense. Hence the identity copula \textit{na} may have past tense reference because tense reference has been anchored in the past by the use of \textit{bin} in the preceding clause. In fact, in this example, a past tense reference of \textit{na} is a plausible option because the speaker’s mother is deceased (unless the speaker considers reference to her mother to be generic in nature): 


\ea%773
    \label{ex:key:773}
    \gll Mi    ném    \textbf{bin}  \textbf{bí} Francisca  Belobe  Toichoa,    porque
mi    mamá  in    ném    \textbf{na} Belobe  Toichoa.\\
\textsc{1sg.poss}  name  \textsc{pst}  \textsc{be}  \textsc{name}    \textsc{name}  \textsc{name}    because
\textsc{1sg.poss}  mother  \textsc{3sg.poss}  name  \textsc{foc}  \textsc{name}  \textsc{name}\\

\glt ‘My name was Francicsa Belobe Toichoa because my mother’s name 
is/was Belobe Toichoa.’ [fr03ft 090]
\z

A further example involving overt TMA marking in an equative clause follows. Sentence \REF{ex:key:774} features the narrative perfective marker \textit{kán} ‘\textsc{pfv}’ followed by \textit{bí} ‘\textsc{be’}. Note that the combination of\textit{ kán} ‘\textsc{pfv}’ with the copula \textit{bí} renders a change of state reading of \textit{bí} just like with any other (inchoative-)stative verb (cf. e.g. \ref{ex:key:326}–\ref{ex:key:327}): 


\ea%774
    \label{ex:key:774}
    \gll So  mí,    mi    yón     e    \textbf{kán}  \textbf{bí}  una  desgracia.\\
so  \textsc{1sg.indp}  \textsc{1sg.poss}  own    \textsc{3sg.sbj}  \textsc{pfv}  \textsc{be}  \textsc{def}  disgrace\\

\glt ‘So as for me, mine [my matter] came to be a disgrace.’ [ab03ay 034]
\z

\textit{Bí} \textsc{‘be’} is also employed instead of \textit{na}\textit{\textup{/}}\textit{nóto} in contexts of reduced finiteness\is{finiteness}. In \REF{ex:key:775}, \textit{bí} occurs as the complement of the modal verb \textit{fít} ‘can’. The form \textit{bí} also appears in subjunctive\is{subjunctive mood} clauses \REF{ex:key:776}. Such clauses are not only inherently future-referring and non-assertive. They also feature reduced tense-aspect marking and are less finite:


\ea%775
    \label{ex:key:775}
    \gll “Kɔ́t”    \textbf{fít}  bí  lɛkɛ    herida.\\
\phantom{“}cut    can  \textsc{be}  like    wound\\

\glt ` “Kɔ́t” can be [mean] like a wound.’ [ye05ce 227]
\z


\ea%776
    \label{ex:key:776}
    \gll \textbf{Mék}    e    \textbf{bí}    sé    Kofí    dɔ́n    kán.\\
\textsc{sbjv}    \textsc{3sg.sbj}  \textsc{be.loc}  \textsc{quot}    \textsc{name}  \textsc{prf}    come\\

\glt ‘(Please) let it be that Kofí has come.’ [dj05ae 032]
\z

Furthermore, \textit{bí} is the only identity copula attested in a context like \REF{ex:key:777} below. In the example, the copula occurs in a subordinate clause featuring the clause linker \textit{wé} ‘\textsc{sub}’. The non-assertive environment of the subordinate clause precludes use of \textit{na}\textit{\textup{/}}\textit{nóto} as copulas. This is presumably due to the fact that these particles realise their core function in identificational and presentational sentences, which are assertive structures \textit{par} \textit{excellence}. 


Additionally, tense\is{tense} reference of the subordinate clause is dependent on the main clause, which is set in the past. These factors contribute to the use of \textit{bí} although the context is finite and there is no overt TMA marking in the subordinate clause in \REF{ex:key:777}:



\ea%777
    \label{ex:key:777}
    \gll \textbf{Frɔn}  wé  a    \textbf{bí}  pikín  a    \textbf{bin}  wánt
kɔmɔ́t  na  dís  kɔ́ntri.\\
from  \textsc{sub}  \textsc{1sg.sbj}  \textsc{be}  child  \textsc{1sg.sbj}  \textsc{pst}  want
go.away  \textsc{loc}  this  country\\

\glt ‘From when I was a child, I wanted to leave this country.’ [ro05ee 027]
\z

A copula clause featuring \textit{bí} ‘\textsc{be’} is negated\is{negation} like any other verbal clause. The negator \textit{nó} ‘\textsc{neg}’ appears in its usual position in the predicate. Compare the following sentence, in which the copula clause is in the potential mood:


\ea%778
    \label{ex:key:778}
    \gll E    \textbf{nó}  \textbf{go}  \textbf{bí}  mecánico.\\
\textsc{3sg.sbj}  \textsc{neg}  \textsc{pot}  \textsc{be}  mechanic\\

\glt ‘He won’t be a mechanic.’ [dj05ae 215]\is{copula!identity}
\z

The element \textit{dé} \textsc{‘be.loc’} functions as a locative-existential copula. Accordingly, this form is used to express relatively transient, less permanent existence in space and time, either on its own or when followed by an adverbial complement\is{complements}. The element \textit{dé} also occurs as a copula in predicate adjective constructions (cf. \sectref{sec:7.6.5}). Hence \textit{dé} may also take adjectives as complements. 


The copula \textit{dé} may occur in intransitive clauses without any complement. Such clauses show that \textit{dé} is semantically relatively rich and has a meaning of its own, namely ‘exist in a place’ or ‘exist in a certain manner’. Compare the question in \REF{ex:key:779a} and the corresponding answer in \REF{ex:key:779b}:



\ea%779
    \label{ex:key:779}
\ea{\label{ex:key:779a}
\gll
Ebongolo  \textbf{dé}?\\
  \textsc{name}    \textsc{be.loc}\\
\glt   ‘Is Ebongolo around/his usual self/fine/alright?’ [ge07fn 180]
}\ex{\label{ex:key:779b}
\gll
Yɛ́s,    e    \textbf{dé}.\\
  yes    \textsc{3sg.sbj}  \textsc{be.loc}\\

\glt   ‘Yes, he’s here/around/his usual self/fine/alright.’ [he07fn 181]
}\z\z

In \REF{ex:key:780}, \textit{dé} takes a locative adverbial phrase introduced by the general locative preposition \textit{na} ‘\textsc{loc}’ as a complement. The adverbial phrase in \REF{ex:key:781} involves the locative noun \textit{nía} ‘be near’:


\ea%780
    \label{ex:key:780}
    \gll E    \textbf{dé}    \textbf{na}  \textbf{grɔ́n}.\\
\textsc{3sg.sbj}  \textsc{be.loc}  \textsc{loc}  ground\\

\glt ‘He is [lying] on the ground.’ [ab03ab 063]
\z


\ea%781
    \label{ex:key:781}
    \gll Yu  fon    \textbf{dé}    \textbf{nía}    \textbf{tébul}.\\
\textsc{2sg}  phone  \textsc{be.loc}  near    table\\

\glt ‘Your phone is near the table.’ [ro05ee 109]
\z

Locative complements of \textit{dé} ‘\textsc{be.loc}’ other than locative adverbs like \textit{yandá} ‘yonder’ in \REF{ex:key:782} rarely appear without a preposition or a locative noun. Where they do, the absence of the locative noun is usually lexically determined. Compare \textit{dé láyf} ‘\textsc{be.loc} life’ = ‘be alive’ in \REF{ex:key:783}. Also note that the copula \textit{dé} receives an imperfective, present tense\is{present tense} interpretation like any other unmarked stative verb in Pichi:


\ea%782
    \label{ex:key:782}
    \gll \'{A}fta    dí  wán  wé  e    \textbf{dé}    \textbf{yandá},  e    bíg.\\
then  this  one  \textsc{sub}  \textsc{3sg.sbj}  \textsc{be.loc}  yonder  \textsc{3sg.sbj}  be.big\\

\glt ‘Then, that one that’s over there, it’s big.’ [li07pe.091]
\z


\ea%783
    \label{ex:key:783}
    \gll Somos  tú  dásɔl  wé  wi  \textbf{dé}    \textbf{láyf}.\\
we.are  two  only    \textsc{sub}  \textsc{1pl}  \textsc{be.loc}  life\\

\glt ‘We are only two who are alive.’ [ab03ay 133]
\z

Sentence \REF{ex:key:784} exemplifies how \textit{dé} is used to express existence in time. In contrast to locative complements, time adverbials like \textit{ívin tɛ́n} ‘evening’ appear as direct complements of the copula \textit{dé} when the intended meaning is ‘location{\fff} in time’ (cf. \sectref{sec:8.2.2} for other temporal relations): 


\ea%784
    \label{ex:key:784}
    \gll Wi  \textbf{dé}    \textbf{íbin}    \textbf{tɛ́n}.\\
\textsc{1pl}  \textsc{be.loc}  evening  time\\

\glt ‘It’s evening.’ [dj05ce 249]
\z

Further, the time is always told in a codemixed Pichi-Spanish construction. The noun phrase employed in telling the time in \ili{Spanish} appears as a complement of the copula \textit{dé} ‘\textsc{be.loc}’ which in turn takes a \textsc{1pl} subject \REF{ex:key:785}. No prepositions are employed in this construction either. Hence here too, there is no formal indication of the adverbial status of the time expression:


\ea%785
    \label{ex:key:785}
    \gll Wi  \textbf{dé}    \textbf{las}    \textbf{dos}  \textbf{y}  \textbf{media}.\\
\textsc{1pl}  \textsc{be.loc}  the\textsc{.pl}  two  and  half\\

\glt ‘It’s two thirty.’ [dj05ce 056]
\z

The form \textit{dé} ‘\textsc{be.loc}’ may also be employed to attribute a relatively transient, non-time-stable property to a subject. Hence, \textit{dé} is encountered as a predicator in predicate adjective constructions involving the few adjectives that Pichi has. One of these is \textit{fáyn} ‘(be) fine’ in \REF{ex:key:786}. As explained in detail in \sectref{sec:7.6.5}, predicate adjective constructions, rather than verbal clauses, are only chosen when the situation is perceived as non-time-stable:


\ea%786
    \label{ex:key:786}
    \gll Dán  tɛ́n  a    \textbf{dé}    \textbf{fáyn}.\\
that  time  \textsc{1sg.sbj}  \textsc{be.loc}  fine\\

\glt ‘That time I was [feeling] fine.’ [ru03wt 024]\is{complements!copula complements}
\z

Another manifestation of the non-time stable character of the situation predicated by \textit{dé} ‘\textsc{be.loc}’ is given in the following three sentences. The copula dé is used when an adverbial complement designates a way of being rather than intrinsic being. Adverbial complements can be a simple manner adverb like \textit{só} ‘so’ \REF{ex:key:787}, a bare noun phrase featuring the generic noun \textit{stáyl} ‘manner, style’ \REF{ex:key:788}, or a prepositional phrase with the similative{\fff} and equative preposition \textit{lɛk} ‘like’ \REF{ex:key:789}:


\ea%787
    \label{ex:key:787}
    \gll Na  \textbf{só}    e    \textbf{dé}.\\
\textsc{foc}  like.that  \textsc{3sg.sbj}  \textsc{be.loc}\\

\glt ‘That’s the way it is.’ [au07se 159]
\z


\ea%788
    \label{ex:key:788}
    \gll E    \textbf{dé}    \textbf{ɔ́da}    \textbf{stáyl}.\\
\textsc{3sg.sbj}  \textsc{be.loc}  other  style\\

\glt ‘It’s different.’ [dj05ae 081]
\z


\ea%789
    \label{ex:key:789}
    \gll \MakeUppercase{A}   wánt  \textbf{dé}    \textbf{lɛk}  Miguel  Ángel.\\
\textsc{1sg.sbj}  want  \textsc{be.loc}  like  \textsc{name}  \textsc{name}\\

\glt ‘I want to be like Miguel Ángel [the way he dresses/acts/looks].’ [ye07ga 007]
\z

By extension, \textit{dé} ‘\textsc{be.loc}’ is also employed whenever an attributed property is questioned directly \REF{ex:key:790} and indirectly \REF{ex:key:791}, or when a property is attributed to a main clause verb in a free adverbial manner clause\is{manner clauses} \REF{ex:key:792}:


\ea%790
    \label{ex:key:790}
    \gll \textbf{Háw}    yu  go  \textbf{dé},    yu  nó  gɛ́t  pikín?\\
how    \textsc{2sg}  \textsc{pot}  there  \textsc{2sg}  \textsc{neg}  get  child\\

\glt ‘How would you be [feel] (if) you had no child?’ [kw03sb 203]
\z


\ea%791
    \label{ex:key:791}
    \gll Bɔt  mí    wánt  sabí    \textbf{háw}    dán    tín    \textbf{dé}.\\
but  \textsc{1sg.indp}  want  know  how    that    thing  \textsc{be.loc}\\

\glt ‘But I \textsc{[emp]} wanted to know how that thing was.’ [ed03sb 147]
\z


\ea%792
    \label{ex:key:792}
    \gll Yu  gɛ́fɔ    lɛ́f=an    \textbf{lɛk}  \textbf{háw}    e    \textbf{dé}.\\
\textsc{2sg}  have.to  leave=\textsc{3sg.obj}  like  how    \textsc{3sg.sbj}  \textsc{be.loc}\\

\glt ‘You have to leave it how it is.’ [hi03cb 065]
\z

Contrary to the time stable copulas \textit{na}\textit{\textup{/}}\textit{nóto} and \textit{bí} described above, \textit{dé} exhibits no irregularities with respect to \textsc{TMA} marking and negation\is{negation} \REF{ex:key:793}. It occurs with the standard negator and any \textsc{TMA} marker compatible with its distribution as a stative verb \REF{ex:key:794}:


\ea%793
    \label{ex:key:793}
    \gll Nó  bɔ́dí    \textbf{nó}  \textbf{dé}    na  pueblo.\\
\textsc{neg}  body  \textsc{neg}  \textsc{be.loc}  \textsc{loc}  village\\

\glt ‘Nobody is in the village.’ [fr03ft 156]
\z


\ea%794
    \label{ex:key:794}
    \gll Sé  “ús=tín  \textbf{kin}  \textbf{dé}   ínsay  dé?”\\
\textsc{quot}  \textsc{q}=thing  \textsc{hab}  \textsc{be.loc}  inside  there\\

\glt ‘(They) said “what is usually in there?”‘ [ed03sb 052]
\z

\subsection{Copula verbs}\label{sec:7.6.2}

Besides the core system of copula expression covered in the previous section, Pichi recruits a number of stative and dynamic verbs in order to express more specific copula meanings linked to the notions of change-of-state and existence in place and time. Copula verbs and their meanings are provided in \tabref{tab:key:7.8}.

\begin{table}
\caption{Copula verbs}
\label{tab:key:7.8}

\begin{tabularx}{\textwidth}{l@{~~}lQll}
\lsptoprule

Type & Verb & Copula meaning & Other meanings & Other functions\\
\midrule
Change of state & \itshape tɔ́n & ‘turn into’ & ‘turn’ & —\\
& \itshape lɛ́f & ‘turn into, become’ & ‘leave, remain’ & Causative verb\\
& \itshape kɔmɔ́t & ‘turn out as’ & ‘go/come out’ & Egressive aspect\\

\tablevspace
Existence in & \itshape gɛ́t & ‘exist’ & ‘get, have’ & —\\
place \& time & \itshape lɛ́f & ‘remain’ & ‘leave’ & Causative verb\\
& \itshape sté & ‘last (long)’ & — & Duration SVC\\
& \itshape pás & ‘exceed in degree’ & ‘pass’ & Comparative SVC\\
& \itshape rích & ‘equal in degree, 

be enough’ & ‘arrive’ & SVC\\
& \itshape dú & ‘be enough’ & ‘do’ & —\\
\lspbottomrule
\end{tabularx}
\end{table}
When employed as a lexical verb, \textit{tɔ́n} means ‘turn, stir’ \REF{ex:key:796}. In its literal sense, \textit{tɔ́n} is employed as a dynamic verb with an agent subject and a patient\is{patient} object \REF{ex:key:795}, or a locative adverbial \REF{ex:key:796}. The collocation \textit{tɔ́n bák} means ‘return’ \REF{ex:key:797}: 


\ea%795
    \label{ex:key:795}
    \gll Yu  gɛ́fɔ    de  \textbf{tɔ́n}=an.\\
\textsc{2sg}  have.to  \textsc{ipfv}  turn=\textsc{3sg.obj}\\

\glt ‘You have to be stirring it.’ [dj03do 057]
\z


\ea%796
    \label{ex:key:796}
    \gll \textbf{Tɔ́n}  na  yu  lɛ́f-hán!\\
turn  \textsc{loc}  \textsc{def}  left.\textsc{cpd}{}-hand\\

\glt ‘Turn left!’ [ye05ce 278]
\z


\ea%797
    \label{ex:key:797}
    \gll Mék    a    gó  dú  smɔ́l  tín    a    \textbf{tɔ́n}    \textbf{bák}.\\
\textsc{sbjv}    \textsc{1sg.sbj}  go  do  be.small  thing  \textsc{1sg.sbj}  turn    back\\

\glt ‘Let me go do something quickly (and) come back.’ [ge07fn 016]
\z

As a copula verb, \textit{tɔ́n} ‘turn’ designates a change of state from one identity to another \REF{ex:key:798}:


\ea%798
    \label{ex:key:798}
    \gll Dɛn-ɔ́l    dɛn  dɔ́n  \textbf{tɔ́n}    \textbf{europeos}  \textbf{dɛn}.\\
\textsc{3pl.cpd-}all  \textsc{3pl}  \textsc{prf}  turn    European.\textsc{pl}  \textsc{pl}\\

\glt ‘They have all turned into Europeans.’ [fr03ft 149]
\z

In contexts other than copula expression and causative formation, \textit{lɛ́f} may be employed as a dynamic verb in transitive clauses with the meaning ‘leave (behind)’ \REF{ex:key:799}: 


\ea%799
    \label{ex:key:799}
    \gll A    \textbf{lɛ́f}    di  tín    dɛn  di  sáy  wé  yu  bin  tɛ́l  mí.\\
\textsc{1sg.sbj}  leave  \textsc{def}  thing  \textsc{pl}  \textsc{def}  side  \textsc{sub}  \textsc{2sg}  \textsc{pst}  tell  \textsc{1sg.indp}\\

\glt ‘I left the things where you told me to.’ [ro05de 025]
\z

The verb \textit{lɛ́f} ‘leave, remain’ also functions as a resultative copula{\fff} in resultative causative constructions {\fff}like the following one (cf. \sectref{sec:9.4.4} for a thorough treatment){\fff}: 


\ea%800
    \label{ex:key:800}
    \gll Yu  go  mék    mék  di  gál  \textbf{lɛ́f}    wet    brok-hát.\\
\textsc{2sg}  \textsc{pot}  make  \textsc{sbjv}  \textsc{def}  girl  remain  with    break.\textsc{cpd}{}-heart\\

\glt ‘You’re going to make that girl become broken-hearted.’ [ge07fn 103]
\z

Besides the verb \textit{gɛ́t} ‘get, have’ (cf. \sectref{sec:7.6.3}), a few other verbs express existence in space and time. When the inchoative-stative verb \textit{lɛ́f} ‘leave, remain’ occurs in an intransitive clause featuring a comitative or locative adverbial, this verb assumes a copula function with the meaning ‘remain (behind), stay temporarily with’ \REF{ex:key:801}. 


\ea%801
    \label{ex:key:801}
    \gll Machyta    \textbf{lɛ́f}    wet    in    fámbul.\\
\textsc{name}    remain  with    \textsc{3sg.poss}  family\\

\glt ‘Machyta has remained (temporarily) with his family’ [ge07ae 213] 
\z

The verb \textit{kɔmɔ́t} ‘come out’ is employed to indicate a change of state in lexicalised collocations involving associative objects\is{associative objects} (cf. also \sectref{sec:9.3.2}). Compare \REF{ex:key:802}:


\ea%802
    \label{ex:key:802}
    \gll \MakeUppercase{A}   de  trén=an    porque  e    go  \textbf{kɔmɔ́t}    pɔ́sin.\\
\textsc{1sg.sbj}  \textsc{ipfv}  train=\textsc{3sg.obj}  because  \textsc{3sg.sbj}  \textsc{pot}  come.out  person\\

\glt ‘I’m bringing him up because [so that] he will turn out to be a (responsible) 
person.’ [au07se 145]
\z

The dynamic verb \textit{sté} means ‘last (a long time)’ as in \REF{ex:key:803}. This verb also expresses excessive duration in an adverbial SVC (cf. \sectref{sec:11.2.5}):


\ea%803
    \label{ex:key:803}
    \gll Bɛ́ta      tín    nó  de  \textbf{sté}.\\
very.good  thing  \textsc{neg}  \textsc{ipfv}  last\\

\glt ‘Good things don’t last.’ [ra07fn 076]
\z

Finally, the occurrence of \textit{pás} ‘pass’ and \textit{rích} ‘arrive’ as inchoative-stative verbs in comparatives like \REF{ex:key:483} and equatives like \REF{ex:key:510} may also be seen as manifestations of a copula-like use of these otherwise dynamic verbs.\is{copula verb} 

\subsection{Existentials}\label{sec:7.6.3}

The locative-existential copula{\fff} \textit{dé} ‘\textsc{be.loc}’, as well as the verb \textit{gɛ́t} ‘have, get, acquire’ both participate in existentials, i.e. constructions which predicate the general existence of an entity. Pichi existentials appear in two types of clauses with respect to number and type of participants: Transitive clauses featuring \textit{gɛ́t} ‘have’ and intransitive clauses featuring \textit{dé} ‘\textsc{be.loc}’. Some of the characteristics of these two types of existentials are given in \tabref{tab:key:7.9}.

%%please move \begin{table} just above \begin{tabular
\begin{table}
\caption{Existential clauses}
\label{tab:key:7.9}

\begin{tabularx}{\textwidth}{p{18mm}lQp{18mm}QQ}
\lsptoprule

Existential verb & Frequency & Syntactic \mbox{relation of} existing entity? & Attested in negative \mbox{existentials?} & Attested with non-finite use? & Attested \mbox{with overt} TMA marking?\\
\midrule
gɛ́t ‘have’ & About half & Object & Marginal & No & Marginal\\
\textit{dé} {\textsc{‘be.loc’}} & About half & Subject & Yes & Yes & Frequent\\
\lspbottomrule
\end{tabularx}
\end{table}

The \textit{gɛ́t}-existential construction occurs in a transitive clause. The subject position is filled by an expletive \textsc{3sg} pronoun, the object position by the existing entity \REF{ex:key:804}. This construction exclusively serves the expression of existential meaning and has no locative connotation. None of the other constructions that follow are uniquely employed to express existential meaning in this way:


\ea%804
    \label{ex:key:804}
    \gll Dís  smɔ́l  bɔ́tul  dɛn  Fanta,  wé  e    \textbf{gɛ́t}  Coca-Cola, e
\textbf{gɛ́t}  Fanta,  e    \textbf{gɛ́t}  limón,  e    báy=an    wán.\\
this  small  bottle  \textsc{pl}  \textsc{name}  \textsc{sub}  \textsc{3sg.sbj}  get  \textsc{name}    \textsc{3sg.sbj}
get  \textsc{name}  \textsc{3sg.sbj}  get  lemon  \textsc{3sg.sbj}  buy=\textsc{3sg.obj}  one\\

\glt ‘These small bottles of Fanta, where there is Coca-Cola, there is Fanta, there is 
Lemon, she bought him one (of them).’ [ab03ab 130]
\z

Pichi has other ways of establishing the type of impersonal reference characteristic for \textit{gɛ́t}-existentials besides a \textsc{3sg} expletive pronoun{\fff}. The verb \textit{gɛ́t} may also occur with an impersonal \textsc{3pl} \REF{ex:key:805} or \textsc{2sg} \REF{ex:key:806} pronoun in clauses that are functionally similar to existentials \REF{ex:key:804}:{\fff}


\ea%805
    \label{ex:key:805}
    \gll Ɔ  \textbf{dɛn}  \textbf{gɛ́t}  problema  fɔ  di  sistema  ɔ  e    sɛ́n=an
na  ɔ́da    empresa,    wé  nóto  Western  Union.\\
or  \textsc{3pl}  get  problem    \textsc{prep}  \textsc{def}  system  or  \textsc{3sg.sbj}  send=\textsc{3sg.obj}
\textsc{loc}  other  company    \textsc{sub}  \textsc{neg}.\textsc{foc}  \textsc{name}\\

\glt ‘Either they have a problem in the system, or she sent it to another company 
which is not Western Union.’ [ge07ac 217]
\z


\ea%806
    \label{ex:key:806}
    \gll Bɔt  yu  dɔ́n  sabí    na  Afrika  \textbf{yu}  \textbf{nó}  \textbf{gɛ́t}  nó
relación  fɔ  mán    lɛk  na  Europa.\\
but  \textsc{2sg}  \textsc{prf}  know  \textsc{loc}  \textsc{place}  \textsc{2sg}  \textsc{neg}  get \textsc{neg}
relation  \textsc{prep}  man    like  \textsc{loc}  Europe\\

\glt ‘But you already know that in Africa you don’t have a 
relationship with a man like in Europe.’ [fr03ft 167]
\z

There are no restrictions on the use of \textit{gɛ́t}-existentials in subordinate clauses. In sentence \REF{ex:key:807}, the existential clause appears in a relative clause introduced by \textit{wé} ‘\textsc{sub}’: 


\ea%807
    \label{ex:key:807}
    \gll Bɔt  na  dán  fɔ́s  tɛ́n  hós    dɛn  \textbf{wé}  e    \textbf{gɛ́t}
dá  piso    dɛn  fɔ  dán  altura  dɛn.\\
but  \textsc{foc}  that  first  time  house  \textsc{pl}  \textsc{sub}  \textsc{3sg.sbj}  get
that  storey  \textsc{pl}  \textsc{prep}  that  height  \textsc{pl}\\

\glt ‘But its those houses of the past where there are those high storeys’ [hi03cb 043]
\z

Copula clauses\is{copula!locative-existential} featuring \textit{dé} \textsc{‘be.loc’} typically acquire an existential reading when they lack a copula complement\is{complements!copula complements}. In these clauses, we find the predicated entity, which may be of varying complexity, in the subject position \REF{ex:key:808}. Since there is no complement to provide further specification, the clause acquires the default locative and manner reading that typifies such \textit{dé}{}-clauses \REF{ex:key:809}:


\ea%808
    \label{ex:key:808}
    \gll Bueno  aunque  dé,    \textbf{bɔkú}  \textbf{interés}  \textbf{económico}  \textbf{dé}.\\
good  although  there  much  interest  economic  \textsc{be.loc}\\

\glt ‘Alright although there, there’s a lot of economic interest.’ [fr03ft 110]
\z


\ea%809
    \label{ex:key:809}
    \gll Bueno,  \textbf{mi}    \textbf{gran-pá}    bin  \textbf{dé}.\\
good  \textsc{1sg.poss}  grand-pa    \textsc{pst}  \textsc{be.loc}\\

\glt ‘Alright, my grandfather was around/fine.’ [fr03ft 166]
\z

Hence, constructions featuring \textit{dé} acquire a locative reading when a locative expression is present. In \REF{ex:key:810}, we find the locative adverbial \textit{na sala} ‘in the hall’:


\ea%810
    \label{ex:key:810}
    \gll Paciente  dɛn  \textbf{dé}    \textbf{na}  \textbf{sala},  yú    dɔ́kta  la  una
yu  de  kán?\\
patient  \textsc{pl}  \textsc{be.loc}  \textsc{loc}  hall    \textsc{2sg.indp}  doctor  the  one
\textsc{2sg}  \textsc{ipfv}  come\\

\glt ‘Patients are in the hall, (and) you doctor, it’s (only)
at one o’clock that you come?’ [ab03ab 118]
\z

Existential clauses featuring \textit{gɛ́t} are not often negated. The data contains only a single negative \textit{gɛ́t}-existential clause, presented in \REF{ex:key:811}. This is probably so because the “true” existential construction featuring \textit{gɛ́t} is subject to an affirmative presupposition:


\ea%811
    \label{ex:key:811}
    \gll Dɛn  de  kɔ́l  dɛn  sé,    \textbf{e}    \textbf{nó}  \textbf{gɛ́t} tɔ́k  na  Píchi.\\
\textsc{3pl}  \textsc{ipfv}  call  \textsc{3pl}  \textsc{quot}    \textsc{3sg.sbj}  \textsc{neg}  get talk  \textsc{loc}  Pichi\\

\glt ‘They’re called, there is no word (for that) in Pichi.’ [dj05be 014]
\z

In contrast, there are many examples of negated \textit{dé}{}-copula \is{copula!locative-existential}clauses with an existential reading, as in the following two examples. Note the occurrence of negative concord in the first of the two following examples: 


\ea%812
    \label{ex:key:812}
    \gll \textbf{Nó}  \textbf{pát}  fɔ  wɔ́l    mɔ́    \textbf{nó}  \textbf{dé}.\\
\textsc{neg}  part  \textsc{prep}  world  more  \textsc{neg}  \textsc{be.loc}\\

\glt ‘There is no other part of the world [where it’s like that].’ [au07se 224]
\z


\ea%813
    \label{ex:key:813}
    \gll “Fam-mán”  \textbf{nó}  \textbf{dé}.\\
\phantom{“}farm.\textsc{cpd}{}-man  \textsc{neg}  \textsc{be.loc}\\

\glt ‘[The word] “Farm-man” doesn’t exist.’ [dj05be 016]
\z

Likewise, the corpus does not reveal any instance of a non-finite \textit{gɛ́t} with an existential sense. Conversely, we once more encounter many examples of non-finite \textit{dé} ‘\textsc{be.loc}’ with an existential reading as in \REF{ex:key:814}:


\ea%814
    \label{ex:key:814}
    \gll Ebanistas  dɛn  \textbf{gɛ́fɔ}    \textbf{dé}.\\
Carpenter.\textsc{pl}  \textsc{pl}  have.to  \textsc{be.loc}\\

\glt ‘Carpenters have to be there/around.’ [hi03cb 042]
\z

The same applies to TMA marking. While quite a few \textit{dé}-existentials are found with overt TMA marking as in \REF{ex:key:815}, there is no such example of a \textit{gɛ́t}-existential. The latter type of existential therefore appears to be prototypical in an additional sense – \textit{gɛ́t} existentials typically predicate a generic situation, which is also marked as such by factative{\fff} tense-aspect:


\ea%815
    \label{ex:key:815}
    \gll Ɛhɛ́  wán  accidente  fɔ  motó  \textbf{bin}  \textbf{dé}.\\
\textsc{intj}  one  accident    \textsc{prep}  car    \textsc{pst}  \textsc{be.loc}\\

\glt ‘Oh yes, there was a car accident.’ [ye03cd 073]
\z

Finally, it is useful to draw attention to the linkages between existential and factive clauses. Factive clauses featuring the copula \textit{dé} are existential clauses with a referentially empty subject position and a complement clause\is{complement clause} introduced by \textit{sé} ‘\textsc{quot}’. The subject is either an expletive \textit{e} ‘\textsc{3sg}’ or a dummy noun\is{dummy noun} like \textit{tín} ‘thing’, as in this example (cf. eg. \ref{ex:key:1139} for further details on factive clauses): 


\ea%816
    \label{ex:key:816}
    \gll \textbf{Di}  \textbf{tín}    \textbf{dé}    \textbf{sé}   mék    e    mék    rabia  wet    mí.\\
\textsc{def}  thing  \textsc{be.loc}  \textsc{quot}    \textsc{sbjv}    \textsc{3sg.sbj}  make  anger  with    \textsc{1sg.indp}\\

\glt ‘The thing is that let her be angry with me.’ [ye05rr 001]
\z

\subsection{Possessives}\label{sec:7.6.4}

Pichi employs a verbal and a copula strategy in the formation of possessive clauses. The verbs \textit{gɛ́t} ‘get, have’ and \textit{hól} ‘hold, keep’ are the principal verbs of possession and express time-stable and non-time-stable possession, respectively. Three collocations involving copulas are also used, albeit less frequently, in order to express possessive relations: \textit{dé fɔ} ‘\textsc{be.loc prep}’ = ‘have’ and \textit{na fɔ} ‘\textsc{foc prep}’ = ‘have’, as well \textit{dé na/fɔ hán} ‘\textsc{be.loc loc/prep} hand’ = ‘have on’. The use of these collocations may also be differentiated along the criterion of time-stability: \textit{dé fɔ} and \textit{na fɔ} express time-stable, and \textit{dé na hán} transient, non-time-stable possession. \tabref{tab:key:7.10} presents some characteristics of possessive clauses.

%%please move \begin{table} just above \begin{tabular
\begin{table}
\caption{Possessive clauses}
\label{tab:key:7.10}

\begin{tabularx}{\textwidth}{l>{\raggedright}p{22mm}Qp{20mm}l}
\lsptoprule
 & Time-stable & Non-time-stable & Possessor & Frequency\\
 \midrule 
Verbal & \textit{gɛ́t} ‘get, have’ & \textstyleTablePichiZchn{hól} ‘\textstyleTableEnglishZchn{keep’} & Subject & Majority\\ \tablevspace
Copula & {\itshape dé fɔ; na fɔ} 
‘be for’ & {\itshape dé na\textup{/}fɔ hán} 
‘be \textsc{loc/prep} hand’ & Prepositional phrase & Minority\\
\lspbottomrule
\end{tabularx}
\end{table}

The verb \textit{gɛ́t} ‘get, have’ expresses permanent, time-stable possession. When \textit{gɛ́t} occurs in a factative marked clause \REF{ex:key:817}, a lexicalised light verb construction \REF{ex:key:818}, an existential construction (cf. \sectref{sec:7.6.3}), or other contexts that propose a generic reading, the verb leans towards the stative meaning ‘own, be in permanent possession’: 


\ea%817
    \label{ex:key:817}
    \gll A    \textbf{gɛ́t}  mɔdɛlɔ́.\\
\textsc{1sg.sbj}  get  mother-in-law\\

\glt ‘I have a mother-in-law.’ [ro05de 009]
\z


\ea%818
    \label{ex:key:818}
    \gll Dí  mán    \textbf{gɛ́t}  \textbf{líba}    ɛ́n,  fɔ  kɔmɔ́t  wet    dís  káyn  bíg  gɛ́l.\\
this  man    get  liver  \textsc{intj}  \textsc{prep}  go.out  with    this   kind    big  girl\\

\glt ‘Than man has guts, right, to go out with such an influential girl.’ [dj05ce 291]
\z

Conversely, when \textit{gɛ́t} co-occurs with a TMA marker with a default or explicit perfective reading \REF{ex:key:819} or sentential aspect suggestive of telicity (i.e. the time clause in \ref{ex:key:820}), an inchoative interpretation of \textit{gɛ́t} as ‘acquire, enter into permanent possession’ is favoured: 


\ea%819
    \label{ex:key:819}
    \gll Di  papá  de  gládin  sé    in    pikín  \textbf{dɔ́n}  \textbf{gɛ́t}  wók.\\
\textsc{def}  father  \textsc{ipfv}  be.glad  \textsc{quot}    \textsc{3sg.poss}  child  \textsc{prf}  get  work\\

\glt ‘The father is happy that his child has found work.’ [dj07ae 073]
\z


\ea%820
    \label{ex:key:820}
    \gll A    \textbf{kin}  \textbf{gɛ́t}  mɔní,  a    kin  fála    húman  dɛn.\\
\textsc{1sg.sbj}  \textsc{hab}  get  money  \textsc{1sg.sbj}  \textsc{hab}  follow  woman  \textsc{pl}\\

\glt ‘(When) I used to receive money, I would chase women.’ [ed03sp 089]
\z

Sometimes we also find the phrases \textit{dé fɔ} ‘\textsc{be.loc} \textsc{prep}’ \is{copula!locative-existential}or \textit{na fɔ} ‘\textsc{be.loc} \textsc{prep}’ \is{copula!identity}expressing time-stable possession. There is no difference in meaning between the two constructions, although \textit{na} ‘\textsc{foc}’ is employed as a time-stable identity copula in other contexts \REF{ex:key:822}: 


\ea%821
    \label{ex:key:821}
    \gll Sɔn    Píchi  \textbf{dé}    \textbf{fɔ}  sɔn    mán    wé  de  síng, 
dɛn  de  kɔ́l  Lapiro.\\
some  Pichi  \textsc{be.loc}  \textsc{prep}  some  man    \textsc{sub}  \textsc{ipfv}  síng
\textsc{3pl}  \textsc{ipfv}  call  \textsc{name}\\

\glt ‘There’s a kind of Pichi used by a man who sings, he’s called 
Lapiro [dé Mbanga].’ [ye05ce 039]
\z


\ea%822
    \label{ex:key:822}
    \gll Di  tín    \textbf{dé}    \textbf{fɔr}=an,    di  tín    \textbf{na}  \textbf{fɔ} ín.\\
\textsc{def}  thing  \textsc{be.loc}  \textsc{prep}=\textsc{3sg.obj}  \textsc{def}  \textsc{thing}  \textsc{foc}  \textsc{prep}  \textsc{3sg.indp}\\

\glt ‘The thing is his, the thing is his (...)’ [dj05ae 239]
\z

The verb \textit{hól} ‘hold, keep’ expresses non-time-stable, temporary possession in a transitive clause like \REF{ex:key:823}. In such contexts, it is best translated as ‘keep’. The temporary nature of possession expressed by \textit{hól} is reaffirmed by the adverbial phrase \textit{durante un mes entero} ‘for one whole month’, which specifies the period of possession: 


\ea%823
    \label{ex:key:823}
    \gll \MakeUppercase{A}   fít  \textbf{hól}    dán    mɔní  \textbf{durante}  \textbf{un}  \textbf{mes}    \textbf{entero}.\\
\textsc{1sg.sbj}  can  keep  that    money  during  \textsc{def}  month  whole\\

\glt ‘I’m able to keep that money for a whole month.’ [ro05rt 049]
\z

Speaker (dj) summarises the difference between \textit{gɛ́t} and \textit{hól} in \REF{ex:key:824}. Note the difference in aspect marking with \textit{hól}, \textit{gɛ́t}, and \textit{dráyb} ‘drive’. Imperfective{\fff} aspect is expressed through factative{\fff} marking with the inchoative-stative verbs \textit{hól} and \textit{gɛ́t}. Meanwhile, it is the presence of \textit{de} ‘\textsc{ipfv}’ that signals imperfective aspect with the dynamic verb \textit{dráyb}:


\ea%824
    \label{ex:key:824}
    \gll “Yu  \textbf{hól}  wán  motó”,  yu  de  dráyb=an,  pero  sé    yu  gɛ́t,
cuando  tienes,    “a    \textbf{gɛ́t}  wán    motó”.\\
\phantom{“}\textsc{2sg}  hold  one  car    \textsc{2sg}  \textsc{ipfv}  drive=\textsc{3sg.obj}  but    \textsc{quot}    \textsc{2sg}  get
when  you.get    \phantom{“}\textsc{1sg.sbj}  get  one    car\\

\glt ` “Yú hól wán motó” (means) you’re driving it, but when you possess it,
when you have it “a gɛ́t wán motó.”’ [dj05ae 223]
\z

The notion of temporary possession expressed by \textit{hól} ‘hold, keep’ may also be applied to a human-possessed \textsc{NP}. A characteristic of West African pedagogy is to confer responsibility for the upbringing of a child to members of the extended family other than the biological parents. Such temporary guardianship is also expressed by \textit{hól}. I leave it to speaker (au) to explain the meaning of \textit{hól} in sentences \REF{ex:key:825} and \REF{ex:key:826}:


\ea%825
    \label{ex:key:825}
    \gll A    \textbf{hól}    mi    brɔ́da  in    pikín,  a    de  {trén  =an}.\\
\textsc{1sg.sbj}  hold    \textsc{1sg.poss}  brother  \textsc{3sg.poss}  child  \textsc{1sg.sbj}  \textsc{ipfv}  train=\textsc{3sg.obj}\\

\glt ‘Because I have guardianship over my brother’s child, I’m bringing him up.’ [au07se 141]
\z


\ea%826
    \label{ex:key:826}
    \gll Bikɔs  e    \textbf{hól}    yú    na  hós    yu  gɛ́fɔ    gɛ́t
di  hóm    trénin.\\
because  \textsc{3sg.sbj}  hold    \textsc{2sg.indp}  \textsc{loc}  house  \textsc{2sg}  have.to  get
\textsc{def}  home  training\\

\glt ‘Because she has guardianship over you in her house you have to receive 
home education.’ [au07se 130]
\z

When \textit{hól} ‘hold, keep’ is employed as a dynamic verb in a transitive clause, it has the literal meaning of ‘hold’, hence the presence of the imperfective marker \textit{de} in the following example: 


\ea%827
    \label{ex:key:827}
    \gll Nó,  na  di  húman  \textbf{de}  \textbf{hól}    di  plét.\\
\textsc{neg}  \textsc{foc}  \textsc{def}  woman  \textsc{ipfv}  hold    \textsc{def}  plate\\

\glt ‘No, it’s the woman that’s holding the plate.’ [ra07se 012]
\z

A second strategy for establishing a non-time-stable possessive relation makes use of the phrasal expression \textit{dé na X hán}/\textit{dé fɔ X hán} ‘be in X’s hand’, where X is the possessor. This phrase is another variant of the copula strategy of possessive clause formation. In such invariably intransitive clauses, the subject instantiates the possessed \textsc{NP} and a prepositional phrase the possessor. In the following example, the transient nature of possession is underscored by the time adverb \textit{náw} ‘now’: 


\ea%828
    \label{ex:key:828}
    \gll George,  mi    móvil  nó  \textbf{dé}    \textbf{na}  \textbf{mi}    \textbf{hán},  
a    nó  gɛ́t  móvil  náw.\\
\textsc{name}  \textsc{1sg.poss}  mobile  \textsc{neg}  \textsc{be.loc}  \textsc{loc}  \textsc{1sg.poss}  hand
\textsc{1sg.sbj}  \textsc{neg}  get  mobile  now\\

\glt ‘George, I don’t have my mobile phone on me, I don’t have 
a mobile phone now.’ [dj05ae 088]
\z

All possessive clauses covered in this section can be negated by standard verb negation\is{negation}. The negator \textit{nó} ‘\textsc{neg}’ is inserted between the personal pronoun and the verb:


\ea%829
    \label{ex:key:829}
    \gll Yu  sabí    sé    yu  \textbf{nó}  \textbf{gɛ́t}  pikín?\\
\textsc{2sg}  know  \textsc{quot}    \textsc{2sg}  \textsc{neg}  get  child\\

\glt ‘Do you know whether you don’t have a child?’ [fr03wt 173]\is{possession verbs}
\z

\subsection{Predicate adjectives}\label{sec:7.6.5}

We are concerned here with a few property items that may be employed as predicate adjectives next to their use as inchoative-stative verbs. The fluidity between adjective and verb with these items shows that, notwithstanding its existence, the verb-adjective distinction is weak in Pichi. Adjectives can be identified by their distribution. Only adjectives may appear as complements to the locative-existential copula \textit{dé} in predicate adjective clauses, such as the following one: 


\ea%830
    \label{ex:key:830}
    \gll Tidé    di  húman  \textbf{dé}    \textbf{fáyn}.\\
today  \textsc{def}  woman  \textsc{be.loc}  fine\\

\glt ‘Today the woman is fine.’ [dj05ae 153]
\z

In \REF{ex:key:830}, \textit{fáyn} ‘be fine’ is used as an adjective and denotes a physical property, namely a body state in an intransitive clause. The predicate adjective construction featuring the copula \textit{dé} translates as ‘be fine, well, healthy’. Contrast this meaning with \REF{ex:key:831}, where \textit{fáyn} is employed as an inchoative-stative verb with the meaning ‘be intrinsically fine’ hence ‘beautiful’. In the latter example, \textit{fáyn} therefore denotes a value:


\ea%831
    \label{ex:key:831}
    \gll Di  húman  \textbf{fáyn}.\\
\textsc{def}  woman  be.fine\\

\glt ‘The woman is beautiful.’ [dj05ae 149]
\z

In the corpus, a handful of property items show the potential to function as predicate adjectives. As a general rule, the perceived time stability of the property determines whether it is used as a time-stable inchoative-stative verb or a non-time-stable adjective. The most consistent time-stability distinction is found with the words \textit{bád} ‘be bad, ill’, \textit{fáyn} ‘be fine, beautiful’, and \textit{gúd} ‘be good, well’. When they occur as adjectives, they denote a body state. When they occur as inchoative-stative verbs, these property items denote a value, an intrinsic property. 


Only these three words are unequivocal members of the small adjective class in Pichi. Beyond that, a few more property items are rarely used as predicate adjectives. \tabref{tab:key:7.11} lists all property items attested in predicate adjective constructions in the corpus.


%%please move \begin{table} just above \begin{tabular
\begin{table}
\caption{Predicate adjectives}
\label{tab:key:7.11}

\begin{tabularx}{\textwidth}{lXlXlX}
\lsptoprule
\textstyleTablePichiZchn{bád}  & ‘ill’ & \textstyleTablePichiZchn{bráyt} &  ‘bright’ & \textstyleTablePichiZchn{frí} &  ‘free’\\
\textstyleTablePichiZchn{fáyn}  &  ‘fine’ & \textstyleTablePichiZchn{wɔwɔ́}  &  ‘messed up’ & \textstyleTablePichiZchn{sló}  &  ‘slow’\\
\textstyleTablePichiZchn{gúd}  &  ‘well’ & \textstyleTablePichiZchn{pyɔ́}    & ‘pure’ & \textit{spɛ́shal} & ‘special’\\
\lspbottomrule
\end{tabularx}
\end{table}

The words in the second and third columns of \tabref{tab:key:7.11} appear as predicate adjectives in the corpus only rarely. For example, the property item \textit{bráyt} ‘be bright, glowing with beauty’ is attested as an adjective where it denotes a visible body state as in \REF{ex:key:832} – the speaker is an elderly lady giving an account of her youth. Compare \textit{frɛ́s} ‘be fresh’ in the same sentence, which is used as an inchoative-stative verb to denote a more lasting body state of freshness or youthfulness:


\ea%832
    \label{ex:key:832}
    \gll Moka  bɔ́y  dɛn  krés    wé  dɛn  sí  lɛk  háw    a    frɛ́s,  
na  so    a    \textbf{dé}    \textbf{bráyt}.\\
\textsc{place}  boy  \textsc{pl}  go.mad  \textsc{sub}  \textsc{3pl}  see  like  how    \textsc{1sg.sbj}  be.fresh  
\textsc{foc}  like.that  \textsc{1sg.sbj}  \textsc{be.loc}  bright\\

\glt ‘The Moka boys went crazy when they saw how fresh I was, that’s how bright 
I looked.’ [ab03ay 059]
\z


\ea%833
    \label{ex:key:833}
    \gll Yu  skín    \textbf{bráyt}    ó.\\
\textsc{2sg}  body  be.bright    \textsc{sp}\\

\glt ‘Your body is really glowing (with beauty).’ [dj07ae 165]
\z

The physical property item \textit{wɔwɔ́} ‘be ugly, messed up’ is used by the same speaker as an adjective in \REF{ex:key:834} and as an inchoative-stative verb in \REF{ex:key:835}. The first example featuring \textit{wɔwɔ́} again expresses a visible state of the street, while the second is more time-stable in its meaning: 


\ea%834
    \label{ex:key:834}
    \gll Di  strít    \textbf{dé}   \textbf{wɔwɔ́}.\\
\textsc{def}  street  \textsc{be.loc}  ugly\\

\glt ‘The street looks messed up.’ [dj05ae 136]
\z


\ea%835
    \label{ex:key:835}
    \gll Di  strít    \textbf{wɔwɔ́},  di  strít    \textbf{chakrá},      
di  strít    nó  \textbf{dé}    \textbf{fáyn}.\\
\textsc{def}  street  be.ugly  \textsc{def}  street  be.destroyed    
\textsc{def}  street  \textsc{neg}  \textsc{be.loc}  fine\\

\glt ‘The street is messed up, the street is destroyed, the street is not fine.’ [dj05ae 134]
\z

Predicate adjective clauses may be marked for \textsc{TMA} like any other copula clause featuring the copula \textit{dé}. Compare the adjective \textit{bád} ‘ill’ in \REF{ex:key:836} with a future tens\is{future tense}e reference:


\ea%836
    \label{ex:key:836}
    \gll Wé  yu  go  fɔdɔ́n  yu  \textbf{go}  \textbf{dé}    \textbf{bád}.\\
\textsc{sub}  \textsc{2sg}  \textsc{pot}  fall    \textsc{2sg}  \textsc{pot}  \textsc{be.loc}  bad\\

\glt ‘When you fall you’ll be in a bad state.’ [ab03ay 114]
\z

Adjectives may also be employed attributively as prenominal modifiers. In this, adjectives behave no differently from other property items (cf. \sectref{sec:5.2.1}). Below, the adjective \textit{fáyn} ‘be fine’ appears as a modifier of \textit{gɛ́l} ‘girl’: 


\ea%837
    \label{ex:key:837}
    \gll Yu  sí  wán    \textbf{fáyn}  \textbf{gɛ́l},  yu  de  gó  tún=an.\\
\textsc{2sg}  see  one    fine    girl  \textsc{2sg}  \textsc{ipfv}  go  tune=\textsc{3sg.obj}\\

\glt ‘(If) you see a fine girl, you go chat her up.’ [au07se 062]
\z

The class of adjectives is closed for words of Pichi origin, since the use of property items as copula complements is lexically restricted. But the predicate adjective construction is a port of entry for Spanish adjectives (cf. \sectref{sec:13.2.2}). 


Finally, I draw attention to the various other means of attributing properties to a noun. Speakers make use of postnominal modification through relative or quotative clause{\fff}s. Other ways of expressing modification are associative constructions and compounding{\fff}. Two strategies of modification serve as a productive means of deriving new property items next to the use of Spanish adjectives in the Pichi predicate adjective construction. A \textit{dé}-copula clause with an adverbial complement {\fff}featuring \textit{wet} ‘with’ \REF{ex:key:838}, as well as light verb constructions involving \textit{gɛ́t} ‘get, have’ \REF{ex:key:839} allow the attribution of a property to a referent: 



\ea%838
    \label{ex:key:838}
    \gll E    hád    fɔ  mék    mék    dɛn  bíl    na  yá    só  bikɔs
di  grɔ́n    e    tú  \textbf{dé}    \textbf{wet}    \textbf{stón}.\\
\textsc{3sg.sbj}  be.hard  \textsc{prep}  make  \textsc{sbjv}    \textsc{3pl}  build  \textsc{loc}  here    like.that  because
\textsc{def}  ground  \textsc{3sg.sbj}  too  \textsc{be.loc}  with    stone\\

\glt It’s hard for them to build here because the ground is too stony.’ [dj05be 111]
\z


\ea%839
    \label{ex:key:839}
    \gll E    hád    fɔ  bíl    na  yá    bikɔs  sé    di  grɔ́n
\textbf{gɛ́t}  \textbf{bɔkú}  \textbf{sansán}.\\
\textsc{3sg.sbj}  be.hard  \textsc{prep}  build  \textsc{loc}  here    because  \textsc{quot}    \textsc{def}  ground
get  much  sand\\
\glt ‘It’s hard to build here because the ground is very sandy.’ [ro05ee 063]
\z

\section{Adverbial modification}\label{sec:7.7}

Pichi adverbials modify verbs and clauses. It is useful to distinguish between adverbs proper and adverbials. I employ “adverbial” as a cover term, which includes adverbs, but also encompasses other clause constituents with the functions of adverbs. Adverbs constitute an underived, largely monomorphemic minor word class\is{word classes} of their own, and unlike other constituents that may function as adverbials (e.g. common NPs), they do not normally appear in the syntactic positions of other word classes. 


Adverbials may occupy a clause-initial, a preverbal, a postverbal and a clause-final position. Some adverbs consist of a single morpheme (e.g. \textit{bambáy} ‘gradually’, \textit{náw} ‘now’), others are lexicalised phrases with idiosyncratic, underivable meanings (for instance. \textit{sɔn.tɛ́n} ‘some.time’ = ‘perhaps’). Other expressions are more or less conventionalised phrases, constituted by means of phrasal syntax (e.g. \textit{bɔkú tɛ́n dɛn} ‘many times, often’), but usually not encountered in non-adverbial functions. Often such noun phrase adverbials are fixed collocations involving generic nouns denoting time (\textit{tɛ́n} ‘time’, \textit{áwa} ‘hour’), manner (\textit{stáyl} ‘style’, \textit{fásin} ‘manner’), and space (\textit{sáy} ‘side’, \textit{plés} ‘place’, \textit{pát} ‘part’). There is thus a smooth transition from more basic monomorphemic adverbs to more or less lexicalised adverbial phrases. 



The expression of degree and manner modification is particularly rich and varied in Pichi and deserves special attention. It should, however, also be pointed out that many adverbial notions are expressed by wholly different means than adverbials. For example, movement verbs may take goal objects, while some spatial and temporal notions may be expressed by motion-direction and adverbial SVCs. Many ideophones function as manner adverbials next to the adverbs of manner covered in this section. 



Equally, many clause linkers are not very different in function from the linking adverbs listed in \tabref{tab:key:7.12} below (e.g. \textit{bikɔs} ‘because’, \textit{adɔnkɛ́} ‘even if’). Further, modal clauses with expletive subjects (e.g. \textit{e fít bí sé} ‘\textsc{3sg.sbj} can be \textsc{quot}’ = ‘it could be that’, and \textit{e fíba sé} ‘\textsc{3sg.sbj} seem \textsc{quot}’ = ‘it seems that’) convey meanings similar to those of sentence adverbs like \textit{sɔntɛ́n} ‘perhaps’ and \textit{mébi} ‘maybe’.


\subsection{Adverbs}\label{sec:7.7.1}

\tabref{tab:key:7.12} presents all monomorphemic adverbs found in the corpus and the most common conventionalised phrasal expressions with adverbial functions. The preferred or canonical syntactic positions are also indicated. The table also contains the two most common Spanish-derived adverbs \textit{pero} ‘but’ and \textit{bueno} ‘alright’. Adverbs with multiple meanings are arranged in all the corresponding “adverb type” sections (e.g. \textit{smɔ́ltɛn} ‘shortly after’ = locative adverb, \textit{smɔ́ltɛn} ‘nearly’ = modal adverb.

%%please move \begin{table} just above \begin{tabular
\begin{table}
\caption{Adverbs}
\label{tab:key:7.12}

\begin{tabularx}{\textwidth}{p{16mm}Ql}
\lsptoprule
Adverb type & Adverbs & Preferred position\\
\midrule
Locative & \textstyleTablePichiZchn{dé} ‘there’, \textstyleTablePichiZchn{yá} \textstyleTablePichiZchn{(só)} ‘here’, \textstyleTablePichiZchn{hía} ‘here’, \textstyleTablePichiZchn{yandá} ‘yonder’, \textstyleTablePichiZchn{aráwn} ‘around’ & Clause final\\
Time & \textit{bambáy} ‘gradually’, \textit{náw} ‘now’, \textit{fɔ́s} ‘first’, \textit{fɔ́s tɛ́n} ‘formerly’, \textit{sɔn tɛ́n dɛn} ‘sometimes’, \textit{smɔ́ltɛn} ‘shortly after’, \textit{wán tɛ́n} ‘(at) once’, \textit{wán wán tɛ́n} ‘from time to time’ & Clause initial\\
& \textit{bɔkú tɛ́n} ‘for a long time’, \textit{lɔ́n tɛ́n} ‘long ago’, \textit{sóté} ‘for a long time’, \textit{mɔ́} ‘again’, \textit{yét} ‘yet’ & Clause final\\
& \textit{wán dé} ‘someday’, \textit{nó wán dé} ‘never’, \textit{ɔ́l tɛ́n} ‘always’, \textit{ɔ́l áwa} ‘all the time’ & Clause initial or final\\
& \textstyleTablePichiZchn{jís/jɔ́s} ‘just’, \textstyleTablePichiZchn{stíl} ‘still’ & Preverbal\\
Degree & \textstyleTablePichiZchn{tú (mɔ́ch)} ‘too (much)’ ,\textstyleTablePichiZchn{só} ‘so much’ & Preverbal\\
& { \textstyleTablePichiZchn{bád} ‘extremely’, \textstyleTablePichiZchn{mɔ́} ‘more’, \textstyleTablePichiZchn{mɔ́-ɛn-mɔ́} ‘more and more’,}

\textstyleTablePichiZchn{sóté} ‘extremely’, \textstyleTablePichiZchn{óva} ‘excessively’, \textstyleTablePichiZchn{sóté} ‘excessively’ & Clause final\\
Linking & \textstyleTablePichiZchn{áfta} ‘then’, \textstyleTablePichiZchn{(e) fínis} ‘then\textstyleTablePichiZchn{’, bɔt /bɛt} ‘but’, \textstyleTablePichiZchn{so} ‘so’, \textstyleTablePichiZchn{na ín} ‘that’s when, that’s why’,\textstyleTablePichiZchn{ dásɔl} ‘then’, \textstyleTablePichiZchn{pero} ‘but’, \textstyleTablePichiZchn{bueno} ‘alright’ & Clause initial\\
Modal \&\newline  \mbox{evaluative} & \textit{bádtɛn} ‘unfortunately’, \textit{smɔ́ltɛn} ‘nearly’, \textit{sɔntɛ́n} ‘perhaps’, \textit{mébi} ‘maybe’ & Clause initial\\
& \textstyleTablePichiZchn{ó} ‘\textsc{sp}’ & Clause final\\
Manner & \textstyleTablePichiZchn{kwík} ‘quickly’, \textstyleTablePichiZchn{haydháyd} ‘secretly’, \textstyleTablePichiZchn{só} ‘like that’, \textstyleTablePichiZchn{fáyn} ‘well’, ideophones\is{ideophones} & Clause final\\
\lspbottomrule
\end{tabularx}
\end{table}

Adverbs that appear at the beginning modify the sentence in its entirety – they have a wide scope. In \REF{ex:key:840}, the linking adverb adverb \textit{pero} ‘but’, the modal adverb \textit{sɔntɛ́n} ‘perhaps’, and the time adverb \textit{bambáy} ‘gradually’ all occur sentence-initially:


\ea%840
    \label{ex:key:840}
    \gll \textbf{Pero}  \textbf{bambáy}    \textbf{bambáy}  \textbf{sɔntɛ́n}  yu  go  sí  di  wán  
wé  go  máred  yú.    \\
but    gradually  \textsc{rep}    perhaps  \textsc{2sg}  \textsc{pot}  see  \textsc{def}  one  
\textsc{sub}  \textsc{pot}  marry  \textsc{2sg.indp}\\

\glt ‘But very gradually perhaps you might find the one who will marry you.’ [ab03ab 204]
\z

Locative and time adverbs may also occur after the verb, in which case they have narrow scope and modify the meaning of the verb alone. In \REF{ex:key:841}, the repeated locative noun \textit{dɔ́n} ‘down’ and the locative adverb \textit{yandá} ‘yonder’ modify the verb \textit{sidɔ́n} ‘sit, stay’: 


\ea%841
    \label{ex:key:841}
    \gll Bɔt  ín    sidɔ́n  \textbf{dɔ́n}    \textbf{dɔ́n}  \textbf{dɔ́n}  \textbf{yandá}.\\
but  \textsc{3sg.indp}  stay    down  \textsc{rep}  \textsc{rep}  yonder\\

\glt ‘But he stays far down over there.’ [ma03ni 026]
\z

The data contains diverse time adverbs. A few of these are monomorphemic, e.g. \textit{bambáy} ‘gradually’ in \REF{ex:key:840} above. Others are more or less idiosyncratic phrases containing the time-denoting generic noun \textit{tɛ́n} ‘time’, as in \textit{bɔkú tɛ́n} ‘much time’ = ‘for a long time’ \REF{ex:key:842} or \textit{dé} ‘day’, as in \textit{wán dé} ‘someday’ \REF{ex:key:843}. Location{\fff}-in-time adverbs, like \textit{wán dé} prefer the clause-initial, duration adverbs like \textit{lɔ́n tɛ́n} ‘long ago’ and \textit{bɔkú tɛ́n} the clause-final position: 


\ea%842
    \label{ex:key:842}
    \gll Nó  chɛ́k=an    \textbf{bɔkú}  \textbf{tɛ́n},    tɛ́l  mí    sé    nó.\\
\textsc{neg}  think=\textsc{3sg.obj}  much  time    tell  \textsc{1sg.indp}  \textsc{quot}    \textsc{neg}\\

\glt ‘Don’t think about it for a long time, tell me “no”.’ [ye07me 034]
\z

Adverbs with generic time reference like \textit{wán dé} ‘someday’ and \textit{ɔ́l tɛ́n} ‘always’ are equally often encountered in the initial as well as the final position: 


\ea%843
    \label{ex:key:843}
    \gll Na  ín    \textbf{wán}    \textbf{dé}  a    bin  tɛ́l  wán    grand  frère     na, 
na  mi    colegio  dé,    \op...\cp{}\\
\textsc{foc}  \textsc{3sg.indp}  one    day  \textsc{1sg.sbj}  \textsc{pst}  tell  one    big     brother    \textsc{loc} 
\textsc{loc}  \textsc{1sg.poss}  college  there\\

\glt ‘That’s why one day, I told one of my seniors in, in my secondary school 
there, (...)’ [ye07ga.003]
\z


\ea%844
    \label{ex:key:844}
    \gll \MakeUppercase{A}   mɔs    gó  Alemania  \textbf{wán}    \textbf{dé}.\\
\textsc{1sg.sbj}  \textsc{obl}    go  \textsc{place}    one    day\\

\glt ‘I absolutely have to go to Germany someday.’ [to07fn 197]
\z

In clauses featuring double-object constructions, speakers may place a time adverbial between the recipient\is{recipient} or beneficiary object\is{beneficiary} and the patient\is{patient} object instead of placing it in the clause-initial or clause final position. This position appears to be focus-induced, since it was encountered more often during the elicitation of adverbials than in natural speech:


\ea%845
    \label{ex:key:845}
    \gll Ebongolo  tɛ́l  mí    \textbf{yɛ́stadé}    in    problema.\\
\textsc{name}    tell  \textsc{1sg.indp}  yesterday  \textsc{3sg.poss}  problem\\

\glt ‘Ebongolo told me about his problem yesterday.’ [dj07ae 347]
\z

The phrase \textit{e fínis} ‘\textsc{3sg.sbj} finish’ = ‘then’ is a stand-alone clause, which may function as a linking “adverb” \REF{ex:key:846}. A formal indication of its hybrid status between clause and adverb is that the personal pronoun \textit{e} ‘\textsc{3sg.sbj}’ is sometimes dropped: 


\ea%846
    \label{ex:key:846}
    \gll A    gó  wás    wet    mi    hán    mí    sénwe 
a    dráy=an,    \textbf{e}    \textbf{fínis}  a    áyɛn=an.\\ 
\textsc{1sg.sbj}  go  wash  with    \textsc{1sg.poss}  hand  \textsc{1sg.indp}  \textsc{emp} 
\textsc{1sg.sbj}  dry=\textsc{3sg.obj}  \textsc{3sg.sbj}  finish  \textsc{1sg.sbj}  iron=\textsc{3sg.obj}\\

\glt ‘I myself went to wash (it) with my own hands, I dried it, then ironed it.’ [dj07re 050]
\z

The two modal adverbs \textit{sɔntɛ́n} ‘perhaps’ (cf. \ref{ex:key:840} above) and \textit{smɔ́ltɛn} ‘nearly’ \REF{ex:key:848} and the evaluative adverb \textit{bádtɛn} ‘unfortunately’ \REF{ex:key:847} are lexicalised phrases involving the generic noun \textit{tɛ́n} ‘time’ as a formative element (cf. \ref{ex:key:103} above). Modal and evaluative adverbs are normally found in the initial position with scope over the entire clause: 


\ea%847
    \label{ex:key:847}
    \gll \textbf{Bádtɛn}      náw,  di  fɔ́s  dɔ́kta  wé  wi  gɛ́t, 
e    nó  dé   ɔ́p  na  ɔspítul.\\
unfortunately  now    \textsc{def}  first  doctor  \textsc{sub}  \textsc{1pl}  get
\textsc{3sg.sbj}  \textsc{neg}  \textsc{be.loc} up  \textsc{loc}  hospital\\

\glt ‘Unfortunately, the first doctor that we had wasn’t up 
(there) in the hospital.’ [ab03ay 078]
\z


\ea%848
    \label{ex:key:848}
    \gll \textbf{Smɔ́ltɛn}   a    bin  fɔ    dáy  dé.\\
nearly    \textsc{1sg.sbj}  \textsc{pst}  \textsc{cond}    die  there\\

\glt ‘I nearly died there.’  [ed07fn 493]
\z

The adverb \textit{mébi} ‘maybe’ \REF{ex:key:849} is not as common as \textit{sɔntɛ́n} ‘perhaps’. Note that \textit{smɔ́ltɛn} \{small.time\} has an entirely opaque sense ‘nearly’ in the example above, and a more transparent, temporal sense ‘shortly (after)’ in \REF{ex:key:850} below.


\ea%849
    \label{ex:key:849}
    \gll \textbf{Mébi}  dɛn  nó  go  bɛ́g  yú    plɛ́nte  fɔ  pé  \op...\cp{}\\
maybe  \textsc{3pl}  \textsc{neg}  \textsc{pot}  beg  \textsc{2sg.indp}  plenty  \textsc{prep}  pay\\

\glt ‘Maybe they won’t ask you to pay a lot (...)’ [hi03cb 011]
\z


\ea%850
    \label{ex:key:850}
    \gll \textbf{Smɔ́ltɛn}   e    mék    hɛɛɛ.\\
shortly    \textsc{3sg.sbj}  make  ‘exhalation’\\

\glt ‘Shortly after, he made [imitates exhalation].’ [ab03ab 086]
\z

The L-toned clause-initial linking adverb \textit{so} ‘so’ \REF{ex:key:851} differs from the H-toned deictic manner adverb \textit{só} ‘like this, like that’ \REF{ex:key:852} in tone alone. The deictic manner adverb \textit{so} ‘like this’ is often focused and fronted in a \textit{na} cleft construction, in order to establish reference to preceding discourse material \REF{ex:key:853} (cf. also \sectref{sec:7.4.3.3}):


\ea%851
    \label{ex:key:851}
    \gll \textbf{So}  di  ɔ́da    wán    de  lístin=an.\\
So  \textsc{def}  other  one    \textsc{ipfv}  listen=\textsc{3sg.obj}\\

\glt ‘So the other one is listening to him.’ [au07se 101]
\z


\ea%852
    \label{ex:key:852}
    \gll E    de  pás  \textbf{só}    lɛk  sé    e    nó  nó    mí    mɔ́.\\
\textsc{3sg.sbj}  \textsc{ipfv}  pass  like.this  like  \textsc{quot}    \textsc{3sg.sbj}  \textsc{neg}  know  \textsc{1sg.indp}  more\\

\glt ‘She was passing by just like that as if she didn’t know me anymore.’ [ru03wt 041]
\z


\ea%853
    \label{ex:key:853}
    \gll Na  \textbf{só}    dɛn  go  mék    yú.\\
\textsc{foc}  like.this  \textsc{3pl}  \textsc{pot}  make  \textsc{2sg.indp}\\

\glt ‘That’s what they would do to you.’ [ab03ay 045]
\z

The H-toned adverb \textit{só} ‘like that’ is also found in the conventionalised collocations \textit{(na) yá só} ‘right here’ \REF{ex:key:854} and \textit{náw só} ‘right now’ \REF{ex:key:855}, where its deictic character provides emphasis: 


\ea%854
    \label{ex:key:854}
    \gll Frɔn    \textbf{na}  \textbf{yá}    \textbf{só}    dɛn  kin  controla  di  húman.\\
from  \textsc{loc}  here    like.that  \textsc{3pl}  \textsc{hab}  control  \textsc{def}  woman\\

\glt ‘They control the woman from right here.’ [ed03sb 158]
\z


\ea%855
    \label{ex:key:855}
    \gll \textbf{Náw}    \textbf{só}    taksí,  nó  extranjero  nó  de  drɛ́b    taksí  mɔ́.\\
now    like.that  taxi    \textsc{neg}  foreigner  \textsc{neg}  \textsc{ipfv}  drive  taxi    more\\

\glt ‘Right now, as for taxis, no foreigner drives taxis anymore.’ [ye07je 177]
\z

Manner adverbs other than \textit{só} ‘like that’ and ideophonic adverbs generally occur after the verb, since they directly modify the meaning of the verb. Compare \textit{kwík} ‘quickly’ and the ideophone \textit{kwáráng} in the two following sentences:


\ea%856
    \label{ex:key:856}
    \gll Bɔt  dá  mɔní  de  fínis    \textbf{kwík}.\\
but  that  money  \textsc{ipfv}  finish  quickly\\

\glt ‘But that money used to finish quickly.’ [ed03sp 088]
\z


\ea%857
    \label{ex:key:857}
    \gll Dɛn  de  plé=an    \textbf{kwáráng}.\\
\textsc{3pl}  \textsc{ipfv}  play=\textsc{3sg.obj}  \textsc{ideo}\\

\glt ‘It is played with this hollow sound (of the seeds 
falling into the pits of the wooden Oware board).’
\z

Pichi has a small set of four preverbal adverbs, which appear in the predicate before the verb. The set includes the time adverbs\textit{ jís}\textit{\textup{/}}\textit{jɔ́s} ‘just’ and \textit{stíl} ‘still’, as well as the degree adverbs \textit{só} ‘so much’, \textit{tú (mɔ́ch)} ‘too much’. The use of the preverbal time adverbs \textit{jís}\textit{\textup{/}}\textit{jɔ́s} and \textit{stíl} coincides with resumptive imperfective\is{resumptive imperfective marking} aspect marking – the adverbs are preceded and followed by \textit{de} ‘\textsc{ipfv}’. The aspect-marking functions of the time adverbs \textit{jís}\textit{\textup{/}}\textit{jɔ́s} ‘just’ and \textit{stíl} ‘still’ are covered in \sectref{sec:6.4.2} and \sectref{sec:6.4.4}, repectively (cf. also for a discussion of the position of preverbal adverbs):\is{aspect}


\ea%858
    \label{ex:key:858}
    \gll Náw    dɛn  de  \textbf{jís}  de  kán.\\
now    \textsc{3pl}  \textsc{ipfv}  just  \textsc{ipfv}  come\\

\glt ‘Now, they’re just coming.’ [ye07je 179]
\z

Preverbal degree adverbs usually occur with gradable property items or light verb constructions which attribute properties as in \REF{ex:key:859}. Hence, sentences like \REF{ex:key:860}, in which a non-gradable verb (i.e. \textit{tɔ́k} ‘talk’), and a dynamic one at that, is preceded by a preverbal degree adverb, are very rare: 


\ea%859
    \label{ex:key:859}
    \gll Yu  \textbf{tú}  \textbf{lɛ́k}  húman.\\
\textsc{2sg}  too  like  woman\\

\glt ‘You’re too much of a womaniser.’ [ge07fn 02]
\z


\ea%860
    \label{ex:key:860}
    \gll E    fíba    lɛk  sé    a    dɔ́n  de  \textbf{tú}  \textbf{tɔ́k}  bɔkú.\\
\textsc{3sg.sbj}  resemble  like  \textsc{quot}    \textsc{1sg.sbj}  \textsc{prf}  \textsc{ipfv}  too  talk  much\\

\glt ‘It seems like I’m talking to much.’ [be07he 015]
\z

Non-gradable verbs are more likely to be modified postverbally by the expression \textit{tú mɔ́ch} ‘too much’ than by preverbal \textit{tú} ‘too (much)’ \REF{ex:key:861}. The phrase \textit{tú mɔ́ch} includes the quantifying adverb \textit{mɔ́ch.} When a verb is modified in this way for superlative degree, the use of \textit{mɔ́ch} is mandatory. The same applies when \textit{tú mɔ́ch} modifies a nominal \REF{ex:key:862}:


\ea%861
    \label{ex:key:861}
    \gll E    de  só    in    sɛ́f  \textbf{tú}  \textbf{mɔ́ch}.\\
\textsc{3sg.sbj}  \textsc{ipfv}  show  \textsc{3sg.poss}  self  too  much\\

\glt ‘He boasts too much.’ [ye07je 133]
\z


\ea%862
    \label{ex:key:862}
    \gll A    de  fíl  \textbf{tú}  \textbf{mɔ́ch}  \textbf{hɔ́t}.\\
\textsc{1sg.sbj}  \textsc{ipfv}  feel  too  much  heat\\

\glt ‘I’m feeling too hot [too much heat].’ [dj07ae 316]
\z

Nonetheless, \textit{tú mɔ́ch} may also be used in preverbal position without any difference in meaning to \textit{tú} ‘too (much)’. The following sentence features both possibilities. While the compound\is{compounding} property item \textit{smɔl.skín} ‘small.body’ = ‘be thin’ is modified preverbally, the property item \textit{dráy} ‘be dry, haggard’ is modified postverbally by \textit{tú mɔ́ch}: 


\ea%863
    \label{ex:key:863}
    \gll Di  pikín  \textbf{tú}  \textbf{mɔ́ch}  \textbf{smɔlskín},  e    \textbf{dráy}  \textbf{tú}  \textbf{mɔ́ch}.\\
\textsc{def}  child  too  much  be.thin    \textsc{3sg.sbj}  be.dry  too  much\\

\glt ‘The girl is too thin, she’s too lean.’ [dj07ae 206]
\z

Somewhat similar to the distribution of \textit{tú (mɔ́ch)} is that of the adverb \textit{só} ‘like that, that much’. When \textit{só} occurs in a preverbal position, it implicitly expresses equative degree and means ‘that much’ \REF{ex:key:864}. However, when \textit{só} appears in the clause-final position, it means ‘like that’ and therefore retains its central meaning as a manner adverb (cf. \ref{ex:key:852} above):\is{preverbal adverbs}


\ea%864
    \label{ex:key:864}
    \gll Dɛn  nó  de  \textbf{só}    \textbf{yús=}an    mɔ́.\\
\textsc{3pl}  \textsc{neg}  \textsc{ipfv}  like.that  use=\textsc{3sg.obj}  more\\

\glt ‘It’s not used that much anymore.’ [ye07je 009]
\z

The word \textit{mɔ́} ‘be more, again’ also functions as a degree adverb and is characterised by an unusual amount of syntactic flexibility. In contexts other than comparison, \textit{mɔ́} may occur clause-finally as a time adverb with the meaning ‘again’ \REF{ex:key:865} and ‘still’ (\ref{ex:key:866}–\ref{ex:key:867}):


\ea%865
    \label{ex:key:865}
    \gll Pút=an    bihɛ́n  \textbf{mɔ́}!\\
put=\textsc{3sg.obj}  behind  more\\

\glt ‘Put it behind [rewind] again!’ [au07se 057]
\z


\ea%866
    \label{ex:key:866}
    \gll Dɛn  sé    nóto  ín    wán,  ɔ́da    wán    dé    \textbf{mɔ́}.\\
\textsc{3pl}  \textsc{quot}    \textsc{neg}.\textsc{foc}  \textsc{3sg.indp}  one    other  one    \textsc{be.loc}  more\\

\glt ‘They said it’s not her alone, there’s yet another one.’ [ed03sb 069]
\z


\ea%867
    \label{ex:key:867}
    \gll E    de  sigue  \textbf{mɔ́}.\\
\textsc{3sg.sbj}  \textsc{ipfv}  continue  more\\

\glt ‘It’s still continuing.’ [ro05rr 003]
\z

In negative clauses, \textit{mɔ́} is best translated as ‘anymore, no longer, not again’. Compare the following examples with the negated dynamic verb \textit{ánsa} ‘answer’ \REF{ex:key:868}, and \REF{ex:key:869} with the negated stative verb and copula \textit{dé} ‘\textsc{be.loc}’:


\ea%868
    \label{ex:key:868}
    \gll E    dé    e    nó  de  \textbf{ánsa} mí    \textbf{mɔ́}.\\
\textsc{3sg.sbj}  \textsc{be.loc}  \textsc{3sg.sbj}  \textsc{neg}  \textsc{ipfv}  answer  \textsc{1sg.indp}  more\\

\glt ‘She was (just) there (and) wasn’t responding to me any more.’ [ru03wt 041]
\z


\ea%869
    \label{ex:key:869}
    \gll Frɔn    Rebola  bajando    e    nó  go  \textbf{dé}    \textbf{mɔ́}.\\
from  \textsc{place}  descending  \textsc{3sg.sbj}  \textsc{neg}  \textsc{pot}  \textsc{be.loc}  more\\

\glt ‘As we descend from Rebola, it [the fog] won’t be there anymore.’ [ye07fn 071]
\z

In \REF{ex:key:870} below \textit{mɔ́} may be analysed as occupying the object position of \textit{tɔ́k} ‘talk, say’ with the meaning ‘more’. Alternatively, \textit{mɔ́} may be seen to function as an adverbial and be translated as ‘still, again continue to’:


\ea%870
    \label{ex:key:870}
    \gll A    nó  de  \textbf{tɔ́k}  \textbf{mɔ́}.\\
\textsc{1sg.sbj}  \textsc{neg}  \textsc{ipfv}  talk  more\\

\glt ‘I was not talking any longer.’ O\textsc{r} ‘I was not saying (anything) more/again.’ [ab03ay 090]
\z

The scope of \textit{mɔ́} may also be narrower than the clause. In \REF{ex:key:871}, \textit{mɔ́} is in the postnominal position and modifies the preceding NP in a way no different from that of the focus particle \textit{sɛ́f} ‘\textsc{foc}’ or the quantifier{\fff} \textit{ɔ́l} ’all’. In \REF{ex:key:872}, \textit{mɔ́} modifies the adverbial \textit{áfta} ‘then’: 


\ea%871
    \label{ex:key:871}
    \gll \textbf{Nó}  \textbf{pát}  \textbf{fɔ}  \textbf{wɔ́l}    \textbf{mɔ́}    nó  dé.\\
\textsc{neg}  part  \textsc{prep}  world  more  \textsc{neg}  \textsc{be.loc}\\

\glt ‘There is no other part of the world [where it’s like that].’ [au07se 224]
\z


\ea%872
    \label{ex:key:872}
    \gll \textbf{\'{A}fta    mɔ́}   a    bin  wók    dís  sén    wók 
wé  a    de  dú. \\
then  more  \textsc{1sg.sbj}  \textsc{pst}  work  this  same  work  
\textsc{sub}  \textsc{1sg.sbj}  \textsc{ipfv}  do\\

\glt ‘Then, additionally, I worked this same job that I’m doing (now).’ [ma03hm 057]\is{degree modification}
\z

Besides the adverbs treated so far, compounds \REF{ex:key:873} or constructions featuring generic nouns of place (i.e. \textit{sáy} ‘side, place’), time (i.e. \textit{tɛ́n} ‘time’ and \textit{dé} ‘day’), and manner (i.e. \textit{stáyl} ‘manner, style’) serve as locative{\fff}, time \REF{ex:key:874}, and manner adverbials \REF{ex:key:875}:


\ea%873
    \label{ex:key:873}
    \gll \textbf{Wok-sáy}    a    de  híɛ    wé  dɛn  de  tɔ́k=an    bɔkú.\\
work.\textsc{cpd}{}-side  \textsc{1sg.sbj}  \textsc{ipfv}  hear    \textsc{sub}  \textsc{3pl}  \textsc{ipfv}  talk=\textsc{3sg.obj}  much\\

\glt ‘(At) work I hear them talk it [Ghanaian Pidgin English] a lot.’ [ye07je 166]
\z


\ea%874
    \label{ex:key:874}
    \gll E    kán    \textbf{sán}  \textbf{tɛ́n}.\\
\textsc{3sg.sbj}  come  sun  time\\

\glt ‘He came (at) noon.’ [dj05ce 050]
\z


\ea%875
    \label{ex:key:875}
    \gll Dɛn  só  di  sɔ́t    \textbf{tú}  \textbf{stáyl}.\\
\textsc{3pl}  sew  \textsc{def}  shirt  two  style\\

\glt ‘The shirt was sewn in two (different) ways.’ [ra07ve 063]
\z

Other than that, Pichi employs noun phrases introduced by prepositions (e.g. \textit{na} ‘\textsc{loc}’, \textit{fɔ} ‘\textsc{prep}’, \textit{to} ‘to’) or locative nouns\is{locative nouns} (e.g. \textit{bifó} ‘before’, \textit{bɔtɔ́n} ‘under’, \textit{kɔ́na} ‘next to’, \textit{míndul} ‘middle’) to form various types of adverbial phrases which provide modification to clauses: 


\ea%876
    \label{ex:key:876}
    \gll \MakeUppercase{A}   pút  di  kí  \textbf{na} pála.\\
\textsc{1sg.sbj}  put  \textsc{def}  key  \textsc{loc}  parlour\\

\glt ‘I put the key in the parlour.’ [to07fn 114]
\z

\subsection{Modification of manner and circumstance}\label{sec:7.7.2}

The corpus contains only few underived manner adverbs (amongst them \textit{kwík} ‘quickly, early’ in \ref{ex:key:856} above). Nevertheless, the possibilities for providing manner modification are particularly rich. They encompass the use of adverbials, ideophones\is{ideophones}, SVCs, secondary predication, compounds, associative constructions, lexicalised iteration, and adverbial clauses of manner. 


The value property item \textit{fáyn} ‘be fine, nice, correct’ is frequently found in clause-final position to provide manner modification. The use of \textit{fáyn} in this way is conventionalised to such an extent that it may be considered an adverb with its own established meaning of ‘nicely, properly, in the right way’ (a similar case is made for \textit{bád} ‘extremely’, cf. \ref{ex:key:897}–\ref{ex:key:898} further below):



\ea%877
    \label{ex:key:877}
    \gll E    fíks  dɛ́n    \textbf{fáyn}.\\
\textsc{3sg.sbj}  fix  \textsc{3pl.indp}  fine\\

\glt ‘She has arranged them properly.’ [li07pe 069]
\z

Another idiosyncratic way of expressing manner modification is through the lexicalised reduplication \textit{haydháyd} ‘secretly’ \REF{ex:key:878}:


\ea%878
    \label{ex:key:878}
    \gll Chico,  yu  dɔ́n  chɔ́p=an    \textbf{haydháyd}.\\
boy    \textsc{2sg}  \textsc{prf}  eat=\textsc{3sg.obj}  secretly\\

\glt ‘Man, you’ve eaten it secretly.’ [ge07fn 333]
\z

Further, Pichi employs the adverb-deriving suffix \textit{{}-wán} \textsc{‘adv’} to form manner adverbs \REF{ex:key:879}, and the generic noun \textit{stáyl} ‘style’ \REF{ex:key:880} in order to form manner-denoting adverbial NPs in clause-final position:


\ea%879
    \label{ex:key:879}
    \gll \'{A}s  dɛn  nɔ́ba  bin  sí  plantí,  dɛn  bin  chɔ́p=an    \textbf{rɔ́n-wán}.\\
as  \textsc{3pl}  \textsc{neg.prf}  \textsc{pst}  see  plantain  \textsc{3pl}  \textsc{pst}  eat=\textsc{3sg.obj}  wrong-\textsc{adv}\\

\glt ‘Since they had never seen plantain before, they ate it the wrong way.’ [ro05ee 062]
\z


\ea%880
    \label{ex:key:880}
    \gll Dɛn    tíf    di  mɔní  \textbf{síkrit}  \textbf{stáyl}.\is{noun phrase adverbials}\\
\textsc{3pl}    steal  \textsc{def}  money  secret  style\\

\glt ‘They stole the money secretly.’ [ra07ve 048]
\z

Likewise, prepositional phrases\is{prepositional phrases} introduced by \textit{fɔ} ‘\textsc{prep}’ may express manner as in the following example: 


\ea%881
    \label{ex:key:881}
    \gll A    wáka  \textbf{fɔ}  \textbf{fút}    wet    mi    maleta.\\
\textsc{1sg.sbj}  walk  \textsc{prep}  foot    with    \textsc{1sg.poss}  suitcase\\

\glt ‘I walked by foot with my suitcase.’ [ab03ay 075]
\z

For one part, biclausal structures are common in providing modifications of circumstance. Compare the following adverbial clauses introduced by \textit{wé} ‘\textsc{sub}’ \REF{ex:key:882} and \textit{sé} ‘\textsc{quot}’ \REF{ex:key:883}: 


\ea%882
    \label{ex:key:882}
    \gll E    gó  na  wók    \textbf{wé}  e    \textbf{klín}.\\
\textsc{3sg.sbj}  go  \textsc{loc}  work  \textsc{sub}  \textsc{3sg.sbj}  be.clean\\

\glt ‘She went to work clean.’ [ra07ve 076]
\z


\ea%883
    \label{ex:key:883}
    \gll Dí  pikín  kɔmɔ́t  \textbf{sé} e\textbf{    dɔtí}.\\
this  child  go.out  \textsc{quot}    \textsc{3sg.sbj}  be.dirty\\

\glt ‘This child went out dirty.’ [ra07ve 016]
\z

Another common way of providing modification to a clause is by means of depictive secondary predication (cf. also \sectref{sec:11.3}). In the depictive adjunct in \REF{ex:key:884}, the secondary predicate \textit{nékɛd} ‘be naked’ provides information about the state of the subject \textit{e} \textsc{‘3sg.sbj’,} while the situation denoted by \textit{kɔmɔ́t} ‘come out’ unfolds:\is{secondary predicates}


\ea%884
    \label{ex:key:884}
    \gll E    kɔmɔ́t    na  rúm    \textbf{nékɛd}.\\
\textsc{3sg.sbj}  come.out  \textsc{loc}  room  be.naked\\

\glt ‘He left the room naked.’ [ra07ve 001]
\z

Modifications of circumstance may also be provided through nominal depictives that come in the guise of prepositional phrases\is{prepositional phrases} introduced by \textit{wet} ‘with’ \REF{ex:key:885} and \textit{lɛk} ‘like’ \REF{ex:key:886}:


\ea%885
    \label{ex:key:885}
    \gll E    pút  di  bɔ́tul  pan  di  tébul  \textbf{wet}    \textbf{di}  \textbf{mɔ́t}    \textbf{dɔ́n}.\\
\textsc{3sg.sbj}  put  \textsc{def}  bottle  pan  \textsc{def}  table  with    \textsc{def}  mouth  down\\

\glt ‘He put the bottle on the table upside-down.’ [li07pe 057]
\z


\ea%886
    \label{ex:key:886}
    \gll Pero    mi    mamá  kán  acepta  di  pikín  \textbf{lɛk}  \textbf{mi}
\textbf{brɔ́da}  \textbf{in}    \textbf{pikín}.\\
but    \textsc{1sg.poss}  mother  \textsc{pfv}  accept  \textsc{def}  child  like  \textsc{1sg}
brother  \textsc{3sg.poss}  child\\

\glt ‘But my mother accepted the child as my brother’s child.’ [fr03ft 122]
\z

The preposition and clause linker \textit{lɛk} may also introduce a prepositional phrase that indicates sameness of manner. Two examples of such “similatives” (\citealt{HaspelmathBuchholz1998}) follow: 


\ea%887
    \label{ex:key:887}
    \gll Mí    nó  lɛ́k  yú    bɔt  wi  fít  dé    \textbf{lɛk}  \textbf{kɔ́mpin}.\\
\textsc{1sg.indp}  \textsc{neg}  like  \textsc{2sg.indp}  but  \textsc{1pl}  can  \textsc{be.loc}  like  friend\\

\glt ‘I don’t love you but we can be (like) friends.’ [ru03wt 029]
\z

The similative collocation \textit{wók lɛk dɔ́kta} functions as a nominal depictive \REF{ex:key:888} (cf. \sectref{sec:11.3} for an extensive discussion of verbal depictives in secondary predications). A similative \textit{lɛk} in \REF{ex:key:889} translates as ‘around’: 


\ea%888
    \label{ex:key:888}
    \gll Di  cubana  húman  de  \textbf{wók}    \textbf{lɛk}  \textbf{dɔ́kta}  na  Malábo.\\
\textsc{def}  Cuban  woman  \textsc{ipfv}  work  like  doctor  \textsc{loc}  \textsc{place}\\

\glt ‘The Cuban woman works as a doctor in Malabo.’ [ro05ee 071]
\z


\ea%889
    \label{ex:key:889}
    \gll Yu  fít  gí  mí    \textbf{lɛk}  \textbf{dos}  \textbf{mil}      só?\\
\textsc{2sg}  can  give  \textsc{1sg.indp}  like  two  thousand  like.that\\

\glt ‘Can you give me around two thousand?’ [be07fn 311]
\z

Similative clauses are introduced by \textit{lɛk sé} ‘like \textsc{quot}’ = ‘as if’, as in the example below:


\ea%890
    \label{ex:key:890}
    \gll E    de  dú  \textbf{lɛk}  \textbf{sé}  e    de    fɛ́n    sɔn    tín.\\
\textsc{3sg.sbj}  \textsc{ipfv}  do  like  \textsc{quot}  \textsc{3sg.sbj}  \textsc{ipfv}    look.for  some  thing\\

\glt ‘He’s pretending to look for something.’ [dj07ae 517]\is{similatives}
\z

Some relations of modification that habitually re-occur tend to be expressed through verb-noun compounds. For example, the depictive secondary predication in \REF{ex:key:891} is more often rendered by \REF{ex:key:892}:


\ea%891
    \label{ex:key:891}
    \gll E    dríng  di  \textbf{watá}  \textbf{kól}.\\
\textsc{3sg.sbj}  drink  \textsc{def}  water  be.cold\\

\glt ‘She drank the water cold.’ [ra07ve 004]
\z


\ea%892
    \label{ex:key:892}
    \gll E    dríng  \textbf{kol-watá}.\\
\textsc{3sg.sbj}  drink  cold.\textsc{cpd}{}-water\\

\glt ‘She drank cold water.’ [ra07ve 003]
\z

\subsection{Modification of degree}\label{sec:7.7.3}

There are various ways of providing degree modification\is{degree modification} in Pichi other than by the means covered in \sectref{sec:7.7.1}. Not all of these involve the use of adverbial constituents. For example, inherently comparative and superlative expressions, cognate objects, some types of focus constructions (i.e. predicate cleft), as well as repetition all provide some form of explicit or implicit modification of degree.\is{emphasis}


Degree modification may also be realised on the suprasegmental level. Vowel-length\-en\-ing and extra-high pitch may indicate a larger amount of intensity, extent, or dimension of a referent, which is generally a property item or an adverbial. The only syllable of the property item \textit{kól} ‘be cold’ in \REF{ex:key:893} is pronounced with an extra-high tone and lengthened. The phonetic transcription is provided in squared brackets: 



\ea%893
    \label{ex:key:893}
    \gll Pero    wé  a    kin  tɔ́ch    in    fút,
in    hán    dé,    na  só    dɛn  \textbf{kó.ól}~\textup{[k\H{o}::l]}.\\
but    \textsc{sub}  \textsc{1sg.sbj}  \textsc{hab}  touch  \textsc{3sg.poss}  leg
\textsc{3sg.poss}  hand  there  \textsc{foc}  like.that  \textsc{3pl}  cold.\textsc{emp}\\

\glt ‘But when I would touch his foot (and) his hand,
they were so extremely cold.’ [ab03ab 066]
\z

Vowel lengthening and extra-high tone are conventionalised with the preposition \textit{sóté} ‘until’. Both phenomena always occur when \textit{sóté} is employed as a clause-final temporal adverb with the meaning ‘for a long time’ or a degree adverb with the meaning ‘extremely’ \REF{ex:key:894}:


\ea%894
    \label{ex:key:894}
    \gll Dɛn  kéch=an    dɛn  bít=an    \textbf{sóté.e}~\textup{[s\H{o}t\H{e}::]}.\\
\textsc{3pl}  catch=\textsc{3sg.obj}  \textsc{3pl}  beat=\textsc{3sg.obj}  until.\textsc{emp}\\

\glt ‘They caught him and beat the hell out of him.’ [pa07fn 556]
\z

Suprasegmental degree modification is performed in accordance with the syllable structure of the modified word. Monosyllabic words bear an extra-high tone over their H-toned syllable. If the syllable ends in a vowel, liquid, or nasal, it may also be lengthened. Two examples for this pattern are \textit{kól} ‘be cold’ in \REF{ex:key:893} above and \textit{fá} ‘be far’ in \REF{ex:key:895} below.


The H-toned syllable of a bisyllabic word may also be lengthened if it ends in a vowel or liquid. Compare \textit{fawe} ‘be far’ in \REF{ex:key:895} below. Both \textit{fá} and \textit{fáwe} in \REF{ex:key:895} are additionally emphasised by means of an extra-high tone: 



\ea%895
    \label{ex:key:895}
    \gll Wántɛn  a    skía,      e    sé  “nó  skía,      a    kɔmɔ́t
\textbf{fá.áwe}~\textup{[fá:we]},  a    kɔmɔ́t    \textbf{fá.á}~\textup{[fá::]}.”\\
suddenly  \textsc{1sg.sbj}  be.scared  \textsc{3sg.sbj}  \textsc{quot}  \phantom{“}\textsc{neg}  be.scared  \textsc{1sg.sbj}  come.out
far.\textsc{emp}      \textsc{1sg.sbj}  come.out  far.\textsc{emp}\\

\glt ‘Suddenly, I became scared, he said “don’t be scared, I come from very far away, 
I come from very far”.’ [ed03sb 176]\is{degree modification}
\z

In contrast, vowel-lengthening for degree modification is not attested with mono- or bisyllabic words with word-final H-toned syllables that end in plosives or fricatives. With this group of words, we only find emphatic extra-high tone or other types of degree modification. For example, in \REF{ex:key:896}, the property item \textit{bíg} ‘be big’ is modified for degree by repetition and the H-tone over both iterations is raised: 


\ea%896
    \label{ex:key:896}
    \gll Dɛn  gɛ́t  wán  \textbf{bíg}  \textbf{bíg} \textup{[b\H{i}g b\H{i}g]} fám.\\
\textsc{3pl}  get  one  big  \textsc{rep} {}     farm\\

\glt ‘They have a huge farm.’ [fr03ft 012]
\z

Property items that do not denote dimension or a physical property and adverbs that do not denote a manner or degree are not usually modified suprasegmentally in this way. One way of providing degree modification to other types of words is by means of the value property item \textit{bád} in clause-final position. For example, in \REF{ex:key:897} the property item \textit{bád} ‘bad’ is employed as a degree adverb with the meaning ‘extremely’. 


\ea%897
    \label{ex:key:897}
    \gll A    de  sɔ́ri      \textbf{bád.}\\
\textsc{1sg.sbj}  \textsc{ipfv}  feel.sorry  extremely\\

\glt ‘I really feel sorry.’ [hi03cb 069]
\z

In \REF{ex:key:898}, \textit{bád} modifies \textit{fáyn} ‘(be) fine’. The example shows that \textit{bád} retains nothing of its lexical meaning of ‘be bad’ when employed in this function. It is a true degree adverbial and may also modify a verb which is the antonym of its lexical source:


\ea%898
    \label{ex:key:898}
    \gll E    fáyn  \textbf{bád,}      e    fáyn  \textbf{bád.}\\
\textsc{3sg.sbj}  be.fine  extremely  \textsc{3sg.sbj}  be.fine  extremely\\

\glt ‘She is really beautiful, she is really beautiful.’ [fr03ft 113]
\z

The sentence-final particle \textit{ó} may also provide degree modification to a sentence \REF{ex:key:899}. The various functions of this particle are covered in detail in \sectref{sec:12.2.4}:


\ea%899
    \label{ex:key:899}
    \gll E    hád    \textbf{ó}.\\
\textsc{3sg.sbj}  be.hard  \textsc{sp}\\

\glt ‘It’s really difficult.’ [ro05fe 037]\is{adverbs}
\z

