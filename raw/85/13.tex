\chapter{Pichi and Spanish in contact}
\il{Spanish}
The integration of Spanish elements into Pichi discourse is thoroughly conventionalised, and encompasses borrowing, calquing and codeswitching. Many of the mixing phenomena that can be observed are not “interactionally meaningful” \citep[20]{Auer1998} and point towards codeswitching as an “unmarked choice” (\citealt{Scotton1993}), i.e. the normal way of speaking Pichi (cf. \citealt{Yakpo2015,Yakpo2017}). I summarily refer to the Pichi-Spanish contact phenomena described in this chapter by the cover term “codemixing”\is{codemixing} \citep{Muysken2000}. This implies patterned and “sedimented” \citep{Auer1999} uses of non-native elements in multilingual interactions. Codemixing therefore forms an integral part of the grammar and pragmatics of Pichi (cf. \citealt{Yakpo2009complexity}, \citealt{Yakpo2018}). In this chapter, all \ili{Spanish} elements are set in bold. 

\section{Patterns of contact}\label{sec:13.1}

Codemixing systematically affects different areas of Pichi grammar and lexicon and it does so with differing frequency and depth. The use of certain lexical items and structures involving Pichi and Spanish material is so conventionalised that they can be said to constitute an integral part of the grammatical system and lexicon of Pichi. \tabref{tab:key:13.1} summarises some of the most conventionalised patterns of Pichi-Spanish codemixing.

%%please move \begin{table} just above \begin{tabular
\begin{table}
\caption{Patterns of Pichi-Spanish contact}
\label{tab:key:13.1}

\begin{tabularx}{\textwidth}{p{22mm}Q}
\lsptoprule

Elements & Description\\
\midrule
Noun phrases & \textsc{SG} and \textsc{PL} Spanish \textsc{NPs} occur with the Pichi definite article \textit{di} and the pluraliser\is{pluraliser} \textit{dɛn.}\is{article}\\

\tablevspace
Verbs & Spanish verbs occur in a \textsc{3sg} present tense invariant form and may only take the suppletive object pronoun \textstyleTablePichiZchn{ín} ‘\textsc{3sg.indp}’.\\

\tablevspace
Adjectives & All Spanish adjectives and past participles occur as complements to the locative-existential copula \textit{dé} ‘\textsc{be.loc’}.\\

\tablevspace
Numerals \&\newline time units & Spanish numerals occur with rising likelihood the higher the number, no Pichi numeral above seven is attested in the corpus, Spanish day names and other time units have been borrowed\is{borrowing}.\\

\tablevspace
Colours & Less basic colours like ‘green’, ‘blue’, or ‘brown’ occur almost exclusively in Spanish.\\

\tablevspace
Adverbials & Spanish adverbs and discourse elements are frequent at the clausal margins.\\

\tablevspace
Other & There are numerous individual structural and lexical borrowings and calques from Spanish.\\
\lspbottomrule
\end{tabularx}
\end{table}

\citet{Muysken2000} identifies three patterns of codemixing that accommodate cross-lin\-guis\-tic mixing phenomena: insertion, alternation, and congruent lexicalisation. All three of these patterns are operative in Pichi-Spanish codemixing. But the type of back-and-forth switching characteristic of much of Pichi discourse points towards a prominent role of congruent lexicalisation: Material from either language is grafted on grammatical structures common to both languages. Consider the following example: 


\ea%1695
    \label{ex:key:1695}
    \gll A    kɔmɔ́t  \textbf{\textit{colegio}}\textbf{\textmd{,}}    a    dé    \textbf{\textit{fuera}}  \textbf{\textit{con}}    \textbf{\textit{mi}}
mísis  \textbf{cuatro}  \textbf{años}    a    nó  ték  bɛlɛ́,
a    nó  lɛ́f    mi    vájin.\\
\textsc{1sg.sbj}  leave  high.school  \textsc{1sg.sbj}  \textsc{be.loc}  outside  with    \textsc{1sg.poss}
matron  four    year.\textsc{pl}  \textsc{1sg.sbj}  \textsc{neg}  take  belly
\textsc{1sg.sbj}  \textsc{neg}  leave  \textsc{1sg.poss}  virginity\\
\glt ‘I came out of high school, I was outside with my guardian for four years, 
I didn’t become pregnant, I didn’t give up my virginity.’ [ab03ay 132]
\z

While the noun \textit{colegio} ‘college’ looks more like an insertion into a Pichi grammatical structure (the noun is left unmarked like a Pichi noun in this position), the switch \textit{fuera con mi} ‘outside with my’ is best understood as an instance of congruent lexicalisation. Each element could be replaced by the corresponding Pichi elements \textit{nadó wet mi}. In this context the possessive pronoun \textit{mi} ‘\textsc{1sg.poss}’ is of particular interest. It is a homophonous diamorph, a morpheme that is identical in form and function in both languages including its suprasegmental feature of low tonedness in Pichi and Equatoguinean Spanish. Besides that, \textit{mi} functions as a possessive pronoun through juxtaposition with the possessed noun in both languages. 


I subjected a smaller section of the corpus consisting of a total of 22,059 words (or tokens, i.e. occurrences of words, irrespective how many times they occur) to a thorough analysis. The subcorpus contains 1475 types (different words). The analysis reveals that the presence of Spanish types and tokens in the Pichi texts varies with word classes in the ways listed in \tabref{tab:key:13.2}.


%%please move \begin{table} just above \begin{tabular
\begin{table}
\caption{Type-token analysis of Spanish words in Pichi discourse}
\label{tab:key:13.2}

\begin{tabularx}{\textwidth}{l rrr rrr}
\lsptoprule
 & \multicolumn{3}{c}{ Types} & \multicolumn{3}{c}{ Tokens}\\
Word class & Pichi & Spanish & Spanish \% & Pichi & Spanish & Spanish \%\\
\midrule
Nouns & 345 & 346 & 50\% & 2748 & 664 & 19\%\\
Verbs & 246 & 94 & 28\% & 3771 & 192 & 5\%\\
Property items & 62 & 48 & 44\% & 450 & 99 & 18\%\\
Numerals & 17 & 28 & 62\% & 166 & 146 & 47\%\\
Prepositions & 16 & 9 & 36\% & 1107 & 54 & 5\%\\
Clause linkers & 6 & 8 & 57\% & 663 & 95 & 14\%\\
\lspbottomrule
\end{tabularx}
\end{table}

With respect to types, \tabref{tab:key:13.2} shows that a total of 50\% of all nouns and approximately 28\% of all verbs that occur are Spanish. Property items (or “adjectives” in Spanish) were counted separately and amounted to a total of 44\% of Spanish types. For numerals, the Spanish percentage stands even higher at 62\%. 


However, the percentage of Spanish tokens (i.e. total instances of occurrences even if the same word occurs several times) reveals a different picture. Numerals still top the list (47\%). But they are followed by a much lower percentage of Spanish nouns (19\%) and adjectives (18\%). This shows that the frequency with which Spanish words are used is considerably lower than the absolute number of Spanish words in Pichi discourse. With the exception of numerals, the Spanish ratio of tokens stands at roughly 20\% of an average text. 


\section{Specific constituents}\label{sec:13.2}

The following four sections describe the specifics of codemixing involving noun phrases, verbs and adjectives, functional elements, and other constituents.

\subsection{Noun phrases}\label{sec:13.2.1}

Inserted Spanish constituents belong to various word classes, but the insertion of content words, and nouns in particular, prevails. Thus we find \textit{novio} ‘fiancé’ and \textit{pueblo} ‘village’\textit{} in \REF{ex:key:1696}. Note that both Spanish nouns are objects\is{objects} of Pichi elements, the first of a verb, the second of a preposition: 


\ea%1696
    \label{ex:key:1696}
    \gll Mék    yu  nó    sé    yu  dɔ́n  gɛ́t  \textbf{novio}    na  \textbf{pueblo},
na  kɔ́ntri.\\
\textsc{sbjv}    \textsc{2sg}  know  \textsc{quot}    \textsc{2sg}  \textsc{prf}  get  boyfriend  \textsc{loc}  village
\textsc{loc}  country\\
\glt ‘You should know that you already have a fiancé in the village, 
in the hometown. [ab03ay 010]
\z

When Spanish nouns are inserted as in \REF{ex:key:1696}, they usually remain bare where Pichi nouns do so, or are accompanied by Pichi determiners and the pluraliser\is{pluraliser} \textit{dɛn} ‘\textsc{pl’} in the same way as Pichi nouns are. In \REF{ex:key:1697}, the definite Spanish noun \textit{paciencia} ‘patience’ is preceded by the Pichi definite article \textit{di}: \is{article}


\ea%1697
    \label{ex:key:1697}
    \gll \textbf{Porque}  fɔ́s,    di  \textbf{paciencia},  yu  nó  go  gɛ́t=an.\\
because  first    \textsc{def}  patience    \textsc{2sg}  \textsc{neg}  \textsc{pot}  get=\textsc{3sg.obj}\\

\glt ‘Because first, the patience, you wouldn’t have it.’ [fr03ft 189]
\z

When a specific Spanish plural noun is inserted, there is a strong likelihood that it will be additionally marked with the postposed Pichi pluraliser \textit{dɛn}, in accordance with the pattern that applies to Pichi count nouns \REF{ex:key:1698}. Conversely, Spanish nouns exhibit a strong tendency to occur devoid of Pichi number and definiteness\is{definiteness} marking where the noun is non-specific as with \textit{rallador} ‘grater’ in the second example:


\ea%1698
    \label{ex:key:1698}
    \gll \'{A}fta    una  báy  di  \textbf{\textit{bloques}}  \textbf{dɛn}  tumára.\\
then  \textsc{2pl}  buy  \textsc{def}  bricks  \textsc{pl}  tomorrow\\

\glt ‘Then you [plural] buy the bricks tomorrow.’ [fr03cd 112]
\z


\ea%1699
    \label{ex:key:1699}
    \gll A    \textbf{ralla}    ín    wet    \textbf{\textit{rallador}}.\\
\textsc{1sg.sbj}  grate  \textsc{3sg.indp}  with    grater\\
\glt ‘I grated it with a grater.’ [dj03do 004]
\z

The occurrence of \textit{pruebas} ‘proofs’ in \REF{ex:key:1700} demonstrates that Spanish nouns may well be devoid of Pichi noun phrase marking, but not necessarily so of the Spanish plural morpheme \{\textit{{}-s}\}: 


\ea%1700
    \label{ex:key:1700}
    \gll Yu  go  gɛ́t  \textbf{prueba-s}.\\
\textsc{2sg}  \textsc{pot}  get  proof-\textsc{pl}\\

\glt ‘You will have proof.’ [ma03sh 013]
\z

This is not surprising however, since in Spanish, determiner-less plural count nouns may have non-specific reference. The semantic overlap between Spanish plural nouns and Pichi bare nouns in codemixing can be seen in \REF{ex:key:1701}. Here the Pichi bare nouns \textit{pía} ‘avocado’ and \textit{sadín} ‘sardine’ are functionally equivalent to the Spanish plural noun \textit{tomates} ‘tomatoes’:


\ea%1701
    \label{ex:key:1701}
    \gll Mí    wet    Rubi    wi  mék    jwɛn-jwɛ́n,  wi  báy  pía,
wi  báy  sadín,  wi  báy  \textbf{tomates},    wi  \textbf{desayuna}.\\
\textsc{1sg.indp}  with    \textsc{name}  \textsc{1pl}  make  \textsc{red}.\textsc{cpd}{}-join  \textsc{1pl}  buy  avocado
\textsc{1pl}  buy  sardine  \textsc{1pl}  buy  tomatoes  \textsc{1pl}  have.breakfast\\

\glt ‘Me and Rubi, we teamed up and bought avocados, we bought sardines, 
we bought tomatoes, we had breakfast.’ [ye03cd 152]
\z

The insertion of larger nominal groups as opposed to single nouns is rarer. In fact, most of the Spanish adjective-noun combinations we encounter are collocations that are somewhat lexicalised in Spanish. Compare \textit{traducción directa} ‘direct translation’ in \REF{ex:key:1702}:


\ea%1702
    \label{ex:key:1702}
    \gll Na  \textbf{\textit{traducción}}  \textbf{\textit{directa}}  e    mék.\\
\textsc{foc}  translation  direct  \textsc{3sg.sbj}  make\\

\glt ‘It’s a direct translation that she made.’ [to03gm 042]
\z

The order of constituents normally remains unchanged when Spanish elements are inserted into a Pichi NP. In \REF{ex:key:1703}, the Pichi quantifier\is{quantifiers} \textit{lás} ‘last’ is used in prenominal position with the inserted Spanish noun \textit{semana} ‘week’. However, note that Spanish also features a quantifier + noun order in NPs (i.e. \textit{la} \textit{última} \textit{semana} ‘(the) last week’):


\ea%1703
    \label{ex:key:1703}
    \gll Ɛf  yu  bin  kán  bigín  \textbf{\textit{las}}  \textbf{\textit{semana}}  yu  bin  fɔ    dɔ́n  fínis    tidé.\\
if  \textsc{2sg}  \textsc{pst}  \textsc{pfv}  begin  last  week  \textsc{2sg}  \textsc{pst}  \textsc{cond}    \textsc{prf}  finish  today\\

\glt ‘If you had begun last week you would have been finished today.’ [dj05ae 057] 
\z

I would assume that the inverse \textsc{NP} constituent order (noun + adjective in the majority of cases) of Spanish NPs blocks the admixture of single Spanish attributive adjectives into Pichi \textsc{NPs} (cf. \citealt{SankoffPoplack1981}). This is largely borne out by the data.


There is, however, some variation, although it is not all that frequent. In \REF{ex:key:1704}, the Spanish adjective \textit{directo} ‘direct’ occurs after the Pichi noun \textit{ɔnkúl} ‘uncle’\textit{} in a Pichi NP and thereby follows the constituent order of a Spanish \textsc{NP}:



\ea%1704
    \label{ex:key:1704}
    \gll Na  wán    ɔnkúl   \textbf{directo},  fɔ  mi    mamá  in    papá
in    fámbul  pát.\\
\textsc{foc}  one    uncle  direct  \textsc{prep}  \textsc{1sg.poss}  mother  \textsc{3sg.poss}  father
\textsc{3sg.poss}  family  part\\

\glt ‘He’s a direct uncle on the part of my mother’s father’s family.’ [fr03ft 051]
\z

In \REF{ex:key:1705}, we find the opposite situation. The Spanish adjective \textit{especial} ‘special’ is in a prenominal position, hence in the syntactic slot of attributively used Pichi property items: 


\ea%1705
    \label{ex:key:1705}
    \gll E    bríng  fís,  e    kúk    sɔn    \textbf{\textit{especial}}  fís,
e    gí  mí    mék    a    chɔ́p.\\
\textsc{3sg.sbj}  bring  fish  \textsc{3sg.sbj}  cook  some  special  fish
\textsc{3sg.sbj}  give  \textsc{1sg.indp}  \textsc{sbjv}    \textsc{1sg.sbj}  eat\\

\glt ‘She brought (a) fish, she cooked a particular fish and gave it to me 
in order to eat.’ [ed03sb 015]
\z

There are other instances of Spanish adjectives that follow Pichi nouns in Pichi NPs. But in these cases, the function of the Spanish words parallels that of some Pichi value property items that are used as adverbials in the same syntactic position. The Spanish adjective \textit{serio} ‘serious’ in \REF{ex:key:1706} may be likened to the Pichi manner adverb \textit{fáyn} ‘well, really’ in \REF{ex:key:1707}:


\ea%1706
    \label{ex:key:1706}
    \gll Dí  wán    go  tɔ́n    plába  \textbf{\textit{serio}}.\\
this  one    \textsc{pot}  turn    trouble  serious\\

\glt ‘This will turn into real trouble.’ [fr03wt 015]
\z


\ea%1707
    \label{ex:key:1707}
    \gll ‘Dí    mán    dé    trɔ́n’  nó  dé    fáyn,  e    nó  gɛ́t
\textbf{\textit{sentido}}    fáyn. \\
 \phantom{‘}this  man    \textsc{be.loc}  strong  \textsc{neg}  \textsc{be.loc}  fine    \textsc{3sg.sbj}  \textsc{neg}  get
meaning    fine\\

\glt ‘“Dí man dé trɔn” is not nice, it doesn’t have a proper meaning.’ [dj05ae 124]
\z

\subsection{Verbs and adjectives}\label{sec:13.2.2}

The low ratio of Spanish verbs as opposed to nouns in the type and token count may be striking at first glance. However, this tendency may stem from the fact that a small number of high frequency Pichi verbs (e.g. \textit{mék} ‘make’, \textit{gɛ́t} ‘get, have’, \textit{gí} ‘give’) participate in conventionalised verb-noun collocations, {\fff} in which a Pichi verb is followed by a Spanish noun (cf. \sectref{sec:9.3.1} for an extensive treatment). Some of these are \textit{gí permiso} ‘give permission’, \textit{mék rabia} ‘be annoyed’, \textit{gɛ́t novio/novia} ‘have a boy/girlfriend’. The collocations also include calques from Spanish. Compare \textit{gí wán vuelta} ‘give one round’ = ‘take a walk’ which is a one-to-one calque of Spanish \textit{dar una vuelta}:


\ea%1708
    \label{ex:key:1708}
    \gll E    de  gí  wán    \textbf{\textit{vuelta}}  kwík.\\
\textsc{3sg.sbj}  \textsc{ipfv}  give  one    round  quickly\\

\glt ‘She’s taking a walk quickly.’ [dj05be 120]
\z

The admixture of Spanish verbs follows established rules. Spanish verbs are always inserted into Pichi clauses in an invariant form of the \textsc{3sg} person of the Spanish present tense\is{present tense} paradigm. This insertion rule is valid without exception across the three regular Spanish verb inflection classes. Due to its frequency, the \textsc{3sg} present tense form is also the default form found in most contact scenarios involving Spanish (\citealt[20–21]{Clements2009}). Examples follow with \textit{controla} ‘control’ (<\textit{controlar}) in \REF{ex:key:1709}, \textit{entiende} ‘understand’ (<\textit{entender}) in \REF{ex:key:1710}, and \textit{sufre} ‘suffer’ (<\textit{sufrir}) in \REF{ex:key:1711}:


\ea%1709
    \label{ex:key:1709}
    \gll Frɔn    na  yá    só    dɛn  kin  \textbf{\textit{controla}}  di  húman.\\
from  \textsc{loc}  here    like.that  \textsc{3pl}  \textsc{hab}  control  \textsc{def}  woman\\

\glt ‘From here they control the woman.’ [ed03sb 158]
\z


\ea%1710
    \label{ex:key:1710}
    \gll Pɔ́sin  go   \textbf{\textit{entiende}}    bɔt  e    nó  dé    \textbf{bien}.\\
person  \textsc{pot}  understand  but  \textsc{3sg.sbj}  \textsc{neg}  \textsc{be.loc}  good\\

\glt ‘One would understand but it isn’t good.’ [dj05ae 043]
\z


\ea%1711
    \label{ex:key:1711}
    \gll E    \textbf{sufre}  wé  náw  dɛn  dɔ́n  lɛ́f=an,    e    dɔ́n  klós.\\
\textsc{3sg.sbj}  suffer  \textsc{sub}  now  \textsc{3pl}  \textsc{prf}  leave=\textsc{3sg.obj}  \textsc{3sg.sbj}  \textsc{prf}  close\\

\glt ‘It [the building] suffered, while now they have abandoned it, it is closed.’ [hi03cb 044]
\z

The \textsc{3sg} invariant form is combined with Pichi TMA markers like any Pichi verb as can be seen by the presence of \textit{kin} ‘\textsc{hab}’ in \REF{ex:key:1709} and \textit{go} ‘\textsc{pot}’ in \REF{ex:key:1710} above. Inserted Spanish verbs may also be reduplicated by the same derivational process that applies to Pichi verbs. Compare \textit{pica-píca} ‘\textsc{red}.\textsc{cpd}{}-cut.up’ = ‘repeatedly cut up (into small pieces)’ in \REF{ex:key:1712}: 


\ea%1712
    \label{ex:key:1712}
    \gll \MakeUppercase{A}   bigín  de  \textbf{\textit{pica}}\textit{{}-}\textbf{\textit{píca}},    wi  fráy  \textbf{patata},  wi  fráy  plantí.\\
\textsc{1sg.sbj}  begin  \textsc{ipfv}  \textsc{red.cpd-}cut.up  \textsc{1pl}  fry  potato  \textsc{1pl}  fry  plantain\\

\glt ‘I began to (repeatedly) snip [the trimmings], we fried potatoes, we fried
plantain.’ [ye03cd 172]
\z

Pichi exhibits a phonologically conditioned suppletive allomorphy in the pronominal system. The lexical pitch configuration of a verb determines the choice of allomorph used for the expression of \textsc{3sg} pronominal object case (cf. \sectref{sec:3.2.5}). Vowel-final verbs with a word-final low tone take the object pronoun \textit{ín} ‘\textsc{3sg.indp}’ – this group includes a few Pichi verbs and all inserted Spanish verbs \REF{ex:key:1713}. This is because the \textsc{3sg} invariant form of the Equatoguinean Spanish verb always features a word-final L-toned vowel:


\ea%1713
    \label{ex:key:1713}
    \gll Fíba    nó  \textbf{\textit{sube}}    ín.\\
fever  \textsc{neg}  go.up  \textsc{3sg.indp}\\

\glt ‘The fever hasn’t risen on him.’ [eb07fn 171]
\z

The form \textit{sigue} (<\textit{seguir}) ‘follow, continue’ \is{loan words}is highly conventionalised in its use\is{auxiliaries}. It is also employed as an auxiliary verb to indicate continuative\is{continuative aspect} aspect in a complement construction: \is{aspect}


\ea%1714
    \label{ex:key:1714}
    \gll \MakeUppercase{A}   go  \textbf{\textit{sigue}}    chɔ́p.\\
\textsc{1sg.sbj}  \textsc{pot}  continue    eat\\

\glt ‘I’ll continue eating.’ [be05 057]
\z

In a similar vein, the verbs \textit{sube} (<\textit{subir}) ‘go up’ and \textit{baja} (<\textit{bajar}) ‘go down’ are far more frequent than their Pichi counterparts\textit{ gó ɔ́p} and \textit{gó dɔ́n} \REF{ex:key:1715}: \is{loan words}


\ea%1715
    \label{ex:key:1715}
    \gll Bɔkú  motó  dɛn  dé    yá    só,    a    nó  nó    sé
Pancho  mék    lɛk  sé    e    de  \textbf{sube}    bihɛ́n
wé   e    \textbf{\textit{baja}}    mɔ́.\\
much  car    \textsc{pl}  \textsc{be.loc}  here    like.that  \textsc{1sg.sbj}  \textsc{neg}  know  \textsc{quot}
\textsc{name}  make  like  \textsc{quot}    \textsc{3sg.sbj}  \textsc{ipfv}  go.up  behind
\textsc{sub}  \textsc{3sg.sbj}  go.down  more\\

\glt ‘So many cars were there, I didn’t know Pancho pretended to 
go up behind us (and) went down again.’ [ye03cd 178]
\z

Spanish adjectives do not only occur as attributes to Pichi nouns. They are systematically inserted into Pichi predicate adjective clauses as complements to the locative-existential copula\is{copula!locative-existential} \textit{dé} \textsc{‘be.loc’} \REF{ex:key:1716}. 


\ea%1716
    \label{ex:key:1716}
    \gll Wán    yáy  \textbf{dé}    \textbf{\textit{blanco}}  e    nó  de  sí.\\
one    eye  \textsc{be.loc}  white  \textsc{3sg.sbj}  \textsc{neg}  \textsc{ipfv}  see\\
\glt ‘One eye is white, it doesn’t see.’ [ye03cd 106]
\z

Neither adjectives nor past participles usually exhibit Spanish-style gender agreement with the subject and are normally inserted in the masculine form. However, past participles always come along with the regular Spanish adjective-deriving morphology \REF{ex:key:1717}.


\ea%1717
    \label{ex:key:1717}
    \gll \MakeUppercase{A}   wánt  dé    \textbf{\textit{flipa}}\textit{{}-}\textbf{\textit{do}}      ɔ́l  áwa,    ɔ́l  áwa.\\
\textsc{1sg.sbj}  want  \textsc{be.loc}  turned.on-\textsc{adj}    all  hour  all  hour\\

\glt ‘I want to be turned on all the time, all the time.’ [ye07ga 012]
\z

I have shown that a handful of Pichi property items may be employed as adjectives and inchoative-stative verbs alike (cf. \sectref{sec:7.6.5}). When used as adjectives, these property items denote a non-time-stable body state\is{body states} and may appear as complements to the copula \textit{dé.} When used as inchoative-stative verbs, these property items denote a time-stable value. The property item \textit{bád} ‘be bad’ displays this kind of behaviour. Hence, \textit{bád} means ‘(intrinsically) bad’ \REF{ex:key:1718} when used as an inchoative-stative verb and ‘ill’ when it appears as a complement to the copula \textit{dé} \REF{ex:key:1719}: 


\ea%1718
    \label{ex:key:1718}
    \gll Sɔn    mamá  dɛn,    dɛn  \textbf{bád.}\\
some  mother  \textsc{pl}    \textsc{3pl}  be.bad\\

\glt ‘Some mothers, they are bad.’ [ab03ay 109]
\z


\ea%1719
    \label{ex:key:1719}
    \gll "E    \textbf{dé}    \textbf{bád}"    min    sé    "e    de  sík".\\
\phantom{‘}\textsc{3sg.sbj}  \textsc{be.loc}  bad    mean  \textsc{quot}    \textsc{3sg.sbj}  \textsc{ipfv}  be.sick\\

\glt ‘\textit{“E dé bad”} means “he’s sick”.’ [ye07je 046]
\z

Spanish also exhibits a distinction based on time-stability with respect to property items. In contrast to Pichi, the distinction may, however, be applied to almost any adjective of the language. Examples \REF{ex:key:1720} and \REF{ex:key:1721} involve the \textsc{3sg} present form of the time-stable identity copula \textit{ser} and the \textsc{2sg} present of the non-time-stable locative-existential copula \textit{estar,} respectively. A comparison of the Pichi examples in \REF{ex:key:1718} and \REF{ex:key:1719} above with the two sentences below show the functional overlap of the relevant constructions in the two languages: 


\ea%1720
    \label{ex:key:1720}
\textsc{Spanish}\il{Spanish}\\
    \gll \textbf{Es}    malo.                            \\
He.is  bad\\

\glt ‘He is bad.’ (Own knowledge)
\z


\ea%1721
    \label{ex:key:1721}
\textsc{Spanish}\il{Spanish}\\
    \gll ¿\textbf{Estás}  mal  hoy?                          \\
You.are  bad  today\\

\glt ‘Do you feel bad today?’ (Own knowledge)
\z

Despite the similarities between the \textit{dé} + property item construction and the Spanish \textit{estar} + adjective construction, all predicatively used Spanish adjectives always appear as complements to the Pichi locative-existential copula \textit{dé}, regardless of whether the denoted property is non-time-stable or time-stable. 


Hence the time-stable property denoted by the Spanish adjective \textit{blanco} ‘white’ appears as a complement to the copula \textit{dé} in \REF{ex:key:1716} above, while the Pichi colour term \textit{wáyt} ‘be white’ can only be employed as a inchoative-stative verb as in \REF{ex:key:1722}:



\ea%1722
    \label{ex:key:1722}
    \gll Di  mán    \textbf{wáyt}.\\
\textsc{def}  man    be.white\\

\glt ‘The man is white.’ [ed05fn 077]
\z

Why is the time-stability distinction not maintained with predicatively used Spanish adjectives? An explanation is that the Pichi construction involving the copula \textit{dé} and an adjectival complement is more compatible with congruent lexicalisation than the use of Spanish adjectives as (inchoative-)stative verbs. With the former pattern, the phrasal syntax of adjectival predication remains identical in both languages. This allows speakers to graft such codemixed constructions onto a common grammatical structure (cf. \citealt{MeechanPoplack1995}). Pichi-Spanish contact in the predicate adjective construction has therefore led to the generalisation of a rather marginal structure specialised to a handful of Pichi property items. The obligatory use of a copula in these mixed collocations may also be seen as a case of structural interference from Spanish where a copula verb \textit{must} be used in predicate adjective constructions.\is{adjectives} 

\subsection{Functional elements}\label{sec:13.2.3}

The most frequently used Spanish functional elements are the cause clause linkers \textit{como} ‘since’ \REF{ex:key:1723} and \textit{porque} ‘because’ \REF{ex:key:1724}. Both linkers form an integral part of the Pichi system of clause linkage and are best seen to have been borrowed into the language:\is{loan words}


\ea%1723
    \label{ex:key:1723}
    \gll \textbf{Como}  wí    de  kɔ́l=an    \textbf{mono}  na  Panyá,  ín    chɛ́k=an
sé    ɛf  e    tɔ́k  sé    wán  mɔnkí,  e    go  dé    fáyn.\\
since  \textsc{1pl.indp}  \textsc{ipfv}  call=\textsc{3sg.obj}  monkey  \textsc{loc}  Spanish  \textsc{3sg.indp}  think=\textsc{3sg.obj}
\textsc{quot}    if  \textsc{3sg.sbj}  talk  \textsc{quot}    one  monkey  \textsc{3sg.sbj}  \textsc{pot}  \textsc{be.loc}  fine\\

\glt ‘Since we \textsc{[emp]} call it “mono” in Spanish, he \textsc{[emp]} understood it such 
that if he said “one monkey”, it would be all right.’ [to03gm 005]
\z


\ea%1724
    \label{ex:key:1724}
    \gll Yu  nɛ́a    gɛ́t  pikín  \textbf{porque}  yu  nɛ́a    máred.\\
\textsc{2sg}  \textsc{neg}.\textsc{prf}  get  child  because  \textsc{2sg}  \textsc{neg}.\textsc{prf}  marry\\

\glt ‘You don’t yet have a child, because you aren’t yet married.’ [ab03ab 204] 
\z

The linkers \textit{como} and \textit{porque} are employed in the same syntactic position as the Pichi equivalents \textit{as} ‘as’ \REF{ex:key:1725} and \textit{bikɔs} ‘because’ \REF{ex:key:1726}, respectively: 


\ea%1725
    \label{ex:key:1725}
    \gll '\textit{As}  in    sísta    dɛn  bin  de  kɔ́l  in    mamá  sé
sísta,  in    de  kɔ́l  in    mamá  sé    sísta.\\
as  \textsc{3sg.poss}  sister  \textsc{3pl}  \textsc{pst}  \textsc{ipfv}  call  \textsc{3sg.poss}  mother  \textsc{quot}
sister  \textsc{3sg.indp}  \textsc{ipfv}  call  \textsc{3sg.poss}  mother  \textsc{quot}    sister\\

\glt ‘As her sisters would call her mother sister, she \textsc{[emp]} would 
call her mother sister.’ [ab03ay 145]
\z


\ea%1726
    \label{ex:key:1726}
    \gll Bɛt  a    dɔ́n  nó  wétin  yu  níd,  \textbf{bikɔs}  wi  gɛ́t  sɔn
prɔ́blɛm  wé  wi  de  tɔ́k  Pichi  na  Malábo.\\
but  \textsc{1sg.sbj}  \textsc{prf}  \textsc{neg}  what  \textsc{2sg}  need  because  \textsc{1pl}  get  some
problem  \textsc{sub}  \textsc{1pl}  \textsc{ipfv}  talk  Pichi  \textsc{loc}  \textsc{place}\\

\glt ‘But I already know what you need, because we have a problem when 
we talk Pichi in Malabo.’ [au07se 005]
\z

\tabref{tab:key:13.3} shows the frequency with which the Spanish linkers \textit{como} and \textit{porque} occur in Pichi sentences in relation to \textit{as} and \textit{bikɔs}. The table indicates that in the overwhelming majority of cases (89\% for \textit{como} and 91\% for \textit{porque}) both conjunctions occur as single constituents in Pichi clauses rather than in clausal switches in which the following material is also in Spanish. The second line of \tabref{tab:key:13.3} shows that these two Spanish function words are established loans. In 76\% of all occurrences, ‘since’ is expressed as \textit{como}, hence only 24\% is expressed with the Pichi equivalent \textit{as}. In 41\% of all cases ‘because’ is expressed as \textit{porque}, so Pichi \textit{bikɔs} occurs as the causal conjunction in 59\% of all cases.

%%please move \begin{table} just above \begin{tabular
\begin{table}
\caption{Distribution and frequency of \textit{como} and \textit{porque}}
\label{tab:key:13.3}

\begin{tabularx}{\textwidth}{lXX}
\lsptoprule

Type of percentage & \itshape como & \itshape porque\\
\midrule
Single constituent switch over total & 89\% & 91\%\\
Spanish conjunction over total & 76\% & 41\%\\
\lspbottomrule
\end{tabularx}
\end{table}
The clause linker\textit{ aunque} ‘although’ occurs so frequently that it is best seen to be fully integrated into the Pichi lexicon as well. In Spanish too, \textit{aunque} is used both as a concessive\is{concessive clauses} or adversative conjunction as in \REF{ex:key:1727} and as a a similative\is{similatives} adverbial as in \REF{ex:key:1728}: 

\ea%1727
    \label{ex:key:1727}
    \gll \textbf{Aunque}     nóto \textbf{paludismo} if  dɛn  gív  yú \textbf{tratamiento} yu  nó  go  dáy.\\
although    \textsc{neg}.\textsc{foc}  malaria    if  \textsc{3pl}  give  \textsc{2sg.indp}  treatment
\textsc{2sg}  \textsc{neg}  \textsc{pot}  die\\

\glt ‘Even if it isn’t malaria, if you are given treatment, you won’t die.’ [fr03ft 061]
\z


\ea%1728
    \label{ex:key:1728}
    \gll Wé  yu  de  mék=an    na  hós,    jɔ́s  ték=an,
pút=an    na  pɔ́t  \textbf{\textit{aunque}}  wán    \textbf{taza}  só.\\
\textsc{sub}  \textsc{2sg}  \textsc{ipfv}  make=\textsc{3sg.obj}  \textsc{loc}  house  just  take=\textsc{3sg.obj}
put=\textsc{3sg.obj}  \textsc{loc}  pot  like    one    cup  like.that\\

\glt ‘When you make it at home, just take it (and) put it into a pot, 
like one cup or so.’ [dj03do 010]\is{borrowing}
\z

The Spanish time clause linker \textit{mientras} ‘while’ occurs less systematically, but it still provides an optional resource for combining clauses: 


\ea%1729
    \label{ex:key:1729}
    \gll \textbf{\textit{Mientras}}    yu  de  sí  sé  di  tín    de  \textbf{transforma}  pɔ́sin
yu  de  kɔ́t  wán    tín    fɔ  in    fínga.\\
while    \textsc{2sg}  \textsc{ipfv}  see  \textsc{quot}  \textsc{def}  thing  \textsc{ipfv}  transform  person
\textsc{2sg}  \textsc{ipfv}  cut  one    thing  \textsc{prep}  \textsc{3sg.poss}  finger\\

\glt ‘While you see that the thing is turning into a human-being you cut 
off a part of its finger.’ [ma03sh 012]
\z

The Spanish coordinator pair \textit{ni – ni} ‘neither – nor, not even’ can express negative disjunction in Pichi utterances. Like in Spanish, \textit{ni} can be used alone \REF{ex:key:1730} or in discontinuous negation\is{negation} \REF{ex:key:1731}. Unlike in Spanish, however, subject disjunction in Pichi requires the kind of negative concord characteristic of other negative clauses in Pichi \REF{ex:key:1731}: 


\ea%1730
    \label{ex:key:1730}
    \gll E    nó  sabí    tɔ́k  \textbf{\textit{ni}}    Panyá,  e    sé
e    wánt  \textbf{muchachita}  \textbf{de}  \textbf{diecisiete}    \textbf{años}.\\
\textsc{3sg.sbj}  \textsc{neg}  know  talk  neither  Spanish  \textsc{3sg.sbj}  \textsc{quot}  
\textsc{3sg.sbj}  want  young.girl  of  seventeen  year.\textsc{pl}\\

\glt ‘He doesn’t even know how to speak Spanish, (and) he says he wants 
a girl of seventeen years.’ [ye03cd 053]
\z


\ea%1731
    \label{ex:key:1731}
    \gll \textbf{\textit{Ni}}    ín    \textbf{\textit{ni}}    in    brɔ́da  dɛn  \textbf{nó}  lán.\\
neither  \textsc{3sg.indp}  neither  \textsc{3sg.poss}  brother  \textsc{3pl}  \textsc{neg}  learn\\

\glt ‘Neither him nor his brother studied.’ [ro05de 145]
\z

In \REF{ex:key:1732}, we find the cardinal numeral \textit{wán} ‘one’ in a peculiar construction with the meaning ‘around’ in combination with quantity expressions.\textit{} When \textit{wán} is employed in this way, it usually modifies \textsc{NPs} containing numerals \REF{ex:key:1732} and time units \REF{ex:key:1733}: 


\ea%1732
    \label{ex:key:1732}
    \gll Yu  jɔ́s  gɛ́t \textit{wán}    \textbf{diecisiete}    \textbf{años}.\\
\textsc{2sg}  just  get  one    seventeen  year.\textsc{pl}\\

\glt ‘You’re just about seventeen years old.’ [ab03ay 105]
\z


\ea%1733
    \label{ex:key:1733}
    \gll Tumɔ́ro    mɔ́nin  tɛ́n,t \textit{wán}  \textbf{las}    \textbf{siete}    só,    a    go  gó  dé.\\
tomorrow  morning  time    one  the.\textsc{pl}  seven  like.that  \textsc{1sg.sbj}  \textsc{pot}  go  there\\

\glt ‘Tomorrow in the morning, around seven or so, I will go there.’ [ye03cd 011]
\z

I attribute this particular usage of the numeral \textit{wán} to structural borrowing from Spanish. In Spanish, the plural indefinite articles \textit{unos, unas} serve the same function \REF{ex:key:1734}. 


\ea%1734
    \label{ex:key:1734}
\textsc{Spanish}\il{Spanish}\\
    \gll Me  faltan    \textbf{unos}  dos  mil      francos.\\
Me  they.lack    one.\textsc{pl}  two  thousand  franc.\textsc{pl}\\

\glt ‘I am short of some 2000 francs CFA.’ (Own knowledge)
\z

\subsection{Other constituents}

Spanish discourse markers and adverbs frequently occur at the beginning of a sentence. Speakers often use Spanish material that is not syntactically integrated into a Pichi clause structure. This includes the high frequency adverbs \textit{bueno} ‘well’ \REF{ex:key:1735} and \textit{pero} ‘but’ \REF{ex:key:1736}. Conversely, the interjection \textit{chico} ‘boy’ \REF{ex:key:1736} is not common in European Spanish. It might have developed in Equatoguinean Spanish and Pichi through mutual reinforcement and calquing of other person-denoting interjections in Pichi and other Equatoguinean languages.


\ea%1735
    \label{ex:key:1735}
    \gll \textbf{Bueno},  so  e    kán  tɛ́l  mí    sé    na  tidé.\\
well    so  \textsc{3sg.sbj}  \textsc{pfv}  tell  \textsc{1sg.indp}  \textsc{quot}    \textsc{foc}  today\\

\glt ‘Well, so she told me that it was today.’ [ed03sb 005]
\z


\ea%1736
    \label{ex:key:1736}
    \gll \textbf{\textit{Pero}  }  \textbf{\textit{chico}},  na  yu  pikín  in    láyf.\\
but    boy    \textsc{foc}  \textsc{2sg}  child  \textsc{3sg.poss}  life\\
\glt ‘But man, it’s your child’s life.’ [bo03cb 133]
\z

The interjection \textit{chico} ‘boy’ in \REF{ex:key:1736} above is more common than other human-denoting Pichi equivalents such as \textit{mán} ‘man’,\textit{ papá} \textit{\textup{‘father’,} }or \textit{mamá} ‘mother’\textit{.} The Spanish noun \textit{mierda} ‘shit’\textit{} is very common as a deprecative interjection \REF{ex:key:1737}:


\ea%1737
    \label{ex:key:1737}
    \gll \textbf{\textit{Mierda}}  \textbf{\textit{mierda}},  ús=sáy  e    pás?\\
shit    \textsc{rep}    \textsc{q}=side  \textsc{3sg.sbj}  pass  \\

\glt ‘Shit, shit, which way did she go?’ [ro05rt 002]
\z

Whole adverbial phrases are also admixed in this way. Like discourse markers, these occur at the beginning or the end of a clause: 


\ea%1738
    \label{ex:key:1738}
    \gll A    fít  hól  dán  mɔní  \textbf{durante}  \textbf{un}  \textbf{mes}    \textbf{entero}.\\
\textsc{1sg.sbj}  can  hold  that  money  during  one  month  entire\\

\glt ‘I can keep that money during an entire month.’ [ro05rt 049]
\z

Alternation may also involve larger syntactically independent chunks of Spanish up to a clause boundary: 


\ea%1739
    \label{ex:key:1739}
    \gll A    bɔ́n    nayntín    twɛnti-fó,    {\textbf{por} \textbf{lo} \textbf{tanto}}
\textbf{ahora}  \textbf{tengo}  \textbf{ochenta}  \textbf{años}.\\
\textsc{1sg.sbj}  be.born  nineteen    twenty.\textsc{cpd}{}-four  therefore
now    I.have  eighty  year.\textsc{pl}\\

\glt ‘I was born in 1924, therefore I am now eighty years old.’ [ab03ay 007]
\z

The Spanish focus syntagma \textit{es que} ‘it is that’ may also be seen as a peripheral element which constitutes an independent syntactic unit \REF{ex:key:1740}. However, \textit{es que} is so much an integral part of the Pichi system of focus marking that it seems like a holophrastic borrowing\is{borrowing} (cf. \sectref{sec:7.4.3} for more). Also note the interesting switch to Spanish at the clausal boundary between the relative clause and the following main clause.


\ea%1740
    \label{ex:key:1740}
    \gll Es  que    húman  wé  e    gɛ́t  bɛlɛ́  
\textbf{siempre}  \textbf{suele}  \textbf{ser}  \textbf{así}.\\
It.is  that    woman  \textsc{sub}  \textsc{3sg.sbj}  get  belly
always  usually  be  like.that\\
\glt ‘It’s that women who are pregnant are always like that.’ [ro03rr 008]
\z

\section{Specific semantic fields}\label{sec:13.3}

Some semantic fields are more regularly affected by codemixing than others. Numerals and other similarly tightly interwoven semantic fields like the expression of time or colour are characterised by the extensive use of Spanish words and structures. In many instances, the corresponding Pichi expressions are no longer used or are falling out of use. The corresponding Spanish words and structures have been borrowed into Pichi. 

\subsection{Numerals, days, and dates}\label{sec:13.3.1}

In natural speech, the occurrence of Pichi cardinal numerals drops rapidly after \textit{trí} ‘three’. The percentages of attributive cardinal numerals of Pichi and Spanish provenance in the corpus are presented in \tabref{tab:key:13.4}. Borrowing \is{loan words}\is{borrowing}has had a profound impact on the Pichi numeral system, where Spanish numerals have substituted all but the basic Pichi numerals below eight. Note that this table only lists the usage of \textit{wán} ‘one’ as a cardinal numeral and does not include \textit{wán} in its use as an indefinite determiner with the meaning ‘a’.

%%please move \begin{table} just above \begin{tabular
\begin{table}
\caption{Use of Pichi numerals}
\label{tab:key:13.4}

\begin{tabularx}{.33\textwidth}{cY}
\lsptoprule

 Numeral & Pichi \%\\
\midrule
 1 & 89\%\\
 2 & 80\%\\
 3 & 63\%\\
 4 & 45\%\\
 5 & 30\%\\
 6 & 40\%\\
 7 & 22\%\\
 8 & 0\%\\
 9 & 0\%\\
\lspbottomrule
\end{tabularx}
\end{table}

The attributive use of Spanish numerals goes along with the insertion of Spanish head nouns – there is no instance of a mixed combination of a Spanish numeral and a Pichi noun:


\ea%1741
    \label{ex:key:1741}
    \gll Lɛ́f=an    mék  e    rích    \textbf{a}  \textbf{los}    \textbf{quince}  \textbf{años}.\\
leave=\textsc{3sg.obj}  \textsc{sbjv}  \textsc{3sg.sbj}  reach  to  the\textsc{.pl}  fifteen  year.\textsc{pl}\\

\glt ‘Leave her, let her reach [the age of] fifteen years.’ [ab03ay 138]\is{cardinal numerals}
\z

When telling the time, Spanish lexical items are fit into a conventionalised mixed construction which does not have an exact equivalent in Spanish. In the Pichi construction, the clock time is an adverbial complement to the locative-existential copula\is{copula!locative-existential} \textit{dé} ‘\textsc{cop’}. The copula, in turn, takes the \textsc{1pl} subject \textit{wi} \REF{ex:key:1743}. In the Spanish construction, the clock time functions as the subject of the identity copula \textit{ser} ‘be’ \REF{ex:key:1743}:


\ea%1742
    \label{ex:key:1742}
    \gll Wi  dé    \textbf{las}    \textbf{cuatro}  \textbf{y}  \textbf{media}.\\
\textsc{1pl}  \textsc{be.loc}  the.\textsc{pl}  four    and  half\\

\glt ‘It’s four thirty.’ [nn07fn 483]
\z


\ea%1743
    \label{ex:key:1743}
    \gll Son      \textbf{\textit{las}}    \textbf{\textit{cuatro}}  \textbf{\textit{y}}  \textbf{\textit{media.}}\\
They.are   the.\textsc{pl}   four    and  half\\

\glt ‘It’s four thirty.’
\z

Equally, the majority of speakers employ Spanish dates\is{loan words}. One of the few tokens of a date featuring Pichi numerals was produced by a lady of more than eighty years of age \REF{ex:key:1745}. I assume this instance and the few other similar ones in the corpus to be holophrastic insertions. This view is supported by the fact that the date in \REF{ex:key:1745} is the speaker’s date of birth and perhaps just as significantly, she was married to a Nigerian in her youth. Other than that, this speaker’s use of numerals parallels the one outlined in \tabref{tab:key:13.4} above:


\ea%1744
    \label{ex:key:1744}
    \gll \textbf{El}  \textbf{diez}  \textbf{de}  \textbf{agosto},  bay  gɔ́d  in    páwa,  a    go  pás  na  yá.\\
the  ten  of  August  by  God  \textsc{3sg.poss}  power  \textsc{1sg.sbj}  \textsc{pot}  pass  \textsc{loc}  here\\

\glt ‘(On) the tenth of August, by the grace of God, I’ll pass by this place.’ [ab07fn 113]
\z


\ea%1745
    \label{ex:key:1745}
    \gll \textbf{Soy}    \textbf{del}    \textbf{veinticuatro},  a    bɔ́n    \textbf{nayntín}  \textbf{twɛnti}{}-\textbf{fó}.\\
{I am}    of.the  twenty-four  \textsc{1sg.sbj}  be.born  nineteen  twenty.\textsc{cpd}{}-four\\

\glt ‘I am of [the year] twenty-four, I was born in nineteen twenty-four.’ [ab03ay 006]
\z

Most speakers are not familiar with Pichi day names and employ the Spanish day nomenclature \REF{ex:key:1746}. Even older speakers rarely if ever use the corresponding Pichi day names \textit{mɔ́nde} ‘Monday’, \textit{tyúsde} ‘Tuesday’, \textit{wɛ́nsde} ‘Wednesday’, \textit{tɔsde} ‘Thursday’, \textit{frayde} ‘Friday’, \textit{sátidé} ‘Saturday’, and \textit{sɔ́nde} ‘Sunday’ \REF{ex:key:1747}:


\ea%1746
    \label{ex:key:1746}
    \gll Dí  \textbf{miércoles}  a    de  gó  Lubá.\\
this  Wednesday  \textsc{1sg.sbj}  \textsc{ipfv}  go  \textsc{place}.’\\

\glt ‘This Wednesday, I am going to Luba.’ [ro05ee 119]
\z


\ea%1747
    \label{ex:key:1747}
    \gll \textbf{Lunes}  na  mɔ́nde,  tyúsde  wé  na  \textbf{martes}.\\
monday  \textsc{foc}  Monday  tuesday  \textsc{sub}  \textsc{foc}  tuesday\\

\glt ‘“Lunes” is Monday. Tuesday that’s “martes”.’ [ro05ee 121]
\z

The elicitation of Pichi day names with two speakers below twenty-eight years was unsuccessful save \textit{sɔ́nde} ‘Sunday’, certainly because of its social importance for religious practice. A speaker above fifty-five years experienced considerable difficulties in retrieving Pichi day names (\ref{ex:key:1748a}–c). \textit{Wɛ́nsde} ‘Wednesday’ was only retrieved after an external input (b) and the elicitation of ‘Thursday’ and ‘Friday’ produced the misnomers \textit{tyúsde} ‘Tuesday’ (c) and \textit{wɛ́nsde} ‘Wednesday’ (d), respectively:


\ea%1748
    \label{ex:key:1748}
\ea{\label{ex:key:1748a}
    \gll  \textbf{Miercoles}    na,  áy,  pero  a    sabí=an.\\
  wednesday  \textsc{foc}  \textsc{intj} but    \textsc{1sg.sbj}  know=\textsc{3sg.obj}\\
\glt   ‘"Wednesday" is, ah [pause], but I know it.’ [ro05ee 123]
}\ex{
\itshape \textbf{Wɛ́nsde}?\\
\glt   ‘Wednesday?’ [ko05ee 124] 
}\ex{\label{ex:key:1748c}
\gll
\textbf{Jueves}    na  tyúsde.\\
Thursday  \textsc{foc}  Tuesday\\
\glt   ‘Thursday is \textit{“tyúsde”}.’ [ro05ee 125]
}\ex{
\gll
Frayde na \textbf{miércoles}.\\
  friday  \textsc{foc}  wednesday\\
\glt   ‘\textit{“Frayde”} is Wednesday.’ [ro05ee 126]
}\z\z

In contrast, Pichi designations for the seasons of the year are fully in use, as shown by the use of the compound noun \textit{ren-sísin} ‘rainy season’ \REF{ex:key:1749} and \textit{amatán} ‘harmattan’ in \REF{ex:key:1750}:


\ea%1749
    \label{ex:key:1749}
    \gll Dís  dé  dɛn  ren-sísin    go  bigín.\\
this  day  \textsc{pl}  rainy.\textsc{cpd}{}-season  \textsc{pot}  begin\\

\glt ‘These days, the rainy season should begin.’ [dj05ce 059]
\z


\ea
\label{ex:key:1750}
\gll
  Wi  de  kɔ́l  yá    só    amatán    dán,    lɛkɛ  sé
\textstylePichiexamplenumberZchnZchn{e} kin  dé    lɛkɛ  \textbf{niebla}.\\
\textsc{1pl}  \textsc{ipfv}  call  here    like.that  harmattan  that    like  \textsc{quot}
\textsc{3sg.sbj}  \textsc{hab}  \textsc{be.loc}  like  fog \\
\glt ‘Here, we call harmattan that, like it’s usually like fog.’ [ye05ce 062]
\z

\subsection{Colours}

Colour terminology was elicited with three speakers between the ages of twenty-one and twenty-seven and with two speakers above the age of fifty-five. The exercise revealed the apparent-time differences in colour terminology contained in \tabref{tab:key:13.5}. Pichi terms are in normal font, variants are indicated by a semicolon. Spanish terms are in italics. \tabref{tab:key:13.5} indicates that the younger speakers employ the basic Pichi colour terms \textit{blák} ‘black’ and \textit{wáyt} ‘white’ consistently. The colours ‘red’ and ‘yellow’ are more frequently referred to by the Spanish terms \textit{rojo} and \textit{amarillo}, respectively, but the Pichi terms \textit{rɛ́d} ‘red’ and \textit{yɛ́lo} ‘yellow’ are also used. All other colours are uniquely referred to by Spanish terms{\fff}. The older group consistently makes use of Pichi \textit{rɛ́d} ‘red’ in addition to the basic colours \textit{blák} and \textit{wáyt}. Meanwhile ‘blue’ and ‘green’ are referred to by the Pichi terms \textit{blú} and \textit{grín} , respectively, or by their Spanish equivalents \textit{azul} and \textit{verde}. 


At least in apparent time, the range of Pichi colour terms appears to have been reduced from the six colours \textit{blák}, \textit{wáyt}, \textit{rɛ́d}, \textit{yɛ́lo}, \textit{blú}, and \textit{grín} with the older group, to the two basic colours \textit{blák} and \textit{wáyt}, supplemented by the less frequent \textit{rɛ́d} and \textit{yɛ́lo} (\tabref{tab:key:13.5}).


%%please move \begin{table} just above \begin{tabular
\begin{table}
\caption{Apparent-time differences in the use of colour terms}
\label{tab:key:13.5}

\begin{tabularx}{\textwidth}{XXX}
\lsptoprule

21–27 years & +55 years & Gloss\\
\midrule
\itshape {blák} & \itshape {blák} & ‘black’\\
\itshape {wáyt} & \itshape {wáyt} & ‘white’\\
\itshape rɛ́d, \textbf{rojo} & \itshape rɛ́d & ‘red’\\
\itshape yɛ́lo, \textbf{amarillo} & \itshape yɛ́lo, \textbf{amarillo} & ‘yellow’\\
\textit{\textbf{azul}} & \textstyleTablePichiZchn{{blú}}\textit{,} \textit{\textbf{azul}} & ‘blue’\\
\textit{\textbf{verde}} & \textstyleTablePichiZchn{{grín}}\textit{,} \textit{\textbf{verde}} & ‘green’\\
\textit{\textbf{naranja}} & \textit{\textbf{naranja}} & ‘orange’\\
\textit{\textbf{rosa}} & \textit{\textbf{rosa}} & ‘pink’\\
\textit{\textbf{violeta}} & \textit{\textbf{violeta}} & ‘violet’\\
\textit{\textbf{marrón}} & \textit{\textbf{marrón}} & ‘brown’\\
\lspbottomrule
\end{tabularx}
\end{table}

Many West African languages, including basilectal Nigerian Pidgin \citep[286]{Faraclas1996} express colours and hues other than ‘black’ and ‘white’ through periphrasis, suprasegmentals and ideophones\is{ideophones}. We also find the expression of colours through periphrasis in Pichi, as in \REF{ex:key:1751} and \REF{ex:key:1752}.


\ea%1751
    \label{ex:key:1751}
    \gll Di  bɔ́y \textbf{yɛ́lo}  lɛkɛ  Chici.\\
\textsc{def}  boy  be.yellow  like  \textsc{name}\\

\glt ‘The guy is yellow like [the guy called] Chici.’ [i.e. He has a light brown skin colour]
\z


\ea%1752
    \label{ex:key:1752}
    \gll Dán  tín     \textbf{yɛ́lo}  lɛk  banána.\\
that  thing  be.yellow  like  banana\\

\glt ‘That thing is yellow like a banana.’ [i.e. It has a bright yellow colour]
\z

The rarity of Pichi colour terms beyond the basic ones of \textit{blák} and \textit{wáyt} with the younger group may therefore be indicative of a departure from the West African composite system of colour denomination towards a European simplex system in which non-basic colours are denoted by specific property items. 


When Spanish colour terms are used attributively, they occur with Spanish head nouns \REF{ex:key:1753}. The corpus contains no examples of mixed collocations involving a Spanish colour denoting property item and a Pichi head noun:



\ea%1753
    \label{ex:key:1753}
    \gll A    tínk    sé    na  \textbf{judías}  \textbf{blancas}  o  no sé.\\
\textsc{1sg.sbj}  think  \textsc{quot}    \textsc{foc}  bean.\textsc{pl}  white.\textsc{pl}  or  \textsc{neg}  I.know\\

\glt ‘I think they’re white beans or so.’ [eb03sp 122]\is{borrowing}
\z

Spanish colour terms also occur as predicate adjectives in the specific type of mixed copula clause involving Spanish adjectives covered in \sectref{sec:13.2.2} above. In contrast, Pichi colour terms are only lexicalised as inchoative-stative verbs.\is{colour terminology}	 

\subsection{Other semantic fields}

Other semantic fields characterised by a high incidence of codemixing involve formalised, institutional domains. One of the few Pichi country names in use is \textit{Panyá} ‘Spain’, the designation for the former colonial power. Spanish lexemes are exclusively employed for country names like \textit{Guinea (Ecuatorial)} ‘Equatorial Guinea’, \textit{Gabón} ‘Gabon’ \REF{ex:key:1754}, ethnonyms like \textit{europeo} ‘European’ or \textit{cameruneses} ‘Cameroonians’ \REF{ex:key:1755}, as well as terms belonging to the state domain such as \textit{problema diplomático} ‘diplomatic problem’ \REF{ex:key:1754}: 


\ea%1754
    \label{ex:key:1754}
    \gll \textbf{Entonces}    wán    \textbf{\textit{problema}}  \textbf{\textit{diplomático}}  kán  dé    entre
\textbf{\textit{Guinea}}  wet    \textbf{\textit{Gabón}}.\\
so      one    problem    diplomatic  \textsc{pfv}  \textsc{be.loc}  between
\textsc{place}  with    \textsc{place}\\

\glt ‘So a diplomatic problem came to be between Guinea and Gabon.’ [fr03ft 007]
\z


\ea%1755
    \label{ex:key:1755}
    \gll \textbf{Cameruneses},    yɛ́s  dɛn  plɛ́nte    yá.\\
Cameroonians  yes  \textsc{3pl}  be.plenty  here\\

\glt ‘Cameroonians, yes they are many here.’ [ma07fn 607]
\z

Also compare the Spanish terms \textit{registro} ‘(civil) registry’ and \textit{registra} ‘(to) register’ in \REF{ex:key:1756}:


\ea%1756
    \label{ex:key:1756}
    \gll A    bin  gɛ́fɔ    chénch  in    ném    na  \textbf{registro}  
a    \textbf{\textit{registra}}  ín.\\
\textsc{1sg.sbj}  \textsc{pst}  have.to  change  \textsc{3sg.poss}  \textsc{name}  \textsc{loc}  register  
\textsc{1sg.sbj}  register  \textsc{3sg.indp}\\

\glt ‘I had to change her name in the register, I registered her.’ [ab03ay 162]
\z

The Pichi lexemes \textit{skul} ‘school’\textit{, gɔ́bna} ‘government’ \REF{ex:key:1757}, and \textit{chɔ́ch} ‘church’ \REF{ex:key:1758} designate these institutions in their general sense and are favoured over their Spanish equivalents \textit{escuela, gobierno}, and\textit{ iglesia}:


\ea%1757
    \label{ex:key:1757}
    \gll E    de  gó  fɔ,  sɔn    \textbf{skúl}    wé  dé    fɔ  \textbf{gɔ́bna}.\\
\textsc{3sg.sbj}  \textsc{ipfv}  go  \textsc{prep}  some  school  \textsc{sub}  \textsc{be.loc}  \textsc{prep}  government\\

\glt ‘She goes to a school that belongs to government.’ [ma03hm 028]
\z


\ea%1758
    \label{ex:key:1758}
    \gll E    sé  e    gó  chɔ́ch  fɔ,  fɔ  Marieta  na  {Ela Nguema},
na  \textbf{catedral}.\\
\textsc{3sg.sbj}  \textsc{quot}  \textsc{3sg.sbj}  go  church  \textsc{prep}  \textsc{prep}  \textsc{place}  \textsc{loc}  \textsc{place}
\textsc{loc}  cathedral\\

\glt ‘She said she went to church at Marieta’s in Ela Nguema, by the cathedral.’ [hi03cb 078]
\z

Meanwhile, the incidence of Spanish lexemes rises with the degree of specificity of words within the semantic fields designated by these superordinates. Thus, we have \textit{catedral} ‘cathedral’ in \REF{ex:key:1758} above, \textit{bolí} ‘pen’ and \textit{cuaderno} ‘exercise book’ \REF{ex:key:1759}, as well as \textit{profe(sor)} ‘teacher’ – though \textit{tícha} ‘teacher’ is also common, however less so beyond primary school. 


\ea%1759
    \label{ex:key:1759}
    \gll Wé,  yu  wánt  báy  \textbf{\textit{cuaderno}},    \textbf{\textit{bolí}}  ɔ́l  dán  tín    dɛn
na  wet    \textbf{dólar}.\\
\textsc{sub}  \textsc{2sg}  want  buy  exercise.book    pen  all  that  thing  \textsc{pl}  
\textsc{foc}  with    dollar\\

\glt ‘While, if you want to buy exercise books, pens, all those things are
with the dollar.’ [ed03sp 096]
\z


\ea%1760
    \label{ex:key:1760}
    \gll Di  \textbf{profesor},    na  bɛ́ta     \textbf{profe}.\\
\textsc{def}  teacher    \textsc{foc}  very.good  teacher\\

\glt ‘The (secondary school) teacher is a very good teacher.’ [dj05be 172]
\z

The preponderance of Spanish lexemes in other semantic fields reflects the asymmetric power relation that holds between Pichi and Spanish in a different way. For example, semantic fields relating to illness and medical treatment that are highly differentiated in other languages of the region (e.g. \ili{Yoruba}, see \citet{Adegbite1993}) probably did not assert itself in Pichi due to the marginalisation of African medical science with the advent of colonialism. In \REF{ex:key:1761}, we therefore find \textit{placenta} ‘placenta’ and \textit{matriz} ‘womb’ for which only the general term \textit{bɛlɛ́} ‘belly, womb’ is recorded and Spanish \textit{membrano} ‘membrane’ which has no equivalent in Pichi: 


\ea%1761
    \label{ex:key:1761}
    \gll Wé  dɔ́kta  ópin,  wé  dɛn  bigín  drɔ́    di,  sɔn    tín    we e    kin  dé    bihɛ́n \textbf{placenta},    na \textbf{membrana},  sɔn    kán  lɛ́f bifó    di  \textbf{matriz},  so  di  \textbf{matriz}  nó  kán  lɔ́k.\\
\textsc{sub}  doctor  open  \textsc{sub}  \textsc{3pl}  begin  draw  \textsc{def}  some  thing  \textsc{sub}
\textsc{3sg.sbj}  \textsc{hab}  \textsc{be.loc}  behind  placenta    \textsc{foc}  membrane  some  \textsc{pfv}  remain
before  \textsc{def}  womb  so  \textsc{def}  womb  \textsc{neg}  \textsc{pfv}  lock\\

\glt ‘When the doctor opened (the womb), they began to draw out the, a certain thing that 
is usually behind the placenta, it’s a membrane, some remained in front of the womb, 
so the womb didn’t close.’ [ab03ay 084]
\z

The systematic use of Spanish items also occurs in semantic fields that designate aspects of material and non-material culture of external origin. In \REF{ex:key:1762}, a car mechanic explains the disadvantages of an Opel ignition cable. Note the Spanish technical terms in the sentence: 


\ea%1762
    \label{ex:key:1762}
    \gll Hɛ́,  a    go  fála      yú    bikɔs  sɔn   \textbf{cable}  dé
wé  na  fɔ  Opel,  yu  \textbf{intenta}  bríng  Opel    in    yón   na
\textbf{corriente},  Opel    de  kɛ́r    bɔkú \textbf{corriente} só    e
nó  go  fít  ɛ́nta    na  dán \textbf{bujía},    yu  go  wánda  sɛ́f.\\
\textsc{intj}  \textsc{1sg.sbj}  \textsc{pot}  accompany  \textsc{2sg.indp}  because  some  cable  \textsc{be.loc}
\textsc{sub}  \textsc{foc}  \textsc{prep}  \textsc{name}  \textsc{2sg}  try    bring  \textsc{name}  \textsc{3sg.poss}  own  \textsc{foc}
electricity  \textsc{name}  \textsc{ipfv}  take    much  electricity  like.that  \textsc{3sg.sbj}
\textsc{neg}  \textsc{pot}  can  enter  \textsc{loc}  that  ignition.plug  \textsc{2sg}  \textsc{pot}  wonder  \textsc{emp}\\

\glt ‘Hey, I’ll accompany you because there’s a cable which is an Opel (cable), (and if) you try to connect the Opel one with electricity, Opel takes a lot of electricity, so it won’t be able to enter that ignition plug, (and) you’ll be very surprised.’ [f103fp 017]
\z

Spanish kinship\is{kinship terminology} terms have also left their mark on the language (cf. also \sectref{sec:12.3}). In \REF{ex:key:1763}, we find \textit{primo} ‘(male) cousin’, a kinship concept that is only rarely expressed by the Pichi term \textit{kɔsín.}


\ea%1763
    \label{ex:key:1763}
    \gll A    tínk    sé    dɛn  papá  na  mi    mamá  in    \textbf{\textit{primo}}.\\
\textsc{1sg.sbj}  think  \textsc{quot}    \textsc{3pl}  father  \textsc{foc}  \textsc{1sg.poss}  mother  \textsc{3sg.poss}  cousin\\

\glt ‘I think that their father is my mother’s cousin.’ [fr03ft 059]
\z

Conversely, the incidence of Spanish words is low in semantic fields characterised by the use of autochthonous technology, such as farming and with designations for locally-grown foodstuffs and other flora. Thus, in \REF{ex:key:1764}, we have \textit{díg grɔ́n} ‘dig ground’ = ‘plough up the ground’, \textit{plánt chɔ́p} ‘(to) plant food’, \textit{gádin} ‘small field, garden’, \textit{jakató} ‘bitter tomato’ and \textit{kíp} ‘grow, rear’, as well as \textit{pamáyn} ‘oil’, and \textit{gadinɛ́ks} ‘egg-plant’ \REF{ex:key:1765}:


\ea%1764
    \label{ex:key:1764}
    \gll A    díg  grɔ́n,  a    plánt chɔ́p,  a    gó  na  gádin,
a    kíp    jakató,    verdura.\\
\textsc{1sg.sbj}  dig  ground  \textsc{1sg.sbj}  plant  food    \textsc{1sg.sbj}  go  \textsc{foc}  garden
\textsc{1sg.sbj}  grow  bitter.tomato  vegetables\\

\glt ‘I ploughed the ground, I planted food, I went to the garden, I grew bitter tomato, 
vegetables.’ [ab03ay 063]
\z


\ea%1765
    \label{ex:key:1765}
    \gll Di  dé  wé  yu  go  níd=an,    yu  go  sé  a    nó  gɛ́t
\textbf{pamáyn},   yu  go  kɔ́t \textbf{gadinɛ́ks}.\\
\textsc{def}  day  \textsc{sub}  \textsc{2sg}  \textsc{pot}  need=\textsc{3sg.obj}  \textsc{2sg}  \textsc{pot}  \textsc{quot} \textsc{1sg.sbj}  \textsc{neg}  get
oil      \textsc{2sg}  \textsc{pot}  cut  egg.plant\\

\glt ‘The day when you would need it, you would say “I don’t have oil”, (and) you would harvest egg-plants.’ [ab03ay 015]\is{codemixing}
\z

