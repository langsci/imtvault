\chapter{Suprasegmental phonology}

Pichi is a tone language. In previous work, I posited that Pichi has a mixed prosodic system in which individual words are either specified for pitch accent or tone (\citealt{Yakpo2009a,Yakpo2009b}), similar to systems claimed for other European-lexifier creoles of the Atlantic basin (e.g. \citealt{Rountree1972,Alleyne1980,Devonish1989,Devonish2002,Good2004,Castillo1998,CastilloFaraclas2006}). In subsequent work on Pichi and comparative work on the prosodic systems of other Afro-European contact varieties (e.g. \citealt{SteienYakpo2017}), I found no evidence that the Pichi lexicon is stratified and that “tonal” and “pitch-accented” words differ with respect to their pitch-related properties or the tonal processes described in this chapter. I therefore treat the prosodic system of Pichi as a tonal system \textit{tout} \textit{court}. In the following, the term “tone class” designates the various fixed pitch patterns that Pichi words fall into. 


The pitch analyses in this chapter were done from connected speech and from words pronounced in isolation using the Praat 5.0 software. The analyses are presented in figures containing a pitch trace and a syllabic segmentation of the utterance. The transcription employed for rendering syllabic segments is orthographic. Nonetheless, phonetic tones are marked on each syllable in the figures for easier recognition.



The approximate pitch values of each syllable are given in Hertz (Hz) on the vertical axis. The horizontal axis provides the time elapsed (1.0 = 1 second). In  the examples in this chapter, the second line contains a phonetic tonal notation of the Pichi utterance above. When a tonal process is described, the relevant Pichi sentence is sometimes repeated after the arrow (→). The second line of the Pichi utterance following the arrow then provides phonetic tone, i.e. the actual pronunciation of the sentence after the tonal process under discussion has taken place. For clarity of presentation, text codes have been omitted with examples in this chapter.


\section{Characteristics of tone}\label{sec:3.1}

\largerpage[2]
Pichi has two distinctive tonemes, namely a High (H) and a Low (L) tone. The language employs lexical and morphological tone, and there is an unevenly distributed number of tone classes. Boundary tones at the right edge of utterances fulfil the pragmatic and grammatical functions of intonation (cf. \sectref{sec:3.4}).

The tone-bearing unit in Pichi is the syllable. Vowels and sonorants serve as tone-bearing segments. Evidence comes from the interaction of lexical tones and boundary tones over utterance-final syllables. In utterance-final position, a boundary tone will associate with the final tone-bearing unit of the utterance. The sonorants /n/, /m/, /l/, and /r/ may bear phonetic tone in Pichi. Hence, an utterance-final /n/, for example, may carry a boundary tone. 

Consider the citation form of \textit{tɛ́n} ‘time’ in \figref{fig:key:3.1}. Here the declarative L\% (L boundary tone), which follows the lexical H tone over /ɛ́/, is spread out over the vowel and the final /n/. Sonorants like /n/ do not, however, bear lexical tone by themselves. Rather, they always bear the tone of the left-adjacent, i.e. preceding, vowel. In contrast, with non-sonorant final segments, tone is only borne by the preceding vowel. The final obstruent in \textit{tɔ́k} ‘talk’ in \figref{fig:key:3.2} cannot bear tone, so the utterance-final declarative L\% is borne by the vowel alone.
\largerpage

%\begin{minipage}{.45\textwidth}
	
%\parbox{.45\textwidth}{
\begin{figure}
	\caption{Citation form of \textit{tɛ́n}}
	\label{fig:key:3.1}
\begin{tikzpicture}
    \draw (0, 0) node[inner sep=0] {
	\includegraphics[width=.44\textwidth]{figures/yakpomod-img3.png}};
    \draw (0, -1.37) node[fill=white] {\footnotesize Tɛ́n.};
\end{tikzpicture}
\end{figure}

\ea\label{ex:key:43}
\gll Tɛ́n.\\
\textsc{hl\%}\\
\glt  ‘Time.’
\z
%}
%\end{minipage}

%\begin{minipage}{.45\textwidth}
%\parbox{.45\textwidth}{
\begin{figure}
	\caption{Citation form of \textstyleTablePichiZchn{tɔ́k}}
	\label{fig:key:3.2}
\begin{tikzpicture}
    \draw (0, 0) node[inner sep=0] {
	\includegraphics[width=.44\textwidth]{figures/yakpomod-img4.png}};
    \draw (0, -1.37) node[fill=white] {\footnotesize Tɔ́k.};
\end{tikzpicture}
\end{figure}

\ea\label{ex:key:44}
\gll Tɔ́k.\\
\textsc{hl\%}\\
\glt ‘Talk.’
\z
%}
%\end{minipage}

\largerpage
When the utterance-final word is a light (vowel-final) monosyllable, the vowel may be lengthened, sometimes up to two beats. I assume that the lengthening of light monosyllables is caused by the metric preference of Pichi for footed tonal domains within the word boundary. Heavy monosyllables with a final non-tone-bearing segment like \textit{tɔ́k} ‘talk’ block the creation of footed domains in utterance-final position. But light syllables leave room for this option. The vowels of the light monosyllables in \figref{fig:key:3.3} and \figref{fig:key:3.4} have been lengthened in order to accommodate the HL contour consisting of the lexical H tone of the monosyllable and the declarative L\% boundary tone. 

\largerpage[2]

\begin{figure} 
\caption{Citation form of \textstyleTablePichiZchn{só}}
\begin{tikzpicture}
    \draw (0, 0) node[inner sep=0] {
    \includegraphics[height=.3\textheight]{figures/yakpomod-img5.png}};
    \draw (0, -1.64) node[fill=white] {\footnotesize Só.};
\end{tikzpicture}
\label{fig:key:3.3}
\end{figure}


   
\ea\label{ex:key:45}
\gll      S\textbf{ó}.\\
\textsc{h}\textbf{\textsc{l\%}}\\
\glt ‘Like that.’
\zlast

\begin{figure}
\caption{Citation form of \textstyleTablePichiZchn{dé}}
\begin{tikzpicture}
    \draw (0, 0) node[inner sep=0] {
    \includegraphics[height=.3\textheight]{figures/yakpomod-img6.png}};
    \draw (0, -1.64) node[fill=white] {\footnotesize Dé.};
\end{tikzpicture}
\label{fig:key:3.4} 
\end{figure} 

\clearpage 

\ea\label{ex:key:46}
\gll    D\textbf{é}.\\
\textsc{h}\textbf{\textsc{l\%}}\\
\glt ‘There.’
\z

\subsection{Distinctive tones}
\largerpage[2]
Pichi contrasts two level tones, a high tone (H) and a low tone (L). H tone is the more active tone in tonal processes: H rather than L participates in tone spreading and is more active in pitch register expansion. Contour tones do not constitute tonemes in their own right. Instead, they result from the succession of a lexical tone and a polar floating tone over a single tone-bearing unit (cf. \sectref{sec:3.2.2}). 



\begin{figure}[b]
\caption{H.L pattern}
\label{fig:key:3.5}
\begin{tikzpicture}
    \draw (0, 0) node[inner sep=0] {
    \includegraphics[height=.3\textheight]{figures/yakpomod-img7.png}};
    \draw (0, -1.64) node[fill=white] {\footnotesize Hásis.};
\end{tikzpicture}
\end{figure}

\begin{figure}[b]
\caption{L.H pattern}
\label{fig:key:3.6}  
\begin{tikzpicture}
    \draw (0, 0) node[inner sep=0] {
    \includegraphics[height=.3\textheight]{figures/yakpomod-img8.png}};
    \draw (0, -1.64) node[fill=white] {\footnotesize Dɔtí.};
\end{tikzpicture}
\end{figure}

\figref{fig:key:3.5} and \figref{fig:key:3.6}  present the pitch trace and segmentation of the two words \textit{hasis} /H.L/ ‘ashes’ and \textit{dɔtí} /L.H/ ‘be dirty’ said in isolation. The two words \textit{hasis} and \textit{dɔtí} represent the tone patterns of the two most frequent tone classes of Pichi (cf. \tabref{tab:key:3.1}). The mean pitch on the L-toned syllable of \textit{dɔtí} is 109.17~Hz, that of the H-toned syllable 129.27~Hz. Hence, the difference in pitch between the H- and L-level tones amounts to 20.1~Hz. With \textit{hásis}, the mean pitch of the H tone is 108.59~Hz, while the mean L tone stands at 99.72~Hz. The difference in mean pitch between H and L therefore stands at 8.87~Hz. This difference is just about half of that between L and H in \textit{dɔtí}.

%%please move \begin{table} just above \begin{tabular
\begin{table} 
\caption{Pitch values}
\label{tab:key:3.1} 
\begin{tabularx}{.5\textwidth}{Xrr}
\lsptoprule
Hertz & \itshape dɔtí & \itshape hásis\\
\midrule
Mean Hz of H    & 129.27 & 108.59\\
Mean Hz of L    & 109.17 & 99.72\\
Highest Hz of H & 132.20 & 110.33\\
Lowest Hz of H  & 127.26 & 107.35\\
Highest Hz of L & 110.78 & 105.83\\
Lowest Hz of L  & 107.47 & 93.50\\
\lspbottomrule
\end{tabularx}
\end{table}

 
 
The relatively small difference in mean pitch between the syllables of \textit{hásis} arises due to the fact that the H tone over the first syllable is carried over into the first half of the following L-toned syllable. In contrast, the L tone of the first syllable of \textit{dɔtí} shows no signs of rightward spreading. 

Words may bear a single or more H or L tones. Compare the pitch traces of the utterance-final tonal words \textit{nyɔ́ní} ‘ant’ and \textit{Bata} ‘Fang’ in the collocations \textit{lɛk nyɔ́ní} ‘like ants’ and \textit{tɔ́k Bata} ‘speak Fang’ in \figref{fig:key:3.7} and \figref{fig:key:3.8}.


\begin{figure}[b]
\caption{H.H pattern} 
\label{fig:key:3.7}
\begin{tikzpicture}
    \draw (0, 0) node[inner sep=0] {
    \includegraphics[height=.3\textheight]{figures/yakpomod-img9.png}};
    \draw (0, -1.64) node[fill=white] {\footnotesize Lɛk nyɔ́ní.};
\end{tikzpicture}
\end{figure}

\begin{figure} 
\caption{L.L pattern}
\label{fig:key:3.8}
\begin{tikzpicture}
    \draw (0, 0) node[inner sep=0] {
    \includegraphics[height=.3\textheight]{figures/yakpomod-img10.png}};
    \draw (0, -1.64) node[fill=white] {\footnotesize Tɔ́k Bata.};
\end{tikzpicture}
\end{figure}

 
Equatoguinean Spanish has been analysed as a tone language, in which the lexical stress characteristic of Spanish has been converted to lexical tone due to contact with the tone languages of Equatorial Guinea (\citealt{Lipski2015,SteienYakpo2017}). Words codeswitched or borrowed from Equatoguinean Spanish are therefore specified for lexical tone just like Pichi words. 

\begin{figure}[h]
\caption{Pitch over Spanish \textstyleTablePichiZchn{abril}}
\label{fig:key:3.9}
\begin{tikzpicture}
    \draw (0, 0) node[inner sep=0] {
    \includegraphics[height=.3\textheight]{figures/yakpomod-img11.png}};
    \draw (0, -1.64) node[fill=white] {\footnotesize Abril.};
\end{tikzpicture}
\end{figure}

\begin{figure}
\caption{Pitch over Spanish \textit{nigeriano}}
\label{fig:key:3.10} 
\begin{tikzpicture}
    \draw (0, 0) node[inner sep=0] {
    \includegraphics[height=.3\textheight]{figures/yakpomod-img12.png}};
    \draw (0, -1.64) node[fill=white] {\footnotesize Na nigeriano.};
\end{tikzpicture}
\end{figure}

 
\figref{fig:key:3.9} and \figref{fig:key:3.10} feature the utterance-final Spanish words \textit{abril} ‘April’ and \textit{nigeriano} ‘Nigerian’, the latter in the collocation \textit{na} \textit{nigeriano} ‘\textsc{foc} Nigerian’ \textit{=} ‘He is a Nigerian’. The pitch configurations over these two words conforms to those of Pichi words with a word-final (\figref{fig:key:3.9}) and a penultimate (\figref{fig:key:3.10}) H tone, respectively. 


   \newpage 
 
\subsection{Lexical and morphological tone}

A small number of monosyllabic roots are distinguished from each other by pitch alone. The list in \REF{ex:key:47} contains most words in the corpus to which this applies. In conformity with a general pattern, (more) functional words are L-toned, while the corresponding content words are H-toned.

\eabox{\label{ex:key:47}
\begin{tabularx}{\textwidth}{ll ll}
L tone &  & H tone & \\
\itshape bay & ‘by’ & \itshape báy & ‘buy’\\
\itshape bɔt & ‘but’ & \itshape bɔ́t & ‘hit with the head’\\
\itshape de & ‘\textsc{ipfv}’ & \itshape dé & ‘day, there’\\
\itshape di & ‘\textsc{def}’ & \itshape dí & ‘this’\\
\itshape lɛk & ‘like’ & \itshape lɛ́k & ‘(to) like’\\
\itshape so & ‘so’ & \itshape só & ‘like this, sew, show’\\
\itshape wet & ‘with’ & \itshape wét & ‘wait’\\
\end{tabularx}
}

However, there are also numerous homophones, which can neither be distinguished segmentally, nor by their pitch properties. The following list contains most homophones in the corpus.

\eabox{\label{ex:key:48}
\begin{tabularx}{\textwidth}{ll ll}
\multicolumn{2}{l}{Homophones} &  & \\
\itshape dé & ‘day, there, \textsc{be.loc}’ & \itshape líf & ‘leaf, live’\\
\itshape an & ‘\textsc{3sg.obj}, and’ & \itshape lɔ́s & ‘loose, louse’\\
\itshape día & ‘deer, expensive’ & \itshape na & \textsc{‘foc,} \textsc{loc’}\\
\itshape bia & ‘beer, bear’ & \itshape nó & ‘know, \textsc{neg’}\\
\itshape bló & ‘blow, relax’ & \itshape nyús & ‘news, use’\\
\itshape fɔ́l & ‘fowl, to rain’ & \itshape pía & ‘avocado, pair’\\
\itshape fɔ́s & ‘first, force’ & \itshape ráyt & ‘right, write’\\
\end{tabularx}
}
\begin{exe}
\sn\parbox[t]{.8\textwidth}{
      \vspace{.7\baselineskip}
\begin{tabularx}{\textwidth}{ll ll}
\itshape fíl & ‘feel, field’ & \itshape rɛ́s & ‘rest, rice’\\
\itshape hát & ‘heart, to hurt’ & \itshape rɔ́n & ‘run, be wrong’\\
\itshape hía & ‘hear, here, year, hair’ & \itshape só & ‘sew, show’\\
\itshape hól & ‘hole, hold, whole’ & \itshape sɔ́t & ‘shirt, short’\\
\itshape (h)ɔ́t & ‘extinguish, hot’ & \itshape tɔ́n & ‘town, turn’\\
\itshape klós & ‘clothing’ & \itshape tú & ‘too (much), two’\\
\itshape kɔ́s & ‘cost, (to) insult’ & \itshape wé & ‘way, \textsc{sub’}\\
\itshape lɛ́f & ‘leave, left’ & \itshape wích & ‘bewitch, which’\\
\end{tabularx}
}
\end{exe}
\bigskip

Morphological tone is employed in the personal pronoun paradigm in order to distinguish morphologically different forms of the same lexeme from one another (e.g. \textit{mi} \textsc{‘1sg.poss’} – \textit{mí} \textsc{‘1sg.indp’}, \textit{dɛn} ‘\textsc{3pl}’ – \textit{dɛ́n} ‘\textsc{3pl.indp}’). Pichi also features a morphological tonal process (cf. \sectref{sec:3.2.4}). In addition, there are three items that derive from a common etymon and are distinguished by pitch alone: \textit{de} ‘\textsc{ipfv}’ – \textit{dé} ‘\textsc{be}.\textsc{loc}’, \textit{di} \textsc{‘def’} – \textit{dí} ‘this’, \textit{go} ‘pot’ – \textit{gó} ‘go’). All low-toned monosyllabic roots are words with more or less grammatical functions, such as personal pronouns (e.g. a ‘\textsc{1sg.sbj}’), determiners (e.g. \textit{di} ‘\textsc{def}’), TMA markers (e.g. \textit{bin} ‘\textsc{pst}’, \textit{kin} ‘\textsc{hab}’), clause linkers (e.g. \textit{ɛf} ‘if’), or prepositions (e.g. \textit{pan} ‘on’). Low-toned function words, except dependent personal pronouns, are listed in \REF{ex:key:49}.

\eabox{\label{ex:key:49}
\begin{tabularx}{\textwidth}{XXXXX}
\multicolumn{2}{l}{Low-toned function words} & \multicolumn{2}{c}{}\\
\itshape di & ‘\textsc{def}’ & \itshape lɛk(ɛ) & ‘like’\\
\itshape sɔn & ‘some, a’ & \itshape na & \textsc{‘loc,} \textsc{foc’}\\
\itshape bin & ‘\textsc{pst}’ & \itshape pan & ‘on’\\
\itshape de & ‘\textsc{ipfv}’ & \itshape to & ‘to’\\
\itshape go & ‘\textsc{pot}’ & \itshape wet & ‘with’\\
\itshape kin & ‘\textsc{hab}’ & \itshape an & ‘and’\\
\itshape mɔs & ‘\textsc{obl}’ & \itshape ɔ & ‘or’\\
\itshape bay & ‘by’ & \itshape ɛf(ɛ) & ‘if’\\
\itshape fɔ & ‘\textsc{prep}’ & \itshape bɔt & ‘but’\\
\itshape frɔn & ‘from’ & \itshape so & ‘so’\\
\end{tabularx}
}

There are, however, limits to this pattern of functional differenciation by tone. The monosyllabic roots \textit{dɔ́n} ‘down, done, \textsc{prf’}, \textit{kán} ‘come, \textsc{pfv’,} \textit{mék} ‘make, \textsc{sbjv}’, \textit{sé} ‘say, \textsc{quot’,} and \textit{wán} ‘one, a’ also have a more grammatical meaning besides their lexical one. Yet, their different functions are covered by segmentally and suprasegmentally identical forms. 


Pichi also exhibits one morphological tonal process. In compounds and morphological reduplication, the H tones over all non-final components are deleted and replaced by an L tone (cf. \sectref{sec:3.2.4}).


\subsection{Tone classes}\label{sec:3.1.3}

About 95 per cent of roots contained in my lexical data-base carry a single H tone over their only, penultimate, or final syllable. Other syllables in these words are L-toned. The remaining 5 per cent of roots feature diverse tone patterns with more than one H, or no H tone. Many (e.g. \textit{nyɔ́ní} ‘ant’ < \ili{Mende} \textit{yɔ́ní} ‘red ant’) but not all (e.g. \textit{ápás} ‘after’ < \ili{English} ‘half-past’) of these words originate from African languages or are monosyllabic function words with an L tone over their only syllable (cf. \ref{ex:key:49}), while words with a single H tone are mostly English-derived. This circumstance speaks to the fact that stress-to-tone conversion took place in the formation of the proto-language of Pichi, as in many other Afro-European creole and non-creole contact languages (e.g. \citealt{Berry1970,Criper1971,Criper-Friedman1990,Alleyne1980,GussenhovenUdofot2010,Steien2015}). 

\tabref{tab:key:3.2} contains a listing of the tone classes of the simplex roots contained in the lexical data base of the corpus (cf. \citealt{Faraclas1996,Good2004}, for pitch classes in Nigerian Pidgin and \ili{Saramaccan}). A few examples are provided for each tone class. Not included in this table are ideophones\is{ideophones}, which feature numerous idiosyncratic tonal patterns and often involve lexicalised reduplication and triplication (cf. \sectref{sec:4.5.3} and \sectref{sec:12.1} for a detailed treatment).

Members of the monosyllabic L-toned tone class only contribute a total of nineteen roots and 2.5 per cent of the total in terms of individual entries and are hence listed as belonging to a minor tone class. The members of this class are, however, mostly function words that constitute the backbone of the grammatical system of Pichi: the personal pronouns \textit{a} ‘\textsc{1sg.sbj}’, \textit{e} ‘\textsc{3sg.sbj}’, \textit{=an} ‘\textsc{3sg.obj}’; the \textsc{TMA} markers \textit{de} ‘\textsc{ipfv}’, \textit{go} ‘\textsc{pot}’, \textit{bin} ‘\textsc{pst}’; the preposition \textit{fɔ} ‘\textsc{prep}’ and the homonymous forms \textit{na} ‘\textsc{loc}’ and \textit{na} ‘\textsc{foc}’ outrank any other root of the language in a frequency count. This makes this tone class perceptually as salient as the H and H.L tone classes. In contrast, the members of the other minor tone classes are each  composed of relatively few lexical words, which together make up 6 per cent of roots in the corpus. 


%%please move \begin{table} just above \begin{tabular
\begin{table}
\caption{Distribution of tone classes over types}
\label{tab:key:3.2}

\begin{tabularx}{\textwidth}{lXrr}
\lsptoprule

{{Tone classes}} & Examples & No. of items & \% of total\\
\midrule 
Major &  &  & \\
\midrule
  H & \textstyleTablePichiZchn{báy} ‘buy’, \textstyleTablePichiZchn{áks} ‘ask’, \textstyleTablePichiZchn{kɛ́r} ‘carry, take’ & 413 & 54.1\\
  H.L & \textstyleTablePichiZchn{drɔ́ngo} ‘be dead drunk’, \textstyleTablePichiZchn{kɔ́mpin} ‘friend’ & 178 & 23.3\\
  L.H & \textstyleTablePichiZchn{bɔkú} ‘be much’, \textstyleTablePichiZchn{sabí} ‘know’, \textstyleTablePichiZchn{watá} ‘water’ & 107 & 14.0\\
\midrule 
{Subtotal} &  & {717} & {91.5}\\

\tablevspace
Minor &  &  & \\
\midrule 
  L & \textstyleTablePichiZchn{de} ‘\textsc{ipfv}’, \textstyleTablePichiZchn{go} ‘\textsc{pot}’, \textstyleTablePichiZchn{sɔn} ‘some, a’, \textstyleTablePichiZchn{fɔ} ‘\textsc{prep}’ & 19 & 2.5\\
  L.H.L & \textstyleTablePichiZchn{ɔspítul} ‘hospital’, \textstyleTablePichiZchn{wahála} ‘trouble’ & 14 & 1.8\\
  H.H & \textstyleTablePichiZchn{nyɔ́ní} ‘ant’, \textstyleTablePichiZchn{sóté} ‘until’, \textstyleTablePichiZchn{sósó} ‘only’, \textstyleTablePichiZchn{ápás} ‘after’ & 11 & 1.4\\
  L.L.H & \textstyleTablePichiZchn{ɔndastán} ‘understand’, \textstyleTablePichiZchn{prɔpatí} ‘property’ & 10 & 1.3\\
  H.L.L & \textstyleTablePichiZchn{kápinta} ‘carpenter’, \textstyleTablePichiZchn{mɛ́rɛsin} ‘medicine’ & 6 & 0.8\\
  L.H.H & \textstyleTablePichiZchn{okóbó} ‘impotent man’ & 3 & 0.4\\
  L.L & \textstyleTablePichiZchn{Bata} ‘\textsc{place’}, \textstyleTablePichiZchn{jɔmba} ‘lover’ & 2 & 0.3\\
\midrule 
{Subtotal} &  & {46} & {8.5}\\
\midrule 
Total &  & {763} & {100.0}\\
\lspbottomrule
\end{tabularx}
\end{table}

\tabref{tab:key:3.2} points to additional characteristics of the corpus. With 54.1 per cent, about half the roots are H-toned monosyllables. Another 25.2 per cent are polysyllabic roots with an H tone over the penultimate syllable (of which a mere 1.8 per cent have more than two syllables). Together, these two groups constitute an overwhelming majority of 79.3 per cent of all roots. An additional 15.3 per cent bear an H tone over the final syllable. Most roots in the corpus, namely 94.6 per cent, therefore carry an H tone over the only syllable, the penultimate syllable, or the final syllable.

It should also be mentioned that many of the \ili{Spanish} items that find their way into codemixed Pichi sentences bear a penultimate H tone in accordance with their original Spanish penultimate syllable stress. This holds in particular for the invariant \textsc{3sg} present insertion form of the \ili{Spanish} verb (cf. \sectref{sec:13.2.2}). Spanish-origin items therefore align with the majority tone classes of Pichi.  


\section{Tonal processes}\label{sec:3.2}

Pitch changes conditioned by various factors may take place within a tonal domain. A tonal domain may be confined to the word, cut across a word boundary in specific phono-syntactic phrases, and involve a whole clause or sentence. The tonal processes attested in the data are described in \sectref{sec:3.2.1} to \sectref{sec:3.2.4}. A summary of these processes is given in \tabref{tab:key:3.3}.

%%please move \begin{table} just above \begin{tabular
\begin{table}
\caption{Tonal processes}
\label{tab:key:3.3}
\small
\begin{tabularx}{\textwidth}{lQQp{1.5cm}}
\lsptoprule
Process & Description & Conditioning factor & Tonal\newline  domain\\
\midrule 
Spreading & H spreads rightwards to L-toned syllable(s) & H spreads rightwards to L-toned syllables & (1) Word\\

\tablevspace
Floating & H is set afloat and docks onto a right-adjacent L-toned segment to form an HL contour tone & Vowel deletion and vowel merging & Adjacent function words\\

\tablevspace
Declination & H tones are progressively lowered across the utterance & (1) Downdrift\is{downdrift}: an H is lowered by a preceding L
\newline 
(2)      Downstep\is{downstep}: an H is lower in pitch than a left-adjacent H & Clause,\newline  sentence \\

\tablevspace
Deletion & The lexical tone is deleted and realised as L & (1) Derivation of compounds and reduplicants
\newline
(2)      Question boundary tone overrides lexical tone & (1) Phonological word
\newline 
(2)         Word\\
\lspbottomrule
\end{tabularx}
\end{table}

\subsection{Tone spreading}\label{sec:3.2.1}

H tones may spread to right-adjacent L-toned syllables within the word boundary. The H tone over the first syllable of \textit{prɔ́mis} ‘promise’ in \figref{fig:key:3.11} spreads to the second syllable.

\begin{figure}
\caption{H tone spreading}
\begin{tikzpicture}
    \draw (0, 0) node[inner sep=0] {
    \includegraphics[height=.3\textheight]{figures/yakpomod-img13.png}};
    \draw (0, -1.69) node[fill=white] {\footnotesize Yu bin prɔ́mis mí mɔní.};
\end{tikzpicture}
\label{fig:key:3.11}
\end{figure}
 


\ea%50
    \label{ex:key:50}
    \glll   Yu  bin  prɔ́mis  mí    mɔní.  →    Yu  bin  \textbf{prɔ́m\'is}  mí  mɔní.\\
\textsc{l}  \textsc{l}  \textsc{h.l}    \textsc{h}    \textsc{l.h}    {}    \textsc{l}  \textsc{l}  \textbf{\textsc{h.h}}    \textsc{h}  \textsc{l.h}\\
\textsc{2sg}  \textsc{pst}  promise  \textsc{1sg.indp}  money\\
\glt ‘You promised me money.’
\z

An environment that is particularly conducive to rightward tone spreading is when the L-toned syllable of a bisyllabic word with an H.L. pattern is hemmed in by the preceding H tone and the H tone of a following object. In \figref{fig:key:3.12}, the L-toned syllable of \textit{fínis} ‘finish’ is raised in pitch approximately to the level of the following object \textit{skúl} ‘school’. The pitch trace in \figref{fig:key:3.13} exemplifies the same process with \textit{vɔ́mit} ‘vomit’ and the following object \textit{chɔ́p} ‘food.

% TODO: put in a cuadrant

\begin{figure}
\caption{H tone spreading}
\label{fig:key:3.12}
\begin{tikzpicture}
    \draw (0, 0) node[inner sep=0] {
    \includegraphics[height=.3\textheight]{figures/yakpomod-img14.png}};
    \draw (0, -1.64) node[fill=white] {\footnotesize Wé a go fínis skúl, (...)};
\end{tikzpicture}
\end{figure}

\begin{figure}
\caption{H tone spreading}
\label{fig:key:3.13}
\begin{tikzpicture}
    \draw (0, 0) node[inner sep=0] {
    \includegraphics[height=.3\textheight]{figures/yakpomod-img15.png}};
    \draw (0, -1.64) node[fill=white] {\footnotesize E de vɔ́mit chɔ́p.};
\end{tikzpicture}
\end{figure}

\ea\label{ex:key:51}
\glll Wé  a    go  fínis    skúl,  \op...\cp{}  →    Wé  a  go  \textbf{f\'inís}  \textbf{skúl},  \op...\cp\\
\textsc{h}  \textsc{l}    \textsc{l}  \textsc{h.l}    \textsc{h}    {}  {}  \textsc{h}  \textsc{l}  \textsc{l}  \textbf{\textsc{h.h}}    \textbf{\textsc{h}}\\
\textsc{sub}  \textsc{1sg.sbj}  \textsc{pot}  finish  school\\
\glt  ‘When I finish school, (...)’\\
\z
\ea\label{ex:key:52}   
\glll E    de    vɔ́mit  chɔ́p.  →  E  de  \textbf{v\'ɔmít}  \textbf{chɔ́p}.\\
\textsc{l}    \textsc{l}    \textsc{h.l}    \textsc{h}    {}    \textsc{l}  \textsc{l}  \textbf{\textsc{h.h}  }  \textbf{\textsc{h}}\is{assimilation of segments}\\
\textsc{3sg.sbj}  \textsc{ipfv}    vomit  food\\
\glt  ‘He is vomiting (the) food.’\\
\z

A second phono-syntactic environment that favours rightward H tone spreading is a modifier-noun phrase. The L-toned syllable of a bisyllabic property item in prenominal position and with an H.L pattern may be raised to H if it is immediately followed by a noun with an initial (or only) H tone. An example for this process is provided in \REF{ex:key:58} further below. In the NP, the L-toned syllable of the modifier \textit{fúlis} ‘foolish’ is raised to an H tone because it is followed by the H-toned noun \textit{mán} ‘man’.

\subsection{Floating}\label{sec:3.2.2}

Pichi makes extensive use of floating boundary tones for the purpose of intonation. Aside from that, a lexical tone may be set afloat when two adjoining vowels merge or one of two adjoining vowels is deleted. Tone floating is particularly likely to occur in the contact zone between an H-toned high-frequency function word and a following L-toned vowel. In \figref{fig:key:3.14}, the final consonant /k/ of \textit{mék} ‘\textsc{sbjv}’ is deleted. This creates a vowel hiatus, which in turn leads to the deletion of the first, higher /e/ of \textit{mék} in favour of the second, lower vowel /à/. The rising-falling contour over \textit{mâ (mék=à)} is clearly visible. 


In \figref{fig:key:3.15}, the final segment of \textit{háw} ‘how’ is deleted and the lexical H tone is set afloat. The vowel merger between /a/ and the following low-toned dependent personal pronoun \textit{e} creates an HL contour tone.

% TODO: put in quadrant

\begin{figure}
\caption{Vowel deletion sets tone afloat}
\label{fig:key:3.14}
\begin{tikzpicture}
    \draw (0, 0) node[inner sep=0] {
    \includegraphics[height=.3\textheight]{figures/yakpomod-img16.png}};
    \draw (0, -1.64) node[fill=white] {\footnotesize Mék a tɛ́l yú di sáy.};
\end{tikzpicture}
\end{figure}

\begin{figure}    
\caption{Vowel merger sets tone afloat}
\label{fig:key:3.15} 
\begin{tikzpicture}
    \draw (0, 0) node[inner sep=0] {
    \includegraphics[height=.3\textheight]{figures/yakpomod-img17.png}};
    \draw (0, -1.64) node[fill=white] {\footnotesize Pút=an lɛk háw e bin dé!};
\end{tikzpicture}
\end{figure}

\ea\label{ex:key:53}
\glll Mék    a    tɛ́l  yú    di  sáy.    →   \textbf{Mâ}    tɛ́l  yú  di  sáy.\\
\textsc{h}    \textsc{l}    \textsc{h}  \textsc{h}    \textsc{l}  \textsc{h}  {}  \textbf{\textsc{hl}}    \textsc{h}  \textsc{h}  \textsc{l}  \textsc{h}\\
\textsc{sbjv}    \textsc{1sg.sbj}  tell  \textsc{2sg.indp}  \textsc{def}  side\\
\glt ‘Let me tell you the place.’\\
\z

\ea\label{ex:key:54}
\glll   Pút=an    lɛk  háw  e    bin  dé!  →  Pút=an  lɛk  \textbf{h\^ɛ}  bin  dé!\\
\textsc{h}=\textsc{l}    \textsc{l}  \textsc{h}  \textsc{l}    \textsc{l}  \textsc{h}    {}  \textsc{h}=\textsc{l}  \textsc{l}  \textbf{\textsc{hl}}  \textsc{l}  \textsc{h}\\
put=\textsc{3sg.obj}  like  how  \textsc{3sg.sbj}  \textsc{pst}  \textsc{be.loc}\\
\glt ‘Put it like it was!’\\
\z

\subsection{Downdrift and downstep}\label{sec:3.2.3}

Downdrift and downstep contribute to a general downward cline of pitch in utterances. An utterance normally begins with a high pitch onset and declines progressively with every lexical tone. Downdrift (indicated by ↓H) causes an H to be lowered by a preceding L tone as in \figref{fig:key:3.16}. The overall effect of downdrift is visible by the roughly equivalent pitch over the initial L-toned personal pronoun \textit{a} ‘\textsc{1sg.sbj}’ and the final H-toned noun \textit{hós} ‘house’.




\ea%55
    \label{ex:key:55}
    \glll   A    mítɔp=an  \textbf{yɛ́stadé}    na  in  \textbf{hós}.\\
\textsc{l}    \textbf{\textmd{\textsc{h}}}\textsc{.h=l}    \textsc{↓}\textbf{\textsc{h}}\textsc{.l.↓}\textbf{\textsc{h}}    \textsc{l}  \textsc{l}  \textsc{↓}\textbf{\textsc{h}}\is{downdrift} \\
\textsc{1sg.sbj}  meet=\textsc{3sg}  yesterday  \textsc{loc}  \textsc{3sg.poss}  house\\
\glt ‘I met him yesterday in his house.’    
\z


The second phenomenon involving declination is downstep\is{downstep} (indicated by –H). In a series of adjacent H tones, each tone may be lowered successively in relation to the preceding one. Downstep is exemplified  by the two successive homophones in \figref{fig:key:3.17} and the iteration in \figref{fig:key:3.18}. We also find downdrift in both examples.

% TODO: put in quadrant
 \clearpage
 
 \begin{figure}[p]
\caption{Downdrift}
\label{fig:key:3.16}
\scalebox{.9}{
\begin{tikzpicture}
    \draw (0, 0) node[inner sep=0] {
    \includegraphics[height=.3\textheight]{figures/yakpomod-img18.png}};
    \draw (0, -1.69) node[fill=white] {\footnotesize A mítɔp=an yɛ́stadé na in hós.};
\end{tikzpicture}
}
\end{figure}

\begin{figure}[p]
\caption{Downstep}
\label{fig:key:3.17}
\scalebox{.9}{
\begin{tikzpicture}
    \draw (0, 0) node[inner sep=0] {
    \includegraphics[height=.3\textheight]{figures/yakpomod-img19.png}};
    \draw (0, -1.64) node[fill=white] {\footnotesize Chɔ́p wé e dɔ́n dɔ́n.};
\end{tikzpicture}
}
\end{figure}

\begin{figure}[p]   
\caption{Downstep}
\label{fig:key:3.18}
\scalebox{.9}{
\begin{tikzpicture}
    \draw (0, 0) node[inner sep=0] {
    \includegraphics[height=.3\textheight]{figures/yakpomod-img20.png}};
    \draw (0, -1.64) node[fill=white] {\footnotesize Wáka sén sén sén.};
\end{tikzpicture}
}
\end{figure}

\clearpage 
\ea\label{ex:key:56}
\glll Chɔ́p  wé  e    \textbf{dɔ́n}  \textbf{dɔ́n}.\\
\textsc{h}    \textbf{\textsc{{}-h}}  \textsc{l}    \textsc{↓}\textsc{h}  \textbf{\textsc{{}-h}} \\
food    \textsc{sub}  \textsc{3sg.sbj}  \textsc{prf}  done\\
\glt ‘Food that is done.’
\z

\ea\label{ex:key:57}
\glll   Wáka  \textbf{sén}    \textbf{sén}    \textbf{sén}.\\
\textsc{h.l}    \textsc{↓}\textsc{h}    \textbf{\textsc{{}-h}}    \textbf{\textsc{{}-h}}\\
walk  same  \textsc{rep}    \textsc{rep}\\
\glt ‘Walk exactly in one line.’
\z

\subsection{Deletion}\label{sec:3.2.4}

Tone deletion occurs in two contexts. In compounds (including reduplications), the lexical H tone over the first component is deleted (also see \citealt{Yakpo2012}). The syllable whose tone has been deleted becomes L-toned. The second component retains its original tone pattern. Tone deletion therefore forms an intrinsic part of a derivational process in Pichi (cf. \sectref{sec:4.3}). The second context in which tone deletion occurs is when a boundary tone overrides the utterance-final lexical tone of a word (cf. \sectref{sec:3.4.4}).

\figref{fig:key:3.19} presents the pitch trace of an NP headed by the noun \textit{mán} ‘man’. The noun is modified prenominally by the verb \textit{fúlis} ‘(be) foolish’, which has an H.L tone pattern. The pitch of the utterance-final H tone over \textit{mán} stands at roughly the same level (albeit slightly downstepped and falling due to declarative intonation) as that of the preceding H tones over the first and second syllables of \textit{fúlis}. Note that the second, lexically L-toned syllable of \textit{fúlis} bears a phonetic H tone due to tonal plateauing (cf. \sectref{sec:3.2.1}).

\begin{figure}
\caption{Simplex noun}
\label{fig:key:3.19}
\begin{tikzpicture}
    \draw (0, 0) node[inner sep=0] {
    \includegraphics[height=.3\textheight]{figures/yakpomod-img21.png}};
    \draw (0, -1.64) node[fill=white] {\footnotesize Fúlis mán.};
\end{tikzpicture}
\end{figure}

\begin{figure}
\caption{Compound noun}
\label{fig:key:3.20}
\begin{tikzpicture}
    \draw (0, 0) node[inner sep=0] {
    \includegraphics[height=.3\textheight]{figures/yakpomod-img22.png}};
    \draw (0, -1.64) node[fill=white] {\footnotesize Mared-mán.};
\end{tikzpicture}
\end{figure}

\ea\label{ex:key:58}
\glll   Fúlis  mán.\\
\textsc{h.h}    \textsc{h}\\
foolish  man\\
\glt   ‘Foolish man.’ 
\z
\ea\label{ex:key:59}
\glll    \textbf{Mared}{}-mán.\\
\textbf{\textsc{l}}.\textsc{l-h}\\
marry.\textsc{cpd}{}-man\\
\glt ‘Married man.’
\z

In contrast, the pitch trace in \figref{fig:key:3.20} above exemplifies tone deletion. The head noun \textit{mán} ‘man’ is also modified by a verb with an H.L pattern, namely \textit{máred} ‘marry, be married’. However, \textit{máred} and \textit{mán} form a single phonological word, the compound noun \textit{mared-mán} ‘married man’. The H tone over the first syllable of \textit{máred} has been deleted in the process and replaced by L (the downward cline over the first syllable is caused by a pitch reset at the beginning of the utterance). At the same time, \textit{mán}, the final component of the compound, retains its H tone (which falls slightly due to its utterance-final position).

Reduplicated verbs show the same suprasegmental characteristics as compound nouns. The pitch trace of the reduplicated (and sentence-medial) monosyllabic \textit{rɔ́n} ‘run’ in \figref{fig:key:3.21} shows an L.H pitch configuration over the two identical components. This parallels the pitch trace over the compound \textit{wach-mán} ‘watchman’ above. Reduplication therefore involves the same derivational process as compounding. The lexical H-tone over the first component is deleted and replaced by an L tone.


\begin{figure}
\caption{Monosyllabic reduplicated verb}
\label{fig:key:3.21}
\begin{tikzpicture}
    \draw (0, 0) node[inner sep=0] {
    \includegraphics[height=.3\textheight]{figures/yakpomod-img23.png}};
    \draw (0, -1.64) node[fill=white] {\footnotesize (...) dí rɔn-rɔ́n (...)};
\end{tikzpicture}
\end{figure}

\begin{figure}
\caption{Bisyllabic reduplicated verb}
\label{fig:key:3.22} 
\begin{tikzpicture}
    \draw (0, 0) node[inner sep=0] {
    \includegraphics[height=.3\textheight]{figures/yakpomod-img24.png}};
    \draw (0, -1.64) node[fill=white] {\footnotesize (...) náw hala-hála.};
\end{tikzpicture}
\end{figure}
 
\ea%60
    \label{ex:key:60}
    \glll  \op...\cp{}  dí  \textbf{rɔn-}rɔ́n \op...\cp\\
{} \textsc{h}  \textbf{\textsc{l}}\textsc{{}-h}\\
{} this  \textsc{red.cpd-}run\\
\glt ‘(...) this running around (...)’
\z

\ea
	\label{ex:key:61}
\glll  \op...\cp{}  náw    \textbf{hala}{}-hála.\\
{} \textsc{h}    \textbf{\textsc{l}}\textstylePichiexamplenumberZchnZchn{.}\textstylePichiexamplenumberZchnZchn{\textsc{l}}\textsc{{}-h.h}\\
{} now    \textsc{red.cpd}{}-shout\\
\glt ‘(...) now, (it was) constant shouting.’\\
\z

\subsection{Pitch range expansion}\label{sec:3.2.5}

In Pichi, certain phonetic features may increase the prominence of a (series of) syllable(s). Segments may be lengthened or may be pronounced with increased volume, they may be pronounced with a breathy or creaky voice, and the speech rate may be slowed down or accelerated for stylistic effect. But there is no stress in Pichi in the sense of an automatic, metrically conditioned culmination of phonetic features as in intonation-only languages. Nor does Pichi make use of intonational melodies spanning the entire (or parts of the) utterance for the realisation of pragmatic functions, since these would override the lexical tone of individual words. Instead, pitch range expansion, and an extra-high tone in particular, are exploited to signal focus and emphasis. Focused or emphasised constituents may bear a higher than usual pitch, an extra-high tone on their H-toned syllable(s). The extra-high tone may spread rightwards onto following L-toned syllables until the word boundary is reached (cf. \sectref{sec:3.2.1}). 

\figref{fig:key:3.23} features the clefted verb \textit{drɔ́ngo} ‘be dead drunk’. In the pitch trace, the emphatic character of the predicate cleft construction \is{predicate cleft}is evident in two ways. The H-toned syllable of \textit{drɔ́ngo} bears an extra-high tone, and the segment /r/ is lengthened for emphasis. The utterance in \figref{fig:key:3.23} shades off into a chuckle from the fifth syllable onwards, which produces a wavering pitch trace:

\begin{figure}
\caption{Predicate cleft and extra-high tone for emphasis}
\label{fig:key:3.23}
\begin{tikzpicture}
    \draw (0, 0) node[inner sep=0] {
    \includegraphics[height=.3\textheight]{figures/yakpomod-img25.png}};
    \draw (0, -1.69) node[fill=white] {\footnotesize Na [drrrɔ́ngò] yu dɔ́n drɔ́ngo};
\end{tikzpicture}
\end{figure}
 


\ea%62
    \label{ex:key:62}
    \glll   Na  [\textbf{drrrɔ́ngò}]    yu  dɔ́n  drɔ́ngo. \\
\textsc{l}  \textbf{\textsc{+h}}\textsc{.l}        \textsc{l}  \textsc{h}  \textsc{h.l}\\
\textsc{foc}  be.dead.drunk  \textsc{2sg}  \textsc{prf}  be.dead.drunk\\
\glt ‘You’re absolutely dead drunk.’     
\z

Elements that fulfil central functions in pragmatically marked contexts are particularly common with extra-high tone, e.g. question elements like \textit{háw} ‘how’, \textit{wétin} ‘what’, \textit{údat} ‘who’, \textit{ús=tín}  ‘what’, the negator \textit{nó}\is{negation}, modifications of degree via repetition\is{repetition} like \textit{bíg bíg} ‘very big’, and the degree adverb \textit{bád} ‘bad, extremely’. Both components of the repetition \textit{bíg bíg} ‘be very big’ in \figref{fig:key:3.24}  carry an extra-high tone. There is no sign of downstep\is{downstep} within the reduplicated sequence: 

\begin{figure}
\caption{Extra-high tone}
\label{fig:key:3.24}
\begin{tikzpicture}
    \draw (0, 0) node[inner sep=0] {
    \includegraphics[height=.3\textheight]{figures/yakpomod-img26.png}};
    \draw (0, -1.69) node[fill=white] {\footnotesize Dɛn bíl=an sɔn bíg bíg hós.};
\end{tikzpicture}
\end{figure}
 


\ea%63
    \label{ex:key:63}
    \glll   Dɛn    bíl=an    sɔn    \textbf{bíg}  \textbf{bíg}    hós.\\
\textsc{h}    \textsc{h=l}      \textsc{l}    \textbf{\textsc{+h}}  \textbf{\textsc{+h}}    \textsc{h}\\
\textsc{3pl}    build=\textsc{3sg.obj}  some  big  \textsc{rep}    house\\
\glt ‘They built him a huge house.’    
\z

Entire clauses or sentences may also be placed under focus \is{prosodic focus} by (a series of) extra-high tones, which thereby (cumulatively) fulfil(s) the same function as emphatic intonation covered in \sectref{sec:3.4.2} further below. There are two principal means of emphasising sentences, which are often used together. The last H tone of the utterance may be raised to an extra-high pitch as in \figref{fig:key:3.25}. Here the H tone of the utterance-final word \textit{mán} ‘man’ has been raised to an extra-high level. The sentence nonetheless bears declarative intonation. The word \textit{mán} still exhibits the utterance-final fall characteristic of declarative intonation (cf. \sectref{sec:3.4.1}) but at a significantly higher pitch level than in a non-emphatic context \REF{ex:key:64}.

\begin{figure}
\caption{Utterance-final extra-high tone for emphasis\is{emphasis}}
\label{fig:key:3.25}
\begin{tikzpicture}
    \draw (0, 0) node[inner sep=0] {
    \includegraphics[height=.3\textheight]{figures/yakpomod-img27.png}};
    \draw (0, -1.69) node[fill=white] {\footnotesize Yu húman de mék jɔmba wet mi mán.};
\end{tikzpicture}
\end{figure}
 


\ea%64
    \label{ex:key:64}
    \glll   Yu  húman  de  mék    jɔmba  wet  mi    \textbf{mán}.\\
\textsc{l}  \textsc{h.l}    \textsc{l}  \textsc{h}    \textsc{l.l}    \textsc{l}  \textsc{l}    \textbf{\textsc{+h}}\textsc{l\%}\\
\textsc{2sg}  woman  \textsc{ipfv}  make  lover  with  \textsc{1sg.poss}  man.\\
\glt ‘Your wife makes love with my husband.’
\z

Secondly, the use of an utterance-final extra-high tone is often accompanied by “pitch range expansion” \citep[276]{Yip2002}. Alternatively, pitch range expansion may be accompanied by the use of the emphatic boundary tone instead of the utterance-final extra-high tone (cf. \sectref{sec:3.4.2}). During pitch range expansion, the pitch range between H and L tones is widened throughout the entire utterance by pronouncing H tones with a higher-than-usual pitch and, optionally, L tones with a lower-than-usual pitch. This creates a strongly undulating pitch contour over the entire utterance.

\figref{fig:key:3.26} graphically depicts the dramatic rises and falls that may characterise pitch range expansion. The female speaker begins with an L-toned \textit{na} at 190 Hz, rises to 490 Hz with H-toned \textit{só}, then falls to an all-time low with \textit{dɛn} at 145 Hz, until the pitch range gradually evens out towards the end of the utterance \REF{ex:key:65}.

\begin{figure}
\caption{Pitch range expansion for emphasis\is{emphasis}}
\label{fig:key:3.26}
\begin{tikzpicture}
    \draw (0, 0) node[inner sep=0] {
    \includegraphics[height=.3\textheight]{figures/yakpomod-img28.png}};
    \draw (0, -1.64) node[fill=white] {\footnotesize Na só dɛn de tɔ́k=an.};
\end{tikzpicture}
\end{figure}
 


\ea%65
    \label{ex:key:65}
    \glll   Na  \textbf{só}    \textbf{dɛn}  \textbf{de}  \textbf{tɔ́k}=an.    \\
\textsc{l}  \textbf{\textsc{+h}}    \textbf{\textsc{+l}}  \textbf{\textsc{+l}}  \textbf{\textsc{+h}}\textsc{=lh\%}      \\
\textsc{foc}  like.that  \textsc{3pl}  \textsc{ipfv}  talk=\textsc{3sg.obj}\\
\glt ‘That’s how they say it.’
\z

\section{Tone-conditioned suppletive allomorphy}\label{sec:3.3}

Pichi features a tone-conditioned suppletive allomorphy (TCSA) of the two pronominal variants \textit{=an} \textsc{‘3sg.obj’} and \textit{ín} ‘\textsc{3sg.indp’}, which may both instantiate (direct and indirect) object case (cf. \sectref{sec:5.4.1} for an overview of the inflection of personal pronouns). Suppletive allomorphy is conditioned by a tonotactic prohibition of immediately adjoining or “string-adjacent” \citep{Suzuki1998} identical tones (cf. also \sectref{sec:2.6.2.2}). Suppletive allomorphy therefore relies on the conditioning environment of vowel hiatus. Further, there are no phonemic long vowels in Pichi. String-adjacent vowels within the same lexical word are always heterosyllabic, and in addition, invariably carry polar tones (cf. \sectref{sec:2.6.2.2}). TCSA can therefore only be triggered when the enclisis of \textit{=an} \textsc{‘3sg.obj’} creates a phonological word. A head with an L-toned vowel-final syllable may therefore not take the vowel-initial L-toned clitic object pronoun \textit{=an}. Instead, the independent (emphatic) personal pronoun \textit{ín} \textsc{‘3sg.indp’} is recruited as a suppletive allomorph. Allomorph distribution according to the phonological class of the host is summarised in \tabref{tab:key:3.4}.

%%please move \begin{table} just above \begin{tabular
\begin{table}
\caption{Distribution of suppletive object pronouns}
\label{tab:key:3.4}

\begin{tabularx}{.8\textwidth}{XXl}
\lsptoprule
{Host class} & {Allomorph} & {Example}\\
\midrule
C/{\longrule}\# & =an & [márè\textbf{d}=àn]\\
\'{V}/{\longrule}\# & =an & [tròw\textbf{é}=àn]\\
\`{V}/{\longrule}\# & ín & [fíb\textbf{à} ín]\\
\lspbottomrule
\end{tabularx}
\end{table}

There is no tonotactic restriction on the enclisis of \textit{=an} with consonant-final hosts like \textit{máred} ‘marry’, since the condition of tonal string-adjacency is not met~\REF{ex:key:66}.


\ea%66
    \label{ex:key:66}
    \gll   E    go  máre\textbf{d}=\textbf{an}.\\
\textsc{3sg.sbj}  \textsc{pot}  marry=\textsc{3sg.obj}\\

\glt ‘S/he’ll marry him/her.’ 
\z

There are no restrictions on the enclisis of \textit{=an} with vowel-final hosts with a word-final H-tone like \textit{trowé} ‘throw, pour away’, since the vowel sequence across the morpheme boundary bears a polar [H.L] tone:


\ea%67
    \label{ex:key:67}
    \gll   A    fít  ték    di  wɔtá  a    trow\textbf{é}=\textbf{an}.\\
\textsc{1sg.sbj}  can  take    \textsc{def}  water  \textsc{1sg.sbj}  throw=\textsc{3sg.obj}\\

\glt ‘I can take the water (and) pour it away.’ 
\z

If the word-final vowel of the host is L-toned, as with \textit{fíba} ‘resemble’, the pitch configuration after enclisis of \textit{=an} across the clitic boundary would be [L.L]. This is an illicit pitch configuration over string-adjacent vowels in Pichi phonological words and triggers the use of suppletive \textit{ín} ‘\textsc{3sg.indp’.} Compare the following two examples:


\ea[*]{%68
    \label{ex:key:68}
    \gll   Yu    \textbf{fíba}=\textbf{an}    bɔkú.\\
\textsc{2sg}  resemble=\textsc{3sg.obj}  a.lot\\
\glt  Intended: ‘You resemble him/her a lot.’ \\
}
\z


\ea%69
    \label{ex:key:69}
    \gll   Yu  \textbf{fíba}      \textbf{ín}    bɔ́ku.\\
\textsc{2sg}  resemble    \textsc{3sg.indp}  a.lot\\

\glt ‘You resemble him/her a lot.’ 
\z

The class of words that features the allomorph \textit{ín} as an object pronoun also includes verbs of Spanish origin. \ili{Spanish} verbs are always inserted into Pichi clauses in the Spanish \textsc{3sg} present tense form, irrespective of their tense-aspect (cf. \sectref{sec:13.2.2}). Examples follow with the verbs \textit{fírma} ‘sign’ (< Span. \textit{firmar}) from the Spanish 1\textsuperscript{st} conjugation class, and \textit{sube} ‘go/bring up’ (< Span. \textit{subir}) from the 3\textsuperscript{rd} conjugation class:


\ea%70
    \label{ex:key:70}
    \gll   Dɛn    nó    \textbf{fírma}  \textbf{ín}    yét.    \\
\textsc{3pl}    \textsc{neg}    sign    \textsc{3sg.indp}  yet\\


\glt ‘They haven't signed it yet.’\\

\z

\ea%71
    \label{ex:key:71}
    \gll   Dán    mán    go  \textbf{súbe}  \textbf{ín}.\\
that    man    \textsc{pot}  bring.up  \textsc{3sg.indp}\\

\glt ‘That man will bring it [the suitcase] up.’
\z

Pichi has a second mechanism next to tone-conditioned suppletive allomorphy to ensure that the requirement of a string-adjacent polar [H.L] tone is not breached. A buffer consonant /r/ can be inserted at the clitic boundary. Epenthesis forestalls the cross-morphemic vowel hiatus and makes the use of the allomorph \textit{ín} unnecessary:


\ea%72
    \label{ex:key:72}
    \gll   Yu  fíba[\textbf{r}]=\textbf{an}    bɔkú.\\
\textsc{2sg}  resemble=\textsc{3sg.obj}  a.lot\\

\glt ‘You resemble him a lot.’ 
\z

Once the epenthetic segment is present, there is no phonotactic difference with a word in which the final consonant forms an integral part of the root like \textit{máred} ‘marry’ in \REF{ex:key:66}. Another example featuring epenthesis follows, involving the general associative preposition \textit{fɔ} ‘\textsc{prep}’. In \REF{ex:key:73}, we find /r/ epenthesis, in \REF{ex:key:74}, suppletive allomorphy:


\ea%73
    \label{ex:key:73}
    \gll   E    tót=an    fɔ[\textbf{r}]=\textbf{an}.\\
\textsc{3sg.sbj}  carry=\textsc{3sg.obj}  \textsc{prep=3sg.obj}\\

\glt ‘He carried it for her.’
\z


\ea%74
    \label{ex:key:74}
    \gll   Dán    tín    dé    \textbf{fɔ}  \textbf{ín}.\\
that    thing  \textsc{be.loc}  \textsc{prep}  \textsc{3sg.indp}\\

\glt ‘That thing is hers.’
\z

Three aspects are noteworthy with respect to /r/ epenthesis in Pichi. Firstly, /r/ insertion is exceedingly rare in natural discourse. In the Pichi corpus, there are less than a dozen instances of /r/ epenthesis in natural discourse, involving a mere handful of lexemes, among them \textit{kɔ́ba[r]=an} ‘cover it’, \textit{klía[r]=an} ‘clear it’, \textit{fía[r]=an} ‘fear him/her’, \textit{fíba[r]=an} ‘resemble him/her, \textit{drɔ́ngo[r]=an} ‘get him/her drunk’, and \textit{fɔ[r]=an} ‘for him/her’. By contrast, the corpus contains hundreds of syntagmas involving the suppletive allomorph \textit{ín.} I could therefore only uncover the distribution of the epenthetic /r/ and its role in TCSA by means of elicitation. Secondly, elicitation revealed that the availability of /r/ epenthesis is subject to considerable idiolectal variation. For some speakers, the use of epenthesis with many verbs is not acceptable, i.e. *\textit{fála[r]=an} ‘follow him/her’, for others it is. All speakers, however, accepted TCSA with all verbs and prepositions, whether belonging to the native Pichi or the non-native Spanish lexical layer. 


The third aspect of interest is that /r/ epenthesis is ungrammatical with Spanish derived verbs, cf. \REF{ex:key:75}. Epenthesis is limited to the native layer of the Pichi vocabulary, thus excluding inserted Spanish verbs from the application of /r/ epenthesis, and limiting them to TCSA alone, hence the ungrammaticality of the following example.



\ea[*]{%75
    \label{ex:key:75}
    \gll   Yu  gɛ́t    fɔ  fírma[\textbf{r}]=\textbf{an}.\\
 \textsc{2sg}  get    \textsc{prep}  sign=\textsc{3sg.obj}\\
\glt Intended: ‘You have to sign it.’
}\z

Pichi words with a word-final L-toned /ì/, e.g. \textit{wɔ́ri} ‘worry’, merit some attention in the context of epenthesis. Such words exhibit the conditioning feature but neither trigger /r/ epenthesis nor TCSA, compare the ungrammatical sentences \REF{ex:key:76} and \REF{ex:key:77}. Other verbs in this group are \textit{sɔ́ri} ‘feel sorry’, \textit{grídi} ‘be greedy’, \textit{hángri} ‘be hungry’, \textit{lési} ‘be lazy’, and \textit{tɔ́sti} ‘be thirsty’. 


\ea[*]{%76
    \label{ex:key:76}
    \gll   Dɛn  wɔ́ri[\textbf{r}]=\textbf{an}    bɔkú. \\
 \textsc{3pl}    worry=\textsc{3sg.obj}    much\\
\glt Intended: ‘They worried him a lot.’
}\z


\ea[*]{%77
    \label{ex:key:77}
    \gll   Dɛn  wɔ́ri    í\textbf{n}    bɔkú.  \\
\textsc{3pl}    worry  \textsc{3sg.indp}  much\\
\glt Intended: ‘They worried him a lot.’ 
}\z

Instead, a word-final nasal /n/ appears at the clitic boundary, thus avoiding the LL vowel hiatus that should trigger suppletive allomorphy, as in \REF{ex:key:78}:


\ea%78
    \label{ex:key:78}
    \gll   Di  tín    sɔ́rin=\textbf{an}         bɔkú.   \\
\textsc{def}  thing  make.sorry=\textsc{3sg.obj}  much\\

\glt ‘This made her feel very sorry.’
\z

Outside of the clitic environment, the wordfinal /ì/ in these words may, but need not be pronounced as a nasalised vowel, as shown in the phonetic transcription in \REF{ex:key:79}:


\ea%79
    \label{ex:key:79}
    \gll   A    sɔ́ri [\textbf{sɔ́r\`{ĩ}}]  sé    e    kíl  di  dɔ́g.\\
\textsc{1sg.sbj}  feel.sorry  {}   \textsc{quot}    \textsc{3sg.sbj}  kill  \textsc{def}  dog\\

\glt ‘I felt sorry that she killed the dog.’
\z

The word-final /n/ in examples like \REF{ex:key:79} is therefore not epenthetic. It is morphologically affiliated to the verbal root and is realised in the clitic environment. The word-final /n/ in verbs like \textit{sɔ́ri} (group 1) has been constructed by analogy with words like \textit{físin} ‘(to) fish’, \textit{hɔ́ntin} ‘(to) hunt’, \textit{mɔ́nin} ‘morning’, \textit{ívin} ‘evening’, and \textit{plantí} ‘plantain’ (group 2). The construction of a word-final /n/ in group 1 words probably occurred in response to the ban on string-adjacent identical tones in the context of cliticisation.

\section{Intonation}\label{sec:3.4}

The functions of intonation are realised by sentence-final particles and utterance-final boundary tones. Pichi boundary tones are floating tones, which are inserted at the right edge of an utterance. These boundary tones serve pragmatic functions by differentiating sentence types, such as declaratives from questions. They also fulfil grammatical functions by linking clauses.


Four boundary tones and contours, represented by <\%> \citep{Pierrehumbert1980}, were identified in the corpus. Their functions with declaratives and questions are summarised in \tabref{tab:key:3.5} (cf. \citealt[18–20]{Cristo1998}).


%%please move \begin{table} just above \begin{tabular
\begin{table}
\caption{Utterance type and boundary tones}
\label{tab:key:3.5}

\begin{tabularx}{.8\textwidth}{XXl}
\lsptoprule

{Boundary tone} & {Declaratives} & {Questions}\\
\midrule
L\% & Non-emphatic & Content\\
LH\% (additive) & Emphatic & {}---\\
& List & {}---\\
${\emptyset}$\% (no tone) & Continuative & {}---\\
& Emphatic & {}---\\
LH\% (substitutive) & {}--- & Yes-no\\
\lspbottomrule
\end{tabularx}
\end{table}

A boundary (contour) tone (henceforth only “boundary tone”) associates with the last syllable of an utterance. A boundary tone (BT) may either form a contour by itself (e.g. question intonation) or arise if the lexical tone (LT) of the utterance-final syllable is polar to the following BT. Otherwise, a BT produces a fall or a level tone over the utterance-final syllable.

\tabref{tab:key:3.6} shows how LTs and BTs interact. The leftmost column contains the word-final LT over the last syllable of the utterance. The top row contains the relevant BT. The boxes in the table contain the (contour) tones over the utterance-final syllable that result from the interaction of LT and BT. These tones represent the phonetic output, the way the tone is actually pronounced. Some of these output tones are level tones, others are contour tones of varying complexity.

%%please move \begin{table} just above \begin{tabular
\begin{table}
\caption{Interaction of lexical tones and boundary tones}
\label{tab:key:3.6}
\small
\begin{tabularx}{\textwidth}{lp{2cm}lXXX}
\lsptoprule

LT/BT & Example & Declarative

L\% & Emphatic LH\% & Cont./Emph.

${\emptyset}$\% & Question

LH\% \\
\midrule
\MakeUppercase{l} & \textit{dɛn} \textsc{‘3pl’} \newline \textit{Píchi} ‘Pichi’ 
\newline 
\MakeUppercase{l}    \MakeUppercase{h.l} & L (fall) & \MakeUppercase{LH} & L (level) & \MakeUppercase{lh}\\

\tablevspace
\MakeUppercase{h} & \textit{gó} ‘go’ \newline \textit{pikín}\textstyleTableEnglishZchn{ ‘child’}
\newline 
 \MakeUppercase{h}    \MakeUppercase{l.h} & HL & \MakeUppercase{hlh} & H & \MakeUppercase{lh}\\


\tablevspace
\MakeUppercase{h} & \textit{bɔbí} \textstyleTableEnglishZchn{‘breast’}
\newline 
\MakeUppercase{l.h} & H & \MakeUppercase{hlh} & H & \MakeUppercase{lh}\\
\lspbottomrule
\end{tabularx}
\end{table}

LTs are not overridden by BTs save in one instance. In yes-no questions\is{yes-no questions}, the utterance-final LT is deleted and replaced by the question boundary contour tone. This is why the rightmost column in \tabref{tab:key:3.6} features the same LH\% boundary tone in the utterance-final position with all tone classes. 

\subsection{Declarative intonation}\label{sec:3.4.1}

Non-emphatic declaratives feature an L\%, which is also found on the right edge of the citation form of words. The declarative L\% causes an utterance-final fall to the bottom of the pitch register. Compare the word-final L-toned syllable of \textit{kɔ́ntri} ‘country’ in \figref{fig:key:3.27}.

\begin{figure}
\caption{Declarative L\% over H.L word}
\label{fig:key:3.27}
\begin{tikzpicture}
    \draw (0, 0) node[inner sep=0] {
    \includegraphics[height=.3\textheight]{figures/yakpomod-img29.png}};
    \draw (0, -1.69) node[fill=white] {\footnotesize A bin wánt kɔmɔ́t na dís kɔ́ntri.};
\end{tikzpicture}
\end{figure}

%%[Warning: Draw object ignored]
 
\ea%80
    \label{ex:key:80}
    \glll   A    bin  wánt  kɔmɔ́t  na  dís  kɔ́ntri.\\
\textsc{l}    \textsc{l}  \textsc{h}    \textsc{l.h}    \textsc{l}  \textsc{h}  \textsc{h.l}\textbf{\textsc{l\%}}\\
\textsc{1sg.sbj}  \textsc{pst}  want  go.out  \textsc{loc}  this  country\\
\glt ‘I wanted to leave this country.’
\z

In contrast, polysyllabic vowel-final words with a final lexical H tone do not usually feature an utterance-final fall in non-emphatic declaratives. They retain their word-final H tone. Compare \textit{bɔbí} ‘breast’ in \figref{fig:key:3.28}.

\begin{figure}
\caption{Unpronounced declarative L\% over L.H word}
\label{fig:key:3.28}
\begin{tikzpicture}
    \draw (0, 0) node[inner sep=0] {
    \includegraphics[height=.3\textheight]{figures/yakpomod-img30.png}};
    \draw (0, -1.69) node[fill=white] {\footnotesize A de gí=an bɔbí.};
\end{tikzpicture}
\end{figure}
 


\ea%81
    \label{ex:key:81}
    \glll   \MakeUppercase{A}   de    gí=an    bɔbí.\\
\textsc{l}    \textsc{l}    \textsc{h=l}      \textsc{l.}\textbf{\textsc{h}}\\
\textsc{1sg.sbj}  \textsc{ipfv}    give  =\textsc{3sg.obj}  breast\\
\glt ‘I’m breast-feeding her.’  
\z

Content questions\is{content questions} feature the same boundary tone as declaratives. Compare the utterance-final fall over the monosyllable in \figref{fig:key:3.29}.

\begin{figure}
\caption{L\% with content question}
\label{fig:key:3.29}
\begin{tikzpicture}
    \draw (0, 0) node[inner sep=0] {
    \includegraphics[height=.3\textheight]{figures/yakpomod-img31.png}};
    \draw (0, -1.69) node[fill=white] {\footnotesize Wétin mék dán wán?};
\end{tikzpicture}
\end{figure}


\ea%82
    \label{ex:key:82}
    \glll   Wétin  mék    dán    \textbf{wán}?  \\
\textsc{h.l}    \textsc{h}    \textsc{h}    \textbf{\textsc{hl\%}}      \\
what  make  that    one\\
\glt ‘What causes this?’\is{declarative intonation}
\z


\newpage 
\subsection{Emphatic intonation}\label{sec:3.4.2}

Emphatic intonation expresses meanings like extra-emphasis, insistence, impatience or reproach. There are two ways of signalling emphasis\is{emphasis} at the sentence level in Pichi\is{emphatic stress}. One way involves the use of the emphatic LH\% boundary tone. A second way involves the use of pitch range expansion (cf. \sectref{sec:3.2.5}).


The emphatic LH\% is an additive contour tone. It succeeds the  lexical tone of the utterance-final syllable, which may therefore count up to three beats in length. Additionally, the last lexical H before the LH\% boundary contour tone is often pronounced with an extra-high tone due to emphasis. This peculiar combination of an extra-high lexical tone and a contour boundary tone creates a highly perceptible utterance-final tonal melody. 



Phonemically, an utterance-final L to which the emphatic LH\% boundary tone associates bears an LHH sequence of tones. Phonetically, the utterance-final syllable is realised as an LH contour. \figref{fig:key:3.30} depicts the utterance-final rise over the L-toned monosyllable \textit{=an} ‘\textsc{3sg.obj’}. 


\begin{figure}
\caption{Emphatic LH\% over L-final word}
\label{fig:key:3.30}
\scalebox{.9}{
\begin{tikzpicture}
    \draw (0, 0) node[inner sep=0] {
    \includegraphics[height=.3\textheight]{figures/yakpomod-img32.png}};
    \draw (0, -1.69) node[fill=white] {\footnotesize E de lé=an.};
\end{tikzpicture}
}
\end{figure}


\ea%83
    \label{ex:key:83}
    \glll   E    de  lé=\textbf{an}.\\
\textsc{l}    \textsc{l}  \textsc{h=}\textbf{\textsc{lh\%}}\\
\textsc{3sg.sbj}  \textsc{ipfv}    lay=\textsc{3sg.obj}\\
\glt ‘She is laying it (on the table).’
\z

When the emphatic boundary tone links with an utterance-final H-toned syllable the resulting contour features an initial rise, an intermediate fall, and a final rise. The utterance-final, extensively lengthened syllable thus bears an HLH contour. Compare the utterance-final H-toned monosyllables \textit{ín} ‘\textsc{3sg.indp}’ and \textit{gó} ‘go’ in \figref{fig:key:3.31} and \figref{fig:key:3.32}.

% TODO: put in quadrant

\begin{figure}
\scalebox{.9}{
\caption{Emphatic LH\% over H-final word}
\label{fig:key:3.31}
\begin{tikzpicture}
    \draw (0, 0) node[inner sep=0] {
    \includegraphics[height=.3\textheight]{figures/yakpomod-img33.png}};
    \draw (0, -1.64) node[fill=white] {\footnotesize Na ín.};
\end{tikzpicture}
}
\end{figure}

\begin{figure}
\caption{Emphatic LH\% over H-final word}
\label{fig:key:3.32}
\scalebox{.9}{
\begin{tikzpicture}
    \draw (0, 0) node[inner sep=0] {
    \includegraphics[height=.3\textheight]{figures/yakpomod-img34.png}};
    \draw (0, -1.64) node[fill=white] {\footnotesize A go gó.};
\end{tikzpicture}
}
\end{figure}  

\ea\label{ex:key:84}
\glll Na  \textbf{ín}.\\
\textsc{l}  \textbf{\textsc{h}}\textbf{\textsc{lh\%}}\\
\textsc{foc}  \textsc{3sg.indp}\\
\glt   ‘That’s it [you should know that].’
\z
\ea\label{ex:key:85}
\glll    A    go  \textbf{gó}.  \\
\textsc{l}    \textsc{l}  \textbf{\textsc{h}}\textbf{\textsc{lh\%}}\\
\textsc{1sg.sbj}  \textsc{pot}  go\\
\glt   ‘I’ll go [you don’t need to remind me to].’ 
\z

\largerpage
An utterance-final, H-toned syllable of a polysyllabic word also bears this contour. Compare \textit{bɔbí} ‘breast’ and \textit{chukchúk} ‘thorn’ in \figref{fig:key:3.33} and \figref{fig:key:3.34}. The two words were pronounced with emphatic intonation during vocabulary elicitation because the speaker expected me to be familiar with them.

\begin{figure}
\caption{H\% over vowel-final L.H word}
\label{fig:key:3.33}
\scalebox{.9}{
\begin{tikzpicture}
    \draw (0, 0) node[inner sep=0] {
    \includegraphics[height=.3\textheight]{figures/yakpomod-img35.png}};
    \draw (0, -1.64) node[fill=white] {\footnotesize Bɔbí.};
\end{tikzpicture}
}
\end{figure}


\begin{figure}
\caption{H\% over obstruent-final L.H word}
\label{fig:key:3.34}
\scalebox{.9}{
\begin{tikzpicture}
    \draw (0, 0) node[inner sep=0] {
    \includegraphics[height=.3\textheight]{figures/yakpomod-img36.png}};
    \draw (0, -1.64) node[fill=white] {\footnotesize Na  chukchúk.};
\end{tikzpicture}
}
\end{figure}

\clearpage 
\ea%86
    \label{ex:key:86}
    \glll   Bɔbí.\\
\textsc{l.}\textbf{\textsc{h}}\textbf{\textsc{lh\%}}\\
breast\\
\glt ‘Breast [that’s self-evident!].’ 
\z

\ea
\label{ex:key:87}
\glll Na  chuk\textbf{chúk}.\\
\textsc{l}  \textsc{l.}\textbf{\textsc{hlh\%}}\\
\textsc{foc}  thorn\\
\glt ‘It’s a thorn [that’s self-evident].’\is{emphatic intonation}
\z

The LH\% boundary contour tone is a loan\is{loan intonation} from (Equatoguinean and, ultimately, European) \ili{Spanish} together with the meanings associated with it. The LH\% contour boundary tone is also employed for list intonation \is{list intonation}(cf. \sectref{sec:3.4.3}). \figref{fig:key:3.35} presents the pitch trace of an utterance in Equatoguinean Spanish.


Compare the contour over the utterance-final L-toned syllable with that borne by the utterance-final L-toned syllable in \figref{fig:key:3.30}. Also compare the emphatic contour over the phonologically independent \textit{sí} ‘yes’ with that of the high-toned \textit{ín} ‘\textsc{3sg.indp}’ in \figref{fig:key:3.31}.


\begin{figure}
\caption{Emphatic intonation in European Spanish}
\label{fig:key:3.35}
\begin{tikzpicture}
    \draw (0, 0) node[inner sep=0] {
    \includegraphics[height=.3\textheight]{figures/yakpomod-img37.png}};
    \draw (0, -1.69) node[fill=white] {\footnotesize Sí vengo.};
\end{tikzpicture}
\end{figure}
 


\ea%88
    \label{ex:key:88}
    \glll   \textbf{Sí}    \textbf{vengo}.\\
\textsc{h}\textbf{\textsc{lh\%}}  \textsc{h.}\textbf{\textsc{lh\%}}\\
yes    I.come\\
\glt ‘Yes [you should know that!], I’ll come.’
\z

\subsection{List intonation}\label{sec:3.4.3}

The additive LH\% boundary tone employed for emphatic intonation is also used for list intonation. As in emphatic declaratives, LH\% associates with the final syllable and creates an LH contour over an utterance-final L-toned syllable and an HLH contour over an utterance-final H-toned syllable. The same intonation contour is once more found in Equatoguinean (and European) Spanish with a similar range of meanings.


The following three pitch traces form part of a list. Take note of the LH contour over the L-toned dependent pronoun \textit{dɛn} ‘\textsc{3pl}’ before the short pause, as well as the LH contour borne by the L-toned final syllable of \textit{manicura} ‘manicure’ in \figref{fig:key:3.36} and \textit{chía} ‘chair’ in \figref{fig:key:3.37}. Compare this with the declarative L\% over \textit{dé} ‘there’, the closing sentence of the list in \figref{fig:key:3.38}.


\begin{figure}[b]
\caption{List intonation}
\label{fig:key:3.36}
\begin{tikzpicture}
    \draw (0, 0) node[inner sep=0] {
    \includegraphics[height=.3\textheight]{figures/yakpomod-img38.png}};
    \draw (0, -1.69) node[fill=white] {\footnotesize A de mék fínga dɛn, manicura, (...)};
\end{tikzpicture}
\end{figure}


\ea%89
    \label{ex:key:89}
    \glll   A    de  mék    fínga  \textbf{dɛn},    manicu\textbf{ra},  \op...\cp\\
\textsc{l}    \textsc{l}  \textsc{h}    \textsc{h.l}    \textbf{\textsc{lh\%}}    \textsc{l.l.}\textbf{\textsc{h.lh\%}}\\
\textsc{1sg.sbj}  \textsc{ipfv}  make  finger  \textsc{pl}    manicure\\
\glt ‘I was making fingers, manicure, (...)’
\z

\begin{figure}
\caption{List intonation}
\label{fig:key:3.37}
\begin{tikzpicture}
    \draw (0, 0) node[inner sep=0] {
    \includegraphics[height=.3\textheight]{figures/yakpomod-img39.png}};
    \draw (0, -1.69) node[fill=white] {\footnotesize (...) a de mék tapete dɛn fɔ chía, (...)};
\end{tikzpicture}
\end{figure}


\ea%90
    \label{ex:key:90}
    \glll   \op...\cp{}  a    de  mék    tapete  dɛn  fɔ  \textbf{chía}, \op...\cp\\
{}  \textsc{l}    \textsc{l}  \textsc{h}    \textsc{l.h.l}    \textsc{l}  \textsc{l}  \textsc{h.}\textbf{\textsc{lh\%}}\\
{}  \textsc{1sg.sbj}  \textsc{ipfv}  make  cloth  \textsc{pl}  \textsc{prep}  chair\\
\glt ‘(...)  I was making chair-drapings, (...)’
\z

\begin{figure}
\caption{Declarative L\% over final item in list\is{declarative intonation}}
\label{fig:key:3.38}
\begin{tikzpicture}
    \draw (0, 0) node[inner sep=0] {
    \includegraphics[height=.3\textheight]{figures/yakpomod-img40.png}};
    \draw (0, -1.69) node[fill=white] {\footnotesize (...) só a bin dé gɛ́t mí mɔní dé.};
\end{tikzpicture}
\end{figure}



\ea%91
    \label{ex:key:91}
\glll   \op...\cp{}  só  a    bin  dé  gɛ́t  mí    mɔní  \textbf{dé}.\\
{}  \textsc{h}  \textsc{l}    \textsc{l}  \textsc{l}  \textsc{h}  \textsc{l}    \textsc{l.h}    \textbf{\textsc{hl\%}}\\
{}  so  \textsc{1sg.sbj}  \textsc{pst}  \textsc{ipfv}  get  \textsc{1sg.poss}  money  there\\
\glt ‘(...) so I was getting my money there.’\is{list intonation}
\z


\subsection{Continuative intonation}\label{sec:3.4.4}

The absence of a boundary tone, usually before a prosodic break (a brief but audible pause), signals continuative intonation. With continuative intonation, the lexical tone of the relevant syllable simply maintains its pitch and is therefore pronounced with the same pitch as it would in utterance-medial position. Continuative intonation functions as a floor-holding device, a juncture marker on the right edge of utterances in order to prepare the ground for following material. Continuative intonation therefore plays an important role in signalling topic and focus next to the particles employed for this purpose (cf. \sectref{sec:7.4}). 


In \figref{fig:key:3.39}, the topical \textsc{NP} \textit{mi láyf} ‘my life’ is set off from the rest of the utterance by a pause. The monosyllable \textit{láyf} ‘life’ bears continuative intonation. Compare this to the utterance-final monosyllable \textit{bád} ‘bad’, which features declarative intonation, signalled by L\%\is{declarative intonation}. The symbol [p] indicates a pause. The pitch trace of the pronoun \textit{e} ‘\textsc{3sg.sbj}’ is slighty distorted due to creaky voice.


\begin{figure}
\caption{Continuative intonation with topicalisation}
\label{fig:key:3.39}
\begin{tikzpicture}
    \draw (0, 0) node[inner sep=0] {
    \includegraphics[height=.3\textheight]{figures/yakpomod-img41.png}};
    \draw (0, -1.69) node[fill=white] {\footnotesize Mi láyf, e, e tránga bád.};
\end{tikzpicture}
\end{figure}

\ea\label{ex:key:92}
\glll Mi \textbf{láyf},  e,    e    tránga \textbf{bád}.\\
\textsc{l}    \textbf{\textsc{h${\emptyset}$}}\textbf{\textsc{\%}  }  \textsc{l}    \textsc{l}    \textsc{h.l}      \textbf{\textsc{hl\%}}\\
\textsc{1sg.poss}  life    \textsc{3sg.sbj}  \textsc{3sg.sbj}  be.strong  bad\\
\glt ‘My life, it, it was really tough.’    
\z

Continuative intonation is also employed as a juncture marker between linked clauses. Here, it may occur alone as a prosodic clause linker between juxtaposed clauses, or in conjunction with an overt clause linker. \figref{fig:key:3.40} and \figref{fig:key:3.41} are two clauses linked in a sequential, temporal relation. The adverbial time clause is introduced by \textit{di} \textit{dé} \textit{wé} ‘(the day) when’ in \figref{fig:key:3.40}. In the example, continuative intonation is found over the rightmost L-toned monosyllable \textit{=an} ‘\textsc{3sg.obj}’. The absence of the utterance-final L\% of declarative intonation halts the fall of the lexical L tone to the bottom of the pitch register.

\begin{figure}
\caption{Continuative intonation with clause linkage} 
\label{fig:key:3.40}
\begin{tikzpicture}
    \draw (0, 0) node[inner sep=0] {
    \includegraphics[height=.3\textheight]{figures/yakpomod-img42.png}};
    \draw (0, -1.69) node[fill=white] {\footnotesize Di dé wé yu go wánt plánt=an,  (...)};
\end{tikzpicture}
\end{figure}

\ea%93
    \label{ex:key:93}
    \glll   Di  dé  wé  yu  go  wánt  plánt=\textbf{an},  \op...\cp\\
\textsc{l}  \textsc{h}  \textsc{h}  \textsc{l}  \textsc{l}  \textsc{h}    \textsc{h=\textbf{l}}\textbf{\textsc{${\emptyset}$}}\textbf{\textsc{\%}}\\
\textsc{def}  day  \textsc{sub}  \textsc{2sg}  \textsc{pot}  want  plant=\textsc{3sg.obj}\\
\glt ‘The day you would want to go plant it (...)’
\z

The second clause in sequence features a lexical H over the utterance-final syllable. Here, continuative intonation produces no effect other than the maintenance of the lexical H tone. Compare \textit{dɔtalɔ́} ‘daughter-in-law’ and \textit{sɔnilɔ́} ‘son-in-law’ in \figref{fig:key:3.41}.

\begin{figure}
\caption{Continuative intonation over non-final clause}
\label{fig:key:3.41}
\begin{tikzpicture}
    \draw (0, 0) node[inner sep=0] {
    \includegraphics[height=.3\textheight]{figures/yakpomod-img43.png}};
    \draw (0, -1.69) node[fill=white] {\footnotesize E go tɛ́l in dɔtalɔ́, sɔnilɔ́,  (...)};
\end{tikzpicture}
\end{figure}

\ea%94
    \label{ex:key:94}
    \glll   E    go  tɛ́l  in    dɔtalɔ́,      sɔni\textbf{lɔ́},  \op...\cp\\
\textsc{l}    \textsc{l}  \textsc{h}  \textsc{l}    \textsc{l.l.h}\textbf{\textsc{${\emptyset}$}}\textbf{\textsc{\%}}      \textsc{l.l.\textbf{h}}\textbf{\textsc{${\emptyset}$}}\textbf{\textsc{\%}}\\
\textsc{3sg.sbj}  \textsc{pot}  tell  \textsc{3sg.poss}  daughter-in-law  son-in-law\\
\glt ‘She would tell her daughter-in-law, son-in-law, (...)’  
\z


Continuative intonation is also used as a stylistic device in ‘unfinished’\textstyleannotationreference{} utterances, such as the one in \figref{fig:key:3.42}. The final syllable retains its H tone or may even rise slightly towards the end. This emphatic variant of declarative\is{declarative intonation!emphatic} intonation is employed for dramatic effect. Compare the utterance-final, H-toned monosyllable \textit{dé} ‘there’.

\begin{figure}
\caption{Continuative intonation for stylistic effect}
\label{fig:key:3.42}
\begin{tikzpicture}
    \draw (0, 0) node[inner sep=0] {
    \includegraphics[height=.3\textheight]{figures/yakpomod-img44.png}};
    \draw (0, -1.69) node[fill=white] {\footnotesize Sɔn hángri kéch mí dé.};
\end{tikzpicture}
\end{figure}


\ea%95
    \label{ex:key:95}
    \glll   Sɔn    hángri    kéch  mí    \textbf{dé}.\\
L    \textsc{h.l}      \textsc{h}    \textsc{h}    \textsc{h}\textbf{\textsc{${\emptyset}$}}\textbf{\textsc{\%}}\\
some  be.hungry  catch  \textsc{1sg.indp}  there\\
\glt ‘I became really hungry there [you wouldn’t believe how much].’\is{continuative intonation}
\z

\subsection{Question intonation}\label{sec:3.4.5}

Yes-no questions are formed with an LH\% contour boundary tone. Contrary to emphatic intonation, question intonation is substitutive: The lexical tone over the utterance-final syllable is replaced by the question LH\%. In this way, the utterance-final syllable of a yes-no question invariably bears an LH contour, irrespective of its original tone. Compare the pitch contour over the L-toned second syllable of \textit{Píchi} ‘Pichi’ in \figref{fig:key:3.43}.

\begin{figure}
\caption{Non-emphatic yes-no question}
\label{fig:key:3.43}
\begin{tikzpicture}
    \draw (0, 0) node[inner sep=0] {
    \includegraphics[height=.3\textheight]{figures/yakpomod-img45.png}};
    \draw (0, -1.69) node[fill=white] {\footnotesize Yu go lán mí Píchi?};
\end{tikzpicture}
\end{figure}

\ea%96
    \label{ex:key:96}
    \glll   Yu    go    lán    mí    Pí\textbf{chi}?\\
\textsc{l}    \textsc{l}    \textsc{h}    \textsc{h}    \textsc{h.}\textbf{\textsc{lh\%}}\\
\textsc{2sg}    \textsc{pot}    teach  \textsc{1sg.indp}  Pichi\\
\glt ‘Will you teach me Pichi?’  
\z

The H tone of the LH\% contour may vary in pitch. While non-emphatic questions exhibit a gentle final rise and may therefore be similar in pitch to continuative intonation\is{continuative intonation}, more emphatic questions yield steeper rises. The more dramatic the rise, the more the question may additionally convey emphatic nuances like counter-expectation or insistence. I assume that in instances where the rise is particularly steep, the H tone component of the LH\% boundary contour tone is raised to extra-high, thus rendering L+H\%. Such an extra-steep rise is particularly common in rhetorical questions, optionally over the L-toned utterance-final question tag \textit{nɔ́} as in the following example.\is{alternative question}

\begin{figure}
\caption{Emphatic yes-no question}
\label{fig:key:3.44}
\begin{tikzpicture}
    \draw (0, 0) node[inner sep=0] {
    \includegraphics[height=.3\textheight]{figures/yakpomod-img46.png}};
    \draw (0, -1.69) node[fill=white] {\footnotesize Una gɛ́fɔ píl dɛ́n nɔ́?};
\end{tikzpicture}
\end{figure}

  
 


\ea%97
    \label{ex:key:97}
    \glll   Una    gɛ́fɔ    píl    dɛ́n    \textbf{nɔ́}?\\
\textsc{l.l}    \textsc{h.l}    \textsc{h}    \textsc{l}    \textbf{\textsc{l+h\%}}\\
\textsc{2pl}    have.to  peel    \textsc{3pl.indp}  \textsc{intj}\\
\glt ‘You [\textsc{pl}] have to peel them, right [you should know that]?’
\z



The utterance-final syllable in the question above exhibits a particularly steep rise. At the same time, emphasis\is{emphasis} is additionally expressed through pitch range expansion. The contrast between H and L tones is widened across the entire utterance as can be seen by the deep troughs in the pitch trace.\is{question intonation}

