\chapter{The spell-out mechanism in Nanosyntax}\label{chapter:nanosyntax}

\section{Introduction}

There are two separate problems that are associated with the term lexicalization. One is \isi{spell-out}, that is the way in which syntactic representations become realized as \isi{morpheme}s. The other is the positions in which these morphemes appear with respect to other morphemes. The positional problem  is sometimes referred to as the prefix vs.\ suffix opposition, which is a little misleading since the issue not only involves the predictions we can make about the placement of morphemes (the ``before or after the stem'' problem), but also the predictions we can make about the amount of affixes a particular syntactic representation is going to be realized by.\footnote{See \citet[135--138, 154--156]{DiSciullo2005} and \cite{Kayne2017} for some recent attempts to derive the prefix vs. suffix distinction from independent properties of grammar.
}%end of fn on Kayne 2016
\par
In order to illustrate these two problems, let us walk through cross-lin\-guis\-tic\-ally attested patterns of genitivite marking on nouns.
The choice to use genitive marking as an illustration of two major problems of lexicalization is motivated by the fact that it is a fairly familiar and well-described domain in the literature. Once the problems of \isi{spell-out} and \isi{morpheme} order are presented using genitive marking, the discussion in the remaining chapters will move to the domains of \ili{Slavic} \isi{verb}s and declarative \isi{complementizer}s.

\section{Two problems of lexicalization}

The first pattern of genitive case marking is found in \ili{Slavic} languages, where the nominal \isi{root} is followed by a single suffix, as shown on the example of the \ili{Polish} noun \textit{win-a} `of wine'.\largerpage
 
\ex. \ili{Polish}
\ag.[]\hspace{-20pt}win-a\\
\hspace{-20pt}wine-\textsc{gen}\\
\hspace{-20pt}\strut `of wine'


\noindent The second pattern is found in languages like Balkan \ili{Romani}, where the genitive case is realized as two separate suffixes on the nominal \isi{root}, as in \ref{vvv}.

\ex.  Balkan \ili{Romani} (\citealt[57]{Friedman1991} as cited in \citealt{Caha2011})\label{vvv}
\ag.[]\hspace{-20pt}\v{c}hav-\'es-koro\\
\hspace{-20pt}boy-\textsc{acc-gen}\\
\hspace{-20pt}`of boy' 

Let us take note of the fact that the suffix \textit{-\'es} is an accusative marker, as in \textit{\v{c}hav-\'es} `boy-\textsc{acc}', while *\textit{\v{c}hav-koro} is ill-formed.\footnote{The \isi{containment} of accusative marker \textit{-\'es} within a complex genitive marker \textit{-\'es-koro} falls within a broader class of morphological containment of cases attested also in  
\ili{Ingush} (\citealt{Nichols1994}), \ili{Estonian} (\citealt{Blevins2008}), \ili{Kazakh} (\citealt{Plakendorf2007}), or \ili{Classical Armenian} (\citealt{Schmitt1981,Caha2013}) and in a list of languages given in \cite{Plank1999}, including \ili{Finnish}, \ili{Karelian}, and \ili{Chukchi}, among others. In \ili{Slavic}, case containment is generally rare but can nevertheless be attested, for instance in the \ili{Prizren-Timok dialect of Serbian} (\citealt{Caha2011b}) or the colloquial form of the \ili{Polish} instrumental plural \textit{ocz-y-ma} `eyes', which contains the syncretic \textsc{nom=acc} suffix \textit{-y}, as shown in: \is{syncretism}

\noindent\parbox{\linguexfootnotewidth}{%
\ex. \ili{Polish}
\ag. ocz-y\\
eye-\textsc{nom/acc.pl}\\
\bg. ocz-y-ma\\
eye-\textsc{inst.pl}\\

}
} %fn on case containment
\par
The third pattern of the lexicalization of genitive case is attested in \ili{English}, where the genitive is realized as a pre-nominal \textit{of}, as in \textit{of wine}. A pre-nominal genitive is also attested as a bound \isi{morpheme} for instance in \ili{Maybrat} (West Papuan):

\ex. \ili{Maybrat} (\citealt[97]{Dol1999})
\ag.[]\hspace{-20pt}amah ro-Petrus\\
\hspace{-20pt}house \textsc{gen}-Petrus\\
\hspace{-20pt}`Petrus' house'

For our purposes, we will treat \isi{preposition}al and prefixal marking as variants of a more general ``pre-'' distribution, as opposed to a ``post-'' distribution (suffixes and postpositions).
\par To sum it up, while \ili{Polish}, \ili{Romani}, and \ili{English} realize genitive case as \isi{morpheme}s, they differ with respect to their amount and placement. This brings us to the following questions that pertain to the core of the lexicalization problem: 

\begin{itemize}
\item What is the source of these differences?
\item  Can we predict whether a language X will lexicalize genitive case -- or any other grammatical features -- as one or more \isi{morpheme}s? \is{feature}
\item If so, then can we predict if these morphemes are going to be linearized as pre- or post-positional  elements? \is{linearization}
\end{itemize}
 
\par A strand of research that has provided methodology to answer these questions is \isi{Nanosyntax}, a theory of the syntax-lexicon interface whose premise is that both the \isi{feature} structure of \isi{morpheme}s as well as their amount and placement are the two results of the way syntactic representations are spelled out (\citealt{Starke2009,Starke2014}). 
\par

\begin{sloppypar}

If we break down the existing methodology of \isi{Nanosyntax}, we find two distinct notions that help us answer the questions listed above, namely (i) phrasal \isi{spell-out} and (ii) the spell-out algorithm. 
Phrasal \isi{spell-out}, the idea that a lexical item corresponds to a phrasal node in a syntactic tree, tells us how syntactic representations become realized as morphemes. The spell-out algorithm, in turn, makes a statement about predicting the placement of \isi{morpheme}s with respect to other morphemes as well as their amount.
\par
Let us discuss in what follows how both tools explain our three patterns of genitive marking on nouns. 



\section{What we already know about how lexicalization works}


\isi{Nanosyntax} (henceforth NS) is a late insertion theory of the architecture of grammar, which assumes a neo-constructionist view of \isi{argument structure}, and whose major premise is that syntactic representation can be submorphemic. This view is consonant both with a growing body of work on the structuralization of lexical semantics (e.g. \citealt{Borer2005,Ramchand08}) and the so-called strong cartographic \is{cartography} thesis, whereby every grammatical \isi{feature} is a head of its own projection in syntax (\citealt[50]{CinqueRizzi2008}).\footnote{The ``one \isi{feature} per one syntactic head'' theorem is also shared by \cite{Kayne2005}, an approach which unlike NS does not assume that terminal nodes of syntactic trees can be smaller than morphemes. \is{terminal node}
} %end of fn on Kayne 2008
 A common platform for neo-constructionist theories is a close correspondence between the mental lexicon and  syntactically relevant features, to the effect that the association between the ``syn'' and ``sem'' of a lexical item is tight, though the specific nature of this association differs among the theories.\footnote{``Neo-constructionist theories'' are understood here as theories of \isi{argument structure} that by and large stem from \citeauthor{HK1993}'s (\citeyear{HK1993,HK2002}) work on syntactic representations of lexical items and, as such, argue that the properties of \isi{verb}al predicates are construed in syntax rather than in a generative lexicon. 
 In constructionist approaches, the meaning of a lexical item, e.g. the minimal meaning of a \isi{verb}al \isi{root}, is both conventionally and partially idiosyncratically associated with pieces of a syntactic structure and argument positions (e.g. \citealt{Goldberg1995,Goldberg2006,Booij2002,Jackendoff2002,Goldberg-Jackendoff2004}). This contrasts with neo-constructionist theories, which rely on more refined syntactic representations that are associated with meaning. The latter position, thus, suggests that there is a more direct and predictable relation between syntactic representations and its interpretation (semantics) (e.g. \citealt{Mateu2002,Borer2003,Borer2005,Ramchand08}).
 See \cite{Levin-Rapp2005}, \citet[19--48]{Acedo2010}, \cite{Ramchand2013}, \cite{Mateu2014}, \cite{Acquaviva-etal-2018} for overviews of the differences between generative theories of lexical semantics.  
} %end of fn on lexical semantics 

\end{sloppypar}

\subsection{Phrasal spell-out}

\is{spell-out} What constitutes a fundamental difference between NS \is{Nanosyntax} and other theories of the syntax-lexicon interface is the nature of the association between the syn-sem properties of a lexical item and its exponence. With this respect, a standard assumption of mainstream generative grammar about constraining spell-out only to terminal nodes of a syntactic representation is also part of \isi{Distributed Morphology} (DM). In DM, an exponent of a lexical item, e.g. $\alpha$ in \ref{DM:X}, realizes a terminal node with pre-packaged \isi{feature} bundles, e.g. the \mbox{[\,F$_{1}$, F$_{2}$, F$_{3}$\,]} bundle in the following illustration (\citealt{HalleMarantz1993,HalleMarantz1994,EN2007,Embick2015}). \is{terminal node}

\ex.\label{DM:X}
\begin{forest}nice empty nodes, for tree={l sep=0.9em,l=0,calign angle=63}
 [XP [\hskip 0.85cm X $\Rightarrow$ $\alpha$\\$^{\textup{{\small [\,F$_{1}$, F$_{2}$, F$_{3}$\,]}}}$ ]]\end{forest}

Limiting the interface between syn-sem properties of lexical items and their exponents to terminal nodes initially looks attractive. However, it comes with the cost of assuming the existence of a separate module, which will combine individual features F$_{1}$, F$_{2}$, F$_{3}$ into \isi{feature} sets that the terminal node in syntax is specified for.\is{terminal node}
The \isi{spell-out} of a featurally complex terminal node in syntax requires the existence of such a pre-syntactic compositional mechanism which construes the features into a set no matter if the set is ordered (a hierarchy) or not (a bundle). The substitution of \isi{feature} bundles for feature hierarchies in DM, thus, does not automatically remove the necessity for a pre-syntactic construal mechanism from the theory. \is{Distributed Morphology}
\par
NS \is{Nanosyntax} makes an opposite claim: \isi{spell-out} targets phrasal nodes, as illustrated in \Next, where features F$_{1}$, F$_{2}$, and F$_{3}$ all project their own phrases in line with the ``one \isi{feature} per one head'' thesis.\footnote{Phrasal spell-out has its origin in \cite{McCawley1968}. Outside NS, \is{Nanosyntax} it has been applied to the analysis of pronouns in \cite{WeermanEvers2002} and \cite{NeelemanSzendroi2007}.
}%end of fn on McCawley 1968

\ex.\label{NS:tree}
\begin{forest}nice empty nodes, for tree={l sep=0.65em,l=0,calign angle=63}
 [F$_{3}$P [F$_{3}$][F$_{2}$P [F$_{2}$][F$_{1}$P [F$_{1}$]]]] {\draw (.east) node[right]{$\Rightarrow$ $\alpha$}; }
\end{forest} 

The upshot of such a scenario is that there is no need for a pre-syntactic mechanism of construal since complex \isi{feature} structures are formed exclusively in syntax. 
\par
There are two immediate consequences resulting from such an alternative. One is that syntactic representations in NS \is{Nanosyntax} are much more fine-grained when compared with representations postulated by theories of grammar that assume the existence of a pre-syntactic lexicon. \is{Distributed Morphology} The other is that the only building block of syntactic structures is an atomic privative \isi{feature} rather than a \isi{morpheme}, abstract (as in late insertion models like DM) or factual (as in lexicalist approaches). 
\par
An essential feature of all late insertion models is the nature of the matching mechanism between the \isi{feature} set in a syntactic node with an exponent of a lexical item. 
\par
In DM, \is{Distributed Morphology} a lexical item can be \textit{underspecified} with respect to the features in the node it spells out. For example, the exponent of a lexical item defined as in \ref{dmle} can spell out the \isi{terminal node} X of the tree in \ref{DM:X}, which is specified for a larger set of features than the lexical item. (In the descriptions of lexical entries, let the symbol ``$\Leftrightarrow$'' indicate the association between the syn-sem structure of a lexical item and its exponence).\is{feature}


\ex. Lexical entry\label{dmle}\\[0.5ex]
 [ F$_{1}$\,] $\Leftrightarrow$ $\alpha$  

\noindent
If there exists another lexical item that meets the condition on insertion, such as the one in \Next, the competition between $\alpha$ and $\beta$ for lexicalizing the \isi{terminal node} X in \ref{DM:X} is resolved by the \textsc{Elsewhere Condition}, which \cite{Halle1997} defines in terms of the greatest number of features in the \isi{terminal node} that are matched by a lexical item.\footnote{This is one of a few approximations of the mechanism of insertion and competition resolution in \is{Distributed Morphology} DM. \cite{Halle1997} unifies underspecification with the Elsewhere Condition into one Subset Principle, \cite{Bobaljik2017} gives a more generic rule of insertion based on pairing a structural description of a lexical item with the features in a syntactic node, among some other versions of the same basic idea.\is{feature}
}%end of fn on Subset Principle

\ex. Lexical entry\\[0.5ex]
[\,F$_{1}$, F$_{3}$\,] $\Leftrightarrow$ $\beta$

Following the Elsewhere logic, the item $\beta$ will win the competition for insertion with the item 
$\alpha$.
\par
A dissenting view is advanced by NS, \is{Nanosyntax} which claims that lexical insertion is governed by the \textsc{Superset Principle}, \is{Superset Principle} defined as in \ref{superset}, which submits that a lexical item (i.e. a lexically stored tree with grammatical features) can be \textit{overspecified} with respect to the features in the syntactic node it spells out.\footnote{See \cite{Caha2018} for a comparison of lexical insertion in NS and DM \is{Distributed Morphology} and the results both mechanisms obtain in explaining the shapes of morphological \isi{paradigm}s. \is{feature}
} %end of fn

\ex. Superset Principle (\citealt{Starke2009})\label{superset}\\[0.5ex]
An exponent of a lexical item is inserted into a syntactic node if its lexical entry has a subconstituent that matches that node. \is{Superset Principle}

On the strength of the \isi{Superset Principle}, the exponent of a lexical item that is defined as in \ref{NS:alfa} will \isi{spell-out} the superset as well as the subsets of the features that make up the syntactic tree in \ref{NS:tree}.

\ex.\label{NS:alfa}  Lexical entry\\[0.5ex]
[ F$_{3}$ [ F$_{2}$ [ F$_{1}$ ]]] $\Leftrightarrow$ $\alpha$ 

When a lexicon of a particular language contains multiple lexical items that are in competition for insertion into a node in syntax, the choice which one gets inserted is governed by the \isi{Elsewhere Principle} defined as in the following:\is{containment}

\ex.  Elsewhere Principle\label{NS:elsewhere}\\
Where several items meet the conditions for insertion, the item containing fewer features unspecified in the node must be chosen.\is{feature}

\pagebreak Thus, if a lexicon contains both lexical entries as in \ref{NS:alfa} and as in:\is{containment}

\ex.\label{NS:beta} Lexical entry\\[0.5ex]
[ F$_{2}$ [ F$_{1}$ ]] $\Leftrightarrow$ $\beta$

then only the superstructure of our tree will be spelled out as $\alpha$ and its subsets will be spelled out as $\beta$, as shown in:

\ex.\label{NS:tree2}
\begin{forest}nice empty nodes, for tree={l sep=0.7em,l=0,calign angle=63}
 [F$_{3}$P [F$_{3}$][F$_{2}$P [F$_{2}$][F$_{1}$P [F$_{1}$]]]{\draw (.east) node[right]{$\Rightarrow$ $\beta$}; }
 ] {\draw (.east) node[right]{$\Rightarrow$ $\alpha$}; }
\end{forest} 

\noindent Note that on the strength of the \isi{Elsewhere Principle} in \ref{NS:elsewhere}, the AP subset of our tree in \ref{NS:tree2} is spelled out as $\beta$ rather than $\alpha$ since the lexical item in \ref{NS:beta} has only one \isi{feature} that is unspecified in the F$_{1}$P node, feature F$_{2}$, while the lexical item in \ref{NS:alfa} has two such features, F$_{1}$ and F$_{2}$. In other words, the lexical item $\beta$ is a better match for the syntactic node F$_{1}$P than the lexical item $\alpha$.\footnote{The \isi{Elsewhere Principle} is informally referred to in the literature on NS as ``the minimize junk principle''.
}% end of fn
\par A central feature of the \isi{spell-out} mechanism in NS \is{Nanosyntax} is that it is attempted after each application of merge, without a delay. 
That is, in order to lexicalize the entire tree in \ref{NS:tree2}, we attempt to \isi{spell-out} each feature, F$_{1}$, F$_{2}$, and F$_{3}$ immediately upon their mergers in the phrase marker. The result is that a lexical entry that matches a bigger tree will always over-ride the entires that match its subconstituents, a principle sometimes referred to as \textsc{Cyclic Over-ride}. \is{Cyclic Over-ride}
\par In connection to the \isi{spell-out}s of the representations in \ref{NS:tree} and \ref{NS:tree2}, let us also point out that the \isi{Superset Principle} applies to an entire phrase marker. That is, features cannot be erased from a grammatical representation and at the end of a cycle every \isi{feature} of a syntactic tree must be realized by a lexical item. Following \cite{Fabregas2007}, this restriction goes by the name  \textsc{Exhaustive Lexicalization Principle} (see also \citealt{Ramchand08}, who formulates essentially the same idea working with a different empirical material than \citealt{Fabregas2007}). \is{Exhaustive Lexicalization Principle}\largerpage[-2]

\subsection{Shortest Move and \isi{linearization}}

The \isi{spell-out} of a syntactic tree is not always going to result in over-ride. For example, the exponent of the following lexical entry

\ex.\label{lex:gam} [ F$_{4}$ ] $\Leftrightarrow$ $\gamma$

will not be inserted in the \isi{root} node of the tree: \is{Shortest Move}

\ex.\begin{forest}nice empty nodes, for tree={l sep=0.65em,l=0,calign angle=63}
 [F$_{4}$P [F$_{4}$][F$_{3}$P [F$_{3}$][F$_{2}$P [F$_{2}$][F$_{1}$P [F$_{1}$]]]{\draw (.east) node[right]{$\Rightarrow$ $\beta$}; }
 ] {\draw (.east) node[right]{$\Rightarrow$ $\alpha$}; }
 ]
\end{forest} 

Due to the strict cyclicity of \isi{spell-out},  F$_{4}$ must be spelled out before another \isi{feature} is merged. Since it is impossible to spell out F$_{4}$ in the tree with  $\gamma$ ``as is'', a different possibility to spell it out is attempted: movement. As indicated in \Next, the evacuation of F$_{3}$P will create the remnant constituent F$_{4}$P, which can then be spelled out as $\gamma$.

\ex.\label{tree:gam}
\begin{forest}nice empty nodes, for tree={l sep=0.65em,l=0,calign angle=63}
 [F$_{4}$P, s sep=30pt [F$_{3}$P, name=2 [F$_{3}$][F$_{2}$P [F$_{2}$][F$_{1}$P [F$_{1}$]]]{\draw (.east) node[right]{$\Rightarrow$ $\beta$}; }
 ] {\draw (.east) node[right]{$\Rightarrow$ $\alpha$}; }
 [F$_{4}$P [F$_{4}$][..., name=1]]{\draw (.east) node[right]{$\Rightarrow$ $\gamma$}; }
]
 \draw[dashed,->,>=stealth] (1) [in=-150,out=-120,looseness=2.7]  to (2);
\end{forest}

\vskip -1.25cm
In \cite{Caha2011}, the movement of the offending node is triggered by the shape of the lexical entry that a remnant constituent can match. For \ref{tree:gam}, this means that the structure of the lexically stored tree in \ref{lex:gam} launches the evacuation of F$_{3}$P. A different rationale is given in \cite{Starke2018}, where movement operations are not triggered by shapes of existing lexical entries and instead take place as part of an ordered set of procedures that are launched whenever a syntactic tree with a newely merged \isi{feature} F is not spelled out ``as is''. I will discuss the details of this \isi{spell-out} procedure in the next section.
\par
As indicated in \ref{tree:gam},
the evacuated node F$_{3}$P adjoins right
above the node that is targeted by \isi{spell-out}, the requirement sometimes referred to as \textsc{Shortest Move}. \is{Shortest Move} This movement takes place in agreement with the Extension Condition, whereby the output of merge must extend the tree at its \isi{root} (\citealt{Chomsky1993}). The evacuated F$_{3}$P creates a non-projecting sister node (a ``specifier'') to the node that is targeted by spell-out. 
\par
Such a structure is mapped onto a linear order of exponents in concert with a simplified version of the Linear Correspondence Axiom (\citealt{Kayne1994}), whose traditional formulation is given in the following: 

\ex.\label{LCA} Linear Correspondence Axiom (\isi{LCA}, \citealt{Kayne1994})\\[0.5ex]
If a non-terminal X asymmetrically c-commands a non-terminal
Y, then all terminal nodes dominated by X will precede all terminal
nodes dominated by Y. \is{terminal node} 

The definition in \ref{LCA} relies on the notion of asymmetric c-command, which distinguishes between categories and its segments, i.e. two directly connected nodes in a tree have the same label.\is{LCA}

\ex. Asymmetric c-command (\citealt[18]{Kayne1994})\\[0.5ex]
X c-commands Y iff:
\a. X and Y are categories and 
\b. no segment of X dominates Y and
\c. every category that dominates X dominates Y

This traditional formulation of the \isi{LCA} relies on both non-terminal and terminal nodes but allows only terminal nodes to linearize. For example, the syntactic representation as in \Next will provide the following statement about the linear order of exponents: \textit{x} precedes \textit{y}.\is{terminal node}

\ex.
\begin{forest}nice empty nodes, for tree={l sep=0.65em,l=0,calign angle=63}
 [YP [XP [X [\textit{x}]]] [YP [Y [\textit{y}]]]]
 \end{forest} 

With lexical items spelling out only non-terminals, the \isi{linearization} axiom must be modified. More precisely, it must be simplified to rely only on non-terminal nodes, as in the following formulation from \cite{Pantcheva2011}: 

\ex. Formulation of the \isi{LCA} for phrasal spell-out (\citealt[135]{Pantcheva2011})\\[0.5ex]
If a non-terminal X asymmetrically c-commands a non-terminal
Y, then whatever spells out X precedes whatever spells out Y.\is{spell-out}

For the tree in \ref{tree:gam}, this means that the spell-out of F$_{3}$P as $\alpha$ and the spell-out of the lower segment of F$_{4}$P as $\gamma$ will map onto the following sequence: $\alpha$ precedes~$\gamma$.

\subsection{*ABA as a consequence of the Superset Principle}\label{caseaba}

A direct consequence of the \isi{Superset Principle} that applies to a \isi{feature} hierarchy rather than to a bundle is the so-called *ABA, which constrains the distribution of syncretic forms in \isi{paradigm}s. We can formulate it after \cite{Bobaljik2007} as in \ref{ABAgeneral}. \is{syncretism} 


\ex. The *ABA generalization\label{ABAgeneral}\\[0.5ex]
In structured sequences (\isi{paradigm}s), a more complex structure and a less complex structure are not realized as form A, if structures that are in between them in terms of complexity are realized as form B. \is{*ABA}

The restriction of syncretic spans to adjacent cells of a paradigm informs us about structural contiguity of its categories and, thus, provides a major tool in discovering functional decomposition in grammar.
\par
For example, let us consider Caha's \citeyearpar{Caha2009} decomposition of cases into sets of cumulatively ordered privative case-forming features K\textsubscript{n} as in \Next, where nominative corresponds to K\textsubscript{n}, accusative to K$_{1}$$+$K$_{2}$, genitive to K$_{1}$$+$K$_{2}$$+$K$_{3}$, and so on.
Due to the description of cases in terms of \isi{feature} cumulation, \ref{case-fseq} comes out as an exocentric representation in the sense that case phrases higher than NomP are construed by both their daughters.
The representation of cases as a sequence of functional heads (\isi{fseq}) follows from the observation that non-accidental case\linebreak\isi{syncretism} targets only adjacent cells of declension \isi{paradigm}s if they are arranged in the order predicted by the hierarchy in \ref{case-fseq}.\footnote{The term ``non-accidental syncretism'' should be understood here simply as identity of exponents which in certain environments become phonologically altered rather than any surface phonological form of a case marker. This is particularly important in the context of \ili{Slavic}, where for example the exponent of the \ili{Polish} nominative masculine suffix of the singular nominal declension is a non-palatalizing [$-$\textsc{atr},$+$back,$+$round,$+$high] yer vowel $U$ and the exponent of the numberless masculine suffix present in the declension of numerals such as \textit{pi\k{e}\'c} `five' is a palatalizing [$-$\textsc{atr},$-$back,$+$round,$+$high] yer vowel $I$. Both yers are subject to deletion unless they lower to /e/ in a defined environment (see \citealt{Guss1980,Rubach1984}). Yers must not be confused with genuinely null exponents in \ili{Polish}, such as the nominative masculine suffix of the singular adjectival declension shown on the example of \textit{du\.zy} `big' in the third column in \tabref{Pol:cases}. See \citet[35--38]{Wiland2009} and the references listed there for a more detailed illustration.
} %end of fn

\ex.\label{case-fseq}
\begin{forest}nice empty nodes, for tree={l sep=0.7em,l=0,calign angle=63}
 [InstP [K$_{6}$][LocP [K$_{5}$][DatP [K$_{4}$][GenP [K$_{3}$][AccP [K$_{2}$][NomP [K$_{1}$]]]]]]]
 ] 
\end{forest} 

 This is illustrated by the examples of case \isi{paradigm}s in \ili{Polish} given in \tabref{Pol:cases}. \is{syncretism}
 
\begin{table}
\caption{Examples of attested case syncretisms in \ili{Polish}}
\label{Pol:cases} 
\begin{tabular}[t]{ l l l l l l l }
\lsptoprule	
     				& `wine'  					& `sir/man' 				& `big' 					& `lamp' 					& `five'\\
    			& \textsc{neu.sg} 			& \textsc{msc.sg} 		& \textsc{msc.sg} 			& \textsc{fem.sg} 			& \textsc{msc}\\\midrule
  \textsc{nom} 		& win-o\cellcolor[gray]{0.65} 	& pan-$U$ 				& du\.zy-$\emptyset$ 		& lamp-a					& pi\k{e}\'c-$I$\cellcolor[gray]{0.65}\\
  \textsc{acc} 		& win-o\cellcolor[gray]{0.65} 	& pan-a\cellcolor[gray]{0.65} 	& du\.z-ego\cellcolor[gray]{0.65} & lamp-\k{e}				& pi\k{e}\'c-$I$\cellcolor[gray]{0.65}\\
  \textsc{gen} 		& win-a 					& pan-a\cellcolor[gray]{0.65} 	& du\.z-ego\cellcolor[gray]{0.65} & lamp-y					& pi\k{e}ci-u\cellcolor[gray]{0.8}\\
  \textsc{dat} 		& win-u 					& pan-u\cellcolor[gray]{0.8} 	& du\.z-emu 				& lampi-e\cellcolor[gray]{0.65}	& pi\k{e}ci-u\cellcolor[gray]{0.8}\\
  \textsc{loc} 		& wini-e 				&  pan-u\cellcolor[gray]{0.8} 	& du\.zy-m 				& lampi-e\cellcolor[gray]{0.65}	& pi\k{e}ci-u\cellcolor[gray]{0.8}\\
  \textsc{inst} 		& win-em 				& pan-em 					& du\.zy-mi 				& lamp-\k{a} 				& pi\k{e}ci-oma\\
\lspbottomrule
\end{tabular}
\end{table}

In all the paradigms shown in the table, syncretic spans include only contiguous regions of the tree in \ref{case-fseq}, which indicates that the lexical entries for particular cases correspond to its constituents, as shown in \ref{cases:wino} for the neuter singular noun \textit{win} `wine'.

\ex.\label{cases:wino}   
\begin{forest}nice empty nodes, for tree={l sep=0.7em,l=0,calign angle=63}
 [InstP [K$_{6}$][LocP [K$_{5}$][DatP [K$_{4}$][GenP [K$_{3}$][AccP [K$_{2}$][NomP [K$_{1}$]]
 ]{\draw (.east) node[right]{$\Rightarrow$ \textit{o}}; }
 ]{\draw (.east) node[right]{$\Rightarrow$ \textit{a}}; }
 ]{\draw (.east) node[right]{$\Rightarrow$ \textit{u}}; }
 ]{\draw (.east) node[right]{$\Rightarrow$ \textit{e}}; }
 ]{\draw (.east) node[right]{$\Rightarrow$ \textit{em}}; }
 ] 
\end{forest}

\noindent
The \isi{Superset Principle} explains the unattested ABA patterns in a straightforward way: since the lexical entry A is contained within the lexical entry B, it is impossible for A to lexicalize a structure bigger than B. For example, since the exponent \textit{-o} in \ref{cases:wino} spells out the accusative structure, which is contained in the genitive structure realized by \textit{-a}, \textit{-o} cannot  spell out the structures that contain genitive at the same time.\is{containment} \is{*ABA}
\par 
Apart from an abundant work on the case \isi{fseq} (e.g. \citealt{Caha2009,Zompi2017,Starke2017}), sequences of syntactic projections have been deduced from \isi{syncretism} falling as a consequence of the \isi{Superset Principle} in the domain of Bantu class markers (\citealt{Taraldsen2010}), spatial adpositions (\citealt{Pantcheva2011}), aspectual prefixes in \ili{Polish} (\citealt{Wiland2012}), negation marking (\citeauthor{DeC2013} \citeyear{DeC2013,DeC2018}), participles (\citealt{Starke2006,LTN}), and wh-pronouns in \ili{German}ic (\citealt{Vangsnes2013}), among others. For some alternative accounts of syncretism see \cite{Stump2001}, \cite{Baerman2005}, \cite{Burzio2007}, \cite{Muller2008}, or \cite{Bobaljik2012}, among others. 

\subsection{The spell-out procedure in Starke (2018)}\label{sec:Starke2018}

\is{spell-out} \is{spell-out algorithm} To illustrate lexicalization patterns of genitive case features attested in \ili{Polish}, \ili{Romani}, and \ili{English}, let us start with the lexical entries in \ref{lex:win}, where the structure in (a) is a stand-in for the \ili{Polish} accusative neuter of the singular declension and the NP in (b) is a stand-in for the nominal \isi{root} \textit{win} `wine'.

\ex. Lexical entries in \ili{Polish}\label{lex:win}
\a. NP $\Leftrightarrow$ \textit{win} `wine'
\b. [ K$_{2}$ [ K$_{1}$ ]] $\Leftrightarrow$ \textit{o}\label{intro:o}

\noindent
The merger of the first \isi{feature} of the case \isi{fseq} on top of the NP \isi{root}, the nomina-tive-forming K$_{1}$, triggers \isi{spell-out} in line with the theorem about a strictly cyclic character of merge and spell-out. However, K$_{1}$ in the tree on the left in \Next does not match any lexical entry in the \ili{Polish} lexicon, which requires its spell-out to be attempted in a different way. For a moment, let us go  with \citeauthor{Caha2011}'s \citeyearpar{Caha2011} idea that movement in syntax is driven by spell-out, which when applied to our case means that all we need to do to spell out K\textsubscript{1} is to evacuate the \isi{root} \textit{win} `wine', as shown on the right side in \ref{mergerandspellout}. \is{spell-out algorithm}

\ex.\label{mergerandspellout} Merger and spell-out of nominative in \ili{Polish}\\[1ex]
\begin{forest}nice empty nodes, for tree={l sep=0.7em,l=0,calign angle=63}
 [NomP, s sep=10pt [K$_{1}$][NP [\textit{win}\\`wine', roof]
 ]]
 \end{forest} 
 \hskip 1cm $\leadsto$ \hskip -0.5cm
\begin{forest}nice empty nodes, for tree={l sep=0.7em,l=0,calign angle=63}
 [~, s sep=15pt [NP, name=tgt [\textit{win}\\`wine', roof]]
 [NomP [K$_{1}$][..., name=t]
]{\draw (.east) node[right]{$\Rightarrow$ \textit{o}}; }]
 \draw[dashed,->,>=stealth] (t) [in=-155,out=-120,looseness=2]  to (tgt);
 \end{forest}

\vskip -2.25\baselineskip
The constituent created in this way matches the lexical entry in \ref{intro:o} and, on the strength of the \isi{Superset Principle}, gets spelled out as \textit{-o}, which comes out as the suffix on the \isi{root} \textit{win}.
\par 
The new cycle begins with the merger of next \isi{feature} in the case \isi{fseq}, the accusative-forming K$_{2}$, as in: 

\ex.\label{intro:K2}
\begin{forest}nice empty nodes, for tree={l sep=0.7em,l=0,calign angle=63}
 [AccP, s sep=0pt [K$_{2}$] [NomP, s sep=10pt [NP [\textit{win}\\`wine',roof]]
 [NomP, s sep=0pt [K$_{1}$]
 ]{\draw (.east) node[right]{$\Rightarrow$ \textit{o}}; }]]
 \end{forest}

Such a structure cannot be spelled out as, again, it is not matched by any existing entry in the \ili{Polish} lexicon. In contrast to \ili{Polish}, a nominal \isi{root} with a sequence of case features K$_{2}$\,$>$\,K$_{1}$ merged on its top, can be spelled out right away in \ili{English}, as shown in \ref{wine}.\largerpage[2]

\ex. Spell-out of the \ili{English} syncretic\is{syncretism} \isi{root} \textit{wine}\label{wine}\\[1ex]
\begin{forest}nice empty nodes, for tree={l sep=0.7em,l=0,calign angle=63}
 [AccP, s sep=10pt [K$_{2}$][NomP [K$_{1}$][NP]]
 ]{\draw (.east) node[right]{$\Rightarrow$ \textit{wine}}; } ]
 \end{forest} 


\noindent
The in situ \isi{spell-out} of the root \textit{wine} together with NomP and AccP captures the fact that all nominative and accusative forms of \ili{English} lexical nouns are syncretic with their \isi{root}s.\footnote{There is no established distinction between closed and open class items in \is{Nanosyntax} NS. While this constitutes a research question of its own, this issue does not have a bearing on the application of phrasal spell-out as long as open class items can be represented as syntactic phrases, the position recently made a case for, on different grounds, in \cite{NU} and \cite{CCW2017}.
} % end of FN
Such a portmanteau \isi{spell-out} is the basic option in which features can be realized as morphology as it does not require any movement operation to facilitate  lexicalization. Let us, thus, call this option \textsc{stay}.\is{spell-out algorithm}
\par
In contrast to \ili{English}, it is clear that neither NomP nor AccP is spelled out by \textsc{stay} in the \ili{Polish} accusative form \textit{win-o} `wine-\textsc{acc}', as the \isi{spell-out} of K$_{2}$ in the tree that looks like in \ref{intro:K2} would over-ride the earlier spell-outs of both the NP \isi{root} \textit{win} `wine' and the nominative suffix \textit{-o}, to the effect that we would have a single portmanteau \isi{morpheme} in their place, counter fact.
\par Since \textsc{stay} fails, the next familiar possibility to spell-out K$_{2}$ is to attempt movement. Let us, thus, call this option \textsc{move}. Unlike in the case of the nominative-forming feature K$_{1}$, however, this time there are two movement possibilities: we can continue with the movement launched in the previous cycle, the \textsc{\textsc{spec-to-spec}} movement of the NP \textit{win}, or we can move the complement of K$_{2}$ (the \textsc{snowballing} of \textit{win-o}). This is a vacuous choice in an approach to lexicalization as in \cite{Caha2011} where spell-out driven movement is teleological, in the sense that it targets those nodes whose evacuation will create a constituent matching an existing lexical entry.\is{spell-out algorithm}
\par An alternative to such a characterization of \isi{spell-out} driven operations is a scenario where we have an unambiguous specification of how to spell-out a \isi{feature}. This is the position taken up in \cite{Starke2018}, who submits that out of the two movement possibilities, spec-to-spec is the first option to try. As shown in \ref{so:wino}, the movement of the \isi{root} \textit{win} lets K$_{2}$ spell-out as part of the accusative superstructure of \textit{-o}, in line with the lexical entry in \ref{intro:o}.  


\ex. Spell-out of the \ili{Polish} accusative \textit{win-o} `wine'\label{so:wino}\\[-1ex]
\begin{forest}nice empty nodes, for tree={l sep=0.75em,l=0,calign angle=63}
 [~, s sep=20pt [NP, name=tgt [\textit{win}\\`wine',roof]][AccP, s sep=-10pt [K$_{2}$] 
 [NomP,s sep=-10pt [..., name=t][NomP [K$_{1}$]]]
 ]{\draw (.east) node[right]{$\Rightarrow$ \textit{o}}; }]
\draw[dashed,->,>=stealth] (t) ..controls +(south west:1.5) and +(south west:2).. (tgt);
 \end{forest}

Consequently, the accusative \textit{-o} surfaces as the suffix. 
\par
Given the lexical entries as in \Next, spec-to-spec movement also facilitates the \isi{spell-out} of K$_{2}$ in the \ili{Romani} \textit{\v{c}hav-\'es} `boy'-\textsc{acc}, as shown in \ref{VR:acc}.

\ex. Lexical entries in Balkan \ili{Romani}
\a. NP $\Leftrightarrow$ \textit{\v{c}hav} `boy'
\b. [ K$_{2}$ [ K$_{1}$ ]] $\Leftrightarrow$ \textit{\'es}

\ex. Spell-out of the Balkan \ili{Romani} accusative \textit{\v{c}hav-\'es} `boy'\label{VR:acc}\\[-1ex]
\begin{forest}nice empty nodes, for tree={l sep=0.75em,l=0,calign angle=63}
[~, s sep=25pt [NP, name=tgt [\textit{\v{c}hav}\\`boy',roof]]
[AccP, s sep=8 [K$_{2}$][NomP [..., name=t]
[NomP, s sep=0pt [K$_{1}$]]]
]{\draw (.east) node[right]{$\Rightarrow$ \textit{\'es}}; }]
\draw[dashed,->,>=stealth] (t) ..controls +(south west:1.5) and +(south west:2).. (tgt);
 \end{forest}


\noindent
The merger of the next case feature in the fseq, the genitive-forming K$_{3}$ reveals that we need both spec-to-spec movement and complement movement to be listed in the \isi{spell-out} algorithm. Whereas the first allows K$_{3}$ to spell-out in \ili{Polish}, it does not in \ili{Romani}. Assuming the lexical entry as in \Next, a stand-in for genitive neuter, then successive-cyclic movement of \textit{win} in \ili{Polish} results in the genitive marker \textit{-a} over-riding the earlier spell-out of the accusative \textit{-o} and getting linearized as the suffix in \textit{win-a} `wine'. \is{linearization} This derivation is shown in \ref{intro:a} below.

\ex. Lexical entry in \ili{Polish}\\[0.5ex]
[ K$_{3}$ [ K$_{2}$ [ K$_{1}$ ]]] $\Leftrightarrow$ \textit{a}


\ex.\label{intro:a} Spell-out of the \ili{Polish} genitive \textit{win-a} `wine'\\[-1ex]
\begin{forest}nice empty nodes, for tree={l sep=0.75em,l=0,calign angle=63}
[~, s sep=20pt [NP, name=tgt [\textit{win}\\`wine',roof]]
[GenP [K$_{3}$][AccP [..., name=t] [AccP, s sep=8pt [K$_{2}$]
[NomP [K$_{1}$]]]]{\draw (.east) node[right]{$\Rightarrow$ \textit{o}}; }]
{\draw (.east) node[right]{$\Rightarrow$ \textit{a}}; }]
\draw[dashed,->,>=stealth] (t) ..controls +(south west:1.75) and +(south west:2.5).. (tgt);
 \end{forest}

\noindent In contrast to the \ili{Polish} genitive \textit{-a}, the genitive marker \textit{-koro} in \ili{Romani} does not over-ride the accusative suffix \textit{-\'es} but stacks as the second suffix. This indicates that the syn-sem structure realized by \textit{-koro} includes only K$_{3}$, as in: 

\ex. Lexical entry in \ili{Romani}\\[0.5ex]\label{koro}
[ K$_{3}$ ] $\Leftrightarrow$ \textit{koro}

This means that an attempt to spell this \isi{feature} out by successive-cyclic movement of the \isi{root} \textit{\v{c}hav} as in \ref{failed:koro} is not going to be successful, as the constituent formed by such a movement is not matched by any existing lexical entry.

\ex.\label{failed:koro}
\begin{forest}nice empty nodes, for tree={l sep=0.75em,l=0,calign angle=63}
[~, s sep=20pt [NP, name=tgt [\textit{\v{c}hav}\\`boy',roof]]
[GenP, s sep=8pt [K$_{3}$][AccP, s sep=8pt [..., name=t] [AccP [K$_{2}$]
[NomP [K$_{1}$]]]{\draw (.east) node[right]{$\Rightarrow$ \textit{\'es}}; }]
]{\draw (.east) node[right]{$\Rightarrow$ no match}; }]
\draw[dashed,->,>=stealth] (t) ..controls +(south west:1.5) and +(south west:2.25).. (tgt);
 \end{forest}

\noindent The failure to \isi{spell-out} requires the derivation to backtrack by trying to move the complement of K$_{3}$. As shown in \Next, the constituent created in this way is matched by the entry in \ref{koro} and \textit{-koro} comes out as the external suffix. \is{backtracking}

\ex.\label{so:vrg} Spell-out of the \ili{Romani} genitive \textit{\v{c}hav-\'es-koro} `boy'\\
\resizebox{\linewidth}{!}{%
\begin{forest}nice empty nodes, for tree={l sep=0.7em,l=0,calign angle=63}
[GenP, s sep=8pt [K$_{3}$][AccP, s sep=8pt [NP [\textit{\v{c}hav}\\`boy' ,roof]]
[AccP, s sep=15pt [K$_{2}$]
[NomP, s sep=0pt [K$_{1}$]]]{\draw (.east) node[right]{$\Rightarrow$ \textit{\'es}}; }
]]{\draw (.east) node[right]{$\Rightarrow$ no match}; }
\end{forest}
\hskip 0.25cm$\leadsto$ %\hskip -1.75cm
\begin{forest}nice empty nodes, for tree={l sep=0.7em,l=0,calign angle=63}
[~, s sep=20pt [AccP, s sep=8pt, name=tgt [NP [\textit{\v{c}hav}\\`boy' ,roof]]
[AccP, s sep=10pt [K$_{2}$]
[NomP [K$_{1}$]]
]{\draw (.east) node[right]{$\Rightarrow$ \textit{\'es}}; }
] [GenP, s sep=-10pt [K$_{3}$][..., name=t]
]{\draw (.east) node[right]{$\Rightarrow$ \textit{koro}}; }]
\draw[dashed,->,>=stealth,overlay] (t) [in=-145,out=-115,looseness=2.85]  to (tgt);
\end{forest}
}\vspace*{\baselineskip}


\noindent Two kinds of movements -- spec-to-spec and snowballing -- derive the genitive marking patterns attested in languages like \ili{Polish} and \ili{Romani} but they fail to derive the pre-nominal genitive marking in languages like \ili{English} from the extension of the accusative structure, the AccP lexicalized as \textit{wine}, by the merger of the next case \isi{feature} in the fseq, K$_{3}$, as shown in \ref{fail:of}.

\ex.\label{fail:of} Merger and attempted spell-out of genitive by \textsc{stay} in \ili{English}\\[1ex]
\begin{forest}nice empty nodes, for tree={l sep=0.7em,l=0,calign angle=63}
 [GenP, s sep=8pt [K$_{3}$][AccP, s sep=10pt [K$_{2}$]
 [NomP [K$_{1}$][NP]]
 ]{\draw (.east) node[right]{$\Rightarrow$ \textit{wine}}; } ]
 ]{\draw (.east) node[right]{$\Rightarrow$ no match}; }
 \end{forest} 
 
\begin{samepage} 
The lack of a specifier created by movement at the previous cycle in \ref{fail:of} leaves us with an attempt to spell out K$_{3}$ by snowballing, as in \Next, which creates a structure that does not correspond to the prepositional \textit{of}, either. \is{preposition}

\ex. 
\resizebox{\linewidth}{!}{%
\begin{forest}nice empty nodes, for tree={l sep=0.7em,l=0,calign angle=63}
 [GenP, s sep=8pt [K$_{3}$][AccP, s sep=10pt [K$_{2}$]
 [NomP [K$_{1}$][NP]]
 ]{\draw (.east) node[right]{$\Rightarrow$ \textit{wine}}; } ]
 ]{\draw (.east) node[right]{$\Rightarrow$ no match}; }
 \end{forest}  
 \hskip 0.35cm$\leadsto$ %%%\hskip -2.05cm
 \begin{forest}nice empty nodes, for tree={l sep=0.7em,l=0,calign angle=63}
[~, s sep=20pt [AccP, s sep=10pt, name=tgt [K$_{2}$][NomP [K$_{1}$][NP]]
]{\draw (.east) node[right]{$\Rightarrow$ \textit{wine}}; } 
[GenP [K$_{3}$][..., name=t]]{\draw (.east) node[right]{$\Rightarrow$ no match}; }]
\draw[dashed,->,>=stealth,overlay] (t) [in=-150,out=-120,looseness=2.15]  to (tgt);
\end{forest}}

\end{samepage}\vspace*{\baselineskip}

\noindent We are, thus, arriving at a situation where genitive cannot be spelled out by \textsc{stay} but applying \textsc{move} does not result in creating constituency which is matched by a lexical entry with K$_{3}$ either. 
\par
An immediate possibility is to assume the \isi{terminal node} K$_{3}$ to lexicalize as \textit{of}, which would make the correct prediction about \textit{of} surfacing in front of \textit{wine}. This is the way \isi{preposition}al case marking is derived in \citeauthor{Caha2009} (\citeyear{Caha2009,Caha2011}). However, the insertion of \textit{of} directly into the terminal K$_{3}$ goes against the thesis that \isi{spell-out} targets only phrasal nodes. Looking at the possibility of spell-out targeting both terminal and non-terminal nodes more globally, an empirical argument against ``pre-'' elements being inserted into terminal nodes is that they would have to comprise only specific markers, certainly not a situation we observe with a considerable subset of prefixes, \isi{particle}s, auxiliary verbs, or \isi{complementizer}s.\is{terminal node} For example, the \ili{English} \textit{with} is a \is{syncretism}syncretic marker of comitative and instrument, \textit{that} is a syncretic form of \isi{demonstrative} pronoun, \isi{complementizer}, and a \isi{relativizer}, etc.
\par
Maintaining the idea that \isi{spell-out} targets only phrasal nodes in syntax, \cite{Starke2018} proposes that the derivation backtracks to the previous cycle, at which point the last resort strategy kicks in: the merger of K$_{3}$ will take place in a parallel subtree and the spell-out of K$_{3}$ will be attempted upon merging the subtree with the mainline derivation. \is{backtracking}
\par
In order to spawn the subderivation of the parallel case \isi{fseq}, \cite{Starke2018} states that what needs to be provided as the base is a nominal feature of the NP (literally, the N head in our representation). In line with the case \isi{fseq} in \ref{case-fseq}, the first case feature to merge with the base feature N is the nominative-forming K$_{1}$, as shown in \ref{culo}. Subsequently, the accusative feature K$_{2}$ is merged in the subderivation, which results in both derivations reaching the same size of the case \isi{fseq}.\footnote{Let us note that the subderivation up to the AccP size is not matched by any existing lexical item, as the sister node to K$_{1}$ is not a complex NP \isi{root}, only an atomic nominal \isi{feature}.  The structure with K$_{1}$, K$_{2}$, and the singleton nominal feature N is not enough to be identified by any lexical entry in the \ili{English} lexicon.
}%end of fn

\ex.\label{culo} Subtree (left) parallel to the mainline derivation (right) in the formation of the \ili{English} 
genitive\\[1ex]
\begin{forest}nice empty nodes, for tree={l sep=0.7em,l=0,calign angle=63}
[AccP, s sep=10pt [K$_{2}$][NomP [K$_{1}$][N]]]
 \end{forest}
 \hskip 1.25cm
\begin{forest}nice empty nodes, for tree={l sep=0.7em,l=0,calign angle=63}
 [AccP, s sep=10pt [K$_{2}$][NomP [K$_{1}$][NP]]
 ]{\draw (.east) node[right]{$\Rightarrow$ \textit{wine}}; } ]
 \end{forest} 
 
\noindent
At this point the merger of the genitive-forming K$_{3}$ takes place in the subderivation, as shown in \ref{sub:K3}.

\ex.\label{sub:K3}
\begin{forest}nice empty nodes, for tree={l sep=0.7em,l=0,calign angle=63}
[GenP, s sep=10pt [K$_{3}$][AccP, s sep=10pt [K$_{2}$][NomP [K$_{1}$][N]]]]
 \end{forest}
 \hskip 1.25cm
\begin{forest}nice empty nodes, for tree={l sep=0.7em,l=0,calign angle=63}
 [AccP, s sep=10pt [K$_{2}$][NomP [K$_{1}$][NP]]
 ]{\draw (.east) node[right]{$\Rightarrow$ \textit{wine}}; } ]
 \end{forest} 

Once the genitive K$_{3}$ is merged in the subderivation, the resulting GenP-subtree is merged with the
mainline and forms a complex left branch, as in \ref{of}. If the \ili{English} lexicon contains the entry like in \Next, then the left branch that contains K$_{3}$ is spelled out as \textit{of}, which surfaces as a ``pre-'' element with respect to the accusative noun, as in \textit{of wine}.\is{containment}

\ex. Lexical entry for the \ili{English} genitive\\[0.5ex]
[ K$_{3}$ [ K$_{2}$ [ K$_{1}$ N ]]] $\Leftrightarrow$ \textit{of}

\ex.\label{of} Merger and spell-out of the the \ili{English} \isi{preposition}al genitive \textit{of}\\
\begin{forest}nice empty nodes, for tree={l sep=0.7em,l=0,calign angle=63}
 [~, s sep=15pt [GenP, s sep=10pt [K$_{3}$][AccP, s sep=10pt [K$_{2}$][NomP [K$_{1}$][N]]]
]{\draw (.east) node[right]{$\Rightarrow$ \textit{of}}; }
 [AccP, s sep=10pt [K$_{2}$][NomP [K$_{1}$][NP]]
 ]{\draw (.east) node[right]{$\Rightarrow$ \textit{wine}}; }]
 \end{forest}

\noindent
A comment about the last resort status of the left branch formation is in order. As \cite{Starke2018} notes, launching the subderivation is a costly operation as it requires the growth of the two parallel trees to be coordinated up to the point of closing in the subderivation with the mainline. The formation of the left branch is hence kept as the final option in the spell-out algorithm.\footnote{Let us recall that in line with the exhaustive lexicalization principle, a failure to spell out a \isi{feature} results in derivation failure. \is{Exhaustive Lexicalization Principle}
}%end of fn on Fabregas rule
\par Deriving the patterns of morphological realization of a syntactic sequence is not the only result of the spell-out procedure that involves what we have called here \textsc{move} and \textsc{subderive}. Namely, these operations also allow us to define the distributional contrast between \mbox{``pre-''} elements (prefixes, \isi{preposition}s, \isi{particle}s, complementizers, etc.) and \mbox{``post-''} elements (suffixes and postpositions) in a structural way. Namely, as \cite{Starke2018} writes, ``pre-'' elements have a binary foot (e.g. the \ili{English} \textit{of}), whereas suffixes have a unary foot (e.g. the \ili{Romani} \textit{-\'es} or \textit{-koro}). The binary foot of ``pre-'' elements is a result of \textsc{subderive}, an operation spawned by the merger of two features; the unary foot of suffixes is a result of \textsc{move}, with a proviso that \isi{spell-out} driven movements do not leave a trace, which is confirmed by the observation that such movements do not show reconstruction or defective intervention effects. 


\subsection{Pointers}\label{sec:pointers}

A central feature of the spell-out procedure discussed so far is that lexical access takes place cyclically -- after each merger of a \isi{feature} in the phrase marker. Such a set up allows for an insertion of a lexical item which is sensitive to a lexical item that has been inserted at an earlier cycle. A tool in NS \is{Nanosyntax} that facilitates a reference to lexical items inserted at previous cycles is called a \textsc{pointer}, which is defined as in \ref{def:pointer} (see also \citealt{TaraldsenNELS,CahaPantcheva2012,Starke2014,GVW2018,CCW-root}). \is{spell-out algorithm}

\ex.\label{def:pointer} A \isi{pointer} is a node in a lexically stored tree that directs to a lexical entry.

A \isi{spell-out} of syntactic \isi{feature} that relies on a \isi{pointer} is illustrated in \Next, where the pointer node is indicated with an arrow.  

\ex.\label{lex:alpha2} \begin{forest}nice empty nodes, for tree={l sep=0.65em,l=0,calign angle=63}
 [F$_{3}$P, s sep=16pt [F$_{3}$][$\beta$, edge+={->, line width=0.15mm}] 
 ]{\draw (.east) node[right]{$\Rightarrow$ $\alpha$}; } 
\end{forest}

Here, the lexical item $\alpha$ is inserted in the phrasal node which includes the feature F$_{3}$ and a constituent that has been spelled earlier out as a lexical item $\beta$.
An essential difference between a lexical entry that involves a \isi{pointer} and one that does not is that the first can spell out syntactic trees that can include only a subset of a structure that is realized by a different lexical item. For example, if the lexical entry for $\beta$ is defined as in:

\ex. [ F$_{2}$ [ F$_{1}$ ]] $\Leftrightarrow$ $\beta$ 

and $\alpha$ is inserted into the node with the \isi{pointer} to $\beta$ in \ref{lex:alpha2}, this means that $\alpha$ can spell out the following syntactic trees:

\begin{multicols}{2}
\ex.
\a.\label{lex:sup:b} \begin{forest}nice empty nodes, for tree={l sep=0.65em,l=0,calign angle=63}
 [F$_{3}$P [F$_{3}$][F$_{2}$P [F$_{2}$][F$_{1}$P [F$_{1}$]]
 ]{\draw (.east) node[right]{$\Rightarrow$ $\beta$}; }
 ]{\draw (.east) node[right]{$\Rightarrow$ $\alpha$}; } 
\end{forest}
\b.\label{lex:sub:b} \begin{forest}nice empty nodes, for tree={l sep=0.65em,l=0,calign angle=63}
 [F$_{3}$P [F$_{3}$][F$_{1}$P [F$_{1}$]
 ]{\draw (.east) node[right]{$\Rightarrow$ $\beta$}; }
 ]{\draw (.east) node[right]{$\Rightarrow$ $\alpha$}; } 
\end{forest}

\end{multicols}

\noindent
\ref{lex:sup:b} includes the superset structure of $\beta$ and \ref{lex:sub:b} its subset. The \isi{pointer} to the lexical entry of $\beta$, thus, allows $\alpha$ to spell-out a structure in \ref{lex:sub:b}, which shrinks in the middle. This result is impossible to obtain under the \isi{Superset Principle} if the lexical entry for $\alpha$ included a constituent [ F$_{3}$ [ F$_{2}$ [ F$_{1}$ ]]].
\par
The \isi{pointer} technology can explain suppletion. For example, while the productive formation of the \ili{English} preterites includes the stem that is identical to the bare form of the \isi{verb}, e.g. \textit{want} and \textit{want-ed}, a subset of the preterites is formed with a suppletive form of the verb, e.g. \textit{give} and \textit{gave}. This can be explained if the suppletive form of the preterite includes a \isi{pointer} to the lexical entry of the bare \isi{verb}. This is illustrated for \textit{gave} in \ref{point:give}, where it spells out the phrasal node PastP which includes the preterite-forming \isi{feature} Past and the pointer to \textit{give}.

\ex.\label{point:give}\begin{forest}nice empty nodes, for tree={l sep=0.65em,l=0,calign angle=63}
 [PastP [Past][\textit{give}, edge+={->, line width=0.15mm}] 
 ]{\draw (.east) node[right]{$\Rightarrow$ \textit{gave}}; } 
\end{forest}

The \isi{spell-out} of PastP as \textit{gave} will take place only if the node pointed to has been earlier spelled out as \textit{give} (not as any other lexical item or constituent).
\par
Other than explaining suppletive \isi{allomorphy}, \isi{pointer}s have been used to explain idioms in \cite{Starke2014} as well as derive syncretic \is{syncretism} alignment in \isi{paradigm}s involving datives, locatives, and allatives in \cite{CahaPantcheva2012} and in pronominal paradigms in \cite{GVW2018}. I will return to pointers in Chapter \ref{chapter:explaining} in an attempt to describe the lexical entry for the \isi{iterative} affix in Czech and \ili{Polish}. 

\section{Summary of the current state of the spell-out procedure}\label{section:Starke2018}

Let us synopsize the spell-out formula in \cite{Starke2018}, which is an unambiguous specification of how to lexicalize a grammatical \isi{feature}, i.e. an algorithm for spell-out:\largerpage[2]  

\begin{description}
\item[Step 1:] \textsc{stay} -- add a \isi{feature} F and spell-out (an in situ spell-out; derives the \ili{English} \textsc{nom/acc} \textit{wine}).
\item[Step 2:] \textsc{spec-to-spec} -- move the node merged in the previous cycle and spell out (derives the suffixal form of the \ili{Polish} genitive \textit{win-a} `of wine').
\item[Step 3:] \textsc{snowball} -- move the complement of the \isi{feature} F and spell out (derives genitive marking in the \ili{Romani} \textit{\v{c}hav-\'es-koro} `of boy').
\item[Step 4:] \textsc{subderive} -- remove F from the mainline derivation and build a phrase marker comprising F, merge it with the mainline derivation and spell out (results in merger of a complex left branch whose spell-out comes out as a ``pre-'' element on the stem; derives the \ili{English} \isi{preposition}al genitive marker \textit{of}). 
\end{description}

Such a procedure predicts that the lexicalization of a feature added to a derivation either keeps the same amount of \isi{morpheme}s (when the added feature is spelled out by the default \textsc{stay}) or adds a morpheme (when it is spelled out by the remaining steps, \textsc{move spec-to-spec}, \textsc{snowball}, or \textsc{subderive}). 

\section{Spell-out resulting in the \isi{reduction} in the number of morphemes}\label{sec:cutback}

\subsection{The problem}

So far we have discussed situations in which the addition of a \isi{feature} to a syntactic representation leads either to the preservation or an increase in the number of morphemes at spell-out. For instance, the addition of the genitive-forming case feature K$_{3}$ to the AccP in \ili{Polish} in example \ref{intro:a} resulted in the genitive suffix \textit{-a} over-riding the accusative suffix \textit{-o}, which preserved the same number of suffixes on the noun. In turn, the addition of K$_{3}$ to the AccP in \ili{Romani} in example \ref{so:vrg} and in \ili{English} in example \ref{of} resulted in the genitive case surfacing as an additional \isi{morpheme}: the outer suffix in \ili{Romani} and the prefix in \ili{English}. 
\par
Let us now consider a situation where the addition of a feature to a syntactic representation gives a different result to the ones discussed so far, namely, instead of a preservation or an increase, it leads to a \isi{reduction} in the number of morphemes at \isi{spell-out}.
\par
In order to illustrate such a scenario, let us suppose that an \isi{fseq} in \Next is lexicalized by a \textit{ROOT} and three affixes \textit{X}, \textit{Y}, \textit{Z}, and that the span that ranges from F$_{1}$ up to F$_{5}$ in this \isi{fseq} is lexicalized by a structure comprising three morphemes: \textit{ROOT-X-Y}. 


\ex. \begin{forest}nice empty nodes, for tree={l sep=0.65em,l=0,calign angle=63}
[F$_{6}$P [F$_{6}$][F$_{5}$P [F$_{5}$][F$_{4}$P [F$_{4}$][F$_{3}$P [F$_{3}$][F$_{2}$P  [F$_{2}$][F$_{1}$P [F$_{1}$]]]]]]]
 \end{forest}
 

\noindent Such a result can be easily obtained with the following list of lexical entries:

\ex.\label{lex:list} 
\a. [ F$_{3}$ [ F$_{2}$ [ F$_{1}$ ]]] $\Leftrightarrow$ \textit{ROOT}\label{lex:ROOT}
\b. [ F$_{4}$ ] $\Leftrightarrow$ \textit{X}\label{lex:X}
\c. [ F$_{5}$ ] $\Leftrightarrow$ \textit{Y}\label{lex:Y}
\d. [ F$_{6}$ [ F$_{5}$ [ F$_{4}$ [ F$_{3}$ ]]]] $\Leftrightarrow$ \textit{Z}\label{lex:Z}

With the spell-out procedure recapped in \sectref{section:Starke2018}, \textit{ROOT} will spell out the range of features from F$_{1}$ to F$_{3}$ by \textsc{stay}, as shown in: 

\ex.\label{f3} \begin{forest}nice empty nodes, for tree={l sep=0.65em,l=0,calign angle=63}
[F$_{3}$P [F$_{3}$][F$_{2}$P [F$_{2}$][F$_{1}$P [F$_{1}$]]]
]{\draw (.east) node[right]{$\Rightarrow$ \textit{ROOT}}; }
 \end{forest}

Next, the merger of F$_{4}$ will take place. The default option for \isi{spell-out}, \textsc{stay}, does not result in lexical insertion since there is no lexically stored tree listed in \ref{lex:list} that matches the syntactic structure that ranges from F$_{1}$ up to F$_{4}$, as indicated in:

\ex.
\begin{forest}nice empty nodes, for tree={l sep=0.65em,l=0,calign angle=63}
[F$_{4}$P [F$_{4}$][F$_{3}$P [F$_{3}$][ [F$_{2}$][ [F$_{1}$]]]
]{\draw (.east) node[right]{$\Rightarrow$ \textit{ROOT}}; }
]{\draw (.east) node[right]{$\Rightarrow$ no match}; }]
 \end{forest}

In this case, the movement of the previously spelled out constituent is attempted: F$_{3}$P \textit{ROOT} moves on top of F$_{4}$P. This movement takes place in line with the \isi{Shortest Move} condition, whereby the evacuated material has to adjoin right above the node where matching takes place. This step is shown in the following: 

\ex. 
\begin{forest}nice empty nodes, for tree={l sep=0.6em,l=0,calign angle=63}
[F$_{4}$P, s sep=45pt [F$_{3}$P, name=tgt [F$_{3}$][ [F$_{2}$][ [F$_{1}$]]]
]{\draw (.east) node[right]{$\Rightarrow$ \textit{ROOT}}; } 
[F$_{4}$P [F$_{4}$][..., name=t]]{\draw (.east) node[right]{$\Rightarrow$ \textit{X}}; }]
\draw[dashed,->,>=stealth,overlay] (t) [in=-150,out=-120,looseness=1.75]  to (tgt);
\end{forest} 

The remnant F$_{4}$P will spell-out as the suffix \textit{X}, since it matches the lexically stored tree in \ref{lex:X}. 
\par
Next, the merger of F$_{5}$ will take place and the situation will repeat: following the evacuation of its complement node F$_{4}$P, the remnant F$_{5}$P will spell out as \textit{Y}, as the constituent formed in this way matches the lexically stored tree in \ref{lex:Y}. This is shown in the following:

\ex.\label{f5} 
\resizebox{\linewidth}{!}{%
\begin{forest}nice empty nodes, for tree={l sep=0.6em,l=0,calign angle=63}
[F$_{5}$P, s sep=-1pt [F$_{5}$] [F$_{4}$P, s sep=25pt [F$_{3}$P [F$_{3}$][ [F$_{2}$][ [F$_{1}$]]]
]{\draw (.east) node[right]{$\Rightarrow$ \textit{ROOT}}; } 
[F$_{4}$P [F$_{4}$][...]]{\draw (.east) node[right]{$\Rightarrow$ \textit{X}}; }]]
\end{forest}
 \hskip -0.25cm $\leadsto$  
\begin{forest}nice empty nodes, for tree={l sep=0.6em,l=0,calign angle=63}
[F$_{5}$P, s sep=25pt [F$_{4}$P,name=tgt, s sep=25pt [F$_{3}$P [F$_{3}$][ [F$_{2}$][ [F$_{1}$]]]
]{\draw (.east) node[right]{$\Rightarrow$ \textit{ROOT}}; } 
[F$_{4}$P [F$_{4}$][...]]{\draw (.east) node[right]{$\Rightarrow$ \textit{X}}; }]
[F$_{5}$P [F$_{5}$] [...,name=t]]{\draw (.east) node[right]{$\Rightarrow$ \textit{Y}}; }
]]
\draw[dashed,->,>=stealth,overlay] (t) [in=-125,out=-115,looseness=1.75]  to (tgt);
\end{forest}
}

In this way, the \textit{Y} \isi{morpheme} will come out as the outer suffix in the tri-mor\-phe\-mic structure \textit{ROOT-X-Y}.
\par\largerpage[-1]
Let us now suppose that along \textit{ROOT-X-Y}, there is also a form \textit{ROOT-Z}, which lexicalizes the span that ranges from F$_{1}$ up to F$_{6}$, that is a span of features which is minimally bigger than the one that is realized by \textit{ROOT-X-Z}.  The question now is: how can the addition of F$_{6}$ at the next cycle, shown in \ref{F6}, result in the \isi{reduction} in the number of suffixes on the \textit{ROOT}: from \textit{ROOT-X-Y} to \textit{ROOT-Z}?

\ex.\label{F6}
\begin{forest}nice empty nodes, for tree={l sep=0.5em,l=0,calign angle=63}
[F$_{6}$P, s sep=-45pt [F$_{6}$] [F$_{5}$P, s sep=35pt [F$_{4}$P,name=tgt, s sep=35pt [F$_{3}$P [F$_{3}$][ [F$_{2}$][ [F$_{1}$]]]
]{\draw (.east) node[right]{$\Rightarrow$ \textit{ROOT}}; } 
[F$_{4}$P [F$_{4}$][...]]{\draw (.east) node[right]{$\Rightarrow$ \textit{X}}; }]
[F$_{5}$P [F$_{5}$] [...,name=t]]{\draw (.east) node[right]{$\Rightarrow$ \textit{Y}}; }]
]{\draw (.east) node[right]{$\Rightarrow$ no match}; }
\end{forest}

There are in principle two possible ways of deriving the \isi{reduction} in the number of suffixes from \textit{ROOT-X-Y} to \textit{ROOT-Z}. One involves \isi{backtracking} and trying an alternative \isi{spell-out} option (the option that kicks in whenever \textsc{stay} is unsuccessful and evacuation of nodes spelled out earlier is required, see \citealt[160--168]{Pantcheva2011}). The other one does not require \isi{backtracking} and, instead, it involves adding \isi{subextraction} to the list of spell-out driven movements. Let us outline both possibilities in turn.

\subsection{Backtracking}\label{sec:back}\largerpage[-2]

The derivation in \ref{F6} with the added F$_{6}$ is not going to surface as \textit{ROOT-Z} if we apply \textsc{stay}, \textsc{move spec-to-spec}, or \textsc{snowball}, since none of these operations reduces the number of affixes. Instead, the \isi{reduction} can be obtained if the derivation backtracks down to F$_{2}$P and, instead of spelling out F$_{3}$ by \textsc{stay} as in \ref{f3},  F$_{3}$ is spelled out following the movement of F$_{2}$P, which is realized as \textit{ROOT} as a subset \isi{spell-out} of the lexical entry in \ref{lex:ROOT} (on the strength of the \isi{Superset Principle} in \pref{superset}). \is{backtracking} As shown in \Next, such an evacuation of F$_{2}$P will allow the F$_{3}$P remnant to be spelled out as \textit{Z}, the subset of the lexical entry in \ref{lex:Z}.

\ex. 
\begin{forest}nice empty nodes, for tree={l sep=0.7em,l=0,calign angle=63}
[F$_{3}$P [F$_{3}$][F$_{2}$P [F$_{2}$][F$_{1}$P [F$_{1}$]]
]{\draw (.east) node[right]{$\Rightarrow$ \textit{ROOT}}; }]
 \end{forest}
 \hskip 0.25cm $\leadsto$ 
\begin{forest}nice empty nodes, for tree={l sep=0.5em,l=0,calign angle=63}
[F$_{3}$P,s sep=45pt [F$_{2}$P, s sep=15pt, name=tgt [F$_{2}$][F$_{1}$P [F$_{1}$]]
]{\draw (.east) node[right]{$\Rightarrow$ \textit{ROOT}}; }
[F$_{3}$P [F$_{3}$][...,name=t]]{\draw (.east) node[right]{$\Rightarrow$ \textit{Z}}; }
]
\draw[dashed,->,>=stealth,overlay] (t) [in=-140,out=-120,looseness=2.25]  to (tgt);
 \end{forest}\vspace*{2\baselineskip}

\noindent The remaining features F$_{4}$, F$_{5}$, and F$_{6}$ will all be spelled out by successive cyclic movement of F$_{2}$P \textit{ROOT}. Such a movement will create intermediate specifier positions, whose sisters can all be spelled out as \isi{morpheme} \textit{Z} in line with the lexical entry in \ref{lex:Z}.\footnote{Let us bare in mind that on the strength of the \isi{Superset Principle}, the remnants left by the evacuation of F$_{2}$P \textit{ROOT} from F$_{3}$P up to F$_{5}$P will spell out as the subset and the remnant F$_{6}$P will spell out as a superset of the lexically stored tree in \ref{lex:Z}.
} %end of fn 
This is illustrated in \ref{so:back}. Such a derivation involving \isi{backtracking} down to F$_{2}$P results in the morphological structure \textit{ROOT-Z}, a desired result.
\par
A theoretical challenge for such an analysis is that it requires backtracking from F$_{6}$ all the way down to F$_{2}$P before \textsc{spec-to-spec} movement of F$_{2}$P \textit{ROOT} can take place. This contrasts with how backtracking applies in the spell-out algorithm articulated in  \sectref{section:Starke2018}, where a failure to spell out \isi{feature} F\textsubscript{n} requires a return to the previous cycle F\textsubscript{n$-$1} and trying a different spell-out option for F\textsubscript{n}. In the situation outlined above, the \isi{reduction} in the number of suffixes on the \textit{ROOT} requires going back a few cycles before a different spell-out option can apply. 

\ex.\label{so:back} Deriving \isi{reduction} in the number of morphemes with \isi{backtracking}\\[1ex]
{\small \begin{forest}nice empty nodes, for tree={l sep=0.5em,l=0,calign angle=63}
[F$_{6}$P,s sep=35pt 
[F$_{2}$P,name=F6, s sep=15pt [F$_{2}$][F$_{1}$P [F$_{1}$]]
]{\draw (.east) node[right]{$\Rightarrow$ \textit{ROOT}}; }
[F$_{6}$P, s sep=15pt [F$_{6}$] [F$_{5}$P,s sep=15pt [...,name=F5] 
[F$_{5}$P,s sep=12pt [F$_{5}$]
[F$_{4}$P,s sep=11pt [...,name=F4] [F$_{4}$P,s sep=10pt [F$_{4}$]
[F$_{3}$P, s sep=10pt [...,name=F3] [F$_{3}$P [F$_{3}$][..., name=F2]]]]]]]
]{\draw (.east) node[right]{$\Rightarrow$ \textit{Z}}; }
]
 \draw[dashed,->,>=stealth,overlay] (F3) to[out=south west,in=south west,looseness=1.55] (F4);
  \draw[dashed,->,>=stealth,overlay] (F4) to[out=south west,in=south west,looseness=1.55] (F5);
   \draw[dashed,->,>=stealth,overlay] (F5) [in=-145,out=-125,looseness=1.75]  to (F6);
      \draw[dashed,->,>=stealth,overlay] (F2) to[out=south west,in=south west,looseness=1.75] (F3);
 \end{forest}} 


\subsection{Subextraction}\label{sec:subext}

The other possibility of deriving the \isi{reduction} in the number of suffixes from \textit{ROOT-X-Y} to \textit{ROOT-Z} is a \isi{subextraction} of a previously spelled out constituent from the specifier node in which it is embedded.
I will continue to refer to this type of \isi{spell-out} procedure simply as \textsc{subextract}.\is{spell-out algorithm} 
\par
In order to illustrate this operation, let us return to \ref{F6}, the cycle where the feature F$_{6}$ becomes merged on top of F$_{5}$P, the structure already spelled out as \textit{ROOT-X-Y}.  
In such a representation, the subextraction of F$_{3}$P \textit{ROOT} from F$_{4}$P (the specifier of F$_{5}$) will create a remnant constituent that comprises features F$_{4}$, F$_{5}$, and F$_{6}$, as shown in \Next.\pagebreak

\ex.\label{so:Z2nd}Deriving \isi{reduction} in the number of \isi{morpheme}s by \textsc{subextract}\\[1ex]
\begin{forest}nice empty nodes, for tree={l sep=0.5em,l=0,calign angle=63}
[F$_{6}$P, s sep=25pt [F$_{3}$P, name=L3 [F$_{3}$][ [F$_{2}$][ [F$_{1}$]]]
]{\draw (.east) node[right]{$\Rightarrow$ \textit{ROOT}}; }
[F$_{6}$P [F$_{6}$] [F$_{5}$P, s sep=35pt [F$_{4}$P [...,name=T3] 
[F$_{4}$P [F$_{4}$][...]]{\draw (.east) node[right]{$\Rightarrow$ \textit{X}}; }]
[F$_{5}$P, s sep=10pt [F$_{5}$] [...]]{\draw (.east) node[right]{$\Rightarrow$ \textit{Y}}; }]
]{\draw (.east) node[right]{$\Rightarrow$ \textit{Z}}; }
]
\draw[dashed,->,>=stealth,overlay] (T3) [in=-160,out=-130,looseness=1.25]  to (L3);
\end{forest}

\noindent As indicated above, the remnant F$_{6}$P created in this way can be spelled out as \textit{Z} if the lexical entry for this exponent is defined as in \Next rather than in \ref{lex:Z} (in other words, the lexically stored tree for the exponent \textit{Z} must look different in the derivation of \textit{ROOT-Z} obtained by \isi{backtracking} and by \textsc{subextract}). \is{subextraction}

\ex.\label{lex:Z2nd} Lexical entry for \textit{Z} (2nd version, alternative to \pref{lex:Z})\\[1ex]
[ F$_{6}$ [[ F$_{4}$ ][ F$_{5}$ ]]] $\Leftrightarrow$ \textit{Z}

The insertion of \textit{Z} into the remnant node F$_{6}$P in \ref{so:Z2nd} will over-ride the earlier spell-outs of \textit{X} and \textit{Y} in a familiar way resulting in the morphological structure \textit{ROOT-Z}, a desired result.\largerpage
\par
A theoretical challenge for such a solution is that a \isi{subextraction} from a specifier that has been formed by movement at an earlier cycle violates the so-called \textsc{Freezing Condition},  \is{Freezing Condition} which can be formalized on the basis of \cite{Wexler-Culicover1980} in the following way:\footnote{The formulation in \ref{freezing} is in fact a paraphrase of Wexler \& Culicover's \citeyearpar[542]{Wexler-Culicover1980} Generalized Freezing Principle, whose formulation as in \Next  below has broader restrictions than extractions from raised phrases.    

\noindent\parbox{\linguexfootnotewidth}{\ex. Generalized Freezing Principle\\[0.5ex] 
A node is frozen if (a) its immediate structure is non-base, or (b) it has been raised.

}

\noindent The range of structures that are constrained by the protasis in (a) is irrelevant to the present discussion.}%end of fn

\ex.\label{freezing} Freezing Condition\\[0.5ex]
A moved constituent becomes an island for extraction. \is{Freezing Condition}

 In \ref{so:Z2nd}, the evacuation of F$_{3}$P \textit{ROOT} takes place from F$_{4}$P \textit{ROOT-X}, a node that has become evacuated and remerged in a successful attempt to spell out F$_{5}$P (as \textit{Y}). 
Assuming the Freezing Condition, the ban on extraction in the representation in \ref{so:Z2nd} is not limited to F$_{3}$P \textit{ROOT} but also to its sister node F$_{4}$P \textit{X}. This issue is not merely theoretical in nature since the extraction of the right branch constituent, i.e. the one that corresponds to F$_{4}$P \textit{X} in \ref{so:Z2nd}, is instantiated by the so-called \textsc{case peeling} \is{peeling} derivation argued for in \citet[\S4]{Caha2009}.
\par
Peeling is argued in \citet[\S4]{Caha2009} to derive case conversions, that is derivations where an NP argument changes its case depending on the syntactic position it occupies.\footnote{The term ``peeling'' has its origin in \citet[195]{Cardinaletti-Starke1999}, who put forth a tripartition of pronouns into clitic\,$<$\,weak\,$<$\,strong. Such a hierarchy is based on structural containment that is described there in terms of \isi{peeling} that applies to layers of syntactic structure: weak pronouns are ``peeled'' strong pronouns, and clitics are ``peeled'' weak pronouns.\is{containment} 
} %end of fn
 For example, case conversion in \ili{English} is overtly visible in passivization involving pronouns, as in \ref{army}, where the accusative object \textit{her} becomes the nominative \textit{she} when it is raised to the subject position.

\ex.\label{army} 
\a. The army promoted her.\textsc{acc} to a higher rank.
\b. She.\textsc{nom} was promoted to a higher rank.

Case conversion between four different morphologically marked cases is observed in `spray/load' alternations in \ili{Slavic}. The alternations involving instrumental, genitive, accusative, and nominative case can be illustrated by the set of sentences with the \ili{Polish} prefixed \isi{verb} \textit{za-\l{}adowa\'c} `load' in \Next, where the case markers that participate in the conversion are bolded. 

\ex.\label{trawa} `Spray/load' alternation in \ili{Polish} (\citealt[241--242]{Wiland-NELS48})
\ag. Jan za\l adowa\l {}  ci\k{e}\.zar\'owk-\k{e} traw-\k{\bf{a}}\label{aaa}\\
Jan-{\sc nom} loaded truck-{\sc acc} grass-{\sc inst} \\
\strut `Jan loaded the truck with grass.'
\bg.\label{gen}za\l adowa-nie traw-{\bf{y}} na ci\k{e}\.zar\'owk-\k{e}\\
load-{\sc ing} grass-{\sc gen} on truck-{\sc acc}\\
\strut  `the loading of the grass on the truck'
\cg. Jan za\l adowa\l {} traw-\k{\bf{e}} na ci\k{e}\.zar\'owk-\k{e}\label{bbb}\\
Jan-{\sc nom} loaded grass-{\sc acc} on truck-{\sc acc}\\
\strut  lit.\ `Jan loaded the grass onto the truck.'
\dg. Traw-{\bf a} zosta\l-a za\l adowa-n-a na ci\k{e}\.zar\'owk-\k{e}\\
grass-{\sc nom} became-{\sc agr} loaded-{\sc prt-agr} on truck-{\sc acc} \\
\strut `The grass was loaded on the truck.'

This set shows the conversion between instrumental, genitive, accusative, and nominative marking on the Figure NP \textit{traw-} `grass', which is linked to the position in which the NP is licensed.    
Assuming the case \isi{fseq} in \ref{case-fseq}, Caha argues that the case conversion is derived according to \ref{case-peeling}, where case-forming features K\textsubscript{n} projected on top of the NP \textit{traw-} `grass' become stranded by the movements of their complement. \is{feature}

\ex. \label{case-peeling}Case \isi{peeling} (\citealt[142--145]{Caha2009})

\vskip -0.5cm
\hskip -1cm \resizebox{\textwidth}{!}{\small{\begin{forest}nice empty nodes, for tree={l sep=0.6em,l=0,calign angle=63}
[~,s sep=15pt [NomP,name=N2 [K$_{1}$][NP [\textit{traw-}\\`grass' ,roof]]]
[, s sep=0pt [...]
[, s sep=10pt [AccP,name=A2 [K$_{2}$][\sout{NomP},name=N1]]
[, s sep=0pt [...]
[, s sep=10pt [GenP,name=G2 [K$_{3}$][\sout{AccP},name=A1]]
[, s sep=10pt [...][InstP [K$_{6}$][DatP [K$_{5}$][LocP [K$_{4}$][\sout{GenP},name=G1]]]]]]]]]]
\draw[dashed,->,>=stealth] (G1) [in=-160,out=-130,looseness=1.1]  to (G2);
\draw[dashed,->,>=stealth] (A1) [in=-160,out=-130,looseness=1.5]  to (A2);
\draw[dashed,->,>=stealth] (N1) [in=-160,out=-135,looseness=1.5]  to (N2);
\end{forest}}}

\vskip -1cm

\noindent An argument for case \isi{peeling} is based on the fact that the case conversions in both the passive transformation and the \ili{Polish} `spray/load' alternation involve a change that is constrained by the case \isi{fseq} in \ref{case-fseq}: a bigger (containing) case converts into a smaller (contained) one, not vice versa. \citet[143--146]{Caha2009} offers a detailed discussion of the role of case selectors in case \isi{peeling}.\is{containment} In essence, the triggering mechanism for case peeling is the presence of selecting heads in the clause, which attract a matching case phrase -- much in the spirit of the probe-goal system of \cite{Chomsky2000}, where the probe attracts a matching goal in its c-commanding domain. For instance, an accusative case selector such as a transitive V head will attract the AccP-layer from its c-commanding domain; a nominative case selector such as the T head will attract the NomP-layer from its c-commanding domain, and so on. The result is that in a single derivation, case-marked NPs will pass through multiple case positions.  As acknowledged in \citet[146]{Caha2009}, such a view stands in opposition to most other theories of case derivation, including \cite{Chomsky2000}. 
\par
In the sense that both \textsc{subextract} illustrated in \ref{so:Z2nd} and case \isi{peeling} in \ref{case-peeling} involve movement out of a moved node, the two violate the \isi{Freezing Condition} defined as in \ref{freezing}. An instantaneous solution to this challenge, based on empirical evidence, is to abandon the description of freezing effects in terms of an all-out ban on extractions from a moved constituent. 
\par
Such a solution is motivated by the fact that, in parallel to evidence in favor of freezing properties of displacement, there is fairly strong evidence for the existence of well-formed extractions from fronted constituents. More precisely, on the one hand extractions have been argued to be blocked from adverbial phrases that have undergone locative inversion in \ili{English} (\citealt{Huy1976}), from extraposed PPs in \ili{English} (\citealt{Wexler-Culicover1980}), from phrases moved to SpecCP (\citealt{LS1992} about \ili{English};  \citealt{Fanselow1987,Grewendorf1989,Muller1998,Muller2010} about \ili{German}), from phrases moved to SpecTP (\citealt{Browning1991,Collins1994,BG2007} about \ili{English}), from preposed constituents that feed remnant movement in \ili{German} (\citealt{Muller1998}), as well as from \ili{English} topicalized PPs (\citealt{Postal1972}) and DPs (\citealt{LS1992}), among others.\footnote{See \cite{Corver2017} for a comprehensive overview of freezing effects. \is{Freezing Condition}
} % end of fn on Corver 2017
\par
On the other hand, examples of felicitous movements from moved constituents include extractions from pied-piped wh-phrases in \ili{Spanish} (\citealt{Torrego1985}), topicalization from subjects in \ili{German} (\citealt{Abels2007}), left-branch extraction of wh-words from fronted wh-phrases in \ili{Polish} (\citealt{Wiland2010}), and object extraction from fronted constituents leading to the non-canonical OVS order in \ili{Polish} (\citealt{Wiland2016}). Likewise, any movement out of an object phrase in canonical SOV languages is going to be an instance of anti-freezing under \citeauthor{Kayne1994}'s \citeyear{Kayne1994} Antisymmetry theory, whereby SOV orders are all derived by object raising from an underlying SVO structure. Yet, as pointed out in \cite{Corver2017}, extractions from objects in SOV languages are attested for instance in \ili{Dutch}, as shown in the following, where the well-formed fronting of \textit{wat} `what' takes place from the preverbal object:

\ex. \ili{Dutch} (\citealt[26]{Corver2017})
\ag.[]\hspace{-22pt}Wat\textsubscript{i} heb jij nog nooit {[ t\textsubscript{i}} voor {dingen ]} gezegd\\
\hspace{-22pt}what have you yet never {} for {things} said\\
\hspace{-22pt}\strut `What kind of things haven't you ever said?'
 
\noindent
There are at least two approaches to freezing that describe it in non-categorical 
\makebox[\linewidth][s]{terms: the feature-driven freezing (\citealt{Bux2008,Lohndal2011}) and Criterial}
Freezing (\citealt{Rizzi2006,Rizzi2007,RizziShlonsky2007}). The feature-driven approach submits that only A-movement for case checking will result in the moving NP becoming opaque for \isi{subextraction}. Under this approach, case peeling -- which is motivated by case selection (i.e. de facto checking) in Caha's work -- should be blocked. In turn, Criterial Freezing submits that while a moving constituent that targets a ``criterial'' (checking) position, becomes opaque to further movements.\is{feature} Subextraction from a constituent in such a position, however, is possible. As pointed out in \citeauthor{Caha2009} \citeyearpar[146--147]{Caha2009}, Criterial Freezing not only renders case \isi{peeling} to be licit but it also correctly predicts that NP-movement into a case position is terminated when a nominative position in the clause structure is reached. This is so since \isi{peeling} involves a \isi{subextraction} from a  constituent merged in its selected position (e.g. movement of NomP from within AccP in \pref{case-peeling}) rather than cyclic movement of the same case layer through different positions in the clause (e.g. no second movement of AccP in lieu of NomP in \pref{case-peeling}). 
\par
Under non-categorical approaches to freezing effects, both \isi{peeling} derivations and \textsc{subextract} are in principle admissible in grammar. \is{Freezing Condition} More specifically, unlike case peeling that is predicted to be admissible under \isi{Criterial Freezing} but not under the feature-driven analysis, \isi{subextraction}s are admissible under both. This is so since no movement leading to the representation in \ref{so:Z2nd} targets a designated checking (or ``criterial'') position or is feature-driven. Instead, all these movements simply form a sister to the node that is targeted by \isi{spell-out} at a given cycle -- in the same way as spec-to-spec and snowballing movements do in the spell-out procedure.\is{spell-out algorithm}

\subsection{Verb stem alternation}

One domain where we find what looks to be a \isi{reduction} in the number of \isi{morpheme}s is a semelfactive-iterative alternation \is{iterative alternation} in Czech and \ili{Polish}, as shown in \Next, where a morphologically more complex \isi{semelfactive} (on the left) alternates with a less complex \isi{iterative} (on the right).

\ex. Czech \label{ch3:Cz}
\ag. 
kop-\textbf{n-ou}-t -- kop-\textbf{a}-t\\
kick-\textsc{n-ou}\textsubscript{theme}-\textsc{inf} {} kick-\textsc{aj}\textsubscript{theme}-\textsc{inf}\\
\strut `give a kick' \hskip 1.4cm `be giving kicks'\label{ch3:kop}
\bg.
\v{s}t\v{e}k-\textbf{n-ou}-t -- \v{s}t\v{e}k-\textbf{a}-t\\
bark-\textsc{n-ou}\textsubscript{theme}-\textsc{inf} {} bark-\textsc{aj}\textsubscript{theme}-\textsc{inf}\\
`give a bark' \hskip 1.4cm `bark repeatedly'

\ex. \ili{Polish}\label{ch3:P}
\ag. 
kop-\textbf{n-\k{a}}-\'c -- kop-\textbf{a}-\'c\\
kick-\textsc{n-ou}\textsubscript{theme}-\textsc{inf} {} kick-\textsc{aj}\textsubscript{theme}-\textsc{inf}\\
\strut `give a kick' \hskip 1.4cm `be giving kicks'
\bg. 
liz-\textbf{n-\k{a}}-\'c -- liz-\textbf{a}-\'c\\
lick-\textsc{n-ou}\textsubscript{theme}-\textsc{inf} {} lick-\textsc{aj}\textsubscript{theme}-\textsc{inf}\\
\strut `give a lick' \hskip 1.4cm `lick repeatedly'

For the present purposes, let us refer to \isi{verb} stems on the left, which denote single-stage events, as \isi{semelfactive}s and to the verb stems on the right, which comprise the \isi{root} and what is glossed here as the \textit{-aj} theme, as \isi{iterative}s.\footnote{\label{fn:glidetruncation}The theme vowel \textit{-aj} surfaces as /a/ before a suffix with a consonant in its onset such as the infinitival -\textit{t} (Cz)\,/\,-\textit{\'c} (Pol) but also before the past participle suffix as in \textit{szczek-a-\l{}} `bark-\textsc{aj-part}'. This is due to a cyclic phonological truncation rule in \ili{Slavic}, whereby glides become deleted before a consonant (see \citealt{Jakobson1948,Rubach1984}, among others): \is{glide truncation}

\noindent\parbox{\linguexfootnotewidth}{\ex.\label{GlideTr}Glide truncation\\[0.5ex] 
j, w $\rightarrow$ $\emptyset$  / \_ C$_{0}$

}

\noindent What is indicated in the glosses in (\pref{ch3:Cz}--\pref{ch3:P}) and later in the text are underlying, ``untruncated'' exponents of theme vowels. \is{thematic suffix}
} %end of fn on Glide Truncation
\par
If we follow the analysis of \isi{iterative}s as categories that in syn-sem terms are more complex than \isi{semelfactive}s (e.g. \citealt{Smith1997,Olsen1997,Egg2017}),
then the alternation in (\pref{ch3:Cz}--\pref{ch3:P}) comes out as puzzling since the iteratives are morphologically less complex than the \isi{semelfactive}s. Thus, if the iterative \textit{-aj} stems are structurally bigger than semelfactive \textit{-n-ou} stems, the \isi{spell-out} of a \isi{feature} added in their formation reduces the number of \isi{morpheme}s. This spell-out problem is outlined in the structural description below on the example of the stem \textit{kop-n-ou} `give a kick' of \ref{ch3:kop} (where VP is a stand-in for a semelfactive \isi{verb} stem and Asp is a stand-in for the feature extending a \isi{semelfactive} stem into an \isi{iterative} stem):

\ex.
\resizebox{\linewidth}{!}{\begin{forest}nice empty nodes, for tree={l sep=0.7em,l=0,calign angle=63}
[\textsc{semelfactive:}\\VP, s sep=10pt [, s sep=10pt
[\textsc{root}  [\textit{kop}\\`kick', roof]
] 
[\textsc{n} 
[\textit{-n}, roof] ]] 
[\textsc{theme} 
[\textit{-ou}, roof]]]
\end{forest}
\hskip 0.25cm $\leadsto$ \hskip 0.25cm
\begin{forest}nice empty nodes, for tree={l sep=0.7em,l=0,calign angle=63}
[\textsc{add feature:}\\AspP, s sep=-5pt [Asp]
[VP, s sep=10pt [, s sep=10pt
[\textsc{root}  [\textit{kop}\\`kick', roof]
] 
[\textsc{n} 
[\textit{-n}, roof] ]] 
[\textsc{theme} 
[\textit{-ou}, roof]]]]
\end{forest}
\hskip -0.25cm $\leadsto$ \hskip 0.75cm
\begin{forest}nice empty nodes, for tree={l sep=0.7em,l=0,calign angle=63}
[\textsc{spell-out:}\\~, s sep=15pt [\textsc{root}  [\textit{kop}\\`kick', roof]] 
[AspP [\textit{-aj}, roof]]]
\end{forest}}

There are in principle two ways to achieve the \isi{reduction} in the number of suffixes on the \isi{root}, from the \textit{-n-ou} sequence down to the single \textit{-aj}: by \isi{backtracking} or by \textsc{subextract}. I will consider both possibilities of deriving this \isi{reduction} in detail in Chapter \ref{chapter:explaining}.\is{subextraction}


\section{Summary and roadmap}

In this introductory chapter, I have outlined an approach to the realization of syntactic trees (i.e. hierarchical \isi{feature} structures) as morphological forms (i.e. linear sequences) based on phrasal \isi{spell-out} and a strictly cyclic lexical access, the two key features of \isi{Nanosyntax}. The strict cyclicity of lexical access  means that  every merger of a feature in a phrase marker is followed by an attempt to match it against the list of lexically stored trees and insert an exponent. If such an attempt is successful, the derivation either terminates (when no more features are merged) or advances to another cycle: the merger of another feature that is followed by an attempt to spell it out.
\par
The spell-out procedure summarized in \sectref{section:Starke2018} involves an order list of procedures that kick in after the merger of a \isi{feature} F, which comprise \textsc{stay}, \textsc{move spec-to-spec}, \textsc{snowball}, and \textsc{subderive}. In the next chapter, I consider the possibility of extending this list by \isi{subextraction}, a natural candidate to be added to the two types of spell-out driven movements along spec-to-spec and snowballing. In particular, I will consider if what looks to be a \isi{reduction} in the number of \isi{morpheme}s that we observe in the semelfactive-iterative alternation \is{iterative alternation} in Czech and \ili{Polish} can be better captured by an analysis based on \isi{backtracking} or by spell-out driven subextraction.\is{spell-out algorithm}
\par
While Chapter \ref{chapter:explaining} explores the possibility of explaining the alternation with \isi{subextraction}, subsequent chapters focus exclusively on the application of the so-far established set of spell-out possibilities -- the ones listed in \sectref{section:Starke2018} -- and do not rely on extending this list with \textsc{subextract}. 
\par
In particular, Chapter \ref{chapter:resolving} discusses the problem of morphological \isi{containment} of the Russian \isi{demonstrative} pronoun \textit{to} in the structure of the declarative \isi{complementizer} \textit{\v{c}-to}. Such a morphological inclusion is paradoxical given the analysis of demonstrative pronouns in \citeauthor{BaunazLander2017} (\citeyear{BaunazLander2017,Baunaz-Lander-Glossa}) as categories that syntactically contain declarative complementizers. The resolution of this paradox is going to rely on accommodating \isi{demonstrative} pronouns without definiteness marking such as the Russian \textit{to} into a cross-categorial \isi{paradigm} with \isi{complementizer}s and definiteness, analyzed as a separate category in the paradigm. The chapter also discusses how the application of the \isi{spell-out} algorithm allows us to explain the differences in the morphological structures of the declarative complementizers in Russian and in \ili{Polish}, another \ili{Slavic} language without definiteness marking.
\par
Chapters \ref{chapter:latvian} and \ref{chapter:basaa} extend the accommodation of non-definite demonstratives into the \isi{paradigm} with the declarative \isi{complementizer} to the languages from outside the \ili{Slavic} group. 
Chapter \ref{chapter:latvian} on Latvian deals with a similar type of morphological \isi{containment} problem as the one observed in Russian. Unlike in Russian, however, the \isi{containment} problem in Latvian concerns the complementizer \textit{k-a}, which is morphologically less complex than the \isi{relativizer} and the interrogative pronoun \textit{k-a-s} `what'. The latter are the categories that are syntactically smaller than the complementizer. \is{wh-pronoun}
\par 
In turn, Chapter \ref{chapter:basaa} resolves a problem with syncretic \is{syncretism} alignment in a \isi{paradigm} Basa\'a, a Bantu language spoken in Cameroon. The Basa\'a paradigm appears to show syncretism between the \isi{demonstrative} pronoun and the relativizer to the exclusion of the declarative \isi{complementizer}. Given the organization of cells in a paradigm with these categories advanced in \citeauthor{BaunazLander2017} (\citeyear{BaunazLander2017,Baunaz-Lander-Glossa}), the Basa\'a \isi{paradigm} is an instance of a \isi{*ABA} violation. It is argued in the chapter that inspecting the syntax behind the offending cells in the paradigm, the *ABA violation in Basa\'a is only apparent. 
\par
Chapter \ref{chapter:conclusion} summarizes the results and points out the gaps in the analyses that remain to be closed in future work.





 
