\chapter{An apparent *ABA violation in Basa\'a}\label{chapter:basaa}

\section{Introduction: an ABA paradigm}

The inclusion of Dem\textsubscript{indef} as the bottom of the hierarchy in \ref{reiteratedfseq} proposed in Chapter \ref{chapter:resolving} constitutes an essential ingredient of sorting out what appears to be an ABA pattern of \isi{syncretism} in \ili{Basa\'a} (\ili{Bantu}, A.43). \is{fseq}

\ex.\label{reiteratedfseq} 
Dem\textsubscript{def}\,$>$\,Comp\,$>$\,Rel\,$>$\,Wh\,$>$\,Dem\textsubscript{indef} \hfill (reiterated)

Namely, as shown in \tabref{ABA:Basaa}, the \ili{Basa\'a} \isi{paradigm} shows a Dem$=$Rel syncretism to the exclusion of Comp.

\begin{table}
\caption{Basa\'a}
\label{ABA:Basaa}
\begin{tabular}[t]{ l l l l l l }
\lsptoprule
\textsc{dem} 	& \textsc{comp} 	& \textsc{rel}  	& \textsc{wh}\\	
\midrule
n\'u\cellcolor[gray]{0.9} & $\emptyset$, & n\'u,\cellcolor[gray]{0.9} & k\'i\'i\\
 & lέ\cellcolor[gray]{0.8} & lέ\cellcolor[gray]{0.8} & \\
\lspbottomrule
\end{tabular}
\end{table}

\noindent The arrangement of the cells in the \ili{Basa\'a} \isi{paradigm}  in the same way as in the Germanic languages, as for instance in \ili{English}, \ili{Dutch}, German or \ili{Swiss German} in \tabref{table:no-wo} (partially repeated from \sectref{sec:paradajm}), results in the violation of the \isi{*ABA} generalization.

\begin{table}
\caption{Germanic}
\label{table:no-wo}
\begin{tabular}[t]{ l l l l l l }
\lsptoprule
& \textsc{dem} 	& \textsc{comp} 	& \textsc{rel}  	& \textsc{wh}\\	
\midrule
English & that\cellcolor[gray]{0.9} & that\cellcolor[gray]{0.9} & that\cellcolor[gray]{0.9} & what\\
Dutch & dat\cellcolor[gray]{0.9} & dat\cellcolor[gray]{0.9} & dat\cellcolor[gray]{0.9} 	& wat\\
German & das\cellcolor[gray]{0.9} & dass\cellcolor[gray]{0.9} & das\cellcolor[gray]{0.9} & was\\
Swiss German & das\cellcolor[gray]{0.9} & dass\cellcolor[gray]{0.9} & $\emptyset$ & was\\ 
\lspbottomrule
\end{tabular}
\end{table}

\noindent The description of the \ili{Swiss German} relative pronoun as the phonologically null marker in \tabref{table:no-wo} requires qualification, which shows a direction toward working out a solution for the refractory \ili{Basa\'a} \isi{paradigm} in \tabref{ABA:Basaa}.

\subsection{Excursus on the Rel-cell in Swiss German}

In \ili{Swiss German}, an invariant particle \textit{wo} introduces both locative relatives, as in \ref{wo-loc}, and headed relative clauses, as in \ref{wo-hr}. It is syncretic with the locative `where'.

\ex. \ili{Swiss German} (\citealt[ex. 42a,b]{vanR2003})
\ag. 
s huss wo de Hans wont\\
the house \textsc{wo} the Hans lives\\
\strut `the house where Hans lives'\label{wo-loc}
\bg.
s f\"ascht wo i gh\"o\"ort han das de Hans anegaat\\
the party \textsc{wo} I heard have that the Hans to.goes\\
\strut `the party that I have heard Hans is going to' \label{wo-hr}

However, \citeauthor{vanR1989} (\citeyear{vanR1989}, \citeyear{vanR2003}) shows that \textit{wo} is not a genuine \isi{relativizer} despite the fact that headed relatives in \ili{Swiss German} are never preceded by a distinct relative pronoun. We can see this, among others, when we compare \ili{Swiss German} with certain other \ili{Upper German dialects} where \textit{wo} either can or must be preceded by a relative \textit{d}-pronoun (see also \citealt{Salzmann2006} and \citealt{BB}).
This is shown in the following examples contrasting \ili{Bavarian} with \ili{Swiss German} (more precisely, the \ili{Z\"urit\"u\"utsch dialect}):

\ex. \ili{Bavarian} (\citealt[216]{Bayer1984})
\ag.[]\hspace{-22pt}I schenk 's dem Kind (des) wo mid da Katz spuid.\\
\hspace{-22pt}I give it the.\textsc{dat} child {\ }\textsc{rel} \textsc{wo} with the cat plays\\
\hspace{-22pt}\strut `I give it to the child that plays with the cat.' 


\ex. \ili{Swiss German} - Z\"urit\"u\"utsch (\citealt[4]{vanR2003}) 
\ag.[]\hspace{-22pt}I sch\"ank 's em chind (*das) wo mit de chatz spilt.\\
\hspace{-22pt}I give it the.\textsc{dat} child {\ \ }\textsc{rel} \textsc{wo} with the cat plays\\
\hspace{-22pt}\strut `I give it to the child that is playing with the cat.' 

The \textit{d}-pronoun strongly appears to qualify as a genuine \isi{relativizer} (in the sense that it belongs to the cross-categorial \isi{paradigm} with the declarative Comp).
\par
The contrast illustrated above, however, begs a question why \textit{wo}-relatives come with a relative pronoun in dialects like \ili{Bavarian} but not in \ili{Swiss German}. 
There is more than one possibility, including an analysis advanced in \citet{PennerBader1995} where it is argued on the basis of the \ili{Bernese dialect} of \ili{Swiss German} that the relative pronoun is a silent \textit{pro}. Also, an interesting insight about \textit{wo}-relatives in the \ili{Z\"urit\"u\"utsch dialect} is offered in \citet{vanR2003}, who argues that they are similar to the so-called aboutness `such that' relatives, which are found in \ili{Japanese} (\citealt[257]{Kuno1973}) and also in \ili{English} (\citealt[157]{Grosu2002}), as in \textit{A mathematical system such that two and two are four is Peano arithmetic}. If on the right track, this account further speaks against classifying \textit{wo} as a relative pronoun.
\par
While working out the right analysis of the \textit{wo}-relatives is a task of its own, what is important for the purposes of the data classification is that \textit{wo} is not a relative pronoun on par with \textit{das} and must therefore be kept separate from the \isi{paradigm} in \tabref{table:no-wo} in a similar way verbal \isi{complementizer}s are kept separate form the paradigm with the nominal complementizer (as for instance in Yoruba or Hausa as seen in (\pref{Yoruba}--\pref{Hausa}) in \sectref{sec:paradajm}).
\par
The point of this observation is that while describing the \ili{Swiss German} relative pronoun either as $\emptyset$ or \textit{wo} does not have consequences for syncretic alignment as neither form shows \isi{syncretism} with the remaining three categories in \tabref{table:no-wo}, the examination of the syntax behind the Dem-cell in \ili{Basa\'a} is going to inform us about the solution to the \isi{*ABA} problem. 

\subsection{Back to the Basa\'a paradigm} 

Perhaps an immediate attempt to resolve the \isi{*ABA} violation in \tabref{ABA:Basaa} is to assume that since the \isi{complementizer} that appears to disrupt the syncretic span between Dem and Rel is phonologically null, then the Comp layer is not projected in \ili{Basa\'a} at all. Such an explanation is challenged by the fact that a dialect of \ili{Basa\'a} does have an overt form of the declarative complementizer \textit{lέ}, as shown in: 

\ex. \ili{Basa\'a} (\citealt[ex. 30a in \S3]{Bassong2010})\label{bassong-ch3}
\ag.[]\hspace{-22pt}mɛ ŋ́-k\^al lέ Tonye a ŋ́-kŋ́ y\`a\'an\'i\\
\hspace{-22pt}I \textsc{pres}-say \textsc{comp} Tomye \textsc{sm} \textsc{pres}-go tomorrow\\
\hspace{-22pt}\strut `I say that Tonye will go tomorrow.'

This variant of the \isi{complementizer} is syncretic with the relativizer, as in shown in the following:\largerpage

\ex. Basa\'a (\citealt[ex. 22b in \S4]{Bassong2010})\label{le-rel}
\ag.[]\hspace{-22pt}ɓa\'ud\'u ɓ\'a gw\v{e} mal\v{e}t lέ a ŋ́-k\^al ɓɔ́ mam\\
\hspace{-22pt}students \textsc{sm} have teacher \textsc{rel} \textsc{sm} \textsc{pres-}tell them things\\
\hspace{-22pt}\strut `The students have a teacher that tells them stories.' 

According to Bassong, the \isi{relativizer} \textit{lέ} is indeclinable and its distribution in \isi{relative clause}s is more restricted than in the case of \textit{n\'u}. An intuitive option would be, thus, to further assume that Comp is a layer of structure that can be skipped -- but only on top of the \isi{paradigm} with the Rel \textit{n\'u} and not on top of the paradigm with the Rel \textit{lέ}. The liaison of these two assumptions, however, is unnecessary if the \ili{Basa\'a} demonstratives are indefinite since, as argued earlier, only definite demonstratives of the type found in Germanic languages are the categories that are structurally bigger than declarative complementizers and \isi{relativizer}s. 
\par
In what follows, I consider a wholesale different approach to resolving the  \isi{*ABA} problem in  \ili{Basa\'a}, the one which relies on inspecting the syntax of the categories behind the Dem and Rel cells in the offending paradigm in \tabref{ABA:Basaa}.

\section{Basa\'a demonstratives}

The first step toward resolving this problem involves contrasting the demonstrative \textit{n\'u} with the Germanic demonstratives and classifying it as the smallest rather than the biggest category in the ``Dem\textsubscript{def}\,$>$\,Comp\,$>$\,Rel\,$>$\,Wh\,$>$\,Dem\textsubscript{indef}'' sequence. The classification of the demonstrative \textit{n\'u} as indefinite, however, requires qualification since \ili{Basa\'a} does have morphological marking of specificity.
\par
 Let us consider the following. \ili{Basa\'a} demonstratives show noun class concord with the noun they apply to. The demonstratives are morphologically distinguished between the proximal (close to speaker), the medial (close to hearer), and the distal (far from speaker and hearer), as shown on the example of class 1 \textit{n\'u} and class 5 \textit{l\'i} below (examples \pref{smierc}--\pref{Bas:liwanda}) are from \citealt{Makasso2010}).\footnote{See \cite{Hyman2003} for an exhaustive list of demonstratives of all nominal classes in \ili{Basa\'a}.
} %end of fn

\ex.\label{smierc} 
\ag. \{ l\'in\'i / l\'i / {l\'i\'i \}} liw\'and\'a \\
 {} 5.\textsc{prox} {} 5.\textsc{med} {} {5.\textsc{dist}} 5.friend\\
\strut `this/that friend'
\bg. \{ n\'un\'u / n\'u / {n\'u\'u \}} mut \\
 {} 1.\textsc{prox} {} 1.\textsc{med} {} {1.\textsc{dist}} 1.person\\
\strut `this/that person'

\noindent In \ili{Basa\'a}, the demonstratives can appear before or after the nouns they modify. Pre-nominal demonstratives receive a focus interpretation, while a noun that is post-modified by a demonstrative is unmarked with respect to information structure (non-focus) and it is obligatorily prefixed with the augment \textit{\'i-}, which marks definiteness/specificity (\citealt{Jenks-etall}), as shown in the following: \is{morpheme}

\ex.\label{Bas:i}
\ag.
\textbf{\'i}-mut\textsubscript{1} n\'u\\
\textsc{aug}-1.person 1.that.\textsc{dem}\\
\strut `that person' 
\bg.
n\'u mut\\
1.that.\textsc{dem} 1.person\\
\strut `THAT person'\label{B:unprefixednoun} 

\noindent
This description holds for all classes of demonstratives and for all values of the proximal-medial-distal contrast:

\ex. \label{Bas:liwanda}
\ag. \{ l\'in\'i / l\'i / {l\'i\'i \}} liw\'and\'a \\
 {} 5.\textsc{prox} {} 5.\textsc{med} {} {5.\textsc{dist}} 5.friend\\
\strut `this/that friend'
\bg.  l\textbf{\'i\textsuperscript{↓}}-w\'and\'a \{ l\'in\'i / l\'i / {l\'i\'i \}}\\
 \textsc{aug.}-5.friend  {} 5.\textsc{prox} {} 5.\textsc{med} {} {5.\textsc{dist}}\\
\strut `this/that friend'

Since these demonstratives do not have definiteness morphology, we can classify them as indefinite on par with \ili{Russian}, \ili{Polish}, \ili{Czech}, and \ili{Latvian} demonstratives. What sets the \ili{Basa\'a} demonstratives apart from the latter is that, descriptively speaking, the first participate in contextual licensing of an augment prefix on the noun they modify, but other than that there is no trace of the Def ingredient in their structure that qualifies them as the biggest category in the sequence in \ref{reiteratedfseq}. \is{morpheme} \is{fseq}
\par
However, the fact that we are able to accommodate indefinite demonstratives as the smallest category in this sequence, which results in the reordering of the cells as in \tabref{table3}, does not resolve the \isi{*ABA} problem but merely pushes it to a different place of the \isi{paradigm} where the non-syncretic Wh is now sandwiched between the syncretic forms for Rel and Dem\textsubscript{indef}. \is{syncretism}

\begin{table}
\caption{Reordered paradigm in Basa\'a}
\label{table3}
\begin{tabular}[t]{ l l l l l l }
\lsptoprule
\textsc{dem}\textsubscript{def} & \textsc{comp} 	& \textsc{rel}  	& \textsc{wh} & \textsc{dem}\textsubscript{indef}\\	
 \midrule
& $\emptyset$, & n\'u,\cellcolor[gray]{0.9} & k\'i\'i  & n\'u\cellcolor[gray]{0.9}\\
& lέ\cellcolor[gray]{0.8} & lέ\cellcolor[gray]{0.8} & \\
\lspbottomrule
\end{tabular}
\end{table}

\section{Non-wh-relatives in Basa\'a}

The key to resolving this problem is the observation that a similar distribution between the  augment \textit{\'i}-prefix on the head noun and a demonstrative pronoun we see in (\pref{Bas:i}--\pref{Bas:liwanda}) holds in headed \isi{relative clause}s, too, with the one essential difference: the augment \textit{\'i}-prefix is optional in relative clauses.
\par 
In both subject and object \isi{relative clause}s in \ili{Basa\'a}, the medial demonstrative pronoun is the one which shows \isi{syncretism} with the relative pronoun. This is  shown below on the example of class 1 medial \textit{n\'u}.

\ex. \citet[153--4]{Makasso2010}\label{Bas:big}
\ag. 
mɛ ŋ́ gwέs m\^ut\textsubscript{i} (n\'u)  [ \_\textsubscript{i}  a y\'e mb\'om ] \\
I \textsc{pres} like 1.person 1.\textsc{rel} {} {} 1.\textsc{sbj} \textsc{cop} 9.big\\
\strut `I like a person that is big/important.' 
\bg. 
mɛ \'n y\'eŋ m\'a\'aŋgέ\textsubscript{i} (n\'u) [ mɛ \'n y\'i \_\textsubscript{i} ]\\
\textsc{1sg} \textsc{pres} seek 1.child \textsc{1.rel} {} \textsc{1sg} \textsc{pr} know\\
\strut `I'm looking at the child that I know.' 

As pointed out in \citet{Makasso2010}, while the augment \textit{\'i}- is obligatory on nouns post-modified by demonstratives, it is optional on nouns that are heads of relative clauses, as shown in \Next, in which case the noun phrase is interpreted as indefinite.

\ex.
\ag.
(\textbf{\'i})-mut\textsubscript{i} n\'u [ \_\textsubscript{i} a b\'i \textsuperscript{↓}jέ b\'ijέk ]\\
\textsc{aug}-1.person 1.\textsc{rel} {} {} \textsc{1.sbj} \textsc{pst} eat 8.food {}\\
\strut `that person that ate the food'\label{RelC:i}
\bg.
n\'u (*\textbf{\'i})-mut\textsubscript{i} [ \_\textsubscript{i} a b\'i \textsuperscript{↓}jέ b\'ijέk ]\\
1.that \textsc{aug}-1.person {} {} \textsc{1.sbj} \textsc{pst} eat 8.food\\
\strut `THAT person that ate the food'\label{RelC:senza-i}


\noindent A two-step analysis of relativization in \ili{Basa\'a} which covers these facts is put forward in \cite{Jenks-etall}, whose central ingredient of the solution the \isi{*ABA} problem involves the derivation of the pre-nominal placement of the demonstrative in the noun phrase from its post-nominal placement, as outlined in \ref{DFCF}.

 \ex.\label{DFCF}
\setlength{\arrowht}{3ex}
\newcommand*\cgdepthstrut{{\vrule height 0pt depth \arrowht width 0pt}}
\renewcommand\eachwordone{\cgdepthstrut\rmfamily}
\renewcommand\glt{\vskip -\topsep}
\let\trans=\glt
\newcommand\arrowex{\setlength{\arrowht}{1ex}\ex}
[\textsubscript{DP} \tikzmark{n}\'u\textsubscript{Dem} (*\textbf{\'i}-) [\textsubscript{NP} mut ]  \tikzmark{t}  ]
 \arrow{t}{n}

\vskip 0.45cm
Such a derivation captures the complementary distribution between the augment marker \textit{\'i-} and the pre-nominal demonstrative in terms of blocking. Specifically, in
 \citeauthor{Jenks-etall}'s \citeyearpar{Jenks-etall} account this instantiates a ``generalized Doubly-filled Comp Filter'' (DFCF), whereby either a head or its specifier can be lexically realized. For \ref{DFCF} it means that \textit{\'i-} in the D-head position cannot be lexicalized when the demonstrative moves to its specifier from a post-nominal position. The analysis advanced here does not depend on the explanation based on a generalized DFCF, instead, it is enough for us to observe that the fronting of Dem blocks the merger of the augment marker. 
\par
The other ingredient of \citeauthor{Jenks-etall}'s \citeyearpar{Jenks-etall} account involves the derivation of \isi{relative clause}s in \ili{Basa\'a} via head raising in the way advanced in \cite{Kayne1994}. Let us note that such an approach to the relative clause formation is in agreement with what has been argued for other \ili{Bantu} languages (see e.g. \citealt{Ngonyani2001} and \citealt{Carstens2005}). 
\par
In \citeauthor{Kayne1994}'s \citeyearpar{Kayne1994} analysis, the head nouns are merged as specifiers of the \isi{relative clause}, which can be selected by the D-head.
 This gives us the following result for the derivation of headed relative clauses (labelled as RelP in the derivations below) with the pre-nominal demonstrative in \ili{Basa\'a}.

\ex. Derivation of a \isi{relative clause} with a post-nominal demonstrative following \citet[34]{Jenks-etall}
\ag.\'i-mut\textsubscript{i} n\'u [ \_\textsubscript{i} a b\'i \textsuperscript{↓}jέ b\'ijέk ]\label{optionalnu} \\
\textsc{aug}-1.person 1.\textsc{rel} {} {} \textsc{1.sbj} \textsc{pst} eat 8.food {}\\
\strut `that person that ate the food'
\b.\label{nu1} 
\begin{forest}nice empty nodes, for tree={l sep=0.6em,l=0,calign angle=63}
[DP$_{1}$, s sep=0pt [D$_{1}$ [\textit{\'i}]]
 [RelP, s sep=15pt  [DP$_{2}$, name=tgt , s sep=15pt  
 [NP [\textit{mut}\\`person', name=tgt2, roof]][{}, s sep=15pt [Op [\textit{n\'u}, roof]]
 [{} [D$_{2}$ [$\emptyset$]][\dots, name=t2]]]]
 [{}, s sep=15pt
 [Rel [$\emptyset$]] [TP [\dots, name=t]
 [T' [\textit{a b\'i \textsuperscript{↓}jέ b\'ijέk}\\`ate the food', roof]]]]]]
 \draw[dashed,->,>=stealth,overlay] (t2) ..controls +(south:1.75) and +(south:1.75).. (tgt2);
 \draw[dashed,->,>=stealth,overlay] (t) [in=-155,out=-125,looseness=3.25]  to (tgt);
\end{forest}

\pagebreak\noindent In the first step of this derivation, the noun phrase \textit{mut} `person' is fronted to a position before the \isi{demonstrative} \textit{n\'u} in its own DP$_{2}$ (described as the ``Op(erator)'' position in \citealt{Jenks-etall}).\footnote{\cite{Jenks-etall} follow \cite{Kayne1994} in labelling the relative clauses simply as CP. RelP is used instead in the diagrams below in order to disambiguate the head of the \isi{relative clause}, Rel, with the head of the clause headed by a \isi{complementizer}, Comp, as these are structurally distinct categories in the strand of research we explore in the present work. This is a technical remark with no consequences for the constituent structure of relative clauses or for the essence of \citeauthor{Jenks-etall}'s \citeyearpar{Jenks-etall} analysis.
} %end of fn
 In the second step, the entire DP$_{2}$ is fronted to the specifier of RelP. The augment marker \textit{\'i}-  spells out the top selecting head D$_{1}$ and comes out as the prefix on the head noun \textit{mut}.
\par 
In \citeauthor{Jenks-etall}'s \citeyearpar{Jenks-etall} account, the post-nominal ``operator'' position of the \isi{demonstrative} does not receive a focus reading when the DP$_{2}$ is in the specifier of the relative clause. In contrast, in the derivation of \isi{relative clause}s with a pre-nominal \textit{n\'u}, the \textit{n\'u} is a genuine demonstrative rather than the ``operator''. In this case, the entire relative DP$_{2}$ is raised out of RelP to a higher position where the demonstrative \textit{n\'u} receives a focus reading, as outlined in \ref{ozzy}.

\ex.\label{ozzy} 
Derivation of a \isi{relative clause} with a pre-nominal \isi{demonstrative} following \citet[35]{Jenks-etall}
\ag.
n\'u mut\textsubscript{i} [ \_\textsubscript{i} a b\'i \textsuperscript{↓}jέ b\'ijέk ]\\
1.that 1.person {} {} \textsc{1.sbj} \textsc{pst} eat 8.food\\
\strut `THAT person that ate the food'\medskip
\b.\label{nu2} 
\begin{forest}nice empty nodes, for tree={l sep=0.7em,l=0,calign angle=63}
 [DP$_{1}$, s sep=20pt  [DP$_{2}$, , s sep=10pt, name=Z
 [Dem [\textit{n\'u}, roof]][{}[D$_{2}$, s sep=12pt [$\emptyset$]][NP [\textit{mut}\\`person', roof]]]]
 [{}, s sep=20pt [D$_{1}$ [$\emptyset$]][RelP, s sep=15pt [\dots, name=tgt ][, s sep=10pt [Rel [$\emptyset$]][TP [\dots, name=t]
 [T' [\textit{a b\'i \textsuperscript{↓}jέ b\'ijέk}\\`ate the food', roof]]]]]]]]
  \draw[dashed,->,>=stealth,overlay] (t) ..controls +(south west:2) and +(south west:1.5).. (tgt);
   \draw[dashed,->,>=stealth,overlay] (tgt) ..controls +(south west:6) and +(west:3.5).. (Z);
\end{forest}

A particularly telling argument in support of such an analysis is that it accounts for the complementary distribution between demonstratives and what (appears to be) a separate relativizer in all types relative clauses involving a gap. The relative clauses involving a gap are subject and object relatives with pre- and post-nominal demonstratives. These are shown in the following:
 
\exg.
\textbf{\'i}-maaŋgέ\textsubscript{i} n\'u (*n\'u) [ mɛ \'n y\'i \_\textsubscript{i} ]\\
\textsc{aug}-1.child {1.\textsc{dem}} \phantom{X}\textsc{1.rel} {} \textsc{1sg} \textsc{pres} know\\
\strut `this/that child that I know'

\exg.
l\'i l\'i-w\'and\'a\textsubscript{i} (*l\'i\textsuperscript{↓}) [ \_\textsubscript{i} l\'i b\'i \textsuperscript{↓}jέ b\'ijέk ] \\
5.\textsc{dem} 5-friend \phantom{X}\textsc{5.rel} {} {} \textsc{5.sbj} \textsc{pst} eat food\\
\strut `THAT friend that ate the food' 

Such a complementary distribution of the medial \isi{demonstrative} pronoun and the relativizer in \isi{relative clause}s involving a gap shows that the relation between these two categories in \ili{Basa\'a} is robust and hence the problematic Dem=Rel \isi{syncretism} to the exclusion of Wh cannot be attributed to an accidental homophony. 
\par
If we follow \citeauthor{Jenks-etall}'s \citeyearpar{Jenks-etall} analysis of the formation of non-wh-relatives in \ili{Basa\'a}, we can directly resolve the \isi{*ABA} problem  present in \tabref{table3}. 
The juxtaposition of the syntax of non-wh-relatives in \ili{Basa\'a} with the syntax of non-wh-relatives in languages like \ili{English} reveals  that the second involves a genuine \isi{relativizer}, which does not form a constituent with the head noun, as outlined by the following example:

\ex. 
\a. the person that found our cat\medskip
\b.
\begin{forest}nice empty nodes, for tree={l sep=0.65em,l=0,calign angle=63}
[DP$_{1}$, s sep=15pt [D$_{1}$ [\textit{the}]]
 [RelP, s sep=20pt  [NP, name=tgt [\textit{person} ,  roof]]
 [{}, s sep=10pt
 [Rel [\textit{that}, roof]] [TP [\dots, name=t]
 [T' [\textit{found our cat}, roof]]]]]]
\draw[dashed,->,>=stealth] (t) [in=-150,out=-125,looseness=1.75]  to (tgt);
\end{forest}

\vskip -0.75cm
\noindent
This contrasts with the \ili{Basa\'a} \textit{n\'u}, which comes out as a genuine \isi{demonstrative} pronoun, which forms a constituent with the head noun. In turn, the relativizer, understood as the head of the relative clause, is null. This result requires the problematic \isi{paradigm} in \ili{Basa\'a} to be rewritten as in \tabref{table4}, which removes the \isi{*ABA} violation with the demonstrative and keeps the syncretic span Comp=Rel in the parallel paradigm with \textit{lέ}.

\begin{table}
\caption{Final version of the Basa\'a paradigm}
\label{table4}
\begin{tabular}[h]{ l l l l l l }
\lsptoprule
\textsc{dem}\textsubscript{def} & \textsc{comp} 	& \textsc{rel}  	& \textsc{wh} & \textsc{dem}\textsubscript{indef}\\	
\midrule
 & $\emptyset$\cellcolor[gray]{0.9}, & $\emptyset$\cellcolor[gray]{0.9}, & k\'i\'i  & n\'u\\
   & lέ\cellcolor[gray]{0.8} & lέ\cellcolor[gray]{0.8} & \\
\lspbottomrule
\end{tabular}
\end{table}

\noindent
The reanalysis of the \isi{paradigm} with a zero relativizer allows us to correctly predict that it will be able to cooccur with elements other than the demonstrative -- class 1 \textit{n\'u} or any other -- in the D head of the \isi{relative clause}. For instance, treating the \ili{English} \textit{that} as a \isi{relativizer}, the head of the relative clause does not need a \isi{demonstrative}, as in:

\ex. John saw \{ three men\slash somebody \} that Mary had fired.

Indeed, as already indicated in the example of a relative clause with a post-nominal demonstrative in \ref{optionalnu}, the null \isi{relativizer} can cooccur with the D head of the \isi{relative clause} that is lexicalized as the \textit{\'i}- prefix.  More generally, as already seen in \ref{Bas:big}, \textit{n\'u} can be generally dropped in both subject and object relative clauses. This optionality holds also with other nominal classes  as shown in the following example from \citet[18]{Jenks-etall}:

\exg. 
h\'inun\'i\textsubscript{i} (h\'i) [ liw\'and\'a l\'i b\'i \textsuperscript{↓}tέhɛ̌  \_\textsubscript{i} ]\\
\textsc{aug}.19.bird \textsc{dem} {} 5.friend 5.\textsc{sbj} \textsc{pst} see \\
\strut `the bird that the friend saw'



\section{Resumptive \isi{relative clause}s}

 A final comment about the \ili{Basa\'a} relative clauses involving resumption is in order. Resumptive relative clauses provide a circumstantial argument that supports both the idea that \isi{relativizer}s in \ili{Basa\'a} are genuine \isi{demonstrative}s as well as the conjecture made earlier on the basis of Slavic, Germanic, and \ili{Latvian} that it is specifically the medial demonstratives that serve as the base category in the  sequence in \ref{reiteratedfseq}.\footnote{The argument is circumstantial in the sense that it depends on a particular analysis of the formation of relative clauses that involve resumption (see for instance \citeauthor{Bianchi2004} \citeyear{Bianchi2004,Bianchi2011} or \citealt[chapters 2--3]{Salzmann2017}).
} %end of fn on resumption
\par
Namely, the complementarity between the \isi{demonstrative} pronoun and (what appears to be a distinct) relativizer is more limited with \isi{relative clause}s that involve resumption. In this environment, it is only the medial demonstrative that cannot co-occur with the relativizer, while the non-syncretic proximal and distal demonstratives can co-occur with the \isi{relativizer}, as shown in the example of object of comparison relative clauses in \ref{tyrionwilldie}.

\ex.\label{tyrionwilldie} Resumptive (object of comparison) \isi{relative clause} (\citealt[27]{Jenks-etall})
\ag.\'i-maaŋgέ\textsubscript{i} \{ n\'un\'u / *n\'u / {n\'u\'u \}} (n\'u) [ ŋgwɔ́ i ye ikέŋ\'i ilέl {ŋyέ\textsubscript{i} ]}\\
\textsc{aug}-1.child {} 1.\textsc{prox} {} \phantom{t}\textsc{1.med} {} {1.\textsc{dist}}
\phantom{l}\textsc{1.rel} {} 9.dog 9.\textsc{sbj} be 9.big exceed {1.\textsc{pron}}\\
\bg.  \{ n\'un\'u / *n\'u / {n\'u\'u \}} maaŋgέ\textsubscript{i} (n\'u) [ ŋgwɔ́ i ye ikέŋ\'i ilέl {ŋyέ\textsubscript{i} ]}\\
 {} 1.\textsc{prox} {} \phantom{t}\textsc{1.med} {} {1.\textsc{dist}} 1.child \phantom{t}\textsc{1.rel} {} 9.dog 9.\textsc{sbj} be 9.big exceed {1.\textsc{pron}}\\
\strut `this/that child that the dog is bigger than' 

This restriction is hard to account for if the \isi{relativizer} is not a genuine \isi{demonstrative} pronoun in the \ili{Basa\'a} relative clauses given that it must show class concord with the head noun, unlike the genuine relativizer \textit{lέ}, as shown in \ref{le-rel}.

\section{Summary}

The resolution of what comes out as an apparent ABA pattern  in the \ili{Basa\'a} \isi{paradigm} is possible if we inspect the syntax behind the offending Rel-cell, in a similar way the description of \textit{wo}-relatives in \ili{Swiss German} indicates that \textit{wo} is not on a par with relative pronouns like the German \textit{das} or the \ili{English} \textit{that}. Specifically, if we follow the analysis of non-wh-\isi{relative clause}s in \ili{Basa\'a} in \cite{Jenks-etall}, the offending relative pronoun turns out to be a genuine DP-internal \isi{demonstrative} that is placed after the head noun. We end up with a picture where overt realization of the cross-categorial paradigm is restricted in \ili{Basa\'a} to its two adjacent cells, in agreement with the \isi{*ABA} generalization and the proposal to insert indefinite \isi{demonstrative}s as the bottom category of the ``Dem\textsubscript{def}\,$>$\,Comp\,$>$\,Rel\,$>$\,Wh\,$>$\,Dem\textsubscript{indef}'' sequence.




