\documentclass[output=paper]{langscibook} 
\ChapterDOI{10.5281/zenodo.2654359}
\author{John E. Joseph \affiliation{University of Edinburgh}}
\title{The resistant embrace of formalism in the work of Émile Benveniste and Aurélien Sauvageot}
\shorttitlerunninghead{The resistant embrace of formalism in the work of Benveniste and Sauvageot}
\label{chap:joseph} 

\abstract{Rarely claimed by linguists as labels for their own work, ``structuralist'' and ``structuralism'' have been more often hurled at others as criticisms. Yet those doing the hurling were themselves often pursuing a similarly formalist analysis, and were not averse to claiming their share of the academic capital that structuralism brought to linguistics. Work by Émile Benveniste (1902--1976) and Aurélien Sauvageot (1897--1988) shows different modes of a ``resistant embrace'' to structuralist formalism, with their resistance centred on a perceived abandonment of attention to phonological and philological detail; and to the role of speakers, a concern that culminates with Benveniste's concept of enunciation. Their reactions are examined here within the framework of two different ways in which structuralism was conceived, one based on holism, the other on universalism.}

\begin{document}
\maketitle

\section{Introduction}
\label{sec:joseph:intro}

Quentin \citet{Skinner1969} expressed concern about the growing use in the history of ideas of the notion of paradigm, which had emerged in the history of art \citep{Gombrich1960} and of science \citep{Kuhn1962}. Skinner argued that it fosters a mythology that how people thought at any given period was more unified than has ever historically been the case. Insofar as we buy into this ``mythology of doctrines'', Skinner writes, quoting Voltaire, ``History then indeed becomes a pack of tricks we play on the dead'' \citep[7, 13--14]{Skinner1969}.\footnote{The quote is from Voltaire's letter to Pierre-Robert Le Cornier de Cideville, 9 Feb. 1757: ``l'histoire […] n'est, après tout, qu'un ramas de tracasseries qu'on fait aux morts''.} 

We are ninety years on from the first uses of the term ``structuralism'' by linguists,\footnote{\citet{Joseph2001} gives details on these early uses, and later ones, as well as on the ``structuralism'' proclaimed in psychology starting in 1907.} first to manifest, then to identify a paradigm, to which some subscribed wholeheartedly, while others resisted no less strongly, and others still, a majority perhaps, were ambivalent. This is not always easy to determine, making it all the more tempting to use the mythology of doctrines to unify the middle decades of the twentieth century into a \isi{structuralist period}.

Not only do linguists no longer apply ``\isi{structuralist}'' to our work, we even struggle to remember what exactly it represented, and why it had the power it did.\footnote{Having detected recently that younger colleagues in my department were using the term ``generative'' in a way that struck me as different from my own use of it, I went around to some of them and asked, ``Are you a \isi{generativist}?'' This included phoneticians and phonologists as well as people who work in pragmatics and syntax, none of them committed Chomskyans like the people with whom I worked at the University of Maryland in the 1980s. Each of my present colleagues whom I queried hesitated for a few seconds, then answered ``yes''. I then asked what ``\isi{generativist}'' means to them. None of them mentioned innateness, or universal grammar, or rules and representations, or principles and parameters, or infinite creativity or any other of the ideas which I associate with \isi{generativism}. Rather, all said that they are generativists because they believe in the existence of a language system which speakers know, and which is the basis of language production and comprehension. This, to me, does not a \isi{generativist} make. It is structuralism, part of the considerable \isi{structuralist} heritage that continued into \isi{generativism}. But to make it the criterion for being a \isi{generativist} is like being asked to define Episcopalianism, and answering that it means believing in God. I have argued for a long time now (since \citealt{Joseph1999}, and most fully in \citealt{Joseph2002whitney}) that ``\isi{American structuralism}'' actually begins not with {\Bloomfield} and {\Sapir} but with Chomsky, or else perhaps the day in 1942 when {\Jakobson} landed in New York. Yet one now meets linguists who think that the idea of a mental language system originated with Chomsky. When it comes to meaning in language, my colleagues often prove to be pre-Saussurean, conceiving of a language as an encoding of a pre-existing external reality — what {\Saussure} rejected as ``nomenclaturism''. They are divided over whether the sound side is essentially mental or acoustic. In these respects they are not yet structuralists, let alone generativists. But they value the label.} Actually, ``\isi{structuralist}'' and ``structuralism'' were more often hurled at others as criticisms than claimed as methodological labels. Yet those doing the hurling were themselves often pursuing a recognizably \isi{structuralist} form of analysis, and were not averse to taking their share of the academic capital that structuralism brought to linguistics. This paper examines some of the modes of resistance to the formalist commitments of \isi{structural linguistics} in mid-twentieth-century France — before the onset of a ``post-\isi{structuralist}'' period — and explores what drove it.

\section{Benveniste and structuralism}
\label{sec:joseph:bevenistestructuralism}

Émile {\Benveniste} (1902--1976) was the most important French linguist of the ``\isi{structuralist period}''. When asked about structuralism in interviews, he did not keep his distance, but answered the question as though he embraced the term and what it stood for (\citealt{Benveniste1968langage}, \citealt{Benveniste1968structuralisme}). His linguistic work became progressively less structural in some senses of that term, starting from when he resumed it after his Swiss exile during the war. Starting in the 1950s and culminating in 1969, he introduced his concept of ``enunciation'', which is the direct opposite to the \isi{structuralist} approach in key respects. As discussed in section \sectref{sec:joseph:laterwork} below, his stated aim was not to replace structuralism, but to supplement it — to provide a parallel mode of enquiry in which the focus is not on the structure of the \emph{langue}, nor on \emph{parole}, but on speakers; on the ``semantic'' rather than the ``semiotic'', in his terms, which seems to make him pre-Saussurean, as does his placing of writing at the centre of language.

He continued to publish prolifically on ancient Indo-European languages, as he had done since the 1920s, and also undertook fieldwork on American Indian languages in Alaska and the Yukon, though he published little of this research. He also produced a small but steady number of papers offering radical revisions to key concepts of linguistic analysis such as person, deixis and performatives, the sort of thing that his teacher Antoine {\Meillet} (1866--1936) had often done. A number of these papers were republished in \citeyear{Benveniste1966} in a volume entitled \emph{Problems in General Linguistics}, of which a second volume appeared in \citeyear{Benveniste1974}. The 1966 volume (henceforth referred to as \textsc{plg 1}, and the 1974 follow-up as \textsc{plg 2}) appeared just at the time when ``structuralism'' as a generalized mode of enquiry was getting established as dominant across the fields that comprise what in France are called the ``human sciences'', and indeed beyond.

The focus of \citet{Benveniste1969vocabulaire} on words, rather than sounds and forms (although they come into the picture in a secondary role), gives it a precarious place within linguistics \emph{tout court}, let alone \isi{structural linguistics}, which treated words as a pre-scientific concept, necessary to refer to when communicating with the general public and specialists in other academic fields, but kept at bay in their formal analyses. This despite two of the core figures of \isi{structural linguistics}, Roman {\Jakobson} (1896--1982) and André {\Martinet} (1908--1999), giving the title \emph{Word} to the journal which they co-founded in New York in 1945. The \citeyear{Benveniste1969vocabulaire} book is his attempt at the sort of structuralism that had spread beyond linguistics. He read the work of his contemporaries such as Georges Dumézil (1898--1986) and Georges Canguilhem (1904--1995) with admiration mixed with an awareness that the philological knowledge they brought to bear in their enquiries was shallow in comparison with his own. Claude Lévi-Strauss (1908--2009) could not see anywhere near so deeply into the cultures he studied as {\Benveniste} could into the remote Indo-European past.

Yet {\Benveniste}, in spite of all his work aimed directly or indirectly at subverting structuralism, never rejected it. In contemplating why, I have been inclined to attribute it to practical concerns: his awareness that structuralism, in promoting linguistics to master science, had brought considerable advantages to the French linguistics establishment and to him as its leader. That may sound like a cynical motive, except that {\Benveniste} was not a {\Jakobson} or a {\Martinet}, men with flamboyant personalities who strove to attract followers and worried about their place in the academic pecking order. {\Benveniste}'s place at the top was assured institutionally, from {\Meillet}'s death until his own forty years later, even during the last seven years when he was paralysed by a stroke and could not speak or write. The advantages which the wide attention to structuralism brought were ones that he personally did not need, but they offered benefits to his students and the other French linguists of whom he was the acknowledged leader.

There was still more to his ongoing semi-commitment to structuralism than the pragmatic benefits for others. Even his late work contains signs that he was drawn to what structuralism promised, in an almost religious way — like an agnostic who never misses church, drawn to the vision and promise he aches to believe in.\footnote{No links to traditional religious thinking are apparent in {\Benveniste}'s work, but see \citet[245--247]{Dosse1997} on Christian interpretations of the semiology of Jacques Lacan (1901--1981), to whom {\Benveniste} was sufficiently close to have contributed an article to the first issue of his journal \emph{La Psychanalyse} \citep{Benveniste1956remarques}, and on displaced Christianity in the work of Louis Althusser (1918--1990), Lacan's ally in the École Normale Supérieure (pp. 294--295).}

It is striking how in a 1968 address \citep[95]{Benveniste1970structure}, and again in a lecture the following January \citep[79]{Benveniste2012}, {\Benveniste} insists that ``the language contains the society''\label{q:joseph:langcontainssoc}.\footnote{``[…] la langue contient la société.'' He adds that ``la langue inclut la société, mais elle n’est pas incluse par elle'' (the language includes the society, but is not included by it) \citep[96]{Benveniste1970structure}. Translations are mine unless otherwise indicated.} \citet{Meillet190506} had been the first to state in print, more than sixty years earlier, that ``a language is a social fact''. But {\Benveniste} is asserting much more than that. To understand why, we can look for example at the brief chapter headed \emph{thémis} in his \emph{Vocabulaire des institutions indo-européennes}:

\begin{quotation}
The general structure of society, defined in its broad divisions by a certain number of concepts, rests on an assemblage of norms which add up to ``law''. All societies, even the most primitive […] are governed by principles of law relating both to persons and to goods. The rules and these norms are traceable in the vocabulary.
\end{quotation}

\begin{quotation}
[…] We can in the first place posit for common Indo-European an extremely important concept, that of ``order''. It is represented by Vedic \emph{r̥ta}, Iranian \emph{arta} ({\Armenian} \emph{aša}, by a special phonetic development). We have here one of the cardinal notions of the legal world of the Indo-European to say nothing of their religious and moral ideas: this is the concept of ``Order'' which governs also the orderliness of the universe, the movement of the stars, the regularity of the seasons and the years; and further the relations of gods and men, and finally the relations of men to one another. Nothing which concerns man or the world falls outside the realm of ``Order''. It is thus the foundation, both religious and moral, of every society. Without this principle everything would revert to chaos. \citep[379--380]{Benveniste20161973}\footnote{``La structure générale de la société, définie dans ses grandes divisions par un certain nombre de concepts, repose sur un ensemble de normes qui constituent un droit. Toutes les sociétés, mêmes les plus primitives, […] sont régies par des principes de droit quant aux personnes et aux biens. Ces règles et ces normes se marquent dans le vocabulaire. […] On peut poser, dès l'état indo-européen, un concept extrêmement important : celui de l'`ordre'. Il est représenté par le védique \emph{r̥ta}, iranien \emph{arta} (avestique \emph{aša}, par une évolution phonétique particulière). C'est là une des notions cardinales de l'univers juridique et aussi religieux et moral des Indo-Européens : c'est l'`Ordre' qui règle aussi bien l'ordonnance de l'univers, le mouvement des astres, la périodicité des saisons et des années que les rapports des hommes et des dieux, enfin des hommes entre eux. Rien de ce qui touche à l'homme, au monde, n'échappe à l'empire de l'`Ordre'. C'est donc le fondement tant religieux que moral de toute société; sans ce principe, tout retournerait au chaos'' \citep[vol. 2, 99--100]{Benveniste1969vocabulaire}.}
\end{quotation}

{\Benveniste} is attuned to the differences among Indo-European societies. Words that are not shared across the languages are interpreted as representing later historical developments. Through close study of texts in which the words occur, {\Benveniste} works to establish their precise meanings, and in so doing to reconstruct the societies themselves. In the case of words shared across all or the great bulk of the family, he is reconstructing the earliest recoverable Indo-European social stratum. The language contains the society.

His remarks about the nature of law are grounded in the findings of this linguistic method, but also apply to the method itself, particularly to the guiding principle of {\Benveniste}'s training, which he embraces even as he resists it. Structuralism is the search for the system that is ``an assemblage of norms which add up to `law'\thinspace''. No wonder its draw was so strong: it ``is the concept of `Order' which governs also the orderliness of the universe, the movement of the stars, […] the relations of men to one another''. It is as true of structure in language as of order in law and society that ``Without this principle everything would revert to chaos''.

Who knows whether {\Benveniste} saw, let alone intended, the reflexivity of his comments? But a few paragraphs on, after going through various Indo-Iranian, Greek and Latin reflexes of this root \emph{ar}--, including Latin \emph{ars} ``art'' and \emph{ritus} ``rite'', Latin \emph{artus} ``joint'' and Greek \emph{árthon} ``joint, limb'', he remarks: ``Everywhere the same notion is still perceptible: order, arrangement, the close mutual adaptation of the parts of a whole to one another […]'' \citep[101]{Benveniste1969vocabulaire}.\footnote{``Partout, la même notion est encore sensible : l'ordonnance, l'ordre, l'adaptation étroite entre les parties d'un tout […]''. \emph{ar}-- is also the root of French \emph{ordre} and English \emph{order}.} If you seek a definition of the \isi{structuralist} view of the language system, look no further.\footnote{The desire to find order in language, with the promise it held out of keeping everything from reverting to chaos, was by no means exclusive to linguists. It was extremely widespread, lying behind movements for language standardization, and for what linguists disparage as prescriptivism. In my view, the descriptivist-prescriptivist dichotomy is ultimately rhetorical, a veneer which masks a shared desire for order — law and order, given how fond linguists have always been of discovering laws comparable to those by which the movement of the stars is explained. One might expect linguists to regard prescriptivism as a phenomenon of language understanding and use, as worthy of study and analysis as their supposedly prescription-free data, but such an outlook is rare. We claim the unique right to define what order is and how it is to be sought, and see it as our duty to stamp out other conceptions of order in language, exposing their ignorance and error and treating them as an even greater threat to order because they decline to acknowledge our unique authority.}

So why did {\Benveniste} not seek unambivalently to be the Galileo of language, reducing the vast chaos of diversity to Order? The clue is in the word ‘reducing’. Reduction is the genius of structuralism. Its ancient and deep-seated appeal in our languages and cultures is evident in {\Benveniste}'s analysis of \emph{thémis}. The one small minority to which it might not appeal are those who actually love the vast diversity of languages, who enjoy nothing more than reading ancient texts in barely-known languages and working through their minute details. In other words, linguists, particularly the sort who entered the field in the nineteenth and the first two-thirds of the twentieth centuries, and who still exist, in reduced numbers.

I am suggesting that the founding tension in structuralism was that it was driven by a reductionist search for order, carried out by people who varied considerably in how fast and how far they thought such reduction could legitimately be taken. Indeed, some of them believed that legitimate knowledge required an accumulation of ever greater data and detail, in direct contrast to the genius of structuralism, and they would form the hard resistance. But my interest here is in the soft resistance of those who embraced the programme even while holding it at bay. Who smoked, but didn’t inhale.

\section{The issue of discontinuity}
\label{sec:joseph:discontinuity}

I shall start from the end of the \isi{structuralist period}, and the critique of \isi{structuralist} linguistics mounted from the 1970s onward by Henri {\Meschonnic} (1932--2000), who belonged to the generation of {\Benveniste}'s students.\footnote{His \emph{magnum opus}, \citet{Meschonnic1982}, sits within a massive output that branches across the disciplinary boundaries which he rejected.}  This is well into the period which, outside France, was being labelled as ``post-\isi{structuralist}'', characterized by resistance to key aspects of \isi{structuralist} work being mounted by academics from fields other than linguistics. Such was the din from without that {\Meschonnic}'s resistance from within did not have the full impact one might have expected. Like the integrationist critique of linguistics mounted by Roy Harris (1931--2015) at Oxford, it produced a clique of devoted followers, along with rejection and enmity from the linguistics mainstream — which however had by now cut itself off from any commitments to a structuralism henceforth associated with the past. This masked its enduring heritage, and made it inevitable that by the late 2010s linguists would no longer distinguish structuralism from \isi{generativism}.

\largerpage[-1]For {\Meschonnic}, the tragedy of structuralism lay in what he called its ``triumphalism of scientizing the discontinuous'' (\emph{triomphalisme d’un scientisme du discontinu}) \citep[20]{Meschonnic2009}. It is true that, if you ask a linguist what linguistics is, the answer you are likely to get is, first, something about the scientific study of language, and then a litany of the sub-specializations, phonology, syntax, semantics, historical linguistics, sociolinguistics etc. and what they are concerned with.\footnote{For an example, see \url{https://www.youtube.com/watch?v=7HOsQDD1Res}.} The existence of those sub-fields, with their division of the labour of analysing sound, order, meaning and the rest each allotted to specialists, is taken to signify the field's progress to a mature state. Specialists tend to avoid treading on each other's turf; yet no one would deny that the ultimate goal is an understanding of language \emph{as a whole}. The institutionalizing of discontinuity is seen as a means to that end, yet, as {\Meschonnic} shows, it has tended to become an end in itself.

Taking inspiration from {\Benveniste}'s (\citeyear{Benveniste1970appareil}) conception of utterance, the perspective which starts from the speaker or writer rather than from the linguistic system, {\Meschonnic} devoted his mature career to exposing the \isi{structuralist} fetishization of the discontinuous and shifting the focus to the \emph{continu} (``continuous'') in language. It is not always clear who did and did not count as a \isi{structuralist} for him, though Ferdinand de {\Saussure} (1857--1913) definitely did not. In fact, when {\Meschonnic} reads out his charge sheet against structuralism, each of the nine crimes is described in its opposition to {\Saussure}. Here are the first three:

\begin{enumerate}
    \item when {\Saussure} says ``system'', a dynamic notion, structuralism says ``structure'', a formal and ahistorical notion;
    \item when {\Saussure} proposes that with language all we have are points of view — a crucial notion: representations — structuralism with the sign presents itself as describing the nature of language;
    \item and {\Saussure} constructs the notion of point of view according to an entirely deductive (rational-logical) internal systematicity, but structuralism created descriptive (empirical) sciences of language […]. \linebreak\citep[20]{Meschonnic2009}\footnote{``1. quand {\Saussure} dit \emph{système}, notion dynamique, le structuralisme dit \emph{structure}, notion formelle et ahistorique ; 2. quand {\Saussure} pose que sur le langage on n'a que des points de vue, notion capitale : des représentations, le structuralisme avec le signe se présente comme décrivant la nature du langage ; 3. et {\Saussure} construit la notion de point de vue selon une systématicité interne toute déductive, mais le structuralisme a fait des sciences du langage descriptives […]''. {\Saussure}'s theory of language was famously laid out in the posthumous \emph{Cours de linguistique générale} \citep{Saussure19221916}.}
\end{enumerate}

Again, he is right about the discontinuous — though it is complicated. Starting in the 1920s, those who got called structuralists, or more rarely, called themselves that, were torn between two urges. One was to reject the methods of an earlier generation who wanted to decompose phenomena into elements. That can be construed as a desire for continuousness. The other was to seek out what connects phenomena to each other, and doing that demanded the decomposition into discontinuous elements that their first urge was to reject.

Early structuralists invoked \isi{Gestalt psychology} as continuous with what they were trying to achieve. By 1945, the first volume of \emph{Word} contains two articles laying out visions of structuralism that superficially overlap, but in fact embody these complexly opposed urges. For the older writer, Ernst {\Cassirer} (1874--1945), systems such as language need to be approached holistically. Language for \citet[110]{Cassirer1945} is organic, ``in the sense that it does not consist of detached, isolated, segregated facts. It forms a coherent whole in which all parts are interdependent upon each other''.

The younger Lévi-Strauss seems at first to be singing from the same hymn sheet when he rejects the analysis of kinship by W. H. R. Rivers (1864--1922) on the grounds that it is concerned merely with an atomistic charting of the details of relationships in some particular society: ``Each terminological detail, each special marriage rule, is attached to a different custom, like a consequence or a vestige: we descend into an orgy of discontinuity'' \citep[37]{LeviStrauss1945}.\footnote{``Chaque détail de terminologie, chaque règle spéciale du mariage, est rattachée à une coutume différente, comme une conséquence ou comme un vestige : on tombe dans une débauche de discontinuité.''}

But a careful reading shows Lévi-Strauss singing in a different key from {\Cassirer}. His concern is not with organicity, but with the failure to take a \emph{universalist} point of view, one that looks past superficial differences to find how kinship systems are fundamentally the same from culture to culture. They must be so, Lévi-Strauss assumes, because the human relationships they encode are the same. Taking maternal uncles as an example, he writes:

\begin{quotation}
We see that the avunculate, to be understood, must be treated as a relationship interior to a system, and that it is the system itself which must be considered in its totality, in order to perceive its structure. \citep[47]{LeviStrauss1945}\footnote{``Nous voyons donc que l'avunculat, pour être compris, doit être traité comme une relation intérieure à un système, et que c'est le système lui-même qui doit être considéré dans son ensemble, pour en apercevoir la structure.''}
\end{quotation}

If this sounds like {\Cassirer}'s holism, the resemblance is deceptive because of what Lévi-Strauss understands by ``system''. It is not like {\Saussure}'s language system, which is specific to each particular language. Lévi-Strauss is not talking about the Yoruba kinship system, as distinct from the Inuit one. He means the \emph{human} kinship system, regarded as a product of evolutionary forces.

There is a double continuity-discontinuity tension at work in this defining \isi{structuralist} moment: on the one hand, holism versus atomism, {\Cassirer}'s tension; on the other hand, the universal versus the language-culture-particular, Lévi-Strauss's tension. {\Meschonnic}'s discontent with the discontinuous falls with\-in the first type. When he takes up arms against the \isi{structuralist} dissociation of language from the body, he cites language-culture-particular examples (notably from the Hebrew of the Old Testament, but also from modern languages) as evidence for a universal language-body continuity. Any language-culture-particular versus universal tension is left aside, or at least pushed into the background.\footnote{The tensions we repress can come back to haunt us, and this, I argue in \citet{Joseph2018languagebody}, is potentially the case with {\Meschonnic}'s approach to the Hebrew-language body.} 

Lévi-Strauss sees the development of phonology, as opposed to the merely physical, empirical study of phonetics, as ``playing for the social sciences the same renewing role as nuclear physics, for example, has played for the exact sciences'' \citep[35]{LeviStrauss1945}.\footnote{``[…] vis-à-vis des sciences sociales, le même rôle rénovateur que la physique nucléaire, par exemple, a joué pour l’ensemble des sciences exactes.''} He locates the renewal in four fundamental points of method identified in 1933 by Nicolai S. {\Trubetzkoy} (1890--1938):

\begin{quotation}
[F]irst, phonology passes from the study of \emph{conscious} linguistic phenomena to the study of their \emph{unconscious} linguistic infrastructure; it refuses to treat terms as independent entities, instead taking as the basis of its analysis the \emph{relations} between terms; it introduces the notion of \emph{system} [...]; and finally it aims at discovering \emph{general laws} either by induction or by logical deduction, [...] which gives them an absolute character. \citep[35]{LeviStrauss1945}\footnote{``[E]n premier lieu, la phonologie passe de l’étude des phénomènes linguistiques \emph{conscients} à celle de leur infrastructure \emph{inconsciente} ; elle refuse de traiter les \emph{termes} comme des entités indépendantes, prenant au contraire comme base de son analyse les \emph{relations} entres les termes ; elle introduit la notion de \emph{système} [...] ; enfin elle vise à la découverte de \emph{lois générales} soit trouvées par induction, soit déduites logiquement, [...] ce qui leur donne un caractère absolu.''}
\end{quotation}

The second and third points, concerning relations and system, are ones {\Cassirer} would have endorsed. But the unconscious is not a concept {\Cassirer} deals with. Although he recognises that consciousness ``grows'' in the child, the dyad suggested by Lévi-Strauss would have been too simplistic for his liking.

When {\Cassirer} talks about atomism and holism, it is on the level of a particular system. Lévi-Strauss talks about atomism and totality of the general system, for example the avunculate, considered universally. The two approaches are not always at cross purposes, only sometimes, but enough to generate an enduring tension within structuralism. A defining moment came in 1955, when the \emph{succès fou} of Lévi-Strauss’s \emph{Tristes Tropiques} defined structuralism for the public at large and for the next generation of scholars. {\Cassirer}’s worries about internal discontinuity were shunted aside, not to vanish but to fester.

For {\Meschonnic} and his contemporaries who took up academic posts with the expansion of the Parisian university system after May 1968, structuralism stood for discontinuity. In linguistics, that meant treating phonology, prosody, morphology, syntax, semantics, semiology and so on as separate levels of language and distinct areas of specialization; and divorcing linguistics from poetics or applied areas such as translation, and ultimately even semiology, though linguistics continued to focus on the sign and combinations of signs as the essence of language. This never sat well with {\Meschonnic}, but since he was teaching linguistics, he could only go so far in opposing it in these early years of his career, though later he would attack it relentlessly.

The greatest damage wrought by this discontinuity, as {\Meschonnic} saw it, was that it resulted in the unifying core of language — rhythm — being relegated to a minor corner, when it should be at the very centre of an investigative enterprise where everything connects to everything else. {\Meschonnic}'s rejection of \isi{structuralist} linguistics can be read as an assertion that structuralism itself was not \isi{structuralist} enough.

\section{Benveniste’s early work: in what sense is it structuralist?}
\label{sec:joseph:earlywork}

Can {\Benveniste}'s earlier work really be called \isi{structuralist}? It is, after all, diachronic rather than synchronic in orientation, where structuralism is usually characterized as having replaced diachronic with synchronic enquiry. But that characterization is flawed — it is based on a misunderstanding of ``diachronic'' as a synonym of ``historical'', when {\Saussure}'s intention in calling for a diachronic linguistics was for it to replace the historical tracing of sound and forms through time with, instead, the comparison of \emph{états de langue} at different points in time, each analysed synchronically. {\Saussure}'s \citeyear{Saussure1879} \emph{Mémoire} on the primitive vowel system of the Indo-European languages is really a synchronic study, a reconstruction of the system at some indeterminate point in the past. {\Benveniste}'s doctoral thesis and first published book (\citeyear{Benveniste1935}) follows the model of {\Saussure}'s \emph{Mémoire} to the extent possible, given that it is a morphological rather than a phonological system that he is reconstructing. 

The first three-quarters of {\Benveniste}'s book consists of focussed surveys of forms and alternations that appear to have been written as separate studies. Not until Chapter 9 does {\Benveniste} explain how they fit together.

\begin{quotation}
All the lines of facts we have traced have led us progressively and by ultimately converging paths to recognize in neuters and adjectives a coherent structure and rule-governed alternations. In turn, these nominal forms posited in their most ancient state reveal principles which, once defined, confronted and grouped, constitute a theory of the Indo-European root. \citep[147]{Benveniste1935}\footnote{``Toutes les lignes de faits que nous avons suivies nous ont acheminé [sic] progressivement et par des voies finalement convergentes à reconnaître aux neutres et aux adjectifs une structure cohérente et des alternances réglées. A leur tour, ces formes nominales posées en leur état le plus ancien révèlent des principes qui, une fois définis, confrontés et groupés, constituent une théorie de la racine indo-européenne.''}
\end{quotation}

Before explaining what that theory is, however, {\Benveniste} sets out his \isi{structuralist} stall, with a sweeping attack on everything written on the subject heretofore:

\begin{quotation}
What has been taught up to now about the nature and modalities of the root is, in truth, a heteroclite assemblage of empirical notions, provisional recipes, archaic and recent forms, all with an irregularity and complexity which defy ordering. \citep[147]{Benveniste1935}\footnote{``Ce qu'on a enseigné jusqu'ici de la nature et des modalités de la racine est, au vrai, un assemblage hétéroclite de notions empiriques, de recettes provisoires, de formes archaïques et récentes, le tout d’une irrégularité et d’une complication qui défient l’ordonnance.''}
\end{quotation}

He illustrates this with a catalogue of reconstructed roots varying from one to five phonemes in length, monosyllabic or disyllabic, with either a vowel or a diphthong as their nucleus, 

\begin{quotation}
with an initial vowel (*\emph{ar}--) or a final vowel (*\emph{po}--); in long degree (*\emph{sēd}--) or zero degree (*\emph{dhək}--); with a long diphthong (*\emph{srēig}--) or a short diphthong (*\emph{bheudh}--), with a suffix or a lengthening, etc. It would be difficult to justify or even to enumerate completely all the types of roots that are attributed to Indo-European. \citep[147]{Benveniste1935}\footnote{``[…] à voyelle initiale (*\emph{ar}--) ou à voyelle finale (*\emph{po}--) ; à degré long (*\emph{sēd}--) ou à degré zéro (*\emph{dhək}--) ; à diphthongue longue (*\emph{srēig}--) ou brève (*\emph{bheudh}--), à suffixe ou à élargissement, etc. On serait en peine de justifier et même d’énumerer complètement tous les types de racines qui sont attribués à l’indo-européen.'' It is interesting that *\emph{ar}--, the root of \emph{order} and its congeners in other Indo-European languages as discussed in section \sectref{sec:joseph:bevenistestructuralism} above, should figure among the examples here.}
\end{quotation}

This is akin to what Lévi-Strauss a decade later will disparage as ``an orgy of discontinuity'' in Rivers' ethnography, which he wants to replace with a structural analysis (see section \sectref{sec:joseph:discontinuity}). One might expect {\Benveniste} to argue that no language could be this complicated, but he does not. His critique extends only to the analysis:

\begin{quotation}
There is here an abuse of words that betrays an indecisive doctrine. The way to arrive at Indo-European is not by piling up the various Indo-Eu\-ro\-pe\-an forms with a verbal theme, nor by projecting into prehistory the particularities of an attested language state. It is necessary to try, through broad comparisons, to find the initial system in its simplest form, then to see what principles modify its economy. It is this mechanism that we are attempting to define here. \citep[147--148]{Benveniste1935}\footnote{``Il y a ici un abus de mots qui trahit une doctrine indécise. On n'obtient pas de l'indo-européen en additionnant les diverses formes indo-européennes d'un thème verbal ni en projetant dans la préhistoire les particularités d'un état de langue historique. Il faut essayer, par de larges comparaisons, de retrouver le système initial sous sa forme la plus simple, puis de voir quels principes en modifient l'économie. C'est ce mécanisme que nous cherchons à définir ici.''}
\end{quotation}

A number of words are striking: {\Saussure}'s \emph{système}, and also \emph{mécanisme}, which occurs repeatedly in the \emph{Cours}; but also \emph{économie}, in a sense more reminiscent of {\Martinet}, {\Benveniste}'s younger contemporary and, in {\Martinet}'s mind at least, his rival. As {\Benveniste} pursues this theme, the word \emph{structure}, absent from the \emph{Cours}, comes to dominate:

\begin{quotation}
The essential thing being the problem of structure, we shall neglect on principle questions of ``value'', ``aspect'' etc. If the definition of the root we arrive at is judged to be valid, these notions of value and aspect will have the morphological basis which they now lack. It will then be the right time to re-examine them. \citep[147--148]{Benveniste1935}\footnote{``L'essentiel étant le problème de la structure, nous négligerons en principe les questions de `valeur', d'`aspect', etc. Si la définition de la racine à laquelle nous aboutirons est jugée valable, ces notions de valeur et d’aspect auront le fondement morphologique qui leur fait encore défaut. Il sera temps alors d’en reprendre l'étude.''}
\end{quotation}

``Value'' does not refer here to Saussurean \emph{valeur}, but to a more particular use of the term by {\Meillet} when writing in his proto-sociolinguistic vein, where he talks about the ``abstract'' and ``concrete'' value of words, linking the abstract to the aristocratic, and to the oldest, most enduring strain of the Indo-European lexicon, whereas the concrete belongs to the peasantry, is imbued with ``affective'' value and is historically unstable. As for ``aspect'', it figures in {\Meillet}'s work mainly in its familiar form, referring for instance to perfective versus imperfective in verbs, but more extensively. For example, ``verbs bearing preverbs offer a nuance of `aspect' different from that of the simple verb: they indicate a process, the end of which is envisaged'' \citep[263--264]{Meillet1931}.\footnote{``[…] les verbes munis de préverbes offrent une nuance d’`aspect' différente de celle du verbe simple : ils indiquent un procès dont le terme est envisagé.'' He gives the contrasting examples of \emph{Nec \textbf{tacui} demens} ``I was mad enough not to keep silent'' (\emph{Aeneid} II, 94) and \emph{\textbf{Conticuere} omnes} ``All fell silent'' (\emph{Aeneid} II, 1), where, since the highlighted verbs have the same root and are both in the perfect tense, the preverb \emph{con}-- is analysed as conveying the perfective aspect.} Although {\Benveniste} has dismissed, or rather postponed, investigation of aspect in the same breath with that of value, aspect in the more usual, limited sense is actually central to his theory of the Indo-European root. What he is doing here is distancing himself from {\Meillet}'s extension of the concept, where {\Meillet} tends to link it with the ``mentality'' of speakers — this despite the fact that {\Meillet} repeatedly places his analyses in opposition to that of ``Mr {\Vossler} and his school'' on the grounds that \emph{they} have recourse to mentality. {\Meillet}'s accounts are not as different from Karl {\Vossler}'s (1872--1949) as his rhetoric would make it appear. Regarding tense and aspect in the development from early Latin to Classical Latin, {\Meillet} writes of how

\begin{quotation}
with the development of a civilization of intellectual character, in which the thinking of the upper echelon takes an exact philosophical turn, and in which children and youths are educated in schools, the notion of ``tense'' takes precedence over the notion of ``aspect''. \citep[270--271]{Meillet1931}\footnote{``[…] avec le développement d'une civilisation de caractère intellectuel, où la pensée des hommes dirigeants prend un tour philosophique exact, où les enfants et les jeunes gens se forment dans les écoles, la notion de `temps' prenne le pas sur la notion d'`aspect'\thinspace''. In his review of \citet{Vossler1932}, \citet[234]{Firth1933} contrasts {\Vossler}'s conception of ``inner language form'' with {\Saussure}'s ```bloodless system of signs' (\emph{langue})''.}
\end{quotation}

Again, a matter of sociolinguistics: the thinking of the upper echelon, education in schools being invoked to explain the rise of rational, which is to say abstract, tense -- not entirely replacing concrete aspect, but taking precedence over it, quite as the upper echelon, the aristocracy, take precedence over the peasantry.

The development of \isi{structural linguistics} is then a story of difference, of differentiating oneself from someone else who is perceived as too psychological, insufficiently concerned with establishing the facts of language structure before offering explanations of them, rendering dubious the sustainability of those explanations. {\Meillet} sees {\Vossler}'s ``idealism'' as too, well, idealist, whereas his own approach is better grounded in ``concrete'' facts. In historical terms {\Meillet} believes that the move from the concrete to the abstract represents progress; and so too in methodological terms, in that his own method proceeds in this way, deducing higher-level explanations from detailed examination of phenomena. Whereas abstraction that is not so deduced, but simply asserted, risks being fantasy, and so is not progress at all; not a nobility destined to rule over the concrete-minded, but a tyranny that the true nobles must resist.

{\Meillet} is polite when rejecting {\Vossler}, keeping his remarks to a minimum. When it comes {\Benveniste}'s turn to make a similar move vis-à-vis {\Meillet}, he is more than polite towards his \emph{cher maître}, not even naming him, just two features of his analysis of Latin, value and aspect in its extended sense, which he says he wants not to eliminate, but to postpone, until the structure of Indo-European is better established. This is not even a criticism of {\Meillet}'s analysis of Latin, a language the structure of which \emph{is} well established — at least, not an overt criticism. But perhaps he is undertaking some distancing from what is said in the Preface to \citet{Meillet1931}, which, after underscoring the role of value and aspect as key features of the analysis, ends with a paragraph thanking {\Benveniste} for helping to revise the text and compiling its index.

One of the curious aspects of \citet{Benveniste1935} is the sizeable gap between the title, ``origins of the formation of nouns'', and the contents, which are not restricted to nouns, but culminate in a ``unitary and constant definition of the Indo-European root and its aspects'' (\emph{une définition unitaire et constante de la racine indo-européenne et de ses aspects}) \citep[170]{Benveniste1935}. This definition says that the ``Indo-European root is monosyllabic, triliterate, composed of the fundamental vowel \emph{ĕ} between two different consonants'', then gives four further specifications about how it may be constituted \citep[170--171]{Benveniste1935}.\footnote{``La racine indo-européenne est monosyllabique, trilitère, composée de la voyelle fondamentale ĕ entre deux consonnes différentes.''} {\Benveniste}'s theory of the Indo-European root was received by Indo-Europeanists somewhat as {\Saussure}'s \emph{Mémoire} had been received: with astonishment at its daring brilliance and respect for its command of linguistic data, mixed with a wait-and-see dubiety that is appropriate with any stunningly simple model, to which scholars are bound to respond with examples that do not appear to fit it.

\largerpage[1]In the longer term, {\Benveniste}'s approach to Indo-European reconstruction has not held up,\footnote{I am grateful to the eminent Indo-Europeanist and {\Benveniste} scholar Georges-Jean Pinault for confirming to me that this is the case.} and has even been rejected as ``brutally reductionist'' \citep[560]{Dunkel1981}. That does at least furnish him with strong credentials as a \isi{structuralist} — but one determined to supplement the formalist approach with serious consideration of what speakers do with language, redeeming his \citeyear{Benveniste1935} promissory note to re-examine notions of value and aspect if his morphological analysis proves valid. This is what he began to do after the war.

\largerpage
\section{The spirit of philology in Sauvageot}
\label{sec:joseph:sauvageot}

Another French linguist of the ``\isi{structuralist period}'', Aurélien {\Sauvageot} (1897--1988) was born in Constantinople, to a Belgian mother and a French father working as an architect for the Sultan. As a student at the British School of Pera the boy, a natural polyglot, learned English and German, and also picked up Greek and Turkish (see Jean-Robert Armogathe's Preface to \citealt[9]{Sauvageot2013}). In 1911 his family returned to Paris, where, preparing for the competitive examination for entry into the École Normale Supérieure, he came to {\Meillet}'s attention. Many years later {\Sauvageot} would recall his first summons to a private meeting at {\Meillet}'s home, in September 1914. The seventeen-year-old made a confession:

\begin{quote}
``Look, Professor, I should tell you straightaway that I have no visual memory''.


``What?''

``No, with me everything happens only with phonic memory, or acoustic if you prefer. I have only auditory images. So I'm really bad at linking what I hear with what’s written, and I can only work on a language insofar as I know how it’s pronounced''.

``Oh, how extraordinary,'' he said to me, ``because I, you see, never hear any auditory image''.

And I said to him: ``But, then, how do you think?''.

And he said to me: ``Well, by sequences of written signs''.

With that a lot of things made sense to me. It was one of the first discoveries I made about {\Meillet}. \citep[193]{Sauvageot1992}\footnote{``Je lui dis : `Ecoutez, Monsieur le Professeur, je vous dis tout de suite que je n'ai pas de mémoire visuelle.' `Comment?' `Non, chez moi tout se passe uniquement avec la mémoire phonique ou si vous voulez acoustique. Je n'ai que des images auditives. Alors, je suis très malheureux pour lier ce que j'entends à ce qui s'écrit et que je ne peux travailler sur une langue que dans la mesure où je sais comment elle se prononce.' `Oh, c'est extraordinaire, me dit-il, parce que moi, alors, voyez-vous, je n'entends jamais une image auditive.' Et je lui dis : `Mais, alors, comment pensez-vous?' Et il me dit : `Eh bien, par séquences de signes écrits.' Alors là, j'ai compris bien des choses. Cela a été une des premières découvertes que j'ai faites de \textsc{Meillet}.''}
\end{quote}

Both are rather extreme cases. {\Sauvageot}'s mind worked as one might expect a blind lad's to, {\Meillet}'s a deaf man's. There are deep differences in how people think, differences we tend to erase, or sort into normal and pathological cases. I shall come back round to {\Sauvageot}’s acute acoustic sensitivity and memory.

In April 1917 {\Sauvageot} received another summons, this time to {\Meillet}'s office. The only French linguist covering Finno-Ugric, Robert Gauthiot (1876--1916), had been killed in the war, and it was decided that {\Sauvageot} would have to replace him, even though he knew no Finno-Ugric language at the time. He asked why he had been chosen, and {\Meillet} replied, ``Why, that's simple, because you were born in Constantinople, you spoke Turkish and you still know a fair bit of Turkish, that's why, and because Turkish is a language whose mechanism is very similar to that of the Finno-Ugric languages'' \citep[194]{Sauvageot1992}.\footnote{```Mais c'est simple, parce que vous êtes né à Constantinople, que vous avez parlé le turc et que vous savez encore pas mal de turc, voilà, et que le turc est une langue dont le mécanisme est très semblable à celui des langues finno-ougriennes'.''}

{\Meillet} sensed the young {\Sauvageot}'s lack of enthusiasm at the prospect, but assured him that it would come. And come it did, very much so, from his arrival in Finland in the summer of 1919 (the date is from \citealt[296]{Perrot2007}), where he began studying Finnish with {\Meillet}'s friend Emil Nestor {\Setälä} (1864--1935). For the rest of {\Sauvageot}'s long life, {\Setälä} would remain one of his principal touchstones not just for Finnish but for the understanding of language generally, rivalled only by {\Setälä}'s Hungarian friend Zoltán Gombocz (1877--1935), along with, of course, {\Meillet}, and the linguist {\Meillet} revered above all others, his own teacher {\Saussure}. These four were not a foursome, but a pair of twosomes; and we can see throughout {\Sauvageot}'s career a tension between what ``structural'' linguistics came to represent, versus the sometimes diametrically opposed concerns of the Finnish and Hungarian philologists.

Jump ahead now thirty years. {\Meillet} had died in 1936. {\Benveniste}, who succeeded him in his chair in the Collège de France, that last surviving royal institution, was king of the nation's linguists. The dauphin, {\Martinet}, had exiled himself to New York after being hounded from the Sorbonne under suspicion of having been a collaborator \citep[see][]{Joseph2016}. Beneath {\Benveniste} were the barons, including {\Sauvageot}. He had occupied the first chair of Finno-Ugric languages in France, in the École Nationale des Langues Orientales Vivantes, Paris, since it was established in 1931, with an interruption from 1941--43 at the insistence of the Vichy government \citep[296]{Perrot2007}, probably because for years he had been a prominent and outspoken member of the Communist Party \citep[see][158]{Chevalier2006}.

{\Sauvageot} was also active in the Institut de Linguistique, which held monthly lectures by the linguistic aristocracy, many of them aimed at surveying the structures of non-Indo-European languages. In \citeyear{Sauvageot1946} {\Sauvageot} published his \emph{Esquisse de la langue finnoise} (Sketch of the Finnish language) in a series called ``L’Homme et Son Langage'' (Man and his language) put out by La Nouvelle Édition in Paris. Three years later, the same book was published by Klincksieck, as the first volume in a new series that {\Sauvageot} started called ``Les langues et leurs structures'' (Languages and their structures).\footnote{Not only were the title and text unchanged, but they bear the copyright and printing date 1946, so apparently the unsold copies were simply given a new cover with the fresher date of 1949. Curiously, the cover and copyright page both give the year 1949, and yet the legally required final page gives the date of printing 24 August 1946, and the legal deposition as the third trimester of 1946 (``Achevé d'imprimer […] le 24 août 1946. Dépôt légal: 3\textsuperscript{e} trimestre 1946''). The book must therefore have been completed by mid-1946.} The 1949 Avertissement (Preface) announces three other volumes as forthcoming in the series, on Modern Greek, Tamil and Berber, and explains that the aim is to create ``a series of descriptive studies bearing on idioms as diverse as possible, each envisaged in isolation, taken in itself'' \citep[7]{Sauvageot1946}.\footnote{``[…] une série d'études descriptives portant sur des idiomes aussi divers que possible, envisagés chacun isolément, pris en soi.''} Each book will ``extricate through an appropriate analysis the characteristics inherent to a given language, grasped at a given moment of its evolution, and reveal the mechanism of the system of functions of which it is constituted'' \citep[7]{Sauvageot1946}.\footnote{``Il s'agit de dégager par une analyse appropriée les caractères inhérents à une langue donnée, saisie à un moment donné de son évolution et d'exposer le mécanisme du système de fonctions dont elle est constitutée.''}

That sounds quite structural — but ``mechanism'' is a loose concept. Used as a metaphor in linguistic work since the nineteenth century \citep[see][]{Joseph2018hj}, it seems to have meant something rather specific to {\Sauvageot}, and perhaps idiosyncratic to him. {\Sauvageot}'s life's work had been determined because, as {\Meillet} said, ``Turkish is a language whose mechanism is very similar to that of the Finno-Ugric languages''. The Preface continues:
 
\begin{quotation}
Up to now it has often been affirmed that a language is an ensemble in which all the parts fit together and the categories that supposedly form the foundation of the structure of a language have been much evoked. Only there has been a negligence in adding to the debate the concrete testimony that must be brought in by the descriptive study of a given state of language. \citep[7]{Sauvageot1946}\footnote{``Jusqu'à présent il a été souvent affirmé qu'une langue est un ensemble où toutes les parties se tiennent et l'on a beaucoup évoqué les catégories qui formeraient le fondement de la structure d'une langue. Seulement on a négligé de verser au débat le témoignage concret que doit apporter l'étude descriptive d'un état de langue donné.''}
\end{quotation}

\largerpage 
What has been affirmed ``up to now'' is the Saussurean conception of the language system. The \emph{toutes les parties se tiennent} is a slight rewording of the famous motto \emph{tout se tient} (everything supports everything else), attributed to {\Saussure} though first used in print by \citet[407]{Meillet1903}. It is probably fair to say that \isi{structural linguistics} does indeed make ``categories'' the foundation of the structure of a language, if categories are put into a binary contrast with ``concrete testimony'', and if that testimony means actual sounds and utterances: phonetics as opposed to phonology, and \emph{parole} as opposed to \emph{langue}. By taking up {\Saussure}'s \emph{état de langue}, state of language, as the place where this concrete testimony is to be found, {\Sauvageot} hints that it is not against {\Saussure} that he is positioning himself, but against later structuralists who claim to be following {\Saussure}'s programme but are perhaps instead betraying it.

{\Sauvageot} specifies that by ``descriptive study'' he does not mean simply enumerating grammatical processes or inventorying the most used paradigms, but rather ``a prospecting effort to penetrate beyond simple grammatical analysis into the domain of expression of which grammar is so to speak only the more or less schematic skeleton'' \citep[7]{Sauvageot1946}.\footnote{``[…] un effort de prospection pour pénétrer par delà la simple analyse grammaticale dans le domaine de l'expression dont la grammaire n'est pour ainsi dire que le squelette plus ou moins schématique.''} This domain of expression includes ``syntax, semantics and vocabulary'', all of which are, he rightly notes, neglected in structural grammars. He wants to get to them through ``sufficiently detailed analytic descriptions of concrete examples of the behaviour of a certain number of linguistic structures'' \citep[7]{Sauvageot1946}.\footnote{``[…] des descriptions analytiques suffisamment poussées, des exemples concrets du comportement d'un certain nombre de structures linguistiques.''} Here nearly every word is charged with potentially polemical meaning: {\Sauvageot} is implicitly accusing \isi{structural linguistics} of being insufficiently detailed in its analysis of individual linguistic structures, of failing to use concrete examples, of taking a broad-brush approach rather than focussing on ``a certain number'' of structures in depth, and of neglecting the ``behaviour'' of the structures, in favour of simple inventories. What he means by behaviour is expanded upon at the start of the book proper: 

\largerpage 
\begin{quotation}
What makes the originality of an idiom is not the presence of this or that particular structural feature but how the structure as a whole is arranged, the use that is made of it and the performance that is obtained from it for the needs of the expression of thought. \citep[13]{Sauvageot1946}\footnote{``Ce qui fait l'originalité d'un idiome, ce n'est pas la présence de tel ou tel trait particulier de structure mais la façon dont l'ensemble de cette structure est agencée, l'usage qui en est fait et le rendement qui en est obtenu pour les besoins de l'expression de la pensée.''} 
\end{quotation}

The concern with ``arrangement'' is an embrace of the core principle of Saussurean linguistics, which {\Sauvageot} thought however had not been adequately buttressed with attention to particular features. The concern with ``performance'' may again be his embrace of {\Saussure}'s all-but-forgotten call for a linguistics of \emph{parole}. The concern with expression of thought reveals a tension within {\Sauvageot} himself, in that he will often insist that language must be understood as a tool of communication, the traditional alternative to representation or self-expression as the primal and formative purpose of language.

The strongest evidence that {\Sauvageot} is taking a polemical stance comes when he claims to be ``observation'' personified: ``In doing this, we have not the least intention of diffusing the theories of a school. We are focussed above all on describing the facts as they present themselves to observation, by disregarding any preconceptions'' \citep[7--8]{Sauvageot1946}.\footnote{``Ce faisant, nous n'avons aucunement l'intention de diffuser les théories d'une école. Nous nous attachons avant tout à décrire les faits tels qu'ils se présentent à l'observation, en faisant abstraction de toute idée préconçue.''} But the language he has used up to this point already belongs to a school, that of \isi{structural linguistics}, and its underlying theory is not immediately dissociable from that language, even when he is positioning himself against aspects of that theory. That positioning is itself a theory, and since this Preface is for a whole series of books by different authors, it looks as though {\Sauvageot} is trying to form a school and to diffuse its theories.

\largerpage
How his treatment of Finnish is distinct from a structural one can be seen from the opening sentence. 

\begin{quote}
To the ear, Finnish seems ``loud'', a bit hoarse and abrupt, the whole spouted rapidly in a rhythm with rather close beats, modulated according to a musical phrasing with rather sharp but descending notes that appear to follow an almost unvarying curve. Finnish discourse knows only a few melodic deviations between the peaks and troughs of modulation. The monotonous repetition of these modulations makes one think right away that the language modulates not in order to express, but only to mark out the elements of the flow.


The vowels ``mark''; they burst joyfully on speakers' lips, whereas the consonants are muffled sometimes to the point of being whispered. \citep[15]{Sauvageot1946}\footnote{``A l'oreille, le finnois fait `sonore', un peu rauque et saccadé, le tout débité rapidement sur un rythme aux alternances assez rapprochées, modulées selon une phrase musicale aux notes plutôt aiguës mais descendantes, qui semble suivre une courbe à peu près invariable. Le discours finnois ne connaît que peu d'écarts mélodiques entre les sommets et les creux de modulation. La répétition monotone de ces modulations fait tout de suite penser que la langue ne module pas pour exprimer, mais seulement, pour démarquer les éléments du débit. Les voyelles `marquent' ; elles éclatent joyeusement sur les lèvres des sujets parlants, tandis que les consonnes sont assourdies parfois jusqu'au chuintement.''}
\end{quote}

To his French colleagues this impressionism would have sounded like a throwback to the nineteenth century, maybe even to \citet{Rousseau1782175561}. In some respects it is — one way to resist the mainstream is to hark back to an earlier age and represent its discourse as a lost truth, as Chomsky would do with his ``Cartesian linguistics'' (\citeyear{Chomsky20091966}). {\Sauvageot} is rescuing a musicality in linguistic analysis that in fact had not been absent in twentieth-century French linguistics, but was always a minority concern and was marginalized with the rise of \isi{structural linguistics}, until its re-emergence with {\Meschonnic}.\footnote{\citet{Saussy2016} revives the history of this alternative tradition, in which {\Meillet} figures as, among other things, supervisor of the doctoral thesis of Milman Parry (1902--1935), whose studies of contemporary oral recitation in the Balkans would revolutionize the understanding of Homeric epic.}

But by the late 1940s, {\Sauvageot} found himself unable actually to get the prom\-ised books on Tamil and Berber out of their signed-up authors, let alone sign up any further authors.\footnote{The books announced were \emph{Esquisse de la langue tamoule} by Pierre Meile (1911--1963) and \emph{Esquisse de la langue berbère} by André Basset (1895--1956).} In his Klincksieck series there finally appeared only his own ``Sketches'' of Finnish and Hungarian (\citeyear{Sauvageot1951}). Given the eccentricity of {\Sauvageot}'s vision, one can imagine that other authors may have felt themselves caught between maintaining their standing among more conventional linguists and producing what he wanted from them, which perhaps could only be managed by someone with his rare ``auditory memory'', plus his double saturation in structuralism and the Finnish and Hungarian philological traditions of {\Setälä} and Gombocz.

\largerpage
The Canadian philosopher Charles \citet{Taylor2016} contrasts what he terms the H-L-C and the H-H-H, where the H-L-C is the Enlightenment outlook inherited by analytic philosphers from Hobbes, Locke and Condillac. The H-H-H is the Counter-Enlightenment ``Romantic'' outlook of Hamann, Herder and {\Humboldt} to which Taylor strives to draw philosophers' attention. These two perspectives are present in modern linguistics, mainly as a result of {\Saussure}'s inclusion in his lectures of both semiology, an Enlightenment inheritance from the \emph{grammaire générale} tradition, and the self-contained language system impervious to change by any individual, an inheritance from German linguists, most directly the Neogrammarians. Their conception of sound laws that followed an exceptionless path ``insofar as they are mechanical'', which is to say physical, represented a neo-Romantic Counter-Enlightenment tradition in which language is something extra-rational, that follows its own organic path, even though the Neogrammarians and {\Saussure} did not go the full naturalist route of Max Müller and others. They however also opened up a breach to allow in some Enlightenment through the role that they allotted to analogy as the only admissible explanation for apparent exceptions to the sound laws. Analogy is a mental rather than a physical process, and rational in nature.

On the theoretical level, {\Meillet}'s structuralism was universalist. He argued that small European languages such as Finnish and Hungarian were doomed by a sort of natural selection that would limit the number of ``languages of civilization'' in Europe (\citealt[279]{Meillet1918}; see also \citealt[209--210]{Sauvageot2013}). That was an Enlightenment position. But unlike {\Meillet}, {\Sauvageot} was H-H-H to the bone. His personal experiences in Finland and Hungary had proved to him that what really determined the present and future vitality of a language was its expressive power. In a book manuscript published posthumously in 1992, he wrote, ``If a language succumbs, it is because it failed in its expressive task'' (\emph{Si une langue succombe, c'est qu'elle faillit à sa tâche expressive}) \citep[160]{Sauvageot1992}. {\Sauvageot} was torn between loyalty to the man to whom he owed everything in career terms, and the Finnish and Hungarian philologists whose view of their language was so much more in accord with his own.


Hence his embracing and resisting of a \isi{structural linguistics} that, in France, saw {\Meillet} as its head. But outside France, it was developing in various directions, above all in Prague, where the terms ``structural'' and ``structuralism'' were being explicitly proclaimed by the start of the 1930s. In the English and French speaking worlds, 1940 is the year when ``\isi{structural linguistics}'' starts to appear regularly \citep[see][]{Joseph2015}. {\Martinet} was the one French linguist in regular contact with the {\PragueLinguisticCircle}, as well as with Copenhagen, where Louis {\Hjelmslev} (1899--1965) and Viggo Brøndal (1887--1942) were laying the ground for glossematics. Perhaps it was {\Meillet}'s death in 1936 that licensed French linguists to be more directly critical of, even hostile to structuralism, particularly as it was being developed in Prague. {\Meillet}'s successor, {\Benveniste}, was himself ambivalent towards it, as became apparent with the article on the arbitrariness of the sign that he contributed to the first issue of {\Hjelmslev}'s \emph{Acta Linguistica} in 1939.

What exactly about structuralism was repellent to {\Sauvageot}? We got some clues earlier in the Avertissement to the \emph{Esquisse de la langue finnoise}, when he wrote about ``a negligence in adding to the debate the concrete testimony that must be brought in by the descriptive study of a given state of language'', and the neglect of syntax, semantics and vocabulary in favour of inventories of phonemes and morphemes. More generally, he objected to the ``dogmatism'' of the structuralists, as his student and later colleague Jean Perrot would report that

\begin{quotation}
{\Sauvageot} was not indulgent towards his colleagues and his hostility to dogmatism led him to severe judgements about these dogmatic theoreticians whom he readily called ``these gentlemen'', and whom he readily maltreated with irony, for example denouncing a manifest error as ``a simple blunder''. In particular he was rather hard with regard to the ``phonologists'', for instance reproaching Lazicius for behaving as a ``disciple of {\Trubetzkoy} and the Prague phonologists, whose excesses and dogmatism he espoused''. \citep[16]{Perrot2009}\footnote{``{\Sauvageot} n'était pas indulgent à l'égard de ses confrères et son hostilité au dogmatisme lui inspirait des jugements sévères sur ces théoriciens dogmatiques qu'il appelait volontiers `ces messieurs', et qu'il malmenait volontiers ironiquement, dénonçant par exemple comme `une simple bévue' une erreur manifeste. Il était en particulier assez dur à l'égard des `phonologistes', reprochant par exemple à Lazicius son comportement de `disciple de Troubetzkoï et des phonologistes de Prague, dont il a épousé les outrances et le dogmatisme'.'' Perrot does not indicate sources for the citations which he indicates. On the Hungarian phonologist Gyula Lazicius (1896--1957), see \citet[288]{Voigt1986}.}
\end{quotation}

French linguists from 1925 to 1950 seem on the one hand to want to claim that they are, along with the Geneva School, the keepers of the Saussurean \isi{structuralist} flame, while on the other hand acting as if structuralism is a foreign perversion. This aspect of linguistic history has to be read in the context of how nationalist feeling was developing in the inter-war period, and again with the anti-imperial wars in Indochina and Algeria in the 1950s, when the Cold War is also central to the plot; as is the ambivalence of linguists to the massive increase in attention and status they gained starting with the success of Lévi-Strauss's \emph{Tristes tropiques} in 1955, and the continuing rise of structuralism as the master science informing the work of Lacan, Merleau-Ponty, Barthes, Greimas and soon the next generation of French intellectuals who became prominent in the 1960s. Among the linguists themselves new tensions arose when {\Martinet} returned in 1955 from his self-imposed exile, distancing himself from structuralism in favour of a ``functionalism'' that combined an ultra-\isi{structuralist} analytical method with genuine challenges to the how structuralists dealt with dialects, bilingualism and social differentiation, challenges that were being pushed further by {\Martinet}'s American student Uriel Weinreich (1926--1967) (\citealt[see, e.g.,][]{Weinreich1954} and \citealt{Joseph2016}).

But Weinreich also had strong allegiances to {\Jakobson}, who had been the first to use the word structuralism in print in the 1920s, and had gone on to redefine it, in conjunction with {\Trubetzkoy} and the {\PragueSchool}, in ways that directly contradicted some of {\Saussure}'s core principles. This is what made French linguists resistant to structuralism as redefined by Prague — partly on intellectual grounds, partly on nationalist ones, in which some degree of jealousy at the success of {\Jakobson} and the {\PragueSchool} in other parts of Europe and America cannot be ruled out. {\Sauvageot}, and to a lesser degree {\Benveniste}, were more forthright about their resistance to structuralism than others of their contemporaries in France, who may have feared being even more marginalized internationally if they overtly challenged Prague, {\Jakobson}, {\Trubetzkoy}, {\Hjelmslev} et al.

But it was not just the formalism, the ``structural'' part of \isi{structural linguistics}, that {\Sauvageot} resisted. It was also ``linguistics'', insofar as it had pushed philology out from the forefront of academic enquiry, to become yesterday's dusty, antiquarian pursuit. With it went attention to the fine details of a language, including how individual writers discovered and exploited its potential resources. \emph{Phil}ology — the \emph{love} of language. Linguistics — its cold, clinical study.

It was a loss that {\Sauvageot}, with his heritage from the philologists who taught him Finnish and Hungarian, sought to rectify — above all in his last major work \citep{Sauvageot1973}, by which time he no longer needed to be concerned with his position within the French linguistic establishment. Even so, he had the reputation of his students to think of; and that may explain why he never published the book manuscript he entitled \emph{La structure du langage} (The structure of language) that appeared in \citeyear{Sauvageot1992}, four years after his death. This book seems determined to subvert some of the basic principles of Saussurean structuralism, let alone its later variants, and offer in their place a vision of language grounded in his long experience of Finnish and Hungarian language and culture, including the ideas of {\Setälä} and Gombocz and the other great linguists and literary figures whose individual impact on the languages he had witnessed. Yet {\Sauvageot} can never escape the shadow of his first linguistics teacher, {\Meillet}, nor does he want to. In this book he aims at a reconciliation in which we find a rare trace of his apparently avowed Marxism. He distinguishes the ``invariants'' of a language, which are its ``structure'', from the ``variants'' which he calls its ``superstructure'' \citep[18]{Sauvageot1992}.

{\Sauvageot}'s encounter with Finno-Ugric philology would limit his embrace of structuralism — a doctrine whose historical nuances we, starting with me, have glossed over. We assume that embracing and resisting are either/or options. History is more complicated than that, which is what makes it interesting. The resisting embrace can have strategic force: someone who only embraces will not push the science forward; someone who only resists will struggle to get a hearing. The resisting embrace can give one an audience for resistant innovations that are heard as progress within the status quo, even if, in his heart, the innovator is committed to overturning it.

\section{Benveniste’s later work: enunciation}
\label{sec:joseph:laterwork}

\citet{Normand1986} traces the development of ``enunciation'' in {\Benveniste}'s work back to papers he published in 1946 and 1949, and notes in particular that his 1954 paper on current trends in linguistics defines a linguist's three principal tasks as being to identify what is described using the word language (\emph{langue}), how to describe this object (linguistic methodology), and thirdly, to confront ``the problem of signification''. Quoting {\Benveniste}, ``Language (\emph{langage}) has as its function to say something. What exactly is this something in view of which language is articulated and how do we delimit it in relation to language itself? The problem of signification is posed'' \citep[7]{Benveniste1954}.\footnote{``Le langage a pour fonction de dire quelque chose. Qu'est exactement ce quelque chose en vue de quoi le langage est articulé et comment le délimiter par rapport au langage lui-même? Le problème de la signification est posé.'' \citet[27--57]{Ono2007} has shown how in {\Benveniste}'s writings from 1945 until the definitive formulations in \citeyear{Benveniste1969semiologie} and \citeyear{Benveniste1970appareil}, the meaning of \emph{énonciation} is often ambiguous, or even indicates quite clearly what he will eventually refer to as \emph{énoncé}. See also \citet{Coquet1987} and \citet{JosephInPress}.}

\emph{Signification} — essentially, meaning — is implicitly conceived here as lying outside the language system (\emph{langue}), while being its \emph{raison d’être}. Signification and enunciation occupy a ``semantic'' realm, distinct from the ``semiotic'' one of the language (see the Editors' Introduction to \citealt[49--51]{Benveniste2012}). Understanding the semantic is the linguist's third task. The wording makes clear that signification lies outside language not just as a \emph{langue} but as the more general \emph{langage} as well, being the something that it is the ``function'' of language and languages to say. The challenge is to identify and delimit meaning with relation to language, which is made difficult because language is itself articulated with this function in view.

{\Benveniste}'s initial presentation of his approach incorporates a question which it provoked in the minds of other structural linguists, as to whether enunciation, as use, was not what {\Saussure} meant by \emph{parole}, speech. He does not directly answer the question, but indicates how his focus is a different one.

\largerpage[-1]
\begin{modquote}
Enunciation is putting the language to work through an individual act of use.
 
But isn't this manifestation of enunciation simply \emph{parole}, the discourse which is produced each time one speaks? — We must take care to focus on the specific condition of enunciation: it is the act itself of producing an utterance, and not the text of the utterance, that is our object. This act is the fact of the speaker who mobilizes the language on his or her own behalf. The relationship of the speaker to the language determines the linguistic features of the enunciation. \citep[80]{Benveniste1970appareil}\footnote{``L'énonciation est cette mise en fonction de la langue par un acte individuel d'utilisation. Le discours, dira-t-on, qui est produit chaque fois qu'on parle, cette manifestation de l'énonciation, n'est-ce pas simplement la `parole'? — Il faut prendre garde à la condition spécifique de l'énonciation : c’est l'acte même de produire un énoncé et non le texte de l'énoncé qui est notre objet. Cet acte est le fait du locuteur qui mobilise la langue pour son compte. La relation du locuteur à la langue détermine les caractères linguistiques de l'énonciation.''}
\end{modquote}

The speaker is not ``speaker'' before the act of enunciation. With enunciation, speaker becomes both speaker and subject; the enunciation positions him or her vis-à-vis the language, while at the same time that relationship shapes the enunciation.

In presenting enunciation not as an alternative to \isi{structuralist} analysis, but as a parallel track, {\Benveniste} can be said to fulfil a wish expressed by the Neogrammarians Hermann Osthoff (1847--1909) and Karl Brugmann (1849--1919), when they remarked that, in the past, ``Languages were indeed investigated most eagerly, but people speaking, much too little'' (\emph{Man erforschte zwar eifrigst die \emph{spra\-chen}, aber viel zu wenig \emph{den sprechenden menschen}.}) \citep[iii]{OsthoffBrugmann1878}. But more striking is how far forward looking the approach is, anticipating ideas of decades later on stance, voice, identity, indexicality, in addition to the direct continuations of enunciation in the work of Antoine Culioli and others in France. Pierre {\Bourdieu}'s (1930--2002) conceptions of language and symbolic power are also grounded in {\Benveniste}, and in fact it was {\Bourdieu} who in \citeyear{Benveniste1969vocabulaire} coordinated the assembling and publication of perhaps {\Benveniste}'s most influential book, the \emph{Vocabulaire des institutions indo-européennes}. It provides the context for understanding what {\Benveniste} means when he says that ``the language contains the society'' (see above, p. \pageref{q:joseph:langcontainssoc}). When he traces the history of a social institution such as ``personal loyalty'' back through each of the branches of the Indo-European language family, adducing precise etymological evidence to show the very different ways in which loyalty was conceived among Celtic, Germanic, Baltic, Slavic, Italo-Roman, Greek and Persian tribes and peoples, the conclusion seems inescapable that the institutional differences among them are historically bound to the language of their enunciation, so deeply as to be ``contained'' not just in the sense of residing within, but in the stronger sense of being prevented from escaping.

{\Benveniste}'s third task of \citeyear{Benveniste1954} can be read as an attempt at responding to the problematizing of meaning that was at the heart of behaviourism, the same problem that motivated {\Bloomfield} to de-psychologize his linguistics, though {\Benveniste} attacks the problem with a different strategy. The insight particular to {\Benveniste} is that the language system and the speaking person occupy different conceptual spheres that nevertheless intersect with one another. He explores this initially, and in greatest detail, in his papers on person and deixis.\footnote{These include, following on from the \citeyear{Benveniste1946} and \citeyear{Benveniste1949} papers cited above, \citet{Benveniste1956nature} and \citet{Benveniste1958}. It is surprising that, in his review of \citet{Benveniste1966}, Winfred P. Lehmann (1916--2007) categorized these papers as ``psycholinguistics'' \citep{Lehmann1968}. Equally surprising is Lehmann's view that ``If in any of his essays {\Benveniste} discusses linguistic theory as such, it is in the first three, which treat the development of linguistics''. In other words, for Lehmann, what {\Benveniste} is doing is not linguistic theory at all, which was a compliment from the pen of a non-Chomskyan American linguist like Lehmann in 1968.} {\Benveniste}'s semiology as laid out in the second half of his \citeyear{Benveniste1969semiologie} paper and the lectures of late 1968 and early 1969 combines the systematicity of a \emph{langue} as conceived by {\Saussure} with the \emph{inter}systematicity assumed by Charles Sanders {\Peirce} (1839--1914). ``There is no trans-systematic sign'', \citet[53]{Benveniste1969semiologie} writes;\footnote{``Il n'y a pas de signe trans-systématique.''} the value of each sign ``is defined solely within the system which integrates it'', which is perfectly Saussurean. Nevertheless, every signifying system other than a language must be interpreted through a language. ``Every semiology of a non-linguistic system must make use of a language to translate it; thus it can exist only through and in the semiology of a language, […] which is the interpretant of all other systems, linguistic and non-linguistic'' \citep[60]{Benveniste1969semiologie}.\footnote{``Toute sémiologie d'un système non-linguistique doit emprunter le truchement de la langue, ne peut donc exister que par et dans la sémiologie de la langue […] ; la langue est l'interprétant de tous les autres systèmes, linguistiques et non-linguistiques.''} And from his last lectures: ``It is the language as system of expression that is the interpretant of all institutions and of all culture'' \citep[83]{Benveniste2012}.\footnote{``C'est la langue comme système d’expression qui est l'interprétant de toutes les institutions et de toute la culture.''} One could argue that this core Benvenistean axiom is implicit in both {\Saussure} and {\Peirce}, but {\Peirce} in particular might have resisted it. It reflects the way a linguist thinks, rather than a psychologist.

The turn the lectures then take, which the article did not, is one that {\Saussure} would certainly have resisted. To say as I have done that {\Benveniste}'s semiology combines the systematicity of a \emph{langue} as conceived by {\Saussure} with the intersystematicity assumed by {\Peirce} is potentially deceptive, because systematicity must be understood in a strong sense for {\Saussure}, and in a weaker sense for {\Peirce}, who places the stress on the ``inter-''. {\Benveniste} criticizes {\Peirce} for ``mistaking'' words for being the whole of language. It is not words, not lexicon, not semantics or even syntax that is the foundation of \isi{structural linguistics}, but phonology and morphology. And yet, when {\Saussure} is teaching semiology, words are what he uses to exemplify the sign; he brings in morphology in his discussion of the associative axis and relative motivation, but sounds hardly figure. Phonemes do not appear to be signs, just constituents of signifiers, even though the differences between phonemes are the ultimate source of signification, and that poses a puzzle: what differentiates a phoneme from a non-speech sound is some sort of signification that this is a signifying sound.

Here {\Peirce}'s idea of ``interpretance'' offers a valuable insight: that the very first meaning of every sign is: ``I am a sign. Interpret me''. And even if {\Benveniste} is right that {\Peirce} only thinks about signification at the level of words, nothing in principle prevents us from extending this insight to the level of phonemes.

Regarding his critique of {\Peirce} for reducing languages to words, it is worth noting how widely known {\Benveniste}'s revered teacher {\Meillet} was for his Latin etymological work, and that {\Benveniste} himself had his broadest impact through his \citeyear{Benveniste1969vocabulaire} \emph{Vocabulaire}, which is word-based. Its focus is on the semantic, and it can be seen as his major practical achievement in the linguistics of enunciation. Yet it shows on every page how understanding the semantic at a deep level requires detailed examination of the semiotic, and how such semantic understanding is in turn what allows us to weigh up alternative analyses of phonological and morphological facts in the semiotic system. {\Benveniste} underscores in his lecture notes ``\emph{the impossibility of reaching the semantic in language without passing through the semiotic plus the grammar}'' \citep[114]{Benveniste1969vocabulaire}.\footnote{``\emph{l'impossibilité d'atteindre le sémantique en langue sans passer par le sémiotique plus la grammaire}.''} {\Peirce} tried to reach the semantic through words alone, without signs, without the language system. {\Saussure} did not deny the self-evident link between the semiotic and the semantic, but observed methodological scruples whereby he, as a grammarian (the term he usually applied to himself), could only pronounce on the semiotic, the semantic being the realm of expertise of psychologists and philosophers.

{\Saussure} and {\Peirce} are for {\Benveniste} the key innovative thinkers of two orders of language and signification. With {\Peirce}, {\Benveniste} folds in the later phenomenology of Edmund Husserl (1859--1938) and of the Husserlian linguist Hendrik Joseph Pos (1898--1955). {\Saussure} stands at the head of the tradition of modern linguistics in which {\Benveniste} himself was trained. For {\Benveniste}, 

\begin{table}
\begin{tabular}{l l}
    {\Peirce} and phenomenology & {\Saussure} and \isi{structural linguistics} \\
    represent the order & represent the order \\
     & \\
    semantic & semiotic \\
    intention/intended & signifier/signified \\
    enunciation & language system \\
    utterance & speech \\
    words and things in the world & signs and social structure
\end{tabular}
\end{table}

\newpage 
Structural linguistics is based on the Saussurean order, which excludes consideration of writing. The new linguistics of enunciation envisioned by {\Benveniste} would combine the two orders, and one of the main aims of his last lectures is to understand how they are bridged by writing.

Insofar as the marginalization of writing is an aspect of structuralism, {\Benveniste}’s last lectures pass unhesitatingly beyond it. The fundamentally philological nature of his etymological work makes it pre-\isi{structuralist}, though in his explanations of the history of individual words the spirit and basic approach of structural method come through. And if the central roles he accords to writing and enunciation make him a post-\isi{structuralist}, that is certainly not a flag he wanted to wave. {\Benveniste} strove to reconcile his vision of the future path of linguistics with its present and past. Or, more precisely, its pasts.

\section{Conclusion}
\label{sec:joseph:conc}

This study has focused on a small set of linguists whom histories of linguistics place in the \isi{structuralist period}, and who embraced formalist principles to a greater or lesser extent while also resisting them. It has examined some of the motives for their resistance, which include a perceived abandonment of attention to phonological and philological detail ({\Benveniste} and {\Sauvageot}), as well as to speakers (the same two, plus {\Martinet}), along with a proclivity towards atomism and discontinuity ({\Meschonnic}). Interpersonal relations, political affiliations and national identity have also come into the picture.

There are other chapters to be added to this story, including the polemic between {\Bloomfield} and Leo Spitzer (1887--1960), with \citet{Spitzer1944} calling out {\Bloomfield}'s mechanism for the reductionism it entailed; C. K. Ogden and the associated figures studied by \citet{McElvenny2018}; {\Hjelmslev}, who never renounced his early Saussurean commitments but moved progressively away from what he saw as the prioritizing of form over function and meaning \citep{Joseph2018hj}; and of course Chomsky, who did battle with the methodological and epistemological commitments of the older generation of linguists who are generally classed as ``American structuralists'', but where Chomsky attacked from a more deeply \isi{structuralist} position.

My aim may seem counter-\isi{structuralist} in trying to undo the paradigm. Yet, deep down, what is this enterprise if not a search for the Order which governs the movement of the stars of the modern science of language, and which is the foundation of our society as its practitioners? Without this principle, everything would revert to chaos.

\sloppy
\printbibliography[heading=subbibliography,notkeyword=this]
\end{document}
