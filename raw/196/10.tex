\chapter{Syntactic problem solving}
\label{sec:10}

After investigating \isi{conscious problem solving activity}, the focus will now be on potentially \isi{subconscious problem solving} in regard to syntax. In the next chapter, we will discuss the influence of the MT output on \isi{syntactic processing}, which might lead to problem solving activities. \isi{Statistical MT systems} seem to have difficulties with syntactic structure, because word order (especially verb positioning) follows different rules in \ili{English} and \ili{German} (e.g. %\label{ref:ZOTEROITEMCSLCITATIONcitationIDzwaLDcWNpropertiesformattedCitationKolssetal2008plainCitationKolssetal2008citationItemsid105urishttpzoteroorgusers1255332itemsQZM5JEF8urihttpzoteroorgusers1255332itemsQZM5JEF8itemDataid105typepaperconferencetitleSimultaneousGermanEnglishlecturetranslationcontainertitleProceedingsofIWSLTpublisherplaceHawaiipage174181eventIWSLTeventplaceHawaiiauthorfamilyKolssgivenMuntsinfamilyWlfelgivenMatthiasfamilyKraftgivenFlorianfamilyNiehuesgivenJanfamilyPaulikgivenMatthiasfamilyWaibelgivenAlexissueddateparts2008schemahttpsgithubcomcitationstylelanguageschemarawmastercslcitationjsonRNDL2k2z0T7Gz}
\citealt{KolssEtAl2008}: 178). Hence, syntax might be of particular interest in regard to problem solving in PE. On the one hand, it might be possible that syntactic structures are primed by the source text structure in all tasks and the MT output in PE and MPE (cf. %\label{ref:ZOTEROITEMCSLCITATIONcitationID5pCCGynApropertiesformattedCitationBangaloreetal2016plainCitationBangaloreetal2016citationItemsid152urishttpzoteroorgusers1255332itemsVQKM26ZXurihttpzoteroorgusers1255332itemsVQKM26ZXitemDataid152typechaptertitleSyntacticVarianceandPrimingEffectsinTranslationcontainertitleNewDirectionsinEmpiricalTranslationProcessResearchpublisherSpringerpublisherplaceHeidelbergNewYorkDordrechtLondonpage211238eventplaceHeidelbergNewYorkDordrechtLondonauthorfamilyBangaloregivenSrinivasfamilyBehrensgivenBergljotfamilyCarlgivenMichaelfamilyGhankotgivenMaheshwarfamilyHeilmanngivenArndtfamilyNitzkegivenJeanfamilySchaeffergivenMoritzfamilySturmgivenAnnegreteditorfamilyCarlgivenMichaelfamilyBangaloregivenSrinivasfamilySchaeffergivenMoritzissueddateparts2016schemahttpsgithubcomcitationstylelanguageschemarawmastercslcitationjsonRNDJ8zoHT01tr}
\citealt{BangaloreEtAl2016}). On the other hand, syntactic error prone MT output might complicate the PE process. It is hypothesised that regular syntactic structures do not cause problems in the translation process for trained translators. However, as the MT system sometimes disarranges syntactic structures in the \isi{target language}, syntax becomes a problem that the translator has to consciously solve in the PE task. Hence, syntactically unacceptable MT output should cause longer production and processing times.


\begin{itemize}
\item H\textsubscript{1}: The production and processing data are significantly different according to the quality of the MT output regarding syntax.



H\textsubscript{01}: The quality of the MT output regarding syntax has no influence on the production and processing data.

  

\item H\textsubscript{2}: Syntax requires conscious problem solving in the PE task, while it is not considered a problematic feature in the TfS task. The processing data are statistically different between professionals and students (or between participants with different experience coefficients), as more experienced participants can handle syntactically less high quality MT output better than less experienced participants.



H\textsubscript{02}: Syntax cannot be categorised as problematic both in PE and TfS. The experience of the participants – independent of the quality of the MT output – has no influence on production and processing times.
\end{itemize}


First, the MT output at a sentence level was evaluated in terms of syntax for all six texts by three raters. The syntax in the MT sentence was categorised as either \textit{acceptable}, \textit{partly acceptable}, or \textit{not acceptable}. This evaluation was done on a sentence basis, which might be subject to discussion, but as %\label{ref:ZOTEROITEMCSLCITATIONcitationIDaxZ7uiEQpropertiesformattedCitationKrings1986plainCitationKrings1986dontUpdatetruecitationItemsid103urishttpzoteroorgusers1255332itemsVMUMBHRNurihttpzoteroorgusers1255332itemsVMUMBHRNitemDataid103typebooktitleWasindenKpfenvonbersetzernvorgehtEineempirischeUntersuchungzurStrukturdesbersetzungsprozessesanfortgeschrittenenFranzsischlernernpublisherGunterNarrVerlagpublisherplaceTbingenvolume291eventplaceTbingenauthorfamilyKringsgivenHansPissueddateparts1986schemahttpsgithubcomcitationstylelanguageschemarawmastercslcitationjsonRNDnKEIgp0fXP}
\citet[197-198, translated J. N.]{Krings1986} emphasises in his study on problem solving in translation:


\begin{quote}
Apart from those two examples, the translators adhere to sentence boundaries unconditionally. […] This might be one of the most exciting results in respect to the translator's problem solving order […]. Despite the strong textual relation between all sentences in both texts, the sentence is – simply characterised by ending with a dot – next to single translation problems the ultimate \isi{translation unit} for the participants.\footnote{Original text: „Abgesehen von diesen beiden Beispielen halten sich die Übersetzer uneingeschränkt an den Satzrahmen. […] Dies ist vielleicht eines der erstaunlichsten Ergebnisse im Zusammenhang mit der [...] diskutierten Reihenfolge in der Problembehandlung. Trotz der starken textuellen Verflechtungen aller Sätze untereinander in den beiden ausgewählten Texten ist der Satz, und zwar rein interpunktorisch verstanden, für die Versuchspersonen neben dem einzelnen Übersetzungsproblem die Übersetzungseinheit schlechthin.“}
\end{quote}


The MT output was considered \textit{acceptable} when syntactic structures did not include any mistakes, \textit{partly acceptable} when some clauses were without errors, but some included errors, e.g. the main clause contained syntactic errors, but the subordinate clause did not, and \textit{not acceptable} when most or all the clauses included errors. We have to keep in mind that the evaluation only considers syntax and no other aspect of the MT output. Hence, if a sentence was categorised as acceptable, it only means that the syntax of the sentence was acceptable, it may still include lexical or grammatical errors. Further, the PE instructions (see \sectref{sec:7:2}) did not specify any direct rules regarding how the participant should treat syntax. However, they were instructed to use “as much raw translation as possible” and change MT “only where absolutely necessary”. However, whether adapting the target syntax is absolutely necessary is very subjective. An inter-rater agreement was calculated with Fleiss' kappa. A total of 41 sentences were assessed by 3 raters. The results were $\kappa=0.52$, $z=8.09$ and $p<0.0001$, which can be interpreted as a moderate agreement.



All in all, the six texts comprise 41 sentences, of which 10 were categorised as \textit{acceptable}, 20 as \textit{partly acceptable}, and 11 as \textit{not acceptable} (find the detailed evaluation in \sectref{sec:Appendix:C} \tabref{tab:C:1}). The MT system retained the sentence boundaries, meaning that there is a target sentence for every source sentence. This influenced the post-edited target text as well. While one source sentence was split into two target sentences in 18 instances in the TfS task, this only happened nine times in the PE task. Similarly, two source sentences were joined to one target sentences twelve times in the TfS task, but in only six instances in the PE task. In the following, we will discuss keylogging and eyetracking data for the difference in syntactic quality of the sentences.


\section{Overview production and processing times}
\label{sec:10:1}

To analyse production and processing in the different tasks, the same parameters are used as in \sectref{sec:10:3} on lexical analyses – \isi{\textit{Dur}}, \isi{\textit{GazeS}}, \isi{\textit{GazeT}}. Further, we will add two new parameters, namely \isi{\textit{FixS}} and \isi{\textit{FixT}}, because they will provide further insight into the processing of the sentences and different behaviour in the tasks. These two parameters are defined as the following:


\begin{quote}
“\textbf{FixS}: Number of fixations on source text unit […]
\end{quote}

\begin{quote}
\textbf{FixT}: Number of fixations on target text unit […].” %\label{ref:ZOTEROITEMCSLCITATIONcitationIDA4kRC2cNpropertiesformattedCitationCarlandSchaeffer2013plainCitationCarlandSchaeffer2013citationItemsid144urishttpzoteroorgusers1255332itemsP4KVCC89urihttpzoteroorgusers1255332itemsP4KVCC89itemDataid144typearticlejournaltitleTheCRITTTranslationProcessResearchDatabaseV14URLhttpbridgecbsdkresourcestprdbTPRDB14pdfauthorfamilyCarlgivenMichaelfamilySchaeffergivenMoritzJissueddateparts2013accesseddateparts2014610schemahttpsgithubcomcitationstylelanguageschemarawmastercslcitationjsonRNDBRPJa1LEK5}
(\citealt{CarlSchaeffer2013}: 22)
\end{quote}


When I dealt with lexical items, I usually considered one word or phrase (i.e. multi-word units). However, I will deal with sentences on the syntactic level, which leads to longer production and processing times, which are influenced by many additional factors other than only syntax. Hence, we will normalise the parameters on a character level for mean values, standard deviations, and first statistical tests. Further, eyetracking data on the source and target texts will be disregarded in the following chapters for the MPE, although they were accessible in the database. When the task was conducted, the participants only worked in one window in the editor. Hence, it is inexplicable why the eyetracking data were separated into source text and target text data and how this separation was conducted. However, the existing eyetracking data were combined to total fixation counts (\textit{TFix}) and gaze durations (\textit{TGaze}) on the whole text – introducing two additional new parameters to the analysis. When the fixation data are combined and normalized per character, the differences between the mean values are not extremely different between the tasks (see \tabref{tab:10:1}), however the differences are highly significant (see \tabref{tab:10:2}).



In the following, \tabref{tab:10:1} provides an overview of the mean values of the parameters per character according to the tasks. It is hypothesised that the PE and MPE task should take less time to produce the target sentences than the TfS task. Further, as the source text becomes less important in the PE task, the gaze data on the source text (\textit{GazeS} and \textit{FixS}) should be smaller than in the TfS task. Finally, gaze on the target text (\textit{GazeT} and \textit{FixT}) should be about the same for both tasks or higher for PE as the MT output is the main source of information. The data for MPE should behave similarly to the PE data. The difference between MPE and PE data is expected to be low.

\begin{table}
\begin{tabular}{l*{6}{S[table-format=4.2]}}
\lsptoprule
 & \multicolumn{3}{c}{Mean} & \multicolumn{3}{c}{SD}\\\cmidrule(lr){2-4}\cmidrule(lr){5-7}
 Parameter & \multicolumn{1}{c}{MPE} & \multicolumn{1}{c}{PE} & \multicolumn{1}{c}{TfS} & \multicolumn{1}{c}{MPE} & \multicolumn{1}{c}{PE} & \multicolumn{1}{c}{TfS}\\
 \midrule 
 Dur & 693.33 & 739.22 & 1077.58 & 585.3 & 535.74 & 690.82\\
 GazeS & {\textendash} & 303.17 & 506.56 & {\textendash} & 792.83 & 907.77\\
 GazeT & {\textendash} & 590.57 & 612.81 & {\textendash} & 1018.37 & 792.64\\
 FixS & {\textendash} & 1.75 & 2.74 & {\textendash} & 4.59 & 4.82\\
 FixT & {\textendash} & 3.59 & 3.17 & {\textendash} & 7.03 & 4.76\\
 TFix & 2.21 & 2.75 & 2.95 & 5.06 & 5.80 & 4.61\\
 TGaze & 470.19 & 457.57 & 559.90 & 1760.07 & 1018.37 & 792.64\\
\lspbottomrule
\end{tabular}
\caption{Mean and SD of the Parameters per Task and per Character}
\label{tab:10:1}
\end{table}


The mean values in \tabref{tab:10:1} confirm that the parameters are highest for TfS, except for \textit{FixT} which is on average higher in PE. Further, the data for MPE are always the lowest, except for \textit{TGaze} which is slightly lower for PE. Interestingly, the total fixation count is higher in PE compared to MPE. This shows that single fixations take longer in MPE than in PE (and also as in TfS), which indicates that processing takes longer in MPE than in the other tasks. This seems reasonable as there is no source text to compare the MT output to, which is expected to be cognitively more demanding when the participants encounter problematic or error prone MT-output. The missing source text on the other hand also explains why there are less fixations in total, because there is less text that the participants need to process.



When longer processing results in longer \isi{fixation durations} but smaller \isi{fixation counts}, it also explains the high fixation counts and the low \isi{total fixation duration} on the target text in PE compared to TfS, because TfS could be considered more cognitively demanding. Furthermore, the higher fixation counts and total fixation durations in the target text in TfS might indicate that the target text becomes more important in the revision task – the source text is only consulted when doubts about the translation arise – and hence is fixated more often and longer in the total session than the source text. Finally, the extraordinarily high standard deviations are very striking and might indicate great differences between individuals and\slash or segments.

\begin{table}
\resizebox{\textwidth}{!}{\begin{tabular}{l *{3}{S[table-format=5.1]S[table-format=<1.5]}}
\lsptoprule
 & \multicolumn{2}{c}{MPE vs. PE} & \multicolumn{2}{c}{MPE vs. TfS} & \multicolumn{2}{c}{PE vs. TfS}\\\cmidrule(lr){2-3}\cmidrule(lr){4-5}\cmidrule(lr){6-7}
 Parameter & U & p & U & p & U & p\\
 \midrule 
 Dur & 43374 & 0.088 & \bfseries 29070 & \bfseries <0.0001 & \bfseries 31479 & \bfseries <0.0001\\
 GazeS & {\textendash} & {\textendash} & {\textendash} & {\textendash} & \bfseries 27342 & \bfseries <0.0001\\
 GazeT & {\textendash} & {\textendash} & {\textendash} & {\textendash} & 43180 & 0.143\\
 FixS & {\textendash} & {\textendash} & {\textendash} & {\textendash} & \bfseries 28406.5 & \bfseries <0.0001\\
 FixT & {\textendash} & {\textendash} & {\textendash} & {\textendash} & 47453.5 & 0.6146\\
 TFix & \bfseries 34600.5 & \bfseries <0.0001 & \bfseries 29100.5 & \bfseries <0.0001 & \bfseries 39212 & \bfseries <0.001\\
 TGaze & \bfseries 41241 & \bfseries 0.0075 & \bfseries 31270 & \bfseries <0.0001 & \bfseries 35280 & \bfseries <0.0001\\
\lspbottomrule
\end{tabular}}
\caption{Mann-Whitney-U-Tests for the Parameters comparing the three tasks (statistically significant results are printed in bold)\label{tab:10:2}}
\end{table}


The tests for significance confirm the impressions (\tabref{tab:10:2}). The data were not distributed normally. Therefore, a \isi{Mann-Whitney-U-test} was conducted. The tests between PE and TfS turned out to be significant for \textit{Dur}, \textit{GazeS}, and \textit{FixS} but not significant for \textit{GazeT} and \textit{FixT} as was expected. For the combined parameter \textit{TFix} and \textit{TGaze} the tests prove significant differences again. To put it in a nutshell, all parameters are significantly higher in TfS when comparing it to PE, except for the eyetracking data on the target text, where there is no significant difference. There is also a significant difference between MPE and TfS for the tested parameter, proving that the parameters are higher for TfS and hence TfS is (at least technically\footnote{%\label{ref:ZOTEROITEMCSLCITATIONcitationIDa5t40FdwpropertiesformattedCitationKrings2001plainCitationKrings2001dontUpdatetruecitationItemsid228urishttpzoteroorgusers1255332items4D665XEKurihttpzoteroorgusers1255332items4D665XEKitemDataid228typebooktitleRepairingtextsempiricalinvestigationsofmachinetranslationposteditingprocessespublisherKentStateUniversityPresspublisherplaceKentOhionumberofpages635sourceLibraryofCongressISBNeventplaceKentOhioISBN9780873386715callnumberP309K75132001shortTitleRepairingtextslanguageengauthorfamilyKringsgivenHansPeditorfamilyKobygivenGeoffreySissueddateparts2001schemahttpsgithubcomcitationstylelanguageschemarawmastercslcitationjsonRNDBclvUyn3qE}
\citet{Krings2001} 
differentiates between temporal, technical (meaning the text \isi{production effort}) and cognitive effort in PE. This differentiation is also used in e.g. %\label{ref:ZOTEROITEMCSLCITATIONcitationIDdMLUtvUspropertiesformattedCitationrtfOuc0u8217Brien2006plainCitationOBrien2006citationItemsid214urishttpzoteroorgusers1255332itemsAE7QW2HGurihttpzoteroorgusers1255332itemsAE7QW2HGitemDataid214typethesistitleMachinetranslatabilityandposteditingeffortAnempiricalstudyusingTranslogandChoiceNetworkAnalysispublisherDublinCityUniversityauthorfamilyOBriengivenSharonissueddateparts2006schemahttpsgithubcomcitationstylelanguageschemarawmastercslcitationjsonRNDH7AcbvCLLT}
\citet{OBrien2006}.})
more demanding. The eyetracking data show significant differences between MPE and PE. The total fixation count is significantly lower for MPE (caused by the missing source text), while the \isi{total fixation duration} is significantly lower for PE. This result supports the assumption that MPE requires more mental effort as was already discussed above.



The tests in \tabref{tab:10:2} were also conducted on the parameter per character to rule out that the length of the sentences influences the results. However, as mentioned above, the values show high \isi{standard deviation} values (\tabref{tab:10:1}). Next to individual differences, this might be caused by the different complexities of the sentences in the tasks and the quality of the MT output for the MPE\slash PE task. Therefore, the next chapter will analyse single sentences with regard to MT quality.


\section{Analysis of the influence of syntactic MT quality}
\label{sec:10:2}

The aim of this chapter is to analyse the syntactic structures on a sentence level. The same parameters will be used for keylogging and eyetracking data that were used in the previous analysis (\isi{\textit{Dur}}, \isi{\textit{GazeS}}, \isi{\textit{FixS}}, \isi{\textit{GazeT}}, \isi{\textit{FixT}}, \isi{\textit{TFix}}, \isi{\textit{TGaze}}). Further, the syntactic quality of the MT output will be taken into consideration.


\subsection{Analysis of production and processing data concerning the quality of the MT output}
\label{sec:10:2:1}

As mentioned in the introduction, the quality of the syntax of in MT output was rated as acceptable, partly acceptable, and not acceptable. Further, it was also shown that the keylogging and gaze data differ for PE and TfS. The hypothesis is that the quality of the MT output influences the data. While \textit{Dur}, \textit{GazeS}, \textit{FixS}, \textit{TGaze}, and \textit{TFix} are still significantly lower for acceptable and partly acceptable MT output in the PE and MPE task in contrast to the TfS task, there is no significant difference when the MT output is considered not acceptable. This does probably not apply for \textit{FixT} and \textit{GazeT}, because there is no significant difference between the two tasks, even if the quality of the MT output is not considered. First, I will look at the mean values for the parameters according to the MT output quality for all three tasks combined in \tabref{tab:10:3}, excluding MPE for \textit{GazeS}, \textit{GazeT}, \textit{FixS} and \textit{FixT}, because only one text was available in the MPE task, and then the values will be separated per task.


\begin{table}
\begin{tabular}{l *{6}{S[table-format=4.2]}}
\lsptoprule
  & \multicolumn{2}{c}{Acceptable} & \multicolumn{2}{c}{Partly Acceptable} & \multicolumn{2}{c}{Not Acceptable}\\\cmidrule(lr){2-3}\cmidrule(lr){4-5}\cmidrule(lr){6-7}
  & \multicolumn{1}{c}{Mean } & \multicolumn{1}{c}{SD } & \multicolumn{1}{c}{Mean } & \multicolumn{1}{c}{SD } & \multicolumn{1}{c}{Mean } & \multicolumn{1}{c}{SD}\\
\midrule 
 Dur & 628.01 & 658.35 & 939.04 & 648.16 & 831.53 & 514.43\\
 GazeS & 605.13 & 1263.54 & 331.21 & 712.23 & 364.37 & 598.04\\
 GazeT & 753.37 & 1209.84 & 535.59 & 796.9 & 591.11 & 787.83\\
 FixS & 3.34 & 6.63 & 1.87 & 4.15 & 1.97 & 3.41\\
 FixT & 4.13 & 8.24 & 3.1 & 5.37 & 3.25 & 4.58\\
 TFix & 3.51 & 6.58 & 2.27 & 4.62 & 2.51 & 4.64\\
 TGaze & 649.71 & 1133.48 & 419.11 & 799.68 & 499.08 & 907.1\\
\lspbottomrule
\end{tabular}
%%please move \begin{table} just above \begin{tabular
\caption{The Mean and SD values of the parameters according to syntactic quality per character (excluding MPE for GazeS, GazeT, FixS, FixT)}
\label{tab:10:3}
\end{table}


As \tabref{tab:10:3} shows, the mean values for production time (\textit{Dur}) increase for partly acceptable and unacceptable quality of the MT output compared to acceptable MT output. However, the highest mean value is assigned to sentences with partly acceptable MT output. One reason could be that partly acceptable sentences are more complex than unacceptable ones. For the remainder of the data, the mean value was the lowest when the MT output was partly acceptable, which is contrary to the result of the production times. We have to keep in mind, though, that the MT output does not influence the TfS task and, therefore, this might (partially) explain why the values are lower for partly acceptable output. Including all three tasks, however, provides the advantage of enabling a general impression on sentence subset. Further, partly acceptable sentences tend to be long, because they often include two or more clauses (of which one is acceptable in terms of syntactic MT output, while the other is not). Hence, the effect of this one syntactic error might be compensated by the length of the overall segment. Maybe sentences with no (syntactic) flaws are more suspicious to the participants than those with few errors and therefore are checked more often. However, why the gaze values are the highest for acceptable MT output is hardly explicable. One would assume that the sentences for which the MT system produces a syntactically acceptable output are less complex and therefore less difficult to process in TfS as well. Other factors such as lexical complexity and the position of the sentence in the text might, however, cause these high gaze values.



Another interesting observation is that the \isi{gaze duration} and \isi{fixation count} on the source text in relation to the target text is higher for acceptable quality than for partly\slash not acceptable quality. One way to explain this could be that more target text reading is necessary when the MT output is bad in PE. Hence, the focus shifts from the source text to the target text, while not especially much attention is necessary for good syntactic quality – neither on the source nor target text. Instead both texts are considered and checked whether they contain the correct information. If the MT output has flaws, however, the participants focus on the target text. Hence, the fixation count and \isi{fixation duration} on the the source text decreases. Next, will we assess the mean values for the parameters according to the the tasks to see whether they shed more light on the unexpected result presented in \tabref{tab:10:4}.

\begin{table}
\begin{tabular}{ll *{6}{S[table-format=4.2]}}
\lsptoprule
 & & \multicolumn{2}{c}{MPE} & \multicolumn{2}{c}{ PE} & \multicolumn{2}{c}{TfS}\\\cmidrule(lr){3-4}\cmidrule(lr){5-6}\cmidrule(lr){7-8}
 & & \multicolumn{1}{c}{Mean } & \multicolumn{1}{c}{SD } & \multicolumn{1}{c}{Mean } & \multicolumn{1}{c}{SD } & \multicolumn{1}{c}{Mean } & \multicolumn{1}{c}{SD}\\
 \midrule 
\multicolumn{7}{l}{Acceptable}\\
\midrule 
& Dur & 373.80 & 483.52 & 468.21 & 515.70 & 1059.68 & 736.11\\
& GazeS & {\textendash} & {\textendash} & 458.99 & 1039.36 & 751.27 & 1446.36\\
& GazeT & {\textendash} & {\textendash} & 712.19 & 1203.41 & 794.56 & 1223.29\\
& FixS & {\textendash} & {\textendash} & 2.7 & 5.98 & 3.98 & 7.20\\
& FixT & {\textendash} & {\textendash} & 4.39 & 10.07 & 3.87 & 5.92\\
& TFix & 3.05 & 5.20 & 3.61 & 8.12 & 3.88 & 6.23\\
& TGaze & 589.79 & 1021.66 & 593.41 & 1116.35 & 770.09 & 1263.19\\\midrule
\multicolumn{7}{l}{Partly Acceptable}\\
\midrule 
& Dur & 864.7 & 620.40 & 833.89 & 529.35 & 1114.55 & 741.42\\
& GazeS & {\textendash} & {\textendash} & 234.49 & 681.17 & 426.05 & 731.26\\
& GazeT & {\textendash} & {\textendash} & 506.6 & 930.97 & 564.02 & 640.66\\
& FixS & {\textendash} & {\textendash} & 1.37 & 4.08 & 2.36 & 4.17\\
& FixT & {\textendash} & {\textendash} & 3.11 & 6.10 & 3.09 & 4.57\\
& TFix & 1.77 & 4.49 & 2.29 & 5.03 & 2.74 & 4.30\\
& TGaze & 380.98 & 941.05 & 378.81 & 794.68 & 495.75 & 638.01\\\midrule
\multicolumn{7}{l}{Not Acceptable}\\
\midrule 
& Dur & 679.91 & 472.88 & 802.83 & 488.98 & 1021.12 & 528.93\\
& GazeS & {\textendash} & {\textendash} & 292.81 & 722.83 & 439.61 & 421.55\\
& GazeT & {\textendash} & {\textendash} & 638.42 & 994.61 & 541.38 & 485.92\\
& FixS & {\textendash} & {\textendash} & 1.64 & 3.95 & 2.34 & 2.71\\
& FixT & {\textendash} & {\textendash} & 3.79 & 5.16 & 2.68 & 3.82\\
& TFix & 2.21 & 5.81 & 2.82 & 4.54 & 2.5 & 3.14\\
& TGaze & 521.06 & 1234.84 & 483.32 & 866.97 & 492.54 & 424.05\\
\lspbottomrule
\end{tabular} 
\caption{The Mean and SD values of the parameters per character according to syntactic quality and task}
\label{tab:10:4}
\end{table}

\begin{figure}[p]
\caption{Visualisations of \tabref{tab:10:4} according to production time Dur}
\label{fig:10:1}
% % \includegraphics[width=\textwidth]{figures/DissertationNitzkeberarbeitet-img29.jpg}
\scalebox{.9}{
\begin{tikzpicture}[trim axis right,trim axis left]
\pgfplotstableread{data/Fig10.1.csv}{\table}
    \pgfplotstablegetcolsof{\table}
    \pgfmathtruncatemacro\numberofcols{\pgfplotsretval-1}
        \begin{axis}[
                    ybar,
                    xtick=data,
                    axis lines*=left,
%                     nodes near coords,
                    ymin=0,
                    xticklabels from table={\table}{P},
                    bar width=3mm,
                    height=5cm,
                    width=\textwidth,
                    ticklabel style={font=\footnotesize},
                    enlarge x limits={0.2},
                    colormap/Accent,
                    cycle list/Accent,
                    legend pos=outer north east,
                    reverse legend
                    ]
            \foreach \i in {1,...,\numberofcols} {
                \addplot+[
                    /pgf/number format/read comma as period, fill
                    ] table [x index={1},y index={\i},x expr=\coordindex] {\table};
                \pgfplotstablegetcolumnnamebyindex{\i}\of{\table}\to{\colname} % Adding column headers to legend
                \addlegendentryexpanded{\colname}
            }   
            \end{axis}    
\end{tikzpicture}
}
\end{figure}

 

\begin{figure}[p]
\caption{Visualisations of \tabref{tab:10:4} according to  total fixation duration on GazeS, GazeT, TGaze}
\label{fig:10:2}
% % \includegraphics[width=\textwidth]{figures/DissertationNitzkeberarbeitet-img30.jpg}
\scalebox{.9}{
\begin{tikzpicture}[trim axis right,trim axis left]
\pgfplotstableread{data/Fig10.2.csv}{\table}
    \pgfplotstablegetcolsof{\table}
    \pgfmathtruncatemacro\numberofcols{\pgfplotsretval-1}
        \begin{axis}[
                    ybar,
                    xtick=data,
                    axis lines*=left,
%                     nodes near coords,
                    ymin=0,
                    xticklabels from table={\table}{P},
                    bar width=4pt,
                    height=5cm,
                    width=\textwidth,
                    ticklabel style={font=\footnotesize},
                    enlarge x limits={0.1},
                    colormap/Accent,
                    cycle list/Accent,
                    legend pos=outer north east,
                    reverse legend
                    ]
            \foreach \i in {1,...,\numberofcols} {
                \addplot+[
                    /pgf/number format/read comma as period, fill
                    ] table [x index={1},y index={\i},x expr=\coordindex] {\table};
                \pgfplotstablegetcolumnnamebyindex{\i}\of{\table}\to{\colname} % Adding column headers to legend
                \addlegendentryexpanded{\colname}
            }   
            \end{axis} 
            \node [align=center, text width=3cm, inner sep=0.25cm,font=\footnotesize] at (2, -0.8) {Acceptable};
            \node [align=center, text width=4cm, inner sep=0.25cm,font=\footnotesize] at (5.6, -0.8) {Partly Acceptable};
            \node [align=center, text width=4cm, inner sep=0.25cm,font=\footnotesize] at (9, -0.8)  {Not Acceptable};
\end{tikzpicture}
}
\end{figure}

 

\begin{figure}[p]
\caption{Visualisations of \tabref{tab:10:4} according to  total fixation count on GazeS, GazeT, TGaze}
\label{fig:10:3}
% % \includegraphics[width=\textwidth]{figures/DissertationNitzkeberarbeitet-img31.jpg}
\scalebox{.9}{
\begin{tikzpicture}[trim axis right,trim axis left]
\pgfplotstableread{data/Fig10.3.csv}{\table}
    \pgfplotstablegetcolsof{\table}
    \pgfmathtruncatemacro\numberofcols{\pgfplotsretval-1}
        \begin{axis}[
                    ybar,
                    xtick=data,
                    axis lines*=left,
%                     nodes near coords,
                    ymin=0,
                    xticklabels from table={\table}{P},
                    bar width=4pt,
                    height=5cm,
                    width=\textwidth,
                    ticklabel style={font=\footnotesize},
                    enlarge x limits={0.1},
                    colormap/Accent,
                    cycle list/Accent,
                    legend pos=outer north east,
                    reverse legend
                    ]
            \foreach \i in {1,...,\numberofcols} {
                \addplot+[
                    /pgf/number format/read comma as period, fill
                    ] table [x index={1},y index={\i},x expr=\coordindex] {\table};
                \pgfplotstablegetcolumnnamebyindex{\i}\of{\table}\to{\colname} % Adding column headers to legend
                \addlegendentryexpanded{\colname}
            }   
            \end{axis} 
            \node [align=center, text width=3cm, inner sep=0.25cm,font=\footnotesize] at (2, -0.8) {Acceptable};
            \node [align=center, text width=4cm, inner sep=0.25cm,font=\footnotesize] at (5.6, -0.8) {Partly Acceptable};
            \node [align=center, text width=4cm, inner sep=0.25cm,font=\footnotesize] at (9, -0.8)  {Not Acceptable};
\end{tikzpicture}
}
\end{figure}

 


First of all, \tabref{tab:10:4} shows that all parameters are the highest for TfS when the MT output is syntactically acceptable, which shows that the participants benefit from the use of the MT system if it works properly. The only exception is fixation count on the target text (\textit{FixT}), which is higher for PE. This again shows that the focus in PE shifts to the target text. Many fixations are necessary to assess the MT output, but the fixations are short if the MT output is acceptable, which leads to a relatively low \isi{total fixation duration} on the target text. However, when the MT output is not acceptable, the gaze data on the target text (\textit{GazeT}) are higher in PE than in TfS, which is in line with previous results. Further, the production times (\textit{Dur}) are much lower in MPE and PE compared to TfS when the MT quality is acceptable. The production time remains the highest in TfS independent of the MT output, which is understandable, because the complete target text has to be produced in TfS but the time increases for PE and MPE and the differences are no longer as obvious. There are no indications, yet that syntax needs to be considered a problem in the translation process. However, the low production times in PE and MPE show that the MT output accelerates the text production task when it is syntactically acceptable.



What we can also observe in \tabref{tab:10:4} is that the gaze data on the source text as well as the gaze data on the total text decrease in all three tasks for partly acceptable quality (when data were collected) and increase again for unacceptable quality, except for TfS. Note, however, that the gaze values on the source text are the highest for the sentence in TfS, in which the MT output quality would be acceptable in the PE and MPE task. This indicates that the processing data are also influenced by the sentence itself and not only by the task. It might be reasonable, for example, to expect that the sentences with partly acceptable quality are often considered less problematic or less complex than the ones with acceptable syntactic MT quality. The latter may include other issues that are not syntax related. The sentences with low syntactic MT output quality might also be the most problematic or most complex, which would also explain why the MT system struggles. On the other hand, the gaze data are even higher for acceptable syntactic quality sentences than for low quality sentences in TfS, which indicates that problems occur in the sentences that are not MT related – both categories result in almost the same processing duration.



Although \isi{text production} (\textit{Dur}) was still the longest in TfS for unacceptable MT output, which is reasonable because the target sentence had to be produced completely from scratch, while in MPE and PE there was still some usable MT output, \textit{TFix} was highest for PE when the MT output was not acceptable and almost equal in MPE and TfS. It was highest, however, in TfS when the MT output was acceptable and partly acceptable and almost equal in PE and MPE, which again indicates that more effort was necessary in MPE, because the participants did not need to process the source text and still the value per character is as high as for PE. Further, \textit{TGaze} was highest for MPE when the MT output was not acceptable, and almost equal for PE and TfS, which supports the argument that much effort was necessary in MPE, especially when the MT output was not acceptable. The \isi{total gaze duration} was also almost equal for MPE and PE when the MT output was acceptable and partly acceptable. These results prove that the decreasing quality of the MT output had a negative influence on the \isi{gaze behaviour} for the PE and MPE task. The high standard deviations of the gaze data in all three tasks also show that there were many individual differences and differences in the segments.


\begin{table}[b]
\begin{tabular}{ll*{3}{S[table-format=4.1]S[table-format=<1.4]}}
\lsptoprule
&  & \multicolumn{6}{c}{Quality of MT output}\\
&  & \multicolumn{2}{c}{Acceptable} & \multicolumn{2}{c}{Partly Acceptable } & \multicolumn{2}{c}{Not Acceptable}\\\cmidrule(lr){3-4}\cmidrule(lr){5-6}\cmidrule(lr){7-8}
& \multicolumn{1}{l}{Parameter} & \multicolumn{1}{c}{U } & \multicolumn{1}{c}{p } & \multicolumn{1}{c}{U } & \multicolumn{1}{c}{p } & \multicolumn{1}{c}{U } & \multicolumn{1}{c}{p}\\
\midrule 
\multicolumn{3}{l}{ PE vs. TfS} \\
\midrule 
 & Dur & \bfseries 1289 & <0.0001 & \bfseries 8539 & <0.0001 & \bfseries 2359 & 0.0042\\
 & GazeS & \bfseries 1840 & 0.0027 & \bfseries 6277 & <0.0001 & \bfseries 1791 & <0.0001\\
 & GazeT & 2469 & 0.6245 & 10147 & 0.0547 & 3283 & 0.773\\
 & FixS & \bfseries 1954.5 & 0.0109 & \bfseries 6418.5 & <0.0001 & \bfseries 1866 & <0.0001\\
 & FixT & 2622.5 & 0.9046 & 10742 & 0.2507 & \bfseries 3920 & 0.0138\\
 & TFix & 2299 & 0.2425 & \bfseries 8884.5 & 0.0004 & 6183 & 0.9198\\
 & TGaze & \bfseries 2096 & 0.0477 & \bfseries 8233 & <0.0001 & \bfseries 2612 & 0.0456\\
\midrule
 \multicolumn{3}{l}{ MPE vs. PE}\\
\midrule 
& Dur & 2381.5 & 0.1352 & 11419 & 0.9014 & 2792.5 & 0.0613\\
& TFix & 2600 & 0.5147 & \bfseries 7442.5 & <0.0001 & \bfseries 2186 & 0.0001\\
& TGaze & 2910 & 0.6014 & \bfseries 8792 & 0.0008 & 2903 & 0.1316\\
\midrule
\multicolumn{3}{l}{ MPE vs. TfS}\\
\midrule 
& Dur & \bfseries 1077 & <0.0001 & \bfseries 8717 & 0.0002 & \bfseries 1861 & <0.0001\\
& TFix & 2344 & 0.1044 & \bfseries 5721.5 & <0.0001 & \bfseries 1974.5 & <0.0001\\
& TGaze & 2398 & 0.156 & \bfseries 6094 & <0.0001 & \bfseries 2252 & 0.0012\\
\lspbottomrule
\end{tabular} 
\caption{Tests for significant differences between the tasks according to quality and parameter (statistically significant results are printed in bold)}
\label{tab:10:5}
\end{table}


\largerpage
All data in \tabref{tab:10:5} were not in a normal distribution. Therefore, a \isi{Mann-Whitney-U test} was conducted. Further, the tests were not directed, but the mean values can be verified in Tables \tabref{tab:10:3} and \tabref{tab:10:4}. When comparing TfS and PE, the results of the tests all proved significance except for \textit{GazeT,} and \textit{FixT} for acceptable and partly acceptable MT output, which confirms the hypothesis for the difference between TfS and PE – except for gaze on the target text. The fixation count on the target text (\textit{FixT}) is significantly higher in PE, which indicates that a lot of effort is required to process syntactically unacceptable MT output. \textit{TFix} was only significantly different for partly acceptable MT output. The parameters were higher for TfS when there was a significant difference, except for \textit{FixT} as mentioned. The MT output was helpful in the PE process and reduced both production time and processing effort. The \isi{gaze behaviour} on the target text is not statistically different between the two tasks, which was expected, due to the previously discussed special role that the target text plays in the PE task. Surprisingly, \textit{GaseS} and \textit{FixS} are still significantly lower for unacceptable MT output. This could be interpreted again by the fact that the participants do not use the source text in PE to correct syntactic mistakes. They rather use the target text to rearrange the MT output into syntactically correct structures.



Although gaze data are available for source and target text in MPE, those data will not be used for the analysis because they were mapped incorrectly, as mentioned above. However, the parameter \textit{Dur}, \textit{TFix}, and \textit{TGaze} will be compared for the remaining task. It is expected that no significant difference can be observed when comparing MPE and PE, because they originate from the same MT output, which is confirmed by the data, except for three parameters, namely \textit{TFix} and \textit{TGaze} for partly acceptable quality and \textit{TFix} for not acceptable quality. These parameters are all higher in the PE task, which might be caused by the existing source text. When the syntactic quality is low(er), the participants might refer to the source text more often and hence more and longer fixations are generated.



Finally, we will compare MPE and TfS. It is expected that the difference is significant for acceptable and partly acceptable quality, but not for not acceptable quality. The assumptions is only partly confirmed by the data (see \tabref{tab:10:5}), because the difference in \textit{TGaze} and \textit{TFix} for acceptable MT output is not significant. The difference between \textit{Dur} for MPE with unacceptable MT output and TfS is still significant (significantly higher for TfS). This might indicate that less changes are made in general in MPE and hence the difference is still significant. The fixation counts are still significantly higher for TfS even when the MT output is syntactically not acceptable. The \isi{total gaze duration}, however, is significantly higher for the MPE task when the MT output is syntactically unacceptable. These two results show that more content needs to be processed in TfS as the target text still needs to be produced (hence, the fixation counts are higher), but the low quality MT output needs longer to be processed reflected in the high \textit{TGaze} numbers. The total fixation durations (\textit{TGaze}) and total fixation counts (\textit{TFix}) on source and target text are lower in MPE and PE than in TfS, when the syntactic MT output is acceptable or partly acceptable. This implies that the mental effort is lower in those tasks, when the syntactic MT output is at least partly acceptable. When the MT output, however, is not acceptable, it does not support the translator, but makes the translation task more difficult.



Finally, a \isi{multiple linear regression} was calculated to evaluate the influence of different independent variables (task, status\slash experience, length of source and\slash or target text, and acceptability of the MT output) on the parameters (\textit{Dur}, \textit{TFix}, \textit{TGaze} for all tasks and \textit{FixS}, \textit{GazeS}, \textit{FixT}, and \textit{GazeT} for PE and TfS). The different participants were set as random effects in the models. As the length of the source (TokS) or the target text (TokT) are included in the model as control variables, the parameters do not have to be normalised.



First, we will consider the parameter \textit{Dur}. The regression reported that neither status nor experience have a significant influence on \textit{Dur}. With the help of an \isi{ANOVA}, it was tested whether the regression model would be improved if the variables were added, but both status ($\chi^2(1)=0.65$, $p=0.4191$) and experience ($\chi^2(1)=1.36$, $p=0.2443$) did not show significance. Hence, the model was calculated considering task and qualitative acceptability as fixed effects, length of source and target text as control variables and participants as random effects. Task has a significant influence on \textit{Dur}, but this depends on the tasks that are compared. When TfS is compared with PE ($t=\pm7.37$\footnote{When we have a parameter in a linear mixed model that is not numeric, the regression takes one factor as a reference. When e.g. TfS is the reference in the parameter task, PE and MPE are tested on TfS and the results are presented for comparing TfS and PE as well as TfS and MPE. However, PE and MPE are not compared. Therefore, we can change the reference. When PE is the reference task, it is compared to MPE, but also again to TfS. The latter yields the same result as the first test, but with a different algebraic sign for t.}, $p<0.0001$) and MPE ($t=\pm9.20$, $p<0.0001$), the differences are significant. However, when MPE and PE are compared, the tasks do not show a significant difference ($t=\pm1.83$, $p=0.0683$). The latter result, however, may have become significant if more data points were available. Those results confirm what was expected from the previous results: The production duration of the target text is significantly higher for the TfS task, as the target text has to be produced completely from scratch. The differences between the MPE and PE tasks are not significant as they start with the same MT output. Acceptability also has a statistically significant influence. While the difference between acceptable and partly acceptable is significant ($t=\pm0.84$, $p<0.0001$) as well as the difference between acceptable and not acceptable sentences ($t=\pm3.69$, $p=0.0002$), the difference between partly acceptable and not acceptable sentences is not significant ($t=\pm1.43$, $p=0.1518$). This implicates on the one hand that a syntactically flawless MT output does impact the production time significantly. On the other hand, this shows that the production time for sentences with which the MT system had some syntactic problems is not statistically different to those for which the MT Output is not acceptable. Further, the control variables are both significant, too (\textit{TokS} ($t=2.09$, $p=0.0371$) and \textit{TokT} ($t=5.28$, $p<0.0001$)), which means that the text production of target text depends on the length of source text and the length of target text, which seems reasonable.


For the next four parameters, we will exclude the MPE task because of the missing source text and the previously mentioned mapping mistakes. First we will look at the gaze on the source text. The model for \textit{FixS} is only influenced by the tasks of the MT output, while \isi{quality} ($\chi^2(1)=0.69$, $p=0.7086$), \isi{status} ($\chi^2(1)=1.58$, $p=0.2093$) and \isi{experience} ($\chi^2(1)=2.31$, $p=0.1283$) do not have an impact. Further, the control variable \textit{TokT} has no additional influence on the model ($\chi^2(1)=0.85$, $p=0.357$), when \textit{TokS} is included, which is to be expected when analysing the fixation on the source text. \isi{Fixation counts} on the source text are significantly different between PE and TfS ($t=\pm5.11$, $p<0.0001$). The source text loses its essence in the PE task because of the target text outline provided by the MT output. The source text is only used as a reference and hence less fixations are necessary. For the same reason, the quality of the MT output has no statistical significant influence on the \isi{total fixation duration} on the source text either. The control variable \textit{TokS} has a statistical influence on the fixation count ($t=3.67$, $p=0.0003$), which is plausible because the longer the segment, the more fixations are expected.



The influence of status and experience of the participants on the model for \textit{GazeS} is hard to judge. When both parameters are tested via the \isi{ANOVA}, both seem to add to the model (as a single parameter, not both parameters in the model; status ($\chi^2(1)=4.11$, $p=0.0425$) and experience ($\chi^2(1)=3.93$, $p=0.0473$)), although the p-value is very close to 0.05. However, when they are both integrated into the model, they do not become significant (status ($t=\pm2.04$, $p=0.0539$), experience ($t=\pm2.00$, $p=0.0582$) – when the task is part of the model). It is not possible to finalise the observation whether or not the \isi{gaze duration} on the source text is dependent on the status\slash experience of the participant. More participants or data may have produced a more obvious result and the assumption has to be re-tested in another experiment. However, the data point towards a potential statistically significant influence, which could not be observed for fixations on the source text. The difference between PE and TfS is again significant ($t=\pm5.98$, $p<0.0001$). Similarly as for \textit{FixS}, the model is not influenced by the syntactic quality ($\chi^2(1)=1.74$, $p=0.4183$), which indicates, as above, that improving the syntactic flaws has no impact on the \isi{gaze duration} on the source text and that the source text is probably not required to repair syntax in the MT output. The control variable \textit{TokS} is again statistically significant ($t=5.65$, $p<0.0001$) in the model for the same reasons as mentioned for \textit{FixS}.



Finally, the gaze data on the target text will be analysed. \isi{Fixation counts} on the target texts (\textit{FixT}) are only influenced by the control variable \textit{TokT} ($t=5.87$, $p<0.0001$), the other parameters do not add to the model (quality ($\chi^2(1)=0.89$, $p=0.6417$), task ($\chi^2(1)=0.17$, $p=0.4686$), status ($\chi^2(1)=0.23$, $p=0.6338$), and experience ($\chi^2(1)=0.12$, $p=0.7283$)). Interestingly, the task is not important for the model. However, this is in line with the former results we found for gaze data on the target text. There is no significant difference between TfS and PE as the MT output is more important than the source text in the PE task. Again, the quality of the MT output does not play a significant role. Finally, we will exclude \textit{TokS} as a control variable, as we focus on fixation counts on the target text and \textit{TokS} does not add to the model ($\chi^2(1)=0.23$, $p=0.6345$) when \textit{TokT} is already included.



We find very similar results for \textit{GazeT}: Only the control variable \textit{TokT} influences ($t=6.88$, $p<0.0001$) the model (\isi{ANOVA} results for the non-influencing data: task ($\chi^2(1)=2.58$, $p=0.1085$), quality ($\chi^2(1)=3.15$, $p=0.2074$), status ($\chi^2(1)=1.13$, $p=0.2888$), and experience ($\chi^2(1)=1.33$, $p=0.2494$)) and we exclude the length of the source text as this would not add to the model ($\chi^2(1)=0.12$, $p=0.726$) when \textit{TokT} is already taken into consideration.



Starting with TFix, we will now analyse the influence of the variables on \isi{fixation counts} and \isi{fixation duration} on the whole text, for which MPE is again taken into consideration. Again, status and experience do not enhance the model (the models were tested again with an ANOVA with the following results: status ($\chi^2(1)=0.00$, $p=0.9732$) and experience ($\chi^2(1)=0.32$, $p=0.5702$)). The difference between the fixation count is significant when comparing all three tasks with each other (TfS vs. PE ($t=\pm2.04$, $p=0.0418$), PE vs. MPE ($t=\pm2.73$, $p=0.0065$) and TfS vs. MPE ($t=\pm4.76$, $p<0.0001$)) with the highest mean value for TfS, followed by PE, and the lowest for MPE (see \tabref{tab:10:1}). This indicates that the most information needs to be processed in TfS in comparison to PE and MPE, which is again very plausible. The source text needs to be processed thoroughly in the TfS task, can be referenced in the PE task and is not available in the MPE task. The fixation counts are again not significantly influenced by the syntactic quality of the MT output ($\chi^2(1)=0.50$, $p=0.7807$). Intriguingly, the length of the source text or the target text significantly add to the model, but not when both factors are included in the model. Hence, it was decided to include TokT in the model as the length of the target text seems to be (slightly) more influential when added to the model ($t=4.36$\footnote{For \textit{TokS}, t would have been 5.04.}, $p<0.0001$) and exclude TokS (value for adding TokS when TokT is already included: $\chi^2(1)=0.03$, $p=0.8524$).



The results for \textit{TGaze} are very similar to those of \textit{TFix}. The status of the participants does not add to the model ($\chi^2(1)=1.69$, $p=0.1934$), and neither does the experience coefficient according to the ANOVA test ($\chi^2(1)=3.89$, $p=0.0486$). The quality categorisation of the MT output of the sentences does not add to the model ($\chi^2(1)=3.19$, $p=0.2024$). The tasks differ significantly when comparing TfS and PE ($t=\pm3.68$, $p=0.0002$) as well as TfS and MPE ($t=\pm4.46$, $p<0.0001$). Between MPE and PE, however, there is no significant difference ($t=\pm0.77$, $p=0.4446$). This result is not as obvious as it may seem at first glance: While a new text needs to be produced in TfS (which makes the total gaze data significantly higher), the MPE and PE texts rely on the same MT output. On the other hand, the source text is missing in the MPE task, which could lead to a significantly shorter \isi{gaze duration} on the whole text. As was already mentioned above, the fixations are on average longer in the MPE task than in the PE task (see \tabref{tab:10:1}), which reflects the fact that there is less to process in MPE as there is only a target text. However the target text is harder to process, because of the missing source text, which prolongs the fixations. The control variable \textit{TokT} will again be used to enhance the model ($t=4.24$, $p<0.0001$) as it is slightly more influential than \textit{TokS}, but adding the latter to the model does not enhance it ($\chi^2(1)=0.01$, $p=0.9134$).


\largerpage
In summary, the \isi{syntactic quality} of the MT output only has a significant impact on the \isi{production times} of the sentences, but not on the \isi{eyetracking data}. This might indicate that \isi{syntactic mistakes} in the MT output do not cause an increased mental effort when participants are confronted with these mistakes. Hence, I conclude that the MT output does not create new problems. The correction of flaws in the syntax of the MT output can be considered a task rather than a problem. No hurdle between the source and target text is created. The participants may have to restructure the sentence, but as they can easily assess what the target syntax should look like, this is only a task. The different tasks, however, influence the models, except for \isi{gaze behaviour} on the source text, which is not statistically different between PE and TfS. This emphasises the importance of the target text again. Although the MT output already created a target text template, this template needs to be processed thoroughly by the translator.



Interestingly, neither the \isi{status} of the participant nor their \isi{experience coefficient} had any significant influence on the parameters. This might have different reasons. Either the participants do not have enough experience in PE and MPE, or syntax is generally not a factor that is influenced by \isi{translation competence} but rather by language competence and hence does not trigger higher processing data. Remember that not even the production time (\textit{Dur}) was influenced by status\slash experience of the participants in the linear mixed models. Another reason might be that neither status nor the experience vector are diverse enough to reveal the differences between single participants with different translation competences. Finally, as was argued before, the text type and the contents are not very close to real life professional translations (the texts are not very domain specific) and hence might blur the differences between the participants.



As was mentioned before, not only the syntax of the MT output influences the quality of the MT output, but many other factors have an impact, too. Therefore, I will contrast two sentences in the next chapter, in which the syntactic quality is the most relevant factor in the MT output.


\subsection{Syntactic analysis on the sentence level excluding non-syntactic factors}
\label{sec:10:2:2}

As was pointed out earlier, the keylogging and eyetracking parameters are potentially not only influenced by the quality of the syntax produced by the MT system, but also by other factors. Hence, we will compare single sentences that are syntactically correct or incorrect, but are flawless in all other aspects in this chapter. Therefore, we will compare the first sentence of Text~4 and Text~5 (both texts do not have a headline). The MT output of sentence one in Text~4 is syntactically not acceptable, but there are no additional problems in the sentence:


\begin{description}
\item[ST:] Although developing countries are understandably reluctant to compromise their chances of achieving better standards of living for the poor, action on climate change need not threaten economic development.
\item[MT output:] Zwar sind die Entwicklungsländer sind verständlicherweise zurückhaltend, ihre Chancen auf Verbesserung des Lebensstandards für die Armen, Maßnahmen gegen den Klimawandel muss nicht bedrohen die wirt-schaftlichen Entwicklung gefährden.
\item[Back Translation:] Although are the developing countries are understandably reluctant, their chances of improving the standards of living for the poor, actions against the climate change need not threaten the economic development endanger.
\end{description}


The main syntactic issue in this sentence is again verb positioning. The doubling of \textit{sind} in the first part is not acceptable, and the positioning in the last two parts is incorrect as well. A possible correction of the MT output could look like the following, when as much of the MT output as possible is used:


\begin{description}
\item[Improved MT output:] Zwar sind die Entwicklungsländer sind verständlicherwei\-se zurückhaltend, ihre Chancen auf Verbesserung des Lebensstandards für die Armen \textit{zu gefährden}, \textit{jedoch müssen} Maßnahmen gegen den Klimawandel die wirtschaftliche Entwicklung \textit{nicht bedrohen}.
\item[Back Translation:] Although are the developing countries understandably reluctant, their chances of improving the standards of living for the poor to endanger, but need actions against the climate change the economic development not threaten.
\end{description}


In comparison, the MT output of sentence one, Text~5 is syntactically unproblematic as well as in other aspects:


\begin{description}
\item[ST:] Sociology is a relatively new academic discipline.
\item[MT output:] Die Soziologie ist eine relativ neue wissenschaftliche Disziplin.
\end{description}


As was already mentioned at the beginning of \sectref{sec:10}, the text corpus in this study only contains 41 sentences. Hence, the selection of MT output sentences that are syntactically correct or not correct and are additionally not influenced by any other factor is very small. Otherwise, two sentences would have been chosen that were more comparable in terms of length. However, the length of the source and target sentence are again included as control variables in the calculations. The parameters are again tested in a linear mixed models for all participants. An individual identifier called \isi{\textit{SegU}} was created for all texts and sentences. It is composed of the text number and the segment number based on the source text segmentation. Hence, sentence one of text four is $\text{SegU}=41$ or sentence one of text five is $\text{SegU}=51$. Consequently, the following analyses only deal with $\text{SegU}=41 \text{\&} 51$. As the sentences differ greatly in length, the length of the source and target text are again considered as control variables. The results shall help us understand whether syntactic quality of the MT output has an impact on the \isi{gaze behaviour} after all, when all other mistakes in the MT output that could influence the PE effort are excluded. This would contradict the results of the previous chapter. Further, the effect on the production time (Dur) should be confirmed by the results.


\begin{table}
\resizebox{\textwidth}{!}{\begin{tabular}{l*{6}{S[table-format=2.2]S[table-format=<1.4]}}
\lsptoprule
 & \multicolumn{2}{c}{SegU } & \multicolumn{2}{c}{Task } & \multicolumn{2}{c}{Status } & \multicolumn{2}{c}{Exp } & \multicolumn{2}{c}{TokS } & \multicolumn{2}{c}{TokT}\\\cmidrule(lr){2-3}\cmidrule(lr){4-5}\cmidrule(lr){6-7}\cmidrule(lr){8-9}\cmidrule(lr){10-11}\cmidrule(lr){12-13}
 Parameter & \multicolumn{1}{c}{χ²(1)} & \multicolumn{1}{c}{p} & \multicolumn{1}{c}{χ²(1)} & \multicolumn{1}{c}{p}& \multicolumn{1}{c}{χ²(1)} & \multicolumn{1}{c}{p}& \multicolumn{1}{c}{χ²(1)} & \multicolumn{1}{c}{p}& \multicolumn{1}{c}{χ²(1)} & \multicolumn{1}{c}{p}& \multicolumn{1}{c}{χ²(1)} & \multicolumn{1}{c}{p}\\
 \midrule 
 Dur & \bfseries  44.61 & \bfseries <0.0001 & \bfseries  13.51 & \bfseries 0.0012 & 2.66 & 0.1032 & 0.67 & 0.4139 & 0 & 1 & 0.26 & 0.6128\\
 FixS & 1.02 & 0.3134 & \bfseries 6.63 & \bfseries 0.0100 & 0.05 & 0.8295 & 0.07 & 0.4029 & 1.02 & 0.3134 & 0.96 & 0.3283\\
 GazeS & 0.93 & 0.3341 & \bfseries 7.57 & \bfseries 0.0059 & 0.07 & 0.7863 & 0.95 & 0.3296 & 0.93 & 0.3341 & 0.86 & 0.3536\\
 FixT & 0.88 & 0.348 & 0.45 & 0.5031 & 0.03 & 0.8701 & 0.92 & 0.338 & 0 & 1 & 0.01 & 0.9105\\
 GazeT & \bfseries 5.07 & \bfseries 0.0243 & 3.42 & 0.0641 & 0.12 & 0.7315 & 0.92 & 0.338 & 0 & 1 & 0.13 & 0.7229\\
 TFix & \bfseries 9.05 & \bfseries 0.0026 & \bfseries  11.13 & \bfseries 0.0038 & 0 & 0.9908 & 0.82 & 0.3647 & 0 & 1 & 0.61 & 0.4333\\
 TGaze & \bfseries  11.75 & \bfseries 0.0006 & \bfseries  11.47 & \bfseries 0.0032 & 0.19 & 0.6634 & 0.66 & 0.4168 & 0 & 1 & 0.26 & 0.6113\\
\lspbottomrule
\end{tabular}}
\caption{Results of ANOVA tests adding to the model excluding MPE for gaze data on source and target text (statistically significant results are printed in bold)\label{tab:10:6}}
\end{table}
 
\tabref{tab:10:6} presents the results of the \isi{ANOVA} tests that focus on the factors influencing the regression model significantly. As we can see in \tabref{tab:10:6}, none of the parameters are influenced by status or experience, which is in line with the previous results. However, neither the length of the source nor the length of the target text is determining for the two sentences. Further, \textit{Dur}, \textit{TFix}, and \textit{TGaze} are the only parameters that are influenced by the segments and the task according to \tabref{tab:10:6}. \textit{Dur} shows a significant difference between the segments ($t=-8.85$, $p<0.0001$) and further varies significantly between PE and TfS ($t=\pm3.30$, $p=0.0021$) as well as MPE and TfS ($t=\pm3.52$, $p=0.0012$), but there is no significant difference between MPE and PE ($t=\pm0.17$, $p=0.8681$). The mean values show that the production duration is much higher for TfS (mean: 150531.0s\footnote{The mean and sd values are not normalised by character in this chapter if not explicitly mentioned as they only underpin the statistic tests and are not part of the analysis themselves.}, sd: 153984.48s) than for PE (mean: 77874.33s, sd: 82589.05s) and MPE (mean: 63186.80s, sd: 77378.49s). Additionally, the mean values reveal the huge differences between the segments (low syntactic quality sentence from Text 4 – mean: 183225.78s, sd: 109143.16s; high syntactic quality sentence from Text 5 – mean: 13487.83s, sd: 21311.16s, which is of course also influenced by the length of the sentences, therefore the mean values are normalised in \tabref{tab:10:6}), which are even more obvious when Task and SegU are separated (see \tabref{tab:10:7}). While there is only very little \isi{production effort} in MPE and PE in segment 51, the \isi{production effort} for both tasks is much higher in segment 41. The differences between MPE and PE, however, are small in segment 41, which demonstrates that the source text is not necessarily vital to correct syntactic errors in the MT output. Further, as the statistically significant differences between PE\slash MPE and TfS show in segment 41, even numerous syntactic mistranslations in the MT output do not prolong the overall production times, meaning that (monolingually) \isi{post-editing} the MT output was still time saving compared to the TfS task.


\begin{table}
\begin{tabular}{*{7}{S[table-format=4.2]}}
\lsptoprule
 & \multicolumn{2}{c}{MPE} & \multicolumn{2}{c}{PE} & \multicolumn{2}{c}{TfS}\\\cmidrule(lr){2-3}\cmidrule(lr){4-5}\cmidrule(lr){6-7}
 \multicolumn{1}{c}{SegU} & \multicolumn{1}{c}{Mean } & \multicolumn{1}{c}{SD } & \multicolumn{1}{c}{Mean } & \multicolumn{1}{c}{SD } & \multicolumn{1}{c}{Mean } & \multicolumn{1}{c}{SD}\\\midrule 
 41 & 655.42 & 262.14 & 701.48 & 266.02 & 1294.66 & 691.70\\
 51 & 18.18 & 21.95 & 50.55 & 104.02 & 713.14 & 464.91\\
\lspbottomrule
\end{tabular}
%%please move \begin{table} just above \begin{tabular
\caption{Dur per character on segment 41 and 51 separated by task}
\label{tab:10:7}
\end{table}


Although the segment was predicted to influence the model for \textit{GazeT}, the parameters did not become statistically significant, when the model was created ($t=\pm2.11$, $p=0.0510$). Hence, we have a total of two parameters (\textit{FixT} and \textit{GazeT}) that are not influenced by any of the predictors and control variables, which means that they are, amongst other things, not statistically dependent on the quality of the MT output. Conclusively, the \isi{gaze behaviour} on the target text is alike no matter what was the task, the syntactic quality of the MT output, the length of the target text, and the status\slash experience of the participants. Similarly, FixS ($t=\pm3.01$, $p=0.0108$) and GazeS ($t=\pm3.29$, $p=0.0055$) differ only significantly in the tasks\footnote{Difference between PE and TfS, MPE is excluded again.}, which can be explained by the fact that PE requires much less focus on the source text which influences the \isi{gaze behaviour}, but they are not influenced by the quality of the MT output.



The total fixation count ($t=\pm19.91$, $p=0.0054$) as well as the \isi{total gaze duration} ($t=\pm19.88$, $p=0.0015$) on both source and target text are influenced by the syntactic quality of the segments. Furthermore, both parameters are statistically different when comparing TfS and PE (\textit{TFix}: $t=\pm2.49$, $p=0.0211$, \textit{TGaze}: $t=\pm3.03$, $p=0.0060$) and TfS and MPE (\textit{TFix}: $t=\pm3.29$, $p=0.0035$, \textit{TGaze}: $t=\pm2.93$, $p=0.0077$), but not when comparing PE and MPE (\textit{TFixT}: $t=\pm0.54$, $p=0.5972$, \textit{TGaze}: $t=\pm0.28$, $p=0.7841$), which could be expected on the one hand because of the similarity of the two tasks, but on the other hand the source text is missing in the MPE task.



All in all, the detachment of the two segments only proved that the syntactic quality of the MT output has an influence on production time and on the overall eyetracking data considering both source and target text, but not on the eyetracking data divided by source and target text for PE and TfS. Therefore, the result of the analysis in \sectref{sec:10:2:1} is confirmed regarding the production time (\textit{Dur}), which is influenced by the syntactic quality, and regarding the gaze data on the source (\textit{FixS} and \textit{GazeS}) and target text (\textit{FixT} and \textit{GazeT}), which are not impacted by the syntactic MT quality. Counting and measuring gaze on both texts resulted in significant differences, which is unexpected. These differences could either indicate an influence of the syntactic quality or that the two sentences were not comparable enough and hence further studies are necessary.


\section{Summary}
\label{sec:10:3}

This chapter analyses the influence of the syntactic quality of the MT output on production and processing data. The analysis was performed on a sentence level and the quality was categorised as acceptable, partly acceptable, and not acceptable. As was expected, the tasks have a significant influence on most production and processing parameters as was proven by the \isi{Mann-Whitney-U-test}s and reinforced by the \isi{linear mixed model}s. The data are significantly higher for TfS than for PE, except for the gaze data on the target text. Similarly, the data are significantly higher for TfS than for MPE, except for the \isi{total gaze duration} in sentences that produced syntactically unacceptable MT for which the opposite applies. The difference between PE and MPE is not significant concerning the production times and the \isi{total fixation duration}, but is significant for \textit{TFix}. \tabref{tab:10:8} summarizes the results of the regression models, exploring which predictors influenced the model and which did not. For all models, the individual participants were used as random effects on the model and the length of the source and target text as control variables.


\begin{table}
 \fittable{
\begin{tabular}{l *{8}{c}}
\lsptoprule
 & \multicolumn{3}{c}{Tasks} & \multicolumn{3}{c}{ Quality} &  & \\\cmidrule(lr){2-4}\cmidrule(lr){5-7}
& MPE vs. PE & MPE vs. TfS & PE vs. TfS & a vs. p & p vs. n & a vs. n & TokS & TokT\\
\midrule
 Dur & \XSolidBrush & \Checkmark & \Checkmark & \Checkmark & \XSolidBrush & \Checkmark & \Checkmark & \Checkmark\\
 FixS & \textendash & \textendash & \Checkmark & \XSolidBrush & \XSolidBrush & \XSolidBrush & \Checkmark & \XSolidBrush\\
 GazeS & \textendash & \textendash & \Checkmark & \XSolidBrush & \XSolidBrush & \XSolidBrush & \Checkmark & \XSolidBrush\\
 FixT & \textendash & \textendash & \XSolidBrush & \XSolidBrush & \XSolidBrush & \XSolidBrush & \XSolidBrush & \Checkmark\\
 GazeT & \textendash & \textendash & \XSolidBrush & \XSolidBrush & \XSolidBrush & \XSolidBrush & \XSolidBrush & \Checkmark\\
 TFix & \Checkmark & \Checkmark & \Checkmark & \XSolidBrush & \XSolidBrush & \XSolidBrush & \XSolidBrush & \Checkmark\\
 TGaze & \XSolidBrush & \Checkmark & \Checkmark & \XSolidBrush & \XSolidBrush & \XSolidBrush & \XSolidBrush & \Checkmark\\
\lspbottomrule
\end{tabular}
}
\caption{Summary of the influential predictors in the linear mixed models (\Checkmark: statistically significant, \XSolidBrush: not statistically significant)\label{tab:10:8}}
\end{table}


Surprisingly, the \isi{status} and the \isi{experience} of the participants had no influence on the linear regression models. Reasons might be (as mentioned in \sectref{sec:10:2:1}) that most participants are not experienced in the PE ad MPE task, that status and the experience vector are not diverse enough to represent the professionalism of the participants, that syntax is not a characteristic of a text that requires special translation skills or that the text types are potentially not realistic translation jobs.



As the analysis has shown, it is difficult to predict the influence of only the syntactic quality on \isi{gaze behaviour}, because too many other MT defects can occur in a sentence that are not syntax related but influence the gaze data, too. The regression models for the overall data set do not yield an influence of the syntactic MT output quality on the gaze values, which was affirmed by the direct comparison of one sentence with syntactically defective MT output and one flawless sentence in \sectref{sec:10:2:2}, which proved that the syntactic quality influences processing time but not \isi{gaze behaviour} directly. The reason for this might be that error prone syntax in the MT output does not cause new problems for the translator when (s)he has to (monolingually) post-edit a text. Although these errors need correction and therefore increase the editing effort, which was shown by the increasing production times (\textit{Dur}), they do not cause problems in the sense defined in \chapref{sec:5}, namely that no new hurdle is created between the source and the target text. Correcting the syntax of MT output can therefore be categorised as a task solving instead of a problem solving activity.



Curiously, the length of the target text as a control variable is often a better predictor than the length of the source text, except (obviously) when we look at the gaze data on the source text. For the two parameters that dealt with the gaze data on the whole text, \textit{TokT} might be a better predictor because \ili{German} translations tend to be longer than the \ili{English} source texts and hence require longer production times and more processing. Further, both text lengths were only essential for \textit{Dur}. When the single segments were compared in \sectref{sec:10:2:2}, neither the length of the source nor the target text influenced the parameter, which is on the one hand reasonable as there is not much overall variety; on the other hand, the two sentences vary drastically in length.



{The first null hypothesis suggested at the beginning of this chapter cannot be rejected in most cases because the syntactic quality of the MT output only influences production times. Hence, the first hypothesis is only true for production times, which are influenced by the quality of the syntactic MT output, but not for \isi{gaze behaviour}.} {However, as the status\slash experience of the participants does not have a significant impact on production and processing data, the second null hypothesis cannot be} {rejected. The second hypothesis could not be proved. In short, syntactic corrections in PE sessions are not categorised as problem solving but as part of a task, which is not dependent on the experience of the translator.}


