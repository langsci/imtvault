\addchap{Acknowledgments}
\epigraph{Hold your breath and count to ten,\newline
And fall apart and start again,\newline
Hold your breath and count to ten,\newline
Start again, start again...\\\textit{Placebo – \ili{English} Summer Rain}\\\\
Sag nicht alles so kompliziert\newline
Weil ich versteh das garantiert nicht\newline
Denk nicht alles so kompliziert\newline
Weil ich versteh, dass das nix wird\\\textit{Wanda – Wenn ich Zwanzig Bin}}

\noindent Two songs and eight lines that came to my mind again and again while preparing this book, although (often) quoted out of context. I regularly remembered the Placebo song during statistical analyses – a completely new field for me at the beginning of this whole project. Although I always enjoyed learning this new, advanced method of data analysis (like presumably most translation students in Germany, I had no clue about statistics, never took a course on data analysis during my B.A.\ or M.A.\ programmes and had to rely on what I had learnt in school, which almost ended with calculating means and medians), it was often incredibly frustrating and time-consuming. Back in the early days I learnt what correlations are and how to perform them in R – my best friend and my worst enemy – but I did not realise that you have to check for normal distribution first to decide on suitable correlations. So I started to correlate data, and correlated more data, and correlated even more data, until somebody finally told me: You need to check for normal distribution first. Hold your breath and count to ten; Fall apart, start again.

The second song by the Austrian band Wanda often accompanied me when I was reading literature that was written in an incredibly complicated way and where the main points were masked (or even hidden) by unusual words and very long sentences (this was usually literature written in \ili{German}). As a result, I often tempered myself during my own writing process and tried to keep it simple and understandable.

Much more essential than music, however, were the people around me, who supported me in so many different ways. First of all, I would like to thank my flatmates and friends who stood by my side although my mood sometimes became unbearable and who still want to be my friends although I have not had much time for them recently: Maike Dankwardt, Sylvana Teifel, Christiana Rohner, Helene Schächtele, Marcella Apple, Lisa Schewe, Sebastian Kriegler, and Simon Bode. Many thanks to my friends who also happen to be professional translators and interpreters (now) – you were always willing to give opinions on matters that concerned me from a practical and not a scientific perspective: Lisa Rüth, Julia Dolderer, Rosa Schröder, Lara Eusemann, and Tina Puetsch. For me, it is very important to keep the reality of the profession in mind when conducting research on translation and translators' behaviour. Many, many thanks to my parents, Heiko and Jutta Nitzke, who always believed in me, always supported me, and always seem to be proud of what I am doing. I could not have done it without you.

By now, I consider most of my colleagues friends, sometimes even close friends. Discussing issues and questions with you helped me a lot in developing my thoughts and approaches. Without your help, some parts of this project would never have been possible for me – both in terms of theoretical approaches and statistical analyses, but also moral support and helping me with my teaching duties: Silke Gutermuth, Katharina Oster, Moritz Schaeffer, Wendy Fox, Sarah Signer (sorry you had to proof-read all of this!), Marcus Wiedmann, Sascha Hofmann, Katja Abels, Don Kiraly, and Tomasz Rozmyslowicz.

Finally, of course, my utmost thanks goes to my supervisors Silvia Hansen-Schirra and Oliver Czulo. Thank you for all your patience, your time, and your willingness to supervise me in the first place. Your guidance through all parts of academia truly helped me to find my way in navigating the entire landscape – from teaching to attending conferences. Thank you for all your essential remarks, hints, and comments that shaped my thinking and, in the end, this book. Last but not least, many thanks to Michael Schreiber who is the third person who jumped into this project and took the time to assess my work.

There are so many more people who helped and\slash or supported me over the years – I cannot name every single one of you, but thanks, thanks, thanks!
