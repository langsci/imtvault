\chapter{Post-editing}
\label{sec:3}

\isi{Post-editing} (PE) "is the correction of raw machine translated output by a human translator according to specific guidelines and quality criteria" %\label{ref:ZOTEROITEMCSLCITATIONcitationID1EMDaqYCpropertiesformattedCitationrtfOuc0u8217Brien2011plainCitationOBrien2011citationItemsid56urishttpzoteroorgusers1255332itemsSFVCRCANurihttpzoteroorgusers1255332itemsSFVCRCANitemDataid56typearticlejournaltitleTowardspredictingposteditingproductivitycontainertitleMachineTranslationpage197215volume25issue3sourceCrossRefDOI101007s1059001190967ISSN0922656715730573authorfamilyOBriengivenSharonissueddateparts2011927schemahttpsgithubcomcitationstylelanguageschemarawmastercslcitationjsonRNDIBayKOG8Sy}
\citet[197--198]{OBrien2011}. Due to an increasing need for translations, more and more companies machine translate texts and send them for editing to (professional) translators afterwards in order to supposedly increase productivity. As %\label{ref:ZOTEROITEMCSLCITATIONcitationIDiUq3NQe2propertiesformattedCitationAllen2003plainCitationAllen2003citationItemsid160urishttpzoteroorgusers1255332itemsWFGP5FUKurihttpzoteroorgusers1255332itemsWFGP5FUKitemDataid160typechaptertitlePosteditingcontainertitleComputersandTranslationAtranslatorsguidepublisherJohnBenjaminsTranslationLibrarypublisherplaceAmsterdamPhiladelphiapage297318volume35eventplaceAmsterdamPhiladelphiaauthorfamilyAllengivenJeffreyeditorfamilySomersgivenHaroldissueddateparts2003schemahttpsgithubcomcitationstylelanguageschemarawmastercslcitationjsonRNDGCDi2suxDy}
\citet{Allen2003} points out, PE introduced a new perspective to translation science, because translators never really had to deal with half finished texts before. One of the most important research questions in PE is hence “to what extent MT output texts are acceptable, and how much human effort is necessary to improve such imperfect texts” (ibid.: 298). From a scientific point of view, PE “is a field in which the human translator and the machine meet – as well as the two disciplines MT and TS [= translation studies, J. N.]” %\label{ref:ZOTEROITEMCSLCITATIONcitationIDGO89vEMJpropertiesformattedCitationrtfuc0u268ulo2014plainCitationulo2014citationItemsid175urishttpzoteroorgusers1255332itemsES4D5SK4urihttpzoteroorgusers1255332itemsES4D5SK4itemDataid175typepaperconferencetitleFromTranslationMachineTheorytoMachineTranslationTheorysomeinitialconsiderationscontainertitleTheFutureofInformationSciencepage3138eventINFuture2013InformationGovernanceauthorfamilyulogivenOliverissueddateparts2014schemahttpsgithubcomcitationstylelanguageschemarawmastercslcitationjsonRNDEmnOwdCdcz}
(\citealt{Culo2014}: 35).



In PE, the target text does not need to be produced from scratch. The translators already have an outline for the final product. Hence, PE and human translation can be considered different tasks. Further, machine translated texts  have different characteristics than human translations. Therefore, PE cannot be seen as another form of proof-reading either. While some mistakes, like spelling and typing errors, hardly ever occur in MT output, some mistakes, e.g. syntatical or lexical ones, would almost never occur in human translation (cf. %\label{ref:ZOTEROITEMCSLCITATIONcitationIDZsliPLpbpropertiesformattedCitationNitzke2016korrekturplainCitationNitzke2016korrekturdontUpdatetruecitationItemsid226urishttpzoteroorgusers1255332itemsTKT6J8G7urihttpzoteroorgusers1255332itemsTKT6J8G7itemDataid226typearticlejournaltitleAuchnurKorrekturlesenPostEditingcontainertitleMDpage2427volume2016issue1authorfamilyNitzkegivenJeanissueddateparts2016schemahttpsgithubcomcitationstylelanguageschemarawmastercslcitationjsonRNDfGtj7mijPd}
\citealt{Nitzke2016korrektur}).


\begin{modquote}
The differences lie in the frequency, repetitiveness, and types of errors. Where a human translator may slip and mistranslate a word once […], the \isi{machine translation} system will plow ahead, consistently mistranslating the same word or phrase over and over again.\footnote{However, this does not entirely apply to statistical and neural MT systems.} (%\label{ref:ZOTEROITEMCSLCITATIONcitationIDiswvsSfXpropertiesformattedCitationKoby2001plainCitationKoby2001citationItemsid227urishttpzoteroorgusers1255332itemsB4K3TNNUurihttpzoteroorgusers1255332itemsB4K3TNNUitemDataid227typechaptertitleEditorsIntroductionPostEditingofMachineTranslationOutputWhoWhatWhyandHowMuchcontainertitleRepairingtextsempiricalinvestigationsofmachinetranslationposteditingprocessespublisherKentStateUniversityPresspublisherplaceKentOhiosourceLibraryofCongressISBNeventplaceKentOhioISBN9780873386715callnumberP309K75132001languageengauthorfamilyKobygivenGeoffreySissueddateparts2001schemahttpsgithubcomcitationstylelanguageschemarawmastercslcitationjsonRND2g6fAgz7pF}
\citealt{Koby2001}: 7)
\end{modquote}


Depending on the purpose of the final post-edited text, PE is divided mainly into \textit{full} and \textit{light} PE\footnote{To my knowledge, this is the most common differentiation. %\label{ref:ZOTEROITEMCSLCITATIONcitationIDR0p8io6OpropertiesformattedCitationAllen2003plainCitationAllen2003citationItemsid160urishttpzoteroorgusers1255332itemsWFGP5FUKurihttpzoteroorgusers1255332itemsWFGP5FUKitemDataid160typechaptertitlePosteditingcontainertitleComputersandTranslationAtranslatorsguidepublisherJohnBenjaminsTranslationLibrarypublisherplaceAmsterdamPhiladelphiapage297318volume35eventplaceAmsterdamPhiladelphiaauthorfamilyAllengivenJeffreyeditorfamilySomersgivenHaroldissueddateparts2003schemahttpsgithubcomcitationstylelanguageschemarawmastercslcitationjsonRNDzwuAh3xDLT}
\citet{Allen2003}, for instance, refers to them as “maximum” and “minimal” PE. The latter is sometimes also referred to as “rapid PE”.}. \isi{\textit{Full PE}} is usually required when the final text is intended for publication. The text must be comprehensible and accurate; grammar and syntax must be flawless. The style of the text is acceptable but does not need to be as good as a human translation. Where low quality is good enough for the final product, \isi{\textit{light PE}} aims to make MT output understandable. According to the TAUS\footnote{Translation Automation User Society } guidelines, a light post-edited text also needs to convey the same meaning as the source text, but style is not important. Further, the text might also contain imperfect grammar or syntax and the reader might realise that the text was generated by a machine. The main aspect is that it delivers the same information as the source text (cf. %\label{ref:ZOTEROITEMCSLCITATIONcitationID1dlSEoHipropertiesformattedCitationMassardoetal2016plainCitationMassardoetal2016citationItemsid174urishttpzoteroorgusers1255332itemsTU2IIZRPurihttpzoteroorgusers1255332itemsTU2IIZRPitemDataid174typereporttitleTAUSMTPostEditingGuidelinespublisherplaceAmsterdameventplaceAmsterdamURLhttpswwwtausnetthinktankarticlesposteditarticlestausposteditingguidelinesauthorfamilyMassardogivenIsabellafamilyMeergivenJaapnondroppingparticlevanderfamilyOBriengivenSharonfamilyHollowoodgivenFredfamilyAranberrigivenNorafamilyDreschergivenKatrinissueddateparts2016accesseddateparts2017316schemahttpsgithubcomcitationstylelanguageschemarawmastercslcitationjsonRNDYX4WtLweQ1}
\citealt{MassardoEtAl2016}: 16-18) In the end, the chosen approach in every PE job is dependent on the purpose and function of the final text and the client – just like translations from scratch.



In the next section, we will take a look at this new field in translation science, which is not all that new in translation practice. Some thoughts on PE development as well as controlled languages and pre-editing will be discussed: \chapref{sec:4} will then concentrate on teaching PE, PE in practice, and PE in research (theoretical and empirical approaches).


\section{The development of post-editing}
\label{sec:3:1}

PE has been a feature of the translation industry for some decades now. Basically, the development of usable MT systems and PE have always gone hand-in-hand (see a more detailed description in %\label{ref:ZOTEROITEMCSLCITATIONcitationIDjBiw0LompropertiesformattedCitationGarcia2009plainCitationGarcia2009citationItemsid173urishttpzoteroorgusers1255332itemsXKNUWNKNurihttpzoteroorgusers1255332itemsXKNUWNKNitemDataid173typearticlejournaltitleBeyondtranslationmemoryComputersandtheprofessionaltranslatorcontainertitleTheJournalofSpecialisedTranslationpage199214volume12issue12authorfamilyGarciagivenIgnacioissueddateparts2009schemahttpsgithubcomcitationstylelanguageschemarawmastercslcitationjsonRND7tYQWNQA87}
\citealt{Garcia2009}). For example, the US Air Force employed 43 people in their MT project (including post-editors) in 1964 with the aim of translating about 100,000 words from \ili{Russian} into \ili{English} per day (cf. ibid.: 295). In the early days (starting around the mid 1950s), the MT process was included a pre-editor who is an expert in the source language, and a post-editor who is an expert in the domain and the \isi{target language}, who did not necessarily have much knowledge of the source language (cf. ibid.: 299-300). As mentioned above, the \isi{ALPAC report} resulted in a big cut in MT financing and also reduced PE activities. The report stated that PE was not cost-effective and that the money should rather be invested in other language studies and tools. Similarly, the first empirical studies on PE can be traced back to the mid 1960, around the time of the ALPAC report. These studies dealt with PE time and rating MT output. After the ALPAC report, almost no empirical, academic research was conducted for a number of decades (cf. ibid.: 300-302). However, research on PE was initiated by non-academics in the 1980s and the introduction of Aslib's annual conference on Translating and the Computer in 1978 also launched many conferences on the topic. (cf. ibid: 302-303) Nonetheless, PE did not vanish entirely but continued to be performed in a less prestigious environment in institutions and companies. (cf. ibid.: 295-296) The MT system \isi{SYSTRAN}, for example, was implemented in the Commission of the European Communities in the late 1970s to assist translators handling the amount of translation work needed in the organisation. During that time, the \isi{European Union} had to deal with “a mere” seven official languages. In the EU context, \citet[236--238]{laurian1984machine} suggested in 1984 that dfferent types of PE needed to be developed for different text types and translation purposes. Further, it was pointed out that post-editors still needed to be very well-trained and proficient, even if they only post-edited rapidly and changed just the major errors. (cf. %\label{ref:ZOTEROITEMCSLCITATIONcitationIDFQpMpeRNpropertiesformattedCitationGarcia2009plainCitationGarcia2009citationItemsid173urishttpzoteroorgusers1255332itemsXKNUWNKNurihttpzoteroorgusers1255332itemsXKNUWNKNitemDataid173typearticlejournaltitleBeyondtranslationmemoryComputersandtheprofessionaltranslatorcontainertitleTheJournalofSpecialisedTranslationpage199214volume12issue12authorfamilyGarciagivenIgnacioissueddateparts2009schemahttpsgithubcomcitationstylelanguageschemarawmastercslcitationjsonRND3HBKkVYk5M}
\citealt{Garcia2009}: 304) With the development of PCs, PE could be performed on screens rather than on paper or a typewriter, which greatly influenced the task as new skills were required from the post-editor such as proficiency in using the keyboard and positioning the cursor. Further, word processing functionalities like using macros, and search and replace functions helped accelerate the PE process. (cf. ibid.%\label{ref:ZOTEROITEMCSLCITATIONcitationIDytHo9wl8propertiesformattedCitationGarcia2009plainCitationGarcia2009citationItemsid173urishttpzoteroorgusers1255332itemsXKNUWNKNurihttpzoteroorgusers1255332itemsXKNUWNKNitemDataid173typearticlejournaltitleBeyondtranslationmemoryComputersandtheprofessionaltranslatorcontainertitleTheJournalofSpecialisedTranslationpage199214volume12issue12authorfamilyGarciagivenIgnacioissueddateparts2009schemahttpsgithubcomcitationstylelanguageschemarawmastercslcitationjsonRNDy9g71V2Cn5}
: 297) The advancing functionalities of PCs and the ongoing research into MT continued to decrease the PE effort, e.g. PE services at the European Commission increased from 30,000 pages in 1990 to 180,000 pages in 1995. (cf. ibid.: 298) While PE remained a relatively unobserved field for many years in translation science, it has been implemented in some organisations for decades now, of which some examples will be introduced in \sectref{sec:4:3}. The application of PE  was organised by the companies and organisations themselves. Guidelines were developed individually and depended on the MT system as well as the needs of the respective company or organisation. However, PE has become a recognised field in translation studies, especially in translation process research, in the last few decades and is studied academically again. Furthermore, prototypes of PE guidelines were developed, e.g. by TAUS\footnote{\url{https://www.taus.net/academy/best-practices/postedit-best-practices/machine-translation-post-editing-guidelines}, last accessed 14 October 2016  and %\label{ref:ZOTEROITEMCSLCITATIONcitationIDbF9uye5ppropertiesformattedCitationMassardoetal2016plainCitationMassardoetal2016citationItemsid174urishttpzoteroorgusers1255332itemsTU2IIZRPurihttpzoteroorgusers1255332itemsTU2IIZRPitemDataid174typereporttitleTAUSMTPostEditingGuidelinespublisherplaceAmsterdameventplaceAmsterdamURLhttpswwwtausnetthinktankarticlesposteditarticlestausposteditingguidelinesauthorfamilyMassardogivenIsabellafamilyMeergivenJaapnondroppingparticlevanderfamilyOBriengivenSharonfamilyHollowoodgivenFredfamilyAranberrigivenNorafamilyDreschergivenKatrinissueddateparts2016accesseddateparts2017316schemahttpsgithubcomcitationstylelanguageschemarawmastercslcitationjsonRNDlJgXkkgQUT}
\citet{MassardoEtAl2016}.}, in recent years. Research began to focus on PE with the increasing use of MT in professional translation situations. Furthermore, \isi{CAT tools}, like CASMACAT\footnote{\url{http://www.casmacat.eu/}, last accessed 13  November 2016.} and MateCat\footnote{https://www.matecat.com/, last accessed 13 November 2016.}, were developed with the purpose of supporting PE and not only translation from scratch. Last but not least, the possibility of integrating MT systems into \isi{translation memory systems} has become a standard feature in most systems, which is highly relevant for professional translators. Combining MT and TM can be the most effective way to approach professional translations, because the advantages of both can be used. The most common way both systems are combined is when no match is found in the TM: The segment is then machine translated, i.e. 100\%-Matches and Fuzzy Matches up to {\textasciitilde}70\% are taken from the TM system and the remaining segments are machine translated.



The study presented by %\label{ref:ZOTEROITEMCSLCITATIONcitationIDmqVa1EdSpropertiesformattedCitationFedericoGaspariAlmaghoutandDoherty2015plainCitationFedericoGaspariAlmaghoutandDoherty2015citationItemsid208urishttpzoteroorgusers1255332itemsTKS8XNJBurihttpzoteroorgusers1255332itemsTKS8XNJBitemDataid208typearticlejournaltitleAsurveyofmachinetranslationcompetencesInsightsfortranslationtechnologyeducatorsandpractitionerscontainertitlePerspectivespage333358volume23issue3authorfamilyGasparigivenFedericofamilyAlmaghoutgivenHalafamilyDohertygivenStephenissueddateparts2015schemahttpsgithubcomcitationstylelanguageschemarawmastercslcitationjsonRNDkZ02skj0Jq}
\citet{GaspariEtAl2015} provides a current and independent overview of the translation market situation with special regard to translation technologies and MT. In total, 438 individuals from around the world responded to an online survey (with the main focus on Europe, though) in November 2012. The study included professional translators, people working in the industry, academia, and governments or institutions. Of the participants, 36\% stated that they did not use MT, 22\% that they did not use MT but instead use \isi{TMS}, 21\% stated that they use MT, and the last 21\% use both. Of the participants who were not already using MT (30\%)\footnote{It is not explained why the numbers do not agree with the numbers in the former question. Probably, participants answered the question, even though they chose the answer that they use MT or MT and TM in the first question.}, 32\% claim that they would (probably) not use MT in the future, 22\% claim they (probably) would, and 16\% were undecided. The follow-up question asked when the participants were planning to adopt MT in the work environment: 228 participants (which equals 82\% of those who were not using MT (yet)) did not plan to use it in the near future, while 60 participants (18\%) would implement MT (very) soon. Interestingly, most of the participants who used MT use free online MT systems (190 participants) and\slash or uncustomised MT systems (187 participants)\footnote{Here again, the numbers are not quite self-explanatory. I assume that some participants who used free online MT systems did not answer the question whether or not their MT system is customised, because it is obviously not, while others did answer it.}. This shows that most participants who used MT did not use it to its full potential. Concerning the 285 participants who answered the question whether or not they were \isi{post-editing}, 38\% reported that they post-edited all MT output, but 30\% also answered that they never post-edit MT output. The others post-edited MT output, but not always. These results raise the question for which purpose the participants used MT output, especially those 30\% who never post-edit. The quality of MT output was rated quite negatively: 24\% of the 289 participants who answered this question rated the quality as poor, 28\% as low, 37\% as medium, 10\% as high, and only 1\% as excellent. Most participants assessed MT output through human evaluation, while only few used automatic evaluation systems. The final MT related question was concerned with the assessment of which PE scenario might be especially interesting for the participants and their work environment. Most opted for improving medium MT output for high quality translations, then improving low MT output for high quality translations, and improving low MT output for medium quality translations was chosen the least often (cf. ibid.: 346-350). In summary, the findings of this study support “other reports of the growing prevalence of MT in the translation industry.” (ibid.: 350)


\section{The influence of pre-editing and controlled language}
\label{sec:3:2}

The effort that a translator needs to make in order to post-edit MT output is strongly related to the quality of the MT output. To improve the quality of the MT output, pre-editing techniques and controlled languages are sometimes used. This chapter will briefly introduce the key aspects of pre-editing and controlled languages and how they influence MT output.



In contrast to PE, \isi{pre-editing} is applied to the source text before the text is machine translated. The aim is to create a source text that can be translated more easily by the machine because predictable, problematic units are changed. The pre-editing rules can be applied manually by a human or automatically by a machine. The pre-editing rules used in the ACCEPT project are an illustrative example. The \isi{ACCEPT} (Automated Community Content Editing PorTal) project was created to enable better MT output for community contents (both in commercial and non-profit environments) for EU citizens. Therefore, pre-editing rules were defined for \ili{English} and \ili{French}, which the system applies (semi)automatically – either the text is analysed automatically, suggestions are made to the user and the user has to choose whether to apply the rules, or the rules are applied automatically without the user's intervention. The source texts are produced in the Acrolinx software that already provides its own rules that highlight problematic text segments for the user. “It is then up to the user to improve the text by reading the help text, choosing a replacement suggestion, manually changing the text or ignoring the marking.” (%\label{ref:ZOTEROITEMCSLCITATIONcitationIDau8O4ys9propertiesformattedCitationACCEPT2012plainCitationACCEPT2012citationItemsid232urishttpzoteroorgusers1255332itemsRKQS9CNRurihttpzoteroorgusers1255332itemsRKQS9CNRitemDataid232typereporttitleDefinitionofPreEditingRulesforEnglishandFrenchpublisherEuropeanCommunitysSeventhFrameworkProgrammeFP720072013undergrantagreementn288769URLhttpcordiseuropaeudocsprojectscnect9288769080deliverables001D21DefinitionofpreeditingrulesforEnglishandFrenchpdfauthorfamilyACCEPTgivenissueddateparts2012accesseddateparts2016114schemahttpsgithubcomcitationstylelanguageschemarawmastercslcitationjsonRNDs0P6vTKYDl}
\citealt[3]{Accept2012eu}) The suggestions should improve spelling, grammar, terminology, readability, and style. However, it turned out that different rules are necessary to improve statistical MT than for standard Acrolinx procedures. Correcting spelling mistakes is a rule that improves both the source text as well as the MT because the statistical MT system usually only recognises words that are in the training data. For the same reason, one rule was specified to exchange lexical items that are unknown to the system with synonyms. However, that does not necessarily improve the source text. Another rule was to rearrange syntactic structures so that they are closer to the \isi{target language}, but might become ungrammatical in the source language. Accordingly, changes were either suggested to the user when the suggestions would improve the source text, or applied automatically when the suggestion would impair the source text as users would probably not accept suggestions that would impair the source text (cf. ibid.: 3-5). Another form of pre-editing was made possible in the \isi{SUSY system}. The user was able to mark units in the text, e.g. when a user marked a headline as a headline, the system would know that it could not analyse the segment as a full sentence but as a phrase. Similarly, the user could mark subordinate clauses or proper names. The user had to set these marks manually, without the help of an automatic instance (%\label{ref:ZOTEROITEMCSLCITATIONcitationIDIUeF8oFMpropertiesformattedCitationWJHutchinsandSomers1992plainCitationWJHutchinsandSomers1992citationItemsid48urishttpzoteroorgusers1255332itemsAX2JS6JRurihttpzoteroorgusers1255332itemsAX2JS6JRitemDataid48typebooktitleAnintroductiontomachinetranslationpublisherAcademicPressIncpublisherplaceLondonnumberofpages320eventplaceLondonURLhttpwwwhutchinswebmeukIntroMTTOChtmauthorfamilyHutchinsgivenWJohnfamilySomersgivenHaroldissueddateparts1992schemahttpsgithubcomcitationstylelanguageschemarawmastercslcitationjsonRNDKHrS3D4Adz}
\citealt{HutchinsSomers1992}: 151-152).



The main aim of \isi{controlled languages} is to improve the readability, translatability, and reusability of texts through consistent and clear writing. Controlled languages are restricted natural languages, not artificial languages such as Esperanto. They are usually applied in very restricted areas of communication including technical documentation or other domain specific texts, e.g. for manuals, instructions, or safety notes. The first controlled language for a technical environment was Caterpillar-Fundamental \ili{English} (today: Caterpillar Technical-\ili{English}), developed by the Caterpillar Tractor Company in the late 1960s. The advantages of controlled languages are that laypersons better understand technical documentation, because readability is improved and misunderstandings and misinterpretations are reduced. Further, localisation of the texts is simplified, costs are reduced, and quality is improved. The texts are more understandable to native and non-native speakers, which is helpful when exporting the product. On the other hand, the text authors are very restrained in the writing process and it takes long for the authors to become acquainted with the new rules (cf. %\label{ref:ZOTEROITEMCSLCITATIONcitationIDaAaZuBDtpropertiesformattedCitationFerleinandHartge2008plainCitationFerleinandHartge2008citationItemsid33urishttpzoteroorgusers1255332itemsBZU3Z9XVurihttpzoteroorgusers1255332itemsBZU3Z9XVitemDataid33typebooktitleTechnischeDokumentationfrinternationaleMrktehaftungsrechtlicheGrundlagenSpracheGestaltungRedaktionundUbersetzungpublisherExpertVerlpublisherplaceRenningensourceOpenWorldCateventplaceRenningenISBN3816925804shortTitleTechnischeDokumentationfrinternationaleMrktelanguageGermanauthorfamilyFerleingivenJrgfamilyHartgegivenNicoleissueddateparts2008schemahttpsgithubcomcitationstylelanguageschemarawmastercslcitationjsonRNDk20clxMW3D}
\citealt{FerleinHartge2008}: 39-45, %\label{ref:ZOTEROITEMCSLCITATIONcitationIDEjdF2hjXpropertiesformattedCitationLehrndorfer1996plainCitationLehrndorfer1996citationItemsid34urishttpzoteroorgusers1255332items5C3QM67Eurihttpzoteroorgusers1255332items5C3QM67EitemDataid34typechaptertitleKontrollierteSprachefrTechnischeDokumentationEinAnsatzfrdasDeutschecontainertitleWissenschaftlicheGrundlagendertechnischenKommunikationpublisherGNarrpublisherplaceTbingensourceOpenWorldCateventplaceTbingenISBN3823345176languageGermanauthorfamilyLehrndorfergivenAnneeditorfamilyKringsgivenHansPissueddateparts1996schemahttpsgithubcomcitationstylelanguageschemarawmastercslcitationjsonRNDN0RnWulaUC}
\citealt{Lehrndorfer1996}).



Some rules can repeatedly be found for controlled languages, e.g., 


\begin{itemize}
\item Use a restricted vocabulary (700-5000 entries).
\item Every word has only one meaning, e.g. “fall” means “move downwards due to gravity” not “decrease”.
\item Use only one word per meaning, e.g. only use “start” not “initiate” or “begin”.
\item Use short words like “make” instead of “manufacture”.
\item Use only defined tenses.
\item Use simple sentences instead of passives or participial constructions.
\item Make sentences and paragraphs short.
\item Every sentence contains only one statement.
\item Avoid noun compounds. (cf. ibid.)
\end{itemize}

To sum up, the purpose of pre-editing is to improve MT output, while controlled languages are supposed to make the source text easier and more well-structured. Pre-editing might improve the source text and controlled languages might improve the MT output, but these are not the objectives of the respective language rules. Accordingly, controlled languages are not necessarily used in the MT context, even if they improve MT output, but to generally simplify the texts.


