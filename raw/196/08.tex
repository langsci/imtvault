\chapter{The questionnaires}
\label{sec:8}

As part of the experiment, the participants were asked to complete two \isi{questionnaires}, one before the experiment and another immediately afterwards. The results of these questionnaires are presented in this chapter. A general introduction to the methodology can be found in \sectref{sec:7:1:2}. The questionnaires focused on social-statistical characteristics of the participants, and ask for facts, behaviour, and opinions. They contained open questions, yes-/no-questions, and Likert scale questions. The latter is a psychometric scale and the scale is used predominantly in social sciences. The responses are usually at least five-ary and every item measures a different intensity of the measured feature\slash phenomenon (cf. %\label{ref:ZOTEROITEMCSLCITATIONcitationIDZUjOgam0propertiesformattedCitationrtfDuc0u246ringandBortz2014plainCitationDringandBortz2014citationItemsid220urishttpzoteroorgusers1255332itemsKJHAHZW5urihttpzoteroorgusers1255332itemsKJHAHZW5itemDataid220typebooktitleForschungsmethodenundEvaluationpublisherSpringerpublisherplaceBerlinHeidelbergedition5eventplaceBerlinHeidelbergauthorfamilyDringgivenNicolafamilyBortzgivenJrgenissueddateparts2014schemahttpsgithubcomcitationstylelanguageschemarawmastercslcitationjsonRNDpT2OwGvxWq}
\citealt[268--269]{DoringBortz2014}). In the following section, values will be calculated for most questions to better illustrate the general opinion of the participants and in which direction it points. The explanation for the single values is presented with the associated question. Some of the questions from these two questionnaires were already briefly discussed in %\label{ref:ZOTEROITEMCSLCITATIONcitationID3MGkrTRMpropertiesformattedCitationCarlGutermuthandHansenSchirra2014plainCitationCarlGutermuthandHansenSchirra2014citationItemsid165urishttpzoteroorgusers1255332items2ME62S8Turihttpzoteroorgusers1255332items2ME62S8TitemDataid165typechaptertitlePostEditingMachineTranslationaUsabilityTestforProfessionalTranslationSettingscontainertitlePsycholinguisticandcognitiveinquiriesintranslationandinterpretationstudiespublisherCambridgeScholarsPublishingpublisherplaceNewcastleuponTynepage145174eventplaceNewcastleuponTyneauthorfamilyCarlgivenMichaelfamilyGutermuthgivenSilkefamilyHansenSchirragivenSilviaeditorfamilyFerreiragivenAlinefamilySchwietergivenJohnWissueddateparts2014schemahttpsgithubcomcitationstylelanguageschemarawmastercslcitationjsonRNDft5RJNIiRp}
\citet{CarlSchwieter2014}. The questionnaires and the data recordings were part of an extensive multilingual translation process research project initiated by the Copenhagen Business School. The data are comparable for different languages and is publicly available (for more information see \sectref{sec:7:2}). Hence, the questionnaires were not adjusted to the research questions at hand, although that might have been necessary to some degree.



All analyses in this and the following chapters were conducted in either Microsoft Excel (mainly for simple calculations and diagrams) or R %\label{ref:ZOTEROITEMCSLCITATIONcitationIDUZRtf2bFpropertiesformattedCitationTeam2013plainCitationTeam2013citationItemsid4059urishttpzoteroorgusers1255332itemsW9PB6MY6urihttpzoteroorgusers1255332itemsW9PB6MY6itemDataid4059typearticlejournaltitleRAlanguageandenvironmentforstatisticalcomputingauthorfamilyTeamgivenissueddateparts2013schemahttpsgithubcomcitationstylelanguageschemarawmastercslcitationjsonRNDk8Uk9Magpt}
(\citealt{R_core_team2013}; mainly for statistical tests).


\section{The questionnaire prior to the experiment}
\label{sec:8:1}

The aim of this questionnaire was to elicit \isi{general information about the participants} (degree of experience, languages) as well as rough \isi{opinions on MT} and its relevance in translation practice.



The first point of interest was the \isi{gender of the participants}: 17 women and 7 men participated in the experiments. Taking into account that the majority of translators are female this is quite a balanced ratio (cf. %\label{ref:ZOTEROITEMCSLCITATIONcitationID4IYMfHg9propertiesformattedCitationrtfHommerichandReiuc0u2232011plainCitationHommerichandRei2011citationItemsid6urishttpzoteroorgusers1255332items536WSTZXurihttpzoteroorgusers1255332items536WSTZXitemDataid6typearticletitleErgebnissederBDMitgliederbefragungauthorfamilyHommerichgivenChristophfamilyReigivenNicoleissueddateparts20114schemahttpsgithubcomcitationstylelanguageschemarawmastercslcitationjsonRNDXdatW5wTPI}
\citet[60]{HommerichReiß2011} reported that 81\% of the participants were female in a members' survey of the BDÜ – one of the biggest associations for translators and interpreters in Germany). Additioally, the focus of this study will not take gender or any socio-cultural aspects into account.



The next question was whether the participants wore glasses or contact lenses, because glasses and contact lenses can influence the quality of the eyetracking data (%\label{ref:ZOTEROITEMCSLCITATIONcitationIDE89bSTgrpropertiesformattedCitationPooleandBall2005plainCitationPooleandBall2005citationItemsid86urishttpzoteroorgusers1255332itemsNHKAVQG3urihttpzoteroorgusers1255332itemsNHKAVQG3itemDataid86typechaptertitleEyeTrackinginHumanComputerInteractionandUsabilityResearchcontainertitleTheEncyclopediaofHumanComputerInteractionpublisherInformationSciencePubpublisherplaceHersheyPALondonpage211219eventplaceHersheyPALondonISBN1591405629authorfamilyPoolegivenAlexfamilyBallgivenLindenJeditorfamilyGhaouigivenClaudissueddateparts2005schemahttpsgithubcomcitationstylelanguageschemarawmastercslcitationjsonRND1qkbzlw8ik}
\citealt{PooleBall2005}: 216). While the use of optical aids is not important to evaluate the performance of the translator themselves, it could rather be used to explain \isi{irregularities in the gaze data}. Seven participants wore glasses, five contact lenses and twelve neither. The gaze precision for the single recordings are indicated in percentage in the Tobii Studio software. For example, if \isi{eye movement} was recorded for 12 minutes in a 15 minute long session, the gaze precision would be 80\%. Reasons for no recorded \isi{eye movement} could either be technical difficulties or situations where the participant was not looking at the screen (and instead was looking at the keyboard while typing, etc.). The mean gaze precision for the single sessions shows differences according to the optical aid: When the participants wore no optical aid 82.9\% of the \isi{eye movement} were recorded (sd:~13.9). This is a little more than what was recorded for the group of participants who wore contact lenses (mean:~75.8\%; sd:~9.4). Finally, only for 58.9\% (sd:~30.8) of the \isi{eye movement} was recorded in the sessions where the participants wore glasses. The differences between the three groups is statistically significant ($p<0.0001$)\footnote{The tests for significance were conducted with a Kruskal-Wallis test because the data were divided into three groups and were not distributed normally. The exact data were Kruskal-Wallis $\chi^2=24.4967$, $\text{df}=2$, $p=\text{\num{4.793e-6}}$  for the test between the percentage of recorded \isi{eye movement} and the type of optical aid, and  Kruskal-Wallis $\chi^2=2.4475$, $\text{df}=2$, $p=0.2941$  for the test between the percentage of recorded \isi{eye movement} and the task.}. The high \isi{standard deviation} observed for the participants who wore glasses might indicate that not all participants with glasses are difficult to record, but maybe some characteristics of the glasses complicate the recording (e.g. the size or the thickness of the lenses, or the frames of the glasses). Further, it was expected that less \isi{eye movement} is recorded for the translation from scratch task, because more text production is necessary and hence it may be plausible that the participants spent more time looking at the keyboard than for the other two tasks. However, when ordered according to task, only minor differences can be recognised (MPE – mean:~74.2, sd:~23.7; PE – mean:~78.2, sd:~19.0; and TfS – mean:~73.5, sd:~21.3), which are not statistically significant ($p=.2941$) and hence will not be discussed further.



The subsequent three questions dealt with the background of the translators: All participants were \ili{German} native speakers and considered \ili{English} their first (83\%) or second foreign language (17\%). The formal translation education of the professionals was on average 5.04 years (sd:~0.92), whereas the students had on average 3.46 years (sd:~1.66) through their studies (see \tabref{tab:key:8:1} – together 4.25 years on average – sd:~1.54).


\begin{table}
\fittable{%
\begin{tabular}{l  *{12}{S[table-format=1.1]}}
\lsptoprule
& \multicolumn{5}{c}{Professional translators} & \multicolumn{7}{c}{Student translators}\\\cmidrule(lr){2-6}\cmidrule(lr){7-13}
years of education & 3 & 4 & 4.5 & 5 & 6 &  1.5 & 2 & 2.5 & 4 & 4.5 & 5 & 6\\
no. of part. & 1 & 1 & 1 & 5 & 4 &  1 & 4 & 1 & 2 & 1 & 1 & 2\\
\lspbottomrule
\end{tabular}
}
\caption{Years of education}
\label{tab:key:8:1}
\end{table}


According to \tabref{tab:key:8:2}, the participants had 4.71 years professional translation experience on average (sd: 6.14). Self-evidently, the group of professional translators had far more translation experience (mean: 9 years, sd: 6.19 years) than the students (mean: 0.42 years, sd: 0.67 years), and the professionals had a broad age range which also explains the high standard deviations.


\begin{table}
 \fittable{%
\begin{tabular}{l  *{13}{S[table-format=2.0]}}
\lsptoprule
& \multicolumn{10}{c}{Professional translators} & \multicolumn{3}{c}{Stud. transl.}\\\cmidrule(lr){2-11}\cmidrule(lr){12-14}
years of exp. & 0 & 0.5 & 4 & 4.5 & 8 & 10 & 12 & 14 & 17 & 20   & 0 & 1 & 2\\
no. of part. & 1 & 1 & 1 & 1 & 2 & 2 & 1 & 1 & 1 & 1 &  8 & 3 & 1\\
\lspbottomrule
\end{tabular}
}
\caption{Years of experience}
\label{tab:key:8:2}
\end{table}


Most of the students are in the middle or the final stage of their studies; some of them even have professional experience. Therefore, they are not beginners and are rather classified as semi-professionals. The high \isi{standard deviation} in the experience values – for the all participants-group as well as for the sub-groups – are the reason for the introduction of a new figure which will better represent the experience of the participants and will unify years at university and professional experience. This value will be called \isi{\textit{experience coefficient}} (see also %\label{ref:ZOTEROITEMCSLCITATIONcitationIDp4Zns2SspropertiesformattedCitationNitzke2016monobplainCitationNitzke2016monobcitationItemsid183urishttpzoteroorgusers1255332itemsUPJZPQ8Rurihttpzoteroorgusers1255332itemsUPJZPQ8RitemDataid183typechaptertitleMonolingualposteditingAnexploratorystudyonresearchbehaviourandtargettextqualitycontainertitleEyetrackingandAppliedLinguisticspublisherLanguageSciencePresspublisherplaceBerlinpage83108eventplaceBerlinauthorfamilyNitzkegivenJeaneditorfamilyHansenSchirragivenSilviafamilyGruzcagivenSamborissueddateparts2016schemahttpsgithubcomcitationstylelanguageschemarawmastercslcitationjsonRNDJbmxz9RzaA}
\citealt{Nitzke2016mono}). The simple equation that is used to calculate the experience coefficient is the following:


\ea
Experience = years of education $*$ 1 + years of professional experience $*$ 2
\z


This new experience coefficient has a number of advantages: First, we have one figure combining experience at university and work experience instead of two. In two cases, the professional translators have less experience than a few students, because they only recently received their degree and did not gather any professional experience during their time at university. Further, the professional experience within both groups is very inhomogeneous (see \tabref{tab:key:8:3}). Although differentiating between students and professionals is appropriate in many scenarios, students who have studied for two years at university and students who have studied for five years and have had one year of professional experience outside university should not necessarily be clustered together. The same applies to professionals who have had two years of professional experience and professionals who have had more than ten years of experience. Finally, a single numeric figure for experience will simplify calculation of statistics, like correlations, between phenomena and experience and is more useful than calculating statistics only according to the two statuses \textit{semi-professionals} and \textit{professionals} (e.g. %\label{ref:ZOTEROITEMCSLCITATIONcitationIDo1CTBOn2propertiesformattedCitationrtfMartuc0u237nezGuc0u243mezetal2014plainCitationMartnezGmezetal2014citationItemsid4052urishttpzoteroorgusers1255332itemsAL4G7W2Curihttpzoteroorgusers1255332itemsAL4G7W2CitemDataid4052typepaperconferencetitleRecognitionoftranslatorexpertiseusingsequencesoffixationsandkeystrokescontainertitleProceedingsoftheSymposiumonEyeTrackingResearchandApplicationspublisherACMpage299302authorfamilyMartnezGmezgivenPascualfamilyMinochagivenAkshayfamilyHuanggivenJinfamilyCarlgivenMichaelfamilyBangaloregivenSrinivasfamilyAizawagivenAkikoissueddateparts2014schemahttpsgithubcomcitationstylelanguageschemarawmastercslcitationjsonRNDFrknKNYeiE}
\citealt{Martinez-GomezEtAl2014} or discussion in %\label{ref:ZOTEROITEMCSLCITATIONcitationIDvmxui6ZXpropertiesformattedCitationSinglaetal2013plainCitationSinglaetal2013citationItemsid4053urishttpzoteroorgusers1255332itemsW7RVNSJPurihttpzoteroorgusers1255332itemsW7RVNSJPitemDataid4053typepaperconferencetitlePredictingposteditorprofilesfromthetranslationprocesscontainertitleProceedingsoftheWorkshoponInteractiveandAdaptiveMachineTranslationAMTAWorkshoppage5160authorfamilySinglagivenKaranfamilyOrregoCarmonagivenDavidfamilyGonzalesgivenAshleighRheafamilyCarlgivenMichaelfamilyBangaloregivenSrinivasissueddateparts2013schemahttpsgithubcomcitationstylelanguageschemarawmastercslcitationjsonRNDJ6KQs4rpuY}
\citealt{SinglaEtAl2013}). \tabref{tab:key:8:3} illustrates which participant has which status and how much experience according to the experience coefficient at the point when the experiments were conducted.


\begin{table}
\begin{tabular}{lcccccccccccc}
\lsptoprule
Participant & 1 & 2 & 3 & 4 & 5 & 6 & 7 & 8 & 9 & 10 & 11 & 12\\
Status & P & S & S & S & S & P & P & S & S & S & S & S\\
Experience & 14 & 6.5 & 3 & 4 & 8 & 45 & 20.5 & 5 & 2.5 & 6 & 6 & 5.5\\
\midrule
Participant & 13 & 14 & 15 & 16 & 17 & 18 & 19 & 20 & 21 & 22 & 23 & 24\\
Status & P & S & P & S & P & S & P & P & P & P & P & P\\
Experience & 6 & 2 & 30 & 2 & 19 & 2 & 40 & 4 & 34 & 13 & 25 & 26\\
\lspbottomrule
\end{tabular}
\caption{Experience coefficient}
\label{tab:key:8:3}
\end{table}


It is assumed that professionals translate a lot more in their everyday life than students at the university, where they “only” attend a few translation courses per semester (if at all) and also concentrate their studies on other topics like translation theory, linguistics, cultural studies, etc. which might indirectly improve their translation behaviour, set the basis for responsible translation choices, and develop the problem solving ability. These activities, however, are not as substantial as full-time translating. Hence, professional experience was weighted stronger than years of study. Further, a learning curve was not implemented in the calculations, because there are no studies known to the author that describe how a learning curve for professionalism in translation would look, and, additionally, other studies used simple years of experience as well for calculating correlations between translation behaviour and experience (e.g.: %\label{ref:ZOTEROITEMCSLCITATIONcitationIDEWzAtqnspropertiesformattedCitationDeAlmeida2013plainCitationDeAlmeida2013citationItemsid221urishttpzoteroorgusers1255332itemsKKNP5KHKurihttpzoteroorgusers1255332itemsKKNP5KHKitemDataid221typethesistitleTranslatingtheposteditoraninvestigationofposteditingchangesandcorrelationswithprofessionalexperienceacrosstwoRomancelanguagespublisherDublinCityUniversityURLhttpdorasdcuie177321THESISGdeAlmeidapdfauthorfamilyDeAlmeidagivenGiselleissueddateparts2013schemahttpsgithubcomcitationstylelanguageschemarawmastercslcitationjsonRNDJ1xaGQVKv8}
\citealt{De_almeida2013}).



%\label{ref:ZOTEROITEMCSLCITATIONcitationID2KVfU7typropertiesformattedCitationrtfMoorkensandOuc0u8217Brien2015plainCitationMoorkensandOBrien2015citationItemsid92urishttpzoteroorgusers1255332items9622WV9Qurihttpzoteroorgusers1255332items9622WV9QitemDataid92typepaperconferencetitlePosteditingevaluationsTradeoffsbetweennoviceandprofessionalparticipantscontainertitleProceedingsofEuropeanAssociationforMachineTranslationEAMTpage7581authorfamilyMoorkensgivenJossfamilyOBriengivenSharonissueddateparts2015schemahttpsgithubcomcitationstylelanguageschemarawmastercslcitationjsonRNDsSv0ZwEgx3}
\citet{MoorkensOBrien2015} examine in their study with nine expert translators (on average 11.3 years translation experience and 4 years PE experience) and 35 undergraduate student translators the productivity\slash speed of the participants, the edit distance of their final product, and their attitude towards PE. The professional group was much faster than the students, while the students tended to edit less of the MT output and they had a more positive attitude towards PE – although almost half of the students had negative feelings about PE, too. Conclusively, this study shows that it might be reasonable to distinguish between professionals and novices because they displayed a different working behaviour and different attitudes towards PE. However, the study does not specify how heterogeneous or homogeneous the participants were. In the analysis chapters of the study at hand, the differentiation between the status of the participants will not be disregarded because visualising data in two groups is much more expressive than according to experience. Further, the status of a participant naturally gives a first impression of the experience. In general, students are less experienced than professionals. When it is tested whether the experience coefficient is significantly different between students and professionals, we get a highly significant result ($W=134.5, p<0.0004$\footnote{The test was conducted with a Mann-Whitney-U-test, because the experience coefficient is not distributed normally.}).



The last experience-related question of the pre-experiment questionnaire asked whether the participants had any experience with PE. In the professional group, half of the participants had post-edited prior to the experiments, while only 17\% of the students had experience with PE. Unfortunately, this question does not provide any insights on how often the participants have post-edited before, whether they do it regularly and whether they gained the PE experience recently or years ago. In future research, this question needs to be more detailed to achieve a better understanding of the PE experience.



The experience of the translators is quite important for the analysis, because it influences the outcome of the translation and, from a problem solving perspective, it is assumed that “[t]he difference in performance between experts and less skilled individuals is not a simple difference in accumulated knowledge about past experience”, but also “appear[s] to reflect differential ability to react to representative tasks and situations that have never been previously encountered” (%\label{ref:ZOTEROITEMCSLCITATIONcitationIDDSo6cRxkpropertiesformattedCitationEricsson2003plainCitationEricsson2003citationItemsid134urishttpzoteroorgusers1255332itemsTWGZGJG8urihttpzoteroorgusers1255332itemsTWGZGJG8itemDataid134typechaptertitleTheacquisitionofexpertperformanceasproblemsolvingcontainertitleThepsychologyofproblemsolvingpublisherCambridgeUniversityPresspublisherplaceCambridgeaopage3183eventplaceCambridgeaoauthorfamilyEricssongivenAeditorfamilyDavidsongivenJanetEfamilySternberggivenRobertJissueddateparts2003schemahttpsgithubcomcitationstylelanguageschemarawmastercslcitationjsonRND4ZTSoFwqlZ}
\citealt[57]{EricssonSternberg2003}). Hence, one could expect that the more experienced translators with no PE experience are better post-editors than semi-professional translators with no PE experience. On the other hand, both tasks are quite different and it also seems plausible that more experienced translators stick to already established translation behaviours and strategies that might not be suitable for PE. Further, experienced translators might even be more reluctant to use MT output (see e.g. %\label{ref:ZOTEROITEMCSLCITATIONcitationIDUuZc24NLpropertiesformattedCitationSilva2014plainCitationSilva2014citationItemsid76urishttpzoteroorgusers1255332itemsAMWPRNSMurihttpzoteroorgusers1255332itemsAMWPRNSMitemDataid76typechaptertitleIntegratingPostEditingMTinaProfessionalTranslationWorkflowcontainertitlePosteditingofMachineTranslationpublisherCambridgeScholarsPublishingpublisherplaceCambridgepage2451eventplaceCambridgeauthorfamilySilvagivenRobertoeditorfamilyOBriengivenSharonfamilyWintherBallinggivenLaurafamilyCarlgivenMichaelfamilySimardgivenMichelfamilySpeciagivenLuciaissueddateparts2014schemahttpsgithubcomcitationstylelanguageschemarawmastercslcitationjsonRNDB7SltZQSUM}
\citealt{Silva2014}). The PE guidelines specified, for example, that style did not need to be addressed. However, it might be hard for an experienced translator who had to apply high quality standards to his\slash her translations for years and decades to discard this aspiration, which might hinder the PE task. Therefore, I assume that translation experience plays a vital role in \isi{problem solving behaviour} (see \sectref{sec:6}).



After collecting general information about the translation background of the participants, the next part of the pre-experiment questionnaire dealt with experience with and opinions on MT. The questions will be stated at the beginning of the paragraph in italics in the following. 



\textit{Q1: How frequently do you use \isi{machine translation}?} -- The participants could choose between different answers – \textit{every day, every 2 to 3 weeks, every month, once or twice a year, never} (see \figref{fig:key:8:1}). These answers are ranked with different values so that the total value for all participants reflects the answers. Therefore, \textit{never} is given the value \textit{zero}, adding one point to every answer that indicates an increased use of MT, ending with the value \textit{four} for the answer \textit{every day}. The resulting values show that the participants use MT quite rarely (mean:~0.71, sd:~1.08); the professionals using it even less (mean:~0.58, sd:~0.79) than the students (mean:~0.83, sd:~1.34). The distribution of the answers is shown in \figref{fig:key:8:1}. According to the numbers, the group uses MT systems less than once or twice a year. This self-assessment of the use of MT systems remains questionable. As explained in \sectref{sec:2:1}, MT is quite common in the world wide web nowadays. Therefore, it might be possible that participants are faced with MT much more often than they care to admit or even realise, because they might not be aware how often they are confronted with MT output when they visit websites. Or they consider the question in a solely professional way and do not consider encounters with MT in their everyday Internet use.


\begin{figure}
\caption{How frequently do you use machine translation?}
\label{fig:key:8:1}
% % \includegraphics[width=\textwidth]{figures/DissertationNitzkeberarbeitet-img5.jpg}
\begin{tikzpicture}[trim axis right,trim axis left]
        \begin{axis}[
                    ybar,
                    xtick=data,
                    axis lines*=left,
                    ymin=0,
                    scaled y ticks = false,
                    colormap/Greys-5,
                    cycle list/Greys-5,
                    legend pos=outer north east,
                    reverse legend,
                    xticklabels={Every day,Every 2–3 weeks,Every month,Once or twice a year,Never},
                    x tick label style={rotate=20,anchor=east,font=\footnotesize},
                    ticklabel style={font=\footnotesize},
                    legend style={font=\footnotesize}
                    ]
                \addplot+[
                     fill=Greys-E,draw=none
                    ] coordinates {(0,0) (1,0) (2,2) (3,3) (4,7)};
                \addlegendentryexpanded{Professional}
                \addplot+[
                     fill=Greys-I,draw=none
                    ] coordinates {(0,1) (1,1) (2,0) (3,3) (4,7)};
                \addlegendentryexpanded{Student}
            \end{axis}    
\end{tikzpicture}
\end{figure} 


\textit{Q2: From your previous experience with \isi{machine translation} outputs, how would you rate your level of satisfaction in relation to \isi{machine translation}?} {}- The range for this questions is \textit{very satisfied}, \textit{somewhat satisfied}, \textit{neutral}, \textit{somewhat dissatisfied}, \textit{highly dissatisfied}. The distributed values have the purpose of reflecting whether the attitude is rather negative (the value will be negative) or rather positive (the value will be positive). Therefore, the answer \textit{neutral} gets the value \textit{0} representing a neutral opinion, \textit{somewhat satisfied} and \textit{very satisfied} the values \textit{1} and \textit{2}, respectively, representing a positive opinion, and \textit{somewhat dissatisfied} and \textit{very dissatisfied} the values \textminus1 and \textminus\textit{2.}


\begin{figure}
\caption{From your previous experience with machine translation outputs, how would you rate your level of satisfaction in relation to machine translation?}
\label{fig:key:8:2}
% % \includegraphics[width=\textwidth]{figures/DissertationNitzkeberarbeitet-img6.jpg}
\begin{tikzpicture}[trim axis right,trim axis left]
        \begin{axis}[
                    ybar,
                    xtick=data,
                    axis lines*=left,
                    ymin=0,
                    scaled y ticks = false,
                    colormap/Greys-5,
                    cycle list/Greys-5,
                    legend pos=outer north east,
                    reverse legend,
                    xticklabels={Highly satisfied,Somewhat satisfied,neutral,Somewhat dissatisfied,Highly dissatisfied,no answer},
                    x tick label style={rotate=20,anchor=east,font=\footnotesize},
                    ticklabel style={font=\footnotesize},
                    legend style={font=\footnotesize}
                    ]
                \addplot+[
                     fill=Greys-E,draw=none
                    ] coordinates {(0,0) (1,2) (2,0) (3,0) (4,6) (5,4)};
                \addlegendentryexpanded{Professional}
                \addplot+[
                     fill=Greys-I,draw=none
                    ] coordinates {(0,0) (1,3) (2,3) (3,1) (4,2) (5,3)};
                \addlegendentryexpanded{Student}
            \end{axis}    
\end{tikzpicture}
\end{figure}

 


Some respondents chose not to answer to this question (see \figref{fig:key:8:2}). A simple explanation for this is that over half of the participants claimed to have never used MT. However, the number of respondents who have never used MT and the number of respondents who did not provide an answer to the question are not equal. More participants answered the question than claimed to have never used MT which shows that they nonetheless have a certain amount of experience and\slash or opinions on MT – a sign that they might have only considered the professional environment or that they have prejudices towards MT output. When non-responses occurred, they were not considered in the calculated value, i.e. the total number of answers decreased. Altogether, the general opinion on MT is very low (mean:~\textminus0.71; sd:~1.36), which is in line with %\label{ref:ZOTEROITEMCSLCITATIONcitationID0wrH4xIOpropertiesformattedCitationGaspariAlmaghoutandDoherty2015plainCitationGaspariAlmaghoutandDoherty2015dontUpdatetruecitationItemsid208urishttpzoteroorgusers1255332itemsTKS8XNJBurihttpzoteroorgusers1255332itemsTKS8XNJBitemDataid208typearticlejournaltitleAsurveyofmachinetranslationcompetencesInsightsfortranslationtechnologyeducatorsandpractitionerscontainertitlePerspectivespage333358volume23issue3authorfamilyGasparigivenFedericofamilyAlmaghoutgivenHalafamilyDohertygivenStephenissueddateparts2015schemahttpsgithubcomcitationstylelanguageschemarawmastercslcitationjsonRNDuI5wYqbil1}
\citet[348--350]{GaspariEtAl2015}, who found in their study, which tries to draw a picture of the translation market, that 52\% of the participants who have used MT judge the quality as low or very low, 37\% said the quality was medium, and only 11\% perceived the quality as high or excellent. However, this is not in line with the findings by %\label{ref:ZOTEROITEMCSLCITATIONcitationIDefrL22DMpropertiesformattedCitationDeAlmeida2013plainCitationDeAlmeida2013citationItemsid221urishttpzoteroorgusers1255332itemsKKNP5KHKurihttpzoteroorgusers1255332itemsKKNP5KHKitemDataid221typethesistitleTranslatingtheposteditoraninvestigationofposteditingchangesandcorrelationswithprofessionalexperienceacrosstwoRomancelanguagespublisherDublinCityUniversityURLhttpdorasdcuie177321THESISGdeAlmeidapdfauthorfamilyDeAlmeidagivenGiselleissueddateparts2013schemahttpsgithubcomcitationstylelanguageschemarawmastercslcitationjsonRNDBe3VpQCHWf}
\citet[157]{De_almeida2013} – most of her participants expressed a neutral or positive opinion of MT in the questionnaire which was filled out before her experiment. The value for the professional group is \textminus1.25 (sd:~1.39). The value for the student group is not as negative but still in the negative area (mean:~\textminus0.16, sd:~1.2). It is rather hard to attach meaning to these opinions. From my subjective experience, translators tend to evaluate MT output from what they would expect from human translations. MT systems, however, cannot deliver fully automatic high quality translation (yet), except in very restricted domains %\label{ref:ZOTEROITEMCSLCITATIONcitationIDxxpT2SVOpropertiesformattedCitationHutchinsandSomers1992plainCitationHutchinsandSomers1992dontUpdatetruecitationItemsid48urishttpzoteroorgusers1255332itemsAX2JS6JRurihttpzoteroorgusers1255332itemsAX2JS6JRitemDataid48typebooktitleAnintroductiontomachinetranslationpublisherAcademicPressIncpublisherplaceLondonnumberofpages320eventplaceLondonURLhttpwwwhutchinswebmeukIntroMTTOChtmauthorfamilyHutchinsgivenWJohnfamilySomersgivenHaroldissueddateparts1992schemahttpsgithubcomcitationstylelanguageschemarawmastercslcitationjsonRNDHDbOpFRIId}
(cf. \citealt[147--149]{HutchinsSomers1992}). Especially, unspecialised systems that are trained on all kinds of data, which most systems available online are, cannot meet these requirements. Further, the mistakes a MT system makes are more obvious and hence easier to deplore than it is to value what the systems are capable of. This negative attitude might even multiply, when the source language is known to the MT user and (s)he can directly compare the source text to the MT output.



\textit{Q3: Do you think that you will want to apply \isi{machine translation} in your future translation tasks?} - The simple yes-no question was extended with \textit{not sure} (see results in \figref{fig:key:8:3}).


\begin{figure}
\caption{Do you think that you will want to apply machine translation in your future translation tasks?}
\label{fig:key:8:3}
% % \includegraphics[width=\textwidth]{figures/DissertationNitzkeberarbeitet-img7.jpg}
\begin{tikzpicture}[trim axis right,trim axis left]
        \begin{axis}[
                    ybar,
                    xtick=data,
                    axis lines*=left,
                    ymin=0,
                    scaled y ticks = false,
                    colormap/Greys-5,
                    cycle list/Greys-5,
                    legend pos=outer north east,
                    reverse legend,
                    xticklabels={yes,no,not sure},
                    x tick label style={font=\footnotesize},
                    ticklabel style={font=\footnotesize},
                    legend style={font=\footnotesize}
                    ]
                \addplot+[
                     fill=Greys-E,draw=none
                    ] coordinates {(0,1) (1,7) (2,4)};
                \addlegendentryexpanded{Professional}
                \addplot+[
                     fill=Greys-I,draw=none
                    ] coordinates {(0,3) (1,5) (2,4)};
                \addlegendentryexpanded{Student}
            \end{axis}    
\end{tikzpicture}
\end{figure}

 


Generally, the participants can hardly see themselves using MT systems in their future translation jobs (17\% could imagine using MT, 50\% could not, and 33\% were not sure). The professionals are not very convinced that they could use MT for their work (58\% of the professional participants decline the suggestion and only one answered with yes), while the students are more open to integrating MT into their future translation tasks (25\% of the student participants can imagine using MT in the future; 42\% cannot). As \figref{fig:key:8:2} showed, the participants are not very convinced of MT output in general. Therefore, the tendencies shown in \figref{fig:key:8:3} are in line with \figref{fig:key:8:2}.



\textit{Q4: In general, how feasible do you think it is to apply \isi{machine translation} to professional translation services?} -- Similar to the question about the satisfaction in relation to MT, the answers were the following: \textit{very likely}, \textit{somewhat likely}, \textit{neutral}, \textit{somewhat unlikely}, \textit{very unlikely}. Accordingly, the distribution of the values was similar: \textit{0} points were given for \textit{neutral}, \textit{2} points for \textit{very likely}, \textit{\textminus2} points for \textit{very unlikely}. The values are quite close to zero in general (mean:~\textminus0.08, sd:~1.10) and for both separate groups, but for the professionals the value is barely positive (mean:~0.08, sd:~1.08), while it is negative for the student group (mean:~\textminus0.25, sd:~1.14). The distributions of the answers can be seen in \figref{fig:key:8:4}.


\begin{figure}
\caption{In general, how feasible do you think it is to apply machine translation to professional translation services?}
\label{fig:key:8:4}
% % \includegraphics[width=\textwidth]{figures/DissertationNitzkeberarbeitet-img8.jpg}
\begin{tikzpicture}[trim axis right,trim axis left]
        \begin{axis}[
                    ybar,
                    xtick=data,
                    axis lines*=left,
                    ymin=0,
                    scaled y ticks = false,
                    colormap/Greys-5,
                    cycle list/Greys-5,
                    legend pos=outer north east,
                    reverse legend,
                    xticklabels={Very likely,Somewhat likely,neutral,Somewhat unlikely,very unlikely},
                    x tick label style={rotate=20,anchor=east,font=\footnotesize},
                    ticklabel style={font=\footnotesize},
                    legend style={font=\footnotesize}
                    ]
                \addplot+[
                     fill=Greys-E,draw=none
                    ] coordinates {(0,1) (1,3) (2,5) (3,2) (4,1)};
                \addlegendentryexpanded{Professional}
                \addplot+[
                     fill=Greys-I,draw=none
                    ] coordinates {(0,1) (1,2) (2,3) (3,5) (4,1)};
                \addlegendentryexpanded{Student}
            \end{axis}    
\end{tikzpicture}
\end{figure}

 


Interestingly, the participants judge this question rather neutrally, which is contradictory to the answers from the last two questions. Professionals consider MT (slightly) feasible for translation services, although they hardly use MT, their general opinion on MT is very low, and they doubt that they will use MT in their own future translation tasks. This could have various reasons. One could be that they might have specialised in areas that are not targeted by MT (literature, advertising, marketing etc.), and can see the potential usability in other domains but not in their own. Further, they might include in their considerations future developments in MT, meaning that they can imagine that MT output might be useful at some point but not now. Another aspect is that professionals are less negative than students, although it was the other way round in the other questions. A reason could be that professional translators are more sensitive about judging the application of tools due to their more profound experience. Compared to the group of professionals, the student group is rather open to MT. They use it more often and have a better attitude towards MT. However, they are more negative about the possibility that MT should find its way into professional translation tasks. Again, we can only try to interpret the meaning. Maybe students are more often confronted with MT, but cannot see how it could be used in a professional environment (which most of them are not familiar with). There may be various other approaches to interpret these findings, but the number of participants is not very representative, so not too much importance should be attached to the different directions indicated by these questions.



Further, questions Q1, Q2, and Q4 were tested statistically for a difference between students and professionals and none of the tests proved significance (t-test or Mann-Whitney-U-test, depending on the distribution of the data): Q1 – $W=70$, $p=0.9222$; Q2 – $W=21$, $p=0.1342$; Q4 – $t=0.7348$, $p=0.4703$.



Unfortunately, the metadata collected in this questionnaires did not cover the fields in which the translators specialise or have specialised and whether they deal with general language texts in their professional work at all. The semi-professional participants might even be more familiar with this text domain, because general language translation is part of the curriculum of the B.A. programme (and can be chosen in the M.A. programme). The translation process might be influenced by the translators' specialisation. As %\label{ref:ZOTEROITEMCSLCITATIONcitationIDM8e76KlKpropertiesformattedCitationLubartandMouchiroud2003plainCitationLubartandMouchiroud2003citationItemsid96urishttpzoteroorgusers1255332items3FHEVFIGurihttpzoteroorgusers1255332items3FHEVFIGitemDataid96typechaptertitleCreativityAsourceofdifficultyinproblemsolvingcontainertitleThePsychologyofProblemSolvingpublisherCambridgeUniversityPresspublisherplaceCambridgepage127148eventplaceCambridgeauthorfamilyLubartgivenToddIfamilyMouchiroudgivenChristopheeditorfamilyDavidsongivenJanetEfamilySternberggivenRobertJissueddateparts2003schemahttpsgithubcomcitationstylelanguageschemarawmastercslcitationjsonRNDP9D4MTCYdh}
\citet[130]{LubartMouchiroud2003} summarise, “high levels of domain knowledge can sometimes bias problem solving, limiting the search space to readily available ideas”. They further argue that many problems can be solved with this “canned knowledge”. Nonetheless, creative problems often require the problem solver to break away from what is already known. Different characteristics of individuals (character traits, domain knowledge, motivation etc.) create the individuals' creative problem solving ability, which explains why some people are able to work creatively in one domain, but are not creative in others – creativity is not only necessary for the arts and literature, but can also be found in engineering, physics, mathematics etc. (ibid.: 130-136)


\section{The retrospective questionnaire}
\label{sec:8:2}

\isi{The retrospective questionnaire} was filled out directly after the experiments and dealt mainly with a \isi{self-assessment of the performance} during the experiment and the \isi{evaluation of the machine output}. The proximity of time has the advantage that the participants are mentally still very involved in the TfS\slash PE\slash MPE tasks. Moreover, the participants did not have to come to the lab a second time for further questions. However, the participants probably were exhausted after this quite extensive study. Hence, it is possible that they did not give much thought to the questions (for more details on advantages and disadvantages of retrospective surveys, especially in connection with think aloud methods see %\label{ref:ZOTEROITEMCSLCITATIONcitationIDSJXniG22propertiesformattedCitationrtfGuc0u246pferich2008plainCitationGpferich2008citationItemsid1487urishttpzoteroorggroups3587items8SGPKEZ2urihttpzoteroorggroups3587items8SGPKEZ2itemDataid1487typebooktitleTranslationsprozessforschungStandMethodenPerspektivencollectiontitleTranslationswissenschaftpublisherNarrpublisherplaceTbingenvolume4eventplaceTbingenISBN9783823364399languagegerauthorfamilyGpferichgivenSusanneissueddateparts2008schemahttpsgithubcomcitationstylelanguageschemarawmastercslcitationjsonRNDTp86E7kl1M}
\citealt{Gopferich2008}). Unfortunately, one of the candidates – a professional translator - was not able to conduct the whole experiment and therefore did not fill out the second questionnaire. All in all, 23 questionnaires are available for assessment, eleven from professional translators, twelve from semi-professional translators. We will mostly continue to work with the original division of the participants in semi-professionals and professionals for illustration reasons at this point of the study.


  
The first question dealt with the \isi{post-editing} task: \textit{Q5: How satisfied are you with the translation you have produced through \isi{post-editing}?} The participants could choose from five answers – \textit{highly satisfied}, \textit{somewhat satisfied}, \textit{neutral}, \textit{somewhat dissatisfied} and \textit{highly dissatisfied}. The self-assessment was rather positive (mean:~0.43, sd:~1.08), especially for the student candidates. The values were calculated in the following manner: \textit{highly satisfied} gained \textit{2} points, \textit{neutral 0} points and \textit{highly dissatisfied \textminus2} points. The professionals reached a general value of 0.18 (sd:~1.25), which can be interpreted as neutral – keeping in mind that over half of them were somewhat or highly satisfied. The students were even more satisfied and assigned their output a general value of 0.67 (sd:~0.89). 75\% of the student participants were somewhat or highly satisfied, which is a lot.


\begin{figure}
\caption{How satisfied are you with the translation you have produced through post-editing?}
\label{fig:key:8:5}
% % \includegraphics[width=\textwidth]{figures/DissertationNitzkeberarbeitet-img9.jpg}
\begin{tikzpicture}[trim axis right,trim axis left]
        \begin{axis}[
                    ybar,
                    xtick=data,
                    axis lines*=left,
                    ymin=0,
                    scaled y ticks = false,
                    colormap/Greys-5,
                    cycle list/Greys-5,
                    legend pos=outer north east,
                    reverse legend,
                    xticklabels={Highly satisfied,Somewhat satisfied,Neutral,Somewhat dissatisfied,Highly dissatisfied},
                    x tick label style={rotate=20,anchor=east,font=\footnotesize},
                    ticklabel style={font=\footnotesize},
                    legend style={font=\footnotesize}
                    ]
                \addplot+[
                     fill=Greys-E,draw=none
                    ] coordinates {(0,1) (1,5) (2,1) (3,3) (4,1)};
                \addlegendentryexpanded{Professional}
                \addplot+[
                     fill=Greys-I,draw=none
                    ] coordinates {(0,1) (1,8) (2,1) (3,2) (4,0)};
                \addlegendentryexpanded{Student}
            \end{axis}    
\end{tikzpicture}
\end{figure}

 


The next question was concerned with the personal assessment of the monolingual \isi{post-editing} task: \textit{Q6: How satisfied are you with the translation you have produced through editing?}\footnote{Unfortunately, the questionnaire did not ask for the satisfaction with the human translation task, which would have been interesting to compare to the two PE tasks.} The answer range was the same as in the question above due to the similar nature of the question. Accordingly, the values were calculated as in the prior question. The evaluations were more negative than for the PE task (mean:~\textminus0.26 and sd:~1.29). The professionals were quite critical of their work and were not satisfied (mean:~\textminus0.68, sd:~1.43). The students were in general neither satisfied nor dissatisfied with their work (mean:~0.08, sd:~1.08), although half of them were somewhat satisfied. The difference between both groups is not significant for PE ($W=52$, $p=0.3557$) and MPE ($W=44.5$, $p=0.18$).


\begin{figure}
\caption{How satisfied are you with the translation you have produced through (monolingual post-)editing?}
\label{fig:key:8:6}
% % \includegraphics[width=\textwidth]{figures/DissertationNitzkeberarbeitet-img10.jpg}
\begin{tikzpicture}[trim axis right,trim axis left]
        \begin{axis}[
                    ybar,
                    xtick=data,
                    axis lines*=left,
                    ymin=0,
                    scaled y ticks = false,
                    colormap/Greys-5,
                    cycle list/Greys-5,
                    legend pos=outer north east,
                    reverse legend,
                    xticklabels={Highly satisfied,Somewhat satisfied,Neutral,Somewhat dissatisfied,Highly dissatisfied},
                    x tick label style={rotate=20,anchor=east,font=\footnotesize},
                    ticklabel style={font=\footnotesize},
                    legend style={font=\footnotesize}
                    ]
                \addplot+[
                     fill=Greys-E,draw=none
                    ] coordinates {(0,1) (1,2) (2,1) (3,3) (4,4)};
                \addlegendentryexpanded{Professional}
                \addplot+[
                     fill=Greys-I,draw=none
                    ] coordinates {(0,1) (1,6) (2,2) (3,3) (4,1)};
                \addlegendentryexpanded{Student}
            \end{axis}    
\end{tikzpicture}
\end{figure}

 


The next step was to find out whether the translators consider the MT output helpful or obstructive. \textit{Q7: Would you have preferred to work on your translation from scratch instead of \isi{post-editing} \isi{machine translation}?} This was a simple yes\slash no question. The responses were very clear: 91\% of the professional translators would rather have translated from scratch than post-edited the MT. The general opinion among the semi-professional translators is similar, but not as striking (75\% of them would rather have translated from scratch). For the group that means 83\% of the participants considered the MT output obstructive.



Similar to one question in the first questionnaire, the translators were asked whether they would integrate MT in their future translation tasks: \textit{Q8: Do you think that you will want to apply \isi{machine translation} in your future translation tasks?} As in the question in the prior questionnaire the simple yes-no question was extended by not sure (see \figref{fig:key:8:7}).


\begin{figure}
\caption{Do you think that you will want to apply machine translation in your future translation tasks?}
\label{fig:key:8:7}
% % \includegraphics[width=\textwidth]{figures/DissertationNitzkeberarbeitet-img11.jpg}
\begin{tikzpicture}[trim axis right,trim axis left]
        \begin{axis}[
                    ybar,
                    xtick=data,
                    axis lines*=left,
                    ymin=0,
                    scaled y ticks = false,
                    colormap/Greys-5,
                    cycle list/Greys-5,
                    legend pos=outer north east,
                    reverse legend,
                    xticklabels={{Yes, at some point},I’m not sure yet,{No, never!}},
                    x tick label style={rotate=20,anchor=east,font=\footnotesize},
                    ticklabel style={font=\footnotesize},
                    legend style={font=\footnotesize}
                    ]
                \addplot+[
                     fill=Greys-E,draw=none
                    ] coordinates {(0,2) (1,5) (2,4)};
                \addlegendentryexpanded{Professional}
                \addplot+[
                     fill=Greys-I,draw=none
                    ] coordinates {(0,4) (1,3) (2,5)};
                \addlegendentryexpanded{Student}
            \end{axis}    
\end{tikzpicture}
\end{figure}

 


Although it is the same question, which had already been asked in the first questionnaire, the results are different (26\% answered \textit{yes, at some point}, 39\% chose \textit{no, never!}, 35\% were not sure – professionals: 18\% yes, 36\% no, 45\% not sure; students: 33\% yes, 42\% no, 25\% not sure). Even more surprising is that the translators were more open to the idea of using MT in future translation tasks, although the participants would rather have translated the post-edited texts from scratch than post-edited the MT output and the general evaluation of the MT output was very bad (see next question). A contrast and a discussion of this question and the different results before and after the experiment will follow in \sectref{sec:8:3}.



The next question was concerned with the assessment of the MT output. The focus of this question will not be on the differences between the students and professionals, although they will be pointed out, but rather on the average evaluation, which resulted in a very negative judgement (see \figref{fig:key:8:8}). \textit{Q8: Based on the \isi{post-editing} task you have performed, how much do you rate \isi{machine translation} outputs on the following attributes - grammaticality, style, overall accuracy and overall quality?}


\begin{figure}
\caption{Based on the post-editing task you have performed, how much do you rate machine translation outputs on the following attributes - grammaticality, style, overall accuracy and overall quality?}
\label{fig:key:8:8}
% % \includegraphics[width=\textwidth]{figures/DissertationNitzkeberarbeitet-img12.jpg}
\begin{tikzpicture}[trim axis right,trim axis left]
        \begin{axis}[
                    ybar,
                    xtick=data,
                    axis lines*=left,
                    ymin=0,
                    scaled y ticks = false,
                    colormap/Greys-5,
                    cycle list/Greys-5,
                    legend pos=north east,
                    width=\textwidth,
                    xticklabels={\textminus2,\textminus1,0,1,2},
                    x tick label style={font=\footnotesize},
                    ticklabel style={font=\footnotesize},
                    legend style={font=\footnotesize}
                    ]
                \addplot+[
                     fill=Greys-E,draw=none
                    ] coordinates {(0,12) (1,6) (2,4) (3,1) (4,0)};
                \addlegendentryexpanded{Grammaticality}
                \addplot+[
                     fill=Greys-G,draw=none
                    ] coordinates {(0,15) (1,4) (2,3) (3,1) (4,0)};
                \addlegendentryexpanded{Style}
                \addplot+[
                     fill=Greys-I,draw=none
                    ] coordinates {(0,7) (1,10) (2,5) (3,1) (4,0)};
                \addlegendentryexpanded{Overall accuracy}
                \addplot+[
                     fill=Greys-K,draw=none
                    ] coordinates {(0,10) (1,9) (2,4) (3,0) (4,0)};
                \addlegendentryexpanded{Overall quality}
            \end{axis}    
\end{tikzpicture}
\end{figure}

 


The following answers were possible for all four categories (in the diagram in \figref{fig:key:8:8} from left to right): \textit{well below average}, \textit{below average}, \textit{average}, \textit{above average}, \textit{well above average}. None of the translators ranked any of the four criteria as \textit{well above average}. For every criterion, a score was calculated in the following manner: \textit{well below average} was attributed \textit{\textminus2} points, average got \textit{0} points and well above average would have been \textit{2} points. The scores given by the translators for the individual criteria were on average all negative. \textit{Grammaticality} was rated \textminus1.26 (sd:~0.92; professionals – mean:~\textminus1.64, sd:~0.67; and students – mean:~\textminus0.92, sd:~1.0); \textit{style}, which is the least important in PE according to most guidelines, was considered the worst criterion with a value of \textminus1.43 (sd:~0.9; professionals – mean:~\textminus1.55, sd:~1.07; and students – mean:~\textminus1.33, sd:~1.07); \textit{overall accuracy} received the “best” evaluation with a value of \textminus1 (sd:~0.85; professionals – mean:~\textminus1.27, sd:~0.9; and students – mean:~\textminus0.75, sd:~0.75); and \textit{overall quality} gained \textminus1.26 points on average (sd:~0.75; professionals: mean:~\textminus1.36, sd:~0.81; and students: mean:~\textminus1.17, sd:~0.72). It was tested whether the ratings of both groups were significantly different, but no significant difference could be observed: \textit{grammaticality} – $W=37.5$, $p=0.0595$; \textit{style} – $W=63.5$, $p=0.8844$; \textit{overall accuracy} – $W=43$, $p=0.1401$; \textit{overall quality} – $W=55$, $p=0.4844$.



These results are in line with the answers from the first two questions and the question whether they would have preferred to translate from scratch as those assessments were quite negative, too. Apparently, the translators were really disappointed by the MT output. The results are a little surprising because other studies have shown that the MT output is not that bad, at least in what efficiency increases are concernded %\label{ref:ZOTEROITEMCSLCITATIONcitationIDKOrUeQUGpropertiesformattedCitationrtfLuc0u228ublietal2013plainCitationLublietal2013citationItemsid100urishttpzoteroorgusers1255332itemsBTZTM4N3urihttpzoteroorgusers1255332itemsBTZTM4N3itemDataid100typepaperconferencetitleAssessingposteditingefficiencyinarealistictranslationenvironmentcontainertitleProceedingsofMTSummitXIVWorkshoponPosteditingTechnologyandPracticepage8391eventWPTPauthorfamilyLubligivenSamuelfamilyFishelgivenMarkfamilyMasseygivenGaryfamilyEhrensbergerDowgivenMaureenfamilyVolkgivenMartinissueddateparts2013schemahttpsgithubcomcitationstylelanguageschemarawmastercslcitationjsonRNDF7tfhwZUIe}
\citep{LaubliEtAl2013}. First, it is doubtful whether the participants can really judge what is above and below average when more than half of them claim that they have never used MT before. Second, the MT quality assessment is very subjective and the participants may have based it on what they would expect from human translators. Further, some questions in the pre-experiment questionnaire already proved a rather negative attitude towards MT or bad experience with MT output (see \figref{fig:key:8:2} and \figref{fig:key:8:3}).



The last two questions focused again on the PE task. \textit{Q9: Based on the \isi{post-editing} task you have performed, which of these statements will you go for?} The aim of this question was to evaluate how much work was left after applying the MT. The participants could choose whether they \textit{had to post-edit ALL the outputs}, \textit{about 75\% of the outputs}, \textit{25--50\% outputs}, or \textit{only VERY FEW outputs}. To create a figurative value, the first answer \textit{I had to post-edit ALL the outputs} was given 0 points, whereas the last answer \textit{I only had to post-edit VERY FEW outputs} got 3 points.


\begin{figure}
\caption{Based on the post-editing task you have performed, which of these statements will you go for?}
% \todo[inline]{Shorter axis desc? - sure why not}
\label{fig:key:8:9}
% % \includegraphics[width=\textwidth]{figures/DissertationNitzkeberarbeitet-img13.jpg}
\hspace*{2cm}
\begin{tikzpicture}[trim axis right,trim axis left]
        \begin{axis}[
                    ybar,
                    xtick=data,
                    axis lines*=left,
                    ymin=0,
                    scaled y ticks = false,
                    colormap/Greys-5,
                    cycle list/Greys-5,
                    legend pos=inner north east,
                    reverse legend,
                    xticklabels={I had to post-edit ALL the outputs,I had to post-edit about 75\% of the outputs,I had to post-edit 25--50\% outputs,I only had to post-edit VERY FEW outputs},
                    x tick label style={rotate=20,anchor=east,font=\footnotesize},
                    ticklabel style={font=\footnotesize},
                    legend style={font=\footnotesize}
                    ]
                \addplot+[
                     fill=Greys-E,draw=none
                    ] coordinates {(0,4) (1,3) (2,4) (3,0)};
                \addlegendentryexpanded{Professional}
                \addplot+[
                     fill=Greys-I,draw=none
                    ] coordinates {(0,3) (1,6) (2,3) (3,0)};
                \addlegendentryexpanded{Student}
            \end{axis}    
\end{tikzpicture}
\end{figure}

 


Although the distribution is a little different, professionals as well as students agree on an average value of 1 (professionals sd:~0.89; students sd:~0.74), meaning that in general the groups felt that they had to post-edit about 75\% of the outputs.



Similarly, the next question asked: \textit{Q10: Based on the \isi{post-editing} task you have performed, how often would you have preferred to translate from scratch rather than \isi{post-editing} \isi{machine translation}?} The acceptable answers were similar: \textit{always}; \textit{in most of the cases (75\% of the outputs or more)}; \textit{in almost half of the cases (approx. 50\%)}; and \textit{only in very few cases (less than 25\%)} with \textit{always} matching 0 points and \textit{only a few cases} 3 points.


\begin{figure}
\caption{Based on the post-editing task you have performed, how often would you have preferred to translate from scratch rather than post-editing machine translation?}
\label{fig:key:8:10}
% % \includegraphics[width=\textwidth]{figures/DissertationNitzkeberarbeitet-img14.jpg}
\hspace*{2cm}
\begin{tikzpicture}[trim axis right,trim axis left]
        \begin{axis}[
                    ybar,
                    xtick=data,
                    axis lines*=left,
                    ymin=0,
                    scaled y ticks = false,
                    colormap/Greys-5,
                    cycle list/Greys-5,
                    legend pos=inner north east,
                    reverse legend,
                    xticklabels={Always,Most of the cases (≥75\%),Half to the cases (approx. 50\%),Very few cases (≤25\%)},
                    x tick label style={rotate=20,anchor=east,font=\footnotesize},
                    ticklabel style={font=\footnotesize},
                    legend style={font=\footnotesize}
                    ]
                \addplot+[
                     fill=Greys-E,draw=none
                    ] coordinates {(0,6) (1,3) (2,1) (3,1)};
                \addlegendentryexpanded{Professional}
                \addplot+[
                     fill=Greys-I,draw=none
                    ] coordinates {(0,4) (1,5) (2,2) (3,1)};
                \addlegendentryexpanded{Student}
            \end{axis}    
\end{tikzpicture}
\end{figure}

 


The value for the professionals is 0.73 (sd:~1.01) and for the students 1 (sd:~0.95). Hence, the first group would have preferred to translate more than 75\% of the texts from scratch, while the students would have preferred to translate about 75\% of the content from scratch. These last two questions showed (again) that the participants were not very satisfied with the MT output and judged that they hardly had any benefits from using the output. Questions Q9 and Q10 might be influenced by the personal opinion about MT and PE even more than Q8 and its sub-questions, because they may want to prefer TfS. As we saw in \sectref{sec:7:3}, PE (and MPE) was (were) more efficient at least regarding time. No significant difference between both groups was found again for the last two questions: Q9 – $W=66$, $p=1$; and Q10 – $W=53$, $p=0.4107$.



As a final note, the last two questions were not worded properly and would have to be adjusted if the study was replicated. The presentation of the single response option is very blurry. When one option is \textit{I had to post-edit all the outputs}, which can be interpreted as 100\% and another \textit{I had to post-edit 25--50\% of the outputs}, the option \textit{I had to post-edit about 75\% of the outputs} would imply everything between 50 and 99\%. Accordingly, the options are not weighted equally. While one option only includes 1\%, another includes 49\%. This improper scaling, however, indirectly shows the emotional situation of the participant. Although the data show that they did not have to post-edit all the output, one third of the professionals and one fourth of the students chose that option. The same applies for the last question. While the question we just discussed left little room for interpretation (95\% might have been considered as \textit{all the output}), the last question directly says that \textit{always} means 100\%, because the next option \textit{in most of the cases (75\% of the outputs or more)} directly states that everything including and above 75\% is included. The option \textit{in almost half of the cases (approx. 50\%)} applies to 25 to 74\% and \textit{only in very few cases (less than 25\%)} for the remaining percentage. The weight of the single items is again unbalanced. The percentage for the participants who opted for \textit{always} is even higher than in the other question: Over half of the professionals and one third of the students chose that option. This is a strong statement against PE. However, we have to bear in mind that the MPE task was the last task conducted in the experiments. Hence, the frustration might not have been that severe for PE, but the participants could have been biased by the MPE task.


\section{Discussion}
\label{sec:8:3}

%\label{ref:ZOTEROITEMCSLCITATIONcitationIDwsErhqnVpropertiesformattedCitationrtfOuc0u8217Brien2002plainCitationOBrien2002citationItemsid1139urishttpzoteroorggroups3587itemsA4J5F4GMurihttpzoteroorggroups3587itemsA4J5F4GMitemDataid1139typepaperconferencetitleTeachingposteditingaproposalforcoursecontentcontainertitleSixthEAMTWorkshoppublisherplaceManchesterUKpage99106eventSixthEAMTWorkshopeventplaceManchesterUKURLhttpwwwmtarchiveinfoEAMT2002OBrienpdfauthorfamilyOBriengivenSharonissueddateparts2002season1511accesseddateparts20121123schemahttpsgithubcomcitationstylelanguageschemarawmastercslcitationjsonRNDHyzDWBCbwB}
\citet[102]{OBrien2002} states that a positive \isi{attitude towards MT} is one of the attributes required by a post-editor to successfully complete the task. When the translators were asked to give their opinion on their previous experiences with MT engines, they claimed to be rather unsatisfied with the MT output (cf. \figref{fig:key:8:2}). The questionnaire did not ask in detail which MT engines the individual participants had gathered experience with or when they had used them. So, it can be assumed that their attitude towards the PE and MPE tasks was rather negative right from the beginning, which might have a negative influence on the final post-edited target text. Further, only 50\% of the professional and 17\% of the semi-professional translators had previous experience with PE, which might indicate that some translators had difficulties with the task itself and do not have any coping strategies, yet. It was not part of the questionnaire, but it seems probable that the \isi{Translog}~II environment was new to most of the participants as the use of \isi{Translog~II} is not very widespread outside of translation process research. Further, the output was produced by an online MT system (\textit{Google Translate}) that is not customized to certain text types or the translators' needs. When MT is used in professional translation environments, the MT engine is usually customized to the special needs of the company (cf. \sectref{sec:4:3}). Therefore, the output of an online MT system does not represent the quality that MT technology can provide. All (these) additional coefficients may have contributed to the negative evaluations of the MT output (cf. \figref{fig:key:8:8})\footnote{As mentioned above for the questionnaires in general, the MT system was also not adjusted, because of the predetermined setup.}.



Another interesting point is that although the participants were unsatisfied with the MT output (cf. \figref{fig:key:8:8}), they claimed that they could hardly see any use in the MT during PE, and would rather have translated from scratch (cf. \figref{fig:key:8:9} and \figref{fig:key:8:10}), they were rather satisfied with the final version of their post-edited texts (cf. \figref{fig:key:8:5}), and not that unsatisfied with their monolingual \isi{post-editing} task (cf. \figref{fig:key:8:6} – especially concerning the self-assessment of the students). One might have expected that the translators would be even less satisfied if the MT output was very poor and the translators could not even refer to the source text. These slightly contradictory assessments could be an indicator of emotional impact and subjectivity. Although the proximity of time has it advantages, the results of this questionnaire might be coloured by the emotions the participants experienced during the experiments. If a participant potentially had problems with a \isi{post-editing} task, his\slash her feelings about the MT output might have been rather bad directly after the experiment, because (s)he may still have been annoyed etc. Therefore, the evaluation of the MT, for example, might be worse than it would have been after a couple of hours\slash days distance from the tasks.\footnote{As the setup of the experiment was predetermined by CRITT, who originally composed the study, these considerations and decisions were not made by myself.}



Further, it is striking that the participants were more satisfied with their PE outcome than with the final text of the MPE task. As the data were not distributed normally, a Mann-Whitney-U-test was conducted which did not show significant differences between the evaluation of PE and MPE task for both professionals and students combined ($W=346.5$, $p=0.0574$), but might have with a higher number of participants. The different results in the self-evaluation might be related to the fact that most participants were unfamiliar with the PE task and probably even less were familiar with MPE because they could not consult the source text (as they do in TfS, too) to confirm or reject the MT output. Furthermore, the available source text in the PE task can be characterised as a valuable aid that is missing in the MPE task, which makes the task naturally more difficult.


\newpage 
In both questionnaires, the participants were asked whether they thought they would apply MT in their future translation tasks (\figref{fig:key:8:3} and \figref{fig:key:8:7}). After examining the satisfaction with the MT output in the experiments, it is expected that fewer translators could imagine using MT for their translation tasks as they are highly dissatisfied with the PE tasks and the MT output. However, almost the opposite is the case: Before the experiment, 8.3\% (one of twelve) of the professionals could imagine implementing MT for their future translation tasks, 58.3\% could not, and 33.3\% were not sure. After the experiment 18.2\% (two of eleven) could imagine implementing MT for their future translation tasks, 36.4\% could not, and 45.5\% were not sure. The percentage of participants who could imagine using MT in their translation tasks (about 9.9\% - two translators) and who were not sure (about 12.2\% - three translators) increased, while the percentage of people who could not imagine using MT in the translation tasks decreased about 22.1\%. The only explanation that comes to mind is that some translators could not imagine a way to use MT at all for their translation work before the experiment. However, the question about the feasibility of MT (cf. \figref{fig:key:8:4}) contradicts this assumption, because the professional translators were rather positive about it.



The tendency is the same for the semi-professional translators, but the picture is not as clear: Before the experiment, 25\% (three out of twelve) of the students could imagine implementing MT into their translation tasks, 41.7\% could not, and 33.3\% were not sure. After the experiment, 33.3\% were open to using MT in the future for translation, 41.7\% could not imagine it and 25\% were not sure. Therefore, the percentage of people who would use MT increased by about 8.3\% and the number who were not sure about using MT decreased by about 8.3\%. The simple explanation would be that one student who was initially not sure about MT in professional translation tasks would not use it. Hence, there was not much change in the evaluation of the feasibility of MT in the student group.

\begin{table}[b]
 \fittable{
\begin{tabular}{l ccc ccc ccc}
\lsptoprule
 & \multicolumn{3}{c}{From \textbf{Yes} to} & \multicolumn{3}{c}{From \textbf{Not} \textbf{sure} to} & \multicolumn{3}{c}{From \textbf{No} to}\\
 \cmidrule(lr){2-4}\cmidrule(lr){5-7}\cmidrule(lr){8-10}
& Yes & Not sure & No & Yes & Not sure & No & Yes & Not sure & No\\
\midrule 
professionals & 0 & 1 & 0 & 1 & 3 & 0 & 1 & 1 & 4\\
students & 2 & 1 & 0 & 2 & 2 & 0 & 0 & 0 & 5\\
\lspbottomrule
\end{tabular}
}
\caption{Changed answers in the two questionnaires}
\label{tab:key:8:4}
\end{table}


\tabref{tab:key:8:4} illustrates the (potential) change of mind after the experiments. It is quite striking that nobody changed his\slash her opinion from \textit{Yes} or \textit{Not sure} to \textit{No}, which, as mentioned above, could have been expected according to the other evaluations. Only two people went from \textit{Yes} to \textit{Not sure}, while the rest of the participants retained or improved their opinion on MT as an aid for translation tasks. This is contradictory to the negative opinions on the MT output and the PE and MPE tasks from the other questions. One argument to explain this tendency may be that some of the translators may have never considered MT systems as a possible tool for their translation tasks and they changed their judgement accordingly after they had gained experience. It would have been interesting to know how many participants recalled the question and how this influenced the answer to the question, but this factor was not checked for.




{To summarise the results, the participants – neither professionals nor semi-professionals - had a negative attitude towards MT before or after the experiments. They did not enjoy the PE task and they ejoyed the MPE task even less. In fact, they would have rather translated the texts from scratch. Nonetheless, the participants would consider MT output as an aid for their future translation tasks even more than before the experiments. In general, the students had a slightly more positive attitude than professional translators. Principally, the attitude towards MT\slash PE seems to have a major influence in the questionnaire data. Although the following analyses (\sectref{sec:9} - \sectref{sec:12}) very often focus on subconscious behaviour, we cannot be certain whether and how much this attitude had a (negative) influence on the results of this study. This is not measurable in the scope of this study. However, it would be very interesting to investigate whether and to what extent this negative attitude had an influence on the participants' behaviour during the tasks.}


