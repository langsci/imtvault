\chapter{Introduction} \label{sec:intro}
The concept of \textit{sentence} occupies a central position in linguistic theory. In (generative\is{Generative grammar}) syntax, well-formed expressions are dominated by a node that is related to sententiality, which was originally labeled as S(entence) \citep[see e.g.][]{chomsky1965} and which has been more recently redefined as the complementizer phrase\is{Complementizer phrase} (CP) or a CP layer \citep{rizzi1997}. This layer is taken to host different speech act-related features, such as sentence mood in \citepos{rizzi1997} ForceP and assertivity \citep{krifka1995}. From a semantic perspective, it is generally assumed that only sentences can be used to perform speech acts and to communicate propositions.

This theoretically motivated requirement for well-formed utterances to be sentential clashes with linguistic reality. \citet{morgan1973} observed more than forty years ago that speakers produce nonsentential utterances which still fulfill the same communicative function as their sentential counterparts. For instance, the utterance in \Next[a] lacks an inflected verb and a subject. At least in this specific context however, \Next[a] is pragmatically interpreted as being meaning-equivalent to the full sentence in \Next[b]. \citet{morgan1973} proposed the term \textit{fragment} for these nonsentential utterances, which I adopt in this book.%
%
\footnote{The term \textit{fragment} is roughly meaning-equivalent to that of \textit{nonsentential utterance} used elsewhere \citep{fernandez.ginzburg2002, barton.progovac2005, stainton2006}. None of these notions is theory-neutral, since \textit{fragment} suggests that the utterance is incomplete, whereas the notion of \textit{nonsentential utterance} implies that it is not underlyingly sentential. Besides being more in line with my empirical findings on sententiality I use the notion \textit{fragment} because it has been proposed earlier in the literature for this phenomenon.}\afterfn%
%

\ex. [Ann and Bill are sharing a pizza. Bill asks Ann:] 
    \a. Another slice? \label{ex:intro-fragment}
     \b. Would you like another slice of pizza?
 
This book addresses two main research questions which are investigated with experimental methods: First, what is the syntactic structure of these expressions? And, second, why do speakers%
%
\footnote{This work focuses on spoken and written language. The term \textit{speaker} refers to the person who produces an utterance and the term \textit{hearer} to the person who processes it. The overarching ideas and results on the question of why people produce a reduced or a syntactically complete utterance might be applied to sign language, what might be investigated in future research.}\afterfn%
%
use fragments at all?

\section{What is the syntactic structure of fragments?}
The apparent violation of standard assumptions about phrase structure in fragments challenges syntactic theory: For instance, a grammar that requires all well-formed structures to contain an inflected matrix verb is not able to derive a bare DP\is{Determiner phrase} fragment like \Last[a]. This mismatch between the nonsentential form and the sentential function of fragments has been relatively extensively investigated in theoretical linguistics \citep[see e.g.][]{morgan1973, ginzburg.sag2000, fernandez.ginzburg2002, schlangen2003, merchant2004,barton.progovac2005, culicover.jackendoff2005, stainton2006, reich2007, weir2014, ott.struckmeier2016}, but there is no consensus on the theoretic analysis of fragments. In particular, it is unclear whether fragments result from ellipsis in full sentences, which are derived by the standard syntax rules, or whether fragments require modifications to syntactic theory that allow for the derivation of subsentential output. Furthermore, for now the competing theories rely almost exclusively on partially conflicting introspective data. The first part of the book (Chapters \ref{sec:chapter-theories} and \ref{sec:chapter-experiments-syntax}) presents a series of acceptability rating\is{Acceptability rating task} and production studies\is{Production task} that investigate the predictions of the competing theories. These experiments provide the first empirical investigation of a set of diverging predictions of the competing theories of fragments.

I focus on three generative\is{Generative grammar} accounts of fragments: the \textit{nonsentential} account\is{Nonsentential account} \citep[e.g.][]{barton.progovac2005, progovac2006}, the information structure\is{Information structure}-based \textit{in situ deletion} account \citep{reich2007, ott.struckmeier2016} and the \textit{movement and deletion}\is{Movement and deletion account} account \citep{merchant2004, weir2014}. These theories make relatively general testable predictions on fragments, such as the requirement that all fragments must be able to appear in the left periphery according to \citet{merchant2004}. I do not investigate HPSG\is{HPSG} accounts of fragments \citep{ginzburg.sag2000, fernandez.ginzburg2002, schlangen2003}, which assign different internal structures that are relatively independent from each other to different types of fragments, depending on the context in which the fragment occurs. An empirical study would have to test all of these structures individually in order to determine the appropriateness of such an account.

The accounts that I investigate differ in particular with respect to two issues: First, whether fragments are underlyingly sentential, and second, whether their generation involves obligatory syntactic movement. The first question is a matter of debate between sentential and nonsentential accounts\is{Nonsentential account} of fragments, whereas the second one is disputed between the different families of sentential accounts. A first series of experiments investigates whether fragments are underlyingly sentential or base-generated nonsentential utterances. These experiments use structural case marking on DP\is{Determiner phrase} fragments as a diagnostic for unarticulated syntactic structure. Since the experiments provide evidence for a sentential analysis of fragments, I then use potential parallelisms between fragment and movement restriction\is{Movement restriction} as a testing ground for obligatory movement in fragments\is{Movement and deletion account}. Taken together, the experiments support the in situ ellipsis account of fragments\is{In situ deletion account}, which has been proposed by \citet{reich2007}. This complements the theoretical debate on the syntax of fragments with empirically validated data and settles the ground for the investigation of the usage of fragments in the second part of this book.

\section{Why do speakers use fragments?}
Generative\is{Generative grammar} theories of fragments determine which form fragments can take, but they do not explain why speakers use fragments at all, and under which circumstances they prefer a fragment over a complete sentence. Corpus\is{Corpus} data show that fragments are relatively frequent, and, the frequent usage of fragments%
%
\footnote{For instance, \citet{fernandez.ginzburg2002} find in a corpus study that 11.15\% of the utterances in a subcorpus of the British National Corpus\is{Corpus} \citep{burnard2000} are fragments.}\afterfn%
%
suggests that speakers have a reason to prefer them over full sentences in particular situations. However, except for a game-theoretic\is{Game theory} approach with very restricted scope by \citet{bergen.goodman2015}, the question of what determines this preference is totally unexplored.

The second part of this book (Chapters \ref{sec:chapter-infotheory} and \ref{sec:chapter-infotheory-experiments}) is dedicated to the investigation of why and when speakers use fragments. An answer to this question requires establishing (i) why the usage of a fragments or a sentence is sometimes (dis)advantageous, and (ii) why \textit{specific} words are preferably omitted in fragments. For instance, in the case of the pizza example in \Last, the speaker might have said \textit{Another slice of pizza} instead, so the choice between competing fragments must be modeled too. At this point, the investigation of the usage of fragments draws on the findings on the syntax of fragments in the first part of the book, since the set of \textit{possible} fragments is necessarily restricted to those which can be derived by syntax. 

The account that I propose assumes that the information-theoretic\is{Information theory} processing principle of Uniform Information Density\is{Uniform Information Density} (UID, \citenob{levy.jaeger2007}) plays a crucial role in the choice of an utterance by the speaker. Two experiments confirm the central predictions of this account: Speakers choose the utterance that makes the most efficient usage of the hearer's processing resources\is{Processing effort}, and they consequently omit words that underutilize these resources but realize words that prevent them from being exceeded. In addition to providing evidence for the infor\-mation-theoretic account of the usage of fragments, these findings have implications both for the research on ellipsis and for the investigation of the choice between alternative utterances in general. The choice between a reduced (elliptical) form and  a complete one is also highly relevant to other ellipsis phenomena like sluicing\is{Sluicing}, gapping\is{Gapping} and verb phrase ellipsis\is{Verb phrase ellipsis}, and it might be instructive to test whether the conclusions on the usage of fragments apply to these ellipses as well. From a broader psycholinguistic perspective, my results contribute to the growing bulk of evidence for effects of information-theoretic\is{Information theory} processing constraints, and specifically for UID\is{Uniform Information Density}, on the preferred form of utterances. This supports several implications of UID\is{Uniform Information Density}, such as the close link between predictability and processing effort, the assumption of audience design\is{Audience design} and the parallel and incremental nature of the human parser\is{Parser, human}.

\section{Chapter overview}

\begin{itemize}\itemsep0em
 \item Chapter \ref{sec:chapter-theories} outlines three main generative theories of fragments and identifies potential testing grounds for them, that is, phenomena with respect to which their predictions differ.
 \item Chapter \ref{sec:chapter-experiments-syntax} presents a series of experiments that test the predictions of the theories presented in Section \ref{sec:chapter-theories}. The experiments suggest that fragments are derived by ellipsis from regular sentences and that their derivation does not involve obligatory movement to the left periphery.
 \item Chapter \ref{sec:chapter-infotheory} briefly reviews information-theoretic\is{Information theory} approaches to the omission of linguistic expressions and presents my information-theoretic\is{Information theory} account of fragment usage.
 \item Chapter \ref{sec:chapter-infotheory-experiments} presents two experiments that test and support the predictions of the infor\-ma\-tion-theoretic account: Predictable words are more likely to omitted and words that increase the predictability of following ones are more often realized. The experiments rely on script-based\is{Script knowledge} event chain\is{Event chain}s as an approximation to extralinguistic context\is{Context, extralinguistic} and a method to estimate word probabilities in elliptical data.
 \item Chapter \ref{sec:chapter-discussion} summarizes and discusses the main results as well as the empirical and theoretical contributions of this book.
\end{itemize}

\section{Defining the notion \textit{fragment}}

\label{sec:intro-fragments}
In the literature on fragments, there is no mutually shared definition of the phenomenon and there is disagreement specifically with respect to which utterances are classified as fragments. This section delimits how the notion \textit{fragment} is used in this book and distinguishes it from other instances of reduced utterances. In order to distinguish fragments from other omission phenomena and instances of apparently incomplete speech, I rely on three criteria: (i) the performance of a speech act \citep{morgan1973}, (ii) the absence of a finite verb, and (iii) the absence of a linguistic antecedent\is{Context, linguistic} within the same utterance. First, a fragment must be used to perform a speech act. This excludes labels \citep{klein1993} like \Next.

\ex. Skim milk / 16 oz. / sugar free \dots

Second, the distinction between fragments and sentences is based on the subsentential character of fragments. What counts as ``subsentential'' depends on the syntactic analysis of the expression in question. For instance, in sentences with null pronouns in argument positions like \Next, where a subject has been omitted, the remainder of the sentence is preserved, whereas in the DP\is{Determiner phrase} fragment in \LLast[a] there is no immediate evidence for any structure above the DP\is{Determiner phrase} level.

\ex. Will be back soon.

Therefore, the most uncontroversial examples of fragments are XPs that are not of the same category as full sentences, which is most easily evidenced by the absence of an inflected verb. For instance, if English\il{English} sentences are TPs\is{Tense phrase}, the bare DP\is{Determiner phrase} \textit{another slice} in \ref{ex:intro-fragment} must be categorized as a fragment based on this criterion. The same holds for any category below TP, like VP\is{Verb phrase}, PP\is{Preposition phrase} or NP\is{Noun phrase}. Consequently, utterances like \Last are not categorized as fragments even though they lack an otherwise obligatory argument, because the auxiliary still evidences that the utterance is a TP. Note that this does not imply that fragments do not ever \textit{contain} TPs\is{Tense phrase} but only that they \textit{are} not TPs\is{Tense phrase} themselves. The complement clause\is{Complement clause} in \Next[a] hence counts as a fragment, because in a full sentence it needs to be embedded under a matrix verb \Next[b] that is missing here. 

\ex. \a. That he'll be back soon.
     \b. John said that he'll be back soon.

Third, unlike antecedent-based ellipses\is{Antecedent-based ellipsis} \citep{reich2011}, such as gapping\is{Gapping}, sluicing\is{Sluicing}, sprouting, and verb phrase ellipsis\is{Verb phrase ellipsis}, fragments do not require an explicit linguistic antecedent\is{Context, linguistic}. There is some disagreement in the literature about whether this condition excludes short answer fragments\is{Fragment, short answer}, too. For instance, \citet{klein1993} distinguishes discourse-initial fragments\is{Fragment, discourse-initial} from what he calls \textit{adjacency pairs}. According to \citet[768]{klein1993}, ellipses in adjacency pairs ``require an explicit linguistic context\is{Context, linguistic}, [\dots] on which the elliptical utterance depends [translation from German\il{German}, R.L.].'' This definition of adjacency pairs clearly includes short answers\is{Fragment, short answer} as \Next. The distinction between short answers\is{Fragment, short answer} and discourse-initial fragments\is{Fragment, discourse-initial} is explicitly made in \citet{reich2011}. In contrast, all of the researchers whose theories I discuss in Chapter \ref{sec:chapter-theories} rely on data from short answers in support of their theories of fragments. This suggests that they adopt, at least implicitly, a uniform analysis of short answers\is{Fragment, short answer} and discourse-initial fragments\is{Fragment, discourse-initial}.

\ex. What did John eat?\\
Pizza.

Even though the status of short answers\is{Fragment, short answer} is theoretically controversial, some of the experiments presented in this book investigate short answers\is{Fragment, short answer}, particularly in the extensions of the experiments by \citet{merchant.etal2013}, who also used short answers\is{Fragment, short answer} in their studies. In experiments \ref{exp:case}--\ref{exp:scripts-rating-case} on default case\is{Default case} marking as well as in experiments \ref{exp:scripts-rating} and \ref{exp:scripts-production}, which investigate the usage of fragments, I use discourse-initial fragments\is{Fragment, discourse-initial} instead. As for the question of whether there is a categorical distinction between adjacency pairs and genuine fragments, from the probabilistic perspective that my information-theoretic\is{Information theory} account implies, it seems compelling to attribute potential differences between short answers\is{Fragment, short answer} and discourse-initial fragments\is{Fragment, discourse-initial} to differences in predictability: Material that has been mentioned in an explicit preceding question will be much more predictable than when it must be inferred from extralinguistic context, and the use of fragments will be therefore more strongly preferred. However, testing this experimentally will be complicated due to the necessary correlation between predictability and the type of context. Therefore, I remain agnostic to the question on whether there is a categorical difference between short answers\is{Fragment, short answer} and discourse-initial fragments\is{Fragment, discourse-initial}. Except for the studies that replicate or follow up on previous experiments involving short answers\is{Fragment, short answer}, I rely on discourse-initial fragments\is{Fragment, discourse-initial}, which are the most uncontroversial instances of fragments.
