\chapter{What motivates Emergent Grammar?}\label{intro_motivation}
\label{ch1} \label{section_cognition}
\largerpage[1.5]
\is{Emergent Grammar|(}There is a striking dichotomy in the way that oral languages make use of sound. Many of the properties of a language's sounds are \is{idiosyncrasy}idiosyncratic, such as which sounds are associated with a particular meaning.\is{meaning} At the same time,  sounds are combined in highly regulated ways -- certain sounds are permitted while others are not (English allows [s] but not [x]);\il{English} certain sounds cooccur in sequences\is{sequence!phonotactics} while others do not  (English allows initial [tr] as in {\it treat}, but not initial *{\it tl}), etc. And some patterns are both regular and idiosyncratic, such as the regularity of \is{assimilation!nasal place}nasal place assimilation in English, coupled with the \is{idiosyncrasy}idiosyncrasy that there are invariant\is{morph!invariant} nasal-final prefixes,  immune to the regularity. The issue at the core of this monograph is the nature, source, and representation of phonological and morphophonological knowledge -- the deep and pervasive regular patterns, the various idiosyncratic forms, and their interaction.\index{invariant morph|see{morph!invariant}} 

What is the source of this knowledge?  At some level, these principles must emerge from the human learner's consideration of the data. The way in which this is achieved is determined by the nature of the learner's mental capacities that drive the acquisition\is{acquisition} of phonology. At one end of the spectrum, phonological acquisition might result from cognitive capacities of a specifically linguistic nature, with general phonological principles encoded as a necessary part of natural language grammar.\is{grammar}  Under this view, an innate\is{innateness} language faculty (\textit{Universal Grammar})\is{Universal Grammar} enables the language learner to cope with learning\is{acquisition} an ambient language by restricting the realm of possible grammars (\citealt{Chomsky+:1968}).

\is{grammar|(}We explore an alternative towards the other end of the spectrum, that general human cognition\is{cognition} enables the learner to characterise patterns appropriate for the data concerned -- there are no, or perhaps few, {\it a priori} regulatory principles specific to language sounds. Under this view, phonological knowledge is acquired\is{acquisition} by domain-general mechanisms, the \textit{Emergent Grammar} hypothesis:\is{Emergent Grammar hypothesis|(} general cognitive principles are the driving force behind the acquisition\is{acquisition} and form of morphophonological grammars, rather than cognitive principles that have evolved specifically for language sound patterns (\citealt{Lindblom:1999}).\footnote{In this, our starting point is similar to that of Construction Grammar\is{Construction Grammar}, e.g., ``what is central here is the conviction that language is on a par with all other human cognitive faculties, by whatever manner they are acquired.'' \citet[1]{Valimaa-Blum:2011}. At this point, there is little work on phonological systems under Construction Grammar\is{Construction Grammar}; \citet{Valimaa-Blum:2011, Hoder:2014, VanDerSpuy:2017} are notable exceptions.}  Under this research program, principles specific to an innate ``language'' module would be posited only if general cognitive principles are inadequate. In all our work within Emergence to date, we have yet to find convincing evidence of the inadequacy of the Emergence hypothesis.  


\begin{dadpbox}{The Emergent Grammar Hypothesis}{box-EG-hypothesis}
General human cognition provides much, if not all, of the necessary scaffolding for the acquisition\is{acquisition} of {morphophonology}, allowing construction of a phonological grammar of the ambient language.
\end{dadpbox}\is{Emergent Grammar hypothesis|)}

Our motivation for this exploration begins with the realisation that, if one were to argue strongly in favour of innate\is{innateness} regulatory principles that were inherently linguistic, it would be important methodologically to eliminate the possibility of adopting comparable principles that were rooted in general, nonlinguistic capabilities.\footnote{As demonstrated for example in \citet{Mielke:2008}, from its inception, Generative Grammar\is{generative phonology} has typically assumed innate phonological principles rather than providing arguments for such innate principles.} In other words, if we wish to strongly argue in favour of innately-encoded linguistic principles, we need to demonstrate that it is impossible to learn the grammar (or a specific aspect of it) using only general cognition\is{cognition}.  As we demonstrate with our case studies here,  accounting for patterns with {minimal -- or no --} recourse to linguistically-specific principles is not only feasible, but in many cases the resulting system is far less complex than the one resulting from  an innatist approach.

\largerpage[-1]
To begin laying out our proposal, it is important to be clear about what we mean by \textit{grammar}. \is{grammar!definition}Grammar is a set of regulatory principles governing a wide array of patterns in some human language; our focus is on the patterns involving sounds. The grammar of a language characterises the recurrent linguistic sounds of that language (the language's ``segmental inventory'')\is{inventory} and \is{generalisation!acquisition|(}generalisations about those sounds. The language's grammar determines what sorts of sequences\is{sequence!phonotactics} of sounds constitute well-formed meaningful strings  of sounds (\textit{phonotactics}),\is{phonotactics} strings such as morphs and words.\is{word}  Additionally, the combination of meaningful sound strings into larger units is also subject to regulatory principles (\textit{morpho-phonotactics},\is{morpho-phonotactics} cases where  well-formedness statements\is{well-formedness condition} about sounds are mediated by morphological categories). There are also regularities concerning productive relations between morphs, that is, regularities whereby one morph is perceived as related to another morph in terms of its meaning\is{meaning} or \is{syntax}syntax, even though the sounds of the forms differ (\textit{Morph Set Relations}, \textit{Morph set Conditions}).\is{Morph Set Relation}\is{Morph Set Condition}  The grammar that we are concerned about, then,  is the mental encoding of phonological and morphophonological regularities such that speakers are able to understand novel forms as well as generate novel forms that are perceived as acceptable in the language (\citealt{Berko:1958}, etc.).\footnote{Properties of inventories\is{inventory} have been given considerable attention in the Emergence literature, for example, in the early work of \citet{Lindblom:1999} and subsequently in work such as \citet{Mielke:2008, Cohn:2011}. How combinatorial patterns emerge and what their nature is has been explored in work such as \citet{Cole:2009}. This is an important aspect of the emergence of a grammar, but not the one that we focus on in this work.}
 

In pursuing this Emergentist approach\is{Emergent Grammar} we follow a large literature, including \citet{Hopper:1987, Lacerda:1995, Deacon:1997, Bybee:1998lexicon, Hopper:1998, Lacerda:1998, Bybee:1999, Lindblom:1999, MacWhinney:1999, Lindblom:2000, Frisch+:2001, Kochetov:2002, Lacerda:2003, Harrison+:2007, OGrady:2008, Beckner+:2009, Cole:2009, Pater:2012,Hopper:2015, McCauley+:2015, Racz+:2015, Weijer:2017, Haspelmath:2020}, inter alia, as well as extending our own work on the topic (\citealt{Mohanan+:2010, Archangeli+:2011Paris, Archangeli+:2012Korea,  Archangeli+:2012McGill, Archangeli+:2012Smith, Archangeli+:2013WCCFL, Archangeli+:2014freeling, Archangeli+:2014lshk, Archangeli+:2015_Frontiers, Archangeli+:2015HKUtone, Archangeli+:2015_K-tone, Archangeli+:2015_YVH, Archangeli+:2016mm, Archangeli+:2017-Setswana, Anghelescu+:2017, Gambarage+:2017, Archangeli+:2018routledge, Archangeli+:2014abidjan}).


In this monograph, we build on such work, aiming to develop our understanding of how human cognition serves to construct a grammar, addressing patterns involving inventories\is{inventory}, phonotactics,\is{phonotactics} morpho-phonotactics\is{morpho-phonotactics} and relations within morph sets. In the next chapter, we illustrate these grammatical components through the lens of a single language. But before we turn to that exemplification, we address briefly how general cognitive properties might give rise to the sorts of phonological patterns just alluded to.\footnote{See \citet{Weijer:2009, vandeWeijer:2012, Weijer:2014, Weijer:2017, Weijer:2019} for a similar research program, couched within Optimality Theory.\is{Optimality Theory}}


\largerpage[-2]
Relevant cognitive properties\is{cognition} include uncontroversial abilities such as the ability to attend\is{attention} to language (\citealt{Pena+:2003}), to remember\is{memory} (\citealt{Meltzoff:1988}), to identify similarities\is{similarity}  (\citealt{Goldstone:1994}), to process sequences\is{sequence!processing}\is{processing} (\citealt{Conway+:2001}), to attend to frequency\is{frequency} (\citealt{Tenenbaum+:2001, Thiessen+:2015}), to form categories\is{category} (\citealt{Ashby+:2005, Ashby+:2011}), to generalise, and to generalise over generalisations (\citealt{Deacon:1997, Gomez+:2000, Saffran+:2007}).\footnote{Perhaps some abilities involve some degree of language-specificity. For example, the neonate attending to linguistic input over other types of input might suggest a preference for speech that is innate\is{innateness} (\citealt{Pena+:2003}), a surmise that is contested in \citet{Lacerda:2003}. Even if there were such a preference, it does not take us far along the spectrum of properties needed for a full characterisation of an adult phonological grammar. Similarly, there is evidence that infants in nonhuman species\is{nonhuman species} attend preferentially to the sounds of their own species, e.g.\ pygmy marmosets (\citealt{Snowdon+:1978}), birds (\citealt{Whaling:2000}), and Japanese macaques (\citealt{Adachi+:2006}), as well as subspecies (birds again, \citealt{Nelson:2000}). \label{species-preference}} 


\begin{example}  \et{Human cognition primitives (a non-exhaustive list)}\label{cognitive_primitives}
    \begin{enumerate}[a.]
    \item {Attention}\label{attention}\is{attention}
    \item {Memory}\label{memory}\is{memory}
    \item {Similarity}\label{similarity}\is{similarity}
    \item {Sequential processing}\label{sequential-processing}\is{sequence!processing}\is{processing}
    \item {Frequency}\label{frequency}\is{frequency}
    \item {Categories}\label{categories}\is{category}
    \item {Generalising} (\& generalising over generalisations)\label{generalising}
    \end{enumerate}
\end{example}

Building such cognitive\is{cognition} scaffolding into an account of phonological patterns is rampantly ``bottom-up''\is{bottom-up} -- especially at the outset. However, as forms are acquired and hypotheses are formed, the nascent grammar\is{Emergent Grammar} also informs continued acquisition:\is{acquisition} grammar construction is not only bottom-up, it is also\is{top-down} ``top-down''.\footnote{See \citet{Rose:2009, Rose+:2011, Rose:2014emergence} for arguments in favour of acquisition involving multiple factors, both bottom-up and top-down.}\is{bottom-up}\is{top-down} This is consistent with language acquisition\is{acquisition} research: as knowledge is acquired, the learner not only adds new knowledge, but also builds generalisations upon existing generalisations (\citealt{Martin+:2013, Curtin+:2014}).  

In order to learn the phonological system of a language, there are certain necessities.\is{acquisition} The learner must experience the language being learned auditorily and/or visually\is{modality} and remember (some of) what is perceived.\is{perception}\footnote{In the acquisition\is{acquisition} of oral language, both audition and vision\is{modality} are relevant (\citealt{McGurk+:1976, Rosenblum+:1997, Burnham+:2004}) from a very early age (\citealt{Coulon+:2013, Weikum+:2007}). In this work, we focus on spoken language and auditory perception. This is a simplifying strategy. We also assume that the essence of the proposals we are making here for spoken languages is similarly relevant for signed languages, {\it mutatis mutandis}.} Early learning\is{acquisition!early} involves breaking up strings of segments. This might be through mechanisms such as tracking transitional probabilities\is{transitions} between units (\citealt{Saffran+:1996san}), or it might be by identifying chunks\is{chunk} within the speech stream (\citealt{Perruchet+:1998}; for discussion of these two possibilities, see \citealt{Black:2018}). The remembered fragments may begin as individual, unanalysed entities, but eventually the learner recognises that some of these fragments are highly similar in some way (due to density in the representational space),\is{acquisition!sets}\is{set!acquisition} and so posits groups based on that similarity. The higher the density of some property defined by similarity of any type, the greater the likelihood of that property being encoded in the grammar.\footnote{Thanks to \name{Ben}{Martin} for helpful discussion of this point.}


Thus, early learning identifies sound chunks and similarities\is{similarity} among those chunks. Learners keep track of chunks they have heard: highly frequent\is{frequency} sound chunks take on greater significance as categories, established as the contrastive\is{contrast} sounds of the language encountered. Constellations of highly frequent properties lead to generalisations about sounds and about patterns. The earliest  generalisations identify the sound  segments and categories\is{category} of the ambient language.\footnote{For similarities\is{similarity} and sound chunks, see  \citet{Newell:1990, Saffran+:2003, Christiansen+:2009, Graf+:2011, Martin+:2013}. On frequency\is{frequency} of chunks\is{chunk} and categorisation,\is{category} see \citet{Rosch+:1976, Plunkett+:1991, Saffran+:1996san, Aslin+:1998, Zacks+:2001, Maye+:2002, Saffran:2003, Newport+:2004-I, Newport+:2004-II, Zacks+:2006, Diessel:2007, Pelucchi+:2009, Seger+:2010, Ellis+:2015, Thiessen+:2015}. And  on generalising, see \citet{Eimas+:1971, Werker+:1981, Werker+:1984, Jusczyk+:1994, Polka+:1994, Pegg+:1997, Stager+:1997, Werker+:2002, White+:2008}.}
As learning progresses,\is{acquisition!early} longer sequences are committed to memory.\is{memory} This allows for frequency observations with respect to position and with respect to substantive properties of the strings encountered, leading to new generalisations. Phonotactics,\is{phonotactics} both sequential and positional, are identified.\footnote{On committing sounds and strings to memory,\is{memory} see \citet{Cristia+:2012, Peperkamp+:2006}; for sequence\is{sequence!acquisition} learning and frequency,\is{frequency} see \citet{Marcus+:1999, Saffran+:2003}.  For further generalisation, see \citet{Kuhl+:1992, Jusczyk+:1993, Jusczyk+:1994, Jusczyk+:1999allophones, Gomez:2002, Martin+:2013}.}


At the point when sound-meaning correspondences\is{sound-meaning correspondence} begin, the earliest items acquired\is{acquisition} are individual vocabulary items (perhaps not corresponding exactly to the adult items); as more items are acquired, those with similar properties are grouped together leading to new generalisations  which serve to identify the morphs\is{morph!acquisition} of the language, to  group morphs together, to identify phonological  relations among morphs in a set, and so on (see \citealt{Seidl+:2005, Gerken+:2008, Gerken+:2015}).  When multiple morphs are identified with the same set of morphosyntactic\is{feature!morphosyntactic} features, the learner\is{acquisition} is faced with the challenge of selecting among the morphs when building words\is{word} -- an assessment\is{assessment} which  relies on phonological and/or morphological \textit{well-formedness conditions}.\is{well-formedness condition}


\begin{dadpbox}{Acquisition of phonological, morphological, and sociolinguistic conditions}{box-acquisition-sequence}
A more complete proposal would consider the \is{acquisition}acquisition of sociolinguistic\is{sociolinguistics} conditions, pragmatic\is{pragmatics} conditions, and so on, considering how and when such conditions are acquired relative to more purely phonological conditions, how children's acquisition of phonology, \is{morphology!acquisition}morphology and sociolinguistic conditions is synchronised or sequential, at what point forms identified as polymorphic for an adult come to be identified as such by the child, and so on. For example, \citet{Roberts:1994, Roberts:1997} compares the alternation\is{alternation!child language} between [t]/[d] and $\emptyset$ among 3 and 4 year olds in South Philadelphia\il{English!South Philadelphia} to that of their care givers.\is{acquisition} Her work finds differences in the extent to which children of this age group reflect adult patterns depending on whether the verbal patterns under investigation are regular or \is{irregular verbs!English}irregular (the weak verbs).  The asymmetric pattern she identifies indicates that, while adults  treat the semi-weak verbs as they treat polymorphic words, children treat the semi-weak as they treat monomorphemic items. This suggests a stage in acquisition\is{acquisition} where the child has yet to generalise\is{generalisation!acquisition|)} a polymorphic pattern for the semi-weak verbs.\\

While such work about phonological usage is important, we do not build it into our discussion here. However, we suspect that the Emergent framework extends readily to usage domains (\citealt{vandeWeijer:2012}): a case in point is the Emergent analysis of diglossia\is{diglossia} in Faifi\il{Faifi} (\citealt{Alfaifi:2020-SAL, Alfaifi:2020phd}).
\end{dadpbox}


To make concrete what such considerations would mean for acquiring a phonological grammar within an Emergentist framework,  we sketch what appears to take place during the acquisition\is{acquisition} of the phonological grammar of a specific language.  Our goal is twofold. On the one hand, we build on the points sketched briefly above, aiming to elucidate how a phonological grammar might emerge. On the other hand, we do so concretely as well as abstractly, showing that the model we are developing allows us to address phonological problems of intricacy and complexity. The phonological literature has established numerous patterns both within and across languages that demand explanation. Within an Emergentist framework, we attempt here to build a model that allows us to address such patterns. One could in principle use almost any language for illustration since the types of phenomena illustrated are familiar across languages: in the next two chapters, we explore the nature of phonological and morphophonological analysis from the Emergentist point of view, using the distribution of vowels in Yangben.\is{grammar|)}\is{Emergent Grammar|)}
