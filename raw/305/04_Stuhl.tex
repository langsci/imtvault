\documentclass[output=paper]{langsci/langscibook} 
\ChapterDOI{10.5281/zenodo.4954479}
\author{Britta Stuhl\affiliation{Freie Universität Berlin} and Christian Zimmer\affiliation{Freie Universität Berlin}}
\title[Northern German in Southern Africa?]{Northern German in Southern Africa? On the phonology of Namdeutsch}

\abstract{This chapter presents a study on the phonology of Namdeutsch, a variety of German spoken in Namibia. Previous literature has called its pronunciation either Standard German or Northern German and the aim of this paper is to determine whether Namdeutsch does share phonological characteristics with Northern German, based on the analysis of two vowel variables and four consonant variables that occur in Northern German. The data for this study stems from the \textit{Deutsch in Namibia} corpus. The analysis reveals that, while not all Northern German variants are common in Namdeutsch, both vowel features and a consonant feature are frequent or very frequent in the data. }
\IfFileExists{../localcommands.tex}{
 \addbibresource{localbibliography.bib}
  \input{../localpackages}
  %Copy this to localcommands.tex

\usepackage[english]{babel}
\usepackage{amsmath}
\usepackage{amssymb,amsfonts,textcomp}
\usepackage{array}
\usepackage{hhline}
\usepackage{hyperref}

\newenvironment{styleStandard}{}{}
\newenvironment{stylelsAbstract}{}{}
\newenvironment{stylelsSectioni}{}{}
\newenvironment{stylelsSectionii}{}{}
\newenvironment{stylelsBulletList}{}{}
\newenvironment{styleBibliographyi}{}{}
\newenvironment{listWWNumxxvleveli}{}{}
\newenvironment{listWWNumxxvlevelii}{}{}
\newenvironment{listWWNumxxvleveliii}{}{}
\newenvironment{listWWNumxxvleveliv}{}{}
\newenvironment{listWWNumixleveli}{}{}
\newenvironment{listWWNumixlevelii}{}{}
\newenvironment{listWWNumixleveliii}{}{}
\newenvironment{listWWNumixleveliv}{}{}

\newcommand\textstyleListLabelxvi[1]{#1}
\newcommand\labellistWWNumxxvleveli{\thelistWWNumxxvleveli.}
\newcommand\labellistWWNumxxvlevelii{\thelistWWNumxxvlevelii.}
\newcommand\labellistWWNumxxvleveliii{\thelistWWNumxxvleveliii.}
\newcommand\labellistWWNumxxvleveliv{\thelistWWNumxxvleveliv.}
\newcommand\labellistWWNumixleveli{[F0B7?]}
\newcommand\labellistWWNumixlevelii{\textstyleListLabelxvi{o}}
\newcommand\labellistWWNumixleveliii{[F0A7?]}
\newcommand\labellistWWNumixleveliv{[F0B7?]}

\newcounter{listWWNumxxvleveli}
\newcounter{listWWNumxxvlevelii}[listWWNumxxvleveli]
\newcounter{listWWNumxxvleveliii}[listWWNumxxvlevelii]
\newcounter{listWWNumxxvleveliv}[listWWNumxxvleveliii]
\newcounter{itemize}  
  \input{../localhyphenation} 
  \togglepaper[1]%%chapternumber
}{}

\begin{document}
\maketitle 


\section{Introduction}
\label{sec:stuhl:1}

One of the lesser-studied varieties of German is spoken in Namibia, in the south-west of Africa. As the result of Namibia being a former German “settler colony”, there is still a vital German-speaking community of roughly 20,000 people living in Namibia today. While most other German varieties outside of Europe have declining speaker numbers, the German-speaking community in Namibia maintains its language through education, media, and social activities. Its vitality makes this extra-territorial variety of German so remarkable (see, e.g., \citealt{wiese_german_2017}).

While there have been some analyses of this variety in the last few decades, there has never been a large database with transcribed speech of Namibian German until now. The creation of the corpus \textit{Deutsch in Namibia} (\textit{DNam}) changed that (cf. \citealt{zimmer_korpus_2020}). 

In the previous research on Namibian German, phonology has typically been dismissed as simply being Standard German and therefore not as interesting as lexical and grammatical features (see below). If anything, it has been said to sound somewhat “Northern German”, which has largely been attributed to the fact that many of the colonists came to Namibia from Northern Germany \citep[564]{bohm_deutsch_2003}. 

The area known as Namibia today was claimed as a German colony from 1884 until 1915 (“German South West Africa”).\footnote{It is important to note that the colony is not the only source of settlement from German-speaking Europe to Namibia, albeit the largest and earliest one. Settlement into Namibia has continued ever since.} As \citet{zimmer_siedlungsgeschichte_nodate} shows, colonists from all German-speaking areas of Europe migrated to this colony, resulting in dialect mixing. A questionnaire study with 157 Namibian German participants shows that, while colonists came from all different language regions, 50\% of participants report their family as coming from a Low German area, i.e. from the northernmost regions of the German-speaking area in Europe (\citealt{zimmer_siedlungsgeschichte_nodate}; see also \citealt{zimmer_wie_nodate}). However, this does not mean that the colonists from those areas spoke Low German – in fact, it is very likely that all colonists spoke not only their regional dialect, but also what has been called “landschaftliches Hochdeutsch” \citep{schmidt_sprachdynamik_2011}. This term, roughly translated as ‘regional High German’, describes prestige varieties in which speakers directly oralised standardised written German. \citet[25]{ganswindt_landschaftliches_2017} states that around the end of the 19th century and the beginning of the 20th century, i.e. the time of the colony German South West Africa, almost every German-speaking person would have been able to speak some version of \textit{regional High German}. These versions of \textit{regional High German} differed from the dialects, but still varied greatly from region to region.

Now that a corpus is available, research on the phonology of Namibian German is more accessible than before and we have decided to re-evaluate the claims in the previous literature. Does Namibian German share phonological characteristics with Northern German?

\section{Namdeutsch and its phonology}
\label{sec:stuhl:2}

Today, the German speaking community in Namibia consists of approximately 20,000 speakers and is remarkably vital. The community uses different terms for its language, including \textit{Südwesterdeutsch}\footnote{This term is derived from \textit{Deutsch-Südwestafrika} (‘German South West Africa’, the name given to the former German colony).}, \textit{Namlish}, and \textit{Namsläng}. We have decided to use the more neutral term \textit{Namdeutsch}. This refers to the language generally spoken in everyday life by Namibian Germans. This study focuses on the phonology of Namdeutsch; for descriptions of the history and structure of German in Namibia see, e.g., \citet{nockler_sprachmischung_1963}; \citet{putz_sudwesterdeutsch_1991, putz_official_1995}; \citet{gretschel_status_1995}; \citet{schmidt-lauber_verkehrte_1998}; \citet{bohm_deutsch_2003}; \citet{shah_german_2007}; \citet{deumert_namibian_2009, deumert_settler_2018}; \citet{ammon_stellung_2014}; \citet{riehl_sprachkontaktforschung_2004}; \citet{kellermeier-rehbein_namslang_2015};
\citet{wiese_deutsch_2014, wiese_german_2017, Wiese_coherence}; \citet{duck_namibia_2018}; \citet{kroll-tjingaete_strukturelle_2018}; \citet{shah_german_2018}; \citet{stolz_weg_2018}; \citet{zappen-thomson_stutzung_2019}; \citet{zimmer_deutsch_2019, zimmer_kasus_2020, zimmer_wie_nodate, zimmer_siedlungsgeschichte_nodate, zimmer_sprachliche_nodate}; \citet{wiese_registerdifferenzierung_2021}; \textcitetv{chapters/05_Bracke}; \textcitetv{chapters/06_Radke}.

So far, no thorough analysis of Namdeutsch phonology has been conducted. Where pronunciation is mentioned at all, authors typically suggest that no difference exists between Namdeutsch and Standard German (\citealt[128]{nockler_sprachmischung_1963}; \citealt[464]{putz_sudwesterdeutsch_1991}). \citet[23]{shah_german_2007} excludes phonological features from her description of Namdeutsch, both because of a lack of previous research and due to her “own observation that they are not nearly as striking as morphological, syntactic and lexical” differences. \citet[564]{bohm_deutsch_2003} states that a large number of German colonists in Namibia came from Northern Germany and therefore Namdeutsch pronunciation is similar to Standard German – this claim stems from the assumption that Standard German is the typical variety spoken in Northern Germany (cf. \citealt{clyne_german_1995}: 29). While this appears to be true especially in comparison with other areas of Germany with regiolects that differ more strongly from Standard German, Northern German does have unique phonological properties (cf. \citealt{elmentaler_norddeutscher_2015}). 

\citet[114]{riehl_sprachkontaktforschung_2004} mentions an overall Northern German pronunciation for Namdeutsch. In the same paragraph, she makes the point that the alveolar trill [r] appears in Namdeutsch, which is unusual due to its general similarity to Northern German. \citet[301]{elmentaler_norddeutscher_2015}, however, find that while the alveolar trill [r] is not very common in Northern German anymore, its use was more widely spread in the area around the turn of the century. Therefore, the existence of [r] does not dispute the idea that Namdeutsch could share phonological characteristics with Northern German – if anything, it might confirm it, considering that [r] appears to have been more common around the time the colonists moved to Namibia.\footnote{Note, however, that [r] is extremely rare in the \textit{DNam} corpus (apart from code-switches to Afrikaans, \citealt{zimmer_sprachliche_nodate}).}

Kellermeier-Rehbein states that, aside from the pronunciation of non-German words, Namdeutsch has what she metaphorically calls a Northern German tinge, a “norddeutsche Färbung” \citep[49]{kellermeier-rehbein_namslang_2015}. She mentions two features as examples: the spirantisation of word-final \nobreakdash-\textit{g}, such as [kʁɪçt] for \textit{kriegt} (‘gets’), and unvoiced plosives that did not change in the High German consonant shift, such as [kɔp] for \textit{Kopf} (‘head’). Her examples were taken from material (e.g. video clips) produced by the Namibian German musician EES \citep[48]{kellermeier-rehbein_namslang_2015}.

\section{Data and Methodology}
\label{sec:stuhl:3}

For this study, we used the corpus \textit{Deutsch in Namibia} (\textit{DNam}, \citealt{zimmer_korpus_2020}), which consists of roughly 225,000 tokens of spoken data. It features three types of recording settings (the recounting of a story in both a formal and an informal setting, free speech, and sociolinguistic interviews). The data was collected in Namibia in 2017 and 2018 and subsequently transcribed, tagged, and compiled in the corpus. Detailed metadata on the speakers is available as well. The entire corpus was used for this study. 

The basis for our comparison with Northern German is the \textit{Norddeutscher Sprachatlas} (‘Northern German Language Atlas’, \citealt{elmentaler_norddeutscher_2015}) and the variables included in this study have been selected from the Northern German variables listed in this publication. Not all variables could be included in this study due to the nature of the data in the \textit{DNam} corpus: The recordings, while of good quality and transcribed well, were not made with the intention of phonetic analysis. This limits the number of features that occur in the data and only variables that naturally occur in the data frequently enough to allow for statistical analyses could be included. Another important aspect that limits our choice of variables, besides availability in the corpus, is how distinctive that feature is in fast, fluent speech. Some features which may be identifiable easily with recordings of word lists are more difficult to correctly identify in fast speech. It was therefore necessary to choose only features that can be clearly identified without a phonetic analysis using software.\footnote{Detailed analyses with Praat were impossible due to the nature of the recordings, as there is background noise of varying volume in most recordings and the group recordings are often too busy to single out specific utterances for a software-based phonetic analysis.}

The methodology of this paper typically follows that of the \textit{Norddeutscher Sprachatlas} (\citealt{elmentaler_norddeutscher_2015}). One significant difference is that non-standard pronunciations limited to single words were excluded from the data, as the focus lies only on variables that occur systematically. Furthermore, all lexical borrowings from the contact languages in Namibia were excluded from the data. 

This article frequently uses the term “Northern German variant”, which is a very broad term. This is intentional. Language use is too varied to easily be able to condense the varieties in all of Northern Germany into one easily described “Northern German”. Therefore, for the purpose of this analysis, we define “Northern German variant” as follows: a variable that occurs at least somewhat frequently in large areas of Northern Germany, i.e. areas in which Low German was/is spoken. While the variable can also occur outside of Northern Germany, it is nonetheless recognised – both by linguists and by speakers themselves – as being a characteristic feature of spoken Northern German, even if not exclusively.

The following section discusses the variables we have chosen for this study: vowel length, raising of long \textit{ä}, a plosive in word-final \textit{-ng}, spirantisation of word-final \textit{-g}, lenition of intervocalic \textit{p, t, k}, and assimilation of intervocalic \textit{nd} and \textit{ld}. Each subsection first describes the variable, followed by its distribution in Northern Germany. Next, we analyse the variable in Namdeutsch using the \textit{DNam} corpus and discuss the influence of sociolinguistic factors.

\section{Analysis}
\label{sec:stuhl:4}
\subsection{Vowel length}
\label{sec:stuhl:4.1}

The realisation of short vowels where long ones would occur in Standard German (see \ref{ex:stuhl:1}--\ref{ex:stuhl:2}), particularly [aː], [oː], [uː], and [iː], is especially prevalent in Northern Germany and has been attributed to Low German (\citealt[135]{martens_niederdeutsch-bedingte_1988}), as Low German has short vowels where High German does not (\citealt[141]{elmentaler_norddeutscher_2015}).\footnote{Here and in the following example pairs, the standard-divergent variant (e.g. {[ʁat]}) precedes the standard variant (e.g. {[ʁaːt]}).} \citet[141]{elmentaler_norddeutscher_2015}, however, note that the realisation of short vowels cannot be traced back only to Low German, as this feature also occurs in other regiolects.

\ea\label{ex:stuhl:1}
\textit{Rad} (‘wheel’): {[ʁat]} vs. {[ʁaːt]} 
\ex\label{ex:stuhl:2}
\textit{Zug} (‘train’): {[tsʊx]} vs. {[tsuːk]} 
\z
 
 

While the feature occurs freely in certain phonetic environments, it is nonetheless connected to specific lexemes quite often, with different lexemes being produced with different vowel lengths in different areas of Northern Germany (\citealt{kleiner_atlas_2011}\footnote{See \url{http://prowiki.ids-mannheim.de/bin/view/AADG/RadQuant} (28 July, 2020).}; \citealt[144]{elmentaler_norddeutscher_2015}; \citealt[65]{kleiner_duden_2015}). For these lexemes, large regional differences exist within Northern Germany and their realisation with a long vowel can in fact be the “Northern German variant”, meaning that the absence of a short vowel does not always equal “not Northern German”. We wanted to quantify the data as broadly as possible. However, a line had to be drawn. Therefore, only short vowels have been included, with no lexeme-specific analyses.

Despite the existence of this feature in other varieties of German, it is nonetheless a defining vocalic feature of the Northern German regiolect. It occurs in the entirety of Northern Germany and is overall fairly common with varying frequencies of at least 20\% and up to 60\%, depending on the region \citep[144]{elmentaler_norddeutscher_2015}. \citet[142]{elmentaler_norddeutscher_2015} state that, while short vowels in place of long ones do exist in other varieties of German as well, they tend to exist primarily in specific lexemes (e.g. \textit{Obst} ‘fruit’ realised as [ʔɔpst] in Southern Germany rather than [ʔoːpst]). 

\begin{sloppypar}
We systematically analysed vowel length in monosyllabic lexemes in the \textit{DNam} corpus, following the methodology used by \citet[144]{elmentaler_norddeutscher_2015}.\footnote{All lexemes ending in \textit{-it} and \textit{-ik} were excluded from the data, as the regional distribution of long and short vowels is different here than for all other phonetic contexts (see, e.g., \citealt{elspas_atlas_2003}, \url{https://www.atlas-alltagssprache.de/politik/} (11 February, 2020); \citealt[151--152]{elmentaler_norddeutscher_2015}).} Overall, these vowels were realised as short vowels 40.4\% of the time in a total of 535 utterances.
\end{sloppypar}

Phonological context plays a role in the occurrence of variables and two variables in particular co-exist frequently. \citet[77]{elmentaler_varietatendynamik_2008} finds that vowel length does not generally seem to be connected to the quality of the following consonant, except in one case: the relation of vowel length to the spirantisation of the following word-final -\textit{g} (see \ref{ex:stuhl:2}). He finds that if the vowel before -\textit{g} is realised as short vowel, -\textit{g} is frequently realised as a fricative – either [x] or [ç] depending on the phonetic environment. However, this does not mean that each spirantisation of -\textit{g} is preceded by a short vowel \citep[77]{elmentaler_varietatendynamik_2008}. The realisation of short vowels before [x] and [ç] mirrors the historical development of German to some degree, as long vowels shortened over time in positions before consonant clusters, especially before [xt] \citep[152]{szulc_historische_1987}.\largerpage

In order to test whether the phonetic context and/or sociolinguistic variables (i.e. \textsc{gender} and \textsc{age}) have an impact on the realisation of word-final -\textit{g} in Namdeutsch, a binomial generalised linear mixed model (GLMM) was fitted (see, e.g., \citealt[278--284]{baayen_analyzing_2008}).\footnote{Our hypotheses behind the sociolinguistic variables were that younger speakers and male speakers (and especially male adolescents) might deviate more strongly from Standard German (see also \citetv{chapters/05_Bracke} on these aspects). These hypotheses hold for all variables studied here.} \textsc{speaker} was integrated as a random effect, which ensures that idiosyncratic behaviour of individual speakers does not skew the results. The first version of the model also contained an interaction term for the two sociolinguistic variables.\footnote{Model specification: vowel\_length {\textasciitilde} gender*age + phonological context + (1{\textbar}speaker). The software R (\citealt{r_core_team_language_2019}) and RStudio (\citealt{rstudio_team_rstudio_2020}) were used for this and all subsequent analyses in this paper. For GLMMs, the package \texttt{lme4} was used \citep{bates_fitting_2015}.} Subsequently, all variables that do not significantly improve the quality of the model were identified and removed. Whilst the phonological context proved to be relevant, the sociolinguistic variables (including the interaction term) did not. Hence, the final version of the model only contains \textsc{phonological\_context} as fixed effect and \textsc{speaker} as a random effect (see \tabref{tab:stuhl:1}).

\begin{table}
\begin{tabular}{l SSS r}
\lsptoprule
 & {Estimate} & {SE} & {$z$} & \multicolumn{1}{c}{$\text{Pr }  (> |z|)$}\\
 \midrule
(Intercept) & 0.9764 & 0.2191 & 4.457 & $<0.001$ ***\\
\midrule
\multicolumn{5}{c}{\textsc{phonological} \textsc{context} (reference level: other\_contexts)}\\
word\_final\_-\textit{g.}plosive & 1.4752 & 0.3245 & 4.546 & $<0.001$ ***\\
word\_final\_-\textit{g.}fricative & -3.9236 & 0.4313 & -9.097 & $<0.001$ ***\\
\lspbottomrule
\end{tabular}
\caption{Results of a GLMM (vowel length)\label{tab:stuhl:1}}
\end{table}

Although the model only contains one fixed effect, it explains a substantial proportion of variance (marginal r\textsuperscript{2} = 0.538; conditional r\textsuperscript{2} = 0.566) and discriminates well (C = 0.938). 87.4\% of all observations are correctly predicted by the model (this rate is significantly higher than the \textit{no information rate}; $p > 0.001$***). Multicollinearity is no problem as all \textit{variation inflation factors} (\textit{VIF}s) are below~2.\largerpage[-2]

These values as well as \tabref{tab:stuhl:1} show that the realisation of the vowel is highly dependent on the phonological context: A final -\textit{g} (realised as a fricative) usually co-occurs with a short vowel, a final -\textit{g} (realised as a plosive) with a long vowel. This mirrors Northern German \citep[77]{elmentaler_varietatendynamik_2008}.

Parallels can also be found as regards the overall frequency: Short vowels appear in place of long vowels fairly frequently in Namdeutsch, as they do in Northern German. With a frequency of 40.4\%, they are about as common in Namdeutsch as they are in Northern German \citep[144]{elmentaler_norddeutscher_2015}. Hence, it can be said that this Northern German feature exists in Namdeutsch.

\subsection{Raising of long \textit{ä}}
\label{sec:stuhl:4.2}

Possibly the most common vocalic feature of Northern German is the raising of the long \textit{ä} from the Standard German [ɛː] to [eː] (see \ref{ex:stuhl:3}--\ref{ex:stuhl:4}). The \textit{Duden Aussprachewörterbuch} states that a pronunciation of \textit{ä} as [eː] is usual in Northern and Eastern Germany as well as in Eastern Austria (\citealt[64]{kleiner_duden_2015}). Moreover, the \textit{Duden} finds that the use of [eː] instead of [ɛː] is common in media as well, excluding traditional news broadcasts \citep[65]{kleiner_duden_2015}. Data by the \textit{Atlas zur Aussprache des deutschen Gebrauchsstandards} (‘Atlas for the pronunciation of the used German standard’) shows that the raised variant is common not only in Northern Germany but also in all of Austria and parts of Bavaria \citep{kleiner_atlas_2011}.\footnote{See \url{http://prowiki.ids-mannheim.de/bin/view/AADG/LangAE} (28 July, 2020).} \citet[104]{elmentaler_norddeutscher_2015} find the feature to be particularly common with a usage of 70--100\% in most of Northern Germany. 

 

\ea 
\label{ex:stuhl:3}
{\textit{Käse}} (‘cheese’): {[keːzə]} vs. {[kɛːzə]} 
\ex
\label{ex:stuhl:4}
{\textit{Mädchen}} (‘girl’): {[meːtçən]} vs. {[}mɛːtçən{]}
 \z

As the raised pronunciation of long \textit{ä} as [eː] rather than [ɛː] is a very common feature of Northern German, we decided to investigate whether the pronunciation of long \textit{ä} is also raised in Namdeutsch. The analysis revealed that the long \textit{ä} is overwhelmingly pronounced as the Northern German variant [eː], with an overall frequency of 97.2\% in a total of 575 occurrences.

This feature, more than any other, is almost exclusively produced in the Northern German variant -- independent of sociolinguistic variables: In a GLMM with \textsc{age}, \textsc{gender}, and the according interaction term as fixed effect and \textsc{speaker} as random effect, no variable turned out to have a significant impact on the vowel realisation.\footnote{The GLMM was calculated according to the procedure described above. To avoid redundancy, the procedure will not be explained in detail again.} 

The data shows that the realisation of long \textit{ä} as [eː] is clearly the typical pronunciation in Namdeutsch, with barely any realisations of [ɛː]. In Northern Germany, the raised variant [eː] is similarly common in most areas, with a typical frequency of over 90\% (\citealt{elmentaler_norddeutscher_2015}: 103). 

Some of the literature on this feature suggests that long \textit{ä} is more frequently realised as [ɛː] in more formal contexts (cf. \citealt{stearns_contemporary_1979}: 151; \citealt{konig_atlas_2019a}: 45), while other researchers find register not to be an important factor in the realisation (cf. \citealt{herrmann-winter_studien_1979}: 141; \citealt{elmentaler_norddeutscher_2015}: 104). \citet[106]{elmentaler_norddeutscher_2015} find that formality plays no role and this feature does not appear to be very salient for Northern German. Our data for Namdeutsch is in line with this, as none of the realisations of [ɛː] were produced in a formal context. In sum, the pronunciation of long \textit{ä} behaves in Namdeutsch just as it does in Northern German and social factors do not make a difference in the distribution of the variants.

\subsection{A plosive in word-final \textit{-ng}}
\label{sec:stuhl:4.3}

Typically, <ng> is pronounced as [ŋ] in Standard German (\citealt{elmentaler_norddeutscher_2015}: 357; \citealt{kleiner_duden_2015}: 68). \citet[233]{konig_atlas_2019b} finds regional realisations of <ng> as [ŋk] in Northern Germany, particularly for lexemes ending in \textit{-ung} (see \ref{ex:stuhl:5}). The \textit{Duden Aussprachewörterbuch} mentions the variant [ŋk] as a possible feature of Northern German and more rarely of Eastern Austrian regiolects (\citealt{kleiner_duden_2015}: 68); this is supported by data from the \textit{Atlas zur Aussprache des deutschen Gebrauchsstandards}, which shows a higher concentration of the realisation of <ng> with a plosive in Northern Germany, the Rhine-Main region, and Eastern Austria (\citealt{kleiner_atlas_2011}).\footnote{See \url{http://prowiki.ids-mannheim.de/bin/view/AADG/LangAE?topic=NgAuslautundvorl} (28 July, 2020).}

 
\ea\label{ex:stuhl:5}  
\textit{Erfahrung} (‘experience’): {[ʔɛɐ̯ˈfaːʁʊŋk]} vs. {[ʔɛɐ̯ˈfaːʁʊŋ]}
 \ex\label{ex:stuhl:6}    
\textit{Ding} (‘thing’): {[dɪŋɡ]} vs. {[dɪŋ]}
 \z
 

The pronunciation of <ng> as [ŋk] (or, in some cases, [ŋɡ] -- see \ref{ex:stuhl:6}) is a variant of most dialects in Northern Germany and as such common in most areas \citep[361]{elmentaler_norddeutscher_2015}. \citet[359]{elmentaler_norddeutscher_2015} find that the plosive is realised in all of Northern Germany, except a small area in the very south-west. Furthermore, the plosive is realised more frequently in the southern area of Northern Germany.

\citet[83]{becker_ruhrdeutsch_2003} hypothesises that the realisation of word-final \textit{-ng} might be part of an ongoing change in Northern German. She found that, while overall older speakers have a higher plosive frequency, younger speakers produced a plosive more frequently in formal situations. This suggests that for younger speakers, the realisation of a plosive for \textit{-ng} might be perceived as the standard \citep[83]{becker_ruhrdeutsch_2003}. Other authors have also found an increase in the realisation of the plosive through diachronic comparisons and apparent-time studies (\citealt{stellmacher_studien_1977}; \citealt{lameli_standard_2004}). \citet[362]{elmentaler_norddeutscher_2015}, on the other hand, have found that in a comparison of their data with that of \citet{pfeffer_grunddeutsch_1975}, the use of the plosive appears to have decreased.

In the \textit{DNam} corpus, <ng> is realised with a plosive in 8.7\% of all cases, or 44 out of 507 utterances. Thus, while the plosive variant exists in Namdeutsch, its frequency is quite low. The distribution of this variant is not dependent on gender or age as a GLMM with these variables (and the according interaction term) shows. 

The rough distinction of Northern German variant and Standard German variant leaves the question of the nature of the plosive open, therefore we decided to analyse the distribution of [ŋk] and [ŋɡ] within the corpus. As the total number of plosives within the data is fairly small, the sample is not very large, but the result is interesting nonetheless: The variant that would be less expected due to the influence of final obstruent devoicing, [ŋɡ], is used more frequently in Namdeutsch – it occurs in 72.7\% of all realisations of a plosive in word-final \textit{-ng} with none of the sociolinguistic variables having a significant impact.

The \textit{Norddeutscher Sprachatlas}, as well as other literature, mentions only [ŋk] as a plosive realisation of \textit{-ng}. The fact that [ŋɡ] is more common in our data than the devoiced [ŋk] is a particularly interesting find, as it leads to questions concerning final obstruent devoicing in Namdeutsch in general. It would be interesting to determine through a systematic analysis whether final obstruent devoicing is also absent in other phonetic contexts and to analyse whether this might be a result of language contact, particularly with English. 

\subsection{Spirantisation of word-final \textit{-g}}
\label{sec:stuhl:4.4}

The spirantisation of word-final \textit{-g} is a prevalent consonant feature of Northern German. It is a process in which word-final \textit{-g}, as well as \textit{g} in word-final \textit{-gt}, is realised as a fricative instead of the plosive [k], which would be expected due to final obstruent devoicing (see \ref{ex:stuhl:7}--\ref{ex:stuhl:8}).\footnote{We decided to use the established term “spirantisation of word-final \textit{-g}” although \textit{g} is of course not word-final if followed by a \textit{-t}.} The fricative is typically [x] or [ç]; however, going south towards Middle German regions, it is increasingly realised as [ʃ] and [ɕ] (\citealt{elmentaler_norddeutscher_2015}: 252). For lexemes ending in unstressed \textit{-ig} or \textit{-igt}, the expected Standard German pronunciation is always [ç] (\citealt{kleiner_duden_2015}: 68). 

 
\ea\label{ex:stuhl:7}   
\textit{lag} (‘lay’): {[laːx]} vs. {[laːk]}
 \ex\label{ex:stuhl:8} 
{\textit{aufgeregt}} (‘excited’): {[ʔaʊ̯⁠fɡəʁeːçt]} vs. {[ʔaʊ̯⁠fɡəʁeːkt]}
\z
 
 

This feature is well-documented (see \citealt{elmentaler_norddeutscher_2015}: 251 for a thorough literature review) and has been attested in the entirety of Northern Germany to varying degrees (cf. \citealt[305]{konig_atlas_2019b}; \citealt{elspas_atlas_2003}).\footnote{See \url{http://www.atlas-alltagssprache.de/runde-1/f15a-b/} (28 July, 2020).} According to the \textit{Duden Aussprachewörterbuch}, the spirantisation of word-final \textit{\nobreakdash-g} is common in Northern and Middle Germany as well as the northernmost areas of South Germany (\citealt{kleiner_duden_2015}: 68). Data from the \textit{Atlas der deutschen Alltagssprache} backs this statement: The fricative is used not only in Northern Germany, but also throughout Middle Germany. What is missing from both the \textit{Duden} and the \textit{Atlas der deutschen Alltagssprache}, however, is the frequency in usage. \citet[302--306]{konig_atlas_2019b} finds the fricative almost exclusively in northern areas. While his data includes frequency, it is not a large-scale study and as such not very representative. 

Despite the feature not being exclusively Northern German, it is still an interesting and important part of Northern German speech. For one, it is a frequent feature that occurs in all varieties of Northern German (\citealt[197]{lauf_regional_1996}; \citealt[2113]{mihm_sprachgeschichte_2000}). Additionally, the spirantisation of word-final \textit{-g} is a rather salient feature of Northern German: According to \citet[159]{berend_standard_2008}, it is a regional marker for Northern German. This awareness of the spirantisation of word-final \textit{-g} being a non-standard variant, whether regional or not, also makes it interesting in another regard. As a variable frequently associated with non-standard and colloquial speech, there is a tendency to hypercorrect it, both in speech and in writing (\citealt{rosenberg_berliner_1986}; \citealt{martens_niederdeutsch-bedingte_1988}; \citealt{eichinger_deutsche_2007}). The hypercorrection generally occurs in the phonetic context of unstressed \textit{-ig} and \textit{-igt}, as those are realised with a fricative in Standard German. While the variant with a realisation of [k] is the regiolectal standard in most of Southern Germany as well as Austria and Switzerland (\citealt{elspas_atlas_2003}; \citealt{kleiner_atlas_2011}), it is a hypercorrected form in Northern Germany.\footnote{See \url{http://www.atlas-alltagssprache.de/runde-1/f14a-c/}, \url{http://prowiki.ids-mannheim.de/bin/view/AADG/IgT?topic=IgAuslaut} and \url{http://prowiki.ids-mannheim.de/bin/view/AADG/IgT} (28 July, 2020).}

In the \textit{DNam} corpus, words ending in \nobreakdash-\textit{g} or -\textit{gt} are more often realised with a fricative than with the Standard German plosive (164 vs. 122 tokens).\footnote{Note that all lexemes ending in \textit{-ig} and \textit{-igt} were excluded from the analysis because their Standard German pronunciation is realised with a fricative.} A GLMM shows that the probability of the fricative increases with the age of the speaker. Additionally, this variant is used more often with words ending in -\textit{gt} (compared to words ending in -\textit{g}). \textsc{Gender} and the interaction term (\textsc{age*gender}) do not significantly improve the model quality and were excluded (see \tabref{tab:stuhl:2}).\footnote{Marginal r\textsuperscript{2} = 0.192; conditional r\textsuperscript{2} = 0.409; C = 0.887; 82.4\% of all observations are correctly predicted by the model (this rate is significantly higher than the \textit{no information rate}; $p > 0.001$***). All VIFs are below~2.}

\begin{table}
\begin{tabular}{l SSS l}
\lsptoprule
 & {Estimate} & {SE} & {$z$} & \multicolumn{1}{c}{$\text{Pr } (>|z|)$}\\
 \midrule
(Intercept) & 1.75574 & 0.49009 & 3.582 & $< 0.001$ ***\\
\midrule
\multicolumn{5}{c}{\textsc{age} (numeric variable)}\\
& -0.05880 & 0.01546 & {}-3.804 & $< 0.001$ ***\\
\midrule
\multicolumn{5}{c}{\textsc{final\_ sound} (reference level: -\textit{g})}\\
-\textit{t} & -1.11129 & 0.33810 & -3.287 & $< 0.01$**\\
\lspbottomrule
\end{tabular}
\caption{Results of a GLMM (spirantisation of word-final \textit{-g})\label{tab:stuhl:2}}
\end{table}

During the auditory analysis, it quickly became clear that [k] and [x] or [ç], plosive and fricative, were not the only two options for the realisation of word-final \textit{-g} as we had initially assumed. Instead, another variant appeared in the data: the deletion of \textit{g}. Deletion refers to the complete absence of a realisation of \textit{g}; there is no glottalisation (see \ref{ex:stuhl:9}).

 
\ea
\label{ex:stuhl:9} 
\textit{gesagt} (‘said’): {[ɡəzaːt]} vs. {[ɡəzaːkt]}
\z 
 

This variant is realised less often than the other two, but its frequency is nonetheless noteworthy (see \tabref{tab:stuhl:3}).\footnote{Deletion also occurs, albeit only with a frequency of 4.3\%, in lexemes ending in \nobreakdash-\textit{ig} and \textit{-igt}, which were excluded from this analysis as they are realised with a fricative in Standard German.}

\begin{table}
\begin{tabular}{cccccc}
\lsptoprule
\multicolumn{2}{c}{{Plosive}} & \multicolumn{2}{c}{{Fricative}} & \multicolumn{2}{c}{{Deletion}}\\\cmidrule(lr){1-2}\cmidrule(lr){3-4}\cmidrule(lr){5-6}
$n$ & \% & $n$ & {\%} &  $n$  & \%\\\midrule
{122} & {39.1\%} & 164 &  {52.6\%} & 26 & 8.3\%\\
\lspbottomrule
\end{tabular}
\caption{Spirantisation of word-final \textit{-g} -- overview}
\label{tab:stuhl:3}
\end{table}


 In a GLMM with realisation (either as plosive or as fricative) vs. deletion as levels of the dependent variable, only the phonological context turned out to be relevant, whilst the sociolinguistic variables did not (see \tabref{tab:stuhl:4}). Deletion occurs more often if a word ends in -\textit{t} (as in \textit{gesagt} ‘said’).\footnote{Note, however, that the overall quality of the model is poor. marginal r\textsuperscript{2}= 0.075; conditional r\textsuperscript{2} = 0.186; C = 0.922; the accuracy of the model is not higher than the \textit{no information rate}.}


\begin{table}
\begin{tabular}{l SSS l}
\lsptoprule
 & {Estimate} & {SE} & {$z$} & \multicolumn{1}{c}{$\text{Pr }  (>|z|)$}\\
 \midrule
(Intercept) & -4.7064 & 0.8152 & -5.773 & $< 0.001$ ***\\
\midrule
\multicolumn{5}{c}{\textsc{word\_ final\_ t} (reference level: no)}\\
yes & 2.2538 & 0.6693 & 3.368 & $< 0.001$ ***\\
\lspbottomrule
\end{tabular}
\caption{Results of a GLMM (realisation vs. deletion of \textit{g})\label{tab:stuhl:4}}
\end{table}


Overall, the Northern German fricative variant is the most common option in Namdeutsch. As such, this matches Northern Germany – while not all areas of Northern Germany use the fricative as frequently as the far North and the West, it is nonetheless the most common realisation in many areas (\citealt{elmentaler_norddeutscher_2015}: 261). Surely, this feature plays a large role in the idea that Namdeutsch “sounds Northern”, as noted by \citet[49]{kellermeier-rehbein_namslang_2015}, especially given that this feature is rather salient, at least in Germany (\citealt{elmentaler_norddeutscher_2015}: 269). Namdeutsch seems to differ from other varieties due to the possibility of deleting \textit{g}, particularly when it is followed by -\textit{t}. This variant is clearly the least common one, yet it still occurs 8.3\% of the time. Possibly, this is a unique feature of Namdeutsch.


Furthermore, it is interesting to see that \textsc{age} has a significant impact on the use of the standard-divergent fricative: Older speakers use this variant more frequently. These differences could be interpreted as an apparent time phenomenon, with the standard-divergent variant decreasing in frequency over time as a result of dialect levelling (for dialect levelling in Namdeutsch, see \citealt{zimmer_siedlungsgeschichte_nodate}). It would be interesting to see how this develops and whether a trend from the Northern German variant towards the Standard German plosive might establish itself in the future.


\subsection{Lenition of intervocalic \textit{p, t, k}}
\label{sec:stuhl:4.5}

In German, obstruents typically exist in pairs of voiceless and voiced: [p]--[b], [t]--[d], [k]--[ɡ], [f]--[v], [s]--[z], [ʃ]--[ʒ] \citep[53]{kleiner_duden_2015}. However, voicing is not the only difference between these sounds with similar manner and place of articulation, but the consonants in each pair also differ in duration, intensity, and tenseness \citep[53]{kleiner_duden_2015}. These differences create the fortis-lenis contrast and obstruents are typically sorted into two categories: the fortis consonants [p], [t], [k], [f], [s], [ʃ] and the lenis consonants [b], [d], [ɡ], [v], [z], [ʒ] \citep[53]{kleiner_duden_2015}. 

The realisation of plosives with regards to the fortis-lenis distinction differs strongly across Germany, Austria, and Switzerland. Fortition and lenition occur with different plosives in different positions within syllables and lexemes all across European German speech. One such feature concerns the lenition of \textit{p, t, k} in intervocalic positions (see \ref{ex:stuhl:10}--\ref{ex:stuhl:12}). This feature differs slightly from the others we have chosen for this study, as it occurs not only or not primarily in Northern Germany, but is also frequent in the south of Western Germany and parts of Austria \citep[69]{kleiner_duden_2015}. The \textit{Duden Aussprachewörterbuch} states that within Northern Germany, this feature is limited to the coastal areas. \citet[129]{martens_niederdeutsch-bedingte_1988} and \citet{auer_hamburger_1998}, however, find lenition to be a common feature in Hamburg. \citet[194--195]{auer_hamburger_1998} finds that sociolinguistic factors play an important role in the distribution of this feature, particularly gender and socioeconomic background: Male speakers and speakers of a lower socioeconomic class are more likely to use lenition. \citet[384]{scheel_hamburger_1963} states that lenition is not very salient and speakers tend to produce a lenis form without being aware of it. \citet[219]{elmentaler_norddeutscher_2015} find that lenition is indeed more common in the far north, especially on the border to Denmark, and quite rare in the southern half of Northern Germany. 

 
\ea\label{ex:stuhl:10}  
\textit{Papa} (‘dad’): {[paba]} vs. {[papa]}
\ex\label{ex:stuhl:11}  
{\textit{bitte}} (‘please’): {[bɪd{ə}]} vs. {[bɪtə]} 
\ex\label{ex:stuhl:12}    
{\textit{Brücke}} (‘bridge’): {[bʁʏɡə]} vs. {[bʁʏkə]}
\z 
 

 \largerpage
While this feature is not exclusive to Northern German, but also exists in other German regiolects, it is nonetheless a feature that is very common in the far north of Northern Germany. Therefore, we decided to include this feature in our analysis, particularly because it would be interesting to see if a feature that is so limited geographically within Northern German occurs in Namdeutsch. The risk, of course, is that even if this feature exists in Namdeutsch, it might not originate from Northern German colonists, but could potentially stem from people from another area and with a different regiolectal background. For that reason, we will limit ourselves to describing the findings.

In order to determine whether the lenition of intervocalic \textit{p, t, k} is frequent in Namdeutsch, we analysed 500 hits which were randomly selected from the \textit{DNam} corpus. Overall, lenition of intervocalic \textit{p, t, k} is not very common, with only 4.6\% of 483 hits realised as a lenis.\footnote{Of the 500 hits, 17 were phonetically unclear and subsequently excluded from the analysis.} Neither of the sociolinguistic variables has a significant influence, nor does the difference between the consonants, i.e. bilabial (\textit{p} and \textit{b}) vs. alveolar (\textit{t} and \textit{d}) vs. velar consonants (\textit{k} vs. \textit{g}), which again was revealed by a GLMM. 
\clearpage

Overall, while lenition of intervocalic \textit{p, t, k} does occur, it is not a very common feature in Namdeutsch. 

\subsection{Assimilation of intervocalic \textit{nd} and \textit{ld}}
\label{sec:stuhl:4.6}

One of the features with the least amount of previous research available to be included in this analysis is the assimilation of \textit{nd} and \textit{ld} in intervocalic positions. In this process, the \textit{d} following an \textit{n} or \textit{l} is assimilated; there are different forms of partial assimilation ranging from a slight assimilation to a deletion (i.e. a complete lack of a plosive, see \ref{ex:stuhl:13}--\ref{ex:stuhl:14}).

 
\ea
\label{ex:stuhl:13}  
\textit{Kinder} (‘children’): {[kɪnɐ]} vs. {[kɪndɐ]}
\ex
\label{ex:stuhl:14}  
\textit{Bilder} (‘pictures’): {[bɪlɐ]} vs. {[bɪldɐ]}
\z
 
 

This feature is attested in Northern Germany, particularly in the very north and in the east \citep[349]{elmentaler_norddeutscher_2015}. It mirrors a similar process in Low German, which explains its spread throughout Northern Germany and the lack of available research on it for the south of Germany, where it can be assumed not to occur \citep[349]{elmentaler_norddeutscher_2015}. There is no mention of the feature in either \citegen{konig_atlas_2019} data, the \textit{Atlas der deutschen Alltagssprache} or the \textit{Atlas zur Aussprache des Deutschen Gebrauchsstandards}. Due to a lack of occurrences, \citet[349]{elmentaler_norddeutscher_2015} did not include \textit{ld} in their analysis. Other studies assume \textit{ld} and \textit{nd} to behave similarly (cf. \citealt{scheel_hamburger_1963}; \citealt{schonfeld_sprache_1989}; \citealt{mihm_sprachgeschichte_2000}). \citet[349--350]{elmentaler_norddeutscher_2015} found assimilated forms of \textit{nd} in the entirety of Northern Germany, with higher frequencies of occurrences in the North and East. This shows that, while the variant is often associated with fast speech, that is not the only factor influencing its use, rather geography seems to also play a role in the frequency. Furthermore, while the assimilation occurs most frequently in free speech, it also exists in more careful, formal speech (\citealt{elmentaler_norddeutscher_2015}: 350).

While different degrees of assimilation exist, the deciding factor for this analysis was whether a plosive was clearly audible. Overall, assimilation occurred in 21.9\% of all cases, which accounts for 108 out of 494 hits.

In a GLMM, only the phonological context turned out to be relevant, whilst the sociolinguistic variables did not (see \tabref{tab:stuhl:5}). Assimilation is more likely if an \textit{l} precedes the \textit{d}.\footnote{Again, the model quality is poor: marginal r\textsuperscript{2}= 0.025; conditional r\textsuperscript{2} = 0.027; C = 0.634; The accuracy of the model is not better than the \textit{no information rate}. All VIFs are below~6. The maximum model was kept in order to avoid (near) singularity.}

\begin{table}
\resizebox{\textwidth}{!}{\begin{tabular}{l SSS l} 
\lsptoprule
& {Estimate} & {SE} & {$z$} & $\text{Pr } (>|z|)$\\
\midrule
(Intercept) & -0.165671 & 0.449125 & -0.369 & $> 0.05$ \\
\midrule
\multicolumn{5}{c}{\textsc{Gender} (reference level: male)}\\
female & 0.297108 & 0.471733 & 0.630 & $> 0.05$\\
\midrule
\multicolumn{5}{c}{\textsc{age} (numeric variable)}\\
& 0.009140 & 0.009272 & 0.986 & $> 0.05$\\
\midrule
\multicolumn{5}{c}{\textsc{preceding\_sound} (reference level: \textit{l})}\\
\textit{n} & 1.350504 & 0.362398 & 3.727 & $< 0.001$***\\
\midrule
\multicolumn{5}{c}{interaction term: \textsc{gender}*\textsc{age}}\\
\textsc{gender(}female\textsc{):age} & {}-0.012735 & 0.012756 & {}-0.998 & $> 0.05$\\
\lspbottomrule
\end{tabular}}
\caption{Results of a GLMM (assimilation of intervocalic \textit{nd} and \textit{ld})\label{tab:stuhl:5}}
\end{table}

Overall, the data reveals that assimilation, while not the typical pronunciation, is common nonetheless in Namdeutsch. 

As the literature on the assimilation of intervocalic \textit{nd} assumes intervocalic \textit{ld} to behave similarly (cf. \citealt{scheel_hamburger_1963}; \citealt{martens_niederdeutsch-bedingte_1988}), we will generally include both in our discussion, despite \textit{ld} not being a part of the \textit{Norddeutscher Sprachatlas}. In the \textit{Norddeutscher Sprachatlas}, the frequency of assimilated \textit{nd} ranges from 5\% to 60\% depending on the area; however, many of these occurrences can probably be traced back to assimilation being a feature of fast speech, particularly in areas with a lower frequency (\citealt{elmentaler_norddeutscher_2015}: 349). Nonetheless, a higher frequency in the north and the east of Germany suggests the existence of assimilation as a feature apart from fast speech. Quantifying the speed of speech for each produced instance of \textit{nd} and \textit{ld} was beyond the scope of this study, but from our familiarity with the data we believe that not all instances of assimilation can be traced back to the speed of speech.

\subsection{Co-occurrence of features}
\label{sec:stuhl:4.7}

Due to the regional distribution of some of the features we have chosen, namely the fact that there is regional variation within Northern Germany (e.g. for the lenition of intervocalic \textit{p, t, k}), we decided to analyse whether some features co-occur more frequently with others. For this analysis, we chose the speakers who realise intervocalic \textit{p, t, k} as a lenis, as lenition is tied to a specific region of Northern Germany (mainly the far north coastal areas of Schleswig-Holstein, see \citealt{elmentaler_norddeutscher_2015}: 219) and it occurs rarely enough in our data to allow for a close analysis of all speakers who produced this variant. Of the 16 speakers who produced a lenition, ten are also present in the data for all other features and thus allowed for a comparison. 

All ten speakers produced a raised long \textit{ä}, eight realised short vowels in place of long vowels, and seven produced spirantisations of word-final \textit{-g}. Assimilation of \textit{nd} and \textit{ld} occurred in the speech of six speakers and five out of ten realised word-final \textit{-ng} with a plosive. As this is a small sample of speakers, it is surely not representative, but it does show something interesting: Of the five speakers who realised both a lenition of intervocalic \textit{p, t, k} and a plosive in word-final \textit{-ng}, three are adolescents who go to the same school. In fact, of the six people from that school, five produced lenitions of intervocalic \textit{p, t, k} in the data. 

This group is very interesting, as they are all students of a German-speaking background who go to the same private school with an instructional language other than German (i.e. English or Afrikaans) and meet regularly with a teacher for private lessons in German, which is the setting they were recorded in.\footnote{Interestingly, these students were also found to use transferred lexical items more frequently than those going to schools with German-language instruction (\citetv{chapters/05_Bracke}).} They all produced Northern German variants, some frequently, with most of them even realising the Northern German variants that occur rarely in the data. The most likely explanation in our opinion is the lack of formal German instruction in school: This group consists of the only adolescents in the corpus who do not take the subject \textit{Deutsch als Muttersprache} (‘German as a first language’) at school. It would be interesting to see whether other adolescents without formal instruction in German as a first language might produce similar phonetic features.

Overall, the data from the analysis of co-occurrences is not particularly representative; nonetheless it shows that speakers who produced intervocalic \textit{p, t, k} as a lenis are likely to realise all other Northern German variants. It also opens the question as to how adolescents without formal instruction in German speak Namdeutsch.

\section{Discussion}
\label{sec:stuhl:5}

Does Namdeutsch sound Northern German? While this study only singles out separate areas of Namdeutsch phonology and is by no means a detailed analysis of all phonetic features that define Northern German, we believe that Namdeutsch does “sound Northern”. Not all features we analysed were common; in fact, some barely occurred at all. However, we believe that the Northern German variants that do exist in Namdeutsch are quite common and are often very distinctive features of Northern German. The spirantisation of word-final \textit{-g}, for example, is a defining feature of Northern German that is quite salient \citep[269]{elmentaler_norddeutscher_2015} and is also common in Namdeutsch. The raising of the long \textit{ä}, one of the most common phonetic features of Northern German, is just as common in Namdeutsch as well, with barely any Standard German variants occurring in the data. Northern German variants in vowel length are also quite frequent in Namdeutsch. These three features as a whole do not represent Northern German, but they are quite common and salient overall; they are defining phonetic features that set Northern German apart from Standard German and set Namdeutsch apart from Standard German, as well. 

In saying this, we challenge the common idea that people from Northern Germany speak only Standard German and therefore phonetic variation is irrelevant (cf. \citealt[128]{nockler_sprachmischung_1963}; \citealt[464]{putz_sudwesterdeutsch_1991}). As the \textit{Norddeutscher Sprachatlas} shows, and as many other studies have shown, Northern German does have phonetic variants that differentiate it from Standard German pronunciation, even if they might not be as obvious and as salient as features from other German varieties. 

However, in agreeing that Namdeutsch does, to some degree, sound Northern German, we are not saying that it sounds only Northern German. Beyond the actual scope of our study, we were able to discover some variants that cannot be traced back to Northern German (or, as far as we can currently tell, German in Germany in general), and as such set it apart. Most obviously, this concerns the deletion of \textit{g} in the final syllable, particularly when followed by -\textit{t}, i.e. the deletion of a consonant at the beginning of a consonant cluster. The apocope of -\textit{t} – the elision of \textit{t} at the end of a word – is a feature of Northern German that is also discussed in the \textit{Norddeutscher Sprachatlas} (\citealt[275]{elmentaler_norddeutscher_2015}) and a common example for deletion, but deletion of \textit{g} is not attested in this context. Interestingly, we also found plosive deletion in a feature involving the consonant cluster -\textit{b}(\textit{s})\textit{t} that we had analysed but ultimately excluded from this study (see \ref{ex:stuhl:15}). 

 
\ea
\label{ex:stuhl:15}
\textit{bleibt} (‘stays’): {[blaɪ̯⁠t]} vs. {[blaɪ̯b⁠t]}
\z
 
 
With 14 occurrences out of 446 overall hits it is not very common, but does nonetheless exist. This leads us to question whether perhaps there is a tendency in Namdeutsch to delete plosives in consonant clusters. This question will be explored in further research.

The analysis also revealed the importance of the speaker’s age as a factor regarding the realisation of word-final \textit{-g.} This could potentially be a sign of a language change away from a more Northern sound and towards Standard German -- younger speakers are more likely to realise the Standard German variant. A move towards Standard German would not be implausible, given the historical and social context. When the speakers who are now adults acquired their language, they did so in large part from other speakers of Namdeutsch. Most German language media, like content on the radio stations, was produced by speakers of Namdeutsch, and until 1996, there was no satellite TV from Germany available in Namibia \citep[25]{kroll-tjingaete_strukturelle_2018}. Now, however, Namibian Germans growing up in Namibia have much more contact with Standard German. This is partially due to the available media from Germany, partially due to the fact that travelling to Germany is easier and cheaper now than it was – and many Namibian Germans frequently travel to Germany – and surely also in large part due to the education system. Most of the adolescents in the \textit{DNam} corpus go to German schools, some of which are private and have teachers from Germany. As the data of students going to German schools outweighs the data of those who go to other schools, the influence of Standard German, partially through being taught by teachers from Germany, should not be underestimated. Preliminary studies suggest an increase in the usage of non-standard features in non-German schools, but no quantitative study was possible due to the scarcity of data from these schools.

It remains to be seen whether the tendency of younger speakers to produce more standard-like forms establishes itself in Namdeutsch or not and whether Namdeutsch might in the future sound less Northern than it does now.

\section*{Acknowledgements}
This work was funded by the \textit{Deutsche Forschungsgemeinschaft} (DFG, German Research Foundation) – SI 750/4-1/2; ZI~1793/1-2.

{\sloppy\printbibliography[heading=subbibliography,notkeyword=this]}
\end{document} 
