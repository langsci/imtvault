\addchap{\lsAcknowledgementTitle} 

 \largerpage

I thank the people who have supported me in this course of developing the ideas in this book, and who have created the conditions in which I could pursue this project. Looking back on the whole process, the key conceptual roadblock to the oscillators/energy levels approach has been allowing myself to sufficiently dissociate analyses of \isi{oscillation} and excitation. It took me about ten years, from the time I began this project, to reach the current state. I starting working on a coupled oscillators approach to syntax in graduate school as a teaching assistant for introduction to syntax and semantics. I had been introduced to the Articulatory Phonology use of oscillators in a seminar by Keith Johnson, and was convinced that coupled oscillations are a general, neurally plausible mechanism of cognition. Keith graciously allowed me to deposit in the Working Papers of the Berkeley Phonology Lab a flawed manuscript which represents those early efforts. In a remarkable turn of luck, in the fall of 2009 as a postdoc at the University of Southern California, I found myself sitting in on a seminar co-taught by Louis Goldstein and Jean-Roger Vergnaud. The aim of the seminar was to explore how a general framework for \isi{phonology} and syntax could be developed, based on coupled oscillations. Sadly, Jean-Roger passed away shortly after that seminar. I put a lot of effort into trying to understand syntax with \isi{oscillation} during that period. At a Conference in honor of Jean-Roger, I presented an analysis which was closer to the current theory, but which nonetheless fell short in dissociating \isi{oscillation} and excitation. I am very grateful to Louis Goldstein, Dani Byrd, and Sarah Bottjer for their support during my time as postdoc. When I started as an assistant professor at Cornell University in 2011, I focused on developing a theoretical model of \isi{articulatory} organization. I received enormous personal and professional support from Abby Cohn and Draga Zec, and from the Cornell Linguistics Department generally. This enabled me to develop the selection-coordination theory of speech production, which is much closer to the oscillators/energy levels framework because it treats \isi{oscillation} and excitation as separate phenomena. Many of the specific syntactic analyses in this manuscript were fleshed out in the summer of 2017 and subsequently while I was on sabbatical as a Visiting Scholar in the Oxford University Phonetics Lab. I am very grateful for the support of John Coleman during this sabbatical period.

