\documentclass[output=paper]{langsci/langscibook} 
\author{Marc van Oostendorp\affiliation{Radboud University and The Meertens Institute}}
\title{The Esperantist background of René de Saussure's work} 
\ChapterDOI{10.5281/zenodo.1306494} %will be filled in at production 
\abstract{\noabstract}
\maketitle
\begin{document}
   \label{sec:esperanto}
\lettrine[loversize=0.1, nindent=0.25em]{R}{ené de Saussure was arguably more an esperantist} 
than a linguist --\linebreak somebody who was primarily inspired by his enthusiasm for the language of
L.\ L.\ Zamenhof, and the hope he thought it presented for the world. His
interest in general linguistics seems to have stemmed from his wish to
show that the structure of Esperanto was better than that of its
competitors, and that it reflected the ways languages work in general.

Saussure became involved in the Esperanto movement around 1906,
apparently because his brother Ferdinand had asked him to participate
in an international Esperanto conference in Geneva; Ferdinand himself
did not want to go because he did not want to become ``compromised''
\citep{kuenzli2015}.  René became heavily involved in the movement, as
an editor of the \emph{Internacia Scienca Revuo} (International
Science Review) and the national journal \emph{Svisa Espero} (Swiss
Hope), as well as a member of the \emph{Akademio de Esperanto}, the
Academy of Esperanto that was and is responsible for the protection of
the norms of the language. Among historians of the Esperanto movement,
he is also still known as the inventor of the \emph{spesmilo}, which
was supposed to become an international currency among Esperantists
\citep{garvia2015}.

At the time, the interest in issues of artificial language solutions
to perceived problems in international communication was more
widespread in scholarly circles than it is today. In the western
world, German was often used as a language of e.g. scholarship and
French as a language of diplomacy and it was unclear that German or
French politicians would ever allow the ``other'' language to grow into
an international language for all domains. English was becoming a
little more used as well, but one could not yet have expected it to
attain the position it has now. Furthermore, there was still a strong
belief in ``technological'' solutions to social problems, and therefore
the idea of a specific artificial language to overcome language
barriers did not seem too far-fetched to many \citep{gordin2015}. In
the period until the Second World War, many linguists also became
involved in issues of artificial language creation, such as Otto
Jespersen, Nikolaj Trubetzkoy and Andr\'e Martinet
\citep{vanoostendorp2004}.

Saussure's work on morphology was very closely related to his interest
in Esperanto. This is clear already from a superficial inspection of
the terminology he introduces in the 1911 text. The vowels \textbf{o,
  a, i} which he uses to denote nouns, adjectives and verbs,
respectively, correspond exactly to the desinences for those lexical
categories in Esperanto (see below). A form such as \emph{kron-ag-o},
which Saussure presents as an ``abstraction'', is actually a well-formed
Esperanto word. The idea was that Esperanto very closely mirrored the
structure underlying all languages.

One can therefore understand the 1911 text in particular as an
extension of Saussure's work on Esperanto (which he had presented
earlier in his 1910 book \emph{La construction logique des mots en
  espéranto} \citep{saussur1910}. As Meillet noted in his review of
the 1911 book~(cf. p. \pageref{sec:meillet-review} above), Saussure's
interest in morphology was triggered by his interest in Esperanto word
formation. The 1911 booklet can basically be seen as a way to show
that Saussure's analysis of Esperanto morphology was right, by turning
it into a universal theory. The 1919 text is clearly a similar
introduction to a text in which Saussure presented a (mostly
orthographic) reform of Esperanto.

Saussure's 1910 analysis of Esperanto itself was actually intended as
an argument against Ido, a proposed reform of Esperanto from 1907
introduced by a team led by the French mathematician, Leibniz scholar
and philosopher Louis Couturat \citep{couturat1907, couturat1908}. Ido
was seen by its proponents as an ``improved'' form of Esperanto, and
several of the improvements concerned morphology. Saussure's aim was
to show that the Idists were wrong on this (he published his work
under the pseudonym \emph{Antido}, showing his attitude towards the
language). Since many of the prominent Idists were French
intellectuals, aiming to devise a language that would be more
``natural'', i.e.~more like French, it is interesting that Saussure
showed how the ``Esperanto'' kind of morphology actually underlies
French.

Saussure's 1910 text is still seen as foundational to the study of
Esperanto word formation, as is a later text from 1915,
\emph{Fundamentaj reguloj de la vort-teorio en Esperanto. Raporto al
  la akademio esperantista} (Foundational rules of the word theory in
Esperanto. Report to the Esperanto Academy,
\citealt{saussure1915}). His own relation with the Esperanto movement
was rather unstable, mainly because he presented a large number of
different (again, mostly orthographic) reform proposals of the
language himself, under names such as Esperantido, Nov-Esperanto and
Esperanto II.

In order to understand this more clearly, I will first lay out some of
the basics of Esperanto word structure before turning to the
criticisms raised against it by Louis de Couturat, the mathematician
who proposed Ido instead and Saussure's response to Couturat, as well
as the relevance of that response to the texts which are presented in
the present volume.

\section{The word structure of Esperanto}

Esperanto is a language project, published by L.\ L.\ Zamenhof in 1887
\citep{zamenhof1887}. It is fair to say that morphology was the module
that got the most attention in Zamenhof's language creation work. The
intention was to make an ``international language'' that could serve as
a second language for everybody in the (Western) world who might need
it. The language therefore needed to be simple, and this simplicity
was mostly a matter of morphology. First, there should be no
exceptions, and secondly, the word stock should be as small as
possible, by exploiting the combinatory possibilities of the language.

\begin{figure}[t]
  \begin{center}
    \includegraphics*[0,0][1.78in,2.5in]{./Photos/Zamenhof.eps}
  \end{center}
  \caption{\label{fig:zamenhof}   L{[udwik]} L{[ejzer]} Zamenhof
    (1859--1917)} 
\end{figure}

\newpage In a ``16 rule grammar'' that Zamenhof published of his language project
\citep{zamenhof1905}, five rules seem particularly relevant:

\begin{itemize}

\item 2. Nouns have the ending -\emph{o}. To form the plural, add the
  ending -\emph{j}. There are only two cases: nominative and
  accusative; the latter can be obtained from the nominative by adding
  the ending -\emph{n}. (\ldots)

\item 3. Adjectives end in -\emph{a}. Cases and numbers are as for
  nouns. (\ldots)

\item 6. The verb does not change for person or number. Forms of the
  verb: present time takes the ending -\emph{as}; past time,
  -\emph{is}; future time, -\emph{os}; conditional mood, -\emph{us};
  imperative mood, -\emph{u}; infinitive mood, -\emph{i}. (\ldots)

\item 7. Adverbs can be formed from adjectives by changing the
  -\emph{a} ending to an -\emph{e} ending (\ldots)

\item 11. Compound words are formed by simple juxtaposition of words
  (the main word stands at the end); the grammatical endings are also
  viewed as independent words.

\end{itemize}

Notice that several parts of Saussure's ``universal'' theory of word
formation already find their expression here. I have already noted the
correspondence of the nominal, adjectival and verbal (infinitival)
endings to the names of categories in Saussure's system. Similarly,
adverbs have an ending -\emph{e}, just as in Saussure's system. They
are considered not to be on a par with the three main categories: the
rule for adverbs explicitly says they are derived from adjectives (the
language actually also has adverbs that are not derived from
adjectives, and those end in -\emph{a\v{u}}: e.g. \emph{hiera\v{u}},
`yesterday'). This is also mirrored in Saussure's 1911 text, which
similarly sees adverbs as derived from adjectives.

The idea that grammatical endings \emph{-o, -a, -i} can be viewed as
independent words is also introduced explicitly in Zamenhof's 1905
rules. Although it is not explicitly stated that the same would be
true for other affixes, it seems logically implied that they are. It
would be strange to assume that in a form like \emph{lern-ej-o}
`school' (literally: learn-location-noun), only \emph{lern} and
\emph{o} count as independent words.

As a matter of fact, affixes in Esperanto can be and are often used as
independent words. In the example just given, \emph{ejo}
(etymologically derived from the German/Yiddish locational suffix
-\emph{ei}) can mean `place' and be used as such in this way (\emph{La
  parko estas bela ejo}, `The park is a nice place'). The language has
approximately 45 affixes, all of them productive, because Zamenhof
believed that it would be easier to learn a language when one did not
have to learn too many separate morphemes.

This decision was sometimes criticized, because it means that
Esperanto words are not directly recognizable to people unfamiliar
with the language, even\linebreak though most morphemes are taken from
Indo-European (mostly Romance) languages. For instance a common word
for `hospital' is \emph{mal-san-ul-ej-o}
(opposite-healthy-person-place-noun). Apart from -\emph{ej}, the other
affixes in this form can also be combined with a desinence to form an
independent word (\emph{mal-e} `to the contrary', \emph{ul-o} `guy';
for some reason these words often have an informal flavour to them.)
Even suffixes that have been borrowed after Zamenhof's time are used
like this: the pejorative -\emph{a\^c} suffix (from Italian
-\emph{accio}) can be used as an adjective in \emph{a\^ca} `ugly,
morally corrupt'.

On the other hand, the desinences themselves cannot really be used as
independent words in Esperanto. For instance, the phrase \emph{la o}
is not used for `the thing' as Saussure's theory would perhaps
predict; instead people say \emph{la a\^{\j}o}, where -\emph{a\^{\j}}
is indeed a deadjectival nominalizing suffix (French -\emph{age}).

Replacing one desinence by another, is however, freely admissable: the
Esperanto system allows for conversion between all categories, so
that, e.g., from the noun \emph{krono} `crown' one can also form the
verb \emph{kroni} `to crown'.

\section{Couturat's criticism and Saussure's answer}

The years after 1907 are sometimes referred to as the \emph{Ido
  crisis} in the literature on the history of the Esperanto
movement. In this year, Louis Couturat proposed his reform plan for
Esperanto, which attracted a considerable following in particular
among the intellectual leadership of the young Esperanto movement at
the time. Although the language was still understandable for somebody
who knew Esperanto (and as a matter of fact originally thought of as
an ``improved dialect''), there were changes in the lexicon, the
orthography, the syntax (e.g. there was no longer an obligatory
accusative for direct objects) and the morphology. The claim was that
these changes would make the language more ``natural'' and hence more
palatable for the world which had not accepted Zamenhof's language in
the 20 years after he had presented it.

\begin{figure}[htb]
  \begin{center}
    \includegraphics*[0,0][2.0in,2.0in]{./Photos/Couturat.eps}
  \end{center}
  \caption{\label{fig:couturat.eps}
    Louis Couturat (1868--1914)}
\end{figure}

A bitter fight between Esperantists and Idists ensued. Saussure took
sides with the former group, although he must have agreed with some of
the reform proposals, in particular with respect to orthography
(Esperanto has letters like \emph{\^c, \^s, \^g} which have not always
been easy to typeset). He did not, however, at all share Couturat's
criticism of the morphological system.

In particular, the Idists reject as unnatural the idea that affixes
can be used as independent words. This is somewhat ironic, as
\emph{ido} is itself an independently used suffix for `offspring'
(\emph{bovo} is `cow', \emph{bovido} `calf'); the name of the language
thus denotes that it is `offspring' of Esperanto. However, the active
Idist (and Esperantist) Gon\c{c}alo Neves (p.c.) informs me that the
only suffix that Idists accept as an independent word is the causative
-\emph{ig}: \emph{igar} means `to make (somebody do something)' (the
infinitival ending in Ido is -\emph{ar} rather than -\emph{i}).

Even more importantly for Saussure, \citet{couturat1907} also rejected
conversion, in particular forming a verb from a noun
directly. Couturat believed that this violated a ``principle of
reversability'' (different from Saussure's ``rule of reversal'') stating
that the relation between words from the same stem should always work
in two directions. If one derives a noun from a verb, the meaning of
that should always be a nominalitzation. However, \emph{krono} in
Esperanto is obviously not the nominalitzation of \emph{kroni}; it does
not mean `coronation' but rather `a crown'. (The word for
`coronation' is \emph{kronado}, with an explict deverbal
nominalizing suffix \mbox{-\emph{ad}}; the form \emph{kronago} which Saussure
gives would also be possible, but is not common.)

This problem was solved by essentially banning direct conversion of
nouns to verbs in Ido. `To crown' in that language is
\emph{kron-iz-ar} with a suffix -\emph{iz} denoting `to cover with',
and `coronation' is therefore \emph{kronizo}. This system was
considered more ``logical'' as one always knows how to form a noun from
a given verb, viz. by replacing -\emph{ar} by -\emph{o}, whereas this
is not possible in Esperanto.

Saussure's answer to this is interesting in its own right. He argues
that, since stems without desinences are actually ``independent words''
(according to Zamenhof's rule in the case of Esperanto, or according
to the universal principles in the 1911 and 1919 texts), they also
have their own categorial specification. We can illustrate this the
example of \emph{kombi} `to comb' and \emph{brosi} `to brush'. The
corresponding nouns, \emph{kombo} and \emph{broso} mean `combing' and
`a brush' respectively. The reason for this, according to Saussure, is
that \emph{komb'} is inherently a verb, and \emph{bros'} inherently a
noun. If one wants to say `a comb' one therefore needs to use an
instrumental (i-o) suffix \emph{il} to say \emph{kombilo}; similarly,
in order to say `brushing' one needs to say \emph{brosado}. The suffix
\emph{ad} is \emph{i-o} and therefore turns the nominal \emph{bros'}
into a verb which then gets nominalized (in his 1911 notation we have
\emph{bros-o-i-o}).\footnote{For further discussion of these matters,
  see \citet{vanOostendorp98:economy-in-esperanto}.}

Instead of Couturat's principle of reversability, Saussure proposed
the principles of \textsc{necessity} and \textsc{sufficiency} in his
Esperanto work, which also play a role in the 1911 and 1919 texts as
universal principles. In the cases at hand, the principle of necessity
says that \emph{broso} does not mean `brushing', as an explicit suffix
denoting the verbal meaning inside that word is missing; the principle
of \emph{sufficiency} on the other hand has it that \emph{brosilo} is
not the word for `brush', as \emph{il} is not necessary to express the
instrumental meaning that is already contained in \emph{bros}. The two
principles actually seem to function as Gricean maxims in the sense
that e.g. \emph{brosilo} is a possible word; it just does not mean
`brush' but `something that can be used to brush, but is probably not
a brush' (\emph{Bran\^co estas brosilo} `A branch is an object to
brush').

It seems important that in the two texts presented in the present
volume, Saussure shows how the analysis can be applied to French, as
France was the primary battleground of the Ido crisis, and the main
criticism of the Idists was that the Esperanto system was an
``unnatural'' one. By showing that the Esperanto system was actually
also underlying in French, Saussure tried to rebut this argument. It
is similarly significant that the 1919 text has a motto by Leibniz;
Couturat was an eminent Leibniz scholar and both Couturat and Saussure
subscribed to the Enlightenment idea that all languages have the same
underlying structure, which is given by the way in which thought
functions in all humans, and that all differences are mere layers put
on top of that universal structure by history. Leibniz, Couturat and
Saussure all believed that the ideal ``universal language'' should
reflect the common grammar as much as possible. The difference in
opinion between Couturat and Saussure was whether this universal
grammar contained the principle of reversability or those of necessity
and sufficiency.

\section{Conclusion} \label{sec:esperanto-conclusion}

Initially, Saussure's ideas met with universal acclaim among
Esperantists. Zamenhof (who would die in 1917), for instance, seems to
have approved Saussure's analysis of the morphology. However, Saussure
was soon to become estranged from the Esperantists, mainly because of
his almost endless attempts to reform the language. In 1921, he was
forced to leave the \emph{Akademio} and in 1925 he declared himself to
be no longer an Esperantist (although he would always maintain that he
was a \emph{dis\^ciplo de Zamenhof}. Still, the principles of
necessity and sufficiency were declared to be official rules of
Esperanto grammar by the \emph{Akademio} in 1957
\citep{garvia2015,kuenzli2015}.


\sloppy
\printbibliography[heading=subbibliography,notkeyword=this] 

\end{document}
