\acused{PF} 
\acused{LF}
\acused{EPP}
\acused{TP}
\acused{CP}
\acused{SOV}
\acused{SVO}

\chapter{Further discussion and conclusion}\label{ch:6}
\largerpage[-1]
In this final chapter, I will review two theoretical concepts, Case and word order, to which the discussion included in this monograph contributes. In the previous discussion of \ac{FI} in Chapter \ref{ch:4}, I proposed that structural case is an uninterpretable D-feature on C (for nominative) and \textit{v} (for accusative) which is valued against the D-feature on a noun phrase. In addition, [\textit{u}D] on C and \textit{v} are parameterized for \ac{EPP} specifications and may be inherited by T and \ac{ASP}, respectively. In the following, I will show how the presence or the absence of overt case morphology in Korean-English and Japanese-English \ac{CS} can be explained by the proposal that structural case is an uninterpretable D-feature on \textit{v}, which is also compatible with cross-linguistic data (\sectref{ch6:sect:case}). I explore \ac{FI} in the C-T domain in Korean and Japanese, in which \ac{CP} is pied-piped and raises to Spec, \ac{CP}. Not only is the proposal of \ac{CP} pied-pipng and raising in the C-T domain entirely parallel to \ac{ASP}P pied-piping and raising in the \textit{v}{}-\ac{ASP} domain (Chapter \ref{ch:4}), but it also accounts for consistent head-final structure in these languages. Section \sectref{ch6:sect:conclusion} concludes the monograph.

\section{Case}\label{ch6:sect:case} 

In many languages, Case is morphologically marked. For instance, nominative Case and accusative Case are marked in \textit{i/ka} and \textit{{}-(l)ul} in Korean and \textit{{}-ga} and \textit{-o} in Japanese, respectively, shown in \REF{ex:142}.

\begin{exe}\ex\label{ex:142}
\begin{xlist}
    \ex \gll Bibi-ka    Joa-wa   hapse-ul      hayss-ta \\
    Bibi-\textsc{nom} Joa-with table-sharing-\textsc{acc} do\textsc{.past-decl} \\ 
    \hspace*{\fill} Korean
    \ex \gll Bibi-ga    Joa-to     aiseki-o      shita \\
    Bibi-\textsc{nom} Joa-with table-sharing-\textsc{acc} do\textsc{.past} \\
    \hspace*{\fill} Japanese
    \glt ‘Bibi shared the table with Joa.’
\end{xlist}
\end{exe}

\hspace*{-1mm}In generative syntax, structural case (or Case), such as nominative or accusative case, is generally assumed to be an uninterpretable feature, [\textit{u}F], of noun phrases, which needs to be valued against a probe, nominative for T and accusative for~\textit{v} (\citealt{Chomsky1995,Chomsky2001}). Assuming that Case is an uninterpretable D-feature on phase heads C (for nominative) and \textit{v} (for accusative) and a head in the nominal extended projection (most likely D) bears a D-feature, I propose that the nominative case is a morphological manifestation of Agree between T and the subject/external argument and the accusative case is a morphological display of Agree between \ac{ASP} and the object/internal argument, as depicted in \REF{ex:143}.

\begin{exe}\ex\label{ex:143}
\begin{tabular}[t]{lc}
a.  & Nominative case   \\
   &  \scalebox{0.99}{\begin{forest} 
   [CP, s sep=5mm 
   [C \\ {[\colorbox{lightgray}{\textit{u}$\upphi$, \textit{u}T, \textit{u}D}]}, name=src       
   ][TP[T \\  {[\textit{u}$\upphi$, \textit{u}T, \textit{u}D]}, name=tgt
   ][\textit{v}P[\acs{SUB} \\ {[$\upphi$, D]}
   ][~~]]]] 
   \draw[->] (src) to[out=south,in=south] (tgt);
   \end{forest}}
   \\
   \tablevspace
   & Agree between T and \acs{SUB} \\ \\
   b. & Accusative case \\
   & 
   \scalebox{0.99}{\begin{forest} 
   [\textit{v}P [\textit{v} \\ {[\colorbox{lightgray}{\textit{u}$\upphi$, \textit{u}Asp, \textit{u}D}]}, name=src  ]
   [\ac{ASP}P [\ac{ASP}P \\ {[\textit{u}$\upphi$, \textit{u}Asp, \textit{u}D]}, name = tgt ] 
   [VP[V][\acs{SUB} \\ {[$\upphi$, D]} ]]]] 
   \draw[->] (src) to[out=south,in=south] (tgt);
   \end{forest}} \\
 & Agree between \ac{ASP} and \acs{OBJ}   
\end{tabular}
\end{exe}

\largerpage[-1]
Interestingly, it seems that overt case morphology on the object is possible only in \ac{OV}-\textit{ha/su} order, but not in \ac{VO}-\textit{ha/su} order in Korean-English and Japanese-English \ac{CS}: for instance, the direct object is overtly marked with the Korean accusative case marker \textit{{}-lul} and the Japanese accusative case marker \textit{{}-o} in \REF{ex:144} whereas the English direct object is not morphologically marked for case in \REF{ex:145}.\footnote{Some of the informants expressed the idea that an insertion of the accusative case particle immediately following the object of the light verb in \REF{ex:145} is not, strictly speaking, impossible, as shown in \REF{exfn78:i} below. 
\begin{exe}\ex\label{exfn78:i}
\begin{xlist}
    \ex \gll \textit{cenyek} \textit{mek-ko} take a walk-\textit{lul}   \textit{hal-kka}?   \\
    dinner eat-and {} {} {~~~~~~~}-\textsc{acc} \textsc{do}\textsc{.fut-coh} \\
    \hspace*{\fill} \textit{Korean}-English  
    \glt ‘Shall we take a walk after dinner?’
    \ex \gll \textit{yooshi}         keep an eye-\textit{o}    \textit{suru-zo} \\
    {well I’m going to} {} {} {~~~~}-\textsc{acc}  do-\textsc{prt}    \\
    \hspace*{\fill} \textit{Japanese-}English 
    \glt  ‘Well, I’m really going to keep an eye on you.’
\end{xlist}
\end{exe}
But if this happens, what is marked with the accusative case is not the object itself but the entire string [\textit{take a walk}] or [\textit{keep an eye}] — in other words, \textit{ha/su} `do' is used as a ``heavy'' verb in \REF{exfn78:i} and assigns accusative case to the nomilaized verbal complement [\textit{take a walk}] in (\ref{exfn78:i}a) and [\textit{keep an eye}] in (\ref{exfn78:i}b). This does not affect the proposal made here.}

\begin{exe}
\ex\label{ex:144}
\begin{xlist}
    \ex \gll too much money-\textit{lul}   spend \textit{hayss-eyo}      \\
    {} {} {~~~~~~~~}-\textsc{acc} {} \textsc{do}.\textsc{past}-\textsc{decl} \\
    \hspace*{\fill} \textit{Korean}-English
    \glt ‘He spent too much money.’
    \ex  \gll one algebra question-\textit{o}     mark \textit{site}      \\
     {} {} {~~~~~~~~~~~}-\textsc{acc} {} \textsc{do} \\
     \hspace*{\fill} \textit{Japanese}-English
    \glt   ‘You mark one algebra question.’
\end{xlist}
\ex\label{ex:145}
\begin{xlist}
     \ex \gll \textit{cenyek} \textit{mek-ko} take a walk   \textit{hal-kka}?   \\
    dinner eat-and {} {} {} \textsc{do}\textsc{.fut-coh} \\
    \hspace*{\fill} \textit{Korean}-English  
    \glt ‘Shall we take a walk after dinner?’
    \ex \gll \textit{yooshi}  keep an eye   \textit{suru-zo} \\
    {well I’m going to} {} {} {}  do-\textsc{prt}    \\
    \hspace*{\fill} \textit{Japanese-}English 
    \glt  ‘Well, I’m really going to keep an eye on you.’
\end{xlist}
\end{exe}

\largerpage[-1]
The presence of case marking in \REF{ex:144} and the absence of case morphology in \REF{ex:145} can be explained by the proposal that the accusative case marking on the object is a morphological indication of Agree between \ac{ASP} and the internal argument: the accusative case marker in \REF{ex:144} is a reflection of D-feature matching between \ac{ASP} and the object, as depicted in \REF{ex:146}.

\begin{exe}\ex\label{ex:146}
\begin{xlist}
\ex \begin{forest}
[\textit{v}P [\textit{v} \\  \textit{ha}\textsuperscript{KR}/\textit{su}\textsuperscript{JP} \\ {[\colorbox{lightgray}{\textit{u}$\upphi$}, \textit{u}Asp, \colorbox{lightgray}{\textit{u}D\textsuperscript{EPP}}]}, name=src  ]
   [\ac{ASP}P [\ac{ASP}P \\ $\varnothing$ \\ {[\textit{u}$\upphi$, \textit{u}Asp, \textit{u}D\textsuperscript{EPP}]}, name = tgt ] 
   [VP[V][\acs{SUB} \\ {[$\upphi$, D]} ]]]] 
   \draw[->] (src) to[out=south,in=south] (tgt);
   \end{forest}
\ex
 \begin{forest}[ \textit{v}P [\textit{v} \\  \textit{ha}\textsuperscript{KR}/\textit{su}\textsuperscript{JP} \\ {[\colorbox{lightgray}{\textit{u}$\upphi$}, \textit{u}Asp, \colorbox{lightgray}{\textit{u}D\textsuperscript{EPP}}]}] [\ac{ASP}P [\textsc{obj}$_i$ {[$\upphi$, D]}, name=tgt] 
 [\ac{ASP}P [\ac{ASP}P{[\sout{\textit{u}$\upphi$}, \sout{\textit{u}D}\textsuperscript{\sout{EPP}}]} \\ $\varnothing$] [VP [~~] [t$_i$, name=src]
 ]]]]
\draw[->,dashed] (src) to[out=west,in=south west] (tgt);
   \end{forest}
\ex \begin{forest}
[\textit{v}P
[\ac{ASP}P$_k$ [\textsc{obj}$_i$][\ac{ASP}$'$ [\ac{ASP}][VP [V][t$_i$]]]]
[\textit{v}$'$ [\textit{v} \\  \textit{ha}\textsuperscript{KR}/\textit{su}\textsuperscript{JP} ][t$_k$]
]]
\end{forest}
\end{xlist}
\end{exe}

In \REF{ex:146} null \ac{ASP} inherits [\textit{u}$\upphi$, \textit{u}D\textsuperscript{\ac{EPP}}] from \textit{v} = \textit{ha}\textsuperscript{\MakeUppercase{kr}}/\textit{su}\textsuperscript{\MakeUppercase{jp}} and agrees with the object, and the object raises to Spec, \ac{ASP}P, delivering \ac{OV} order (\ref{ex:146}b). Agree between the probe \ac{ASP} and the goal/object is reflected as the accusative case marking in which [\textit{u}D] on \ac{ASP} is spelled out as the accusative marker \textit{-lul} in Korean and \textit{{}-o} in Japanese.\footnote{\textrm{Similarly, \citet{Kim1993} proposes that structural case must be overtly realized when the object moves out of the VP: more specifically, when the object moves to the specifier of µP above VP, which corresponds to A}\textrm{\textsc{sp}}\textrm{P in \REF{ex:146}, the object must be morphologically marked with the accusative Case marker in Korean.}}

On the other hand, in \REF{ex:147}, which is the underlying structure for the examples in \REF{ex:145}, \ac{FI} from \textit{v} to \ac{ASP} does not take place due to the fact that \ac{ASP} is lexicalized by an English light verb such as \textit{take} and \textit{keep}. So, \ac{ASP} does not agree with the object. Instead, feature matching occurs between \textit{v} and \ac{ASP}P and \ac{ASP}P raises to left of \textit{v} = \textit{ha}\textsuperscript{KR}/\textit{su}\textsuperscript{JP}, resulting in \ac{VO} order. In other words, the absence of the accusative case marker in the Korean-English and Japanese-English \ac{CS} examples in \REF{ex:145} can be explained by the proposal that accusative case marking on the object is a morphological manifestation of Agree between \ac{ASP} and the internal argument/object and there is no Agree between \ac{ASP} and the object in the \ac{VO}-ordered code-switched sentences between Korean/Japanese and English, as depicted in \REF{ex:147}.\footnote{\textrm{The account of case morphology in Korean-English and Japanese-English \ac{CS} provided in this monograph differs from my earlier analysis in \citet{DenDikkenShim2011}, in which Case was proposed as [}\textrm{\textit{u}}\textrm{Asp] instead of a D-feature.}}

\begin{exe}\ex\label{ex:147}
\begin{xlist}
    \ex \begin{forest}
     [\textit{v}P [\textit{v} \\ \textit{ha}\textsuperscript{KR}/\textit{su}\textsuperscript{JP} \\ {[\textit{u}$\upphi$, \textit{u}Asp\textsuperscript{EPP}, \textit{u}D\textsuperscript{EPP}]}, name=src]
    [\ac{ASP}P {[Asp, T, $\upphi$, D]} 
    [\ac{ASP} {[Asp, T]} \\ LV\textsubscript{ENG}, name=tgt  ]
    [ OBJ \\ {[$\upphi$, D]}]
    ]]
    \draw[->] (src) to[in = south west, out = south] node[pos=0.5]{\ding{54}} (tgt);
    \end{forest}
    
\ex\adjustbox{width=0.9\textwidth}{
    \begin{forest}
     [\textit{v}P [\textit{v} \\ \textit{ha}\textsuperscript{KR}/\textit{su}\textsuperscript{JP} \\ {[\textit{u}$\upphi$, \textit{u}Asp\textsuperscript{EPP}, \textit{u}D\textsuperscript{EPP}]}, name=src]
    [\ac{ASP}P \\ {[Asp, T, $\upphi$, D]} 
    [\ac{ASP} \\ {[Asp, T]}  ]
    [ OBJ \\ {[$\upphi$, D]}] { \draw (.east) node[right]{~~~~~~$\Uparrow$}; }
    ]{ \draw (.east) node[right]{\textsc{feature percolation}}; }]
    \end{forest}}
    
    \ex \begin{forest}
    [\textit{v}P [\ac{ASP}P$_i$, name = tgt [\ac{ASP}][\textsc{obj}]]
    [\textit{v}$'$ [\textit{v}][t$_i$, name=src]
    ]]
    \draw[->,dashed] (src) to[in = west, out = south] (tgt);
    \end{forest}
\end{xlist}
\end{exe}

\citet{Richardson2003} presents a similar account of the accusative Case/case in Russian: the accusative case marker in Russian is a result of Agree between \ac{ASP} and the internal argument. Yet, she claims that it is \emph{\Acl{ASP}} feature matching between \ac{ASP} and the internal argument/object: an (uninterpretable) aspectual feature on \ac{ASP} (more precisely [$\pm$telic]) is valued against the interpretable aspectual feature on the object. However, as discussed earlier in Chapter \ref{ch:3} (\sectref{ch3:sect:3.2.1}), telicity is not in fact tied one-to-one to accusative Case: telic unaccusative constructions do not involve accusative Case feature checking, thus it provides no direct link between accusative Case and telicity (or aspect in general).\footnote{\textrm{Richardson also argues that the nominative case in Russian is morphological manifestation of ${\upphi}$-(complete) feature on T in a feature matching relationship with the external argument. By contrast, the dative case is morphological manifestation of ${\upphi}$-incomplete T and arises in the absence of any feature matching relationship \citep[56]{Richardson2003}.}} 

Cross-linguistic data supports the proposal that Case is a D-feature and accusative case marking is a morphological indicator of Agree between \ac{ASP} and the object. In Turkish and Hebrew, for instance, accusative case-marked objects denote definite/specific readings whereas indefinite/non-specific readings arise with their morphologically null or unmarked counterparts as exemplified in \REF{ex:148}. Assuming that definite/specific readings arise from DPs, not from \acp{NP}, definite objects/DPs are overtly marked for the accusative case both in (\ref{ex:148}a) and (\ref{ex:148}b), whereas case marking on indefinite objects/\acp{NP} is not possible in (\ref{ex:148}a') and (\ref{ex:148}b'), showing that the accusative case is related to a D-feature. 

\begin{exe}\ex\label{ex:148}
\begin{xlist}
    \ex\label{ex:148a} \gll Mehmet kitab-i oku-du      \\
    {} book-\textsc{acc} read-\textsc{past} \\
    \hspace*{\fill} Turkish
    \glt  ‘Mehmet read the book.’
    \exi{a.\parbox{0mm}{$'$}} \gll  Mehmet kitab oku-du      \\
     {} book read-\textsc{past} \\
    \glt  ‘Mehmet read a book.’ (\citealt{NagaOzcelik2012})
    \ex\label{ex:148b} \gll ha-seret    her’a     et-ha-milxama  \\
    the-movie showed \textsc{acc}-\textsc{det-}war\\ 
    \hspace*{\fill} Hebrew
    \glt  ‘The movie showed the war.’
    \exi{b.\parbox{0mm}{$'$}} \gll ha-seret    her’a     ha-milxama  \\
    the-movie showed \textsc{det-}war \\
    \glt   ‘The movie showed a war.’  (\citealt{Aissen2003})
\end{xlist}
\end{exe}

Additionally, in Palauan only a definite object can trigger agreement on the verb and an indefinite object triggers no agreement, as in the contrast shown in (\ref{ex:149}a) and (\ref{ex:149}b). Since Palauan is not a nominative-accusative language, the agreement morpheme \textit{ii} on the verb, which agrees with the object \textit{bilis} ‘dog’ in (\ref{ex:149}a), may not be considered equivalent to the accusative marker. However, along the lines of the current proposal that (accusative) case is a morphological reflex of Agree between a probe and a goal (object) with the matching D-feature, \ac{DP} object-verb agreement in (\ref{ex:149}a) can be explained as well. 

\begin{exe}\ex\label{ex:149}
\begin{xlist}
    \ex \gll te-'illebed-ii     a bilis a rengalek    \\
    3\textsc{pl-perf}-hit-3\textsc{sg} {} dog  {}   children \\
    \hfill Palauan
    \glt ‘The kids hit the dog.’
    \ex ,\gll te-'illebed    a bilis a rengalek      \\
    3\textsc{pl-perf}-hit  {} dog  {}   children \\
    \glt ‘The kids hit a dog/some dog(s).’ (\citealt{Georgopoulos1991})
\end{xlist}
\end{exe}

\citet{Kim1993} argues that the specific/non-specific distinction is transparent and related to accusative case marking in Korean as well. For instance, specificity seems to be closely related to the presence of the accusative marker in \REF{ex:150}.

\begin{exe}\ex\label{ex:150}
\begin{xlist}
    \ex{\gll   ~~Bibi-ka     etten   haksayng-ul po-ass-ta\\
    ~~Bibi-\textsc{nom} certain student-\textsc{acc} see-\textsc{past-decl} \\}
    \hfill \hspace*{-5mm}Korean
    \ex[*]{ \gll Bibi-ka etten haksayng po-ass-ta \\
    Bibi-\textsc{nom} certain student see-\textsc{past-decl}\\
    \glt ‘Bibi saw a certain student.’   modified from \citet[23]{Kim1993}
    }
\end{xlist}
\end{exe}

However, some of his data need further verification: Korean informants that I consulted did not agree with the (un)acceptability/grammaticality patterns of many examples Kim provides, and definiteness or specificity may not arise from the presence of overt case morphology in Korean and Japanese. Instead, accu\-sa\-tive-case marked nouns may induce a \textit{focal} reading in Korean and Japanese, as exemplified in (\ref{ex:151}) (\citealt{Jun2005,Ko2000,LeeH2006b,Matsuda1995}).

\begin{exe}\ex \label{ex:151}
\begin{xlist}
    \ex \gll (yesnal-ey)   Bibi-ka    han/etten  yeca-lul        mannass-ta\\
    long.time-\textsc{loc} Bibi-\textsc{nom} one/some woman\textsc{{}-acc} meet.\textsc{past-decl}\\\hfill \hspace*{-5mm}Korean
    \glt ‘(A long time ago) Bibi met a woman.’     (focalized/emphatic)
    \ex \gll  (yesnal-ey)   Bibi-ka    han/etten  yeca   mannass-ta   \\
    long.time-\textsc{loc} Bibi-\textsc{nom} one/some woman meet.\textsc{past-decl} \\
    \glt ‘(A long time ago) Bibi met a woman.’  modified from \citet[58, (5)]{AhnCho2007}
\end{xlist}
\end{exe}

One may question whether there is a D-feature present on noun phrases in Korean or Japanese since neither language possesses overt definite or indefinite articles and bare noun phrases may be interpreted as definite or indefinite in a given context. Thus, some researchers argue that a determiner phrase (DP) is not projected in these languages (\citealt{Chierchia1988,Fukui1986,Tomioka2003}). By contrast, many others argue for the presence of DPs in Korean and Japanese (\citealt{Furuya2009,Kakegawa2003,Koike1999,ParkSY2008,Tateishi1987}), and the proposal has been made that case markers represent D heads (\citealt{Kakegawa2003,Koike1999,Tateishi1987}), which provides supporting evidence for the present proposal that Case is a D-feature. Thus, I maintain that Case is an uninterpretable D-feature on the phase heads C (for nominative) and \textit{v} (for accusative). When T inherits [\textit{u}D] from C, T agrees with a goal/external argument with the matching D-feature and the nominative case on the goal/external argument reflects such an Agree relation between the probe T and the goal. Likewise, when \ac{ASP} inherits [\textit{u}D] from \textit{v}, \ac{ASP} agrees with a goal/internal argument bearing the matching D-feature, which manifests as the accusative case marker on the goal in Korean-English and Japanese-English \ac{CS}. 

\section{Word order} \label{ch6:sect:WordOrder}

In Chapter \ref{ch:4}, I explained how \ac{FI} occurs from \textit{v} to  \ac{ASP} in English in comparison with Korean and Japanese, from which \ac{VO} and \ac{OV} order are derived, respectively. In the following, I extend the mechanism of \ac{FI} to the C-T domain and show how head-initial structure/\ac{SVO} order in English and head-final structure/\ac{SOV} order in Korean/Japanese are generated.

\subsection{English}\label{ch6:sect:English}

In this monograph, I have proposed that English heavy/lexical verbs represent V and light verbs are A\textsc{sp.} I have also identified feature specifications on the phase heads C and \textit{v} in English, as repeated in \REF{ex:152}. Based on these, now I show how \ac{SVO} order is derived via \ac{FI} from \textit{v} to \ac{ASP} and C to T. We begin with lexical verbs and then look at light verbs afterwards. 

\begin{exe}\ex\label{ex:152}
\begin{tabular}[t]{ll}
    C & [\textit{u}$\upphi$, \textit{u}T, \textit{u}D\textsuperscript{\ac{EPP}}] \\
    \textit{v} &   [\textit{u}$\upphi$, \textit{u}Asp, \textit{u}D\textsuperscript{(\ac{EPP})}]
\end{tabular}
\end{exe}

The structure in \REF{ex:153} shows that \ac{FI} takes place at both the upper phase \ac{CP} and the lower phase \textit{v}P when an English heavy/lexical verb lexicalizes V. In the lower \textit{v}P phase, all of \textit{v}’s features are inherited by \ac{ASP}, as a result of which \ac{ASP} agrees with V and the object. Since none of the features on \textit{v} are \ac{EPP}-specified, no movement is induced. Following the proposal that V overtly raises to \textit{v} in English \citep{Chomsky1995}, I propose that V raises to \ac{ASP} and the \ac{ASP} + V complex further moves to \textit{v}, as depicted in \REF{ex:154}.

\ea \label{ex:153}\adjustbox{width=0.95\textwidth}{\begin{forest}
for tree={s sep=0, inner sep=1, l=1}
[CP [C \\ {[\colorbox{lightgray}{\textit{u}$\upphi$, \textit{u}T, \textit{u}D\textsuperscript{EPP}}]}, name = src1 ]
[TP [T \\ {[\textit{u}$\upphi$, \textit{u}T, \textit{u}D\textsuperscript{EPP}]}, name = tgt1 ]
[\textit{v}P [SUB \\ {[$\upphi$, D]} ]
[\textit{v}$'$
[\textit{v} \\ {[\colorbox{lightgray}{\textit{u}$\upphi$, \textit{u}Asp, \textit{u}D}]}, name = src2 ]
[A\textsc{sp}P [A\textsc{sp}P \\ {[\textit{u}$\upphi$, \textit{u}Asp, \textit{u}D]}, name = tgt2 ]
[VP [HV \\ {[Asp, T]}] [OBJ \\ {[$\upphi$, D]}]
]]]]]]]
\draw[->] (src1) to[out=south,in=south] (tgt1);
\draw[->] (src2) to[out=south,in=south] (tgt2);
\end{forest}}

\ex\label{ex:154}
\footnotesize
\begin{forest}
for tree={s sep=0, inner sep=1, l=1}
[CP [C \\ {[\colorbox{lightgray}{\textit{u}$\upphi$, \textit{u}T, \textit{u}D\textsuperscript{EPP}}]}, name = src2 ]
[TP [T \\ {[\textit{u}$\upphi$, \textit{u}T, \textit{u}D\textsuperscript{EPP}]}, name = tgt2 ]
[\textit{v}P [SUB \\ {[$\upphi$, D]} ]
[\textit{v}$'$
[(\textit{v} + A\textsc{sp}P + V) \\{[Asp, T]} , name = tgt1 ]
[A\textsc{sp}P [ (A\textsc{sp}P + V), name = src/tgt ]
[VP [(V), name = src1 ] [OBJ \\ {[$\upphi$, D]}]
]]]]]]]
\draw[->] (src2) to[out=south,in=south] (tgt2);
\draw[->,dashed] (src1) to[out=south west,in=south] (src/tgt);
\draw[->,dashed] (src/tgt) to[out=south west,in=south] (tgt1);
\end{forest}
\z

\largerpage[3]
After all \textit{v}'s features are transferred to \ac{ASP}, \textit{v} is no longer a probe; none of the probing features on \textit{v} are active, following the principle of \textit{Expiration}. Instead, \textit{v} (more precisely, the amalgam \textit{v} + \ac{ASP} + V) now bears [Asp, T] features after V-to-\ac{ASP}-to-\textit{v} raising. After the complement of \textit{v}P is transferred to \ac{PF} and  \ac{PF}, the derivation continues. In the \ac{CP} phase, T inherits all of C’s features. T agrees with the subject, which bears the matching [$\upphi$, D] features, and also with the \textit{v} + \ac{ASP} + V complex head, which has a [T] feature. The \ac{EPP} property of a D-feature on T triggers subject raising to Spec, \ac{TP}. As a result, \ac{SVO} order is obtained in \REF{ex:155}. The nominative Case on the subject is a morphological manifestation of Agree between the probe T and the goal/subject with the matching D-feature.

\ea \label{ex:155}
\footnotesize
\begin{forest} 
[CP [C]
[TP [\textbf{SUB}$_i$]
[T$'$ [T]
[\textit{v}P [t$_i$]
[\textit{v}$'$ [\textit{v} + A\textsc{sp} + \textbf{V} ]
[A\textsc{sp}P [(A\textsc{sp} + \textbf{V})]
[VP [(V)][\textbf{OBJ}]
]]]]]]]
\end{forest}
\z\vspace*{-1cm}

\clearpage
One note should be made here: at the \textit{v}P phase, \ac{FI} from \textit{v} to \ac{ASP} must occur prior to V to \ac{ASP} raising. If V raises to \ac{ASP} first, the lexically filled \ac{ASP} bears the [Asp, T] features of the light verb and it is no longer a valid head to inherit \textit{v}’s features: only a featureless non-phase functional head can inherit the phase head’s probing features. 

On the other hand, when \ac{ASP} is lexicalized by an English light verb, \ac{FI} takes place only at the upper phase \ac{CP}, but not at the lower phase \textit{v}P, as shown in \REF{ex:156}. \ac{FI} from \textit{v} to \ac{ASP} is blocked because the lexically filled \ac{ASP} is not qualified to be a recipient head to inherit features from \textit{v} (the principle of validation). As a result, \textit{v} agrees with A\textsc{spP} which bears all the features of \ac{ASP} and the object. Similar to the derivation in \REF{ex:154}, \ac{ASP} raises to \textit{v} after \textit{v}’s features are valued, as a result of which the complex \textit{v} + \ac{ASP} head bears the T-feature \REF{ex:157}.\footnote{I assume that, similar to a lexical verb which represents V and raises to \textit{v}, a light verb lexicalizing \ac{ASP} raises to \textit{v} in English.} 

\ea \label{ex:156}
\adjustbox{width=0.9\textwidth}{ 
\begin{forest} 
[CP [C \\ {[\colorbox{lightgray}{\textit{u}$\upphi$, \textit{u}T, \textit{u}D\textsuperscript{EPP}}]}, name = src1 ]
[TP  [T  \\ {[\textit{u}$\upphi$, \textit{u}T, \textit{u}D\textsuperscript{EPP}]}, name = tgt1]
[\textit{v}P [SUB  \\ {[$\upphi$,D]}]
[\textit{v$'$}[\textit{v}  \\ {[\textit{u}$\upphi$, \textit{u}Asp, \textit{u}D]}, name = src2 ]
[A\textsc{sp}P \\ {[Asp, T, $\upphi$, D]} 
[A\textsc{sp} \\ LV\textsubscript{{[Asp, T]}}, name = tgt2]
[OBJ  \\ {[$\upphi$,D]}]{ \draw (.east) node[right]{~~~~~~$\Uparrow$}; } ] { \draw (.east) node[right]{\textsc{feature percolation} }; }
]]]]
\draw[->] (src1) to[out=south,in=south] (tgt1);
\draw[->] (src2) to[out=south,in=west] node[pos=0.3]{\ding{54}}  (tgt2);
\end{forest}}
\ex\label{ex:157}
\adjustbox{width=0.9\textwidth}{ 
\begin{forest} 
[CP [C \\ {[\colorbox{lightgray}{\textit{u}$\upphi$, \textit{u}T, \textit{u}D\textsuperscript{EPP}}]}, name = src1 ]
[TP  [T  \\ {[\textit{u}$\upphi$, \textit{u}T, \textit{u}D\textsuperscript{EPP}]}, name = tgt1]
[\textit{v}P [SUB  \\ {[$\upphi$,D]}]
[\textit{v$'$}[\textit{v} + \textsc{Asp}  \\ {[Asp, T]}, name = tgt2 ]
[A\textsc{sp}P  [(A\textsc{sp}), name = src2]
[OBJ  \\ {[$\upphi$,D]}] ] 
]]]]
\draw[->] (src1) to[out=south,in=south] (tgt1);
\draw[->,dashed] (src2) to[out=south,in=south]  (tgt2);
\end{forest}}
\z

In the next \ac{CP} phase, T inherits all of C’s features and enters multiple agree relations with the subject to value the ${\upphi}$ and D-features and with the complex~\textit{v} head to value the T-feature. Due to the \ac{EPP} property on the D-feature on the probe T, the subject raises to Spec, \ac{TP} and \ac{SVO} order is derived, as in \REF{ex:158}. In summary, \ac{SVO} order is always derived in English whether the verb is heavy or light.

\ea \label{ex:158}
\begin{forest}
[CP [C]
[TP [\textbf{SUB}$_k$]
[T$'$ [T]
[\textit{v}P [t$_k$]
[\textit{v}$'$[\textit{v} + \textbf{A\textsc{sp}}$_i$]
[A\textsc{sp}P [(A\textsc{sp})][\textbf{OBJ}]]]]]]]
\end{forest}
\z

\subsection{Korean and Japanese}\label{ch6:sect:KRJN}

\REF{ex:159} shows the feature matrices of C and \textit{v} in Korean and Japanese, and we will see how \ac{SOV} order is derived via \ac{FI} from C to T and from \textit{v} to A\textsc{sp.} 

\ea\label{ex:159}\begin{tabular}[t]{ll}
    C & [\textit{u}$\upphi$, \textit{u}T\textsuperscript{EPP}, \textit{u}D\textsuperscript{\ac{EPP}}] \\
    \textit{v} &   [\textit{u}$\upphi$, \textit{u}Asp\textsuperscript{EPP}, \textit{u}D\textsuperscript{\ac{EPP}}]
\end{tabular}
\z

In the \textit{v}-\ac{ASP} domain \REF{ex:160}, \ac{ASP} inherits [\textit{u}$\upphi$, \textit{u}D\textsuperscript{\ac{EPP}}] from \textit{v} and agrees with the object, after which  the object raises to Spec, \ac{ASP}P to satisfy the \ac{EPP} requirement on \ac{ASP} and so \ac{OV} order is derived within \ac{ASP}P. Then [\textit{u}Asp\textsuperscript{\ac{EPP}}] on \textit{v} is valued against the matching feature on V and the maximal projection of V is triggered to move to Spec, \textit{v}P. However, the lexical projection of \ac{VP} cannot move by itself and pied-pipes \ac{ASP}P. Thus, \ac{OV} order is derived in the \textit{v}P phase as in \REF{ex:161}, with the detailed process already explained in Chapter \ref{ch:4}.

\ea\label{ex:160} \adjustbox{width=0.9\textwidth}{\begin{forest}
[\textit{v}P [\textit{v} \\ {[\colorbox{lightgray}{\textit{\textit{u}$\upphi$}}, \textit{u}Asp\textsuperscript{EPP}, \colorbox{lightgray}{\textit{u}D\textsuperscript{EPP}}]}, name = src] 
[A\textsc{sp}P [Asp \\ {[\textit{u}$\upphi$, \textit{u}D\textsuperscript{EPP}]}, name = tgt]
[VP [V \\ {[Asp, T]}] [OBJ \\ {[$\upphi$, D]}]]
{\draw (.east) node[right]{[\textit{u}D\textsuperscript{EPP}] on A\textsc{sp} triggers \textsc{obj} raising }; }
]
{\draw (.east) node[right]{~~~~~~~~~~[\textit{u}Asp\textsuperscript{EPP}] on \textit{v} triggers \textsc{obj} raising }; }
]
\draw[->] (src) to[out=south,in=south] (tgt);
\end{forest}}

\ex \label{ex:161}\adjustbox{width=0.9\textwidth}{\begin{forest}
[\textit{v}P [AspP [OBJ$_j$ \\ {[$\upphi$, D]} ][A\textsc{sp}$'$
[A\textsc{sp} \\ {[\sout{\textit{u}$\upphi$}, \sout{\textit{u}D}\textsuperscript{\sout{EPP}}]}][VP [V \\ {[Asp, T]} ][t$_i$]
]]]
[\textit{v}$'$ [\textit{v} \\ {[\colorbox{lightgray}{\textit{u}$\upphi$}, \sout{\textit{u}A\textsc{sp}}\textsuperscript{\sout{EPP}}, \colorbox{lightgray}{\textit{u}D\textsuperscript{EPP}}]}][t$_k$]]]
\end{forest}}
\z

The derivation continues in a higher \ac{CP} phase where \textit{v}P merges with T and \ac{FI} happens from C to T in upper \ac{CP} phase as in \REF{ex:162}. Since feature specifications on C and \textit{v} are entirely parallel in Korean and Japanese, as we see in \REF{ex:159}, \ac{FI} from C to T is entirely parallel to \ac{FI} from \textit{v} to A\textsc{sp.} \textsc{T} inherits [\textit{u}$\upphi$, \textit{u}D\textsuperscript{\ac{EPP}}] from C and agrees with the subject. The \ac{EPP} on the $\upphi$ feature on T triggers subject raising to Spec, \ac{CP}. C also agrees with VP, which has the matching T feature. The \ac{EPP} property on the T-feature on C triggers \ac{CP} movement to the left of C, resulting in consistent head-final structure in Korean and Japanese. In the following, I will explain how \ac{CP} raises after C agrees with \ac{VP} \REF{ex:162}.

\ea\label{ex:162}\adjustbox{width=0.9\textwidth}{\begin{forest}
[CP [C \\ {[\colorbox{lightgray}{\textit{\textit{u}$\upphi$}}, \textit{u}T\textsuperscript{EPP}, \colorbox{lightgray}{\textit{u}D\textsuperscript{EPP}}]}, name = src] 
[TP [T \\ {[\textit{u}$\upphi$, \textit{u}D\textsuperscript{EPP}]}, name = tgt]
[\textit{v}P [SUB \\ {[$\upphi$, D]}] [v$'$ [\textit{v}][A\textsc{sp}P]]
{\draw (.east) node[right]{[\textit{u}D\textsuperscript{EPP}] on T triggers SUB raising }; }
]
{\draw (.east) node[right]{~~~~~~~~[\textit{u}T\textsuperscript{EPP}] on C triggers TP raising }; }
]]
\draw[->] (src) to[out=south,in=south] (tgt);
\end{forest}}

\ex\label{ex:163}
\adjustbox{width=0.9\textwidth}{ 
\begin{forest}
[CP [C \\ {[\colorbox{lightgray}{\textit{u}$\upphi$}, \textit{u}T\textsuperscript{EPP},\colorbox{lightgray}{ \textit{u}D\textsuperscript{EPP}}]}, name = src]
[TP[T \\ {[\textit{u}$\upphi$, \textit{u}D\textsuperscript{EPP}]}, name = tgt]
[\textit{v}P [SUB \\ {[$\upphi$, D]}]
[\textit{v}P [A\textsc{sp}$_k$ [OBJ$_i$ \\ {[$\upphi$, D]}]
[A\textsc{sp}$'$ [A\textsc{sp}]
[VP [V \\ {[Asp, T]}]
[t$_i$]]]]
[\textit{v}$'$ [\textit{v} \\  {[\colorbox{lightgray}{\textit{u}$\upphi$}, \sout{\textit{u}Asp}\textsuperscript{\sout{EPP}}, \colorbox{lightgray}{\textit{u}D\textsuperscript{EPP}}]} ][t$_k$]]]]]]
\draw[->] (src) to[out=south,in=south] (tgt);
\end{forest}
}
\z

In \REF{ex:163} T inherits [\textit{u}$\upphi$, \textit{u}D\textsuperscript{EPP}] from C and agrees with the subject, and the subject raises to Spec, \ac{CP} due to the \ac{EPP}-specification on the D-feature of T.\footnote{Multiple specifiers of \textit{v}P  does not violate \textit{multiple agree under antisymmetry} here. As discussed in Chapter \ref{ex:4}, antisymmetry is a \ac{PF} constraint, and since the subject moves to Spec, \ac{CP} and leaves a trace, only one specifier of \textit{v}P is spelled out at the  \ac{PF}.} The [\textit{u}T] on C still needs to be valued, and V has the corresponding feature. Also, the maximal projection of the goal must raise to satisfy \ac{EPP} on C, therefore it induces VP movement. But we now know that \ac{ASP}P must be pied-piped by VP movement. However, \ac{ASP}P cannot raise in \REF{ex:163} according to \textit{Criterial Freezing}.

\begin{exe}\ex\label{ex:164} \textit{Criterial} \textit{Freezing}  \hspace*{\fill}  (\citealt[11, (32)]{Rizzi2004}) \\
A phrase meeting a criterion is frozen in place.
\end{exe}

In a case of (successive-cyclic) movement, a chain is formed where the head of the chain indicates the final landing position of a moved element and the foot of the chain shows its base-generated position in \REF{ex:165}. Rizzi calls the last merging position of X a criterial position at which X fulfills the criterion (or ‘features’ in Chomsky’s term) of a probe via spec-head configuration. The element in a criterial position is frozen and cannot be extracted any further in the derivation.

\begin{exe}
\ex\label{ex:165} < X\textit{i},  ~~~ \ldots ~~~ t\textit{\textsubscript{i} }>
\end{exe}

Returning to \REF{ex:163}, \ac{ASP}P raises to Spec, \textit{v}P and fulfills the criterion of the probe \textit{v}, [\textit{u}Asp] including its \ac{EPP} specification via a spec-head agreement. In other words, \ac{ASP}P is at a criterial position and is frozen in place. Thus, \ac{ASP}P cannot move to Spec, \ac{CP} in a probe-goal relationship and it may conceivably pied-pipe \textit{v}P, as in \REF{ex:166}.

\ea\label{ex:166}
\footnotesize
\begin{forest}
[*CP
[\textit{v}P
[t\textsubscript{SUB},draw]
[\textit{v}P 
[A\textsc{sp}P [OBJ]
[A\textsc{sp}$'$ [A\textsc{sp}][VP[V][t\textsubscript{OBJ}]]
]]
[\textit{v}$'$ [\textit{v}][t\textsubscript{A\textsc{sp}P}]]]
]
[C$'$ [C]
[TP [SUB,draw]
[T$'$ [t][t\textsubscript{\textit{v}P}]]]]]
\end{forest}
\z

\largerpage[3]
However, \textit{v}P movement to Spec, \ac{CP} violates the Proper Binding Condition \citep{Fiengo1977} and is ruled out: the trace of the subject is not c-commanded by its antecedent and therefore not properly bound in \REF{ex:166}. Since neither \ac{ASP}P nor \textit{v}P can be pied-piped by VP, it turns out that the smallest unit that VP can pied-pipe is \ac{CP}, which further raises to Spec, \ac{CP} in Korean and Japanese (cf. \citealt{Kayne1994}). Analogous to \ac{ASP}P pied-piping and raising in the \textit{v}-\ac{ASP} domain, \ac{CP} is pied-piped and raises in the C-T domain. As a result, surface order \textsc{sovc}(omplementizer) is derived, as illustrated in \REF{ex:167}.

\ea\label{ex:167}
\footnotesize
\begin{forest}
[CP[TP [\textbf{SUB}]
[T$'$ [T]
[\textit{v}P [t\textsubscript{SUB}]
[\textit{v}P [A\textsc{sp}P [\textbf{OBJ}]
[A\textsc{sp}$'$ [A\textsc{sp}]
[VP [\textbf{V}][t\textsubscript{SUB}]]]]
[\textit{v}$'$ [\textit{v}][t\textsubscript{A\textsc{sp}P}]]
]]
]]
[C$'$ [\textbf{C}][t\textsubscript{TP}]]]
\end{forest}\z
\clearpage

While \textit{v}P cannot raise to Spec, \ac{CP} in Korean and Japanese, \textit{v}P raising is not categorically ruled out across languages. As a matter of fact, \textit{v}P can raise to Spec, \ac{CP}: various types of so-called VP topicalization (arguably involving fronting of \textit{v}P) are found in Germanic V2 languages, exemplified in \REF{ex:168}.

\ea\label{ex:168}
\ea\gll ~[\textit{\textsubscript{v}}\textsubscript{P} dem Peter      ein Buch      gegeben]\textsubscript{3} hat die  Claudia    gestern     t\textsubscript{3} \\
 {} \textsc{art}  Peter\textsubscript{DAT} a    book\textsubscript{ACC} given       has \textsc{art} Claudia\textsubscript{NOM} yesterday \\\hfill German
\ex \gll  ~[\textit{\textsubscript{v}}\textsubscript{P} t\textsubscript{1} gelesen]\textsubscript{3} hat das Buch\textsubscript{1} keiner t\textsubscript{3} \\
{} {} read  has the book  no-one \\
\ex \gll ~[\textit{\textsubscript{v}}\textsubscript{P} dem Peter       t\textsubscript{2} gegeben]\textsubscript{3} hat die  Claudia       ein Buch\textsubscript{2} t\textsubscript{3}  \\
{} \textsc{art}  Peter\textsubscript{DAT} {}    given  has \textsc{art} Claudia\textsubscript{NOM} a    book\textsubscript{ACC} \\
\ex \gll ~[\textit{\textsubscript{v}}\textsubscript{P} t\textsubscript{1} t\textsubscript{2} gegeben]\textsubscript{3} hat die  Claudia     dem Peter\textsubscript{1}     ein Buch\textsubscript{2}~t\textsubscript{3}   \\
{} {} {} given  has \textsc{art} Claudia\textsubscript{NOM} \textsc{art}  Peter\textsubscript{DAT} a    book\textsubscript{ACC} \\
adopted from \citet[187, (5)]{Muller1998}
\z\z

In (\ref{ex:168}a), the entire \textit{v}P is fronted to the beginning of the sentence, which has been argued to involve raising to Spec, \ac{CP} (\citealt{BestenWebelhuth1987}, \citealt{Thiersch1985}). Similarly, the \textit{v}P is topicalized in (\ref{ex:168}b-d). However, the \textit{v}Ps in (\ref{ex:168}b-d) are remnant \textit{v}Ps including the trace of the scrambled direct object in (\ref{ex:168}b), the trace of the direct object in a double object construction in (\ref{ex:168}c), and the trace of both the direct object and the indirect object in (\ref{ex:168}d). These traces inside the \ac{VP} are not bound, violating the Proper Binding Condition. Yet, the sentences remain grammatical in remnant \textit{v}P topicalization in German. On the contrary, we have seen that \textit{v}P raising in Korean and Japanese is ruled out because it violates the Proper Binding Condition. How can we explain this conflicting pattern between \textit{v}P topicalization in Germanic languages, on the one hand, and the ban on \textit{v}P raising in Korean and Japanese, on the other hand?

\hspace*{-1mm}I suggest that purely \ac{EPP}-driven movement as in \textit{v}P raising in Korean and Jap\-a\-nese cannot reconstruct (\citealt{NevinsAnand2003}), whereas topicalization, which is not a case of pure \ac{EPP}-driven movement (because it leads to the interpretive effect of topic-comment articulation), as in \textit{v}P raising in Germanic languages, can reconstruct, thus allowing an unbound trace in the VP to be bound at its reconstructed base position in V2 Germanic languages. It is worth investigating cross-linguistic variation regarding \textit{v}P raising, including various verb-initial languages such as Niuean, which is arguably a \ac{VP} fronting language exhibiting \textsc{vso} order (\citealt{Massam2000,Massam2001}). However, this I will leave for future research.

I have shown how \ac{FI} takes place in both \ac{CP} and \textit{v}P phases in English, Korean and Japanese, which explains how head-initial structure in English and head-final structure in Korean and Japanese are derived. In Korean and Japanese, various types of pied-piping are observed, \ac{ASP}P pied-piping in the \textit{v}{}-\ac{ASP} domain and \ac{CP} pied-piping in the C-T domain, which describe consistent head-final structure in these languages. Pied-piping is observed in various domains across languages, and PP pied-piping in \textit{wh}{}-movement in English is one of many examples. In (\ref{ex:169}a, b), for instance, the object moves up to Spec, \ac{CP} triggered by the \ac{EPP} specification on the \textit{wh}{}-feature on C. On the other hand, the goal/object may pied-pipe its mother node PP and move up to Spec, \ac{CP} in (\ref{ex:169}a', b'). 

\begin{exe}\ex\label{ex:169} 
    \begin{xlist}
    \ex\label{ex:169a} Who did you speak to?
    \exi{a.\parbox{0mm}{$'$}} To whom did you speak?
    \ex\label{ex:169b} What do you mix your vodka with?
    \exi{b.\parbox{0mm}{$'$}} With what do you mix your vodka?
\end{xlist}\end{exe}

Researchers have proposed various accounts for pied-piping constructions like those in \REF{ex:169}, such as feature percolation from the \textit{wh}-object to its dominating node PP or restricting the \textit{wh}-feature in a local domain where no phrasal boundary must be present between the probe and the goal. However, a feature percolation approach to pied-piping has been criticized for its overgeneralization, and the nature of pied-piping is still open to debate. Instead, pied-piping may be dealt with differently under the \ac{FI} system, as proposed in this monograph. 

\section{Conclusion of the monograph}\label{ch6:sect:conclusion}

The main purpose of this monograph was to investigate word order variation in \ac{CS}, with particular focus on the distribution of \ac{OV} and \ac{VO} orders in Korean-English and Japanese-English \ac{CS}. Taking the view that linguistic variation is due to different morphosyntactic features on functional categories rather than lexical ones (the Borer-Chomsky conjecture), the role of functional/light verbs was explored in comparison with lexical/heavy verbs, with respect to \ac{OV}-\ac{VO} variation in Korean-English and Japanese-English \ac{CS}. In addition, the role of syntactic flexibility of a code-switched phrase was also investigated, assuming that while the internal argument of a syntactically flexible phrase is subject to \ac{CS}, a less flexible phrase undergoes \ac{CS} as a unit. The results from the \ac{CS} judgment task and the syntactic flexibility judgment task from 28 Korean-English and 8 Japanese-English bilingual speakers confirmed the two research hypotheses of the study: selection between English heavy verbs and light verbs within a code-switched constituent and the degree of syntactic flexibility of a code-switched phrase are related to \ac{OV}-\ac{VO} variation in \ac{CS}.

There are several implications of the findings of the present study for an understanding of bilingual speakers’ linguistic competence and their linguistic creativity as well as human language forms in general. As argued in this monograph, light verbs may not represent the same syntactic category across languages; for instance, Korean and Japanese-type light verbs lexicalize the functional category \textit{v} whereas English-type light verbs realize \ac{ASP}. With access to a larger set of functional categories drawn from different languages, which may vary in their morphological forms, bilingual speakers are able to construct a wider range of \ac{CS} sentences, whose patterns may not be found in monolingual grammar. Otherwise, bilinguals are just like monolinguals in the sense that their grammars also reflect Universal Grammar, which is claimed to govern monolingual grammars in generative linguistics. Thus, \ac{CS} provides us with richer data to test with stronger confidence, the validity of linguistic theories and proposals primarily intended to account for the grammatical patterns of monolingual grammar. By studying the diverse and creative patterns of \ac{CS}, we are at a better disposal to understand how languages are parameterized similarly or differently in a given domain, which is the topic that generative linguists have pursued for a long time.

To provide a structural analysis of how \ac{OV} and \ac{VO} orders are distributed and derived in Korean-English and Japanese-English \ac{CS}, I first reviewed different approaches to word order in generative linguistics, non-derivational approaches such as the head directionality approach, derivational approaches and minimalist approaches. While the old head parameter approach states that the lexical head takes its complement either to its left or to its right, a revised head parameter has been proposed by restricting parametric variation into functional heads. The idea of treating functional heads and their morphosynctacic features as the locus of linguistic variation has been further developed in the Minimalist Program. 

The derivational approach, on the other hand, assumes that all languages share the same underlying structure from which various surface orders are derived. Among different proposals under the derivational approach, two opposite views were closely examined with respect to \ac{OV} and \ac{VO} derivations: (a) \ac{OV} is derived from \ac{VO} via object movement (\citealt{Kayne1994}) and (b) \ac{VO} is derived from \ac{OV} by verb movement (\citealt{Haider1992,Haider2000}). While word order is regarded to be established after syntax in the minimalist approaches, Kayne’s idea that the head-complement order (e.g. \ac{VO}) is universal was adopted in the minimalist syntax framework and his \acf{LCA} has been modified as a rule/condition/constraint applied to \ac{PF} and implemented in the bare phrase structure, in which structure is built via Merge between two lexical items directly. Although the notion of \textit{Merge} was adopted as a way to build syntactic structure in this monograph, the tree structure advocated in the X-bar theory, which precedes the bare phrase structure, was used for expository reasons to show the role of functional categories with respect to linguistic variation such as word order. 

The proposal that object shift is triggered by an \ac{EPP} specification of a syntactic category \textit{v} was further developed in the \textit{v}{}-\ac{ASP} structure, where Korean and Japanese light verbs represent \textit{v} and English light verbs lexicalize \ac{ASP}. And it was argued that features are inherited from \textit{v} to \ac{ASP} according to the four principles (\textit{Obligation}, \textit{Validation}, \textit{Selection}, and \textit{Expiration}) and three operational rules (\textit{Multiple} \textit{Agree} \textit{under} \textit{Asymmetry}, \textit{Earliness} and \textit{Economy}). The distribution of \ac{OV} and \ac{VO} orders in Korean-English and Japanese-English \ac{CS} was argued to be a result of object raising in derivations: when \ac{FI} occurs from \textit{v} = \textit{ha}\textsuperscript{KR}/\textit{su}\textsuperscript{JP} to \ac{ASP}, the object raises to Spec, \ac{ASP} triggered by the \ac{EPP} property on A\textsc{sp.} As a result, \ac{OV} order is derived. If \ac{FI} does not occur, on the other hand, the object stays in situ and the underlying \ac{VO} order surfaces. The \ac{FI}-based account of word order in monolingual and bilingual grammars shows that \ac{FI} is designed to facilitate a derivation to proceed economically and efficiently in the syntax, reflecting the very spirit of Minimalist syntax.
