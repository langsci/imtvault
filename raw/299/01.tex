\chapter{Introduction}\label{ch:1}

\section{Code-switching}\label{ch1:sect:CS}

\acused{CS} 
\acused{EPP}

\textit{Code-switching} (\ac{CS})  refers to the concurrent use of more than one language in a conversation, which is commonly observed in bilingual speech. Bilingual speakers may alternate from one language to another when they converse with other bilingual speakers. The term \textit{bilingual} \textit{speakers} or \textit{bilinguals} in this monograph is defined as people who speak more than one language, and given this definition there is no distinction made between \textit{bi}-linguals and \textit{multi}-linguals. Under this view, \ac{CS} may occur in any subset of the languages that the bilingual speaks, and it minimally involves two languages.

Much of the literature on \ac{CS}, especially in earlier years, focused on various social and pragmatic functions of \ac{CS} (e.g. \citealt{Auer1995,Barker1972,Gumperz1976,Zentella1995}), centering around the inquiry of social motivations for switching. Some of the earlier sociolinguistic research (e.g. \citealt{Labov1970}, \citealt{Lance1975}) characterizes \ac{CS} as a random occurrence and does not explain why switching may occur between sentences (\textit{inter-sentential} \ac{CS}) as in \REF{ex:1} or within a single sentence (\textit{intra-sentential} \ac{CS}) as in \REF{ex:2}.

\begin{exe} % sets up the top-level example environment \ex\label{here} Here is one. % example with running number
\ex\label{ex:1} I’m not very much in a hurry. \textit{Nuva noveye na khasoda khmbe}.	 \\
`I'm not very much in a hurry. If you have some soda to offer, let me have it.'   \hspace*{\fill} English-\textit{Lwidakho}\footnote{Lwidakho is a dialect of the Luyia language, which is spoken in Kenya and Uganda.} \\
  \hspace*{\fill} Adapted from \citet{Scotton1982}; quoted in \citet[]{MyersScotton2007} \\

\ex\label{ex:2} 
\begin{xlist} 
    \ex Right to 104th Street \textit{donde tenía una casa} which were furnished rooms. \\	
	‘Right to 104th Street where I had a house which were furnished rooms.’ \\
	  \hspace*{\fill}  English-\textit{Spanish} (\citealt[35]{SankoffPoplack})
    \ex 
    \gll Ka-yxxes bezzaf Dyal \textit{generaties} \textit{voorbijgaan}. \\
    it-must much of     generations pass \\
    \hspace*{\fill}  Arabic-\textit{Dutch} \\ \hspace*{\fill} (\citealt[139]{Nortier1990}, quoted in \citealt{Muysken1995})
\end{xlist} 
\end{exe} 

\newpage
Over the years, linguists have observed that \ac{CS} is not distributed randomly in the utterance: on the contrary, while \ac{CS} is possible in certain places, it seems to be forbidden in other positions in the utterance. Based on this observation, they have tried to identify the sites where switching is possible and impossible and why this is the case. It has been agreed that there are restrictions on \ac{CS} and different proposals have been put forth to formulate these constraints imposed on the patterns of \ac{CS}. Proposals that were mostly made in the 1980s and 1990s are best understood as language-pair specific rules or \ac{CS}-specific constraints: there are specific rules applied to or constrained in \ac{CS} (e.g. ‘the free morpheme constraint’ in \citealt{Poplack1980}, ‘the equivalence constraint’ in \citealt{Poplack1980,SankoffPoplack}, ‘the dual structure principle’ in \citealt{SridharSridhar1980}, ‘the closed class constraint' in \citealt{Joshi1985}, ‘the government constraint’ in \citealt{DiSciulloetal1986}, ‘the matrix code principle’ in \citealt{Kamwangamalu1989}, and ‘the functional head constraint’ in \citealt{Belazietal1994}). Yet, most of these proposals have been challenged by empirical data offered in subsequent work (e.g. \citealt{Lietal2018}).

On the other hand, more recent proposals hypothesize that both monolingual and bilingual grammars are subject to the same grammatical principles, which I will call \textit{the universal approach to \ac{CS}}. Researchers who adopt the universal approach to \ac{CS} have offered grammatical accounts that explain the patterns of \ac{CS}, and most of these accounts are claimed to be universal in the sense that the proposed accounts can be applied to any language pairs in \ac{CS}, therefore universal (e.g. \citealt{Belazietal1994,Chan2003,Chan2008,GonzalezLopez2011,GonzalezLopez2012,Lopez2020,MacSwan1999,Mahootian1996,Shim2013,Shim2016,Woolford1983}). Nonetheless, the claimed status of universality of most of these accounts has been questioned by cross-linguistic data, and there is little agreement of the precise nature regarding the rules involved in \ac{CS}. 

\largerpage[-1]
Despite this, the universal approach has several advantages over the proposal of \ac{CS}-specific rules or constraints. At the theoretical level, the universal model assumes the same underlying grammar for both monolinguals and bilinguals. Hence, we do not have to postulate a new theory or a set of rules that are specifically applied to \ac{CS}. This sense of `economy' or `parsimony' at the theoretical level also seems to be appealing in terms of language acquisition and language processing. Should there be an additional set of rules imposed on switching, a bilingual child who code-switches would be required to acquire a maximum set of grammars, such as his/her first language grammar, second language grammar and the rule of switching that may differ from the grammars of his/her two languages. Even if we assume that bilingual acquisition does not involve two separate grammars but one grammar with two sets of lexical/vocabulary items,\footnote{This is the view taken up by \citet{MacSwan1999}, who assumes that a bilingual speaker has access to two sets of lexicons from his/her two languages (L1 and L2) and constructs a sentence by drawing lexical items from them. The selected lexical items then feed one syntactic computational system, as a result of which \ac{CS} manifests. \citet{Lopez2020}, on the other hand, adopts the framework of Distributed Morphology, in which there is no real lexicon and what feeds the syntax is roots, not lexical items. According to this view, bilinguals have one linguistic competence just like monolinguals. In this monograph, I adopt the view proposed by MacSwan and assume that a bilingual speaker has access to a composite set of lexicons in his languages, and structure is built via one computational system using these lexical items.} 
the view of having a separate \ac{CS} grammar predicts that the bilingual child’s language acquisition is predicted to slow down due to mandatory learning of the maximum set of rules available in his/her language repertoire. Under these circumstances, it is not clear why \ac{CS} prevails among many, if not all, bilingual speakers. Thus, the universal approach, which provides a unified account of monolingual and bilingual speech, seems to be a better approach than \ac{CS} specific rules or constraints both at the theoretical level and at the acquisition level. To put it differently, there is no fundamental difference between monolingual and bilingual speakers in terms of their linguistic competence. By advocating the universal approach to \ac{CS}, \citet[6]{Lopez2020} says the following: 

\begin{quote}
When bilinguals code-switch, they do not simply go back and forth from one language to another. Nor is code-switching about inserting words of one language into the other or alternating from one language to the other. These shallow descriptions do not provide us with insight into the properties of code-switching because code-switching involves establishing a network of dependencies among the disparate constituents that conform a sentence structure.
\end{quote}

Taking the universal approach to \ac{CS}, this monograph investigates \ac{CS} between two language pairs, Korean and English and Japanese and English, which exhibit several common linguistic features that need to be explained. Due to their canonical word order differences, Korean and Japanese being \ac{SOV} and English \ac{SVO}, a code-switched sentence between Korean and English and between Japanese and English can take, in principle, either \ac{OV} or \ac{VO} order, following the grammars of the two languages of switching, as exemplified in \REF{ex:3} and \REF{ex:4} respectively.

\newpage

\begin{exe}
\ex\label{ex:3} \acs{OV}-ordered code-switched sentences: 
    \begin{xlist}
        \ex \gll Wonderful ideas-\textit{lul} \textit{mani} \textit{naynoh}-\textit{un}     \textit{kes} \textit{kath-ayo}.\\
        {} {~~~~~~}-\textsc{acc} a.lot present-\textsc{rel} thing seem-\textsc{decl}\\\hfill \hspace*{-5mm}\textit{Korean}-English
        \glt 	‘(They) seem to present many wonderful ideas.’	         \citep{Park1990}
        \ex 
        \gll Only small prizes \textit{moratta-ne}. \\
        {} {} {} get-\textsc{past} \\ \hfill \textit{Japanese}-English
        \glt 	‘(We) got only small prizes.’     \citep{Nishimura1986}
    \end{xlist}
\ex\label{ex:4} \acs{VO}-ordered code-switched sentences:
    \begin{xlist}
        \ex \gll 	I like koki. Koki ’s good. \\
        {} {} meat meat \\  \hfill \textit{Korean}-English
        \glt ‘I like meat. Meat’s good.’           \citep{Choi1991}
        \ex \gll We never knew \textit{anna} \textit{koto} \textit{nanka}. \\
        {} {} {} such thing sarcasm \\   \hfill \textit{Japanese}-English
        \glt 	‘We never knew such a thing as sarcasm.’              \citep{Nishimura1986}
    \end{xlist}
\end{exe}

To account for \ac{OV}-\ac{VO} variation in \ac{CS}, several researchers have proposed similar structural analyses, whose main claim is summarized in \REF{ex:5}.

\begin{exe}
\ex\label{ex:5} The language of the verb determines the position of the object in both monolingual and bilingual contexts (\citealt{MacSwan1999,Mahootian1993,Nishimura1997}).\footnotemark
\end{exe}

\footnotetext{\citet{MacSwan1999} adopts the Minimalist Program as a theoretical framework for his analyses, but he does not take the view that functional categories are considered as the locus of word order variation, as proposed in the Minimalist Program. Instead, he proposes that the lexical verb determines the placement of the object in \ac{CS}, which is the view by the head parameter approach. The head parameter approach will be explained in Chapter \ref{ch:2}.}

The claim in \REF{ex:5} predicts that in Korean-English and Japanese-English \ac{CS}, word order would be determined by the language of the verb: if the verb comes from either Korean or Japanese, \ac{OV} order would be obtained, following their respective grammars. On the other hand, if the verb is provided from English, \ac{VO} order would be expected, reflecting the default order of the language. This is confirmed by the above-mentioned Korean-English and Japanese-English \ac{CS} examples in \REF{ex:3} and \REF{ex:4}. In \REF{ex:3}, the verb comes from either Korean (\textit{naynoh} ‘present’ in (\ref{ex:3}a)) or Japanese (\textit{moratta} ‘get’ in ((\ref{ex:3}b)), thus exhibiting \ac{OV} order. If  the verb comes from English as in \REF{ex:4}, the code-switched sentences show \ac{VO} order.

However, the proposal in \REF{ex:5} fails to account for the order of the examples in \REF{ex:6}, where the verb comes from English (\textit{apply} in (\ref{ex:6}a) and \textit{mark} in (\ref{ex:6}b)). It is predicted that the sentence would be linearized in \ac{VO} order, following the language of the verb, which is English. Contrary to predictions, the surface order is \ac{OV} in \REF{ex:6}.

\begin{exe} \ex \label{ex:6}
    \begin{xlist}
        \ex \gll assistantship apply \textit{hay} \textit{noh-ass-eyo} \\
         {} {}  do.\textsc{lnk} put-\textsc{past}-\textsc{decl}		\\ \hfill 	\textit{Korean}-English
        \glt 	‘(I) applied for an assistantship.’   \citep{Park1990}
        \ex \gll one algebra question-\textit{o} mark-\textit{shite}				 \\
         {} {} {~~~~~~~~~~~~~}-\textsc{acc} {~~~~~~~~~}-do \\ \hfill \textit{Japanese}-English
        \glt	‘(You) mark one algebra question.’  \citep{Nishimura1995}
    \end{xlist}
\end{exe}

The limitations of the proposal in \REF{ex:5} has been noted by \citet{Chan2003,Chan2008}, who argues that the problematic cases that are not explained by \REF{ex:5} involve light verbs. For example, we see the Korean light verb \textit{ha} in (\ref{ex:6}a) and the Japanese light verb \textit{su} in (\ref{ex:6}b).\footnote{Korean and Japanese light verbs will be discussed in \sectref{ch1:sect:KRJP}.} Chan refers to the constructions of the type in \REF{ex:6} as ‘mixed compound verbs’ in which the complex verb consists of a host verb and a light verb provided from different languages in \ac{CS}: the host verb is in English (\textit{apply} and \textit{mark}) and the light verb comes from Korean or Japanese in \REF{ex:6}. With this observation, Chan proposes the following:

\begin{exe}
\ex\label{ex:7}  The complex verb of the light verb constructions behaves the same way as a simplex or a compound verb from the language of the light verb.
\end{exe}

However, as Chan questions himself, whether the complex verbs in \REF{ex:6} are genuine compounds is a controversial issue. For example, (sentential) negation \textit{an} ‘not’ can intervene between the lexical verb \textit{apply} and the Korean light verb \textit{ha} in (\ref{ex:6}a), which is shown in (\ref{ex:8}a). In fact, the only possible position of the negation marker \textit{an} is between the two verbs, \textit{apply} and \textit{ha}; the negation marker \textit{an} cannot precede the verb \textit{apply}, as shown in \REF{ex:8}.\footnote{All Korean-English bilingual speakers that I have consulted unanimously agreed on their judgements of the sentences in \REF{ex:8}.}

\begin{exe} \ex \label{ex:8}
    \begin{xlist} \judgewidth{*}
        \ex[]{ \gll assistantship apply \textit{an} \textit{hay} \textit{noh-ass-eyo} \\
         {} {}  \textsc{neg} do.\textsc{lnk} put-\textsc{past}-\textsc{decl}		\\}
        \ex[*]{\gll assistantship \textit{an} apply  \textit{hay} \textit{noh-ass-eyo} \\
         {} \textsc{neg} {} do.\textsc{lnk} put-\textsc{past}-\textsc{decl}		\\}
         \glt 	‘(I) did not apply for an assistantship.’ \\
    \end{xlist}
\end{exe}

The placement of \textit{an} with respect to the lexical verb \textit{apply} in an example of Korean-English \ac{CS} in \REF{ex:8} contrasts with the placement of \textit{an} in \REF{ex:9}, which is a serial verb construction in Korean where two verbs, \textit{ssip} ‘chew’ and \textit{mek} ‘eat’, form a compound and share the argument structure. Unlike the example in (\ref{ex:8}a), the two verbs cannot be separated by the negation marker \textit{an} in (\ref{ex:9}a), which must appear before the compound verb as in (\ref{ex:9}b). This subsequently disproves Chan's claim that the complex verb of a light verb construction acts similarly to a compound verb from the language of the light verb. 

\begin{exe}\ex \label{ex:9}
    \begin{xlist}
        \ex[*]{\gll Joa-ka koki-lul    ssip-e   	  an   mek-see-ta \\
        Joa-\textsc{nom} meat-\textsc{acc} chew-\textsc{lnk} \textsc{neg} eat-\textsc{past}-\textsc{decl} \\}
        \ex[]{ \gll Joa-ka 	   koki-lul    an    ssip-e       mek-see-ta \\
        	Joa-\textsc{nom} meat-\textsc{acc} \textsc{neg} chew-\textsc{lnk} eat-\textsc{past}-\textsc{decl} \\
        \glt  ‘Joa did not chew-and-eat meat.’}
    \end{xlist}
\end{exe}

\citet{Chan2008} further proposes that the language of functional categories determines the position of their complements and light verbs instantiate the functional category \acs{I(nfl)} and the position of its \ac{VP} complement. In other words, if the light verb comes from Korean or Japanese, the \ac{VP} complement will be placed before the light verb, which correctly describes the patterns in \REF{ex:6}: the \ac{VP} in English precedes either the Korean light verb \textit{ha} in (\ref{ex:6}a) or the Japanese light verb \textit{su} in (\ref{ex:6}b).\footnote{In the Minimalist Program, I(nfl) takes \textit{v}P, not \ac{VP} as its complement.} Yet, this account does not explain why the object also appears before the English lexical verb inside the \ac{VP} in \REF{ex:6}. One may argue that the language of the light verb also determines the position of the object, therefore, the Korean and Japanese style \ac{OV} order is derived in \REF{ex:6}. However, this does not seem to be the case. 

In \REF{ex:10}, the Korean light verb \textit{ha} and the Japanese light verb \textit{su} take the \ac{VP} complement to their left, but the object follows the verb, exhibiting the English style \ac{VO} order, which is in contrast with the \ac{OV} order inside the \ac{VP} in \REF{ex:6}. To summarize, when a \ac{VP} is code-switched into English in Korean-English and Japanese-English \ac{CS}, the \ac{VP} precedes the Korean light verb \textit{ha} or the Japanese light verb \textit{su} and the linear order between the object and the verb inside the \ac{VP} may alternate between \ac{OV} and \ac{VO} orders.

\begin{exe} \ex \label{ex:10}
    \begin{xlist}
    \ex \gll catch up cold \textit{ha-myen}\footnotemark	\\
    {} {} {} do-if \\    \hfill \textit{Korean}-English
    \glt ‘If (you) catch up a cold \ldots’(\citealt{Park1990}) \\

    \ex \gll \textit{yooshi} keep an eye \textit{suru-zo} \\
    	{well I’m going to} {} {} {} do-\textsc{prt} \\\hfill \textit{Japanese}-English
    \glt ‘Well, I’m really going to keep an eye on you.’  (\citealt{NambaND})
    \end{xlist}
\end{exe}

\footnotetext{The phrase ‘catch up cold’ is ungrammatical in English, which must be ‘catch a cold’. Perhaps the speaker who uttered the sentence may not be a balanced bilingual speaker but a second language (L2) learner of English. The detailed description of the bilingual speakers included in the study is not provided in \citet{Park1990}.}

Chan treats examples such as \REF{ex:10} as exceptional cases whose word order is not predicted by his proposal in \REF{ex:7}. For such exceptional examples, he reasons that corpus \ac{CS} data between \ac{OV} and \ac{VO} languages show that the \ac{VO} sequence in light verb constructions in \ac{CS} is rarer than the \ac{OV} pattern in a similar environment. However, lower frequency of \ac{VO} order \textit{per se} does not justify Chan’s decision that they are not subject to universal principles. The \ac{CS} literature clearly shows that both \ac{OV} and \ac{VO} orders exist in various \ac{OV}-\ac{VO} language pairs in \ac{CS} (e.g. Hindi-English, Punjabi-English, Tamil-English), and \ac{OV}-\ac{VO} variation in \ac{CS} should be accounted for, which is the aim of this monograph. Although Chan’s analysis fails to correctly account for the \ac{OV}-\ac{VO} variation documented in the \ac{CS} literature, he rightly points out that code-switched sentences that vary between \ac{OV} and \ac{VO} orders involve a light verb in diverse language pairs, which was not noticed in the earlier studies. 

The role of light verbs in \ac{CS} was investigated in great detail in a few subsequent studies (e.g. \citealt{GonzalezLopez2011,GonzalezLopez2012} for Spanish-German; \ac{CS} \citealt{Shim2011,Shim2013,Shim2016} for Korean-English and Japanese-English \ac{CS}). For instance, in German the object is normally placed before the verb in a construction involving an auxiliary or a modal verb, as shown in \REF{ex:11}. However, when Spanish-German \ac{CS} involves the Spanish light verb \textit{hacer} ‘do’ as in \REF{ex:12}, the object cannot precede the verb but must follow it, thus exhibiting Spanish-style \ac{VO} order.

\begin{exe}\ex\label{ex:11}   
    \begin{xlist}
        \ex{\gll ~~Hans hat die Bücher verkauft. \\
           ~~Hans has the books   sold \\}\jambox*{German}
         \ex[*]{\gll Hans hat verkauft die Bücher\\
         Hans has sold       the books \\
        \glt 	‘Hans has sold the books.’}
         \ex[]{ \gll  Hans muss die Bücher verkaufen. \\
        Hans must the  books	 sell	\\}
        \ex[*]{\gll Hans muss verkaufen die Bücher.  \\
         Hans must  sell the books \\
        \glt ‘Hans has to sell the books.’	 	    Modified from \citet[42,~(15)]{GonzalezLopez2012}}
    \end{xlist}
    \ex\label{ex:12} 
    \begin{xlist}
        \ex{\gll ~~\textit{Juan} \textit{hizo} verkaufen die Bücher.  \\
        ~~Juan did   sell	         the books \\}
          \hfill \textit{Spanish}-German
         \ex[*]{\gll \textit{Juan} \textit{hizo} die Bücher verkaufen \\
         Juan  did  the books	sell \\
        \glt ‘Juan sold the books.’}
         \ex[]{ \gll \textit{Juan} \textit{ha}  \textit{hecho} verkaufen die Bücher. \\
         Juan has done   sell	      the books \\}
        \ex[*]{\gll \textit{Juan} \textit{ha} \textit{hecho} die Bücher verkaufen. \\
         Juan  has done	 the books    sell\\
       \glt  ‘Juan has sold the books.’	  Modified from \citet[42,~(16)]{GonzalezLopez2012}}
    \end{xlist}
\end{exe}

\citeauthor{GonzalezLopez2011} analyze the Spanish verb \textit{hacer} ‘do’ as a light verb lexicalizing \textit{v}, and the order of \ac{VP} in \REF{ex:12} is determined by \textit{v} in Spanish, which is parameterized to take the object to the right of the verb.\footnote{In the Minimalist Program, a syntactic category \textit{v} represents a \textit{light verb} to which V(erb) overtly raises \citep[315]{Chomsky1995}.} The claim that \citeauthor{GonzalezLopez2011} make is summarized below. 

\begin{exe}
\ex \label{ex:13} The order of the verb and its complement/direct object is determined by \textit{v}, which has a binary feature that decides whether the object should be linearized to the left or to the right of the verb. 
\end{exe}

They further argue that if \textit{v} comes from the lexicon of an \ac{OV} language, the order will be \ac{OV}. By contrast, if \textit{v} is extracted from the lexicon of a \ac{VO} language, the constituents in the \ac{VP} will surface in \ac{VO} order. The account by González-Vilbazo and López is similar to \citegen{Chan2008} in the sense that the functional category represented by a light verb (\textit{v} in González-Vilbazo and López and I in Chan) is parametrized and determines word order. However, their proposal is different from Chan’s: what determines word order is not the \textit{language} of the light verb, but the feature specification of \textit{v}, which may be parameterized differently across languages. In this respect, the proposal by González-Vilbazo and López is grounded on the fundamental concepts assumed in minimalist syntax: the locus of linguistic variation is due to (morphosyntactic) features specified on functional categories. Yet, they do not discuss how these features play a role in deriving different word orders. Instead, they claim that features on \textit{v} are binary, taking either a preverbal object (e.g. German) or a post verbal object (e.g. Spanish). Interestingly, in \textit{hacer} constructions of Spanish-German \ac{CS}, one may find \ac{OV} order in addition to \ac{VO} order, as exemplified in \REF{ex:14}, which seems to challenge their proposal.

\begin{exe}\ex \label{ex:14}
    \begin{xlist}
    \ex \gll \textit{Juan} \textit{hizo} verkaufen die Bücher.		\\
     Juan did  sell   the books\\
      \hfill \textit{Spanish}-German
    \ex ~[\textsubscript{\textit{v}P} \textit{Juan} \textit{v}\textsubscript{\textsc{sp}} = \textit{hizo} [\textsubscript{\ac{VP}} verkaufen die Bücher]]
    \ex \gll \textit{Juan} \textit{hizo} \ldots ~die Bücher verkaufen. \\
    Juan did \ldots ~the books   sell\\
    \ex ~[\textsubscript{CP} \textit{Juan} \textit{hizo} \ldots] [\textsubscript{\textit{v}P} [\textsubscript{\ac{VP}} die Bücher verkaufen] \textit{v}\textsubscript{\textsc{gr}} = $\varnothing$] \\
    \end{xlist} 
      \hfill Modified from \citet[52,~(59)]{GonzalezLopez2012}
\end{exe}
	
The Spanish-German \ac{CS} examples in (\ref{ex:14}a) and (\ref{ex:14}c) have the same lexical material from the two languages, including the Spanish light verb \textit{hacer}. Nonetheless, the exhibited word order of the \ac{VP} differs: \ac{VO} and \ac{OV}, respectively. González-Vilbazo and López argue that the \ac{VO} vs \ac{OV} contrast found in Spanish-German \ac{CS} in \REF{ex:14} is due to the selection of \textit{v} from different languages: \textit{v} comes from Spanish in (\ref{ex:14}a) and from German in (\ref{ex:14}c), which are parameterized differently. The Spanish light verb \textit{hacer} lexicalizes \textit{v} in (\ref{ex:14}a) and the object is linearized to the right of the verb. In (\ref{ex:14}c), on the other hand, \textit{v} is null (which is not spelled out/pronounced by a lexical item) and comes from German, as a result of which the object is linearized to the left of the verb. They analyze the code-switched sentence in (\ref{ex:14}b) as a \textit{v}P, which is separate from the main clause \ac{CP} and includes an elided site. They call this hanging \textit{v}P ‘an orphan’, which is not integrated as a constituent into a larger clause structure (cf \citealt{Haegeman1991}, \citealt{Shaer2003}). Thus, the underlying structure of the sentence in (\ref{ex:14}a) and that in (\ref{ex:14}c) differ from each other, as indicated in (\ref{ex:14}b) and (\ref{ex:14}d). 

To support their analysis, González-Vilbazo and López show that the prosodic structure of the \ac{VO}-ordered \textit{hacer}-construction in (\ref{ex:14}a) and that of the \ac{OV}-ordered \textit{hacer}-construction in (\ref{ex:14}c) are different. While it is not allowed to pause between \textit{hizo} (hacer) and the following \ac{VP} in \REF{ex:14}a, the sentence in (\ref{ex:14}c) is only acceptable when there is a pause between \textit{hizo} and the \ac{VP}. Based on this, they argue that the prosodic isolation of the \ac{VP} reflects the syntactic isolation of the \ac{VP} in (\ref{ex:14}c), justifying the structure proposed in (\ref{ex:14}d). According to them, the structure in (\ref{ex:14}d) is the only possible structure for \ac{OV}-ordered \textit{hacer} light verb constructions in Spanish-German \ac{CS}. In addition, González-Vilbazo and López present a volume of syntactic and other linguistic evidence to support their orphan approach to (\ref{ex:14}c), such as extraction, anaphor binding, scrambling, prosody. Readers are recommended to refer to González-Vilbazo and López (2012) for details.

 While González-Vilbazo and López’s analysis successfully accounts for the \ac{VO} vs \ac{OV} order found in \textit{hacer} light verb constructions in Spanish-German \ac{CS}, their proposal cannot be extended to the case of Korean-English and Japanese-English \ac{CS}. Unlike Spanish-German \ac{CS}, there is no noticeable difference between \ac{OV}-ordered and \ac{VO}-ordered \textit{ha} light verb constructions in Korean-English \ac{CS} and \textit{su} light verb constructions Japanese-English \ac{CS}: for instance, \textit{ha}/\textit{su} light verb constructions do not exhibit any prosodic difference between \ac{OV} and \ac{VO} orders in Korean-English and Japanese-English \ac{CS}. Thus, \ac{OV}-\ac{VO} variation in Korean-English and Japanese-English \ac{CS} cannot be explained by the position of the \textit{v}P, either inside or outside the matrix clause, as argued by González-Vilbazo and López for Spanish-German \ac{CS}. 

The unresolved problem of the distribution of \ac{OV} and  \ac{VO} orders in Korean-English and Japanese-English \ac{CS} was taken up in \citet{Shim2011}, who analyzed Korean-English and Japanese-English \ac{CS} data available in the literature. Some of the examples that are presented in her study are reproduced below. 

\begin{exe}
\ex\label{ex:15}   \ac{OV}-ordered code-switched sentences
    \begin{xlist}
        \ex \gll 	too much money-\textit{lul}  spend \textit{hayss-eyo}	\\
          {} {} {~~~~~~~~~~~}-\textsc{acc} {} do.\textsc{past}-\textsc{decl} \\ \hfill \textit{Korean}-English
        \glt 	‘(He) spent too much money.’   \citep{NishimuraYoon1998}
        \ex \gll 	Jasmin, aunt-\textit{nun} \textit{ce} store own \textit{hay} \\
        {} {~~~~~~~}-\textsc{top} that {} {} do.\textsc{decl}		\\  \hfill	\textit{Korean}-English
        \glt 	‘For Jasmine, her aunt owns that store.’  \citep{Choi1991}
        \ex \gll Meena, basket-\textit{an-eyta}  all the toys-\textit{lul}    \textit{ppali} put \textit{ha-ko} \textit{cip-ey} \textit{ka-ca}		 \\
        {} {~~~~~~~~~~}-inside-\textsc{loc} {} {} {~~~~~~}-\textsc{acc} quickly {} do-\textsc{comp} home-\textsc{loc} go-\textsc{adh} \\ \hfill	\textit{Korean}-English
        \glt 	‘Meena, let’s put all the toys in the basket quickly and go home.’  \citep{Lee1991}
        \ex \gll Chinese food order \textit{suru-tte} \\
        {} {} {} do-\textsc{quo} \\ \hfill \textit{Japanese}-English
        \glt ‘(He) said that (they) will order Chinese food.’ \citep{Nishimura1995}
    \end{xlist}
\newpage
\ex\label{ex:16}  \ac{VO}-ordered code-switched sentences
    \begin{xlist}
        \ex \gll catch up cold 	\textit{ha-myen} \\
        {} {} {} do-if \\  \hfill	\textit{Korean}-English
        \glt 	‘If (you) catch up a cold \dots’	 \citep{Park1990}
        \ex \gll 	put the burden on myself \textit{ha-ketunyo}	 \\
       {} {} {} {} {} do-\textsc{decl} \\ \hfill	\textit{Korean}-English
        \glt  	‘(I) put the burden on myself.’  \citet{Park1990}
        \ex \gll \textit{yooshi} keep an eye {suru-zo} \\
        {well I’m going to} {} {} {} do-\textsc{prt} \\ \hfill	\textit{Japanese}-English
        \glt 	‘Well, I’m really going to keep an eye on you.’ \citep{NambaND}
    \end{xlist}
\end{exe}

Shim suggests that the distribution of \ac{OV}-\ac{VO} orders in Korean-English and Japanese-English \ac{CS} is closely related to the \textit{heavy} vs \textit{light} distinction of the verb within the code-switched constituent; while the verbs in \REF{ex:15} are \textit{heavy} in the sense that they deliver lexical semantic information to their clausal structure (e.g. \textit{spend} in \ref{ex:15}a, \textit{own} in \ref{ex:15}b, \textit{order} in \ref{ex:15}d), the verbs in \REF{ex:16} (e.g. \textit{catch} in \ref{ex:16}a, \textit{keep} in \ref{ex:16}c) are \textit{light} and have little semantic content of their own. She also postulates that \ac{OV}-\ac{VO} variation in Korean-English and Japanese-English \ac{CS} could also be related to the size of the code-switched constituent: the verb and the internal argument/object are separately code-switched into English in \REF{ex:15} without changing the \ac{OV} order of Korean or Japanese. In contrast, idioms such as \textit{catch} \textit{a} \textit{cold}, \textit{keep} \textit{an} \textit{eye} \textit{(on)} in \REF{ex:16} may be listed in the lexicon and switched as a unit, thus, the internal \ac{VP} structure is intact and preserves the  \ac{VO} order of English during switching. These two hypotheses, namely (i) selection of light verbs and (ii) idiomaticity, were tested against the judgments solicited from a small number of Korean-English and Japanese-English bilingual speakers. Based on the results from the questionnaires, she argues that the distribution of \ac{OV} and  \ac{VO} orders in Korean-English and Japanese-English \ac{CS} is related to (a) the status of the verb (heavy vs light) within a code-switched constituent and (b) the compositionality of the code-switched constituent. Based on this, she concludes the following:

\begin{exe}
\ex\label{ex:17}
\begin{xlist}
    \ex Selection of light verbs results in \ac{OV}-\ac{VO} variation in both non-id\-i\-o\-mat\-ic (literal) phrases and compositional idioms. More specifically, when light verbs are selected from Korean or Japanese, \ac{OV} is generated, following the grammar of Korean or Japanese. On the contrary, selection of English light verbs results in the English-style  \ac{VO} order in the derivation.
    \ex Compositional idioms undergo aspectual composition, similar to non-idiomatic phrases, while non-compositional idioms do not undergo aspectual composition in the syntactic derivation.\footnotemark  
    \ex Non-compositional idioms undergo code-switching as a unit, and the internal order of the code-switched phrase is maintained throughout the derivation.
\end{xlist}
\end{exe}

\footnotetext{Shim proposes a \textit{v}P structure where the \textit{v} head selects an \acs{ASP}P (\acl{ASP} phrase) \acused{ASO} and the Korean light verb \textit{ha} and the Japanese light verb \textit{su} lexicalize v. She also argues that the aspectual feature on \textit{v}, [$\pm$stative], must match the corresponding feature on \ac{ASP}. Although she does not provide a clear definition of aspectual composition, she explains that the aspect of a predicate is not determined by the feature on the \ac{ASP} head per se, but it is composed within the domain of the projection of \ac{ASP}, including its complement. In this monograph, I differ from my own earlier view in \citet{Shim2011} and argue that it is the \ac{VP}, not the \acs{ASP}P, where the aspectual properties of the predicate is determined. This will be discussed in detail in Chapter \ref{ch:3} (\sectref{ch3:sect:3.2.1}). }

\section{The present study}\label{ch1:sect:PresentStudy}

The present study is an updated version of \citegen{Shim2011} case study with a series of modification. Prior to explaining how the present study has been modified from \citet{Shim2011}, let us begin by providing two research questions that were explored in \citet{Shim2011}. 

\begin{exe}
\ex\label{ex:18}\textbf{ Research Question 1} \\
What is the role of light verbs in \ac{CS}? How is \ac{OV}-\ac{VO} variation in Korean-English and Japanese-English \ac{CS} related to the choice between heavy/lexical verbs and light/functional verbs? 
\end{exe}

Lexical categories are also called ‘content words’ or ‘open class’ or ‘substantive morphemes’, and bear a specific semantic content of their own (e.g. nouns, verbs, adjectives). Functional categories, on the other hand, do not bear such idiosyncratic meaning but carry out various grammatical functions (e.g. tense), which are also referred to as ‘abstract words’ or ‘closed class’ or ‘grammatical morphemes’. As mentioned before, the locus of linguistic variation is claimed to be functional categories rather than lexical categories in the minimalist framework. Also, a syntactic functional category \textit{v} represents a \textit{light} \textit{verb} and the precise syntax of light verbs differs from language to language (\citealt{Adger2003,Butt2003} etc.), which leads to linguistic variation, including word order. Adopting these views, this study explored how light verbs from typologically different languages, such as English, Korean, and Japanese, play a role in deriving \ac{OV} and  \ac{VO} orders in Korean-English and Japanese-English \ac{CS}. The following hypothesis was made with respect to the first research question.

\begin{exe}
\ex\label{ex:19} \textbf{Research Hypothesis 1} \\
Assuming that linguistic variation is determined by the way features are parameterized in functional categories and how these features are valued in syntactic derivations, \ac{OV}-\ac{VO} variation in Korean-English and Japanese-English \ac{CS} will be determined by feature specifications on functional categories represented by light verbs in Korean, Japanese and English and how these features are valued in syntactic derivations.
\end{exe}


In addition to the role of light verbs in \ac{CS}, this study also examined various idiomatic expressions in English and how they contributed to the distribution of \ac{OV} and  \ac{VO} orders in Korean-English and Japanese-English \ac{CS}. Since there is little consensus in the literature on what constitutes idioms or even compositionality, a working definition of \ac{VP} idioms was made. 

\begin{exe}
\ex\label{ex:20} The term \textit{\ac{VP} idiom} refers to a \ac{VP} where the verb takes a complement and the verb and its complement together deliver a non-literal, idiosyncratic reading.
\end{exe}

Taking into account that idioms are not stored as a chunk but built in the syntax just like non-idiomatic phrases (\citealt{EpsteinSeely2006,Nediger2017}) and they differ with respect to their relative syntactic flexibility, the second research question and the hypothesis were constructed in the following way. 

\begin{exe}
\ex\label{ex:21} \textbf{Research Question 2} \\
Does syntactic flexibility play a role in \ac{OV}-\ac{VO} variation in Korean-English and Japanese-English \ac{CS}? Are both syntactically flexible {phrases}  and inflexible phrases subject to \ac{CS}?

\ex\label{ex:22}  \textbf{Research Hypothesis 2} \\
Syntactically flexible phrases and inflexible phrases will behave differently with respect to word order derivation in \ac{CS}. More specifically, while the internal argument of a syntactically flexible phrase is subject to \ac{CS}, a syntactically inflexible phrase is frozen and undergoes \ac{CS} as a unit. Hence, the internal order of the phrase will be maintained throughout the derivation.
\end{exe}

In the following, light verbs and idioms, which were investigated in relation to \ac{OV}-\ac{VO} order variation in Korean-English and Japanese-English \ac{CS}, are explained in further detail. 
	
\subsection{Light verbs and light verb constructions}\label{ch1:sect:LV}

\subsubsection{English}\label{ch1:sect:English}

The term \ac{LV} was first introduced by \citet{Jespersen1965}, who referred to the verbs in English V+\acs{NP} (\acl{NP}) \acused{NP} constructions such as \textit{have}, \textit{take}, \textit{make}, \textit{give}, \textit{get}, and \textit{do} in \textit{have} \textit{a} \textit{look}, \textit{take} \textit{a} \textit{walk}, \textit{make} \textit{a} \textit{plunge}, \textit{give} \textit{a} \textit{sigh}, \textit{get} \textit{a} \textit{move} \textit{on}, \textit{do} \textit{a} \textit{bunk} ‘run away’, etc. Jespersen describes a light verb as “an insignificant verb, to which the marks of person and tense are attached, before the really important idea” (1965: 117). As their very name implies, light verbs are ``light'' in the sense that they have little semantic content of their own, and contrast with \textit{lexical} or \textit{full} or \textit{heavy} verbs which deliver idiosyncratic lexical semantic information to their clausal structure. Examples in \REF{ex:23} illustrate the contrast between heavy and light verbs.

\begin{exe}\ex\label{ex:23}
\begin{xlist}
    \ex\label{ex:23a} Bibi made a snowman.
    \exi{a.\parbox{0mm}{$'$}} Bibi made a dash across the puddle.
    \ex\label{ex:23b} Bibi took Joa to Bordeaux.
    \exi{b.\parbox{0mm}{$'$}} Bibi took a walk to Bordeaux.
\end{xlist}
\end{exe}

The verbs in (\ref{ex:23}a, b) are lexical/heavy/full verbs (\textit{made}, \textit{took}) while the same verbs are used as light verbs in (\ref{ex:23}a', b'). For instance, the verb \textit{make} in (\ref{ex:23}a) has the meaning of creating or bringing about a physical entity out of something, but \textit{make} in (\ref{ex:23}a') does not convey this meaning at all. Instead, it combines with the noun phrase \textit{a} \textit{dash}, and the whole phrase \textit{make} \textit{a} \textit{dash} seems to be used as a single predicate in the sentence: \textit{make} \textit{a} \textit{dash} and \textit{take} \textit{a} \textit{walk} are equivalent to \textit{dash} and \textit{walk}, respectively. 

Jespersen identified the following verbs as light verbs in English; \textit{have}, \textit{take}, \textit{make}, \textit{give}, \textit{get}, and \textit{do}, which appear in various light verb constructions productively. In addition, other verbs such as \textit{pay}, \textit{offer}, \textit{put}, \textit{raise}, and \textit{hold} may also participate in light verb constructions (\citealt[296,~1093]{HuddlestonPullum2002}), as shown in \REF{ex:24}. 

\begin{exe}\ex\label{ex:24}
\begin{xlist}
\ex offer an apology, a suggestion
\ex pay attention (to), a call (to), a visit (to)
\ex put the blame (on), an end/stop (to)
\ex raise an objection (to)
\ex hold the belief
\end{xlist}
\end{exe}

\citeauthor{HuddlestonPullum2002} refer to light verbs as ‘light uses of verbs’ (2002: 290), which differs from their heavy uses. To put it differently, a verb may be used as a heavy verb or a light verb and participate in a light verb construction. Light verbs are similar to auxiliaries, which lack inherent lexical meaning. In addition, light verbs straddle the divide between a functional and a lexical category since they originate from lexical elements but do not predicate the same way as main verbs do (\citealt[4]{Butt2003}).

Researchers have also proposed that light verbs should be acknowledged as an independent syntactic category (\citealt{Adger2003,Butt2003}, among others) and \textit{v} is argued to represent a light verb to which V(erb) overtly raises (\citealt{Arad1999,Chomsky1995,Harley1999,Marantz1997}). Yet, the precise syntax of light verbs varies across languages. The idea of \textit{v} goes back to \citet{Chomsky1957}, who introduced it for auxiliaries and modals. In recent syntactic analyses, however, auxiliaries and modals have been re-analyzed as an independent syntactic category such as \textsc{aux} and \textsc{mod}, which inherently lack any lexical meaning (\citealt{Butt2003}). \Acp{LV}, on the other hand, may or may not have an effect on the argument structure of the clause, but they may contribute to the event structure although they do not have full semantic charge as their counterpart heavy/full verbs do.\footnote{In this monograph, I will argue that \acp{LV} should be distinguished from \acp{LVC} and inherently light verbs do not contribute to the event structure/aspect of the predicate in English. This will be discussed in Chapter \ref{ch:3} (\sectref{ch3:sect:3.2.1}).} In other words, light verbs remain at an intermediate stage between the full verb and the auxiliary on the cline of grammaticalization (\citealt{Hook1974}), as \REF{ex:25} shows. 

\begin{exe}
\ex \label{ex:25} 	(heavy) verb > light verb > auxiliary > clitic > affix
\end{exe}

Assuming that heavy or lexical verbs represent a lexical category V(erb) where\-as light verbs lexicalize a functional category such as \textit{v}, as proposed in the Minimalist Program, we may extend this view to the idea that the verbs occurring in light verb constructions normally belong to a syntactic category V, but may also lexicalize \textit{v} in their light uses. However, among those verbs that participate in light verb constructions (e.g \textit{have}, \textit{take}, \textit{make}, \textit{give}, \textit{get}, \textit{do} \textit{pay}, \textit{offer}, \textit{put}, \textit{raise}, and \textit{hold}) only a subset of them have been re-analyzed as ‘true light verbs’ by different researchers, which characterize one or more abstract functional heads in all of their uses. Examples are provided in \REF{ex:26}.

\begin{exe}\ex\label{ex:26}
\begin{xlist}
    \ex \textit{have} = \textsc{be} + \textsc{to} \hfill \citep{Benveniste1966,DenDikken1995,Kayne1993} \\
    \textit{have} = P\textsubscript{HAVE}			\hfill \citep{Harley1997} \\
    \textit{have} = \textit{v}\textsubscript{BE} + \textsc{appl} \hfill \citep{Kim2012}
    \ex \textit{get} = \textsc{become} \hfill \citep{McIntyre2005a,McIntyre2005b} \\
    \textit{get} = \textsc{inch} \hfill \citep{Shim2006}
    \ex \textit{give} = \textsc{cause} \textit{v}\textsubscript{CAUSE} + P\textsubscript{HAVE} \hfill \citep{Harley2003}
\end{xlist}
\end{exe}

All of the syntactic analyses provided in \REF{ex:26}  and other similar accounts adopt the so-called lexical decompositional theory in generative semantics, championed by \citet{Dowty1979}, in which abstract operators such as \textsc{caus(e)}, \textsc{do}, and \textsc{become} are proposed to convey the nuanced meaning of a predicate. Such abstract operators do not directly correspond to the English lexical verbs such as \textit{cause}, \textit{do}, and \textit{become} but stand for abstract semantic units and characterize abstract elements in the syntax as well.\footnote{Instead of Dowty's \textrm{\textsc{become}} \textrm{operator, \citet{Shim2006} uses} \textrm{\textsc{inch}}\textrm{, to indicate pure inchoativity, for the reasons argued in her paper.}} Thus, while verbs like \textit{cause}, \textit{do}, and \textit{become} are categorized as V, which is a lexical category, abstract elements such as \textsc{caus}, \textsc{do}, and \textsc{become} instantiate a functional category, variants of \textit{v} in the structure. 

Assuming that the analyses in \REF{ex:26} are on the right track of capturing the core properties of these verbs as true light verbs, we now face the task of distinguishing verbs like these, which are inherently light verbs, from the verbs that are normally heavy or lexical verbs but have additional light uses. In this regard, we need to clarify the terms \textit{light} \textit{verbs}, \textit{light} \textit{uses} \textit{of} \textit{verbs}, and \textit{light} \textit{verb} \textit{constructions}, which have been used interchangeably in the literature. As \citet{Butt2003} points out, there exist a great diversity of analyses and terminology of light verbs in the literature. Thus, I provide a new working definition of light verbs and light verb constructions for the purpose of the present study.

\begin{exe}\ex \label{ex:27}
\begin{xlist}
    \ex A \textbf{light verb} lacks idiosyncratic lexical meaning of its own, but only lexicalizes an abstract functional head. 
    \ex 	In a \textbf{light verb construction}, the verb does not contribute any lexical-semantic information, but only its complement does. Both heavy and light verbs may participate in light verb constructions. 
\end{xlist}
\end{exe}

Based on the definition in (\ref{ex:27}a), eight English verbs were selected and included as light verbs in the present Korean-English and Japanese-English \ac{CS} study, as listed in \REF{ex:28}.

\begin{exe}\ex\label{ex:28}
\textit{have} \textit{=} \textsc{be} \textsc{+} \textsc{to}  \\
\textit{give} = \textsc{caus} \textsc{[be} \textsc{+} \textsc{to]}  \\   
  \textit{get} = \textsc{inch}           \\
  \textit{take} = \textsc{inch} \\
  \textit{make} = \textsc{caus} + exist \\        
  \textit{keep} = \textsc{caus} \textsc{+} \textsc{be} \\
\textit{hold} = \textsc{caus} \textsc{+} \textsc{be} \\
\textit{raise} = \textsc{caus} \textsc{+} \textsc{go} (up)
\end{exe}

While we leave a detailed discussion of the English light verbs listed in \REF{ex:28} in Chapter \ref{ch:5} (\sectref{ch5:sect:5.7}), it is important to note that these light verbs may also participate in light verb constructions (e.g. \textit{have} \textit{a} \textit{look}, \textit{get} \textit{a} \textit{sense}, \textit{give} \textit{a} \textit{smile}). In contrast to light verbs, the term \textit{heavy} \textit{verb} refers to a lexical verb, such as \textit{pay} or \textit{play}, and they may also participate in a light verb construction (e.g. \textit{pay} \textit{a} \textit{visit} or \textit{play} \textit{a} \textit{joke}). 

\begin{exe} \ex \label{ex:29}
\begin{xlist}
    \ex Light verbs in a non-light verb construction	(e.g.\textit{have a sister})
    \ex Heavy verbs in a non-light verb construction	(e.g. \textit{pay the bill})
    \ex Light verbs in a light verb construction 	(e.g. \textit{have a look})
    \ex Heavy verbs in a light verb construction 	(e.g. \textit{pay a visit})
\end{xlist}
\end{exe}

The syntax of English light verbs will be discussed in Chapter \ref{ch:3} (\sectref{ch3:sect:3.2.1}) and in Chapter \ref{ch:5} (\sectref{ch5:sect:5.7}).

\subsubsection{Korean and Japanese}\label{ch1:sect:KRJP}

Since Jespersen’s original coinage, the terms \textit{light} \textit{verbs} and \textit{light} \textit{verb} \textit{constructions} have been adopted by a number of researchers to analyze various forms of complex predicates in many languages (e.g. \citealt{Campbell1989,Cattell1984,Kearns1989} for English; \citealt{Ahn1991,ParkK1992} for Korean; \citealt{Dubinsky1997,GrimshawMester1988} for Japanese; \citealt{Butt1995,ButtGeuder2001} for Urdu; \citealt{KarimiDoostan1997} for Persian; \citealt{Wittenberg2016} for German, to name a few). 

One of the most productive forms in light verb constructions involves the verb meaning ‘do’ across languages. Korean and Japanese also have light verb constructions with \textit{ha} ‘do’ and \textit{su} ‘do’, respectively. For instance, in \REF{ex:30}, the Korean light verb \textit{ha} and the Japanese light verb \textit{su} are attached to the Chinese origin verbal noun \textit{hapsek} and \textit{aiseki} ‘table-sharing’. 

\begin{exe}\ex\label{ex:30}
\begin{xlist}
\ex \gll Bibi-nun Joa-wa 	hapsek-hayss-ta	\\
Bibi-\textsc{top} Joa-with	table.sharing-do.\textsc{past}-\textsc{decl} \\
\hfill Korean
\ex \gll Bibi-nun Joa-to 	aiseki-shita \\
Bibi-\textsc{top} Joa-with	table.sharing-do.\textsc{past} \\
Japanese
\glt ‘Bibi shared a table with Joa.’	  Modified from \citet[(2)]{GrimshawMester1988}
\end{xlist}\end{exe}

Korean and Japanese light verb constructions have been extensively discussed in the literature (e.g.\citealt{Ahn1991,Bak2011,Chae1996,ChoiWeschler2001,Dubinsky1994,GrimshawMester1988,Jun2003,Kim1991,ParkK1992,SaitoHoshi2000,ShimadaKordoni2003}), and a large body of the literature on this topic discusses the argument structure of the light verb, focusing on the fact that the verbal noun that the light verb is attached to may be marked with the accusative marker, -(\textit{l})\textit{ul} in Korean or -\textit{o} in Japanese.

\begin{exe}\ex\label{ex:31}
\begin{xlist}
    \ex \gll Bibi-nun Joa-wa 	hapsek-ul	     hayss-ta.	\\
   	Bibi-\textsc{top} Joa-with	table.sharing-\textsc{acc} do.\textsc{past}-\textsc{decl} \\
      \hfill Korean
    \ex \gll Bibi-nun Joa-to 	aiseki-o	     shita.	\\
    	Bibi-\textsc{top} Joa-with	table.sharing-\textsc{acc} do.\textsc{past} \\
      \hfill Japanese
    \glt ‘Bibi shared a table with Joa.’	\\
\end{xlist}
\end{exe}

As discussed in detail in \citet{Shim2013}, researchers differ on how to analyze the contrast between [a bare verbal noun + \textit{ha/su}] in \REF{ex:30} and [an accusative marked verbal noun + \textit{ha/su}] in \REF{ex:31}. These analyses can be roughly divided into three different trends: (a) \textit{ha}/\textit{su} is used as a light verb in both [a bare verbal noun + \textit{ha/su}] and [an accusative-marked verbal noun + \textit{ha/su}] constructions (\citealt{Jung2003,ParkK1992,ParkK1995,GrimshawMester1988}), (b) \textit{ha}/\textit{su} is always a heavy/lexical verb in both [a bare verbal noun + \textit{ha/su}] and [an accusative-marked verbal noun + \textit{ha/su}] constructions (\citealt{Terada1990}), and (c) \textit{ha}/\textit{su} is a light verb in [a bare verbal noun + \textit{ha/su}] and a heavy verb in [an accusative-marked verbal noun + \textit{ha/su}] constructions (\citealt{Ahn1991,Miyamoto1999,UchidaNakayama1993}). To date, there is still a debate among researchers as to whether \textit{ha/su} is a light verb or a heavy verb in the [accusative-marked verbal noun + \textit{ha/su}] form. Yet they have converged on the view that \textit{ha/su} is used as a light verb in the [bare verbal noun + \textit{ha/su}] construction. Thus, we conclude that \textit{ha} in Korean and \textit{su} in Japanese in the [bare verbal noun + \textit{ha/su}] construction are light verbs, and can be included in the present Korean-English and Japanese-English \ac{CS} study.\footnote{\textrm{Having defined} \textrm{\textit{ha}} \textrm{and} \textrm{\textit{su}} \textrm{are light verbs in [bare verbal noun +} \textrm{\textit{ha/su}}\textrm{] constructions, they will be glossed as} \textrm{\textsc{do}} \textrm{rather than ‘do’ in corresponding examples in order to indicate their status as a functional category.} } 

\subsection{Idioms} \label{ch1:sect:Idioms}

The notion of compositionality of idioms, which \citet{Shim2011} adopts for her analysis of Korean-English and Japanese-English \ac{CS}, has been contested by a number of researchers despite the fact that it has been widely adopted in many studies. According to \citet{NunbergEtAl1994}, idioms are divided into two groups based on their semantic compositionality. Most idioms (e.g. \textit{take} \textit{advantage} \textit{of}, \textit{pull} \textit{strings}) are in fact relatively ‘compositional’ in the sense that the idiomatic reading is composed fairly transparently from the component parts. ‘Non-compositional’ idioms (e.g. \textit{kick} \textit{the} \textit{bucket}, \textit{shoot} \textit{the} \textit{breeze}), on the contrary, do not compose their meanings from their components, but the idiomatic meaning is assigned to the whole phrase.\footnote{\textrm{Instead of} \textrm{\textit{compositional}} \textrm{vs} \textrm{\textit{non}}\textrm{{}-}\textrm{\textit{compositional}} \textrm{idioms, \citet{NunbergEtAl1994} use the terms} \textrm{\textit{idiomatically} \textit{combining} \textit{expression}}\textrm{s vs.} \textrm{\textit{idiomatic} \textit{phrases.}}} \citeauthor{NunbergEtAl1994} propose that while compositional idioms have the syntax of non-idiomatic expressions, non-compositional idioms are stored in the lexicon as complete phrases (1994: 497, 515). Following this distinction, Shim argues that compositional idioms and non-compositional idioms are predicted to behave differently in \ac{CS} and derive different word orders: while compositional idioms are not frozen as a chunk and thus their internal arguments are subject to \ac{CS}, similar to non-idiomatic/literal phrases, non-compositional idioms are listed in the lexicon and undergo \ac{CS} as a whole.  

However, the distinction between compositional and non-compositional (or similarly, decomposable vs. non-decomposable) idioms does not seem to hold uniformly among researchers. One example of such controversial cases is \textit{spill} \textit{the} \textit{beans} ‘reveal a secret’ or ‘divulge secretive information’, which is categorized either as a compositional/decomposable idiom or a non-compositional/non-decomposable idiom by different researchers (e.g. \citealt{CacciariGlucksberg1995,GibbsNayak1989,NunbergEtAl1994} vs \citealt{Abeille1995,Horn2003}). Those who argue that the idiom \textit{spill} \textit{the} \textit{beans} is compositional explain that the idiomatic or figurative meaning is distributed over the parts, with the verb \textit{spill} denoting ‘divulge, reveal’ and the \ac{NP} \textit{beans} being associated with ‘the information/secret that is divulged’ under the assumption that the availability of these meanings of each constituent depends on the presence of the other item which is imposed by conventionality. On the contrary, others argue that the relationship between the concrete spilling of beans and its figurative reading of releasing the information is not transparent, for \textit{beans} is not directly associated with the meaning ‘a secret’ or ‘information’ in any other contexts. Thus, non-transparent idioms such as \textit{spill} \textit{the} \textit{beans} and \textit{break} \textit{the} \textit{ice} should be further distinguished from semantically transparent idioms such as \textit{lay} \textit{down} \textit{the} \textit{law}, \textit{pay} \textit{attention}, etc., in which parts of the idioms have literal meanings in these expressions or they may retain their idiomatic meanings outside of idiomatic phrases. 

  While the view on the semantic properties of idioms, such as compositionality or analyzability or transparency, varies to a large extent in the literature, it seems that the syntactic behavior of idioms is less of a contentious issue. Researchers converge on the view that idiomatic expressions vary with regard to syntactic versatility (\citealt{Nediger2017,TabossiEtAl2009}). Roughly speaking, idioms can be categorized into three groups based on their syntactic behavior: (i) one that is syntactically fully flexible; (ii) one that is syntactically less flexible; and (iii) one that is syntactically inflexible or frozen. Corresponding examples are provided in (\ref{ex:32}--\ref{ex:34}), most of which are adapted from \citet{Horn2003} and \citet{Schenk1995}. The symbol `\#' indicates that the sentence is grammatical, but the idiomatic reading is not available.

\begin{exe}
\ex\label{ex:32}
    \begin{xlist}
        \ex	Care was taken of all of the orphans.					
    	\ex Great care seemed to be taken of the refugees by the government.
    	\ex The care that they took of the infants was more than adequate.
    	\ex How much care did they take of the infants?
    \end{xlist}
\ex\label{ex:33} \judgewidth{\#}
    \begin{xlist}
        \ex	The beans were spilled (by Bibi).						
    	\ex The beans appeared to be spilled when he opened his mouth.
    	\ex[\#]{The beans that Bibi spilled caused us a lot of trouble.}
    	\ex[\#]{Which beans did Bibi spill?}
    \end{xlist}
\ex\label{ex:34}
    \begin{xlist}
        \ex[\#]{The bucket was kicked by all of the bad guys.}
    	\ex[\#]{The bucket seems to be kicked by Bibi.}
    	\ex[\#]{The bucket Bibi kicked was astonishing.}
    	\ex[\#]{Which bucket did Bibi kick?}
    \end{xlist}
\end{exe}
					
In \REF{ex:32}, the idiom \textit{take care of} undergoes various syntactic operations such as passivization (\ref{ex:32}a), raising (\ref{ex:32}b), relativization (\ref{ex:32}c) and wh-question formation (\ref{ex:32}d), and its meaning also remains the same in all of these examples. 

On the other hand, \textit{spill} \textit{the} \textit{beans} may partake only in a limited number of syntactic operations such as passivization (\ref{ex:33}a) and raising (\ref{ex:33}b), but the DP argument of \textit{spill} \textit{the} \textit{beans} does not occur as the head of a relative clause (\ref{ex:33}c) or in a \textit{wh}-question (\ref{ex:33}d), where its idiomatic interpretation is no longer available. Finally, the degree of syntactic flexibility of \textit{kick} \textit{the} \textit{bucket} is heavily restricted, and none of the sentences in \REF{ex:34} deliver the figurative reading of the phrase ‘(to) die’. Although there seems to exist inter-speaker variation with respect to what extent they accept each sentence in (\ref{ex:32}--\ref{ex:34}) on their idiomatic readings, it is clear that the degree of syntactic flexibility varies from idiom to idiom. In fact, it has been proposed that idioms are also built in the syntax just like literal, non-idiomatic phrases (\citealt{EpsteinSeely2006,Nediger2017}). 

Several researchers argue that there is a strong correlation between semantic analyzability and syntactic flexibility of idioms and that compositional idioms are more syntactically flexible than non-compositional idioms (e.g. \citealt{GibbsNayak1989,Nediger2017,Nunberg1978,NunbergEtAl1994}). Especially in the field of psycholinguistics, the notion of semantic compositionality was further developed as the idiom decomposition hypothesis by Gibbs and his colleagues (\citealt{GibbsEtAl1989,GibbsNayak1989}), who propose that the processing of semantically compositional idioms differs from non-compositional idioms: compositional idioms are involved in the same mechanism of lexical retrieval and syntactic parsing of literal expressions, whereas non-compositional idioms require lexical recognition and retrieval directly from the lexicon. They also found in an experimental study that participants have a clear intuition about the distinction between compositional and non-compositional idioms and the processing of compositional idioms was faster than that of non-compositional idioms. However, in subsequent research that replicated their studies, these findings were not confirmed or contradicted (\citealt{TitoneConnine1994,TabossiEtAl2008}), revealing that people do not share clear and systematic intuitions on the semantic compositionality of idioms and the processing of non-compositional idioms did not differ from or was faster than the processing of compositional idioms. Thus, the notion of semantic compositionality of idioms and its relation to syntactic flexibility has been questioned by a number of linguists and remains subject to further investigation (\citealt{Abeille1995}, \citealt{AckermanWebelhuth1997,Horn2003,Jackendoff1997,Ruwet1991,LibbenEtAl2008}). 

In light of these considerations, the present study investigated the syntactic flexibility of idioms rather than their semantic compositionality in relation to the distribution of \ac{OV} and \ac{VO} orders in Korean-English and Japanese-English \ac{CS}. 

\section{The goal and organization of the monograph}\label{ch1:sect:Outline}

The major goal of this monograph is to provide a grammatical account which explains how \ac{OV} and \ac{VO} orders are systematically distributed in Korean-English and Japanese-English \ac{CS}. Based on the findings from previous research on this topic (\citealt{Shim2011}), two potential factors that seemed to contribute to \ac{OV}-\ac{VO} variation in Korean-English and Japanese-English \ac{CS} were identified, namely, the status of a verb (heavy vs light) within a code-switched constituent and the syntactic flexibility of the code-switched phrase, which were constructed as the two research questions of the present study in \REF{ex:18} and \REF{ex:21}. These two research questions were investigated in an experimental study, which will be presented in \chapref{ch:2}. The overall results of the experimental study confirmed the two research hypotheses stated in \REF{ex:19} and \REF{ex:22}.

After presenting an experimental study on Korean-English and Japanese-En\-glish \ac{CS} in Chapter \ref{ch:2}, I discuss in Chapters \ref{ch:3} and \ref{ch:4} the key theoretical assumptions that are made in this monograph and propose the mechanism of feature inheritance, which will be used to account for \ac{OV}-\ac{VO} variation in Korean-English and Japanese-English \ac{CS} in Chapter \ref{ch:5}. I adopt a derivational approach to word order proposed by \cite{Kayne1994}, who claims that \ac{OV} order is derived from \ac{VO}. After comparing this view against other competing approaches in Chapter 3, I conclude that the difference between \ac{OV} languages (e.g. Korean and Japanese) and \ac{VO} languages (e.g. English) is explained by the idea that the object moves to the left of the verb in \ac{OV} languages whereas the object stays in situ in \ac{VO} languages. The idea that \ac{OV} is derived from \ac{VO} will be implemented in the framework of minimalist syntax, which hypothesizes that linguistic variation is due to the morpho-syntactic contents/features of functional categories.

By taking this line of thought, features on various functional categories will be discussed, and \ac{OV} order in Korean and Japanese and \ac{VO} order in English will be explained under the feature inheritance system that is developed in Chapter \ref{ch:4}. While feature inheritance was originally proposed for the clausal domain (T inherits features from C), it will be extended to the verbal domain (\acs{ASP} inherits features from \textit{v}) to account for \ac{OV} order in Korean and Japanese and \ac{VO} order in English. Based on the feature inheritance mechanism developed in Chapter 4, I explain the results of the experimental study reported in Chapter \ref{ch:2} and propose that \ac{OV} and \ac{VO} orders in Korean-English and Japanese-English \ac{CS} are a result of the success or failure of feature inheritance from \textit{v} to \acs{ASP}. If feature inheritance occurs, object movement ensues, resulting in \ac{OV} order. If feature inheritance does not take place, the object stays in situ, leading to \ac{VO} order. 

Chapter \ref{ch:6} concludes the monograph. 
