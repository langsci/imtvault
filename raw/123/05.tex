







\chapter{Attributive constructions in the Jewish dialect of Zakho} \label{ch:JZax}
\renewcommand{\defaultDialect}{\JZax}
\section{Introduction}

The study of the AC system of \JZax\footnote{I maintain the J. (=Jewish) abbreviation in the dialect name, since there exists also a Christian (Chaldean) dialect of Zakho\il{NENA!Zakho (Christian)}, as reported by \citet{HobermanZakho}. Due the scarcity of information on this dialect, it is not included in the current survey.} is made easy due to the fact that the main source on which I rely, \citet{CohenZakho},  uses a conceptual framework similar to ours. {A more exact formulation would be that Cohen's framework inspired ours.} Indeed, chapters 2 and 4 of his work, \enquote{The attributive relationship} and \enquote{Apposition} respectively, address directly the issues at hand. A similar and concise analysis is also presented in \citet{CohenNucleus}.
The present survey, therefore, repeats to some extent the claims presented in these sources. Nevertheless, the current treatment is innovative in the classification of the construction according to the typology discussed in \sref{ss:typology_here}. This permits, moreover, a transparent comparison of this dialect's system to the other dialects discussed in the following chapters. 

As noted in \sref{ss:intro_NPstructure}, \citet[20--27]{CohenZakho} identifies a system of definite and indefinite \isi{determiners} in the grammar of \JZax, an analysis which I adopt here.











Cohen's examples are based mainly on Polotsky's\ia{Polotsky, Hans J.@Polotsky, Hans J.} unpublished transcribed Zakho texts, as well as some published sources (see \cite[5--8]{CohenZakho} for details).\footnote{Cohen's examples refer to the page number of Polotsky's transcribed texts. As these texts are as yet unpublished, I have not reproduced this number in the citations, but rather referred to Cohen's page and example numbers.} Two other sources used in this chapter are \citet{Avineri}, a collection of texts of \JZax, and \citet{SabarDictionary}. The latter is a dictionary devoted to the Cis-Zab\il{Cis-Zab NENA dialects} Jewish dialects of north-west Iraq (see \ref{ss:intro_dialects}), of which \JZax constitutes an important part.  I only use examples which are clearly identifiable as \JZax examples (by virtue of their source, or being explicitly marked as such). 

\citet{CohenZakho, SabarDictionary} use the sign <\transc{ı}> to denote the phonemic \isi{schwa} (\phonetic[ə]\~\phonetic[ɪ]), while \citet{Avineri} uses the sign <\transc{i}>. For consistency with other dialects, I use instead the <\transc{ə}>  symbol (see \pageref{ss:examples_format}). 

This chapter is organised as follows: In \ref{ss:JZax_Poss_Pro} I discuss the usage of the possessive pronominal suffixes. Subsequently, the two main ACs of \JZax are discussed: the \isi{construct state} construction is treated in \ref{ss:JZax_CST} and the \isi{analytic linker construction} in \ref{ss:JZax_Lnk}, the latter being the richest one in terms of paradigmatic variation. The marking of \secns by \gen* case, which is compatible with both these constructions, is treated in \ref{ss:JZax_gen}. A rare  case of the \isi{double annexation} construction is discussed in \sref{ss:JZax_DAC}, while the \isi{juxtaposition} construction is handled in \ref{ss:JZax_juxt}. \Ref{ss:JZax_conclusions} gives some general conclusions and comparative remarks. 




\section{Possessive pronominal suffixes (X-y.\poss)} \label{ss:JZax_Poss_Pro}

A pronominal \secn can be realised as a possessive \isi{pronominal suffix} (=\poss), which  attaches directly to the stem of the \prim noun. 

\acex{Noun}{Pronoun}{321}
{ēha brāt-i}
{\dem.\fem{} daugther-\poss.1\sg}
{this daughter of mine}
{CohenZakho}{98 (28)}

If there are any further \secns, they follow the inflected noun:

\acex{Noun}{Pronoun}{322}
{ʾaw brōn-e rūwa}
{\definite.3\masc{} son-\poss.3\masc{} big.\masc}
{this older son of his}
{CohenZakho}{98 (29)}

As noted in \sref{ss:synt_axis}, the term \concept{possessive suffix} is traditional;  its usage is in fact wider than denoting solely possession, similarly to other ACs. For instance, it can attach to an infinitive to denote one of its arguments (here its subject):

\acex{Infinitive}{Pronoun (subject)}{330}
{u\cb{} zaʿ-li b-əd ʾīzāl-i əl knəšta}
{and\cb{} be\_lost-1\sg{} in-\cst{} go.\inf-1\sg{} to synagogue}
{And I got lost while going to the synagogue.}
{CohenZakho}{99 (34)}

Moreover, it can attach to a preposition to denote its complement:

\largerpage
\acex{Preposition}{Pronoun}{452}
{əmm-i}
{with-\poss.1\sg}
{with me}
{CohenZakho}{451}\antipar\newpage

\section{The construct state construction (X.\textsc{cst} Y)} \label{ss:JZax_CST}

\JZax has two \cst* markers: the suffix \transc{-əd\~-ət} and \isi{apocope} (phonological reduction).\footnote{\citet[92]{CohenZakho} relates the apocopate form to the ancient \abs*, i.e.\ the \il{Aramaic!Early}{Early Aramaic} free indefinite form. It seems more appropriate to relate it directly to the \il{Aramaic!Classical}Classical Aramaic \isi{construct state}. In any case, for singular nouns the two forms were mostly identical. Some apocopate \cst* forms can not be easily traced to ancient Aramaic \isi{construct state} forms, and must be considered innovated, such as the form \transc{bax}, apocopate form of \foreign{baxta}{woman, wife} \citep{CohenNucleus}. See the discussion in \sref{ss:apcopate}.}
\citet[97]{CohenZakho} states: \enquote{the functional distinction between them is not clear, and for now it must be regarded as mere [free] variation}. He gives  the following two examples, in which the noun \foreign{xabra}{word} (\pl\ \transc{xabre}) is put in \cst*. Note that in this case, both markings neutralize the number distinction.

\acex{Noun}{Noun}{318}
{xabr-ət xōr-e}
{word-\cst{} friend-\poss.3\masc}
{the words of his friend}
{CohenZakho}{97 (2)}

\acex{Noun}{Noun}{317}
{(p\cb{}) xabər xōr-e}
{in\cb{} word.\cst{} friend-\poss.3\masc}
{(by) the words of his friend}
{CohenZakho}{97 (21)}

The \isi{apocopated form} is constructed by removing the \free*-cum-number suffixes \foreign{-a}{\free.\sg} or \foreign{-e}{\free.\pl}. Due to the resulting consonant cluster at the end of \transc{*xabr}, an \isi{epenthetic} \transc{ə} is inserted.





\largerpage
Nouns (including infinitives, see \sref{ss:JZax_inf_head}) as well as adjectives can appear with both \cst* markers, while pronouns as well as adverbials appear only with the \ed suffix.\footnote{As adverbials do not have a full free form one cannot postulate an apocopate form for them, making the above statement trivial for this category.} \Prim pronouns typically appear with clausal \secns (see \ref{ss:JZax_clausal}).



Rarely, the \cst* suffix appears NP finally, but only when the NP ends with an adjective, as in the following example (=\example{338bis}):

\acex{Noun Phrase}{Noun}{338}
{[gōra qamāy]-ət [d-anya baxt-āsa]}
{man first.\masc-\cst{} \gen-\dem.\pl{} woman-\pl}
{The first husband of these women}
{CohenZakho}{101 (49)}



As discussed in \sref{ss:d_vs_ed}, such examples are accommodated under the analysis of the \ed suffix as a \isi{phrasal suffix}. 




\subsection{Pronominal, ordinal and adverbial \secns}
\label{ss:JZax_adv_secn}

There is a variety of categories which can occupy the \secn position: pronouns, ordinal or PPs (\isi{adverbial} phrases). The \prim in each case can be marked by both \cst* markings:

\acex{Noun}{Pronoun}{370}
{šəmm-ət gyāne}
{name-\cst{} \refl-\poss.3\masc}
{his own name}
	{CohenZakho}{115 (106)}
	
\acex{Noun}{NP headed by pronoun}{373}
{ʾaqlās [xa mənn-u]}
{feet.\pl.\cst{} one from-3\pl}
{the feet of one of them}
{CohenZakho}{115}

\acex{Noun}{Ordinal}{305}
{baxt-ət tre}
{wife-\cst{} two}
{the second wife}
{CohenZakho}{95 (6)}

\acex{Noun}{Ordinal}{347}
{gōr tre}
	{man.\cst{} two}
{(the) second man}
{CohenZakho}{84 (7)}

\acex{Noun}{\PP}{378}
{rubʿ-ət mənn-u}
{quarter-\cst{} from-\poss.3\pl}
{a quarter of them}
{CohenZakho}{117 (113)}

\acex{Noun}{\PP}{377}
{ū\cb{} ʾan nāš ʾəmm-e}
{and\cb{} \definite.\pl{} person.\cst{} with-\poss.3\masc}
{and the people with him}
{CohenZakho}{117 (112)}\antipar\newpage



\subsection{Clausal and infinitival \secns} \label{ss:JZax_clausal}
\largerpage
Clausal \secns, whether full clauses or infinitival phrases, can appear regularly in the CSC. 
	
\acex{Noun}{Infinitival Phrase}{367}
{ṭēr-ət maḥkōye}
{bird-\cst{} speak.\inf}
{a speaking bird}
{CohenZakho}{111 (88)}

\acex{Noun}{Infinitival Phrase}{362}
{waʿd-ət [īsāya dīd-e]}
{time-\cst{} come.\inf{} \lnk-\poss.3\masc}
{time of his coming}
{CohenZakho}{111 (87)}

\acex{Noun}{Infinitival Phrase}{325}
{(mṭē-le) waʿəd [ʾīsāy-ət d-anya ṭḷāha xūr-āsa]}
{came-3\masc{} time.\cst{} come.\inf-\cst{} \gen-\dem.\pl{} friend-\pl}
{The time arrived for the three friends’ coming}
{CohenZakho}{98 (32)}

The following example is especially interesting, since the \secn consists of two conjoined infinitives. Moreover, it is split by the occurrence of the \isi{copula} \transc{la}, marking the entire CSC as a predicate. In \JZax (as in many other \ili{NENA} dialects), the \isi{copula} is a second-position \isi{clitic} with respect to the predicate phrase.\footnote{In other words, the \isi{copula} typically appears after the first minimal unit of the predicate, be it a free noun or a CSC.
A similar behaviour is exhibited by the \Syr \isi{enclitic} personal pronouns, which can be seen as precursors of the \isi{copula}; see \citet[121ff.]{GutmanVanPeursen} for a discussion and examples.} Yet  as this example clearly shows, the \isi{copula} cannot split the \cst* \prim from its \secn and thus appears instead after the first conjoined \secn.\footnote{Such cases should be clearly differentiated from cases where the \isi{copula} itself is part of the \secn, in which case it can appear directly after the \prim in a special attributive form (see \sref{ss:JZax_genitive_clauses}).}

\acex{Noun}{Conjoined infinitives}{363}
{urx-ət [īzāla] \cb{}la [ū\cb{} la dʾāra]}
{way-\cst{} go.\inf{} \cb{}\cop.3\fem{} and\cb{} \neg{} return.\inf}
{It is a road of going and not returning.}
{CohenZakho}{111 (87)}\antipar

As stated, full clauses can appear as well in the \secn position. The \prim noun can have various functions in the \isi{relative clause} (subject, object, etc.)

\acex{Noun (subject)}{Clause}{296}
{šaps-ət zəl-la}
{week-\cst{} went-3\fem}
{Last week (lit. The week that passed)}
{Avineri}{191 (1617)}

\acex{Noun (subject)}{Clause}{295}
{ʾō məṭər [kuš-le xapča]}
{\definite.\masc{} rain.\cst{} fell-3\masc{} a-little}
{This rain which fell (=rained) a little}
{Avineri}{171 (1274)}

\acex{Noun (object)}{Clause}{319}
{xabr-ət mír-rē-la}
{word(s)-\cst{} said-\agent3\masc-\dat3\fem}
{the words he told her}
{CohenZakho}{97 (24)}

\acex{Noun (object)}{Clause}{320}
{xabər [mxē-la baxt-e b-ət ʾarya]}
{word(s).\cst{} struck-3\fem{} wife-\poss.3\masc{} in-\cst{} lion}
{the words his wife said about the lion}
{CohenZakho}{97 (23)}

\largerpage[1.5]
Interrogative pronouns can  head the CSC with clausal \secns. In such cases they act as a head of a \concept{free relative}, loosely speaking. In this construction, they are always marked by the \cst* suffix.\footnote{As \citet[96]{CohenZakho} notes, the free forms of the interrogative pronouns are used as markers of either direct or indirect questions. No apocopate forms of the interrogative pronouns are found in \JZax.}

\acex{Pronoun}{Clause}{303}
{(mən) mā-d mər-rī-lox}
{from what-\cst{} told-\agent1\sg-\dat2\masc}
{from what I told you}
{CohenZakho}{94 (4)}

\acex{Pronoun}{Clause}{313}
{manī-t hāwe ḥmīla qām-ox (b-āse l-qṭāla)}
{who-\cst{} \sbjv.be.3\masc{} stood.\resl.\masc{} before-2\masc{} \fut-\pass\_\aux.3\masc{} to-kill.\inf}
{whoever is standing in front of you will be killed}
{CohenZakho}{96 (14)}

To this group one may tentatively add the pronoun/quantifier \foreign{kud}{each}:

\acex{Pronoun}{Clause}{300}
{ku-d [ʾāwəz-lox hawūsa]}
{each-\cst{} \sbjv.do.3\masc-\dat.2\masc{} favor}
{anyone who does you a favor}
{CohenZakho}{94 (1)}

In the above example I analyse \transc{kud} as having a base-form \transc{ku} augmented by a \cst* suffix. However, a free form of \transc{kud} is not attested  \citep[94, fn. 5]{CohenZakho}, though historically it is very probably derived from \transc{*kull + d} \citep[181b]{SabarDictionary}.\footnote{The form \transc{kull}, meaning \transl{all}, is used in \JZax only together with possessive pronominal suffixes. Compare this with \example{605} of  \Qar and see further \citet[282f.]{KhanQaraqosh}.} Moreover, as \citet[3]{CohenDetermination} notes, disregarding its \isi{construct state} marking, it can be equated with a determiner of the \JZax system. Thus, one may reasonably question whether this should be seen as a genuine instance of \cst* marking, or rather a fossilized remnant of it. Interestingly, instances of \transc{ku} are found in \Nrt\  \citep[181b]{SabarDictionary}, but these may be cases of back-formation.

Finally, note that the interrogative \foreign{ēma}{which} can be complemented directly by a noun-phrase, which embeds in it a clause. In this case too, the \cst*-marked \isi{interrogative pronoun} functions rather as a determiner:

\acex{Pronoun}{Noun Phrase}{316}
{ēm-ət [julle d\cb{} [ʾājəb-le ləbb-ox]] (ṭḷōb mən d-e kwīna)}
{which-\cst{} clothes \lnk\cb{} please-3\masc{} heart-\poss.2\masc{} take.\imp{} from \gen-\definite{} tent}
{Ask for whatever clothes which please you from that tent}
{CohenZakho}{96 (16)}



\subsection{Adverbial \prims} \label{ss:JZax_Adv_head}

Many adverbials are in fact nouns used adverbially, such as the noun \foreign{waʿda}{time}  (thus \example{362} can be understood adverbially: \transl{when he comes}). These behave as nouns, having  potentially two \cst* forms: one suffixed and one apocopate, besides their free forms.  



In contrast to these, there are true prepositions or conjunctions (often mono-syllabic, or shorter). Some of these  take a \cst* suffix obligatorily when complemented by a syntactic word, be it a noun or a pronoun. Such is the case of \foreigngloss{ʾəmm-əd}{with-\cst} (\reex{346}): 


\acex{Preposition}{Noun}{346}
{ʾəmm-əd gōra}
{with-\cst{} man}
{with the man}
{CohenNucleus}{82}


\acex{Preposition}{Pronoun}{372}
 {əmm-ət gyān-e}
 {with-\cst{} \refl-3\masc}
 {(together) with itself}
 {CohenZakho}{115 (106)}
 
 One can nevertheless identify the base form, as it  appears together with the possessive pronominal suffixes (see \example{452}). 
 Moreover, it is clear that it stems etymologically from classical Aramaic \transc{\texthebrew{עִם} ʿĭm} \citep[97a]{SabarDictionary}.
 
Some prepositions allow for two forms to appear before syntactic words: one with the suffix \ed and one without it. The shorter form should not be seen as an apocopate form, as it is not derived from a full free form (except for those prepositions derived from nouns), but should rather be seen as the \concept{simple form} of the preposition (see also the system of \Barw in \vref{tb:Barw_prep}). Functionally, the simple form is equivalent to a \cst* marked form, yet from the point of view of the classification system used here it represents an unmarked \prim.  In this vein, contrast the following \cst* marked example with \example{352}, representing an unmarked \isi{juxtaposition} construction.\footnote{The preposition \foreign{b-}{in} should be kept apart from its homonym  particle  \transc{b-}, which joins the infinitive to form a gerund  \parencites[99, fn. 9]{CohenZakho}[103a]{SabarDictionary}. Note, moreover, that  the preposition \transc{b-əd} can precede an infinitive (without forming a gerund), as in \example{330}. \label{ft:JZax_gerund}} 




 
 

 

 
 \acex{Preposition}{Noun}{351}
 {b-əd bəllūre}
 {in-\cst{} flute}
 {with a flute}
 {CohenNucleus}{85 (16)}


\largerpage[2]
The \cst* marked form can precede phrasal and clausal \secns:

\acex{Preposition}{Infinitival Phrase}{364}
{ʾəmm-ət [maʾōre dīd-a]}
{with-\cst{} transfer.\inf{} \lnkd-3\fem}
{while transferring her}
{CohenZakho}{111 (87)}\antipar\newpage 

\acex{Preposition}{Participial Phrase}{405part}
{(u\cb{} šqīl-a-le nunīsa) b-ət [kēf-e ʾəsya]}
{and\cb{} took.\patient3\fem-\agent3\masc{} fish in-\cst{} joy-\poss.3\masc{} came.\resl}
{(And he took the fish) with his joy attained (=happily).}
{CohenZakho}{122 (142)}


\acex{Adverbial}{Clause}{343}
{(uz-le ōha ʾərba) māṭu-t mír-rā-le}
{did-3\masc{} \dem.\masc{} sheep how-\cst{} said-3\fem-\dat.3\masc}
{(The sheep did) as she told him.}
{CohenZakho}{104 (67)}




\subsection{Adjectival \prims} \label{ss:JZax_Adj_Head}
 
Adjectives heading the CSC typically yield a superlative reading, whenever the \secn is a \pl* noun or pronoun:

\acex{Adjective}{Noun}{335}
{rūw-ət ganāw-e}
{big.\masc-\cst{} thief-\pl}
{the biggest/head thief}
{CohenZakho}{100 (41)}\antipar

 
\acex{Adjective}{Pronoun}{465}
{ʾaw zōr kull-u}
{\definite.\masc{} small.\cst{} all-3\pl}
{the smallest of them all}
{CohenZakho}{100 (42)}

 

	Note that the \prim adjective exhibits the gender and number features of the referent, while the \secn has \pl* marking.\footnote{The superlative preceding the qualified noun is clearly an areal phenomenon: It is present in \Arab, \Sor \parencites[68]{MacKenzie}[19]{ThackstonSorani}, and \Kur \citep[28]{ThackstonKurmanji}. In the latter, the superlative adjective is sometimes marked as \cst*. Compare also to the \Syr \example{1116} as well as \example{981}. One reviewer suggested this is semantically motivated, as superlatives establish a unique reference similarly to \isi{determiners} which are typically pre-nominal.
	\label{ft:JZax_superlative}} 

A formally related construction, though functionally different, is the \concept{emotive genitive}.\footnote{This term was coined by \citet{HopkinsEmotive}, who identifies the characteristic syntactic and semantic features of the construction and exemplifies it from various \ili{Semitic} languages, including \ili{NENA}.  \citet{GaiRare} attempts to trace this construction back to \Akk. \label{ft:emotive_genitive}} In this construction, loaded with some emotional emphasis, the semantic head of the construction is expressed as a \secn, while the adjectival \prim agrees with it:

\acex{Adjective}{Noun}{337}
{pappūk-ət xmāra}
{pitiful.\masc-\cst{} donkey}
{the poor donkey, that poor of a donkey}
{CohenZakho}{100 (46)}

\citet[270]{GaiHighLoad}, analysing a similar construction of \CUrm, explains it as follows: \blockquote{[B]y inserting the subordinating \textit{d}
the noun is converted to \textit{nomen rectum}, i.e., a subordinated
one [=\secn of the CSC], and by lowering the status of the noun, the
status of the adjective, its subordinator in the nominal
phrase, rises. [...] Thus, the communicatively more important element
has the more important syntactic status, while the communicatively
secondary element has the secondary syntactic
status.}

\citet[101]{CohenZakho} mentions a third case of an adjectival \prim, namely the use of the \Arab loanword \foreign{ġēr}{other}. However, it is not clear synchronically why it should be considered as an adjective rather than a type of quantifier or pronoun. 

\subsection{Infinitival \prims} \label{ss:JZax_inf_head}
Infinitives  can be expanded by one of their arguments in the CSC. As a general rule, infinitive of transitive verbs are expanded by their object argument, while those of intransitive verbs are expanded by their subject. Most often the infinitives are marked by the \cst* suffix, but also the apocopate form is available.

\acex{Infinitive}{Noun (subject)}{328}
{(mōnəx-li čūča əl) ʾīzāl-ət ʾərba}
{looked-1\sg{} a\_little to go.\inf-\cst{} sheep}
{I looked a little at the  sheep's walking.}
{CohenZakho}{99 (36)}

\largerpage
\acex{Infinitive}{Noun (subject)}{329}
{(pəš-la waʿəd) ʾīzal jwanq-e l\cb{} nēčīr}
{became-3\fem{} time.\cst{} go.\inf.\cst{} youngster-\pl{} to\cb{} hunt}
{It was the time of the youngsters’ going out for a hunt.}
{CohenZakho}{99 (35)}\antipar\newpage 

\acex{Infinitive}{Noun (object)}{331}
{(ū\cb{} škəl-lu b-ət) ʾwāz-ət ṣandūqa ta ḥakōma}
{and\cb{} began-3\pl{} in-\cst{} make-\cst{} box for king}
{They began preparing a box for the king.}
{CohenZakho}{99 (37)}\antipar

\acex{Infinitive}{Noun Phrase (object)}{332b}
{(\ldots{} mən) ʾwāz [xōrūs bnās mīr-e ū\cb{} ṗāšā-ye]}
{\ldots{} than make.\inf.\cst{} friendship.\cst{} daughter.\pl.\cst{} emir-\pl{} and\cb{} pasha-\pl}
{(Befriending you pleases me more than) befriending daughters of emirs and pashas.}
{CohenZakho}{99 (38)}\antipar

 
Like other nouns (\examples{378}{377}), \cst* marked infinitives can also be complemented by PPs, serving as indirect objects  \citep[117]{CohenZakho}:

\acex{Infinitive}{\PP\ (indirect object)}{379}
{(ʾēn nāša la\cb{} g-sōʾa m\cb{}) mēnōx-əd ʾəbb-u}
{eye.\cst{} man \neg\cb{} \ind-satiate.3\fem{} from\cb{} look.\inf-\cst{} in-3\masc}
{The man’s eye is not satiated of looking at him.}
{Avineri}{164 (1197)}\antipar

 

	\acex{Infinitive}{\PP\ (indirect object)}{457}
{la\cb{} g-barya hēǰ kwāš mənn-a}
{\neg{} \ind-happen.3\fem{} still descend.\inf.\cst{} from-3\fem}
{It is still impossible to descend from it (the plane).}
{Avineri}{104 (453)}\antipar\antipar
 
\section{The analytic linker construction (X \textsc{lnk} Y)} \label{ss:JZax_Lnk}
\subsection{Introduction}
The common form of the pronominal \lnk* in \JZax is \transc{dīd}. It may be analysed as an expansion of the \il{Aramaic!Classical}Classical Aramaic \lnk* \d with an overt \cst* suffix yielding \transc{də-} + \ed = \transc{dīd}. Yet   contrary to regular cases where a \cst* suffix appears, the \lnk* \transc{dīd} is compatible with possessive pronominal suffixes, as in \example{304part}, and thus the ending \transc{-īd} must be seen as an integral part of the \lnk* itself.\footnote{Diachronically, one finds the same form before possessive pronominal suffixes in \JBA, and it may be related as well to the \Syr \transc{dil-}. See \sref{ss:did_lnk} for a discussion of its development.}

\acex{Noun}{Pronoun}{304part}
{qaḥra dīd-ox}
{grief \lnkd-2\masc}
{your grief}
{CohenZakho}{95 (5)}

In the context of a pronominal \secn realised as a \isi{possessive suffix}, the \lnk* is in syntagmatic \isi{complementary distribution} with the \isi{proclitic} \lnk* \d:  The base \transc{did-} is used exclusively with the monosyllabic \sg* possessive pronouns, while \d is used with the bisyllabic plural ones; see \vref{tb:Amd_gen_pron} \citep[453]{CohenZakho}.

\acex[\JZax]
{Noun}{Pronoun}{1920}
{līšāna d-ēni}
{language \lnk-\poss.1\pl}
{our language}
{CohenZakho}{85 (347)}


 In other contexts, the \isi{proclitic} \d is restricted almost exclusively to clausal \secns, thus functioning similarly to a \rel* (see \sref{ss:JZax_lnk_clause}, but see also the rare exceptions in \example{447}, \example{354} and \example{306}). 




The basic use of the \lnk* is to create an indirect \isi{attributive relationship} between a \prim noun (whether explicit or implicit) and a \secn noun. This relationship is indirect, since it is the pronominal \lnk* that stands in \concept{direct attributive relationship} with the \secn. The \prim noun, which is most frequently explicit, stands syntactically in \isi{apposition} with the linking pronoun, and is never marked for \cst* in \JZax:

\acex{Noun}{Noun}{314a}
	{xa ṭēra dīd ḥukūm}
{\indef{} bird \lnkd{} sovereignty}
{a royal bird}
{CohenZakho}{96 (18)}




Pronouns which are realised as independent words, such as the reflexive pronouns, are treated in the same way as
nouns:



\acex{Noun}{Pronoun}{416}
{(sē-le əl) bāžer dīd gyān-e}
{came-3\masc{} to city \lnkd{} \refl-3\masc}
{(he came to) his own town}
{CohenZakho}{132 (176)}

Whenever the \secn consists both of a \isi{pronominal suffix} and a full noun, the \lnk* is repeated:



\acex{Conjoined Nouns}{Pronoun+Noun}{357}
{[ḥāl ū\cb{} quṣta] dīd-i ū\cb{} dīd xa ṭēra}
{situation and\cb{} story \lnkd-1\sg{} and\cb{} \lnkd{} \indef{} bird}
{the story of me and a bird}
{CohenZakho}{106 (74)}

\largerpage[-1]
Note, \textit{in passim}, that in the last example the \prim itself consists of a conjunction of two nouns (being in this case an idiomatic expression). Noun phrases are quite common as \prims of the ALC, especially those consisting of a noun modified by an adjective:

\acex{Noun Phrase}{Noun}{340}
{[ʾáy ʾasəqsa turta] dīd ḥakōma}
{\definite.\fem{} ring(\fem) broken.\fem{} \lnkd{} king}
{the broken ring of the king}
{CohenZakho}{102 (51)}

Compare the last example to the less common construction exhibited in \example{338}.

Semantically, it is noteworthy that in some cases the adjective seems to have a wide scope over the entire AC, notwithstanding its syntactic position:\footnote{This is not related specifically to \JZax, but it is interesting to note the phenomenon. An alternative analytic possibility is to see the adjective as a \prim of an embedded ALC: [Noun [Adj. \lnk\ Noun]].}

\acex{Noun Phrase}{Noun}{341}
{(āna \ldots ) [brōna yakāna] dīd yəmm-i}
{1\sg{} \ldots{} son only \lnkd{} mother-\poss.1\sg}  
{(I am [...]) my mother's only son.}
{CohenZakho}{102 (52)}




\subsection[Verbal nouns as members of the ALC]{Verbal nouns as members of the analytic linker construction} \label{ss:JZax_VerbalNouns_heads}

The notion of \concept{verbal nouns} should be understood here as nominal elements which can participate in a verbal construction. In \JZax, these can be infinitives or certain \isi{participles}. 

Infinitives can  appear both as \prims and as \secns of the ALC. As \secns, they are analogous to nouns:

\acex{Noun}{Infinitive}{417}
{narʾa dīd [qṭāʾ-ət dār-e]}
{axe \lnkd{} cut-\cst{} tree-\pl}
{an axe to cut trees}
{CohenZakho}{132 (177)}

\largerpage
When an infinitive is the \prim of the ALC, the \secn is an argument of the infinitive, either its subject or object.  In the latter case, if the infinitive is part of a verbal periphrastic expression, one could argue that the relation to its nominal complement is a \isi{completive relation} rather than an attributive one (see \sref{ss:threeRel} and \cite[cf.][100]{CohenZakho}). However, it is interesting to see that formally this relation is expressed by the same construction. 



Thus, in the following example the infinitive \foreign{mēsōye}{to bring} functions both as the \prim of an ALC with its objective argument as a \secn, and as a \secn of a wider ALC. Note that in the contrast to the object \foreign{nāše}{people}, the locative \isi{adverbial}  \foreign{qam məšpaṭ}{before the court} is not marked by a \lnk*.

\acex{Infinitive}{Noun (object)}{365}
{(húl-lē-lu šūla dīd) mēsōye dīd nāše qam məšpaṭ}
{gave-\agent3\masc-\dat3\pl{} task \lnkd{} bring.\inf{} \lnkd{} people before court}
{(He gave them) the task of bringing the people in front of court.}
	{CohenZakho}{111 (87)}
	
Pronominal arguments are also linked by means of the same construction, as can be seen in examples \vref{ex:362} and \vref{ex:366}.

Parallel to \example{357}, whenever both a pronominal and a nominal argument are expressed, the \lnk* is repeated:

\acex{Infinitive}{Pronoun+Noun (subjects)}{356}
{ʾīxāla\footnotemark{} dīd-a u\cb{} dīd  gōr-a}
{eat.\inf{} \lnkd-\poss-3\fem{} and\cb{} \lnkd{} man-\poss.3\fem}
{her and her husband’s food (eating)}
{CohenZakho}{106 (73)}

\footnotetext{The infinitive \transc{ʾīxāla} functions here practically as a normal noun, denoting \transl{food}. Yet  in general it still functions as an infinitive \citep[see][93]{SabarDictionary}, and thus deserves its place here.}	

Of special interest are cases in which the object of an infinitive is expressed pronominally on a \lnk*, while it appears in immediate \isi{apposition} to an explicit nominal object. These are in fact cases of \concept{prolepsis},  discussed by \citet[142--4]{CohenZakho}. Note that there are two possible realisations of the full object: First, it may be a \secn of an ALC standing in \isi{apposition} to the first \lnk*. In this case the nominal may be marked by \isi{genitive case} (on which see \ref{ss:JZax_gen}).

\acex{Infinitive}{Pronoun/Noun (object)}{459}
{u\cb{} pəš-la mahōye dīd-a, dəd d-ē baxta sməx-ta}
{and\cb{} became-3\fem{} assist\_delivery.\inf{} \lnkd-3\fem{} \lnkd{} \gen-\dem.\fem{} woman pregnant-\fem}
{She started to assist her in delivery, this pregnant woman.}
{Avineri}{53 (3)}

Alternatively (and more frequently), only the explicit object appears, standing in \isi{apposition} with the pronominal \secn. In this case it is not marked as genitive, since it does not stand in direct \isi{attributive relation} with a \lnk*. This is shown in the following example, taken from the same story as the previous one. 

\acex{Infinitive}{Pronoun/Noun (object)}{460}
{pəš-la mandōye dīd-e ʾaw tūma}
{became-3\fem{} throw.\inf{} \lnkd-3\masc{} \definite.\masc{} garlic}
{She started to throw it, the garlic.}
{Avineri}{53 (6)}
 
It is important to note that the object of an infinitive may also be introduced by the accusative/locative preposition \transc{ʾəl} \citep[see][96]{SabarDictionary}, in which case an AC is not used. The  functional equivalence of the two constructions, as apparent in the following example, suggests indeed that the AC serves in such cases to instantiate a \isi{completive relation}.
The following example illustrates the two possibilities:

\acex{Infinitive}{Pronoun (object)}{334}
{(ū\cb{} pəš-la ṭāl-a maṛaq) nḥāqa ʾəll-e ū xpāqa dīd-e}
{and\cb{} became-3\fem{} for-3\fem{} desire touch.\inf{} \acc-3\masc{} and hug.\inf{} \lnkd-3\masc}
{(and it became her desire) to touch him and hug him}
{CohenZakho}{(40)}

Not only infinitives can act as \prims of this construction, but also resultative \isi{participles}, which form part of the analytic perfect tense. Here again, functionally this is a \isi{completive relation}, which is formally realised as an AC. The following example is analogous to \example{460}.

\acex{Participle}{Pronoun (object)}{438}
{ṭamāha wē-lu mukīm-e dīd-a ʾay bāžər}
{why \cop.\pst{}-3\pl{} blackened.\resl-\pl{} \lnkd-3\fem{} \definite.\fem{} city}
{why  had  they blackened the town}
{CohenZakho}{144 (7)}

\subsection{Clausal \secns} \label{ss:JZax_lnk_clause}
\largerpage
Clausal \secns (i.e., relative clauses) can follow both the \lnk* \transc{dīd} and the shorter form \d, apparently in free variation.  Note that the form \d is typically reserved in \JZax for clausal \secns.  
In both cases, the clausal \secn  stands in direct \isi{attributive relation} with the \lnk*, which represents the modified \prim. In this respect it is similar to a relative pronoun, except that it is external to the \isi{relative clause}; indeed, inside the \isi{relative clause} one normally finds a second pronominal index representing the \prim.


\acex{Noun (subject)}{Clause}{387}
{(ta) [d-aw gōra] dīd [wḗ-wa-le faqīr]}
{for \gen-\definite.\masc{} man \lnkd{} \cop.\pst-\pst-3\masc{} poor}
{(to) the man who was poor}
{CohenZakho}{120 (121)}

\acex{Noun}{Clause}{426}
{(rīš) kursi d\cb{} wē-la tūta rēš-e}
	{on chair(\masc) \lnk\cb{} \cop.\pst-3\fem{} sat.\resl.\fem{} on-3\masc{}}
{(on) the chair on which she was sitting}
{CohenZakho}{134 (189)}

\acex{Noun Phrase}{Clause}{427}
{u\cb{} [trḗ sūsə-wāsa dīd-i] dīd [ʾāna g-rakw-en ʾəll-u]}
{and\cb{} two horse-\pl{} \lnkd{}-1\sg{} \lnkd{} 1\sg{} \ind-ride-1\sg{} on-3\pl}
{and two of my horses which I ride}
{CohenZakho}{135 (190)}



Pronouns may also act as  \prims of the ALC with clausal \secns. Such is the case of the \isi{interrogative pronoun} \foreign{mani}{who} in the following example. Note that in this example \transc{mani} itself acts as a \secn of a CSC headed by the pronoun \foreign{kud}{every}:

\acex{Pronoun}{Clause}{312}
{ku-d [mani dīd [yāwəl pāre ta ṣədāqa]]}
{every-\cst{} who \lnkd{} \sbjv.give.3\masc{} money for charity}
{each one who gives money for charity…}
{CohenZakho}{96 (13)}

Contrast this example with examples \ref{ex:313} and \vref{ex:300}, in which both \transc{mani} and \transc{kud} are \prims of a \cst*-marked AC with a clausal \secn. Note that \transc{kud} cannot act as the \prim of the ALC  due to its inherent \cst* marking.


\subsection{Numerals as ordinal \secns}

Numerals serving as ordinal \secns occur frequently in this construction:

\acex{Noun}{Ordinal}{309}
{ʾō gabāra dīd tre}
{\dem.\masc{} hero \lnkd{} two}
{the second hero}
{CohenZakho}{95 (11)}

\acex{Noun}{Ordinal}{307}
{ē baxta dīd ṭḷāha}
{\dem.\fem{} woman \lnkd{} three}
{the third wife}
{CohenNucleus}{85 (12)}

As \citet[85]{CohenNucleus} points out, a phrase headed by a \isi{pronominal linker} is the syntactic counterpart of an adjective. This is especially clear in the case of ordinal numerals, since the ordinal \transl{first} is always expressed as a morphological adjective:

\acex{Noun}{Ordinal}{461}
{ē baxta qamē-sa}
{\dem.\fem{} woman first-\fem}
{the first wife}
{CohenNucleus}{85 (11)}

For an elaboration of this point in the context of \Syr, see the discussion in \sref{ss:syr_adj}.

\subsection{Adverbials as \secns}

Adverbials, whether true adverbs, or PPs, can occur following a linking pronoun:

\acex{Noun}{Adverb}{375}
{ku-d žaġīl dīd tam}
{every-\cst{} worker \lnkd{} there}
{any worker (who was) there}
{CohenZakho}{116 (109)}

\acex{Noun}{\PP}{415}
{aw gōra dīd go māya}
{\definite.\masc{} man \lnkd{} in water}
{the man in the water}
{CohenZakho}{132 (175)}

\acex{Noun}{Adverbial Phrase}{447}
{xabra d\cb{} [la l\cb{} dūk-e]}
{word \lnk\cb{} \neg{} to\cb{} place-\poss.3\masc}
{inappropriate word (lit. word not in its place)}
{CohenZakho}{217 (11)}

Given that the \lnk* \d typically precedes clausal \secns, the last example's \secn may be understood as a reduced clause, lacking a \isi{copula}.

Note that adverbials (prepositions or conjunctions) cannot serve as \prims of the ALC. This is not surprising, since a pronominal element cannot in general represent a preposition. Moreover, adverbials by virtue of their function are equivalents of \cst* nouns which require a complement, and are as such incompatible with the \lnk*. When a \lnk* occurs after a preposition, it refers anaphorically to an implicit \prim (see \example{304}).


\subsection{Linkers without an explicit \prim} \label{ss:JZax_lnk_zero_head}


As explained in \ref{ss:Analytic_AC}, the \lnk* is seen as pronominal since it  is capable of heading an AC without any explicit nominal preceding it, functioning analogously to a \cst* noun. Yet  in \JZax at least, the \lnk* is different from fully fledged pronouns in that it does not in general replace an entire determined noun-phrase, but rather only the \isi{head noun} of a bare noun-phrase (see \vref{tb:NP_struc}). Thus, whereas a normal pronoun would not typically follow a determiner, the pronominal \lnk* usually requires a determiner to precede it, as the following examples show. In the majority of cases the determiner is definite, but not always, as \example{368} shows.\footnote{Following the terminology presented in \vref{tb:NP_struc}, the normal pronouns represent a determined NP (or DP), while the \lnk* represents a bare NP.} 
For clarity of exposition, I put a \zero\ symbol in the examples, where an overt \prim could have occurred.



\acex{\zero}{Ordinal}{306}
{ʾay \zero{} d\cb{} treʾ}
{\definite.\fem{} \zero{} \lnk\cb{} two}
{the second (one\textsubscript{\fem})}
{CohenZakho}{95 (7)}

\acex{\zero}{Ordinal}{348}
{ʾaw \zero{} dīd tre}
{\definite.\masc{} \zero{} \lnkd{} two}
{the second (one\textsubscript{\masc})}
{CohenNucleus}{84 (8)}

This construction is also common with clausal \secns, yielding a kind of \concept{free relative}:

\acex{\zero}{Clause}{423}
{ʾay \zero{} d\cb{} g-əbá-wā-le}
{\definite.\fem{} \zero{} \lnk\cb{} \ind-want.\agent3\fem-\pst-\patient3\masc}
{the one\textsubscript{\fem} who wanted him}
{CohenZakho}{134 (183)}

\acex{\zero}{Clause}{350}
{ʾaw \zero{} dīd hāyē \cb{}b-e əll-i}
{\definite.\masc{} \zero{} \lnkd{} \sbjv.be.3\masc{} \cb{}in-3\masc{} \acc-1\sg}
{the one\textsubscript{\masc} who can defeat me}
{CohenNucleus}{84 (9)}

In some cases, as in the following example, the linker is seemingly coalesced with the \isi{genitive marking} of the \isi{copula} (see the discussion of \example{398}). Alternatively, such cases could be analysed as case of asyndetic free relative clauses (see \example{434}) with an attributive marked \isi{copula}. 

\acex{\zero}{Clause}{404}
{ʾan \zero{} d-īlu ʾəsye m\cb{} qabəl mənn-an}
{\definite.\pl{} \zero{} \opt\lnk.\gen-\cop.\pl{} came.\resl.\pl{} from\cb{} before from-1\pl}
{those who came before us}
{CohenZakho}{122 (141)}

One may wonder whether the \isi{determiners} themselves should not be simply analysed as pronouns in the \prim position of an AC. To this \citet[134]{CohenZakho} answers: \textquote{The \textsc{det}s \textit{ay}, \textit{aw} and \textit{an} are neither pronouns nor do they function as such, and hence cannot be suspected to be antecedents. In all these examples, what we have are in effect determined complex nominal syntagms.} Other cases, however, do cause an analytical ambiguity. Such is the case of \foreign{xa}{one} in the following example, where it can be analysed both as a \isi{indefinite determiner} or as an indefinite pronoun.  \citet[111f.]{CohenZakho} seems to imply that \transc{xa} in this position is a determiner.\footnote{One may argue that the long version \phonetic[xā] present in the example represents necessarily a pronoun \citep[cf.][191]{SabarDictionary}. Yet  the vowel-lengthening may be due to prosodic reasons, like the lack of a stress-bearing nominal following \transc{xa}, and as such cannot be taken as a clear indication of grammatical status. See also the similar \example{433} where \transc{xa} does not have a long vowel.}  Note, moreover, that the noun phrase introduced by \transc{xa} is itself the \secn of CSC headed by \foreign{pumma}{mouth}.

\acex{Pronoun/\zero}{Clause}{368}
{(mən pumm-ət) xā \opt{\zero}  d\cb{} [la ʾāwəz gāzənda mən baxt-e]}
{from mouth-\cst{} \indef.\pro/\textsc{det} \opt{\zero} \lnk\cb{} \neg{} \sbjv.do.3\masc{} complaint from wife-\poss.3\masc}
{(from the mouth of) someone who does not complain about his wife}
{CohenZakho}{112 (89)}

In some cases, however, the \lnk* does appear without an overt determiner or nominal element immediately preceding it, and these require some further examination.\footnote{I dismiss cases where \transc{dīd} forms part of a conjoint attributive complex, such as in examples \vref{ex:357} or \vref{ex:356}, as in such cases one can argue that both \transc{dīd} phrases are leaning on the same \prim noun.} The first case occurs whenever the \lnk* is complemented by a \isi{pronominal suffix}, such as in the following examples (the relevant expressions are marked in \textbf{bold}):\footnote{\citet[138]{CohenZakho} writes on the complex \lnk-\poss: \enquote{The latter is a pronoun, rather than an adjective}, but at the same time \transc{dīd/d-} is considered to be on its own a \enquote{construct-state pronoun} \citep[452]{CohenZakho}.}

\acex{\zero}{Pronoun}{304}
{mā\cb{} xulla qaḥra dīd-ox zōdan-ta \cb{}la mən \zero{} \textbf{dīd-i}}
{what\cb{} \textsc{rhetoric} grief(\fem) \lnkd-2\masc{} superior-\fem{} \cb{}cop.3\fem{} from \zero{} \lnkd-1\sg{}}
{Is your grief superior to mine?}
{CohenZakho}{95 (5)}

\acex{\zero}{Pronoun}{349}
{ha\cb{} wē-la šasəqsa dīd-a ʾəmm-i ū\cb{} kaffīya dīd-a ū\cb{} \zero{} \textbf{dīd-i} lēw-u kəs-li}
{here\cb{} \cop-3\fem{} ring \lnkd-3\fem{} with-1\sg{} and\cb{} scarf \lnkd-3\fem{} and\cb{} \zero{} \lnkd-1\sg{} \neg.\cop-3\pl{} at-1\sg}
{Here is her ring with me and her scarf, but mine are not with me}
{CohenNucleus}{84, fn. 7}

Such cases call for two analyses: either the combination \lnkd+\poss\ has been  grammaticalised as an independent genitive pronoun (possibly due to the inherent \isi{definiteness} of the \isi{pronominal suffix}\footnote{Recall that the \isi{definiteness} of the \ili{Semitic} CSC is typically determined by the \secn; see \sref{ss:CSCdet}.}), or it simply forms a bare NP that has a \zero\ determiner (which is defined by \citet[454]{CohenZakho} to be \enquote{±definite; generic}). The latter option may be more adequate, since a similar example of a linker followed by a \isi{possessive pronoun} is found with an overt determiner (but note the somewhat unexpected non-definite meaning):

\acex{\zero}{Pronoun}{437}
{ay \zero{} d-ōhun (mərta \cb{}la)}
{\definite.\fem{} \zero{} \lnk-3\pl{} said.\resl.\fem{} \cb{}\cop.\fem}
{One\textsubscript{fem} of them (has said).}
{CohenZakho}{138, fn. 36}


Another case where a \lnk* construction appears without an overt determiner is in predicative position. In this position it may well be a bare NP, as the predicate position is quite flexible syntactically (it can accommodate as well bare adjectives or \isi{adverbial} phrases). The following two examples constitute a question/answer pair, both having a predicative \transc{dīd}. 

\acex{\zero}{pronoun}{311a}
{\zero{} dīd mani \cb{}le ōqadda māl? }
{\zero{} \lnkd{} who \cb{}\cop.3\masc{} so\_much property}
{Whose is so much property?}
{CohenZakho}{95 (12)}

\acex{\zero}{noun}{311b}
{g-əmr-i: \zero{} dīd flāna nāša \cb{}le}
{\ind-say-3\pl{} \zero{} \lnkd{} certain man \cb{}\cop.3\masc}
{They say: it is of a certain man.}
{CohenZakho}{95 (12)}

Finally, there is a quite different usage of the \d morpheme without a nominal \prim, namely its usage as a \isi{complementizer}, such as in the following examples:

\acex{Verb}{Clause}{397}
{(u\cb{} xa lá\cb{} k-īʾē-wa) d\cb{} ʾanya trēʾ baxt-ā́s d-ō gōra \cb{}lu}
{and\cb{} one \neg\cb{} \ind-know-\pst{} \comp\cb{} \dem.\pl{} two woman-\pl.\cst{} \gen-\dem.\masc{} man \cb{}\cop.3\pl}
{(But nobody knew) that these two are wives of the same man.}
{CohenZakho}{121 (136)}

\acex{Verb}{Clause}{406}
{(la rʾəš-le) d\cb{} d-īla ḥməl-ta məlʾḗl mənn-e}
	{\neg{} felt-3\masc{} \comp\cb{} \gen-\cop.3\fem{} stood.\resl-\fem{} above from-3\sg}
{He didn't feel she is standing above him.}
{CohenZakho}{124 (143)}

\largerpage
One may reasonably argue that when used as a \isi{complementizer}, \transc{dīd/d-} is not part of  an AC, and is thus distinct from the \lnk*.\footnote{Paradigmatically, it can be replaced by other complementizing particles, such as \foreign{ʾənnu}{that} or \foreign{hakan}{whether}, attesting to its different status. See also the discussion regarding the similar \Syr \examples{978}{975}.} Further evidence to this is adduced by the fact that in some cases, such as in example \ref{ex:397} the \isi{copula} does not appear in its genitive form, as expected in an AC. 

\section{Genitive marking of \secns} \label{ss:JZax_gen}

As discussed in \ref{ss:d_gen}, all \isi{determiners} and demonstratives of \JZax which  start with  a glottal-stop or a vowel have a special genitive allomorph, formed by removing the \isi{glottal stop} and prefixing the \gen* \d marker.
These include the definite articles (see \vref{tb:jzax_gen_det}), some demonstratives and the interrogative determiner \foreign{ēma}{which}. As mentioned there, the discovery and analysis of this phenomenon in \JZax is due to \citet{CohenNucleus}, but it occurs in other dialects as well.  As the  facts motivating this analysis are the clearest in \JZax,  I repeat them here briefly.

First, as stated above, the \d prefix appears only before a closed set of \isi{determiners} and demonstratives, and not before other any \transc{ʾ} or vowel-initial word (compare \vref{ex:363}), thus excluding a simple phonological conditioning of its appearance.\footnote{Clearly, such a phonological conditioning took part in the diachronic emergence of the marker, but it is no longer operative. In \Nrt (texts from the 17\th century) one finds examples like \foreign{šəmm-əd d-ʾəlāha}{name of God} \citep[38, §2]{SabarDictionary}.} 

Second, note that the \d prefix appears after \prims with a suffixed \cst* marker, \prims marked by apocopate \cst*, as well as unmarked invariable \prims:

\acex{Noun}{Noun}{359}
{pumm-əd d-aw nāša}
{mouth-\cst{} \gen-\definite.\masc{} man}
{the mouth of the man}
{CohenZakho}{107 (76)}

\acex{Noun}{Noun}{358}
{bēs d-aw gōra}
{house.\cst{} \gen-\definite.\masc{} man}
{the house of the man}
{CohenZakho}{107 (75)}

\acex{Noun}{Infinitival Phrase}{366}
{sabab [d-o ʾīzāla dīd-ax]}
{reason(\invar) \gen-\definite{} go.\inf{} \lnkd-2\fem}
{the reason of your going}
{CohenZakho}{111 (87)}


The first example proves that the \d prefix is distinct from the \cst* marker, while the second and third examples show that it occurs also when no \ph/d/ segment is called for by a \cst* marking. 

The last point can also be exemplified when the \isi{genitive marking} follows an \isi{adverbial} which normally do not get the \transc{-əd} suffix:

\acex{Preposition}{Noun}{361}
{mən d-ay xzēna}
{from \gen-\definite{} treasure}
{from the treasure}
{CohenZakho}{108 (77)}

In contrast to other dialects, the \gen* marking is obligatory whenever its appearance conditions are met. The very few exceptions listed by \citet[108, fn. 15]{CohenZakho} can probably be explained by speech \textit{lapsi} (or transcription errors), rather than a systematic optionality.  

As we will see below, the \gen* marker is also distinct from the linking pronoun, with which it can co-occur.

\subsection{Genitive marking following the linker}

Since the \lnk* stands in direct \isi{attributive relationship} with its complement (the \secn), the latter is marked by \isi{genitive case} whenever possible:

\acex{Noun}{Noun}{355}
{ʾōda dīd d-aw gōra}
{room \lnkd{} \gen-\definite.masc{} man}
{the room of the man}
{CohenZakho}{106 (72)}

A \gen* marker following the short \lnk* \d is also found, as in the following example (=\example{354bis}). In the original source it is written as a separate word, but very likely it procliticizes to the following word. Given that \d typically comes before clausal \secns (see \sref{ss:JZax_lnk_clause}), it is possible that it appears here as a phonetic simplification of the sequence \transc{dīd d-o} > \transc{d\cb{} d-o}.


\acex{Noun Phrase}{Noun}{354}
{(mən) [d-o bəhna rwīxa] d\cb{} [d-o jwanqa]}
{from \gen-\dem.\masc{} breath(\masc) wide.\masc{} \lnk\cb{} \gen-\dem.masc{} youngster}
{(from) the patience of this youngster}
{CohenZakho}{106 (71)}


\subsection{Genitive marking of clauses} \label{ss:JZax_genitive_clauses}

As we have seen above (\sref{ss:JZax_clausal}), clauses may act as \secns. In some cases, their \secn status is marked by the very same prefix \d.  Indeed, \JZax has developed a special series of \textit{genitivally} marked copulas, which consist of the normal (indicative present) \isi{copula} preceded by \d. While the \isi{copula} is the marked element, the scope of the marking should be understood as the entire \secn clause. As \citet{CohenNucleus} notes, this innovation of \JZax is similar to an \ili{Akkadian} construction, but no direct influence can be adduced.\footnote{It should be noted that the \isi{genitive marking} of clauses is different from the \concept{subjunctive mood} (i.e. \transc{šaqəl} forms without a pre-verbal particle) which exists in \JZax and other dialects. While the subjunctive mood is frequently found with embedded clauses, it  adds a semantic mood value to the utterance, in contrast to the \gen* marker, which is a pure grammatical mark of \secn position. Moreover, the subjunctive form can appear in matrix clauses, as \example{315} shows.} 

\acex{Noun}{Clause}{384}
{(b-əd) ḥaqq-əd d-īlu ʾəsye mən mōṣəl}
{in-\cst{} price-\cst{} \gen-\cop.3\pl{} came.\resl{} from{} M.}
{at the price which they had come to you from Mosul}
{CohenZakho}{119 (116)}

\acex{Noun}{Clause}{383}
{(psōx) xā sandū́q [d-īle mutwa go qurnīs-ət čappe]}
{open.\imp{} \indef{} chest \gen-\cop.3\masc{} placed.\resl{} in corner-\cst{} left}
{open a chest that is placed in the left corner}
	{CohenZakho}{119 (119)}

Thus, the \gen* marking appears both after nouns with suffixed \cst\ marker or in apocopate \cst\ form. It is worthwhile noting that the \d prefix occurs even when the \isi{copula} is not the first element of the attributive clause, though this must occur quite rarely, as the following example is unique in my survey:

\acex{Noun}{Clause}{298}
{(qam mesē-li ʾəl) d-ay qaṣər [maṭməryam d-īla ʾəll-e]}
{\pst{} brought.\agent3\pl-\patient1\sg{} to \gen-\definite{} castle.\cst{} VM. \gen-\cop.3\fem{} on-3\masc}
{(They brought me to) the castle, on which (the statue of) Virgin Mary is.\footnotemark}
{CohenZakho}{119 (117)}

\footnotetext{If  \transc{dīla} is assumed to open the \isi{relative clause}, it yields the quite odd interpretation \enquote{the castle of Virgin Mary, whose (statue) is on it}.}

The \isi{genitive marking} appears also in ACs headed by \isi{adverbial} conjunctions, whether they are invariable or \cst* marked:

\acex{Conjunction}{Clause}{392}
{ mən hīng-əd d-īla hwīsa}
{from then-\cst{} \gen-\cop.3\fem{} born.\resl.\fem }
{from the time she was born}
{CohenZakho}{120 (127)}

\acex{Conjunction}{Clause}{394}
{čukūn d-īw-ət qarīwa l\cb{} ləbb-i}
{since \gen-\cop-2\masc{} close.\masc{} to\cb{} heart-\poss.1\sg}
{since you are close to my heart}
{CohenZakho}{121 (130)}\antipar 
\newpage 

The same genitive \isi{copula} is used also for clausal \secns of the ALC, i.e.\ following the \lnk*: 

\acex{Noun (subject)}{Clause}{386}
{ō gōra dīd d-ī-le go namūsīy-e}
{\dem.\masc{} man \lnkd{} \gen-\cop-3\masc{} in bed-\poss.3\masc}
{the man who is in his bed}
{CohenZakho}{119 (120)}

\acex{Noun (subject)}{Clause}{401}
{ʾē baxta d\cb{} d-ī-la baxt-e}
{\dem.\fem{} woman \lnk\cb{} \gen-\cop.3\fem{} woman-\poss.3\masc}
{that woman who is his wife}
{CohenZakho}{122}

In the last example the \d \lnk* cliticizes to the \d genitive marker. This resulting \transc{d\cb{}d} cluster is sometimes simplified to a de-geminated \phonemic{d}, serving in both functions \citep[cf.][122]{CohenZakho}:

\acex{Noun (subject)}{Clause}{398}
{ay ʾurxa d-īl-a msukar-ta mən qam mšalxāne ū\cb{} ganāwe}
{\definite.\fem{} road \lnk.\gen-\cop-3\fem{} closed.\resl-\fem{} from before robbers and\cb{} thieves}
{the road which is closed because of robbers and thieves}
{CohenZakho}{121 (137)}



The \gen* marking of clauses, on the other hand, is possible only whenever the \secn clause uses the indicative \isi{copula}. When no such \isi{copula} is present, such as when a form of the verb \transl{to be} is used, no \gen* marking is apparent. This is exemplified by \example{313} and possibly also by \example{433}. Similarly the existential particle (glossed \exist), combined here with the preposition \foreign{b-}{in} to denote ability,  has no \isi{genitive marking}:

\acex{Pronoun}{Clause}{315}
{(šaql-axni) mā-t [ʾī-b-an mṭāš-ax go jēbē-ni]}
{\sbjv.take-1\pl{} what-\cst{} \exist-in-1\pl{} \sbjv.hide-1\pl{} in pocket.\pl-\poss1\pl}
{Let’s take whatever we can hide in our pockets.}
{CohenZakho}{96 (20)}



\section{The double annexation construction (X-y.\poss\ \textsc{lnk}-y)} \label{ss:JZax_DAC}

In general the DAC is not used in \JZax. A rare usage of it occurs when the \secn is pronominal, in which case a \isi{possessive pronoun} can be suffixed both to the \prim noun and to a \lnk*. This yields some added pragmatic emphasis. 

\acex{Noun}{Pronoun}{369}
{baxt-i dīd-i}
{woman-\poss.1\sg{} \lnkd-1\sg}
{my own wife}
{CohenZakho}{113, fn.\ 19}


\section{Juxtaposition (X Y.\opt{\agr})} \label{ss:JZax_juxt}

By \concept{juxtaposition} I mean a construction in which the two members of the AC are put adjacent to each other, without any further marking (except for the possibility of agreement, in which case it is \isi{juxtaposition-cum-agreement}).
This type of construction is reserved in \JZax for several quite distinct cases, which are detailed below.

\subsection{Adjectival attribution}

Adjectives are normally directly juxtaposed after the \prim noun. Syntactically, they stand in \isi{apposition} with the \prim noun; the \isi{attributive relationship} itself is expressed indirectly by agreement of the adjective with the \isi{head noun}. Such cases are termed here accordingly \concept{juxtaposition-cum-agreement}. The inflection of adjectives, however, is mostly restricted to adjectives of Aramaic origin, especially with regard to the gender feature.

\acex{Noun}{Adjective}{440}
{xa jwanqa sqīl-a}
{\indef{} youth(\masc) beautiful-\masc}
{a beautiful youth}
{CohenZakho}{214}

\acex{Noun}{Adjective}{441}
{xa xamsa sqəl-ta}
{\indef{} maiden(\fem) beautiful-\fem}
{a beautiful maiden}
{CohenZakho}{214}

The ordinal \foreign{qamāya}{first} acts as an adjective, as is seen in \example{461}. 

Borrowed adjectives sometimes do not inflect (or inflect only for number). This is the case of  the adjective \foreign{ʿāqəl}{wise}, borrowed from \ili{Arabic} \citep[246a]{SabarDictionary}. Such cases are truly zero-marked ACs:

\acex{Noun}{Adjective}{439}
{ʾaxōna ʿāqəl}
{brother wise(\invar)}
{(the) wise brother}
{CohenZakho}{214}

Occasionally an inflecting definite determiner precedes the adjective, instead of the \prim noun \citep[215]{CohenZakho}. The reasons for this are unclear, and may be related to some unknown semantic or stylistic factors. One syntactic possibility is that it marks the grammatical features of a non-inflecting (or partially inflecting) adjective, as in the following example (=\example{462bis}):


\acex{Noun}{Adjective}{462}
{axōna aw rūwa}
{brother \definite.\masc{} big.\sg\footnotemark}
{the older brother}
{CohenZakho}{214}

\footnotetext{Note that \foreign{rūwa}{big} inflects only for number in \JZax \citep[288b]{SabarDictionary}.}

Another motivation might be the occurrence of a possessed noun as the \prim, which normally is not marked by a determiner preceding it:

\acex{Noun Phrase}{Adjective}{450}
{axōn-e aw ʿāqəl}
{brother-\poss.3\masc{} \definite.\masc{} wise(\invar)}
{his wise brother}
{CohenZakho}{214}

Recall, however, that similar examples occur also in \Syr (see \example{2029}).\footnote{Another possibility suggested by Eran Cohen (p.c.) is that the determiner imitates the position of the \Kur \lnk* \ez*, which precedes the adjective (see \sref{ss:kurd_lnk_adj} as well \vref{par:adj_secn}). For possible relationship to the \ili{Semitic} heritage see \vref{ft:PatElDefinite}. \label{ft:ex462}} 




\subsubsection{Inverse order juxtaposition (Y X)}

\citet[214, fn. 2]{CohenZakho} mentions that \enquote{in a small number of cases}, the adjective precedes the noun, creating  an \isi{inverse \isi{juxtaposition} construction}:\footnote{Recall that the title of examples follows always the order \Prim-\Secn.}

\acex{Noun}{Adjective}{443}
{aw fīta ṭūra}
{\definite.\masc{} huge(\invar) rock(\masc)}
{the huge rock\footnotemark}
{CohenZakho}{214, fn. 2}

\footnotetext{\citet[172a]{SabarDictionary} lists \transc{ṭūra} as \transl{mountain}, but I follow here the translation given by \citeauthor{CohenZakho}.}

\acex{Noun}{Adjective}{456}
{ʾē fīta brāta}
{\dem.\fem{} huge daughter}
{this grown-up girl}
{Avineri}{124 (680)}

Interestingly, \citet{SabarDictionary} mentions a variant of this construction with the \cst* suffix, which is very similar to the \concept{emotive genitive} construction presented  in \example{337}.

\acex{Noun}{Adjective}{463-4}
{xa fīt-a/əd gōra}
{\indef{} huge-\free/\cst{} man}
{a huge person}
{SabarDictionary}{263a}

Thus, it may be that {inverse juxtaposition}\isi{inverse juxtaposition construction} of adjectives has some {emotive}\is{emotive genitive} value as well, but this question has not been investigated in the scope of this work. 

\subsection{Adverbial \prims} \label{ss:JZax_Juxt_Adv_head}

As mentioned in \sref{ss:JZax_Adv_head}, prepositions which are not explicitly marked with a \cst* \ed suffix cannot be considered to be marked by \isi{apocope}, since they do not have a free form (unless they are derived from  nouns). In these cases they are merely juxtaposed before their complement, as in the following example (contrast with \example{351}):

\acex{Preposition}{Noun}{352}
{b\cb{} bəllūre}
{in\cb{} flute}
{with a flute}
{CohenNucleus}{85 (15)}


 Nonetheless, most prepositions induce an \isi{attributive relationship}, as is clear from the occurrence of genitive marked \secns following them, such as in  \example{361}.

\citet[104]{CohenZakho} does mention, however, one preposition, \foreign{bēb}{with}, which does not induce \isi{genitive case}, and thus formally always realises a \isi{juxtaposition} pattern:\footnote{\citet[108]{SabarDictionary} lists this preposition as being possibly from Kurdish origin, while \citet[121]{MutzafiBetanure} gives a possible Aramaic etymology \transc{b-ēh b-} consisting of the preposition \foreign{b-}{in} repeated with a \isi{proleptic pronoun}. Given that Kurdish \Kur prepositions induce \isi{oblique case}, the Aramaic etymology may be the correct one.}

\acex{Preposition}{Pronoun}{345}
{bēb ʾāwa}
{with 3\masc}
{with him}
{CohenZakho}{104 (70)}

It is interesting to note that a preposition can be diachronically derived from an apocopate \cst* noun, without being any more synchronically connected to it. Such is the case of the preposition \foreign{rəš}{on}, derived from the noun \foreign{rēša}{head}, still present in the dialect. The former is a phonologically reduced form of the \cst* of the latter, \transc{rēš}.\footnote{The derivation of the preposition \transl{on} from the noun \transl{head} is probably a pattern borrowing from Kurdish, where the word \transc{ser} has the same two meanings \citep[206]{Noorlander}. \label{ft:reš}} While synchronically the noun can also occur as \transc{rēšəd}, the preposition is invariable. The contrast is neatly shown in the following example (continuing \example{314a}):

\acex{Preposition}{Noun Phrase}{314b}
{rəš [rēš-ət manī-t yātū-wa]}
{on head-\cst{} who-\cst{} \sbjv.sit-\pst}
{on the head of whomever it would sit}
{CohenZakho}{96 (18)}




\subsection{Adverbial \secns}

Nouns, as well as pronouns, can be modified by at least some \isi{adverbial} \secns without any AC marking, neither on the \prim nor on the \secn:

\acex{Noun}{\PP}{448}
{xá quṭeʾfa mux quṭeʾf-ət ʾədyo ġzē-li}
{\indef{} cluster like cluster-\cst{} today saw-1\sg}
{a cluster like the cluster I saw today}
{CohenZakho}{217 (12)}

\acex{Pronoun}{\PP}{449}
{xá mənn-ēni}
{one from-1\pl}
{one of us}
{CohenZakho}{217 (14)}





\subsection{Clausal \secns} \label{ss:JZax_Asyndetic_Rel_Clauses}

Clausal \secns can follow an indefinite \prim noun asyndetically, i.e.\ without any particular marking. \citet[138]{CohenZakho} tentatively relates this pattern to \ili{Arabic} influence, whether direct or indirect, as in \ili{Arabic} too the construction is confined to indefinite \prims.\footnote{\textquote[{\cite[137f.]{CohenZakho}}][.]{Most asyndetic adjective clauses [... are] perhaps modeled after the \ili{Arabic} ([but] this phenomenon also occurs  in \ili{NENA} dialects that are clearly outside the \ili{Arabic} speaking area)}. See discussion in \sref{ss:asyndetic_relatives}.}

\acex{Noun}{Clause}{431}
{(u\cb{} ʾāna gəb-ēn ṭāʾ-ət-tə) [xa xamsa] [hōya mən məšpāḥá bāš]}
{and\cb{} 1\sg{} \ind.want-1\sg{} look-\agent2\masc{}-\dat.1\sg{} \indef{} maid \sbjv.be.3\masc{} from family good }
{I want you to look for a maid (who) is from a good family.}
{CohenZakho}{137 (214)}


Such asyndetic clauses may be related to examples like the following:

\acex{Noun}{Clause(?)}{451}
{(mpəq-le mən xá\cb{} ʾāl) [xa ʾarya] [ʾēn-e smōq-e]}
{came\_out-3\masc{} from one\cb{} side \indef{} lion eye-\poss.3\masc{} red-\pl}
{From one side came out a lion, his eyes red}
{CohenZakho}{225 (63)}

\citet[225]{CohenZakho} classifies such cases as \concept{non-clausal adjectival nexus}. In his view, while these are expressions of a \isi{predicative relation} (a \concept{nexus}, see \sref{ss:threeRel}), they are not clausal, since they lack a \isi{copula}.\footnote{Cf. \citet[257]{GoldenbergSemitic} who writes: \enquote{such syntagms are not conceived as asyndetically embedded sentences, which in Neo-Aramaic would require a \isi{copula} [...] In other words, such syntagms might be said to incorporate the cohesive or relational, but not the assertive, constituent of the nexus.}} Nevertheless, since nominal clauses lacking a \isi{copula} do occur (albeit rarely) in \ili{NENA} dialects, one may relax the usage of the term \concept{clause} to include such cases as well. 


The morpheme \transc{xa}, followed by an asyndetic attributive clause, presents the same analytical difficulty found in \example{368}. Either one analyses it as an indefinite pronoun (\transl{someone}), which is followed directly by an attributive clause, and is not marked as \cst* (contrast with the \cst* pronouns in \examples{303}{316}), or one analyses it as an \isi{indefinite determiner}, followed by a \zero\ \prim noun + \isi{relative clause}, which  renders the structure parallel to the above examples:

\acex{Pronoun/\zero}{Clause}{433}
{(la ġzē-lu) xa \opt{\zero} [šəmm-e hāwe qaramā́n]}
{\neg{} saw-3\pl{} \indef.\pro/\textsc{det} \opt{\zero} name-\poss.3\masc{} \sbjv.be.3\masc{} Q.}
{They did not find anyone whose name is Qaraman.}
{CohenZakho}{137 (216)}

Note that in either analysis, \transc{xa} introduces a discourse referent, which is referred to in the \secn clause by the 3\masc\ pronominal indices. 

The latter analysis may also be preferred due to the rare occurrence of a similar structure introduced by a \textit{definite} determiner. In such examples, as the following, I also posit the occurrence of a \zero\ \prim.\footnote{Note though that such cases are unusual, in that a definite \prim would normally not allow an asyndetic \isi{relative clause}. \citet[138]{CohenZakho} mentions that such examples are only found in one source \citep{SabarAgonies}, which is a unique genre of autobiographical account.} 

\acex{\zero}{Clause}{434}
{an \zero{} k-sēm-i mənn-an}
{\definite.\pl{} \zero{} \ind-hate-3\pl{} from-1\pl }
{those who hate us}
{CohenZakho}{138 (217)}




\subsection{Infinitival \prims}

As shown in \ref{ss:JZax_inf_head}, infinitives can be marked by \cst* morphology. This marking, however, is not obligatory, probably due to the verbal nature of infinitives: When followed by an object argument, the infinitive can induce a \isi{completive relation}, which is manifested by the unmarked \isi{juxtaposition} construction, rather than a marked \isi{attributive relation}. Contrast the following example with \example{332b} (which is the continuation of this example):



\acex{Infinitive}{Noun}{332a}
{(bəš xlē-la ʾəll-i) ʾwāza xōrūs-ax (mən \ldots)}
{more pleased-3\fem{} to-1\sg{} make.\inf{} friendship-\poss.2\fem{} than \ldots}
{Befriending you (pleases me more than befriending emirs' and pashas' daughters.)}
{CohenZakho}{99 (38)}

\section{Conclusions} \label{ss:JZax_conclusions}

\JZax has three types of AC markers (\cst* marking, the \lnk*, and \gen* case), but only two can co-occur at the same time, as the \cst\ and \lnk\ marking are in \isi{complementary distribution}. 
In other words, \JZax has two slots of marking the AC, presented in \vref{tb:JZax_loci}. Note that these slots do not correspond directly to head-marking vs.\ dependent marking, as the first slot opposes the head-marking \cst* markers attaching to the \prim with the dependent  marking \lnk*, which attaches syntactically and phonologically to the \secn. 


\begin{table}[h]
\centering
\begin{tabular}{lc cr}
\toprule
  & 1		& 2		 &   \\		
  \midrule  
X & \begin{tabular}{c} \cst\ (\ed, Ap.) \\ \lnk\ (\d, \transc{dīd})  \\ 
	\end{tabular}	& ± \gen (\d) & Y \\
\bottomrule
\end{tabular}
\caption{AC markers in \JZax}\label{tb:JZax_loci}
\end{table}

The dialect presents both innovative and conservative aspects of the AC system, as compared to \il{Aramaic!Classical}Classical Aramaic (of which I take \ili{Syriac} as the main point of comparison). First and foremost, it conserves the classical \ili{Semitic} logic of the \isi{attributive relation} marking: this relation must be marked either directly on the \prim noun (by means of \cst* marking), or indirectly by means of a pronominal \lnk*. Moreover, the distinction between nominal/clausal attribution, in which the \isi{attributive relationship} is overt, and \isi{adjectival attribution}, in which it is covert (formally realised only by agreement) is strictly kept. On the other hand, \JZax innovated the morphological material available for marking the \cst*, most notably the suffixed \cst* \ed marker, and to a lesser degree novel apocopate forms. \citet[121ff.]{CohenEzafe} suggests that the development of the \ed suffix is related to \isi{pattern replication} from the co-territorial \Kur, yet an internal development stemming from \isi{encliticization} of the \il{Aramaic!Classical}Classical Aramaic \d \lnk* to the \prim is possible as well (see discussion in \sref{ss:role_contact}). 

Another related morphological innovation is the introduction of the variant form \transc{dīd} as an independent linker, in contrast to a base appearing only with possessive pronominal suffixes as in \JBA.  This may be explained by the resemblance of the \transc{dīd} \lnk* to a \d \lnk* augmented by the \cst* \ed suffix; see \sref{ss:did_lnk} for a discussion. Note that all the above discussed innovations do not change the basic logic of the system, but rather affect only the forms involved.


Structurally more innovative is the introduction of \isi{genitive marking} in certain morphological environments (certain \isi{determiners}, and the indicative \isi{copula}). While \isi{genitive marking} existed in ancient \ili{Semitic} languages, it is unknown in prior strata of Aramaic,\footnote{But see \vref{ft:gen_Samal}.} and must be considered as an innovation. It is clearly innovative in that it constitutes an additional marker of ACs (on top of the \cst* marking or the \lnk*). Moreover, in some cases, such as ACs headed by simple prepositions, it reveals the existence of an \isi{attributive relation} which could in earlier strata only be posited abstractly. In this respect, the innovation of \isi{genitive marking} is more important structurally than the innovation of new \cst* forms, although it affects only a restricted number of grammatical items. Nonetheless, the overall effect of this innovation on the attributive system is small, as it only adds  morphological marking in restricted cases, without restructuring any AC.

\largerpage
Another point which can be considered innovative is the usage of the \lnk* in marking completive relations, following infinitives as well as resultative \isi{participles} (see \sref{ss:JZax_VerbalNouns_heads}). This phenomenon, however, is situated at the periphery of the attributive system, and is in fact related to the development of periphrastic verbal constructions.

Another innovation, as compared to earlier strata of Aramaic, is the possibility of introducing relative clauses directly after \prims marked for \cst*, irrespective of the type of marking (\isi{apocope} or the \ed suffix). While this possibility existed in ancient \ili{Semitic} languages (such as \BHeb), it must have been reintroduced in \ili{NENA}. Indeed, the usage of clausal \secns after apocopate \cst* \prims is specific to the J. Cis-Zab\il{Cis-Zab NENA dialects} group, as it is only clearly attested there.\footnote{See \vref{tb:app_cst} and the following \Amd example for another J. Cis-Zab\il{Cis-Zab NENA dialects} dialect showing this construction: 

\protectedex{\ex.[(i)] \Amd: \textbf{Noun--Clause} \label{ex:1604} \nopagebreak \bg.[] ḥil yom gawər-wa-la \\ until day.\cst{} marry.\agent3\masc-\pst-\patient3\fem \\ \transl{until the day he married her}
\ifx\relaxGreenblattAmidya\relax73\else\citep[73]{GreenblattAmidya}\fi \par }
} To be sure, it does not appear in \ili{Syriac}. This innovation can be understood as filling a syntactic   gap (or asymmetry) in the system which existed in previous Aramaic dialects, including Syriac, since it extends the possibility of marking ACs with nominal \secns either by the linking pronoun or by the \cst* marking to clausal \secns as well. As such, it can be said to be an internally motivated development (due to the general force of \concept{analogy}), but see again \citet[123]{CohenEzafe} who suggests that this possibility is due to \isi{pattern replication} from \Kur.
 In the realm of relative clauses, one finds also the innovation of asyndetic relative clauses (see \sref{ss:JZax_Asyndetic_Rel_Clauses}), which is known in other \ili{Semitic} languages, but not in Aramaic. Here, the suggestion of borrowing from \ili{Arabic} \citep[138]{CohenZakho} seems plausible; see further discussion in \sref{ss:asyndetic_relatives}. 

 
As for the phonological material, \JZax is clearly conservative, as it recycles the same material for the new morphological devices. The D-markers derive from the \il{Aramaic!Classical}Classical Aramaic \lnk* \d, which by various cliticization and re-analysis processes  yielded the \cst* suffix \ed (see \sref{ss:neo-CSC}) and the \gen* prefix \d (\sref{ss:genitive_development}). The \lnk* \transc{dīd} existed already in \il{Aramaic!Classical}Classical Aramaic (in particular in \JBA) as a pronominal base of the independent genitive pronouns, but it has been reanalysed in \JZax as an independent \lnk*, capable of introducing full nominal \secns (see \sref{ss:did_lnk}). Thus, no morphemes are borrowed from contact languages.

 \largerpage
In summary, notwithstanding the possibility of \isi{language contact}, it seems that most of the \JZax features regarding the AC system can be explained, at least in principle, by processes of internal development. These processes \enquote{shuffle around} morphemic material (making essentially analytical forms synthetic), but keep the essential logic of the system intact. 


