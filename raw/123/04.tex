

\chapter{The D-markers in NENA dialects} \label{ch:synchrony}









\section{Introduction}
In

all \ili{NENA} dialects one finds AC markers which are cognate with the \Syr \lnk* \transc{d(a)-} (discussed in \sref{ss:syr_ALC}), which must have existed also in the \ili{NENA} precursors of the \il{Aramaic!Classical}Classical Aramaic period. These are easily identifiable by virtue of containing a segment \phonemic{d} or \phonemic{t}, accompanied optionally by a \isi{schwa} before or after it. The probable unified historical origin of these markers has led many scholars to analyse them synchronically as variants of one and the same particle, termed in general \enquote{the \isi{annexation}/genitive particle \textit{d}}. For example, \citet[25]{JastrowHertevin} writes in his description of \Her: \blockquote{Die Annexion eines Nomens an ein vorangehendes Leitwort (Genetivverbindung) erfolgt durch die Genetivpartikel \textit{d}, die zwischen die beiden Nomina tritt.} This particle is assumed to have different allomorphs, varying both regarding its voice feature (\phonemic{d} vs.\ \phonemic{t}), and -- more importantly -- regarding its attachment as an \isi{enclitic} to the \prim or as a \isi{proclitic} to the \secn.\footnote{The phonological attachment is not always clearly stated in the descriptions, as sometimes one finds the \textit{d} written as a separate word standing between the two. This is the usual practice in Jastrow's description cited above.} This is most clearly stated by \citet[44]{FassbergChalla}: \blockquote{The Cl[assical] Aram[aic] relative pronoun \texthebrew{דִּי/דְּ} [=di-/d-] has allomorphs in J[ewish] Challa: \textit{-əd}, \textit{-d}, \textit{d-}, \textit{ʾəd-}.} Other scholars use in general the same practice, including Khan in his detailed descriptions of various \ili{NENA} dialects.  

While I agree that these elements must all be ultimately derived from a common ancestor morpheme \d (see the diachronic treatment in \cref{ch:diachrony1}), one of the claims advanced in this study is that these forms, which I subsume under the name \emph{D-markers}, represent in fact  different morphemes, differing in their morpho-syntactic distribution and, to a finer degree, in their pragmatic implication. More precisely, I distinguish between 1) a head-marking (phrasal) suffix \ed, equivalent functionally to \cst* marking, 2) a dependent-marking pronominal \isi{proclitic} \d,\footnote{Note that the \cb symbol, marking \isi{clitic} boundaries, is only used in glossed examples. For aesthetic and typographic reasons, in the running text I use the hyphen as marking a general morpheme boundary.} and 3) a dependent-marking genitive (phrasal) prefix  \d. In this I follow \citet{CohenNucleus, CohenZakho} in his description of \JZax, although his terminology is somewhat different. In the view advocated here, the separate existence of these morphemes can be observed in most \ili{NENA} dialects, albeit with differing levels of clarity. 

As distinct morphemes, one would expect their semantic-pragmatic (their \concept{signifié} or \concept{meaning} in a broad sense) to differ as well. This is indeed the case, but as this difference is very fine, I shall not use it as a decisive criteria. Instead, I shall establish their separate linguistic existence relying chiefly on their distributional properties. Accordingly, as written above, two shall be characterized as \concept{affix}\textsc{es} and one as a \concept{clitic}.

Needless to say, these terms are hotly debated in recent linguistic literature, and their relevance to typology, or linguistics in general, has been cast in doubt \citep[cf.\ most recently][]{HaspelmathClitics}. Therefore, in the first section of this chapter, I discuss my understanding of these terms, and why they should be relevant for this study. In the second section, I present the usage of these terms in a similar and highly relevant linguistic debate, namely the analysis of a \ili{Persian} AC marker, the so-called \ez*. This debate is interesting not only due to the similar analytical and methodological problems posed by this marker, but also due to the similar properties shared by the \ez* and the \ili{NENA} AC markers, bearing in mind the contact situation between the \ili{Iranic} languages using the \ez* (chiefly Kurdish and \ili{Persian}) and \ili{NENA} dialects. Finally, in the last two sections, I use the concepts discussed in the former sections to affirm the distinction between the three above-mentioned \ili{NENA} morphemes.

\largerpage
The claims put forward in this chapter bear a general character, claiming validity for all \ili{NENA} dialects. Yet, as these dialects are quite different (some could be considered to be different languages), it is clear that not all the details of my claims would be true for each and every dialect. Some dialects, moreover, are  not well enough documented at this stage in order to test these claims. To overcome this methodological difficulty, I establish the claims with examples of \Barw, possibly the best documented dialect so far \citep{KhanBarwar}, with sporadic examples from other dialects. Given that \citet{CohenNucleus} has put forward a similar analysis for \JZax, I hope in this way to show that the three morphemes are differentiable at least in principle across \ili{NENA}, opening the way for further research on this question.  Moreover, as I posit a clitic-affix continuum, I have a certain methodological leeway: The three morphemes may be diverging to different degrees in each dialect, yet still show the same general pattern. While in this chapter I take a bird's-eye view perspective, recall that in the subsequent chapters four dialects (\JZax, \Qar, \JSan  and \JUrm) are examined in further detail.

\section{Clitics, affixes, and phrasal affixes} \label{ss:clitics_affixes}

In this study, I use the notions \emph{clitics} and \emph{affixes} to designate different types of bound morphemes,  i.e.\ morphemes which cannot stand by themselves, but have to attach to other morphemes in order to form a self-standing phonological word, i.e.\ a stress-unit.\footnote{In the scope of this research, dealing exclusively with stress-accent languages, I equate the notion of \isi{phonological word} with the notion of stress unit: a sequence of morphemes carrying exactly one stress-accent. Cf.\ \citet[39]{GarbellUrmi}: \enquote{A minimal free form constituting a stress unit is a word.}} While these terms  are commonplace in current linguistic analysis, they are not always used alike  by different linguists, and thus I find it necessary to clarify my understanding of these terms in the context of this study.\footnote{Needless to say, I use the terms \emph{suffixes} and \emph{\isi{enclitic}s} to denote those bound morphemes which attach backwards, and \emph{prefixes} respectively \emph{\isi{proclitic}s} to denote those which attach forwards.} 

 In current linguistic thinking, the most common distinction between the two types is related to their syntactic status:  While affixes are considered to be building blocks of words, clitics have the syntactic and distributional properties of full-fledged words, but happen to be phonologically deficient in that they are typically devoid of stress.
  Thus, in a recent characterisation of clitics the framework of Canonical Typology, \citet[140]{SpencerLuisCanonical} write: \enquote{the canonical \isi{clitic} is an element which has the form of a canonical affix [i.e.\ it is phonologically bound] and the distribution of a canonical function word}. 

Indeed, the distinction between affixes and clitics lies to a large extent on a modular view of language, differentiating two sub-systems: One, \concept{morphology}, has to do with the \textit{make-up} of {words}, while the second, \concept{syntax}, has to do with \textit{arrangement} of words in discourse. In the last century, this distinction has received various formalisations (and at times denied altogether), but the core insight justifying this distinction remained the same: While words can combine together to form sentences in a seemingly limitless fashion, the building blocks of words  combine in a much more limited and regular fashion. In this view, affixes fall under the domain of morphology while clitics are treated syntactically. Thus, the bond between a host and an affix is stronger than the one holding with a \isi{clitic}.\footnote{This notion can be formalised as a transition probability: The probability of an affix following a prosodic (and syntactic) host should be higher than the probability of a \isi{clitic} following its prosodic host. In the scope of this study I leave this notion at an intuitive level.} A corollary of this is that clitics can attach phonologically to a certain word while forming a syntactic constituent with another word \citep[cf.][]{Klavans}.

\largerpage
Yet the keystone of this partition, the notion of the \concept{word}, or more precisely the morpho-syntactic word, has remained an elusive concept and defied any attempt to define it  cross-linguistically. This has recently led \citet{HaspelmathWord} to reject the validity of this notion and subsequently the dichotomy between morphology and syntax (see more references there). In a follow-up article, \citet{HaspelmathClitics} claims that the notion of \isi{clitic} cannot be defined consistently cross-linguistically, and therefore he suggests to dispense with it altogether.

While \citeauthor{HaspelmathClitics}'s argumentation is compelling, his suggestion to abolish this terminology is not helpful. He suggests to replace these terms with the terms \concept{plenimorph} and \concept{minimorph}. Not only would this add to the terminological inflation in linguistics, but these terms do not address the issue at hand: Plenimorphs are simply lexical bases, while minimorphs are functional morphemes, covering both affixes and clitics.
Indeed, it may be that the notion of \isi{clitic} is not useful in a large scale typological survey where clearly defined comparative concepts are needed. However, in a smaller study, covering typologically or genealogically related languages, these notions can be proven useful, as  they permit to discern between different levels of attachment of morphemes to their hosts, on the one hand, and distinguish distributional classes on the other hand. Moreover, in the field of \isi{language change} and \isi{grammaticalisation}, it is a well known fact that independent morphemes can become more and more bound to their hosts, finally becoming grammatical affixes. As this is a gradual process, it is convenient to use the label \concept{clitic} to designate intermediate stages of such processes.\footnote{A neat analysis of such a process, showing that the same source morpheme may be grammaticalised differently as a \isi{clitic} and as an affix, is given by \citet{LahiriBengali} treating the TAM system of Bengali.}

Thus, my suggestion is to understand the notions of \concept{affix} and \concept{clitic} not as absolute terms, but rather as relative terms, which can be fully appreciated only in the context of a study of a specific language, or a group of related languages. Hence, one can use different criteria proposed in the literature, such as those famously proposed by \citet{ZwickyPullum}, to gauge the proximity of a morpheme to its host, and not as categorical distinctions. Needless to say, \concept{morphology} and \concept{syntax} are not considered in such a view as distinct modules of language, but rather as \textit{end points} on a continuum which permit us to classify different types of linguistic signs.\footnote{In this respect, my approach is  similar to that of Canonical Typology, which defines grammatical notions in terms of \enquote{ideals} with possible deviations; see \citet{BrownChumakina}.} 

Of special importance for us is the \concept{selectivity criterion} proposed by \citet{ZwickyPullum}. Since inflection is normally associated with a specific word-class (nominal inflection, verbal inflection, etc.), morphemes which attach only to a specific word-class are typically \isi{affix}es, especially if they stand in grammatical opposition with other such morphemes, encoding different values of a specific grammatical feature associated with the stem.\footnote{In this respect, note that while \citet[511]{ZwickyPullum} observe that \enquote{special cliticization and inflection can look much alike}, they still keep the two notions strictly apart. In applying this logic they show that the \ili{English} ending \transc{n't} (as in \transc{can't, won't} etc.), which seemingly is a reduced \isi{clitic} form of the \ili{English} negator \transc{not}, is in fact not a \isi{clitic} but rather an inflectional affix of certain \ili{English} verbs. Thus,  \concept{inflection} in my terminology should be understood as word-internal (morphological) marking, which is typically realised by means of affixes, and is opposed to marking by clitics.} 
  From a cognitive point of view, this probably means that these morphemes are associated mentally with this word-class. From a \isi{language change} perspective, if they originated in free forms, it means that their usage has been maximally restricted to a sub-part of the lexicon, reaching the endpoint of a \isi{grammaticalisation} process. Bound morphemes which have a freer distribution, on the other hand, are typically \concept{clitic}\textsc{s}, as they conserve better the distributional freedom associated with phonologically independent forms, i.e.\ words. 

Of course, in the process of becoming an affix, an element may become more selective but still keep some features of free forms. In this study, one such important feature is \concept{phrasal scope}. An affix, attaching to a single base, has typically scope only over that base. Conversely, a free word, or a \isi{clitic}, can have  scope over a syntactic phrase. To designate the intermediate state of affairs, I shall use the term \concept{phrasal affix} for bound morphemes which are selective with respect to their hosts, but exhibit phrasal scope. This term can be conveniently contrasted with the term \concept{word-level affixes}, to denote the typical affixes. 

To conclude, in the realm of phonologically bound morphemes, I shall distinguish three types of morphemes, on the basis of the above criteria:\footnote{While these criteria are sufficient for concise definitory purposes, more criteria will be examined in the following discussions.}  


\begin{description}

\item[Clitics] Phonologically bound morphemes showing syntactic distribution and \linebreak scope of free morphemes (i.e.\ words).

\item[Phrasal affixes] Bound morphemes tied to the inflectional system of a specific grammatical category (part-of-speech), but exhibiting wide scope over \linebreak phrases.

\item[Word-level affixes] Bound morphemes tied to the inflectional system of a specific grammatical category, and having narrow scope over the inflected word alone.

\end{description}

This is schematically presented  in \vref{tb:continuum}.

\begin{table}[h!t]
\fittable{
\begin{tabular}{l cccc r}
\toprule
 			& \hspace*{0.5cm}	& Word affixes 	& Phrasal affixes 	& Clitics 			& \\
\multicolumn{1}{r}{More bound} 	& \multicolumn{4}{c}{$\xleftrightarrow{\hspace*{7.5cm}}$} & Less  \\
 Phon. dependent 	&	&	+	 		&		+			& 		+			& bound \\
 Phrasal Scope&	&	-			&		+			&		+			& \\
 Selective &	&	+			&		+			&		-			& \\
\bottomrule
\end{tabular}
}
\caption{The Affix--Clitic Continuum} \label{tb:continuum}
\end{table}

\section{The Persian Ezafe: Clitic or phrasal affix?} \label{ss:ezafe_dispute}

In several \ili{Iranic} languages, ACs are marked by a morpheme known as \ez* (the \ili{Persian} adaptation of the \ili{Arabic} term \foreign{\textarabic{إِضَافَة} {iḍāfa}}{annexation}), which is typically attached phonologically to the \prim. A detailed description of the usage patterns of the \ez* in Kurdish dialects is given in \cref{ch:Kurdish}. Here, I shall consider the case of the \Per \ez*, and briefly survey the controversy accompanying its grammatical analysis. A more detailed discussion can be found in \citet[\S 3.1]{HaigLinker}, as well as in the papers cited below.

In \ili{Persian}, the \ez* is a \transc{-(y)e} morpheme attaching to the \prim of an AC, whether the \prim is phrasal or not, as in the following illustrative example of \citet{SamvelianEzafe}:

\arabex[\Per]{Noun Phrase}{Noun}{1917}
{این کتابِ کهنه‌یِ بی ارزشِ مریم}
{in [[ketâb-e kohne]-ye bi arzeš]-e maryam}
{\dem{} book-\ez{} ancient-\ez{} without value-\ez{} M.}
{This ancient worthless book of Maryam.}
{\citep[606]{SamvelianEzafe}}

Analyses of the \ez* provide two competing accounts of the syntactic status of the \ez*. The basic question is whether the \ez* forms a morpho-syntactic constituent with the \prim (to which it attaches phonologically) or with the \secn. In the latter case, it must be a \isi{clitic}, since there is a mismatch between its phonological attachment and syntactical attachment, while in the former case it could be seen as an inflectional affix, encoding \cst*.\footnote{Of course, in contrast to the \ili{Semitic} \cst* the \ez* must mark nouns modified by adjectives as well. This difference has to do with the nature of adjectives in \ili{Persian}. In contrast to \ili{Semitic} adjectives which denote \textit{bearers of qualities} and thus embody within them an \isi{attributive relation} (see \sref{ss:Goldenberg_typology}), the \ili{Persian} adjectives denote \textit{qualities} alone and thus must stand in explicit \isi{attributive relation} with the noun referring to the bearer of the quality.} The earliest formulation of this debate, which I am aware of, is given by \citet[41]{Fairbanks}, who treats the \ez* as a preposition associated syntactically with the \secn, but gives the following remark in an endnote: 
\blockquote[{\cite[43, note 1]{Fairbanks}}]{In conversation with \name{Charles A.}{Ferguson} he has pointed out to me that the \textit{izafe} may be considered an inflection of the preceding noun, an inflection that would mark the noun as one that is determined. This is another indication of the tenuousness of the distinction between inflections and prepositions or postpositions. I would prefer to consider the immediate constituents of \textit{kitab e buzurg} [book \ez{} big] \transl{the big book} as \textit{kitab / e buzurg}, although \textit{e} is \isi{enclitic} to \textit{kitab}. This is the equivalent to considering \textit{I'll go} to have the immediate constituents \textit{I / ll go} although \textit{ll} is \isi{enclitic} to \textit{I.}}

These two competing analyses may affect our understanding of the \ez* as head-marking or dependent-marking (see \sref{ss:head_vs_dep}), if these concepts are understood as indicating syntactic association. Indeed, some authors  associate the \ez* with the notion of \concept{case}, which is typically understood as a dependent-marking device. \citet{Samiian1994}, working within X-bar theory, sees the \ez* as a \enquote{dummy case assigner}, while \citet{LarsonYamakido} treat it as a \gen* case-marker. In both accounts, the \ez* is syntactically associated with the \secn.\footnote{\citet{LarsonYamakido} discuss mostly data from Kurdish \ili{Zazaki}, but they apply their analysis to \ili{Persian} as well.} A similar position, in the framework of LFG, is advocated by \citet{BögelButt, BögelButtSulger}, who analyse the \ili{Urdu} \ez* (which is borrowed from \ili{Persian}) as forming a constituent with the \secn, notwithstanding its phonological attachment with the \prim.\footnote{To be exact, they allow for two possible analyses: X [\ez\ Y] or [X \ez\ Y] but they rule out the structure [X \ez] Y.}

It is not my intention to discuss these proposals in detail, but they raise an interesting methodological question: given that the \ez* in \ili{Persian} is \textit{always} phonologically attached to the \prim, why do these authors prefer to analyse it as a \isi{clitic} syntactically attached to the \secn, thus implying a phonology-syntax mismatch?\footnote{In \Kur Kurdish such an analysis may be more plausible, as an independent \ez* can appear without an overt \prim, thus forming a constituent with the \secn. The situation in Kurdish is discussed in more detail in \ref{ch:Kurdish}.}

One answer is that the \ez* is marked not on the head of the phrase it modifies, but rather phrase-finally. For example, in case of a conjoined phrase acting as a \prim, the \ez* appears only once phrase-finally:

\arabex[\Per]
{Cojoined NPs}{Noun}{1918}
{کلاهِ سفید و لباسِ  زردِ مریم}
{[kolâh-e sefid va [lebâs-e zard]]-e maryam}
{hat-\ez{} white and dress-\ez{} yellow-\ez{} M.}
{Maryam’s white hat and yellow dress}
{\citep[630]{SamvelianEzafe}}

Yet, as \citet[624]{SamvelianEzafe} notes, wide scope over coordination does not necessarily entail \isi{clitic} status. Indeed, the unique marking of conjoined nouns by one affix is a phenomenon known in the literature by the name of \concept{suspended affixation}, coined by \citet{LewisGrammar} in the context of \ili{Turkish} (for more recent studies, see \cite{KabakSuspended} or \cite{BroadwellSuspended}). The similar phenomenon where a suffix belonging to a head is marked phrase-finally on a complement is known in the literature under the name \concept{Suffixhäufung} \citep[50]{PlankIntro}.


Another, meta-linguistic reason may lie in the fact that the \textit{architecture} of formal grammars have been geared toward the syntax of major European languages, which are mostly dependent-marking. Therefore, they do not provide easy provisions for head-marking morphemes. Thus \citet{HaigLinker} comments: \enquote{all the proposals [...] are faced with the same dilemma: how to fit the Ezafe particle into a theoretical framework which provides no category that readily accommodates it.}\footnote{The difficulty to admit an analysis in which the \ez* is head-marking is explicitly stated by \citet[317]{BögelButt}: \enquote{This is problematic because the head is difficult to access for agreement purposes and it is also difficult to state a constraint that just when the XP+\textit{ezafe} is initial, the XP is restricted to be nominal (or a PP). The licensing of the modifying XP also becomes a matter of stating a long distance dependency between the \textit{ezafe} and the modifying constituent that must be propagated up and down through various levels of the tree.} Note that these are technical difficulties related to the grammatical framework, rather than conceptual difficulties.}

If, on the other hand, the notion of \isi{construct state} is seen as part of the \concept{state} grammatical category, as advocated in \sref{ss:state}, then the phonological attachment of the \ez* to the \prim can be
 easily accounted to in terms of head-marking state inflection.  Such an approach (using different terminology) is advanced by \citet{SamvelianEzafe, SamvelianHead}, who treats the \ez* as a \enquote{head-marking inflectional affix}.\footnote{See also \citet[11--17]{ThackstonKurmanji} who treats the \Kur \ez* as \enquote{\isi{construct case}}. The term \enquote{construct} is appropriate, but not so the term \enquote{\isi{case}}, which should be reserved for dependent-marking\is{dependent marking} morphology.}

In her view, the \ez* can be seen as part of nominal morphology in \ili{Persian}. With the exception of some lexically determined prepositions (which may be of nominal origin), it can only follow nominal elements, and it is barred from appearing after finite verbs. Moreover, as a nominal inflectional affix, it stands in opposition with other similar affixes, namely the indefinite marker \transc{-i}  and the so-called pronominal enclitics (which are in fact affixes, according to this view), i.e.\ they are in paradigmatic \isi{complementary distribution}.\footnote{A similar conclusion is reached by \citet[338f.]{Kahnemuyipour2003}, working within the Phrasal Phonology framework, who sees the \ez* as well as the other post-nominal morphemes discussed above as inflectional in nature, and thus prefers to call them \enquote{suffixes}.} All these affixes are special in that they do not attach to the head-noun of the phrase they modify, but rather phrase-finally, though always after a nominal element.\footnote{The pronominal enclitics can also occur after finite verbs to designate their complements, but in this position they exhibit different paradigmatic oppositions, and should arguably be treated differently. This is somewhat reminiscent of the situation in \ili{Semitic} languages, where the pronominal suffixes can appear as complements both after verbs and nouns.} Similarly, they all have wide scope over coordination.

Such an approach is preferable in my opinion, as it does not stipulate a mismatch between phonology and syntax, which, by the principle of scientific simplicity (\concept{Ockham's razor}) should only be called upon as a last resort. 


\section{The \d proclitic vs. the \ed suffix} \label{ss:d_vs_ed}
\largerpage
\subsection{Introduction}

\citet{CohenNucleus} is the first to clearly state the different syntactic status of the \d \isi{proclitic} and the \ed suffix. Cohen's analysis relies on the assumption that the \isi{proclitic} \d  retains the pronominal function of the Syriac pronominal \lnk* \d, in that it can head an NP.\footnote{Recall that unlike most pronouns which replace an entire NP, the pronominal \lnk* heads a complex NP. This is due to the fact that it replaces a \cst* noun, which must be followed by a complement. See also \sref{ss:Analytic_AC}.}
 This possibility is clearly illustrated in the following poetic \Qar example (=\example{551}):\footnote{In \JZax the regular form of the \lnk* is in fact \transc{dīd}. The form \d is almost exclusively restricted to clausal \secns, except in the context of possessive pronominal suffixes, where the two forms are allomorphs. See \sref{ss:JZax_Lnk}.}  

\acex[\Qar]
{d\textsubscript{\lnk}}{Noun}{551bis}
{kə-mzámri də\cb{} ʾurxàṯa.ˈ}
{\ind-sing.3\pl{} \lnk\cb{} roads}
{(The people) of the roads sing.}
{KhanQaraqosh}{279 {[Poetry 29]}}

In \Barw \d headed NPs are found in predicative positions:\footnote{This example is complicated by the fact that one would expect here a \gen* marking of the \dem* as well.  See \vref{ft:Barw_gen} for a possible solution.}

\acex[\Barw]
{d\textsubscript{\lnk}}{Determined Noun}{1368}
{lɛ́wət d\cb{} áwwa ʾàθṛaˈ}
{\neg.\cop-2\masc{} \lnk\cb{} \dem.\near.\masc{} country}
{You are not one of this country.}
{KhanBarwar}{112 {[A25:82]}}


Cohen argues that if  the \ed form is equivalent to the \d \isi{proclitic}, it must also be a \isi{pronominal linker} (a \concept{pronominal nucleus} in his terminology), representing pronominally the \prim to which it is attached.   He notes, however, that the suffixed form \ed attaches not only to nominal \prims but also to prepositional  \prims, as in the examples below.  Since a preposition is not a noun, it cannot be said to be represented pronominally by a pronominal element, and thus one must conclude that the \ed form is not a pronominal element, but rather a pure \prim marker (\concept{nucleus marker} in his terminology), different from the pronominal \isi{proclitic} \d.\footnote{See also the discussion following the \Syr \examples{996}{1120}.}  


\acex[\JZax]
{Preposition}{Noun}{346bis}
{ʾəmm-əd gōra}
{with-\cst{} man}
{with the man}
{CohenNucleus}{82}

\acex[\Barw]
{Preposition}{Noun}{1411}
{qám-ət gə̀ppa}
{before-\cst{} cave}
{before the cave}
{KhanBarwar}{442 {[A8:28]}}

\largerpage
Cohen's argument is convincing but it may be undermined by inverting it, claiming that a possible conclusion is rather that the \d \isi{proclitic} is a pure relational marker rather than a pronominal one (explaining away the cases where it heads an NP as a kind of \prim ellipsis). Alternatively, one may claim that the \ed  attaching to some prepositions is a lexicalised element, which should be kept apart from the \ed suffix following a nominal \prim. Instead of trying to refute such attacks, I shall substantiate Cohen's claim on different grounds, showing that the distribution of the \d morph is different from the \ed morph. In essence, I shall argue that while the \d morph is indeed a \isi{proclitic}, the \ed morph is better seen as a nominal suffix, since it stands in paradigmatic opposition to other nominal suffixes. In this claim, I reproduce the line of argumentation of \citet{SamvelianHead, SamvelianEzafe}, who argues that the \ili{Persian} and \Kur Ezafe morphemes should be analysed as inflectional affixes. 

It is important to stress that \ed and \d cannot be considered allomorphs: Their presence is  normally not conditioned by any grammatical or phonological factors, but is rather a deliberate choice of the speaker (reflecting some stylistic or pragmatic choice; see below \sref{ss:sem_diff}). For example, given the \Barw \prim \foreign{kθawa}{book} and \secn \foreign{qaša}{priest}, both \transc{kθaw-ət qaša} and \transc{kθawa t-qaša} are grammatical expressions meaning \transl{the book of the priest} \citep[488]{KhanBarwar}. On the other hand, the two morphs cannot be considered as free variants either (disregarding for the sake of argument the fine difference in function), as in some grammatical environments, detailed below, their distribution is different. 



Let us first consider the forms attaching to the \secn, namely \d and its variant forms \transc{də-\,\~ʾəd-\,\~t-}.\footnote{The exact form  depends on the dialect as well as some syllabic constraints, which are immaterial to the current discussion. For \Barw, \citet[396]{KhanBarwar} claims the unvoiced \transc{t-} form to be the basic one.} These elements are phonologically bound forms, as they normally lack stress, forming instead a  stress-unit together with  the first word of the \secn. In doing so, they show very low selectivity of their host (if any at all), and can attach to nouns, adjectives, verbs (being part of an attributive clause), the negator \transc{la} and various adverbs.\footnote{Note this is true only of dialects which make use of \d as a general \lnk*. Some dialects, such as \JZax, use it principally before clausal \secns, in which case it attaches normally to verbs. See \vref{tb:X_d-Y}.} In this respect, it seems reasonable to treat them as  proclitics. 

Similarly, these elements  show no sensitivity to the morpho-syntactic structure of the \prim, as long as it is nominal in nature (i.e.\ it has a nominal head). They can, moreover, be separated from the \prim  by intervening material, and stand in a separate intonation group. This is expected under the analysis of \d as a \isi{pronominal head}, separate from its antecedent (the \prim), and forming an NP together with the \secn. Given that a syntactic head is typically conceived of as  a syntactic word (i.e., not an inflectional element) this fact too establishes it as a \isi{proclitic}.  

What about the form \ed? At first sight, in accordance with the quotes given at the beginning of the chapter, one may consider \ed to be an \isi{enclitic} version of the same morpheme. Evidence for this includes the fact that \ed can appear phrase-finally, having wide scope over the entire phrase, such as in the following examples:\footnote{The dialects of \Arb, \JSul and \Qar exhibit also examples of phrase-final \ed marking, but only after NPs consisting of conjoined nouns (see \sref{ss:conj_crit}).}

\acex[\Barw]{Noun Phrase}{Noun}{1384}
{[ʾo\cb{} bɛ́θ-a zór]-ət yə̀mm-i.}
{\defi.\masc\cb{} house-\free.\sg{} small-\cst{} mother-\poss.1\sg}
{the small house of my mother}
{KhanGenitive}{76}

\acex[\JZax]{Noun Phrase}{Noun}{338bis}
{[gōra qamāy]-ət [d-anya baxt-āsa]}
{man first.\masc-\cst{} \gen-\dem.\pl{} woman-\pl}
{The first husband of these women}
{CohenZakho}{101 (49)}


A closer look, however, reveals some differences that distinguish it from the \d \isi{proclitic}, rendering it rather a \isi{phrasal suffix}. Following \citet{SamvelianEzafe}, I apply some of the different criteria of \citet{ZwickyPullum} to show that the \ed morph is rather inflectional in nature, while the \d morph is (as anticipated) a \isi{clitic}. 

\subsection{Distinguishing factors}

\subsubsection{Selectivity with respect to the host}

As stated above, the \d \isi{proclitic} is not selective at all. The \ed morph, on the other hand, is rather selective: With the exception of some lexically determined adverbials (mostly prepositions but also some conjunctions; see \examples{346bis}{1411}), it can only occur directly after nominals (nouns, pronouns and adjectives). Indeed, NP-finally, it can only attach to nominals.\footnote{Note that in principle an NP containing a \isi{relative clause} may end in a verb or an adverb, but the \ed morph does not attach to such NPs.}

\largerpage
\subsubsection{Arbitrary gaps}

The \d \isi{proclitic} does not show any arbitrary gaps in its distribution. The \ed morph, on the other hand, shows idiosyncratic gaps in its attachment to prepositions. While some prepositions require the \ed suffix before full nominal complements, other prepositions cannot occur with it. Yet another class of prepositions can co-occur optionally with the \ed suffix. Before pronominal complements, realized as possessive suffixes, the \ed suffix is typically absent.

The exact distribution of the \ed suffix with different prepositions is different from dialect to dialect. As an example, the distribution of \ed with \Barw prepositions is given in \vref{tb:Barw_prep} \citep[432--445]{KhanBarwar}.\footnote{I disregard cases of  \phonemic{d} segments appearing before demonstrative pronouns following prepositions, as this is considered to be the \gen* \d discussed in \sref{ss:d_gen}.} Note that the forms that do not take an \ed suffix generally lack stress, and pro-cliticize to their complement, while those that take the \ed suffix are phonologically independent.

\begin{table}
\centering
\begin{tabular}{l l}
\toprule

No suffix 					& Optional 							\\
\midrule
\foreign{ax-}{like}		& \foreign{bahs-/báhsət}{about}	 \\
\foreign{b-}{in, at}	& \foreign{bar-/báθər/báθrət}{about}	 \\
\foreign{bēn-/bēl-}{between} & \foreign{qam-/qámət}{before}	\\		
\foreign{gu-}{in}		& \foreign{xo-/xót}{under} 			\\ 
\foreign{hal-}{until}	& \\ 
\foreign{kəs-}{by}	& Mandatory \\ \cline{2-2}
\foreign{l-}{to} &  \\
\foreign{mən-/m-}{from} & \foreign{barqúlət/barqúlət}{opposite}  \\
\foreign{qa-}{for} &  \foreign{č̭ənnək̭ɛ́rət}{around} \\
\foreign{reš-}{upon} &  \foreign{qámθət}{in front of}\\
\foreign{t-la-}{without} [<\lnk+\neg] & \foreign{šáwpət}{instead}\\
\foreign{ṭla-/ta-}{for} & \\
\bottomrule
\end{tabular}
\caption{Distribution of \ed suffix with \Barw prepositions before a full nominal} \label{tb:Barw_prep}
\end{table}


\subsubsection{Morpho-phonological idiosyncrasies} \label{ss:morpo_phon_idio}

The \d \isi{proclitic} does not show any major morpho-phonological idiosyncrasies, although it presents variant forms which may be motivated phonologically. The same is true in general for the \ed suffix as well. In \Amd, however, there is one exceptional form discussed by \citet[71, fn.\ 27]{GreenblattAmidya}: the \Arab loanword \foreign{jamaʿa}{community} takes the \cst* form \transc{jamaʿa-t} rather than the expected \transc{*jamaʿ-əd}. Whether this is a true morphological idiosyncrasy replicating the \ili{Arabic} \cst* morphology or a phonetic artefact related to the presence of the pharyngeal \phonemic{ʿ} is hard to tell without further investigation. 



\subsubsection{Morphological paradigm} \label{ss:morph_paradigm}

The most important criterion according to \citet[627]{SamvelianEzafe} to distinguish an affix from a \isi{clitic} is the criterion of \concept{haplology}, devised by \citet{Miller} following \citet{ZwickySupressing}. In fact, this criterion boils down to showing that a morpheme stands paradigmatically \concept{in opposition} (i.e., in a paradigmatic \isi{complementary distribution}) with other morphemes, forming in essence a morphological paradigm, and thus revealing its affixal nature (see also \sref{ss:theoretical_framework}). Indeed, this is spelt out in greater clarity, from a structural perspective, in \citet{SamvelianHead}. These arguments are in essence repeated here for the case of \ili{NENA} dialects.

 The \ed morpheme does not simply attach to its host,   but rather stands \concept{in opposition} with a set of other nominal-final morphemes, most conspicuously the Aramaic nominal inflectional endings \transc{-a}\~\transc{-e}, but also the possessive pronominal suffixes.\footnote{The \ed morpheme stands also in opposition with  \isi{definite suffix} \transc{-ake}, borrowed from \Sor, in the dialects which have it, such as \JKoy or \JSul \parencites[62]{MutzafiKoySanjaq}[232]{KhanSulemaniyya}.}  Indeed, one never finds an \ed morph attaching to a noun or a preposition ending in a \isi{possessive suffix}, even if their scope is different. In such cases the \d \isi{proclitic} must be used, as stated explicitly by \citet[490]{KhanBarwar}, who gives the following example:
 
 \acex[\Barw]{Noun}{Noun}{1861}
 {jull-éy t\cb{} yáwne}
 {clothes-\poss.3\pl{} \lnk\cb{} doves}
 {their clothes of doves}
 {KhanBarwar}{490 {[A14:80]} }
 
  The nominal endings \transc{-a}\~\transc{-e} are clearly inflectional suffixes as they have single-word scope, as is the case with the possessive pronominal suffixes in most dialects.\footnote{Quite exceptional are the dialects of \JSan and \JSul, in which the possessive pronominal suffixes may be phrase-final, such as in the example \foreign{[ʾaxon-a ruw]-i}{my elder brother} (brother-\free{} big-\poss.1\sg) \parencites[251]{KhanSanandaj}[262]{KhanSulemaniyya}. This may very well be under influence of \Sor, which allows similar constructions, such as \foreign{[bira gewr]-ek-em}{brother big-\defi-\poss.1\sg} \citep[\War, ][81]{MacKenzie}.} 
  Since the \ed morph stands in opposition to these suffixes, it forms part of the same inflectional system. Thus, most naturally it should also be treated as an inflectional suffix.\footnote{One may argue that there is no principal restriction disallowing a \isi{clitic} to stand in opposition with an affix. I'm unaware of such analyses in the literature, but in any case, such an analysis goes against the very spirit of the notion of \isi{paradigmatic opposition}. In the structuralist tradition, which I follow here, the fact that two elements stand in opposition is an evidence that they share the same privilege of occurrence and thus the same grammatical status, for instance part-of-speech category. Thus, unless other reasons are invoked, one would naturally assume that affixes and clitics should appear in different paradigmatic slots.}
  
  It should be noted, that even if one maintains the \isi{enclitic} status of \ed, the fact that it cannot follow these suffixes differentiates it from the \isi{proclitic} \d, which has no such restriction. Formally, I oppose the licit construction [X-\poss\ \dlnk Y] to the illicit combination *[X-\poss\ed Y]. This alone should suffice in showing that these are two different linguistic units, which cannot merely by analysed as phonological variants of one and the same entity.\footnote{A similar claim is made by \citet[357]{SamvelianHead} to distinguish the \Kur suffixed \ez* from the independent (\enquote{demonstrative}) \ez* (see her examples (47) and (48)). In \ili{NENA} the situation is clearer than in \Kur, since the suffix \ed is typically added to a nominal stem, and not to a fully fledged noun.} Since the functional load of the two elements is different (as shall be examined below), I take this difference to be morphemic. 
  
 As a suffix, the \ed morph marks \cst*, while the \transc{-a\~-e} ending marks \free* combined with number.  This permits us to recognize three potential morphemic slots following the \ili{NENA} nominal stem (with optional marking of pronominal \secns): Gender + Number + State. The actual realisation of these slots, however, is subject to much fusion and idiosyncrasy (especially regarding the marking of gender), so that very often only one or two distinct suffixes are discernible on top of the stem. Some typical examples, reoccurring in many dialects, are given in \vref{tb:morphemic_slots}.\footnote{Note that the table is not meant to be exhaustive, given the richness of morphological patterns of \ili{NENA} nouns. }
 
 \begin{table}[ht!]
	\centering
 \begin{tabular}{l l l}
 \toprule
 Gender 							& Number & State [+ pron.] \\
 \midrule
 \zero\ (\textsc{m})				& \multicolumn{2}{c}{\textit{-a} (\sg.\free)} \\
 \zero\ (\textsc{m})				& \multicolumn{2}{c}{\textit{-e} (\pl.\free)} \\
 \multicolumn{2}{c}{\textit{-an} (\textsc{m.pl})}	& \textit{-e} (\free) \\
 \midrule
 \multicolumn{2}{c}{\textit{-t} (\textsc{f.sg})}	& \textit{-a} (\free) \\
 \multicolumn{2}{c}{\textit{-aṯ} (\fpl)}	& \textit{-a} (\free) \\
 \midrule 
 \zero\ (\textsc{m})				& \textsc{nil} & \textit{-əd} (\cst) \\
  \multicolumn{2}{c}{\textit{-an} (\textsc{m.pl})}	& \textit{-əd} (\cst) \\
 \midrule
 \multicolumn{2}{c}{\textit{-t} (\textsc{f.sg})}	& \textit{-əd} (\cst) \\
 \multicolumn{2}{c}{\textit{-aṯ} (\fpl)}	& \textit{-əd} (\cst) \\
 \midrule 
 \zero\ (\textsc{m})				& \textsc{nil} & \textit{-i} (\cst+\poss.1\sg) \\
 \bottomrule
		
		
 \end{tabular}
 \caption[Some typical \ili{NENA} nominals suffixes]{Some typical \ili{NENA} nominals suffixes (\textsc{nil}=neutralised feature)} \label{tb:morphemic_slots} 
 \end{table}



\subsubsection{The conjunction criterion} \label{ss:conj_crit}

\citet[630]{SamvelianEzafe} shows that the \ez* has wide scope over a conjoint NP (see \example{1918}). This is true also for the \ed suffix, as in the following examples:

\acex[\Barw]{Conjoined Nouns}{Noun}{699}
{[bab-a \cb{}w yəmm]-ət yala}
{father-\free{} \cb{}and mother-\cst{} child}
{the father and the mother of the boy}
{KhanBarwar}{488 {[A15:9]}}

\acex[\Qar]{Conjoined Nouns}{Noun}{535bis}
{[wánat\cb{} u toráθ]-əd Baġdèdə}
{sheep\cb{} and cows-\cst{} B.}
{the sheep and cows of Qaraqosh}
{KhanQaraqosh}{276 {[F:1]}}

As noted in the discussion of the \ili{Persian} example, such examples are characteristic of \isi{clitic} status. For example, \citet[134]{SpencerLuisCanonical} state that a \enquote{\isi{clitic} canonically takes wide scope over a coordinated phrase}. Yet, such cases are also accommodated by the analysis of \ed as a \isi{phrasal affix}.\footnote{Under this analysis, the \free* suffix \transc{-a} in \example{699} is analysed as a default nominal suffix, compatible with the phrase-final \cst* suffix.} Moreover, one may analyse such examples as  cases of \concept{suspended affixation}. If such an analysis is accepted, then suspended affixation manifests itself as an areal phenomenon, as it encompasses \ili{NENA}, \ili{Iranic} languages, and \ili{Turkish} (for which the term was coined). In another respect, the single \cst* marking conserves the  logic of classical \ili{Semitic} languages, in which it is generally not possible to conjoin directly two nouns  marked for \cst*, and instead alternative formulations are used.\footnote{Recall that in \MHeb such constructions are not found; see \example{MHeb_conj}. See also the highly unusual \Syr \example{1009}.}  Indeed, in such a view, the first noun \foreign{baba}{father} is in fact not under the morphological scope of the \ed suffix, but rather its relationship with \foreign{yala}{child} is inferred pragmatically.    

Yet, in contrast to the \ili{Persian} \ez*, the analysis of \ed as a suffix becomes truly clear in those dialects, \Barw and \JUrm, which show an alternative construction in which both  conjoined nouns are marked by the \ed suffix, though no explicit conjunction appears:\footnote{\Example{700} from \Barw is given by \citeauthor{KhanBarwar} as \transc{ʾaqlət iðə d-ay-baxta}, but the attachment of the \transc{d} segment to the following demonstrative can be seen as a product of syllabification of fast speech, since clearly the \isi{head noun} has a \cst* ending.} 

\acex[\Barw]{Conjoined Nouns}{Noun}{700}
{[ʾaql-ət ið-əd] ay baxta}
{leg-\cst{} hand-\cst{} \defi.\fem{} woman}
{the legs and hand of that woman}
{KhanBarwar}{488 {[A10:10]}}

\acex[\JUrm]{Conjoined Nouns}{Noun}{136bis}
{[id-əd reš-əd] [gor-aw]}
{hand-\cst{} head-\cst{} man-\poss.3\fem}
{the hands and head of her husband}
{GarbellUrmi}{86}

The possibility of repeating the \ed suffix on both nouns indicates that in these dialects the \ed marker has shifted, at least partially, from being a \isi{phrasal suffix} to a word-level suffix.\footnote{In \JUrm, in fact, one hardly finds any evidence of \ed being phrasal at all. Note, moreover, that the repetition of the \ed suffix on each conjoint is coupled, both in \Barw and in \JUrm, with the lack of an over coordination conjunction. The occurrence of this construction may very well be related to \ili{Turkish} influence, which allows the \isi{asyndetic conjunction} of possessed and non-possessed alliterative nouns. See the discussion in \ref{ss:JUrmi_CST} regarding \examples{136}{1923}.} This corroborates the idea that in all dialects \ed has acquired some affixal features, albeit in different degrees. Moreover, these dialects provide the clearest evidence for a distributional differentiation between the \d \isi{proclitic} and the \ed suffix, as there is no motivation for the \d \isi{proclitic} to be doubled in this environment. 



\subsubsection{Prosodic autonomy} \label{ss:prosody}

The syntactic autonomy of the \lnk* \d is corroborated by the prosodic structure, as it may stand at the beginning of a prosodic phrase, separate from the \prim. This fact can be observed in the grammatical descriptions of G. Khan, who indicates prosodic phrase boundaries by a small vertical line ˈ.

\acex[\Barw]{Noun}{Noun}
{1375}
{rìxaˈ t\cb{} xa\cb{} kàlləšˈ}
{smell \lnk\cb{} \indef\cb{} carrion}
{the scent of a carrion}
{KhanBarwar}{399 {[C8:5]}}


 The \ed suffix, on the hand cannot induce a prosodic break: It is bound to the \prim serving as its host, and in general is followed immediately by the \secn in the same prosodic phrase (just as \d is followed immediately by the \secn). Thus, aside from some exceptional cases,  the \ed suffix stands always in the middle of a prosodic phrase. 
 
 The difference in prosodic autonomy has implications for the semantics of the different forms (see next section), but bears also on their syntactic status, as mediated by language acquisition.
According to recent language acquisition theories, functional elements standing at the edges of prosodic phrases play a special role in acquiring syntactic structure, as they serve to \textit{tag} the prosodic phrase with a syntactic label \citep{christophe2008, GutmanBootstrapping}. Note that the \d \lnk* can serve in such a function, effectively becoming a functional head of its NP, while \ed cannot. This may explain the mechanism of \isi{language change} as well: The \isi{encliticization} of \d  as \ed leads to the loss of its role as tagging the category of its phrase in the process of language acquisition. It thereby loses its status as a syntactic head and is consequently reanalysed as an inflectional suffix. 
 
\subsubsection{Semantic differentiation} \label{ss:sem_diff}

The above criteria show that on distributional grounds the \d \isi{proclitic} and the \ed suffix should be treated as two separate morphemes. As such, one would expect each to have a different semantic load. The exact semantic difference between the two constructions, however, is difficult to pinpoint and is outside the scope of this work.\footnote{A similar question arises in the field of Syriac, where the exact usage conditions of the CSC, the ALC and the DAC are compared. See for example \textcites{Meyer}[Ch.\ 2]{WilliamsKings}[Ch.\ 4]{JoostenMatthew}.} In this respect, the comments of \citet[489ff.]{KhanBarwar}, the most detailed study of a \ili{NENA} dialect, may be illuminating (the emphasis is mine):

\blockquote{The structural difference between the \textit{kθawət qaša}  construction [=CSC] and the \textit{kθawa t-qaša} construction [=ALC] reflects different degrees of prosodic bonding
between the nouns. The first noun in the \textit{kθawa t-qaša}  construction is
prosodically more independent than the first noun in the \textit{kθawət qaša}
construction. [...] The \textit{kθawa t-qaša} construction is a \enquote{heavier} form of coding than the more
compact \textit{kθawət qaša} construction. This heavy coding is sometimes used
to give particular \textbf{salience} to a newly introduced referent that plays an
important role in the discourse [...] When these referents are mentioned subsequently in the discourse, they
are typically present with the lighter coding of the \textit{kθawət qaša} construction [...] The heavy coding of the \textit{kθawa t-qaša} construction may be used to give
\textbf{prominence} to the clause as a whole. [...] The looser prosodic connection between the two components in the \textit{kθawa t-qaša} construction is sometimes used as a device to give \textbf{prominence} to
the dependent noun rather than to the phrase as a whole.}

In essence, the \ed suffix creates a prosodically and pragmatically tighter bond between the \prim and the \secn, presenting them as one NP, while the \d \isi{proclitic} presents the two elements as two separate NPs mediated by the \isi{pronominal linker} \d, which stands in \isi{apposition} with the first NP.\footnote{A similar view is advanced by \name{Hans J.}{Polotsky} in his yet unpublished grammatical notes about \JZax (\name{Eran}{Cohen}, p.c.).} Formally, one can contrast [X\ed Y]\textsubscript{NP} with \mbox{X\textsubscript{NP\textsubscript{1}} $\leftrightarrow$ [\dlnk Y]\textsubscript{NP\textsubscript{2}}}.
 This, in turn, permits the speaker to assign some sort of pragmatic emphasis to one of the component NPs, both of them, or the clause as a whole. 














\section{The \d proclitic vs.\ the \d genitive prefix} \label{ss:d_gen}
\largerpage
\subsection{Introduction}

An important analytic discovery made by \citet{CohenNucleus} is the distinction between two separate \d shaped morphemes: One is the \isi{proclitic} \lnk* (\concept{pronominal nucleus} in Cohen's terms) and the other is a \d prefix marking \gen* case on a handful of  morphemes, mostly definite \isi{determiners} and demonstrative pronouns\footnote{This includes also the  interrogative determiner \foreign{ēma}{which}. See discussion in \vref{ft:d_ema}.} which begin with a \isi{glottal stop} (sometimes left untranscribed or unpronounced\footnote{Thus, these words can also be considered to be \concept{vowel-initial}. I leave open the question whether the initial \isi{glottal stop} is merely a phonetic artefact.}) in the non-genitive (or  unmarked) case, which is replaced\footnote{Note that in the description of \Diy  by \citet[93]{NapiorkowskaDiyana} the \d prefix precedes a \isi{glottal stop} in the genitive pronouns, rather than replacing it (e.g.\ \foreigngloss{d-ʾawwa}{\gen-\dem.3\masc}). The initial \isi{glottal stop} is in fact not a strict requirement: In \JZax \citet[109]{CohenZakho} lists also the distal demonstratives/\isi{determiners} \transc{wā(ha)} (\masc) and \transc{yā(ha)} (\fem) as having  genitive forms \transc{dwā(ha)} and \transc{dyā(ha)} respectively. In other dialects the distal forms are \transc{ʾawāha} and \transc{ʾayāha}, which may be the ancestral forms of the forms in \JZax.} by the \d in the \isi{genitive case} (\concept{attributive} marking in Cohen's terms\footnote{\citet[90]{CohenNucleus} identifies the same prefix also before the subordinated present \isi{copula}, to which the term \concept{genitive} seems inappropriate, yet the term \concept{attributive} may fit. For other dialects, however, the establishment of subordinate form of the \isi{copula} could not be made with certainty. On the other hand, the \d prefix is found sporadically before the deictic adverb \foreign{axxa}{here} in other dialects, maybe since it contains an implicit demonstrative \transl{in \textit{this} place}. \label{fn:d_gen_axxa}}).
 In other words, these morphemes (all of pronominal origin) inflect for case, as is illustrated in \vref{tb:jzax_gen_det} for the definite \isi{determiners}.


\begin{table}[ht!]
\centering
\begin{tabular}{c c}
\toprule
Case & Determiner \\
\midrule
\begin{tabular}{l r}
-\textsc{gen} & \textit{ʔ-} \\
+\gen		 & \d \\
\end{tabular} 
&
\begin{tabular}{l r}
\textit{aw} & \masc \\
\textit{ay} & \fem \\
\textit{an} & \pl \\
\end{tabular} \\
\bottomrule

\end{tabular}
\caption[Case-inflected definite \isi{determiners} in \JZax]{Case-inflected definite \isi{determiners} in \JZax  \citep[88]{CohenNucleus}} \label{tb:jzax_gen_det}
\end{table}


The main reason for establishing this category is that the \d prefix appears where a \lnk* is not expected, namely after \prims already marked for \cst* (by the \ed suffix or by \isi{apocope}), after prepositions (whether they are marked by \ed or not) and after the \lnk* itself.\footnote{Interestingly, also in \NMand possessive pronominal suffixes attach to a \d base after certain prepositions and \cst* loan-nouns. \citet{HaberlRelative} claims that this modern \d is not related to the \CMand \lnk* but arose rather from the metathesis of the final two root consonants of \CMand \foreign{qadmia}{to, for}. Yet, if the analysis of the \ili{NENA} \d \gen* prefix advocated here is correct, it may be tentatively suggested that the \NMand \d is the very same \gen* marker (with a more limited distribution), reflecting an areal phenomenon preceding the emergence of the modern Aramaic dialects. \label{ft:NMand_Gen}}

\largerpage
 These  possibilities are illustrated in the following \JZax and \Barw examples:

\acex[\JZax]
{Noun}{Noun}{297}
{brāt d-ay baxta}
{daughter.\cst{} \gen-\defi.\fem{} woman}
{the daughter of the woman}
{CohenZakho}{110}\antipar

\acex[\JZax]
{Preposition}{Noun Phrase}{354bis}
{mən [[d-o bəhna rwīxa] d\cb{} [d-o jwanqa]]} 
{from \gen-\defi.\masc{} breath(\masc) wide.\masc{} \lnk\cb{} \gen-\defi.\masc{} youngster}
{from the patience of this youngster}
{CohenZakho}{106 )(71)}\antipar

\newpage

\acex[\Barw]
{Noun}{Noun}{1380}
{ahwált-ət d-ò našaˈ}
{condition-\cst{} \gen-\defi.\masc{} man}
{the condition of that person}
{KhanBarwar}{399 {[B5:8]}}

\acex[\Barw]
{Preposition}{Noun}{1412}
{qám [d-áyya qə̀ṣṣət]}
{before \gen-\dem.\near.\fem{} story{} }
{before this story}
{KhanBarwar}{442 {[A16:5]}}

 Analysing the apparently spurious \d prefixes as \gen* case markers provides a clear justification for their appearance. Formally, we can distinguish between the patterns [X.\cst\ \dgen Y]\textsubscript{NP} and \mbox{X\textsubscript{NP\textsubscript{1}} $\leftrightarrow$ [\dlnk\  Y]\textsubscript{NP\textsubscript{2}}}, treating prepositions and the \lnk* itself as being functionally in \cst*. 
 
 
 It is worthwhile noting, that such an analysis implies a revolution in Aramaic grammar: while Proto-\ili{Semitic} is supposed to have case markers (on the evidence of \ili{Akkadian} and \CArab), Aramaic had lost all case marking by its earliest attestations (beginning of the first millennium BCE).\footnote{A notable exception is found in some Aramaic inscriptions from Sam'al dating from the 8\th\ century BCE, where \mpl\ nouns conserve an archaic distinction between nominative and oblique cases \citep[117]{Dion}. \label{ft:gen_Samal}} Thus, the \gen* \d represents a structural innovation in Aramaic.\footnote{Traditionally, the \lnk* \d itself is sometimes called \concept{nota genitivi}, but due to its pronominal status, it should not be assimilated with a genitive \isi{case marker}; see \citet[253, fn.\ 27]{GoldenbergSemitic}.} Note that the \d segment acting as a \gen* prefix is highly selective, since it attaches virtually only to demonstratives and definite \isi{determiners} (but see \vref{fn:d_gen_axxa}).

A similar distinction between the two \d morphemes is made in the native grammar of \citet[41f.]{Marogulov} treating \CUrm. He distinguishes between a pronominal \d (which he writes as a separate particle) and a prefix \d attached to demonstrative pronouns after prepositions, which in his terms has no function (\foreignquote{french}{le son \textit{d} n'ayant aucune fonction spéciale}):

\acex[\CUrm]
{Preposition}{Noun}{Marogulov}
{qə d-o nəşə}
{to \gen-\dem.\masc{} man}
{to this man}
{Marogulov}{41}

\newpage 
Note that in the \gen* function, the \d element cannot stand at the beginning of a prosodic phrase, unlike the \lnk* \d  (see discussion in \sref{ss:prosody} about the importance of the prosodic autonomy). 

Taking into account more dialectal data, further justification for their differentiation can be found, as presented below. 

\subsection{Distinguishing factors}

\subsubsection {Phonological shape}
In \Barw, the \d linker is normally realised as \transc{t-} \citep[396]{KhanBarwar}. If it attaches to a word beginning with a \isi{glottal stop}, it is normally realised as an unaspirated \transc{ṱ-}, which is phonetically expected.\footnote{This is also the case before the indicative \isi{copula} appearing in subordinate clauses.} 

\acex[\Barw]
{Noun}{Noun}{1374}
{tre\cb{} pə́nxe ṱ\cb{} ə́rxe}
{two\cb{} grinding\_stones \lnk\cb{} watermill }
{two stones of a watermill}
{KhanBarwar}{399 {[A24:13]}}


Appearing, however, before the demonstrative pronouns, normally beginning with a \isi{glottal stop}, it takes rather the voiced form \d, as in \examples{1380}{1412}. This is expected, if the \d morph does not attach before the \isi{glottal stop}, but rather replaces it, as in \vref{tb:jzax_gen_det}.\footnote{In other respects, the data from \Barw sometimes obscures the presence of a \gen* prefix as it marks the appearance of only one \transc{d} segment. Thus, one finds \foreign{čádra d-ò-malka}{the tent of that king} instead of the expected \transc{čádra d\cb d-ò-malka} and \foreign{ríxə d-o-xámra}{the smell of that wine} instead of \transc{ríxət d-o-xámra}. Especially in the second case, it seems plausible to assume that the actual forms are subject to a process of phonetic de-gemination (cf.\ \cite[122]{CohenZakho} who speaks of \enquote{phonetic simplification} in similar cases). \label{ft:Barw_gen}}

\subsubsection {Dialectal distribution} \label{ss:gen_dialectal}

Also in dialects where the \d \lnk* has disappeared as such one finds the \d prefix before \isi{determiners}. This is the case in \JSan, where one can find the \d prefix even after the \ez* suffix:\footnote{In contrast to \JZax, however, the usage of the \d \gen* prefix is optional in \JSan.}

\acex[\JSan]{Noun}{Noun}{4bis}
{fešár-e d-o màe}
{pressure-\ez{} \gen-\dem.\masc{} water}
{the pressure of the water}
{KhanSanandaj}{200 {[A:59]}}

The fact that the \d prefix is retained in such environments while the \d \lnk* in general has disappeared is easily explained if one considers the two to be separate morphemes, subject to different \isi{language change} processes. 

\subsubsection{Phrase-internal marking}

Similarly to the \cst* suffix \ed, which in most dialects is a \isi{phrasal suffix} (see \example{699}), the \gen* \d prefix in  \JZax must be analysed as phrasal prefix, 
judging by examples such as the following:


\acex[\JZax]{Noun}{Conjoined nouns}{check}
{ʾuz-lu xıṭbe dīd d-áw jwanqa ū\cb{} ʾáy xamsa}
{made-3\pl{} wedding \lnk{} \gen-\defi.\masc{} youngster and\cb{} \defi.\fem{} maiden}
{they made the wedding of that youngster and that maiden}
{CohenZakho}{304 (205)}

The analysis of \d as a \gen* prefix is justified by the fact that it follows  \lnk* \transc{dīd}, and cannot therefore assume a pronominal role. However, it marks only the initial determiner \transc{aw}, and not the subsequent determiner \transc{ay}. Thus, when \d is used as a genitive  inflectional morpheme of \isi{determiners}, it appears only as a phrase-initial marker, or, in other words, as a phrasal prefix.\footnote{\citet{CohenZakho} brings a case of phrase-internal \d marker attaching to the \isi{copula} (\example{298}). This \d is arguably different from the \gen* one, as it attaches to a quasi-verbal form, although Cohen subsumes both as markers of the \concept{attributive} function (see \vref{fn:d_gen_axxa}).}

In \Barw, on the other hand, one finds cases where the \gen* \d occurs phrase-internally. This happens in cases where a \secn noun is further modified by an adjective which is preceded by a determiner. In such cases, the internal determiner is marked as \gen*.

\acex[\Barw]{Noun}{Noun Phrase}{1409}
{gnáy-ət [táwra d-o\cb{} gòṛa]}
{fault-\cst{} \hspace{0.7ex}ox \gen-\defi.\masc\cb{} big.\masc}
{the fault of the big ox}
{KhanBarwar}{517 {[D2:19]}}

\largerpage
Adjectives in \Barw stand in \isi{apposition} to their head-noun. Therefore, the \d marker is not marking a dependency relationship between the adjective \transc{goṛa} and the noun it modifies (\foreign{tawra}{ox}). Rather, it is induced in virtue of the entire NP being in \gen* case due to its \isi{attributive relation} with the \prim noun \foreign{gnaya}{fault}. Similarly to the \ed suffix in \Barw, it seems that also the \d prefix has undergone a further development to become  a word-level marker, permitting it to appear phrase-internally.\footnote{Unfortunately I could not come across any examples of conjoint genitives mirroring exactly the \JZax example.} Note that there is no motivation for a \lnk* \d to appear in this position.

A similar example is found in \CArd, where the \gen* marking is induced by a prepositional head:

\acex[\CArd]{Preposition}{Noun Phrase}{1788}
{tla [ōǰax d-ay xēta]}
{for clan(\fem) \gen-\defi.\fem{} other.\fem}
{for the other clan}
{KrotkoffAradhin}{49 {[113]}}

\section{Conclusions}

In this chapter, I have shown that on various synchronic distributional grounds, one can distinguish between three different morphemes in \ili{NENA} related to the \il{Aramaic!Classical}Classical Aramaic \isi{proclitic} \lnk* \d. One of them is simply a retention of this morpheme in \ili{NENA}, conserving in essence its pronominal nature, while the two other morphemes are pure relational markers: the head-marking \cst* suffix \ed and the dependent-marking \gen* \isi{case marker} \d. 

The distinction between the three morphemes is primarily based on distributional reasons. The different environments where they can be found are summarised in \vref{tb:dist_3morph}.

\begin{table}[h!]
\centering
\begin{tabular}{l r}
\toprule
\Prim				& 	\Secn				\\
\midrule
N.\ stem-\transc{əd}\textsubscript{\cst}			&	\multirow{2}{*}{Y}		\\
Prep-\transc{əd}\textsubscript{\cst}				& 			\\
\midrule
X					& 	\dlnk	Y\\
\dlnk\				&			Y\\
\midrule
X.\cst\footnotemark				& 
\multirow{3}{*}{\begin{tabular}{r@{}} \dgen\textsc{det} Y \\ \dgen \dem \end{tabular}} \\
Prep.\opt{\cst}				& \\
\opt{X} \dlnk				& \\
\bottomrule
\end{tabular}
\caption[Distribution of D-markers]{Distribution of D-markers: a) \cst* suffix \ed; b) \lnk* \d; c) \gen* prefix \d (X and Y indicate any phrasal \prim or \secn).} \label{tb:dist_3morph}
\end{table}

\footnotetext{Recall that in this context the notation .\cst\ marks any type of \cst* marking, be it an \ed suffix, \isi{apocope} or an \ez* morpheme; see \vref{ft:cst_glossing}. Note also that prepositions need not be explicitly marked as \cst* in order to induce \isi{genitive case}.}

I have conceptualised this distinction in terms of the strength of the bond between these morphemes and their hosts, calling the two latter morphemes \isi{affix}es, while maintaining the \isi{clitic} nature of the original \lnk*. Their affixal nature stems from the high selectivity of the hosts: the \ed suffix appears almost exclusively on nominal hosts, while the \d prefix is restricted to \isi{determiners} and demonstratives (which are related historically). On the other hand, their \isi{clitic} origin is apparent in the fact that both these morphemes show phrasal placement: in general they  appear either phrase-finally (\ed) or phrase-initially (\d), yet some dialects show a further development in that they allow these markers to appear phrase-internally, as word-level inflectional suffixes. 




