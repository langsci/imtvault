\documentclass[output=paper]{LSP/langsci} 
\ChapterDOI{10.5281/zenodo.1291922}
 % \usepackage{makecell}

\title{Editorial}
% \subtitle{Special issue on technologies for a multilingual europe}

\author{Georg Rehm\and Felix Sasaki\and Daniel Stein\lastand Andreas Witt}
\abstract{\noabstract}
\maketitle

\begin{document}
\noindent{}The roots of this special issue of \emph{Translation: Computation,
  Corpora, Cognition}  
  go all the way back to 2011. At the end of
September of that year, the guest editors organised a workshop at the
Conference of the German Society for Computational Linguistics and
Language Technology (\textsc{gscl}), which took place in Hamburg. The topic of
the \textsc{gscl}~2011 conference -- ``Multilingual Resources and Multilingual
Applications'' -- had already set the stage for our pre-conference
workshop on September 27, 2011, which put special emphasis on
``Language Technology for a Multilingual Europe''.

Our intention behind this workshop was to bring together various groups
concerned with the umbrella topics of multilingualism and language
technology, especially multilingual technologies. This encompassed, on
the one hand, representatives from research and development in the
field of language technologies, and on the other hand users from
diverse areas such as, among others, industry, administration and
funding agencies. Two examples of language technologies that we
mentioned in the call for contributions were Machine Translation and
processing of texts from the humanities with methods drawn from language
technology, such as automatic topic indexing, and text mining, as well
as integrating numerous texts and additional information across
languages.

What these kinds of application areas and research and development in
language technology have in common is that they either rely --
critically -- on language resources (lexicons, corpora, grammars,
language models etc.) or produce these resources. A multilingual
Europe supported by language technology is only possible if an
adequate and interoperable infrastructure of resources (including the
related tooling) is available for all European and other important
languages. It is necessary that the aforementioned groups and other
communities of developers and users of language technology stand as a
single homogenous community. Only if all members of our (quite
heterogeneous and hitherto mostly fragmented) community stand together
and speak with one voice, it will be possible to assure the long-term
political acceptance of the ``Language Technology'' topic in Europe.

The Workshop ``Language Technology for a Multilingual Europe'' was
co-or\-ga\-nised by two \textsc{gscl} working groups (\emph{Text Technology} and
\emph{Machine Translation}) and \textsc{meta-net}
(\url{http://www.meta-net.eu}). \textsc{meta-net}, an \textsc{eu}-funded Network of Excellence,
is dedicated to building the technological foundations of a multilingual
European information society. To this end, \textsc{meta-net} is forging \textsc{meta}, the
Multilingual Europe Technology Alliance.

This special issue of \emph{Translation: Corpora, Computation,
  Cognition} includes the majority of the papers presented at the \textsc{gscl}~2011
Workshop ``Language Technology for a Multilingual Europe'', held at
the University of Hamburg on September 27, 2011, along with several
additional contributions.

The first article, ``Machine Translation -- Past, Present and
Future'', provides an overview of what must be considered the
essential core of multilingual technologies. Setting the stage, Daniel
Stein looks at the history of \textsc{mt} and discusses current approaches and
future perspectives. The backgrounds of the next two articles are two
interlinked \textsc{eu}-funded initiatives. Georg Rehm describes the Network of
Excellence \textsc{meta-net}, which consists of 60 research centres in 34
European countries, and its goal to build the technological
foundations of a multilingual Europe. He provides a summary of one of
the key outcomes of the initiative, ``The \textsc{meta-net} Strategic Research
Agenda for Language Technology in Europe''. In his article ``Metadata
for the Multilingual Web'', Felix Sasaki provides an overview of the
Internationalization Tag Set (\textsc{its}) which will become a \textsc{w\oldstylenums{3}c}
recommendation later in 2013. \textsc{its}~\oldstylenums{2.0} is one of the key results from
the European Union-funded Multilingual Web project.

The second part of this special issue contains six full research
papers. First is Uwe Reinke with a paper on the ``State of the Art
in Translation Memory Technology'', that focused upon technologies
applied by human translators. He takes a detailed look at major
concepts and recent trends in research and also in commercial Translation Memory (\textsc{tm})
systems, with an emphasis on integrating \textsc{mt} into \textsc{tm}, data exchange
formats, and approaches of improving the information retrieval
performance of \textsc{tm} systems. As a complement to the technologies used by
translators, Melanie Siegel examines ``Authoring Support for
Controlled Language and Machine Translation'', i.\,e., language
technologies that help and assist authors to produce high quality
documents. She concludes that it is necessary to combine methods
from authoring support and \textsc{mt} and to make them integrated tools in the
production and translation process. The paper ``Integration of Machine
Translation in On-line Multilingual Applications'' by Mirela-Stefania
Duma and Cristina Vertan takes a look at a difficult and challenging
problem that \textsc{mt}, especially statistical \textsc{mt}, is confronted with domain
adaptation. The method employed by the authors for this task is
language model interpolation, which produces good results even when
only sparse domain-specific training data is available. This, in turn,
is an advantage for less-resourced languages. The next article
concentrates on a specialised application that provides help for users
of monolingual or crosslingual search. In ``Disambiguate Yourself --
Supporting Users in Searching Documents with Query Disambiguation
Suggestions'', Ernesto William De Luca and Christian Scheel describe a
semantic approach and a corresponding architecture and prototype for
making more sense of queries as they are typed in by the user. The
penultimate article, ``Multilingual Knowledge in Aligned Wiktionary
and OmegaWiki for Translation Applications'', goes back to the topic
of \textsc{mt}. Michael Matuschek, Christian M.~Meyer, and Iryna Gurevych take
a look at multilingual lexical-semantic resources and their role in translingual technologies. They focus on two crowd-sourced
resources and present methods for aligning these resources in order to
combine them on the level of word senses, this way providing increased
coverage and improved interoperability. In the final article, Igor
Leturia and colleagues present ``The BerbaTek project for Basque:
Promoting a less-resourced language via language technology for
translation, content management and learning''. In this joint project
between companies and research centres, the partners developed several
technologies for the Basque language which is, as the \textsc{meta-net} study
``Europe's Languages in the Digital Age'' pointed out, among the 21
European languages in danger of digital extinction.

Since we held the workshop, there have been quite a few very positive
developments in the area of multilingual language technologies
\emph{from} Europe \emph{for} Europe. Among those developments are a
new series of projects funded by the European Commission such as, for
example, QTLaunchPad, or additional projects around the open source
machine translation system Moses. In addition, \textsc{meta-net} organised its
third \textsc{meta-forum} conference in June 2012, which was attended by more
than 250 participants from the domains of research, industry, administration, and
politics. Important milestones for the work of \textsc{meta-net} were the
publication of the \emph{\textsc{meta-net} Language White Papers} (September
2012) and the \emph{\textsc{meta-net} Strategic Research Agenda for
  Multilingual Europe 2020} (January 2013). While the first funded
phase of the initiative came to an end on January 31, 2013, there will
be a fourth \textsc{meta-forum} conference later this year
(\url{http://www.meta-forum.eu}). Among the topics of \textsc{meta-forum} 2013
are upcoming opportunities for multilingual technologies in the
frameworks of Connecting Europe Facility (\textsc{cef}) and Horizon
2020. On February 29, 2012, the Common Language Resources
and Technology Infrastructure (\textsc{clarin}) received \textsc{eu-eric} legal
status, as the second
European Research Infrastructure overall. These initiatives clearly demonstrate the emphasis the
European Commission lays on the further development of language
resources and language technology.

This special issue would not have been possible without the help of
several colleagues. First of all, the guest editors would like to
thank all authors who contributed articles to this special issue and
those who presented papers at the workshop back in September 2011. We
would like to thank the reviewers who provided valuable and helpful
feedback to all authors. Many thanks are also due to our colleague
Sarah Weichert (\textsc{dfki}) who supported us in a critical phase during the
preparation of this special issue. Finally, we would like to express
our gratitude towards the editors of \emph{Translation: Computation,
  Corpora, Cognition}, especially Oliver Čulo, who not only made it
possible that we could publish the results of our workshop in this
journal but also constantly supported us whenever necessary.\\\\
 
\noindent Georg Rehm, Felix Sasaki, Daniel Stein, Andreas Witt \hfill
June 21, 2013


\printbibliography[heading=subbibliography,notkeyword=this]

\end{document}
 
