\documentclass[output=paper,
modfonts
]{langscibook} 
\bibliography{localbibliography}


\input{localpackages.tex} 

%Copy this to localcommands.tex

\usepackage[english]{babel}
\usepackage{amsmath}
\usepackage{amssymb,amsfonts,textcomp}
\usepackage{array}
\usepackage{hhline}
\usepackage{hyperref}

\newenvironment{styleStandard}{}{}
\newenvironment{stylelsAbstract}{}{}
\newenvironment{stylelsSectioni}{}{}
\newenvironment{stylelsSectionii}{}{}
\newenvironment{stylelsBulletList}{}{}
\newenvironment{styleBibliographyi}{}{}
\newenvironment{listWWNumxxvleveli}{}{}
\newenvironment{listWWNumxxvlevelii}{}{}
\newenvironment{listWWNumxxvleveliii}{}{}
\newenvironment{listWWNumxxvleveliv}{}{}
\newenvironment{listWWNumixleveli}{}{}
\newenvironment{listWWNumixlevelii}{}{}
\newenvironment{listWWNumixleveliii}{}{}
\newenvironment{listWWNumixleveliv}{}{}

\newcommand\textstyleListLabelxvi[1]{#1}
\newcommand\labellistWWNumxxvleveli{\thelistWWNumxxvleveli.}
\newcommand\labellistWWNumxxvlevelii{\thelistWWNumxxvlevelii.}
\newcommand\labellistWWNumxxvleveliii{\thelistWWNumxxvleveliii.}
\newcommand\labellistWWNumxxvleveliv{\thelistWWNumxxvleveliv.}
\newcommand\labellistWWNumixleveli{[F0B7?]}
\newcommand\labellistWWNumixlevelii{\textstyleListLabelxvi{o}}
\newcommand\labellistWWNumixleveliii{[F0A7?]}
\newcommand\labellistWWNumixleveliv{[F0B7?]}

\newcounter{listWWNumxxvleveli}
\newcounter{listWWNumxxvlevelii}[listWWNumxxvleveli]
\newcounter{listWWNumxxvleveliii}[listWWNumxxvlevelii]
\newcounter{listWWNumxxvleveliv}[listWWNumxxvleveliii]
\newcounter{itemize} 

\title{Linguistic Accommodation}  

\author{%
 Hanna Ruch\affiliation{University of Zurich}\lastand 
 Carlota de Benito Moreno\affiliation{University of Zurich}
}

% \chapterDOI{} %will be filled in at production
% \epigram{}

\abstract{
Abstract goes here
}

\begin{document}
\maketitle

\section{Introduction} \label{Section_Introduction}
% * <rikvangijn@gmail.com> 2018-06-27T19:02:21.757Z:
% 
% General: I think it is really good. The right tone, mostly the right level of elaboration. I can only hope that the other contributions are as good as yours. There were a few typos/grammar errors, and I have taken the liberty to change them in the text. Further comments you'll find in the text (best use "rich text" to properly see them). Thanks for your excellent work!
% 
% 
% ^ <max.wahlstrom@helsinki.fi> 2018-06-28T13:45:47.022Z:
% 
% I agree! Really good work! My minor interventions can be tracked by comparing the current version with the one saved as Rik's comments 18-06-27 in History & Revisions. One general comment: Rik is a fan of tables, numbered lists, and subsectioning (and I'm a fan of his tables, lists, and subsections). At times, I was left wondering, whether adding subsections might be helpful in bringing organization to some of the lengthier sections, and, more importantly, since the argumentation nevertheless runs smoothly, add some visual clarity. I'll point these out below. Also, a three-point list could maybe be done with your proposed model...
% 
% ^ <hanna.ruch@uzh.ch> 2018-08-03T14:40:44.079Z.

We define linguistic accommodation as the adjustments speakers make to become linguistically more (convergence) or less (divergence) similar to an interlocutor or to a social environment. When occurring in a single interaction or experiment over minutes or hours, we will refer to these adjustments as \textit{short-term accommodation}. \textit{Long-term accommodation} will be used when accommodation takes place over months or weeks, for instance, after a speaker has moved to a new region or moved in with a new flatmate.\footnote{Some authors instead use the term "second dialect acquisition" \citep{siegel_second_2010} and prefer this term over long-term accommodation as it denotes permanent changes \citep{chambers_dialect_1992}. In this chapter, we will nonetheless refer to long-term accommodation as one of our main aims is a comparison between long- and short-term accommodation.} %As to linguistic background, most studies on accommodation have dealt with situations where both speakers have the same mother tongue and regional background. 
% * <rivka.eisner@uzh.ch> 2018-09-17T11:18:36.978Z:
% 
% > second dialect acquisition
% If this is Siegel's term, then probably best to use quotes and indicate a page number in parentheses. 
% 
% ^ <cdebenitomoreno@gmail.com> 2018-10-31T15:40:15.024Z.
% * <rikvangijn@gmail.com> 2018-06-26T08:32:07.070Z:
% 
% Is there any reason for this that you might be able to mention here?
% 
% ^ <max.wahlstrom@helsinki.fi> 2018-06-28T13:49:43.650Z:
% 
% What came to my mind in this connection is that with different varieties, some types of L2 research, interference studies etc., may address similar questions. Not sure if this is worth mentioning though. 
% 
% ^ <hanna.ruch@uzh.ch> 2018-07-09T09:56:34.863Z:
% 
% Hm, now I am not so convinced anymore by this statement. It is true for recent work, but not when considering early work within the CAT framework. So I commented this out. (In the case of the recent studies, where it is true, I think the reason is availability of speakers...).
%
% ^ <hanna.ruch@uzh.ch> 2018-08-03T14:40:55.218Z.

As we will see below, accommodation can be observed at all linguistic levels and can involve the adoption of single elements such as lexical items \citep{brennan_conceptual_1996}, but also more subtle shifts, for instance, a change in speech rate \citep{putman_conception_1984} or degree of regional accent \citep{bourhis_language_1977}. Accommodation can further be observed as categorical switches from one language to another in bilingual speakers \citep{giles_towards_1973}, and is therefore related to language choice and code-switching. %Add CrossReference  Code-Switching chapter.
Given the focus of this book, this chapter primarily discusses situations involving speakers of different dialects or languages. Research dealing with interlocutors from the same region will nonetheless be included to shed light on the role of linguistic and extralinguistic factors.
% * <rikvangijn@gmail.com> 2018-06-26T08:37:30.758Z:
% 
% > code-switching
% Remember to add a cross-reference here. More generally (note to editors): how will we deal with CRs within the book, do we add them to the reference list? Or do we use some alternative CR system (direct links would be good in any case).
% 
% ^ <max.wahlstrom@helsinki.fi> 2018-06-28T13:56:34.162Z:
% 
% Good question. Adding them to references is not my preferred way, as that might be a bit weird even in traditional symposia. I suggest we use plain text references à la "see Chapter N, Code Switching in this volume". We can add a common, easy-to-find place holder as a comment, e.g. %CrossReference. Eventual linking must (and will) be done at a late stage of the typesetting.
%
% ^ <hanna.ruch@uzh.ch> 2018-07-09T09:59:34.475Z:
% 
% I added %CrossReference so we don't forget.
%
% ^.
% * <rikvangijn@gmail.com> 2018-06-26T08:35:46.730Z:
% 
% > talkers
% Is that a commonly used term in accomodation research? Rather than e.g. speakers or interlocutors?
% 
% ^ <hanna.ruch@uzh.ch> 2018-07-09T10:04:24.550Z:
% 
% I think it is mainly used in phonetic and psycholinguistic work ("model talker" is very common and more often used than "model speaker"). I have seen that Hanna L. doesn't use this term, so I changed it here as well.
% 
% ^ <hanna.ruch@uzh.ch> 2018-07-09T10:12:17.812Z:
% 
% Oops, I just closed the comments on BrE vs AmE spelling, sorry! My preference is BrE (but I don't think it really matters as long as we're consistent throughout the book...).
% 
% ^ <hanna.ruch@uzh.ch> 2018-08-03T14:41:10.713Z.

Accommodation can also involve non-verbal communication and other kinds of social behavior \citep{hall_behavioural_2013, dijksterhuis_perception-behavior_2001}. For this reason, the phenomenon has been studied not only in linguistics, but in a range of other disciplines too. The focus of early research on dyadic communication in psychology primarily fell on whether speakers converge, for the purposes of understanding interview dynamics \citep[e.g.][]{matarazzo_interviewer_1963} or investigating the relationship between personality and imitative behavior \citep[e.g.][]{natale_convergence_1975}. In the 1970s, accommodative processes came to the attention of social psychologists, who also investigated the role of language and accent in person perception and inter-group processes. %Based on this early research, Speech Accommodation Theory \citep{giles_accent_1973,giles_towards_1973,giles_speech_1975} was developed, which attributed accommodation primarily a social function. Originally applied to speech, the theory was later refined to explain both verbal and non-verbal communicative behavior \citep[Communication Accommodation Theory, CAT;][]{giles_accommodation_1991}. CAT sees convergence and divergence as speakers' communicative strategies to express social distance in an interaction \citep[293]{giles_communication_2007}.
% * <rivka.eisner@uzh.ch> 2018-09-20T12:46:03.512Z:
% 
% > came to the fore of social psychologists
% Suggested change:  replace with either "came to the attention of social psychologists" or "came to the fore of [or, perhaps better, use "in"] social psychology" (if the latter is chosen, the rest of the sentence would need to be adapted to fit with the change to "social psychology" rather than its practitioners).
% 
% ^ <cdebenitomoreno@gmail.com> 2018-10-31T15:41:19.296Z.
Given its main interest in social and inter-group processes, the focus of social psychology fell on \textit{whether} speakers accommodated and to what extent these changes were perceptible to listeners.
% * <max.wahlstrom@helsinki.fi> 2018-06-28T13:57:26.098Z:
% 
% Ok, using \citep works, we've used \parencite. This is mainly a note to myself lest I forget to check in the end that these compile the same way.
% 
% ^ <hanna.ruch@uzh.ch> 2018-08-03T14:41:24.627Z.

For linguistics, however, it is crucial to understand \textit{what} linguistic features are subject to the process of accommodation, an issue that was soon taken up by sociolinguists, with \cite{coupland_accommodation_1984} being the first to concentrate on specific linguistic variables. Most of this early sociolinguistic work \citep[e.g.][]{rickford_adressee-_1994,coupland_accommodation_1984,bell_language_1984,selting_levels_1985} aims at understanding style-shifting. \cite{trudgill_dialects_1986} is probably the first to apply Communication Accommodation Theory (CAT) to dialect contact and dialect change. He formulates the idea that long-term changes in linguistic behavior (i.e. long-term accommodation) are based on repeated short-term accommodation, and further suggests that accommodation between speakers underlies linguistic change at the community level (see below). \cite{niedzielski_linguistic_1996} propose several ways in which CAT could inform our understanding of language contact phenomena, and encourage linguists to integrate CAT into their research.
% * <rivka.eisner@uzh.ch> 2018-09-17T12:04:32.608Z:
% 
% > CAT
% The first time one uses an acronym, write out the full name and include abbreviation in parentheses.  Subsequently, just the acronym can be used.  Here this would be: "the first to apply 'communication accommodation theory' (CAT) to dialect contact."
% Probably also good to include a page number here for the citation if Trudgill coined the term?
% 
% ^ <cdebenitomoreno@gmail.com> 2018-10-31T15:42:50.682Z.
% * <rivka.eisner@uzh.ch> 2018-09-17T11:51:27.403Z:
% 
% > e.g.][]{rickford_adressee-_1994,coupland_accommodation_1984,bell_language_1984,selting_levels_1985} 
% This note is for this specific location and also throughout the text.  You will want to decide if lists of parenthetical citations will be ordered by 1) ascending date of publication, 2) descending date of publication, or 3) alphabetically.  If you are following a specific style guide, refer to their instructions for guidance. If not, then decide on one choice and make sure it is consistently used throughout the text.  As an example, for Harvard author-date style, one lists in alphabetical order (and with commas after the author): (e.g. Bell, 1984; N. Coupland, 1984; Rickford & McNair-Knox, 1994; Selting, 1985).  
% 
% ^ <cdebenitomoreno@gmail.com> 2018-10-31T15:43:23.538Z.
% * <rikvangijn@gmail.com> 2018-06-26T08:44:52.900Z:
% 
% > behaviour
% Note to editors: decide on British vs. American spelling.
% 
% ^ <max.wahlstrom@helsinki.fi> 2018-06-28T14:05:48.794Z:
% 
% Yep. Note also leveling vs. levelling, generalization vs. generalisation, favour, also, here quotation marks follow AmE. etc. My preference is AmE, but on the other hand Eleanor promised to revise the text (although I remember her saying that she doesn't mind AmE).
% 
% ^ <hanna.ruch@uzh.ch> 2018-07-09T10:10:20.302Z.

Apart from sociolinguistics, accommodation has been examined in other fields of linguistics too. In recent years, the phenomenon has been extensively studied in cognitive psychology and psycholinguistics. This work has often used more controlled laboratory settings, and the research aims are mainly oriented toward understanding the \textit{mechanisms} rather than the social \textit{functions} of convergence, also referred to as \textit{alignment} or \textit{entrainment}. %In these fields, the most influential model is the Interactive Alignment Model \citep[IAM;][]{pickering_toward_2004}. IAM sees convergence as an automatic process which results from a link between speech perception and speech production, similar to the priming mechanism, which is constantly activated during speech processing \citep{pickering_toward_2004}. In some cases, it is difficult to separate accommodation from priming. We follow \cite{pickering_toward_2004} who see priming as the underlying mechanism of accommodation, whereas accommodation is the process of mutual linguistic adjustments in its communicative context.
Although many studies on short-term accommodation take place in socially impoverished settings, these experiments have much to say about the linguistic and cognitive factors favoring or inhibiting accommodation. %The mechanism and functions of accommodation can be analysed separately, as suggested by \cite{ruch_function_2018}. 
% * <rikvangijn@gmail.com> 2018-06-26T08:52:08.850Z:
% 
% > underlying
% Head noun missing
% 
% ^ <hanna.ruch@uzh.ch> 2018-07-09T10:14:18.421Z:
% 
% Added "mechanism"
%
% ^ <hanna.ruch@uzh.ch> 2018-08-03T14:41:28.758Z.
In interactional linguistics, in contrast, speakers' mutual adjustments are of interest to understand discourse structure and dynamics as well as the construction of identity through language \citep[e.g.][]{chakrani_arabic_2015,nilsson_dialect_2015}, while structural patterns are only of secondary interest.
%Instead of structural patterns or quantitative differences, qualitative patterns, for instance, the repetition (i.e. recycling) of words (Beispiele) are given analytical preference. For the analysis of structural linguistic patterns, however, quantitative analysis rather than qualitative research is of primary interest. For this reason, we focus on the former.

More recently, accommodation has also been studied in applied linguistics and psychology. For instance, accommodation has been used to assess communication quality in health communication (...), %add in examples/refs
% * <rivka.eisner@uzh.ch> 2018-09-21T12:23:36.157Z:
% 
% >  (...), 
% I assume this is an area where you will be adding a citation?
% 
% ^.
and it is analyzed to improve human-machine interaction  \citep[e.g.][]{linnemann_as_2016}. This work mainly uses holistic or listener-based approaches to quantify accommodation, and does not usually analyze what specific linguistic features speakers accommodate to. Over the last years, finally, written computer-mediated communication has also been examined with regard to accommodation \citep[e.g.][]{Danescu-Niculescu-Mizil2011}.
% * <max.wahlstrom@helsinki.fi> 2018-06-28T14:19:37.042Z:
% 
% > health communication (...)
% Just a place holder so you won't forget to sort this out.
% 
% ^.

As outlined above, the idea of relating accommodation research to the study of language contact and change is not new. So far, however, there has not been nearly enough empirical research on this issue. The present chapter reviews empirical research on linguistic accommodation, focusing on those aspects which are relevant to the study of language and dialect contact. Emphasis will therefore be put on studies involving speakers from different dialects or languages. We will start by discussing theoretical models on the relationship between accommodation and contact-induced language change. We will then review the existing literature on accommodation to address the following questions: First, what are the linguistic patterns resulting from short-term and long-term accommodation? Second, what linguistic and extralinguistic factors favor or mitigate accommodation processes? And third, to what extent are these patterns compatible with the idea that contact-induced language change is initiated in individual interactions? We will conclude by proposing directions for future research and by elaborating how accommodation research could further inform our understanding of deep time and societal language contact phenomena.
% * <rivka.eisner@uzh.ch> 2018-09-17T12:13:35.704Z:
% 
% > hardly been
% This would sound more complete if written "hardly been any empirical research" or, if that is too broad/bold a claim, one could say "there has not been nearly enough empirical research."
% 
% ^ <cdebenitomoreno@gmail.com> 2018-10-31T15:45:28.663Z.
% * <rikvangijn@gmail.com> 2018-06-26T09:03:29.485Z:
% 
% In our chapter on linguistic areas, we have a different structure: introduction includes a subsection on terms and concepts, and the relation of the chapter topic to contact linguistics forms part of the intro. Then a section on approaches, and then the processes and patterns. We should discuss what is the best way (maybe also taking into account what Hanna Lantto and Danae have been doing). I have to say the, in our chapter, the terms and concepts section does not really add a lot, so from our point of view, I would be tending towards striking it and instead explain terms and concepts when they become relevant. Up for discussion...
% 
% ^ <max.wahlstrom@helsinki.fi> 2018-06-28T14:30:42.306Z:
% 
% Hanna L. (who is ready with her text) uses the same as here, second section with the heading: "Relevance of the topic for other fields". < this should, btw, be changed to Code Switching and contact-induced language change / the study of language contacts. The easiest solution for this would be for us (Areas) to promote 1.3 into 2.
% On the second point: I'd like to keep our terminological discussion, we also address the history of the field in the Intro. Another, section-level discrepancy comes with methodology that we (I & Rik) don't address by that title, but within "Approaches". This is also what Hanna L. does. For you, I think approaches could work as well, without virtually any changes to the content.
% 
% ^ <hanna.ruch@uzh.ch> 2018-07-09T10:17:33.927Z:
% 
% I don't have much to add on this - I would wait until the other chapters are ready and the look for the best way how to do it across chapters.
%
% ^.

\subsection{Linguistic accommodation and contact-induced language change}
% * <max.wahlstrom@helsinki.fi> 2018-06-29T14:08:20.618Z:
% 
% In this section, I think the structure would benefit from subsectioning, especially highlighting your own, original contribution. For instance: 2.1 From individual to community – from short term to long term 2.2 The receiving community 2.3 Toward an improved model of...
% 
% ^ <hanna.ruch@uzh.ch> 2018-07-09T10:47:12.452Z:
% 
% Thanks for this suggestion; that makes sense to me! I added three subsections:
% 2.1 From the individual to the community
% 2.2 From short-term to long-term
% 2.3 Toward an improved change-by-accommodation model
%
% ^ <hanna.ruch@uzh.ch> 2018-08-03T14:41:44.835Z.
Linguistic accommodation has been a crucial element of language change models. In what follows, we review two of the most relevant proposals of said models, namely, how individual change turns into social change and how short-term change becomes lasting change. Last we call attention into several other points where the role of accommodation in language change must be studied.
\subsubsection*{From the individual to the community}

It has long been assumed that linguistic innovations spread via face-to-face contact between individual speakers. As early as the 1930s, \citet[476-477]{bloomfield_language_1933} describes how individual speech habits are shaped by those who the speaker has interacted with before. He also postulates that persons with power and prestige are imitated to a greater extent than less socially influential individuals, and that imitators themselves will become models in later interactions. He further formulates the assumption that, with a few exceptions, ``the process does not rise to the level of discussion,'' and that not all linguistic forms are equally likely to be imitated: ``The adjustments are largely minute and consist in the favoring of speech-forms more often than in the adoption of wholly new ones. A great deal of adjustment probably concerns non-distinctive variants of sound'' \citep[476-477]{bloomfield_language_1933}. Bloomfield thus already describes the general principles of what will later be determined linguistic accommodation, and relates the phenomenon to dialect leveling and linguistic change. 
% * <rivka.eisner@uzh.ch> 2018-09-17T12:40:51.693Z:
% 
% > 1933
% 
% For this sentence, how about: "As early as the 1930s, Bloomfield (1933) describes how individual speech habits are shaped by those with whom an individual has interacted with before."
% 
% ^ <cdebenitomoreno@gmail.com> 2018-10-31T15:46:14.673Z.
% * <rivka.eisner@uzh.ch> 2018-09-17T12:32:44.964Z:
% 
% > .'
% Include page citation here since you are directly quoting. Probably it is "(476-477)"? 
% 
% It seems like you may not need the earlier citational reference to Bloomfield (e.g. the second one in the paragraph) because it is clear you are still talking about the same author and you are not directly quoting there. 
% 
% ^ <cdebenitomoreno@gmail.com> 2018-10-31T15:48:49.007Z.
% * <rivka.eisner@uzh.ch> 2018-09-17T12:27:13.848Z:
% 
% > socially less
% Reverse word order here: "less socially influential individuals"
% 
% ^ <cdebenitomoreno@gmail.com> 2018-10-31T15:49:19.989Z.
% * <rivka.eisner@uzh.ch> 2018-09-17T12:19:11.956Z:
% 
% > the individual speech habits are shaped by whom an individual has interacted with before.
% 
% Please change to:  "describes how individual speech habits are shaped by those who an individual has interacted with before."  
% 
% Also, to avoid the repetition of "individual" in the same sentence, could the latter one  (e.g. "an individual") be changed to "a person" instead?
% 
% ^ <cdebenitomoreno@gmail.com> 2018-10-31T15:50:15.444Z.
% * <rikvangijn@gmail.com> 2018-06-27T07:20:45.862Z:
% 
% We should take up this bit in the introduction to the book as well.
% 
% ^.

% to quote: ``text''
A more explicit model of the relationship between linguistic accommodation and dialect leveling and change is formulated by \cite{trudgill_dialects_1986}. He draws a link between social psychologists’ CAT and the questions of linguistic diffusion in space, i.e., the micro and the macro level of dialect contact. ``Clearly, if a linguistic feature has spread from one region to another, it must have spread from one speaker to another, and then on to other speakers and so on. But how exactly are linguistic forms transmitted from one geographical area to another at the level of the individual speaker'' \citep[39]{trudgill_dialects_1986}? Trudgill suggests that accommodation is the mechanism of diffusion at the micro level, arguing that ``if a speaker accommodates frequently enough to a particular accent or dialect [. . .] then the accommodation may in time become permanent, particularly if attitudinal factors are favorable'' \citep[39]{trudgill_dialects_1986}. This idea was then taken up by \cite{auer_role_2005}, who refine the so-called change-by-accommodation model. According to their model, short-term shifts may, through repeated interactions, accumulate into long-term accommodation and thus lead to innovation in an individual's speech habit. Given favorable network structures and the critical mass of speakers displaying an innovative feature, the innovation may – again via accommodation – spread to other speakers and lead to linguistic change at the community level. 
% * <rivka.eisner@uzh.ch> 2018-09-17T12:56:19.318Z:
% 
% > (...)
% If you are omitting parts of Trudgil's writing (and "(...)" is not included in his own text), then it is most typical to use brackets instead, with single spaces between the periods, e.g.: "accent or dialect [.  .  .] then the accommodation . . ."
% 
% ^ <cdebenitomoreno@gmail.com> 2018-10-31T15:51:26.197Z.
% * <rivka.eisner@uzh.ch> 2018-09-17T12:54:10.939Z:
% 
% > r?'' \citep[39]{trudgill_dialects_1986}.
% 
% Write the "?" at the end of the full cite here: "level of the individual speaker" (Trudgil 1986: 39)?
% 
% ^ <cdebenitomoreno@gmail.com> 2018-10-31T15:51:50.445Z.
% * <rivka.eisner@uzh.ch> 2018-09-17T12:44:30.894Z:
% 
% > Communication Accommodation Theory
% Since this is the second time "communication accommodation theory" has come up in the chapter, the abbreviation can be used without reference to the full name. Also, perhaps include a little more reference here, e.g.: "link between social psychologists' understanding of [or use of?] CAT and the questions of. . .".
% Also: Check if this name should be capitalized or not?  If yes, then please correct per my comment earlier in the text (p. 2 where the term is introduced) & anywhere else the full name of the theory is written.
% 
% ^ <cdebenitomoreno@gmail.com> 2018-10-31T15:52:56.318Z.
% * <rivka.eisner@uzh.ch> 2018-09-17T12:29:58.329Z:
% 
% > was 
% You may want to change this to "is" to keep consistent with the present tense. 
% 
% ^ <cdebenitomoreno@gmail.com> 2018-10-31T15:53:30.096Z.

\subsubsection*{From short-term to long-term}

Although the change-by-accommodation model is widely acknowledged in linguistics, there is a lack of empirical evidence for the postulated idea that repeated short-term accumulates into long-term accommodation.
\cite{auer_role_2005} compare several sociolinguistic case studies on short-term or long-term accommodation with linguistic change taking place at the community level. Based on the reviewed case studies, they conclude that patterns observed in individual speakers do not parallel the change described at the community level. As a result, the authors question the change-by-accommodation model. Their evaluation of the model is mainly based on studies of accommodation which used auditory-phonetic methods. It is thus possible that with more sophisticated, acoustic methods, more subtle shifts in pronunciation could be observed, or that other linguistic levels, for instance, morphology or syntax, behave differently from phonology. 
 
To our knowledge, the only study so far that systematically analyzes variability over a short and a longer time period is \cite{sonderegger_medium-term_2017}. This work investigates phonetic variability for five phonological variables (three vowels, stop aspiration and voicing and /t/-deletion) and compares the variability over days with the variability over months in 12 speakers of UK Big Brother. The authors' approach permits studying variability – and accommodation – in a closed communication system where the speakers only communicate among them, and with nobody from outside the house. They found that variability over days is very common for all speakers and all five variables they looked at. Some speakers showed a time trend for some variables (i.e. lowering of F2 over several weeks). For many speakers and many variables, however, by-day variability did not accumulate into a time-trend, and there was no evidence for overall accommodation despite frequent interaction. The only clear evidence for convergence was found for two individuals who also formed a close social bond. \cite{sonderegger_medium-term_2017} confirm the effect of linguistic as well as social factors on time-dependent phonetic variability, but they also show that, in their data, short-term trends only occasionally accumulate into longer-term changes. The authors speculate that this is the case because individual speakers present large differences in terms of pronunciation plasticity. Based on their findings, \citet{sonderegger_medium-term_2017} speculate that accent change over years may vary even more  between different speakers because long-term changes themselves are assumed to build upon medium-term changes. The study suggests that, as short-term accommodation, medium-term dynamics of phonetic variables is mediated by social and linguistic factors as well as individual differences. \cite{sonderegger_medium-term_2017} relate the important individual differences in phonetic plasticity to the different roles of individuals in the adoption and spread of sound change \citep[reminiscent of the contrasts between early adopters and innovators;][]{milroy_linguistic_1985}.
% * <rivka.eisner@uzh.ch> 2018-09-20T13:10:55.824Z:
% 
% > To our knowledge, the only study so far that systematically analyzes variability over a short and a longer time period is \cite{sonderegger_medium-term_2017}. 
% When describing the studies in this paragraph, the tense shifts between present and past.  It would be good to choose to refer to other studies in either the past or present tense and then stay with that tense throughout the chapter for the sake of consistency.
% This note regarding tense applies to this specific paragraph/instance and to all other locations where colleagues' work is described throughout the text. For example, it seems that up until this point, other studies have been described in the present tense (e.g. p. 3-4). But closer to the end of the document (e.g. p. 14), the past tense is used more often.  Please check document and decide which tense makes the most sense/is most consistently used in your field.  Please note that if you decide to use the past tense, you can still change to the present tense at times (even within the same sentence) when you are referring to, for example, theories whose results are still considered correct today.    
% 
% ^ <cdebenitomoreno@gmail.com> 2018-10-31T15:54:02.948Z.
% * <rivka.eisner@uzh.ch> 2018-09-20T13:09:54.974Z:
% 
% > suggest 
% 
% Replace with "suggests": "the study suggests that."
% 
% ^ <cdebenitomoreno@gmail.com> 2018-10-31T15:54:32.625Z.
% * <rivka.eisner@uzh.ch> 2018-09-18T11:25:45.089Z:
% 
% > early adopters and innovators;
% Is "early adopters and innovators" an example here? Perhaps insert "i.e." as done above.
% 
% ^ <cdebenitomoreno@gmail.com> 2018-10-31T15:56:19.430Z:
% 
% Hanna, I've added the "reminiscent of…", is that ok?
%
% ^.
% * <rivka.eisner@uzh.ch> 2018-09-18T11:19:23.052Z:
% 
% You may want to write out "and" instead of using the "&" sign within the text.  Please choose whichever is most acceptable within your field and/or with your chosen style guide.  This note is for the entire text. 
% 
% ^.

Further indirect evidence for a more complex relationship between short-term and long-term accommodation comes from studies on long-term accommodation \citep[see][]{ruch_function_2018}. First, most adults hardly ever acquire a second dialect perfectly even after living for several years in a new social environment \citep{siegel_second_2010}. And second, there are examples of accent reversion showing that repeated short-term accommodation does not necessarily accumulate over time and, therefore does not necessarily lead to long-term accommodation. For instance, the British journalist and radio presenter Alistair Cooke  first converged toward American English after having migrated from the UK to the USA, but shifted back to his British English accent (i.e. reversed his accent) in later life \citep{reubold_dissociating_2015}.  Similar findings are reported by \cite{werlen_zwischen_2006}, who investigate how speakers from Valais, a canton in the southern part of Switzerland, change their pronunciation after relocating to Berne. Two years after relocating, five out of 18 participants used \textit{less} Bernese variants than shortly afterwards. More longitudinal studies and more research comparing short- and long-term accommodation within individuals are needed to empirically establish this relationship.
% * <rivka.eisner@uzh.ch> 2018-09-18T11:34:23.609Z:
% 
% > reubold_dissociating_2015}
% 
% In the text, the formatting here looks a little strange. Were Reubold & Harrington the one's studying Alistair Cooke's speech? If so, perhaps move the reference to the end of the sentence. Or, maybe because it is his speech that converged rather than he himself, it could be written: "For instance, it was found that the speech [or speech patterns?] of radio presenter Alistair Cooke first converged toward American English after having. . . later in life (Reubold & Harrington 2015)."
% 
% *"Towards" is more commonly used in UK English and "toward" is more common in US English. 
% 
% ^ <cdebenitomoreno@gmail.com> 2018-10-31T16:02:11.610Z.
% * <rivka.eisner@uzh.ch> 2018-09-18T11:30:33.173Z:
% 
% > over time and therefore, does not necessarily lead to long-term accommodation.
% 
% Here the comma can be moved from after "therefore" to after the word "time": "accumulate over time, and therefore does not necessarily"
% 
% ^ <cdebenitomoreno@gmail.com> 2018-10-31T16:02:48.170Z.
% * <rivka.eisner@uzh.ch> 2018-09-18T11:28:19.316Z:
% 
% > do
% The word "do" can be deleted here. It is not necessary. 
% 
% ^ <cdebenitomoreno@gmail.com> 2018-10-31T16:02:58.000Z.
% * <rikvangijn@gmail.com> 2018-06-27T07:22:03.845Z:
% 
% > decrease in convergence
% Does this mean that earlier convergence is lost subsequently?
% 
% ^ <hanna.ruch@uzh.ch> 2018-07-09T12:46:27.621Z:
% 
% I describe the examples in more detail; I hope it is clearer now.
%
% ^ <hanna.ruch@uzh.ch> 2018-08-03T14:41:57.275Z.
% * <rikvangijn@gmail.com> 2018-06-27T07:21:36.470Z:
% 
% > accent reversion
% What is that?
% 
% ^ <hanna.ruch@uzh.ch> 2018-07-09T12:46:31.736Z:
% 
% I describe the examples in more detail; I hope it is clearer now.
%
% ^ <hanna.ruch@uzh.ch> 2018-08-03T14:42:00.481Z.
% * <rikvangijn@gmail.com> 2018-06-27T07:22:47.017Z:
% 
% I thought the above paragraph was perhaps a bit too short, and would require some elaboration for non-specialists (especially the two terms mentioned in the comments above).
% 
% ^ <hanna.ruch@uzh.ch> 2018-07-09T12:46:35.404Z:
% 
% OK; see comments above.
%
% ^ <hanna.ruch@uzh.ch> 2018-08-03T14:42:03.865Z.

\subsection{Toward an improved change-by-accommodation model}

In order to increase the insightfulness of the change-by-accommodation model, in this section we suggest a number of lines of research that have not received as much attention as others, but which seem to us to be of crucial importance to shed light on the role of accommodation in language change. 
% * <rivka.eisner@uzh.ch> 2018-09-18T11:44:16.565Z:
% 
% > the insightfulness, 
% "Increase the insightfulness" of what?  Could it be written: "In order to increase the insightfulness of these ____ studies, we suggest"?
% 
% ^ <cdebenitomoreno@gmail.com> 2018-10-31T16:03:41.698Z.
% * <rikvangijn@gmail.com> 2018-06-27T07:24:50.548Z:
% 
% > Apart from these caveats,
% Maybe rather something like "In order to increase its insightfulness in the matter of short- versus long-term accommodation," (or something similar) 
% 
% ^ <hanna.ruch@uzh.ch> 2018-07-09T12:47:59.954Z:
% 
% OK
%
% ^ <hanna.ruch@uzh.ch> 2018-08-03T14:42:06.758Z.
First, it is assumed that long-term accommodation is relevant to understanding contact-induced language change such as, for instance, dialect leveling \citep{trudgill_dialects_1986}. If a group of speakers moves from region A to region B, this may eventually lead to innovation or contact-induced change in variety B. However, studies on long-term accommodation most commonly focus on linguistic effects of exposure to a new linguistic environment within \textit{mobile} speakers, that is, in variety A, and does not usually address variability within the receiving community.
% * <rikvangijn@gmail.com> 2018-06-27T07:27:26.671Z:
% 
% > dialect leveling
% Insert a cross-reference to the dialect chapter
% 
% ^.
A possible exception is \citet{klee_andean_2006}, who study the speech of Andean migrants in Lima and also analyzed a control group of lower-class Limeños, the social group most likely to be in contact with the migrant population. \cite{klee_andean_2006} find no evidence of change within the receiving community's speakers as a result of contact with migrants. That is, this study does not support the idea that migrants spread linguistic feature to a new community. \cite{Escobar2007}, on the contrary, suggests that migrant speakers brought Andean Spanish features into \textit{costeño} Spanish spoken in Lima, although she considers this influence to be restricted to syntactic features of low sociolinguistic salience.
% * <rivka.eisner@uzh.ch> 2018-09-18T11:53:07.731Z:
% 
% > At least
% Word choice?  Is it a good thing that the study does not support this kind of change?  Would "However" work better as a more neutral term here?
% 
% ^ <cdebenitomoreno@gmail.com> 2018-10-31T16:04:41.966Z.
% * <rivka.eisner@uzh.ch> 2018-09-18T11:50:18.096Z:
% 
% > ound no evidence for the receiving community's speakers to change as a result of contact to the migrated individuals.
% 
% Suggested change: "Klee & Caravedo (2006) found no evidence of change within the receiving community's speakers as a result of contact with those who had recently migrated."
% 
% ^ <cdebenitomoreno@gmail.com> 2018-10-31T16:05:39.356Z.
% * <rivka.eisner@uzh.ch> 2018-09-18T11:48:25.790Z:
% 
% > \citep{klee_andean_2006} 
% 
% The parentheses should be moved to encompass just the date: "Klee & Caravedo (2006)..."
% 
% ^ <cdebenitomoreno@gmail.com> 2018-10-31T16:07:32.785Z.
% * <rivka.eisner@uzh.ch> 2018-09-18T11:47:07.035Z:
% 
% > i.e. 
% Could "i.e." be omitted here?  One could write "the social group" or "a social group likely to be in contact," etc.  
% 
% ^ <cdebenitomoreno@gmail.com> 2018-10-31T16:08:15.997Z.
% * <rikvangijn@gmail.com> 2018-06-27T07:28:35.929Z:
% 
% There is this paper by Naess and Jenny, who study a Burmese and  Astronesian societies in which it seems that immigrants have had a profound impact on the languages of the receiving speech communities. I don't know if this is beyond the scope of the chapter, but it might be worth to have a look at: Næss, Åshild; Jenny, Mathias (2011).  Who changes language? Bilingualism and structural change in Burma and the Reef Islands. Journal of Language Contact, 4(2):217-249.
% 
% ^ <hanna.ruch@uzh.ch> 2018-08-03T14:42:14.152Z.
Ideally, future work will concentrate not only on mobile individuals, but also investigate the possible effects on the receiving community. We argue that in order to understand contact-induced change, the receiving community is as important as the migrating individuals. Long-term changes in the speech of mobile individuals, on the other hand, are ideal scenarios to shed light on dialect attrition within individuals.

Second, although most research on accommodation has dealt with adult speakers, children may be as relevant as adults to test and refine the change-by-ac\-com\-mo\-dation model. On the one hand, children are generally acknowledged to acquire a second dialect more quickly and more easily than adults \citep{siegel_second_2010} when moving to a new environment. At the same time, they seem to be sensitive to linguistic variation from early on. For instance, \cite{jones_development_2017} shows that even some of the 4-5-years old listeners they study are able to distinguish their own regional variety from other varieties of American English, and \cite{khattab_phonetic_2013} describes how three children between 5 and 10 years of age converge and diverge in the use of local, standard and non-native phonetic features in English when interacting with their mothers. 
Children might be relevant to dialect leveling and change for several reasons. They may acquire a dialect imperfectly, bringing D1 features into D2, but may also become bidialectal speakers, that is, become fluent in both dialects while still separating them. For instance, they may use D1 at home, and D2 in school and elsewhere. Finally, children may also end up with a mixed variety \citep{chambers_dialect_1992,tagliamonte_howd_2007}, what \cite{klee_andean_2006} see as a possible source for dialect leveling and change, presupposing a critical mass of speakers.
% * <rivka.eisner@uzh.ch> 2018-09-21T11:12:32.950Z:
% 
% > ,
% This comma can be deleted.
% 
% ^ <cdebenitomoreno@gmail.com> 2018-10-31T16:24:34.673Z.
% * <rivka.eisner@uzh.ch> 2018-09-18T11:58:14.786Z:
% 
% > Finally, children may also end up with a mixed variety \cite{chambers_dialect_1992,tagliamonte_howd_2007}, what \cite{klee_andean_2006} see as a possible source for dialect leveling and change, presupposing a critical mass of speakers.
% 
% Please check where parentheses are placed within this sentence. Right now, they do not look accurate. 
% 
% ^ <cdebenitomoreno@gmail.com> 2018-10-31T16:30:48.498Z.

%(Describe link here to the change-by-accommodation model, perhaps by thinking of a specific situation and what the role of the children might be. Ideas: Children might not acquire a new dialect perfectly and this way bring some D1 features into D2. Or they might become bidialectal and speak D1 at home... Or they may stop separating the two varieties and acquire a mixed variety, what \cite{klee_andean_2006} suggest may be related to dialect leveling. Or they might pick up on their parent's "assimilated features", which might result either in mixed varieties or in generalizing these features? It really seems to me that focusing on individuals who have moved to a different place is not on the right track, since we have little evidence that "foreigners/inmigrants" affect the speech of their new home –except for dialect leveling cause by large population movements—: for the spread of dialect features through space, I think exposure through mobile individuals and inter-community events (such as balls, for instance) might be more important... Here comes in also the fact that, when studying dialect contact we tend to focus on distinct varieties (i.e. American vs. British English), but for the spread of dialect features through space we might be talking about very similar varieties, which differ not even in a single feature, but in the contexts of that features –I'm thinking in the spread of /s/ aspiration from the South of Spain, for instance–.)
Third, the model remains rather vague in how exactly contact between speakers takes place, and the kinds of situations facilitating either short- or long-term accommodation. A central question is whether short- or long-term accommodation has more impact on a given linguistic variety. More concrete predictions and, ideally, their empirical validation would allow to link these ideas to issues of areal linguistics. %CrossReference to chapters on Dialect Areas and Linguistic areas.
% * <rivka.eisner@uzh.ch> 2018-09-18T12:12:54.608Z:
% 
% > linguistics 
% Missing a period after "linguistics."
% 
% ^ <cdebenitomoreno@gmail.com> 2018-11-01T08:49:42.724Z.
% * <rivka.eisner@uzh.ch> 2018-09-18T12:10:01.780Z:
% 
% > a variety
% 
% Would it be "a variety" here or just "variety"?  Is "variety" a technical term that does not need to be qualified -- e.g.  "a variety of__" ?
% 
% ^ <cdebenitomoreno@gmail.com> 2018-11-01T08:51:19.620Z.
For instance, it has been found that travel times correlate with linguistic similarity \citep{jeszenszky_exploring_2017}.
% * <rikvangijn@gmail.com> 2018-06-27T07:40:11.813Z:
% 
% > areal linguistics
% Add cross-reference to chapters on dialect areas and linguistic areas.
% 
% ^.
We can think of at least two scenarios leading to these patterns. First, speakers are more likely to move to close-by, culturally and linguistically similar areas \citep[e.g.][]{falck_cultural_2016}. In this case, linguistic similarity would by induced by the mobile speakers' influence on the local dialect. Alternatively, places within shorter travel distances might favor frequent short-term contacts, for instance, through trading, commuting, etc. In the latter case, dialect change and leveling would take place through repeated short-term accommodation in face-to-face interactions.

\section{Approaches}
% * <rikvangijn@gmail.com> 2018-06-27T08:41:43.007Z:
% 
% > Methods
% Rename into approaches?
% 
% ^ <hanna.ruch@uzh.ch> 2018-07-09T13:46:19.263Z:
% 
% For experimentally-orientend work, "methods" is much more common (I think "approaches" would be weird...).
%
% ^ <hanna.ruch@uzh.ch> 2018-08-03T14:42:40.290Z.

In this section, we present the most common methods used in accommodation research. We will start by presenting the methodological approaches to short-term accommodation and then discuss the most common methods that have been used to study long-term accommodation.
% * <rivka.eisner@uzh.ch> 2018-09-18T12:44:39.914Z:
% 
% > to then
% Change to "and then" here.
% 
% ^ <cdebenitomoreno@gmail.com> 2018-11-01T09:00:18.709Z.

\subsection{Short-term accommodation}
Studies on short-term accommodation can roughly be divided into two types: dialogue studies and shadowing tasks. Dialogue studies analyze recorded dialogues between speakers, mostly between unacquainted persons. In most study designs, the participants are given a collaborative task such as describing a route on a map to their interactant \citep[i.e. map task, e.g.][]{pardo_phonetic_2006} or finding the differences on otherwise identical pictures \citep[i.e. diapix task, e.g.][]{kim_phonetic_2013}, whereas in other work, participants are asked to converse freely \citep[e.g.][]{schweitzer_convergence_2013}. Dialogue studies represent more natural speech situations than shadowing tasks, making them suitable to investigate socio-psychological issues such as the relationship between accommodation and speaker perception.
% * <rikvangijn@gmail.com> 2018-06-27T08:42:15.091Z:
% 
% > more natural
% Than what? Perhaps use st. like "more or less" or else use the explicit comparison to shadowing tasks here already.
% 
% ^ <hanna.ruch@uzh.ch> 2018-07-09T14:03:45.534Z:
% 
% Done.
%
% ^ <hanna.ruch@uzh.ch> 2018-08-03T14:42:42.965Z.

So-called shadowing tasks \citep{goldinger_echoes_1998,shockley_imitation_2004,babel_dialect_2010}, in contrast, involve more controlled situations, which makes them particularly appealing to study the effect of linguistic factors. The experiments typically comprise three phases: (1) recording the participants' baseline productions, (2) listening to the speech of a model speaker over headphones and (3) recording the participants' post-task speech. Post-task productions are then compared with baseline productions to see whether the participants became linguistically more similar, that is, whether they converged to the model speaker. Variations of the paradigm have been implemented in web-based experiments \citep{weatherholtz_socially-mediated_2014} or in experiments involving nonhuman model speakers \citep[e.g.][]{beckner_participants_2016}.
% * <rikvangijn@gmail.com> 2018-06-27T08:44:35.238Z:
% 
% > on the contrary
% in contrast / on the other hand
% 
% ^ <hanna.ruch@uzh.ch> 2018-07-09T14:04:53.722Z:
% 
% Done
%
% ^ <hanna.ruch@uzh.ch> 2018-08-03T14:42:45.134Z.
The listening task can consist of isolated words \citep[e.g.][]{goldinger_echoes_1998} or a longer passage \citep[e.g.][]{yu_phonetic_2013,weatherholtz_socially-mediated_2014}. Sometimes, listeners are asked to repeat each word separately, while in other cases, the listening and speaking tasks are taken in blocks, implying a longer pause between the listening task and the post-task production.


The methods to assess accommodation also vary considerably across studies and subdisciplines. Dialogue studies have often assessed accommodation by asking independent listeners to judge the similarity of dialogue excerpts \citep{pardo_phonetic_2006,kim_phonetic_2013}. This approach has been used in several shadowing tasks too \citep[e.g.][]{goldinger_echoes_1998}, turning out to be a very useful method for assessing the global similarity of isolated words. In other phonetically-oriented studies, specific parameters are measured \citep[e.g.][]{babel_dialect_2010,de_looze_investigating_2014} which, however, correlated only marginally  with perceived similarity as assessed by independent listeners \citep{pardo_phonetic_2013,walker_repeat_2015,abel_cognitive_2016,pardo_phonetic_2017}. Research on lexical, syntactic, or morphological accommodation usually quantifies the frequency of occurrence of the linguistic variants of interest \citep[e.g.][]{beckner_participants_2016,weatherholtz_socially-mediated_2014}.
% * <rivka.eisner@uzh.ch> 2018-09-18T13:03:58.832Z:
% 
% > s.
% Please ignore the following comment below: "*Change placement of "However" to beginning of sentence and omit comma after "which."
% I posted the above note in the wrong comment box and then accidentally "closed" the conversation when I meant to fix it -- can't figure out how to delete it or change it. My apologies!
% 
% ^ <cdebenitomoreno@gmail.com> 2018-11-01T09:07:58.040Z.
% * <rivka.eisner@uzh.ch> 2018-09-18T12:53:10.089Z:
% 
% > In other phonetically-oriented studies, specific parameters are measured \citep[e.g.][]{babel_dialect_2010,de_looze_investigating_2014} which, however, correlated only marginally  with perceived similarity as assessed by independent listeners \citep{pardo_phonetic_2013,walker_repeat_2015,abel_cognitive_2016,pardo_phonetic_2017}
% Consider changing to: "However, in other phonetically-oriented studies, specific parameters are measured ... which correlated only marginally with perceived...".
% 
% ^ <cdebenitomoreno@gmail.com> 2018-11-01T09:07:59.463Z.
% * <rivka.eisner@uzh.ch> 2018-09-18T12:50:29.416Z:
% 
% > shadowing tasks, too 
% Consider changing to: "used in several shadowing tasks as well (e.g. Goldinger 1998), where it..."
% 
% *Change placement of "However" to beginning of sentence and omit comma after "which."
% 
% ^ <rivka.eisner@uzh.ch> 2018-09-18T12:56:55.217Z.

These differences in research design as well as in the quantification of accommodation make comparisons across studies difficult. For these reasons, in Section \ref{Section_Factors}, rather than compare the degree of accommodation or other details across studies, we will organize the findings of accommodation according to the research questions outlined in \ref{Section_Introduction}: What are the linguistic patterns resulting from short- and long-term accommodation? What linguistic and extralinguistic factors favor or mitigate accommodation processes?
% * <rikvangijn@gmail.com> 2018-06-27T08:47:30.179Z:
% 
% > Therefore, rather than comparing the degree of accommodation or other details across situations and studies, we will organize the findings of accommodation research to address the questions outlined in the introduction to this chapter. 
% 
% This is a bit cryptic. It refers to section 5, right? Maybe put a cross-reference here.
% 
% ^ <hanna.ruch@uzh.ch> 2018-07-09T14:17:44.377Z:
% 
% I have changed it and added a cross-reference. Better now?
%
% ^ <hanna.ruch@uzh.ch> 2018-08-03T14:43:00.025Z.


\subsection{Long-term accommodation}
Studies on long-term accommodation typically focus on speakers who have moved from their region of origin to a place where a variety different from their own is spoken. Studies on long-term accommodation are frequently framed within a sociolinguistic approach. This means that they typically rely on semi-spontaneous speech, often collected by means of sociolinguistic interviews \citep[][among many others]{shockey_all_1984, auer_subjective_1998, romera_prosodic_2013}. 
% * <rivka.eisner@uzh.ch> 2018-09-18T13:12:31.313Z:
% 
% > on production data
% Would it be "on the production data" here?
% 
% ^ <cdebenitomoreno@gmail.com> 2018-11-01T09:14:01.174Z.
% * <rikvangijn@gmail.com> 2018-06-27T08:49:00.926Z:
% 
% > among many others).
% Fix the bracketing
% 
% ^ <hanna.ruch@uzh.ch> 2018-07-09T14:27:48.773Z:
% 
% Done - but there are still ":" which don't want to disappear...
%
% ^ <cdebenitomoreno@gmail.com> 2018-11-01T09:13:40.882Z:
% 
% This is the same thing I was talking about above… It has to do  with the citation style, I think, which separates the reference it self from whatever is in brackets with a colon. That's cool for pages but not for other stuff. Maybe Tanja can help us with this?
%
% ^.
In longitudinal studies, the same speakers are recorded several times after having moved to a new region, what allows tracking an individual's linguistic shifts over time. Probably because of the considerable logistic effort needed, longitudinal studies are rather rare \citep[but see][]{shockey_all_1984,auer_subjective_1998,reubold_dissociating_2015}.

An exception, however, are studies on the effect of accommodation on children and youngsters, which are often longitudinal. For instance, \citet{chambers_dialect_1992} records his speakers twice in a two-year period, while \citet{tagliamonte_howd_2007} record their participants every weekend starting 6 months after having moved from Canada to England. The often large time lapses between interviews are due to logistic challenges. In \citet{tagliamonte_howd_2007}, however, the subjects are the children of the first author, a fact that facilitated data collection.
% * <rivka.eisner@uzh.ch> 2018-09-18T13:16:40.757Z:
% 
% > at the same time
% This phrase, "at the same time," can be deleted so that it reads:  "In Tagliamonte & Molfenter (2007), however, the subjecs were the children of the first author, a fact that facilitated data collection."
% 
% ^ <cdebenitomoreno@gmail.com> 2018-11-01T09:14:26.850Z.
% * <rivka.eisner@uzh.ch> 2018-09-18T13:14:53.570Z:
% 
% > mostly 
% Word choice. Perhaps change to "high incidence"?  E.g. "the high incidence of large time lapses between interviews..."
% 
% ^ <cdebenitomoreno@gmail.com> 2018-11-01T09:15:40.235Z.
At any rate, the majority of investigations concerned with long-term accommodation rely on data collected once in time for each subject. Usually, the participants' speech after migrating is then compared to existing, general descriptions of their linguistic variety \citep{shockey_all_1984, trudgill_dialects_1986, MolinaMartos2010} or to non-mobile speakers from their place of origin \citep{PalaciosAlcaine2007,Fernandez2013}. To investigate the effect of time of exposure on accommodation, usually time spent in the new environment is used as a predictor \citep{shockey_all_1984, romera_prosodic_2013, erker_contact_2016}, although this parameter of course does not necessarily correlate with the actual amount of linguistic exposure to the new variety. In comparison to short-term studies, which often follow a controlled, experimental protocol, longer-term changes in speech are much more difficult to trace back to specific factors. Some studies have used questionnaires in order to gain additional information about the speakers' social environment or attitudes \citep[e.g.][]{pesqueira_cambio_2008}.
% * <rivka.eisner@uzh.ch> 2018-09-18T13:19:18.384Z:
% 
% > the new variety
% Similar question as before: does "variety" need qualifying?  E.g. "exposure to the new speech pattern variety," or... perhaps it is fine the way it is if this is an already specified term in your field.
% *This note applies to other occurrences within the chapter as well.
% 
% ^ <cdebenitomoreno@gmail.com> 2018-11-01T09:16:18.013Z.


\section{Patterns and processes}
% * <rikvangijn@gmail.com> 2018-06-27T07:59:57.294Z:
% 
% > processes
% You mainly talk about patterns here. Given the psychological literature, can you say something about the processes as well, i.e. what is the driving cognitive force behind accommodation, is there any relation to the type of conversation?
% 
% ^ <max.wahlstrom@helsinki.fi> 2018-06-29T14:28:38.614Z:
% 
% A very good suggestion. Maybe this could be done by moving a paragraph from the Intro here (maybe adding two subsections: 3.1 Processes; 3.2 Patterns. Also, in terms of processes, I don't know what you think about it, but perhaps a couple of sentences about the typical extra linguistic settings that expose individuals to situations where accomodation may become more frequent, migrations, urbanization...
% 
% ^ <hanna.ruch@uzh.ch> 2018-07-09T13:40:25.357Z:
% 
% That sounds reasonable to me. I have organised the section according to Max' suggestion and moved the aspects related to processes from the Introduction to the Processes subsetion.
% 
% ^ <hanna.ruch@uzh.ch> 2018-08-03T14:42:24.050Z.
It is useful to distinguish between patterns of accommodation, i.e. its possible outcomes, and the processes whereby accommodation takes place. We discuss both in what follows.
\subsection{Patterns}

\cite{giles_accommodation_1991} distinguish between three  accommodative patterns: convergence, divergence and maintenance. Convergence describes the situation where speakers become more similar to their dialogue partner or a model speaker. In divergence, individuals become more dissimilar to their conversation partner or to a model speaker. Maintenance, finally, denominates the case where an individual does not shift toward or away from another speaker, but largely maintains her or his way of speaking. In dialogues, convergence and divergence can be reciprocal, but also asymmetric in the sense that one, but not the other speaker, converges or diverges. \cite{giles_accommodation_1991} further note that speakers may converge on some parameters, while diverging on others.
% * <rivka.eisner@uzh.ch> 2018-09-21T11:19:35.505Z:
% 
% >  but not the other speaker
% Insert another comma here: "in the sense that one, but not the other speaker, converges or diverges."
% 
% ^ <cdebenitomoreno@gmail.com> 2018-11-01T08:51:34.148Z.
% * <rivka.eisner@uzh.ch> 2018-09-18T12:17:55.814Z:
% 
% > converge on some, while diverging on other parameters.
%  
% Consider specifying here: "may converge on some ____, while diverging on other parameters."
% 
% ^ <cdebenitomoreno@gmail.com> 2018-11-01T08:53:46.060Z.
% * <rivka.eisner@uzh.ch> 2018-09-18T12:15:31.069Z:
% 
% > distinguish three  accommodative patterns. 
% 
% Consider writing: "Giles, J. Coupland & N. Coupland (1991) distinguish between three accommodative patterns: convergence, divergence, and maintenance."
% 
% ^ <cdebenitomoreno@gmail.com> 2018-11-01T08:54:22.318Z.

As is apparent from the present chapter and from previous work reviewing accommodation studies \citep{ruch_function_2018}, convergence seems to occur much more frequently than divergence. One possible explanation for this bias is that alignment is the default pattern and, as a consequence, is observed much more frequently than maintenance or divergence \citep[see][]{dijksterhuis_perception-behavior_2001}. However, another possible explanation is that, given that convergence is the expected result, divergence is not as thoroughly scrutinized by researchers, or simply that null results or divergence are more difficult to publish. This could have led to a publication bias toward convergence. For syntactic accommodation, divergence indeed seems to receive some support in the literature: In order to actively engage with their interlocutor, speakers seem to use complementary structures rather than repetition \citep{healey_divergence_2014}.
% * <rivka.eisner@uzh.ch> 2018-09-18T12:20:16.148Z:
% 
% > looked for
% Word choice: maybe consider "scrutinized," "examined," or "sought out" or something else? In all three examples, the word "for" would be deleted.  
% 
% ^ <cdebenitomoreno@gmail.com> 2018-11-01T08:57:01.859Z.
% * <rikvangijn@gmail.com> 2018-06-27T07:53:06.548Z:
% 
% > be a plausible result
% Perhaps rather: "receive some support in the literature" (?)
% 
% ^ <hanna.ruch@uzh.ch> 2018-07-09T13:41:28.943Z:
% 
% OK
%
% ^ <hanna.ruch@uzh.ch> 2018-08-03T14:42:26.968Z.

Given that analysis and quantification of accommodation largely differ across studies, it is extremely difficult to describe linguistic patterns in accommodation more generally. As mentioned above, socio-psychological work so far has mainly focused on whether accommodation was perceivable and has therefore worked with perceptual, more holistic measures of accommodation. Work within computational linguistics, too, has used holistic measures \citep{lewandowski_talent_2012,de_looze_investigating_2014}, however, often without relating them to linguistically interpretable categories. More recent work within linguistics and psycholinguistics has studied a limited number of specific linguistic features, mostly on one level of linguistic description only, for instance, voiceless stops \citep{nielsen_specificity_2011}, vowel quality \citep{babel_dialect_2010}, past tense formation \citep{beckner_participants_2016}, or the English dative alternation \citep{weatherholtz_socially-mediated_2014}. Most studies involving dialect contact deal with phonetics or phonology, perhaps because in this area, dialectal differences are most obvious and better described than, for instance, in morphology, syntax, or pragmatics. Furthermore, when working with spontaneous or semi-spontaneous speech, it is more feasible to get a sufficient number of tokens for phonetic or phonological features than, for instance, for syntax or lexis.
% * <rivka.eisner@uzh.ch> 2018-09-18T12:25:11.467Z:
% 
% > , for instance, in morphology, syntax, or pragmatics.
% I notice that here and in the preceding sentence the "Oxford comma" is being used, but elsewhere in the manuscript it is not.  It would be good to review the document and decide whether or not to use the Oxford comma throughout. Then, commas can be added or subtracted where necessary. 
% 
% ^ <cdebenitomoreno@gmail.com> 2018-11-01T08:57:21.637Z.

For these reasons, in Section \ref{Section_Factors} we will prescind from enumerating different linguistic phenomena observed in accommodation research, but instead group and discuss the observed patterns according to the linguistic and extralinguistic factors favoring or inhibiting accommodation.

\subsection{Processes}

The two most influential models dealing with the processes underlying accommodation are the so-called Communication Accommodation Theory (CAT) and the Interactive Alignment Model (IAM).
CAT \citep{giles_accent_1973,giles_towards_1973,giles_speech_1975} was developed in the field of social psychology and attributes accommodation primarily a social function. Convergence and divergence are seen as the speakers' communicative strategies to express social closeness of social distance in an interaction \citep[293]{giles_communication_2007}. The model thus focuses more on the ultimate \textit{function} of accommodation rather than on its underlying mechanisms.
% * <rivka.eisner@uzh.ch> 2018-09-18T12:37:50.925Z:
% 
% > ;
% Semicolon can be deleted here.
% 
% ^ <cdebenitomoreno@gmail.com> 2018-11-01T08:59:05.187Z.
% * <rivka.eisner@uzh.ch> 2018-09-18T12:33:46.150Z:
% 
% > Communication Accommodation Theory and the Interactive Alignment Model.
% > Communication Accommodation Theory \citep[CAT;]
% 
% Communication Accommodation Theory was referred to early on in the chapter as well.  Perhaps if you are going into more detail with the terms here, you can just introduce the abbreviation (i.e. CAT) here.  Or, you could do it in both places -- e.g. the first time it is referenced in the chapter and again here. 
% 
% ^ <cdebenitomoreno@gmail.com> 2018-11-01T08:58:55.246Z.

IAM \citep{pickering_toward_2004} has its origins in cognitive psychology and sees convergence as an automatic process which results from a link between speech perception and speech production. This link is similar to the priming mechanism and is constantly activated during speech processing \citep{pickering_toward_2004}. In some cases, it is difficult to separate accommodation from priming. We follow \cite{pickering_toward_2004} who see priming as the underlying mechanism of accommodation, whereas accommodation is the process of mutual linguistic adjustments in its communicative context.
% * <cdebenitomoreno@gmail.com> 2018-11-01T09:03:04.075Z:
% 
% >  \citep[IAM]{pickering_toward_2004} 
% Hanna, da haben wir ein Problem… Du hast "(Pickering & Garrod 2004; IAM)" gewollt, aber es scheint, dass das nicht geht und "(Pickering & Garrod 2004: IAM)" ist ein bissche komisch, oder? Ich würde "(Pickering & Garrod 2004) (IAM)" als Alternative vorschlagen, was denkst?
% 
% ^ <cdebenitomoreno@gmail.com> 2018-11-01T10:32:46.319Z.
% * <rivka.eisner@uzh.ch> 2018-09-18T12:38:50.805Z:
% 
% > ;
% Semicolon can be deleted here.
% 
% ^ <cdebenitomoreno@gmail.com> 2018-11-01T09:02:57.271Z.

At first sight, the two models might seem conflicting, because a phenomenon which results from an automatic process is not necessarily assumed to have a social function. However, the two models can also be seen as complementary and, as is for instance common practice in biology, mechanism and function can be studied independently from each other \citep{ruch_function_2018}.


% * <rikvangijn@gmail.com> 2018-06-27T07:57:43.608Z:
% 
% This points towards section 5 if I interpret it correctly. In our chapter, your 3 (patterns and processes) and 4 (methods - called approaches in our chapter) are reversed. It seems feasible to do that for your chapter, too, especially because the discussion on short- versus long-term accommodation gets a natural extension in the methods section.
% 
% ^ <hanna.ruch@uzh.ch> 2018-07-09T14:02:37.358Z:
% 
% I remember that we discussed this and came to the conclusion that it is better to first explain what kind of outcomes there can be before discussing the methods. I'd wait to see the solutions of the other chapters, and then change it accordingly.
% 
% ^.




\section{Factors} \label{Section_Factors}

We will now discuss the findings from the accommodation literature with respect to evidence for linguistic and extralinguistic factors. As much as possible, findings from long-term studies will be compared with those from short-term studies to explore the extent to which short- and long-term accommodation could potentially be based on the same mechanisms and governed by similar constraints.
% * <rivka.eisner@uzh.ch> 2018-09-21T11:32:25.049Z:
% 
% > far
% Consider changing to "much" as in: "As much as possible."
% 
% ^ <cdebenitomoreno@gmail.com> 2018-11-01T09:16:34.709Z.
% * <rikvangijn@gmail.com> 2018-06-27T12:42:14.040Z:
% 
% > We will now discuss the findings from the accommodation literature with respect to evidence for linguistic and extralinguistic factors. We will organize existing studies according to their evidence for linguistic and extralinguistic factors.
% Lot of recurring words within a short space
% 
%
% ^ <hanna.ruch@uzh.ch> 2018-07-09T14:40:37.103Z:
% 
% I deleted the second sentence.
%
% ^ <hanna.ruch@uzh.ch> 2018-08-03T14:43:11.861Z.
\subsection{Linguistic factors}
From a linguistic point of view, accommodation studies seek to answer two important questions. First, what kind of linguistic features are more susceptible to convergence, and second, what factors favor or inhibit this process?
% * <rivka.eisner@uzh.ch> 2018-09-18T13:22:45.397Z:
% 
% >  be converged on
% Would "convergence" be a better choice here?  Then it would read: "more susceptible to convergence, and second..."
% 
% ^ <cdebenitomoreno@gmail.com> 2018-11-01T09:17:02.985Z.
A number of studies have highlighted the role of \textit{salience} in long-term accommodation. Salience can be defined as perceptual conspicuousness of a linguistic element \citep{lenz_zum_2010}. Since it arises in context, it cannot be defined in absolute terms.  Salience of a linguistic element is assumed to be affected by acoustic, cognitive and sociolinguistic factors \citep{auer_anmerkungen_2014}.\footnote{Other work has used so-called subjective criteria to operationalize salience \citep[see examples reviewed in][]{wilson_types_2011,macleod_critical_2015}. Criteria based on the researcher's perspective, however, are problematic because they hinder comparisons across studies, and because salience as perceived by language users themselves, rather than the researcher, is arguably most relevant \citep[see]{macleod_critical_2015}. See Section \ref{Section_Discussion} for further argumentation and examples.}
% * <rikvangijn@gmail.com> 2018-06-27T12:43:38.205Z:
% 
% > Salience can be defined as perceptual conspicuousness of a linguistic element 
% Should it be mentioned here whether this is meant in an absolute sense or in a relative sense?
% 
% ^ <hanna.ruch@uzh.ch> 2018-07-09T14:54:48.422Z:
% 
% I added a sentence to make this clear.
%
% ^ <hanna.ruch@uzh.ch> 2018-08-03T14:43:14.699Z.
Several studies report more convergence toward a second dialect for salient features of the D2 \citep{auer_subjective_1998,pesqueira_cambio_2008,wilson_types_2011,romera_prosodic_2013}. That is, salient features of a variety seem to be more easily picked up by D2 speakers. 
However, convergence for salient features does not always occur and seems to be mediated by social attitudes. For instance, it has been noted that, while D2 stereotypes are rarely adopted (sometimes they are even diverged on), D1 stereotypes are easily abandoned and, consequently, result more easily in convergence \citep{trudgill_dialects_1986,erker_contact_2016}. \citeauthor{Escobar2007}'s \citeyear{Escobar2007} finding, that only syntactic features with low salience have been transferred from (highly stigmatized) Andean Peruvian Spanish to \textit{costeño} Peruvian Spanish, points in the same direction. Research on short-term accommodation is generally consistent with these findings, suggesting that some linguistic features are more easily adopted than others \citep{babel_dialect_2010,walker_repeat_2015}. \citet{babel_dialect_2010} argues that New Zealanders possibly converge less toward Australian KIT and TRAP vowels (/ɘ/ and /ɛ/ in New Zealand, /ɪ̠/ and /æ/ in Australian English) because these are particularly salient Australian features to New Zealanders. Similar arguments can be found in \citet{walker_repeat_2015} for the variable imitation across different vowels and varieties of English. However, in none of these publications is salience quantified empirically, and thus the findings remain speculative. A possible exception is \cite{macleod_effect_2012}, a study that explicitly investigated the role of perceptual salience on short-term accommodation. Salience is assessed here by means of a dialect recognition test, and features contributing more to dialect recognition are considered to be more salient. Interestingly, perceptual salience is able to predict the degree, but not the direction of accommodation, which instead seems to depend on the participants' attitudes toward the interlocutor's dialect and toward the new social environment. % Wilson's (2011) remarks on salience are interesting, but I'm not sure where (or if) to mention them. He notes that salience is a subjective concept defined differently by every author. He operationalizes it on the basis of frequency of the features, its areal distribution and social acceptability, speaker awareness and complexity and creates a sort of "salience value" for each feature. A good point to make could be _not_ to do that: analysing each parameter separately improves comparability across studies and makes it easier to discuss which parameters are more relevant.
% * <rivka.eisner@uzh.ch> 2018-09-21T11:48:55.371Z:
% 
% Suggested change: "In contrast, Kim, Horton & Bradlow (2011) found convergence for some pairs, and divergence for others, between native and non-native interlocutors in English."
% 
% ^ <cdebenitomoreno@gmail.com> 2018-11-01T09:25:50.591Z:
% 
% I can't find where this goes!! Has something disappeared? Can't find anything similar in the last print version I have…
%
% ^ <cdebenitomoreno@gmail.com> 2018-11-01T11:19:49.834Z:
% 
% Found it!
%
% ^ <cdebenitomoreno@gmail.com> 2018-11-01T11:19:51.779Z.
% * <rivka.eisner@uzh.ch> 2018-09-21T11:35:53.171Z:
% 
% Insert commas & word change: "Escobar's 2007 finding, that only syntactic features with ... Peruvian Spanish, points in the same direction."
% 
% ^ <cdebenitomoreno@gmail.com> 2018-11-01T09:19:00.168Z.
% * <rivka.eisner@uzh.ch> 2018-09-18T13:41:21.203Z:
% 
% I notice that when discussing the work of other colleagues, the tense changes from past to present in this paragraph.  It would be best to stay consistent with the tense -- e.g. choose either the present or past tense when referring to other studies.  This note is for this specific location,  but it also applies to the whole chapter. Early on, I noticed the present tense being used to discuss other colleague's work, then in a different section it changed to past tense, and here it is a combination.  Please check and consider changing to one or the other tense for consistency throughout.
% 
% ^ <cdebenitomoreno@gmail.com> 2018-11-01T09:19:31.958Z.
% * <rivka.eisner@uzh.ch> 2018-09-18T13:28:39.662Z:
% 
% > salience was quantified empirically
% Change to: "However, in none of these publications was salience quantified empirically, and thus the findings remain speculative."  
% 
% The sentence could also be broken in two: "However in none of these publications was salience quantified empirically. Thus, the findings remain speculative."
% 
% ^ <cdebenitomoreno@gmail.com> 2018-11-01T09:20:14.518Z.
% * <rivka.eisner@uzh.ch> 2018-09-18T13:26:42.635Z:
% 
% > into
% Change to "in" so that it reads: "points in the same direction."
% 
% ^ <cdebenitomoreno@gmail.com> 2018-11-01T09:20:34.318Z.
% * <max.wahlstrom@helsinki.fi> 2018-06-29T14:31:21.124Z:
% 
% > KIT and TRAP
% Can you spell these out as well! These can be tricky for people outside of English phonetics.
% 
% ^ <hanna.ruch@uzh.ch> 2018-07-09T15:06:28.423Z:
% 
% I added a parenthesis with IPA transcription of the vowels in each variety.
%
% ^ <hanna.ruch@uzh.ch> 2018-08-03T14:43:19.405Z.
% Hanna: he puesto una nota a pie de página más arriba.


Another important factor seems to be \textit{intelligibility}. D1 phonetic features that frequently cause misunderstandings with D2 speakers are more susceptible to accommodation \citep{trudgill_dialects_1986}. \citet{shockey_all_1984}, for instance, observes a higher decrease of /t/-flapping than /d/-flapping in speakers of American English who have moved to Britain, a result that might be explained by the low frequency of /t/-flapping, but not /d/-flapping, in British English. Given that /t/-flapping potentially leads to misunderstandings in British English, American speakers seem to accommodate more easily toward British English for this variable. Similarly, the fact that lexical differences are highly salient and can cause severe and obvious comprehension difficulties \citep{trudgill_dialects_1986} might explain why the lexicon is usually the first linguistic level to be affected by accommodation \citep{Bonomi2010,chambers_dialect_1992}.
Results from short-term studies are generally consistent with these findings. In a dialogue study, \cite{hwang_phonetic_2015} find that non-native speakers of English pronounce plosive and vowel contrasts in a more English-like way in words with a phonological competitor. They interpret this result as evidence for accommodation to the pragmatic needs of the listener.
A seminal study on functional constraints in short-term accommodation was conducted by \citet{nielsen_specificity_2011}. She tests the effect of lengthened and shortened voice onset time (VOT; i.e. amount of aspiration or voicing of a plosive) in /p/ on its imitation. Interestingly, participants imitate lengthened, but not shortened VOT. This result is interpreted with the phonological status of VOT in English: While lengthening VOT (i.e. aspiration) does not have phonological consequences, VOT shortening may lead to a confusion of /p/ with /b/ in minimal pairs such as \textit{pan} vs \textit{ban}.
% * <rikvangijn@gmail.com> 2018-06-27T12:45:22.786Z:
% 
% >  VOT
% Spell out
% 
% ^ <hanna.ruch@uzh.ch> 2018-07-09T15:14:58.124Z:
% 
% OK
%
% ^ <hanna.ruch@uzh.ch> 2018-08-03T14:43:21.779Z.

Yet another linguistic variable that favors imitation is \textit{linguistic variability}. In a comparison between mobile and non-mobile adult speakers of American English, \citet{bowie_effect_2000} finds that, in the long term, phonological variables that are currently undergoing linguistic change are more susceptible to adaptation than more stable features. As for short-accommodation, \citet{watt_levels_2010} observe that an interviewer in the Scottish-English border region is more inclined to converge toward their interviewees for variable than for stable linguistic features. Similar results come from one of the few studies exploring morphological convergence. Using an adapted version of Asch's conformity experiment (1951), \citet{beckner_participants_2016} test whether human participants are influenced in their way of forming the English simple past by human or robotic peers. The participants' morphology is influenced by humans, but not robots, and in verbs with variable past tense formation (e.g. \textit{dream - dreamt/dreamed}) the subjects are more likely to imitate the human peer's choice.
It has also been claimed that free variation  (i.e. altering the pronunciation of one phoneme in every context) is more prone to accommodation than conditioned variation (where the pronunciation of a sound is affected only in some contexts) \citep{trudgill_dialects_1986,siegel_second_2010}. \cite{chambers_dialect_1992} rephrases this constraint by distinguishing between simple and complex phonological rules. Simple rules (such as /t/-voicing in English) are categorical in the sense that they have no exceptions, while complex rules (such as vowel backing in English) do not automatically apply in all contexts. In his study of anglophone Canadian youngsters in the south of England he finds that Canadian /t/-voicing is abandoned faster (implying convergence toward British English) than the British process of vowel backing was acquired. \cite{wilson_types_2011} finds similar results for speakers of Moravian who had moved to Prague and converged to Common Czech, although he notes that rules are seldom without exception and prefers to use the term ``semi-simple rules.''
% * <rivka.eisner@uzh.ch> 2018-09-21T12:28:12.909Z:
% 
% > ''. 
% In other locations, it looks like you are putting punctuation within the quotes.  Please change to: "semi-simple rules."
% 
% ^ <cdebenitomoreno@gmail.com> 2018-11-01T09:27:22.403Z.
% * <rikvangijn@gmail.com> 2018-06-27T12:48:42.887Z:
% 
% > he finds that /t/-voicing was abandoned faster than vowel backing was acquired.
% Here I was a bit confused.  Are these English speaking Canadians? And the two properties, are they part of British English or Canadian English, and then who abandons what and who acquires what?
% 
% ^ <hanna.ruch@uzh.ch> 2018-07-11T08:33:36.754Z:
% 
% Carlota, I added "anglophone" - is this correct?
%
% ^ <cdebenitomoreno@gmail.com> 2018-07-30T16:04:44.093Z:
% 
% Correct, I've also specified what is Canadian and what British, I hope it's not too much of a tongue twister.
%
% ^ <hanna.ruch@uzh.ch> 2018-08-03T14:43:26.975Z.
% * <rikvangijn@gmail.com> 2018-06-27T12:46:49.470Z:
% 
% > changes
% Do you really mean changes here? Or patterns/rules?
% 
% ^ <hanna.ruch@uzh.ch> 2018-07-11T08:35:15.352Z:
% 
% @Carlota, do you mean "categorical variation"? Perhaps you could clarify this by adding an example out of Trudgill (1986)?
%
% ^ <cdebenitomoreno@gmail.com> 2018-07-30T16:22:19.572Z:
% 
% I do mean change (I realised it's Siegel's terminology) – but not in the "language change" sense, but in the "describing a phonetic process" sense, does it work now with the clarifications?
%
% ^ <hanna.ruch@uzh.ch> 2018-08-03T14:43:40.840Z.

There is also evidence that accommodation is affected by \textit{lexical factors}. For instance, for Argentinians who had moved to Mexico City, \citet{pesqueira_cambio_2008} finds more phonetic accommodation in highly frequent words, a result that can be explained by the enhanced degree of exposure for these items. However, in some short-term studies, shadowers are found to converge \textit{less} toward their model speakers in high-frequency words \cite{goldinger_echoes_1998,goldinger_episodic_2004,babel_dialect_2010,nielsen_specificity_2011}. This apparent contradiction between short- and long-term studies can be solved by considering high-frequency words in long-term studies as words that are repeated more often, and therefore, provide the speakers with a higher degree of exposure to these words. The results from short-term studies, in contrast, have been explained by the episodic traces left by the tokens heard, which are assumed to be less influential in high compared to low-frequency words \cite{goldinger_echoes_1998}; an interpretation that is in line with Exemplar Theory \citep{pierrehumbert_exemplar_2001}. However, a recent, very comprehensive study on short-term accommodation \citep{pardo_phonetic_2017} is not able to replicate the main effects of frequency found in earlier work, but instead finds an interaction between speaker gender and word frequency (see below). 
% * <rivka.eisner@uzh.ch> 2018-09-18T13:48:04.157Z:
% 
% >  towards
% Toward is more commonly used in American English.  This is a small note, and you can decide if you wish to change the word here & elsewhere or not.
% *This note applies to the entire chapter.
% 
% ^ <cdebenitomoreno@gmail.com> 2018-11-01T09:27:47.851Z.

In long-term studies, D1 phonetic features have been found to be more likely to persist in words where these features were lexicalized \citep{auer_subjective_1998} or in forms which do not exist in D2 at all \citep{pesqueira_cambio_2008}. Similarly, words that exclusively exist in D2 seem to facilitate the adoption of D2 phonetic features \citep{pesqueira_cambio_2008}. In line with these results, \citet{Bonomi2010} observes that discursive markers and words related to the new cultural reality are adopted first by Spanish-speaking individuals who have migrated from Latin America to Spain and Italy.

In order to become a relevant force driving language change, accommodation not only must show some consistency across speakers, but should also generalize across the lexicon and across different syntactic constructions.
Some evidence for \textit{generalizability} comes from short-term studies. For instance, in her shadowing task, \cite{nielsen_specificity_2011} finds that speakers of American English not only imitate lengthened VOT in items with word-initial /p/, but also generalize this sub-phonemic specificity to new instances of /p/ and even words with initial /k/. \cite{beckner_participants_2016} find that some of their participants generalized the  morphological pattern heard from the model speaker (regular past tense formation in English) to new verbs.

% * <max.wahlstrom@helsinki.fi> 2018-06-29T14:58:25.542Z:
% 
% A lot of things before actually concluding :) Maybe put distance to last. Btw, is this the same as typological distance (discussed in Areas). A brief sentence about the compatibility of the terms?
% 
% ^ <hanna.ruch@uzh.ch> 2018-07-11T08:47:12.685Z:
% 
% Yes, I'd say it's the same, but here applied to dialects. 
% I just realized that we didn't actually discuss levels of linguistic description. To my knowledge, there is no study so far that explicitly compares this...
% 
% ^ <hanna.ruch@uzh.ch> 2018-08-03T14:43:52.288Z.
There is some disagreement on how \textit{linguistic distance} between the systems in contact influences accommodation. \cite{kim_phonetic_2011} find more convergence between speaker pairs of American English who are from largely the same dialect region than between speaker pairs from different dialect regions. On the other hand, \cite{babel_evidence_2012} finds the most convergence for exactly those vowels and participants who differ most from the model speaker. Large phonetic distance between the participants and the model speaker also favor phonetic convergence in a study by \cite{walker_repeat_2015}. The findings mentioned above \citep{beckner_participants_2016,bowie_effect_2000,watt_levels_2010}, that synchronic intra-speaker variability favors convergence, offer yet another interpretation: speakers will more readily take up and use a variant that is a plausible token of their own distribution for the same linguistic variable \citep[for evidence from an agent-based model, see][]{harrington_/u/-fronting_2017}. 
% * <rivka.eisner@uzh.ch> 2018-09-18T13:56:43.548Z:
% 
% > The findings mentioned above \citep{beckner_participants_2016,bowie_effect_2000,watt_levels_2010} that synchronic intra-speaker variability favors convergence offer yet another interpretation: 
% 
% Add in commas here: "The findings mentioned above (...), that synchronic intra-speaker variability favors convergence, offer yet another interpretation:..."
% 
% ^ <cdebenitomoreno@gmail.com> 2018-11-01T09:28:19.464Z.
% * <rivka.eisner@uzh.ch> 2018-09-18T13:54:47.376Z:
% 
% > participants and model speaker
% 
% Consider changing to: "the participants and model speaker."
% 
% ^ <cdebenitomoreno@gmail.com> 2018-11-01T09:28:53.478Z.
% * <rivka.eisner@uzh.ch> 2018-09-18T13:53:47.415Z:
% 
% > found most convergence
% Consider changing to: "found the most convergence"
% 
% ^ <cdebenitomoreno@gmail.com> 2018-11-01T09:29:08.910Z.

While the focus of this chapter is on dialect contact, it is worth mentioning that accommodation has also been found to occur between bilingual speakers with varying degrees of L2 proficiency. Over longer time periods, the predominant linguistic environment has been shown to not only affect a speaker's L2, but also her L1. For instance, in a bilingual speaker of Portuguese and English, VOT is longer or shorter, respectively, after a stay of several months in Brazil or the USA, respectively \citep{sancier_gestural_1997}. \citet{tobin_phonetic_2017} partly replicate these findings for a larger set of Spanish-English bilinguals with Spanish as a dominant language. The speakers' VOT in English voiceless stops drifts toward that of the ambient language (Spanish or English), however, no drift is observed for VOT in Spanish, which is the speakers' L1. 
\cite{chang_rapid_2012} studies American English learners of Korean and finds that already after a few weeks in Korea with intensive Korean classes, the English speakers' L1 is phonetically influenced by the L2. In a subsequent study, \cite{chang_novelty_2013} shows that the phonetic drift toward L2 is less pronounced in more experienced learners.

An interesting aspect of these findings is that the ambient language not only affects the language currently heard and spoken by the speakers, but also their other, ``inactive'' language. These effects on the L1 are often considered cases of linguistic attrition and have been shown to affect all linguistic levels, including morphosyntax. \citet{kaufman_morphological_1991}, for instance, analyze the effect of English on Hebrew in a 2-year-old after moving from Israel to the US. Their longitudinal study shows how Hebrew inflectional and derivational morphology are simplified, resulting in a personal mixed variety \citep[][see also Chapter 5 of this handbook]{kaufman_morphological_1991}.%check if CrossReference to Chapter 5 should be added.
% * <rivka.eisner@uzh.ch> 2018-09-18T14:27:58.435Z:
% 
% > analyse
% Change to: "analyze" (American English spelling).
% 
% ^ <cdebenitomoreno@gmail.com> 2018-11-01T09:29:26.786Z.
% * <rivka.eisner@uzh.ch> 2018-09-18T14:02:01.001Z:
% 
% > Interesting about these findings is that the ambient 
% Consider changing to: "An interesting element within these findings is that..." or "An interesting aspect of these findings is that..."
% 
% ^ <cdebenitomoreno@gmail.com> 2018-11-01T09:29:50.863Z.
% * <rikvangijn@gmail.com> 2018-06-27T12:55:10.364Z:
% 
% > chapter 5
% CHECK this
% 
% ^.
 % see also chapter 5 of this handbook*
 % se refiere esto a Kaufmann & Aronoff o a nuestro libro?
 % Carlota: El nuestro :) 


Short-term studies involving conversations between L2 and L1 speakers are to some extent compatible with these findings. \cite{lewandowski_talent_2012} finds mutual phonetic convergence between German speakers and native speakers of English in English conversations. Interestingly, native English speakers converge even though prior to the dialogue they have been instructed not to do so.
In contrast, \cite{kim_phonetic_2011}, find convergence for some pairs and divergence for others, between native and non-native interlocutors of English. The authors argue that the heavily-accented L2 English of most of their non-native speakers might have enhanced the processing load and therefore inhibited convergence \citep{kim_phonetic_2011}.
% * <rivka.eisner@uzh.ch> 2018-09-18T14:30:35.409Z:
% 
% > have enhanced processing load
% Consider revising to: "might have enhanced the processing load"
% 
% ^ <cdebenitomoreno@gmail.com> 2018-11-01T09:30:26.593Z.
\cite{berry_phonetic_2017} analyze two vocalic contrasts in Spanish and Dutch speakers of English. Prior to the dialogue, Spaniards produce the /\textipa{E}/-/\textipa{\ae}/, but not the /\textipa{i}/-/\textipa{I}/ contrast, while Dutch participants produce the latter, but not the former phonological contrast. During a conversation in English with a Dutch native speaker, Spaniards converge toward their Dutch confederate by merging /\textipa{E}/-/\textipa{\ae}/ and unmerging /\textipa{i}/-/\textipa{I}/.

Taken together, these results suggest that not only categories in an L2 but also in an L1 are more malleable than previously thought. \cite{hwang_phonetic_2015} analyze two phonological contrasts in conversations between Korean speakers of English in a separate collaborative task with a native speaker of English and a partner who speaks English with a heavy Korean accent. Participants converge toward the English native speaker, but only \textit{after} the latter has produced the phonological contrasts of interest. No convergence toward the Korean confederate is observed, however. Based on their results, the authors conclude that accommodation is better explained as as result of priming, not as a way of affiliating with the conversation partner. \cite{kootstra_syntactic_2010} find similar results for Dutch-English bilinguals in situations with code-switching. In an experimental setting, they find that the utterances of the confederate have an effect on the speaker's word order in both their L1 and their L2. While \citeauthor{kootstra_syntactic_2010} interpret their results with the Interactive Alignment Model, they could also be interpreted in terms of CAT (i.e. convergence as an attempt to affiliate with the interlocutor) or in terms of priming.
% * <rivka.eisner@uzh.ch> 2018-09-18T14:32:50.460Z:
% 
% > towards
% Toward (more common in American English).  See also the following sentence & throughout the chapter. 
% 
% ^ <cdebenitomoreno@gmail.com> 2018-11-01T09:31:08.963Z.


\subsection{Extralinguistic factors} 
A common finding of most research on accommodation is that there are important differences between individual speakers in the extent, and sometimes also the direction, of accommodation \citep[e.g. ][]{yu_origins_2013,macleod_effect_2012,babel_evidence_2012,werlen_zwischen_2006,evans_plasticity_2007}. In some cases, these individual differences can be traced back to individual differences in, for instance, attitudes, personality, or exposure to new linguistic environment. In other cases, interaction-related variables are able to explain at least some of the variability. In what follows, we will again compare findings from long-term studies against results from research on short-term accommodation where this is possible.
% * <rivka.eisner@uzh.ch> 2018-09-21T11:53:10.642Z:
% 
% > In the following, 
% Add in a qualifier here -- e.g. "In the following sections" or "section."
% 
% ^ <cdebenitomoreno@gmail.com> 2018-11-01T09:31:37.537Z.

\subsubsection{Speaker-related factors}
% * <max.wahlstrom@helsinki.fi> 2018-06-29T15:02:18.906Z:
% 
% Maybe instead of two subsubsections four: Age, Gender, Language/linguistic attitudes, Interaction-related factors
% 
% ^ <hanna.ruch@uzh.ch> 2018-07-11T08:58:07.438Z:
% 
% I personally prefer to leave it as is: speaker-related factors, interaction-related factors, because age, gender, attitudes are more fine-grained than interaction-related factors. But I like the idea of italicising the factors to give the section more structure.
%
% ^.
There is some evidence for the role of \textit{speaker age} in accommodation. When exposed to a new linguistic environment for a longer time period, children acquire a new dialect faster than adults and, in some cases, they acquire it almost completely \citep{chambers_dialect_1992,siegel_second_2010,tagliamonte_howd_2007}.  \citet{chambers_dialect_1992} distinguish between early and late acquirers. Children younger than 7 are typically early acquirers and reach native-like levels in the second dialect, while adolescents older than 14 are typically late acquirers and will not completely acquire the second dialect. 
% * <rikvangijn@gmail.com> 2018-06-27T13:19:14.711Z:
% 
% > When exposed
% Maybe explicitly mention the factor age before this sentence, or else italicise it.
% 
% ^ <hanna.ruch@uzh.ch> 2018-07-11T09:44:50.996Z:
% 
% OK (both done)
%
% ^ <hanna.ruch@uzh.ch> 2018-08-03T14:44:12.530Z.
In fact, many studies highlight that, similar to second language acquisition, adolescents and adults hardly ever master second dialects \citep{siegel_second_2010}. For his sample of 39 Moravians living in Prague, \cite{wilson_types_2011} reports on only two subjects who acquired native-levels for the phonetic and morphological variables studied. The big majority (36 out of 39) of the participants accommodate to variable extents and one speaker even maintains their accent. These findings are consistent with the differences found between first with second generation migrants in \cite{klee_andean_2006}: While Andean migrants who have move to Lima maintain many of their Andean Spanish features, their Lima-born children are almost indistinguishable from other Limeños \citep[for the linguistic effect of having non-native parents, see][]{payne_factors_1980}.
% * <rivka.eisner@uzh.ch> 2018-09-21T12:31:47.994Z:
% 
% > while Andean migrants 
% Most often in the chapter, a capital letter is used after the colon.  Suggested change here: "While Andean migrants."
% 
% ^ <cdebenitomoreno@gmail.com> 2018-11-01T09:32:16.208Z.
Another example for imperfect acquisition comes from intermediate forms. Sometimes, D1 variants change toward intermediate variants between D1 and D2 (so-called interdialect forms). For instance, \citet{PalaciosAlcaine2007} observes that, after having moved to Madrid, adolescents from Ecuador tend to both abandon the evidential values of their native compound past tenses and to use these tenses more often, as typical for Madrid speech. However, their use still differs from that of Madrid speakers and thus represents a mixed use. 
In some long-term studies, hyperdialectalisms are observed, which can be interpreted as a result of overgeneralization \citep{trudgill_dialects_1986}.
\cite{klee_andean_2006}, for instance, find that some Andean migrants show higher frequencies of /s/-aspiration and and /s/-elision than native Limeños. In line with these results, migrants are commonly perceived to neither speak D1, nor D2 \citep{siegel_second_2010}, but an intermediate or mixed dialect.
Very few studies so far have been concerned with the relationship between age and short-term accommodation. In line  with the age-effects reported for long-term accommodation, \citet{nielsen_phonetic_2014} finds that in a shadowing task, children imitate lengthened VOT to a greater extent than adults. However, more research is needed to understand how short-term accommodation evolves across the life-span and, in particular, in childhood. 
% * <rikvangijn@gmail.com> 2018-06-27T13:22:22.244Z:
% 
% > Compatibly
% In line with?
% 
% ^ <hanna.ruch@uzh.ch> 2018-07-11T09:45:07.281Z:
% 
% OK
%
% ^ <hanna.ruch@uzh.ch> 2018-08-03T14:44:15.169Z.

In the sociolinguistic literature,  \textit{speaker gender} and its relation to linguistic variation has been extensively studied. Women have often been ascribed a crucial role in language change \citep{labov_intersection_1990}, and some long-term studies suggest that women are more prone to converge to a new variety than men.
% * <rikvangijn@gmail.com> 2018-06-27T18:34:58.831Z:
% 
% Here mention the term gender as a factor (and italicise)
% 
% ^ <hanna.ruch@uzh.ch> 2018-07-11T09:45:16.600Z:
% 
% Both done
%
% ^ <hanna.ruch@uzh.ch> 2018-08-03T14:44:17.611Z.
For instance, Argentinean women use a higher percentage of Mexican Spanish phonetic forms than men after residing for several years in Mexico City \citet{pesqueira_cambio_2008} and \citet{MolinaMartos2010} observes that female Latin-American immigrants in Madrid use more European Spanish courtesy forms than men. In the latter study, however, women also show more-negative attitudes toward Madrid speech than their male compatriots. This finding is interpreted as a sign of women attempting to improve their social status by converging toward the local norms.
% * <rikvangijn@gmail.com> 2018-06-27T13:25:10.299Z:
% 
% > treatments
% Constructions? Forms?
% 
% ^ <hanna.ruch@uzh.ch> 2018-07-11T09:46:45.713Z:
% 
% Carlota?
%
% ^ <cdebenitomoreno@gmail.com> 2018-07-30T16:24:38.280Z:
% 
% Replaced by "forms"
%
% ^ <hanna.ruch@uzh.ch> 2018-08-03T14:44:20.892Z.

Gender differences in accommodative behavior have been interpreted in various ways. For instance, \citet[20-21]{giles_accommodation_1991} look at them in the context of social power relations, similar to the situation that salespersons converge more to their clients than vice-versa. \cite{chambers_dialectology_1998} hypothesize that women, perhaps as a result of fewer opportunities for occupational achievement (still relevant today), tend to fulfill a higher number of different social roles than men. As a result, women come into contact with more people within more different social environments, and therefore ``must master a wider repertoire of linguistic variants than men'' \citep[85]{chambers_dialectology_1998}. \citet{willemyns_accent_1997} suggest that gender differences in accommodative behavior may be related to women being more affective than men, and 
% * <rivka.eisner@uzh.ch> 2018-09-20T10:19:53.081Z:
% 
% > As a result, women get in touch with more different people in more different social environments, and therefore 
% 
% Suggested change: "As a result, women come into contact with more people, and more people within different social environments, and therefore"
% 
% ^ <cdebenitomoreno@gmail.com> 2018-11-01T09:33:09.725Z.
% * <rikvangijn@gmail.com> 2018-06-27T13:25:33.839Z:
% 
% > remind
% hypothesize
% 
% ^ <hanna.ruch@uzh.ch> 2018-07-11T09:47:21.579Z:
% 
% OK
%
% ^ <hanna.ruch@uzh.ch> 2018-08-03T14:44:23.218Z.
% * <rikvangijn@gmail.com> 2018-06-27T13:23:05.542Z:
% 
% > ascribed to different factors
% interpreted in various ways?
% 
% ^ <hanna.ruch@uzh.ch> 2018-07-11T09:54:19.697Z:
% 
% Yes, much better!
%
% ^ <hanna.ruch@uzh.ch> 2018-08-03T14:44:26.633Z.
\citet{namy_gender_2002} relate these differences with gender-related differences in sensitivity to indexical variation, that is, systematic linguistic variation associated with extralinguistic factors such as the social background of the speaker or the social context in which the communication takes place . \citet{namy_gender_2002}  assume that differences in sensitivity to indexical variation might themselves be related to social or affiliative motives. \cite{tagliamonte_howd_2007} also observe gender differences in the acquisition of the British English glottal stop by their Canadian youngsters. However, they also note that these differences parallel the sociolinguistic distribution of the variants in the indigenous population. Rather than seeing an effect of the child's gender, they see their results as an example for how children acquire socio-indexical variation.
% * <rikvangijn@gmail.com> 2018-06-27T13:27:15.278Z:
% 
% > indexical variation
% Briefly explain
% 
% ^ <hanna.ruch@uzh.ch> 2018-07-11T10:19:12.287Z:
% 
% I added a sentence.
% 
% ^ <hanna.ruch@uzh.ch> 2018-08-03T14:44:29.378Z.
Two recent, very comprehensive studies \cite{pardo_phonetic_2017,pardo_comparison_2018}, in contrast, are not able to replicate the gender effects reported in earlier studies. Overall, no differences in degree of convergence are observed between women and men. Interestingly, however, women appear to be slightly more sensitive to factors influencing convergence: In \cite{pardo_phonetic_2017}, speaker gender interacts with lexical frequency, with women being more prone to imitate model speakers in low frequency words. The authors suspect that gender effects in earlier shadowing tasks might be driven by the use of low-frequency words in some studies or by individual model speakers. In \cite{pardo_comparison_2018}, which assesses convergence in shadowing tasks \textit{and} conversations, women's accommodative behavior is less consistent across tasks then men's. Again, this result suggests that women are more sensitive to factors that seem to mediate linguistic accommodation.
% * <rivka.eisner@uzh.ch> 2018-09-20T10:26:00.769Z:
% 
% > task
% Suggested change: "tasks" 
% 
% ^ <cdebenitomoreno@gmail.com> 2018-11-01T09:33:30.436Z.

One of the most relevant factors to explain individual variability is a speaker's \textit{attitudes}. Speakers with more favorable attitudes toward a new variety and the receiving community (measured as, for instance, the speakers' willingness to stay or their plans to return) have been found to accommodate to a greater extent than those with less-positive attitudes in several long-term studies \citep{VandenBerg1988,werlen_zwischen_2006,pesqueira_cambio_2008,romera_prosodic_2013,Mick2013,reubold_dissociating_2015}.  
% * <rikvangijn@gmail.com> 2018-06-27T18:35:31.949Z:
% 
% > attitudes
% Italicize
% 
% ^ <hanna.ruch@uzh.ch> 2018-07-11T09:22:31.201Z:
% 
% OK
%
% ^ <hanna.ruch@uzh.ch> 2018-08-03T14:44:32.511Z.
% I would suggest to put this part here because it is on long-term, but it should be better integrated with the paragraph above. % Carlota: I think I wrote this to try to summarize why attitudes affect both long-term and short-term acquisition? The last part of the sentence is making reference to positive attitudes to the specific interlocutor… Lo hablamos cuando nos reunamos, si quieres :) 
Hence attitudes toward one's own and the new linguistic variety seem to play a crucial role in long-term accommodation \citep[see][]{Caravedo2010}. On the one hand, they may affect an individuals willingness to integrate in the receiving community and, on the other hand, these attitudes seem to be related to establishing new social relationships.

Studies on short-term accommodation found comparable results for the role of a speaker's attitudes. \citet{macleod_effect_2012} observes that Argentinian speakers with plans to stay in Madrid are more likely to converge toward a Madrid speaker than those with less-positive attitudes toward their new social environment. However, in this study short-term effects are not easily separable from long-term effects because at the time of the study, the participants had been living in Madrid for different amounts of time. 
Similarly, more-positive attitudes toward the interlocutor lead to more convergence in a number of other studies \citep{yu_phonetic_2013,schweitzer_convergence_2013,babel_dialect_2010,babel_evidence_2012}, or to less divergence in a few others \citep{schweitzer_social_2014}. 



\subsubsection{Interaction-related factors}

Some effects on accommodation have been shown to depend neither on linguistic, nor on speaker-specific factors, but may be better explained by the specific situation in which an interaction takes place. 
For instance, the way a model speaker is presented (either positively or negatively) affects the extent to which participants imitate the model speaker's long VOT in a shadowing task \citep{yu_phonetic_2013}. In an earlier study, however, a similar manipulation did not affect the participants' degree of accommodation \citep{babel_dialect_2010}. The findings mentioned above are generally compatible with long-term studies showing that  positive attitudes toward the new social environment facilitate convergence toward the new linguistic variety \citep{werlen_zwischen_2006,macleod_effect_2012,pardo_phonetic_2012}.
% * <rivka.eisner@uzh.ch> 2018-09-20T10:30:19.067Z:
% 
% > his 
% 
% It is unclear who "his" refers to. Is this the "presenter" in the study?  It would be good to clarify this point here.
% 
% ^ <cdebenitomoreno@gmail.com> 2018-11-01T09:34:08.036Z.

The only investigation so far which directly compared accommodation in shadowing tasks and real interactions \citep{pardo_comparison_2018} finds that the degree of convergence (as assessed by independent listeners in a perception task) is very similar across tasks. Overall, degree of convergence within speakers is not correlated across tasks. A weak correlation between degree of convergence in the two types of tasks is found for male participants only, but not for females. This finding is important because it suggests that results from non-interactive tasks cannot easily be generalized to speech in more natural, interactive settings \citep{pardo_comparison_2018}. 
% * <rivka.eisner@uzh.ch> 2018-09-20T10:34:54.020Z:
% 
% > two type of tasks 
% 
% Make "type" plural here: "two types of tasks."
% 
% ^ <cdebenitomoreno@gmail.com> 2018-11-01T09:34:20.500Z.
% * <rivka.eisner@uzh.ch> 2018-09-20T10:33:19.605Z:
% 
% > that degree of 
% 
% Insert "the" here: "that the degree of."
% 
% ^ <cdebenitomoreno@gmail.com> 2018-11-01T09:34:41.976Z.

Research on dialogues by \cite{pardo_phonetic_2006} and \cite{pardo_phonetic_2013} shows that the specific \textit{communicative role} an interlocutor has in a conversation can also affect accommodation. If convergence was based on exposure alone, we would expect less active dialogue partners to converge less than participants who speak more. However, \cite{pardo_phonetic_2006} and \cite{pardo_phonetic_2013} find that for vowel quality and speech rate, information givers converge more toward information receivers than vice-versa. \cite{pardo_phonetic_2013} explain their findings in the sense of social affiliation: Speakers who are more interested in the information transfer (i.e. the information givers), are more inclined to affiliate with their dialogue partners and therefore converge more.
% * <rivka.eisner@uzh.ch> 2018-09-21T12:01:32.153Z:
% 
% > However \cite{pardo_phonetic_2006,pardo_phonetic_2013} 
% Suggested change: "However, Pardo (2006) and Pardo, Jordan, et al. (2013) found that..."
% 
% ^ <cdebenitomoreno@gmail.com> 2018-11-01T09:39:27.242Z.
% * <rikvangijn@gmail.com> 2018-06-27T18:38:48.241Z:
% 
% > attempted
% tempted? Or perhaps better: inclined
% 
% ^ <hanna.ruch@uzh.ch> 2018-07-11T09:22:48.213Z:
% 
% OK
%
% ^ <hanna.ruch@uzh.ch> 2018-08-03T14:44:36.806Z.

A number of phonetic studies suggest that convergence is contingent on \textit{cognitive load}. \cite{abel_cognitive_2016} find that speakers converge only in a simple, but not a difficult collaborative task, and \cite{berry_phonetic_2017} find more convergence of Spaniards toward Dutch speakers of English in an informal than in a formal situation. Furthermore, convergence is positively correlated with a participant's proficiency in English in this study. These findings suggest that in a situation with lower processing costs, speakers pay more attention to their interaction partner's speech, and therefore are more likely to converge \citep{yu_phonetic_2013,abel_phonetic_2011,berry_phonetic_2017}. 
% * <rivka.eisner@uzh.ch> 2018-09-20T10:43:22.954Z:
% 
% > towards
% Same as previous notes: "Toward" is more commonly used than "towards" in American English.
% 
% ^ <cdebenitomoreno@gmail.com> 2018-11-01T09:38:39.542Z.
% hay algo que se puede escribir sobre la situatión y long-term studies? Estoy pensando en algo como: la situación del migrante en general, si se siente cómodo o no en el nuevo lugar, etc... esto también es algo más bien temporal que podría ir aquí, no?


% no sé si esto no iría mejor en la parte sobre la metodología de long-term studies? Habías escrito algo similar ahí... quizá lo podemos borrar aquí?

 %Although the methodology used in long-term studies is not particularly favorable to study the role of interaction-related factors, they all share the underlying assumption that accommodation is caused through face-to-face contact and hence depend on face-to-face interactions. Thus, long-term studies often highlight the role played by duration of contact (used as a proxy for frequency of interactions) in the degree of accommodation shown by the subjects \citep[see, for instance,][]{shockey_all_1984,PalaciosAlcaine2007,pesqueira_cambio_2008,Mick2013}. Having a life-partner from the receiving community is also sometimes used as a proxy for frequency of interaction \citep{pesqueira_cambio_2008}. 






\section{Discussion and outlook} \label{Section_Discussion}
% * <rikvangijn@gmail.com> 2018-06-27T18:40:31.617Z:
% 
% > Conclusions
% This section is more like a discussion. Perhaps it turns out that this is generally the case in the chapters, and that we should rename these final sections as "discussion".
% 
% ^ <max.wahlstrom@helsinki.fi> 2018-06-29T15:10:10.224Z:
% 
% True, could work for everybody. Hanna L. has Discussion and future directions, which would work here as well. And – I hope I haven't annoyed you by this – some subsections right after the first paragraph could work, e.g.: Key factors in accomodation; Toward a better model for the study of accomodation (or the like...)
%
% ^ <hanna.ruch@uzh.ch> 2018-07-11T09:23:55.496Z:
% 
% From the beginning I was never convinced by having a Conclusion, but no Discussion section, but Conclusions is what we agreed on. I am much more convinced by having Discussion and Future Directions. Let's change it! :-)
% 
% ^ <hanna.ruch@uzh.ch> 2018-08-03T14:44:45.198Z.

The main aim of this chapter has been to compare short- and long-term accommodation and to discuss their relevance to the change-by-accommodation model. We will start by summarizing our findings to then formulate new research questions and highlight promising areas for future research.
Among the linguistic factors affecting accommodation, both intelligibility and linguistic variability show consistent results between long- and short-term studies. According to the reviewed literature, linguistic features that hinder intelligibility as well as features exhibiting synchronic variation are accommodated faster than other linguistic features. The effect of lexical factors such as word-frequency appeared to differ between long-term and short-term studies. While low-frequency words facilitate convergence in short-term studies, long-term studies find that more frequent words were more prone to convergence. This apparent contradiction can be resolved by considering degree of exposure: Over a longer time period, high-frequency words are the ones that speakers are more likely to hear. In short-term studies, in particular, in experimental settings, the degree of exposure is more stable between words. The negative effect of high lexical frequency on short-term accommodation can be explained with Exemplar Theory, which states that more frequent forms are more entrenched and therefore less influenced by recently heard forms. 

On the other hand, while salience is one of the most-studied factors in the accommodation literature, the many different approaches to the concept prevent a direct comparison between different studies, both across and within long- and short-term accommodation. \cite{auer_anmerkungen_2014} distinguishes three types of criteria that contribute to the perceptual salience of a linguistic feature: acoustic-auditory factors, cognitive factors and sociolinguistic factors. Given that these factors are not independent from each other (e.g. a longer, acoustically salient vowel is more prone to acquire sociolinguistic salience), different aspects of salience are hard – if not impossible – to operationalize. It seems to us that a more fruitful approach to the study of salience would entail a listener-based approach \citep{macleod_critical_2015,ruch_role_2018}. Instead of estimating salience based on theoretical criteria from a researcher's perspective \citep{auer_subjective_1998,trudgill_dialects_1986}, listener-based approaches work with experiments or questionnaires. For instance, \cite{ruch_role_2018} uses a perception experiment to operationalize the salience of phonetic features in two Swiss German dialects. Native listeners of Grison and Zurich German were asked to identify the dialect of spoken isolated words which contained different segmental cues to one of the two dialects. By measuring sensitivity and reaction time it is possible to order the different segments according to their salience. The most salient dialect features are also the ones people from all over German-speaking Switzerland most frequently mention when asked to describe the dialects in an online questionnaire. This suggests that a first and feasible approach to learn about salient features of a variety is by asking (naive) listeners to describe  how they recognize speakers of the variety in question. 
% * <rivka.eisner@uzh.ch> 2018-09-21T12:06:37.954Z:
% 
% > Last,
% Suggested change: perhaps use "finally" here?  "Last" would imply a list, but I do not see where there has been a list of element/themes, etc. in the prior sentences.  
% Also: This is an instance in the chapter where the paragraph is very long. Here and in other areas where there are very long paragraphs, I would suggest looking for places where it makes sense to divide the long paragraph into 2-3 shorter ones.  
% 
% ^ <cdebenitomoreno@gmail.com> 2018-11-01T09:42:17.852Z.
% * <cdebenitomoreno@gmail.com> 2018-07-30T16:27:52.661Z:
% 
% > the cantons of Zurich and Grisons
% Where they were from was missing, I added this, is it correct?
% 
% ^ <hanna.ruch@uzh.ch> 2018-07-31T07:28:08.817Z:
% 
% Thanks!
%
% ^ <hanna.ruch@uzh.ch> 2018-08-03T14:44:48.820Z.

% * <rikvangijn@gmail.com> 2018-06-27T18:57:28.045Z:
% 
% > agglutinating
% considering?
% 
% ^ <hanna.ruch@uzh.ch> 2018-07-11T09:32:48.416Z:
% 
% OK
%
% ^ <hanna.ruch@uzh.ch> 2018-08-03T14:44:51.537Z.
% En qué aspectos de saliencia has pensado? Creo que habría que explicar un poco más. No estoy segura si esto funciona - o asea, si realmente se pueden separar estos aspectos - porque unos aspectos de saliencia dependen de otros... Pero lo que podríamos quizá proponer es "an experimental approach to salience, in which salience is operationalized by experimentally measuring its contribution to dialect recognition (MacLeod 2014)". 
% * <rikvangijn@gmail.com> 2018-06-27T18:58:15.074Z:
% 
% I agree with the comment here, it would be nice if you could just say a little more about how this breaking down of salience would look, maybe giving an example of one aspect that one could try and isolate is enough already.
% 
% ^ <hanna.ruch@uzh.ch> 2018-07-11T12:31:42.135Z:
% 
% I am not vey convinced that "breaking down salience" really works. I added a couple of sentences to explain why - and to suggest an alternative approach.
%
% ^ <hanna.ruch@uzh.ch> 2018-08-03T14:44:55.646Z.
%Algunos que encuentro en la bibliografía: phonetic and perceptual distinctness between varaints of a variable, lexicalization, mutual intelligibility; stigmatization, linguistic change, phonological contrast, frequency, areal distribution, social acceptance, speaker awareness, complexity, continuous vs. dichotomous, useage in code-aternation, representation un lay dialect writing, stereotyping/mimicking (a partir de Wilson 2011 y Auer/Barden/grosskpf 1998) 

Among the extra-linguistic factors, attitudes and age show the most consistent effects between long- and short-term studies. More positive attitudes toward the contact variety and a younger age seem to facilitate convergence toward a different dialect. However, more research is needed on children and adolescents, for whom short-term accommodation is still underresearched. The role of gender, in contrast, is controversial in accommodation. Some studies find that women converge more than men, in both short- and long-term. However, such gender differences in accommodative behavior surface only in few studies. Furthermore, recent research has not been able to replicate gender differences from earlier research. The few studies investigating the role of cognitive load so far find that accommodation is more likely to occur when cognitive load is lower, however, more research is needed to confirm these effects. To our knowledge, the role of cognitive load in long-term accommodation has not been studied to date. A possible way to address this issue is by conducting a longitudinal study with several sessions over a longer time period. In these sessions, participants can be exposed to a model speaker in two different conditions: One in which the participants solve an easy task and one in which they solve a difficult task and therefore have fewer cognitive resources to attend to the model's speech \citep[see][]{abel_effect_2015}. The hypothesis to be tested is that speech heard while solving an easy task will leave more traces over the long-term than speech heard while solving a difficult task.
% * <rivka.eisner@uzh.ch> 2018-09-20T10:54:41.955Z:
% 
% > One in which the participants solve an easy, and one in which they solve a difficult task and therefore have fewer cognitive resources to attend to the model's speech 
% 
% Suggested change: "One in which the participants solve an easy task, and one in which they solve a difficult task and therefore have fewer..."
% 
% ^ <cdebenitomoreno@gmail.com> 2018-11-01T09:42:43.494Z.
% * <rivka.eisner@uzh.ch> 2018-09-20T10:49:52.300Z:
% 
% > towards
% Same note as above for the word "towards" (2x in this sentence).  If you wish to change "towards" to "toward" throughout the chapter, it would be helpful to refer back to the  word document so you can use the search function to find each instance of the word's use. 
% 
% ^ <cdebenitomoreno@gmail.com> 2018-11-01T09:43:37.482Z.
% * <rikvangijn@gmail.com> 2018-06-27T19:16:07.576Z:
% 
% > A possible way to address this issue could be 
% A possible way (...) is or 
% A way (...) could be
% 
% ^ <hanna.ruch@uzh.ch> 2018-07-11T09:34:36.626Z:
% 
% OK
%
% ^ <hanna.ruch@uzh.ch> 2018-08-03T14:44:58.092Z.
% * <rikvangijn@gmail.com> 2018-06-27T19:13:12.793Z:
% 
% > First research
% Initial, preliminary? Early? Not exactly sure what is meant here.
% 
% ^ <hanna.ruch@uzh.ch> 2018-07-11T09:40:29.632Z:
% 
% I changed it to "The few studies"
%
% ^ <hanna.ruch@uzh.ch> 2018-08-03T14:45:00.787Z.
% * <rikvangijn@gmail.com> 2018-06-27T19:09:13.395Z:
% 
% >  On the contrary, the role of gender in accommodation is controversial. 
% The role of gender, in contrast, is controversial
% 
% ^ <hanna.ruch@uzh.ch> 2018-07-11T12:43:48.587Z:
% 
% OK
%
% ^ <hanna.ruch@uzh.ch> 2018-08-03T14:45:02.374Z.
% * <rikvangijn@gmail.com> 2018-06-27T19:08:05.042Z:
% 
% > Within
% Among
% 
% ^ <hanna.ruch@uzh.ch> 2018-07-11T12:46:08.211Z:
% 
% You mean "Within the extra-linguistic factors" -> "Among the extra-linguistic factors" , right?
%
% ^.

From our literature review, several gaps within accommodation research have become evident, which open up the way for new research directions. In particular, the relationship between short- and long-term accommodation, as well as their role in models of language change, remain speculative. First, in long-term accommodation the focus so far has been on migrant communities. However, in order to shed light on how accommodation may drive linguistic change, studying the receiving community is as essential as investigating migrating individuals. Second, in both short- and in long-term studies the focus has been on adults, who typically show an imperfect acquisition of a new variety. The role of children, who are faster and more complete acquirers of new varieties (and languages) deserves more attention too, and should be better integrated in the change-by-accommodation model. % Sollten wir das hier nicht etwas genauer ausführen? Vielleicht einen Teil des Textes von oben (Model) hierher verschieben?
% * <rivka.eisner@uzh.ch> 2018-09-20T11:08:54.558Z:
% 
% >  a 
% Would it be "a" or "the" here?  Is there one primary "change-by-accommodation model"? The call for "integration" suggests that it may be "the" -- e.g. integration into a specific model (or series/set of models).   I defer to your knowledge of the field here to determine what is correct. 
% 
% ^ <cdebenitomoreno@gmail.com> 2018-11-01T09:44:41.140Z.
% * <rivka.eisner@uzh.ch> 2018-09-20T11:06:36.651Z:
% 
% > show imperfect acquisition of a new variety
%  
% Perhaps add in "an" here: "show an imperfect acquisition"?
% 
% ^ <cdebenitomoreno@gmail.com> 2018-11-01T09:45:06.339Z.
% * <rivka.eisner@uzh.ch> 2018-09-20T11:02:38.828Z:
% 
% > the migrant community.
% Suggested change: "migrant communities."
% Seems like it should be a general reference to migrant communities rather than to a specific migrant community. If it is a specific one, then include the name of/distinguishing factors of this group (e.g.  Peruvians in Spain or the Peruvian migrant community in Madrid).
% 
% ^ <cdebenitomoreno@gmail.com> 2018-11-01T09:45:30.144Z.
% * <rivka.eisner@uzh.ch> 2018-09-20T10:59:43.441Z:
% 
% > In particular, the relationship between short- and long-term accommodation as well as their role in models of language change remain speculative. 
% Suggested change (2 places to insert commas): "In particular, the relationship between short- and long-term accommodation, as well as their role in models of language change, remain speculative."
% Also: would "roles" be plural here?  Is there more than one role for short- and long-term accommodation in this instance?
% 
% ^ <cdebenitomoreno@gmail.com> 2018-11-01T09:46:08.565Z.
% * <rivka.eisner@uzh.ch> 2018-09-20T10:57:36.317Z:
% 
% > gaps of accommodation research
% 
% Suggested change: "gaps within accommodation research."
% 
% ^ <cdebenitomoreno@gmail.com> 2018-11-01T09:46:23.020Z.
Third, to better understand linguistic accommodation, its underlying mechanism and its ultimate social function, a broader set of languages needs to be studied. As is evident from the current literature review, research on accommodation so far has mostly focused on well-known Indo-European languages and western communities. Similarly, work on accommodation has typically dealt with phonetics and phonology (especially in short-term studies). More research on different linguistic phenomena and, in particular, direct comparisons between different linguistic levels is crucial to shed light on the mechanisms and constraints of accommodation.
% * <rikvangijn@gmail.com> 2018-06-27T19:23:47.466Z:
% 
% > their underlying mechanism, and their ultimate social functions,
% If "their" refers to linguistic accommodation (which would make sense), it should be "its"
% 
% ^ <hanna.ruch@uzh.ch> 2018-07-11T09:42:31.187Z:
% 
% OK
%
% ^ <hanna.ruch@uzh.ch> 2018-08-03T14:45:13.083Z.
Last, the striking methodological differences between short- and long-term studies hinder comparability between them. Short-term accommodation research, which typically relies on experimental settings, could benefit from more interactive settings fostering spontaneous speech in order to study social factors. This is of particular importance because, as mentioned above, accommodative behavior across tasks is not necessarily the same for the same speaker \citep{pardo_comparison_2018}. Similarly, long-term studies, which so far have mostly relied on sociolinguistic interviews, should use more controlled settings too, to allow for comparability across subjects and  with non-migrant control groups. Finally, longitudinal studies will be crucial to offer a more accurate picture of accommodation over longer periods of time. So far, time of exposure has been studied by comparing different individuals. However, given the large inter-speaker variability that pervades published accommodation research, longitudinal studies with data from the same speakers across time are crucial to understand accommodation and, in particular, the role of exposure.
% * <rivka.eisner@uzh.ch> 2018-09-21T12:16:26.666Z:
% 
% > At the same time
% Could a different phrase be used here to avoid the use of the word "same" (since it is used 2x in the preceding sentence)?  What about: "Similarly," or "Accordingly," or "In a similar light," etc.?
% 
% ^ <cdebenitomoreno@gmail.com> 2018-11-01T09:50:36.300Z.
% * <rivka.eisner@uzh.ch> 2018-09-20T11:15:02.326Z:
% 
% > in the long run.
% Suggested change: "over time" or "over longer periods of time."
% 
% ^ <cdebenitomoreno@gmail.com> 2018-11-01T09:50:52.715Z.
% * <rikvangijn@gmail.com> 2018-06-27T19:30:13.823Z:
% 
% From the perspective of linguistic areas, it would also be nice to have more studies of language pairs rather than dialect pairs, and perhaps also situations involving more than two varieties.
% 
% ^ <hanna.ruch@uzh.ch> 2018-07-11T09:43:28.837Z:
% 
% This is perhaps something for the general discussion at the end of the book?
%
% ^.



\section*{Abbreviations}
\section*{Acknowledgements}

\printbibliography[heading=subbibliography,notkeyword=this]
% * <max.wahlstrom@helsinki.fi> 2018-06-28T14:25:08.236Z:
% 
% Re: the references. To keep things simple, remove DOIs, and give URL only in case the source is available only online, or the online version differs from the printed version (most importantly having different page enumeration).
% 
% ^ <hanna.ruch@uzh.ch> 2018-07-11T15:17:50.001Z:
% 
% Done!
%
% ^ <hanna.ruch@uzh.ch> 2018-08-03T14:45:28.458Z.

\end{document}
