\documentclass[output=paper]{LSP/langsci}
\author{Elisabet Engdahl\affiliation{University of Gothenburg}
}
\title{Expletive passives in Scandinavian -- with and without objects}
% \epigram{Change epigram}
\abstract{\citet{Holmberg2002Expletives} proposes an account for the variation concerning expletives, participial agreement and word order in periphrastic passives in the Mainland Scandinavian languages in terms of parameters. In this short article, the predictions of Holmberg’s proposal are evaluated against a corpus study of expletive passives. It turns out that only Norwegian 1 (\textit{bokmål}) behaves as expected given Holmberg’s parameter settings; it lacks participle agreement and only displays the PCP DO word order, with few exceptions. Danish, which has the same parameter settings as Norwegian 1, is shown to have had the DO PCP order in earlier stages and this order is still used in many dialects. Norwegian 2 (\textit{nynorsk}) and Swedish are predicted to allow both the PCP DO order and the DO PCP order, but it is shown that Norwegian 2 uses the same order as Norwegian 1, PCP DO, whereas Swedish – to the limited extent that the periphrastic passive is actually used in expletive passives – uses the DO PCP order. In both Danish and Swedish, the DO PCP order is facilitated by an incorporated negation in the DO, just as in active clauses, a fact that should presumably be reflected in the analysis.
}
\ChapterDOI{10.5281/zenodo.1116777}

\maketitle

\begin{document}



\section{Introduction}%1
\largerpage

The interplay between \isi{agreement} and word order in expletive \isi{passive} constructions in \ili{Mainland Scandinavian} has received considerable attention starting with \cite{ChristensenTaraldsen1989}. At first glance, the pattern seems quite clear: when the direct object (DO) precedes the participle (PCP), the latter shows \isi{agreement}, but when the PCP precedes the DO, the form of the PCP is consistently neuter singular, as shown by the \ili{Swedish} examples in \REF{ex:engdahl:1}.\footnote{I follow \citet[104]{Holmberg2002Expletives} in glossing the expletive subject as \textsc{ex} and non-\isi{agreeing} participles simply as \textsc{pcp}. Agreeing participles are glossed as \textsc{c} for common \isi{gender} singular, \textsc{n} for neuter singular. The \isi{gender} distinction is neutralised in the plural, glossed \textsc{pl}. 
}

% \settowidth\jamwidth{(Sw.)}
\ea%1
    \label{ex:engdahl:1}
    \ili{Swedish} (\citealt{Holmberg2002Expletives}: 86)
\ea\label{ex:engdahl:1a}
\gll Det \textbf{blev}     \textbf{skrivet} /     *\textbf{skrivna}     tre     böcker   om     detta. \\
  \textsc{ex}   became written.\textsc{n} /    written\textsc{.pl}   three books     about   this \\
\glt ‘Three books were written about this.'
\ex\label{ex:engdahl:1b}
\gll    Det     \textbf{blev}     tre     böcker \textbf{\textit{*skrivet}} /     \textbf{\textit{skrivna}}     om     detta.\\
  \textsc{ex}   became three  books     written.\textsc{n} / written.\textsc{pl} about  this \\
  \glt ‘Three books were written about this.'
\z
\z

% \settowidth\jamwidth{(Sw.)}
% \ea%1
%     \label{ex:engdahl:1}
% \ea\label{ex:engdahl:1a}
% \gll Det \textbf{blev}     \textbf{skrivet} /     *\textbf{skrivna}     tre     böcker   om     detta. \\
%   \textsc{ex}   became written.\textsc{n} /    written\textsc{.pl}   three books     about   this \\\jambox{(Sw.)}
% \ex\label{ex:engdahl:1b}
% \gll    Det     \textbf{blev}     tre     böcker \textbf{\textit{*skrivet}} /     \textbf{\textit{skrivna}}     om     detta.\\
%   \textsc{ex}   became three  books     written.\textsc{n} / written.\textsc{pl} about  this \\
% \z
% \z
\noindent In his detailed study of these constructions, \citet{Holmberg2002Expletives} proposes several parameters in order to account for the variation. One parameter determines whether or not the expletive and the participle have ϕ-features.\footnote{See the helpful survey in the Appendix (\citealt[125f]{Holmberg2002Expletives}).
} In \ili{Swedish}, both the expletive \textit{det} (‘it’, neut. sing.) and the participle are assumed to have ϕ-features. Consequently the participle can agree either with the expletive or with the DO and both orders are possible, as shown in \REF{ex:engdahl:1}. In \ili{Danish}, both the expletive \textit{der} (‘there’) and the participle lack ϕ-features and only the PCP DO order should be possible, see \REF{ex:engdahl:2} (cf. Holmberg p. 104). \ili{Norwegian} displays more variation; the \textit{bokmål} varieties (Holmberg’s \ili{Norwegian} 1) use \textit{det} as expletive and lack participle \isi{agreement} \REF{ex:engdahl:3}, whereas the \textit{nynorsk} varieties (Holmberg’s \ili{Norwegian} 2) have \isi{agreeing} participles \REF{ex:engdahl:4} and hence are predicted to allow the order DO PCP.\footnote{In addition Holmberg identifies a third variety, \ili{Norwegian} 3, which uses the locative expletive \textit{der} but has participle \isi{agreement}. He also notes that there is actually more dialectal variation in Norway. This is confirmed in a recent study by \citet{AaEtAl2014}.}


\ea%2
    \label{ex:engdahl:2}
    \ili{Danish} (\citealt{Holmberg2002Expletives}: 104)
\ea[]{
\gll  Der      \textbf{blev}  \textbf{skrivet}  tre     bøger   om     dette.\\
 \textsc{ex}    became written.\textsc{pcp} three  books  about  this \\ 
 \glt ‘Three books were written about this.'}
\ex[*]{
\gll    Der   \textbf{blev}  tre     bøger \textbf{skrivet}  om     dette.\\
 \textsc{ex}     became three  books written.\textsc{pcp}  about  this\\}
\z
\z


\ea%3
    \label{ex:engdahl:3}
    \ili{Norwegian} 1 (cf. \citealt{Holmberg2002Expletives}: 104)
\ea[]{\label{ex:engdahl:3a}
\gll  Det   \textbf{ble}  \textbf{skrevet}  tre     bøker   om     dette.\\
  \textsc{ex}     became written.\textsc{pcp} three  books  about  this\\
  \glt ‘Three books were written about this.'}
\ex[*]{\label{ex:engdahl:3b}
\gll    Det   \textbf{ble}  tre     bøker \textbf{skrevet}  om     dette.\\
    \textsc{ex}  became three  books written.\textsc{pcp}  about  this\\}
\z
\z


\ea\label{ex:engdahl:4}
\ili{Norwegian} 2 (cf. \citealt{Holmberg2002Expletives}: 104)
\ea
\gll Det   \textbf{vart}  \textbf{skrive} /     *\textbf{skrivne}  tre  bøker   um     dette.\\
   \textsc{ex}     became   written.\textsc{n} /  written.\textsc{pl}    three    books  about  this \\
   \glt ‘Three books were written about this.'
\ex
\gll Det   \textbf{vart}  tre     bøker *\textbf{skrive} /     \textbf{skrivne}     um     dette.\\
 \textsc{ex}     became three  books   written.\textsc{n} / written.\textsc{pl} about  this\\
 \glt ‘Three books were written about this.'
\z
\z

\noindent Another parameter proposed by \citet[106f]{Holmberg2002Expletives} is whether the \isi{Participle} Phrase (PrtP) is a phase or not, in the sense of \citet{Chomsky2001}.\footnote{This parameter is necessary in order to account for the word order and \isi{agreement} patterns in corresponding structures in English and \ili{Icelandic}, see \citet[105]{Holmberg2002Expletives}.} In \ili{Norwegian} 2 and \ili{Swedish}, where PrtP is assumed to be a phase, the participle is “formally stronger” and the PrtP is “more sentence-like” than in \ili{Danish} and \ili{Norwegian} 1. If the PrtP is not a separate phase, examples like \REF{ex:engdahl:3a} in \ili{Norwegian} 1 will consist of a single array with the expletive merged with VP, shown in \REF{ex:engdahl:5a} before spell-out and spelled out as \REF{ex:engdahl:5b}.

\ea%5
\label{ex:engdahl:5}
\ili{Norwegian} 1 (\citealt{Holmberg2002Expletives}: 106)
\ea \label{ex:engdahl:5a}
C [\textsubscript{TP} det T [\textsubscript{AuxP} bli [\textsubscript{PrtP} t Prt [\textsubscript{vP} V DP]]]]\\
\ex \label{ex:engdahl:5b}
\gll Det  \textbf{ble}   \textbf{skrevet} mange    bøker.\\
    \textsc{ex}    became  written.\textsc{pcp}  many    books\\
    \glt ‘Many books were written.'
\z
\z

\noindent If PrtP is a separate phase, as in \ili{Swedish}, the lexical array is divided into two subarrays, according to \citet[106]{Holmberg2002Expletives}. One contains C, T and the \isi{auxiliary} and the other contains the participle, V and the DP. The expletive may belong to either array, which accounts for the two word orders. If the expletive belongs to the second subarray, the derivation will be as in \REF{ex:engdahl:5}, but if it belongs to the first subarray, the DP object has to move to SpecPrtP in order to satisfy the EPP-feature on the head. Holmberg’s illustration is given in \REF{ex:engdahl:6} (cf. the \ili{Swedish} example in \REF{ex:engdahl:1b}).


\ea%6
    \label{ex:engdahl:6}
    \ili{Swedish} (\citealt{Holmberg2002Expletives}: 107)
\ea \label{ex:engdahl:6a}
C [\textsubscript{TP} det T [\textsubscript{AuxP} t bli [\textsubscript{PrtP} DP Prt [\textsubscript{vP} V t]]]]\\
\ex \label{ex:engdahl:6b}
\gll Det  \textbf{blev} många  böcker    \textbf{skrivna}.\\
   \textsc{ex}    became  many  books    written\textsc{.pl}\\
\glt ‘Many books were written.'
\z
\z


\noindent In this article I show that the pattern of variation is more complex than assumed by Holmberg and that other factors need to be taken into account, in particular whether or not the object has an incorporated \isi{negation}.

\section{Transitive expletive constructions, word order and agreement}%2

Before discussing to what extent the patterns shown in (\ref{ex:engdahl:1})--(\ref{ex:engdahl:4}) reflect the ways expletive passives are used, a few words about the distribution of the two \isi{passive} forms in \ili{Mainland Scandinavian} are in order, viz. the periphrastic and the morphological \isi{passive}. For obvious reasons, \citet{Holmberg2002Expletives} limits his discussion to periphrastic passives, i.e. passives formed with an \isi{auxiliary} and a participle, as shown in (\ref{ex:engdahl:1})--(\ref{ex:engdahl:4}).\footnote{In \ili{Danish}, \ili{Norwegian} 1 and \ili{Swedish}, the \isi{auxiliary} is \textit{bli} ‘become’ (\textit{blive} in \ili{Danish}); in \ili{Norwegian} 2 and some \ili{Swedish} dialects, the preferred \isi{auxiliary} is \textit{varda} ‘become’.} The morphological \isi{passive} is formed by adding -\textit{s} to the infinitive or the tensed form of the verb. The choice of \isi{passive} form – periphrastic \isi{passive} or \textit{s}{}-\isi{passive} – depends on several factors such as genre, tense, mood, animacy of the subject, control, event structure and to some extent lexical preferences (see \citealt{Sundman1987}, \citealt{Engdahl1999,Engdahl2006} and \citealt[47--61]{Laanemets2012} for overviews and \citealt{DeCuypereEtAl2014} for a multivariate statistical analysis). The data in the next three subsections come from the extensive corpus study in \citet{Laanemets2012}, complemented by some specific searches for \isi{impersonal passive}s.\footnote{\citet{Laanemets2012} extracted \textit{s}{}- and \textit{bli(ve)}{}-passives from comparable written and spoken corpora in \ili{Danish}, \ili{Norwegian} (\textit{bokmål}) and \ili{Swedish} and annotated around 11 300 \isi{passive} examples.}

\subsection{Swedish} %2.1

In \ili{Swedish} there is a clear preference for the \textit{s}{}-\isi{passive} in general; \textit{s}{}-\isi{passive} is used in 97\% of all \isi{passive} verb phrases in written texts (newspapers and novels) and in 85\% of all \isi{passive} phrases in informal conversations \citep[92]{Laanemets2012}. This also applies to transitive expletive passives; only 1–3\% are \textit{bli}{}-passives, varying somewhat with genre.\footnote{\citet[Chapter 3]{Hedlund1992} discusses \textit{bli}{}-passives without mentioning their limited distribution. Periphrastic passives with \textit{få} ‘get’ are discussed in \citet{Larsson2012}.} This means that Holmberg’s examples in \REF{ex:engdahl:1} are rather unusual. The normal way of conveying this message in \ili{Swedish} would be with an \textit{s}{}-\isi{passive} as in \REF{ex:engdahl:7}.

\ea%7
    \label{ex:engdahl:7}
    \ili{Swedish}\\
\gll Det   har   \textbf{skrivits}     tre  böcker   om     detta.\\
     \textsc{ex} has    written.\textsc{s}    three    books    about  this\\
     \glt ‘Three books have been written about this.'
\z

Among the 3176 \ili{Swedish} \isi{passive} examples analysed by Laanemets, there were 108 \isi{impersonal passive}s with expletive subjects and of these only three were transitive \textit{bli}{}-passives. One example from spoken \ili{Swedish} is shown in \REF{ex:engdahl:8}.

\ea%8
    \label{ex:engdahl:8}
    \ili{Swedish}, spoken (\citealt{Laanemets2012})\\
\gll men jag har   en   känsla   av att det \textbf{blir}  inte   någonting \textbf{gjort}  där   ändå \\
     but   I    have  a    feeling    of that \textsc{ex} becomes not something done\textsc{.n} there still\\
    \glt ‘but I have a feeling that  still nothing gets done there’
\z

\noindent All three examples had the word order DO PCP. They resemble the authentic examples in \REF{ex:engdahl:9}.

\ea%9
    \label{ex:engdahl:9}
    \ili{Swedish} \citep[31]{Engdahl1999}
\ea
\gll Det \textbf{blev}     inte   så mycket \textbf{sagt}     kanske.\\
     \textsc{ex} became not so  much\textsc{.n}   said\textsc{.n}   maybe\\
     \glt ‘Not much was  said, maybe'.
\ex
\gll     Men då     \textbf{blev}  det   ingenting    \textbf{gjort}.\\
    but    then  became \textsc{ex}     nothing\textsc{.n}      done\textsc{.n}\\
    \glt ‘But nothing got done then.'
\z
\z

\noindent The examples in \REF{ex:engdahl:8} and \REF{ex:engdahl:9} sound quite natural, unlike \REF{ex:engdahl:1}. Note that they all contain a negative element, either the \isi{negation} \textit{inte} ‘not’ or \textit{ingenting} ‘nothing’\textit{.} In order to find a wider range of examples, Anu Laanemets and I carried out a search in an 800 million subcorpus of \textit{Korp,} looking for instances of this pattern, i.e. \textit{det}, followed or preceded by a form of the lemma BLI, with an optional adverb or \isi{negation}, a quantifying pronoun or numeral, a noun and a participle.%
\footnote{We searched in newspapers, novels and blogs using the schematic search string in (i):
\ea%
\texttt{%
\{det BLI $\mid$ BLI det\}  []\{0,1\} \{INGEN  $\mid$ MYCKEN  $\mid$ MÅNGEN  $\mid$ NÅGON  $\mid$ artikel  $\mid$ pronomen  $\mid$ grundtal \} []\{0,1\} PCP ej-NN
}
\z
See \citet{EngdahLaanemetsl2015opersonlig} for details about the corpus searches.
}
The search produced 283 examples which gives us a relative frequency of 0.4 per million words. This can be compared to transitive expletive \textit{s}{}-passives as in \REF{ex:engdahl:7} which were used around 50 times per million words in the same corpora, i.e. a hundred times more often. 

Some representative examples from the corpus search are given in \REF{ex:engdahl:10}.\footnote{The whole dataset with our annotations is available: \url{https://svn.spraakbanken.gu.se/sb-arkiv/pub/engdahl/Opersonlig\_passiv}.} The participle agrees with the preceding DO, as predicted.

\ea%10
\label{ex:engdahl:10}
\ili{Swedish} (\textit{Korp})
\ea \label{ex:engdahl:10a}
\gll  Det \textbf{blev}     ingen   post     \textbf{utdelad}   alls     igår.\\
   \textsc{ex} became no post.\textsc{c}  distributed.\textsc{c}  at-all  yesterday\\
   \glt ‘No post  whatsoever  was  distributed  yesterday.'
\ex \label{ex:engdahl:10b}
\gll Jag sitter där   vid   datorn   och   ska   skriva,   jag   vet   vad   jag   ska   göra men det \textbf{blir}  ändå inget  \textbf{gjort}.  \\
    I    sit  there  by  computer.\textsc{def} and shall  write I know  what  I shall  do    {but} \textsc{ex}  becomes still  nothing\textsc{.n}  {done}\textsc{.n}\\
\glt ‘I sit there in front of the computer, about to write, I know what I should do, but still nothing gets done.’
\ex \label{ex:engdahl:10c}
\gll Utan     deras   försörjning   och   rimliga  villkor,  \textbf{blir} det inga filmer \textbf{gjorda}, inga böcker \textbf{skrivna},     inga låtar \textbf{komponerade}.\\
 without  their   support  and reasonable  conditions become \textsc{ex}  no    films    made.\textsc{pl}  no    books  written.\textsc{pl}  no    songs  composed.\textsc{pl}\\
 \glt ‘Without their support and reasonable conditions, there won't be any films made,  books written or songs composed.'
\z
\z

\noindent This type of expletive \isi{passive} is used primarily when an expected result does not occur: about two thirds of the hits are negated. The construction is also used to emphasize that a result was obtained, \REF{ex:engdahl:11a}, often with a numeric specification, as in (11b,c), cf. \REF{ex:engdahl:1b}.

\ea%11
    \label{ex:engdahl:11}
 \ili{Swedish} (\textit{Korp})   
\ea \label{ex:engdahl:11a}
\gll   “så hit     med en   skyffel   så   det   \textbf{blir}  något    \textbf{gjort}.” \\
 so here with a shovel so  \textsc{ex} becomes  something\textsc{.n} done\textsc{.n}\\
    \glt ‘Hand me a shovel so that something gets done.’

\ex \label{ex:engdahl:11b}
\gll   {I går}    \textbf{blev}  det bara två     mål     \textbf{insläppta},    \\
    yesterday  became \textsc{ex} only two goals  let-in.\textsc{pl}\\
    \glt ‘Yesterday only two goals were let in.'

\ex \label{ex:engdahl:11c}
\gll Allt som allt   \textbf{blev}  det   fem     hus     \textbf{byggda}.  \\
     all    as  all    became \textsc{ex}    five  houses    built\textsc{.pl}\\
    \glt ‘Altogether there were five houses built.’
\z
\z

\noindent We also searched for the order PCP DO and found one example, see \REF{ex:engdahl:12a}, where the participle is in the neuter singular form.

\ea%12
    \label{ex:engdahl:12}
    \ili{Swedish} (\textit{Korp})
    \ea[]{\label{ex:engdahl:12a}
\gll Så  det \textbf{blev}  inte   \textbf{skrivet}  någon  berättelse om  loppet. \\
    so \textsc{ex}  became  not    written.\textsc{pcp}  any    story.\textsc{c} about  race.\textsc{def}\\
     \glt ‘So no story about the race was written.'}
   
\ex[?]{\label{ex:engdahl:12b}
\gll Så det \textbf{blev} ingen berättelse om loppet \textbf{skriven}.\\
so \textsc{ex}  became  no    story.\textsc{c} about  race.\textsc{def}    written.\textsc{c}  \\
}
\ex[]{\label{ex:engdahl:12c}
\gll Så det \textbf{blev} ingen berättelse \textbf{skriven} om loppet.\\
so \textsc{ex}  became  no    story.\textsc{c}    written.\textsc{c} about  race.\textsc{def}  \\
}

\z
\z

\noindent This example is actually quite similar to Holmberg’s \REF{ex:engdahl:1a}; note the complex noun phrase placed after the participle. Placing the entire noun phrase before the participle is less felicitous \REF{ex:engdahl:12b}, whereas splitting it up is OK \REF{ex:engdahl:12c}, just as in Holmberg’s \REF{ex:engdahl:1b}.

We can conclude that practically all the authentic examples in \ili{Swedish} have the DO PCP order and that the DO is very often negated. The opposite order is grammatical, but used very sparingly, primarily when some other factor such as weight influences the word order. One way of integrating this finding with Holmberg’s analysis would be to assume something along the following lines: whether the expletive belongs to the first or the second subarray depends on the complexity of the DP and whether or not there is a \isi{negation} present.

\section{Danish} %2.2
In \ili{Danish}, the \textit{blive}{}-\isi{passive} and  the \textit{s}{}-\isi{passive} are distributed more evenly than in \ili{Swedish}. \textit{s}{}-\isi{passive} is primarily used in the present tense and with infinitives, especially following modal verbs. The periphrastic \textit{blive}{}-\isi{passive} dominates all the other tenses.  \citet{HeltoftJakobsen1996} claim that the choice of \isi{passive} form reflects a mood distinction in \ili{Danish}; \textit{s-}\isi{passive} is used in objective statements whose validity is independent of the speaker, whereas \textit{blive}{}-\isi{passive} is preferred when the speaker makes a subjective judgment about some event that s/he has first hand knowledge about.\footnote{This view is also put forward in the \ili{Danish} reference grammar  \citep[747ff]{HansenHeltoft2011}. See \citet[101ff]{Laanemets2012} for a critical assessment.} Among the 4765 \ili{Danish} \isi{passive} examples analysed by \citet{Laanemets2012}, roughly 10\% (474) were \isi{impersonal passive}s and of these 185 were transitive \textit{blive}{}-passives, as illustrated in \REF{ex:engdahl:13}.

\ea%13
    \label{ex:engdahl:13}
    \ea
    \ili{Danish}, spoken (\citealt{Laanemets2012})\\
\gll der \textbf{bliver}   næsten   ikke \textbf{optaget}    nye elever\\
    \textsc{ex}  becomes  almost    not    admitted.\textsc{pcp}  new  pupils\\
    \glt ‘Hardly any new pupils are admitted.'
\ex
    \ili{Danish}, written (\citealt{Laanemets2012})\\
\gll     Der er \textbf{blevet} \textbf{produceret} flere terrorister   i     de sidste år pga. den politik,\\
    \textsc{ex}    is    become  produced.\textsc{pcp} more    terrorists in   the  last  years because-of  that  policy\\
    \glt ‘More terrorists  have been produced in recent years  because of that policy.'
\z\z

\noindent All of these examples had the word order PCP DO, without participle \isi{agreement}, as expected on Holmberg’s analysis. In order to find out if the DO PCP order is used at all, we carried out a similar search to the one in \ili{Swedish} in the 56 million word corpus \textit{KorpusDK.} We found altogether eleven examples, eight of which were negated, see (14b,c).

\ea%14
    \label{ex:engdahl:14}
    \ili{Danish} (\textit{KorpusDK})
\ea \label{ex:engdahl:14a}
\gll Hver   gang {der} \textbf{bliver} en   ny \textbf{indlagt}, skal man sætte sig   ind i   patientens  journaler.\\
every time  \textsc{ex} becomes a new admitted.\textsc{pcp} shall one  put  \textsc{refl} into in  patient.\textsc{def.poss} notes\\
    \glt ‘Every time a new patient is admitted, one has to familiarize oneself with his/her notes.’
\ex\label{ex:engdahl:14b}
\gll     Der \textbf{blev}     ingenting \textbf{sagt},    før  det ringede   på   døren,\\
  \textsc{ex} became  nothing    said\textsc{.pcp} before \textsc{ex} rang on  door\textsc{.def}\\
  \glt ‘Nothing was said before the  door bell rang.'
\ex\label{ex:engdahl:14c}
\gll  Ifølge    SAS \textbf{blev} der ingen fejl \textbf{fundet}     på nogle af flyene,\\
     according  SAS  became  \textsc{ex}  no    fault  found\textsc{.pcp}  on  any  of  planes.\textsc{def}\\
     \glt ‘According to SAS, no fault was found on any of the planes.'
\z
\z

\noindent These examples resemble the \ili{Swedish} ones except that the participles lack \isi{agreement}. The DO PCP order is also used in spoken \ili{Danish}, as shown in \REF{ex:engdahl:15}. The examples come from the \textit{Nordic Dialect Corpus} (\textit{NDC}).

\ea%15
    \label{ex:engdahl:15}
    
    \ea
    \ili{Danish}  (\textit{NDC}, østjylland2)
    
\gll  der \textbf{blev} inte   noget \textbf{gjort} ved det\_der \\
	\textsc{ex}  became not anything done.\textsc{pcp} with that \\
    \glt ‘Nothing was done with that.'
\ex\label{ex:engdahl:15b}
\ili{Danish}   (\textit{NDC}, fyn2)

\gll     klokken   fem     om morgenen der     \textbf{blev}     der én     \textbf{skudt}     ned  \\
  {clock.}\textsc{def}  five  in  morning    there    became \textsc{ex} {one}\textsc{.c} {shot.}\textsc{pcp} {down}\\
  \glt ‘At 5 o’clock in the morning one person was shot down there.’
\z
\z

\noindent According to \citet{Pedersen2017}, the DO PCP order is, or has been, possible in all \ili{Danish} dialects and is still the preferred order in Sønderjylland (North Schleswig) as shown in \REF{ex:engdahl:16}.

\ea%16
\label{ex:engdahl:16}
\ili{Danish}, Sønderjylland  (K. M. Pedersen, p.c.)\\
\gll da     \textbf{blev}  der en stor   gryde   grød      \textbf{kogt}    hver     dag  \\
    then  became \textsc{ex}  a   large    pot porridge\textsc{.c}  {cooked}\textsc{.pcp}  every  day\\
    \glt ‘Then a large  pot of porridge was  cooked  every day.'
\z

\noindent Note that the participle has the neuter singular form even when placed after a non-neuter object in \REF{ex:engdahl:15b} and \REF{ex:engdahl:16}. In older \ili{Danish}, when the DO PCP order was more common, \isi{agreeing} participles were used, as shown in the following examples from \citet{Høysgaard1752}, supplied by K. M. Pedersen (e-mail, April 2015). Later grammars such as \citet{Mikkelsen1894,Mikkelsen1911} do not have any examples with \isi{agreeing} participles.

\ea%17
    \label{ex:engdahl:17}
    \ili{Danish} \citep[327]{Høysgaard1752}\\
    
    \gll Der \textbf{blev} en sølvske \textbf{staalen}.\\
	 \textsc{ex} \textit{became}  \textit{a} \textit{silverspoon}\textsc{.c}  \textit{stolen}\textsc{.c}\\
     \glt ‘A silver spoon was stolen.'\\
     \z

     \newpage 
\ea
\label{ex:engdahl:17b}
\ili{Danish} \citep[345]{Høysgaard1752}\\
\gll     Der \textbf{blev}     en   Död  \textbf{udbaaren}.  \\
     \textsc{ex}  became  a    dead\textsc{.c}   {out-carried}\textsc{.c}\\
     \glt ‘A dead person was carried out.'
\z


% \ea%17
%     \label{ex:engdahl:17}
%     \ili{Danish} \citep[327]{Høysgaard1752}\\
%     \ea
%     \gll Der \textbf{blev} en sølvske \textbf{staalen}.\\
% 	 \textsc{ex} \textit{became}  \textit{a} \textit{silverspoon}\textsc{.c}  \textit{stolen}\textsc{.c}\\
%      \glt 'A silver spoon was stolen.'\\
% \ex
% \citep[345]{Høysgaard1752}\\
% \gll     Der \textbf{blev}     en   Död  \textbf{udbaaren}.  \\
%      \textsc{ex}  became  a    dead\textsc{.c}   {out-carried}\textsc{.c}\\
%      \glt 'A dead person was carried out.'
% \z
% \z
\noindent In contemporary \ili{Danish}, only the dialect spoken in the island of Bornholm has \isi{agreeing} participles, see the example in  \REF{ex:engdahl:18}.\footnote{\citet{Pedersen2013} shows that the use of \textit{s}{}-\isi{passive} in Bornholm also resembles the \ili{Swedish} pattern.}

\ea%18
    \label{ex:engdahl:18}
    \ili{Danish}, Bornholm (K. M. Pedersen, e-mail, April 2015)\\
\gll     Dær   \textbf{ble}    ejnj   værja      \textbf{tesatter}.\\
 \textsc{ex} became a sword.\textsc{masc} added.\textsc{masc} \\\
 \glt ‘A sword was added.'
\z

\noindent We conclude that although the dominant word order pattern in modern \ili{Danish} is PCP DO, the DO PCP order, without participle \isi{agreement}, is available for many dialect speakers and is often used with quantified, especially negated objects like \textit{ingenting}. It would be interesting to look closer at the diachronic development of the modern \ili{Danish} system.

\section{Norwegian} %2.3.

The distribution of \textit{s}{}- and \textit{bli}{}-\isi{passive} in \ili{Norwegian} \textit{bokmål} (Holmberg’s \ili{Norwegian} 1) resembles the situation in \ili{Danish}. \textit{S}{}-\isi{passive} is only used in the present tense and infinitives. Among the 3096 examples analysed by \citet{Laanemets2012}, 238 were \isi{impersonal passive}s, of which 87 transitive \textit{bli}{}-passives, see the examples in \REF{ex:engdahl:19}.

\ea%19
\label{ex:engdahl:19}
\ili{Norwegian} 1 (\citealt{Laanemets2012})\\
\ea
\gll for  det  \textbf{ble}  \textbf{bygd}  veldig   mye   akkurat   den   tida    \\
     {because} \textsc{ex}  became  built.\textsc{pcp}  very  much  exactly    that  time.\textsc{def}  \\
     \glt ‘because a lot was built right at that time'
\ex
\gll     Det  \textbf{ble}   ikke   \textbf{funnet}  tekniske   bevis  i   kvinnens   leilighet.\\
  \textsc{ex}    became  not  found.\textsc{pcp}  technical  evidence    in woman.\textsc{def.poss} {flat}\\
  \glt ‘No technical evidence was found in the  womans  flat.'
\z
\z

\noindent All the examples in \citet{Laanemets2012} had the order PCP DO, again as expected, and quantified objects were common. Using the same procedure as for \ili{Swedish} and \ili{Danish}, we investigated if the word order DO PCP is used in \ili{Norwegian} 1. We searched in a 41.4 million word subcorpus of \textit{Leksikografisk bokmålskorpus} (LBK) but only found a few examples.

\ea%20
    \label{ex:engdahl:20}
    \ili{Norwegian} (\textit{LBK})\\
    \ea
\gll  Dermed     \textbf{blir}  det mye   vanndamp   \textbf{fordelt}      på hver dråpe. \\
with-this becomes  \textsc{ex} much steam distributed.\textsc{pcp} on every drop
\\
\glt ‘This way a lot of steam  is distributed over each drop.'
\ex\label{ex:engdahl:20b}
\gll     Ifølge  Amnesty International \textbf{ble} det   5.000   uskyldige   \textbf{drept}. \\
  according  Amnesty  International  became \textsc{ex}    5000  innocent    killed.\textsc{pcp}\\
  \glt ‘According to Amnesty International, 5000 innocent people were killed.'
\z\z

\noindent We did not find any examples with negated pronouns or other quantified expressions, like \textit{ingenting}, before the participle, i.e. \ili{Norwegian} counterparts to \REF{ex:engdahl:10} in \ili{Swedish} or \REF{ex:engdahl:14b} in \ili{Danish}. This is presumably linked to the fact that \ili{Norwegian} speakers are much less likely to prepose negated objects than \ili{Danish} and \ili{Swedish} speakers (see below).

    With respect to \textit{nynorsk,} Holmberg’s \ili{Norwegian} 2, there is variation in the choice of expletive and whether or not the participle shows \isi{agreement}, but apparently not much variation with respect to word order (see \citealt{Åfarli2009}; \citealt[218ff]{AaEtAl2014}). The order PCP DO dominates strongly, just as in \ili{Norwegian} 1. Only one example with a preposed negated DO was found in the Oslo corpus of \textit{nynorsk} (3.8 million words), see \REF{ex:engdahl:21}. It is not possible to tell whether the participle agrees with \textit{det} or \textit{ingenting}, since both are neuter.

\ea%21
    \label{ex:engdahl:21}
    \ili{Norwegian} 2 (\textit{Oslo corpus})\\
\gll   Ei   lang   stund   \textbf{vart}  det   ingenting   \textbf{sagt}.\\
a long while became  \textsc{ex} nothing.\textsc{n} said.\textsc{n}
 \\
 \glt ‘For a long while nothing was said.'
\z

\noindent This resembles the examples found in \ili{Swedish} and \ili{Danish}. However, speakers of \ili{Norwegian} 2 are less willing to accept preposed objects with numerical attributes, as in (11b,c) and \REF{ex:engdahl:20b}.
The fact that \ili{Norwegian} 2 speakers accept the DO PCP order when the DO is negated distinguishes them from \ili{Norwegian} 1 speakers, but more informant studies are clearly needed here.

\subsection{The NEG-DO PCP order}
We have seen that when a direct object precedes the participle in expletive transitive \textit{bli(ve)}{}-passives in \ili{Swedish} and \ili{Danish}, it is very often negated. This pattern is also used with active participles in \ili{Danish} and \ili{Swedish}, see \REF{ex:engdahl:22}.

\ea%22
    \label{ex:engdahl:22}
    \ea
    \ili{Danish} (\citealt{Engels2012}, example (6))\\
\gll    Manden     havde   måske     ingenting   sagt.     \\
man\textsc{.def} had maybe  nothing  said\textsc{.pcp}
	\\
    \glt ‘Maybe the man hadn’t said anything.’
\ex
\ili{Swedish} (\citealt{Engels2012}, example (6))\\
\gll     Mannen     hade   kanske   ingenting   sagt.     \\
  {man}\textsc{.def}    had    maybe    nothing    said\textsc{.pcp}\\
  \glt ‘Maybe the man hadn’t said anything.’
\z
\z

This word order is often described as stylistically marked and reserved for formal and literary genres. However, \citet{Engels2012} found that it is used both in spoken language and in blog texts on Google. She investigated the positioning of negated objects with five frequent verbs (the \ili{Scandinavian} counterparts of \textit{say, hear, see, get} and \textit{do}) and found that 33\% preceded the participle in \ili{Danish} and 15\% in \ili{Swedish}, compared to 0\% in \ili{Norwegian} (see \citealt[Table 1]{Engels2012}).\footnote{In \ili{Swedish}, preposing of negated objects is also possible in \textit{s}{}-\isi{passive}.
\begin{xlist}[(ii)]
\exi{(i)} \gll Det   har   ingenting   sagts   (*ingenting)   om   detta.\\
   \textsc{ex}   has nothing said.\textsc{s} {} about this\\ 
   \glt ‘Nothing has been said about this.’
\exi{(ii)}\label{ex:engdahl:22b} 
\gll    Det   har ({\upshape\textsuperscript{??}}mycket)   sagts ({\upshape\textsuperscript{ok}}mycket)   om   detta.\\
   \textsc{ex}  has {} said.\textsc{s} much about this  
   \\
   \glt ‘Much has been said about this.'
\end{xlist}
}
\noindent It thus seems that one additional factor that affects the word order options is whether the language allows for incorporated negative objects to precede the participle. In \ili{Swedish}, where \textit{bli}{}-passives are unusual, they are primarily used with negated objects. In \ili{Danish}, where expletive transitive \textit{blive}{}-passives normally have the word order PCP DO, most of the exceptions involve negated objects. And in \ili{Norwegian} 1, where preposed negated objects are rare, we hardly find any deviations from the PCP DO order.

\section{Double object constructions}
\citet{Holmberg2002Expletives} also discusses the word order options in \is{DOC}double object constructions. For \ili{Swedish}, he gives examples where either both objects follow the participle \REF{ex:engdahl:23a} or where the indirect object (IO) precedes and the direct object follows the participle, \REF{ex:engdahl:23b} .

\ea%23
    \label{ex:engdahl:23}
    \ea \label{ex:engdahl:23a}
    \ili{Swedish} (\citealt[87]{Holmberg2002Expletives})
    \\
\gll    Det   \textbf{blev}  \textbf{givet}     pojken     presenter.\\
     \textsc{ex}    became   given.\textsc{n}  boy\textsc{.c.def} presents\\
     \glt ‘The boy was given presents.'
\ex \label{ex:engdahl:23b}
\ili{Swedish} (\citealt[114]{Holmberg2002Expletives})\\
\gll     Det   \textbf{blev}  inte   många barn  \textbf{givna}     presenter den   julen.  \\
         \textsc{ex}    became  not    many  children  given.\textsc{pl}  presents    that Christmas.\textsc{def}\\
         \glt ‘Not may children were given presents that  Christmas.'
\z
\z

\noindent The orders shown in \REF{ex:engdahl:23} are grammatical, but hardly used. It is somewhat more common for both objects to precede the participle, especially if the indirect object is a pronoun, as also pointed out in \citet{börjarsvincent}. In that case the participle agrees with the direct object.

\ea%24
    \label{ex:engdahl:24}
    \ili{Swedish} (\citealt[4:387]{TelemanEtAl1999})\\
    \ea
\gll  Det   skulle   \textbf{bli}    oss en belöning   \textbf{tilldelad}.  \\
  \textsc{ex}    should  become  us  a    reward.\textsc{c}  {awarded.}\textsc{c}.\\
  \glt ‘We  were supposed to  receive an award.'
\ex\label{ex:engdahl:24b}
\gll     Det   \textbf{blev}  oss inte   mycket     \textbf{anförtrott}.\\
  \textsc{ex}    became us    not    much.\textsc{n}    {confided.}\textsc{n}\\
  \glt ‘Not much was  confided to us.'
\ex\label{ex:engdahl:24c}
\gll Det   \textbf{blev}  ingen  särskilt   mycket     \textbf{anförtrott}.\\
  \textsc{ex}    became nobody very much.\textsc{n}    {confided.}\textsc{n}\\
  \glt ‘Not much was confided to anybody.'
\z
\z

\noindent Note that the pronominal indirect object is shifted across the \isi{negation} in \REF{ex:engdahl:24b}. In \REF{ex:engdahl:24c} the \isi{negation} is incorporated into the indirect object \textit{ingen} (‘nobody’).\footnote{In this respect, the expletive double object passives differ from active versions.  Whereas negated direct objects can be preposed, as shown in (23b), preposing a negated indirect object is not felicitous in \ili{Swedish}. 
\begin{xlist}[(i)]
 \exi{(i)} \gll ?* Vi har ingen anförtrott särskilt mycket.\\
 {} we have nobody confided  very much
                \\
                \glt Intended: ‘We have not confided very much to anybody.'

\end{xlist}

\noindent This was brought to my attention by Björn Lundqvist (e-mail, May 2016) who mentioned a similar observation concerning \ili{Norwegian} in \citet[22]{Lødrup1989}.
}

\noindent In both \ili{Norwegian} 1 and 2, it seems that only the order PCP IO DO is used, see \REF{ex:engdahl:25}.\footnote{The order IO PCP is found in \ili{Norwegian} \textit{bli}{}-passives with extraposed clauses, as shown in (i).

\begin{xlist}[(i)]
 \exi{(i)} \ili{Norwegian} (\textit{LBK}) \\
 \gll Det \textbf{blir}  meg ofte \textbf{fortalt} at israelske soldater   scorer så høyt   på   motivasjon.    \\
   \textsc{ex}  beccomes me often told.\textsc{pcp} that 
       Israeli soldiers    score     so high  on  motivation\\
       \glt ‘I am often told that Israeli soldiers score high on motivation.'
\end{xlist}
\citet{EngdahLaanemetsl2015opersonlig} argue that this type should not be analysed as expletive passives, one reason being that they are grammatical in English, (ii), where expletive passives are ungrammatical (cf. \citealt{CarnieHarley2005}).
\begin{xlist}[(ii)]
\exi{(ii)}It has to be said that the budget proposal is unlikely to pass.
\end{xlist}
}


\ea%25
    \label{ex:engdahl:25}
    \ili{Norwegian} 2 (\citealt[845]{FaarlundEtAl1997})\\
\gll   	   Det   \textbf{blei}  \textbf{nekta}    oss   adgang.\\
  \textsc{ex}    became  denied\textsc{.pcp}  us    admittance\\
  \glt ‘We were denied admittance.'
\z

 \noindent Also in \ili{Danish}, the preferred order is PCP IO DO, as in the following examples from \textit{KorpusDK}.

\ea%26
    \label{ex:engdahl:26}
    \ili{Danish} (\textit{KorpusDK})
    \ea \label{ex:engdahl:26a}
    
\gll    Jeg skulle være naturlig,   der   \textbf{blev}  ikke \textbf{pålagt}    mig   noget.\\
I should be  natural  \textsc{ex} beccame not imposed.\textsc{pcp} me anything
\\
\glt ‘I was supposed to be natural, nothing was imposed on me.'
\ex
\gll     Der   \textbf{bliver}     \textbf{pålagt}    børn  et   alt   for   stort     ansvar    i dag.\\
  \textsc{ex}   becomes  imposed.\textsc{pcp}  children  a    too for big responsibility  in day\\
  \glt ‘A too big responsibility is imposed on children today.'
\z
\z

\noindent Our corpus searches also produced some examples with IO preceding PCP, as in the examples in \REF{ex:engdahl:27}.

\ea%27
\label{ex:engdahl:27}
\ili{Danish} (\textit{KorpusDK})
\ea
\gll De     udførte  blot   de   opgaver, der   \textbf{blev}  dem \textbf{pålagt}      af   folketinget,\\
  they  carried-out  just  the    tasks that  became  them  imposed.\textsc{pcp}  by  parliament\textsc{.def}\\
  \glt ‘They only carried out the tasks that  had been imposed on them by the parliament.'
\ex
\gll   Vent og     se,   hvem   {der} {\textbf{bliver}} {dig} {\textbf{tildelt}}.\\
  wait  and  see  who     that  becomes    you  assigned.\textsc{pcp}.\\
  \glt ‘Wait and see who is asssigned to you.'
\z
\z

\noindent However, these are not expletive transitive constructions but ordinary passives where the DO has been relativized or questioned. In modern \ili{Danish}, the expletive pro-form \textit{der} is also used as relativizer (‘that’) in subject \isi{relatives} and questions. Consequently examples may be ambiguous between an expletive and a personal \isi{passive}, as discussed in  \cite{EngdahlLaanemets2015prepositional}. An example is given in \REF{ex:engdahl:28a} which can be analysed as a \isi{relative} clause with either an expletive \isi{passive} \REF{ex:engdahl:28b}, or a personal \isi{passive} \REF{ex:engdahl:28c}.

\ea%28
    \label{ex:engdahl:28}
    \ili{Danish} (\citealt[314]{EngdahlLaanemets2015prepositional})
    \ea\label{ex:engdahl:28a}
    
\gll  Det er   det   forlig,    \textit{der}   \textbf{bliver}  \textbf{refereret} til. \\
   \textsc{ex}  is  this   settlement \textsc{der}  becomes    referred\textsc{.pcp}  to\\
\ex\label{ex:engdahl:28b}
     Det er det forlig\textsubscript{i},  [ \textsubscript{CP} [\textit{Ø}][\textsubscript{IP} \textit{der} bliver refereret til \textit{e}\textsubscript{i} ]]\\
\glt ‘It is this settlement there are references to.’
\ex\label{ex:engdahl:28c}
 Det er det forlig\textsubscript{i},  [\textsubscript{CP}[ \textit{der}][\textsubscript{IP} \textit{e}\textsubscript{i} bliver refereret til ]]\\
\glt ‘It is this settlement that is being referred to.’
\z
\z

\section{Concluding remarks} %4

Of the investigated language varieties, \ili{Norwegian} 1 (\textit{bokmål}) stands out as the only one that behaves as expected given Holmberg’s parameters; it lacks participle \isi{agreement} and only displays the PCP DO word order, with few exceptions. \ili{Danish}, which has the same parameter settings as \ili{Norwegian} 1, apparently had the DO PCP word order in earlier stages and this still shows up in many dialects.  \\
The assumed parameter settings for \ili{Swedish} and \ili{Norwegian} 2 (\textit{nynorsk}) predict that these languages should allow both word orders. Nevertheless, there is very little evidence for this in actual use. The languages differ furthermore in which pattern is preferred; the PCP DO order is hardly used in \ili{Swedish}, but is the preferred order in \ili{Norwegian} 2, just as in \ili{Norwegian} 1.   

Although expletive \textit{bli}{}-passives are very infrequent in \ili{Swedish} compared with expletive \textit{s}{}-passives, corpus studies have revealed a characteristic pattern where a quantified, often negated, DO precedes the participles, as illustrated in \REF{ex:engdahl:10}. The same type of DO occasionally appears preceding the participle in \ili{Danish} which suggests that there may be a correlation between the availability of NEG-DO PCP order in expletive passives and in active clauses.

\section*{Acknowledgments}

I am grateful to Anu Laanemets for discussions and for help with the corpus searches for this article. I would also like to thank Filippa Lindahl and Henrik Rosenkvist for comments on an earlier draft and members of the Grammar seminars in Gothenburg and Lund for stimulating questions and comments, in particular Lars-Olof Delsing, Verner Egerland, Gunlög Josefsson, Erik Petzell, Halldór Sigurðsson and Øystein Vangsnes.

 
\section*{Corpora}

\textit{Korp}: \url{http://spraakbanken.gu.se/korp/}

\noindent
\textit{KorpusDK}: \url{http://ordnet.dk/korpusdk}

\noindent
\textit{Leksikografisk bokmålskorpus}:     \url{http://www.hf.uio.no/iln/tjenester/kunnskap/sprak/korpus/skriftsprakskorpus/lbk}

\noindent
\textit{Nordic Dialect Corpus} 
\url{http://www.tekstlab.uio.no/nota/scandiasyn/}

\noindent
\textit{Oslo Corpus of Tagged \ili{Norwegian} Texts}   
\url{http://www.tekstlab.uio.no/norsk/nynorsk/index.html}


% \section*{Abbreviations}
\newpage 
\largerpage
\sloppy
\printbibliography[heading=subbibliography,notkeyword=this]
\end{document}
