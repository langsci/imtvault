\documentclass[output=paper]{LSP/langsci} 
\author{Makiko Mukai \affiliation{University of Kochi}}
\title{Semantic characteristics of recursive compounds} 
% \epigram{Change epigram}
\abstract{In this paper I propose a structure for recursive compounds, such as peanut butter sandwich in Phase Theory \citep{Chomsky2008}. I propose that a root without a categorical feature is merged with a category-determining feature \citep{Marantz1997} in the narrow syntax and another root is merged to form a compound word. I also argue that another root without categorical feature is merged to form a right-branching recursive compound. On the other hand, a linking element is there for the sake of asymmetry (cf. \citealt{Okubo2014}): it checks the head of the two-member compound and another [${\surd}$ROOT \textit{n}]
% \todo{We have one opening bracket here, but two closing ones} 
can be merged. As a result the final categorising nominal head is the head of the whole compound word.} 
\ChapterDOI{10.5281/zenodo.1116775}

\maketitle

\begin{document}
\section{Introduction}

% Recursion of words or phrase is a fundamental property of human language that potentially differentiates human language both from other human cognitive domains and known communication systems in animals \citep{Hauser2002}. In this paper I define it is a phenomenon of embedding structures within structures in cyclic fashion to create words or phrases, as complex and long as we like. The followings are examples from English (\ref{ex:mukai:1}--\ref{ex:mukai:2}), \ili{Mainland Scandinavian} (\ref{ex:mukai:3}--\ref{ex:mukai:4}) and Japanese (\ref{ex:mukai:5}--\ref{ex:mukai:6}). 



\ea%1
\label{ex:mukai:1}
{[mail [delivery service]]   }  
\glt‘delivery service of mails‘
\z



\ea%2
\label{ex:mukai:2}
{[[chocolate chip] cookie] }
\glt‘cookie cooked with flakes of chocolate‘ 
\z


\ea%3
\ili{Swedish}\\\label{ex:mukai:3}
{[barn-[bok-klub]] }    \\{}
[child-[book-club]]                      
\glt‘book club for children‘ 
\z


\newpage 
\ea%4
\ili{Swedish}\\\label{ex:mukai:4}
{[[röd-vins-s]-flaska]}  \\{}
[[red-wine-LE]-bottle] 
\glt‘bottle for red wine‘

 \z

\ea%5
\label{ex:mukai:5}
\ili{Japanese}\\
{[doitsu [bungaku kyookai]] }     \\{}
[Germany [literature association]] 
\glt‘literature association in Germany‘ 
\z

\ea%6
\ili{Japanese}\\\label{ex:mukai:6}
{[[nyuugaku shiken] taisaku] }    \\{}
[[entrance exam] study]    
\glt‘study for entrance exam‘
\z


The interpretation of the whole compounds\is{compound}, is, for example,\textit{ book club for children}, not \textit{club for children’s books} in \REF{ex:mukai:5}. This \isi{recursive} compound is called right-branching recursive compounds. In contrast, in the examples (\ref{ex:mukai:2},\ref{ex:mukai:4},\ref{ex:mukai:6}) the modifier at the right hand expands the already-made compound.\is{compound} This type is called left-branching recursive compounds. 

\section{Proposed structure} 

According to \citet{Miyagawa2015} merge is the \isi{recursive} operation of the language faculty. I follow this claim and use \isi{Phase} Theory \citep{Chomsky2008,Marantz1997} for a structure of compounds.\is{compound} I propose structures for right-branch\-ing and left-branching recursive compounds in \isi{Phase} Theory. 

\ea%7 
\label{ex:mukai:7}
\begin{forest}
[n
[${\surd}$] [n
  [${\surd}$] [n
    [${\surd}$] [n]
   ]
  ]
]
\end{forest}
\z
 
\newpage  
The structure \REF{ex:mukai:7} is derived as follows. Once the two-member compound\is{compound} is derived, another derivation can take place. Another root without any features is merged. This is the derivation of the right-branching \isi{recursive} compounds, like (\ref{ex:mukai:1},\ref{ex:mukai:3},\ref{ex:mukai:5}). If one assumes that both constituents of the compound are merged with category-defining element, the LF does not see which element is the head, and the derivation crashes at the LF level. So in my proposed structure, only one root is merged with a cateogory-defining head, turning the root into an \textit{n}. This is labelling in terms of \citet{Chomsky2008}. The head of the whole compound\is{compound} is the category-defining element. The whole compound is transferred to the interpretational representation and spelled out as a phase \citep{Chomsky2008}. 

\ea%8
    \label{ex:mukai:8} 
    \begin{forest}
     [n, for tree={calign=fixed edge angles}, delay={where content={}{shape=coordinate}{}} [ 
			     [ [$\surd$,tier=preword] [ [ [$\surd$,tier=word] [\st{n},tier=word,name=n] ]]] [L, tier=word,name=L]
			    ] [ [$\surd$,tier=preword] [n,tier=preword] ] 
			  ]
     \draw[<-] (n) -- (L);		
     \end{forest}
\z


For left-branching \isi{recursive} compounds,\is{compound} there is a linking element in left-branching recursive compounds, phonetically realised in \ili{Mainland Scandinavian} but not in Japanese or English (see \ref{ex:mukai:4}). I propose that the linking element has an uninterpretable feature (cf. \citealt{Okubo2014}) and checks the category-defining feature. The resulting structure is sent to the interpretational component and spelled out as phase. 

The resulting structure is merged with another root, which is merged with a category-defining head in parallel. As a result the head of the whole compound is the right-most category-defining head and this \isi{compound} is transferred to the interpretational representation and spelled out as phase. 

\section{Conclusion} 

In this paper, the author proposed a structure for \isi{recursive} compounds\is{compound} in \isi{Phase} Theory. If the linking morpheme does not check the categorical features of the non-head, the structure will be impossible, having two heads. Thus, in the languages without recursive compounding, there is no linking element. Assuming that the two-member is a phase we can capture the word-like accent characteristic, as opposed to phrase-like right-branching \isi{recursive} compounds.\is{compound}  


% \section*{Abbreviations}
% \todo[inline]{maybe add some Abbreviations to this chapter? - The ones used in trees might be used differently then in the Leipzig Rules?}
% \section*{Acknowledgements}
I would like to thank Anders Holmberg, Hideki Kishimoto, and Shigeru Miyagawa for their discussions on this topic. Also, I would like to thank the 8 English native speakers on their judgments for the data. 

\printbibliography[heading=subbibliography,notkeyword=this]
\end{document}