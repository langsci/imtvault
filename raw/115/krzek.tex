\documentclass[output=paper]{LSP/langsci}
\author{Małgorzata Krzek \affiliation{Newcastle University}
}
\title{The null subject parameter meets the Polish impersonal -NO/-TO construction}
% \epigram{Change epigram}
\abstract{This squib argues that null generic inclusive subjects are found in consistent null subject languages not only in the passive voice, as maintained by \citet{FassiFehri2009}, but also in the active voice – in the so-called -NO/-TO construction. However, the null subject of the -NO/-TO construction is not logophoric, so it does not receive its inclusive reading by being anchored to the Speech Act, where the [Speaker] and the [Addressee] features are located \citep{DAlessandro2007,Sigurðsson2004,Bianchi2003Tense}. It is proposed that the interpretation of the null subject of the -NO/-TO construction is dependent on a binding relation with a null Topic \citep{FrascarelliHinterhölzl2007} that is merged in the C-domain.}
\rohead{\thechapter\hspace{0.5em}The null subject parameter}

\ChapterDOI{10.5281/zenodo.1116779}

\maketitle

\begin{document}
\section{Introduction}

According to \citet[12]{RobertsHolmberg2010}, there are four types of \isi{null subject} languages (NSL):
\begin{enumerate}
\item Expletive \isi{null subject} languages (\ili{German}, \ili{Dutch})
\item Partial \isi{null subject} languages (\ili{Finnish}, Russian)
\item Consistent \isi{null subject} languages (\ili{Italian}, \ili{Greek})
\item Discourse \isi{pro-drop} languages  (Chinese, Indonesian)
\end{enumerate}

Expletive \isi{null subject} languages allow for subject expletives to be null. Partial \isi{null subject} languages allow for a generic subject to be null, as in \REF{ex:2.krzek}, but 3\textsuperscript{rd} person subjects have to be overt, as in \REF{ex:3.krzek}.

\ea{\label{ex:1.krzek}
\langinfo{Finnish}{}{\citealt[200]{Holmberg2010finnish}}\\
\gll Tässä   istuu   mukavasti.\\
 here     sits     comfortably\\
\glt ‘One can sit comfortably here.’ 
}
\z

\ea{\label{ex:2.krzek}
\langinfo{Finnish}{}{\citealt[539]{Holmberg2005}}\\
\gll *(Hän) puhuu   englantia.\\
 S/he  speak.\oldstylenums{3}\textsc{sc} English\\
\glt ‘S/he speaks English.’ 
}
\z

In consistent NSL all subject pronouns regardless of the person and tense can be null. Indefinite \isi{null subjects}, on the other hand, have to be overt. \citet[92]{Holmberg2010Null} illustrates this difference by contrasting \ili{Brazilian Portuguese} (BP), a partial NSL, with European Portuguese (EP), a consistent NSL.

\ea{\label{ex:3.krzek}
\langinfo{Brazilian Portuguese}{}{\citealt[92]{Holmberg2010Null}}\\
\gll É assim que faz       o   doce.\\
  is thus   that makes the sweet\\
\glt ‘This is how one makes the dessert.’
}
\z

\ea{\label{ex:4.krzek}
\langinfo{European Portuguese}{}{\citealt[92]{Holmberg2010Null}}\\
\gll É assim   que \ul{\textbf{se}} faz     o   doce.\\
  is thus   that \ul{\textbf{SE}}   makes   the   sweet \\
\glt ‘This is how one makes the dessert.’
}
\z

In BP the subject pronoun corresponding to the English \textit{one} is null. In EP the overt pronoun \textit{se} is used. \citet{Holmberg2010Null} notes that this generalization only concerns those generic pronouns that have an inclusive reading; that is, they denote people in general including speaker and the addressee. On the other hand, pronouns that express exclusive generic reading, which is equivalent to generic \textit{they} in English (as in \textit{They eat a lot of cheese in France}), can be null in consistent NSL.

The reason why this is the case is that, according to \citet{Holmberg2005,Holmberg2010Null}, consistent NSL have an unvalued D-feature in T(ense), which is valued by an A-\isi{Topic} \citep{Frascarelli2007}. This means when a null ΦP (‘phi-phrase’; 3\textsuperscript{rd} person deficient pronoun) enters into an \isi{Agree} relation with T and, as a result of this, is incorporated in T, it can be interpreted as definite, referring to an individual or a group. But it also means that a \isi{null subject} cannot have a generic interpretation; is, it cannot refer to people in general. Therefore, in order to express a generic meaning, consistent NSL have to resort to a variety of ‘overt strategies’. Thus, they may express it with an overt pronoun of \textit{SI/SE-}type. Partial NSL, on the other hand, do not have an uD in T that could be valued by an A-\isi{Topic}. As a result, an incorporated ΦP can only receive an indefinite interpretation.

More recently, \citet{FassiFehri2009} has argued for a qualification of these generalizations, claiming that generic inclusive null pronouns are actually found in consistent NSL, contra \citet{Holmberg2005,Holmberg2010Null}, but only in the \isi{passive} voice.

In this squib, I present evidence from \ili{Polish}, a consistent NSL (  \citealt{Sigurðsson2009}), that null generic inclusive subjects are found in the active voice – in the so-called -NO/-TO construction. I also show that the \isi{passive} construction identified in \citet{FassiFehri2009} shares a number of morphosyntactic properties with the -NO/-TO construction, suggesting that the construction in question may need to be actually reanalyzed as an active construction. The observation that null generic subjects can be found in consistent NSL suggests that a more fine-grained typology of \isi{null subjects} is needed.

First, I present a brief overview of morphosyntactic properties of the \ili{Polish} -NO/-TO construction and compare them to those of an \ili{Arabic} \isi{passive} construction identified in \citet{FassiFehri2009}. Next, I discuss possible interpretations of the null pronoun in the -NO/-TO construction and touch upon some of the possible consequences it may have for the internal structure of pronouns \citep{HarleyRitter2002}.

\section{The morphosyntactic properties of the -NO/-TO construction in Polish}

The -NO/-TO construction uses an uninflected verb form with a -NO\textit{/}{}-TO\textit{} suffix and can only refer to the past. It has been classified as ‘active indefinite’, and not \isi{passive} \citep{Kibort2004,Dziwirek1994,Śpiewak2000} \footnote{The \isi{passive} analysis of the -NO/-TO construction has been supported by the diachronic argument; that is, the -NO/-TO form was historically a neuter nominal \isi{passive} participle used with neuter \isi{passive} subjects \citep{Siewierska1988,Kibort2004}.} . The reason for this is that it can occur with transitive and intransitive verb types and with accusative case on the direct object argument. It is illustrated by the examples in \REF{ex:5.krzek}

\ea\label{ex:5.krzek}
   \ea
\gll Bywa\textbf{no} tam     często.\\
      were.\textsc{imp}  there   often\\
\glt ‘[One/They] used to come/be there often.’

    \ex
\gll  Dopiero w 1988   roku odczu\textbf{to}     ponownie potrzebę   odtworzenia Towarzystwa   Przyjaciół   ‘Ossolineum’.\\
only      in 1988   year felt.\textsc{imp}     again        need      reconstitution, Society             Friends     ‘Ossolineum’\\
\glt ‘It wasn’t until 1988 that [one/they] felt the need to reconstitute the Society of the Friends of “Ossolineum”.’                 (adapted from \citealt[259]{Kibort2004})
    \ex
\gll Kupowa\textbf{no}     tutaj   dużo      chleba.\\
     bought.\textsc{imp}   here    a-lot-of  bread\\
\glt      ‘[One/They] bought a lot of bread here.’
\z
\z

What is more, the construction in question is ungrammatical with a \isi{passive} \isi{auxiliary} and a \isi{passive} by-phrase \citep{Lavine2005}, as given in \REF{ex:6.krzek}.\footnote{For a full overview of the differences between the -NO/-TO construction and the \isi{passive}, see \citet{Kibort2004}.} 

\ea\label{ex:6.krzek}
  \gll   (*Zostało)      znaleziono         pieniądze w  restauracji (*przez kelnera). \\
    (\textsc{aux.pass})  found.\textsc{imp} money     in restaurant    (by   waiter)\\
\glt ‘[One/They] found money in the restaurant.’
\z

As for the \isi{null subject} of the -NO/-TO construction, the fact that it is projected is confirmed by the fact that it participates in control and binding.\footnote{\citet{Babby1998} maintains that there is no subject in the -NO/-TO construction at any level of representation. The affixation of the \isi{passive} morpheme ensures the dethematisation of the subject whereas the impersonal inflectional ending -o is used only when the external argument (i.e. subject) is not selected. This, according to \citet{Babby1998}, confirms that the sentence is truly subjectless. \citet{Babby1998} argues that the canonical subject position non-obligatory, and suggests that in the -NO/-TO construction it is simply not projected.} \citet{BondarukEtAl2003} observe that the -NO\textit{/}{}-TO\textit{} impersonals can share their subjects with embedded infinitive clauses \REF{ex:7.krzek}, with present and past participle forms, and in subject-raising constructions.

\ea%7
    \label{ex:7.krzek}

    \gll Próbowano          zrozumieć    ten    problem.\\
         tried.\textsc{imp} understand.\textsc{inf}  this.\textsc{acc} problem.\textsc{acc} \\
\glt ‘[One/They] tried to understand this problem.'
\z

With regard to binding, \citet{Kibort2004} observes that the covert subject of the -NO\textit{/}{}-TO is also capable of binding reflexive and reflexive-possessive pronouns that need to be bound by the subject. The former is illustrated by the example in \REF{ex:8.krzek}.

\ea{\label{ex:8.krzek}
\langinfo{Polish}{}{\citealt[273]{Kibort2004}}\\
\gll Oglądano      siebie/się w lustrze.\rmfnm\\
 looked.\textsc{imp} self/\textsc{SIĘ} in mirror\\
\glt ‘[One/They] looked at oneself/themselves in the mirror.' 
}
\z

\footnotetext{In \ili{Polish} the reflexive pronoun \textit{siebie} ‘self\textsubscript{\textsc{acc}}’ is, in very restricted contexts, interchangeable with a multifunctional enclitic form \textit{się} (see \citealt{Nagórko1998} and \citealt{Kibort2004}).}

The \isi{null subject} of the -NO/-TO construction has been argued to be either pro\textsubscript{arb} \citep{Dziwirek1994} or PRO\textsubscript{arb}  (e.g. \citealt{Maling1993,Lavine2005}). However, contrary to PRO found in infinitival clauses, the null pronoun in the subject position in the -NO/-TO is always interpreted as human. Secondly, the \isi{null subject} of the -NO/-TO does not require control \citep{Kibort2004}, contrary to PRO. Finally, the subject of the -NO/-TO construction is only compatible with adjectival predicates that are \textsc{masc}.\textsc{pl} whereas the PRO\textsubscript{arb} in \ili{Polish} uncontrolled infinitivals patterns with adjectival predicates that are \textsc{masc}.\textsc{sg}, as in \REF{ex:9.krzek} \citep[footnote 26]{Lavine2005}.

\ea%9
    \label{ex:9.krzek}
\ea
\gll Jest ważne [PRO być   szczęśliwym / *szczęśliwymi].\\
      is   important  PRO to.be happy.\textsc{instr}.\textsc{sg} / *happy.\textsc{instr}.\textsc{pl}\\
\glt  'It is important to be happy.’

\ex
\gll PRO wyglądano na  *szczęśliwego/ szczęśliwych.\\
         PRO look for *happy.\textsc{masc}.\textsc{acc}.\textsc{sg}/ happy.\textsc{masc}.\textsc{acc}.\textsc{pl}\\
         \glt ‘They looked happy.’
\z
\z

\section{\citet{FassiFehri2009}}

\citet{FassiFehri2009}, focusing on data from \ili{Arabic}, confirms that in \ili{Arabic}, just as in \ili{Italian}, null 3\textsuperscript{rd} person pronouns can only receive a definite/referential reading (i.e. \textit{she/he}). They cannot be interpreted as non-referential or generic. A generic or arbitrary interpretation can, however, be found, as \citet{FassiFehri2009} observes, when a verb appears in its \isi{passive} form, as in \REF{ex:10.krzek} and \REF{ex:11.krzek}.

\ea{\label{ex:10.krzek}
\langinfo{Arabic}{}{\citealt[4]{FassiFehri2009}}\\
\gll Y-u-jlas-u     hunaa  waqt-a       l-istiraahat-i.\\
3-\textsc{pass}{}-sit-\textsc{ind}    here   time-\textsc{acc}   the-brake-\textsc{gen}\\
\glt ‘One sits here at brake time.’ 
}
\z

\ea{\label{ex:11.krzek}
\langinfo{Arabic}{}{\citealt[6]{FassiFehri2009}}\\
\gll Wa-y-u-xraj-u         la-hu    yawm-a     l-qiyaamat-i      kitaab-an.\\
and-3-\textsc{pass}{}-bring-\textsc{ind}    to-him    day-\textsc{acc}    the-resurrection-\textsc{gen} book-\textsc{acc} \\
\glt ‘And someone will bring to him a book the day of the resurrection.’
}
\z

As illustrated by \REF{ex:10.krzek} and \REF{ex:11.krzek}, the kind of \isi{passive} construction discussed by \citet{FassiFehri2009} can occur with both transitive and intransitive verbs, and it does not support a \textit{by}{}-phrase. Contrary to personal passives, in the \isi{passive} construction in question objects are not promoted to the subject position, and they retain their accusative case, as in \REF{ex:11.krzek} above. What is more, the \isi{null subject} of the \ili{Arabic} construction binds reflexives/reciprocals, and it controls the subject of a participial clause. This is illustrated by the examples in \REF{ex:12.krzek} from   \citet[17]{FassiFehri2009}.

\ea{\label{ex:12.krzek}
\langinfo{Arabic}{}{\citealt[17]{FassiFehri2009}}\\
\gll Y-u-ģ-t-asal-u          hunaa.\\
3-\textsc{pass}-ref-wash-\textsc{ind} here \\
\glt ‘One washes oneself here.’

\ex
\gll Y-u-tasallalu ard-an fard-an ʕabra l-hawaajizi daʕimiina baʕ d-un baʕd-an. \\
3-\textsc{pass}-infiltrate   individual-\textsc{acc}  individual-\textsc{acc} across   the-barriers supporting.\textsc{pl}.\textsc{acc} each-\textsc{nom} each-\textsc{acc}\\
\glt 'People will infiltrate through barriers, supporting each other.’\\
}
\z


A very brief overview of the morphosyntactic properties of this \ili{Arabic} \isi{passive} suggests that the properties displayed are not those typical of canonical passives, as identified \citet{Blevins2003}\footnote{For \citet[512]{Blevins2003} ‘passivisation is a detransitivising operation that deletes a subject term in the argument structure of a verb’. The logical subject can then be reintroduced into the structure by means of an oblique phrase. Impersonalised verb forms, on the other hand, ‘preserve the lexical transitivity of their input retain an unexpressed subject that characteristically determines an active indefinite interpretation and may even provide an antecedent for reflexive pronouns’ \citep[508]{Blevins2003}.} , but rather strikingly similar to those of the \ili{Polish} -NO/-TO construction, which has been traditionally analysed by \ili{Slavic} linguists as ‘active indefinite’ (\citealt{Wierzbicka1966,Doros1975,Brajerski1979,Bogusławski1984,Siewierska1988}; and \citealt{Rozwadowska1992}). It may be then that this \ili{Arabic} construction should be reanalysed as active. Space limitations, however, do not allow for a more in-depth analysis of this issue to be carried out here.

\section{The interpretation of the subject in the -NO/-TO construction in Polish }

The covert subject of the -NO\textit{/}{}-TO\textit{} impersonal triggers masculine plural marking on adjectival and nominal predicative complements, suggesting that the \isi{null subject} is specified as \oldstylenums{3}\textsc{pl}.\textsc{masc}. Despite its specification, however, it can be used with reference to participants that are other than masculine, plural or speaker and addressee exclusive \citep{Kibort2004}. \citet{Kibort2004} notes that as long as the inflectional criteria are fulfilled, the construction can be found in a variety of contexts, implying that the referent of the agent is non-masculine, as in \REF{ex:13.krzek}; or that it is other than 3\textsuperscript{rd} person or plural, as in \REF{ex:14.krzek}.

\ea{\label{ex:13.krzek}
\langinfo{Polish}{}{\citealt[284]{Kibort2004}}\\
\gll Kochano swoich mężów.\\
loved.\textsc{imp}     own.\textsc{acc}   husbands.\textsc{acc}\\
\glt ‘[They] loved [their] husbands.’
}
\z

\ea \ea{\label{ex:14.krzek}
\langinfo{Polish}{}{\citealt[284, footnote 19]{Siewierska1988}}\\

\gll Mówiono o tym wyżej. \\
talked.\textsc{imp} about this higher\\
\glt ‘[One] discussed this above.’ (meaning: ‘As I/we said above’)



\ex
\langinfo{Polish}{}{\citealt[285]{Kibort2004}}\\
\gll Proszę pani,     ja się   nie     awanturuję, tylko proszę,   żeby  mi        wydano     zaświadczenie. \\
please madam,     I  \textsc{refl}   \textsc{neg}   brawl.\oldstylenums{1}\textsc{sc}    only  ask.\oldstylenums{1}\textsc{sc}  that    me.\textsc{dat} issued.\textsc{imp}   certificate.\textsc{acc} \\
\glt ‘Madam, I am not brawling, but only asking that [one] would issue the certificate to me.’ (meaning: ‘... I am only asking you to issue the certificate to me’, said by a customer to an uncooperative clerk)      



\ex 
\label{ex:15.krzek}
\gll A      w tym roku   na co  wydaliśmy  najwięcej? \\
    and   in this year  on what spent.\oldstylenums{1}\textsc{pl}  the-most\\
\glt ‘And what did we spent the most on this year?’
\ex
\gll W   tym  roku  najwięcej  wydawano  na czynsz. \\
        in  this  year    the-most    spent.\textsc{imp}    on rent \\
\glt ‘This year [we] spent the most on rent.’
}
\z
\z

The sentences in \REF{ex:14.krzek} demonstrate that the subject of the -NO/-TO construction can refer to a group of people that includes the speaker and the addressee, suggesting that generic inclusive reading of the \isi{null subject} pronoun is possible in the -NO/-TO. This observation has further consequences. Firstly, it shows that null inclusive generic subjects are available in consistent NSL in active sentences. If this is the case, then the typology of \isi{null subject} languages should be revisited. Another point worth mentioning with respect to the subject of the -NO/-TO is a possible bearing it may have on the feature geometry of pronouns (\citealt{HarleyRitter2002}). Contrary to the subject of the \ili{Italian} impersonal SI construction \REF{ex:16.krzek} or \ili{Polish} SIĘ construction, the subject of the -NO/-TO construction is not logophoric. It means that it does not refer back to the ‘reporting’ speaker (in \ref{ex:17I.krzek}).

\ea%16
\label{ex:16.krzek}
\citep[173]{DAlessandro2007}\\
\gll	 \ul{Maria e Gianni} hanno raccontato che \ul{si} era   mangiato bene in  quel locale.\\
          \ul{Maria and Gianni}  have   told         that  \ul{si}  was  eaten     well  in  that  place\\
\glt ‘Maria and Gianni have told that they had eaten well at that place.’
\z

\ea%17
\label{ex:17I.krzek}
\gll	 Maria i       Paweł   powiedzieli że             oglądano        te      filmy      często. \\
[Maria and Paweł]\textsubscript{i} said.\oldstylenums{3}\textsc{pl}    {that pro\textsubscript{j/*i}} watched.\textsc{imp}  these movies   often\\
\glt           (Intended) ‘Maria and Paweł said that they watched these movies often.’
\z

\citet{DAlessandro2007} reports that \textit{si} in \REF{ex:16.krzek} is logophoric. This means that it refers back to the person who reports what happens, rather than to the person who utters the whole sentence. \textit{si} then receives its inclusive interpretation by being anchored to the Speech Act, where the [\isi{Speaker}] and the [\isi{Addressee}] features are located (\citealt{DAlessandro2007,Sigurðsson2004,Bianchi2003Tense}). It is, however, not clear how an inclusive interpretation is achieved with the subject of the -NO/-TO, as it does not refer back to the reporting speaker. It is possible that one of the reasons why the \isi{null subject} of the \ili{Polish} construction cannot refer to the ‘reporting’ speaker may have to do with a more general ban on it being bound. Consider the examples in \REF{ex:17.krzek} below.

\ea%17
    \label{ex:17.krzek}
	 \ea\label{ex:17.krzeka}
\gll  Marysia                       słuchała                                           muzyki       kiedy   [pro\textsubscript{i}] gotowano. \\
      Marysia\textsubscript{j/*i}   listented.\oldstylenums{3}\textsc{sc}.\textsc{fem} music        when  [pro\textsubscript{i}]   cooked.\textsc{imp}\\
\glt  ‘Marysia listened to music when [they/people] cooked.’

 \ex\label{ex:17.krzekb}
\gll  Marysia                   słucha                               muzyki     kiedy      [pro\textsubscript{j}]                   gotuje.\\        
      Marysia\textsubscript{j}  listens.\oldstylenums{3}\textsc{sc}  music      when      [pro\textsubscript{j}]    cooks.\oldstylenums{3}\textsc{sc}\\
\glt  ‘Marysia listens to music when she cooks.’
\z
\z

In \REF{ex:17.krzekb} the main clause subject \textit{Marysia} is coreferential with the \isi{null subject} of the subordinate clause. In \REF{ex:17.krzeka}, on the other hand, such coreferentiality between \textit{Marysia} and a \isi{null subject} in the subordinate clause is not possible. I propose that this may well be caused by the difference in the feature-geometry make-up of pronouns (\citealt{HarleyRitter2002}).\footnote{%
 \begin{forest}
 [Pronoun
  [Participant
    [\isi{Speaker}] [\isi{Addressee}]
  ]
  [Individuation
    [Group] [Minimal [Augmented]] [Class
				    [Animate [Masculine] [Feminine]] [Inanimate]
    ]
  ]
 ]
 \end{forest}\newline\citep{HarleyRitter2002}} To be more specific, it may be that the [Participant] feature in the geometry is underspecified with respect to the [\isi{Speaker}] and the [\isi{Addressee}] features, such that the [Participant] feature cannot be specified any further. Alternatively, it may be that the [Participant] feature is deleted altogether. This null pronoun is then similar to 3\textsuperscript{rd} person pronouns for which the [Participant] feature either does not exist in their featural make-up or is present but underspecified, and as such they can only be bound by Topics and not by logophoric features. Now in order to explain how the inclusive interpretation is attained, I propose that the interpretation of the \isi{null subject} of the -NO/-TO construction will depend on a binding relation with the null \isi{Topic}\footnote{An initial investigation suggests that it may be an Aboutness \isi{Topic} \citep{Frascarelli2007}, but more research is needed to establish whether this is really the case.} \citep{FrascarelliHinterhölzl2007} that is merged in the C-domain. Consider the extract in \REF{ex:18.krzek}. The examples (\ref{ex:18.krzek}ii) and (\ref{ex:18.krzek}iii) are answers to (\ref{ex:18.krzek}i).

\ea%18
\label{ex:18.krzek}
\begin{xlist}[(iii)]
\exi{(i)}
\gll Na co   my Polacy  wydawaliśmy   najwięcej   w ubiegłym   roku? \\
            on what  we Poles  spent.\oldstylenums{1}\textsc{pl}        most     in last     year\\
\glt ‘What did we spent on most last year?’

\exi{(ii)}
\gll My sądzimy,   że       wydawano najwięcej   na czynsz. \\
             we\textsubscript{i}   think.\oldstylenums{1}\textsc{pl}    {that   [pro\textsubscript{j/?i}]}  spent.\textsc{imp}   most     on rent   \\

\glt ‘We think that [people living in Poland/they/we?] spent most on rent.’

\exi{(iii)}
\gll Eksperci    sądzą,     że       wydawano najwięcej   na czynsz. \\
             experts\textsubscript{i}    think.\oldstylenums{3}\textsc{pl}    {that  [pro\textsubscript{j/*i}]}  spent.\textsc{imp}   most     on rent   \\
\glt ‘Experts think that [people living in Poland] spent most on rent.’
\end{xlist}
\z

In \REF{ex:18.krzek} speaker (i) introduces \textit{my Polacy} ‘we Poles’ as a \isi{Topic}. This \isi{Topic} is then re-merged as a silent copy in the C-domain in (ii).  The \isi{null subject} (\textit{pro}) in (ii) refers back \textit{my Polacy} ‘we Poles’ as it is an established \isi{Topic}. If \textit{my} ‘we’ in (ii) has the same referent as \textit{my} ‘we’ in (i), which is a \isi{Topic}, then the \textit{pro} in (ii) may accidentally be  coreferential with \textit{my} ‘we’ in (ii). Crucially, however, for my informants \textit{my} ‘we’ in (ii) does not have to be coreferential with \textit{pro}, and for some of them it cannot. In other words, those who think do not have to be/cannot be those who spent most on rent in (\ref{ex:18.krzek}ii). In (\ref{ex:18.krzek}iii) again, the \isi{Topic} \textit{my Polacy} ‘we Poles’ is remerged in the C-domain, and the \isi{null subject} \textit{pro} refers back to that \isi{Topic}, and it cannot be coreferential with the subject \textit{eksperci} ‘experts’. These data suggest that for the \isi{null subject} in the -NO/-TO construction to receive a generic interpretation, it needs to be bound by a \isi{Topic}.

\section{Conclusion}
This squib presented evidence that \ili{Polish}, a consistent NSL, has an impersonal active construction whose subject can receive an inclusive interpretation. The \ili{Polish} construction shares a number of morphosyntactic properties with a type of a \isi{passive} construction in \ili{Arabic} \citep{FassiFehri2009} – a consistent NSL as well – the subject of which can also receive a generic interpretation. It is, however, clear that the range of occurrence of inclusive generic subjects in these languages is very restricted. In the -NO/-TO construction the generic interpretation arises only when the \isi{null subject} is bound by a \isi{Topic} that has a generic referent.\footnote{According to \citet[707]{Frascarelli2007}, an indefinite DP can be a \isi{Topic} when it is intended as specific indefinite; that is, when it is used to refer to specific type of referent.} It remains to be investigated whether there is any relation between the uninflected verb form used in the -NO/-TO construction and the availability of a generic interpretation that a \isi{null subject} occurring in it can receive.



\section*{Abbreviations}
Abbreviations used in this article follow the Leipzig Glossing Rules’ instructions for word-by-word transcription, available at: \url{https://www.eva.mpg.de/lingua/pdf/Glossing-Rules.pdf}.
% \section*{Acknowledgements}

\sloppy
\printbibliography[heading=subbibliography,notkeyword=this]
\end{document}
