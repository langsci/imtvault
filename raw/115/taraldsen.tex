\documentclass[output=paper]{LSP/langsci}
\author{Tarald Taraldsen\affiliation{University of Tromsø}}
\title{Places}
% \epigram{Change epigram}
\abstract{In Norwegian, a locative PP can occur as the subject of the copula just in case the complement of the copula is a relative construction with \textit{sted} or \textit{place}, both meaning `place', as its head noun. I examine the properties of this construction and ultimately propose an analysis based on a specific view of locative PPs as well as a novel assumption about the ways A-movement and A’-movement may interact.
}
\ChapterDOI{10.5281/zenodo.1116759}

\maketitle

\begin{document}
\section{Introduction}\label{sec:taraldsen:1}
In this article, I will look at some curious properties of \ili{Norwegian} sentences like those in (\ref{ex:taraldsen:1}--\ref{ex:taraldsen:2}):\footnote{The meaning is not `In Tromsø there is a nice place to live' which would be the meaning of (i):
\begin{xlisti}
	\exi{(i)}\label{ex:taraldsen:i}\gll I Tromsø er det et bra sted å bo.\\
        in Tromsø is it a nice place to live.\\
\end{xlisti}
}

\ea%1
    \label{ex:taraldsen:1}
    \gll I Tromsø er et bra sted å bo.\\
       in Tromsø is a nice place to live. \\
    \z

\ea%2
    \label{ex:taraldsen:2}
	\gll   I Tromsø er et sted det er morsomt å arbeide.\\
	      in Tromsø is a place it is fun          to work\\
\z

I will present evidence that the initial PPs in (\ref{ex:taraldsen:1}--\ref{ex:taraldsen:2}) are in the usual subject position. After rejecting an alternative analysis in \sectref{sec:taraldsen:3}, I will also argue that these PPs are derived subjects raised to the subject position of the copula from inside the \isi{relative} clause\footnote{I take \textit{et bra sted å bo} in \REF{ex:taraldsen:1} to contain an infinitival \isi{relative} clause, ignoring the question how such constructions relate to Tough Movement constructions like \textit{Dette stedet er bra å bo på} – ‘This place is nice to live in’. The fact that the stranded \isi{preposition} cannot be left out in the Tough Movement constructions (see the comments on example \ref{ex:taraldsen:35} in \sectref{sec:taraldsen:4.1}) suggests that the relation cannot be too tight.} and will discuss the theoretical issues that arise from this (\sectref{sec:taraldsen:4}).

A key fact about sentences like (\ref{ex:taraldsen:1}--\ref{ex:taraldsen:2}) is that the head noun of the \isi{relative} construction must be \textit{sted} or \textit{plass}, which both means `place'.\footnote{%
Plass can replace sted in (\ref{ex:taraldsen:1}--\ref{ex:taraldsen:2}), as in (i--ii), and all other grammatical examples in the text:
\begin{xlisti}[(ii)]  
	\exi{(i)}\label{ex:taraldsen:3i}
	\gll I Tromsø er en bra plass å bo.\\
	    in Tromsø is a nice place to live\\
	\exi{(ii)}\label{ex:taraldsen:3ii}
	\gll I Tromsø er en plass det er morsomt å arbeide.\\
	     in Tromsø is a place it is fun to work\\
\end{xlisti}
} Correspondingly, a key element in the analysis I suggest, is the special status of these nouns in the formation of locative expressions.

\section{Some basic facts}\label{sec:taraldsen:2}

I will begin by identifying the special properties that sentences like (\ref{ex:taraldsen:1}--\ref{ex:taraldsen:2}) have.

\subsection{Spatial PPs as subjects of copulative sentences}\label{sec:taraldsen:2.1}

In (\ref{ex:taraldsen:1}--\ref{ex:taraldsen:2}), the locative PP \textit{i Tromsø} `in Tromsø' is linked by the copula to a predicate consisting of a \isi{relative} clause headed by a noun:

\ea%3
    \label{ex:taraldsen:3}
    \gll I Tromsø er et bra sted å bo.\\
       in Tromsø is a nice place to live \\
\z

\ea
	\label{ex:taraldsen:4}
	\gll   I Tromsø er et sted det er morsomt å arbeide.\\
	      in Tromsø is a place it is fun to work\\
\z

The usual tests suggest that the PP is really the subject:

\ea%5
    \label{ex:taraldsen:5}
    \ea\gll Derfor er i Tromsø blitt et bra sted å bo.\\
	    therefore is in Tromsø become a nice place to live\\
    \ex\gll Nå er i Tromsø blitt et sted det er morsomt å arbeide.\\
         now is in Tromsø become a place it is fun  to work\\
         \z
\z


\ea%6
    \label{ex:taraldsen:6}
  \ea 	\gll  I Tromsø synes å være et bra sted å bo.\\
	      in Tromsø seems to be  a  nice place to live\\
  \ex	\gll I Tromsø påstås å være et sted det er morsomt å arbeide.\\
	    in Tromsø is.claimed to be a place it is fun     to work\\
\z\z


\ea%7
    \label{ex:taraldsen:7}
	  \ea\gll I Tromsø mener vi (*at)  er et bra sted å bo.\\
		  in Tromsø think we (*that) is a nice place to live\\
	  \ex\gll I Tromsø synes vi (*at) er et sted det er morsomt å arbeide.\\
		  in Tromsø think we (*that) is a place it is fun      to work\\
\z\z

The examples in \REF{ex:taraldsen:5} show that the PP appears between an \isi{auxiliary} in the V2-position and a participle just like ordinary subjects. Those in \REF{ex:taraldsen:6} show PPs undergoing raising-to-subject, and the examples in \REF{ex:taraldsen:7} illustrate the *\textit{that}-trace effect triggered by extraction of PPs like those in (\ref{ex:taraldsen:1}--\ref{ex:taraldsen:2}).\footnote{\ili{Norwegian} speakers show variation with respect to *\textit{that}-t effect. Speakers who tolerate at ‘that’ in \textit{Hvem tror du at har vunnet?} ‘who think you that has won etc., should also allow it in \REF{ex:taraldsen:7}.}

\subsection{The importance of the relative clause}\label{sec:taraldsen:2.2}

The \isi{relative} clause is essential:

\ea%86
    \label{ex:taraldsen:8}
    \gll *  I Tromsø er et bra sted nord for Polarsirkelen.\\
	 {} in Tromsø is a nice place north of {the Arctic Circle}\\
\z

\noindent\REF{ex:taraldsen:8} contrasts with \REF{ex:taraldsen:9}, where the subject is not a PP:

\ea%9
    \label{ex:taraldsen:9}
    \gll Tromsø er et bra sted nord for Polarsirkelen.\\
	 Tromsø is a nice place north of {the Arctic Circle}\\
\z

\subsection{The importance of the head noun}\label{sec:taraldsen:2.3}

It is also essential that the head noun of the \isi{relative} clause be \textit{sted} or \textit{plass} (both `place'):

\ea%10
    \label{ex:taraldsen:10}
    \ea[*]{\gll I Tromsø er en bra by å bo.\\
		in Tromsø is a nice city to live\\}
    \ex[*]{\label{ex:taraldsen:8b}
\gll I Tromsø er en by det er morsomt å arbeide.\\
		in Tromsø is a city it is   fun         to work\\}
    \z
\z
This is presumably related to the fact that \textit{sted} and \textit{plass} are the only nouns that can form a locative adjunct without a (overt) \isi{preposition}:

\ea%11
\label{ex:taraldsen:11}
\ea \gll Vi arbeidet (på) {det samme stedet/den samme plassen} i tre år.\\
         we worked (at)  {the same place}           for three years\\
\ex \label{ex:taraldsen:11b} \gll Vi har nettopp besøkt {et sted/en plass} vi bodde (på) i 1981.\\
         we have just    visited {a  place}             we lived (at)  in 1981\\
\z\z

\ea
\label{ex:taraldsen:12}
\ea \gll Vi arbeidet *(i) den samme byen  i tre år.\\
         we worked *(in) the same city     for three years\\
\ex\label{ex:taraldsen:12b} \gll Vi har nettopp besøkt en by vi bodde *(i) for ti år siden.\\
	 we have just    visited a  city we lived *(in) {}  ten years ago\\
\z\z

\subsection{No stranded preposition in the relative clause}\label{sec:taraldsen:2.4}

If a stranded \isi{preposition} is inserted into the \isi{relative} clause in \REF{ex:taraldsen:10}, just as in \REF{ex:taraldsen:12b}, the outcome is still ungrammatical, in contrast with \REF{ex:taraldsen:14}:

\ea%13
\label{ex:taraldsen:13}
\ea[*]{\gll I Tromsø er en bra by å bo i.\\
          in Tromsø is a nice city to live in.\\}
\ex[*]{\gll I Tromsø er en by det er morsomt å arbeide i.\\
	    in Tromsø is a city it  is  fun         to work   in.\\}
\z\z

\ea%14
    \label{ex:taraldsen:14}
    \ea \gll Tromsø er en bra by å bo i.\\
	     Tromsø is a nice city to live in.\\
    \ex \gll Tromsø er en by det er morsomt å arbeide i.\\
	     Tromsø is a city it is fun to work in.\\
	 \z
\z

Likewise, the stranded \isi{preposition}, which is optional in \REF{ex:taraldsen:11b}, makes (\ref{ex:taraldsen:1}--\ref{ex:taraldsen:2}) ungrammatical:

\ea%15
    \label{ex:taraldsen:15}
\ea[*]{I Tromsø er et bra sted å bo på.\\
       in Tromsø is a nice place to live at\\}
\ex[*]{I Tromsø er et sted det er morsomt å arbeide på.\\
       in Tromsø is a place it is  fun         to work    in\\}
\z
\z

In this case, the subject must lose its \isi{preposition} exactly as in \REF{ex:taraldsen:14} and \REF{ex:taraldsen:9}:\footnote{These sentences are also fine without a stranded \isi{preposition} in the \isi{relative} clause, just like (\ref{ex:taraldsen:1}--\ref{ex:taraldsen:2}):

\begin{xlist}
 \exi{(i)}
 \begin{xlista}
 \ex \gll Tromsø er et bra sted å bo.\\
	  Tromsø is a nice place to live\\
 \ex \gll Tromsø er et sted det er morsomt å arbeide.\\
	  Tromsø is a place it is fun to work\\
 \end{xlista}
\end{xlist}
}

\ea
\label{ex:taraldsen:16}
\ea \gll Tromsø er et bra sted å bo på.\\
          Tromsø is a nice place to live at\\
\ex \gll Tromsø er et sted det er morsomt å arbeide på.\\
         Tromsø is  a place it  is  fun         to work    at\\
\z\z
\subsection{Summary}\label{sec:taraldsen:2.5}

The data I have reviewed, gives rise to the following questions:

\ea%17
    \label{ex:taraldsen:17}
    \ea\label{ex:taraldsen:17a} Why must the predicative noun be \textit{sted} or \textit{plass} when the subject of the copula is a PP?
    \ex\label{ex:taraldsen:17b} Why must there be a \isi{relative} clause modifying the predicative noun?
    \ex\label{ex:taraldsen:17c} Why can’t there be a stranded \isi{preposition} in the \isi{relative} clause?
    \z
\z

In the next section, I will sketch two ways of providing answers to these questions. Both ultimately turn on where PPs can be introduced by external merge, but make different assumptions as to where exactly that is.

\section{Two analytical options}\label{sec:taraldsen:3}

The first analysis suggested below answers question \REF{ex:taraldsen:17a} by saying that when the subject of the copula is a PP, the complement of the copula must be a PP as well. Then, the contrast between (\ref{ex:taraldsen:1}--\ref{ex:taraldsen:2}) and sentences like \REF{ex:taraldsen:10} follows, if \textit{sted} and \textit{plass} license a silent locative P, but no other noun does, as suggested by the contrast between \REF{ex:taraldsen:11} and \REF{ex:taraldsen:12}. However, this account requires untenable \isi{auxiliary} assumptions to provide answers to (\ref{ex:taraldsen:17b}--\ref{ex:taraldsen:17c}). The second analysis answers questions (\ref{ex:taraldsen:17b}--\ref{ex:taraldsen:17c}) directly by claiming that a PP subject must be a derived subject, but an answer to \ref{ex:taraldsen:17a} will only be forthcoming in \sectref{sec:taraldsen:4}.

\subsection{Categorial matching}\label{sec:taraldsen:3.1}

Suppose we take the grammaticality of \REF{ex:taraldsen:18} without \textit{på} to mean that \textit{sted} and \textit{plass} allow a locative \isi{preposition} to be silent:

\ea%18
    \label{ex:taraldsen:18}
    \ea \gll Vi arbeidet (på) {det samme stedet/den samme plassen} i tre år.\\
	     we worked (at)  {the same place}                        for three years\\

    \ex \gll Vi har nettopp besøkt {et sted/en plass} vi bodde (på) i 1981.\\
	     we have just    visited {a  place}             we lived (at)  in 1981\\
    \z
\z

In \REF{ex:taraldsen:19}, P represents the silent locative \isi{preposition}:

\ea%19
    \label{ex:taraldsen:19}
    \ea \gll Vi arbeidet P {det samme stedet/den samme plassen} i tre år.\\
	we worked  {}  {the same place}                        for three years\\
    \ex \gll Vi har nettopp besøkt {et sted/en plass} vi bodde P i 1981.\\
             we have just    visited {a  place}             we lived {}  in 1981\\
\z \z
Then, the obligatoriness of the overt \isi{preposition} in \REF{ex:taraldsen:12}  may be taken to show that only \textit{sted} and \textit{plass} license a silent P:

\begin{exe}%1
    \exr{ex:taraldsen:12}
    \ea \gll Vi arbeidet *(i) den samme byen  i tre år.\\
         we worked *(in) the same city     for three years\\
    \ex \gll Vi har nettopp besøkt en by vi bodde *(i) for ti år siden.\\
	 we have just    visited a  city we lived *(in) {} ten years ago\\
\z
\end{exe}

Correspondingly, (\ref{ex:taraldsen:1}--\ref{ex:taraldsen:2}) might be taken to contain silent prepositions too, as in \REF{ex:taraldsen:21}:

\ea%20
    \label{ex:taraldsen:21}
   \ea \gll I Tromsø er {[\textsubscript{PP} P} et bra sted å bo.]\\
	    in Tromsø is {}  a nice place to live\\
   \ex \gll I Tromsø er {[\textsubscript{PP} P} et sted det er morsomt å arbeide.]\\
	    in Tromsø is {}  a place it is fun          to work\\
\z\z

But \REF{ex:taraldsen:10} may not:

\begin{exe}
    \exr{ex:taraldsen:10}
    \ea[*]{\gll I Tromsø er en bra by å bo.\\
		in Tromsø is a nice city to live\\}
    \ex[*]{\label{ex:taraldsen:10b}\gll I Tromsø er en by det er morsomt å arbeide.\\
		in Tromsø is a city it is   fun         to work\\}
    \z
\end{exe}


\ea%23
    \label{ex:taraldsen:23}
    \ea[*]{\gll I Tromsø er {[\textsubscript{PP} P} en bra by å bo ].\\
		in Tromsø is     {}    a nice city to live\\}
    \ex[*]{\gll I Tromsø er {[\textsubscript{PP} P} en by det er morsomt å arbeide].\\
		in Tromsø is     {}   a city it is   fun         to work\\}
    \z
\z

Then, the ungrammaticality of \REF{ex:taraldsen:10} might be due to a mismatch between the category of the subject and the category of the complement of the copula:

\ea%24
    \label{ex:taraldsen:24}
    \ea \gll [\textsubscript{PP} I Tromsø] er [\textsubscript{DP} en bra by å bo].\\
              {}  in Tromsø is    {}   a nice city to live\\
    \ex \gll [\textsubscript{PP} I Tromsø] er [\textsubscript{DP} en by det er morsomt å arbeide].\\
             {} in Tromsø is    {}    a city it is   fun     to work\\
\z\z

It should be clear that this approach does not presuppose that \textit{er} `is' has the semantics of an “identificational copula”. In fact, \textit{er} is to be regarded as an identity function passing on the denotation of its complement. The complement of \textit{er}, then, is the predicate that would have to be applicable to the subject, but the type of things the predicate applies to may be determined by its syntactic category. Thus, the analysis we are examining is ultimately based on the assumption that the syntactic categories DP and PP correspond to different \isi{semantic} types.\footnote{\label{fn:taraldsen:6}Sentences like \textit{Tromsø er i Nord Norge} - Tromsø is in Northern Norway - are fine. In these, \textit{er} can be replaced with \textit{ligger} `lies' or `is situated', an option not available when the subject is a PP as in (\ref{ex:taraldsen:1}--\ref{ex:taraldsen:2}) or when \textit{er} has an adjectival complement. That is, \textit{er} `is' can also be assigned a meaning such that its complement is not predicated of the subject the way it is in (\ref{ex:taraldsen:1}--\ref{ex:taraldsen:2}).}

But to answer question \REF{ex:taraldsen:17b}, we must also assume that a \isi{preposition} cannot be merged to the complement of the copula so that \REF{ex:taraldsen:8} cannot be analyzed as in \REF{ex:taraldsen:24}:

\begin{exe}
 \exr{ex:taraldsen:8}
  \gll *  I Tromsø er et bra sted nord for Polarsirkelen.\\
       {} in Tromsø is a nice place north of {the Arctic Circle}\\
\end{exe}

\ea%26
    \label{ex:taraldsen:26}
    \gll [\textsubscript{PP} I Tromsø] er {[\textsubscript{PP} P} et bra sted nord for Polarsirkelen].\\
          {}  in Tromsø is    {}     a nice place north of {the Arctic Circle}\\
\z
Then, (\ref{ex:taraldsen:1}--\ref{ex:taraldsen:2}) must be derived as indicated in \REF{ex:taraldsen:27}:

\ea%27
    \label{ex:taraldsen:27}
    \ea \gll [\textsubscript{PP} I Tromsø] er {[\textsubscript{CP}[\textsubscript{PP} P} et bra sted] å bo \st{PP}].\\
              {} in Tromsø is       {}   a nice place to live\\
    \ex \gll [\textsubscript{PP} I Tromsø] er [\textsubscript{CP}[\textsubscript{PP} P et sted] det er morsomt å arbeide \st{PP}].\\
              {} in Tromsø is       {} {}   a place  it   is fun          to work \\
    \z
\z

If so, we also have answer to question \REF{ex:taraldsen:17c}. Given the stranded \isi{preposition}, the sentences in \REF{ex:taraldsen:15} must parsed as in \REF{ex:taraldsen:29}:

\begin{exe}
 \exr{ex:taraldsen:15}
  \ea[*]{\gll I Tromsø er et bra sted å bo på.\\
       in Tromsø is a nice place to live at\\}
  \ex[*]{\gll I Tromsø er et sted det er morsomt å arbeide på.\\
       in Tromsø is a place it is  fun         to work    in\\}
  \z
\end{exe}

\ea\label{ex:taraldsen:29}
  \ea \gll [\textsubscript{PP} I Tromsø] er [\textsubscript{CP}[\textsubscript{DP} et bra sted] å bo [\textsubscript{PP} på \st{DP}]].\\
             {}                in Tromsø  is     {}                                a nice place to live {} at\\
  \ex \gll [\textsubscript{PP} I Tromsø] er [\textsubscript{CP}[\textsubscript{DP} et sted] det er morsomt å arbeide [\textsubscript{PP} på \st{DP}].\\
             {}               in Tromsø  is    {}                                  a place  it is  fun      to work             {}        at\\
  \z
\z

But the derivation indicated in \REF{ex:taraldsen:27} would be a “head raising” derivation of the \isi{relative} constructions where the raised constituent is a PP, and although the head raising analysis may be justified when the head is a NP or DP (see \sectref{sec:taraldsen:4.3} below), extending it to PPs raises a number of problems. In particular, it begs the question why the silent P in \REF{ex:taraldsen:27} cannot be replaced with an overt \isi{preposition}:

\ea%30
    \label{ex:taraldsen:30}
    \ea[*]{I Tromsø er på et bra sted å bo.\\
           in Tromsø is at a nice place to live\\}
    \ex[*]{I Tromsø er på et sted det er morsomt å arbeide.\\
           in Tromsø is at a place it is fun          to work\\}
\z\z

In fact, head-raising must be allowed to pied-pipe a \isi{preposition} only when the complement of the \isi{preposition} is a wh-phrase. Thus, \REF{ex:taraldsen:31a} is acceptable (in a formal register), but \REF{ex:taraldsen:31b} is not:

\ea%24
    \label{ex:taraldsen:31}
    \ea[]{\label{ex:taraldsen:31a} \gll Vi fant et sted på hvilket det er morsomt å arbeide.\\
				     we found a place at which it is fun         to work\\}
    \ex[*]{\label{ex:taraldsen:31b} \gll Vi fant på et sted det er morsomt å arbeide.\\
				      we found at a place it is fun        to work\\}
\z\z

Hence, the matching account seems to rest on untenable assumptions.

\subsection{The subject PP comes from the relative clause}\label{sec:taraldsen:3.2}

The second line of analysis I will look at, is based on the assumption that a PP may not appear in the subject position of the copula by external merge. This may follow from proposals like those in \citet[282--313]{Kayne1999}, which, among other things, are designed to account for subject/object asymmetries with respect to prepositional complementizers.

If so, we are led to conclude that a subject PP is always a derived subject, a PP formed below the subject position and subsequently raised, as in sentences with  “locative inversion”. But then the PP subject in (\ref{ex:taraldsen:1}--\ref{ex:taraldsen:2}) must be a derived subject too.

When we ask where the subject PP in (\ref{ex:taraldsen:1}--\ref{ex:taraldsen:2}) comes from, the only possible answer seems to be that it actually has been extracted from the \isi{relative} clause:

\ea%25
  \settowidth\jamwidth{=(1)}
    \label{ex:taraldsen:32}
    \gll\relax [\textsubscript{PP} i Tromsø] er [ et bra sted å bo \st{PP}].\\
                     {}           in Tromsø is  {}  a nice place to live\\\jambox{=\REF{ex:taraldsen:1}}
\z

This analysis provides a straightforward explanation why \REF{ex:taraldsen:8} and \REF{ex:taraldsen:15} are ungrammatical:

\begin{exe}%6
\exr{ex:taraldsen:8}
\ea\gll *  I Tromsø er et bra sted nord for Polarsirkelen.\\
        {} in Tromsø is a nice place north of {the Arctic Circle}\\
\z
\end{exe}

\begin{exe}
 \exr{ex:taraldsen:15}
  \ea[*]{\gll I Tromsø er et bra sted å bo på.\\
       in Tromsø is a nice place to live at.\\}
  \ex[*]{\gll I Tromsø er et sted det er morsomt å arbeide på.\\
       in Tromsø is a place it is  fun         to work    in\\}
  \z
\end{exe}

In \REF{ex:taraldsen:8}, there is no position the subject PP could have moved from, since there is no constituent modifiable by a PP. In \REF{ex:taraldsen:15}, there is a position modifiable by a PP (the VP headed by \textit{bo} `live'), but the subject PP cannot have moved from that position, since there is a stranded P. Thus, we have answers to the questions (\ref{ex:taraldsen:17b}--\ref{ex:taraldsen:17c}).

On the other hand, the new analysis does not yet provide an answer to question \REF{ex:taraldsen:17a}, i.e. it doesn’t explain why no other noun can replace \textit{sted} or \textit{plass} in (\ref{ex:taraldsen:1}--\ref{ex:taraldsen:2}). It also raises the question how a locative PP manages to raise to the subject position of the copula from inside a \isi{relative} construction. In the next section, however, I will suggest an answer to this question which also leads to an answer to question \REF{ex:taraldsen:17a}.

\subsection{Summary}\label{sec:taraldsen:3.3}

I began this section by sketching an apparently simple account of (\ref{ex:taraldsen:1}--\ref{ex:taraldsen:2}) based on categorial matching, This account would provide an answer to question \REF{ex:taraldsen:17b}, but cannot answer questions \REF{ex:taraldsen:17a} and \REF{ex:taraldsen:17c} without adding assumptions that were seen to be untenable. Thus, I suggested a different analytical option based on the assumption that the PP subject in (\ref{ex:taraldsen:1}--\ref{ex:taraldsen:2}) must be a derived subject moved out of the \isi{relative} clause. This analysis will be more fully developed in the next section.

\section{The proposal}\label{sec:taraldsen:4}

To develop the analysis sketched in \sectref{sec:taraldsen:3.2}, I will first attempt to capture what is special about \textit{sted} and \textit{plass}. This will provide a way of understanding how a locative PP can move out of the \isi{relative} clause in the derivation of (\ref{ex:taraldsen:1}--\ref{ex:taraldsen:2}) just in case the head of the \isi{relative} construction is \textit{sted} or \textit{plass}.

\subsection{What’s special about \textit{sted}?}\label{sec:taraldsen:4.1}

Saying that \textit{sted} and \textit{plass} can be locatives without an overt \isi{preposition} because they have the unique property of licensing a silent locative \isi{preposition}, seems to beg the question why exactly only \textit{sted} and \textit{plass} should have this property. There is also an empirical issue. Consider first \REF{ex:taraldsen:35}, where the stranded \isi{preposition} cannot be omitted:

\ea%26
    \label{ex:taraldsen:35}
    \gll Dette stedet er bra å bo *(på).\\
	  this place is nice to live  at\\
\z

\REF{ex:taraldsen:35} with the stranded \isi{preposition} is simply a Tough Movement construction with a stranded \isi{preposition} analogous to \textit{This problem is hard to talk about}. But why couldn’t \REF{ex:taraldsen:35} without \textit{på} `at' simply have a stranded silent P instead of \textit{på}?

The answer to that might be that the P cannot remain silent when stranded. But then we have a problem with the following:

\ea%27
    \label{ex:taraldsen:36}
\ea
 Vi besøkte et sted vi hadde bodd i fem år\\
 \glt        we visited  a place we had lived for five years
\ex Tromsø er et bra sted å bo.
\glt         Tromsø is a nice place to live.
\z
\z

In these, \textit{sted} originates as (part of) a locative modifier in the \isi{relative} clause. If \textit{sted} can only be a locative modifier when accompanied by a silent or overt \isi{preposition}, there must be a silent P in \REF{ex:taraldsen:36} which is either stranded or has been carried along under relativization (assuming for the sake of the argument that the head-raising analysis can be extended to PPs in spite of the problem noted in \sectref{sec:taraldsen:3.1}). If we conclude from \REF{ex:taraldsen:35} that a stranded \isi{preposition} cannot be silent, we must also say that the P associated with \textit{sted} actually has been pied-piped in \REF{ex:taraldsen:36}. But this runs up against the problem that overt prepositions cannot be pied-piped in this way in sentences otherwise similar to \REF{ex:taraldsen:36}:

\ea%28
    \label{ex:taraldsen:37}
\ea \label{ex:taraldsen:37a}
\gll Vi besøkte en by vi hadde bodd \textbf{i} i fem år.\\
           we visited a city we had   lived in for five years\\

\ex \label{ex:taraldsen:37b}
\gll *Vi besøkte \textbf{i} en by vi hadde bodd i fem år.\\
          we visited in a city we had  lived for five years\\
\z
\z

This may be due either to the way movement works in the derivation of relatives (that is, a P can be pied-piped only when its complement is a wh-phrase) or to the fact that \textit{besøke} `visit' selects a DP complement, while relativizing a PP as in \REF{ex:taraldsen:37b} makes it impossible to analyze the \isi{relative} construction as a DP. Either way, we are now led to conclude that a silent P associated with \textit{sted} in \REF{ex:taraldsen:36} can be neither stranded nor pied-piped. In other words, there cannot be a silent P associated with \textit{sted} in \REF{ex:taraldsen:36}.

This leads me to abandon the idea that \textit{sted} and \textit{plass} functioning as locative modifiers must come with a silent P. Instead, I submit that these nouns are able to be locative modifiers without a \isi{preposition} (silent or otherwise) because they are inherently locative, i.e. because they mean `place'.

Putting this in slightly more precise terms, I propose that a noun whose meaning is just `place' can be used as a locative modifier providing a spatial coordinate for an eventuality without needing a \isi{preposition} to create this relation. This is in fact what we see in \REF{ex:taraldsen:36}.

From this point of view, what sets \textit{sted} and \textit{plass} apart from \textit{by} `city' and other nouns, is that only the former can be pure expressions of location.

\subsection{Places and things}\label{sec:taraldsen:4.2}

Given the preceding, one may well wonder why \textit{sted} ever co-occurs with a locative \isi{preposition}, as it optionally does:

\ea%29
    \label{ex:taraldsen:38}
    \gll Vi bodde (på) et sted i Nord-Norge.\\
         we lived  (at)  a place in {Northern Norway}\\
\z

To approach this question, we should first ask the question what the \isi{preposition} is actually doing in sentences like \REF{ex:taraldsen:39}:

\ea%30
    \label{ex:taraldsen:39}
  \gll	  Vi bodde *(\textbf{i}) en by i Nord-Norge.\\
         we lived *(in) a city in {Northern Norway}\\
\z

I have already suggested that a locative \isi{preposition} is not always needed to license a locative modifier. I will now propose that locative prepositions create a relation between a purely place-denoting noun and another noun. In (30), the other noun is \textit{by}, and I suggest that the structure of \textit{i en by} `in a city' is roughly as in \REF{ex:taraldsen:40}:

\ea%31
    \label{ex:taraldsen:40}
    \gll [ {STED} [ {i} [ en by ]]] \\
         {}  place {}   in {}  a city\\
\z

That is, \textit{i} `in' assigns a space denoted by silent \textit{sted} in its Spec as the location of the city picked as the denotation of \textit{en by} `a city'. The difference between \textit{i} `in' and \textit{på} `on, at' is that \textit{i} associates this space with the interior of an object denoted by its complement, while \textit{på} associates it with the surface of that object.\footnote{The distribution of  \textit{på} `on, at' vs. \textit{i} `in' raises additional issues that will be ignored here. For example, place names denoting cities in the inland  or islands admit \textit{på}, e.g. \textit{på Hamar}, \textit{på Island} `on Iceland', while names of coastal cities require \textit{i}, e.g. \textit{i Oslo}, \textit{i Tromsø}.}  But the \isi{preposition} is not otherwise instrumental in creating a locative modifier. Only \textit{sted} is.

Thus, \textit{sted} as a locative modifier does not need a \isi{preposition} when its denotation is not to be associated with the denotation of another noun phrase. Therefore, \REF{ex:taraldsen:38} without \textit{på} `at' can be analyzed as in \REF{ex:taraldsen:41}, without a silent P:\footnote{I abstract away from \isi{V2} movement and the question whether \textit{et} `a' is a D or part of NP, which seems immaterial at this point.}

\ea%32
    \label{ex:taraldsen:41}
    \gll Vi [\textsubscript{VP} bodde [\textsubscript{NP} et sted i Nord-Norge]].\\
         we {}                   {lived (at)} {}  a place in {Northern Norway} \\
\z

The fact that the \isi{preposition} \textit{på} may nevertheless occur in \REF{ex:taraldsen:38}, can then be accounted for by attributing two distinct interpretations to \textit{sted}: It can denote a space, as in \REF{ex:taraldsen:41}, but it can also denote a ``thing” (located in some space), just like \textit{by} `city' On the second interpretation, it can only be a locative modifier by having the \isi{preposition} \textit{på} associating it with a space-denoting STED just as in \REF{ex:taraldsen:40}:

\ea%33
    \label{ex:taraldsen:42}
    \gll Vi [\textsubscript{VP} bodde {[ STED [} på {[\textsubscript{NP}} et sted i Nord-Norge]]].\\
         we {}                  lived {}        (at)  {} a place in {Northern Norway}\\
\z

Returning now to the fact that the stranded \isi{preposition} cannot be omitted in \REF{ex:taraldsen:35}, I tentatively suggest that the subject of a Tough Movement construction may denote ``things”, but not spaces:

\begin{exe}
\exr{ex:taraldsen:35}
\gll   Dette stedet er bra å bo *(på).\\
       this place is nice to live  at\\
\end{exe}
Then, \textit{på} is obligatory in \REF{ex:taraldsen:35} for the same reason as in \REF{ex:taraldsen:39}.

\subsection{ The head-raising analysis of relative constructions}\label{sec:taraldsen:4.3}

To complete the analysis of sentences like \REF{ex:taraldsen:36} and explain the contrast between \REF{ex:taraldsen:36} and \REF{ex:taraldsen:45}, we need to adopt the head-raising analysis of relatives advocated by \citet{Vergnaud1974} and \citet{Kayne1994} among others.

\begin{exe}
\exr{ex:taraldsen:36}
   \ea
\gll Vi besøkte et sted vi hadde bodd i fem år.\\
         we visited  a place we had lived for five years\\
\ex
\gll Tromsø er et bra sted å bo. \\
         Tromsø is a nice place to live\\
\z
\end{exe}

\ea%34
    \label{ex:taraldsen:45}
    \ea
\gll Vi besøkte en by vi hadde bodd *(i) i fem år.\\
         we visited  a place we had lived in for five years\\
\ex
\gll  Tromsø er et bra sted å bo *(i).\\
         Tromsø is a nice place to live in\\
\z
\z
The contrast between \REF{ex:taraldsen:36} and \REF{ex:taraldsen:45} follows immediately on the head-raising analysis:

\ea%35
    \label{ex:taraldsen:46}
    \begin{forest}
    [DP [et] [\isi{CP} [sted,name=sted] [\isi{CP} [vi bodde \st{sted} i 1981,name=vi,roof]]]]
    \path[-Stealth] (vi.270) edge [bend left,in=90,out=90,looseness=2] (sted);
    \end{forest}
\z

\ea%36
    \label{ex:taraldsen:47}
    \begin{forest} for tree={nice empty nodes}
    [DP [en] [\isi{CP} [by, name=BY] [\isi{CP} [vi bodde] [ [STED] [\isi{CP} [*(i) \st{by} i 1981, roof, name=by]]]]]]
    \path[-Stealth] (by.220) edge [bend left,in=90,out=90] (BY);
    \end{forest}
\z

But on a derivation involving operator-movement, the difference between \textit{by} and location-denoting \textit{sted} is neutralized at the point of the derivation where the decision to merge a \isi{preposition} must be made:

\ea%37
    \label{ex:taraldsen:48}

% % % 	  [\textsubscript{DP} et [\textsubscript{NP} sted [\textsubscript{CP} Op [\textsubscript{CP} vi bodde *(på) {\textless}\textbf{Op}{\textgreater} i 1981 ]]]]
    \begin{forest}
    [DP [et] [NP [sted] [\isi{CP}, s sep=2cm [Op,name=Op] [\isi{CP} [vi bodde *(på) \st{Op} i 1981,roof, name=vi]]]]]
    \path[-Stealth] (vi.330) edge [bend left,in=90,out=90] (Op);
    \end{forest}
\z

\ea%38
    \label{ex:taraldsen:49}
    \begin{forest}
    [DP [en] [NP [by] [\isi{CP}, s sep=2cm [Op,name=Op] [\isi{CP} [vi bodde *(i) \st{Op} i 1981,roof, name=vi]]]]]
    \path[-Stealth] (vi.320) edge [bend left,in=90,out=90] (Op);
    \end{forest}
\z
The head-raising analysis of relatives will be crucial in what follows.

\subsection{ Where does \textit{sted} come from in (\ref{ex:taraldsen:1}--\ref{ex:taraldsen:2})?}\label{sec:taraldsen:4.4}

On the head-raising account of \isi{relative} constructions, the analysis sketched in \sectref{sec:taraldsen:3.2} seems to run up against a serious problem: Where does \textit{sted} `place', the head of the \isi{relative} clause in (\ref{ex:taraldsen:1}--\ref{ex:taraldsen:2}), come from, if the subject PP originates as a locative modifier inside the \isi{relative} clause?:

\begin{exe}%1
    \exr{ex:taraldsen:1}
    \gll I Tromsø er et bra sted å bo.\\
       in Tromsø is a nice place to live \\
\end{exe}

\begin{exe}
	\exr{ex:taraldsen:2}
	\gll   I Tromsø er et sted det er morsomt å arbeide.\\
	      in Tromsø is a place it is fun to work\\
\end{exe}

In particular, it would seem as if \textit{sted} and \textit{i Tromsø} cannot both start out as locative modifiers in the \isi{relative} clause.

But in \sectref{sec:taraldsen:4.2}, I proposed that a silent \textit{STED} occurs inside locative PPs as in \REF{ex:taraldsen:52}:

\ea%39
    \label{ex:taraldsen:52}
\gll 	 [   STED [     i  [ Tromsø ]]] \\
        {}  place  {}   in {}  Tromsø\\
\z

Taking STED to be a regular syntactic object in \REF{ex:taraldsen:52}, in fact a noun phrase, we can now entertain the possibility that movement can apply to it. If so, the structure of (\ref{ex:taraldsen:1}--\ref{ex:taraldsen:2}) at a point of the derivation where the PP has not yet raised to the subject position, may be as in \REF{ex:taraldsen:53}, still assuming the head-raising analysis of relatives:\footnote{In \ref{ex:taraldsen:53}--\ref{ex:taraldsen:54}, the indefinite article \textit{et} and the adjective \textit{bra} are taken to be merged onto the \isi{relative} \isi{CP}, like the definite article \textit{the} in \citegen{Kayne1994} analysis of relatives, but it may also be possible to replace \REF{ex:taraldsen:53} with (i) or (ii):

\begin{xlisti}[(ii)]
\exi{(i)}
\gll  er [\textsubscript{CP} [\textsubscript{NP} et bra \textbf{sted}] [    å bo         [\st{[{\textsubscript{NP}} et bra sted]} [ i [ Tromsø ]]]]]]\\
       is   {}                           {}         a nice     place       {}  to live       {}                             {}  in {} Tromsø\\
\exi{(ii)}
\gll  er [\textsubscript{DP} et [\textsubscript{CP}[\textsubscript{NP} bra sted] [ å bo [\st{[{\textsubscript{NP}} bra sted]} [ i [ Tromsø ]]]]]] \\
      is   {}                a                {}                       nice    place  {}      to live              {}        {}  in  {} Tromsø\\
\end{xlisti}

Deciding between the options will in part turn on determining the structure of \textit{et bra sted i Tromsø} `a nice place in Tromsø' in sentences in like (iii):

\begin{xlisti}[(iii)]
\exi{(iii)}
\gll Vi fant et bra sted i Troms.ø\\
     we found a nice place in Tromsø\\
\end{xlisti}}

\ea%40
    \label{ex:taraldsen:53}
  \ea
\gll  er [\textsubscript{DP} et bra [\textsubscript{CP} \textbf{sted} [ å bo [\st{sted} [ i [ Tromsø ]]]]]] \\
          is    {}   a nice  {}   place {} to live         {} {}     in {} Tromsø\\
\ex
\gll  er [\textsubscript{DP} et [\textsubscript{CP} \textbf{sted} [ det er morsomt å arbeide           [\st{sted} [ i [ Tromsø ]]]]]] \\
          is  {}     a  {}    place  {} it    is  fun        to work  {} {} in {} Tromsø\\
\z
\z
(I’m assuming that \textit{sted} can only be silent when it remains in the Spec of a \isi{preposition}.)

Then, either the remnant [ \st{sted} [ \textit{i} [ \textit{Tromsø} ]]] or just [ \textit{i} [ \textit{Tromsø} ]] raises to the subject position. Assuming that the remnant raises, (\ref{ex:taraldsen:1}--\ref{ex:taraldsen:2}) are parsed as in \REF{ex:taraldsen:54}:

\ea%41
    \label{ex:taraldsen:54}

\ea \label{ex:taraldsen:54a}
\gll  [\st{sted} [ \textbf{i} [ \textbf{Tromsø} ]]] er [\textsubscript{DP} et bra [\textsubscript{CP} sted [ å bo       {\st{[sted [}} i [ Tromsø ]]]]]] \\
           {}    {}         in  {}  Tromsø  {}   is   {}    a nice   {}  place {} to live {} in {} Tromsø\\
\ex \label{ex:taraldsen:54b}
\gll  [\st{sted} [ \textbf{i} [ \textbf{Tromsø} ]]] er [\textsubscript{DP} et [\textsubscript{CP} sted [ det er morsomt   å arbeide \st{[sted [ i [ Tromsø ]]]}]]] \\
          {}     {}     in    {}    Tromsø     {}   is   {}                a       {}             place {}  it is  fun    to work    {\hspaceThis{\textit{[sted~[}} in \hspaceThis{~} Tromsø} \\
\z
\z

Notice that a parallel derivation is not available to \REF{ex:taraldsen:55}: On the assumptions made in \sectref{sec:taraldsen:4.2}, no noun other than \textit{plass} `place' can replace \textit{STED} in \REF{ex:taraldsen:52}:

\ea%8
    \label{ex:taraldsen:55}
\ea[*]{
\gll I Tromsø er en bra by å bo.\\
        in Tromsø is a nice city to live\\}
\ex[*]{
\gll    I Tromsø er en by det er morsomt å arbeide.\\
        in Tromsø is a city it is   fun         to work\\}
\z
\z

Thus, \REF{ex:taraldsen:55} is excluded because \textit{by} `city' has no position in the \isi{relative} clause to originate from.

Notice also that on the analysis in \sectref{sec:taraldsen:4.2}, this still correlates with the fact that \textit{by} cannot be a prepositionless locative, unlike \textit{sted}:

\ea%9
    \label{ex:taraldsen:56}
\ea
\gll  Vi arbeidet (på) {det samme stedet/den samme plassen} i tre år.\\
         we worked (at)  {the same place}                        for three years\\
\ex
\gll   Vi har nettopp besøkt {et sted/en plass} vi bodde (på) i 1981.\\
         we have just    visited {a  place}    we lived (at)  in 1981\\
\z
\z

\ea%10
    \label{ex:taraldsen:57}
  \ea
\gll   Vi arbeidet *(i) den samme byen  i tre år.\\
          we worked *(in) the same city     for three years\\
\ex
\gll    Vi har nettopp besøkt en by vi bodde *(i) for ti år siden.\\
           we have just    visited a  city we lived *(in)  ten years ago \\
\z
\z

Thus, our current set of hypotheses also provides a satisfactory answer to question \REF{ex:taraldsen:17a}.

\subsection{ Locality and minimality}\label{sec:taraldsen:4.5}

We are still left with the problem that the analysis in \sectref{sec:taraldsen:3.2} must allow the PP to undergo \isi{A-movement} out of \isi{relative} clause.

In the derivation leading to (\ref{ex:taraldsen:1}--\ref{ex:taraldsen:2}) via the structures in \REF{ex:taraldsen:54}, the PP moves to an A-position from a position inside the \isi{relative} clause. This is of course at odds with standard assumptions. Relative constructions are generally assumed to be islands for any kind of movement. In addition, \isi{A-movement} is not expected to cross intervening A-positions such as the covert subject of the infinitive in \REF{ex:taraldsen:54a} (not shown in the representations) and the expletive subject \textit{det} `it' in \REF{ex:taraldsen:54b}. This is in fact what \isi{Relativized Minimality} is designed to exclude.

The proposal in \sectref{sec:taraldsen:4.4} suggests a solution. The basic intuition is that the \isi{A-movement} of the PP leading to \REF{ex:taraldsen:54} is in a sense parasitic on the \isi{A’-movement} of \textit{sted} `place'.

Taking island conditions and minimality as constraints on derivations, I want to suggest that since A’-moved \textit{sted} is subextracted from [ \textit{sted} [ \textit{i} [ \textit{Tromsø} ]]], the remnant [\st{sted} [ \textit{i} [ \textit{Tromsø} ]]] can be accessed by movement as if it were sitting in the same position as the previously moved \textit{sted}.

An immediate objection to this might be that \textit{sted} is moved to an A’-position (Spec-\isi{CP}, on our analysis inherited from \citealt{Kayne1994}) so that moving the remnant as if it were sitting in that position would make the movement of the remnant to the subject position similar to improper movement. However, if “\isi{relative} clause extraposition” is analyzed as the outcome of movement of the “head” of a \isi{relative} construction (stranding the rest of the \isi{relative} clause) as proposed by \citet{Kayne1994}, the grammaticality of sentences like \REF{ex:taraldsen:58} shows that the head noun can undergo \isi{A-movement}:

\ea%42
    \label{ex:taraldsen:58}
	  A man appeared who we had never seen before
\z

That is, although the head noun has raised to Spec-\isi{CP} by \isi{A’-movement} (on the head-raising analysis), it can still go on to raise to a subject position. Correspondingly, saying that the remnant containing the PP can raise to the subject position as in \REF{ex:taraldsen:52} because it can move as if it were in the position held by \textit{sted}, the head noun of the \isi{relative} construction, would appear less obviously incorrect.

Crucially, this derivation only gives rise to sentences where the location associated with the subject PP is co-extensive with the space denoted by \textit{sted} `place', as in (\ref{ex:taraldsen:1}--\ref{ex:taraldsen:2}). With the verb \textit{ligge} `lie, be located within', a subject must be associated with a proper subspace of the location denoted by the locative complement:

\ea%43
    \label{ex:taraldsen:59}
  \ea
\gll  Tromsø ligger (på) et bra sted å bo.\\
          Tromsø lies     (on) a nice place to live\\
\ex
\gll   Tromsø ligger (på) et sted det er bra å bo.\\
          Tromsø  lies   (on) a  place it is nice to live\\
\z
\z

These are similar to:

\ea%44
    \label{ex:taraldsen:60}
  \gll	  Tromsø ligger i Nord-Norge.\\
         Tromsø lies   in Northern Norway\\
\z

\newpage 
Correspondingly, we correctly predict the impossibility of substituting \textit{ligger} `lies' for \textit{er} in sentences like (\ref{ex:taraldsen:1}--\ref{ex:taraldsen:2}) (see footnote \ref{fn:taraldsen:6}):\footnote{The sentences in \REF{ex:taraldsen:61} are fine with the initial PP as a fronted adverbial of the sort seen in \textit{I Tromsø ligger Ishavskatedralen} – In Tromsø lies The Arctic Cathedral. The following are ungrammatical:

\begin{xlisti}[(ii)]
 \exi{(i)}
 \gll *  Nå ligger i Tromsø et bra sted å bo. \\
      {} now lies in Tromsø a nice place to live\\
 \exi{(ii)}
 \gll  *  I Tromsø synes å ligge et bra sted å bo. \\
       {} in Tromsø seems to lie a nice place to live \\
\end{xlisti}}

\ea%45
    \label{ex:taraldsen:61}
    \ea
\gll	* I Tromsø ligger et bra sted å bo.\\
        {}  in Tromsø lies   a  nice place to live\\

 \ex
\gll * I Tromsø ligger et sted det er morsomt å arbeide. \\
     {} in Tromsø lies   a place it   is  fun  to work\\
\z
\z

\section{Conclusion}\label{sec:taraldsen:5}

In this article, I have primarily endeavored to characterize the puzzles surrounding the existence of \ili{Norwegian} sentences like (\ref{ex:taraldsen:1}--\ref{ex:taraldsen:2}). I have also suggested a line of analysis that seems plausible to me, but clearly stands in need of much elaboration in order to fit into current syntactic theories.
 

\printbibliography[heading=subbibliography,notkeyword=this]
\end{document}
