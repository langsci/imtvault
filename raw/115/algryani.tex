\documentclass[output=paper]{LSP/langsci}
\author{Ali Algryani\affiliation{The Libyan Academy}}
\title{Ellipsis in Arabic fragment answers}
% \epigram{Change epigram}
\abstract{Fragment answers are short answers to questions consisting of non-sentential XPs that convey the same propositional content as complete sentential answers. This squib discusses the syntax of ellipsis in Arabic fragments answers focusing on whether or not ellipsis in fragmentary utterances contains syntactic structure and whether, if so, such fragmentary XPs can be derived via A-bar movement to a clause-initial position plus TP deletion at PF in a way similar to that of \citet{Merchant2004}. It is argued that ellipsis in Arabic fragment answers contains syntactic structure and therefore can be analysed as TP ellipsis derived by focus movement of the remnant to a left peripheral position followed by deletion of the TP constituting the background information. Such an analysis captures some morpho-syntactic effects such as morphological case-matching, preposition-stranding, and islands effects.
}
\ChapterDOI{10.5281/zenodo.1116781}

\maketitle

\begin{document}

\section{The syntax of ellipsis in fragment answers}\label{sec:algryani:1}


Fragment answers are short answers to questions consisting of non-sentential XPs. Such XPs, however, convey the same propositional content as full sentential answers
\citep{Merchant2004}. Fragmentary utterances, such as (1B), have been analysed according to non-structural and structural approaches. While the former argue against positing a structure in ellipsis at any level of representation, that is, there is no more structure than what is pronounced (see \citealt{Progovac2006,Casielles2006}), the latter assume that ellipsis in such utterances contains invisible syntactic structure (\citealt{Merchant2004,Merchant2006,Krifka2006,vanCraenenbroeck2010}).


\ea%1
\label{ex:algryani:1}
	 A: Who did she see?


 B: John.
\z

There are several arguments that seem to speak in favour of the non-elliptical approach. One comes from the facts seen in (2a,b).


\ea%2
 \label{ex:algryani:2}

	 Who ate the pizza?


\ea \label{ex:algryani:2a} Me/him/them.


\ex \label{ex:algryani:2b} *I/he/they.
\z
\z


\citet{Progovac2006} takes the absence in \REF{ex:algryani:2a} of structural nominative case, which is assigned in T, as an indication that such fragments are complete syntactic objects (NPs), not TPs. The lack of a tense projection in the structure explains why the NPs \textit{me/him/them} surface in the (default) accusative case. The ungrammaticality of \REF{ex:algryani:2b} is ascribed to the fact that the pronouns \textit{I/he/them} contain unchecked nominative \isi{Case} features. In contrast, a subject pronoun in an answer such as \textit{I did} surfaces in the nominative case, as is expected given that nominative case assignment requires a tense projection.


Another arguments given by \citet{Progovac2006} against the ellipsis analysis of fragment answers comes from verbal utterances. These too can be analysed as base-generated phrases. The verb in (3B) surfaces in the bare infinitive form which is not expected if such a verbal answer is derived from a full sentential source, as in \REF{ex:algryani:4}. The absence of tense and verbal \isi{agreement} on the verb \textit{play} in (3B) is ascribed to the lack of a tense node, which in turn suggests that such an utterance is better analysed as a base-generated VP (see \citealt{Progovac2006} and \citealt{Casielles2006} for further discussion).


\ea%3
 \label{ex:algryani:3}

	 A: What did Andres do?


 B: Play volleyball.
\z

\ea%4
 \label{ex:algryani:4}

	 Andres plays volleyball.
\z

In the structural approach, utterances like \REF{ex:algryani:1} are analysed as the result of a deletion process. \citet{Merchant2004}, for instance, provides an analysis of fragment answers in which the fragment answer is fronted to a clause-peripheral position and the remainder of the sentence is deleted. Accordingly, the fragment answer in \REF{ex:algryani:1}, i.e. \textit{John}, originates as an object of the verb \textit{saw} and it moves to a clause initial position while the rest of the clause is elided, that is, not pronounced. This is illustrated in the tree diagram in \REF{ex:algryani:5}.


\ea%5
 \label{ex:algryani:5}
 \begin{forest}
  [FP
    [spec [John\textsubscript{i}]]
    [F' [F\textsubscript{[E]}] [\st{TP}
				 [\st{she saw ~~~~ t}\textsubscript{i},roof]]
    ]
  ]
 \end{forest}
\z 

There is evidence for such an analysis based on morphological case marking, \isi{preposition} stranding, and binding effects. For instance, in languages where case is marked morphologically, it has been argued that the remnant in short answers can bear only the same case as it would display in full answers, as in \REF{ex:algryani:6} from \ili{Greek}. The short answer in (\ref{ex:algryani:6}a) can be explained as follow: the remnant DP fragment answer starts as a subject bearing the nominative case, as is expected in full answers prior to ellipsis. The short answer in (\ref{ex:algryani:6}b) is ungrammatical due to its accusative case.


\ea%6
 \label{ex:algryani:6}
	 \ili{Greek} \citep[75]{Merchant2006} \\
	\begin{xlist}
	\exi{Q:}
	\gll  Pjos idhe tin Maria?\\
	      {who.}\textsc{nom} saw the {Maria?}\\
	\glt   ‘Who saw Maria?’
	 \exi{a.}
	 \gll  {\upshape A:} {\textit{O Giannis}}.\\
		{}     {The Giannis.}\textsc{nom}\\
	 \exi{b.}
	 \gll {\upshape A:} *{\textit{Ton Gianni}}.\\
		 {}     {The Giannis.}{\textsc{acc}} \\
	\end{xlist}
\z

{The p-stranding phenomenon also argues in favour of the ellipsis analysis. \isi{P-stranding} is permitted in fragment answers only if it is permitted in sentential answers. In \REF{ex:algryani:7}, \isi{preposition} stranding is unacceptable since \ili{Greek} is a non-p-stranding language; the \isi{preposition} in such cases has to be pied-piped. In a p-stranding language such as} {Norwegian}{, both options are available, as in \REF{ex:algryani:8}, indicating that only constituents that are independently able to move in a language can be fragment answers in that language.}

\newpage 
\ea%7
 \label{ex:algryani:7}
	 \ili{Greek} \citep[685--686]{Merchant2004}\\
\ea \label{ex:algryani:7a}
\gll  Me pjon milise i Anna? \\
      with whom spoke the Anna?\\
\glt `Who did Anna speak with?'

\ex \label{ex:algryani:7b}
\gll  Me ton Kosta.\\
with the Kostas\\

\ex \label{ex:algryani:7c} 
*\textit{Ton Kosta}.\\
	 
\z
\z


\ea\label{ex:algryani:8}
\ili{Norwegian} {\citep[685--686]{Merchant2004}}\\
\ea \label{ex:algryani:8a}
\gll Hvem har Per snakket med? \\
 Who has Per talked with?\\
\ex \label{ex:algryani:8b} \textit{Mary}.
\z
\z


Finally, DP fragments show the distribution regulated by the Binding Theory just like their sentential counterparts. The anaphor \textit{himself} in \REF{ex:algryani:9a} is acceptable as a fragment answer despite the absence of any antecedent. This can be explained under the assumption that there is a clausal structure in the ellipsis site hosting the antecedent, which in such a case satisfies Condition A of the Binding Theory, which stipulates that an anaphor has to be bound in its governing category (see \citealt{Merchant2004,Merchant2006}).


\ea%9
 \label{ex:algryani:9}
	 Who does John like?\\
	\ea \label{ex:algryani:9a} Himself.
	\ex \label{ex:algryani:9b} John\textsubscript{i} likes himself\textsubscript{i}.
\z\z

This squib provides an overview of the syntax of ellipsis in \ili{Arabic} fragment answers. It is organised as follows: \sectref{sec:algryani:2} presents fragment answers in Standard \ili{Arabic} and discusses the interaction between ellipsis and information structure. \sectref{sec:algryani:3} puts forward an analysis for fragment answers in \ili{Arabic}. Finally, \sectref{sec:algryani:4} presents the conclusion.


\section{Fragment answers in Arabic}\label{sec:algryani:2}


Fragment answers exist in \ili{Arabic}. Speakers of the language often answer a question with a phrase, a fragment of a sentence, rather than with a full sentence. Such non-sentential fragments are, however, interpreted as full sentential structures. Fragment answers can be DPs, PPs or VP, as in (\ref{ex:algryani:10}--\ref{ex:algryani:12}).


\ea%10
 \label{ex:algryani:10}
\begin{xlist}[A:]
\exi{A:}
\gll Maða ištarat Hind-un?\\
 what bought.\textsc{3fs} Hind-\textsc{nom}\\
\glt  ‘What did Hind buy?’

\exi{B:}
\gll Kitaab-an.\\
 book-\textsc{acc}\\
\glt  \textsc{‘A} book.’
\end{xlist}
\z


\ea%11
 \label{ex:algryani:11}
\begin{xlist}[A:]
 \exi{A:}
\gll Maʕa man ðahabat Hind-un?\\
     with whom went.\textsc{3fs} Hind-\textsc{nom}\\
\glt  ‘With whom did Hind go?’

\exi{B:}
\gll  Maʕa Zayd-en.\\
 with Zayd-\textsc{gen}\\
\glt  ‘With Zayd.’
\end{xlist}
\z


\ea%12
 \label{ex:algryani:12}
\begin{xlist}[A:]
\exi{A:} \gll Maða faʕalat Hind-un b-ssayyarat-i?\\
     what did.\textsc{3fs} Hind-\textsc{nom} with-the-car-\textsc{gen}\\
\glt ‘What did Hind with the car?’

 \exi{B:}
\gll \textit{baʕat-ha.} \\
 sold.\textsc{3fs}{}-it\\
 \end{xlist}
 \z

Ellipsis in fragment answers is linked to information structure, since the remnant is interpreted in terms of focus which can be informational or identificational (see \citealt{Brunetti2003,Busquets2006,Kolokonte2008}). \isi{Focus} can be expressed in \ili{Arabic} in two different means: a focused constituent can appear in situ or in a left peripheral position, as in \REF{ex:algryani:13}. The former is perceived as new informational focus, while the latter is normally interpreted as contrastive/identificational focus (see \citealt{Moutaouakil1989,AounEtAl2010} for discussion).


\ea%13
 \label{ex:algryani:13}

  Standard \ili{Arabic} \citep[202]{AounEtAl2010}\\
\ea
\gll šariba zayd-un ŠAY-AN.\\
 drank.\textsc{3ms} zayd-\textsc{nom} tea-\textsc{acc}\\
 \glt ‘Zayd drank TEA.’

 \ex
 \gll ŠAY-AN šariba zayd-un.\\
 tea-\textsc{acc} drank.\textsc{3ms} zayd-\textsc{nom} \\
 \glt ‘It was tea that Zayd drank.’
 \z
 \z


\section{Analysis of Arabic fragment answers}\label{sec:algryani:3}


Fragment answers in \ili{Arabic} display some morpho-syntactic effects that are also found in their full sentential counterparts. For instance, the morphological case-marking effect is evident in \ili{Arabic} as in \REF{ex:algryani:14}, where the fragment answer can only bear accusative case it would bear in a sentential answer (\ref{ex:algryani:14}C).


\ea%14
\label{ex:algryani:14}
\begin{xlist}[A:]
\exi{A:}
\gll Maða ištarat Hind-un?\\
 what bought.\textsc{3fs} Hind-\textsc{nom}\\
\glt  ‘What did Hind buy?’\\

\exi{B:}
\gll kitaab-\textbf{an}. / *kitaab-un\\
 book\textsc{-acc} {} book\textsc{-nom}\\

\exi{C:}
\gll Hind-un ištarat kitaab-\textbf{an}.\\
 Hind-\textsc{nom} bought.\textsc{3fs} book\textsc{-acc}\\
\end{xlist}
\z

The remnant in \REF{ex:algryani:14} bears the accusative case, indicating that it originates as an object of the verb \textit{ištarat} \textit{‘bought’}, where it is assigned accusative case. The remnant undergoes focus movement to a left peripheral position followed by \isi{TP} deletion at PF, as illustrated in the tree diagram in \REF{ex:algryani:15}.\footnote{An alternative idea could be that the remnant, e.g. in \REF{ex:algryani:14}, might be in situ, that is, in the \isi{TP}, and that all of the \isi{TP} except for the constituent that surfaces as a remnant elides, as in (i), is unacceptable since it would entail that a syntactic operation can apply to a string of words that do not make up a constituent.
\begin{xlist}[(i)]
 \exi{(i)}
 \gll \st{ištarat} \st{Hind-un} kitaab-an.\\
    bought.\textsc{3fs} Hind-\textsc{nom} book-\textsc{acc}\\
\glt `a book'\\
\end{xlist}}
As for the interpretation of the remnant, it is interpreted as new informational focus given that it is not in contrast with any existing information but rather it expresses new information that is not shared by the speaker and the addressee.\footnote{The same is true of English examples like \REF{ex:algryani:1} and similar cases in \ili{Italian} and \ili{Greek} (\citealt{Brunetti2003,Kolokonte2008}).}


\ea%15
 \label{ex:algryani:15}
  \begin{forest}
    [FP
      [Spec [kitaab-an\textsubscript{i},name=kitaab]]
      [\st{TP}
	[Spec] [T'
	[T [i\v{s}tarat\textsubscript{i},name=istarat]]
	[VP [Spec [Hind-un]] [V' [V [t\textsubscript{i},name=ti]] [DP [t\textsubscript{i},name=tii]]]
        ]
      ]
    ]
    ]
  \path[-{Stealth[]},draw] (ti.south) node[fill=black,circle, inner sep=1pt] {} to [bend left=45,looseness=.8]  (istarat);
  \path[-{Stealth[]},draw] (tii.south) node[fill=black,circle, inner sep=1pt] {} to [bend left=60,in=130] (kitaab.220);
  \end{forest}
\z


The \isi{preposition} stranding argument of \citet{Merchant2004} can also be extended to \ili{Arabic}. \ili{Arabic} is a non-p-stranding language; p-stranding is not permitted in fragment/short answers, as in (16A1) nor in full answers (16A3). The p-stranding effect can be accounted for by the movement-plus-deletion analysis, according to which the remnant PP \textit{maʕa Zayd-en} ‘with Zayd’ starts as a complement of the verb \textit{taḥadaθat} ‘talked’ and moves up to the left periphery before the entire \isi{TP} gets deleted, as shown in \REF{ex:algryani:17}. The ungrammaticality of (A1) can be ascribed to the ban on p-stranding in the language.


\ea%16
 \label{ex:algryani:16}
\gll maʕa man taḥadaθat Hind-un?\\
 with who talked.\textsc{3fs} Hind-\textsc{nom}\\
\glt ‘With whom did Hind talk?’

\ea
\gll *\textit{Zayd}.\\
 Zayd\\

\ex
\gll  \textit{maʕa Zayd-en}.\\
 with Zayd{}-\textsc{gen}.\\

\ex
\gll *Zayd-en taḥadaθat Hind-un maʕa. \\
 Zayd-\textsc{gen} talked.\textsc{3fs} Hind-\textsc{nom} with \\
 \z
 \z


\ea%17
 \label{ex:algryani:17}
  \begin{forest}
    [FP
      [Spec [maʕa Zayd-en\textsubscript{i},name=maca]] [\st{TP} [Spec] [T'
	[T [taḥadaθat\textsubscript{j},name=taha]]
	[VP [Spec [Hind-un]] [V' [V [t\textsubscript{j}, name=tj]] [PP [t\textsubscript{i},name=ti]]]]
	]
      ]
    ]
  \path[-{Stealth[]},draw] (tj.south) node[fill=black,circle, inner sep=1pt] {} to [bend left=45,looseness=.8]  (taha);
  \path[-{Stealth[]},draw] (ti.south) node[fill=black,circle, inner sep=1pt] {} to [bend left=60,in=125] (maca.220);
  \end{forest}
\z 


Finally, a third argument in favour of the assumption that the remnant undergoes \isi{A-bar movement} to the left periphery is the fact that the remnant in fragment answers is sensitive to island domains. \citet{Merchant2004} shows for English that if the correlate to a fragment answer is within an island, then only the sentential answer is possible. The same is true in \ili{Arabic}, as shown in (\ref{ex:algryani:18}--\ref{ex:algryani:19}).



\ea%18
\label{ex:algryani:18}
Adjunct island\\
\begin{xlist}[A:]
 \exi{A:}
\gll Hal ʔatat liʔana-ka lam tadʕu Hind-an?\\
 \textsc{q} came.\textsc{3fs} because-you \textsc{neg} invited.\textsc{2ms} Hind-\textsc{acc}\\
\glt  ‘Did she come because you didn’t invite Hind?’\\

\exi{B:}
\gll *la, Omar-an.\\
 no, Omar-\textsc{acc}.\\

\exi{C:}
\gll la, ʔatat liʔana-ka lam tadʕu Omar-an.\\
 no came.\textsc{3fs} because-you \textsc{neg} invited.\textsc{2ms} Omar-\textsc{acc}\\
\glt ‘No, she came because you didn’t invite Omar.’
\end{xlist}
\z

\pagebreak
\ea%19
\label{ex:algryani:19}
\isi{Relative clause} island \\
\begin{xlist}
\exi{A:}
\gll hal istalamat al-ressalat-a allati kataba-ha li-Zaynab?\\
 \textsc{q} received.\textsc{3fs} the-letter-\textsc{acc} that wrote.\textsc{3ms}-it to-Zaynab\\
\glt  ‘Did she receive the letter that he wrote to Zaynab?’~

\exi{B:}
\gll  *la, li-Zayd-en.\\
 no to-Zayd-\textsc{gen}. \\

\exi{C:}
\gll la, istalamat al-ressalat-a allati kataba-ha li-Zayd-en.\\
no received.\textsc{3fs} the-letter-acc that wrote.\textsc{3ms}-it to-Zayd-\textsc{gen}\\
 \glt  ‘No, she received that the letter that he wrote to Zayd.’\\
\end{xlist}
\z


The ungrammaticality of (\ref{ex:algryani:18}B) and (\ref{ex:algryani:19}B) is expected if we assume that the fragment DPs derive from the structures in (C) and that they have moved across island domains to the left periphery.


\section{Conclusion}\label{sec:algryani:4}
\ili{Arabic} fragment answers contain syntactic structure and can be derived by focus movement of the remnant to the left periphery followed by \isi{TP} ellipsis. Such a movement-plus-deletion analysis is based on evidence from morphological case-matching, preposition-stranding as well as island effects. The remnant is interpreted as new informational focus, indicating that new information focus can appear in the left periphery in the context of ellipsis.


\sloppy
\printbibliography[heading=subbibliography,notkeyword=this]
\end{document}
