\documentclass[output=paper]{LSP/langsci}
\author{Ciro Greco\affiliation{Ghent University}\and
	Liliane Haegeman\affiliation{Ghent University}\lastand
	Trang Phan\affiliation{Ghent University/University of Hai Duong}
}
\title{Expletives and speaker-related meaning}
\abstract{In our paper, we investigate a set of pronominal forms that have lost their referential meaning and might at first sight be analyzed as expletives. First, we  discuss the case of Finnish, which, though a pro-drop language, displays an element \textit{sitä} with expletive function; and the case of Dominican Spanish, another pro-drop language which seems to have an expletive \textit{ello} but in which, unlike Finnish, the expletive conveys a speaker-related meaning. In addition, we also examine the case of Vietnamese, a radical pro-drop language which also seems to deploy an expletive \textit{nó} with discourse value, and the case of the Flemish element \textit{tet}, which has lost its referential value and also has a discourse function. From these data it emerges that independently of the satisfaction of formal EPP-requirements, some languages can employ expletive or expletive-like elements for discourse-related reasons in those contexts where regular expletives are required in languages like English. The data discussed here lead to a more complex picture of the nature of expletives and their function in the grammar.}
\ChapterDOI{10.5281/zenodo.1116757}

\maketitle

\begin{document}
% Key words: \textit{expletives, speaker-related meaning, \ili{Finnish}, Dominican Spanish, Vietnamese, Flemish}

\section{Introduction: expletives as formal devices}\label{sec:Greco:1}
\subsection{Characterizing expletives}

Traditionally, expletives have been defined as elements inserted at some point in the structure to satisfy purely formal requirements, such as, for instance, the \isi{EPP}, which requires subject position to be filled in \isi{finite} clauses \citep{Chomsky1981,Chomsky1995}. Under this conception, expletives are a \textit{last resort} device deployed whenever no regular (overt) subject is available to satisfy the formal requirement in question, either because there is no overt subject argument, as with weather or impersonal constructions, or because the relevant argument fails to attain the canonical subject position, as in existential and presentational sentences. Some patterns for English are illustrated in (\ref{ex:Greco:1}--\ref{ex:Greco:3}): in each example set, the (b) sentence illustrates the pattern in which the contentful subject argument does not reach its canonical position and an expletive element is inserted: in the existential patterns in (\ref{ex:Greco:1}b) and (\ref{ex:Greco:2}b) the expletive is \textit{there}, with an extraposed clausal subject in (\ref{ex:Greco:3}b) the expletive is \textit{it}:

\ea%1
    \label{ex:Greco:1}
    \ea\label{ex:Greco:1a} \textit{Many students are arriving from Italy.}
    \ex\label{ex:Greco:1b} \textit{There are now many students arriving from Italy.}
   \z
\z

\ea%2
    \label{ex:Greco:2}
    \ea\label{ex:Greco:2a} \textit{A workable solution to this problem does not exist.}
    \ex\label{ex:Greco:2b} \textit{There does not exist a workable solution to this problem.}
   \z
\z

\ea%3
    \label{ex:Greco:3}
    \ea\label{ex:Greco:3a} \textit{That the students accepted the new regulations is surprising.}
    \ex\label{ex:Greco:3b} \textit{It is surprising that the students accepted the new regulations.}
   \z
\z

From the literature it emerges that cross-linguistically, canonical expletives share a number of properties. (i) Being inserted to satisfy a formal requirement, they are obligatory in the relevant contexts because, in their absence, the specific formal requirement would not be satisfied, leading to ungrammaticality. For instance, in English omission of the expletive subjects in the (b)-examples above leads to ungrammaticality because the canonical subject position has to be filled in English, i.e. SpecTP, or SpecSubjP in a cartographic approach \citep{Rizzi2007}. (ii) Though expletive elements usually have the form of~an existing contentful element (e.g. 3\textsuperscript{rd} person pronoun, locative adverb), expletives are taken to be semantically empty, at least when deployed as formal devices satisfying subject-related grammar requirements. For instance, though originally a locative adverb, English \textit{there} in \REF{ex:Greco:1b} and \REF{ex:Greco:2b} does not contribute any locative or other semantics.\footnote{Weather expletives might differ from other types of expletives with respect to their \isi{semantic} content \citep{Bolinger1977}. For a (controversial) example of a meaningful use of an expletive, \textit{er}, in \ili{Dutch}, see \citet{Mohr2005}.} Being semantically empty, expletives cannot be focused or contrasted. For instance, they typically are prosodically reduced, and cannot receive focal stress. Moreover, expletives do not undergo \isi{A’-movement} to the left periphery since this type of movement is specialized for the encoding of scope-discourse functions{.}\footnote{There arises a conceptual tension with respect to \citegen{Rizzi2007} assumption that expletives formally satisfy the subject criterion, itself a condition implying a \isi{semantic} component. We will not try to solve this issue here.}

(iii) The picture outlined above leads to a crosslinguistic prediction: \isi{pro-drop} languages should not display overt expletives, because in these languages the \isi{EPP} can be satisfied through some alternative mechanism (for proposals see, a.o. \citealt{Rizzi1982,AlexiadouAnagnostopoulou1998,HolmbergEtAl2009}). Thus, the contrasts between English and \ili{Italian} illustrated in (\ref{ex:Greco:4}--\ref{ex:Greco:5}) have been traced back to the availability of an alternative way to satisfy the \isi{EPP} in \ili{Italian}, which is unavailable in English, and have led to a view in which the presence of expletives is related directly to the \isi{pro-drop} parameter:

\settowidth\jamwidth{(\ili{Italian})}
\ea%4
    \label{ex:Greco:4}
    \ea[*]{\textit{(It) rains.}}
    \ex[]{
    \gll Piove.\\
         rains\\\jambox{(\ili{Italian})}}
  \z
\z


\ea%5
    \label{ex:Greco:5}
    \ea[*]{\textit{(There) have arrived three girls.}}
    \ex[]{\gll Sono  arrivate   tre   ragazze.\\
	     are   arrived   three   girls\\\jambox{(\ili{Italian})}}
    \z
\z



\subsection{Exceptional expletives}

The predictions that follow from the characterization of expletives above are broadly speaking correct in that, typically, (i) expletives are not optional, (ii) they lack \isi{semantic} content, and (iii), \isi{pro-drop} languages do not display expletives as extensively as non-\isi{pro-drop} languages do \citep{Newmeyer2005}, confirming the hypothesis that their presence correlates with the negative setting of the \isi{null subject} parameter.

However, additional research reveals that even in languages which allow non overt subjects there are occurrences of what seem to be expletive elements, \textstylea{suggesting that the correlation with a negative setting of the \isi{pro-drop} parameter is not categorical}. Apparent \textstylea{expletive elements have been attested in \ili{Finnish}, Dominican Spanish and Vietnamese.} The \textstylea{distribution and the properties of the ‘expletives’ in question closely resemble those of canonical subject expletives: typically,} they are pronominal elements without referential value and \textstylea{occupying a position in the higher portion of the inflectional layer.}

\textstylea{Since the customary function of expletives (namely, to satisfy a subject-related \isi{EPP} requirement) can be fulfilled differently in \isi{pro-drop} languages, the question is what function these elements perform in these systems. Do they also serve to satisfy some formal requirement or can they be employed for other purposes and, if the latter, do they make any \isi{semantic} contribution?}

\textstylea{In what follows we will examine such cases in more detail. We will discuss the cases of Dominican Spanish and Vietnamese, two \isi{pro-drop} languages. Our analysis will reveal that the relevant expletives are fully optional devices which convey a speaker-related meaning.}

\textstylea{In particular, we will show that in Vietnamese, the relevant expletive element appears to be allowed only in those contexts where regular, semantically vacuous expletives are required in non-\isi{pro-drop} languages, like English. This suggests that even though the expletive does not fulfill the function of being a subject place holder, it maintains some connection with the subject position. We will then turn to West Flemish, a non \isi{pro-drop} language, in which an expletive-like element appears in a position in the high IP-layer and conveys a speaker-oriented meaning.}

The expletive-like elements which we examine seem to be distributionally alike: they all occupy a high position in the IP layer. However, we will show that, unlike Vietnamese, West Flemish expletive-like elements are not restricted only to the constructions that require expletives in non-\isi{pro-drop} languages. We will suggest that this difference can be captured by the articulation of high IP-layer into specialized subject positions \citep{Kiss1996,Rizzi2007,Cardinaletti2004} and optional discourse-related positions \citep{Uriagereka2004,Grohmann2000}.

\textstylea{This paper is organized as follows: \sectref{sec:Greco:2} and \sectref{sec:Greco:3} discuss} \textit{sitä} \textstylea{in Finnish} and \textit{ello} in Dominican Spanish respectively: we will see that, unlike \ili{Finnish} \textit{sitä}, the expletive \textit{ello} conveys a speaker-related meaning. \sectref{sec:Greco:4} illustrates the expletive-like element \textit{nó} in Vietnamese, a radical \isi{pro-drop} language. We will show that \textit{nó} also seems to encode discourse meaning. In \sectref{sec:Greco:5} we turn to \textit{tet} in Flemish, a non \isi{pro-drop} language. \textit{Tet} is a pronominal element which has lost its referential value, has a discourse function and again it is located in the high IP-area.

\section{Expletives in pro-drop languages: Finnish \textit{sitä}}\label{sec:Greco:2}
\largerpage

\citet{HolmbergEtAl2002} have shown that correlating the presence of expletives with a negative setting of the \isi{pro-drop} parameter is an oversimplification: \ili{Finnish}, a \isi{pro-drop} language, displays what look like overt expletives in a subset of cases where expletives are expected in non-\isi{pro-drop} languages.

\citet{HolmbergEtAl2002} show that \ili{Finnish} can be classified as a \isi{pro-drop}\footnote{\ili{Finnish} is classified as a \textit{partial null-subject language} in the typology in \citet{HolmbergEtAl2009}. This implies that null referential subjects are restricted to 1\textsuperscript{st} and 2\textsuperscript{nd} person, while 3\textsuperscript{rd} person subjects can only be null when bound by a higher argument \citep{Holmberg2005,Holmberg2010Null}.}\textstylea{ }\textstylea{language with referential \isi{null subjects} \REF{ex:Greco:6a} and with \isi{null subjects} with weather verbs \REF{ex:Greco:6b}. However, some expletive elements can (and sometimes must) appear in pre-verbal position, precisely in those contexts typically requiring expletives in non-\isi{pro-drop} languages. As is the case for the canonical expletives, \citet{HolmbergEtAl2002} argue that the relevant expletives do not contribute to the interpretation of the sentence. One such expletive is the element} \textstylea{\textit{sitä}},\footnote{For the sake of completeness, we add that \ili{Finnish} has a second expletive, \textit{se,} the nominative pendant of \textstylea{\textit{sitä} }\textstylea{\citep[100, note 3]{HolmbergEtAl2002},} which is inserted as the subject of weather verbs and in constructions with an extraposed clause. For reasons of space, we cannot discuss this element.} a partitive form of the 3\textsuperscript{rd} person singular non-human pronoun. \REF{ex:Greco:6c} illustrates the use of \textit{sitä} \textstylea{in presentational sentences:}\footnote{\citet[81--83]{HolmbergEtAl2002} also discuss verb-initial sentences without expletives. We cannot go into these here for reasons of brevity.}

\settowidth\jamwidth{(\ili{Finnish})}
\ea%6
    \label{ex:Greco:6}
    \citep[75]{HolmbergEtAl2002}\jambox{(\ili{Finnish})}
    \ea[]{\label{ex:Greco:6a}\gll Olen     väsynyt.\\
	     be.\oldstylenums{1}\textsc{sg}   tired\\
	  \glt ‘I’m tired.’}
    \ex[]{\label{ex:Greco:6b}\gll Sataa     (vettä).\\
	      Rains   (water)\\
	  \glt ‘It is raining.’}
    \ex[*]{\label{ex:Greco:6c}\gll (Sitä) leikkii   lapsia    kadulla.\\
		SITÄ  play    children   in.street\\
	   \glt ‘There are children playing  in the street.’}
    \z
\z

\textit{Sitä} immediately precedes the inflected verb or \isi{auxiliary}, as in (\ref{ex:Greco:6}c), and \textstylea{follows left-peripheral focalized constituents, as in \REF{ex:Greco:7}.} \citet{HolmbergEtAl2002} argue that \textstylea{\textit{sitä}} does not occupy the specifier of \isi{TP}, but rather the specifier of the topmost \textstylea{topic-related} \textstylea{functional projection in the inflectional domain; the specifier of this projection is filled by an argument with the feature [-\isi{Foc}]. When no suitable argument with the feature [-\isi{Foc}] is available,} \textstylea{\textit{sitä}} \textstylea{is inserted:}

\newpage 
\ea%7
    \label{ex:Greco:7}
    \citep[93]{HolmbergEtAl2002}\jambox{(\ili{Finnish})}
    \ea[]{
    \gll \textsc{nama} \textsc{lapset} sitä olisivat    oppineet   uimaan.\\
	these children   \textsc{sitä}   have.\textsc{cond}.\oldstylenums{3}\textsc{pl}   learn     to.swim\\
    \glt ‘\textsc{These children} would have learned to swim.'}
    \ex[*]{\gll Sitä \textsc{nama} \textsc{lapset} olisivat    oppineet   uimaan.\\
		 \textsc{sitä}  these children   have.\textsc{cond}.\oldstylenums{3}\textsc{pl}  learn     to.swim\\}
    \z
\z

\textstylea{On the basis of distributional facts such as those above, \citet{HolmbergEtAl2002} conclude that expletive}\textstylea{\textit{ sitä}} \textstylea{satisfies a formal EPP-requirement, associated with a topic projection in the inflectional domain that dominates the projection encoding subject \isi{agreement}; they suggest that the relevant projection might be the high functional projection ‘FP’ postulated by \citet{Uriagereka2004} for \ili{Romance} and that its availability is related from the general properties of \ili{Finnish} as a Topic-prominent language (see \citealt{Kiss1995}).}

\textstylea{The patterns discussed by \citet{HolmbergEtAl2002} provide evidence that, although it is generally true that languages that can dispense with overt subjects do not require expletives in the same way as non null-subject languages like English do, \isi{pro-drop} systems may still feature expletives. The behavior of} \textstylea{\textit{sitä}}\textstylea{, thus, reveals that the correlation between the distribution of expletives and the null-subject parameter is more complex than originally thought. At the same time,} \textstylea{\textit{sitä}} \textstylea{appears to be employed to fulfill a function similar to that fulfilled by prototypical subject expletives, namely that of satisfying a formal EPP-requirement of some kind.}

\section{Expletives and Discourse Functions: \textit{ello} in Dominican Spanish}\label{sec:Greco:3}

As highlighted above, one implicit assumption in the literature is that the prototypical expletive \textstylea{is inserted for formal reasons and lacks} interpretive effects. However, this generalization has also been challenged. \textstylea{For a number of \ili{Romance} \isi{pro-drop} languages, neuter strong pronouns and demonstratives have been reported to act as optional expletive subjects (see \citealt{Bartra-Kaufmann2011} for an overview); a number of these have been claimed to contribute to the discourse interpretation of the sentence. One such case is the expletive use of the pronoun} \textstylea{\textit{ello}} \textstylea{reported for Dominican Spanish (DS).}

\textstylea{The pronoun} \textstylea{\textit{ello}} occurs in configurations which in the non-\isi{pro-drop} languages typically require an expletive, such as impersonal and weather constructions and with unaccusative post-verbal subjects \citep{BullockEtAl2009,Martínez2011,Muñoz2014,GuptonEtAl2014}:

\settowidth\jamwidth{(DS)}
\ea%8
    \label{ex:Greco:8}
    \citep[156]{Muñoz2014}\jambox{(DS)}
    \ea \gll (\textbf{Ello})   tiene    que  haber    otro    paso.\\
	ELLO  should  that  {to be}    other    path\\
    \glt ‘It should be other paths.’
    \ex \gll (\textbf{Ello})  no  está  lloviendo  aquí  pero  allá    sí.\\
	      ELLO  not  is  raining  here  but  there    yes\\
	 \glt ‘It is not raining here, but it is there.’ \textstylea{\citep[57]{BullockEtAl2009}}
    \ex \gll (\textbf{Ello})   casi     no   ha pasado   ni   u   vehicolo.\\
	      ELLO almost   not   has passed   no   a   vehicle\\
	\glt ‘Almost no vehicle has passed.’
    \z
\z

This use of \textit{ello} is incompatible with an overt pre-verbal subject \citep[65]{Martínez2011}. Because of its complementary distribution with a pre-verbal DP subject, the position of \textit{ello} has been equated with the canonical subject position, i.e. SpecTP:

\ea%9
    \label{ex:Greco:9}
    \citep[65]{Martínez2011}\jambox{(DS)}
    \gll {\upshape *} \textbf{Ello} \textbf{yo}   no   sé   {por qué}   mi papá   me puso    Almeida.\\
	      {} ELLO  I   not   know why          my dad   me called.\oldstylenums{3}\textsc{sg}  Almeida\\
    \glt \hspaceThis{*}`I don't know why my dad named me Almeida.'
    \z

To all intents and purposes, DS \textit{ello} has the properties of an expletive: it is formally like a pronominal element, it lacks referential content, it occupies a high IP-position, it occurs in the contexts that display expletives in the non-\isi{pro-drop} languages. Unlike regular expletives, though, \textit{ello} is optional. In line with the generalization that \isi{pro-drop} languages typically lack expletives, \citet{Muñoz2014} points out that the pronominal system of DS is currently changing as speakers tend to produce more overt pronouns than European Spanish speakers \citep{OtheguyEtAl2007}, suggesting that in fact DS is losing its \isi{pro-drop} properties. In this scenario, the occurrence of an overt expletive would no longer be unexpected and rather than complicating the picture it would indeed corroborate the hypothesis that the presence of overt expletives correlates with a negative setting of the \isi{pro-drop} parameter (however formulated).

As mentioned, \textit{ello} lacks referential content and, in this respect, appears to be like a regular expletive. However, exploring observations in \citet{MartínEtAl1999} and \citet{HinzelinEtAl2007} signal that, while indeed non-referential, dislocated uses of DS \textit{ello} encode \textit{point of view}. They identify the pronoun as a left-peripheral discourse marker conveying the speaker’s commitment to the proposition:

\ea%10
    \label{ex:Greco:10}
    \citep[173]{HinzelinEtAl2007}\jambox{(DS)}
    \gll Ello...  así  decían.\\
	ELLO  so   say.\textsc{imp}.\oldstylenums{3}\textsc{pl}\\
    \glt ‘Well, that’s how they were saying it.’
    \z

While \citet{HinzelinEtAl2007} focus on dislocated \textit{ello} \REF{ex:Greco:10}, \citet[344--345]{GuptonEtAl2014} extend the analysis of \textit{ello} as a \textit{point-of-view}\textit{} discourse marker to IP-internal expletives. They also argue that DS does not behave like partial null-subject languages or non-null-subject languages, but is more like archaic \ili{Romance} \isi{pro-drop} languages such as European Portuguese and Galician in that it has the other identifying properties such as (sporadic) finite-verb enclisis, clitic tripling, and personal infinitives.

Pursuing \citegen{Uriagereka2004} proposal, \citet{GuptonEtAl2014} propose that \textit{ello} occupies the specifier position of a projection FP dominating \isi{TP} which encodes the speaker’s point of view. Observe that the position assigned to \textit{ello} by \citet{GuptonEtAl2014} is similar to that associated by \citet{HolmbergEtAl2002} with \ili{Finnish} \textit{sitä}\textstylea{, but while the latter is not associated with any \isi{semantic} content, DS} \textstylea{\textit{ello} }\textstylea{conveys speaker-related meaning.}

The conclusions in \citet{GuptonEtAl2014} are tentative and further work is needed to substantiate their analysis and explore its impact for other similar pronominal elements in \ili{Romance} but, if their interpretation of the role of DS \textit{ello} is correct, it supports the idea that expletives can be associated with interpretive content.

\section{Vietnamese \textit{nó}}\label{sec:Greco:4}
Like many East Asian languages (e.g. Chinese, Japanese, Korean and \ili{Thai}), Vietnamese is a radical \isi{pro-drop} language \citep{Huang1984} without \isi{agreement} marking on the verb and in which arguments can be freely omitted: (\ref{ex:Greco:11}a) illustrates subject omission, (\ref{ex:Greco:11}b) object omission:

\settowidth\jamwidth{(Vietnamese)}
\ea%11
    \label{ex:Greco:11}
    \ea{ \gll Mary   thích   Tom.   Và Ø c\~ung  thích  Peter.\\
	     Mary   like     Tom   and   Ø  also    like     Peter\\}\jambox{(Vietnamese)}
	\glt ‘Mary\textsubscript{i} likes Tom. She\textsubscript{i} also likes Peter.’
    \ex \gll Mary   thích   Tom.   Nhưng Peter   không   thích Ø.\\
	     Mary   like    Tom   but      Peter     \textsc{neg}    like   Ø\\
	\glt ‘Mary likes Tom\textsubscript{i}. But Peter does not like him\textsubscript{i}.’
    \z
\z

Surprisingly then, in spoken Vietnamese, in addition to its referential use, the pronoun \textit{nó} optionally appears in contexts typically displaying expletive subjects in non-\isi{pro-drop} languages \citep{NguyenEtAl2011,Dao2012}. Like prototypical expletives, Vietnamese \textit{nó} is formally related to a pronoun, it lacks referential content and it cannot be focused. In (\ref{ex:Greco:12}a), \textit{nó} appears to be the subject of a weather predicate, in (\ref{ex:Greco:12}b) it occurs with an existential predicate, and in (\ref{ex:Greco:12}c--\ref{ex:Greco:12}e) it occurs with unaccusative predicates. In all these cases, \textit{nó} is non-referential:

\ea%12
    \label{ex:Greco:12}
	\ea \label{ex:Greco:12a} \gll (\textbf{Nó})  mưa  bây-giờ  đấy.\\
			NÓ  rain  now    \textsc{prt}\\\jambox{(Vietnamese)}
		\glt ‘It is about to rain now.’
	\ex \label{ex:Greco:12b} \gll (\textbf{Nó})  không  có  cái  bút  nào.{\upshape\footnotemark}\\
		NÓ  \textsc{neg}  exist  \textsc{clf}  pen  any\\
		\glt ‘There are no pens.’
	\ex \label{ex:Greco:12c} \gll (Nó)   ngã  thằng  bé.\\
			  NÓ  fall  \textsc{cls}  boy\\
		\glt ‘A/the boy fell.’
	\ex \label{ex:Greco:12d} \gll (Nó)   chết  cá  tao.\\
			  NÓ   die  fish  mine\\
		\glt ‘My fish died.’
	\ex \label{ex:Greco:12e} \gll (Nó)   cháy  cái      nhà     kho.\\
		    NÓ     burnt   \textsc{clf}  house  store\\
		\glt ‘A warehouse burned.’
	\z
\z
\footnotetext{(\ref{ex:Greco:12}b) is ambiguous between the existential and a possessive interpretation with \textit{nó} interpreted as a referential subject pronoun ‘(S)he doesn’t have any pen’. We only discuss the existential reading.}

\subsection{The interpretation of nó}

In contrast with \ili{Finnish} \textstylea{\textit{sitä}}, but in line with some proposals concerning DS \textit{ello,} Vietnamese \textit{nó} does contribute to the interpretation of the clause. Specifically, inserting \textit{nó} narrows down the contexts in which the sentence is appropriate in terms of speaker-related epistemic specificity \citep{GrecoEtAltoappear}.

We first illustrate the interpretive effect brought about \textit{nó} in existential patterns. Existential sentences like \REF{ex:Greco:13} are ambiguous between being either generic statements asserting (or denying) the existence of an entity in general or being contextual statements about the existence of an entity in a specific situation: \REF{ex:Greco:13} either denies the existence of ghosts in general or it denies the presence of ghosts in the context of utterance (while not excluding their existence as such):

\ea%13
    \label{ex:Greco:13}
    \ea \gll Không    có  ma.\\
		  \textsc{neg}    exist  ghost\\\jambox{(Vietnamese)}
	\ex \isi{Generic}: ‘Ghosts do not exist.’
	\ex Contextual: ‘There are no ghosts speaking of a certain place/time.’
\z
\z

Inserting \textit{nó} restricts the domain of validity of the assertion that ‘there are no ghosts’ to a specific context, thus narrowing down the contextualization potential of the containing sentence.

\ea%14
    \label{ex:Greco:14}
    \ea[]{\gll \textbf{Nó}   không  có  ma.\\
					 NÓ   \textsc{neg}  exist  ghost\\\jambox{(Vietnamese)}}
	\ex[\#]{\isi{Generic}: ‘Ghosts do not exist.’}
    \ex[]{Contextual: ‘There are no ghosts speaking of a certain place/time.’}
\z
\z

The ‘contextualizing’ effect of \textit{nó} is also found in sentences with post-verbal unaccusative subjects (\ref{ex:Greco:12c}--\ref{ex:Greco:12e}). These structures are thetic sentences whose \isi{semantic} contribution is to assert the existence of an eventuality of a certain kind \citep{Ladusaw1994}. Typically, these sentences can be uttered out of the blue and they can be used as answers to questions like ‘\textit{What happened}?’. In a thetic sentence, the subject is represented as part of the predicative nucleus (e.g. as a mere participant of an event). \REF{ex:Greco:15} asserts the existence of an event of burning involving a warehouse as the main participant. In thetic sentences, \textit{nó} contributes the implication that the eventuality expressed in the clause is specifically identifiable in or anchored to a given context:

\ea%15
    \label{ex:Greco:15}
    \gll (Nó)    cháy  cái  nhà  kho.\\
         NÓ     burnt      \textsc{clf}    house    store\\\jambox{(Vietnamese)}
    \glt      ‘A warehouse burned.’
\z

This contextualization effect of \textit{nó} appears to be speaker-related: in thetic sentences\textit{ nó} is only felicitous in contexts in which the speaker disposes of sufficient background information to report on a specific event. \REF{ex:Greco:16} and \REF{ex:Greco:17} illustrate the speaker-anchoring achieved by \textit{nó}.

\begin{description}
\item[\textsc{Context} 1:\\]
After meeting a friend who told him that there had been a fire in New York last week and that a warehouse burned down, the speaker utters \REF{ex:Greco:16} as a report.
\ea%16
    \label{ex:Greco:16}
    \gll (\textbf{Nó})  cháy  cái  nhà  kho   rồi\\
	NÓ  burnt  \textsc{clf}  house  store   already\\\jambox{(Vietnamese)}
    \glt ‘A warehouse burned.’
    \z

In this context, information available to the speaker allows him to supply specific spatial and temporal coordinates for the eventuality he’s referring to. In this context \textit{nó} is appropriate, although not obligatory.

\item[\textsc{Context} 2:\\]
The speaker has seen on the television that there had been a fire and that a warehouse has burned down but lacks any further information about this event such as its temporal and locative coordinates. All he knows is that an event of burning took place. In this context, the speaker may utter \REF{ex:Greco:16}, but, crucially, inserting \textit{nó} would be infelicitous:

\ea%17
    \label{ex:Greco:17}
    \gll (\#\textbf{Nó})  cháy  cái  nhà  kho   rồi\\
	NÓ  burnt  \textsc{clf}  house  store   already\\\jambox{(Vietnamese)}
    \glt ‘A warehouse burned.’
\z
\end{description}

Though space prevents a fuller discussion of this point, the crucial requirement for the insertion of the expletive nó appears to be the possibility of the speaker having a specific event in mind (see \citealt{GrecoEtAltoappear}). In this respect, the discourse-related meaning of \textit{nó} can be conceived of as related to some form of speaker-oriented epistemic specificity \citep{Hellan1981,Farkas2002}.

Even when the conditions for its use are met, \textit{nó} is never obligatory, since the contextualization effect can be conveyed implicitly in the context of utterance: inserting \textit{nó} restricts the felicitous contexts of the utterance to a subset of the contexts available \textstylea{without the expletive}.

\subsection{The syntax of nó}

The Vietnamese IP-domain displays a rigidly ordered array of functional morphemes, such as pre-verbal temporal and aspectual markers (\citealt{Duffield2013,Phan2013}), the topmost of which is the future marker \textit{sẽ.} In what looks like its expletive use, the pronoun \textit{nó} occupies a position dominating this element: \REF{ex:Greco:18} illustrates the relevant pattern with the weather verb \textit{mưa} (‘rain’), \REF{ex:Greco:19} illustrates the existential pattern and \REF{ex:Greco:20} illustrates unaccusative \textit{ngã} (‘fall’):

\ea%18
    \label{ex:Greco:18}
	\ea
    \gll Nó  sẽ  mưa  bây-giờ  đấy.\\
	NÓ  \textsc{fut}  rain  now           \textsc{prt}\\\jambox{(Vietnamese)}
    \glt ‘It will rain now.’
	\ex
	\gll *Sẽ  nó  mưa  bây-giờ  đấy.\\
		\textsc{fut}  NÓ  rain  now    \textsc{prt}\\
	\glt ‘It will rain now.’
\z
\z

\ea%19
    \label{ex:Greco:19}
    \ea
    \gll Nó  sẽ  không  có  cái  bút  nào.\\
	NÓ  \textsc{fut}  \textsc{neg}  exist  \textsc{clf}  pen  any\\
    \glt    ‘There will be no pens.’
    \ex
 	\gll  *Sẽ  nó  không  có  cái  bút  nào. \\
      \textsc{fut}  NÓ  \textsc{neg}  exist  \textsc{clf}  pen  any\\
    \glt       ‘There will be no pens.’
    \z
\z

\ea%20
    \label{ex:Greco:20}
\ea
    \gll {Nó} {sẽ}   ngã  thằng    bé.\\
     NÓ  \textsc{fut}  fall  \textsc{clf}    little	\\
    \glt      ‘A/The boy will fall.’
\ex \gll  *\textit{Sẽ} nó  ngã  thằng    bé.\\
      \textsc{fut}  NÓ  fall  \textsc{clf}    little\\
    \glt       ‘A/The boy will fall.’
       \z
 \z

Vietnamese also displays left peripheral scope-discourse markers. For example, \textit{thì} and \textit{là} are associated with topicalized constituents. Following \citet{Rizzi1997}, we analyze these markers as the heads of projections whose specifiers host topicalized constituents:

\ea%21
    \label{ex:Greco:21}
	\ea
    \gll Thằng  Nam  thì/là    sẽ  ăn  cái  này  đấy.\\
	      \textsc{clf}  Nam  \textsc{top}/\textsc{top}  \textsc{fut}  eat  \textsc{clf}  this  \textsc{prt}\\\jambox{(Vietnamese)}
    \glt ‘As for Nam, he will eat this thing.’
	\ex \gll Cái  này  thì/là    thằng  Nam  sẽ  ăn  đấy.\\
			\textsc{clf}  this  \textsc{top}/\textsc{top}  \textsc{clf}  Nam  \textsc{fut}  eat  \textsc{prt}\\
		\glt ‘As for this thing, Nam will eat it.’
    \ex \gll Lúc  khác  thì/là    thằng  Nam  sẽ  ăn  cái  này đấy.\\
  			time  other  \textsc{top}/\textsc{top}  \textsc{clf}  Nam  \textsc{fut}  eat  \textsc{clf}  this \textsc{prt}\\
		\glt   ‘At another time, Nam will eat this thing.’
	\z
\z

As illustrated in \REF{ex:Greco:22}, in its expletive use, \textit{nó} remains lower than the left-periph\-er\-al markers \textit{thì} and \textit{là}:

\ea%22
    \label{ex:Greco:22}
    \ea
    \gll 	(*\textbf{nó}) Trên  bàn  (*\textbf{nó)}   \textbf{thì/là}    \textbf{(nó)}  sẽ  không có cái   bút   nào \\
			NÓ  On  table  NÓ  \textsc{top}/\textsc{top}   NÓ  \textsc{fut}  \textsc{neg}  exist {clf}   pen   any \\\jambox{(Vietnamese)}
\glt     ‘On the table, there will be no pens.’
    \z
\z

In addition, \textit{nó} cannot occur to the left of overt pre-verbal subjects, be they referential DPs or personal pronouns:

\ea%23
    \label{ex:Greco:23}
    \gll  *Nó thằng Nam/tao/mày s\~{e}  gạp  Hòa  ngày-mai. \\
     NÓ   \textsc{clf}   Nam/I/you    \textsc{fut}   meet   Hòa   tomorrow\\\jambox{(Vietnamese)}	
     \glt    ‘Nam/I/you will meet Hòa tomorrow.’ 
    \z

From the distributional data, we conclude that \textit{nó} occupies a position in the highest portion of the inflectional layer, immediately dominated by the left-pe\-riph\-er\-al topic projection:

\ea%24
    \label{ex:Greco:24}
 thì\textsubscript{[\isi{Topic}]} > là\textsubscript{[\isi{Topic}]}  > \uline{nó}  >   sẽ\textsubscript{[Future]} > đã\textsubscript{[Perfect]} > đang\textsubscript{[durative]}  > VP
\z

Assuming that \textit{nó} occupies a high position in the inflectional domain, two avenues can be envisaged to identify the nature of its position: one explores the subject properties of \textit{nó,}  the other explores its speaker-related discourse properties. We discuss these in turn.

The specificity effect of \textit{nó} and the fact that it anchors the proposition to the speaker’s context provides additional empirical support that, while non-refer\-en\-tial, expletives can encode speaker-oriented meaning. Pursuing this line of thinking, \textit{nó} could be associated with a high discourse-related functional projection in the IP domain which encodes \textit{point of view}. This conclusion would be close to that reached for DS \textit{ello} by \citet{GuptonEtAl2014}. It also implies that a high projection in the IP-layer may convey discourse-related functions that are otherwise instantiated in the left periphery.

However, any account of the syntactic position of \textit{nó} has to capture the fact that, besides the \isi{semantic} contribution, \textit{nó} is in complementary distribution with pre-verbal subjects, as illustrated in \REF{ex:Greco:23}. This suggest that \textit{nó} retains some subject properties and could be related to the hypothesis that there is a specialized subject position in the inflectional domain with a subject of predication feature. This projection attracts referential subjects in a number of cases, yielding a structure like \REF{ex:Greco:25}:

\ea%25
    \label{ex:Greco:25}
	[\textsubscript{IP} . . . DP\textsubscript{i [+subject-of-predication]} … [\textsubscript{vP} . . . t\textsubscript{i} . . . ]]
\z

In a number of languages, however, thetic predicative structures leave the subject in-situ, without attracting it to the high IP-field. To capture the complementary distribution of pre-verbal subjects and \textit{nó}, one might propose that \textit{nò} appears only in thetic structure where the referential subject is either absent or left in-situ and that in these structures \textit{nó} occupies the pre-verbal position, namely the position occupied by the referential subject in structures like \REF{ex:Greco:25}.

\citet{Rizzi2006} relates the ‘subject of predication’ property in \REF{ex:Greco:25} to a specialized projection for the subject, SubjP, reinterpreting the \isi{EPP} feature standardly associated with T in terms of a Subject Criterion. One might then propose that \textit{nó} is located in SubjP and assume that in Vietnamese Subj may encode specificity (in a way that is reminiscent of \citealt{Kiss1996} and \citealt{Cardinaletti2004}).

\section{West Flemish \textit{tet}}\label{sec:Greco:5}

In this section we turn to another non-referential element which is formally related to a pronoun and which might at first sight be labeled as ‘expletive’: pleonastic particle \textit{tet} in West Flemish (WF), which is not a \isi{pro-drop} language. Like \ili{Finnish} \textstylea{\textit{sitä}}, DS \textit{ello} and Vietnamese \textit{nó}, the element will be shown to occupy a high position in the inflectional domain and, like DS \textit{ello} and Vietnamese \textit{nó}, it will be shown to convey discourse-related meaning.

In contrast with Vietnamese \textit{nó}\textit{,} however, WF \textit{tet} does not show a complementary distribution with any type of overt subjects: it is compatible with all \isi{finite} clauses\footnote{ For detailed discussion see also \citealt{Haegeman2008}. \textit{Tet} is compatible with infinitival clauses that allow an overt nominative subject. For reasons of space we cannot discuss this here.} and can co-occur with both lexical subjects and the existential expletive \textit{er.} As illustrated by \REF{ex:Greco:26}, in \isi{finite} sentences with a full DP subject, \textit{tet} can be inserted to the immediate left of the canonical subject position. In all instances, \textit{tet} is optional. In the contemporary WF dialect described here, the form \textit{tet} does not have any referential use.

\settowidth\jamwidth{(WF)}
\ea%26
    \label{ex:Greco:26}
    \ea \gll Morgen   goa (tet)  Valère niet kommen.\\
    	    Tomorrow   goes \textsc{tet}   Valere not come\\\jambox{(WF)}
             \glt `Tomorrow Valère is not coming.'
	\ex \gll … dat      (tet)    Valere nie goa kommen\\
		     … that    \textsc{tet}    Valère not goes come\\
            \glt `...that Valère isn't coming.'
    \z
\z

The nature of the form \textit{tet} is unclear but it merits some discussion. \citet[209--210]{DeVogelaer2005} speculates that it may derive from a strong masculine or neuter pronoun (see \citealt{DeVogelaerEtAl2008}). Instead of \textit{tet}, other Flemish dialects and the regional variety of Flemish referred to as the \textit{tussentaal} \citep{CaluweEtAl2013} deploy a strong form of the nominative masculine pronoun \textit{hij}, a form which definitely has a clear co-existing referential use \citep{DeVogelaerEtAl2008,GuéronEtAl2012}. For reasons of space, these alternative forms are not discussed in this paper but for completeness’sake we illustrate the use of \textit{hij} with some examples attested in the informal spoken language by a Brabant speaker \REF{ex:Greco:27a} and a Ghent speaker \REF{ex:Greco:27b}:

\ea%27
    \label{ex:Greco:27}
	\ea \label{ex:Greco:27a} \gll We moeten wij uitprikken   en   dat   telt   hij niet mee.\\
         we must     we logout   and   that   counts \textit{hij} not with.\\\jambox{(WF)}
	\glt ‘We have to log out and that does not count.’
	\ex \label{ex:Greco:27b} \gll Dat kan hij later ook.\\
     that can \textit{hij} later too\\
    \glt `We can do that later too.'
    \z
\z

Our discussion focuses on the use of \textit{tet} in the WF dialect of Lapscheure. \sectref{sec:Greco:5.1.} discusses its syntactic position. \sectref{sec:Greco:5.2.} turns to its interpretive effect. \sectref{sec:Greco:5.3.} discusses the syntax of \textit{tet} and \sectref{sec:Greco:5.4.} briefly turns to its development.

\subsection{The distribution of \textit{tet}}\label{sec:Greco:5.1.}

West Flemish is not a \isi{pro-drop} language in the standard sense\footnote{If subject clitics are the \isi{spell out} of \isi{agreement} features on C or on V \citep{BennisEtAl1984}, one might argue that WF has a \isi{null subject}.} and the language systematically deploys expletive subjects. \REF{ex:Greco:28a} illustrates weather verbs, \REF{ex:Greco:28b} illustrates extraposed subject clauses, \REF{ex:Greco:28c} and \REF{ex:Greco:28d} illustrate existential patterns. As a generalization, indefinite subjects in WF cannot occupy the canonical subject position and expletive insertion is obligatory, including in transitive patterns \REF{ex:Greco:28d}:

\ea%28
    \label{ex:Greco:28}
    \ea \label{ex:Greco:28a} \gll Vrydag   goat =  t   regenen.\\
		    Friday   goes {}    it   rain\\\jambox{(WF)}
		\glt  ‘It is going to rain on Friday.’
	\ex \label{ex:Greco:28b} \gll T’is     nie woar   dat   ze   vrijdag   moet werken.\\
		    {it is}     not true   that   she   Friday   must work\\
		\glt ‘It is not true that she must work on Friday.’
	\ex \label{ex:Greco:28c} \gll kpeinzen   dat=ter   veel   volk     goat kommen.\\
		    {I think}   {that   there}   much   people   will come\\
		\glt ‘I think that many people will come.’\\
	\ex \label{ex:Greco:28d} \gll kpeinzen   dan=  der   veel studenten   dienen boek   goan kopen.\\
			{I think}   that.\textsc{pl}  there   many students   that   book   will buy\\
		\glt `I think that many students will buy that book.’
	\z
\z

The element \textit{tet} can be inserted in all \isi{finite} clauses, in embedded clauses (\ref{ex:Greco:29}a,b), in non-subject-initial \isi{V2} root clauses (\ref{ex:Greco:29}c,d) and in subject-initial \isi{V2} root clauses (\ref{ex:Greco:29}d):

\ea%29
    \label{ex:Greco:29}
    \ea \gll Ik peinzen   dat   tet   Valere   vrydag   moet werken.\\
		    I think   that   \textsc{tet}   Valere   Friday   must work\\\jambox{(WF)}
		\glt ‘I think that Valery must work on Friday.’
	\ex \gll Oa   tet   Valere   vrydag   moet werken…\\
			 if   \textsc{tet}  Valere   Friday   must work…\\
		\glt ‘If Valery must work on Friday…’
	\ex \gll Woar     is   tet   menen paraplu?\\
			 Where   is   \textsc{tet}  my umbrella\\
		\glt ‘Where is my umbrella?’
	\ex \gll Vrydag   moet   tet   Valere werken.\\
			 Friday   must   \textsc{tet}  Valere work\\
		\glt ‘On Friday Valery must work.’
	\ex \gll Valère   moet   tet  vrydag   werken.\\
	    	 Valère    must  \textsc{tet}  Friday    work\\
	    \glt `Valery must work on Friday.’
	\z
\z

The distribution of \textit{tet} is not sensitive to the nature of the subject, in particular it can co-occur with a DP subject \REF{ex:Greco:29}, with a clitic subject (\ref{ex:Greco:30}a), with a clitic subject doubled by a full pronominal subject (\ref{ex:Greco:30}b), in sentences with expletive subjects with weather verbs (\ref{ex:Greco:30}c), in extraposition patterns with expletive \textit{t} (\ref{ex:Greco:30}d), as well as in existential sentences with expletive \textit{der} (\ref{ex:Greco:30}e). In the dialect described (cf. \citealt{DeVogelaerEtAl2008}), \textit{tet} cannot itself take on the function of the expletive, omission of the expletives in (\ref{ex:Greco:30}c--e) systematically leads to ungrammaticality:

\ea%30
    \label{ex:Greco:30}
    \ea \gll Oa=ze   tet   vrydag   moet werken …\\
    	    if=she   \textsc{tet}   Friday   must work\\\jambox{(WF)}
    \ex \gll Oa=ze   tet zie     vrydag moet werken …\\
    	    if=she   \textsc{tet} she   Friday must work\\
    	\glt ‘If she must work on Friday…’
    \ex \gll oat=*(t)  tet   vrydag regent …\\
    	    if=it     \textsc{tet}  Friday rains\\
    	\glt `if it rains on Friday...’
    \ex \gll oat=*(t)   tet   woar   is   dat=ze   vrydag  moet werken …\\
             if=it     \textsc{tet}  true   is   that=she   Friday   must work\\
        \glt ‘If it’s true that she must work on Friday…’

    \ex \gll oat=*(der)   tet   veel volk     komt…\\
    	     if=there   \textsc{tet}  much people   comes…\\
    	\glt ‘If there are many people coming…’
    \z
\z
The occurrence of \textit{tet} is independent of the nature of the predicate, it is compatible with all types of predicates including, for instance, transitive patterns with subjects in the canonical subject position:

\ea%31
    \label{ex:Greco:31}
    \gll dat   tet   Valère   dat   niet   gezeid   eet\\
         that   \textsc{tet}   Valère     that   not   said     has\\\jambox{(WF)}
\z

Linearly, \textit{tet} occupies a fixed position: it follows the (\isi{agreeing}) \isi{complementizer} and any subject (or object) clitics that may have adjoined to that, and it immediately precedes the canonical subject position. Importantly, apart from the object clitics \textit{t, ze} and \textit{der, tet}  is the only constituent that can separate the \isi{complementizer} from the definite subject. Interjections and discourse particles  or adverbial adjuncts cannot be inserted in this position:

\ea%32
    \label{ex:Greco:32}
    \ea[*]{
    \gll Oa   toch     Valere   moet werken…\\
	if   \textsc{part}     Valere   must work…\\\jambox{(WF)}}
    \ex[*]{
    \gll Oa   vrydag   Valere   moet werken…\\
         if   Friday   Valere   must work…\\}
    \z
\z

Nor can such elements separate \textit{tet} from the \isi{complementizer} (\ref{ex:Greco:33}a,b) or from the canonical subject (\ref{ex:Greco:33}c,d):

\ea%33
    \label{ex:Greco:33}
    \ea[*]{\gll Oa   toch/vrydag   tet   Valere   moet werken…\\
           if   \textsc{part}/Friday  \textsc{tet}    Valere   must work…\\\jambox{(WF)}}
    \ex[*]{\gll Oa   tet   toch/vrydag  Valere   moet werken…\\
	       if   \textsc{tet}    \textsc{part}/Friday  Valere   must work…\\}
	\z
\z

In root clauses, \textit{tet} immediately follows the inflected verb from which it can only be separated by clitics. In non subject-initial \isi{V2} \REF{ex:Greco:34a} \textit{tet} precedes the definite DP subject, to which it is adjacent. In subject-initial \isi{V2} sentences \REF{ex:Greco:34b} \textit{tet} follows the \isi{finite} verb (see \citealt{CraenenbroeckEtAl2007} for the relevance of these data for the analysis of \isi{V2}):

\ea%34
    \label{ex:Greco:34}
    \ea \label{ex:Greco:34a} \gll Vrydag   moet   (*toch)   tet   (*toch) Valere werken.\\
    	    Friday   must   (*\textsc{part})   \textsc{tet}  (*\textsc{part}) Valere work\\\jambox{(WF)}
            \glt `Friday, Valère has to work.'
	\ex \label{ex:Greco:34b} \gll Valere   moet   (*toch)   tet   (toch)   vrydag werken.\\
		     Valere   must   (*\textsc{part})   \textsc{tet}  (\textsc{part}) Friday work\\
	\glt `Friday, Valère has to work.'
    \z
\z

\subsection{The interpretation of \textit{tet}}\label{sec:Greco:5.2.}
The element \textit{tet} lacks referential content and co-occurs with any kind of subject \citep{Haegeman2008}. Unlike \ili{Finnish} \textit{sitä}, but like DS \textit{ello} and Vietnamese\textit{ nó, tet} makes an interpretive contribution to the clause by narrowing down the contextualization possibilities for the utterance. However, the \isi{semantic} contribution of \textit{tet} is not identical to that of \textit{nó}. While the Vietnamese expletive relates to the speaker’s epistemic state, \textit{tet} introduces speaker-related emphasis and contrasts the containing utterance with the discourse. By inserting \textit{tet,} the speaker signals that the propositional content of the utterance containing \textit{tet} conflicts with some contextually salient assumptions. For example, the \textit{wh-}question in \REF{ex:Greco:35a} asks for the identity of a person. The unmarked answer to \REF{ex:Greco:35a} is \REF{ex:Greco:35b}. \REF{ex:Greco:35c}, with \textit{tet,} will be a felicitous answer to \REF{ex:Greco:35a} if, for some reason, Valère’s presence is unexpected to the speaker and conflicts with his discourse background:


\ea%35
    \label{ex:Greco:35}
    \ea \label{ex:Greco:35a} \gll Wien is dadde?\\
    		 who   is that\\\jambox{(WF)}
		\glt ‘Who’s that?’
	\ex     \label{ex:Greco:35b} \gll Dat   is Valère.\\
		     that   is Valère\\
		\glt ‘That’s Valère.’
	\ex     \label{ex:Greco:35c} \gll Dat   is tet   Valère!\\
		     that   is \textsc{tet}  Valère \\
		\glt ‘That’s Valère!’
	\z
\z

Recall that \textit{tet} is never obligatory. The conflict in contextualization need not be encoded, or the speaker may achieve the effect differently, for instance by stressing \textit{Valère} in (\ref{ex:Greco:35}b).


(\ref{ex:Greco:29}c), repeated here as \REF{ex:Greco:36}, illustrates the same point: without \textit{tet}, it is a neutral question about the location of the speaker’s umbrella, with \textit{tet} the question is appropriate if the umbrella is unexpectedly missing:


\ea%36
    \label{ex:Greco:36}
    \gll Woar    is    (tet)   menen paraplu?\\
         where    is  \textsc{tet}  my       umbrella\\\jambox{(WF))}
/glt `Where is my umbrella?'
\z


Given its discourse function, one might be inclined to assimilate \textit{tet} to dis\-course-related adverbs, particles, or interjections. However, as we have discussed, such elements are distributionally different.


\subsection{The syntax of \textit{tet}}\label{sec:Greco:5.3.}

Because \textit{tet} to some extent alternates with focal stress, one might associate it with the left peripheral FocP \citep{Rizzi1997}. This is not plausible, though, because \textit{tet} occurs in \textit{wh}{}-questions \REF{ex:Greco:36}. If the \textit{wh}{}-constituent \textit{woar} (‘where’) occupies the specifier of the root FocP, the position of \textit{tet} must be lower than the left-peripheral FocP. \textit{Tet} follows the \isi{complementizer} and it precedes the definite subject DP. These data suggest that \textit{tet} occupies\textit{} a high IP-related functional position. If definite DP subjects occupy the canonical subject position (i.e. the specifier of \isi{TP} or SubjP), the functional projection hosting \textit{tet,} FP, must immediately dominate the projection hosting the subject. The fact that \textit{tet} occurs to the right of clitic subjects follows if these are cliticized to the C-domain, as is commonly assumed.  \REF{ex:Greco:37a} is a schematic representation:



\ea%37
    \label{ex:Greco:37a}
    [\textsubscript{CP} [\textsubscript{C} da]   [\textsubscript{FP} tet [\textsubscript{F}]   [\textsubscript{TP} Valere vrydag moet werken]]]
\z

All \isi{V2} clauses are derived by \isi{finite} \isi{V movement} to C \citep{CraenenbroeckEtAl2007}. It follows from \REF{ex:Greco:37a} that in \isi{V2} clauses \textit{tet} will be adjacent to the \isi{finite} verb in \isi{V2} sentences, from which it can only be separated by those clitics that can themselves right-adjoin to the \isi{finite} verb in C.\footnote{An alternative is that the relevant projection in the low left periphery, but this approach would have important ramifications. In particular, if \textit{tet} is in a left-peripheral projection, the \isi{complementizer} \textit{dat} and the \isi{finite} verb in \isi{V2} patterns must themselves occupy a higher left peripheral position, the nature of which would need to be clarified.} In line with \citet{CraenenbroeckEtAl2007}, \citet{CraenenbroeckEtAl2012,GuéronEtAl2012} propose that FP is Uriagereka’s FP, \citep{Uriagereka2004,Carrilho2008}, and following \citet{Grohmann2000} they reinterpret the projection as one encoding Point of View:

\ea \label{ex:Greco:37b} \isi{CP} > \isi{FinP} > PovP > \isi{TP} \z
\subsection{Cross-speaker variation and the nature of \textit{tet}} \label{sec:Greco:5.4.}
Though, informally speaking, \textit{tet} appears to be located somewhere in a ‘subject zone’ of the clause, and is sandwiched between the clitic subject and the full pronominal subject (\ref{ex:Greco:30}b), \textit{tet} cannot be assimilated to the expletives which satisfy a formal requirement because such expletives in fact co-occur with \textit{tet}.

Note that the wide distribution of \textit{tet} or its analogue \textit{hij} in some varieties of Flemish, including that described here, is not shared by all speakers. Based on a native speaker questionnaire, \citet[272, 278]{DeVogelaerEtAl2008} speculate that the current distribution of \textit{tet}/\textit{hij} is a recent extension which has taken it beyond its original doubling function. The strong pronouns originally served as ‘topic markers’ used to double third person clitic subjects, including expletive subjects. At this stage, the doubling pronoun matched the clitic pronoun in \isi{gender} and \isi{number}. The pronouns could also be used to double an expletive clitic subject. In their extended use, the elements \textit{hij, (t)jij} or \textit{tet} have come to be used more liberally and co-occur with all subjects, regardless of their \isi{gender} and \isi{number}. With the extension, the restriction by person and \isi{number} features postulated for the topic marking function of the doubling pronouns has been lost. We speculate that it is at this point that the pronominal elements lost their phi features, i.e. their nominal properties. With the loss of the nominal properties, then, the element has acquired a new discourse function and a wider distribution.

\section{Recycling expletives as discourse particles}\label{sec:Greco:6}
In this paper we started out from the fairly standard view of expletive elements as pronominals which have lost their referential content and have become place holders for the subjects in contexts in which a formal requirement imposes the presence of a subject and in which no suitable DP subject can fulfill the requirement. The standard view on expletives leads to a set of generalizations: (i) they are generally unexpected in \isi{pro-drop} languages, (ii) they are semantically vacuous, (iii) they are not optional.

In our paper, we investigate a set of pronominal forms that have lost their referential meaning and might at first sight be analyzed as expletives. The data discussed lead to a more nuanced view of the nature of expletives, in which the generalizations outlined above seem to be challenged. With respect to the correlation between the availability of expletives and the \isi{pro-drop} nature of a language, there are cases, like \ili{Finnish} \textit{sitä}\textit{,} where a \textstylea{\isi{pro-drop} language may still employ expletive elements in a subset of contexts, if needed because of EPP-requirements.}

In addition, the case of DS \textit{ello} illustrates a class of expletives or expletive-like elements without referential content which, though retaining the distributional properties of expletives, seems to have acquired a discourse-related meaning. Pursuing this point, \textstylea{we have discussed two additional instances of pronominal forms that have lost their referential meaning and seem to have acquired a discourse function.}

Vietnamese \textit{nó} is a pronominal form without referential content that has acquired some discourse-related meaning: \textit{nó} serves to narrow down the contextualization properties of the utterance that contains it. WF \textit{tet} originates as a strong pronominal form, it \textstylea{has lost its referential value and it has the discourse function of constraining} the contextualization of the containing utterance to those contexts where the utterance’s propositional content conflicts with the speakers’ prevalent assumptions.

Since it is in complementary distribution with lexical subjects and is restricted to certain predicate types, we proposed that Vietnamese \textit{nó} is located in a dedicated subject projection that encodes specificity and which is otherwise occupied by lexical subjects. Differently, WF \textit{tet,} while originating as a strong pronominal doubler of, among others, an expletive subject clitic, and while being located in what appears to be the subject portion of the clause, never takes on any subject function and never competes with a subject constituent for the same position. \textstylea{We propose that} \textstylea{\textit{tet}} \textstylea{appears in an optional position encoding} \textstylea{\textit{point of view}} \textstylea{which is not subject-related.}

The data we have discussed here lead to a more complex picture of the nature of expletives and their function in the grammar. The elements we have discussed here all share the property that they are pronominal forms having lost referential value, the hallmark of the prototypical expletive, but while the prototypical expletive has a purely formal function, DS \textit{ello,} Vietnamese \textit{nó} and Flemish \textit{tet} are pronominal elements which, having lost their referential meaning, seem to have acquired discourse-related functions.

\section*{Acknowledgements}
We dedicate this paper to Anders Holmberg, whose work on \ili{Finnish} expletives has sparked our own. Our research was funded by FWO Belgium as part of project 2009-Odysseus-Haegeman-G091409.

\section*{Abbreviations}
Abbreviations used in this article follow the Leipzig Glossing Rules’ instructions for word-by-word transcription, available at: \url{https://www.eva.mpg.de/lingua/pdf/Glossing-Rules.pdf}.

The non-standard abbreviation used:
\begin{tabbing}
\textsc{prt} \= Particle \kill
\textsc{prt} \> Particle
\end{tabbing}

\sloppy
\printbibliography[heading=subbibliography,notkeyword=this]
\end{document}
