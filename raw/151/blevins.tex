\documentclass[output=paper,
modfonts
]{LSP/langsci}
\ChapterDOI{10.5281/zenodo.495434}

%\input{localpackages.tex}

%% TRV should be caps, not small caps (note here for LSP copyeditors and proofers)

\title{Between natural and unnatural phonology: The case of cluster-splitting epenthesis }
\author{Juliette Blevins\affiliation{The Graduate Center, CUNY}}

\abstract{A widely recognized feature of loan-word phonology is the resolution of clusters
by vowel epenthesis. When a language lacking word-initial clusters borrows words from
a language with initial \#TRV sequences, T an oral stop and R a liquid, it is common to
find vowel epenthesis, most typically vowel-copy, as in, for example: Basque <g\textbf{u}rutze> ‘cross’ from Latin <cruce(m)>; Q’eqchi’ <k\textbf{u}rus> ‘cross’ from Spanish <cruz> ‘cross’, or Fijian <k\textbf{o}losi> ‘cross’ from English <cross>. The phonological rule or sound change responsible for this pattern is sometimes called ``cluster-splitting epenthesis": \#TRV\textsubscript{i} > \#TV\textsubscript{(i)}RV\textsubscript{i}. The most widely accepted explanation for this pattern is that vowel epenthesis between the oral stop and the following sonorant is due to the vowel-like nature of the TR transition, since \#TRV\textsubscript{i} is perceptually similar to \#TV\textsubscript{(i)}RV\textsubscript{i}. A fact not often appreciated, however, is that cluster-splitting epenthesis is extremely rare as a language-internal development. The central premise of this chapter is that \#TRV\textsubscript{i} in a non-native language is heard or perceived as \#TV\textsubscript{(i)}RV\textsubscript{i} when phonotactics of the native language demand TV transitions. Without this cognitive component, cluster-splitting epenthesis is rare and, as argued here, decidedly unnatural.}


\begin{document}
\maketitle

\section{Introduction}

Diachronic\is{diachrony} explanations have been offered for both natural and unnatural
sound patterns in human spoken languages. Building on the Neogrammarian
tradition, as well as the experimental research program of \citeauthor{ohala1971a} (e.g. \citeyear{ohala1971a, ohala1974a, ohala1993a}), it is argued that natural sound patterns, like final obstruent
devoicing, nasal place \isi{assimilation}, vowel harmony, consonant lenition,
and many others, arise from regular sound changes with
clear phonetic bases (\citealt{blevins2004}, \citeyear{blevins2006b}, \citeyear{blevins2008a}, \citeyear{blevins2015a}; \citealt{anderson2016a}).
Unnatural sound patterns, in contrast, cannot be explained in terms of
simple phonetic processes. Case studies of unnatural patterns show more
complex histories: in some cases more than one natural change has been
collapsed; in others, a natural change is reanalyzed or inverted; in
still others, \isi{analogy}\is{analogical} has extended a sound pattern whose origins are
morphological, not phonological; and combinations of all of these paths
can also be observed \citep{bach1972a,anderson1981,buckley2000a,vaux2002a,blevins2008a,blevins2008b,blevins2008c,garrett2009a,anderson2016a}.

However, typological study of regular \isi{sound change} reveals certain kinds
of \isi{sound change} that are neither wholly natural nor wholly unnatural.
For example, the shift of *\#kl \textgreater{} \#tl, documented in at
least three different language families, appears to have a natural basis
in perception, since {[}kl{]} and {[}tl{]} clusters are acoustically
similar and confused with each other. However, the rarity of this sound
change is associated with structural factors: misperception of {[}kl{]}
as {[}tl{]} is strongly associated with the absence of phonological /tl/
clusters in a language. In this case, the absence of a sound pattern,
/tl/, influences cognition, making listeners more likely to perceive
{[}kl{]} as {[}tl{]} \citep{blevinsju2009a}. The presence of a
contrast can also facilitate particular types of \isi{sound change}. As first
argued by \citet{deChene1979}, \isi{compensatory lengthening} sound
changes are strongly associated with pre-existing \isi{vowel length}
contrasts. This statistical tendency is argued to arise from
phonetically natural vowel lengthening in pre-sonorant and open syllable
contexts \citep{kavitskaya2002}, combined with the cognitive effects of
structural \isi{analogy},\is{analogical} where pre-existing categories act as attractors in
the course of \isi{language acquisition} (\citealt[150--155]{blevins2004} and Kavitskaya, this volume).

In this contribution, I offer another example of regular \isi{sound change}
that is neither wholly natural nor wholly unnatural, and highlight the
role of human cognition in cases where it has occurred.
Cluster-splitting \isi{epenthesis} is of interest, not only because of its
rarity as a regular \isi{sound change}, but in how it advances our
understanding of sound patterns compared to 20th century models \citep{anderson1985a}, and emphasizes the extent to which historical
linguists, phonologists, and phoneticians still need the cognitive
scientist \citep{anderson2001a}.

\section{Cluster-splitting {epenthesis} in {loanword} phonology}

A widely recognized feature of \isi{loanword} \isi{phonology} is the resolution of
clusters by vowel \isi{epenthesis}. When a language lacking word-initial
clusters borrows words from a language with initial \#TRV sequences, T
an oral stop and R a liquid, it is common to find vowel \isi{epenthesis}, most
typically vowel-copy, as illustrated in \REF{ex:blevins:1}.\footnote{For these and
  other examples, see: \citet{blevins2016a} on \ili{Latin} loans in
  \ili{Basque}; \citet{decasparis1997a} on \ili{Sanskrit} loans in Indonesian; \citet{campbell2013a} on Colonial \ili{Spanish} loans into \ili{Mayan} languages; and \citet{kenstowicz2007a} on Fijian \isi{loanword} phonology. A reviewer notes that some
  varieties of Q'eqchi' permit initial /CR/ clusters. Proto-\ili{Mayan} had
  only *CVC syllables, and \ili{Mayan} \emph{kurus} arguably reflects
  borrowing of Colonial \ili{Spanish} \emph{cruz} into a language which lacked
  initial CR clusters.}


If \isi{loanword} \isi{phonology} is taken as evidence for properties of
phonological\is{phonology} grammars \citep{hyman1970a}, a phonological\is{phonology} rule describing this
pattern can be stated as in \REF{ex:blevins:2}.

\begin{minipage}{\linewidth}
\ea \label{ex:blevins:1} Cluster-splitting \isi{epenthesis} in \isi{loanword} phonology 
%\begin{table}
\begin{tabular}{lllll}
a. 	& Source language	& \ili{Latin}	&  \emph{crucem} & `cross' \\
	& Target language	& \ili{Basque}	 &  g\textbf{u}rutze & `cross' \\
& & & & \\
b.	& Source language 	& \ili{Sanskrit}  & \emph{klēśa} &  `defilement' \\
	& Target language 	& Indonesian	&  k\textbf{e}lesa  & `indolent' \\
& & & & \\
c.	& Source language	& \ili{Spanish} &  \emph{cruz} & `cross' \\
	& Target language 	& Q'eqchi' & k\textbf{u}rus & `cross' \\
& & & & \\
d.	& Source language 	& \ili{English} & \emph{cross} & `cross' \\
	& Target language 	& Fijian  & k\textbf{o}lósi  & `cross' \\
\end{tabular}
%\end{table}
\z
\end{minipage}
\smallskip


\ea \label{ex:blevins:2} Cluster-splitting vowel-\isi{epenthesis}
	\begin{itemize}
	\item[] \#TRV\textsubscript{i} → \#TV(\textsubscript{i})RV\textsubscript{i}
	\end{itemize}
\z

Within the structuralist and generative traditions detailed in \citet{anderson1985a}, the locus of explanation for this type of \isi{epenthesis} lies in
phonotactic differences between the source language and the target
language. Under this general account, the speaker of the target language
hears a word pronounced in the source language, constructs a
phonological\is{phonology} representation with an initial \#TR cluster based on this
hearing, but then alters this phonological\is{phonology} representation in line with
the phonotactics of the speaker's native language which lacks initial
\#TR clusters (e.g. \citealt{broselow1987a}; \citealt{ito1989a}).

\largerpage[-1]
Typological studies of \isi{loanword} \isi{phonology} and advances in our
understanding of speech perception have given rise to 21st century
treatments of these patterns that are more explanatory in accounting not
only for a ``repair" but for the specific type of sound pattern that
results. At present, the most widely accepted explanation for the sound
pattern in \REF{ex:blevins:2} combines two new findings in speech perception, one
related to perceptual similarity, and the other related to perceptual
illusions. A first component of the analysis is that vowel-\isi{epenthesis}
between the oral stop and following sonorant is due to the vowel-like
nature of the TR transition \citep{fleischhacker2001a, fleischhacker2005a,Kang2011, berent2013a, broselow2015a}. \citet{fleischhacker2001a, fleischhacker2005a} argues that the general
pattern is determined by perceptual similarity: initial TR clusters are
more perceptually similar to TVR than VTR. An important \isi{aspect} of her
work is the distinction between initial \#TR clusters and initial \#sT
clusters, which rarely show vowel-splitting \isi{epenthesis}, and defy purely
phonotactic accounts. A second component of the analysis relates to
specific structural differences between the source and target languages.
Under the perceptual account, perception of \#TR by native speakers of
languages that lack initial \#TR is biased: these speakers will tend to
hear a vowel between the oral stop and the following liquid, even if no
vowel is present. Experimental work supporting the existence of
illusionary vowels for Japanese speakers presented with CC clusters was
presented in \citet{dupoux1999a}, and has been supported by much
subsequent work (see \citealt{peperkamp2002a,kang2003a,Kang2011,kabak2007a,davidson2012a}), including a range of studies
showing vowel percepts in TR clusters (\citealt{berent2013a}, and works cited
there).\footnote{This approach is not agreed upon by all researchers.
  See \citet{uffmann2007a} and \citet{hall2011a} for discussion.}

Given this evidence, one might conclude that the sound pattern described
in \REF{ex:blevins:2} is both natural and common, having a clear phonetic explanation
\citep{blevins2004, blevins2008a, blevins2015a}. As a natural, phonetically-based sound
pattern one might expect many instances of reconstructed word-initial
*TR clusters to be resolved by a \isi{sound change} parallel to the synchronic
rule in \REF{ex:blevins:2}. A \isi{sound change} of this kind might be even more common than
expected on phonetic grounds due to \isi{markedness} proposals stating that
complex\is{complexity} onsets are less preferred than simple onsets \citep{prince1993a,kager1999a}. As I show below, these expectations are not
borne out, suggesting that cognitive bias in the context of novel
stimuli plays a central role in cluster-splitting \isi{epenthesis}.

\section{Cluster-splitting {epenthesis} as regular sound change}

Very few well-studied and widely agreed upon proto-languages are
reconstructed with initial *TR clusters at the oldest stages. One
exception is Proto-\ili{Indo-European}, reconstructed with *TR clusters as
well as other initial cluster types \citep[64--65]{fortson2011}. Some widely
agreed upon Proto-\ili{Indo-European} reconstructions with initial *TR
clusters are shown in \REF{ex:blevins:3}.

\begin{exe}\label{ex:blevins:3}
\ex Word-initial *TR in Proto-\ili{Indo-European} \\
\begin{tabularx}{.85\textwidth}{lX}
a. *gras- `eat'  & Cf. \ili{Vedic} \emph{grásate} `eats, feeds', Greek
\emph{grástis} `green fodder, grass', \ili{Latin} \emph{grāmen} (\textless{} *gras-men) `grass, fodder' \\
& \\
b.  *prek\textsuperscript{j}- `ask' & Cf. \ili{Vedic} \emph{pṛccháti} `asks',
\ili{Latin} \emph{precor} `I entreat', German \emph{fragen}, Tocharian B \emph{prek}- \\
&\\
c. *trejes `three' & Cf. Lycian \emph{tri}-, \ili{Vedic} \emph{tráyas}, Greek
\emph{treĩs,} \ili{Avestan} \emph{θrayō,} \ili{Latin} \emph{trēs} \\
\end{tabularx}
\end{exe}

The \ili{Indo-European} language family is relatively large, relatively
diversified, and relatively well-studied in comparison with other
language families of the world. According to \emph{Ethnologue}, there
are approximately 445 living \ili{Indo-European} languages at present, and
linguists agree that the major subgroups of Anatolian, Indo-Iranian,
Greek, Italic, Celtic, Armenian, Tocharian, and Balto-Slavic have had
long independent developments. If cluster-splitting vowel-\isi{epenthesis} in
\REF{ex:blevins:2} has a natural phonetic basis in perceptual similarity as outlined
above, then a \isi{sound change} like \REF{ex:blevins:4} might be expected to have occurred
numerous times in the \ili{Indo-European} language family.

\begin{exe} \label{ex:blevins:4}
\ex Cluster-splitting vowel-\isi{epenthesis} as \isi{sound change}
\begin{itemize}
\item[] \#TRV\textsubscript{i} \textgreater{} \#TV(\textsubscript{i})RV\textsubscript{i}
\end{itemize}
\end{exe}

However, cluster-splitting vowel \isi{epenthesis} as a regular \isi{sound change} is
rare in the \ili{Indo-European} language family. *TR clusters are inherited
intact in all of the major subgroups, and sound changes affecting these
clusters at later stages of development are of distinct types (e.g.
palatalization of *l in Romance *Tl clusters; loss of *p in Celtic).

Indeed, within the entire \ili{Indo-European} language family, there appears
to be only one clear instance of a regular \isi{sound change} like
\REF{ex:blevins:4}.\footnote{\citet[302--303]{fortson2011} mentions evidence for a \isi{sound change}
  similar to cluster-splitting \isi{epenthesis} in Oscan, an extinct Italic
  language known from inscriptions from approximately 400 BC - 100 CE,
  as in \emph{aragetud} `with money' (cf. Lat. \emph{argentō)} and
  \emph{sakarater} `it is consecrated' (cf. Lat. \emph{sacratūr}).
  However, the *rg cluster continued in \emph{aragetud} was arguably
  heterosyllabic R.T (as opposed to tautosyllabic TR) and initial TR
  clusters are continued intact in Oscan as in \emph{trístaa, trííbúm,
  prúfatted} (op cit.).} The \isi{sound change} in question appears to have
occurred in relatively recent times, in the transition from Middle
Persian to Modern Persian (aka New Persian, Farsi, Dari, Tajiki), or,
perhaps more generally, from Middle to Early New Iranian.\footnote{In
  his discussion of East and West Iranian dialectology, \citet[21]{windfuhr2009} states the reflexes of initial \#CC-clusters as showing a
  distinct areal distribution: ``insertion of a short vowel, CVC-, along
  the Zagros, including the NW tier I from Kurdish, Zazaki to the SW
  Fars and Larestan dialects, as opposed to initial vowel, VCC-,
  elsewhere", while in the east, Balochi and most East Iranian languages
  allow initial clusters.} While the specific \isi{sound change} is rarely
stated as in \REF{ex:blevins:4}, it is implied. For example in their chapter on modern
Persian and Tajik phonology, \citet[427--428]{windfuhrperry2009} describe
the language as having \isi{syllable} onsets consisting of only a single
consonant, and note that ``The inherited initial clusters have been
resolved by prothetic or \isi{epenthetic} vowels, either of which could become
standardized, e.g. st-: \emph{stār} `star' \textgreater{}
\emph{setāre/sitora}, br: \emph{brādar} `brother' \textgreater{}
\emph{barādar/barodar...}"  \citep[428]{windfuhrperry2009}. The \isi{epenthesis} process
described is identical to that schematized in \REF{ex:blevins:4}, and it is also
characteristic in \isi{loanword} phonology,\is{phonology} on which much more has been
written (see, e.g. \citealt{strain1968a}, \citealt{karimi1987a}). Illustrative examples
comparing Middle Persian inherited clusters to Modern Persian \#CVC
sequences are shown in \REF{ex:blevins:5}.

\ea \label{ex:blevins:5} One case of cluster-splitting \isi{epenthesis} in \ili{Indo-European}: Modern Persian 
\begin{table}
\begin{tabular}{llll}
Middle Persian & Modern Persian & gloss & PIE\\
a. brādar  & barādar & `brother' & *b\textsuperscript{h}réh\textsubscript{2}ter- \\

b. griftan & gereftan, & `grab, take' &  *g\textsuperscript{h}rebh\textsubscript{2}- \\
 & giriftan & & \\

c. draxt & daraxt  & `tree' &  *drew- `wood' \\

d. griy- & geri- & `to cry' & *g\textsuperscript{h}reh\textsubscript{2}d- \\
\end{tabular}
\end{table}
\z

A second case of cluster-splitting \isi{epenthesis} \isi{sound change} is found in
the Siouan-Catawba language family, a small group of languages in North
America that includes Crow, Hidatsa, Mandan, Lakota, Dakota,
Assiniboine, Yanktonai, Stoney, Sioux Valley, Chiwere (aka
Iowa-Missouria-Otoe), Hoocąk (aka Winnebago), Omaha-Ponca, Ponca,
Kanza/Kaw, Osage, Quapaw, Biloxi, Ofo, Tutelo, Saponi, Catawba and
Woccon. The diachronic\is{diachrony} process known as Dorsey's Law \citep{dorsey1885a} is a
\isi{sound change} taking Proto-Siouan *TRV to
\#TV\textsubscript{i}RV\textsubscript{i} in Hoocąk (aka
Winnebago).\footnote{Dorsey's Law also refers to the resulting
  synchronic sound pattern in Hoocąk. It also applies to medial
  clusters. Since the \isi{syllabification} of medial TR is ambiguous
  cross-linguistically, discussion is limited here to initial \#TR
  where, at least utterance-initially, sequences constitute unambiguous
  complex onsets.}\is{complexity} Examples from \citet{rankin2015a} are shown in \REF{ex:blevins:6}.

\begin{exe} \label{ex:blevins:6}
\ex One case of cluster-splitting \isi{epenthesis} in Siouan: Dorsey's Law in
Hoocąk
\begin{table}
\begin{tabular}{llll}
Chiwere & Hoocąk & Proto-Mississippi-Valley &  gloss\\

a. églųñį & waki/kųnųnį & *krų́rį & `forget' \\

b. glé & keré & *kre & `go back to' \\

c. wa/brú & ru/purú `plough' & \emph{*}prú & `powder' \\
\end{tabular}
\end{table}
\end{exe}

While the time-depth of Siouan-Catawba is thought to be 2,000--3,000
years \citep{parks2001a}, Hoocąk and Chiwere are considered to be
closely related and even sometimes treated as dialects of a single
language \citep{miner1979a}.\footnote{\citet[25]{miner1979a} begins his article with
  the statement that: ``Winnebago and Chiwere ... are, in spite of their
  geographical separation in historical times, very closely related and
  enjoy a high degree of mutual comprehensibility." He also notes on the
  same page (footnote 1) that ``Winnebago-Chiwere is sometimes referred
  to in the literature simply as Chiwere."} Given this, Dorsey's Law
must be a relatively recent development.

Outside of the Persian and Hoocąk cases, it is difficult to find
convincing cases of cluster-splitting \isi{epenthesis} as a diachronic\is{diachrony}
development. And here lies the central point of interest. Given that
cluster-splitting \isi{epenthesis} is common in \isi{loanword} \isi{phonology} \REF{ex:blevins:2}, and
appears to be a natural phonetically-motivated process, why is it rarely
attested as a regular \isi{sound change}? Why, out of more than 440
\ili{Indo-European} languages, is there only one clear case of a
\#TRV\textsubscript{i} \textgreater{}
\#TV(\textsubscript{i})RV\textsubscript{i} \isi{sound change}? And how should
we understand the Siouan sister-languages Chiwere and Hoocąk, where
Chiwere continues \#TRV, but Hoocąk does not?

I suggest that cluster-splitting \isi{epenthesis} is neither wholly natural
nor wholly unnatural: non-phonetic structural and cognitive factors are
involved. The structural condition is that cluster-splitting \isi{epenthesis}
occurs only when speakers of a language that \emph{lacks} initial TR
clusters begin to acquire a language that \emph{has} initial TR
clusters. It is only under this circumstance that the perceptual
illusion of \#TRV as \#TVRV arises (cf. \citealt{dupoux1999a}), with this
perceptual illusion constituting the cognitive catalyst for phonological\is{phonology}
change. An important component of this model is that regular sound
changes of this kind will only occur under special types of language
contact, where speakers dominant in a language that lacks initial
consonant clusters suddenly (or without extensive exposure) acquire a
language with \#CR-clusters.\footnote{For a similar proposal regarding
  paragoge (final vowel insertion), see \citet{ng2015a}, a
  dissertation supervised by Steve Anderson.} If
extensive exposure occurs, perceptual illusions of phantom vowels will
weaken, lowering the probability of \isi{epenthesis} as regular \isi{sound change}.
Let us now evaluate this proposal with respect to the two cases of
diachronic\is{diachrony} cluster-splitting \isi{epenthesis} documented above.

Modern Persian \isi{phonology} has had significant influence from \ili{Arabic} and
Turkic. \ili{Arabic} loans constitute about half of the lexicon, and some
estimate that of the most frequent vocabulary, at least 25\% is \ili{Arabic}
\citep{perry2004a, perry2005a}. Turkic loans also exist and there is a long history
of Persian-Turkic bilingualism as well as Turkic ``Persianization". Could
\isi{acquisition} of Persian by \ili{Arabic} or Turkic speakers be the source of
Modern Persian cluster-splitting \isi{epenthesis}? I believe the answer is
yes. More specifically, I suggest that the Persianization of Turkic
people, such as the one occurring during the Ghaznavid dynasty
(977--1186), and extending over large parts of Iran, was a critical
factor in the evolution of cluster-splitting \isi{epenthesis} in Modern
Persian. Turkic languages have phonotactics that appears to be most
important in triggering cluster-splitting-\isi{epenthesis}: they disallow
complex\is{complexity} onsets in word-\isi{initial position} (and elsewhere). Under this
scenario, Middle Persian underwent rapid phonological\is{phonology} change, as it was
acquired by native speakers of Turkic languages across Iran. How early
the process began is unknown, though it could have begun as early as the
10th century when Turkic speakers came to the area, or in the 11th and
12th centuries, when a large migration of Oghuz Turks resulted in the
gradual ``Turkification" of Azerbaijan and Anatolia \citep{frye2004a}. \citet{key2012a}, who focuses on \isi{morphosyntactic} effects of contact, suggests that
Turkic influence may date from the Safavid state (1501--1736) ``the rulers
of which were Persianized Turks who spoke a variety of Middle
Azerbaijanian that might actually have been a mixed language
incorporating Ottoman elements \citep[228]{stein2005a}."\footnote{\citegen{key2012a}
  study of \isi{differential object marking} in Turkic and Persian identifies
  Iranian Azerbaijan as an isogloss for this feature.} \citet{frye2004a}
presents a distinct view of the Safavids as Turkicized Iranians, but
most seem to agree that it was the post-Islamic migration of Turks, as
opposed to \ili{Arabic} speakers, that had the most linguistic influence in
the area: `` ...the Turks who came, especially beginning from the tenth
century, moved in sufficient numbers to change the linguistic map of the
whole area. (op cit.)"

Though Classical \ili{Arabic} also disallows \isi{onset} clusters, there are several
reasons to doubt \ili{Arabic} as the source of cluster-splitting \isi{epenthesis} in
Modern Persian. First, evidence from early loans into Classical \ili{Arabic}
shows common prothetic vowels, with \isi{epenthesis} the exception (cf. \ili{Arabic}
\emph{ʔiklīl} `crown , wreath' from Syriac \emph{klīlo}, \ili{Arabic}
\emph{ʔiqlīm} `region' from Greek \emph{klīma}; but also \ili{Arabic}
\emph{dirham} `money' from Greek \emph{drakhmi}; \citealt{bueasa2015a}). Second,
the influence of \ili{Arabic} on Middle Iranian languages came, primarily,
through translation of religious texts into \ili{Arabic}, and through
\isi{acquisition} of \ili{Arabic} by writers and thinkers who used it as a prestige
language. This socialization process was notably different from the
Persianization of Turkic people referred to above, and resulted in
significant loans, but no obvious evidence of \ili{Arabic} influence on
Persian grammar.

I hypothesize that cluster-splitting \isi{epenthesis} in the history of
Persian arose as a result of contact between speakers of Turkic
languages, which did not allow complex\is{complexity} onsets, and speakers of Middle
Iranian languages with initial \#TR-clusters. As Turks became
Persianized, they acquired Persian (and, perhaps, other Middle Iranian
languages). In this process, cognitive effects of CV(C) \isi{syllable}
structure resulted in the perception of illusory vowels in \#TR-initial
words (cf. \citealt{dupoux1999a}), giving rise to the change in
pronunciation schematized in \REF{ex:blevins:4}. Under this account, the rarity of
sound changes like \REF{ex:blevins:4} is attributed to three factors: first, initial
\#TR clusters are relatively stable over time, so \REF{ex:blevins:4} is unexpected as a
language-internal development; second, a \isi{sound change} of this kind
requires contact between two distinct language types, one language which
lacks complex\is{complexity} onsets and another which has word-initial \#TR; a third
factor is the nature of the language contact involved, which must
include social factors that demand rapid and full \isi{acquisition} of the
language with \#TR clusters despite minimal previous exposure.\footnote{This
  process is distinct from creolization, since the starting point here
  is not a pidgin. Interestingly, many Creoles show initial complex\is{complexity}
  onsets \citep{klein2013a}, consistent with the view here, that they are
  relatively stable, and not particularly ``marked''.} Only when these
last two conditions are met will cluster-splitting \isi{epenthesis} occur as a
regular \isi{sound change}.\footnote{An anonymous reviewer notes that if
  future generations have access to the donor language, and that
  language is prestigious, one may see a shift involving adoption of the
  donor phonotactics.}

Can the same hypothesis involving language contact of a very specific
type account for the evolution of Dorsey's Law in Hoocąk (Winnebago)? I
believe so. Oral histories suggest that the split between Hoocąk,
traditionally spoken between Green Bay and Lake Winnebago in present-day
Wisconsin, and Chiwere, once spoken south and west of Hoocąk territory,
occurred sometime in the mid-16th century, a time-line consistent with
the great similarity between the two languages.\footnote{Though the
  homeland of the Siouan-Catawba language family is widely debated, oral
  histories and archeological remains are consistent with (pre)-Hoocąk
  occupation of land between Green Bay and Lake Winnebago (in
  present-day northeast Wisconsin) in pre-contact times.} This would
make the mid-16th century the earliest time at which Hoocąk could have
developed cluster-splitting \isi{epenthesis}, an innovation not found in
Chiwere \REF{ex:blevins:6}. By the time Jean Nicolet made contact with the ``Ho-Chunk"
in 1634, with an estimated population of 8,000 or more, their culture
was very similar to that of surrounding Algonquian tribes, they were
completely encircled by speakers of Algonquian languages, and the
language had a significant number of borrowings from Central Algonquian
languages (\citealt{radin1990a}; \citealt[17]{pfister2009a}).\footnote{By the late 1650s, the
  Hoocąk population may have been as few as 500 people, with great
  cultural devastation. This drastic decrease in population is
  attributed to a storm-related accident, epidemics (due to European
  contact), and/or battle losses to neighboring tribes \citep{edmunds1978a, radin1990a}.} I suggest that sometime between the mid-16th and
mid-17th centuries, (pre-)Hoocąk was acquired by speakers of neighboring
Algonquian languages. Since none of the Central Algonquian languages had
initial \#TR clusters, cognitive effects of \#CV(C) \isi{syllable} structure
resulted in the perception of illusory vowels in \#TR-initial words (cf.
\citealt{dupoux1999a}), giving rise to Dorsey's Law. As with the contact
scenario sketched for Modern Persian above, the evolution of
cluster-splitting \isi{epenthesis} is associated not only with these
structural-cognitive factors, but also with a specific type of language
contact: external social factors demanding rapid and full \isi{acquisition} of
a language, (pre-)Hoocąk, with initial \#TR clusters by speakers of a
language Central Algonquian language with only simple \#C-onsets
word-initially.

\section{Concluding remarks}

The \isi{typology} of \isi{sound change} may seem like an odd place to uncover
significant evidence of cognitive forces that are independent of
universal phonetics, or evidence against widely assumed \isi{markedness}
constraints.\is{constraint}\footnote{Though other instances of non-phonetic cognitive
  forces have been suggested. See, for example, \citet{blevins2009b}, where lexical competition is argued to play an active role in
  shaping the \isi{typology} of \isi{sound change}.} Yet, this study of
cluster-splitting \isi{epenthesis} as regular \isi{sound change} suggests that
typological studies of this kind may illuminate our understanding of the
role of human cognition in shaping sound patterns, and the extent to
which general aspects of memory, category formation, similarity metrics,
and \isi{analogy}\is{analogical} contribute to their evolution \citep{blevinsblevins2009a}.
Contrary to widely assumed \isi{markedness} constraints\is{constraint} treating all complex\is{complexity}
onsets as marked or dispreferred, the \isi{typology} of \isi{sound change} suggests
that word-initial \#TR clusters are phonotactically stable. On the other
hand, in the rare cases where these clusters undergo regular
cluster-splitting \isi{epenthesis}, this \isi{epenthesis} is not a simple case of
"syllable-repair”.\is{syllable} Rather, native-language \#CV-structure in
language-contact situations results in the perception of phantom vowels
which take on phonological\is{phonology} status when speakers of \#CV-initial
languages must quickly, and with little earlier familiarity, acquire a
language with \#TR clusters. This, I suggest, was the original situation
of Turkic speakers acquiring Persian, and of Central Algonquians
acquiring Hoocąk. Unlike many other common sound patterns, regular
cluster-splitting \isi{epenthesis} does not have a simple phonetic
explanation, and is not known as a purely language-internal development.
By examining other sound changes with this profile, we may,
unexpectedly, learn even more about the human mind.

%\section*{Abbreviations}
%\section*{Acknowledgements}

{\sloppy
\printbibliography[heading=subbibliography,notkeyword=this]
}


\end{document}

