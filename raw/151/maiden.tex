\documentclass[output=paper,
modfonts
]{LSP/langsci}


%\input{localpackages.tex}
\ChapterDOI{10.5281/zenodo.495446}
\title{Romansh allomorphy (Again!)}
\author{Martin Maiden\affiliation{Oxford University}}

\abstract{This essay resumes a debate which has continued for some years between
me and Stephen Anderson regarding the correct analysis of a complex set
of data from the verb morphology of the Romansh dialect of Savognin.
Anderson believes that the data are an example of ``phonologically
conditioned allomorphy'', whilst I maintain that they exemplify
``morphomic'', or autonomously morphological, alternation patterns, whose
only phonological motivation lies in diacrhrony. I reply below to
Anderson's most recent analysis of the data, by discussing reasons in
support of my ``morphomic'' account. I conclude, however, by considering
the possibility that our two accounts may be too exclusivist in their
respective ``phonologizing'' and ``morphologizing'' stances, and that they
are not necessarily wholly incompatible.
}

\begin{document}
\maketitle

\section{Introduction}\label{introductionM}

Some readers might regard this essay as an example of chutzpah, or
downright impertinence, but it is a sincere mark of respect for Steve
Anderson that I feel able to disagree with him even in a collection
published in his honour. One would not do this with a less
intellectually generous scholar. What follows is, in fact, a further
instalment in an amicable difference of opinion I have had with him for
some years (see \citealt{anderson2008a}; \citealt{anderson2011a}; \citealt{maiden2011a}; \citealt{anderson2013stem}) concerning the analysis of a set of data from some Romansh\il{Romansh}
dialects, and principally, the Surmiran\il{Surmiran} variety spoken in Savognin.
Readers, and not least the honorand himself, may be feeling that there
is little left to say. That there is reason to continue the debate is
suggested, for example, by Andrea Sims' deft review of the issues \citep[202--206]{Sims15:book}, to which I return in my conclusion.

Anderson's analysis displays not only his characteristically penetrating
theoretical rigour, but also a quite formidable grasp of the data.
Reasons of space, and the fact that the data are lucidly laid out by him
in previous publications (e.g., \citealt{anderson2008a}, \citeyear{anderson2011a}) permit me to do no
more here than summarize them: Savognin has a recurrent pattern of
\isi{vocalic alternations}]\is{alternation} in the verb \isi{root} (or ``stem''),\is{stem} such that one
alternant occurs in the \isi{root} of the singular and third person forms of
the present indicative imperative, throughout the present subjunctive,
and also in third conjugation infinitives, while another occurs in the
\isi{root} in the remainder of the paradigm.\is{paradigm} What is involved originates as
\isi{alternation} in vowel quality, phonologically conditioned by \isi{stress}. The
distinctions between stressed\is{stress} and unstressed positions may additionally
manifest as differences in the number of syllables\is{syllable} in the root,\is{root} and in
sundry consonantal alternations,\is{alternation} including \isi{metathesis}. In fact, what
\ili{Savognin} (and Romansh generally) exhibits is an unusually florid
manifestation (in respect of the range of different \isi{alternation} types
involved) of a pattern of \isi{alternation} recurrently attested across
Romance languages, and I have argued extensively (e.g., \citealt{maiden2005,maiden2011c}) that it arose historically because of stress-related\is{stress} vowel
differentiation, but then became ``autonomously morphological'' or
``morphomic''\is{morphome} in nature, being no longer determined by stress,\is{stress} but simply
by the heterogeneous set of \isi{paradigm} cells, one property of which is
that they are ``rhizotonic'' (i.e., stressed on the lexical root).\is{root} This
set I label (for reasons that are here unimportant) the ``N-pattern''.
Important diachronic\is{diachrony} proof that the N-pattern is independent of
phonological causation is the rise in various \ili{Romance languages} of
\isi{alternation} patterns whose phonological\is{phonology} content cannot possibly ever
have been determined by \isi{stress} (including lexical \isi{suppletion}), but which
replicates the pattern originally ``etched out'' by the effects of \isi{stress}.
The distribution of \isi{alternation} found in Savognin constitutes a
widespread variant of this ``N-pattern''. \footnote{Savognin is among a
  number of Romance dialects where the N-pattern alternant also appears
  in the first and second persons \isi{plural} present subjunctive. The
  reasons are complex and not immediately relevant here \citep[cf.][]{maiden2012}.} There is no space to recapitulate all the data and arguments,
but crucially Anderson (e.g., \citeyear[25]{anderson2010a};\citeyear[34f]{anderson2011a};\citeyear[10,16,23]{anderson2013stem};\citeyear[173]{anderson2016r}) accepts the ``morphomic'' nature of the ``N-pattern'' and its
importance in determining morphological\is{morphology} change in the Romance verb
generally. He believes, however, that modern Savognin (and other Romansh\il{Romansh}
varieties: cf. \citealt{anderson2013stem}) is, in effect, a ``special case'', indeed a
prime example of `\isi{phonologically conditioned allomorphy}'.\is{phonology} He
convincingly shows that the \isi{alternation} types involved have often become
so disparate, and refractory to unique underlying representation, that
they must be represented directly in the grammar, but he also argues
that the alternant-pairs characteristically contain one member more
suited to appear under stress, and another more suited to appear in
unstressed position \citep[18]{anderson2011a}, and that the alternants are
accordingly conditioned by the presence vs absence of \isi{stress}. He also
shows that the position of \isi{stress} in Savognin is systematically
predictable from the segmental content of the word-form. It follows that
the alternations\is{alternation} are fully predictable from the position of stress,\is{stress} and
that appeal to the ``N-pattern'' is inappropriate. My response, \emph{in
nuce}, has been that there exist nonetheless within Savognin
morphological phenomena that really are irreducibly ``morphomic'' and
follow the N-pattern. Given this, I say that Anderson misses an
important generalization by divorcing the \isi{vocalic alternations}\is{alternation} from
clear-cut cases of the N-pattern. Anderson's response is, in effect,
that my alleged N-pattern examples are secondary effects of the
principle of stress-related\is{stress} \isi{allomorph} \isi{selection}, and that invocation of
the morphome risks missing another significant generalization, namely
that the alleged stress-related\is{stress} alternations\is{alternation} found in the verb are found
across the grammar, outside the verb.

I need to comment briefly on a methodological assumption. Steve Anderson
objected to me orally some years ago (cf. also \citealt[13f.;17]{anderson2011a})
that I could not draw inferences about \ili{Savognin} from apparently parallel
developments in other \ili{Romance languages} for which my ``morphomic''
analysis seemed correct, observing quite reasonably that the inhabitants
of Savognin were ``not Romance linguists'' (and therefore could not know
about what happens in other Romance languages). My very delayed response
is, in effect: ``Yes, but they are still Romance speakers''. Let us
suppose that in some other Romance variety, distant both historically
and geographically (a real case is Romanian), virtually identical
patterns of \isi{alternation} are found, except that this time they are
clearly morphomic; let us further assume that the analysis appropriate
to, say, Romanian is perfectly \emph{possible} for Romansh, even though
rival possible accounts exist. Obviously native speakers of Romansh are
not native speakers of \ili{Romanian}, nor are they Romance linguists enjoying
Olympian vistas over comparative Romance morphology: nothing we could
say about Romanian could ever be definitively probative for Romansh.
What does not follow, however, is that the comparative evidence can be
ignored. Speakers of both languages obviously have the same mental
endowment, and both languages have inherited much that is structurally
common, particularly with regard to the organization of the inflexional
morphology of the verb. What this means is that an analysis which is
\emph{justified} for Romanian deserves consideration as \emph{plausible}
for Romansh. The Romance languages ought not to be treated as
hermetically isolated entities: rather, the analysis of one variety
should always be allowed to inform that of another. That, in fact, is
one of the reasons for doing Romance \isi{linguistics} from a comparative
perspective (in fact, there is no other way of doing it), and in the
following pages the analysis will frequently be guided, with all due
caution, by comparisons and inferences across cognate but separate
varieties. I now examine the facts which seem to me to require
acknowledgement of the morphomic N-pattern in Savognin.

\section{Suppletion: {dueir} and {deir}}

The verb \emph{dueir} `must, be obliged to (cf. German \emph{sollen})'
(from \ili{Latin} \textsc{debere}) plays a central role in the debate, because
it has a \isi{suppletive} root \isi{allomorph} in precisely that set of cells which,
in other verbs, displays the ``stressed'' alternant (\tabref{tab:maiden:1}):


%\begin{table}
\begin{longtable}[]{@{}lllllll@{}}
\caption{\label{tab:maiden:1}\emph{Dueir} in \ili{Savognin}}\\
\lsptoprule
\textsc{inf} & \emph{duéir} & & & & &\tabularnewline
\midrule
\endhead
\textsc{pst.part} & \emph{duía} & & & & &\tabularnewline
\textsc{ger} & \emph{duónd} & & & & &\tabularnewline
& \textsc{1sg} & \textsc{2sg} & \textsc{3sg} & \textsc{1pl} &
\textsc{2pl} & \textsc{3pl}\tabularnewline
\textsc{prs.ind} & \emph{stó} & \emph{stóst} & \emph{stó} & \emph{duágn}
& \emph{duéz} & \emph{stón}\tabularnewline
\textsc{prs.sbjv} & \emph{stóptga} & \emph{stóptgas} & \emph{stóptga} &
\emph{stóptgan} & \emph{stóptgas} & \emph{stóptgan}\tabularnewline
\textsc{ipf.ind} & \emph{duéva} & \emph{duévas} & \emph{duéva} &
\emph{duévan} & \emph{duévas} & \emph{duévan}\tabularnewline
\textsc{cond} & \emph{duéss} & \emph{duéssas} & \emph{duéssa} &
\emph{duéssan} & \emph{duéssas} & \emph{duéssan}\tabularnewline
\lspbottomrule
\end{longtable}
%\end{table}

The \isi{suppletive} forms are taken from another verb, \emph{stueir} `must,
be necessarily the case (cf. \ili{German} \emph{müssen})', which unlike
\emph{dueir} continues to have its own complete \isi{paradigm} (Table 2):


%\begin{table}
\begin{longtable}[]{@{}lllllll@{}}
\caption{\emph{Stueir} in Savognin}\\
\lsptoprule
\textsc{inf} & \emph{stuéir} & & & & &\tabularnewline
\midrule
\endhead
\textsc{pst.part} & \emph{stuía} & & & & &\tabularnewline
\textsc{ger} & \emph{stuónd} & & & & &\tabularnewline
& \textsc{1sg} & \textsc{2sg} & \textsc{3sg} & \textsc{1pl} &
\textsc{2pl} & \textsc{3pl}\tabularnewline
\textsc{prs. ind} & \emph{stó} & \emph{stóst} & \emph{stó} &
\emph{stuágn} & \emph{stuéz} & \emph{stón}\tabularnewline
\textsc{prs. sbjv} & \emph{stóptga} & \emph{stóptgas} & \emph{stóptga} &
\emph{stóptgan} & \emph{stóptgas} & \emph{stóptgan}\tabularnewline
\textsc{ipf} & \emph{stuéva} & \emph{stuévas} & \emph{stuéva} &
\emph{stuévan} & \emph{stuévas} & \emph{stuévan}\tabularnewline
\textsc{cond} & \emph{stuéss} & \emph{stuéssas} & \emph{stuéssa} &
\emph{stuéssan} & \emph{stuéssas} & \emph{stuéssan}\tabularnewline
\lspbottomrule
\end{longtable}
%\end{table}

For \citet{anderson2008a,anderson2010a} \emph{dueir} is a defective verb and its
pattern of \isi{alternation} is a matter of phonologically\is{phonology} conditioned
\isi{allomorphy}: he believes that the explanation of the \isi{suppletion} is that
\emph{dueir} lacks a stressed\is{stress} stem-alternant, having only unstressed
/dʊ/. Since /dʊ/ contains a vowel whose phonological characteristics
debar it from occurring under stress,\is{stress} speakers in effect plug the
resultant phonological gap by borrowing appropriate stressed forms from
a near synonym of \emph{dueir}, namely \emph{stueir}. My view, from
comparative and diachronic evidence, is that it is highly unlikely that
\emph{dueir} could ever have been, in any relevant sense, ``defective'',\is{defectiveness}
and that even if it were, the filling of the alleged gap could have
nothing to do with phonology.\is{phonology} Indeed, any explanation in terms of
phonological conditioning crucially fails to account for the fine
details of the \isi{allomorphy}. If I am correct, and what we observe is a
pattern of \isi{allomorphy} identical in distribution to the vocalic
alternations,\is{alternation} yet independent of phonology,\is{phonology} then in principle whatever
explains the paradigmatic\is{paradigm} distribution of forms of \emph{dueir} should
also be available to explain the \isi{vocalic alternations}.\is{alternation} Indeed,
considerations of economy would lead us to prefer that single
explanation. This is a view that I have expounded before (e.g., \citealt[46--49]{maiden2011a}), while Anderson, in his latest discussion \citeyear[8]{anderson2013stem} states
that there are no new facts, and that we simply disagree. I think that
the facts remain very important, and I (re-)present them below in a
slightly revised form.

\largerpage
The `\isi{defectiveness}'\is{defectiveness} of \emph{dueir} is the effect, not the cause, of the
\isi{suppletion}. All \isi{suppletive}\is{suppletion} verbs whose morphology reflects the incursion
of forms of the \isi{paradigm} of one lexeme on the \isi{paradigm} of another are,
in one sense, ``defective''. If, for example, \citet[89f.n5]{grisch1939a},
\emph{DRG} s.v. \emph{dovair}, p.378, \citet[155;158]{decurtins1958a}, or
\citet[165f]{signorell1987a}, describe Romansh reflexes of
\textsc{debere} as ``defective'', this simply means that there are parts
of the \isi{paradigm} occupied by forms which are patently unconnected
(diachronically or synchronically) with \emph{dueir}, and which
obviously are connected with \emph{stueir}. This does not mean that the
\isi{paradigm} of the lexeme meaning `must' somehow has ``holes'' in it.

One might object that the independent existence of \emph{stueir} as a
verb with its own full \isi{paradigm} (and indeed with its own distinctive
meaning, as I explain shortly) is grounds to view those forms of it
which appear inside \emph{dueir} as synchronic interlopers\is{synchrony} drafted in to
occupy otherwise ``empty territory''. Such reasoning would force us into
the highly counterintuitive position of claiming, for example, that
Savognin \emph{esser} `to be' is ``defective'' in respect of its past
participle, because the latter has the form \emph{sto} which is also
(and transparently) the past participle of a different verb, \emph{star}
`to stand'. This is actually a case where there was,
\emph{historically}, \isi{defectiveness}\is{defectiveness}: the \ili{Latin} ancestor of \emph{esser}
had no past participle, for semantic reasons. It is only with the rise
in early Romance of verbal periphrases\is{periphrasis} comprising auxiliary + past
participle, that the verb `be' needs to fill the past participle ``slot'',
and it does so (in some regions) by supplying the missing form from the
past participle of \textsc{stare} `stand'. But the idea that, in modern
Romansh, the verb `be' lacks a past participle and ``borrows'' it from
\emph{star} seems peculiar, given that the verb `be' itself, and the use
of forms of it involving its past participle, are utterly basic. Indeed,
to the best of my knowledge no grammarian of the Romance languages has
ever described the wholesale \isi{suppletion} of `be' in the Romance languages
as involving ``\isi{defectiveness}''. If one can analyse Savognin \emph{esser}
as \isi{suppletive} but not defective, then surely the same analysis should be
available for \emph{dueir}.

My principal difficulty with Anderson's analysis of \emph{dueir} is that
I see absolutely no motivation to view this verb as defective, beyond
the morphological facts which are the explanandum. All its cells are
well and truly \emph{filled} --- and it is almost inconceivable that a
subset of present-tense cells of a verb expressing such a basic meaning
could ever be empty. Here, again, a comparative perspective is useful.
Virtually all Romance languages conserve reflexes of \textsc{debere},
with a full inflexional\is{inflection} paradigm,\is{paradigm} and no Romance languages show any sign
of \isi{defectiveness} in the verb meaning `must', whatever its origin: there
is no reason for parts of its \isi{paradigm} to be missing. Many Romansh
dialects indeed have a full \isi{paradigm} all of whose forms still continue
\textsc{debere} (see \citealt[152f.]{decurtins1958a}; \emph{DRG} s.v.
\emph{dovair}), and the rhizotonic forms (usually \emph{dé}-) are
robustly attested from the earliest records, including in central
dialects (of which Savognin is one); cf. also \citet[30;32]{anderson2010a}.
There is simply nothing in the \isi{phonology} of these dialects, either
synchronically or diachronically, which could have determined
\emph{deletion} of such stressed\is{stress} forms, and the \isi{defectiveness} certainly
cannot be explained as a phonological effect of \isi{stress}.\footnote{Could a
  stress-based\is{stress} account be salvaged if, unlike Anderson, one said that
  \emph{any} kind of alternation, including an \isi{alternation} where one of
  the alternants was zero, could be effected by stress?\is{stress} Given that the
  position of \isi{stress} in Savognin is predictable on grounds of segmental
  phonological\is{phonology} content, one can hardly invoke the case of zero
  alternants which would, by definition, lack any segmental content. The
  best one could say is that zero forms appear in those parts of the
  \isi{paradigm} where \isi{stress} \emph{would normally be expected} to appear. But
  then one would have to ask: ``Where would \isi{stress} normally be expected
  to appear?'', and the answer would be purely morphological: ``the
  N-pattern''.}

\citet[32]{anderson2010a} suggests that ``the primary factor in the emergence of
\isi{defectiveness} in Surmiran \emph{dueir}, as well as the complementary
pattern in the Engadine languages, was the morphologization\is{morphology} of the vowel
alternations\is{alternation} in Swiss \ili{Rumantsch}. If we hypothesize that this was
combined with reduced use of the verb due to competition with others
such as \emph{stueir}, it could well have led to the present situation with
only one \isi{stem} conserved''. I discuss later the nature of the
``competition'' from \emph{stueir}, which involves overlap and
replacement, not \isi{defectiveness}. As for the alternation,\is{alternation} regular sound
change would indeed have given rise to a unique \isi{alternation} between a
stressed\is{stress} alternant /de/, and unstressed alternant /du/ (in the latter
the back vowel is the result of an adjustment of the unstressed front
vowel triggered in the environment of a following labial consonant: cf.
\ili{Italian} \textsc{3sg.prs.ind} \emph{déve} vs \textsc{inf} \emph{dovére}).
But the notion that Romansh would eliminate an \isi{alternation} type because
it was ``morphologized''\is{morphology} (or, perhaps better, idiosyncratic and
unpredictable), especially in such a high-frequency verb, seems
unlikely, particularly given that Romansh is notable for retaining
extreme and idiosyncratic patterns of vocalic alternation,\is{alternation} even in
isolated verbs which surely have a much lower frequency of use. \citet{rogers1972a}, in an analysis of Surselvan, lists no fewer than eleven sets of
vocalic \isi{alternation} each apparently limited to just one verb, with
meanings such as `vomit', `scythe', `drivel' - all without sign of
resort to \isi{defectiveness}; see also my discussion of Savognin \emph{deir}
`say', below. One might add that the most natural response to any
idiosyncratic type of \isi{alternation} would surely be not to create a ``gap'',
by jettisoning one alternant, but to attempt some kind of ``repair'' by
analogically\is{analogical} remodelling the alternants to be less different.\footnote{In
  any case, Savognin stems\is{stem} do sometimes have ``inappropriate'' forms. Thus
  \citet[32]{anderson2011a} discusses verbs such \emph{baitár} `babble' which
  has a stem suitable for \isi{stress} only, but which is is nonetheless used
  throughout the paradigm in unstressed environments as well.} The last
thing one expects is a reaction resulting in an \isi{alternation} which, by
virtue of being \isi{suppletive}, is even stranger than the rejected original.

Viewed in a comparative-historical perspective, the notion that the
Savognin reflexes of \textsc{debere} could be in any significant sense
``defective'' seems most unlikely. And even if it were defective, the
\isi{suppletive} filling of the alleged gaps would take place because the
\emph{gaps} needed filling, not specifically because of the lack of a
``stressed''\is{stress} alternant. What that perspective does suggest, however, is a
different scenario (see also \citealt[46--49]{maiden2011a}), which involves not
``gaps'' but ``overabundance'' (cf. \citealt{thornton2011}), in which more than one
realization became available for certain cells of the \isi{paradigm} of
\emph{dueir}, and in which one of the alternative realizations
ultimately prevails. This situation arises from particular
discourse-related\is{discourse} circumstances. The reflexes of \textsc{debere} are
subject in Romansh and beyond to intensive competition from other nearly
synonymous alternatives.\footnote{See further \citet[49]{sturzinger1879};
  \citet[84]{tagliavini1926a}; \citet[64]{kramer1976a}; \citet{maiden2011b}} I have no
space to detail the facts or the mechanisms (see \citealt{maiden2011a}),\footnote{In addition to the sources cited by Maiden (2011a), see
  also \citet[166f.]{pult1879a} for the \isi{suppletive} introduction of forms of the
  verb `want' in the dialect of Sent.} but essentially what appears to
be involved is ``face-saving'': speakers avoid the charge of moral
obligation inherent especially in present indicative forms of
\emph{dueir} by resorting to alternatives such as expressions equivalent
to `ought' (e.g., conditional forms of the verb), or expressions meaning
``absolute'' (rather then ``moral'') necessity, which is exactly expressed
by \emph{stueir}. This tendency created, I suggested, a situation in
which the frequent use of \emph{stueir} alongside \emph{dueir} in the
present tense led to effective synonymy between the two forms,
eventually resolved by replacing \emph{dueir} with \emph{stueir}
according to the familiar pattern of \isi{alternation} (the ``N-pattern'')
associated with vocalic \isi{allomorphy} (a type of reaction to synonymy
attested elsewhere in Romance: cf.\citealt{maiden2004a, maiden2006a}). The same
paradigmatic\is{paradigm} distribution, reflexes of \textsc{debere} alternating this
time with those of \textsc{conuenire} (originally meaning `be fitting'),
emerges from \emph{ALDII} (maps 829--833; 836--838), for Peio (point 54)
and S. Michele all'Adige (point 66). The disappearance of reflexes of
\textsc{debere} from certain cells of the present indicative and the
present subjunctive of \emph{dueir} never involved ``\isi{defectiveness}'', and
has never had anything to do with phonology. The perception of
``\isi{defectiveness}'' is a synchronic \emph{effect} of the \isi{suppletion}.

Crucially, the \isi{suppletive} fusion of \emph{dueir} and \emph{stueir} in
Savognin is of a significantly different kind from the alleged
\isi{phonologically conditioned allomorphy} of the \isi{vocalic alternations}.\is{alternation} The
latter is a \emph{binary} correspondence between alternants and stress:\is{stress}
one alternant is selected under stress, the other elsewhere. The
putative relation between the \emph{dueir - stueir} \isi{alternation} and
\isi{stress} can only be described as binary at a level which is in fact
lacking any phonological content. For what, in the case of the
\isi{suppletion}, is allegedly selected by \isi{stress} is not \emph{a form}
correlated with stress,\is{stress} but a whole array of phonologically and
morphologically different forms. As Anderson himself points out
\citeyear[124]{anderson2008a},\footnote{\citet[29]{anderson2010a} even calls this verb
  ``\isi{suppletive}''.} ``it is not just a single stem,\is{stem} but the full range of
irregular forms of \emph{stueir} (\emph{ia stò, te stost, el stò, els
ston}; Subjunctive \emph{ia stoptga}, etc.) that replaces those of
\emph{dueir} where \isi{stress} would fall on the stem''.\footnote{As mentioned
  earlier, if we ask ``where \isi{stress} would fall on the stem'', the answer
  is ineluctably morphological: in the cells of the singular and third
  person present and imperative and in the present subjunctive. It seems
  to me useless to say, instead, ``wherever the endings would be
  unstressed'' because the third person singular has no ending, and the
  distribution of the remaining, unstressed, endings turns out to be the
  morphomic N-pattern.} More exactly: ``the first and second person
singular, and third person, forms of the indicative of \emph{stueir} are
mapped onto the first and second person singular, and third person,
forms of the indicative of \emph{dueir}, and the present subjunctive
cells of \emph{stueir} are mapped onto the corresponding cells of
\emph{dueir}''. Only this way do we get the observed distribution. In
effect, it is not ``a stem'', but an entire, morphomically defined,
``slab'', of the \isi{paradigm} of \emph{stueir}, a set of full word-forms
replete with their own internal allomorphic idiosyncrasies, that has
been mapped onto \emph{dueir}. A rule of phonologically\is{phonology} conditioned
\isi{allomorphy} involving \isi{stress} could in principle select \emph{a} stressed
\isi{root} \isi{allomorph} of \emph{stueir} and introduce it into \emph{dueir}, but
it could not necessarily insert the \emph{right} \isi{root} \isi{allomorph} in the
relevant cell. A rule that identifies a morphologically-defined portion
of the \isi{paradigm} as that in which the replacement of one lexeme by the
other can do just that.

\emph{Dueir} exemplifies lexical \isi{suppletion} (``incursion'', in the
terminology of \citealt{corbett2007a}), where one historically distinct lexeme
obtrudes on the inflexional\is{inflection} \isi{paradigm} of another lexeme. Another
diachronic\is{diachrony} route to \isi{suppletion} is regular \isi{sound change} so extreme in its
effects that the synchronic result is \isi{allomorphy} such that the
alternants bear no phonological relation to each other. Savognin has at
least one case of phonologically\is{phonology} induced \isi{suppletion}, namely \emph{deir}
`say', which inflects as in Table 3:


%\begin{table}
\begin{longtable}[]{@{}lllllll@{}}
\caption{ \emph{Deir} in Savognin}\\
\lsptoprule
\textsc{pst.part} & \emph{détg} & & & & &\tabularnewline
\midrule
\endhead
\textsc{ger} & \emph{schónd} & & & & &\tabularnewline
& \textsc{1sg} & \textsc{2sg} & \textsc{3sg} & \textsc{1pl} &
\textsc{2pl} & \textsc{3pl}\tabularnewline
\textsc{prs. ind} & \emph{déi} & \emph{déist} & \emph{déi} &
\emph{schágn} & \emph{schéz} & \emph{déian}\tabularnewline
\textsc{prs. sbjv} & \emph{schéia} & \emph{schéias} & \emph{schéia} &
\emph{schéian} & \emph{schéias} & \emph{schéian}\tabularnewline
\textsc{ipf} & \emph{schéva} & \emph{schévas} & \emph{schéva} &
\emph{schévan} & \emph{schévas} & \emph{schévan}\tabularnewline
\textsc{cond} & \emph{schéss} & \emph{schéssas} & \emph{schéssa} &
\emph{schéssan} & \emph{schéssas} & \emph{schéssan}\tabularnewline
\lspbottomrule
\end{longtable}
%\end{table}

I cannot here retrace the phonological history of this verb in detail.
Suffice to say that the historically underlying root was *dik-, and that
sound changes involving, in particular, \isi{deletion} of unstressed\is{stress} vowels
and \isi{assimilation} of resulting consonant clusters, created the modern
situation. The rather unusual present subjunctive of this verb happens
to show the effects of \isi{analogical} levelling in favour of the originally
arrhizotonic first and second person \isi{plural} form stems\is{stem} togther with the
associated stressed\is{stress} inflexional ending (cf. \citealt{decurtins1958a} for the
reflexes of \textsc{ambulare / ire}, \textsc{dicere, uenire, habere,} or
\textsc{sapere} in Samedan, Parsons, and Razen for other Romansh
examples of this kind; further discussion also in Maiden, in progress).
That aside (and there is every reason to believe that the \emph{déi}-
root originally occurred as expected in the present subjunctive), this
verb shows N-pattern \isi{suppletion}. \citet[17]{anderson2011a} acknowledges that it
is ``genuinely \isi{suppletive}'' and that ``the choice of \isi{stem} is determined by
\isi{morphosyntactic} features''.\is{features} He defines in the same way some other, less
radically \isi{suppletive} verbs, for which I give here just the present-tense
forms, e.g, (\emph{vu})\emph{léir} `want', \emph{néir} `come' (Table 4):

%\begin{table}

\begin{longtable}[]{@{}lllllll@{}}
\caption{(\emph{Vu})\emph{leir} and \emph{neir} in Savognin}\\
\lsptoprule
\textsc{prs. ind} & \emph{ví} & \emph{vót} & \emph{vót} & \emph{léin} &
\emph{léz} & \emph{vóttan}\tabularnewline
\midrule
\endhead
\textsc{prs. sbjv} & \emph{víglia} & \emph{víglias} & \emph{víglia} &
\emph{víglian} & \emph{víglias} & \emph{víglian}\tabularnewline
\lspbottomrule
\end{longtable}\addtocounter{table}{-1}

\begin{longtable}[]{@{}lllllll@{}}
\lsptoprule
\textsc{prs. ind} & \emph{vígn} & \emph{vígnst} & \emph{vígna} &
\emph{nín} & \emph{níz} & \emph{vígnan}\tabularnewline
\midrule
\endhead
\textsc{prs. sbjv} & \emph{vígna} & \emph{vígnas} & \emph{vígna} &
\emph{vígnan} & \emph{vígnas} & \emph{vígnan}\tabularnewline
\lspbottomrule
\end{longtable}
%\end{table}

If we acknowledge that \emph{deir} and some other verbs have
(near-)\isi{suppletive} patterns determined synchronically by \isi{morphosyntactic}
features,\is{features} then we have to admit the presence of the morphomic N-pattern
in Savognin. Yet if we say that the \isi{vocalic alternations}\is{alternation} are still a
matter of ``\isi{phonologically conditioned allomorphy}'', then the fact that
they show exactly the same paradigmatic\is{paradigm} distribution becomes
uncomfortably coincidental.

\section{The ``augment''}\label{the-augment}

The ``augment'' is a functionally empty formative which, in certain cells
of the inflexional\is{inflection} \isi{paradigm} of the verb, occurs between the lexical \isi{root}
and desinences denoting tense,\is{tense} mood, person, and number (for detailed
discussions of its nature and origins, which lie in \ili{Latin} and
proto-Romance Aktionsart suffixes,\is{suffixation} see especially \citealt{maiden2003};\citeyear[249--53]{maiden2011c};\citeyear[715f.]{maiden2016}). In \ili{Latin}, the relevant affixes were
restricted to imperfective-\isi{aspect} forms, but had no restrictions for
person, number, or tense.\is{tense} In most Romance languages, augments are
associated with particular inflexion\is{inflection} classes (in Romansh, usually the
first and fourth conjugations), and have become restricted to certain
cells of the inflexional\is{inflection} \isi{paradigm} defined by tense,\is{tense} mood, person, and
number. In Savognin, the augment occurs solely in the singular and third
person forms of the present indicative, and in all forms of the present
subjunctive. That is to say, of course, that it has exactly the same
paradigmatic\is{paradigm} distribution as the ``stressed''\is{stress} vocalic alternants, a fact
which clearly suggests a link between them. Thus first conjugation
\emph{luschardár} `strut', and fourth conjugation \emph{tradéir}
`betray' (Table 5):
 
 \begin{table}
\caption{The augment in Savognin first and fourth conjugation verbs}
 \fittable{
\begin{tabular}{@{}lllllll@{}}
\lsptoprule
& \textsc{1sg} & \textsc{2sg} & \textsc{3sg} & \textsc{1pl} &
\textsc{2pl} & \textsc{3pl}\\
\midrule 
\multicolumn{2}{l}{First conjugation}\\
\midrule
\textsc{prs. ind} & \emph{luschardésch} & \emph{luschardéschas} &
\emph{luschardéscha} & \emph{luschardágn} & \emph{luschardéz} &
\emph{luschardéschan}\\
\textsc{prs. sbjv} & \emph{luschardéscha} & \emph{luschardéschas} &
\emph{luschardéscha} & \emph{luschardéschan} & \emph{luschardéschas} &
\emph{luschardéschan}\\
\textsc{ipf.ind} & \emph{luschardéva} & \emph{luschardévas} &
\emph{luschardéva} & \emph{luschardévan} & \emph{luschardévas} &
\emph{luschardévan}\\
% \lspbottomrule
% \end{tabular} 
% }
% \fittable{
% \begin{tabular}{@{}lllllll@{}}
% % \caption*{Fourth conjugation}\\
% \lsptoprule
\tablevspace
\multicolumn{2}{l}{Fourth conjugation}\\
\midrule
% & \textsc{1sg} & \textsc{2sg} & \textsc{3sg} & \textsc{1pl} & \textsc{2pl} & \textsc{3pl}\\
% \midrule 
\textsc{prs. ind} & \emph{tradésch} & \emph{tradéschas} &
\emph{tradéscha} & \emph{tradígn} & \emph{tradíz} &
\emph{tradéschan}\\
\textsc{prs. sbjv} & \emph{tradéscha} & \emph{tradéschas} &
\emph{tradéscha} & \emph{tradéschan} & \emph{tradéschas} &
\emph{tradéschan}\\
\textsc{ipf.ind} & \emph{tradíva} & \emph{tradívas} & \emph{tradíva} &
\emph{tradívan} & \emph{tradívas} & \emph{tradívan}\\
\lspbottomrule
\end{tabular}
}
\end{table} 

\largerpage
It is undisputed that the distribution of the Romance augment cannot be
explained, diachronically or in modern varieties, as the output of any
kind of phonological process. The view that I have developed (see, e.g.,
\citealt{maiden2005,maiden2011a,maiden2011c}) is that the redistribution of the alternant
from \ili{Latin} to Romance is \emph{purely morphologically}\is{morphology} determined, and
reflects sensitivity to a paradigmatic\is{paradigm} pattern created, originally, as
an effect of \isi{vocalic alternations}\is{alternation} between stressed\is{stress} and unstressed
vowels: the same pattern can be shown to have provided a ``template'' for
the \isi{suppletive} \isi{merger} of distinct lexical verbs in various Romance
languages (notably, the verb `go'). I see no reason why what we see in
Savognin, and more generally in Romansh, should be viewed any
differently: the distribution of the augment appears a matter of pure
morphology,\is{morphology} and given that the \isi{vocalic alternations}\is{alternation} have the same
distribution as the augment, they too can be accounted for in the same,
purely morphological, terms.

Anderson views the facts, in effect, in terms of a kind of
``\isi{defectiveness}'': verbs showing the augment lack a stressed\is{stress} vocalic
alternant, and the augment is inserted wherever this occurs. Since the
augment is inherently stressed,\is{stress} the preceding root-form\is{root} must be
unstressed, and the lack of a stressed\is{stress} root \isi{allomorph} is thereby
resolved. My view is that this analysis inverts cause and effect: it is
not the case that the augment is applied because there is no stressed\is{stress}
root \isi{allomorph} but, rather, that there is no stressed root \isi{allomorph}
because the relevant cells of the \isi{paradigm} are specified as being filled
by forms containing the augment. This latter analysis has the immediate
advantage of avoiding the problem of arbitrary stipulation of
\isi{defectiveness} in one set of cells only. After all, if stressed\is{stress}
alternants can be defective, why should not unstressed alternants too?
Why do we not also find, that is, verbs with a stressed\is{stress} alternant but
not an unstressed one? And if the distribution of the augment is
dictated by the need to plug a phonological ``gap'', how is it that such
gaps only occur in first and fourth conjugation verbs, precisely the
inflexion\is{inflection} classes to which the augments are historically restricted
across the Romance?

Discussion of the Savognin augment has tended to focus on first
conjugation verbs, where it is most productive, but where it still only
constitutes a subset (and apparently a minority) of such verbs. We
should not forget that the augment also appears in the great
\emph{majority} of fourth conjugation verbs (characterized by
infinitives in -\emph{eir}), a class comprising dozens of lexemes and
endowed with some \isi{productivity}. If we follow Anderson, this means that
almost all of the fourth conjugation is characterized by lack of a
stressed\is{stress} alternant. Nothing logically excludes this, but it seems
counterintuitive to say that a major, semi-productive, inflexion\is{inflection} class
is, in effect, ``defective'' in most of its present tense. Nobody would
countenance such an analysis for the cognate Romance varieties
(Daco-Romance, Italo-Romance, Catalan) where the fourth conjugation
behaves in this way.

\citet[22]{anderson2011a} points out that the augment frequently appears in
neologisms, including where speakers feel doubt about the identity of
the stressed\is{stress} \isi{allomorph}. This does not mean, however, that the augment is
\emph{usually} a response to perceived lack of a stressed\is{stress} alternant.
Using the Savognin first conjugation verb \emph{luschardár} `strut'
(exemplified above), \citet[122]{anderson2008a} observes that: ``The use of this
pattern {[}...{]} has the advantage that the speaker does not need to
retrieve any information about the specific \isi{alternation} pattern of the
\isi{stem} in order to produce all of the correct forms. Otherwise, it would
be necessary to choose {[}...{]} among a variety of possibilities such
as *\emph{luscharda}, *\emph{luscheirda}, *\emph{luschorda},
*\emph{laschurda}, *\emph{laschorda}, etc. Each of these patterns is
more or less secure with reference to at least some verbs in the
Surmiran lexicon, but the availability of the \isi{paradigm} {[}given above{]}
makes it possible to avoid the choice when positive evidence is not
readily available.'' The problem here is that that there \emph{is}
unequivocal evidence for the stressed\is{stress} vowel. This verb is transparently
and directly derived from the nominal \emph{luschárd} `dandy, fop, vain,
proud', which actually contains, moreover, a highly frequent stressed
pejorative suffix -\emph{árd}. In this case, the identity of the ``right''
stressed alternant is patent. This is in fact true of a large number of
other verbs that take -\emph{esch}, all transparently derived from nouns
or adjectives whose stressed vowel is known (examples from \citealt{signorell2001a}), such as those give in Table 6:


%\begin{table}
\begin{longtable}[]{@{}lllll@{}}
\caption{Nouns, adjectives, and derived verbs in Savognin} \\
\lsptoprule
Basic noun/adjective & & Infinitive & \textsc{3sg}.prs.ind
&\tabularnewline
\midrule
\endhead
\emph{cisél} & & \emph{ciselár} & \emph{ciseléscha} &
`chisel'\tabularnewline
\emph{dimóra}\footnote{In fact, \emph{dimóra} and \emph{fírma} below may
  be derived from the corresponding verbs (cf. \citealt[517]{thornton2004a}, and
  \citealt{cortelazzo1988} ss.vv. \emph{dimorare, firmare} for
  this type in \ili{Italian}). If this holds for Savognin, then these verbs do
  possess a `stressed' alternant.}
 & & \emph{dimorár} &
\emph{dimoréscha} & `dwell (-ing)'\tabularnewline
\emph{discrédit} & & \emph{discreditár} & \emph{discreditéscha} &
`discredit'\tabularnewline
\emph{fáx} & & \emph{faxár} & \emph{faxéscha} & `fax'\tabularnewline
\emph{figúra} & & \emph{figurár} & \emph{figuréscha} &
`figure'\tabularnewline
\emph{fílm} & & \emph{filmár} & \emph{filméscha} & `film'\tabularnewline
\emph{fírma} & & \emph{firmár} & \emph{firméscha} & `sign
(-ature)'\tabularnewline
\emph{guíd} & & \emph{guidár} & \emph{guidéscha} &
`guide'\tabularnewline
\emph{líber} & & \emph{liberár} & \emph{liberéscha} &
`free'\tabularnewline
\emph{nivél} & & \emph{nivelár} & \emph{niveléscha} &
`level'\tabularnewline
\emph{ódi} & & \emph{odiiér} & \emph{odiéscha} & `hate'\tabularnewline
\emph{penél} & & \emph{penelár} & \emph{peneléscha} & `paint
(-brush)'\tabularnewline
\emph{schicána} & & \emph{schicanár} & \emph{schicanéscha} &
`chicane'\tabularnewline
\emph{teléfon} & & \emph{telefonár} & \emph{telefonéscha} &
`telephone'\tabularnewline
\emph{unifórm} & & \emph{uniformár} & \emph{uniforméscha} &
`uniform'\tabularnewline
\emph{vagabúnd} & & \emph{vagabundár} & \emph{vagabundéscha} &
`bum'\tabularnewline
\lspbottomrule
\end{longtable}
%\end{table}

What such derived forms lack is not a ``stressed''\is{stress} alternant but an
\emph{unstressed} one: what appears in the verb is simply the \isi{root} of
the corresponding rhizotonic nominal form. Yet there is no sign of
attempts to invent a predictable ``unstressed'' counterpart for the
stressed vowel of the base-form (e.g., \textsc{inf} **\emph{udiiér} from
the noun \emph{ódi} after the model of \textsc{3sg.prs.ind} \emph{dórma}
`sleeps' vs \textsc{inf} \emph{durméir}) and quite simply the derived
verb-forms preserve the segmental identity of the base form. Many
scholars have suggested that in Romance generally the augment serves to
obviate \isi{root} \isi{allomorphy} that might otherwise occur in the lexical root.\is{root}
There are various reasons why this view does not account for the facts
(cf. \citealt[251f.]{maiden2011c}), but note that in any case this kind of
``solution'' comports a paradox: one type of \isi{alternation} is merely
replaced by another, that between the augment and its absence, the
augment itself retaining an irreducibly ``N-pattern'', morphomic,
distribution.

\citet{anderson2013stem}broadens his survey beyond Savognin, arguing for a
similar analysis for other Romansh varieties. In fact, in dialects where
\isi{stress} has a somewhat different distribution from Savognin, the augment
duly follows that distribution. Thus \citet[21f.]{anderson2013stem}, in Vaz
(Valbella; data from \citealt{ebneter1981a}) the first person \isi{plural} present
indicative, in addition to arrhizotonic forms, as in Savognin, also has
rhizotonic forms with unstressed desinence -\emph{an}, and the predicted
``stressed'' stem: e.g., \textsc{inf} \emph{amblidár} `forget',
\textsc{1sg} \emph{amblóid,} \textsc{1pl} \emph{amblidáin}, but
\textsc{inf} \emph{amprastár} `lend', \textsc{1sg} \emph{amprést,}
\textsc{1pl} \emph{ampréstan}. Sometimes, the same verb has two possible
first person \isi{plural} present indicative forms, e.g.: \textsc{inf}
\emph{gudáir} `enjoy', \textsc{1sg} \emph{giód,} \textsc{1pl}
\emph{gudáin} / \emph{giódan}. In verbs taking the augment, there are,
correspondingly, forms in \textsc{1pl} -\emph{áin} without augment
(e.g., \textsc{inf} \emph{adorár} `adore', \textsc{1sg} \emph{adorésch,}
\textsc{1pl} \emph{adoráin}), and forms in \textsc{1pl} unstressed
-\emph{an} duly showing it (e.g., \textsc{inf} \emph{habitár} `inhabit',
\textsc{1sg} \emph{habitésch,} \textsc{1pl} \emph{habitéschan}).
According to \citet[23]{anderson2013stem} such behaviour poses a ``problem'' for the
morphomic account, because ``it is fairly clear that the \isi{stem} \isi{alternation}
and the appearance of -\emph{esch} {[}\ldots{}{]} are tied directly to
the position of stress,\is{stress} even where this is potentially variable, and not
to a fixed set of morphological categories''. With respect to Anderson, I
think that it poses no such problem: all it shows is that whatever
principle governs the stressed\is{stress} \isi{root} \isi{allomorph} also governs the augment.
In this particular case, in fact, we are not dealing with a change in
position of \isi{stress} at all: rather, we have a \isi{syncretism} such that the
first person \isi{plural} form tends to be ``taken over'' by the third person
\isi{plural}. This is quite systematic in Vaz (and elsewhere), and occurs
independently of \isi{stem} \isi{stress} (for example, in non-present forms).

More revealing is the case presented in \citet[45f.]{maiden2011a}, of
distributional \emph{discrepancy} between augment and stressed\is{stress} vocalic
alternant. The dialects of the Val Müstair (see \citealt[132]{schorta1938a}) tend
to place \isi{stress} on the \isi{root} of the infinitive in all
conjugations.\footnote{Some second conjugation verbs are exceptions.}
Indeed, they are unique among Romance languages in generally having
rhizotonic infinitives even in the first conjugation, as shown in Table
7:\footnote{This phenomenon is mentioned by \citet[35]{sturzinger1879} and
  \citet[518f.]{huonder1901}, and is amply confirmed by data from Val Müstair
  covered by \emph{ALDI}/\emph{II}. See also \citet[222]{grisch1939a} for Vaz,
  \citet[51]{candrian1900a} for Stalla, \citet[135]{soler1991a} for Schams.}

%\begin{table}
\begin{longtable}[]{@{}llll@{}}
\caption{Rhizotony in Val Müstair first conjugation infinitives}\\
\lsptoprule
\textsc{arare} & \textgreater{} & ˈarər & `plough'\tabularnewline
\midrule
\endhead
\textsc{captare} & \textgreater{} & ˈcatər & `find'\tabularnewline
\textsc{filare} & \textgreater{} & ˈfilər & `spin'\tabularnewline
\textsc{ieiunare} & \textgreater{} & jaˈynər & `fast'\tabularnewline
\textsc{laetare} & \textgreater{} & ˈlai̯dər & `spread
dung'\tabularnewline
\textsc{pescare} & \textgreater{} & ˈpɛʃcər & `fish'\tabularnewline
\textsc{scopare} & \textgreater{} & ˈʃkuər & `sweep'\tabularnewline
*titˈtare & \textgreater{} & ˈtɛtər & `suckle'\tabularnewline
\textsc{iantare} & \textgreater{} & ˈjai̯ntər (or janˈtar) &
`breakfast'\tabularnewline
\emph{telefonare} & \textgreater{} & teleˈfonər &
`telephone'\tabularnewline
\lspbottomrule
\end{longtable}
%\end{table}

\largerpage[-1]
Schorta states, however, that \isi{root} \isi{stress} in first conjugation
infinitives systematically fails to happen in that class of first
conjugation verbs having the element \textit{-aj-}, which I identify as the
augment.\footnote{I must acknowledge here a different, and very careful,
  analysis of these facts by \citet*[291--310]{kaye2015}, for whom \textit{-aj-}
  is not the 'augment' but part of the stressed\is{stress} lexical \isi{root} of the
  verb, whose historically regular unstressed counterpart is \textit{-j-}
  (*bateˈdjare \textgreater{} \textit{bɐˈtjar}; *baˈtedja \textgreater{}
  *bɐˈtaja). In a form such as \textit{bɐˈtjaja}, on Kaye's account, the
  unstressed element -j- has been analogically\is{analogical} generalized into the \isi{root}
  of the stressed alternant, originally of the type *bɐˈtaja \textless{}
  *baˈtedja \citep[307]{kaye2015}; resistance of \textit{bɐˈtjar} (\textless{}
  *bateˈdjare) to \isi{stress} retraction is then, Kaye suggests\citeyear[309]{kaye2015}, a
  function of the degree of phonological difference between stressed and
  unstressed root.\is{root} I would observe that \textit{-aj-} is exactly the expected
  reflex of the proto-form of the augment (although rarely attested in
  the rest of Romansh, where is has been supplanted by \textit{-eʃ-}), and that
  it is not clear why the \isi{root} found in the root-stressed present-tense
  forms of the verb would be phonologically disfavoured in root-stressed
  infinitives. In fact, even if \textit{-aj-} turns not to be, in origin, an
  ``augment'', such an analysis suggests that speakers have effectively
  analysed \textit{bɐtj-} as the lexical root,\is{root} treating \textit{-aj-} as a kind of
  excrescent element following it, and one that occurs just in the
  N-pattern cells. That is to say that its synchronic status is
  equivalent to that of the augment in other verbs.} Here, \isi{stress} always
remains on the ending of the infinitive: e.g., \textsc{inf} \textit{bɐˈtjar}
`baptize', \textsc{3sg.prs} \textit{bɐˈtjaja}; \textsc{inf} \textit{biˈar} `build',
\textsc{3sg.prs} \textit{biˈaja}; \textsc{inf} \textit{guˈjar} `dare', \textsc{3sg.prs}
\textit{guˈjaja}. The same holds of fourth conjugation infinitives, but
apparently only if they belong to that minority of verbs that lack the
augment: Schorta cites \textsc{inf} \textit{fiˈnir} 'finish', a verb which takes
the augment; compare this with, e.g., \textsc{inf} \textit{ˈbwɔʎər} `boil' a verb
that does not show the augment). Now the most likely explanation of why
the augment does not appear in the infinitive here is that, in Romance
languages generally, root-stress\is{stress}\is{root} in infinitives is limited to third
conjugation verbs (cf. \citealt[201f.]{maiden2011c}; \citeyear[509]{maiden2016}), all other classes
having non-rhizotonic infinitives. The augment, however, is
characteristic solely of the fourth and first conjugations, not the
third. The third, while a relatively small and unproductive class,
contains some of the semantically most basic, and highest frequency,
verbs, and the appearance of \isi{root} \isi{stress} in the Val Müstair fourth and
first conjugations is almost certainly modelled, therefore, on the third
conjugation. If no augment can appear in first conjugation infinitives,
it is because the distribution of the augment is \emph{morphologically}
specified, and that specification happens to exclude infinitives.

In \citet{maiden2011a} I argued as follows: if the function of the augment is
in effect to supplete the absence of a stressed\is{stress} root-alternant,\is{root} and if
infinitives in the Val Müstair are generally root-stressed, then we
should expect verbs with the augment duly to show that augment in the
infinitive, in lieu of root-stress\is{stress}\is{root} (e.g., \textsc{inf} **bɐˈtjajər
rather than the actually occurring bɐˈtjar). The fact that the augment
never appears in the infinitive therefore also suggests that its
distribution is independent of stress,\is{stress} and purely morphologically
specified. The only way to ``save'' the stress-based\is{stress} account (and this is
what \citealt{anderson2013stem} does) is to claim that the augment is inherently
limited to ``tensed'' forms, and is therefore not available for the
infinitive. He observes, in support of his view, that it does not occur,
either, in \isi{participles} or in ``related non-verbal forms''. Since the
augment also appears in second person singular imperatives, it is not
clear that ``tensed'' is quite the right term: it might be more accurate
to say that the domain of the augment involves cells with values for
person and number. But now Anderson's claim must be that the augment is
selected in those parts of the \isi{paradigm} specified for person and number
where \isi{stress} would otherwise fall on the root,\is{root} while my account can
easily be reformulated, if we wish, as also applying to those parts of
the \isi{paradigm} specified for person and number, except for the first and
second persons \isi{plural} present indicative. Both accounts acknowledge that
the phenomenon is heavily \emph{morphologized}, and applies over a
domain whose definition corresponds to no natural phonological or
\isi{morphosyntactic} class. Even Anderson's account is, I submit, implicitly
``morphomic''.

\section{The generality of the alternations: derivational
morphology}\label{the-generality-of-the-alternations-derivational-morphology}

Anderson's analysis gains support from the fact that the vocalic
alternations\is{alternation} that occur within the verb also occur outside it: nouns and
adjectives with stressed\is{stress} \isi{derivational affixes}\is{derivation} show the corresponding
``unstressed'' vocalic alternants in the derived forms. The sensitivity of
these alternations\is{alternation} to \isi{stress} is thereby argued to be a general property
of the grammar, and not a peculiarity of verb morphology.\isi{morphology} Take, for
example, the behaviour of the vocalic alternants in derivational\is{derivation}
\isi{morphology} (Table 8), as presented by Anderson (\citeyear[28--30]{anderson2011a};\citeyear[13--17]{anderson2013stem}):


%\begin{table}
\begin{longtable}[]{@{}lllll@{}}
\caption{Vocalic alternants in Savognin derivational morphology}\\
\lsptoprule
& Verb & & Basic noun & Derived nouns\tabularnewline
\midrule
\endhead
Infinitive & \textsc{3sg.prs.} & \textsc{pst.part} & &\tabularnewline
\emph{guttár}

`drip' & \emph{gótta} & & \emph{gót}

`drop' & \emph{gutélla}

`drip'\tabularnewline
\emph{liiér}

`bind' & \emph{léia} & & \emph{léia} `union' & \emph{liadéira}
`\isi{binding}'\tabularnewline
\emph{néiver}

`snow' & \emph{néiva} & \emph{navía } & \emph{néiv}

`snow' & \emph{naváglia}

`big snowfall'\tabularnewline
\lspbottomrule
\end{longtable}
%\end{table}

In reality, this might be no more than the residual, and synchronically
more or less accidental, effect of historical differentiation of vowel
quality according to stress.\is{stress} This is suggested by the fact that there
are derived forms with stressed suffixes\is{suffixation} (and therefore with unstressed
roots)\is{root} where, nonetheless, the \emph{stressed} alternant occurs. Thus
Table 9:

%\begin{table}

\begin{longtable}[]{@{}ll@{}}
\caption{Discrepancy between vowel \isi{alternation} in verbs and derived forms} \\
\lsptoprule
Verb & Derived forms\tabularnewline
\midrule
\endhead
Infinitive & \textsc{3sg.prs.}\tabularnewline
\emph{satgér}

`dry' & \emph{sétga}\tabularnewline
\emph{accumpagnér} `accompany' & \emph{accumpógna}\tabularnewline
\emph{durméir }

`sleep' & \emph{dórma}\tabularnewline
\lspbottomrule
\end{longtable}
%\end{table}

\newpage 
Similar phenomena recur elsewhere in Romansh, as \citet[20]{anderson2013stem}
points out: thus Vallader has \emph{scóula} `school', \emph{scolár} `to
school' (\textsc{3sg.prs.ind} \emph{scóula}), \emph{scolaziún}
`education', yet diminutives \emph{scoulína} `kindergarten',
\emph{scoulétta} `craft school'. \citet[28]{anderson2011a} deals with such
apparent counterexamples by assuming an architecture of the grammar in
which \isi{morphology} and \isi{phonology} ``interact cyclically (with some
appropriate subsystem) of the \isi{phonology} applying to adjust the results
of each stage of morphological elaboration of a form''. The \isi{selection} of
the stressed\is{stress} or unstressed \isi{stem} alternant operates only during the
``first cycle''; once the stem-shape\is{stem} is determined `the decision is not
revisited on subsequent cycles'.\is{derivation} A ``stressed'' base will then persist
through later cycles, even if it is itself no longer stressed: the
derivational counterexamples can now be explained in terms of the cycle
on which they occur.

One immediate objection is that saying that an apparently phonological
phenomenon is confined to a particular ``cycle'' is in fact to concede
that it is ``morphologized''\is{morphology} (the cycles being defined, precisely, as
stages of ``morphological elaboration of a form''), and restriction of a
phonological\is{phonology} rule to some morphologically defined domain is to introduce
into the analysis a considerable degree of \isi{arbitrariness} (precisely one
of the things that for Anderson consitutes an objection to the purely
``morphomic'' analysis). Things look even more arbitrary, and
``morphological'', if we consider that we now have to say that the domain
of the \isi{phonologically conditioned allomorphy}\is{phonology} is defined over two quite
disparate sets of forms: ``tensed'' forms of the verb (at least in
Vallader) and the ``first cycle'' in derivation.\is{derivation} A more fundamental
difficulty is that it is not always true that the ``stressed''\is{stress} \isi{stem}
persists unchanged after the ``first cycle'': why do we have, say, derived
forms \emph{accumpagnedér} or \emph{durmiglión} with ``unstressed''\is{stress}
alternants, yet \emph{accumpognamáint} or \emph{dormulént} with
``stressed''\is{stress} alternants?\footnote{For an inconclusive discussion of these
  data, see \citet[171]{wolf2013a}.} Actually, the predicted \isi{selection} of the
``unstressed''\is{stress} alternant usually occurs in words belonging to inherited
vocabulary, but not in neologisms, which led me to conclude \citep[41]{maiden2011a} that what we have is evidence of the ``death'' of phonological\is{phonology}
\isi{selection} of the allomorphs, now reflected only in traditional
vocabulary. This claim \emph{tends} to be reflected in the behaviour of
adjectives showing reflexes of \ili{Latin} -\textsc{abilis, -ibilis}
(equivalents of \ili{English} -\emph{able}, -\emph{ible}): the ``popular''
reflex by direct inheritance from \ili{Latin} (-\emph{evel}) displays the
``unstressed''\is{stress} alternant (e.g., \emph{\emph{ludével}} `praiseworthy'),
while the ``learned'' ­\emph{-ábel}/\emph{-íbel} displays the ``stressed''\is{stress}
alternant (e.g., \emph{\emph{accumodábel}} `accommodatable'). One might,
perhaps, want to assign -\emph{evel} to the ``first cycle'', and
-\emph{abel} to the second, but even this does not work too well, for we
find occasional examples of the distinctive ``unstressed''\is{stress} alternant with
\emph{-ábel}/\emph{-íbel}: e.g., \emph{schliíbel} `soluble',
\emph{bavábel} `drinkable', \emph{purtábel} `portable', \emph{duvrabel}
`usable'.

\citet[15]{anderson2013stem} observes an ``asymmetry'', in that the counterexamples
to his claim all involve the appearance of a ``stressed''\is{stress} \isi{stem} that does
not bear stress,\is{stress} while no examples exist in which an ``unstressed'' \isi{stem}
appears under stress.\is{stress} But this is not proof that that the \isi{stem}
alternants are sensitive to stress.\is{stress} On such evidence as Anderson
presents (and from \citealt{signorell2001a}), the small inventory of possibly
derived forms involving a stressed\is{stress} \isi{stem} displays that \isi{stem} simply
because it is the phonologically regular result of their etyma (e.g.,
\emph{prescháint} `present' (adj.) \textless{} \textsc{praeséntem}). In
any case, it is perfectly possible that some forms such as
\emph{prescháint} are not ``derived'', but are the base forms from which
the corresponding verbs are derived. Anderson also observes \citeyear[18]{anderson2013stem}
that in Surselvan (in fact, more widely) only ``unstressed''\is{stress} \isi{root}
alternants appear in factive verbs formed with the suffix\is{suffixation} -\emph{ent}-
or -\emph{ant}-,\footnote{Cf. \citet[103f.]{signorell1987a}.} where the
lexical \isi{root} is systematically unstressed\is{stress} (e.g., \textsc{inf}
\emph{béiber} `drink', \textsc{1pl.prs} \emph{buéin}, but factive
\textsc{inf} \emph{buentár} `cause to drink', never **\emph{beibentár}).
He cites similar phenomena in Puter \citep[10]{anderson2013stem}, such as
\emph{stanglantér} from \emph{stáungel} `tired', but says that
``{[}m{]}ore research is needed to establish the generality of the
phenomenon''. Here I concur: we need to be sure that \isi{selection} of the
``unstressed''\is{stress} \isi{stem} is synchronically productive and therefore
psychologically real. Otherwise, all we may have is the regular,
\isi{lexicalized}, outcome of old sound changes in the relevant derived forms.
In any case, there do seem to be examples of factive verbs in
-\emph{entar} that bear the ``stressed''\is{stress} \isi{root} \isi{allomorph}. \citet[291f.]{jaberg1939a} lists Surselvan examples most of which indeed bear the
``unstressed''\is{stress} \isi{allomorph}, but also gives \emph{dormentár} `put to sleep'
(cf. \textsc{inf} \emph{durmír} `sleep', \textsc{1sg.prs.ind}
\emph{dorm}), and a case in which the unstressed\is{stress} vowel of the derived
form, while phonologically plausible, does not correspond to that of the
base verb (\emph{scumpentár} `cause to be saved, heal', from
\textsc{inf} \emph{scampár} `save', \textsc{3sg.prs.ind} \emph{scómpa}).

Even leaving counterexamples aside (and I acknowledge that they are not
many), it is not obviously necessary to invoke stress:\is{stress} given that the
vast majority of cells of the inflexional\is{inflection} \isi{paradigm} of any Romansh verb
are arrhizotonic, one might equally say that what we call the
``unstressed''\is{stress} \isi{stem} is the default, on which affixally derived forms are
usually built. An apparent counterargument to such an approach might
come (cf. \citealt[17]{anderson2013stem}) from the fact that, in cases of derivation
where the \isi{stress} falls on the root,\is{root} it is always the ``stressed''
\isi{allomorph} that appears: e.g., Surmiran \emph{cumónd} `order' (cf.
\textsc{inf} \emph{cumandár}, \textsc{1sg.prs.ind} \emph{cumónd}),
\emph{clóm} `call' (cf. \textsc{inf} \emph{clamár}, \textsc{1sg.prs.ind}
\emph{clóm}), \emph{gartétg} `success' (cf. \textsc{inf}
\emph{gartagér}, \textsc{1sg.prs.ind} \emph{gartétg}), \emph{dórma}
`narcotic' (cf. \textsc{inf} \emph{durmír}, \textsc{3sg.prs.ind}
\emph{dórma}), \emph{stéma} `esteem' (cf. \textsc{inf} \emph{stimár},
\textsc{1sg.prs.ind} \emph{stéma}). Significantly, however, \citet[51]{signorell1987a} describe such forms, pre-theoretically, as ``deriving from
a finite form'': what we may have here is simply nominalization of first
or third person singular verb-forms (which happen to contain stressed\is{stress}
roots),\is{root} not a derivational\is{derivation} process which specifies a stressed\is{stress} root, and
thereby must select the ``stressed''\is{stress} alternant. In short, while it is
indeed true that many patterns of root-alternation\is{alternation}\is{root} found in the verb
recur across the grammar, this is largely a historical residue, not
necessarily evidence of an active synchronic phonological principle.

\section{Conclusion}\label{conclusionM}

\citet[23]{anderson2013stem} accepts that ``morphological categories play a role
(e.g. in constraining the appearance of -\emph{esch} to tensed forms of
first and fourth conjugation verbs)'', but asserts that `there is no
warrant for invoking the further step of complete and arbitrary
morphological categorization that would be implied by associating the
variation with a morphome'. I suggest that the data are in fact already
inextricably permeated with ``arbitrary'' morphological specifications: in
Savognin, and in Romansh at large, the morphomic N-pattern is really
present. The need to specify that the alleged phonological\is{phonology} principle
only applies to ``tensed'' verb forms (for Val Müstair), or to certain
levels of derivational \isi{morphology} makes that principle itself
``arbitrary''. Given that the behaviour of \isi{suppletion} in \emph{dueir} and
the distribution of -\emph{esch} are, as I have argued, incompatible
with the ``phonological''\is{phonology} account, attempts to account for the identically
distributed \isi{vocalic alternations}\is{alternation} in phonological terms become
superfluous. Finally, given that countless Romance varieties do have
genuinely morphomic patterns of the kind attested in Savognin, treatment
of Savognin as a special case is what may be ``unwarranted''. And
yet\ldots{}.

I do not think that Savognin can be presented as the perfect example of
``\isi{phonologically conditioned allomorphy}'' that Anderson claims, and yet
one must ask whether Anderson's insight -- that right across the grammar
there is a close correlation between \isi{stress} and the \isi{selection} of
alternants, might be at risk of being abandoned too lightly. My
criticism has been that there exist some cases where such an analysis
does not ``work'', and that since we need independently to invoke the
N-pattern even for Savognin, we should do so for all types of
\isi{alternation} which follow that pattern. But there is an unspoken
assumption here which may need to be challenged, and it involves what
might be described as the ``ghettoization of the morphomic''. The classic
examples of ``morphomic'' phenomena as adduced by \citet{Aronoff94:book} make the
case for the existence of a ``morphomic level'' of linguistic analysis
precisely because they are not plausibly explicable as effects of
phonological, syntactic, or semantic conditioning: they are cases of
``morphology by itself''.\is{morphology} In morphologists' enthusiasm to assert the
existence of genuinely morphological phenomena, much weight has been
placed on the notion of the ``\isi{autonomy}'' of \isi{morphology} (witness the titles
of \citealt{maiden2005}, or \citealt{maiden2011}). While there are very good
reasons to proclaim loudly that ``autonomously morphological'' phenomena
exist, the search for them should not become a reductivist obsession,
nor is there any good reason to suppose that there cannot exist
phenomena which contain a very high degree of purely morphological
determination, while yet also possessing some degree of phonological or
other conditioning.

The seeds of a possible compromise appear in \citet{maiden2013a} (actually, in
the same volume as, and immediately following, \citealt{anderson2013stem}). This
deals with patterns of consonantal \isi{alternation} in \ili{Italian} verbs
historically caused by two different kinds of palatalization.
Synchronically, the result is that phonologically quite disparate types
of alternant tend overwhelmingly to conform to a common distribution
such that one alternant occurs in all (or most) forms of the present
subjunctive, and in the first person singular and third person \isi{plural}
forms of the present indicative, but nowhere else in the paradigm. There
are powerful arguments (see, e.g., \citealt{maiden1992a},\citeyear{maiden2011a}\citeyear[205--63]{maiden2011c}) to
say that this pattern has lost all phonological\is{phonology} causation and is
genuinely morphomic. In \citet{maiden2009a} I had been extremely critical of
attempts by \citet{burzio2004a} to force a synchronic phonological\is{phonology} analysis of
the modern \ili{Italian} facts, quite often by what is, in effect, the
illegitimate resurrection of long dead phonological conditioning
environments. For the most part, these are criticisms I stand by: it
cannot be said too often that \isi{morphology} suffers from a kind of
``phonologizing bias'' which too readily dismisses morphological analyses
of the data, and is far too prone to give credence to
phonologically-oriented accounts, even at the expense of postulating
conditioning environments lacking plausible synchronic justification.
Burzio's analysis appeared to me an example of this kind, but
observation of some of the fine details of the \isi{diachrony} of the
alternations\is{alternation} at issue later caused me to moderate my view.

One type of \isi{alternation} involved an opposition between velar consonants
and palatal affricates, the latter arising, historically, through
palatalization and affrication of velars before front vowels. Now it is
beyond reasonable doubt that there has existed no productive process of
palatalization/affrication of velars before front vowels for over a
millennium, such a putative process being massively counterexemplified
by the existence of unmodified velars before front vowels from the time
of the earliest documents, including within the \isi{paradigm} of the verbs at
issue. The principal fact\footnote{But see also \citet[31--35]{maiden2013a}.}
which made me revise \citep{maiden2013a} the ``morphological exclusivism'' of my
earlier treatments of the subject, however, was the observation that in
medieval \ili{Italian} a certain type of \isi{analogical} \isi{innovation} affecting verbs
displaying the relevant types of alternation,\is{alternation} whereby the root of the
present subjunctive was optionally extended into \isi{gerund} forms with the
ending ­-\emph{endo}, was strikingly, and systematically, \emph{blocked}
just where the result would have been a velar consonant followed by a
front vowel. Informally: ``don't allow a velar alternant before a front
vowel if an alternative (and more phonologically ``natural'') palatal
alternant is also available''. Thus Table 10 (where ``**'' means ``not
occurring''):


%\begin{table}
\begin{longtable}[]{@{}lll@{}}
\caption{Analogically reformed subjunctives in old Tuscan} \\
\lsptoprule
subjunctive & inherited

\isi{gerund} & \isi{gerund} analogically

reformed on subjunctive\tabularnewline
\midrule
\endhead
\emph{possa} `can' & \emph{potendo} & \emph{possendo}\tabularnewline
\emph{ve/}dʤ\emph{/a} `see' & \emph{vedendo} &
\emph{ve/}dʤ\emph{/endo}\tabularnewline
\emph{te/}ɲɲ\emph{/a} `hold' & \emph{tenendo} &
\emph{te/}ɲɲ\emph{/endo}\tabularnewline
\emph{abbia} `have' & \emph{avendo} & \emph{abbiendo}\tabularnewline
\emph{pia/}tʧ\emph{/a} `please' & \emph{pia/}ʧ\emph{/endo} &
\emph{pia/}tʧ\emph{/endo}\tabularnewline
\emph{di/}k/\emph{a} `say' & \emph{di/}ʧ/endo &
\emph{**di/}k/\emph{endo}\tabularnewline
\emph{pian}/ɡ/\emph{a} `weep' & \emph{pian}/ʤ/\emph{endo} &
\emph{**pian}/g/\emph{endo}\tabularnewline
\lspbottomrule
\end{longtable}
%\end{table}

While it would have been impossible to explain the distribution of the
alternants in \emph{purely} phonological terms (for the reasons, see
\citealt[25--31]{maiden2013a}), this behaviour clearly suggests residual sensitivity
to \emph{phonologically plausible} environments for the distribution of
certain alternants.

In short, while Anderson's analysis of the Savognin data seems to me too
``phonologizing'', it may be that my own approach, insisting on purely
morphological aspects of the phenomenon, has been too ``morphologizing'',
and both approaches seem to be subject to the questionable assumption
that redundancy must be eliminated from the analysis. I do not think
that Savognin is as ``pure'' an example of ``phonologically conditioned
\isi{allomorphy}'' as Anderson believes, but the possibility that speakers are
sensitive to the recurrent correlation between certain types of
\isi{alternation} and \isi{stress} should not be sacrificed too hastily on the altar
of formal economy. As \citet[205f.]{Sims15:book} observes, our two approaches
need not in fact be mutually exclusive. We have probably reached the
point where only appropriately devised psycholinguistic experimentation
would be able to tell us more about the Savognin data. However that may
be, morphologists and Romance linguists are truly in Steve Anderson's
debt for having focused our attention so sharply on these fascinating
data.


\section*{Abbreviations}
\begin{tabular}{ll}
{AIS} & \citet{jaberg1940a}\\
{ALDI} & \citet{goebl1998a} \\
{ALDII} & \citet{goebl2012a}\\
\end{tabular}

\section*{Acknowledgements}
I would like to thank Hans-Olav Enger for some very helpful remarks on an earlier draft of this chapter.

{\sloppy
\printbibliography[heading=subbibliography,notkeyword=this]
}
% \todos

\end{document}
