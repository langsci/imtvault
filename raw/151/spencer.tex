\documentclass[output=paper,
modfonts
]{LSP/langsci}



\input{localpackagesspencer.tex}

%%Copy this to localcommands.tex

\usepackage[english]{babel}
\usepackage{amsmath}
\usepackage{amssymb,amsfonts,textcomp}
\usepackage{array}
\usepackage{hhline}
\usepackage{hyperref}

\newenvironment{styleStandard}{}{}
\newenvironment{stylelsAbstract}{}{}
\newenvironment{stylelsSectioni}{}{}
\newenvironment{stylelsSectionii}{}{}
\newenvironment{stylelsBulletList}{}{}
\newenvironment{styleBibliographyi}{}{}
\newenvironment{listWWNumxxvleveli}{}{}
\newenvironment{listWWNumxxvlevelii}{}{}
\newenvironment{listWWNumxxvleveliii}{}{}
\newenvironment{listWWNumxxvleveliv}{}{}
\newenvironment{listWWNumixleveli}{}{}
\newenvironment{listWWNumixlevelii}{}{}
\newenvironment{listWWNumixleveliii}{}{}
\newenvironment{listWWNumixleveliv}{}{}

\newcommand\textstyleListLabelxvi[1]{#1}
\newcommand\labellistWWNumxxvleveli{\thelistWWNumxxvleveli.}
\newcommand\labellistWWNumxxvlevelii{\thelistWWNumxxvlevelii.}
\newcommand\labellistWWNumxxvleveliii{\thelistWWNumxxvleveliii.}
\newcommand\labellistWWNumxxvleveliv{\thelistWWNumxxvleveliv.}
\newcommand\labellistWWNumixleveli{[F0B7?]}
\newcommand\labellistWWNumixlevelii{\textstyleListLabelxvi{o}}
\newcommand\labellistWWNumixleveliii{[F0A7?]}
\newcommand\labellistWWNumixleveliv{[F0B7?]}

\newcounter{listWWNumxxvleveli}
\newcounter{listWWNumxxvlevelii}[listWWNumxxvleveli]
\newcounter{listWWNumxxvleveliii}[listWWNumxxvlevelii]
\newcounter{listWWNumxxvleveliv}[listWWNumxxvleveliii]
\newcounter{itemize} 

\author{Andrew Spencer\affiliation{University of Essex}}

\title{Split-morphology and lexicalist morphosyntax: The case of transpositions} 
\ChapterDOI{10.5281/zenodo.495456}

% \chapterDOI{} %will be filled in at production
% \epigram{}

\abstract{
One of Anderson's many contributions to morphological theory is the claim that morphology is split between  syntactically mediated inflection and lexically mediated derivation. In Minimalist morphosyntax all morphology is syntax. This means that the split morphology proposal is not meaningful for that model. In lexicalist models, however, the split morphology hypothesis manifests itself as a distinction between direct accessibility to syntactic representations (inflection proper) and lack of accessibility. However, there are construction types which bring the inflection-derivation distinction into question. One of these is the transposition, as illustrated by the ubiquitous deverbal participle. This is a mixed category, being at once a form of a verb yet having the external syntax of an adjective. It is thus unclear which side of the split participles fall. Similarly, participles seem to be an embarrassment for the Word-and-Paradigm models of inflection which have become dominant since Anderson first introduced them to contemporary theorizing. This is because they seem to require us to define a ``paradigm-within-a-paradigm'' (or ``quasi-lexeme-within-a-lexeme'').

I provide an analysis of Russian participles within Stump's PFM2 model, deploying the model of lexical representation developed in \citet{Spencer13:book}, which fractionates representations into more finely grained subcategories than is usual. I take a participle to be the adjectival representation of a verb, coded directly by means of a set-valued feature, REPR. I show how  a set of rules can be written which will define the adjectival paradigm as a set of forms belonging to the overall paradigm of the original verb lexeme. The rules define a partially underspecified lexical entry for the participle (``quasi-lexeme''), which has essentially the same shape as the lexical entry for  an (uninflected) simplex adjective. Thus, the participle's lexical entry is that of an adjective, just as though we were dealing with derivation, but it realizes the verbal properties of voice/aspect and it shares its semantics and lexemic index with its base verb, just as in the case of verb inflection. The participle thus straddles the split, but in a principled fashion.
}

\begin{document}
\maketitle

\newpage 
\section{Introduction: Morphological architecture}\label{sec:architecture}
\largerpage
Since \citet{AndersonSR77:inflection}
(re-)introduced to generative grammar the traditional notion of ``word-and-paradigm'',\is{word-and-paradigm}\is{paradigm} 
and particularly the ground-breaking work of  \citet{Matthews72}
on  \ili{Latin}  %\isl{Latin}
inflection,\is{inflection} morphologists have been grappling with the challenge of providing an adequate characterization of the key notions ``word'' and ``paradigm''.\is{paradigm}%

Central to this debate has been the fate of the Bloomfield/Harris %\isa{Bloomfield}%\isa{Harris}
interpretation of the morpheme\is{morpheme} 
concept  \parencite{AndersonSR15:morpheme}
and the notion of ``\isi{Separationism}'' \is{Separationism} 
\parencite{Beard95:book}.
The ``word'' notion presupposes (at least) a word/phrase distinction. (In morpheme-based\is{morpheme} approaches no such distinction is necessary and all \isi{morphology} and syntax can be subsumed under a model of morphotactics.) There are very well known problems with any attempt to find necessary and sufficient conditions for the \textquoteleft concrete\textquoteright\ instantiations of word \textemdash\ the \isi{phonological word}, (inflected) word form, syntactic word, even. This is generally on account of incomplete grammaticalization, which strands constructions and formatives in a limbo between the status of function word \textendash\ \isi{clitic} \textendash\ affix,\is{clitic}
\isi{compound} element\textendash\ affix,\is{affix} analytical syntactic construction \textendash\  \isi{periphrasis}\is{periphrasis}
and so on. However, the \textquoteleft abstract\textquoteright\ notions of word are no less problematic, specifically the lexeme and the \isi{morphosyntactic} word (i.e. an inflected word form together with the \isi{morphosyntactic} property array that it realizes).  Defining the set of \isi{morphosyntactic} words often requires us to make decisions about what constitutes a word form, which brings us back to the issue of clitics, \isi{periphrasis} and so on. It also requires us to make sometimes arbitrary decisions about \isi{morphosyntactic} property sets (MPSs) in the light of form-content mismatches such as (some types of) \isi{syncretism} \parencite{Baerman:etal05:book}, %
overabundance \parencite{Thornton12:overabundance}, %
\isi{defectiveness} \parencite{Sims15:book}, %
and deponency  \parencite{Baerman:etal07}, %
as summarized in \citet{Stump16:book}.

Inflectional\is{inflection} properties, and word-oriented \isi{functional categories} generally, such as \isi{definiteness} (for nouns) or modality (for verbs) in \ili{English}, seem to presuppose an inflection\is{inflection} $\sim$ \isi{derivation} dichotomy that is notoriously hard to pin down. Broadly speaking it distinguishes the creation of new lexical items/units (generally, Saussurean signs pairing a cognitive meaning with a set of forms) from forms of a lexical item/unit. The component of grammar that defines new lexical units or lexemes is derivational\is{derivation} morphology.\is{morphology} However,  as \citet{Spencer13:book} %
itemizes in some detail, such a (canonical) inflection/derivation\is{inflection}\is{derivation} dichotomy represents just two poles of a scale of types of \isi{lexical relatedness}.\is{lexical relatedness} 
Some of the intermediate types of relatedness pose problems for any clean characterization of the lexeme concept 
\parencite[part of the problem of ``\isi{lexemic individuation}'',][]{Spencer16:individuating}.\is{lexemic individuation}

 One way of characterizing the core of the inflection/derivation\is{inflection}\is{derivation} distinction is the notion of \isi{split morphology}\is{morphology} 	\parencite[][and subsequent references]{AndersonSR82}.\is{split morphology}
 The essence of the distinction can be thought of as an interface claim: \isi{inflection} \isi{interfaces} directly with syntax (\textquoteleft \isi{inflection} is what is relevant to syntax\textquoteright). The obverse to this claim is that \isi{derivation} \isi{interfaces} with the lexicon,\is{lexicon} in the sense that \isi{derivation} is what gives rise to expansion of the lexical stock (as well as defining relatedness between already fixed lexical entries), in other words \isi{derivation} is \textquoteleft what is relevant for the lexicon\textquoteright.\is{lexicon} Anderson implements this architectural claim by saying that it is the rules of syntax themselves which define inflectional\is{inflection} morphology. \is{morphology}
  
  \newpage 
This move raises the important question of what model of syntax we are presupposing. Most versions of the \isi{Minimalist Program}\is{Minimalist Program}
presuppose something very close to the model of morphotactics proposed by the American Structuralists:\is{Structuralism}
the atoms of representation are morphemes,\is{morpheme} \isi{morphology} and syntax are identical (it is therefore a terminological choice whether we think of sentence construction as morphotactics or syntax), and the notions lexeme, word form, inflection,\is{inflection} derivation,\is{derivation} inflectional \isi{paradigm} are at best heuristic descriptive terms which cannot be given a coherent definition within the model. The natural syntactic framework for investigating a word-and-paradigm,\is{paradigm} or rather, lexeme-and-paradigm\is{paradigm} approach to \isi{inflection} is, perhaps, a lexicalist, or \isi{constraints-based},\is{constraints-based}\is{constraint}
model \parencite{Miller:Sag97,Sadler:Spencer01,Sadler:Nordlinger06:stacking}. %
 In that case the question of \isi{split morphology}\is{morphology} takes on a somewhat different \isi{aspect}. Rather than claiming that syntactic rules \emph{construct} inflected forms as such, we must say that inflected forms, compared with derived lexemes, are permitted to  interact in specific ways with syntactic representations, or equally that inflected forms bear properties which are visible to syntactic representations and principles. 
 
 The obvious way to implement this idea is to say that the abstract characterization of an inflected word includes a \isi{morphosyntactic} description which overlaps with that of a corresponding syntactic representation. A concrete version of this type of overlap is seen in the form-content mapping, as defined in Stump’s notion of \isi{paradigm} linkage %
 \parencite{Spencer:Stump13:Hungproncase, Stewart:Stump07,Stump02:paradigmlinkage,Stump06:heteroclisis,Stump16:book,Stump16:MorphMetatheory}.	%
Stump distinguishes morphological\is{morphology} properties, the FORM paradigm,\is{paradigm} from syntactic properties, the CONTENT paradigm.\is{paradigm} By default these are homologous, but there are many instances of mismatch. For example, \ili{Latin} syntax distinguishes singular and \isi{plural} number and a variety of cases,\is{case} including dative and ablative, but those two cases are never distinguished morphologically for any lexeme in the \isi{plural}. On the other hand, \ili{Spanish} verbs have two distinct subparadigms\is{paradigm} for the imperfect subjunctive but that distinction is nowhere reflected in the syntax. Other mismatches include deponency and \isi{periphrasis}. To a limited extent we can say that the form-content \isi{paradigm} distinction is a reflex of covert \isi{split morphology}:\is{morphology} such a distinction is not definable for derivational\is{derivation} morphology. \is{morphology}
 
 In lexicalist models, the derivational\is{derivation} \isi{morphology} $\sim$ \isi{lexicon} interface operates over property sets which don’t play a direct role in syntax. The hedge ``direct'' is important: typically, \isi{derivation} \emph{is} relevant to syntax, in the sense that it changes a lexeme’s \isi{morphosyntactic} class. More subtly, \isi{derivation} may make appeal to \isi{argument structure}\is{arguments} realization (witness \ili{English} Subject Nominalizations, \emph{able}-Adjective formation and so on). But if it is assumed that lexemes have a representation of their word class \isi{argument structure} and other relevant properties, then such syntactically expressed
relations can be defined over lexical representations, as extensively argued in the \isi{constraints-based}\is{constraint} literature  %
\parencite[see][for a review]{Wechsler14:book}.
This is effectively a statement of the doctrine of \isi{lexical integrity},	\is{lexical integrity}
at the abstract level of representation as defined by %
\citet{Ackerman:LeSourd97}.

The conclusion to be drawn is that \isi{morphology} interfaced with a \isi{constraints-based}\is{constraint} syntax needs to be split in essentially the way proposed by Anderson, but as an abstract architectural property, which sometimes bears a rather complex relation to concrete morphophonological expression. Inflectional\is{inflection} \isi{morphology} maps to syntactic representations in a way in which derivational\is{derivation} \isi{morphology} is unable to, while derivational \isi{morphology} serves to define (specific kinds of) \isi{lexical relatedness}. However, there remain interesting cases of violations of \isi{lexical integrity} with derivational\is{derivation} relations, in which syntax appears to have access to the internal structure of the derived word.  This paper will argue that such phenomena require us to extend the \isi{scope} of the split in \isi{morphology} in a way that takes the notion of ``lexeme'' as syntactic atom seriously, and which ultimately provides conceptual motivation for a lexicalist interpretation of \isi{split morphology}. \is{morphology}

A case in point is the class of denominal (relational) adjectives in many languages, which allow one noun to modify another by taking on the morphosyntax of an adjective. In European languages, including \ili{English} and \ili{Russian}, %\isl{Russian}
such adjectives respect \isi{lexical integrity}, in the sense that the base noun is \isi{opaque}  to syntax. For example, the base noun \textex{kniga} ‘book’ in the \ili{Russian} relational adjective \textex{knižnyj} ‘pertaining to a book/books’ is \isi{opaque} to agreement, government or any other syntactic process, just as in \ili{English}  the noun \textex{tide} in \textex{tidal} is \isi{opaque}. For instance, the phrases \textex{poderžannaja kniga} ‘second-hand book’ and \textex{knižnyj magazin} ‘book shop’ do not gives us \textex{*poderžannaja/poderžannyj knižnyj magazin}, and although we can say \textex{high tide} and \textex{tidal barrier} we can’t say \textex{*high tidal barrier}. The importance of these observations is that there are languages  in which just such attributive modification into a derived adjective is possible  (see the discussion of Tungusic and Samoyedic examples in \cite{Nikolaeva08}, and also the detailed discussion of the Samoyedic language Selkup %\isl{Selkup}  
in \citealt[Chapter 10]{Spencer13:book}).

The relational adjectives of \ili{Russian} and \ili{English}, however, share  one important property with the Tungusic and Samoyedic  pure relational adjectives, namely, they have precisely the same lexical \isi{semantics} (cognitive content) as the base noun. This leaves us with the question of how to explain why in some languages relational adjectives are \isi{opaque} and in others they are transparent to attributive modification.   

\citet{Spencer13:book} argues that the crucial difference between true transpositional relational adjectives of, say, Selkup,  and the ``fake'' transpositions of \ili{English}/\ili{Russian} is that true transpositions are effectively forms of the base noun lexeme, while the \ili{English} and \ili{Russian} relational adjectives are distinct lexemes, though ones which have a semantic representation identical to that of their base, what \citet[275]{Spencer13:book} calls a transpositional lexeme. Other types of transpositional lexeme include \ili{English} property nominalizations (\textex{kindness, sincerity, \ldots}), deverbal nominalizations such as \textex{destruction}, and participial forms which have been converted into qualitative adjectives such as \textex{(very) boring/bored, charming, excited, \ldots}.  A relational adjective which is a true transposition permits inbound attributive modification because it is, in an important sense, still a noun, just as a noun \isi{stem} marked for number, case,\is{case} possession or \isi{definiteness} is still a noun. 

One consequence of this reappraisal of the morphology$\sim$syntax\is{morphology} interface is that the crucial divide can no longer be straightforwardly equated with a traditional distinction between \isi{inflection} and derivation.\is{derivation} It is not appropriate to think of a deverbal participle or a relational adjective as merely an inflected form of a verb or noun, because that participle or adjective will in general inflect like an adjective, not like a verb/noun.  However, following \citet{Haspelmath96}, \citet[Chapter 10]{Spencer13:book} argues for an enrichment of the traditional notion of the \isi{inflection} \isi{paradigm} to include an attribute REPRESENTATION (taken from \ili{Russian} descriptive practice). Thus, a participle is the adjectival representation of the verb, and as such it can have its own adjectival inflectional\is{inflection} paradigm.\is{paradigm} It thus has the outward appearance of an autonomous lexeme, but appearances are deceptive. Rather, the transposition is a ``quasi-lexeme'', \label{quasilexeme} and the transpositional relationship therefore represents a particularly striking instance of a deviation from inflectional\is{inflection} canonicity.%
\footnote{Other such deviations are certain forms of evaluative \isi{morphology} (cf ``the diminutive \emph{form of} a noun'') and grammaticalized \isi{argument structure}\is{arguments} alternations\is{alternation} (cf ``the passive/anti-passive/applied/\isi{causative}\is{passive} \emph{form of} a verb'').} %
Like transpositions, these are not usually described under the heading of morphology-syntax\is{morphology} mismatches, and like transpositions they are not discussed in Stump’s (\citeyear*{Stump16:book}) otherwise very detailed survey. 

In the model of lexical representation argued for in \citet{Spencer13:book} the notion ``form of a lexeme'' in this somewhat extended sense is reflected very simply: all forms of a lexeme share their Lexemic Index. This leads us to propose a (no doubt too strong) hypothesis about the nature of the split in morphology:\is{morphology} \medskip

\noindent

\textbf{Principle of Lexemic Transparency}

\begin{tabular}{@{}p{28em}}

Let D be a word derived by some regular morphological process from a word B,  possibly of different \isi{morphosyntactic} category. 
If \isi{morphosyntactic} processes treat D in the same manner as they would treat the base word, B, even  where the category of D is such that we would not otherwise expect it to be subject to such processes, then D is a form of the lexeme B (shares B’s Lexemic Index).
\end{tabular}\bigskip

I shall argue that the  architectural equivalent of splitting inflectional\is{inflection} from derivational\is{derivation} \isi{morphology} is this modification of the notion of \isi{lexical integrity}: if \isi{morphology} defines a set of forms of a lexeme, rather than defining a new, autonomous lexeme, then those forms will show lexical \isi{transparency}. Derived lexemes, however, show lexical \isi{opacity} (one reflex of which is the more familiar property of \isi{lexical integrity}). This paper will illustrate that proposal on the basis of the behaviour of \ili{Russian} deverbal \isi{participles}. These are particularly useful. First, \ili{Russian} adjectival morphosyntax is very clearly distinguished from noun or verb morphosyntax, so it is easy to show that the \isi{participles} behave like adjectives. Second, the \ili{Russian} \isi{past tense} and conditional mood are expressed by a form which is historically a participle and which show participle-like agreement, but which has been reanalysed as a verb form (the l-participle). This contrasts in important ways with the true \isi{participles}. Third, like many languages, \ili{Russian} often converts its \isi{participles} into true qualitative adjectives. These have (almost) exactly the same set of forms as the \isi{participles} but their syntax is no different from that of a simplex adjective. The true \isi{participles} are like the l-participle in showing lexical \isi{transparency} with respect to the base verb. They therefore both appear on the inflectional\is{inflection} side of the \isi{split morphology}.\is{morphology} This is because they are forms of the verb’s paradigm,\is{paradigm} and do not constitute independent lexemes in their own right. They contrast with the converted \isi{participles}, which are autonomous lexemes and hence \isi{opaque} with respect to the verb properties of their (etymological) base lexeme.

\nocite{RussGramm80-I}
%Lexical representations and \isi{lexical relatedness} 
\section{Lexical representations and lexical relatedness}\label{sec:lexreprel}

\subsection{Introduction}

Our discussion will require us to be explicit about a number of aspects of lexical representation and the way that words, in the broadest senses of the term, are related to each other. 
I shall adopt a generalized form of Stump’s (\citeyear*{Stump01:book}) Paradigm Function Morphology,\is{paradigm}\is{morphology} which I have called \emph{Generalized Paradigm Function Morphology}, GPFM (\citealt{Spencer13:book}). The GPFM model is designed to permit us to use the machinery of PFM to describe types of \isi{lexical relatedness} which go beyond normal inflectional\is{inflection} morphology.\is{morphology} It thus extends the lexical representations that \isi{morphology} has access to by incorporating representations of syntactic properties and the lexical semantic representation of words. In GPFM lexemes have to be individuated by means of an arbitrary index, the Lexemic Index (defining something like the key field in a database). One of the reasons for this is because it is arguably not possible in the general case to individuate lexical representations of lexemes in terms of any of the linguistically relevant properties that can be ascribed to a lexical representation. In addition, however, the index serves an important role in distinguishing certain types of morphological relatedness.%


\subsection{Lexical representations}	\label{sec:lexrep}

I begin with the descriptive representational  apparatus required to characterize an inflected word form, taking \isi{inflection} to be an uncontroversial category for the sake of exposition. I then generalize the representational format to provide a characterization of the lexemic entry.

A word has a minimum of three contentive attributes (together with a fourth, its Lexemic Index, LI): FORM, SYN(TAX), SEM(ANTICS). %
The SYN attribute records idiosyncratic selectional or collocation properties, but its main component is the  \isi{argument structure} attribute, ARG-STR. This records thematic argument\is{arguments} arrays in the standard fashion (notated here as x, y, \ldots\ variables). However, it also includes a semantic function (sf) role. 

For nouns and verbs the sf roles are the ``R'' and ``E'' roles respectively, familiar from the literature. The ``R'' (for ``referential'') 
argument\is{arguments} is canonically associated with lexical entries whose SEM value belongs to the ontological class of \textit{Thing}.
It thus identifies those predicates that typically denote (concrete or abstract) objects and that can serve as the lexical head of a referring expression, i.e. a canonical noun.  Thus, the ``R'' argument\is{arguments} of \textsc{tree}%
\footnote{Where relevant, I adopt the standard convention of putting the name of a lexeme in \textsc{small caps}.} %
 corresponds to the ``x'' variable in the semantic representation λx.TREE(x). It is the argument\is{arguments} that is the target of attributive modifiers: \emph{(tall) tree} \parencite{Spencer99:transpositions}.
See \citet[16; 55\textendash 59]{Lieber04:book} for concrete examples of the R role being deployed in morphology.\is{morphology}
The ``E'' (for ‘event(uality)’) argument\is{arguments} (sometimes written as ``e'' or ``s'' (for ``situation'')) is canonically associated with lexical entries whose SEM value belongs to the ontological class of \textit{Event}. It thus identifies those predicates that typically denote states or events (eventualities)  and that can serve as the lexical head of a clause, i.e. a canonical verb.  Thus, the ``E'' argument\is{arguments} of \textsc{fall} corresponds to the ``e'' variable in the (neo-Davidsonian) semantic representation λe,x.{FALL}(e,x). For attributive modification (principal role of the traditional adjective class) I assume a semantic function role labelled ``A''. This is coindexed to a noun’s R sf role to represent attributive modification. All adjectives which function as attributive modifiers, including relational adjectives and \isi{participles}, have an ``A'' semantic function role.

I assume the SEM attribute is essentially a formula in predicate calculus defined over the ontological types \textit{Thing, Event, Property} \parencite{Jackendoff90}, %
and perhaps others, corresponding loosely to the \isi{morphosyntactic} categories of Noun, Verb, Adjective. I remain here agnostic as to whether the categories N, V, A are universal and if so, in what sense. I assume that some languages also have a category of Adverb, and also transpositional morphosyntax, adjective-to-adverb (as in \ili{English} \emph{ly}-suffixation),\is{suffixation} verb-to-adverb (\isi{gerund}) and noun-to-adverb (found in Selkup, for example), but I will not have much to say about that category here.

Adpositions may mandate a further ontological category of, say, \textit{Relation}, but I ignore that too. The SEM attribute can be thought of as a label for an encyclopaedic representation, such as λx.CAT(x) or λx,y.WRITE(x,y), but sometimes including linguistically encoded information relevant to semantic interpretation that cannot simply be consigned to an undifferentiated encyclopedia, for instance, λx,y,δ.SIMILAR\_TO(x,y,δ) ∧  CAT(y) ∧ DIMENSION(δ), ‘similar to the property of ``cat'' in some dimension, δ’, or λx,y.AGAIN(WRITE(x,y)) `to re-write something’.


The FORM attribute is essentially a record of the word’s morphology.\is{morphology} Assuming an articulated inflectional\is{inflection} system, complete with arbitrary inflectional\is{inflection} classes and possibly other purely morphological,\is{morphology} paradigm-based\is{paradigm} properties, the FORM attribute needs to specify all the information needed to locate the word form in the appropriate inflectional\is{inflection} paradigm.\is{paradigm} The first property is the morpholexical category, MORCAT. This will typically be derived by default from the syntactic category of the representation (the SYN|CAT attribute),%
\footnote{Given the complexities of category mixing it is better to dispense entirely with morphological or syntactic category labels such as  ``verb'', ``adjective''. The required lexical classes can be defined over other aspects of representation much more efficiently and it is not difficult in \isi{constraints-based}\is{constraint} models to ensure that those aspects of representation are accessible to rules of morphosyntax. However, for convenience of exposition I will continue to talk of (morphological or syntactic) verbs, adjectives and so on.}
 %
but that default mapping is not infrequently overridden, sometimes in rather complex ways. 

The next property is largely defined by reference to the syntactic category of the word form/lexeme, namely, the morpholexical signature, MORSIG. This specifies all those \isi{morphosyntactic} properties  for which an element of that MORCAT is obligatorily inflected. An inflected word has to have a specification of the \isi{morphosyntactic} property set (or sets), MPSs, that it realizes. In the case of \isi{syncretism} this may be a (natural or unnatural) class of MPSs. %
For example, a \ili{Russian} adjective is obligatorily inflected for at least the properties of number, gender and case,\is{case} and these features are therefore listed in the MORSIG (a gradable adjective is also inflected for comparative and superlative forms). We will see in \S\ref{sec:Russparalink} that the conception of MORSIG assumed in \citet{Spencer13:book} can  be enriched and extended to include the FORM-CONTENT \isi{paradigm} distinction introduced in \citet{Stump06:heteroclisis} and subsequent work.

Finally, the representation has to specify a phonological form for the word, through an  attribute FORM|PHON. The precise characterization of the PHON entry, in general, will be given by the rules of inflectional\is{inflection} morphology.\is{morphology} I will assume that one \isi{aspect} of the PHON representation will be a specification of the STEM\is{stem} on which the inflected form is based, but I ignore this refinement because it will not be relevant to the question of \isi{split morphology}.\is{morphology}

The actual inflected word forms of the lexeme are defined by inflectional rules, which apply to the pairing $\langle$\pounds, σ$\rangle$, where σ is a complete, permissible feature\is{features} set for the lexeme with Lexemic Index \pounds. The lexemic representation has to include all the idiosyncratic morphological information relevant to a lexeme’s realized paradigm.\is{paradigm} In the next subsection I summarize the way that the Paradigm\is{paradigm} Function  can be generalized to define  not only \isi{inflection} but all the systematic forms of relatedness. 

One \isi{aspect} of these representations is worth noting. In keeping with the inferential-realizational assumptions underlying our model of \isi{inflection} the SEM attribute remains constant for all inflected forms. In particular, there is no characterization at the level of the lexical representation of a word form (much less the level of lexemic representation) of the \isi{semantics} of, say, \isi{tense} or number. What this means is that in the syntax the  VP  which is headed by a \isi{past tense} form verb  may, ceteris paribus, be  interpreted as referring to an event situated prior to speech time. However, since ‘\isi{past tense}’ forms are also used in sequence of \isi{tense} constructions, irrealis conditionals and so on, ‘past time’ is only the default interpretation. 

\subsection{Lexical relatedness}	\label{sec:lexrel}

We can now ask what types of systematic relatedness lexemic entries (i.e. lexemes) can exhibit. \citet{Spencer13:book} argues extensively that we can find pretty well all the logically possible types as defined by the very crude but simple artifice of defining non-trivial differences in the four principal attributes, FORM, SYN, SEM, LI. For instance, if we consider  pairs of representations of distinct lexemic entries, \pounds$\sb{1}$, \pounds$\sb{2}$, i.e. those with distinct LIs, then we can identify several different types of relatedness (usually all treated as derivation). \is{derivation}

Suppose that the lexemes \pounds$\sb{1}$, \pounds$\sb{2}$ are distinct in FORM, SYN, SEM representations. Suppose also that the FORM/SEM representations of \pounds$\sb{2}$ subsume or properly include (in some sense) those of \pounds$\sb{1}$. Then we have standard (canonical) derivational\is{derivation} morphology,\is{morphology} \lxm{drive} $\Rightarrow$ \lxm{driver}. %
Languages sometimes define derived lexemes without changing the FORM attribute at all, however. %
A case in point is that of adjectives which are converted wholesale to nouns without any change in \isi{morphology} (\citealt{Spencer02:gender};
see also the discussion of \textit{Angestellte(r)} nouns in \citealt{Spencer13:book}). We will later see examples of \isi{derivation} in which FORM/SYN/LI attributes are changed but without changing the meaning (transpositional lexeme). 

Now let’s consider what happens if we keep the LI constant, that is, we consider intra-lexemic relatedness. To begin with, let us assume that only the FORM attribute can change. In the canonical case this is the same as inflectional\is{inflection} morphology.\is{morphology}  Ignoring for the present the transpositional interpretation of the \isi{participles} as a verbal noun or as verbal adjectives, we can  say that all inflected forms of \lxm{sing} are forms of a verb. We have to be a little cautious when referring to syntactic properties: the syntactic distribution of any given inflected form is, in general, distinct from that of other forms. The 3sg subject agreement form of a verb does not occur in the same syntactic positions as the 3pl form. The properties that give rise to these differences, however, are precisely the MPSs which bifurcate into FORM/CONTENT paradigms.\is{paradigm} This means that we must enrich lexical representations in the obvious way: FORM paradigms\is{paradigm} are defined over features typed as FORM features,\is{features} and CONTENT paradigms\is{paradigm} are defined over features typed as SYN features,\is{features} with the proviso that the two sets of \isi{features} are identical by default. Modulo the CONTENT paradigm,\is{paradigm} then, in canonical \isi{inflection} the SYN value of a given word form is identical to that of the other forms of that lexeme. This means that most inflection is what Booij (\citeyear*{Booij94}) would call contextual inflection.\is{inflection}%
\footnote{This includes Booij’s parade examples of inherent inflection,\is{inflection} \isi{past tense} and \isi{plural} number. See \citet[77--82]{Spencer13:book} for critical discussion of Booij’s distinction.} %

Likewise, the lexeme \lxm{sing} denotes the same event type in all of its inflected forms, and in that sense all word forms share the same SEM representation. In \citet{Spencer13:book} I argue that there are certain types of \isi{inflection} that enrich the semantic representation of the base lexeme, whilst remaining inflectional. Certain kinds of Aktionsart marking, as well as semantic \isi{case} marking often have this characteristic, as do \isi{causative} \isi{argument structure}\is{arguments} alternations.\is{alternation}  In traditional descriptions of languages with such \isi{inflection} we often find terminological vacillation, as linguists are unsure whether to label, say, the iterative form or the \isi{causative} of a verb inflectional\is{inflection} or derivational\is{derivation} (and similar problems afflict evaluative morphology). \is{morphology}

The GPFM descriptive framework proposed in \citet{Spencer13:book} generalizes the PFM model 
so that all forms of \isi{lexical relatedness}, from contextual \isi{inflection} to derivation,\is{derivation} are defined over four principal attributes of a lexical representation. This requires us to generalize the notion of the Paradigm\is{paradigm} Function to that of a Generalized Paradigm\is{paradigm} Function  (GPF), which is like the Paradigm Function except that it consists of  four component functions, \textit{f$\sb{form}$, f$\sb{syn}$, f$\sb{sem}$, f$\sb{li}$}.  For canonical \isi{derivation} the GPF introduces non-trivial changes to all four components (including the LI). For the converted adjectives and \emph{Angestellte(r)} nouns the \textit{f$\sb{form}$} function has no effect (it can be thought of as a kind of identity function). For most inflection,\is{inflection}  the f$\sb{syn, sem, li}$ functions are the identity function, because the GPF simply realizes inflectional properties of the lexeme at the FORM level.  
   
In the GPFM model the lexemic representation is defined in terms of the Lexemic Index and a completely underspecified (empty) feature\is{features} set, \emph{u}, for example, $\langle$\lxm{put}, \emph{u}$\rangle$,  a special case of the GPF.  %
This maximally underspecified GPF defines just those properties of a lexeme (identified by its Lexemic Index) that are completely idiosyncratic and completely unpredictable. However, although such a representation reflects the traditional notion of a maximally compact, non-redundant  dictionary entry, it is not a representation that can serve as the direct input to rules of inflection.\is{inflection} This is because the lexemic entry has to be specified for those inflectional properties that it can and must inflect for. This set is defined by the morphological\is{morphology} signature, MORSIG. In \citet[199]{Spencer13:book} I make this rather obvious point explicit in the \emph{Inflectional Specifiability Principle}.  In the current context this can be stated as follows: a lexeme is inflected for a given MPS, iff that MPS is defined in its MORSIG.

In \citet{Spencer13:book} I treat the MORSIG attribute as part of the FORM \isi{paradigm} of a lexeme. However, we know that the FORM and CONTENT paradigms\is{paradigm} of a lexeme can differ substantially. For this reason, it is necessary to enrich the SYN attribute of a lexemic entry with a (possibly distinct) MORSIG attribute. 
The values of the (FORM and CONTENT) MORSIG attribute are for the most part predictable from other aspects of the lexical representation. First, the FORM MORSIG attribute is generally projected from the CONTENT MORSIG and by default the two attributes are identical. Second, to some extent the meaning of the lexeme can determine the content of the MORSIG attribute. Most importantly, however, the MORSIG (which, recall, is essentially a record of the properties for which a lexeme inflects) is largely projectable from various aspects of the SYN attribute, notably the ARG-STR attribute. Thus, a lexeme with the SYN|ARG-STR value \lab E\lab\ x, \ldots\rab\rab\ (i.e. a verb) will by default have the syntax and \isi{morphology} of a verb. The lexemic representation needs to be enriched to include purely idiosyncratic information, such as irregular \isi{stem} forms, irregular inflections,\is{inflection} defective cells or subparadigms,\is{paradigm} and so on.\is{irregularity} Technically, this can easily be achieved in GPFM by defining very specific functions for particular properties over the LIs of the lexemes concerned. For instance, the irregular \isi{past tense} of \lxm{put} can be defined by a function  defining the  STEM\textsubscript{pst}\is{stem} form for the pairing \lab\lxm{put}, \featval{u}\rab: GPF(\lab\lxm{put},  \{STEM\textsubscript{pst}|PHON\}\rab) = /pʊt/ or similar. This will override any less specific (in practice, any other) statement of \isi{past tense} morphology.\is{morphology} Similarly, a defective lexeme such as \lxm{forgo} (lacking a \isi{past tense} form) will have a tense-specific\is{tense} GPF(\lab\lxm{forgo}, \{\featname{tense}:pst \}\rab) = \textsf{undefined}. This again will override any other statement, including the GPF(\lab\lxm{go}, \{\featname{tense}:pst \}\rab) = /wɛnt/, which applies to one other verb based on \lxm{go} (cf \textex{underwent}) and hence is less specific.

The role of the MORSIG attribute can be simply illustrated by the \ili{English} \isi{plural}. Any lexeme with the SYN|ARG-STR|\lab R \rab\ value licenses MORSIG|\featname{num}:\{sg, pl\}, provided that its SEM attribute specifies it as a count noun. For a mass noun the MORSIG value will be just \featname{num}:\{sg\}, while for a plurale tantum noun it will be \featname{num}:\{pl\}. The \{pl\}, resp. \{sg\} values for such nouns are therefore undefined, so that a GPF(\lab\lxm{sincerity}, \{\featname{num}:pl\}\rab) or %
GPF(\lab\lxm{scissors}, \{\featname{num}:sg\}\rab) will not correspond to any legal output.

\largerpage[-1]%longdistance
In summary, I assume a representation for a dictionary entry of a very traditional kind: it is minimally redundant, specifying just the unpredictable phonological and semantic information, together with any \isi{morphosyntactic} information that cannot be projected from the phonological and semantic specifications. 
 For an entirely well-behaved lexeme belonging to the default \isi{inflection} class for that word category, this is all the information that is required, but that is only because the morphological signature can be projected from that entry too. 


To all intents and purposes GPF collapses with PFM for most cases of inflectional\is{inflection} morphology.\is{morphology} However, the model is designed to cover all types of \isi{lexical relatedness}, up to regular derivation,\is{derivation} but using essentially the same machinery as PFM. For instance, for the \isi{derivation} of a Subject Nominal such as \lxm{driver} from \lxm{drive} we would have the partial GPF shown in (\ref{driver}), where $\mathcal{V}$ is the Lexemic Index of a verb lexeme and δ is the derivational\is{derivation} feature\is{features} which defines the Subject Nominal formation process.

%\noindent\begin{minipage}{\linewidth}
\begin{exe}
\ex	\label{driver}

	\begin{xlist}

\ex	f$\sb{li}$($\langle$$\mathcal{V}$, δ$\rangle$) = Δ($\mathcal{V}$), where Δ is a function over LIs corresponding to the derivational feature δ

\ex	f$\sb{sem}$($\langle$$\mathcal{V}$, δ$\rangle$) = 	 [\mbox{\textit{Thing}} λx, PERSON(x) $\wedge$ $\mathcal{P}^{\prime}$(x)],  where $\mathcal{P}^{\prime}$ is a suitable form of the semantic representation of the lexeme $\mathcal{V}$
	
\ex	f$\sb{syn}$($\langle$$\mathcal{V}$, δ$\rangle$) = \featval{u}

	

\ex	f$\sb{form}$($\langle$$\mathcal{V}$, δ$\rangle$) = Z$\oplus$er, where Z is the δ-selected stem form of $\mathcal{V}$

	\end{xlist}
\end{exe}%\end{minipage}\bigskip

\noindent
This corresponds to a novel lexemic entry which is exactly like that in (\ref{driver}) except that it is  defined over the pairing $\langle$Δ$(\mathcal{V}),\textit{u}\rangle$. If \lxm{drive} is the LI of the verb \lxm{drive}, then Δ$(\mathcal{V})$ is the LI of the derived lexeme \lxm{driver} and   $\langle$Δ$(\mathcal{V}),\textit{u}\rangle$ defines its lexemic entry.

The representation in (\ref{driver}) lacks any syntactic specification. This is because that specification can be given by a default mapping from the SEM attribute, by virtue of the Default Cascade (\citealt{Spencer13:book}: 191\textendash 194). Under that principle,  regularly derived lexemes have their principal \isi{morphosyntactic} properties projected from their semantic representations, in accordance with the notional model of parts of speech. Thus, \lxm{driver} is of ontological category \emph{Thing} and therefore by default has the semantic function role R. I assume that the SEM attribute includes an indication that the lexeme denotes a countable \emph{Thing} so that the lexeme licenses the full MORSIG|NUM:\{sg, pl\}. However, the lexeme in (\ref{driver}) is derived, not simplex. In \citet{Spencer13:book} I propose a Category Erasure Principle, under which the \isi{morphosyntactic} properties of a base lexeme in \isi{derivation} are deleted so that they can be overwritten by the Default Cascade. However, given maximally underspecified lexemic entries to start with this is probably not necessary: the word formation rule interpretation of the GPF takes the only information it has available in a lexical entry, that is the \isi{phonology} of the \isi{root} and the meaning, and modifies it, say, by adding an affix,\is{affix} and by enriching the SEM representation systematically, for instance, by adding a predicate. The Default Cascade then specifies the underspecified properties, including the two MORSIG representations. If the base lexeme’s entry includes non-default specifications such as irregular\is{irregularity} inflected forms or non-standard argument\is{arguments} realization, these will override the Default Cascade. Illustrative examples of simple \isi{inflection} and \isi{derivation} are provided in the Appendix.

\largerpage[-1]
I have sketched the GPFM approach to \isi{inflection} on the one hand, and standard \isi{derivation} on the other hand. Non-standard types of \isi{derivation} are handled by overriding some of the defaults in the GPF. However, this still  leaves us with the somewhat intriguing, but very widespread set of relatedness types in which FORM/SYN attributes are altered, much as in derivation,\is{derivation} but without changing either the SEM representation or, crucially, the LI attribute, one of the class of relatedness types often called a ``transposition''. The parade example of this type is, perhaps, the deverbal participle, the ‘adjectival representation’ of a verb.
 
Transpositions pose particular difficulties for a simple interpretation of the inflection-derivation\is{inflection}\is{derivation} divide (and, indeed, for any architecture with the equivalent of \isi{split morphology}):\is{morphology} on the one hand, a participle is generally taken to be ‘a form of’ the base verb (regular \isi{participles} are never given their own entry in a traditional dictionary, for instance). On the other hand, a participle is morphologically and syntactically a kind of adjective. If \isi{morphology} is split, on which side do transpositions fall?

\section{Paradigm Linkage: the problem of transpositions} \label{section:linkage}


\citet{Spencer:Stump13:Hungproncase} %
\parencite[see also][]{Bonami:Stump16:PFM,Stump16:book} %
\is{paradigm|(}describe a variant of the Paradigm Function Morphology\is{morphology} model, PFM2, which explicitly distinguishes two types of paradigm. FORM paradigms determine the mapping between MPS’s and the word forms realizing them; CONTENT paradigms determine the set of grammatical distinctions a lexeme needs to make in a syntactic representation. The two paradigms are related by rules of paradigm linkage. In the default case the two paradigms align perfectly: a morphological \isi{plural} noun is syntactically \isi{plural} and vice versa. However, there are numerous mismatches. A simple case in point is provided by \ili{English} perfect and \isi{passive} \isi{participles}. These have distinct syntax (they collocate with different auxiliaries, and only the \isi{passive} participle can be used attributively), but they are never distinguished morphologically. Thus, the (single) FORM property %
\parencite[say, VFORM:en-ptcp, or the output of a function f\textsubscript{\textit{en}}, as in ][]{Aronoff94:book} % 
has to map to two CONTENT properties. \citet{Stump16:book} provides an extensive survey of the principal types of mismatch encountered in the world’s inflectional\is{inflection} systems.

I will argue that we can regard the FORM/CONTENT paradigm distinction as a reflex of \isi{split morphology}\is{morphology} for \isi{constraints-based}\is{constraint} lexicalist models of syntax. The reasoning is very simple \textemdash\ no corresponding FORM/CONTENT distinction can be mandated for derivation.\is{derivation} Indeed, given standard PFM2 assumptions it is difficult to imagine what such a distinction could mean.

\largerpage[-2]
Incorporating the FORM-CONTENT paradigm distinction into GPFM has consequences for the way in which lexical representations are organized. We have seen that  the FORM attribute of a lexical representation is defined in part in terms of its MORSIG, which determines precisely those properties a lexeme must inflect for. In the original GPFM model the MORSIG attribute is only defined at the level of FORM paradigms, therefore. Since CONTENT paradigms are not always congruent to FORM paradigms we will need to draw the appropriate distinction in lexical representations. The obvious way to do this is to assume that each lexeme is associated with \emph{two} (at least) MORSIG attributes, one of which is a value  of the FORM attribute and the other of which is a value of the SYN attribute. We can call these f-MORSIG, c-MORSIG respectively.%
\footnote{A number of authors have proposed that cells in a lexeme’s paradigm can be realized by multiword, periphrastic, constructions \parencite{Sadler:Spencer01,Brown:etal12:periphrasis,Bonami15:collocation}. \citet{Popova:Spencer17:FDSL11}, following suggestions by \citet{Bonami15:collocation}, propose that such constructions demand additional CONTENT feature\is{features} sets specifically to define the content of the periphrastic expression, which is often at odds with the default feature\is{features} content of the words which make up that expression. Periphrasis therefore provides substantial  motivation for a FORM/CONTENT or m-/s-feature\is{features} \parencite{Sadler:Spencer01} distinction.%
} %
 

\citet[253]{Stump16:book} defines paradigm linkage in terms of the Paradigm Function and a \textbf{\textit{Corr}} function. Given a  \isi{lexemic index} \pounds, a set of MPSs σ, τ,  and a \isi{stem} form, Z, the function \Corr{ \pounds,σ} delivers a form correspondent $\langle$Z,τ$\rangle$, such that if \PF{Z,τ} = $\langle$w,τ$\rangle$, then \PF{\pounds,σ} = $\langle$w,τ$\rangle$. By default the FORM and CONTENT \isi{features} are identical, that is the set σ = τ, as defined by a set of property mappings (\textbf{\textit{pm}}), that is, 
\textbf{\textit{pm}}$(\sigma)$ = σ. However, the function \textbf{\textit{pm}} is called upon whenever there is a mismatch between FORM and CONTENT properties. This is the case, for instance, with  deviations such as \isi{syncretism}, deponency and so on. In the case of active$\sim$passive deponency the property mapping maps the morphological \isi{passive} voice forms to the syntactic active paradigm and leaves the morphological active voice forms undefined. 

We can ask how  paradigm linkage would work for derivation,\is{derivation} by taking the example of a derivational feature, δ, such as the privative denominal adjective feature, \textit{privadj}, which derives \textex{friendless} from \textex{friend}, as described in Stump (2001:252--60).\is{features}%
\footnote{Stump does not discuss derivation,\is{derivation} or, indeed, transpositions in his works on paradigm linkage.} %
Here, we are  dealing with trivial (two-celled) paradigms, so that the featural mismatches of inflectional\is{inflection} paradigms will not be found, and we can work with just a single derivational feature, δ. 

\begin{exe} \ex \label{friendless}

Let \Corr{\lxm{friend}, \textit{privadj}} be the correspondence function for \textit{privadj} applied to the lexeme \lxm{friend}. 

\Corr{\lxm{friend}, \textit{privadj}} = $\langle$Z,δ$\rangle$, where Z = |friend|, the stem of \lxm{friend}, and δ = \textit{privadj}.  If \PF{Z,δ} = $\langle$w,δ$\rangle$, then \PF{\pounds,δ} = $\langle$w,δ$\rangle$, hence, \PF{\lxm{friend}, \textit{privadj}} = $\langle$friendless, \textit{privadj}$\rangle$. 
\end{exe}
This seems very straightforward, but there is a hidden difficulty, not immediately apparent from a language like \ili{English}, with limited inflection.\is{inflection} Consider a hypothetical language just like \ili{English} but in which nouns and adjectives have entirely different inflectional\is{inflection} paradigms, or, indeed, consider a derivational\is{derivation} process such as that of Subject Nominalization which derives \lxm{driver} from \lxm{drive}. What we have to ensure is that the output lexeme is (automatically, by default) inflected as a noun, as opposed to the base lexeme which is inflected as a verb. As it stands, the Paradigm Function applied to a pairing of Lexemic Index and derivational\is{derivation} feature\is{features} will not deliver what Stump calls the realized paradigm of the output lexeme. At best, it might define the (or an) uninflected \isi{stem} form of the derived word. Moreover, without significantly altering the nature of the \textbf{\textit{Corr}} function it will be impossible to define additional lexical information such as inflectional class membership, idiosyncratic \isi{stem} forms or other deviations from default inflection,\is{inflection} or indeed any non-default or purely morphological\is{morphology} property of the output.%
\footnote{It is also not clear how the additional semantic predicate of the output would be defined.}%

A direct solution to this problem might be to enrich the content of the derivational\is{derivation} feature,\is{features} δ, so that it incorporates the full set of paradigmatic oppositions realizable by the derived lexeme. Thus, the Subject Nominal feature\is{features} (say \textit{subjnom}) could be defined as the complex [\textit{subjnom}, NUMBER:α]. This would mean that we would have to define the Paradigm Function so that it defines each inflected form of the output lexeme, rather than defining an underspecified lexemic entry, which then gets inflected by the standard inflectional\is{inflection} rules applying to words of that category. For instance, the Paradigm Function could take the form \PF{\lxm{drive},\{\textit{subjnom}, NUMBER:sg\}} = $\langle$driver, \{\textit{subjnom}, NUMBER:sg\}$\rangle$, \PF{\lxm{drive},\{\textit{subjnom}, NUMBER:pl\}} = $\langle$drivers, \{\textit{subjnom}, NUMBER:pl\}$\rangle$. We could call this approach the ‘full-listing approach’.


Now, the full-listing approach would have the rather peculiar consequence that the word forms \{\textex{driver, drivers}\} would be inflected forms of the verb \lxm{drive}, and there would be no such thing as a \lxm{driver} lexeme. Not only is this entirely counter-intuitive, and at variance with any sensible distinction between \isi{inflection} and derivation,\is{derivation} it becomes even more counter-intuitive when we see recursive derivation.\is{derivation} It would entail that the noun \textex{reprivatizability} was an inflected form of the adjective \textit{private}. The problem is that the \textbf{\textit{Corr}} function needs to be able to define not a cell in a realized paradigm (what syntacticians  refer to, misleadingly, as the lexical entry of a word(form)), but it has to define a featurally underspecified lexemic entry (an autonomous dictionary entry in traditional terms). I therefore reject the full-listing approach in favour of that adopted in the GPFM model. 

 We are now in a position to examine the case of transpositions. The problem we must address is how to ensure that the transposition is assigned to its own inflectional\is{inflection} paradigm, proper to its new \isi{morphosyntactic} category, whilst still in some sense remaining part of the inflectional\is{inflection} paradigm of the base lexeme. %
 Recall that I have proposed adopting a class of features,\is{features} [REPRESENTATION:\{\ldots\}] to define the paradigm space of a transposition.  I now consider the way this feature\is{features} can be deployed to define the inflectional\is{inflection} paradigm-within-a-paradigm of a transposition. %
Let ρ = [\featname{REPR}:κ] for some transpositional relation κ, e.g. a verb-to-adjective (participle) transposition. I assume that the GPF applied to the pairing $\langle$\pounds,$\rho$$\rangle$ defines a partially specified representation for the transposition. Normally, when the Paradigm Function applies to a pairing of LI and MPS the MPS has to represent a complete and coherent set of \isi{features} sufficient to define the inflected form. However, this is not strictly speaking a property of the PFM system itself. In principle, we could define a partial paradigm for a lexeme by reference to just a proper subset of the \isi{features} required to define any fully inflected word form. For instance, suppose that verbs in a language inflect for a variety of tense-aspect-mood-voice\is{tense} (TAMV) series, and that in addition they show subject agreement. We could, in principle, define the \isi{stem} sets for the TAMV categories independently of the agreement morphology,\is{morphology} by simply not specifying the subject agreement properties, effectively defining a set of ``screeves'' for the language \parencite[as in the Georgian descriptive tradition, ][141\textendash 45]{AndersonSR92:book}. This makes use of the same notion of feature underspecification used in GPFM to define derivation,\is{derivation} but relativized to a specific feature set. \is{features}

The GPF for a transposition has to specify the derived \isi{morphosyntactic} category, κ, and, by default, the f-/c-MORSIG attribute, associated with that category, together with additional morphological\is{morphology} properties inherited from the base, such as TAMV properties. If the Paradigm Function were to apply in the same way for transpositions as for other inflected forms then the \Corrfn\ function would have to provide the form correspondents for each inflected form, but it would not be able to provide a lexemic entry for the uninflected transposition. In other words, standard application of PFM2 principles would give rise to a full listing interpretation of the paradigm. It would not be possible to provide any characterization of the transposition as a quasi-lexeme (see p.~\pageref{quasilexeme}). %p29
This generates many of the same conceptual and technical problems as the application of \Corrfn\ to derivation.\is{derivation} %
One additional consequence  is that  it would be difficult to describe the very common situation in diachronic\is{diachrony} change in which a participle is reanalysed as a qualitative adjective, often without significantly changing the lexical \isi{semantics} of the original verb lexeme (transpositional lexeme). If we treat the participle as akin to a lexeme, then we can easily define the diachronic\is{diachrony} \isi{reanalysis} over that representation, just as we would for ordinary derivational\is{derivation} conversion. 
One of the problems with the full-listing approach is that it is not clear how we could  permit the transposition to inherit  MORSIG properties of the derived  category (Adjective in the case of \isi{participles}). We therefore need to define the GPF in such a way that it allows us to define a quasi-lexeme as part of the paradigm of the base.\is{paradigm|)}

First note that application of the Default Cascade to define derived categories is entirely excluded in the case of (pure) transpositions, since these preserve the meaning, and hence the ontological category, of their base; indeed, this is the whole point of a transposition. The most natural assumption is that  the transpositional morphosyntax effects a shift in the syntactic categorization and that morphological\is{morphology} recategorization falls out as a consequence. (In fact, the extent to which the transposition acquires derived categorial properties and loses those of its base is subject to a good deal of cross-linguistic variation,\is{variation} tempered by poorly understood typological tendencies. See \citet{Malchukov04:book} for discussion in the context of  action nominal transpositions.) This is effectively a weaker instantiation of the Default Cascade.  Following \citet{Spencer99:transpositions,Spencer13:book}, I will assume that the shift in syntactic representation is actually a modification of the \isi{argument structure} representation; %
specifically, the definition of a complex sf role (see below). %
 The crucial point is  that the transpositional GPF changes the SYNCAT value of the base lexeme to that of a mixed category and  this, ceteris paribus, will automatically entrain a shift in the morphological\is{morphology} category and hence the MORSIG attributes. 

The basic machinery is only hinted at in \citet[Chapter ten]{Spencer13:book}. In order to develop an explicit account we first need to refine the definition of the REPR(ESENTATION) attribute. I will therefore assume that the REPR feature\is{features} takes an ordered pair as value, not a singleton element, [REPR:$\langle$κ,λ$\rangle$], where κ, λ range over lexical categories. Thus, a participle represents a verb as an adjective and hence  bears the specification [REPR:$\langle$V,A$\rangle$], while a predicatively used noun will have the specification [REPR:$\langle$N,V$\rangle$]. We then couple these feature\is{features} values to syntactic category specifications in the obvious way, stated  informally in (\ref{morcat}).

\begin{exe}	\ex	\label{morcat}

Given [REPR:$\langle$κ,λ$\rangle$], for λ = N, V, A. 

Then SYNCAT = λ,

and by default, MORCAT = λ
\end{exe}


In  the GPFM framework the [REPR] attribute will be associated with an appropriate enrichment of the semantic function role of the ARG-STR attribute. A  verb base ARG-STR includes E, the event semantic function role, $\langle$E, $\langle$x,\ldots$\rangle\rangle$. That of the participle is enriched by addition of the A semantic function role, to  become (simplifying somewhat) $\langle$A$\sb{i}$$\langle$E, $\langle$x$\sb{i}$,\ldots$\rangle\rangle\rangle$, 
where the co-indexing indicates that the element of which the  participle is predicated is an element of the argument\is{arguments} array of the base verb lexeme (the highest such argument\is{arguments} for \ili{Indo-European} type \isi{participles}, though not necessarily so in other languages). 
This representation reflects the mixed categorial nature of the participle. While its main external syntax \parencite{Haspelmath96} is now that of an adjective, it retains the eventive ARG-STR  of the base verb, which permits it, on a language specific basis, to realize a number of verb properties. These typically include the realization of internal \isi{arguments} as verb dependents (complete with quirky \isi{case} marking). The E semantic function role also permits modification by adverbials targeting event \isi{semantics}. 

Exactly which adjectival properties are acquired and, more crucially for \isi{participles}, which verb properties are lost or retained, differs cross-linguistically. The \ili{Indo-European} participle, for instance, typically retains the active$\sim$passive\is{passive} alternation,\is{alternation} but may also retain \isi{aspect} (Slavic) or \isi{tense}  \parencite[\ili{Sanskrit}; ][]{Lowe15:book}. As attributive modifiers, the  \ili{Indo-European} \isi{participles} can only modify a noun which expresses the participle’s highest (subject) argument,\is{arguments} effectively making them into heads of subject-oriented relative clauses, but in other languages there is much greater freedom in the choice of argument\is{arguments} that can be relativized on by the participle %
\parencite[for discussion see][]{Spencer16:ptcprels}. 

In \S\ref{sec:Russparalink} ‘Paradigm linkage rules for \ili{Russian}’ I present a more detailed analysis of \ili{Russian} \isi{participles} in which this basic schema for transpositions is expanded upon.

\section{\ili{Russian} participles}  \label{sec:Russianptcps}


In this section we look at the set of four \isi{participles} that are regularly associated with \ili{Russian} verbs. Before we can consider these, however, we need to understand  \ili{Russian} verb \isi{inflection} and the place of the \isi{participles} in that system. I begin with an overview of the grammatical distinctions as a whole made by verbs, in other words, the CONTENT paradigm,\is{paradigm} before considering the actual morphological forms themselves. We encounter the familiar problem that there is no consensus on  just what the oppositions are and how they relate to each other, and so some of my descriptive decisions will be motivated in part by expositional convenience. I illustrate with the second conjugation transitive\is{transitivity} verb \lxm{udarʹitʹ} ‘hit’, whose \isi{imperfective} \isi{aspect} series is formed by shifting to the first conjugation, \lxm{udarʹatʹ}.  

%
The simplest categories are the infinitive and  \isi{imperfective} (``present'') and perfective (``past'') gerunds. These are indeclinable. Telic verbs such as \lxm{udarʹitʹ/udarʹatʹ}, ‘hit’, require \isi{imperfective} and perfective \isi{aspect} forms for most of their paradigm.\is{paradigm} We can distinguish three moods, indicative, imperative, conditional. The imperative is straightforward and I will ignore it here. The indicative distinguishes three tenses,\is{tense|(} present (for im\isi{perfectives} only), past, future. I return to the conditional below.
%

Transitive\is{transitivity} verbs show an active$\sim$passive\is{passive} voice opposition. However, the voice system is complicated by the fact that \isi{imperfective} verbs are able to take the reflexive suffix\is{suffixation} \textex{\nobreakdash-sʹa/sʹ}. This has the basic function of giving a reflexive/reciprocal meaning (though similar meanings can also be expressed with fully-fledged reflexive/reciprocal pronouns). The reflexive form also has a wide variety of other uses, including the \isi{passive} \parencite{Gerritsen90:book}. This means that we should set up a reflexive voice category and define the \isi{imperfective} \isi{passive} in terms of this, but I set this task aside since it is not directly relevant. The perfective verbs express the \isi{passive} \isi{alternation} periphrastically, with BE + perfective \isi{passive} participle. At the level of the CONTENT \isi{paradigm} we need to be able to define [VOICE:\{act,pass\}] for both perfective and \isi{imperfective} verbs, therefore.  

Present tense is expressed morphologically, but only for \isi{imperfective} verbs. There is no dedicated tense marker, and in effect, present tense is realized by the person/number subject agreement morphology.\is{morphology} The future tense is expressed periphrastically by \isi{imperfective} verbs, by means of BE + the infinitive form. For perfective verbs the future is expressed by a \isi{paradigm} which is essentially the same as the present tense \isi{paradigm} for \isi{imperfective} verbs. Thus, the present tense  of the \isi{imperfective} verb \lxm{pʹisatʹ} ‘write’ is \textex{pʹišu, pʹišeš,\ldots} and the  future tense  of the prefixed perfective verb form \lxm{napʹisatʹ} is \textex{napʹišu, napʹišeš,\ldots}.

One of the main challenges of the \ili{Russian} verb system is the representation of the \isi{past tense}. This is derived historically from a periphrastic perfect tense series formed by  BE and a resultative participle expressed by a suffix\is{suffixation} \textex{-l}, the l-participle. Syntactically, the l-participle behaved like a predicative adjective, agreeing with the subject in number and gender, but not in person. The auxiliary BE was lost, leaving the l-participle and its adjective-like agreement as the sole exponent of \isi{past tense}. The  agreement inflections\is{inflection} on an l-participle such as \textex{(na)pʹisal} ‘wrote’ are almost identical to those on a predicative (short-form) adjective such as \lxm{mal} ‘short’:  \textex{(na)pʹisal, (na)pʹisala, (na)pʹisalo, (na)pisalʹi} vs \textex{mal, mala, malo, maly}. The only real difference is that in the \isi{plural} the l-participle stem is palatalized but not the predicative adjective stem: \textex{(na)pʹisalʹi} vs \textex{maly}. 

The simple way of analysing the \isi{past tense} construction would be to take the \textex{-l} formative to be an exponent of \isi{past tense} and define two distinct sets of subject agreement rules for present/future and past tenses. However, the l-participle has one other significant usage which precludes this direct analysis. The conditional mood is expressed by means of the invariable particle \textex{by}, a kind of freely distributed en\isi{clitic} (it can occur anywhere in the clause except absolute \isi{initial position}). This can co-occur with the infinitive, (\ref{skazatby}). 

% \begin{minipage}{\linewidth}
\noindent\begin{exe} \ex \label{skazatby}

\gll	Eslʹi by skazatʹ pravdu, \ldots\\
	if BY say.\glossfeat{inf} truth\\
      \glt `To be honest, \ldots’

\end{exe}
% \end{minipage}\bigskip
 
\noindent However, it is much more often found with the l-participle, (\ref{skazalby}). 

% \begin{minipage}{\linewidth}
\noindent\begin{exe} \ex \label{skazalby}

\gll	Eslʹi by ty skaza-l pravdu, \ldots\\
	if BY you say-\glossfeat{l-ptcp} truth\\
\glt	‘If you told/were to tell/had told the truth, \ldots’
\end{exe}
% \end{minipage}\bigskip
 
\noindent
As is indicated in the gloss, the conditional is tenseless, and serves as the translation equivalent of past or non-past conditional in \ili{English}.\is{tense|)}
The existence of the conditional construction means that we cannot regard the l-participle simply as the exponent of \isi{past tense}. In fact, it is an instance of what, since \citet{Aronoff94:book} morphologists have called a ``morphome'', \is{morphome}
that is,  a pure morphological form, which serves as a \isi{stem} for building up inflected word forms, but which realizes no MPSs on its own and whose distribution is not mandated by any semantic, phonological or other non-morphological properties.\is{morphology} The CONTENT \isi{paradigm} for \ili{Russian} is summarized in Table~\ref{table:RussCONTENT}, but ignoring the true \isi{participles}. 

\begin{table} [h]
	
		\begin{tabular}{p{1em}p{3.5em}ll} \lsptoprule
\multicolumn{2}{l}{ASPECT}		&\isi{imperfective}					&perfective			\\\cmidrule(rl){3-4}
\multicolumn{2}{l}{INFINITIVE}		&udarʹa-tʹ 						&udarʹi-tʹ 			\\
\multicolumn{2}{l}{GERUND}		&udarʹa-ja						&udarʹi-v(ši)		\\
\multicolumn{2}{l}{IMPERATIVE}	&udarʹa-j(te)!				&udarʹ(te)!			\\
\multicolumn{2}{l}{TENSE}													\\
	&present	&udarʹa-ju, -eš, \ldots				&<none>			\\
	&future	 &bud-u, -eš, \ldots udarʹatʹ 			&udarʹ-u, -iš		\\
	&past		&udarʹa-l, -a, -o, -ʹi				&udarʹi-l, -a, -o, -ʹi		\\
\multicolumn{2}{l}{CONDITIONAL}	&udarʹa-l, -a, -o, -ʹi + by			&udarʹi-l, -a, -o, -ʹi + by	\\
\multicolumn{2}{l}{PASSIVE}	&udarʹatʹ-sʹa, etc					&(byl) udaren, -a, -o, -y	\\	\lspbottomrule
		\end{tabular}

\caption{\ili{Russian} verb CONTENT paradigm for \lxm{udarʹitʹ/udarʹatʹ} ‘hit’}
	\label{table:RussCONTENT}

	
\end{table}


%%%%%%%%%%%%    MORPHOLOGY	%%%%%%%

I turn now to the morphological\is{morphology} or FORM paradigm.\is{paradigm} %
We can divide \ili{Russian} verb forms into five groups. The first is the infinitive and the second the set of two indeclinable \isi{gerund} forms. The third is the set of finite forms, including the imperative mood. In practice, these are limited to the present (non-past) forms showing subject agreement in person/number. The fourth is the l-participle. Finally, we have  the set of (declinable) active and \isi{passive} \isi{participles} discussed earlier. These are tabulated in Table~\ref{table:RussFORM}.

\begin{table} [h]
	
		\begin{tabular}{p{1em}p{6em}ll} 	\lsptoprule
\multicolumn{2}{l}{ASPECT}		&\isi{imperfective}					&perfective			\\ \cmidrule(rl){3-4}
\multicolumn{2}{l}{INFINITIVE}		&udarʹa-tʹ 						&udarʹi-tʹ 			\\
\multicolumn{2}{l}{GERUND}		&udarʹa-ja						&udarʹi-v			\\
\multicolumn{2}{l}{IMPERATIVE}	&udarʹa-j(te)!				&udarʹ(te)!			\\
\multicolumn{2}{l}{TENSE}													\\
	&present/future	&udarʹa-ju, -eš, \ldots				&udarʹ-u, -iš			\\
\multicolumn{2}{l}{L-PTCP}	&udarʹa-l, -a, -o, -ʹi				&udarʹi-l, -a, -o, -ʹi		\\
\multicolumn{2}{l}{REFLEXIVE}		&udarʹatʹ-sʹa, etc					&<none>	\\	\lspbottomrule
		\end{tabular}

\caption{\ili{Russian} verb FORM paradigm for \lxm{udarʹitʹ/udarʹatʹ} ‘hit’}	\label{table:RussFORM}

	
\end{table}

Summarizing Table~\ref{table:RussFORM}, the verb can/must inflect for \isi{aspect}. The verb system also  shows voice alternations.\is{alternation} However, these are effected either through  reflexive \isi{morphology} (\isi{imperfective} verbs) or periphrastically (perfective verbs) so voice proper lacks a dedicated, purely morphological exponent, except for the true \isi{participles}. The infinitive, gerunds and the l-participle do not show any \isi{tense} oppositions. The subject agreement shown by l-\isi{participles} differs from that of finite verb forms in that it is defined in terms of MPSs proper to predicative adjectives, not those of finite verbs.

\begin{table}[ht]

\begin{tabular}{lll}\lsptoprule

ASPECT		&\{ipfv,pfv\}\\
VFORM	&INFINITIVE\\
		&TENSE:\{prs, fut, pst\}\\
		&IMPERATIVE:\{sg, pl\}\\
		&CONDITIONAL:\{yes, no\}\\
REFLEXIVE	&\{yes, no\}\\
AGRSUBJ	&PERSON:\{1, 2, 3\}\\
		&NUMBER:\{sg, pl\}\\
		&GENDER:\{m, f, n\}\\	
VOICE	&\{ACTIVE, PASSIVE\}\\ 
REPR	&\{\lab V,A\rab, \lab V,Adv\rab\}\\
\lspbottomrule

\end{tabular}

\caption{CONTENT feature array}
\label{table:fsets:CONTENT}
\end{table}%
  
   

In  Tables  \ref{table:fsets:CONTENT}, \ref{table:fsets:FORM} I provide a summary list of the \isi{features} which populate the CONTENT and FORM paradigms.\is{paradigm} %
I have provided only basic labels for the various MPSs. Ideally, we would want to know how they are grouped together, if at all, in the two paradigms.\is{paradigm} This is a difficult question, and I finesse it by just assuming what is effectively a list structure for the MPSs, with a number of dependency statements between them. Thus, I have not distinguished, say, a finite from a non-finite set of forms or constructions. However, for the purposes of giving a broad-brush characterization of the morphology$\sim$syntax\is{morphology} mapping this is probably not a problem. 
 

\begin{table}

\begin{tabular}{lll}	\lsptoprule
ASPECT		&\{ipfv,pfv\}\\		
VFORM	&INFINITIVE\\
		&TENSE:\{prs-fut\}\\
		&IMPERATIVE:\{sg, pl\}\\
		&L-PTCP\\
REFLEXIVE	&\{yes, no\}\\
AGRSUBJ	&PERSON:\{1, 2, 3\}\\
		&NUMBER:\{sg, pl\}\\
		&GENDER:\{m, f, n\}\\	
REPR	&\{\lab V,A\rab, \lab V,Adv\rab\}\\
\lspbottomrule
\end{tabular}
\caption{FORM feature array} 
\label{table:fsets:FORM}

\end{table}

\largerpage
From this overview we can see that there is a very clear divide between the CONTENT \isi{paradigm} MPSs required to describe the system as a whole and the FORM \isi{paradigm} MPSs required to describe the individual word forms. These tables ignore the inflectional paradigms\is{paradigm} of the \isi{participles}, of course, but adding them will just serve to emphasize the CONTENT$\sim$FORM disparity.

\ili{Russian} has four \isi{participles}, active$\sim$passive\is{passive} and perfective$\sim$\isi{imperfective}, which are typical attributive modifiers with the agreement morphosyntax of standard adjectives. %
However, they retain a variety of verb properties, making them into typical examples of mixed categories. Thus, in  (\ref{upravljajushchego}),  the \isi{imperfective} active participle  \textex{upravlʹajuščij} takes a temporal PP adjunct and assigns instrumental \isi{case} to its complement,\is{complement} just like the finite verb (\ref{upravljaet}).

\begin{exe}	\ex	\label{upravljajushchij}
	\begin{xlist}
\ex	\label{upravljajushchego}\gll	(čelovek-a), upravlʹa-jušč-ego v tečenʹie mnogo let mestnoj školoj\\
		(the.person\glossfeat{[m]-gen.sg}) run-\glossfeat{prsptcp-m.gen.sg}	in course many of.years local.\glossfeat{instr}  school.\glossfeat{instr}\\
\glt		‘of (the person) (who has been) running the local  school for many years’
\ex	\label{upravljaet}\gll	Ivanov upravlʹaet v tečenʹie mnogo let mestnoj  školoj\\
		Ivanov runs	in course many of.years local.\glossfeat{instr}  school.\glossfeat{instr}\\
\glt		‘Ivanov has been running the local primary school for many years’
	\end{xlist}
\end{exe}

The \isi{participles} are often described as present/\isi{past tense} forms, but their \isi{semantics} is essentially aspectual and they fit somewhat better into the overall verb scheme if they are treated as perfective$\sim$\isi{imperfective} pairs. They are summarized in Table~\ref{table:upravitptcps} \parencite[cf][361]{Wade92:book}.

\begin{table}[h]
	
		\begin{tabular}{lll}\lsptoprule

Aspect	&\isi{imperfective}	&perfective\\  \cmidrule(rl){2-3}

Active	&\textex{upravlʹaju-šč-}	&\textex{upravʹi-vš-}\\

Passive	&\textex{upravlʹa-em-}	&\textex{upravlʹ-on(n)-}\\ \lspbottomrule

		\end{tabular}
	

\caption{\ili{Russian} \isi{participles} of the verb \lxm{upravʹitʹ/upravlʹatʹ} ‘control’}	\label{table:upravitptcps}
\end{table}

\newpage
Perfective \isi{aspect} \isi{participles}, for semantic reasons, usually only have past time reference.%
\footnote{But see the counter-examples in the Academy of Sciences grammar \emph{Russkaja Grammatika, I}: 667, \textex{pred’javlʹajuščij} ‘presenting’, \textex{vzvolnujuščij} ‘exciting’, \textex{sdelajuščij} ‘doing’, \textex{smoguščij} ‘being able’.} %
The \isi{imperfective} \isi{participles} realize a time relative to the main verb of the clause, so that ‘present tense’ is a particularly misleading label for these forms \citep[375]{Wade92:book}. 
 
\begin{exe} \ex	\label{begushchuju}

\gll	Ja vʹidel/vʹižu 	sobak-u, 				bega-jušč-uju {po beregu}\\
	I saw/see		dog\glossfeat{[f]-acc.sg} run-\glossfeat{actprsptcp-f.acc.sg}  {along the shore}	\\
\glt	‘I saw/see the dog  running along the shore’
\end{exe}
The participle differs from the finite form in this respect. Example (\ref{begaet}) would only be possible with either the meaning ‘(dog) which usually runs along the shore’ or as a somewhat marked form of the historic present (cf \emph{Russkaja Grammatika I}: 665).

\begin{exe} \ex	\label{begaet}

\gll	Ja vʹidel sobak-u, kotor-aja begaet {po beregu}\\
	I	saw	dog\glossfeat{[f]-acc.sg} which-\glossfeat{f.nom.sg} runs {along the shore}	\\
\glt	‘I saw the dog which is running along the shore’
\end{exe}
In this respect, \ili{Russian} \isi{participles} are just like their \ili{English} counterparts, of course. 

%
The \isi{participles} have a number of properties aligning them with verbs. In addition to  realizing the purely verbal (eventive) categories of tense-aspect-voice,\is{tense} the active \isi{participles} can take reflexive forms, either as reflexive variants of non-reflexives, inheriting all the \isi{semantics} of the reflexive forms, \textex{upravlʹatʹ(sʹa)} ‘manage, control’ $\sim$ \textex{upravlʹajuščijsʹa}, or as inherent reflexives (with no non-reflexive counterpart), \textex{bojatʹsʹa ‘fear’ $\sim$ bojaščijsʹa}. As we have seen, syntactically, the \isi{participles} retain the verb’s \isi{argument structure},\is{arguments} including quirky \isi{case} assignment to complements,\is{complement} such as instrumental in the case of 
\textex{upravlʹajuščijsʹa} (see examples (\ref{upravljajushchij})) and genitive in the case of \textex{bojaščijsʹa}.
%

%
On the other hand, the \isi{participles} have a number of adjectival properties. The most  salient \isi{morphosyntactic} property is that of attributive adjective agreement together with the morphological property of belonging to a well-defined adjectival inflectional\is{inflection} class. (As attributive modifiers the \isi{participles} can be restrictive or non-restrictive, like other attributive modifiers, including relative clauses.) However, they can also be used as predicates with the copula \lxm{bytʹ} ‘be’ or with ``semi-copulas'' such as \lxm{statʹ } ‘become’, \lxm{kazatʹsʹa} ‘seem’,  \lxm{ostatʹsʹa} ‘remain’, and others. Most commonly it is the perfective \isi{passive} \isi{participles} that can be used as predicates but active \isi{participles} can also be found in this role \parencite[291, \S2346]{RussGramm80-II}. The \isi{passive} (though not the active) \isi{participles} can also appear as predicates, in the so-called short form, a typically adjectival property.%
\footnote%
{Present active \isi{participles} can be used in the short form, however, when they are converted into true, qualitative, adjectives \parencite[666]{RussGramm80-I}.
}
%


\begin{exe} \ex	\label{podan}

\gll	užin uže poda-n\\
	supper.\glossfeat{m.sg}	already 	serve-\glossfeat{passptcp.m.sg}\\
\glt	‘Supper has already been served’
\end{exe}
  
Given this brief descriptive summary of the basic facts of \ili{Russian} \isi{participles} we can turn to the central questions: how do we represent \isi{participles} in a formal, \isi{constraints-based}\is{constraint} grammar with an inferential-realizational morphology?\is{morphology} How do these representations relate to the inflection$\sim$derivation\is{inflection}\is{derivation} divide and the issue of \isi{split morphology}? \is{morphology}

\section{Paradigm linkage rules for Russian} \label{sec:Russparalink}


In the extension to GPFM presented here, the FORM  \isi{paradigm} and the CONTENT \isi{paradigm} are modelled by the attributes f-MORSIG, c-MORSIG. However, those types of \isi{lexical relatedness} which modify the MPSs of a representation, such as transpositions, will ipso facto modify the content of Stump's FORM/ CONTENT paradigms\is{paradigm} and the f-/c-MORSIG attributes. The GPF for such types of relatedness therefore has to specify that novel content, by stipulation, if necessary. In the rules I propose below I show how this can be achieved for \ili{Russian} conjugation. The reference to MORSIG is taken to mean c-MORSIG by default and by default, this is identical to f-MORSIG. 

	
I shall begin by specifying the CONTENT and FORM MORSIG attribute for non-participial verb categories.
We need to define the f-MORSIG in part in terms of the c-MORSIG and in part independently. The default is the identity mapping from c-MORSIG to f-MORSIG. The c-MORSIG list shown earlier in Table \ref{table:RussCONTENT} is defined for any lexical representation whose ARG-STR includes the E semantic function role.  The MPSs that are shared across the CONTENT and FORM paradigms\is{paradigm} of \ili{Russian} verbs are fairly limited (cf. Tables \ref{table:RussCONTENT} and \ref{table:RussFORM} above). They are (ignoring for the present the \isi{participles} and gerunds):  ASPECT:\{ipfv, pfv\}, VFORM:\{INFINITIVE, IMPERATIVE:\{sg, pl\}\}, VOICE:\{act, pass\}, and AGRSBJ:\{PERSON/NUMBER/GENDER\}. 


The c-features ASPECT, VFORM:\{INFINITIVE, IMPERATIVE:\{sg, pl\}\} have relatively straightforward f-feature correspondents.\is{features} I shall ignore INFINITIVE and IMPERATIVE for present purposes. AGRSUBJ is also a FORM property but with some complications and I return to it  when I discuss the l-participle. 

The status of the TENSE\is{tense} feature\is{features} is a little unclear. At the CONTENT level there are clearly three values, \featval{prs, fut, pst}, but only the \featval{prs} and \featval{fut} values have a FORM correspondent, and even then the value of the FORM correspondent is a composite \featval{prs/fut}, and therefore not a direct correspondent of either c-[TENSE:prs] or c-[TENSE:fut]. The c-TENSE:\{pst\} property is realized by the morphomic l-participle, and not by any dedicated f-TENSE:\{pst\} property. I shall therefore assume a univalent FORM property, TNS, itself a value of VFORM, realizing c-TENSE:\{prs, fut\} depending on the value of ASPECT. %
This replaces the atom-valued TENSE:\{present-future\} shown in Table \ref{table:RussFORM} above. %
The property VOICE:\{act, pass\} is intriguing. It is a CONTENT property of the verb system but it is not a FORM property of any part of the verb system proper, outside of the participle subsystem. However, the \isi{participles} distinguish active and \isi{passive} sets, and the perfective passive participle is actually part of the periphrastic \isi{exponence} of the syntactic (CONTENT) VOICE property. Moreover, the two passive \isi{participles} have the \isi{passive} SYN|ARG-STR representation, namely, \ldots\lab E\lab(x), y, \dots\rab, where (x) denotes the demoted active subject argument\is{arguments} role. Therefore, VOICE is both part of the CONTENT and the FORM paradigm,\is{paradigm} albeit with a somewhat complex realization, which will require the f-MORSIG attribute to be modified by a feature\is{features} co-occurrence statement restricting f-VOICE to the \isi{participles}. %
The other CONTENT \isi{paradigm} \isi{features} are represented by forms which are effectively periphrastic. It is for these reasons  that the FORM MPSs have to be defined independently, as shown below. 
 
 One attribute that is shared across FORM/CONTENT paradigms\is{paradigm} is REPRESENTATION. The only reason for a language to define a true transpositional category is to allow a lexeme to assume the syntactic distribution of a word of a different class, so REPR clearly has to be a CONTENT feature.\is{features} In languages such as \ili{Russian}, in which \isi{participles} are marked morphologically, this also means that REPR is a FORM feature.\is{features} %
 I shall assume that the REPR attribute as applied to verbs has three values. The first is \featval{plain} (or \lab V,V\rab), the ‘identity representation’, and the default value. Where no indication of REPR is given the default is to be understood.  The second value of REPR is \lab V, A\rab, defining the four \isi{participles}.   %
 The third value is \lab V,Adv\rab,%
 \footnote{The semantic function role label Adv stands proxy for whatever the appropriate way is of defining adverbs as distinct from adjectives.} %
defining the \isi{imperfective} and perfective gerunds. I shall ignore the gerunds for simplicity of exposition. This means that I shall only mark the \isi{participles} explicitly.

We now define the content of the c-MORSIG attribute by reference to the verb’s SYN value. The aspects, and voice are defined with the mapping shown informally in (\ref{c2fverbs}). This will apply to any representation whose ARG-STR contains the E sf role. In practice, this means verbs, \isi{participles}, and gerunds.


\begin{exe}	\ex	\label{c2fverbs}

\ldots\ E \ldots\ $\Rightarrow$ ASPECT, VOICE $\subset$ MORSIG 
\end{exe}
I return to the c-MORSIG|CONDITIONAL, TENSE:\{pst\} mappings below when I discuss the status of the l-participle.

%%%%%%%%%%%%%%%%%%%%%%%%%%%%%

The  feature\is{features} array defining \isi{participles} is given in Fig.~\ref{fig:Russptcps}. This is the ``derived'' MORSIG attribute for any lexical representation defined by the feature REPR:\lab V,A\rab. %
The property sets labelled \fbox{1}\, in Fig.~\ref{fig:Russptcps} come from the MORSIG attribute of the base verb %
by virtue of (\ref{c2fverbs}). %
The property sets \fbox{3}\,, \fbox{4} are those which ensure that the \isi{participles} are inflected like adjectives. I return to these when I have introduced adjective inflection. \is{inflection}


\begin{figure}[h]
%\[	
%\left[
%\begin{array}{ll}
%MORSIG & 
%\left[ \begin{array}{ll}
% ASPECT & \{ipfv, pfv\}\,\fbox{1}\\
% VOICE & \{act, pass\}\,\fbox{2}\\
% CONCORD & \left[ \begin{array}{ll} NUMBER		&\{sg, pl\}\\
 %GENDER &\{m, f, n\}\\
 %CASE &\{nom, \ldots\} ]\, \end{array}\right]\fbox{3} \\
% MORCLASS & adj\,\fbox{4} \end{array}\right]
% \end{array}
% \right]
%\]
%\todo[inline]{commented out avm}	
 		\begin{avm}
 [MORSIG  [
 ASPECT		&\{ipfv, pfv\}\,\hpsgtag{1}\\
 VOICE		&\{act, pass\}\,\hpsgtag{2}\\
 CONCORD &[NUMBER		&\{sg, pl\}\\
 GENDER		&\{m, f, n\}\\
 CASE			&\{nom, \ldots\}
 		]\,\hpsgtag{3}\\
 MORCLASS	&adj\,\hpsgtag{4}
 ]
 ]
\end{avm}
\caption{Feature structure for \ili{Russian} participles} 	\label{fig:Russptcps} 
	\end{figure}

The key to my analysis of transpositions is to incorporate part of the analysis of derivational\is{derivation} \isi{morphology} into the definition of the transposition’s entry. Given a feature\is{features} pairing \lab\pounds,ρ\rab, where \pounds\ is a \isi{lexemic index} and ρ contains a value of REPR (for instance, [REPR:\lab V,A\rab] for \isi{participles}), the GPF applied to this pairing leaves the LI and the SEM representations of the base unchanged, but enriches the SYN|ARG-STR attribute by creating a complex semantic function role \lab A$\sb{i}$\lab E\lab x$\sb{i}$, \ldots\rab\rab\rab. The coindexation guarantees that the noun head modified by the participle is identified with the highest thematic argument of the base verb’s argument\is{arguments} array, i.e. its SUBJECT. %
The FORM function component of the GPF defines the \isi{stem} set for the participle, but underspecifies all other FORM information, including the MORSIG (and the CONTENT \isi{paradigm} MORSIG is also underspecified by the SYN function). 


The lexemic entry for a typical transitive\is{transitivity} \ili{Russian} verb such as \lxm{udarʹitʹ/udarʹatʹ} is that shown in Fig.~\ref{fig:lexentryudarit’}, where $\mathcal{V}$ stands for the verb’s \isi{lexemic index}. The lexemic entry’s value for LI is just the LI of the lexeme, of course (the GPF here does not describe  a process of derivational\is{derivation} morphology).\is{morphology} %
Note that for this lexeme the SYN attribute, too, is completely underspecified, lacking even the ARG-STR attribute. The value of that attribute is determined by default from the \emph{Event} ontological type of the SEM representation. However, the REPR feature\is{features} which defines transpositions introduces a realization rule which has to be defined over a specified ARG-STR representation. Therefore, the ARG-STR attribute has to be part of the lexeme’s (SYN attribute’s) morpholexical signature (inflection-like \isi{argument structure}\is{arguments} alternations\is{alternation} such as \isi{passive} or antipassive impose a similar requirement). In this respect, the transpositions are like \isi{inflection} and not like derivation.\is{derivation} %


We can informally state the realization rule which defines adjectival representations of lexemes (transpositions-to-adjective) as a function α from ARG-STR representations to ARG-STR representations, as shown in (\ref{trans2A}), where (\ref{transfromV}) defines a participle’s ARG-STR and (\ref{transfromN}) defines that of a relational adjective.

\begin{exe}
\ex	\label{trans2A}
	\begin{xlist}
\ex	\label{transfromV} α(\lab E\lab x,\ldots\rab\rab) = \lab A\textsubscript{i}\lab E\lab x\textsubscript{i},\ldots\rab\rab\rab

\ex	\label{transfromN} 	α(\lab R\rab) = \lab A \lab R\rab\rab
	\end{xlist}
\end{exe}

Given the lexemic entry in Fig.~\ref{fig:lexentryudarit’} and the realization rules for \ili{Russian} morphology,\is{morphology} the GPF for the \isi{imperfective} active participle, \textex{udarʹajušč-} will map to a partially underspecified lexical representation, as shown in Fig.~\ref{fig:udarjajushchlexentry}.  A representation such as this is the ‘quasi-lexeme’ discussed earlier. Like a simple lexemic entry, or an entry defined by a derivational\is{derivation} GPF, it needs to have its MORSIG attributes specified in order to be inflectable. I turn now to how those MORSIG entries are defined. 


\begin{figure}
	
\begin{tabular}{lll}

\multirow{6}{*}{f\textsubscript{\textit{form}}	(\lab $\mathcal{V}$, \featval{u} \rab)}	& \multirow{6}{*}{=}	&			\begin{avm}
[STEM0 [PHON	&udarʹ \\ 
		MORCLASS &V|CONJ2
		]\\
STEM1 [PHON	&udarʹaj \\
		MORCLASS	&V|CONJ1
		]
]
		\end{avm}	\\\addlinespace[1em]


f\textsubscript{\textit{syn}}(\lab $\mathcal{V}$, \featval{u} \rab)		& =	& \featval{u}						\\	\addlinespace[1em]
f\textsubscript{\textit{sem}}(\lab $\mathcal{V}$, \featval{u} \rab)		& =	& λx,y[HIT(x,y)]	\\	\addlinespace[1em]
f\textsubscript{\textit{li}}(\lab $\mathcal{V}$, \featval{u} \rab)		& =	& \featval{u}						\\
		
\end{tabular}	
	
\caption{Lexemic entry for \lxm{udarʹitʹ/udarʹatʹ}}	\label{fig:lexentryudarit’}
\end{figure}


%%%%%%%%

\begin{figure}

	\begin{tabular}{lll}
\multirow{4}{*}{FORM(\lab $\mathcal{V}$, \{REPR:\lab V,A\rab\}\rab)} 	&\multirow{4}{*}{=}	&
		\begin{avm}
[STEM0|PHON  &udarʹajušč \\
MORSIG &\featval{u}
]
\end{avm}
	\\	\addlinespace[1em]
\multirow{4}{*}{SYN(\lab $\mathcal{V}$, \{REPR:\lab V,A\rab\}\rab)} &\multirow{4}{*}{=}	&	\begin{avm}
[ARG-STR		&\lab A$\sb{i}$\lab E\lab x$\sb{i}$,y\rab\rab\rab\\
MORSIG &\featval{u}
]
\end{avm}\\	\addlinespace[1em]

SEM(\lab $\mathcal{V}$, \{REPR:\lab V,A\rab\}\rab)	&= &identity function
 \\	\addlinespace[1em]
LI(\lab $\mathcal{V}$, \{REPR:\lab V,A\rab\}\rab) &= &identity function
		
\end{tabular}	

	\caption{Underspecified entry for  \textex{udarʹaju\v{s}\v{c}(ij)}}	\label{fig:udarjajushchlexentry}

\end{figure}


\begin{figure}
%\todo[inline]{commented out avm}	
 \begin{avm}
[ MORSIG 	[ASPECT: $\lbrace$u$\rbrace$	\\
 		VOICE: $\lbrace$u$\rbrace$\\
 		CONCORD:$\lbrace$u$\rbrace$\\  
 		MORCLASS: ADJ|DECL1/2
 		]
]
  \end{avm}
% \bigskip

\caption{MORSIG for \ili{Russian} participles}	\label{fig:ptcpMORSIG}
	
\end{figure}


The partially specified MORSIG we need to be able to define for \isi{participles} is that shown in Fig.~\ref{fig:ptcpMORSIG}. The ASPECT/VOICE properties are shared with verbs. This can be achieved by writing the rules defining the MORSIG of verbs and \isi{participles} in such as way as to refer either to the ``outermost'' E semantic function role of the ARG-STR attribute, or the ``embedded'' E role found with \isi{participles}.%
\footnote{\ili{Russian} action nominals are transpositional lexemes and not true transpositions.} %
The CONCORD attribute,  \fbox{3}\,,  comes from the generic c-MORSIG of an adjective, shown in (\ref{adjmorsig}).%
\footnote{Members of the semantically defined class of quality or scalar adjectives will also have the feature COMPARISON added to their MORSIG to define comparative/superlative morphology.\is{morphology}} %




\begin{exe}	\ex	\label{adjmorsig}

\ldots\ A \ldots\  $\Rightarrow$  [\featname{CONCORD}:\{NUMBER, GENDER, CASE\}] $\subset$ MORSIG
\end{exe}

The [MORCLASS adj] specification, \fbox{4}\,, is strictly speaking a stipulation, except that in \ili{Russian} (in contrast to, say, \ili{Latin}) all \isi{participles} belong to the default adjectival class, DECL1/2. %


Given this machinery we can now account for the transpositional mixed category of participle by application of the (quasi-inflectional)\is{inflection} generalized \isi{paradigm} function applying to a verb and delivering its participial forms, triggered by the [REPR] feature.\is{features} These representations will be underspecified for (adjectival) CONCORD features.\is{features}  For concreteness, consider the \isi{imperfective} active participle, \textex{udarʹajušč-}. Given ρ = \{REPR:\lab V,A\rab, ASPECT:ipfv\,\hpsgtag{1}\,, VOICE:act\,\hpsgtag{2}\,\}, then for $\mathcal{V}$ = \lxm{udarʹitʹ/udarʹatʹ}, the GPF(\lab$\mathcal{V}$,$\rho$\rab) will apply to %
a lexical representation which is derived from the lexemic entry for \lxm{udarʹitʹ/udarʹatʹ} %
whose MORSIG attributes have been fully specified, allowing the lexeme to be inflected (in the broad sense of this term, including ``inflection''\is{inflection} for participle formation).
This GPF will deliver a partially underspecified lexical representation for the participle. The GPF will specify the participle’s \isi{stem} form(s), the ASPECT, VOICE \isi{features} which define that particular participle, and the enriched ARG-STR attribute with complex semantic function role. %

The lexical representation which is input to the GPF is shown in Fig.~\ref{fig:lexrepudarit} and the lexical representation of the participle is shown in (\ref{gpfudarjajushchij}). %
In Fig.~\ref{fig:lexrepudarit},  STEM0 denotes the perfective stem,\is{stem} which is effectively the lexeme’s root.\is{root} CONJ2 is second conjugation, and %
this means that most inflectional\is{inflection} rules will be defined over another stem,\is{stem} \textex{udarʹi-}, derived by regular rules of \isi{stem} formation. %
STEM1 denotes the \isi{imperfective} stem,\is{stem} a member of CONJ1, the first conjugation, whose inflectional\is{inflection} \isi{stem} is therefore \textex{udarʹaj-}.  Attributes which belong to both FORM and SYN (CONTENT) paradigms\is{paradigm} are tagged to make them more easily identifiable.%






\begin{figure}
 \begin{avm}
 [		
 FORM &[		& PHON &[[STEM0	PHON|udarʹ				\\
 			MORCLASS	V|CONJ2
 			]							\\
 			[STEM1 \\
 			MORCLASS V|CONJ1
 			] ]\\
 
 			&MORSIG &[ASPECT:\featval{ipfv, pfv}\,\hpsgtag{1}			\\
 					VOICE:\featval{act, pass}\,\hpsgtag{2}			\\
 					VFORM:[TNS											\\
 								LPTCP								\\
 								\ldots
 								]									\\
 					CONCORD:\{\featname{NUMBER, PERSON, GENDER}\}\,\hpsgtag{3}		\\
 					REPR:\{\lab V,A\rab, \lab V,Adv\rab\}\,\hpsgtag{4}\\
 					Other verb FORM MPSs\,\hpsgtag{6}	]
  			]						\\	
 SYN &[		&ARG-STR:&\lab E\lab x$\sb{i}$,y\rab\rab\,\hpsgtag{5}				\\
 			&MORSIG &[ASPECT:\{\featval{ipfv, pfv}\}\,\hpsgtag{1}			\\
 					VOICE:\{\featval{act, pass}\}\,\hpsgtag{2}			\\
 					TENSE:\{\featval{pst, prs, fut}\}					\\
 					CONCORD:\{\featname{NUMBER, PERSON, GENDER}\}\,\hpsgtag{3}		\\
 					REPR:\{\lab V,A\rab, \lab V,Adv\rab\}\,\hpsgtag{4}\\
 					ARG-STR:\lab sf role \lab $\theta$ array\rab\rab\,\hpsgtag{5}\\
 					{Other verb CONTENT MPSs\,\hpsgtag{6}}	]\\
 		&\ldots
 ]	
 										\\
 SEM &λx,y\[\textsubscript{\textit{Event}}HIT\(x,y\)\]
 								\\
 LI &\lxm{udarʹitʹ/udarʹatʹ}							\\
 ]
 \end{avm}
  \caption{Lexical representation of the lexeme UDAR'IT'/UDAR'AT' `to hit'}\label{fig:lexrepudarit}
\end{figure}



\begin{exe} \ex \label{gpfudarjajushchij}

f\textsubscript{\featval{form}} = STEM-iap $\oplus$ \v{s}\v{c} = udarʹaju-\v{s}\v{c}-

f\textsubscript{\featval{syn}} = ARG-STR:\lab A$\sb{i}$\lab E\lab x$\sb{i}$,y\rab\rab\rab

\end{exe}
\noindent where STEM-iap denotes the \isi{imperfective} active participle stem,\is{stem} derived from STEM1. 

We have now achieved our goal of defining \isi{participles}. In effect, we have defined the lexemic entry for a class of adjectives, whose peculiarity is that they are marked for verbal voice and \isi{aspect} and they share their \isi{semantics} and \isi{lexemic index} with a parent verb. 

%%%%%  True \isi{participles} vs l-ptcp

It is instructive to compare the behaviour of the true \isi{participles} with that of the l-participle. While the true \isi{participles} are mixed categories, the l-participle is essentially a verb form with unusual agreement morphology.\is{morphology} It is not entirely clear how best to account for the peculiar agreement properties of the l-participle in the \isi{past tense} and conditional constructions, but the simplest (if somewhat crude) way to do this is to assume that verbs in general agree with their subjects in person, number, gender features,\is{features} but that in the past/conditional forms agreement is restricted to number, gender, while in the present tense forms it is restricted to person, number.%
\footnote{A more sophisticated approach might define agreement in a morphology-driven\is{morphology} fashion by stating that the agreement \isi{features} that the syntax can manipulate are restricted to those that can be expressed by a particular morphological form, so that it is the morphological\is{morphology} MORSIG that determines which features trigger agreement, even in identical syntactic positions.} %

I conclude, then, that the  l-participle is a verb form that inflects just like a (subtype of) predicative (short-form) adjective. Assuming a feature\is{features} {ADJDECL} covering adjectival \isi{morphology} generally, we can distinguish several sub-types of declension, including that for the predicative adjectives, [{ADJDECL}:predadj]. The l-\isi{participles} will belong to a subtype of this class, [{ADJDECL}:predadj:lptcpdecl]. The {ADJDECL} feature is part of the MORSIG attribute of ordinary adjectives, as defined by the \isi{paradigm} linkage rule in (\ref{adjmorsig2}).%
\footnote{As adjectives, we might expect the \isi{participles} to have predicative forms, too. This is true, however, only for the \isi{passive} \isi{participles}, especially the perfective passive,  which has a special stem form ending in a singleton /n/, \textex{napʹisan} ‘written’, in contrast to the attributive form with geminated /nn/: \textex{(v speške) napʹisannaja (zapʹiska)} ‘(a hastily) written (note)’.} %

\begin{exe}
\ex
\label{adjmorsig2}

\ldots\ A \ldots\ $\Rightarrow$ ADJDECL $\subset$ MORSIG
\end{exe}


%%%%% Comparison with transpositional lexemes
We  also compare true \isi{participles} with qualitative adjectives derived by conversion from \isi{participles}. These are a type of transpositional lexeme. The theoretical significance of this type of \isi{lexical relatedness} for current morphological\is{morphology} models was first identified, as far as I am aware in \citet[275]{Spencer13:book}, where I discuss \ili{English} words such as \lxm{prepositional}, from \lxm{preposition}. These look like relational adjectives (noun-to-adjective transpositions), because their lexical semantic content seems to be identical to that of their base noun (\textex{prepositional phrase} means the same as the \isi{compound} \textex{\isi{preposition} phrase}, for instance). However,  the \ili{English} adjectives are syntactically \isi{opaque}: \textex{monosyllabic prepositional phrase} does not mean ‘phrase headed by a monosyllabic \isi{preposition}’ but only ‘monosyllabic phrase headed by a \isi{preposition}’ (though that interpretation is possible for the compound,\is{compound}  \textex{monosyllabic \isi{preposition} phrase}). In other words, the adjectival expression only has the structure \textex{monosyllabic [prepositional phrase]}, not \textex{[monosyllabic \isi{preposition}]al phrase}.


\ili{English} has a large number of qualitative adjectives derived by conversion from \isi{participles}, which are also instances of transpositional lexemes \parencite{Spencer16:MorphMetatheory}:  \textex{amazed/ amazing, bored/boring, challenged/challenging, interested/interesting, \ldots}. In very many cases it is not possible to identify a meaning difference between the adjective and the etymological verb base: \textex{This book bores me/This book is boring}. \ili{Russian} \isi{participles} likewise are often converted into qualitative adjectives: \textex{potrʹasaju\v{s}\v{c}ij} ‘amazing’, \textex{vyzyvaju\v{s}\v{c}ij} ‘provocative, defiant, challenging’. In \ili{Russian} it is often possible to determine that a word with the shape of a participle is actually an independent adjective, since only true adjectives have the short predicative form: \textex{uspexi potrʹasaju\v{s}\v{c}i} (\isi{plural}, from \textex{potrʹasaju\v{s}\v{c}}) ‘the-progress is-amazing’ (lit. ‘the-successes are-amazing’), \textex{ego povedenʹie vyzyvaju\v{s}\v{c}e} ‘his behaviour is-defiant’. True \isi{participles} do not have the predicative form (Russkaja Grammatika II, p.666). \ili{English} converted \isi{participles} fail to inherit the complementation\is{complement} properties of the base verb: \textex{The obstacle course challenged the stamina of the athletes$\sim$The obstacle course was very challenging (*the stamina of the athletes)}. \ili{Russian} transpositional adjective lexemes behave likewise. The transitive\is{transitivity} base verb \lxm{potrʹasatʹ} ‘to amaze’ gives us \textex{uspexi potrʹasalʹi nas} ‘the-successes amazed us’, but the (active) transpositional lexeme cannot take a direct object: \textex{*uspexi potrʹasaju\v{s}\v{c}i nas}. The true participle remains transitive: \textex{potrʹasaju\v{s}\v{c}ie nas uspexi} ‘progress (successes) which amaze us’.

We thus have a double dissociation of properties: on the one hand, we have the l-participle which has the form of a predicative adjective but which realizes finite (tense/ mood)\is{tense} verb properties and retains the full complementation\is{complement} properties of the verb, and on the other hand we have participle-like adjectives which, while allowing predicative adjective forms, lack all the crucial morphosyntax of verbs. In between we have the true \isi{participles}, with the external morphosyntax of an adjective but the complementation\is{complement} properties of the base verb. 

\section{Conclusions: Transpositions and split morphology}


I have argued in this paper for a view of morphosyntax which recognizes a word/phrase distinction and, given that, a distinction between (abstract) lexemes and (concrete) inflected word forms of those lexemes, valid for very nearly all known languages. I have also assumed that languages can increase their stock of lexemes by means of derivational\is{derivation} morphology,\is{morphology} and that in some cases this is sufficiently regular to be regarded as  paradigmatic,\is{paradigm} hence, part of the grammatical system proper. The inflection/derivation\is{inflection}\is{derivation} distinction is controversial, however, because it is often difficult to know where the boundary actually lies and where to place intermediate types of \isi{lexical relatedness}. The transpositions, as exemplified by the \ili{Russian} participial system discussed here, represent a particularly troublesome case-in-point.


Participles and other transpositions are often treated as a type of derivational\is{derivation} morphology,\is{morphology} because they involve a shift in word class, but this is a wrong characterization. Participles are part of a verb’s paradigm,\is{paradigm} they are not a type of lexical stock expansion \parencite{Beard95:book}. Nonetheless, \isi{participles} inflect like adjectives, not like verbs and thus seem to straddle the inflection/derivation\is{inflection}\is{derivation} divide in a way that calls that very distinction into question, and, on the face of it, even provides support for models in which all \isi{morphology} is just syntax by other means (Minimalism/\isi{Distributed Morphology})\is{morphology} or in which all syntax is just \isi{morphology} by other means (American morphemics). Here I have claimed that, on the contrary, we can only make sense of \isi{participles} against a set of background assumptions that contradict  monolithic  models in which there is no autonomous \isi{morphology} module (Aronoff's ``morphology-by-itself''), so that morphology is no more than syntax by other means. The crucial observation is that derivational\is{derivation} \isi{morphology} induces a type of lexical \isi{opacity} which is lacking in transpositions, which, by contrast, show the kind of lexical \isi{transparency} associated with inflected forms.

\newpage 
In the GPFM model, canonical \isi{derivation} is a relation between maximally underspecified (minimally redundant) lexical representations (lexemic entries), consisting of a specification of the basic \isi{root} form (FORM|PHON) and the ontological/semantic representation (SEM). The \isi{morphosyntactic} properties are then projected from these by default mappings. Canonically, \isi{derivation} enriches the PHON and the SEM representations, and the Default Cascade then specifies the \isi{morphosyntactic} properties of the derived lexeme. One consequence is that there will then be no ``trace'' left of the \isi{morphosyntactic} properties of the base lexeme, such as its word class or its \isi{argument structure}.\is{arguments} This automatically guarantees most of the lexical \isi{opacity}/\isi{lexical integrity} effects familiar from the literature. In some (noncanonical) cases it may be necessary to stipluate overrides of lexical information as part of the derivational\is{derivation} Generalized Paradigm\is{paradigm} Function. This might be true if, for instance, a base lexeme belongs to a lexical category which is not the default for its ontological class. For instance, a language with a distinct lexical category of (qualitative) adjective may also have non-derived \isi{stative} verbs whose denotations are of the ontological type Property, and which by the Default Cascade should be adjectives, not verbs. Such a verb would have to have its ARG-STR|\lab E \ldots \rab\ value prespecified.  If such a lexeme were the input to a verb-to-noun derivational\is{derivation} function (nominalization), then that ARG-STR would have to be overridden by the nominalization function, replacing it wholesale with the ARG-STR|\lab R\rab\ value. Exactly how such cases are to be handled has to remain a matter for future research.


No such \isi{opacity} is found with canonical inflection: in general, the syntax treats a noun as a noun no matter what its number, case,\is{case} \isi{definiteness}, \ldots\ value. Thus, although a locative \isi{case} marked noun would normally be restricted to functioning as an adjunct or the \isi{complement} of a class of adpositions, it can still be modified by an adjective, just like any other noun form, so that for a noun ‘house’, \textex{new house-\glossfeat{loc}} means ‘at a new house’. In this respect, a locative \isi{case} marked noun typically differs from a  derived denominal lexeme denoting a location. Many languages have a denominal derivational\is{derivation} marker meaning ‘place where there is/are NOUN, place associated with NOUN’: N-\glossfeat{place}. Typically, the base noun, N, is not accessible to morphosyntax, so that  an expression such as \textex{new house-\glossfeat{place}} could only mean ‘new place where there is a house/are houses’ and not ‘place where there is/are a new house(s)’.  Thus, \isi{inflection} differs from \isi{derivation} in being lexically transparent. 

The significance of these rather obvious points about inflected forms is that we are far less able to make similarly categorical claims where transpositions are concerned. A participle behaves in the syntax \emph{to some extent} as though it were an inflected verb form, but not entirely. In GPFM the lexical representation of a transposition exhibits \isi{transparency} by virtue of being a member of the base lexeme’s (extended) paradigm,\is{paradigm} that is, by bearing the same LI as the base. In this respect it is not an autonomous lexeme. On the other hand, the transposition exhibits the external syntax of a distinct (derived) lexical category. The extension to the GPF proposed here permits us to model this ``inflectional-paradigm-within-a-paradigm''\is{inflection}\is{paradigm} effect in a way that reflects the lexical \isi{transparency} of the transposition while also allowing us to state restrictions on full \isi{transparency} as a restriction on the MORSIG of the transposition. 

A crucial \isi{aspect} of the analysis is the distinction between FORM/CONTENT properties (m-/s-features).\is{features} %
Without at  least this level of differentiation we cannot make sense of \ili{Russian} \isi{participles} and we cannot distinguish them from the verbal l-participle form or the transpositional lexemes.  A second crucial \isi{aspect} of the analysis of any type of transposition is the Lexemic Index (LI). In the general case, there is no combination of lexical properties that will uniquely serve to individuate lexemes, but it is nonetheless essential to impose such an individuation to account for the patterns of \isi{lexical relatedness} that are found across languages, specifically, to distinguish true \isi{participles} from departicipial converted adjectives (transpositional lexemes). 

The combination of the FORM/CONTENT \isi{paradigm} distinction (or its equivalent) and the Lexemic Index allow us to define not just canonical \isi{inflection} but also non-canonical intra-lexemic types of relatedness such as that shown by transpositions. That combination also serves to reconstruct the split in the \isi{morphology} argued for originally by Anderson. Indeed, \isi{split morphology}\is{morphology} is entailed by this set of assumptions, except that in a \isi{constraints-based}\is{constraint} model the split is defined in terms of access: inflectional\is{inflection} \isi{morphology} is that which permits syntax to retain access to the properties of the base lexeme (lexical \isi{transparency}), while derivational\is{derivation} \isi{morphology} permits no such access (lexical \isi{opacity}/integrity).   

The more articulated view of morphosyntax proposed here allows us to pose a question which was not at the forefront of debate over the question of \isi{split morphology},\is{morphology} as far as I am aware: which side of the split do transpositions fall on? The answer, given the foregoing, is ``both''. The transposition is inflectional\is{inflection} by virtue of preserving the base’s LI and by virtue of the, at least partial, \isi{transparency} of the base’s properties. It is derivational\is{derivation} by virtue of the fact that it defines the \isi{paradigm} of a quasi-lexeme, within the \isi{paradigm} of the base. But it would be difficult to make sense of this conclusion without assuming the basic split in the first place.



% \appendix
% \setlength{\parindent}{0em}
% \setlength{\parskip}{6pt}

\section*{Appendix: Illustration of the lexical representations assumed}

%\todo{this is dangling}
%\textsc{drive} = GPF(\lab\textsc{drive},u\rab) =

\begin{figure}[h]
	\begin{centering}
		\begin{tabular}{lll}
\begin{avm}
[FORM	&STEM0 |draɪv|						\\
SYN		&\textemdash						\\
SEM		&\[\textsubscript{\textit{Event}} λx,y.\textbf{drive}\(x,y\)\]						\\
LI		&\textsc{drive}
]
\end{avm}
&\multirow{6}{*}{$\Rightarrow$}			\\	\addlinespace[2em]
\begin{avm}
[FORM	&[STEM0 &|draɪv|						\\
		   MORSIG	&[TNS  		&\{prs, pst\}			\\
		   			VFORM 	&\{ing-form, ed-form, base\} \\
				    AGR 		&\{3sg, none\}]
		   ]		    									\\
SYN		&[SYNCAT &V									\\
		   ARG-ST &\<E\<x,y\>\>							\\
		   MORSIG	&[TNS 	&\{prs, pst\}					\\
		   		    ASP	&\{simple, prog, perf\}			\\
				    AGR 		&\{3sg, none\}				\\
				\ldots
				    ]
		]							\\
SEM		&\[\textsubscript{\textit{Event}} λx,y.\textbf{drive}\(x,y\)\]						\\
LI		&\textsc{drive}
]
\end{avm}
		\end{tabular}
	\end{centering}
\caption{Illustration of the application of the GPF for standard inflection.}	\label{fig:ajs:drive}
\end{figure}

	%
\bigskip

To specify, say, the 3sg form \textit{drives}, the GPF(\lab \textsc{drive},\{3sg\}\rab) applies to the output of Figure \ref{fig:ajs:drive} to specify the FORM value |STEM0⊕z|, leaving other aspects of the representation unchanged. This is equivalent to the operation of the \isi{paradigm} function in PFM1 and the output of the \textbf{\textit{Corr}} function in PFM2.
\clearpage


\begin{exe}
\ex \textsc{drive} $\Rightarrow$ \textsc{driver}

Where δ = SubjNom, Σ = [\textsubscript{\textit{Event}} λx,y.\textbf{drive}(x,y)], GPF(\lab\textsc{drive},δ\rab) =
\end{exe}

\begin{figure}[h]
	\begin{centering}
		\begin{tabular}{lll}
\begin{avm}
[FORM	&STEM0 |draɪv|						\\
SYN		&\textemdash						\\
SEM		&Σ   						\\
LI		&\textsc{drive}
]
\end{avm}
&\multirow{6}{*}{$\Rightarrow$}			\\ \addlinespace[2em]
\begin{avm}
[FORM	&$\[$STEM0 &STEM0\(\textsc{drive}\)⊕|ə|						\\
SYN		&\textemdash						\\
SEM		&$\[$\textsubscript{\textit{Thing}} λx.PERSON\(x\) ∧ Σ 						\\
LI		&δ\(\textsc{drive}\)
]
\end{avm}
		\end{tabular}
	\end{centering}
\caption{Illustration of the application of the GPF for standard derivation}	\label{fig:ajs:driver}
\end{figure}

The output of this GPF then undergoes specification of MORSIG by the Default Cascade, which defines the derived lexeme as a syntactic and morphological noun.






\section*{Abbreviations} 


 \begin{tabularx}{.45\textwidth}{lQ} 
AGRSUBJ & subject agreement     \\
act & active         \\
ARG-STR &  {argument structure}     \\
CONJ & conjugation       \\
DECL & declension       \\
GPF & Generalized Paradigm Function    \\
GPFM & Generalized Paradigm Function Morphology\\
ipfv & {imperfective}        \\
LI &  {lexemic index}       \\
L-PTCP, l-ptcp & l-participle      \\
\\
\end{tabularx}
\begin{tabularx}{.45\textwidth}{lQ}
MORCAT & morpholexical category \\
MORCLASS & morphological class     \\
MORSIG & morpholexical signature    \\
MPS &  {morphosyntactic} property set    \\
pass & passive         \\
PF & Paradigm Function      \\
PFM & Paradigm Function Morphology    \\
pfv & perfective        \\
PHON & phonological form     \\
REPR & REPRESENTATION      \\
SEM &  {semantics}        \\
SYN & syntax         \\
\\
\end{tabularx}

\newpage 
\section*{Acknowledgements}
I am grateful to two anonymous referees and to Larry Horn for very helpful commentary and for correcting a number of typos and other errors. I would also like to express my gratitude to Steve Anderson for his pioneering work in getting morphology accepted as a legitimate subject for generative grammar, and also to thank him for his enlightening and entertaining contributions to the field,  admirably limpid,  often witty, and always thought-provoking.

{\sloppy
\printbibliography[heading=subbibliography,notkeyword=this]
}


\end{document}
