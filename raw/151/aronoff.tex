\documentclass[output=paper,
modfonts
]{LSP/langsci}


%%\input{localpackages.tex}

%%%Copy this to localcommands.tex

\usepackage[english]{babel}
\usepackage{amsmath}
\usepackage{amssymb,amsfonts,textcomp}
\usepackage{array}
\usepackage{hhline}
\usepackage{hyperref}

\newenvironment{styleStandard}{}{}
\newenvironment{stylelsAbstract}{}{}
\newenvironment{stylelsSectioni}{}{}
\newenvironment{stylelsSectionii}{}{}
\newenvironment{stylelsBulletList}{}{}
\newenvironment{styleBibliographyi}{}{}
\newenvironment{listWWNumxxvleveli}{}{}
\newenvironment{listWWNumxxvlevelii}{}{}
\newenvironment{listWWNumxxvleveliii}{}{}
\newenvironment{listWWNumxxvleveliv}{}{}
\newenvironment{listWWNumixleveli}{}{}
\newenvironment{listWWNumixlevelii}{}{}
\newenvironment{listWWNumixleveliii}{}{}
\newenvironment{listWWNumixleveliv}{}{}

\newcommand\textstyleListLabelxvi[1]{#1}
\newcommand\labellistWWNumxxvleveli{\thelistWWNumxxvleveli.}
\newcommand\labellistWWNumxxvlevelii{\thelistWWNumxxvlevelii.}
\newcommand\labellistWWNumxxvleveliii{\thelistWWNumxxvleveliii.}
\newcommand\labellistWWNumxxvleveliv{\thelistWWNumxxvleveliv.}
\newcommand\labellistWWNumixleveli{[F0B7?]}
\newcommand\labellistWWNumixlevelii{\textstyleListLabelxvi{o}}
\newcommand\labellistWWNumixleveliii{[F0A7?]}
\newcommand\labellistWWNumixleveliv{[F0B7?]}

\newcounter{listWWNumxxvleveli}
\newcounter{listWWNumxxvlevelii}[listWWNumxxvleveli]
\newcounter{listWWNumxxvleveliii}[listWWNumxxvlevelii]
\newcounter{listWWNumxxvleveliv}[listWWNumxxvleveliii]
\newcounter{itemize} 
\ChapterDOI{10.5281/zenodo.495459}
\title{Darwinism tested by the science of language}

\author{%
Mark Aronoff\affiliation{Stony Brook University}
}

% \sectionDOI{} %will be filled in at production
% \epigram{}

\abstract{%
Linguistics enjoyed great success in the last half of the 19\textsuperscript{th} century. The use of tree diagrams to express the genetic relations between languages spread from linguistics to evolutionary biology. The achievements of the Neogrammarians in establishing sound laws, however, led to a realization that the exceptionless laws of language change bore no resemblance in kind to the laws of natural science. Language \isi{evolution} had no principled basis akin to natural selection.  Saussure solved this problem in the \textit{Cours} by rooting linguistic theory in synchronic states of language rather than historical change, thus relegating diachronic\is{diachrony} linguistics to a minor position in the field. In recent decades, the field of cultural \isi{evolution} has allowed for the application of well-established principles from evolutionary\is{evolution} biology and ecology. The application of one of these, Gause's principle of competitive exclusion, to central problems of morphology has produced good results, suggesting prospects for the revival of evolutionary explanation in language along the lines of what linguists envisioned a century and a half ago.
}

\begin{document}
\maketitle


\section{Introduction}

Steve Anderson refuses to forget the past.  Along with \ia{Matthews, Peter}Peter Matthews, throughout his career he has reminded the community of theoretical linguists, especially morphologists, of the continuity of our culture.  His two most recent publications, one of them a tribute to Matthews \citep{andersontoappeara,andersontoappearb}, deal with the history of morphology. The following brief essay is a small tribute to Steve's life-long effort to demonstrate that we can understand how we think and act today only to the extent that we also understand where our thinking comes from. In it, I trace the rise and fall of the relation between \isi{linguistics} and evolutionary\is{evolution} biology in the half century between the publications of the foundational work of modern biology, Darwin's \citetitle{Darwin1859}, in 1859, and the foundational work of modern \isi{linguistics}, Saussure's \citetitle{Saussure1916}, in 1916.

In 1859, historical (also termed evolutionary at the time) \isi{linguistics} was on a meteoric trajectory as one of the most successful of academic disciplines, having provided precise demonstrations that numerous modern languages were related in specific ways. In the next thirty years, it would produce even more remarkable results and {Darwin}\ia{Darwin, Charles} himself would invoke its successes.

Saussure, author in 1878 of one of the most spectacular works in historical \isi{linguistics}, did a complete \textit{volte face} in his posthumous book forty years later, advocating that the field concentrate on description  of single language states rather than historical relations. He rejected evolutionary\is{evolution} accounts of language for one sweeping reason: the lack of an explanatory framework. Saussure's argument quickly turned the field away from \isi{diachrony} to \isi{synchrony}.

A century later, though, it has become clear that an evolutionary\is{evolution} account of what Darwin himself was referring to when he wrote that ``The survival or preservation of certain favoured words in the struggle for existence is natural \isi{selection}'' \citep[61]{Darwin1871} can provide just such a framework, allowing historical explanation of language to return to its rich relation with \isi{evolution} and evolutionary\is{evolution} theory.

\section{Schleicher, Haeckel, and stem trees}

They were friends, both professors at the University of \isi{Jena}, who shared a love of botany and gardening.  August Schleicher, the linguist, born in 1821, was 13 years older than Ernst Haeckel, the embryologist, and had been on the faculty since 1857. Haeckel had taken up his post as professor of comparative anatomy in 1862, soon after receiving his doctorate in zoology. Haeckel had been captivated by Darwin's landmark 1859 work and had recommended it to Schleicher. Schleicher opened his 1863 pamphlet, \citetitle{Schleicher1863} with an acknowledgement so notable that I repeat the passage here in its entirety.\footnote{All passages from \citet{Schleicher1863} are quoted from Alex V. W. Bikkers's 1869 \ili{English} translation.}\is{Schleicher}

\begin{quote}
You would leave me no peace until I began reading Bronn's [1860] translation of the much discussed work of Darwin \citetitle{Darwin1859}. I have complied with your request; I have waded through the whole of the book, in spite of its being rather clumsily arranged, and heavily written in a curious kind of German, and the greater part of the work I was tempted to read again and again. My first thanks are now offered to you for those repeated inducements of yours which ended in my study of this incontestably remarkable work. \citep[13--14]{schleicher1869}
\end{quote}

Schleicher had very recently achieved academic renown as the author in 1861/2 of the two-volume \citetitle{Schleicher1861}. He would die five years later in 1868 at the age of 47.  Haeckel\ia{Haeckel, Ernst} would outlive his friend by over a half century and become the greatest Continental disseminator of Darwinian thought through his bestselling book \citetitle{Haeckel1868}, published in the year of Schleicher's death and translated into \ili{English} as \textit{The History of Creation}. Haeckel, as an embryologist, is perhaps most famous for the slogan ``ontogeny recapitulates phylogeny'' and the associated erroneous theory. Together, though, as described by \citet{Burrow1972}, \citet{O'Hara1996}, and \citet{gontier2011}, the two friends are responsible for the use of tree diagrams in depicting evolutionary relations among languages first \citep{Schleicher1861,Schleicher1863} and then species \citep{Haeckel1866}. \citet{Haeckel1874} presented \textit{Stammbäume} lit. `stem trees’, of both the historical \isi{evolution} of \ili{Indo-European languages} and the ‘\isi{pedigree of man}' together in the same book.  What Haeckel had learned from Schleicher during the few years that they were colleagues in a small university was that the \isi{evolution} of languages and the evolution of \isi{species} were sufficiently analogous to warrant the use of the same diagrammatic method to describe the two. The method survives to this day in both fields, but little else remains in common between the two.  How did they move so far apart?

\section{Darwinism tested by the science of language}

Schleicher went further than \isi{analogy}.  The main point of his 1863 pamphlet, only 48 pages long in the original German edition, shorter than many academic journal articles today, was that the results of historical \isi{linguistics} over the previous half century constituted a successful ``test'' of Darwin's theory of evolution. First, Schleicher asserted that languages were what he called ``organisms of nature.'' He needed this to be true in order to directly test Darwin's theory, which deals with natural organisms, by applying it to languages.\is{evolution}

\begin{quote}
Languages are organisms of nature; they have never been directed by the will of man; they rose, and developed themselves according to definite laws; they grew old, and died out. They, too, are subject to that series of phenomena which we embrace under the name of ``life.'' The science of language is consequently a natural science; its method is generally altogether the same as that of any other natural science. \citep[20--21]{Schleicher1863}
\end{quote}

\begin{quote}
The rules now, which Darwin lays down with regard to the species of animals and plants, are equally applicable to the organisms of languages, that is to say, as far as the main features are concerned. \citep[30]{Schleicher1863}
\end{quote}

Here is where Schleicher understood that \isi{linguistics} had a contribution to make.  As is well known, Darwin acknowledged in his introduction that he had no direct evidence for the application of his theory to ``the variability of \isi{species} in a state of nature'' \citep[4]{Darwin1859}. The closest he could come was ``a careful study of domesticated animals and of cultivated plants'' (ibid.), which is why he devoted the first chapter of his book to ``Variation under Domestication.'' Only in the last quarter century have we been able to observe \isi{evolution} at work, most notably in Richard Lenski's Long Term Experimental Evolution Project (e.g, \citealt{Tenaillonetal2016}). Schleicher offered \isi{language evolution} as the only tangible proof of evolution available at the time:

\begin{quote}
Nobody doubts or denies any longer that the whole Indogermanic family of speech -- Indic, Iranic, (old Armenian, Persic, \& c.,) Hellenic, Italic, (\ili{Latin}, Oscan, Umbrian, with the daughters of the former) Keltic, Slavonic, Lithuanian, Teutonic or German, that all these languages, consisting of numerous species, races and varieties, have taken their origin from one single primitive form of the Indo-Germanic family. \citep[34]{Schleicher1863}\il{Indo-European}
\end{quote}

\begin{quote}
We are actually able to trace directly in many idioms that they have branched off into several languages, \isi{dialects} \& c., for we are in a position to follow the course of some, nay, of whole families of them during a period of more than two thousand years, since a faithful picture of them has been left us in writing. This, for instance, is the case with \ili{Latin}. \citep[41--42]{Schleicher1863}
\end{quote}

\section{Max Müller}

Darwin learned of Schleicher's work in an indirect way but in time to mention it in his second book, a dozen years after the publication of his first \citep{Darwin1871}. The 1863 pamphlet received a very positive brief anonymous review in a short-lived British weekly, the \textit{Reader}, in 1864 but appears to have attracted little attention in England, where German was not commonly read.\footnote{ \citet{alter1999} speculates that the writer of the \textit{Reader} piece was Frederic William Farrar, who was later responsible, as Dean of Westminster Abbey, for Darwin's interment there.} The \ili{English} translation, though, published in 1869, caused a stir. Max Müller himself, the best-known linguist and popularizer of language study in the country, reviewed it at some length in the first volume of \textit{Nature}, a successor to the \textit{Reader} as a general science periodical for the public \citep{muller1870}.  Darwin quoted from this review in his most famous passage on language.  In the review, Müller acknowledged the power of Schleicher's \isi{analogy}: ``He thinks rightly that the genesis of \isi{species}, as explained by Mr. Darwin, receives a striking illustration in the genealogical system of languages''\is{language evolution} \citep[257]{muller1870}; ``No reader of Mr. Darwin's books can fail to see that an analogous process pervades the growth of a new species of language, and of new species of animal and vegetable life'' \citep[258]{muller1870}.

Müller disagreed with Schleicher\ia{Schleicher, Ernst} on a number of points.  
The least noticed but most insightful was his objection to Schleicher's claim that \isi{linguistics} is a natural science.
It is natural only in that ``languages are not produced by the free-will of individuals . . . 
[T]he freedom of the individual is necessarily limited by the pressure exercised by all upon all. 
Speech in its very nature is mutual'' \citep[258]{muller1870}, a point that presaged Saussure's observation on the social nature of \textit{langue}. The second point was Müller's hobby-horse, his idiosyncratic idea that ``In tracing the origin of species, whether among plants or animals, we do not begin with one perfect type of which all succeeding forms are simple modifications. . . It is the same with languages'' \citep[258]{muller1870}. Müller here betrays his complete misunderstanding of \isi{Darwinism} as well as of mainstream \isi{linguistics} of his time, whose proponents -- most prominently William Dwight Whitney  -- he sparred with throughout his career \citep{Whitney1875,alter1999}. It is no wonder that Darwin had little use for Müller, with whom he also disagreed on a more fundamental issue: the continuity of humans with other creatures. This passage of Müller's is important enough to be cited in its entirety:

\begin{quote}
A much more striking \isi{analogy}, therefore, than the struggle for life among separate languages, is the struggle for life among words and grammatical forms which is constantly going on in each language. Here the better, the shorter, the easier forms are constantly gaining the upper hand, and they really owe their success to their own inherent virtue. Here, if anywhere, we can learn that what is called the process of natural \isi{selection},\is{evolution} is at the same time, from a higher point of view, a process of rational elimination; for what seems at first sight mere accident at the dropping of old and the rising of new words, can be shown in most cases to be due to intelligible and generally valid reasons. Sometimes these reasons are purely phonetic, and those words and forms are seen to prevail which give the least trouble to the organs of pronunciation. At other times the causes are more remote. We see how certain forms of grammar which require little reflection, acquire for that very reason a decided numerical preponderance; become, in fact, what are called regular forms, while the other forms, generally the more primitive and more legitimate, dwindle away to a small minority, and are treated at last as exceptional and irregular.\is{irregularity} In the so-called dialectic\is{dialects} growth of languages we see the struggle for life in full play, and though we cannot in every instance explain the causes of victory and defeat, we still perceive, as a general rule, that those forms and those words carry the day which for the time being seem best to answer their purpose. \citep[258]{muller1870}
\end{quote}

Darwin evidently approved of the argument, for he was generous enough to cite Müller's last point in \citetitle{Darwin1871} the following year:\footnote{\citet{Dingemanse2013} is a fascinating blog-post about the evolution of the citation of this passage in the last decade.  Most have attributed the observation to Darwin and neglect to note that Darwin had himself directly credited and cited Müller.}

\begin{quote}
We see variability\is{variation} in every tongue, and new words are continually cropping up; but as there is a limit to the powers of the memory, single words, like whole languages, gradually become extinct. As Max Müller has well remarked: -- ``A struggle for life is constantly going on amongst the words and grammatical forms in each language. The better, the shorter, the easier forms are constantly gaining the upper hand, and they owe their success to their own inherent virtue.''{ }To these more important causes of the survival of certain words, mere novelty may, I think, be added; for there is in the mind of man a strong love for slight changes in all things. The survival or preservation of certain favoured words in the struggle for existence is natural \isi{selection}.\is{evolution} \citep[60--61]{Darwin1871}\is{change}
\end{quote}

\section{\isi{Darwinism} and the laws of language}

With Müller's approving review of Schleicher's pamphlet, and more importantly with Darwin's endorsement, \isi{linguistics} had entered the mainstream of scientific discourse, a long-time goal of its practitioners that persists to this day. But not for long.  In the end, the most important legacy of Darwinian thinking for 19\textsuperscript{th} century \isi{linguistics} was the clarification of the \isi{scope} of the term \textit{law}\is{law (of language)} as it applies to language.  As late as Müller's review, linguists could still see their ultimate goal as the formulation of general laws of language origin and structure on a par with the laws of physics and chemistry, perhaps based on Darwin's theory of evolution.\is{evolution}  What linguists discovered instead were the one-off contingent laws of \isi{sound change}, from \isi{Grimm's law} to \isi{Verner's law}, startling but of no general significance beyond their purported exceptionlessness.  Linguists could still call them laws, though, not on account of generality but because of their \isi{regularity}.  This was, however, small consolation. As Osthoff and Brugmann so memorably declared in their Neogrammarian credo:\is{Neogrammarians}

\begin{quote}
First, every \isi{sound change}, inasmuch as it occurs mechanically, takes place according to laws that admit no exception. That is, the direction of the sound shift is always the same for all the members of a linguistic community except where a split into \isi{dialects} occurs; and all words in which the sound subjected to the change appears in the same relationship are affected by the change without exception. \citep[204]{osthoffandbrugmann1878}
\end{quote}

Hermann Paul stressed a couple of years later that sound laws in no way resemble those of physics and chemistry, but were statements of regular but contingent historical facts. To those who had aspired to gain for the science of language a place among the natural sciences, laws of this character would have been disappointing:

\begin{quote}
Can we assert uniformity of sound-laws? In the first place, we must fully understand what we mean, generally speaking, by a sound-law. The word ‘law' is itself used in very different senses, and this fact induces errors in its application. The idea of sound-law is not to be understood in the sense in which we speak of ‘laws' in Physics or Chemistry, nor in the sense of which we were thinking when we contrasted exact sciences with historical sciences. Sound-law does not pretend to state what must always under certain general conditions regularly recur, but merely expresses the reign of uniformity within a group of definite historical phenomena. \citep[57]{Paul1880}
\end{quote}

\section{Ferdinand de Saussure and the end of evolution}

The greatest individual achievement of Neogrammarian\is{Neogrammarians} historical \isi{linguistics} was Saussure's \emph{M\'emoire sur le syst\`eme primitif des voyelles dans les langues indo-europ\'eennes}.  Written in a fury and printed in fascicles one by one in 1878 when Saussure was a 21-year-old student, it was the only sizable work of his lifetime. He published brief articles on scattered topics afterwards, apparently unable to find a unifying vision.

Saussure returned to his native Geneva in 1891 as Extraordinary Professor of \ili{Indo-European} languages. He was not named Ordinary Professor of General Linguistics until December of 1906, an additional responsibility added to \ili{Sanskrit}, comparative philology, and the occasional language course. Between 1906 and 1911, he gave three biennial series of lectures on general \isi{linguistics} as required by his new position, all covering similar ground, to his devoted but few students and colleagues \citep[617]{Joseph2012}. We will never know whether he intended to publish this work.  Saussure died in February 1913.  His students gathered their notes together and organized them, publishing the results in 1916 in tribute to their late master as the \citetitle{Saussure1916}, whose reputation gradually grew, until it became justly regarded as the founding document of modern theoretical \isi{linguistics}. The \citetitle{Saussure1959} remains influential today across a broad range of disciplines.

The \citetitle{Saussure1959} comprises for the most part an attempt to establish a new foundation for the science of language. The first chapter of the \textit{Cours}, ``A glance at the history of \isi{linguistics},'' is five pages long in Baskin's 1959 translation. Of \isi{traditional grammar}, Saussure writes that ``It lacked a scientific approach'' \citep[1]{Saussure1959}. He credits \iai{Bopp} with ``realiz[ing] that the comparison of related languages could become the subject matter of an independent science'' \citep[2]{Saussure1959}, noting that he could not have succeeded without Jones's ‘discovery' of \ili{Sanskrit}, which is ``exceptionally well-fitted to the role of illuminating the comparison [with Greek and \ili{Latin}]'' \citep[2]{Saussure1959}.  Importantly for both Sausssure and us, ``the comparative school . . . did not succeed in setting up the true science of \isi{linguistics}. It failed to seek out the nature of its study'' \citep[3]{Saussure1959}. ``Not until around 1870 did scholars begin to seek out the principles that govern the life of languages'' \citep[4]{Saussure1959}. He credits \citet{Whitney1875} and the \isi{Neogrammarians} with realizing that language is ``a product of the collective mind of linguistic groups'' \citep[5]{Saussure1959} and not ``an organism that develops independently'' \citep[5]{Saussure1959}, as Schleicher had claimed in rhetorical support of his Darwinian argument. Saussure concludes the chapter by stating that ``the fundamental problems of general \isi{linguistics} still await solution,'' a solution that he proceeds to outline in the rest of the book.

One of the most important components of Saussure's solution was the observation that the science of language could be divided into the analysis of single states of a language, \textit{états de langue} -- synchronic\is{synchrony} \isi{linguistics} -- and the analysis of a succession of such states -- diachronic\is{diachrony} \isi{linguistics}. Tellingly, the chapter devoted to this fundamental distinction was entitled ``Static and evolutionary \isi{linguistics},'' \citep[79]{Saussure1959}, a title that echoes the earlier connection between \isi{linguistics} and \isi{evolutionary biology}.\is{evolution} He used the term \textit{diachronic}\is{diachrony} interchangeably with \textit{evolutionary}\is{evolution} but \textit{diachronic}\is{diachrony} eventually won out, presumably because it lacked any suggestion of a connection to biological evolution; perhaps also because of the bad reputation that the term \textit{evolution}\is{evolution} had gained when applied to the study of the human language faculty since being banned by the Société de Linguistique de Paris in 1866.  Saussure made clear that \isi{synchrony} was more important in the last two paragraphs of the chapter:

\begin{quote}
\textit{Synchronic linguistics} will be concerned with the logical and psychological relations that bind together coexisting terms and form a system in the collective mind of speakers.

\textit{Diachronic linguistics},\is{diachrony} on the contrary, will study relations that bind together successive terms not perceived by the collective mind but substituted for each other without forming a system.
\end{quote}

It is fair to say that Saussure's distinction was the most important factor leading to a shift in the focus of \isi{linguistics} in the century since.  In the last chapter of the book, ``Concerning retrospective \isi{linguistics},'' Saussure reduces \isi{diachrony} to \isi{synchrony} and thus dispenses with the former in a single argument.  Diachrony,\is{diachrony} for Saussure, is simply ``an infinite number of photographs, taken at different times'' \citep[212]{Saussure1959}.

The photographic \isi{analogy} is striking.  The Lumière brothers had perfected the \textit{cinématographe} in 1895, only a decade before Saussure's first lectures on general \isi{linguistics} and it had quickly grown in popularity in Saussure's \ili{French}-speaking world. Cinematography allowed for the depiction of passage through time as a sequence of successive photographs or states: a moving picture is a succession of photographs shot and projected at regular very short intervals. The individual photographs matter much more than the interval, which is always the same.  No single transition is of interest. The depiction of the passage of time is simply an illusion created by the sequence of static photographs. We cannot know if the cinema had any influence on Saussure's thought, but he was very clear in asserting that ``From the speakers' point of view, \isi{diachrony} does not exist; speakers deal only with a state.'' \citep[594]{Joseph2012}.

Saussure had struggled through his life with the problem that the field to which he had devoted his career, evolutionary \isi{historical linguistics}, had not been able find any principled theoretical basis. Once he sat down to provide a theory of language, the result was a theory of what we now call linguistic structure or grammar, not of historical linguistic evolution. He could come to terms with this conclusion only by killing the field that had borne him. His reduction of \isi{diachrony} to \isi{synchrony} and his double insistence that a linguistic system must be synchronic and that diachronic\is{diachrony} \isi{linguistics} is not systematic in this sense was the most important factor leading to the radical shift that the field underwent in the next few decades. By 1945, the synchronic system and Saussure had won. Modern \isi{linguistics} \textit{was} synchronic \isi{linguistics} and attempts to tie \isi{linguistics} to \isi{evolution} in any way had been abandoned.\footnote{As late as 1929, though, Edward Sapir could still proudly proclaim, in an article entitled ``The status of \isi{linguistics} as a science,'' that ``Many of the formulations of comparative \ili{Indo-European} \isi{linguistics} have a neatness and a \isi{regularity} which recall the formulae, or the so-called laws, of natural science.'' \citep[160]{sapir1929}. One could write a book about this confusion of formulae with laws. Saussure understood the difference.}

\section{Principles of \isi{cultural evolution}}

Historical or evolutionary \isi{linguistics} had been one of the most successful academic enterprises of the nineteenth century, amassing concrete results such as the establishment of historical language families and the reconstruction of a number of proto-languages.  Saussure's conjecture on the vowel system of \ili{Indo-European}, for example, was confirmed by the decipherment of \ili{Hittite} in the early 20\textsuperscript{th} century and the observation made by \citet{Kurylowicz1935} that a number of consonants in the recently deciphered ancient \ili{Hittite} language, not found in other \ili{Indo-European} languages, lined up nicely in their distribution with the  \textit{coefficients sonantiques} that Saussure had proposed based solely on historical analysis.  The problem that Saussure confronted in his theoretical work was that, unlike Darwinian evolutionary\is{evolution} biology, which was grounded in the great insight of natural \isi{selection}, the field had no explanatory basis. His solution was to dismiss the field.

All biologists agree that, in the memorable words of Theodosius \citet{dobzhansky2013}, ``nothing in \isi{biology} makes sense except in the light of \isi{evolution}.''  Darwin's theory of natural \isi{selection} provides a satisfying sweeping explanation for the origin and evolution of all biological species, while the modern synthesis provides the genetic\is{genetics} mechanism that underpins reproductive success. As the \isi{Neogrammarians} noted themselves (see above), there are no equivalent principles in \isi{historical linguistics}.  All the ‘laws', exceptionless though they may be, are contingent facts.

Can there be general principles of linguistic change?\footnote{\citet{Kurylowicz1935} set out a set of six laws of \isi{analogy},\is{analogical} but these are far from general principles.} Saussure certainly did not propose any, but there are reasons for optimism.  William Labov, whom many consider to be the most important linguist of our time, published a massive work in three volumes \citep{Labov1994,Labov2001,Labov2010} entitled just that: \textit{Principles of Linguistic Change}. \citet{blevins2004} has written an influential book entitled \citetitle{blevins2004}. Others, notably \citet{deboer2001}, \citet{galantucci2005}, Simon Kirby \citep{verhoefetal2014}, and Kenny Smith \citep{kirbyetal2015} have looked at emergent systems of language based on evolutionary\is{evolution} models. None of these have direct ties to \isi{Darwinian theory}. Some, though, are firmly within the tradition of what has come to be called \isi{cultural evolution}.

The founder of \isi{cultural anthropology}, Edward B. Tylor, had embraced \isi{evolution} from the start \citep{Tylor1871} but he entwined \isi{evolution} with material progress, providing fodder for a long detour into \isi{social Darwinism}, an idea also traceable in part to Ernst Haeckel, and to related eugenics movements around the globe. \isi{Franz Boas}, the most prominent anthropologist of his time, took up the flag against all things evolutionary in culture \citep{boas1928} and drove \isi{evolution} from the field for close to a century in both anthropology and the closely related sociology. The idea that culture could and should be studied from an evolutionary\is{evolution} point of view raised its head occasionally (e.g., \citealt{Sahlins1960}), with Marvin Harris especially exploiting the notion of an ecological niche as an explanatory device in cultural adaptation (e.g., \citealt{Harris1979}), but the most important development in anthropology was the anti-empirical direction of the field under the influence of \citet{Geertz1973}, which was profoundly anti-evolutionary\is{evolution} and even anti-explanatory. This line of thought soon led to a great weakening of the \isi{scope} of the field of \isi{cultural anthropology}.

The disintegration of cultural anthropology left an opening for a more biologically oriented study of human behavior, often called \isi{behavioral ecology}. A major thrust of this biological approach was the direct application of insights from \isi{evolutionary biology} to \isi{cultural evolution}.\is{evolution} One of the most influential lines of work in this direction was led by a duo made up of a biologist (Peter Richerson) and an anthropologist (Robert Boyd), who, over a thirty-year collaboration, have written three influential books \citep{boydandricherson1985,boydandricherson2005,Richersonandboyd2005} and many articles in which they directly and precisely apply principles from evolutionary theory to human culture, with many examples, works which have unfortunately had little influence in \isi{linguistics}. Their framework provides a simple answer to Saussure's concern about the lack of principled explanation in evolutionary\is{evolution} \isi{linguistics}: \isi{cultural evolution} can be explained using precise methods and we can start by exploiting principles taken quite directly from \isi{evolutionary biology}.  Applying these methods to language requires that we first step back from the position that has dominated our field for the last half century and accept Sapir's position: language is a product of the interaction of biology and \isi{culture} and we cannot understand it by confining ourselves to one or the other.  Once we adopt this position, we can look at how languages evolve culturally, on the basis of well-established principles.\footnote{This research has nothing to say about the evolution of the language faculty, only about the evolution of individual languages.}

Taking direct inspiration from Richerson and Boyd, over the last half decade I have shown that a simple well-known principle from ecology, Gause's law of competitive exclusion \citep{Lotka1925,Volterra1926,Volterra1931,Gause1934}, provides very satisfying explanations for a variety of important long-standing problems in \isi{morphology} and lexicology, including the absence of lexical synonyms, morphological \isi{productivity}, \isi{allomorphy}, the existence of inflectional\is{inflection} classes, and the relation between morphology and writing \citep{lindsayandaronoff2013,aronoffandlindsay2015,aronoff2016,bergandaronofftoappeara}.

\section{Conclusion}

The moment I discovered Gause's principle, I was seized by an Andersonian impulse that I could not shake until I had satisfied myself that I understood what had had happened in the relationship between \isi{linguistics} and \isi{evolutionary biology} since Darwin and Schleicher. Why, as Morris Halle pronounced many years ago, did Saussure never publish the \citetitle{Saussure1959} if it was so important?  Why does mainstream academia pay lip service at best to Saussure's most accomplished work, the \citetitle{Saussure1878}?  Why did this greatest linguist of his time publish so little? Why did historical \isi{linguistics}, the most successful human science of the 19\textsuperscript{th} century, fall into the tenuous position that it holds today? And finally, how should we approach the relation between evolutionary\is{evolution} theory and \isi{linguistics} today?  In this piece, I have begun to answer these questions for myself, in the profound belief, which I share with Steve Anderson and Peter Matthews, that we cannot get lasting answers unless we understand the basis of the questions that drive our work and thought. Steve himself has recently written cogently about taking evolutionary\is{evolution} biology seriously in any discussion of language. Notably, in \citet{anderson2013}, he has reminded us that that the general properties of languages are not necessarily attributable solely to the language faculty.  Many may have an external basis, and some may have been incorporated into the language faculty by natural \isi{selection} itself.

{\sloppy
\printbibliography[heading=subbibliography,notkeyword=this]
}
%\section*{Abbreviations}
%\section*{Acknowledgements}


\end{document}