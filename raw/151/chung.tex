\documentclass[output=paper,
modfonts
]{LSP/langsci}   


%\input{localpackages.tex}
%%Copy this to localcommands.tex

\usepackage[english]{babel}
\usepackage{amsmath}
\usepackage{amssymb,amsfonts,textcomp}
\usepackage{array}
\usepackage{hhline}
\usepackage{hyperref}

\newenvironment{styleStandard}{}{}
\newenvironment{stylelsAbstract}{}{}
\newenvironment{stylelsSectioni}{}{}
\newenvironment{stylelsSectionii}{}{}
\newenvironment{stylelsBulletList}{}{}
\newenvironment{styleBibliographyi}{}{}
\newenvironment{listWWNumxxvleveli}{}{}
\newenvironment{listWWNumxxvlevelii}{}{}
\newenvironment{listWWNumxxvleveliii}{}{}
\newenvironment{listWWNumxxvleveliv}{}{}
\newenvironment{listWWNumixleveli}{}{}
\newenvironment{listWWNumixlevelii}{}{}
\newenvironment{listWWNumixleveliii}{}{}
\newenvironment{listWWNumixleveliv}{}{}

\newcommand\textstyleListLabelxvi[1]{#1}
\newcommand\labellistWWNumxxvleveli{\thelistWWNumxxvleveli.}
\newcommand\labellistWWNumxxvlevelii{\thelistWWNumxxvlevelii.}
\newcommand\labellistWWNumxxvleveliii{\thelistWWNumxxvleveliii.}
\newcommand\labellistWWNumxxvleveliv{\thelistWWNumxxvleveliv.}
\newcommand\labellistWWNumixleveli{[F0B7?]}
\newcommand\labellistWWNumixlevelii{\textstyleListLabelxvi{o}}
\newcommand\labellistWWNumixleveliii{[F0A7?]}
\newcommand\labellistWWNumixleveliv{[F0B7?]}

\newcounter{listWWNumxxvleveli}
\newcounter{listWWNumxxvlevelii}[listWWNumxxvleveli]
\newcounter{listWWNumxxvleveliii}[listWWNumxxvlevelii]
\newcounter{listWWNumxxvleveliv}[listWWNumxxvleveliii]
\newcounter{itemize} 
\ChapterDOI{10.5281/zenodo.495450}
\title{Another way around causatives in Chamorro}

\author{%
 Sandra Chung\affiliation{University of California, Santa Cruz}
}

% \chapterDOI{} %will be filled in at production
% \epigram{}

\abstract{%
In \citegen{anderson1992} theory of a-morphous morphology, the traditional observation that
\isi{inflection} occurs ``outside of'' derivation follows from the assumption that only lexically
complete stems can instantiate morphosyntactic representations. Anderson discusses an apparent counterexample to the traditional observation that involves causative verbs and number agreement in the Austronesian language Chamorro. Anderson defuses the apparent counterexample by proposing, following \citealt{durie1986}, that Chamorro number agreement is a derivational, rather than inflectional, process. I show that there is a different way of finessing the issue that preserves the intuition that Chamorro number agreement is inflectional. This alternative takes the causative ‘prefix’ to be a prosodically deficient verb, in the overall spirit of \citealt{anderson2005}.
}

\begin{document}
\maketitle

\section{Introduction}

In Anderson's theory of a-morphous morphology, the traditional
observation that inflection occurs ``outside of'' \isi{derivation} follows
from the assumption that only lexically complete stems\is{stem} can instantiate
\isi{morphosyntactic} representations. \citet[127--128]{anderson1992} discusses an
apparent counterexample to the traditional observation from \ili{Chamorro}, an
\ili{Austronesian} language of the Mariana Islands. \ili{Chamorro} has \isi{causative}
verbs which, according to previous accounts, are formed by attaching the
prefix \emph{na'}- to a verb or adjective (see e.g. \citealt{baker1985,gibson1980}, \citealt[108]{safford1904}, and \citealt[247--249]{topping1973}). The
point of interest is that \emph{na'}- can attach to a verb or adjective
that already shows \isi{number agreement}. Assuming that \emph{na'}- is
derivational\is{derivation} but \isi{number agreement} is inflectional,\is{inflection} the fact that
\emph{na}'- can occur ``outside of'' \isi{number agreement} is problematic.
Anderson defuses the apparent counterexample by proposing, following
\citet[364--365]{durie1986}, that \ili{Chamorro} \isi{number agreement} is a derivational,\is{derivation}
rather than inflectional,\is{inflection} process.

Here I explore a different way of finessing the issue, one that
preserves the intuition that \ili{Chamorro} \isi{number agreement} is inflectional.\is{inflection}
The key to this alternative is to take the \isi{causative} ``prefix'' to be a
\isi{prosodically deficient} verb, in the spirit of Anderson's \citeyear{anderson2005} approach
to clitics as phrasal affixes.\is{affix} \ili{Chamorro} has a small class of
\isi{prosodically deficient} verbs that are instances of Zwicky's \citeyear{zwicky1977a} bound\is{bound morpheme}
words.\is{morpheme} These forms have the morphosyntax of verbs, but cannot serve as
phonological words on their own. Instead, they must remedy their
prosodic deficiency by undergoing stray \isi{adjunction} to the phonological
word to their immediate right, which is always the first phonological
word of their complement.\is{complement}

I show that much of the distinctive profile of \ili{Chamorro} causatives is
accounted for if the \isi{causative} \emph{na'} is treated as a prosodically
deficient verb that selects a vP complement.\is{complement} Moreover, once this route
is taken, \ili{Chamorro} causatives no longer pose a threat to the
``outside-inside'' order of \isi{inflection} and derivation,\is{derivation} even if number
agreement is inflectional.\is{inflection} This is because the \isi{causative} \emph{na'} that
can appear ``outside of'' \isi{number agreement} is not, in fact, derivational\is{derivation}
morphology, but rather the \isi{prosodically deficient} content of a syntactic
verb.

\sectref{sec:chung:2} of this paper gives a mini-introduction to the morphosyntax of
\ili{Chamorro} clauses. \sectref{sec:chung:3} presents the basics of causatives and their
interaction with the language's two types of subject-verb agreement.
\sectref{sec:chung:4} looks closely at Durie's \citeyear{durie1986} evidence that \ili{Chamorro} number
agreement is derivational and concludes that it is not decisive. Then,
\sectref{sec:chung:5} gives an overview of \ili{Chamorro}'s \isi{prosodically deficient} verbs.
\sectref{sec:chung:6} proposes that the \isi{causative} \emph{na'} is a prosodically
deficient verb and explores some positive consequences of this proposal.
\sectref{sec:chung:7} handles some challenges, and \sectref{sec:chung:8} concludes.

\section{\ili{Chamorro} Morphosyntax in Brief}\label{sec:chung:2}

\ili{Chamorro} is a head-initial language that allows predicates of all major
category types and a range of null arguments.\is{arguments} When the predicate is a
verb or adjective, the default \isi{word order} of the clause is Predicate
Subject Object Other, but the order of \isi{arguments} and adjuncts after the
predicate is flexible (see \citealt{chung1998} and the references cited
there).\footnote{Most of the \ili{Chamorro} examples cited here are from the CD database,
  which consists of some 30,000 sentences constructed by Chamorros in
  the CNMI as illustrative examples for the revised \ili{Chamorro}-\ili{English}
  dictionary. Other examples are from published sources listed in the
  references; unattributed examples are from my fieldwork.}

\begin{exe}
\ex \label{ex:chung:1}
\begin{xlist}
\ex \label{ex:chung:1a} 
\gll Ha baba si Antonio i petta.\\
\textsc{p.agr} open \textsc{unm} Antonio the door.\\
\glt `Antonio opened the door.'
\ex \label{ex:chung:1b} \gll K\emph{um}åti i neni sa' ma'å'ñao ni sanye'yi'.\\
\textsc{n.agr.}cry the baby because \textsc{n.agr.}afraid \textsc{obl} spider\\
\glt `The child cried because she's afraid of the spider.' (CD, entry for
\emph{sanye'yi'}) 
\z
\z

DPs are case-marked\is{case} with a pro\isi{clitic} that occurs to
their immediate left. There are three morphological\is{morphology} cases:\is{case} unmarked,
local, and oblique. Subjects, direct objects, possessors, predicate
nominals, the objects of most overt prepositions, and DPs that occupy
topic or \isi{focus} position occur in the unmarked case,\is{case} which is overtly
realized only when the DP is a proper name. Otherwise, DPs that denote
locations or goals occur in the local case;\is{case} most other types of DPs
occur in the oblique case.\is{case}

\ea \label{ex:chung:2}
\gll Ma rikuknisa si Esthera ni finatton-ña gi hunta.\\
\textsc{p.agr} recognize \textsc{unm} Esther \textsc{obl} arrival-\textsc{poss} \textsc{loc} meeting\\
\glt `They acknowledged Esther for her coming to the meeting.' (CD, entry for \emph{rikuknisa})
\z

Predicates that are verbs or adjectives show subject-verb agreement via
forms that also indicate mood (realis vs. irrealis) and are sensitive to
transitivity.\is{transitivity} There are two types of subject-verb agreement.
\emph{Person-and-number agreement} (glossed \textsc{p.agr} in the
examples\textsc{)} is realized via forms that could be analyzed as
proclitics or prefixes, but are written as separate words in the
\ili{Chamorro} \isi{orthography}; see the \isi{paradigm} in (\ref{tab:persnumagr}).\footnote{\ili{Chamorro} has
  various standard and nonstandard orthographies (see \citealt[Appendix A]{chung1998}). The \isi{orthography} used here, which was officially adopted
  by the CNMI legislature in 2010, differs in small ways from the
  transcription used in \citealt{chung1998}, and more substantially from earlier
  spelling systems, including the official \isi{orthography} on Guam.} In the
realis mood, this type of agreement is found only on transitive\is{transitivity} verbs;
in the irrealis mood, it is found on all verbs and adjectives.

\ea \label{ex:chung:3} Person-and-Number Agreement\\
%\begin{table}
	\begin{tabular}{lcc}
	\lsptoprule
	& Realis & Irrealis \\
	\midrule
	1\textsc{sg}& hu & (bai) hu / bai \\
	
	2\textsc{sg}& un & un\\
	
	3\textsc{sg}& ha & u\\
	
	1\textsc{incl.du/pl} & ta & (u)ta\\
	
	1\textsc{excl.du/pl} & in & (bai) in\\
	
	2\textsc{du/pl} & en & en\\
	
	3\textsc{du/pl} & ma & u (\textsc{intr}) / uma (\textsc{tr}) \\
	\lspbottomrule
	\end{tabular}
%\caption{Person-and-Number Agreement.}
\label{tab:persnumagr}
%\end{table}
\z

%\begin{exe}
%\ex \label{ex:chung:} Person-and-Number Agreement
%\begin{xlist}
%\ex \label{ex:chung:} Realis Forms\\
%\begin{tabular}[t]{@{}p{1in}l}
%1 sg. & hu \\
%
%2 sg. & un \\
%
%3 sg. & ha \\
%
%1 incl. du./pl. & ta\\
%
%1 excl. du./pl. & in\\
%
%2 du./pl. & en\\
%
%3 du./pl. & ma\\
%\end{tabular}
%\ex \label{ex:chung:} Irrealis Forms\\
%\begin{tabular}[t]{@{}p{1in}l}
%1 sg. & (bai) hu / bai \\
%
%2 sg. & un\\
%
%3 sg. & u\\
%
%1 incl. du./pl. & (u)ta\\
%
%1 excl. du./pl. & (bai) in\\
%
%2 du./pl. & en\\
%
%3 du./pl. & u {[}intransitive{]} / uma {[}transitive{]}\\
%\end{tabular}
%\end{xlist}
%\end{exe}

\emph{Number agreement} (glossed \textsc{n.agr)} is realized via a
prefix or infix;\is{infix} see the \isi{paradigm} in (\ref{tab:numagr}). This type of agreement is
found only on intransitive verbs and adjectives.\footnote{The choice
  between the two realizations of realis singular \isi{number agreement} is
  determined lexically. Generally, -\emph{um}- is used for event
  predicates, as well as for state predicates in the \isi{inchoative} \isi{aspect};
  otherwise, the agreement is generally unrealized for state predicates.
  But there are exceptions. The realizations of \isi{plural} \isi{number agreement}
  have a final /n/ that undergoes the \isi{alternation} known as nasal
  substitution.}

\ea \label{ex:chung:4} Number Agreement\\
%\begin{table}
	\begin{tabular}{lcc}
		\lsptoprule
		& Realis & Irrealis \\
		\midrule
		\textsc{sg/du} & -um- / --- & --- \\
		\textsc{pl} & man- & fan- \\
		\lspbottomrule
	\end{tabular}
%	\caption{Number Agreement.}
	\label{tab:numagr}
%\end{table}
\z

%\begin{exe}
%	\ex \label{ex:chung:} Number Agreement
%	\begin{xlist}
%		\ex \label{ex:chung:} Realis Forms\\
%			\begin{tabular}[t]{@{}p{1in}l}
%			Sg./Du. & -um- /  --- \\
%			Pl. & man- \\
%			\end{tabular}
%		\ex \label{ex:chung:} Irrealis Forms\\
%			\begin{tabular}[t]{@{}p{1in}l}
%			Sg./Du.  & --- \\
%			
%			Pl. & fan-\\
%			\end{tabular}
%	\end{xlist}
%\end{exe}

Notice that dual is aligned with \isi{plural} for the purposes of
\isi{person-and-number agreement}, but with singular for the purposes of
\isi{number agreement}. This will become important later.

Both types of agreement are the default realizations of subject-verb
agreement for predicates of the relevant type, and fully productive;
e.g. they can be added to recently borrowed words (as in \ref{ex:chung:5a}), even
when the borrowings are creative or innovative (as in \ref{ex:chung:5b}).

\ea \label{ex:chung:5}
\ea \label{ex:chung:5a}
\gll Man-meeting ham gi Lunis.\\
\textsc{n.agr-}have.meeting we \textsc{loc} Monday\\
\glt `We had a meeting on Monday.' (CNMI Senate session SJ 17--22: 20)

\ex \label{ex:chung:5b} \gll  {Bai hu} ``love-mark'' i kurason-mu.\\
\textsc{p.agr} love-mark the heart-\textsc{poss}\\
\glt `I will ``love-mark'' your heart.' (EM 60)
\z
\z

Finally, the two types of agreement have overlapping distributions.
Transitive\is{transitivity} verbs show only \isi{person-and-number agreement} (see \ref{ex:chung:1a}, \ref{ex:chung:2},
\ref{ex:chung:5b} and \ref{ex:chung:6a}); intransitive predicates in the realis mood show only
\isi{number agreement} (see \ref{ex:chung:1b}, \ref{ex:chung:5a}, and \ref{ex:chung:6b}); but intransitive predicates
in the irrealis mood show both. Note that when the two types of
agreement co-occur, \isi{person-and-number agreement} occurs ``outside
of'' -- i.e. to the left of -- \isi{number agreement} (see \ref{ex:chung:6c}).

\ea \label{ex:chung:6}
\ea \label{ex:chung:6a}
\gll Hu afuetsas gui' para u atan yu'.\\
\textsc{p.agr} compel her \textsc{fut} \textsc{p.agr} look.at me\\
\glt `I compelled her to (lit. that she would) look at me.' (CD, entry for \emph{afuetsas})

\ex \label{ex:chung:6b} \gll Durånti-n i tinaitai, bula mang-åti.\\
during-\textsc{l} the prayer many \textsc{n.agr}-cry.\\
\glt `During the prayer, many cried.' (CD, entry for \emph{durånti})

\ex \label{ex:chung:6c} \gll Ti para u fang-åti i famalåo'an.\\
not \textsc{fut} \textsc{p.agr} \textsc{n.agr}-cry the women\\
\glt `The women are not going to cry.'
\z
\z 

With this much in place, let us now zero in on causatives.

\section{Causatives}\label{sec:chung:3}

Previous accounts describe \ili{Chamorro} causatives as formed by adding the
prefix \emph{na'}- to a verb or adjective (see e.g. \citealt{baker1985,gibson1980,safford1904,topping1973}). This process creates a
derived\is{derivation} transitive\is{transitivity} verb with an additional argument,\is{arguments} which denotes the
causer. The causer argument\is{arguments} is realized as the subject of the \isi{causative};
the subject of the original predicate (henceforth the \emph{inner}
predicate) is realized as the direct object of the \isi{causative}; and the
direct object of the inner predicate, if any, is realized as an oblique
(see \citealt{gibson1980}). To illustrate, the causatives \emph{na'baba} `make
open', \emph{na'kåti} `make cry', and \emph{na'ma'å'ñao} `make afraid,
frighten' are derived, respectively, from the transitive verb
\emph{baba} `open' (cf. \ref{ex:chung:1a}), the intransitive verb \emph{kåti} `cry'
(cf. the first clause of \ref{ex:chung:1b}), and the adjective \emph{ma'å'ñao}
`afraid' (cf. the second clause of \ref{ex:chung:1b}).

\ea \label{ex:chung:7}
\ea \label{ex:chung:7a}
\gll In na'-baba si Antonio ni petta.\\
\textsc{p.agr} \textsc{caus-}open \textsc{unm} Antonio \textsc{obl} door\\
\glt `We made Antonio open the door.'

\ex \label{ex:chung:7b} \gll Ha na'-kåti si Gene i lahi-ña anai ha lalåtdi.\\
\textsc{p.agr} \textsc{caus-}cry \textsc{unm} Gene the son-\textsc{poss} when \textsc{p.agr } scold\\
\glt `Gene made his son cry when he scolded him.' (CD, entry for \emph{kåti})

\ex \label{ex:chung:7c} \gll Un na'-ma'å'ñao yu' ni taklalo'-mu.\\
\textsc{p.agr} \textsc{caus-}afraid me \textsc{obl} great.anger-\textsc{poss}\\
\glt `You made me afraid with your great anger.'
\z
\z

Gibson's \citeyear{gibson1980} investigation of the syntax of \ili{Chamorro} causatives
established three points that will be in the spotlight here. First,
causatives have the morphosyntax of the language's transitive\is{transitivity} verbs
\citep[86--91]{gibson1980}. Like other transitive\is{transitivity} verbs, they can occur in the
passive.\is{passive}\footnote{Passive\is{passive} verbs are formed with the \isi{infix} -\emph{in}- or
  the prefix \emph{ma}-. The choice between -\emph{in}- and \emph{ma}-
  is determined primarily by the number of the \isi{passive} \isi{agent}:
  -\emph{in}- appears when the \isi{agent} is singular, \emph{ma}- when the
  \isi{agent} is dual/\isi{plural} or implicit (see \citealt[38, note 8]{chung1998}).}

\begin{exe}
\ex \label{ex:chung:8}
\begin{xlist}
\ex \label{ex:chung:8a}
\gll Ma-na'-gimin i patgun åmut ni ti dinanchi.\\
\textsc{n.agr.pass}-\textsc{caus-}drink the child medicine \textsc{comp} not \textsc{n.agr.}right\\
\glt `The child was made to drink medicine that was not right.' (CD, entry for \emph{tumaiguihi})

\ex \label{ex:chung:8b} \gll Kulan n\emph{in}a'-ma'å'ñao i biha nu esti na klåsi-n tinanum.\\
sort.of \textsc{n.agr.pass.caus-}afraid the old.lady \textsc{obl} this \textsc{l} type-\textsc{l} plant\\
\glt `The old lady was kind of made afraid by this type of plant.' (MAK 2)
\end{xlist}
\end{exe}

They can also occur in the antipassive.\footnote{Antipassive verbs are
  usually formed with the prefix \emph{man-/fan}-. However, some
  transitive\is{transitivity} verbs have \isi{suppletive} antipassive forms (e.g. the
  antipassive of \emph{kånnu'} `eat' is \emph{chotchu}); others have
  antipassive forms identical to their transitive\is{transitivity} forms (e.g.
  \emph{gimin} `drink'). The antipassive of a \isi{causative} is formed by
  shifting primary \isi{stress} to the \isi{causative} prefix.}

\begin{exe}
\ex \label{ex:chung:9}
\gll Mu-nana'-gupu papaloti si Juanito gi kantu-n tåsi.\\
\textsc{n.agr-ap.caus-}fly.\textsc{prog} kite \textsc{unm} Juanito
\textsc{loc} edge-\textsc{l} ocean\\
\glt `Juanito is flying a kite (lit. making a kite fly) by the seashore.'
(CD, entry for \emph{na'gupu})
\end{exe}

And they can be used to create \isi{reciprocals} -- derived\is{derivation} intransitive verbs,
formed with the stressed prefix \emph{á}-, which are \ili{Chamorro}'s
primary means of expressing reciprocal meaning.

\begin{exe}
\ex \label{ex:chung:10}
\gll Kao um-á-na'-patcha hamyu ni feggun?\\
\textsc{q} \textsc{n.agr-recip-caus-}touch you.\textsc{pl} \textsc{obl} stove\\
\glt `Did you two make each other touch the stove?'
\end{exe}

Second, causatives can be derived\is{derived} from verbs that are morphologically
complex\is{complexity} (see \citealt[114--121]{gibson1980}). The causatives in the examples in
(\ref{verbder}) are derived from verbs -- surrounded by brackets -- that are \isi{passive}
(\ref{ex:chung:11a}--\ref{ex:chung:11b}), antipassive \REF{ex:chung:11c}, and reciprocal \REF{ex:chung:11d}.

\begin{exe}
\ex \label{ex:chung:11}\label{verbder}
\begin{xlist}
\ex \label{ex:chung:11a}
\gll In na'-{[}ma-baba{]} as Antonio.\\
\textsc{p.agr} \textsc{caus-pass-}open \textsc{obl} Antonio\\
\glt `We made it be opened by Antonio.'

\ex \label{ex:chung:11b} \gll Bai na'-{[}s\emph{in}aolak{]} hao nu i ma'estra.\\
\textsc{p.agr} \textsc{caus-pass.}spank you \textsc{obl} the teacher\\

\glt `I will let you be spanked by the teacher.' (CD, entry for \emph{sinaolak})

\ex \label{ex:chung:11c} \gll  I bakulu-hu ha-na'-{[}fang-gånna{]} yu'.\\
the shooter.marble-\textsc{poss} \textsc{p.agr-caus-ap-}win me\\
\glt `My shooter marble made me win.' (CD, entry for \emph{båkulu})

\ex \label{ex:chung:11d} \gll Ma na'-{[}á-dispatta{]} i dos tåotao ni mumu.\\
\textsc{p.agr} \textsc{caus-recip-}separate the two person \textsc{comp} \textsc{n.agr.}fight\\
\glt `They separated (lit. caused to separate from each other) the two people
who were fighting.' (CD, entry for \emph{na'ádispatta})
\end{xlist}
\end{exe}


As these observations might lead one to expect, causatives derived\is{derivation} from
morphologically complex\is{complexity} verbs can themselves occur in the passive,\is{passive}
antipassive, or reciprocal.\footnote{Although it is possible in
  principle for causatives formed from a verb in any voice to occur in
  any voice, the naturally occurring data suggest that some combinations
  are more frequent than others. When the \isi{causative} is active transitive\is{transitivity}
  or passive,\is{passive} the inner predicate can be active (transitive\is{transitivity} or
  intransitive), passive,\is{passive} antipassive, or reciprocal. When the \isi{causative}
  is antipassive or reciprocal, the inner predicate is most often active
  (transitive\is{transitivity} or intransitive).} The verbs in boldface in \REF{ex:chung:12} are the
\isi{passive} of a \isi{causative} derived from a passive verb (in \ref{ex:chung:12a}) and the
\isi{passive} of a \isi{causative} derived from an antipassive verb (in \ref{ex:chung:12b}).

\begin{exe}
\ex \label{ex:chung:12} \begin{xlist}
\ex \label{ex:chung:12a}
\gll  {\dots} yan maseha håyi malago'-ña i Lahi-ña para u \textbf{n\emph{in}a'-{\ob}ma-tungu'{\cb}} Gui'.\\
{ } and ever who \textsc{wh.}want-\textsc{poss} the son-\textsc{poss} \textsc{fut}
\textsc{p.agr} \textsc{pass.caus-pass-}know he\\
\glt `{\dots}and whoever his Son wants to cause Him (lit. that He be caused) to
be known by.' (NT 124)

\ex \label{ex:chung:12b} \gll \textbf{N\emph{in}a'-{[}fañ-otsut{]}} anai ma-nå'i måolik na kunseha.\\
\textsc{n.agr.pass.caus-ap-}repent when \textsc{n.agr.pass-}give good
\textsc{l} advice\\

\glt `She repented (lit. was caused to repent) when she was given good
advice.' (CD, entry for \emph{na'fañotsut})
\end{xlist}
\end{exe}

A \isi{causative} can even be derived\is{derivation} from the \isi{passive} of a \isi{causative}, as \REF{ex:chung:13}
shows.

\begin{exe}
\ex \label{ex:chung:13}
\gll Si Josephine ha \textbf{na'-{[}ma-na'-{[}suha{]}{]}} i atgoya gi gui'eng-ña.\\
\textsc{unm} Josephine \textsc{p.agr} \textsc{caus-pass-caus-}go.away the
nose.ring \textsc{loc} nose-\textsc{poss}\\
\glt `Josephine had her nose ring removed (lit. caused the nose ring to be
caused to go away).' (CD, entry for \emph{atgoya})
\end{exe}

Third, the inner predicate -- the verb or adjective from which a
\isi{causative} is derived -- does not show person-number agreement. But,
surprisingly, the inner predicate \emph{does} show \isi{number agreement} (see
\citealt[112--114]{gibson1980}). Inner predicates that are intransitive agree with
the DP that would have been their subject via irrealis \isi{number agreement},
which is unrealized in the singular/dual, but spelled out as the prefix
\emph{fan}- in the \isi{plural}. This \isi{number agreement} is not realized on the
inner predicates in (11--13), because the DPs that would have been their
subjects are singular/dual (e.g. the null pronoun `it' in \REF{ex:chung:11a},
\emph{hao} `you (sg.)' in \REF{ex:chung:11b}, \emph{yu'} `me' in \REF{ex:chung:11c}), but it is
overt on the inner predicates in \REF{ex:chung:14}, because the DPs that would have
been their subjects are \isi{plural}. (Note that the inner predicates in \REF{ex:chung:14}
are clearly not agreeing with the subject of the \isi{causative}, which is
singular.)

\begin{exe}
\ex \label{ex:chung:14} \begin{xlist}
\ex \label{ex:chung:14a}
\gll  Hu na'-{[}fang-gupu{]} i petbus.\\
\textsc{p.agr} \textsc{caus-n.agr-}fly the dust\\
\glt `I made the (particles of) dust fly around.' (CD, entry for \emph{na'gupu})

\ex \label{ex:chung:14b} \gll Ha na'-{[}fan-luhan{]} ham.\\
\textsc{p.agr} \textsc{caus-n.agr-}afraid us\\
\glt `{[}The wind{]} scared us (lit. made us afraid).' (CD, entry for \emph{diripenti})

\ex \label{ex:chung:14c} \gll Ha na'-{[}fan-ma-kotti{]} i guåtdia, ya ha na'-{[}fan-ma-punu'{]}.\\
\textsc{p.agr} \textsc{caus-n.agr-pass-}try the guard and \textsc{p.agr} \textsc{caus-n.agr-pass-}kill\\
\glt `He had the guards brought to trial, and had them killed.' (NT 235)

\ex \label{ex:chung:14d} \gll I abisu ha na'-{[}fan-man-unungu'{]} i taotao na {\ldots}\\
the alarm \textsc{p.agr} \textsc{caus-n.agr-ap-}know.\textsc{prog} the person that\\
\glt `The alarm is letting the people know that{\dots}{[}the typhoon is close{]}.
(CD, entry for \emph{abisu})
\end{xlist}
\end{exe}

\citet{baker1985} used the relative order of the \isi{plural} \emph{fan}- with
respect to the \isi{causative} and \isi{passive} affixes\is{affix} to argue for the Mirror
Principle. As he observed, ``clear examples of agreement morphemes\is{morpheme} that
can appear intermixed with GF-rule morphemes\is{morpheme} seem quite unusual'' \citep[386]{baker1985}. What matters here is that the \isi{plural} \emph{fan}- in the examples
in \REF{ex:chung:14} occurs ``inside of'' -- i.e. to the right of -- the \isi{causative}
\emph{na'}-. Assuming that \emph{fan}- is inflectional\is{inflection} but \emph{na'}-
is derivational,\is{derivation} this ordering appears to counterexemplify the
traditional claim that \isi{inflection} always occurs ``outside of''
derivation.\is{derivation}

\section{Number agreement revisited}\label{sec:chung:4}

A natural question to raise at this point is whether \ili{Chamorro} number
agreement might be derivational\is{derivation} as well.

\subsection{Is it derivational?}\is{derivation}

As \citet[127--128]{anderson1992} observes, this question is answered in the
affirmative by \citet{durie1986}, who contends that across languages, verbal
number -- whether realized by \isi{stem} \isi{suppletion} or productive
affixation\is{affix} -- is ``selectional concord'' (i.e. derivational)\is{derivation} as opposed to
``agreement''. Durie's evidence for this claim comes from various
languages, including \ili{Chamorro}. In the \isi{suppletion} cases he examines (in
e.g. Huichol), verbal number is sensitive to semantic roles like patient
or affected participant, not to syntactic relations like subject.
\ili{Chamorro} \isi{number agreement} does not conform to this pattern, but instead
cross-references the (surface) subject regardless of semantic role; this
is one way that it behaves like a paradigmatic\is{paradigm} case of agreement. Still,
Durie argues that \isi{number agreement} in \ili{Chamorro} is ``inherent verbal
Number morphology''\is{morphology} \citep[364]{durie1986} whereas \isi{person-and-number agreement} is
inflectional,\is{inflection} on the basis of the following:

\begin{itemize}
\item Number agreement distinguishes \isi{plural} from non\isi{plural} (i.e. \isi{plural} from
singular/dual), but the number feature\is{features} on nouns and pronouns
distinguishes singular from nonsingular (i.e. singular from
dual/\isi{plural}), so ``{[}t{]}here is no {[}±\isi{plural}{]} feature\is{features} for the verb
to agree with'' \citep[364]{durie1986}.

\item Number agreement can have an overt pronoun as antecedent, whereas
\isi{person-and-num\-ber agreement} cannot.

\item Number agreement appears in infinitives, imperatives, and attributive
modifiers, whereas \isi{person-and-number agreement} does not.

\item Number agreement is preserved in lexical derivations,\is{derivation} such as
causatives (see above), whereas \isi{person-and-number agreement} is not.
\end{itemize}

These may look like good reasons for classifying \isi{number agreement} as
derivational\is{derivation} -- a move that would make it unsurprising in the extreme
that the \isi{plural} \emph{fan}- can occur ``inside of'' the \isi{causative}
\emph{na'}-. But further examination suggests a more equivocal picture.

\subsection{A second look}

Consider, to begin with, the claim that \ili{Chamorro} nouns and pronouns have
a different number feature\is{features} than what is registered by \isi{number agreement}.
The specific claim is that nouns and pronouns employ the feature\is{features}
{[}±singular{]} -- they distinguish singular from dual/\isi{plural} -- whereas
\isi{number agreement} employs the feature\is{features} {[}±\isi{plural}{]} -- it distinguishes
singular/dual from \isi{plural} (see the \isi{paradigm} in (\ref{tab:numagr})). Assuming that
inflectional\is{inflection} \isi{morphology} is the spell-out of syntactic features,\is{features} the
disconnect between these \isi{features} might seem to pose an insuperable
problem for the view that \isi{number agreement} is inflectional\is{inflection} (but see
below).

Overt pronouns in \ili{Chamorro} do indeed employ the feature\is{features}
{[}±singular{]} -- they distinguish singular from dual/\isi{plural}, as
observed explicitly by e.g. \citet[308]{safford1903}. The second person
independent pronouns \emph{hågu} and \emph{hamyu}, for instance, differ
in that \emph{hågu} refers to just one addressee, while \emph{hamyu}
refers to two or more addressees. The other overt pronouns are similar.
It is less obvious how number is handled in nouns, because most \ili{Chamorro}
nouns do not show obligatory number inflection.\is{inflection} Just a handful of nouns,
listed in (\ref{ex:numnouns}), are inflected\is{inflection} obligatorily, and somewhat irregularly,
for number.

\ea \label{ex:chung:15}\begin{tabular}[t]{@{}lllll}
	a. & \multicolumn{1}{c}{Singular} & \multicolumn{1}{c}{Dual} & \multicolumn{1}{c}{Plural} & \\
	& che'lu & chume'lu & mañe'lu & `sibling'\\
	b. & %\multicolumn{2}{c}{Singular/Dual} & \multicolumn{1}{c}{Plural} & \\
	%& 
	\multicolumn{2}{c}{\lefttabular{låhi}} & lalåhi & `man, son'\\
	&\multicolumn{2}{c}{\lefttabular{palåo'an}} & famalåo'an & `woman'\\
	&\multicolumn{2}{c}{\lefttabular{påli'}} & mamåli' & `priest'\\
	&\multicolumn{2}{c}{\lefttabular{påtgun}} & famagu'un & `child'\\
	&\multicolumn{2}{c}{\lefttabular{saina}} & mañaina & `parent'\\
	\end{tabular}\label{ex:numnouns}
\z

The noun \emph{che'lu} has separate forms for singular, dual, and
\isi{plural}. The other nouns have forms which are usually termed ``singular''
and ``\isi{plural}'' (e.g. \citealt[302--304]{safford1903}, \citealt[325]{topping1973}),
but actually distinguish singular/dual from \isi{plural}. That is, they employ
the feature\is{features} {[}±\isi{plural}{]}. The examples in \REF{ex:chung:16} reveal that when these
nouns refer to just two individuals, they are realized in the
singular/dual form, not the \isi{plural} form.

\ea \label{ex:chung:16}
\ea \label{ex:chung:16a}
\gll  Um-iskuekuela i dos påtgun sanlagu.\\
\textsc{n.agr-}attend.school.\textsc{prog} the two child continental.US\\
\glt `The two children are attending school in the continental U.S.' (CD,
entry for \emph{sanlagu})
\ex \label{ex:chung:16b} \gll Dos na palåo'an u fang-gugulik trigu.\\
two \textsc{l} woman \textsc{p.agr} \textsc{n.agr.ap-}grind.\textsc{prog} grain\\
\glt `Two women will be grinding grain.' (NT 48)
\z
\z

The claim that the nouns in (\ref{ex:numnouns}b) align dual with singular is supported
by naturally occurring data.\footnote{Native speakers' judgements trend
  in the same direction, but are more forgiving. For instance, when
  asked which of the following two forms she would use to refer to two
  children, one speaker commented that \emph{i dos påtgun} `the two
  children' (with the singular/dual form of the noun) was better for
  her, but that \emph{i dos famagu'un} (with the \isi{plural} form of the
  noun) ``will be understood in most circumstances''.} There are 30
instances in the CD database, and 23 instances in the first 150 pages of
the \ili{Chamorro} New Testament (NT), of these nouns occurring in explicitly
dual DPs -- DPs whose noun is preceded by the numeral \emph{dos} `two'.
In 51 out of the combined 53 instances, the noun occurs in the
singular/dual form.

It is now clear that \ili{Chamorro} pronouns employ the feature\is{features}
{[}±singular{]}, but obligatorily inflected\is{inflection} nouns employ the feature\is{features}
{[}±\isi{plural}{]} or -- in the case of \emph{che'lu} -- both features.\is{features} This
makes it reasonable to suppose that \ili{Chamorro} DPs are specified for
{[}±singular{]} \emph{and} {[}±\isi{plural}{]}, even though in the vast
majority of cases, these \isi{features} have no DP-internal realization. But
then the way that number is handled by the agreement system is
compatible with the idea that both types of agreement are inflectional.\is{inflection}
Person-and-\isi{number agreement} simply registers one of the number \isi{features}
(namely, {[}±singular{]}), while \isi{number agreement} registers the other
({[}±\isi{plural}{]})

I now turn to Durie's other evidence that \isi{number agreement} is
derivational.\is{derivation} It consists of the following:

- Number agreement can have an overt pronoun as antecedent, but
\isi{person-and-num\-ber agreement} cannot. (The only pronouns that can antecede
\isi{person-and-number agreement} are null pronouns; see also \citealt[30--31]{chung1998}.) Durie takes these facts, which are illustrated in \REF{ex:chung:17}, to show
that \isi{person-and-number agreement} is ``anaphoric'', but \isi{number agreement}
is not.

\ea \label{ex:chung:17}
	\ea \label{ex:chung:17a} \gll  Yayas (gui').\\
		\textsc{n.agr.}tired s/he\\
		\glt `S/he is tired.'
	\ex \label{ex:chung:17b} \gll  Ha fåhan (*gui') i lepblu.\\
		\textsc{p.agr} buy s/he the book\\
		\glt `S/he bought the book.'
	\z
\z

Now, the contrast in \REF{ex:chung:17} \emph{could} ultimately reflect a difference
between \isi{derivation} and inflection.\is{inflection} But it is equally likely that it
flows from some linguistic notion of ``efficiency'' or ``brevity'' (cf.
Grice) plus the featural content of the two types of agreement.
Person-and-\isi{number agreement} encodes exactly the same \isi{features} as
\ili{Chamorro} pronouns -- namely, person \isi{features} and {[}±singular{]} -- so a
ban that prevents this type of agreement from being anteceded by an
overt pronoun contributes to the goal of minimizing redundancy. A
comparable ban on \isi{number agreement} would have no rationale, because
\isi{number agreement} encodes a different feature\is{features} -- {[}±\isi{plural}{]}.

- Number agreement appears in infinitives, imperatives, and attributive
modifiers, but \isi{person-and-number agreement} does not. Consider the
imperative in \REF{ex:chung:18}.

\ea \label{ex:chung:18}
\gll (*En) Fan-man-hokka' sa' bula pineddung mångga gi egga'an.\\
\textsc{p.agr} \textsc{n.agr-ap-}pick because \textsc{n.agr.}many fallen.\textsc{l} mango \textsc{loc} morning\\
\glt `Go and do some picking, because there were many fallen mangos in the
morning.' (CD, entry for \emph{poddung})
\z

To the extent that this observation is valid,\footnote{In conjoined
  imperatives, the leftmost imperative verb does not show
  \isi{person-and-number agreement}, but verbs in subsequent conjuncts
  generally show irrealis \isi{person-and-number agreement} as well as number
  agreement (if applicable). The embedded ``clause'' in \isi{restructuring}
  constructions can either be inflected\is{inflection} like an infinitive or show
  realis \isi{person-and-number agreement}; see 6.2.} it \emph{could} bear on
the contrast between \isi{derivation} and inflection,\is{inflection} but other explanations
are possible. Suppose, for instance, that \isi{number agreement} realizes a
feature\is{features} of small v, whereas \isi{person-and-number agreement} realizes
\isi{features} of T. Then \isi{number agreement} would be expected to appear in
infinitives and imperatives, because these constructions are at least
vPs; there might be no similar expectations for person-and-number
agreement. I will adopt a version of this approach below. As for
attributive modifiers, it should be noted that \ili{Chamorro} allows relative
clauses to precede or follow the head NP; it also allows relative
clauses whose head NP is null (see \citealt{borja2015}). The
attributive modifiers that show \isi{number agreement} can straightforwardly
be analyzed as predicates of one or another of these relative clause
types.

- Finally, \isi{number agreement} is claimed to be preserved in lexical
derivations,\is{derivation} such as causatives and what Durie calls ``nominal
derivatives''. Causatives are, of course, the focus of investigation
here. The ``nominal derivatives'' are not, in fact, derived nouns but
rather relative clauses whose head NP is null. Two of Durie's examples
are given below, with the spelling normalized. In these constructions,
the word that shows \isi{number agreement} is the verb of the relative clause,
which happens to be intransitive.

\ea \label{ex:chung:19}
	\ea \label{ex:chung:19a}\gll  i h\emph{um}ånao \\
		the \textsc{n.agr.}go\\
		\glt `the (one) who went' (translated by Durie as `the goer')
	\ex \label{ex:chung:19b}\gll i man-hånao\\
		i \textsc{n.agr-}go\\
		\glt `the (ones) who went' (translated by Durie as `the goers (\textgreater{}
		2)')
	\z
\z

Notice that when the verb of the relative clause is transitive,\is{transitivity} it can show
\isi{person-and-number agreement}; see the relative clauses in brackets
below.\footnote{The verb of the relative clause can also show
  \isi{wh-agreement}, but that is irrelevant here.} This too is expected if
these constructions are relative clauses.

\ea \label{ex:chung:20}
	\ea \label{ex:chung:20a}\gll  Abånsa {[}i un chochogui{]}.\\
		advance the \textsc{p.agr} do.\textsc{prog}\\
		\glt `Go forward with the (thing) which you are doing.' (CD, entry for
		\emph{abånsa})
	\ex \label{ex:chung:20b} \gll Hu angokku na para un cho'gui {[}i hu faisin hao{]}.\\
		\textsc{p.agr} trust \textsc{comp} \textsc{fut} \textsc{p.agr} do the \textsc{p.agr} ask
		you\\
		\glt `I trust that you will do the (thing) which I ask you.' (CD, entry for
		\emph{angokku})
	\z
\z

In the end, the evidence cited by Durie provides no firm basis for
classifying \isi{number agreement} as derivational\is{derivation} \emph{or} inflectional.\is{inflection} But
then we are back to the original conundrum: why can the \isi{plural}
\emph{fan}- occur ``inside of'' the \isi{causative} \emph{na'}-? I propose to
answer this question by analyzing the \isi{causative} \emph{na'} not as a
derivational\is{derivation} prefix, but as a \isi{prosodically deficient} verb.

\section{Prosodically deficient verbs}\label{sec:chung:5}

The proposal to analyze the \isi{causative} \emph{na'} as a prosodically
deficient verb assimilates it to a very small class of frequently used
\ili{Chamorro} verbs. This class contains the intransitive verb
\emph{malak/falak} `go to, head to, depart for' and the transitive\is{transitivity} verb
\emph{fa'} `pretend'.\footnote{I represent these verbs without dashes in
  order to highlight the fact that they are not prefixes. Note that
  \emph{malak/falak} is an \emph{m/f} verb; its initial consonant is
  realized as /m/ in the realis mood or when preceded by \isi{plural} number
  agreement, but as /f/ otherwise. \emph{Fa'} is, confusingly,
  homophonous with a prefix \emph{fa'}- that creates derived\is{derivation} verbs
  meaning `make (into, with)'. This prefix attaches productively to
  nouns (e.g. \emph{fa'hånum} `liquefy', from \emph{hånum} `water,
  liquid'; \emph{fa'denni'} `prepare with hot sauce', from \emph{donni'}
  `hot pepper'), and less productively to adjectives (e.g.
  \emph{fa'baba} `deceive', from \emph{båba} `bad'; \emph{fa'tinas}
  `make', from \emph{tunas} `straight'). The verb \emph{fa'} `pretend'
  and the derivational\is{derivation} prefix \emph{fa'}- are treated as the same \isi{affix}
  by  \citet[176--77]{topping1973}.} Both verbs are clearly the
content of lexical categories; they are not derivational\is{derivation} prefixes. Like
other verbs, they serve as the predicates of clauses, show subject-verb
agreement, are inflected\is{inflection} for mood and \isi{aspect}, and so on. More
significantly for our purposes, they select a functional projection as
their complement.\is{complement}

\emph{Malak/falak} `go to' selects a DP that is linked to its goal
argument.\is{arguments} This DP, which is bracketed in \REF{ex:chung:21}, can include determiners
(see \ref{ex:chung:21a}) and modifiers (\ref{ex:chung:21a}--\ref{ex:chung:21b}); it can also consist of a place name
\REF{ex:chung:21c} or an interrogative pronoun \REF{ex:chung:21d}. This range of expansions
reveals that syntactic incorporation, however analyzed, is not involved.

\ea \label{ex:chung:21}
	\ea \label{ex:chung:21a} \gll  Man-malak {[}i Pala na kasinu{]} ham.\\
		\textsc{n.agr-}go.to the Pala \textsc{l} casino we\\
		\glt `We went to the Pala casino.' (CD, entry for \emph{kasinu})
	\ex \label{ex:chung:21b} \gll Ti ya-hu malak {[}ottru tånu'{]}.\\
		not like-\textsc{poss} \textsc{n.agr.infin.}go.to other land\\
		\glt `I don't like to go to other places.' (CD, entry for \emph{gåstu})
	\ex \label{ex:chung:21c} \gll Yanggin gaigi hao Saipan ya para un falak {[}Tinian{]}, siempri h\emph{um}ånao hao luchan.\\
		if \textsc{n.agr.}be.at you Saipan and \textsc{fut} \textsc{p.agr} \textsc{n.agr.}go.to
		Tinian indeed \textsc{n.agr.}go you south\\
		\glt `If you are on Saipan and traveling to Tinian, you will have to go
		south.' (CD, entry for \emph{luchan})
	\ex \label{ex:chung:21d} \gll Malak {[}månu{]} hao nigap?\\
		\textsc{n.agr.}go.to where? you yesterday\\
		\glt `Where did you go yesterday?' (CD, entry for \emph{malak})
	\z
\z

\emph{Fa'} `pretend' selects a finite realis TP complement.\is{complement} This
embedded TP can have a predicate of any major category type, and when
the predicate is a verb or adjective, it shows subject-verb agreement,
as expected.

\ea \label{ex:chung:22}
\ea \label{ex:chung:22a} \gll In fa' {[}in tingu' i ti un tungu'{]}.\\
\textsc{p.agr} pretend \textsc{p.agr} know the not \textsc{p.agr} know\\
\glt `We (excl.) pretend we know what you don't know.' (from a conference speech)

\ex \label{ex:chung:22b} \gll Ma tutuhun ma fa' {[}man-kubåtdi siha{]}.\\
\textsc{p.agr} begin \textsc{p.agr} pretend \textsc{n.agr-}cowardly they\\
\glt `They began to pretend that they were afraid.' (NT 343)

\ex \label{ex:chung:22c} \gll Ha fa' {[}sen-metgut gui'{]}.\\
\textsc{p.agr} pretend \textsc{n.agr.}extremely-strong he\\
\glt `He pretended to be extremely strong.'

\ex \label{ex:chung:22d} \gll Ha fa' {[}i anghit gui'{]} si Juan sa' gaigi i nobiå-ña.\\
\textsc{p.agr} pretend the angel he \textsc{unm} Juan because
\textsc{n.agr.}be.at the girlfriend-\textsc{poss}\\
\glt John is acting like an angel (lit. pretending he is the angel) because
his girlfriend is here' (CD, entry for \emph{ånghit})
\z
\z

The distinctive property of these verbs is that they are prosodically
deficient; they are not phonological words and cannot bear primary
stress.\is{stress} They remedy this deficiency by undergoing stray \isi{adjunction} to
the \isi{phonological word} to their immediate right, which (in \ili{Chamorro}) is
always the first \isi{phonological word} of their complement.\is{complement}\footnote{In
  prosodic theory, stray \isi{adjunction} is the operation that incorporates
  elements that are not parsed as prosodic units at a given level of
  prosodic structure into an adjacent prosodic unit at that level; see
  e.g. \citealt[13]{anderson2005}. The text assumes that in the cases under
  discussion, stray \isi{adjunction} literally produces an \isi{adjunction}
  structure. As Nick Kalivoda observes, another possibility is that a
  \isi{prosodically deficient} verb simply becomes a daughter of the
  \isi{phonological word} to its immediate right.} In \REF{ex:chung:21c}, for instance,
stray \isi{adjunction} attaches \emph{falak} to the \isi{phonological word}
\emph{Tinian} (as shown in (\ref{ex:strayadjtrees}a)); in \REF{ex:chung:22a}, it attaches \emph{fa'} to
the \isi{phonological word} \emph{in tingu'}, which itself consists of an
agreement pro\isi{clitic} adjoined to the \isi{phonological word} \emph{tingu'}
`know' (as shown in (\ref{ex:strayadjtrees}b)).

\ea \label{ex:chung:23}
\begin{tabularx}{\linewidth-0.1em}[t]{lXlX}
	a. &
	\begin{forest}
		[ω
		[falak]
		[ω
		[Tinian]
		]
		]
	\end{forest} &
	b. &
	\begin{forest}
		[ω
		[fa']
		[ω
		[in]
		[ω
		[tingu']
		]
		]
		]
	\end{forest} \\	
\end{tabularx}\label{ex:strayadjtrees}
\z

Morphophonological processes which affect verbs, but whose domain is the
\isi{phonological word}, cannot affect a \isi{prosodically deficient} verb directly.
Instead, they must target the \isi{phonological word} that immediately
dominates it. In \ili{Chamorro}, for instance, the progressive \isi{aspect} is
realized via \isi{reduplication} of the primarily stressed\is{stress} CV of the
predicate. When \emph{malak/falak} or \emph{fa'} occurs in the
progressive, the CV that is reduplicated is the primarily stressed\is{stress} CV of
the \isi{phonological word} that immediately dominates them (which is also the
primarily stressed\is{stress} CV of the \isi{phonological word} to which they are
adjoined). See \REF{ex:chung:24}.

\ea \label{ex:chung:24}
\ea \label{ex:chung:24a}
\gll  Siempri {[}malak i tetenda{]} yu'.\\
indeed \textsc{n.agr.}go.to the store.\textsc{prog} I\\
\glt `I will definitely be going to the store.'

\ex \label{ex:chung:24b} \gll  Ha {[}fa' mudodoru{]} ha' gui'.\\
\textsc{p.agr} pretend \textsc{n.agr.}stupid.\textsc{prog} \textsc{emp} he\\
\glt `He is just pretending that he is stupid.' (CD, entry for \emph{mudoru})
\z
\z

The same holds true for other processes that are sensitive to prosodic
structure. Among the overt pronouns of \ili{Chamorro} are a set of weak
pronouns which are \isi{second position} clitics (e.g. \emph{yu}' `I',
\emph{hao} `you (sg.)'). These weak pronouns occur right after the first
phonological phrase of the intonational phrase corresponding to their
clause (see \citealt{chung2003}). Because most \ili{Chamorro} clauses have predicates
that are verbs or adjectives, and most verbs or adjectives are
phonological words that project a phonological phrase of their own, a
weak pronoun is usually positioned right after them (see e.g. \ref{ex:chung:21c}).
But when the verb is \isi{prosodically deficient}, a weak pronoun is -- as
expected -- positioned right after the \isi{phonological word} (and
phonological phrase) dominating it. The relevant \isi{phonological word} is
enclosed in brackets below.

\ea \label{ex:chung:25}
\ea \label{ex:chung:25a}
\gll Tåtnai {[}malak Luta{]} yu'.\\
never \textsc{n.agr.}go.to Luta I\\

\glt `I've never been to Rota.' (CD, entry for \emph{tåtnai})

\ex \label{ex:chung:25b} \gll  {[}Ha fa' gof-maolik{]} gui' na tåotao.\\
\textsc{p.agr} pretend \textsc{n.agr.}very-good he \textsc{l} person\\
\glt `He pretended to be a very good person.' (CD, entry for \emph{fa'})
\z
\z

\section{Causative {na'} as a \isi{prosodically deficient} verb}\label{sec:chung:6}
The preceding should be enough to suggest why it would be helpful to
reanalyze the \isi{causative} \emph{na'-} as a \isi{prosodically deficient} verb.
Then the exuberance of its interplay with voice, agreement, and the like
can be attributed to the fact that it combines morphosyntactically with
the material on its left, but merely prosodically with the material on
its right.

\subsection{Proposal}

I propose to flesh out the details of this \isi{reanalysis} as follows.
Suppose that instead of a \isi{causative} prefix \emph{na'}-, \ili{Chamorro} has a
\isi{prosodically deficient} verb \emph{na'} `make, let, cause', which selects
a small clause complement\is{complement} -- specifically, a vP complement.\is{complement} In \ili{Chamorro},
small v selects a complement\is{complement} that is VP or AP, so the verb \emph{na'}
will occur in syntactic structures of the type shown in (\ref{ex:na'trees}) (with
specifiers omitted for convenience).

\ea \label{ex:chung:26}
\begin{tabularx}{\linewidth-0.1em}[t]{lXlX}
	a. &
	\begin{forest}
		[VP
			[V
				[na']
			]
			[vP
				[v]
				[VP]
			]
		]
	\end{forest} &
b. &
\begin{forest}
		[VP
			[V
				[na']
			]
			[vP
				[v]
				[AP]
			]
		]
	\end{forest} \\	
\end{tabularx}\label{ex:na'trees}
\z

The V or A that heads the embedded VP or AP in (\ref{ex:na'trees}) corresponds to what
was referred to earlier as the inner predicate. Because the inner
predicate has small v in its functional layer, and small v is
responsible for voice, the inner predicate can be a verb that is
passive,\is{passive} antipassive, or reciprocal. At the same time, when the verb
\emph{na'} occurs as the predicate of a finite clause, it will project
its own small v (not represented in (\ref{ex:na'trees})), so it can independently be
passive,\is{passive} antipassive, or reciprocal. This will account for much of the
exuberant interplay that causatives exhibit.

What about subject-verb agreement? In \ili{Chamorro}, person-and-number
agreement is always realized to the left of \isi{number agreement}. This makes
it reasonable to suppose that the two types of agreement spell out
\isi{features} of different functional heads, where the head whose \isi{features}
are spelled out by \isi{person-and-number agreement} is the higher of the two.
Now, word order aside, finite clauses in \ili{Chamorro} have a familiar
architecture in which the functional layer of the clause contains (at
least) T and small v (see \citealt{chung1998,chung2004}). Let us assume, then,
that T is specified for finiteness, mood, \isi{aspect}, and the person and
number of the DP in its \isi{specifier} (= the subject). The relevant number
feature \is{features}here is, of course, {[}±singular{]}. These \isi{features} of T are
spelled out by \isi{person-and-number agreement} when the predicate is
transitive\is{transitivity} or the mood is irrealis; see (\ref{tab:persnumagr}). Let us make the further,
more interesting assumption that small v is specified for the number of
the DP in \emph{its} \isi{specifier} via the feature\is{features} {[}±\isi{plural}{]}. This
feature\is{features} of v is spelled out by \isi{number agreement} when the predicate is
intransitive; see (\ref{tab:numagr}).\footnote{A reviewer asks how \isi{transitivity} is
  folded into the picture. I assume that T's \isi{features} are spelled out as
  \isi{person-and-number agreement} when T shares \isi{features} with transitive\is{transitivity}
  small v -- a small v that assigns abstract Case.\is{case} Small v's number
  feature\is{features} is spelled out as \isi{number agreement} when small v does not
  assign abstract Case.\is{case} See 7.2.} In the finite clauses of interest
here, T has a vP complement,\is{complement} the DP in vP's \isi{specifier} raises to the
\isi{specifier} of T, and \isi{number agreement} spells out some \isi{features} of T
(finiteness and mood) as well as the number feature\is{features} of small v. The
mechanisms responsible for the multiple \isi{exponence} of finiteness and mood
are irrelevant here. What matters is that in structures in which vP is
the \isi{complement} of the verb \emph{na'}, \isi{number agreement} is spelled out
with ``irrealis'' forms: as the prefix \emph{fan-} when the DP in small
v's \isi{specifier} is {[}+\isi{plural}{]}, and with no realization otherwise.

Let us now turn to the \isi{prosody}. The verb \emph{na'} is prosodically
deficient, so in the prosodic structure corresponding to (\ref{ex:na'trees}) it will
undergo stray \isi{adjunction} to the \isi{phonological word} to its immediate
right, which is always the first \isi{phonological word} of its complement.\is{complement}
Assuming -- crucially -- that the word order of the small clause
\isi{complement} has already been determined, this \isi{phonological word} will be
the content of a verb or adjective. The verb or adjective may be
morphologically complex\is{complexity} and may begin with the \isi{plural} \emph{fan}-. In
other words, stray \isi{adjunction} will lead to one of the outcomes
schematized in (\ref{ex:strayadjna'}).

\ea \label{ex:chung:27}
\begin{tabularx}{\linewidth-0.1em}[t]{lXlX}
	a. &
	\begin{forest}
			[ω
				[na']
				[ω]
			]
	\end{forest} &
b. &
\begin{forest}
				[ω
					[na']
					[ω
						[fan-\dots\dots]
					]
				]
	\end{forest} \\	
\end{tabularx}\label{ex:strayadjna'}
\z

Overall, this proposal gives a remarkably successful account of the
\isi{morphosyntactic} profile of \ili{Chamorro} causatives presented in \S3.
Causatives have the morphosyntax of transitive\is{transitivity} verbs (see (8--10))
because \emph{na'} is, in fact, a transitive verb. The prosodic
deficiency of this verb makes it appear to be a prefix -- and therefore
derivational\is{derivation} morphology -- but the appearance is illusory. Like other
\isi{prosodically deficient} verbs, the verb \emph{na'} selects a \isi{complement}
that is a functional projection -- namely, vP -- and undergoes stray
\isi{adjunction} to the \isi{phonological word} to its immediate right, which is
always the first \isi{phonological word} of its complement.\is{complement} For independent
reasons, this word is always the content of a verb or adjective (see
\REF{ex:chung:7}). The verb or adjective projects a vP in the syntax, so it is
inflected\is{inflection} for \isi{number agreement} (see \REF{ex:chung:14}) and can be morphologically
complex\is{complexity} (see (11--12)). Moreover, the claim that \emph{na'} is a verb, as
opposed to derivational\is{derivation} morphology, makes it unsurprising that it can be
the content of both the main verb and an embedded verb in recursive
structures like \REF{ex:chung:13}.

Note, finally, that the proposal is consistent with the way that
\emph{na'} interacts with morphophonological processes whose domain is
the \isi{phonological word} or phonological phrase. When \emph{na}' occurs in
the progressive \isi{aspect}, the CV that is reduplicated is the primarily
stressed\is{stress} CV of the \isi{phonological word} that immediately dominates it (see
\citealt[79--81]{gibson1980}).\footnote{The progressive \isi{aspect} in these examples
  must be interpreted as affecting the \isi{causative} \emph{na'}; it cannot
  be interpreted as affecting the inner predicate of the \isi{causative}. See
  especially \REF{ex:chung:28b}.} (For consistency, I continue to use the parsing and
glossing conventions adopted earlier for causatives, even though
\emph{na'} is now analyzed as a \isi{prosodically deficient} verb.)

\ea \label{ex:chung:28}
\ea \label{ex:chung:28a}
\gll Esta {[}n\emph{in}a'-chachatkuentus{]} ni malangu-ña.\\
already \textsc{n.agr.pass.caus-}speak.incoherently.\textsc{prog} \textsc{obl} sickness-\textsc{poss}\\
\glt `Her sickness is making her speak incoherently.' (CD, entry for \emph{chátkuentus})

\ex \label{ex:chung:28b} \gll  Hu ripåra na un {[}na'-malilisia{]} mampus i palabråk-ku.\\
\textsc{p.agr} notice \textsc{comp} \textsc{p.agr} \textsc{caus-}malicious.\textsc{prog}
too.much the word-\textsc{poss}\\
\glt `I noticed that you really are making my words malicious.' (CD, entry
for \emph{malisia})
\z
\z

Further, weak pronouns are positioned not immediately after \emph{na'},
but right after the \isi{phonological word} (and phonological phrase) that
dominates it.

\ea \label{ex:chung:29}
\gll {[}Man-na'-hanao{]} ham åbiu para i man-disgrasiåo.\\
\textsc{n.agr-ap.caus-}go we support for the \textsc{n.agr-}in.accident\\
\glt `We sent help for those involved in that accident.' (CD, entry for
\emph{åbiu})
\z

This is what we expect from a \isi{prosodically deficient} verb; see \S4.

\subsection{Consequences}

If this new approach turns out to be correct, \ili{Chamorro} causatives no
longer provide a counterexample to the traditional observation that
\isi{inflection} is ``outside of'' derivation.\is{derivation} Instead, the \isi{causative}
\emph{na'} is a \isi{prosodically deficient} verb, and its relative order with
respect to \isi{morphology} which it happens to be prosodically attached to,
but which belongs morphosyntactically with a different predicate, is
immaterial. The result stands even if \ili{Chamorro} \isi{number agreement} is taken
to be inflectional,\is{inflection} as in 6.1. This is why we embarked on the
investigation in the first place.

Further, and interestingly, the small clause \isi{complement} of the verb
\emph{na'} turns out to fill a gap in the \isi{paradigm} of \ili{Chamorro}
complementation.\is{complement} As might be expected, the language has various types of
clausal complements,\is{complement} including finite clauses, infinitive clauses, and
the embedded ``clause'' of \isi{restructuring} constructions. Finite clauses and
infinitive clauses are clearly TPs. They differ in that finite clauses
are specified for mood and can have an overt subject, whereas infinitive
clauses are mood-invariant and cannot have an overt subject \citep[see][64--68]{chung1998}. Embedded ``clauses'' of \isi{restructuring} constructions are
similar to infinitive clauses in these respects, but smaller \citep[see][]{chung2004}. Given the claim in 6.1 that \isi{person-and-number agreement} realizes
\isi{features} of T, these embedded ``clauses'' are best analyzed as defective
TPs, as proposed by \citealt{bhatt2005} for Hindi-Urdu -- TPs whose head is
parasitic on the T of the clause under which they are embedded.

The three types of clausal complements\is{complement} just described show number
agreement and \emph{some} \isi{person-and-number agreement}. Finite clauses
make full use of the agreement paradigms\is{paradigm} in (\ref{tab:persnumagr}--\ref{tab:numagr}). Infinitive clauses
show realis \isi{number agreement} when their predicate is intransitive and
the invariant \isi{infix} -\emph{um}- when it is transitive.\is{transitivity} Embedded
``clauses'' of \isi{restructuring} constructions show realis \isi{number agreement} when their
predicate is intransitive and either realis \isi{person-and-number agreement},
or the \isi{infix} -\emph{um}-, when it is transitive.\is{transitivity}

If \emph{na'} truly is a verb, then its small clause \isi{complement} differs
from the other types of clausal complements\is{complement} just mentioned along all of
these dimensions. The small clause \isi{complement} of the verb \emph{na}' is
merely a vP -- even smaller than the embedded ``clause'' of
\isi{restructuring} constructions -- but it can have an overt subject. And, because it is
merely a vP, it shows (irrealis) \isi{number agreement} but no
\isi{person-and-number agreement} at all.\footnote{Interestingly, \ili{Chamorro} has
  at least one other verb that can select a vP complement:\is{complement} the
  imperative verb \emph{cha'}- `don't, shouldn't, better not'. As
  expected, the vP \isi{complement} of \emph{cha'}- (a) does not show
  \isi{person-and-number agreement}, but (b) when intransitive, does show
  irrealis \isi{number agreement}. Less expectedly, the \isi{specifier} of this vP
  is always controlled PRO, and the verb or adjective from which vP is
  projected must be inflected\is{inflection} for progressive \isi{aspect}. Thanks to Pranav
  Anand for questions that uncovered this.}

None of this language-particular fine detail is theoretically necessary
or even expected, of course. But it is reassuring that the vP \isi{complement}
posited by our alternative approach to causatives can be integrated
smoothly into the overall picture of complementation\is{complement} in \ili{Chamorro}.

\section{Other aspects of the morphosyntax of causatives}\label{sec:chung:7}

Other aspects of the profile of causatives present more of a challenge
to the proposal just outlined. I discuss two such aspects below, with
the aim of showing that they can be handled relatively straightforwardly
once the right infrastructure is installed. One set of facts involves
\isi{wh-agreement}; the other involves morphological\is{morphology} case.\is{case}

\subsection{Wh-agreement}

When a constituent undergoes wh-movement\is{movement} in \ili{Chamorro}, the verb or
adjective on which it depends shows a special morphological\is{morphology} agreement
called \isi{wh-agreement} (see \citealt{chung1998} and the references cited there).
This special agreement, which supersedes the normal forms of normal
subject-verb agreement, signals the grammatical relation of the
wh-trace -- whether it is a subject, direct object, or oblique. For
instance, when the wh-trace is a direct object, \isi{wh-agreement} is realized
by the \isi{infix} -\emph{in}- and nominalization of the verb. Nominalization
is indicated, among other things, by the fact that the subject is
cross-referenced by (suffixal)\is{suffix} possessor agreement (glossed
\textsc{poss}) rather than subject-verb agreement. Compare the sentence
in \REF{ex:chung:30a}, in which the verb shows \isi{person-and-number agreement}, with the
constituent question in \REF{ex:chung:30b}, in which the verb shows \isi{wh-agreement}.

\ea \label{ex:chung:30}
\ea \label{ex:chung:30a}
\gll  Hu kåkannu' i gollai.\\
\textsc{p.agr} eat\textsc{.prog} the vegetable\\
\glt `I'm eating vegetables.' (CD, entry for \emph{nos})

\ex \label{ex:chung:30b} \gll Håfa k\emph{in}annono'-mu?\\
what? \textsc{wh.}eat-\textsc{poss.prog}\\
\glt `What have you been eating?' (from a tape-recorded narrative)
\z
\z

In earlier work I analyzed \isi{wh-agreement} as the result of feature\is{features} sharing
in abstract Case\is{case} between a wh-trace and the T that most immediately
commands it. The shared Case\is{case} feature\is{features} is then spelled out on the verb or
adjective that projects T's complement.\is{complement} I will adopt this analysis here,
noting that in minimalist syntax, abstract Case\is{case} is often reconfigured in
terms of the \isi{syntactic head} that licenses the relevant DP via Agree.

Let us now turn to \isi{causative} sentences and consider the DP described at
the beginning of \S3 as the subject of the inner predicate. The
proposal we are exploring treats this DP as the subject of the small
clause \isi{complement} of \emph{na'} -- in other words, as the \isi{specifier} of
the embedded vP in the schematic diagram below. (This \isi{specifier} is
represented to the right for convenience; see \citealt{chung1998} on the
derivation of predicate-first \isi{word order} in \ili{Chamorro}.)

\ea \label{ex:chung:31}\label{ex:caustree}
	\begin{forest}
		[v\1
			[v\sub{1}]
			[VP
				[V
					[na']
				]
				[vP
					[v\1
						[v\sub{2}]
						[VP]
					]
					[DP]
				]
			]
		]
	\end{forest}
\z

The small clause subject is in an ECM configuration, so it is licensed
by Agree with the small v that immediately commands \emph{na'} (=
v\textsubscript{1} in the diagram in (\ref{ex:caustree})) in essentially the same way
as if it were the direct object of \emph{na'}. This licensing is
confirmed by \isi{wh-agreement}: when the small clause subject undergoes
wh-movement,\is{movement} \isi{wh-agreement} signals that the wh-trace is a direct object
(see \citealt[82, 164]{gibson1980}).\footnote{In \REF{ex:chung:32b}, the construction of
  interest is a prenominal relative clause (in brackets), and what has
  undergone wh-movement\is{movement} is -- depending on one's assumptions -- either a
  null relative operator or else the head NP \emph{hånum} `water'.}

\ea \label{ex:chung:32}
\ea \label{ex:chung:32a}
\gll Håyi na pilotu n\emph{in}a'-baståm-mu?\\
who? \textsc{l} pilot \textsc{wh.caus}-quit-\textsc{poss}\\
\glt `Which pilot did you fire (lit. make quit)?'

\ex \label{ex:chung:32b} \gll Ha na'-moderåtu si Lillian i {[}n\emph{in}a'-maipen-ña{]} hånum.\\
\textsc{p.agr} \textsc{caus-}moderate \textsc{unm} Lillian the \textsc{wh.caus-}hot-\textsc{poss} water\\
\glt `Lillian moderated (the temperature of) the water that she was making
hot.' (CD, entry for \emph{moderåtu})
\z
\z

Next, consider structures in which the inner predicate is transitive\is{transitivity} and
so the small clause \isi{complement} of \emph{na'} contains a direct object.
This embedded direct object is licensed by Agree with the small v that
immediately commands the inner predicate (= v\textsubscript{2} in (\ref{ex:caustree})).
Therefore, when it undergoes wh-movement,\is{movement} \isi{wh-agreement} signals that the
wh-trace is a direct object (see \citealt[197]{gibson1980}).

\ea \label{ex:chung:33}
\gll Håfa n\emph{in}a'-li'e'-ña si Maria nu hågu?\\
what? \textsc{wh.caus-}see-\textsc{poss} \textsc{unm} Maria \textsc{obl} you\\
\glt `What did Maria show you (lit. cause you to see)?'
\z

Not only does \isi{wh-agreement} register the same Case\is{case} feature\is{features} for both types
of wh-traces, but in both constructions the verb on which the agreement
is realized is the higher verb, namely, the \isi{causative} \emph{na'}. It is
this verb that is infixed\is{infix} with -\emph{in}- and undergoes nominalization,
as can be seen from the fact that its subject (the causer) is the DP
cross-referenced by possessor agreement. It may seem surprising that
\isi{wh-agreement} is realized on the higher verb, given that the wh-traces in
\REF{ex:chung:32} and \REF{ex:chung:33} are \isi{arguments} of the inner predicate. But the pattern
follows from the syntactic structure proposed for causatives in \S6.1, plus the independently motivated assumption that \isi{wh-agreement}
involves feature\is{features} sharing between the wh-trace and T. Because small
clauses do not contain T, a wh-trace in the small clause \isi{complement} of a
\isi{causative} must share its abstract Case\is{case} feature\is{features} with the matrix T. As
usual, the shared Case\is{case} feature\is{features} is spelled out on the verb or adjective
that projects T's complement,\is{complement} which in this case is the \isi{causative}
\emph{na'}.

It may seem even more surprising that the possessor agreement that ought
to be realized on the nominalized verb is spelled out on what is
apparently the inner predicate. I contend that what lies behind this
unusual spell-out is the prosodic deficiency of \emph{na'}. In \ili{Chamorro},
affixes\is{affix} must attach to phonological words. This point emerges most
clearly for suffixes,\is{suffix} perhaps because suffixation\is{suffix} invariably causes
primary \isi{stress} to shift to the penultimate syllable of the suffixed\is{suffix}
word. Since \emph{na'} is not a \isi{phonological word}, but rather
\isi{prosodically deficient}, the \isi{suffix} that realizes possessor agreement
must attach instead to the \isi{phonological word} immediately dominating
it -- the \isi{phonological word} formed by stray \isi{adjunction} of \emph{na'} to
the inner predicate. This, I claim, is responsible for the unusual
location of possessor agreement in (32--33).\footnote{A reviewer asks if
  -\emph{in}- infixation\is{infix} might target the \isi{phonological word} containing
  the relevant verb. It might indeed. However, what matters here is that
  infixation\is{infix} does not target the \isi{phonological word} consisting only of
  the inner predicate (which, recall, is distinct from the phonological
  word consisting of the inner predicate plus \emph{na'}). This can be
  seen from the ill-formedness of *\emph{na'-l\textbf{in}i'e'-ña} as
  opposed to \emph{n\textbf{in}a'-li'e'-ña} `she caused to see'. More
  generally, it is hard to locate \ili{Chamorro} evidence that prefixes and
  infixes\is{infix} must attach specifically to phonological words (as opposed to
  just any phonological material).}

One might wonder how the same facts would be handled by a more
traditional analysis of \ili{Chamorro} causatives that treats \emph{na'-} as a
derivational\is{derivation} prefix. Such an analysis could deal straightforwardly with
the spell-out facts just described, because it takes the combination of
\emph{na'}- plus the inner predicate to be a complex\is{complexity} word (and therefore
a \isi{phonological word}). It would, however, have more trouble with the
evidence provided by \isi{wh-agreement} that both the subject and direct
object of the inner predicate are licensed by (different instances of)
small v. This is because the more traditional analysis assumes that
there is just one verb, and therefore just one small v, in the
structure.

It should be noted that \ili{Chamorro} has no double object verbs -- no verbs
whose small v licenses more than one DP as a direct object. Verbs of
transfer, for instance, have just one DP that activates the object form
of \isi{wh-agreement} when it undergoes wh-movement\is{movement} -- namely, the DP that
realizes the theme (not the DP that realizes the goal; see \citealt[161--163]{gibson1980}). What this means is that a more traditional analysis of
\ili{Chamorro} causatives will have to stipulate that the derived\is{derivation} \isi{causative}
verb, exceptionally, has \emph{two} \isi{arguments} that activate this form of
\isi{wh-agreement}. But no such stipulation is needed in the small clause
analysis of this construction, as we have just seen.

\subsection{Morphological case}\is{case|(}

I mentioned in \S2 that \ili{Chamorro} has three morphological\is{morphology}
cases -- unmarked, oblique, and local -- and that subjects and direct
objects occur in the unmarked case. We must now confront the fact that a
\isi{causative} sentence has just two DPs in the unmarked case: the subject of
\emph{na'} and the subject of the inner predicate. The direct object of
the inner predicate, if there is one, occurs in the oblique case. See
the examples below.

\ea \label{ex:chung:34}
\ea \label{ex:chung:34a}
\gll Hu na'-ayao si Isidro ni kareta.\\
\textsc{p.agr} \textsc{caus-}borrow \textsc{unm} Isidro \textsc{obl} car\\
\glt `I let Isidro borrow the car.' (CD, entry for \emph{ayao})

\ex \label{ex:chung:34b} \gll Maila' ya {bai hu} na'-li'i' hao ni cha'ka gi kodu-mu.\\
come and \textsc{p.agr} \textsc{caus-}see you \textsc{obl} rat \textsc{loc} arm.muscle-\textsc{poss}\\
\glt `Come and let me show you (lit. I will make you see) the rat in your arm muscle.' (CD, entry for \emph{chå'ka})
\z
\z

This pattern raises a question. Given the \isi{wh-agreement} evidence that the
subject and direct object of the inner predicate are licensed in the
same way (by a small v), why do these DPs occur in different
morphological\is{morphology} cases?

In minimalist syntax, one way of resolving disconnects between
morphological\is{morphology} case and morphological\is{morphology} agreement is to take case to
reflect some mechanism other than licensing by a \isi{syntactic head}. The
mechanism usually invoked is case competition (also known as dependent
case assignment; see \citealt{marantz1991} and many others since). The leading
idea behind case competition is that if two DPs are in the same local
domain, independent of each other, and not already case-marked, the
presence of one DP will cause the other DP to be assigned case.

\citet{baker2015} develops a theory of structural case in which various
aspects of case competition are parameterized, including the local
domain in which the DPs occur and the specifics of the c-command
relation holding between them. In his theory, dependent case can be
assigned in two local domains, VP and TP, which are the spell-out
domains of phases. Significantly for our purposes, the evidence that VP
is a local domain comes, in part, from \isi{causative} sentences in \ili{Chamorro}.
\citet[137--139]{baker2015} assumes that \ili{Chamorro} causatives are
morphologically complex verbs,\is{complexity} and therefore \isi{causative} sentences have a
single VP that contains the complex\is{complexity} verb's direct object (= the subject
of the inner predicate) and can contain another DP (= the direct object
of the inner predicate). The dependent case assignment that he proposes
for \ili{Chamorro} is essentially as follows.

\ea \label{ex:chung:35} \citet{baker2015} on dependent case assignment in \ili{Chamorro}
\ea \label{ex:chung:35a} 
Suppose DP\textsubscript{1} has not been marked for case. If
DP\textsubscript{1} is c-commanded by DP\textsubscript{2} and both are
in the VP domain, assign DP\textsubscript{1} oblique case;

\ex \label{ex:chung:35b} Otherwise, assign DP\textsubscript{1} unmarked case.
\z
\z

\newpage 
As he observes, this case assignment handles the distribution of oblique
versus unmarked case in \isi{causative} sentences as well as other clause
types (e.g. clauses constructed from verbs of transfer).\footnote{The
  local case does not enter into the picture, because it is not a
  structural case.}

Obviously, this proposal does not mesh well with the analysis of
\ili{Chamorro} causatives being explored here. The small clause structure I
proposed in \S6.1 for causatives locates the subject and the
direct object of the inner predicate in different VP domains (see (\ref{ex:caustree}));
this will prevent dependent case assignment from occurring. However,
Baker's theory of case allows structural case to be assigned under Agree
or through case competition, and this suggests other options.

Baker takes unmarked case to be the default case in \ili{Chamorro}. Suppose we
take the opposite position and declare oblique case to be the default
case. Then the task becomes to assign unmarked case to the various types
of \ili{Chamorro} DPs that exhibit it.\footnote{The unmarked case is also used
  for predicate nominals and the objects of most overt prepositions. The
  oblique case is also used for various DPs treated in \citealt{chung1998} as
  objects of null prepositions: \isi{passive} agents, instruments, and DPs
  that realize the complements\is{complement} of antipassive verbs, other intransitive
  predicates, and nominalized verbs. It is unclear whether the proposals
  for case assignment in the text can, or should, be extended to these
  other uses.} Among the DPs that occur in the unmarked case are
subjects, possessors, and DPs in topic/focus\is{focus} position. These DPs are the
specifiers of the functional heads T, D, and C, which license them via
Agree (see \citealt{chung1998}). Moreover, each licensing relation gives rise to
some type of morphological agreement: \isi{person-and-number agreement} (for
subjects), possessor-noun agreement (for possessors), or operator-C
agreement (for DPs in topic/focus\is{focus} position). All this suggests that
unmarked case is assigned to these DPs under Agree.

Direct objects also occur in the unmarked case, where the ``direct
object'' of a \isi{causative} sentence is the inner predicate's subject but not
the inner predicate's object. Since direct objects -- including the inner
predicate's object -- are licensed by transitive\is{transitivity} small v via Agree, the
obvious move is to try to get their case to follow from a more limited
version of that relation. I claim that unmarked case is assigned to
these DPs under Agree, but \emph{only when transitive\is{transitivity} small v is
selected by T}. The italicized extra requirement may look stipulative.
But there is evidence from several areas of \ili{Chamorro} grammar that
feature\is{features} sharing occurs between small v and the T that selects it. Number
agreement spells out not only the number feature\is{features} of small v but the
finiteness and mood \isi{features} of the T that selects it (see \S6.1).
Further, the morphological operations responsible for person-animacy
effects in \ili{Chamorro} require that this feature\is{features} sharing extend to person
and other \isi{features} of the DPs licensed by these heads (see \citealt{chung1998,chung2014}). This feature\is{features} sharing can be achieved in multiple ways which,
frankly, are not of particular interest. What is relevant is that case
assignment to direct objects can now be understood as follows: unmarked
case is assigned by transitive\is{transitivity} small v under Agree, but only when it
shares \isi{features} with T.

This achieves the desired outcome. In \isi{causative} sentences, the subject
of the inner predicate will be assigned unmarked case, because it is
licensed in an ECM configuration by a small v (= v\textsubscript{1} in
(\ref{ex:caustree})) that shares \isi{features} with T. But the direct object of the inner
predicate will \emph{not} be assigned unmarked case, because it is
licensed by the embedded small v (= v\textsubscript{2} in (\ref{ex:caustree})), which
does not enter into a feature\is{features} sharing relation with T. This DP will
instead be assigned oblique case by default.

The Agree-based case assignment that I have just proposed is summarized
below.

\ea \label{ex:chung:36} Agree-based case assignment in \ili{Chamorro}
	\ea \label{ex:chung:36a} Assign unmarked case to DP if it is licensed by T, D, C, or by a
transitive\is{transitivity} small v that shares \isi{features} with T;
	\ex \label{ex:chung:36b} Otherwise, if DP has not been marked for case, assign oblique case.
	\z
\z

This case assignment handles the distribution of unmarked versus oblique
case in \isi{causative} sentences as well as clauses constructed from verbs of
transfer and various other clause types. In other words, it has the same
empirical coverage as Baker's dependent case assignment for \ili{Chamorro},
but does not require causatives to be analyzed as complex\is{complexity}
verbs.\footnote{Mark Baker (personal communication) observes that a
  dependent case account of \ili{Chamorro} morphological case can be
  maintained if the VP embedded under \emph{na'} is what he calls a
  'soft phase'. For reasons of space, the details are not spelled out
  here.} A more sustained comparison of the two approaches to \ili{Chamorro}
case assignment is better left for another time. My goal here is merely
to show that it is possible to give a coherent description of
morphological case in \isi{causative} sentences within the small clause
analysis I propose. \is{case|)}

\section{Conclusion}\label{sec:chung:8}

\ili{Chamorro} has many types of inflectional\is{inflection} material that could perfectly
well be analyzed as affixes\is{affix} or clitics; for instance, the material that
realizes \isi{person-and-number agreement} (and -- conceivably -- even the
material that realizes \isi{number agreement}). I hope to have shown here that
the same freedom of analysis, when extended to material that is
apparently derivational,\is{derivation} can have thought-provoking theoretical
consequences.


\section*{Abbreviations}
% The following
%   abbreviations are used: 
\begin{tabularx}{.45\textwidth}{lQ}
\textsc{ap}  &  antipassive\\
\textsc{caus}  &   {causative}\\
\textsc{comp}  &  complementizer\\
\textsc{emp}  &  emphatic\\
\textsc{fut}  &  future\\
\textsc{infin}  &  infinitive\\
\textsc{loc}  &  local\\
\textsc{n.agr}  &   {number agreement}\\
\textsc{obl}  &  oblique\\
\end{tabularx}
\begin{tabularx}{.45\textwidth}{lQ}
\textsc{p.agr}  &   {person-and-number agreement}\\
\textsc{pass}  &  passive\\
\textsc{poss}  &  possessor agreement\\
\textsc{prog}  &  progressive\\
\textsc{q}  &  question\\
\textsc{recip}  &  reciprocal\\
\textsc{unm}  &  unmarked\\
\textsc{wh} &   {wh-agreement}\\
\end{tabularx}
\is{causative}\is{number agreement}\is{person-and-number agreement}\is{wh-agreement}


\newpage 


\noindent Sources for the Examples:

\begin{tabularx}{\textwidth}{lX}
CD & CD database: The database for the revised \ili{Chamorro}-\ili{English}
dictionary. Saipan, CNMI and University of California, Santa Cruz\\

EM & Borja, Joaquin F., Manuel F. Borja, and Sandra Chung. 2006.
\emph{Estreyas Marianas: Chamorro}. Saipan, CNMI: Estreyas Marianas
Publications.\\

MAK & Marciano, Dolores. n.d. \emph{Månnge' na Alåguan Kalamåsa}.
National Dissemination and Assessment Center, CSULA\\

NT & 2007. \emph{Nuebo Testamento} {[}The \ili{Chamorro} New Testament{]}.
Diocese of Chalan Kanoa, CNMI\\
\end{tabularx}


\section*{Acknowledgements}
This paper is for Steve, who for many years has been   my teacher, mentor, intellectual role model, and friend.

{\sloppy
\printbibliography[heading=subbibliography,notkeyword=this]
}

% \todos
\end{document}
