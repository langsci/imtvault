\addchap{Preface}
\begin{refsection}

%content goes here

While linguistic theory is in continual flux as progress is made in our ability to understand the structure and function of language, one constant has always been the central role of the word. \textit{On Looking into Words} is a wide-ranging volume spanning current research into word-based morphology, morphosyntax, the phonology-morphology interface, and related areas of theoretical and empirical linguistics.  The 26 papers that constitute this volume extend morphological and grammatical theory to signed as well as spoken language, to diachronic as well as synchronic evidence, and to birdsong as well as human language. Central concerns of the volume's authors include morphological gaps, learnability, increases and declines in productivity, and the interaction of different components of the grammar. 

The contributions in the volume explore linked synchronic and diachronic topics in phonology, morphology, and syntax (in particular, inflection and cliticization) and their implications for linguistic theory, and are informed by data from a wide range of language families. Languages discussed in the contributions include Ancient and Modern Greek (Hale, Kavitskaya), Macedonian (Kaisse), Sanskrit, Finnish, and Sakha (Kiparsky), Middle Indo-Aryan (Deo), Rumantsch (Maiden), Latin and French (Timberlake), Russian (Spencer), Icelandic and Faroese (Thráinsson), Welsh (Hammond), Hebrew (Bat-El, Horvath \& Siloni), Limbu and Southern Sotho (Stump), Lusoga (Hyman \& Inkelas), Yidiny (Round), Japanese (de Chene), and Tsotsil Mayan (Aissen), as well as ASL and other signed languages (Lepic \& Padden, Napoli), as well as English.  

The papers are offered as a tribute to Stephen R. Anderson, the Dorothy R. Diebold Professor of Linguistics at Yale, who is retiring at the end of the 2016--2017 academic year and who has long been a major figure in several of the fields under investigation in the volume: phonology and morphology (and the history of scholarship on those topics), the of morphology-syntax interface, the relation of theory to data, the evolution of human language and its relation to animal communication systems. The contributors -- friends, colleagues, and former students of Professor Anderson -- are all important contributors to linguistics in their own right. The central contributions that Anderson has made to so many areas of linguistics and cognitive science, drawing on synchronic and diachronic phenomena in diverse linguistic systems, are represented through the papers in the volume.

The papers in Part I of the volume focus on phonology at the level of the word and below. Ellen Kaisse draws on evidence from literary Macedonian to investigate why stress assignment is word-bounded, despite the existence of higher-level prosodic interactions -- that is, she discusses the reasons why lexical stress is specifically \textit{lexical}. Dasha Kavitskaya revisits a robust generalization from work by de Chene and Anderson that predicts compensatory lengthening should be structure preserving. She argues that a class of apparent exceptions to this generalization have independently motivated analyses consistent with the generalization. Juliette Blevins looks at the role of language contact in determining cross-linguistic strategies for the resolution of stop + liquid clusters through vowel epenthesis as a recurrent feature of loan-word phonology. Erich Round's contribution addresses issues of learnability in determining the class of possible grammars. He discusses the implications of word-final deletion in Yidiny (Pama-Nyungan) for the theory of exceptionality, which he shows cannot be represented in purely phonological or purely diacritic terms. Like several papers in the collection, Höski Thráinsson's examines the diachronic underpinnings of synchronic linguistic structure. He uses contrastive evidence from alternations in Modern Icelandic and Modern Faroese to argue for different trajectories in historical change in u-umlaut and accounts for differing degrees of synchronic productivity and transparency of the umlaut rules in the two languages.

The papers in Part II concentrate on issues in morphological theory, in several cases also drawing on the interaction of synchronic and diachronic factors. Martin Maiden argues for the use of comparative and historical evidence in deciding between possible synchronic interpretations of phenomena. He revisits paradigmatic phenomena in Rumantsch and investigates whether allomorphy in verb stems is phonologically conditioned or is evidence for paradigm autonomy. Brent de Chene's paper looks at an apparent argument for a syntactic approach to derivational stem morphology (as proposed in the theory of Distributed Morphology) in languages like Japanese and shows that a closer examination of the facts supports the approach defended by Anderson on which it is just inflectional (and not derivational) processes that lend themselves to a syntactic analysis. Anderson's A-Morphous Morphology is an item-and-process approach that takes words to be related to each other by morphological processes; Semitic languages, with their celebrated triconsonantal roots, were explicitly excluded from the purview of Anderson's analysis. In her contribution, Outi Bat-El -- invoking data from both adult and child language -- argues that in fact the Anderson-style item-and-process framework can be naturally extended to provide an empirically grounded account of word relations in Modern Hebrew, with the word/stem taken as basic and the putative triconsonantal root seen as a mere residue of phonological elements.  Mike Hammond applies data from English and Welsh to the examination and measurement of morphological complexity, applying the statistical framework of Input Optimization developed by Hammond himself in earlier work to questions arising in morphological investigation. Larry Hyman and Sharon Inkelas (with Fred Jenga) address a different type of question for morphological theory, the nature of and motivation for multiple exponence. Lusoga, a Bantu language spoken in Uganda, exhibits multiple exponence of causative, applicative, subjunctive and perfective suffixes and the apparent intermingling of derivational and inflectional affixation processes, properties which Hyman \& Inkelas see as posing challenges for theories of morphology that seek to minimize redundancy and treat derivation as distinct from (and invariably ordered before) inflection. Finally, Charles Yang's paper examines a different issue for morphological theory, that of morphological defectiveness arising when a productive process fails to be recognized by the language learner. He reconsiders the \textit{*amn't} problem -- the absence (in most dialects of English) of an inflected negation for the first person singular copula -- through the lens of his Tolerance Principle, a corollary of the Elsewhere Principle motivated by the behavior of speakers with respect to paradigm gaps in English, Russian, Spanish, and other languages.  As in other papers in this section (and in the volume more generally), Yang extends his view to consider the implications of the processes he describes for language acquisition, variation, and change.  

Part III encompasses a set of issues relating morphology to syntax, drawing on complex synchronic and diachronic evidence from a diverse set of languages.  Sandy Chung's paper returns to the interaction of causative marking and number agreement in Chamorro, which has been recognized as a challenge for the traditional view that inflectional affixes invariably attach outside derivational ones. While Anderson proposed taking number agreement to be derivational, Chung argues that the apparent causative prefix is actually a prosodically deficient verb, thus dissolving the apparent counterexample to the inflection-outside-derivation while preserving the intuition that number agreement is inflectional. Another contribution to morphosyntactic investigation in this section is Randy Hendrick's argument for analyzing the English preposition of (appearing in degree inversion expressions like \textit{more complex of a theory}) as a phrasal affix rather than a syntactic head, thus predicting certain distributional features of the construction and allowing an explanation of the appearance of \textit{of} in fractional expressions (\textit{one third of a cup}).  

Several of the papers in this section examine the nature of clitics. Mark Hale investigates the syntax underlying the distribution of clitics that fall under Wackernagel's Law, which describes the tendency of clitics to appear in second (or, now, “Wackernagel”) position in the sentence. Utilizing diachronic data from Ancient Greek and Indo-Iranian, Hale provides a detailed examination of cases in which there are multiple demands on a Wackernagel position and of the devices for resolving those conflicts, ultimately taking Wackernagel's Law to be epiphenomenal on other phenomena. Judith Aissen similarly considers exceptional clitic placement, but in Tsotsil (Mayan); while Hale argues for syntactic constraints on exceptionality based primarily on Attic Greek, Aissen focuses on the role of phonology, and in particular prosodic constraints, in the determination of the placement of clitics on the right periphery (rather than in the syntactically conditioned second position). Like Hale, Deo and Kiparsky examine diachronic shifts in Indo-Aryan, albeit toward different ends. Ashwini Deo's article traces the morphosyntax of ergative alignment loss in Indo-Aryan. Anderson and others have While much work, including a seminal paper by Anderson, has sought to describe how languages develop ergative case marking, Deo turns the tables by seeking to develop an understanding of how nominative-accusative marking can emerge in historically ergative systems, as in the gradual de-ergativization of Middle Indo-Aryan based on independent shifts in argument realization in that language. Paul Kiparsky leverages a synchronic analysis of gerunds and nominalizations across a variety of languages (Vedic Sanskrit, Finnish, Sakha (Yakut), English) to argue for a lexicalist rather than syntactically derived analysis of agent nominalizations; gerunds, on the other hand, are argued to be verbal at all levels of the syntactic derivation. Thus, we have a neatly symmetrical system: while gerunds are verbs that bear a Case feature (via their INFL head), transitive nominalizations (\textit{pace} Baker) are nouns with an Aspect feature. The final two papers in the (Morpho)syntax component of the volume bear on the status of inflectional morphology within the general theory of grammar. Like other authors in the volume, Andy Spencer takes as a jumping off point Anderson's “split morphology” hypothesis -- the view that derivational morphology is lexically mediated while inflectional morphology is syntactically mediated.  Along with the gerunds and nominalizations explored in Kiparsky's contribution, another test case for this view is posed by participles. Spencer investigates Russian participles and other transpositions utilizing a generalized version of Stump's Paradigm Function Morphology. He concludes that while the participle's actual lexical entry is that of an adjective (as lexicalist theory predicts), it realizes the verbal properties of voice/aspect, sharing the semantic properties and lexemic index of its verbal base, just as in the case of verb inflection. In his paper, Greg Stump focuses on the viability of Anderson's conception of rule block for inflectional processes within a grammar and on rule interaction. His evidence is provided by multiple exponence in Limbu, a Tibeto-Burman language of Nepal, and by polyfunctionality in Southern Sotho, a Bantu language of southern Africa.  He develops principles of rule conflation that are independently motivated and converge to explain the outcome when a single rule participates in more than one block. 

The volume concludes with a set of papers that explore linguistic theory from a broad perspective, while also extending the issues of words and rules from the structure and evolution of spoken language to the properties of signed languages and birdsong.  Mark Aronoff, who is (along with Anderson) a key figure in the development of word-based morphology, turns his attention here to the mismatch between the nature of linguistic rules and that of biological evolution (despite the influence of the results of neogrammarians on Darwin and other prime movers in the establishment of the principles of natural selection). This mismatch led Saussure to abandon his earlier productive focus on diachrony in favor of a single-minded characterization of synchronic linguistics. Aronoff borrows Gause's principle of competitive exclusion from evolutionary ecology as a means for reviving the evolutionary approaches of Saussure and his 19th century predecessors. Besides the dichotomy of diachrony and synchrony, Saussure is also celebrated for his distinction between \textit{langue} and \textit{parole}, the latter notion often dismissed by generative linguists as a matter of peripheral interest. Alan Timberlake revisits Saussure's treatment of the shift in stress rules from Latin to French and points to the crucial role of parole in accounting for the relevant shifts not only in that case but in the alternations dictating the appearance of intrusive /r/ (\textit{the idear of it}) and reanalysis (\textit{a napron > an apron}) in English.  The papers by Newmeyer and Horvath \& Siloni reconsider theoretical issues in generative grammar and its rivals.  While parameters have long been adduced in grammatical theory as a means to account for cross-linguistic variation, Fritz Newmeyer's paper re-examines the role of parametric explanation and argues that despite its success in leading generative scholars to the discovery of cross-linguistic generalizations, macro- and micro-parameters rest on shaky ground empirically and conceptually in current theoretical scholarship. Newmeyer goes on to survey some promising alternative approaches for capturing the relevant insights.  Julia Horvath and Tal Siloni compare two theoretically distinct approaches to idioms.  They argue that the distributional differences between phrasal and clausal idioms with respect to the diatheses with which they co-occur represent a serious problem for the approach of Construction Grammar, which treats linguistic knowledge as a network of stored constructions, while generative grammar, which allows for a computational system (providing unstored derivational outputs) as well as a storage module, is more capable of offering an empirically adequate account of these phenomena.

The of iconicity in sign languages representations is explored in the papers by Lepic \& Padden and Napoli.  On the a-morphous theory of word structure proposed by Anderson, the word and not the morpheme is the fundamental building block of morphology. As earlier posited by Jackendoff and Aronoff, the task of Word Formation Rules is analysis rather than creation; the speaker's knowledge encompasses the complex relationships among the words in her lexicon. Ryan Lepic and Carol Padden provide further support for this theory by drawing on the properties of lexical iconicity (and the gradual loss of iconicity) in American Sign Language. Donna Jo Napoli's paper also deals with iconicity in non-spoken linguistic modalities, although she extends her domain from ASL to a wide range of the world's signed languages. She uncovers biologically motivated “chains of iconicity” consisting of mappings from perceptual systems into visual realizations that evolve along semantic lines, a natural development that helps explains why unrelated signed languages exhibit similar signs for abstract concepts. The final contribution is by Louis Goldstein, a pioneer in the development of articulatory phonology, a gesture-based approach to the problem of how messages in the mind of the sender are transferred to the mind of the receiver through actions of the body. In his paper, Goldstein explores the ways such messages are conveyed by humans (through spoken or signed words) and by birds; his study demonstrates that despite their real and instructive differences, birdsong, like speech, is a complex motor behavior with an essential role played by temporal patterning. 


The word has always been at the crux of descriptive and theoretical research on phonology, morphology, syntax, and semantics. By looking into words, the contributors to this volume shed light on both specific subfields and the interfaces among them, on how languages work and how they change over time.  The papers in this volume, in addressing linguistic structure and linguistic relationships, help further our understanding of individual languages and of language as a unified phenomenon.  

%\printbibliography[heading=subbibliography]
\end{refsection}

