\addchap{Preface}
\begin{refsection}

The \isi{present} volume, \textit{Advances in Formal \ili{Slavic} Linguistics 2016}, marks a delectable double premiere: It initiates both the book series \textit{Open \ili{Slavic} Linguistics} as a whole, and its sub-series of collective volumes on formal \ili{Slavic} linguistics. 

\textit{Open \ili{Slavic} Linguistics} aims at publishing high quality books with a focus on \ili{Slavic} languages on the empirical side, which at the same time reflect the state of the art and current developments in general linguistics. Its core principles are strict adherence to a genuine Open Access policy and to quality control through double-blind peer review. The series takes a broad linguistic perspective and invites monographs and topical collective volumes from virtually all subdisciplines. This may include theoretically oriented work on \ili{Slavic} linguistic phenomena, advanced empirical\slash experimental work on \ili{Slavic} languages, as well as handbooks, introductions and companions to the linguistic analysis of a given language. The defining characteristics of the series is that it seeks a solid grounding in up-to-date theoretical and empirical methods, fosters mutual understanding of linguists across object languages and subdisciplines, and seeks to contribute both to narrowly defined \ili{Slavic} linguistics and to general linguistics and linguistic typology. 

\textit{Advances in Formal \ili{Slavic} Linguistics 2016} presents a selection of high quality papers authored by young and senior linguists from around the world and contains both empirically oriented work, underpinned by up-to-date experimental methods, and more theoretically based contributions. The volume covers all major linguistic areas, including morphosyntax, \isi{semantics}, pragmatics, phonology, and their mutual interfaces. The particular topics discussed range from argument structure, \isi{word order}, case, \isi{agreement}, tense, aspect, and the left clausal periphery to segmental phonology. The thematic breadth and analytical depth of the contributions reflect the vitality of the field of formal \ili{Slavic} linguistics and testify to its relevance for the global linguistic endeavor. 

Early versions of the papers included in this volume were presented at the conference on Formal Description of \ili{Slavic} Languages 12 or at the satellite Workshop on Formal and Experimental Semantics and Pragmatics, which were held in Berlin on 7--10 December 2016 -- the year referred to in the title of the volume. Half of the submitted abstracts made it into the 44 presentations of the conference. The 21 papers in the \isi{present} volume were developed from these contributions in the course of a further thorough reviewing process. Neither the original conference nor the \isi{present} volume would have been possible without the readiness of so many experts to devote their time and thoughts to the critical evaluation and helpful commenting of their colleagues' research papers. We wish to express our gratitude both to the 75 anonymous reviewers of the original conference abstracts, and to the more than 50 external reviewers for the \isi{present} volume. Their commitment testifies to the liveliness and ambition of the field of \ili{Slavic} linguistics. This book would have also been impossible without our student assistants, Bella Badt, Justina Bojarski, Andrei Koniaev and Jake Walsh, and the invaluable help of the Language Science Press editors Sebastian Nordhoff and Felix Kopecky. We gratefully acknowledge their efforts and support. Finally, we would like to acknowledge the authors themselves. Open Access publishing is a collective endeavor and we appreciate the authors' willingness to collaborate with us closely not just on linguistic and scientific issues, but also on editorial matters. We sincerely hope that the authors and readers of this volume will share our conviction that it has been worthwhile.

\null\hfill Denisa Lenertová, Roland Meyer, Radek Šimík \& Luka Szucsich\\
\null\hfill Berlin, 14 December 2018
% \printbibliography[heading=subbibliography]
\end{refsection}

