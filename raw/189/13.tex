\documentclass[output=paper,modfonts,newtxmath,hidelinks]{langscibook} 
\ChapterDOI{10.5281/zenodo.2545531}

\title{General-factual perfectives: On an asymmetry in aspect choice between western and eastern Slavic languages}

\author{Olav Mueller-Reichau\affiliation{University of Leipzig}}

\abstract{The paper addresses the issue of microvariation within Slavic aspect. Specifically, it investigates perfective general-factuals, which appear in Czech and Polish but not in Russian. It is shown that perfective aspect is used in Czech and Polish when the semantics of the VP of the sentence is such that reference is limited to unique events, or when reference to a unique event is contextually determined. 
Assuming that semantic aspects operate over VP-meanings,
it is then argued that the semantics of perfective aspect in Polish and Czech includes a completedness condition and a uniqueness condition whereas the semantics of the Russian perfective, more strongly, encodes target state validity. This difference categorically bans perfective aspect from general-factual contexts in Russian, but not in Czech and Polish.

\keywords{microvariation, perfective, general-factual, target state, uniqueness, VP}
}

\begin{document}
\maketitle
\shorttitlerunninghead{General-factual perfectives}

\section{Introduction}\label{13:s1} 

The present paper contributes to the discussion of microvariation within the realm of \ili{Slavic} aspect. 
As is well-documented, the distribution 
of \isi{perfective} and \isi{imperfective} verb forms among contexts is not constant within the \ili{Slavic} family (see, among others, \citealt{Stunova1991,Stunova1993,Breu2000ZurPositiondes,Petruchina2000,Dickey2000,Dickey15,Dickey18,Gehrke02,Wiemer08,Rivero10,Alvestad13,Gattnar13,Berger13,Rivero14,Duebbers15,FK15,FK18}).
Although there is typological reason to speak of ``the \ili{Slavic}-style aspect'' (e.g. \citealt{Dahl85,Plungjan11}),
it would be utterly wrong to consider the \isi{aspectual} systems of the \ili{Slavic} languages all the same. 

The pioneering study on microvariation of aspect in \ili{Slavic} is \citet{Dickey2000}. Based on disagreeing patterns of aspect choice (\isi{perfective} / \isi{imperfective}),
Dickey analyzes the \ili{Slavic} languages as clustering around two poles on a scale. The western languages represent one pole, the eastern languages the other one. 
\ili{Polish} and \ili{Serbian} and \ili{Croatian} are diagnozed as occupying an intermediate region, as these languages share properties with languages of the western as well as with 
languages of the eastern group, see \tabref{13:sd1}.




\begin{table}
\centering
%    \begin{minipage}{\linewidth}
      \captionof{table}{\citet[5]{Dickey2000}}
      \label{13:sd1}
      \begin{tabularx}{0.6\textwidth}{l l l}
\lsptoprule \textsc{west} & \textsc{transitional} & \textsc{east}\\
\midrule     &  $\quad\quad\quad\quad$ {Polish} & $\,${Russian} \\
   Sorbian  &    & $\,$\ili{Belarusian} \\
    {Czech}  &    &  $\,${Ukrainian}\\ 
  $\,$  {Slovak}  &    &  \\
   $\,\,\,\,\,$  {Slovene}  &    & {Bulgarian} \\
        \multicolumn{2}{l}{$\quad\quad\quad\quad$Serbo-Croatian} & \\
\lspbottomrule
\end{tabularx}
%    \end{minipage}
\end{table}



\noindent\citet{Dickey15} presents a revision of the \citeyear{Dickey2000} picture. The most important innovation is that the South \ili{Slavic} languages (apart from \ili{Slovene}) are no longer classified 
as members of the western or eastern groups, but are classified separately, see \tabref{13:sd2}.

\begin{table}
\centering
%    \begin{minipage}{\linewidth}
      \captionof{table}{\citet[36]{Dickey15}}
      \label{13:sd2}
     \begin{tabularx}{0.7\textwidth}{l l l l}
\lsptoprule     \multicolumn{4}{l}{$\quad\quad\quad\quad\quad\quad\,\quad\quad$\textsc{north}} \\
\midrule  \textsc{west}$\quad\quad\quad\quad$ & \multicolumn{2}{l}{\textsc{transitional}} & \textsc{east}\vspace{-0.0cm}\\
 \midrule  {Czech}  &   \multicolumn{2}{l}{$\quad\,\,$Polish} & {Russian} \\
  {Slovak}   &   \multicolumn{2}{l}{} &  {Ukrainian} \\
    Sorbian  &   \multicolumn{2}{l}{} & \ili{Belarusian} \\
     {Slovene} &   \multicolumn{2}{l}{} & \\
\midrule    \multicolumn{4}{l}{$\quad\,\quad\quad\quad\quad\quad\quad\quad$\textsc{south}} \vspace{-0.0cm}\\
\midrule    \multicolumn{2}{l}{\textsc{transitional} (tending west)} & \multicolumn{2}{l}{\textsc{east} (with deviations)$\quad\quad$}\vspace{-0.0cm}\\
%\rowcolor{grau} \multicolumn{2}{|l|}{\small{(TENDING WEST)}} & \multicolumn{2}{|l|}{\small{(WITH DEVIATIONS)}}\vspace{-0.2cm}\\
 \midrule   \multicolumn{2}{l}{Bosnian} & \multicolumn{2}{l}{Bulgarian}\\
    \multicolumn{2}{l}{Croatian} & \multicolumn{2}{l}{Macedonian}\\
    \multicolumn{2}{l}{Serbian} & \multicolumn{2}{l}{}\\
 \lspbottomrule
\end{tabularx}
%    \end{minipage}
\il{Ukrainian}
\end{table}








In this paper, I will be concerned with \ili{Czech}, \ili{Polish} and \ili{Russian}. 
For the present purposes, therefore, the move from \citet{Dickey2000} to \citet{Dickey15} is by and large irrelevant. 
What matters is that \ili{Czech} is treated as a member of 
the western group, that \ili{Russian} is counted as an instance of the eastern group, 
and that \ili{Polish} is treated as a language sharing properties with both these groups.

More specifically, I will look at the \isi{aspectual} behavior of these three languages in general-factual usage.
General-factual contexts are particularly interesting from a \isi{comparative} point of view.
The \ili{Russian}-biased general wisdom is that general-factuals call for \isi{imperfective aspect}. As has been observed, among others, by \citet{Dickey2000}, however, 
there are certain general-factual contexts in which \ili{Czech} speakers, for instance, resort to \isi{perfective} forms.  
The aim of the present study is twofold. The first goal is to describe the kinds of contexts in which the western language \ili{Czech} displays general-factual perfectives, whereas 
the eastern language \ili{Russian} displays general-factual imperfectives. Since the theoretical prediction for ``transitional'' \ili{Polish} is unclear, 
we will always have a look at the choice that speakers of \ili{Polish} 
make in the respective cases. As we will see, and as noted in 
\citet[101]{Dickey2000}, with respect to aspect choice in general-factual contexts \ili{Polish} is not ``in between'', but 
follows the \ili{Czech} pattern.  
The second goal, in turn, is to explain the described differences by tracing them back to differences in the underlying 
\isi{semantics} of perfectivity. 

The paper is structured as follows: 
In \sectref{13:s2} I will introduce the phenomenon of \isi{perfective} general-factuals in \ili{Czech} and \ili{Polish}. 
In \sectref{13:s3} I will discuss and reject the hypothesis (proposed by \citealt{Dickey2000}) that these cases can be traced back to underlying achievement verbs. 
In \sectref{13:s4} I will discuss and reject the hypothesis (suggested by \citealt{Cummins1987}) that the decisive factor is lack of volition. 
In \sectref{13:s5} I will discuss and reject the hypothesis (brought up by myself) that \isi{perfective} general-factuals are explicable in terms of event uniqueness.
In \sectref{13:s6}, however, I will argue that the uniqueness hypothesis is not entirely on the wrong track, showing that 
it will produce correct results if it is relativized to the syntactic domain of the VP.
In \sectref{13:s8} the situation in \ili{Russian} will be taken into account. I will explain why general-factual contexts are per se incompatible
with \isi{perfective aspect} in \ili{Russian}, and what this reveals about differences in the \isi{semantics} of the respective \isi{aspectual} categories
in the western and eastern \ili{Slavic} languages under consideration. 
\sectref{13:s9} concludes the paper. 

\section{General-factual perfectives}\label{13:s2} 
Somewhere in the world wide web, a young \ili{Russian}-speaking lady tells us ten facts about herself.\footnote{\url{https://ask.fm/Nailyuta}} 
We are invited to read that she prefers to drink tea without sugar (fact 1), that she is 18 years old but feels like 16 (fact 8), that she once started 
piano lessons but soon quit in favor of choreography (fact 3), and so on. Of relevance for us is fact 6. The young woman is telling us that she has once fallen from a tree. 
The \ili{Russian} sentence that she uses to express that is \REF{13:dereva}:\footnote{I reduce grammatical information in the gloss to a relevant minimum. \textsc{ipf} is for \isi{imperfective}, \textsc{pf} is for \isi{perfective aspect}. Other abbreviations are explained at the end of the paper.}


\ea \label{13:dereva}
\gll Ja padala s dereva.\\
I fell.\textsc{ipf} from tree\\
\glt `I (once) fell from a tree.'
\z


\noindent This is a canonic instance of a \ili{Russian} general-factual \isi{imperfective}. A similar one is sentence \REF{13:shifer}, which the young lady uses to convey fact 7:

\ea \label{13:shifer}
\gll Na menja padal šifer. \\
on me fell.\textsc{ipf} roof \\
\glt `I was (once) hit by a piece of roof.' 
\z

\noindent \ili{Russian} general-factuals are characterized by reference to a single completed event only vaguely located in \isi{past} time, with 
\isi{verbal} morphology always being \isi{imperfective}.\footnote{Note that the definition of general-factuals used here does not cover cases of `presuppositional' (\citealt{Gronn2004}) / `actional' (\citealt{Padu1996}) / `anaphoric' (\citealt{Mehlig2011}) imperfectives. Note furthermore that I restrict the scope of the term to \isi{past tense} contexts, which is debatable.} What is interesting is that, if our young lady was \ili{Czech}-speaking, she would have used the \isi{perfective verb} form to convey her message:
 
\ea \label{13:stromu}
\gll Jako mal\'a jsem spadla ze stromu. \\
as small \textsc{aux} fell.\textsc{pf} from tree\\
\glt `As a child I (once) fell from a tree' 
\z

\noindent What about \ili{Polish}? \ili{Polish} turns out to pattern like \ili{Czech}:

\ea \label{13:drzewa}
\gll Jako dziecko spad\l{a}m z drzewa. \\
as child fell.\textsc{pf} from tree\\
\glt `As a child I (once) fell from a tree.' 
\z

\noindent The following pair of examples contrasting \ili{Polish} general-factual perfectives \REF{13:portfel} and \ili{Russian} general-factual imperfectives \REF{13:koselek} is taken from \citet{Wiemer01}:

\ea \label{13:portfel}
\gll Czy Pan kiedykolwiek zgubił swój portfel? \\
\textsc{q} sir ever lost.\textsc{pf} \textsc{refl} briefcase\\
\glt `Have you ever lost your briefcase?' 
\z

\ea \label{13:koselek}
\gll Vy kodga-nibud' terjali svoj košelek? \\
you ever lost.\textsc{ipf} \textsc{refl} briefcase \\
\glt `Have you ever lost your briefcase?'
\z


\noindent The kind of data discussed so far are described in \citet[95ff.]{Dickey2000}. 
It is important not to overlook that in other cases of general-factuals, 
\ili{Czech} and \ili{Polish} resort to \isi{imperfective aspect}, just like \ili{Russian} does. The examples \REF{13:korovu} to \REF{13:krowe} may serve as illustration.   

\begin{exe}
\ex\label{13:korovu}
\gll Ty kogda-nibud' doila korovu? \\
you ever milked.\textsc{ipf} cow\\
\glt `Have you ever milked a cow?' 
\end{exe}
\begin{exe}
\ex\label{13:kravu}
\gll U\v{z} jsi n\v{e}kdy dojila kr\'avu? \\
already \textsc{aux} ever milked.\textsc{ipf} cow \\
\glt `Have you ever milked a cow?'
\end{exe}
\begin{exe}
\ex\label{13:krowe}
\gll Czy kiedykolwiek doi\l{a}\'s krow\k e? \\
\textsc{q} ever milked.\textsc{ipf} cow\\
\glt `Have you ever milked a cow?'
\end{exe}

\noindent We saw that \ili{Czech} and \ili{Polish} form \textit{perfective} general-factuals, but that they do not always do so. 
It is only for a subset of general-factuals that 
these languages deviate from the \isi{imperfective} coding holding in \ili{Russian} throughout. 
The question that arises is: what precisely characterizes the contexts in which speakers of \ili{Czech} and \ili{Polish}  
use \isi{perfective} forms to denote completed \isi{past} events only vaguely located in time? 




\section{Achievements?}\label{13:s3} 
The first hypothesis to be discussed stems from \citet{Dickey2000}, reemphasized in \citet{Dickey18}. According to Dickey, the use of \isi{imperfective aspect} in the languages of the 
western group presupposes a temporal extension of the denoted event. Given this, 
speakers will have to resort to \isi{perfective aspect} whenever the predicate of the sentence is based on an achievement verb:
``In the west [...] the impv forms of achievement verbs are unacceptable in contexts where one otherwise expects the impfv'' (\citealt[124]{Dickey2000}).

The idea may be restated in terms of the following hypothesis.
\begin{exe}
\ex\label{13:hypo1}
\textbf{Hypothesis H1:} Perfective aspect is used in general-factuals whenever the verb is an achievement verb because achievement verbs  do not supply the temporally extended events required by \isi{imperfective aspect} in \ili{Czech} and \ili{Polish}.
\end{exe}

\noindent This builds on Dickey's general conclusions about \isi{aspectual} semantic differences between western and eastern languages. According to \citet[107--109]{Dickey2000}, 
the western \isi{imperfective} expresses the notion of \textsc{quantitative temporal indefiniteness}, characterized as ``the assignability of a situation 
to several points in time''. The eastern \isi{imperfective}, by contrast,   
expresses the notion of \textsc{qualitative temporal indefiniteness}, which is described as ``the non-assignment of a situation to 
a unique location relative to other states of affairs''. 

Consider example \REF{13:drzewa}, for instance. Here the predicate is formed on the basis of a lexical verb 
which is arguably analyzable as characterizing achievement events.
Being an achievement, the verb does not supply ``several points in time'', which is, 
according to Dickey, a prerequisite for using the western \isi{imperfective}. 
Therefore, in this case, the choice of \isi{imperfective aspect} is no option for the speaker of \ili{Polish}, and she has to use the \isi{perfective} instead.  

There is, however, counterevidence to Dickey's proposal. To see why, consider the following example from \ili{Russian} first:


\begin{exe}
\ex\label{13:lakeru}
\begin{itemize}
 \item[A:] \gll A Niagarskij vodopad kogda-nibud' zamerzal?\\
and N. {water falls} once froze.over.\textsc{ipf}\\
\glt `Did the Niagara Falls ever freeze over?'\vspace{0.2cm}
\item[B:] \gll Da, esli ja ne o\v{s}ibajus',  odna\v{z}dy zamerzal. \\
yes if I not make.mistake once froze.over.\textsc{ipf} \\
\glt `Yes, if I am not mistaken, it once froze over.'
\end{itemize}
\end{exe}

\noindent As can be seen and as expected, \ili{Russian} speakers use \isi{imperfective aspect} here. Now, as can be seen in \REF{13:lake} and \REF{13:lakepo}, speakers of \ili{Czech} and speakers of 
\ili{Polish} would use \isi{perfective aspect} when expressing the same thing:

\begin{exe}
\ex\label{13:lake}
\begin{itemize}
 \item[A:] \gll A Niagarsk\'y vodop\'ad n\v{e}kdy zamrzl?\\
but N. waterfalls ever froze.over.\textsc{pf}\\
\glt `Did the Niagara Falls ever freeze over?'\vspace{0.2cm}
\item[B:] \gll Ano, pokud se nem\'yl\'im, tak jednou zamrzl. \\
yes if \textsc{refl} not.mislead then once froze.over.\textsc{pf} \\
\glt `Yes, if I am not mistaken, it once froze over.'
\end{itemize}
\end{exe}

\begin{exe}
\ex\label{13:lakepo}
\begin{itemize}
 \item[A:] \gll Czy wodospad Niagara kiedy\'s zamarz\l{?}\\
\textsc{q} {water falls} N. ever froze.over.\textsc{pf}\\
\glt `Did the Niagara Falls ever freeze over?'\vspace{0.2cm}
\item[B:] \gll Tak, je\'sli si\k e nie myl\k e, kiedy\'s zamarz\l{.} \\
yes if \textsc{refl} not mislead once froze.over.\textsc{pf} \\
\glt `Yes, if I am not mistaken, it once froze over.'
\end{itemize}
\end{exe}


\noindent Above we saw that, according to Dickey's explanation of general-factual perfectives, 
the respective predicates are \isi{perfective} because of a conflict between 
the meaning of the \isi{imperfective} and the lexical meaning of the verb, and that the conflict arises with achievement verbs. 
Accordingly, the reason why \REF{13:lake} and \REF{13:lakepo} have \isi{perfective} predicates should be that these predicates are formed from 
achievement verbs lacking a process component in their lexical-semantic structure. 
The problem is that, if \REF{13:lake} and \REF{13:lakepo} were based on verbs lacking such a component, 
we would not expect these verbs to be (easily) used for denoting ongoing processes. As a matter of fact, however,
they may be used in that function, quietly and without fuss. Consider the \ili{Polish} example in \REF{13:jezioro}:

\begin{exe}
\ex\label{13:jezioro}
\gll Jezioro zamarza! \\
lake freeze.over.\textsc{ipf}\\
\glt `The lake is freezing over.'
\end{exe}

\noindent The sentence can be found on the internet, written above a photograph that shows a half-frozen lake. 
It is further elaborated by the following text:\footnote{\url{http://tarnobrzeskie.eu/2016/01/23/jezioro-zamarza-zdjecia/}} 

\begin{quote}
 Po raz pierwszy tej zimy woda w Jeziorze Tarnobrzeskim zacz\k e\l{a} 
zamarza\'c dalej ni\.z tylko kilkadziesi\k at centymetr\'ow od brzegu.\\
$\,$[`This winter for the first time the water of Lake Tarnobrzeg froze further than for just some dozens of centimetres from the lakeside.'] 
\end{quote}

\noindent Example \REF{13:jezioro} proves that the \ili{Polish} predicate meaning `freeze over' characterizes 
temporally extended events. 
Thus, it does supply ``several points in time''. According to Dickey's reasoning, this implies that the predicate should be lexically capable 
of taking on \isi{imperfective} morphology. But then, why does it not show up in the \isi{imperfective} in \REF{13:lakepo}?

One might, of course, object that the argument misses the point because \ili{Polish} is not classified as a genuine western language within Dickey's system. 
Fair enough, but consider the \ili{Czech} equivalent to \REF{13:jezioro}:

\begin{exe}
\ex\label{13:jezoro}
\gll Jezero (\hspace{-2pt} pr\'av\v{e}) zamrz\'a. \\
lake {} right.now freeze.over.\textsc{ipf} \\
\glt `The lake is freezing over (right now).'
\end{exe}

\noindent There is an alternative way of understanding Dickey's proposal.\footnote{Thanks to an anonymous reviewer for pointing that out to me.} Maybe the claim is that 
the sentences \REF{13:lake} and \REF{13:lakepo} denote achievements \textit{because} they are \isi{perfective}. 
Following this suggestion, we should perhaps restate H1:  
\begin{exe}
\ex\label{13:hypo1bar}
\textbf{Hypothesis H1':} Perfective aspect is used in general-factuals whenever the speaker wants to refer to an achievement event because the use of the \isi{imperfective} in \ili{Czech} and \ili{Polish} is restricted to reference to temporally extended events.
\end{exe}

\noindent Yet the problem remains. Note that the situations referred to in \REF{13:lake} and \REF{13:lakepo} \textit{are} temporally extended. As a matter of fact, the freezing over of a waterfall does never happen all of a sudden. It is a very time-consuming process indeed. Given that ``in the default conceptualization there is a process component in these situations'' (\citealt[78]{Dickey18}), H1' predicts that the natural translation of the \ili{Russian} \REF{13:lakeru} into \ili{Czech} or \ili{Polish} should make use of an \isi{imperfective} verb form. What is actually chosen, however, is a \isi{perfective verb} form. This raises the unanswered question: why should the speaker want to present the freezing of the Niagara Falls as an instantaneous event?    

I think that it is fair to conclude that, without further modification, Dickey's solution to the puzzle of general-factual perfectives fails to explain cases like \REF{13:lake} and \REF{13:lakepo}.\footnote{\citet{FK15} raise a similar concern about Dickey's analysis of the western \isi{imperfective}.}

\section{Volition?}\label{13:s4} 
The next idea to be discussed has been stated by \citet{Cummins1987} as a generalization to account for the situation in 
\ili{Czech}:\footnote{In the quote, I have replaced Cummins' ``constative I'' by the synonymous ``general-factual \isi{imperfective}''.}
\begin{quote}
 \ili{Czech} absolutely prohibits the general-factual \isi{imperfective} in all low-voli\-tional predicates. This restriction admits no exception [...]:
 all \ili{Czech} general-factual imperfectives have predicates with high agentivity.\\\mbox{}\hfill(\citealt{Cummins1987}:41) 
\end{quote}
For the sake of the argument let us suppose an intuitive understanding of volition, according to which it is ``the cognitive process 
by which an individual decides on and commits to a particular course of action.''\footnote{\url{https://en.wikipedia.org/wiki/Volition_(psychology)}} Given that, Cummins' law may suggest the following 
hypothesis.

\begin{exe}
\ex\label{13:hypo2}
\textbf{Hypothesis H2:} Perfective aspect is used in general-factuals whenever the speaker wants to refer to a non-volitional event because (for some unclear reason) general-factual imperfectives in \ili{Czech} and \ili{Polish} are restricted to volitional actions. 
\end{exe}

\noindent This may, indeed, account for the cases that we came across with so far. 
Sentences like \REF{13:stromu} report on accidental events, and accidents are by definiton not accompanied by the individual's decision 
on the course of events. Also sentences like \REF{13:lake} may be accounted for, as the event participant is \isi{inanimate} and, hence,
void of volition.

Nevertheless, the approach as it stands is not tenable. This has been shown in \citet[101--102]{Dickey2000}. Consider the following examples:  

\begin{exe}
\ex\label{13:prkna}
\gll U\v{z} jsi z toho prkna n\v{e}kdy sko\v{c}ila? \\
already \textsc{aux} from that diving.board ever jumped.\textsc{pf} \\
\glt `Have you ever jumped from that diving board?'
\end{exe}

\begin{exe}
\ex\label{13:trampolin}
\gll Czy kiedykolwiek skoczy\l{e}\'s z tej trampoliny? \\
\textsc{q} ever jumped.\textsc{pf} from that diving.board  \\
\glt `Have you ever jumped from that diving board?'
\end{exe}

\noindent These sentences clearly report on volitional actions, and yet the \isi{perfective} form is used.
If lack of volition was the explanation for the use of \isi{perfective aspect} in general-factual contexts, 
as the hypothesis H2 suggests, examples like \ili{Czech} \REF{13:prkna} and \ili{Polish} \REF{13:trampolin} should not exist.  
So appealing as it may seem at first sight, we have to look for a better explanation. 

\section{Uniqueness?}\label{13:s5} 
The third hypothesis that I would like to check may be stated as follows:

\begin{exe}
\ex\label{13:hypo3}
\textbf{Hypothesis H3:} Perfective aspect is used in general-factuals whenever the speaker wants to refer to an event which is unique in the relevant context because 
perfectivity semantically expresses uniqueness in \ili{Czech} and \ili{Polish}. 
\end{exe}

\noindent To make sense of that, let us assume that the \isi{aspectual} operators in \ili{Czech} and \ili{Polish} have the following \isi{semantics}:\footnote{For ease of readability, I will not indicate \citeauthor{Krifka98}'s (\citeyear{Krifka98}) temporal trace function $\tau(e)$,
which maps events onto their run time. Thus, wherever $e$ is related to $t$ in the semantic representations to follow, this is meant to express that $\tau(e)$ is related to $t$.}

\begin{exe}
\ex\label{13:aspuniq}
\sx{IPF}${}= \lambda P \lambda t \exists e [P(e) \wedge e \bigcirc t]$\\
\sx{PF}${}= \lambda P \lambda t \exists e[P(e) \wedge e \subseteq t \wedge \neg \exists e'[ P(e') \wedge e' \neq e]]$
\end{exe}

\noindent Informally speaking, the PF-operator includes a completedness requirement ($e \subseteq t$) as well as 
a uniqueness condition ($\neg \exists e'[ P(e') \wedge e' \neq e]$). The former requires that the denoted event must have reached 
its culmination point, and the latter requires that there is no possibility or, at least, no expectancy of a second event realization of the same type 
in the discourse context. The IPF-operator, by contrast, imposes only a very vague condition on interpretation ($e \bigcirc t$). All that it requires is that the 
event time should, in this or that way, overlap the reference time (cf. \citealt{Gronn2004}). 

Given these assumptions, why do unique events call for perfectivity? Note that the two operators in \REF{13:aspuniq} are of the same semantic type,
differing only in specificity of content (every event that fulfills $e \subseteq t$ is an event that fulfills $e \bigcirc t$).
Therefore, the two \isi{aspectual} operators may legitimately be analyzed as forming a 
Horn-scale (\citealt{Sonne06,Sonne07}). As they are located on a Horn-scale, the use 
of the less specific \isi{imperfective} marker will trigger the conversational \isi{implicature} that the speaker 
lacks evidence for using the more specific \isi{perfective} marker. If the speaker wanted to avoid inviting this inference, because she does have 
sufficient evidence for categorizing the event as completed and unique, she would have to use the \isi{perfective}. The use of the \isi{imperfective} 
would otherwise misinform the hearer by suggesting that the event is either non-unique or non-completed. Taking into account that the latter option is out in general-factual contexts (as general-factuals always report on completed events, see above), we may rewrite H3 as H3':

\begin{exe}
\ex\label{13:hypo3bar}
\textbf{Hypothesis H3':} Perfective aspect is used in general-factuals whenever the speaker wants to refer to an event which is unique in the relevant context because imperfectivity in \ili{Czech} and \ili{Polish} general-factuals implies reference to non-unique events. 
\end{exe}

\noindent Note, by the way, that if accidental events imply uniqueness (and I shall argue that they do), Cummins' law (``\ili{Czech} absolutely prohibits the general-factual \isi{imperfective} in all low-volitional predicates'') may be viewed as a special case:
If the expression of a unique, completed event attracts \isi{perfective aspect}, and if accidents represent a special kind of unique events, 
then the expression of an accident should likewise attract \isi{perfective aspect}. 

Hypothesis H3 gains further plausibility in view of the fact that necessarily unique events (i.e. cases where world knowledge makes 
event repetition unlikely) require \isi{perfective aspect}. Note that these sentences do not represent cases of general-factuals, as general-factuals require the event property to be in principle replicable 
(e.g. \citealt[58]{Padu1996}). 

\begin{exe}
\ex\label{13:baumfaell}
\gll Jako mal\'a jsem \{\hspace{-2pt} pok\'acela / *\hspace{-2pt} k\'acela\} n\'a\v{s} jedin\'y strom. \\
as small \textsc{aux} {} felled.\textsc{pf} {} {} felled.\textsc{ipf} our single tree \\
\glt `When I was young, I felled our one and only tree.'
\end{exe}
\begin{exe}
\ex\label{13:baumfaellpf}
\gll Kiedy by\l{e}m m\l{o}dy, \'sci\k a\l{e}m nasze jedyne drzewo. \\
when was small felled.\textsc{pf} our
single tree \\
\glt `When I was young, I felled our one and only tree.'
\end{exe}

\noindent And yet H3 and H3' are, like the previous hypotheses, confronted with counterevidence. Consider the following \ili{Czech} dialogue. 

\begin{exe}
\ex\label{13:blinddarm}
\begin{itemize}
 \item[A:] \gll U\v{z} mu odstra\v{n}ovali {slep\'e st\v{r}evo?} \\
already him took.out.\textsc{ipf} appendix\\
\glt `Has his appendix been removed?'\vspace{0.2cm}
\item[B:] \gll Ano, u\v{z} mu odstra\v{n}ovali {slep\'e st\v{r}evo.} \\
yes already him took.out.\textsc{ipf} appendix \\
\glt `Yes, his appendix has been removed.'
\end{itemize}
\end{exe}

\noindent What A and B are talking about here is a non-repeatable, i.e. unique event (everything else would enforce the conceptualization of 
an absurd scenario where a formerly removed appendix is re-implanted). According to hypothesis H3', this should rule out \isi{imperfective} 
aspect in favour of the \isi{perfective}. Contra to that prediction, however, the \isi{imperfective} appears to be well suited to figure in the \ili{Czech} example \REF{13:blinddarm}.

According to a comment by an anonymous reviewer, the situation in \ili{Polish} seems to be the same as in \ili{Czech}:  

\begin{exe}
\ex\label{13:blinddarmpo}
\gll Czy mu  \{\hspace{-2pt} wycinali / wyci\k eli\}  {\'slep\k a kiszk\k e?} \\
\textsc{q} him {} took.out.\textsc{ipf} {} took.out.\textsc{pf} appendix \\
\glt `Has his appendix been removed?'
\end{exe}

\noindent Here, too, it is possible to use an \isi{imperfective} verb form under reference to a completed event, which is in conflict with H3/H3'.\footnote{According to the reviewer, the use of the \isi{perfective} form leads to an interpretation involving target state relevance (see \sectref{13:s8}). It needs to be checked whether target state relevance is indeed obligatory when the \isi{perfective} is used in \REF{13:blinddarmpo}. If yes: Does it follow from the \isi{semantics} of the \isi{perfective}? Then \ili{Polish} would approximate the \ili{Russian} pattern. Or does it rather follow from pragmatic inferences, presumably in competition with the \isi{imperfective}? I must leave this issue open.}

We have to conclude that, as it stands, the uniqueness hypothesis seems to be falsified.  

\section{Uniqueness!}\label{13:s6} 

In this section, I elaborate on hypothesis H3. The idea is to take the syntactic structure of the sentence into account 
and relativize the \textit{semantic} uniqueness condition to the domain of the AspP. The new hypothesis (which is actually not ``new'' but merely more precise) will then be (\ref{13:hypo4}). 

\begin{exe}
\ex\label{13:hypo4}
\textbf{Hypothesis H4:} Perfective aspect is used in general-factuals whenever the speaker wants to refer to an event which is unique in the relevant context because perfectivity semantically expresses AspP-uniqueness in \ili{Czech} and \ili{Polish}. 
\end{exe}

\noindent Let me explain. Above I proposed the denotations stated in \REF{13:semas}.

\begin{exe}
\ex\label{13:semas}
\sx{IPF}${}= \lambda P \lambda t \exists e [ P(e) \wedge e \bigcirc t ]$\smallskip\\
\sx{PF}${}= \lambda P \lambda t \exists e [ P(e) \wedge e \subseteq t \wedge \neg \exists e'[ P(e') \wedge e' \neq e ]]$
\end{exe}

\noindent Now I remind of that these semantic assumptions presuppose the syntactic assumptions stated in \REF{13:synas}:

\begin{exe}
\ex\label{13:synas}
{} [{\dots} [\un{AspP} \{PF/IPF\} [\un{VP} {\dots} V {\dots} ]]]
\end{exe}

\noindent What \REF{13:semas} basically says is that the use of a \isi{perfective} form will always impose on interpretation the conditions of completedness
and uniqueness. What \REF{13:synas} adds to that is that these interpretive conditions enter in above the syntactic level of VP (see \citealt{Tate11,Tate13} for a defense). It is thus the \isi{semantics} of the VP that the functions PF and IPF operate on. Several consequences follow from this kind of grammatical architecture.

The first consequence to be noted here is that if the VP-property entails event uniqueness, \isi{perfective aspect} will have to be used.
This prediction seems to be borne out (for the sake of space I will only use \ili{Czech} examples):

\begin{exe}
\ex\label{13:baum}
\gll Jako mal\'a jsem [\un{VP} pok\'acela n\'a\v{s} jedin\'y strom].\\
as small \textsc{aux} {} felled.\textsc{pf} our single tree \\\hfill = \REF{13:baumfaell}
\glt `When I was young, I felled our one and only tree.'
\end{exe}

\noindent In \REF{13:baum}, the VP-property is one that can be realized only once in a given world. The VP thus narrows down the denotation set to unique events. According to \REF{13:semas} and \REF{13:synas}, this strictly calls for the \isi{perfective} (when presupposing completedness) because the speaker cannot but refer to a unique event. This prediction is in line with the use of \isi{perfective aspect} observed in (\ref{13:baum}).

%\begin{exe}
%\ex\label{13:baumrep}
%Jako mal\'a jsem [\un{VP} spadla ze stromu]  
%\end{exe}

Let me now turn to the second consequence that follows from the above made assumptions, specifically concerning general-factuals. If the VP does not restrict denotation to unique events, then on semantic grounds alone the \isi{perfective} is neither required nor excluded. Perfective aspect may be used, but \textit{if} it is used, the expression of event uniqueness introduced by it should be pragmatically motivated. Below I present three contexts in which the pragmatic felicity of \isi{perfective} use is met because expressing uniqueness is what the speaker wants to convey (the list is not meant to be exhaustive).

Context 1: The choice of the \isi{perfective} expressing uniqueness is felicitous because the speaker wants to refer to an accident. This is the case in \REF{13:stromu}, repeated here for convenience.  
 
\begin{exe}
\ex\label{13:baumrep}
\gll Jako mal\'a jsem [\un{VP} spadla ze stromu]. \\
as small \textsc{aux} {} fell.\textsc{pf} from tree\\
\glt `As a child I (once) fell from a tree' 
\end{exe}

\noindent In \REF{13:baumrep}, the speaker reports on an accidental event. It lies in the very concept of an accident that it is unexpected.  
If, unexpectedly, an accident happens to occur once (twice...), it will not be expected to occur a second time (third time...).  
Given this, communicating the existence of an accident, as in \REF{13:baumrep}, or requesting the existence of an accidental event, as in \REF{13:portfel} from above, will always invite an inference of uniqueness. It follows from hypothesis H4 that the \isi{perfective} is to be used because otherwise the event would be understood as non-unique and, hence, as non-accidental.

Context 2: The choice of the \isi{perfective} is felicitous because the speaker refers to an action that requires unusual skills. \REF{13:prknarep}, for instance, refers to a dare. It should be read here as an answer to \REF{13:prkna}:

\begin{exe}
\ex\label{13:prknarep}
\gll Ano, jako mal\'a jsem z toho prkna [\un{VP} sko\v{c}ila].\\
yes as small \textsc{aux} from that diving.board {} jumped.\textsc{pf} \\
\glt `As a child I (once) jumped from that diving board.'
\end{exe}

\noindent Here, arguably, the speaker answers the question of whether she has performed an action that (from the point of view of the questioner at least) requires extraordinary courage of those who perform it. 
Given this, the speaker may assume that the addressee (= questioner) takes the occurrence of such an action as unlikely. 
Similar to the case of accidents, it then follows that if the speaker states that she has performed the action once, she may be sure not to be expected to having performed it a second time. Thus, the expression of uniqueness, which H4 attributes to the use of \isi{perfective aspect}, is well grounded in the context of \REF{13:prknarep}.

\REF{13:gol} shows a similar example. Again, the kind of event is such that already one event realization will count as something special 
(from \citealt[115]{FK18}):

\begin{exe}
\ex\label{13:gol}
\gll U\v{z} jste n\v{e}kdy \{\hspace{-2pt} dal / *\hspace{-2pt} d\'aval\}   g\'ol? \\
already \textsc{aux} ever {} gave.\textsc{pf} {} {} gave.\textsc{pf} goal\\
\glt `Have you ever scored a goal?'
\end{exe}

\noindent Context 3: The choice of \isi{perfective aspect} is felicitous because the speaker refers to an extraordinary event, see \REF{13:lakerep}. 

\begin{exe}
\ex\label{13:lakerep}
\gll V minul\'em stolet\'i Niagarsk\'y vodop\'ad [\un{VP} zamrzl]. \\
in last century N. waterfalls {} froze.over.\textsc{pf}\\
\glt `In the last century Niagara Falls froze over.'
\end{exe}

\noindent It is very difficult to imagine that the Niagara Falls freezes over completely. 
Thus, already one such event is unexpected. If it turns out to have taken place, 
we will not expect it to take place a second (let alone third, fourth, ...) time. Being about an unlikely event, \REF{13:lakerep} conveys uniqueness,
and the attested choice of \isi{perfective aspect} is correctly predicted by H4. 


Finally, I turn to the case that rendered the hypothesis H3 wrong. 

%\begin{exe}
%\ex\label{13:blinddarm3}
%Ano, u\v{z} mu [\un{VP} odstra\v{n}ovali slep\'e st\v{r}evo] \hfill = %\REF{13:blinddarm}
%\end{exe}

\begin{exe}
\ex\label{13:blinddarm3}
\gll Ano, u\v{z} mu [\un{VP} odstra\v{n}ovali {slep\'e st\v{r}evo}].\\
yes already him {} took.out.\textsc{ipf} appendix \\\hfill = \REF{13:blinddarm}
\glt `Yes, his appendix has been removed.'
\end{exe}


\noindent As can be seen, I have identified the element \textit{mu} `him.\textsc{dat}’ as being located outside of the VP. This might seem debatable, but see \citet{Dvorak10} for independent evidence in support of the assumption that benefactive \textit{mu} is 
base-generated above VP. 
The point is that, if this syntactic decision can be maintained, 
the VP of \REF{13:blinddarm3} will turn out to supply a property describing a repeatable event. Countless appendisectomies 
are being carried out at the moment in the hospitals of the world.  
This does not deny the uniqueness intuition that we feel in view of \REF{13:blinddarm3}. The intuition is real, but it arguably comes in 
by semantic composition taking place above VP. Since as a matter of fact every person has at most one appendix, 
the meaning of \textit{mu} serves as a referential anchor for the 
otherwise non-specific meaning of \textit{slep\'e st\v{r}evo} `appendix’. As a consequence, once the semantic contribution of \textit{mu} is taken into account, 
the appendix will be understood to be a specific one. This, in turn, referentially anchors the whole event.
What is described now is no longer a repeatable event, but a unique one. 

Crucially, our new hypothesis H4 does \textit{not} dictate perfectivity for \REF{13:blinddarm3}. Since H4 incorporates the assumption that the \isi{aspectual} operators PF and IPF take VP-meanings as input, and since the VP of
\REF{13:blinddarm3} does not involve uniqueness, the use of \isi{imperfective aspect} is not ruled out on
semantic grounds. H4 predicts that the \isi{imperfective} can be used in contexts where the uniqueness of the event is pragmatically irrelevant to what the speaker wants to convey.

Example \REF{13:abendessen} shows a similar case (adopted from \citealt{Cummins1987}): 


\begin{exe}
\ex\label{13:abendessen}
\gll U\v{z} jsem ve\v{c}e\v{r}el. \\
already \textsc{aux} had.supper.\textsc{ipf} \\
\glt `I've already had supper.'
\end{exe}

\noindent Speakers of \ili{Czech} may refuse an invitation to supper by uttering \REF{13:abendessen}. The utterance will be felt 
to address a unique event, and the use of \isi{imperfective} morphology runs counter to the predictions of H3. Hypothesis H4,  
by contrast, may account for why the \isi{imperfective} is allowed in that case. Again, we have to pay attention to the VP:

\begin{exe}
\ex\label{13:dinner}
\gll U\v{z} jsem [\un{VP} ve\v{c}e\v{r}el].\\
already \textsc{aux} {} had.supper.\textsc{ipf} \\\hfill = \REF{13:abendessen}
\glt `I've already had supper.'
\end{exe}



\noindent In \REF{13:dinner}, as in \REF{13:blinddarm3}, the VP does not describe a unique event.  
The uniqueness-expectation associated with the sentence likewise enters in above VP, i.e. on account of further information provided by the linguistic and non-linguistic context within which
the VP appears. The relevant pieces of information stem from: First, the (dropped) subject, which refers to a specific person as the agent of the event (the speaker). 
Secondly, the topic time, which is a specific day (today). Thirdly, script-knowledge which says that supper is normally taken once per day.    
In sum, the VP does not determine the uniqueness of the event in \REF{13:abendessen}, the use of the \isi{perfective} is therefore not mandatory, and \isi{imperfective aspect} remains an option according to hypothesis H4. 

To sum up, the observations made above amount to the following picture for \ili{Czech} and \ili{Polish} (which is valid not only for general-factual contexts):\footnote{The reader should bear in mind that the exposition presupposes that reference to completed events is intended.}

\begin{itemize}
 \item If the VP-property describes a kind of event that allows for one event realization at most, \isi{perfective aspect} must be used. 
% \item If the VP-property describes a kind of event whose realization entails the existence of more than one event realization, \isi{imperfective aspect} must be used. 
\item If the VP-property does not limit the event realization to be a singleton, 
both \isi{perfective} and \isi{imperfective aspect} are in principle possible. 
\end{itemize}

\noindent In the latter case it is upon the speaker to decide on pragmatic grounds whether the denoted event should be 
understood as unique. If signaling uniqueness was intended by the speaker (because she perhaps wanted to refer to an accident, to a dare or to a sensational news event),
she would have to use a \isi{perfective verb} form. 
If, on the other hand, uniqueness is not what the speaker wants to signal, she should use an \isi{imperfective} verb. 



%$PF \Rightarrow \lambda P \lambda t \exists e . P(e) \wedge e \subseteq t \wedge \neg \exists e'. P(e') \wedge e' \neq e $\\
%$IPF \Rightarrow \lambda P \lambda t \exists e . P(e) \wedge e \bigcirc t $\vspace{0.3cm}\\

\section{Taking Russian into account}\label{13:s8} 
As we saw above, \ili{Czech}/\ili{Polish} and \ili{Russian} general-factuals do not pattern alike. The story told above takes care of the former languages. 
I have proposed denotations for PF and IPF in \ili{Czech} and \ili{Polish} that predict the \isi{aspectual} choices made by the speakers of these languages. 
The open question is: why does \ili{Russian} deviate from \ili{Czech}/\ili{Polish} in general-factual contexts? 

My answer to that question follows \citet{Stunova1991} who traces the differing distributions of the \isi{aspectual} markers in \ili{Czech} and \ili{Russian} 
back to a difference in semantic content of the respective \isi{perfective} category, the \isi{imperfective} category being treated in \ili{Czech} as well as in \ili{Russian} 
as ``an unmarked member of the \isi{aspectual} opposition'' 
(\citealt[297]{Stunova1991}). \citeauthor{Stunova1991}'s (\citeyear{Stunova1991}) results are summarized in \REF{13:czru}:\footnote{It should be noted that the conclusions 
in \citet{Stunova1993} differ from those in \citet{Stunova1991}.}

\begin{exe}
\ex\label{13:czru}
PF\un{Czech}${}\rightlsquigarrow\textsc{totality}$\smallskip\\
PF\un{Russian}${}\rightlsquigarrow\textsc{totality}+\textsc{connectedness}$
\end{exe}

\noindent I propose to reinterpret the feature of `totality' as comprising the features (conditions) `completedness' and `uniqueness'. Given that move, 
Stunov\'a's \isi{semantics} for the \ili{Czech} \isi{perfective} will be in perfect harmony with the conclusions that I have arrived at. 
The remarkable thing is that Stunov\'a's result is derived from empirical observations based on entirely different linguistic 
``parameters'' (in the sense of \citealt{Dickey2000}) than mine. While I am concerned here with the choice of \isi{perfective} or \isi{imperfective aspect} 
in general-factual contexts, \citet{Stunova1991} discusses aspect choice in sequences of events,
in the historical \isi{present}, in generics and in pluractionals.  

Stunov\'a's feature `connectedness' is adopted from Barentsen (outlined in \citealt{Barentsen95,Barentsen98}). 
According to \citeauthor{Barentsen98}'s (\citeyear[45]{Barentsen98}) informal description, an event is ``connected'' if it is viewed from the perspective 
of the changes that it is imposing on its environment. Given this, Barentsen's notion is virtually identical (or at least very similar) to 
\citeauthor{Gronn2004}'s (\citeyear{Gronn2004}) pragmatic notion of
target state relevance, which he derives from the semantic condition of target state validity.\footnote{Here is where the difference between the two notions lies: while target state relevance determines that the event produces an occasion for subsequent events, connectedness is more broadly construed allowing alternatively for that the event starts from the final state created by a preceding event; see \cite[81ff.]{Dickey18} for discussion on that point.} 
The notion of target state validity is formally defined by means of the condition $f$\hspace{-2pt}\uncnst{end}$(t)  \subseteq f\hspace{-2pt}$\uncnst{target}$(e)$.\footnote{The condition $f$\hspace{-2pt}\uncnstfn{end}$(t) \subseteq f$\hspace{-2pt}\uncnstfn{target}$(e)$
requires the reference time to end when the target state is in force.}  

Given all this, we may rewrite \REF{13:czru} as \REF{13:czru1}:

\begin{exe}
\ex\label{13:czru1}
PF\un{Czech}${}\rightlsquigarrow \textsc{completedness}+\textsc{uniqueness}$\\
PF\un{Russian}${}\rightlsquigarrow\textsc{completedness}+\textsc{uniqueness}+\textsc{target state validity}$
\end{exe}

\noindent Now, formally, \REF{13:czru1} may be stated as \REF{13:czru2}:

\begin{exe}
\ex\label{13:czru2}
\sx{PF\un{Czech}}${}= \lambda P \lambda t \exists e [ P(e) \wedge e \subseteq t \wedge \neg \exists e'[ P(e') \wedge e' \neq e ]]$\smallskip\\
\sx{PF\un{Russian}}${}= \lambda P \lambda t \exists e [ P(e) \wedge e \subseteq t  \wedge \neg \exists e'[ P(e') \wedge e' \neq e] \wedge f$\hspace{-2pt}\uncnst{end}$(t)  \subseteq f\hspace{-2pt}$\uncnst{target}$(e)]$
\end{exe}

\noindent The ``semantically unmarked'' \isi{imperfective} will be the same in all of the discussed languages:\footnote{\label{13:fnm}I wish to point out that under the proposed analysis (which closely follows \citealt{Gronn2004}) the \isi{imperfective} is, in fact, not unmarked/unspecified, but rather radically \textit{under}specified in comparison to the \isi{perfective}. Thus, the approach is \textit{not} Jakobsonian. The meanings in \REF{13:czru2} and \REF{13:czru3} all represent Hauptbedeutungen in the sense of \citet[178]{Kury60}.}

\begin{exe}
\ex\label{13:czru3}
\sx{IPF}${}= \lambda P \lambda t \exists e [ P(e) \wedge e \bigcirc t  ]$
\end{exe}

\noindent In \REF{13:czru2}, target state validity is implemented in the \ili{Russian} \isi{perfective} operator as an additional condition besides completedness and uniqueness.
It should be noted, however, that the semantic content of target state validity by itself implies the conditions of uniqueness and completedness (\citealt[342--344]{Mittw08}).
Accordingly, \REF{13:czru2} may be reduced to \REF{13:czru4}:  

\begin{exe}
\ex\label{13:czru4}
\sx{PF\un{Czech}}${}= \lambda P \lambda t \exists e [ P(e) \wedge e \subseteq t \wedge \neg \exists e'[ P(e') \wedge e' \neq e ]]$\smallskip\\
\sx{PF\un{Russian}}${}= \lambda P \lambda t \exists e [ P(e) \wedge f$\hspace{-2pt}\uncnst{end}$(t)  \subseteq f\hspace{-2pt}$\uncnst{target}$(e)]$
\end{exe}

\noindent Now back to the initial question of why \ili{Russian} deviates from the \ili{Czech} and \ili{Polish} pattern in the way it does. The answer is that, given the \ili{Russian} \isi{perfective} 
operator as stated in \REF{13:czru4}, it will be ruled out for semantic reasons in any general-factual context.
The condition of target state validity, and thus the \isi{perfective} operator, is per se incompatible with general-factuals. 
To meet the condition of target state validity,
the event has to have a specific reference time. General-factuals, by contrast, require the event to be located in 
a reference time which is ``big and floating'' 
(\citealt[273]{Gronn2004}; see \citealt{omr16} for an explanation as to why this is so).  

The incompatibility of general-factual interpretations and target state validity being associated with \isi{perfective aspect} 
is, crucially, independent of whether or not the denoted event is unique. 
This is a non-trivial result, as it runs counter to \cite{Dickey18}'s claim that ``[t]he only way to establish that an event [...] is unique in time is to specify the temporal (and causal) context of the event in question. And this can only be done by providing information about prior and subsequent situations''.

An event that has a specific reference time is necessarily unique, but a unique event does not have to have a specific reference time.
This is what sets \ili{Russian} apart from \ili{Czech} and \ili{Polish}, i.e. why \ili{Russian} excludes general-factual perfectives, whereas \ili{Czech} and \ili{Polish} allow for them under the described circumstances.

\section{Conclusions}\label{13:s9} 
In this paper, I have addressed the variation in aspect choice in general-factual contexts between \ili{Czech}, \ili{Polish} and \ili{Russian}. 
I have argued that the asymmetry between \ili{Czech} and \ili{Polish} on the one hand, and \ili{Russian} on the other hand, 
should be related to a difference in the \isi{semantics} of the respective \isi{perfective} operators. While perfectivity in the former languages 
introduces the condition that the denoted event is completed and unique, perfectivity in \ili{Russian} more strongly requires 
that the reference time ends when the target state is in force. The \isi{imperfective} operator is in each of these languages semantically vague in that it requires no more than that the reference time overlaps the event time. 

I have shown that my conclusions are in line with much of the existing descriptive and theoretical literature on \ili{Slavic} aspect.
Specifically, I have made a case 
for the following claims:
\begin{itemize}
\item The two \isi{aspectual} grammemes form a Horn scale, with the \isi{imperfective} being semantically less specified than the \isi{perfective} -- in line with \citet{Sonne06} and \citet{Gronn2004}, and reminiscent of traditional explanations based on markedness (e.g. \citealt[15--16]{Maslov1984}).
\item The point just noted does not only hold for \ili{Russian}, but for \ili{Czech} and \ili{Polish} as well -- in line with \citet[116]{FK18}, but contra \citet[105]{Dickey2000}.
\item The \isi{perfective} operator in \ili{Russian} entails target state validity -- in line with \citet{Gronn2004}, and arguably compatible with \citet{Barentsen98} and \citet{Dickey2000}.
 \item The \ili{Russian} \isi{perfective} category has a more specific content than the \ili{Czech} \isi{perfective} category -- in line with \citet{Stunova1991} and \citet{Dickey18}.
 \item Aspect is syntactically located outside of (above) the VP -- in line with, e.g., \citet{Tate11}.
 \end{itemize}

\noindent Still, many questions remain open. How do the generalizations that I derived from general-factual contexts agree with the patterns of \isi{aspectual} variation observed in other
contexts (``parameters'')? The closeness to \citeauthor{Stunova1991}'s results gives rise to optimism, but these things have to be checked.  

I wish to conclude with a further argument that one might bring forward in support of the story told in this paper. 
\citet[112]{Dickey2000} reports that the \ili{Polish} \isi{perfective} in \REF{13:win} is possible given the following scenario: 
The speaker, who had instructed the hearer to air the room beforehand, has entered the room, the hearer is around, and the (only) window is closed at the moment.   
This possibility of \isi{perfective aspect} is in sharp contrast to the case of \ili{Russian}, where the 
use of a \isi{perfective verb} would strictly require the window to be open. 

\begin{exe}
\ex\label{13:win}
\gll Czy otworzy\l{e}\'s ju\.z okno? \\
\textsc{q} opened.\textsc{pf} already window\\
\glt `Did you already open the window?'
\end{exe}

\noindent Drawing on a suggestion made by \citet[84]{Dickey18}, I speculate that the absence of target state validity in the \ili{Polish} \isi{perfective} operator provides the reason why the \isi{perfective} is usable here despite result annullment, 
and that the significance of uniqueness (that there is the expectancy of a single event realization) explains why the \isi{perfective} is indeed used in the particular context at hand. 
This points to but one out of many intriguing issues that await investigation in the field of inner-\ili{Slavic} \isi{aspectual} variation.  






\section*{Abbreviations}

\begin{tabularx}{.5\textwidth}{@{}lQ@{}}
\textsc{aux} & auxiliary \\
\textsc{(i)pf} & (im){perfective}\\
\end{tabularx}%
\begin{tabularx}{.5\textwidth}{@{}lQ@{}}
\textsc{q}&polar question marker\\
\textsc{refl}&reflexive\\
\end{tabularx}


\section*{Acknowledgements}
I would like to thank the audience of FDSL 12 and two anonymous reviewers for valuable feedback. 
My gratitude also goes to 
Anna Artwińska, Petr Biskup, Mojmír Dočekal, Małgorzata Gałaś-Prokopf, Wojciech Roskiewicz, Danuta Rytel-Schwarz, Yulia Sorokina, Lenka Vávrová, Marcin Wągiel, Maria Yastrebova and Dagmar Žídková-Gunter. Finally I wish to thank the editors of this volume and Radek Šimík in particular for thoughtful and instructive comments. 
All remaining shortcomings are, of course, my own.  

{\sloppy
\printbibliography[heading=subbibliography,notkeyword=this]
}
\clearpage 
\end{document}
