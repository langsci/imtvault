\documentclass[12pt,a4paper]{book}
\usepackage[utf8]{inputenc}
\usepackage[tone]{tipa} 
\usepackage[english]{babel}

%Bibliographieren
\usepackage[backend=biber, style=authoryear, maxnames=99]{biblatex}
\addbibresource{MVCliterature.bib}

\usepackage{expex}
\defineglwlevels{d} 
\usepackage{pstricks}
\usepackage{pst-xkey}
\usepackage{pst-jtree} 
\usepackage{pbox}
%\usepackage{setspace} 
\usepackage[margin=10pt,font=footnotesize,
labelsep=endash]{caption}

\usepackage{tabularx}
\newcolumntype{L}[1]{>{\raggedright\arraybackslash}p{#1}} % linksbündig mit Breitenangabe
\newcolumntype{C}[1]{>{\centering\arraybackslash}p{#1}} % zentriert mit Breitenangabe
\newcolumntype{R}[1]{>{\raggedleft\arraybackslash}p{#1}} % rechtsbündig mit Breitenangabe



%epigraph settings
\makeatletter
\newenvironment{chapquote}[2][2em]
  {\setlength{\@tempdima}{#1}%
   \def\chapquote@author{#2}%
   \parshape 1 \@tempdima \dimexpr\textwidth-2\@tempdima\relax%
   \itshape}
  {\par\small\hfill---\ \chapquote@author\hspace*{\@tempdima}\par\bigskip}
\makeatother

\usepackage{multicol}
\usepackage{colortbl}
\usepackage{graphicx}
\usepackage{array}
\usepackage{booktabs}
\usepackage{rotating}

\newcolumntype{P}[2]{%
  >{\begin{turn}{#1}\begin{minipage}{#2}\small\raggedright\hspace{0pt}}l%
  <{\end{minipage}\end{turn}}%
}

%Headers fancyhdr
\usepackage{fancyhdr}
\pagestyle{fancy}% muss vor \renewcommand{\sectionmark} stehen
\fancyhf{}
\fancyhead[EL]{\thepage}% gerade Seiten, links
\fancyhead[ER]{\leftmark}% gerade Seiten, rechts
\fancyhead[OL]{\rightmark}% ungerade Seiten, links
\fancyhead[OR]{\thepage}% ungerade Seiten, rechts
\renewcommand{\sectionmark}[1]{\markboth{\thesection{} #1}{}}
\renewcommand{\subsectionmark}[1]{\markright{\thesubsection{} #1}
}

\usepackage{avm}
\avmfont{\sc}
\avmsortfont{\scriptsize\it}
\usepackage{tree-dvips}
%\usepackage{amsfonts}
%\usepackage{amssymb}
\usepackage{longtable}
\usepackage{multirow}
\usepackage{fontenc} 
\usepackage{placeins} % Befehl FloatBarrier verhindert deplatzierte Tabellen, Bilder etc
\usepackage[nonumberlist,nopostdot,acronym,shortcuts]{glossaries}
\setacronymstyle{long-sc-short}
\author{Volker Unterladstetter}
\title{Multi-verb constructions in Eastern Indonesia}

%Glossaries settings
\loadglsentries{abbreviations.tex}
\makenoidxglossaries

%subsubsections included in toc
\setcounter{tocdepth}{4}
\setcounter{secnumdepth}{4}

%Document starts here
\begin{document}
\begin{titlepage}
\centering
\vspace*{0.5in}
\begin{Huge}
\rule{\linewidth}{0.5mm}\\
Multi-verb constructions in Eastern Indonesia
\rule{\linewidth}{0.5mm}\\
\end{Huge}
\vspace{2in}

\begin{large}\bfseries
Volker Unterladstetter\par
\end{large}
\end{titlepage}

% Roman numbering for first pages
\setcounter{page}{1} \pagenumbering{roman}

\tableofcontents

\newglossarystyle{acronym_index}{
\setglossarystyle{index}% base this style on the index style
\renewenvironment{theglossary}{
\begin{multicols}{2}\begin{footnotesize}\begin{description}}{\end{description}\end{footnotesize}\end{multicols}} 
% the default environment for list is \begin{description} \end{description}
}


\addcontentsline{toc}{section}{\textit{Abbreviations}}
\printnoidxglossary[type=acronym, style=acronym_index, title={Abbreviations}]


\pagebreak
\thispagestyle{empty}

\vspace*{2in}
\begin{chapquote}[30pt]{Marie Luise Kaschnitz}
{\small\noindent Nur ein Wort und ein Wort und ein Wort\\
Wahllos aus dem Sprachnetz gerissen\\
Zueinandergeschleudert\\
Umarmen sich\\
Sind sogleich eine\\
Sind eine Welt.}
\end{chapquote}

\pagebreak


% Arabic numbering for subsequent pages
\pagenumbering{arabic} \setcounter{page}{1}

\include{1_Introduction}

\include{Area_of_Eastern_Indonesia}

\include{2_Theoretical_foundation}

\include{3_Grammatical_properties}

\include{4_Semantic_properties} 

\include{5_Construction_types}

\include{7_Discussion}

\nocite{*}
\addcontentsline{toc}{chapter}{Literature}
\printbibliography
\end{document}

