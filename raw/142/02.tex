\chapter{An evolving landscape}\label{chap2}

Around a decade ago, an important review article entitled ``Language evolution in the laboratory'' \citep{scott2010language} was published in \textit{Trends in Cognitive Sciences}. Its central message, in my opinion, was that it was becoming possible, at last, to approach at least certain aspects of language evolution in a scientific manner. This was a sharp departure from over a century of statements declaring that language evolution was a mystery.

Remnants of this old attitude still exist \citep{hauser2014mystery}; they typically invoke in a tedious fashion the 1866 ban on all discussion of the evolution of language imposed by the Linguistic Society of Paris; they also frequently cite Lewontin's pronouncement that we will never know why cognition evolved the way it did \citep{lewontin1998evolution}. But things have changed quite dramatically over the past two decades, so much so that it has become possible to contemplate ``controlled hypothesis-testing through experimentation'' \citep{motamedi2019evolving} in the domain of language evolution.

I still recall being told as a graduate student that the topic of language evolution was more a matter of science fiction than science, and that this was best left as a domain of study for after retirement. Today, some of the brightest students I know are actively engaged in this field, illustrating the massive progress made over the past 20 years, well attested in the Proceedings of the Evolang conference series, as well as in the creation of centers for the study of language evolution in Edinburgh and more recently Z\''{u}rich. The main change (still ongoing), to my mind, is the resistance to exploring hypotheses until they can be formulated in a way that can be put to the test. A change from `I think \textit{x}' to `I think \textit{x} and I can test \textit{x} doing \textit{y}.'

The efforts of members of the Centre for Language Evolution at the University of Edinburgh, led by Simon Kirby, have shown how combining the development of artificial languages (mini-grammars) in a laboratory setting \citep{kirby2008cumulative,kirby2015compression}, as well as agent-based modelling approaches controlling for biases that language users in the lab bring to the task in an unconscious manner \citep{thompson2016culture}, reveals how learnability and expressivity pressures shape grammars. Subsequent work from other centers (e.g. \cite{raviv2019larger,raviv2020language,raviv2021makes}) also experimentally demonstrates how communicative contexts impact grammar formation and the emergence of new languages. While it is often said that such work only addresses language change (`glossogeny'), and not language evolution proper (language phylogeny, the emergence of the modern language capacity),\footnote{Terminology introduced in \cite{hurford1990nativist}.} I do not find this dichotomy particularly useful, and believe that a continuum of cognitive biases that interact with changing communicative conditions from which language-readiness emerges, shaping the range of grammars acquired, is a more adequate stance (more on this in chapter \ref{chap4}).

The same year the review article by \cite{scott2010language} appeared, the first draft of the Neanderthal genome was published \citep{green2010draft}, starting a revolution that continues unabated to this day \citep{reich2018we}. As we will see later on, the successful retrieval of ancient DNA, from a few skeletal remains and now even cave sediments, and of ancient proteins, allows us to ask questions at an unprecedented level of resolution and dramatically changes what we mean by ``fossil record''. The debt we owe to Svante Pääbo and his collaborators is hard to overstate \citep{paabo2014neanderthal,meyer2012high,prufer2014complete,prufer2017high,mafessoni2020high,slon2017neandertal,vernot2021unearthing,zavala2021pleistocene,welker2016palaeoproteomic,welker2020dental}.

Yet this massive amount of data that is now accessible would be ``empty'' if it were not for the progress made in linking the genotype and the phenotype. In the domain of language, the work pioneered by Simon Fisher on \textit{FOXP2} is the gold standard \citep{lai2001forkhead}, and arguably one of the most significant achievements in the language sciences in the past twenty five years \citep{fisher2009foxp2,fisher2015genetics,fisher2019human,den2021molecular}. It has taught us that for all the intricacies and levels of analyses separating genes and behavior, careful work can illuminate central issues that Lenneberg could only dream of when he wrote his classic book, \textit{Biological Foundations of Language}, over fifty years ago \citep{Lenneberg1967biological}.

Equally important for the success of what is sometimes called ``evolinguistics'' is the dramatic shift of perspective that took place in the domain of comparative psychology. This is well-captured in the following passage from \cite{de2010towards}:

\largerpage
\begin{quote}	
Over the last few decades, comparative cognitive research has focused on the pinnacles of mental evolution, asking all-or-nothing questions such as which animals (if any) possess a theory of mind, culture, linguistic abilities, future planning, and so on. Research programs adopting this top-down perspective have often pitted one taxon against another, resulting in sharp dividing lines. Insight into the underlying mechanisms has lagged behind \ldots \clearpage


A dramatic change in focus now seems to be under way, however, with increased appreciation that the basic building blocks of cognition might be shared across a wide range of species. We argue that this bottom-up perspective, which focuses on the constituent capacities underlying larger cognitive phenomena, is more in line with both neuroscience and evolutionary biology.
\end{quote}

In the domain of language, calls for recognizing an ever broader ``community of descent'', to borrow a phrase from \cite{darwindescent}, are more and more frequent \citep{lattenkamp2018vocal}. Far from being rhetorical, these calls demonstrate how much one can learn about our kind by studying behavior in numerous species in accordance with Tinbergen's multi-level approach.

As Ernst Mayr was fond of saying, ``evolutionary biology [unlike physics] is a historical science, [where] one constructs a historical narrative, consisting of a tentative reconstruction of the particular scenario that led to the events one is trying to explain'' \citep{mayr2000darwin}. Narratives will continue to dominate evolutionary investigations into language, but crucially, thanks to the progress made in key areas that I singled out above, these narratives are enriched with, and constrained by, ``numbers''. Hypotheses can now be put to the test.

It becomes very apparent in this context that simple narratives, appealing as they may appear, are hopelessly misguided. Recalling the words of H. L. Mencken, ``For every complex problem there is an answer that is clear, simple, and wrong''. What more complex problem is there than the problem of language evolution?

Accordingly, the simple, clear, ``minimalist'', and influential evolutionary scenario advocated by Berwick and Chomsky in their book \textit{Why Only Us} \citep{berwick2016only} must be wrong.\footnote{If I am right, this has non-trivial ramifications for the minimalist program. Over the years, talk of optimization, efficiency, etc., which occupied center stage in the early days of the program, has been replaced by a focus on evolutionary considerations. If such considerations lead to an impasse, the program as a whole may indeed have been (at best) premature.} I have tried to say so on several occasions \citep{boeckx2017not,martins2019language,de2020evolutionary}. Very briefly: it is wrong because it disregards the comparative evidence (`only us'), it fails to appreciate the multi-level approach required to link genotype and phenotype (claiming that a single mutation yields the simple, atomic operation ``merge''), it keeps the discussion at the logical level, without attempting to even sketch a plausible path to testing it, and does not engage with the many lessons coming from the great discoveries in paleo-sciences over the past decade.

The reason I have spent time arguing against Berwick and Chomsky's narrative is not only because it was proposed by influential linguists, but because it is representative of a family of approaches that linguists remain attracted to: it presupposes that other animals don't have much to teach us about the core of our language faculty, because essentially they are non-linguistic creatures. The gap between them and us is a chasm. It also takes for granted that our language capacity is very recent in evolutionary terms, going back maybe 150 000 years. As such, so the claim goes, there was very little time to evolve a ``kludgy'' language organ (cf. \cite{marcus2009kluge}). Accordingly, a narrative must be developed that keeps the core language faculty essentially free of evolutionary tinkering.

Such a narrative (in many ways, the culmination of the minimalist program envisaged by Chomsky) clashes with recent attempts to attribute a significant portion of our ``modern'' language faculty to the last common ancestor shared with our closest extinct relatives \citep{dediu2013antiquity,dediu2018neanderthal}. It also clashes with mounting evidence for a complex, temporally very extended, mosaic-like evolution of our lineage \citep{scerri2018did,Bergstrom}. Also, it makes certain assumptions about how many changes can be favored by natural selection within a relatively short window of time which are not obviously true---indeed, very implausible \citep{de2020evolutionary}. Last, but not least, it grants too much power to linguistic theorizing. As argued in \cite{martins2019language}, it is fallacious to draw a direct correspondence between the formal structure of a computational operation and the biological changes that would lead to it.\footnote{In their reply to \cite{martins2019language}, \cite{berwick2019all} completely---and surprisingly---miss this point; see \cite{martinsboeckx_clar} for illustration.} It is what theoretical linguists would love to be able to do: it would make their theoretical work immediately relevant for evolutionary claims. But it is logically incorrect. This is precisely why, in my opinion, evolutionary considerations impact how we do theoretical linguistics, or how we see the import of that work. If there is no such direct correspondence, if the link between genotype and phenotype is very complex indeed, I do not see any alternative to painstakingly developing linking hypotheses that, we hope, progressively spell out what it means to say that our linguistic condition is part of our human (biological) condition.

I want to insist once more on the importance of debunking simple accounts like Berwick and Chomsky's. It may well be that there will be certain behaviors or artifacts or anatomical traits that we can confidently ascribe exclusively to members of our species that ``emerged'' recently. Right now this is being questioned, but I would not be surprised if we are left with a small set of recent ``\textit{sapiens}-exclusive'' properties (brain changes giving rise to our globular skull, use of complex symbiotic tools like the bow and arrow, and some aspects of figurative art are fairly good bets in my current opinion), but crucially, even if the evidence settles along these lines, it should not be used to argue for a recent cognitive revolution that matches a minimalist vision of the language faculty. Rather, such evidence will have to be integrated into the complex mosaic of language that evolution has constructed over an extended period of time.

This is certainly a major lesson I learned from thinking about Darwin's problem: Evolutionary considerations invalidate certain theoretical frameworks that fail to come to grips with the ``complex dynamical system'' nature of language. The next two chapters deal with other lessons that pertain to a broader range of approaches, and implicate a larger number of researchers: even those linguists that readily accept that the evolutionary trajectory of our language capacity was long and complex still subscribe to certain views that I think we would do well to abandon. I'll focus on three such views here. One is that somehow, there is at least one aspect of language (typically, some aspect of syntax) that makes our language capacity special, and that as a result forms some sort of barrier in a comparative setting. Another is the belief that linguistic theory matters and that one's theory of language evolution depends on one's theory of language. And third, the claim that because languages don’t leave fossils, the evidence for studying the evolution of language is too sparse. These three claims are incorrect.
