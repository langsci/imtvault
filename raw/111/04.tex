\chapter{Markings of information structure}
\label{chapter4}
\setcounter{enums}{0}


\noindent The main goal of this chapter is to find the range of
possible expressions with respect to information structure. Different
languages employ different marking systems, and the linguistic means
of conveying information structure meanings includes: (i)
\isi{prosody}, (ii) \isi{lexical markers}, (iii) syntactic
positioning, and (iv) combinations of these \citep{gundel:99}. This
chapter explores how these meanings are specifically realized in
various languages.  This contributes to typological studies of human
languages, and also carries weight with implementing a grammar library
for information structure within the \lingo Grammar Matrix
customization system
\citep{bender:flickinger:05,drellishak:09,bender:etal:10}.  Because
users of that system are referencing the actual linguistic forms in
their language it is important that the library that they use
systematize linguistic realizations in a sufficiently fine-grained
way.


This chapter is structured as follows: Section \ref{4:sec:prosody} addresses
prosodic means of expressing information structure.\is{prosody} The
present work does not directly implement constraints on prosodic
patterns into the system, but presents a flexible representation for
them to set the ground work for a further developed system.
Section \ref{4:sec:lexical} looks into \isi{lexical markers} responsible for
\isi{focus} and \isi{topic} from a cross-linguistic viewpoint. These are
classified into three subclasses: affixes, adpositions, and
modifiers.\is{adposition} Section \ref{4:sec:syntactic} surveys positioning
constraints on information structure components in human language.





\section{Prosody}
\label{4:sec:prosody}

In much of the previous work on this \isi{topic}, prosody has been presumed
to be the universal means of marking information structure
\citep{gundel:99,buring:10}.\is{prosody} Many
previous papers have studied information structure with special
reference to how it is marked by prosody.  \citet{bolinger:58} argues
that there are two types of pitch accents in \ili{English}; the A and
B-accents (i.e.\ H* and L+H* in the ToBI format respectively).
\citet{jackendoff:72} creates a generalization about the correlation
between pitch accents and information structure components: A and
B-accents in English are responsible for marking constituents as \isi{focus}
and topic respectively.\footnote{Admittedly, there are quite a few
  recent and comprehensive studies of the interaction between prosody
  and information structure, such as \citet{ladd:08}, 
  \citet{chen:12}, and many others. Their analyses may help model information structure
  in a cross-linguistic perspective. Nonetheless, the present study
  does not enter into the deeper details of them, mainly because the
  current model basically aims to be used for text-based processing
  systems. }  The way in which A and B accents structure information
is exemplified in \myref{exe:ab-accent}, in which \textsc{small caps}
represents the A-accent, and \textbf{boldface} represents the
B-accent. The constituent semantically associated with aboutness bears
the B-accent in \ili{English}, and because it refers to aboutness,\is{aboutness} is
identified as the topic in the present study.  The constituent
corresponding to the \textit{wh}-word in the question \textit{What did
  Kim read?} bears the A-accent, which gives a \isi{focus} meaning.


\myexe{\enumsentence{\label{exe:ab-accent}
\begin{tabular}[t]{ll}
Q: & {What about Kim? What did Kim read?}\\
A: & {\textbf{Kim} read the \textsc{book}.}\\
\end{tabular}}}


In the following subsections I explore the details of three
perspectives on incorporating prosodic information into grammatical
structures.  This is done with an emphasis on application in the
creation of an information structure library as a tool for grammar
engineering.


\subsection{Prosody as a widespread means of marking}
\label{4:ssec:universal}

Since \citet{jackendoff:72}, quite a few studies have explored the
connection between prosodic patterns and information structure in
languages, including \ili{English} \citep{steedman:00}, German
\citep{buring:03}, Portuguese \citep{frota:00}, \ili{Japanese} and
\ili{Korean} \citep{ueyama:jun:98}.  However we should not assume that
every language employs prosody for marking information structure.  In
fact there are several counterarguments to treating prosody as a
language-universal way to express \isi{focus} and/or \isi{topic}.



My cross-linguistic survey reveals several languages with no means of
expressing information structure through prosody.  For instance, it is
reported that \ili{Yucatec Maya} employs no prosodic marking for
expressing information structure.  Instead, syntactic functions
indicate these relations without an interaction with prosody
\citep{kugler:etal:07}.  In \ili{Akan}, prosodic patterns also have
little to do with expressing \isi{focus}, and instead a focused item must
occupy the clause-initial position with one of several morphological
markers \citep{drubig:03}.\is{clause-initial} Likewise, \ili{Catalan},
in which syntactic operation is responsible for marking information
structure, has a rather weak (or even null) correlation between
prosody and information structure meanings
\citep{engdahl:vallduvi:96}.\is{prosody} Hence, the assumption that
prosody is a language-universal means of marking information structure
is not valid.  That is to say, using prosody for expressing
information structure is clearly widespread, but not universal
\citep{drellishak:09}.


\subsection{Mappings between prosody and information structure}
\label{4:ssec:conditions}


There seems to be no clear consensus with respect to mappings between
prosody and information structure even in \ili{English}. Contra to
\citeauthor{jackendoff:72}'s claim, (i) \citet{kadmon:01},
\citet{buring:03}, and \citet{oshima:08} argue that B-accents are
specifically responsible for contrastive topics,\is{contrastive topic}
rather than \isi{topic} in a broad sense.
(ii) \citet{steedman:00} argues that B-accents mark
theme, and additionally associates information structure meanings with
boundary tones.  (iii) \citet{hedberg:06} regards the use of a
B-accent as a contrastive marker for both \isi{focus} and topic
(i.e.\ either \is{contrastive focus} or contrastive topic). (iv) More
recently, \citet{constant:12} explores how semantic and pragmatic
behavior is influenced by a specific prosodic `rise-fall-rise' pattern
in \ili{English} (transcribed in the ToBI format as [L*+H L- H\%]), as
illustrated in \myref{exe:constant}. That is, there are three
components: The first `rise' corresponds to [L*+H], `fall' to [L-],
and the second `rise' to [H\%].\footnote{The main argument
  \citet{constant:12} provides is that the `rise-fall-rise' intonation
  involves a regular conventional implicature, acting as a \isi{focus
    sensitive} \isi{quantifier} over assertable alternative
  propositions.}

\myexe{\enumsentence{\label{exe:constant}
\begin{tabular}[t]{llllll}
A: & \multicolumn{5}{l}{Why isn't the coffee here?}\\ 
B: & I don't know. I was & \textit{expecting} & there to be & coffee ...\\ 
   & & L*+H & L- & H\% \\
   & & & \multicolumn{2}{l}{\citep[409]{constant:12}}\\
\end{tabular}}}


\noindent \citeauthor{constant:12} investigates the correlations
between `rise-fall-rise' intonation and \isi{contrastive topic} intonation.
\citeauthor{constant:12} denies the previous assumption that the
former is a subclass of the latter.



Among the varied claims, I follow \citeauthor{hedberg:06}'s argument,
mainly because \citeauthor{hedberg:06}'s classification is firmly
based on an acoustic analysis of naturally occurring spoken data
\citep{hedberg:sosa:07}: A-accents are responsible for non-contrastive
\isi{focus}, while B-accents are responsible for \isi{topic} and contrast in
\ili{English}. 




The debate presented above is largely concerned with which prosodic
pattern has which effect on information structure, and the nature of
the mapping between prosody and information structure. However, there
exist some circumstances in which prosody is not involved in the
articulation of information structure (even in \ili{English}).
\citet{fery:krifka:08} argue prosodic patterns are not obligatorily
related to information structure even in \ili{English}.\is{prosody}
For example, the association between prosody and \isi{focus} can be canceled
in the context of Second Occurrence Focus.  A second occurrence focus
is an expression that falls within the scope of a \isi{focus sensitive
  operator} (e.g.\ \textit{only} in English), but is a repeat of an
earlier focused occurrence \citep{partee:99,beaver:etal:07,
  fery:ishihara:09}. The repeatedly focused item prosodically differs
from the previously focused one (i.e.\ ordinarily focused), and is
normally devoid of a specific pitch accent responsible for marking
focus.  Because \textit{vegetables} in (\ref{exe:fery:krifka:08:sof}b)
is combined with a \isi{focus sensitive} item \textit{only}, it would
be interpreted as containing \isi{focus} meaning, but that meaning is
already given in (\ref{exe:fery:krifka:08:sof}a).




\myexe{\eenumsentence{\toplabel{exe:fery:krifka:08:sof}
\item Everyone already knew that Mary only eats [vegetables]\mysub{F}.
\item If even [Paul]\mysub{F} knew that Mary only eats
  [vegetables]\mysub{SOF},\\ then he should have suggested a different
  restaurant. \citep[215--216]{partee:99}}}


\noindent \myref{exe:fery:krifka:08:sof} is a clear counterexample to
\citeauthor{halliday:67}'s claim that what is focused should carry new
information as `vegetables' in (\ref{exe:fery:krifka:08:sof}b) has
already been mentioned.  In addition, while the \textit{vegetables} in
(\ref{exe:fery:krifka:08:sof}a) bears an A-accent, the repeated
occurrence in (\ref{exe:fery:krifka:08:sof}b) does not.  According to
\citet[132]{fery:krifka:08}, ``there are only weak correlates of
accent, and no pitch excursions in the postnuclear position.''. 
This means that the \isi{focus} meaning in this case is
not directly invoked by the A-accent.




These findings indicate that prosodic patterns do not always reliably
reveal information structure.\footnote{\citet{fanselow:07} provides a
  view against this. The claim is that the connection between
  information structure and syntax is mediated by prosody,\is{prosody}
  with no direct link between information structure and syntax.  I do
  not follow this, because my cross-linguistic survey reveals that
  some languages, such as Catalan \citep{engdahl:vallduvi:96}, Akan
  \citep{drubig:03}, and Yucatec Maya \citep{kugler:etal:07}, have a
  system with very weak or no interaction between prosody and syntax
  with respect to \isi{focus}.} In other words, prosodic prominence is
merely a tendency; it is neither a sufficient nor a necessary
condition for conveying information structure meanings even in
languages whose markings are largely dependent on prosody
(e.g.\ \ili{English}) \citep{rochemont:86,drubig:03}.





\subsection{Flexible representation}
\label{4:ssec:flexible}


Prosody makes a contribution to information structure in many
languages,\is{prosody} even if the relationship between prosodic
marking and information structure is complicated. However, in some
contexts, especially processing of texts that were originally written
(rather than transcribed speech), we do not have access to prosodic
information anyway.  Given that our processing system is usually
text-based, currently it is almost impossible for us to resolve the
phonological patterns of sentences, including intonation contour and
pitch accents.  The best way to handle prosodic marking is to allow
for \isi{underspecification} in such a way that prosodic information
can be later added into the formalism.  \citet{kuhn:96} in the same
context suggests an underspecified representation for information
structure, noting that even prosodic marking of information structure
often yields ambiguous meanings, which cannot in general be resolved
in sentence-based processing.  The present work employs
underspecification for representing information structure when the
meaning is not fully solved by prosody.\is{underspecification} In
principle, this would allow for refining the representation
monotonically.



\section{Lexical markers}
\label{4:sec:lexical}


According to my cross-linguistic survey, there are three subtypes of
\isi{lexical markers} that assign information structure roles; (i) affixes,
(ii) adpositions, and (iii) modifiers.\is{adposition}

Quite a few languages have specific affixes to signal \isi{focus}, \isi{topic},
and \isi{contrast}, as exemplified in the following \ili{Rendile} (Cushitic,
Afro-Asiatic, spoken in northern Kenya) examples, in which two affixes
are used to express an argument focus (i.e.\ \textit{{\'e}} by an
enclisis process) and a predicate focus (i.e.\ \textit{{\'a}} by a
proclisis process) respectively \citep{lecarme:99}.


\myexe{\eenumsentence{\label{exe:rel}
\item \shortex{2}
  {{\'i}nam-{\'e} & yimi}
  {boy-\textsc{foc} & came}
  {`\textsc{The boy} came.'}
\item \shortex{2}
  {{\'i}nam & {\'a}-yimi}
  {boy & \textsc{foc}-came}
  {`The boy \textsc{came}.' [rel] \citep[277]{lecarme:99}}}}


\noindent Some languages use affixes responsible for \isi{topic} meanings;
for instance, \nun in \ili{Korean} is used to signal information structure
meanings (\tdl{contrast-or-topic} in the current work), and is in
complementary distribution with ordinary case morphemes (e.g.\ \ika
for nominatives, \lul for accusatives).



\myexe{\enumsentence{\label{exe:nun}
\shortex{3}
  {ku & kay-nun & cic-e}
  {\textsc{det} & dog-\textsc{nun} & bark-\textsc{decl}}
  {`The \textbf{dog} barks.'  [kor]}}}



\noindent Unlike the \isi{focus} affixes used in (\ref{exe:rel})
(i.e.\ \textit{{\'e}} and \textit{{\'a}}) which directly signal the
information structure roles of the constituent, \nun in \ili{Korean}
is not deterministic. The word which \nun is attached to can be
ambiguously interpreted. This is addressed in Section \ref{5:sec:lex} in
detail.




Clitics are also often employed to express information structure. A
clitic, somewhere between morpheme and word, is a linguistic
item that is syntactically independent, but phonologically dependent.
Clitics used for information structure markings can be subclassed into
two types; adpositions and modifiers.\footnote{Note that 
I do not argue that all adpositions are necessarily enclitics.} 
Adpositions are responsible for
information structure markings in \ili{Japanese}.\is{adposition} In
\myref{exe:wa}, the adposition \wa is responsible for conveying
contrast or \isi{topic}.



\myexe{\enumsentence{\label{exe:wa}
\shortex{3}
  {inu & wa & hoeru.}
  {dog & \textsc{wa} & bark}
  {`The \textbf{dog} barks.'  [jpn]}}}



\noindent On the other hand, clitics that have nothing to do with case
marking can also be used as \isi{lexical markers} for information
structure. They are regarded as modifiers in the current work. For
instance, \citet{man:07} presents two types of \ili{Cantonese} lexical
particles that mark NPs for information structure roles: \textit{aa4}
and \textit{ne1} as the \isi{topic} marker and \textit{aa3}, \textit{laa1},
and \textit{gaa3} as \isi{focus} markers, respectively.

\myexe{\eenumsentence{\label{exe:man:part}
\item \shortex{8}
{nei1 & bun2 & syu1 & \myemp{aa4} & ngo5 & tai2gwo3 & hou2do1 & ci3}
{\textsc{def} & \textsc{clf} & book & \textsc{part} & 1\textsc{sg} & read.\textsc{exp} & many & times}
{`As for this book, I have read it for many times.' [yue]}
\item \shortex{6}
{keoi5 & \myemp{aa3} & bun2 & syu1 & ngo5 & bei2zo2}
{3.\textsc{sg} & \textsc{part} & \textsc{clf} & book & 1\textsc{sg} & give.\textsc{perf}}
{`It is him/her who I have given the book to.' [yue] \citep[16]{man:07}}}}


\noindent Clitics are made use of to designate the \isi{topic} and/or the
\isi{focus} in other languages, too. For example, \ili{Cherokee} (a native
American language (Iroquoian), still spoken in Oklahoma and North
Carolina) employs a second-position clitic \textit{=tvv} as the focus
marker, meaning it immediately follows the focused word as shown below
\citep{montgomery:08}.

\myexe{\eenumsentence{\label{exe:chr}
\item\shortex{2}
{ayv=tvv & yi-tee-ji-hnooki}
{1\textsc{pro}=\textsc{fc} & \textsc{irr-dst-1a}-sing.\textsc{imm}}
{`I am going to sing it.'}
\item\shortex{2}
{no\'okwu=tvv & ji-tee-a-asu\'ula-a}
{now=\textsc{fc} & \textsc{rel-dst-3a-}wash.hands:\textsc{imm-imm}}
{`He just washed his hands.' [chr] \citep[152]{montgomery:08}}}}



As noted above, the present study defines three subtypes of lexical
markers for expressing information structure: (i) affixes, (ii)
adpositions,\is{adposition} and (iii) modifiers.\footnote{Someone may
  claim that what I regard as an adposition in a given language is a
  modifier or something. Admittedly, I am concerned with finding the
  full range of potential ways to mark information structure. This
  enables the users of the \lingo \isi{Grammar Matrix} system to have
  flexibility in describing what they see in their language following
  the meta-modeling idea of \citet{poulson:11}.}




The differences among them are as follows: First, (i) affixal markers
such as \nun in \ili{Korean} always behave dependently within the
morphological system (as shown in \ref{exe:nun}). In contrast,
adpositions\is{adposition} (e.g.\ \isi{lexical markers} in
\ili{Japanese}) and modifiers (e.g.\ particles in \ili{Cantonese}
 and \ili{Cherokee}) are dealt
with as separate words in the language. Second, if a language employs
a non-affixal marker to express information structure, there are two
options: (ii) If a non-suffixal marker is used to express information
structure and the language employs adpositions, the marker is regarded
as an adposition, too. In other words, when a language makes use of
case-marking adpositions, and the adpositions are in complementary
distribution with a lexical marker of information structure (as in
Japanese), the marker is subtyped as an adposition.\is{lexical
  markers} (iii) Otherwise, the lexical marker is regarded as a
modifier.



According to my survey, there are four constraints on lexical markers
for information structure. They are presented in the following
subsections.\is{lexical markers}



\subsection{Multiple markers}
\label{4:ssec:multiple}

Human languages can have multiple lexical markers for expressing
either focus or \isi{topic}, with different syntax from each other. Turning
back to the \ili{Rendile} example \myref{exe:rel}, \textit{{\'e}} is
used for nominals, while \textit{{\'a}} is a verbal \isi{focus}
marker. There are similar cases in other languages, too: For example,
Akan employs two focus markers; one is \textit{na} that appears only
in sentential replies, and the other is \textit{a} that shows up only
with a short answer \citep[4]{drubig:03}.


\myexe{\eenumsentence{\label{exe:aka}
\item[Q:]\shortex{4}
{Hena & na & Ama & rehwehw\textepsilon?}
{who & \textsc{foc} & Ama & is.looking.for}
{`Who is it that Ama is looking for?'}
\item[A1:]\shortexnt{4}
{Kofi & na & *(Ama & rehwehw\textepsilon)}
{Kofi & \textsc{foc} & Ama & is.looking.for}
\item[A2:]\shortex{4}
{Kofi & a & (*Ama & rehwehw\textepsilon)}
{Kofi & \textsc{foc}}
{`(It is) \textsc{Kofi} (that Ama is looking for)' [aka] \citep[5]{drubig:03}}}}



Sometimes, multiple lexical markers can be used
simultaneously:\is{lexical markers} \citet{schneider:09} argues that
\ili{Abma} has four markers expressing information structure:
 \textit{ba} as a comment marker, and \textit{tei} as a focus
marker.  \textit{Ba} and \textit{tei} can appear together before the
predicate to designate comment plus \isi{focus} (i.e.\ predicate focus), but
the latter should be immediately preceded by the former as presented
in \myref{exe:app} below.


\myexe{\enumsentence{\toplabel{exe:app}
\shortex{6}
{... & ba & tei & te & ba=i=te & Liwusvet=nga.}
{& \textsc{comm} & \textsc{foc} & 3\textsc{sg}.\textsc{pfv} & \textsc{neg}.1=be=\textsc{part} & Liwusvet=\textsc{neg}.2}
{`... but it wasn't Liwusvet.' [app] \citep[5]{schneider:09}}}}




\subsection{Positioning constraints}
\label{4:ssec:position}


Lexical markers can occur before or after a phrase that is assigned an
information structure role by the markers.\is{lexical markers} For
instance, in \ili{Rendile}, \textit{{\'e}} in (\ref{exe:rel}a) is a
suffix, and \textit{{\'a}} in (\ref{exe:rel}b) is a prefix.
(\ref{exe:buli}) is an example in \ili{Buli}, in which the \isi{focus}
marker \textit{k\`{a}} precedes the focused constituent. In
contrast, the focus marker \textit{ny\={a}} in \ili{Ditammari} is
preceded by the focused constituent, as shown in
(\ref{exe:ditammari}). \footnote{Both languages belong to the language
  family of Niger-Congo/Gur.}


\myexe{\eenumsentence{\toplabel{exe:buli}
\item[Q:] What did the woman eat?
\item[A:]\shortex{4}
{\`{o} & \textipa{N}\`{o}b & k\`{a} & t\'{u}\'{e}.}
{3\textsc{sg} & eat & \textsc{fm} & beans}
{`She ate \textsc{beans}.' [bwu] \citep[133]{fery:krifka:08}}}}

\myexe{\eenumsentence{\toplabel{exe:ditammari}
\item[Q:] What did the woman eat?
\item[A:]\shortex{4}
{\`{o} & d\={\i} &  y\={a}t\~{u}r\`{a} & ny\={a}.}
{3\textsc{sg}& eat & beans & \textsc{fm}}
{`She ate \textsc{beans}.' [tbz] \citep[133]{fery:krifka:08}}}}




\subsection{Categorical restriction}
\label{4:ssec:categorical}


There is a categorical restriction on the phrases with which lexical
markers can be combined. Phrases can be nominal, verbal, and even
adverbial; for instance, adverbial categories in \ili{Korean} and
\ili{Japanese} can be \wa and \onun-marked.  Choice of lexical markers
can also be dependent on category; in \ili{Rendile} as shown in
\myref{exe:rel}, an affix \textit{{\'e}} is attached to only nouns
such as \textit{{\'i}nam} `boy', while a prefix \textit{{\'a}} is
exclusively used with verbs such as \textit{yimi} `came'.\is{lexical
  markers} That means, each lexical marker has a constraint on which
category it can be used for, which also needs to be represented as
lexical information.


\subsection{Interaction with syntax}
\label{4:ssec:interaction-syn}



In some languages that employ lexical markers for expressing
information structure,\is{lexical markers} lexical markers interact
with syntactic operations.  One well known case of this interplay
between lexical markers and \isi{syntactic positioning} is
\isi{scrambling} constructions in \ili{Korean} and \ili{Japanese}
\citep{choi:99,ishihara:01}. Similarly, in \ili{Akan}, focused items
obligatorily (i) occupy sentence-initial position and (ii) immediately
precede \isi{focus} markers such as \textit{na} and \textit{a} as already
illustrated in \myref{exe:aka} \citep[4]{drubig:03}.  A comparable
phenomenon can be found in the \ili{Buli} example \myref{exe:buli}:
According to \citet{fery:krifka:08}, if a focused constituent is
sentence-initial, the focus marker \textit{k\`{a}} can be
used. \ili{Cherokee}, as demonstrated in \myref{exe:chr}, employs the
clitic \textit{tvv} to signal focus, and the focused constituent with
\textit{tvv} should be followed by any other constituents in the
sentence (i.e.\ it should be clause-initial).\is{clause-initial}



\section{Syntactic positioning}
\label{4:sec:syntactic}



Information structure roles are often associated with specific
positions in a clause. It is well-documented that the realization of
information structure has much to do with word order, and this
relationship can be cross-linguistically 
captured \citep{zubizarreta:98,van:05,mereu:09}.  For example, although word
order in \ili{Spanish} is relatively free in comparison with
\ili{English}, there are still ordering constraints in Spanish that
hinge on information structure \citep{zagona:02}.  Moreover, according 
to \citet{li:thompson:76}, every language has one or more syntactic
device(s) for expressing information structure.



Before discussing specific syntactic positions, it is necessary to
look into how information is structured in the \isi{basic word order}
in a language.\is{syntactic positioning} Languages have different
unmarked \isi{focus} positions, depending largely, but not entirely, on
their neutral word order. For example, in \ili{English}, narrow focus on the
object is a case of unmarked narrow focus, while narrow focus on the
subject is a case of marked narrow focus.\is{narrow focus}  An ordinary example of a
narrow focus can be found in Q/A pairs in which the object plays the
role of focus as provided in \myref{exe:narrow:focus:van}.


\myexe{\enumsentence{\label{exe:narrow:focus:van}
\begin{tabular}[t]{ll}
Q: & {What did Kim read?}\\ 
A: & {Kim read the \textsc{book}.}\\
\end{tabular}}}


\noindent \citet{van:05} captures a generalization about the
relationship between word order type and the most unmarked position of
narrow \isi{focus}: In SVO languages, it is the last position in the
core clause (e.g.\ \ili{English}) or the immediate \isi{postverbal}
position (e.g.\ \ili{Chiche{\^w}a}). In verb-final languages, the
unmarked focus position is the immediate \isi{preverbal} position
(e.g.\ \ili{Korean} and \ili{Japanese}). In VOS languages, it is the
immediate postverbal position (e.g.\ \ili{Toba Batak}).



The present study does not place an information structure constraint
on sentences in the unmarked word order for two reasons.


First, the clause-initial items in subject-first or \isi{V2 languages} are
ambiguous when it comes to \isi{focus}/\isi{topic}
\isi{fronting}.\is{clause-initial} For instance, note
\myref{exe:ydd:topicalization} in \ili{Yiddish}.  Given that
declarative clauses in Yiddish are both SVO and V2 \citep{jacobs:05},
the constituent that occurs in the sentence-initial position is the
subject in the default word order.  What is to be considered at the
same time is that focus/topic fronting is productively used in Yiddish
as exemplified below \citep{jacobs:05}.


\myexe{\eenumsentence{\label{exe:ydd:topicalization}
\item\shortexnt{5} 
{Der ler{\textschwa}r & \v{s}rajbt & di zacn & mit krajd & afn tovl.}
{`The teacher & writes & the sentences & with chalk & on the blackboard.' (neutral)}
\item\shortex{5} 
{Di zacn  & \v{s}rajbt & der ler{\textschwa}r & mit krajd & afn tovl.}
{the sentences & writes & the teacher & with chalk & on the blackboard}
{`It's the sentence (not mathematical equations) that the teacher is writing with chalk on the blackboard.'}
\item\shortex{5} 
{mit krajd &  \v{s}rajbt & der ler{\textschwa}r & di zacn & afn tovl.}
{with chalk & writes & the teacher & the sentences & on the blackboard}
{`It's with chalk (not with a crayon) that that the teacher is writing the sentence on the blackboard.'}
\item\shortex{5} 
{afn tovl & \v{s}rajbt & der ler{\textschwa}r & di zacn &  mit krajd.}
{on the blackboard & writes & the teacher & the sentences & with chalk}
{`It's on the blackboard (not the notepad) that that the teacher is writing the sentence with chalk.' [ydd] \citep[224]{jacobs:05}}}}


\noindent Thus, without reference to the context, we cannot clearly
say which information structure meaning the subject carries when the
sentence is in V2 order. That is, the subject \textit{Der
  ler{\textschwa}r} in (\ref{exe:ydd:topicalization}a) may or may not
be associated with \isi{focus}.  Another example can be found in
\ili{Breton} (a V2 language).\is{V2 languages} In the Q/A pair, what
is focused in (\ref{exe:bre:focus}A) is the fronted item
\textit{Mar\'i} (the rheme and the new information in
\citeauthor{press:86}'s terminology). In this case, the word order of the
sentence is SVO.

\myexe{\eenumsentence{\label{exe:bre:focus}
\item[Q:]\shortex{3}
{P\'iv & a wel & Yann\'ig?}
{who & sees & Yannig}
{`Who sees Yannig?'}
\item[A:]\shortex{3}
{Mar\'i & a wel & Yann\'ig}
{Marie & sees & Yannig}
{`Maries sees Yannig.' [bre]  \citep[194]{press:86}}}}

\noindent However, the sentence \textit{Mar\'i a wel Yann\'ig} itself,
if it were not for the contextual information, sounds
ambiguous. \citeauthor{press:86} argues that in the sentence
\textit{Yann\'ig} could well be the subject of the sentence (i.e.\ in
an OVS order). If \textit{Yann\'ig} is the subject, \isi{focus} is
assigned to the fronted object \textit{Mar\'i}. In other words, a
Breton sentence \textit{Mar\'i a wel Yann\'ig} conveys two potential
meanings like either \textit{It is Marie who sees Yannig.} (when the
sentence is SVO) or \textit{It is Marie who Yannig sees.} (when the
sentence is OVS).  Note that (\ref{exe:bre:focus}A) in which the focus
is associated with the subject is ambiguous because \ili{Breton} is a
V2 language,\is{V2 languages} and therefore the subject, in itself,
can be interpreted as either as focused or just unknown. In the
analysis I propose later the information structure value of the
constituents in situ (e.g.\ the subjects in
\ref{exe:ydd:topicalization} and \ref{exe:bre:focus}A) is left
underspecified.

\largerpage[2]
Second, unmarked \isi{focus} positions in different languages also
deeply interact with phonological variation.\footnote{This has to do
  with the so-called p-movement \citep{zubizarreta:98}, which
  indicates an indirect interface between information structure and
  syntax. Given that nuclear-stress position is relatively fixed (in
  some languages at least; cf. non-plastic accent,
  \citealt{vallduvi:90}) and focus should be maximally prominent
  \citep{buring:10}, the focused item needs to be in the right
  (i.e.\ stressed) position.}  \citet{ishihara:01} argues that two
types of stresses have an effect on the unmarked position; one is
N-stress (Nuclear stress), and the other is A-stress (Additional
stress). According to \citeauthor{ishihara:01}, A-stress is not
required, while every sentence presumably bears N-stress, and the
position of the N-stress is rather fixed in a
language.\footnote{\citet{ishihara:01} offers this argument based on a
  lot of previous phonological studies, but not seeing a large number
  of languages (e.g.\ \ili{Japanese}, \ili{Korean}, \ili{Basque},
  etc.). Thus, we may not say that these rules are meant to be
  universals. Nonetheless, \citeauthor{ishihara:01}'s argument still
  has a significance in that it is well discussed how different types
  of sentential stresses impact forming information structure of
  sentences in a default word order.}  Thus, N-stress is realized in
the same position almost invariably even if constituents shift their
order (e.g.\ through inversion, \isi{scrambling}, etc.). For example,
the following sentences in Japanese \myref{exe:n-stress:jpn} and
Ondarroa \ili{Basque} \myref{exe:n-stress:basque}, in which
\'{\xspace} and \^{\xspace} stand for the N-stress in each language,
show that the position of N-stress (\isi{preverbal} in both languages)
does not shift to reflect the change in word order.\il{Japanese}

\myexe{\eenumsentence{\toplabel{exe:n-stress:jpn}
\item\shortex{4}
{Taro-ga & kyoo & h\'on-o & katta}
 {Taro-\textsc{nom} & today & book-\textsc{acc} & bought}
{`Taro bought a book today.'}
\item\shortexnt{4}
{Taro-ga & hon-o & ky{\'o}o & katta}
{Taro-\textsc{nom} & book-\textsc{acc} & today & bought [jpn] \citep[145]{ishihara:01}}}}


\myexe{\eenumsentence{\toplabel{exe:n-stress:basque}
\item\shortex{4}
{Jonek & M\^{i}ren & ikus\'{i} & ban.}
{John.\textsc{erg} & Miren & see.\textsc{tu} & \textsc{aux}.\textsc{pst}}
{`Jon saw \textsc{Miren}.'}
\item\shortex{4}
{Miren & J\^{o}nek & ikus\'{i} & ban.}
{Miren.\textsc{erg} & John & see.\textsc{tu} & \textsc{aux}.\textsc{pst}}
{`\textsc{Jon} saw Miren.' [eus] \citep[22]{arregi:00}}}}



\noindent N-stress has a tendency to fall on the \isi{preverbal}
position in OV languages as shown in \textit{h\'on-o} and
\textit{ky{\'o}o} \myref{exe:n-stress:jpn} and \textit{M\^{i}ren} and
\textit{J\^{o}nek} in \myref{exe:n-stress:basque}, while it tends to
fall on the \isi{postverbal} position in VO languages
(e.g.\ \ili{English}). By contrast, since A-stress lays an additional
emphasis on a specific word, its position can vary depending on what
the speaker wants to emphasize (i.e.\ \isi{focus}). With respect to the
presence of A-stress, \citeauthor{ishihara:01} proposed a rule: Any
material that follows an A-stress must be deaccented.



Combining the three factors presented thus far, (i) basic word order,
(ii) N and A-stresses, and (iii) the unmarked position for narrow
focus,\is{narrow focus} we can explain the reason why an object normally bears the
\isi{focus} of a sentence in an unmarked way at least in the languages
presented so far.  A-stress, as mentioned, does not show up unless it
is necessary for the speaker to emphasize something.  In the absence
of an A-stress, the word with N-stress is the most stressed
constituent in the sentence.  N-stress in a sentence has a strong
tendency to fall on the object in both OV and VO languages. In
addition, subjects have a strong tendency to be topics. Most languages
have a spot in the syntactic structure which is the unmarked position
for topics, and subjects tend to fall in that part of the syntactic
structure \citep{lambrecht:96}. Hence, the unmarked marking of focus
tends to fall on objects.


The present study does not deal with the unmarked positions of \isi{topic}
and \isi{focus}. We cannot identify them without deterministic clues that
reveal their information structure meanings. The different positions
of focus outlined in the next section are those which are not in the
most neutral word order in each language.



\subsection{Focus position}
\label{4:ssec:focus-position}

Some languages assign a specific position to signal \isi{focus}. It is
evident that the position in this case is primarily motivated by the
necessity to mark narrow focus on a single constituent in the
non-neutral word order. For example, if a language employs SVO by
default, and the canonical focus position of the language is
clause-final, then the object in SVO is not considered as necessarily
containing focus.\is{clause-final} This is because sentences in the
default word order allow for all possibilities in information
structure.


According to \citet{fery:krifka:08} and my own survey, there are four
positions that human languages employ to designate narrow focus; (i)
\isi{clause-initial}, (ii) \isi{clause-final}, (iii) \isi{preverbal},
and (iv) \isi{postverbal}. In the following subsections, each position
is exemplified and the languages that use the strategy are enumerated.


\subsubsection{Clause-initial position}
\label{4:sssec:clause-initial}


Narrow focus can be assigned to the clause-initial position in some
languages,\is{narrow focus} including \ili{English} (e.g.\ focus/topic \isi{fronting}
constructions), \ili{Ingush} \citep{nichols:11},
\ili{Akan} \citep{drubig:03}, \ili{Breton} \citep{press:86},
\ili{Yiddish} \citep{jacobs:05}, and \ili{Hausa}
\citep{hartmann:zimmermann:07,buring:10}.\is{narrow focus}



The representative example in \myref{exe:inh:focus} is from
\ili{Ingush} (a Northeast Caucasian language, spoken in Ingushetia and
Chechnya).  Ingush is a head-final language except for predominantly
V2 order in main clauses \citep{nichols:11}.\is{V2 languages} In \myref{exe:inh:focus},
the first element in each sentence is associated with focus.

\myexe{\eenumsentence{\label{exe:inh:focus}
\item\shortex{4}
  {Cuo & diicar & suona & jerazh.} 
  {3s.\textsc{erg} & D.tell.\textsc{wp} & 1s.\textsc{dat} & these} 
  {`\textit{She} told me them (=stories).' (focus on \textit{she})} 
\item\shortex{4}
  {Suona & diicar & cuo & yzh.} 
  {1s.\textsc{dat} & D.tell.\textsc{wp} & 3s.\textsc{erg} & 3p} 
  {`She told \textit{me} them (=stories).' (focus on \textit{me}) [inh] \citep[687]{nichols:11}}}}


\ili{Hausa} is also known to use the clause-initial position for
marking \isi{focus} \citep{buring:10}. As is exemplified in the Q/A pair
presented in (\ref{exe:hau:ch4}Q-A1) and (\ref{exe:hau:ch4}Q-A12, the
focused constituent in Hausa (replying to the \textit{wh}-question)
can appear first or can be realized in situ.  That is to say,
there are two types of foci in Hausa, namely ex situ focus
(\ref{exe:hau:ch4}A1) and in situ focus (\ref{exe:hau:ch4}A2)
\citep{hartmann:zimmermann:07}.

\myexe{\eenumsentence{\label{exe:hau:ch4}
\item[Q:]\shortex{3}
{M{\`e}e & suk{\`a} & kaam{\`a}a?}
{what & \textsc{3pl.rel.perf} & catch}
{`What did they catch?'}
\item[A1:]\shortex{4}
{\textbf{Kiifii} & (n{\`e}e) & suk{\`a} & kaam{\`a}a.}
{fish & \textsc{prt} & \textsc{3pl.rel.perf} & catch}
{`They caught \textsc{fish}.'}
\item[A2:]\shortex{3}
{Sun & kaam{\`a}a & \textbf{kiifii}.}
{\textsc{3pl.abs.perf} & catch & fish}
{`They caught \textsc{fish}.' [hau] \citep[242--243]{hartmann:zimmermann:07}}}}



\noindent There are two types of languages with respect to focus
position. One obligatorily places focused elements in a specific
position, and the other optionally does. Hausa is of the latter type.
\ili{Ingush} and \ili{English} belong to former.\is{clause-initial}  


Even if a language does not always assign \isi{focus} to the clause-initial
position, it can sometimes make use of clause-initial focus, which is
called focus/topic \isi{fronting} in the current analysis.\footnote{As
  mentioned several times, this kind of syntactic operation is often
  called topicalization \citep{prince:84,man:07}.}  Old information is
sometimes focus-marked as in \myref{exe:gussenhoven:reactivating}
where the replier wants to say that \textit{she} does not merely know
\textit{John}, but dislikes him.\footnote{Another example is already
  given in \myref{exe:fery:krifka:08:sof}, which is called Second
  Occurrence Focus (Section \ref{4:ssec:conditions}).}

\myexe{\enumsentence{\label{exe:gussenhoven:reactivating}
\begin{tabular}[t]{ll}
Q: & {Does she know \textsc{John}?}\\
A: & {\textsc{John} she \textsc{dislikes}. \citep[96]{gussenhoven:07}}\\
\end{tabular}}}


\noindent Hence, an \ili{English} sentence in which the object is not
in situ (e.g.,\ \textit{John she dislikes.}), if we do not
consider the accents, can be read ambiguously (e.g.,\ either
\textit{It is John who she dislikes.} or \textit{As for John, she
  dislikes him.}).  These matters are revisited in the next chapter in
terms of discrepancies between meaning and marking of information
structure.\is{clause-initial}
For the moment, suffice it to say that the clause-initial
position can be employed to narrowly mark the \isi{focus} of the sentence in
many languages including English.


\subsubsection{Clause-final position}
\label{4:sssec:clause-final}

Second, narrow focus can be licensed in clause-final position in some
languages.\is{clause-final} These include \ili{Russian}
\citep{neeleman:titov:09},\footnote{In Russian, \isi{non-contrastive focus}
  (i.e.\ \tdl{semantic-focus} in the taxonomy of the present study)\is{semantic focus}
  shows up sentence-finally, whereas contrastive focus is fronted
  \citep{neeleman:titov:09}.}\is{narrow focus}  \ili{Bosnian Croatian Serbian},
\ili{American Sign Language} \citep{petronio:93,churng:07}, and some
Chadic languages such as \ili{Tangale} and \ili{Ngizim}
\citep{drubig:03}.\is{narrow focus} For example, in Russian, if (i) a constituent
corresponds to the \textit{wh}-word in a given question, and thereby
is narrowly focused and (ii) the accent does not designate the \isi{focus},\is{clause-final}
it can occupy the clause-final position as presented
below.\footnote{The second answer in \myref{exe:neeleman:titov:focus}
  is in the most unmarked word order in Russian.}


\myexe{\eenumsentence{\toplabel{exe:neeleman:titov:focus}
\item[Q:]\shortex{4}
  {Kto & dal & Kate & knigu?}
  {who & gave & Kate.\textsc{dat} & book.\textsc{acc}}
  {`Who gave a book to Kate?'}
\item[A:]\shortex{4}
  {Kate & knigu & dala & \textsc{anja}.}
  {Kate.\textsc{dat} & book.\textsc{acc} & gave & Anna}
  {`\textsc{Anna} gave a book to Kate.' (focus on the subject)}
\item[Q:]\shortex{4}
  {\v{C}to & Anja & dala & Kate?}
  {what.\textsc{acc} & Anna & gave & Kate.\textsc{dat}}
  {`What did Anna give to Kate?'}
\item[A:]\shortex{4}
  {Anja & dala & Kate & \textsc{knigu}.}
  {Anna & gave & Kate.\textsc{dat} & book.\textsc{acc}}
  {`Anna gave a \textsc{book} to Kate.' (focus on the direct object)}
\item[Q:]\shortex{4}
  {Komu & Anja & dala & knigu?}
  {who.\textsc{dat} & Anna & gave & book.\textsc{acc}}
  {`Who did Anna give a book to?'}
\item[A:]\shortex{4}
  {Anja & dala & knigu & \textsc{Kate}.}
  {Anna & gave & book.\textsc{acc} & Kate.\textsc{dat}}
  {`Anna gave a book to \textsc{Kate}.' (focus on the indirect object) [rus] \citep[515]{neeleman:titov:09}}}}


\noindent \ili{Russian}, in which the most unmarked word order is SVO,
is known for its \isi{free word order} of constituents.  However,
\citet{rodionova:01}, exploring variability of word order in Russian
declarative sentences, concludes that the word order in Russian is
influenced by different types of \isi{focus}, namely narrow, predicate, and
sentential focus.


The same phenomenon holds in \ili{Bosnian Croatian Serbian} as
exemplified in \myref{exe:hbs:focus}; (\ref{exe:hbs:focus}a)
represents an unmarked word order in the language (SVO), but the
subject in (\ref{exe:hbs:focus}b) \textit{Slavko} is postposed to mark
\isi{focus} meaning overtly through syntax.



\myexe{\eenumsentence{\toplabel{exe:hbs:focus}
\item \shortex{3}
  {Slavk-o & vid-i & Olg-u}
  {Slavko.\textsc{m}-\textsc{sg}.\textsc{nom} & see-3\textsc{sg} & Olg-3.\textsc{f}.\textsc{sg}.\textsc{acc}}
  {`Slavko sees \textsc{Olga}' (the unmarked word order)}
\item \shortex{3}
  {Olg-u & vid-i & Slavk-o}
  {Olga.\textsc{f}-\textsc{sg}.\textsc{acc} & see-3\textsc{sg} & Slavko.\textsc{m}-\textsc{sg}.\textsc{nom}}
  {`\textsc{Slavko} sees Olga.' (focus on the subject) [hbs]}}}




\subsubsection{Preverbal position}
\label{4:sssec:preverbal}

Third, the (immediately) \isi{preverbal} position is another site that
signals \isi{focus}. Languages that assign \isi{narrow focus} to the preverbal
position include \ili{Basque} \citep{urbina:99}, \ili{Hungarian}
\citep{kiss:98,szendroi:01}, \ili{Turkish} \citep{icssever:03}, and
\ili{Armenian}
\citep{comrie:84,tamrazian:91,tamrazian:94,tragut:09,megerdoomian:11}.
Basque, for instance, is a language in which focus marking heavily
depends on sentence positioning. This is similar to the situation in
\ili{Catalan} \citep{vallduvi:92,engdahl:vallduvi:96} and \ili{Yucatec
  Maya} \citep{kugler:etal:07}.\is{narrow focus} The syntactic device for marking
narrow focus in Basque is to assign focus immediately to the left of
the verb as exemplified in \myref{exe:urbina:focus}. While
(\ref{exe:urbina:focus}a) conveys neutral information structure
(i.e.,\ all constituents are underspecified from the view of the
present study.), in (\ref{exe:urbina:focus}b--c), the subject
\textit{Jonek} `Jon', being adjacent to the verb \textit{irakurri}
`read', should be read as conveying focus meaning.


\myexe{\eenumsentence{\toplabel{exe:urbina:focus}
\item\shortex{4}
{Jonek & eskutitza & irakurri & du}
{Jon & letter & read & has}
{`Jon has read the letter.' (SOV)}
\item\shortex{4}
{Jonek & irakurri & du & eskutitza}
{Jon & read & has & letter}
{`\textsc{Jon} has read the letter.' (SVO)}
\item\shortex{4}
{Eskutitza, & Jonek & irakurri & du}
{letter & Jon & read & has}
{`\textsc{Jon} has read the letter.' (OSV) [eus] \citep[312]{urbina:99}}}}


\noindent \citet{crowgey:bender:11} also employ the \textit{wh}-test
for identifying \isi{focus} in \ili{Basque}: Both (\ref{exe:eus:joshua}b--c)
are grammatical sentences in Basque, but (\ref{exe:eus:joshua}c)
cannot be used as an answer to
(\ref{exe:eus:joshua}a).\is{felicity-conditions} This distinction in
felicity-conditions shows that focused constituents should appear in
the immediately preverbal position.

\myexe{\eenumsentence{\toplabel{exe:eus:joshua}
\item\shortex{5}
{Liburu & bat & nork & irakurri & du?}
{book & one.\textsc{abs.sg} & who.\textsc{erg.sg.foc} & read.\textsc{perf} & \textsc{3sgO.pres.3sgA}}
{`Who has read one book?'}
\item\shortex{5}
{Liburu & bat & Mirenek & irakurri & du.}
{book & one.\textsc{abs.sg} & Mary.\textsc{erg.sg.foc} & read.\textsc{perf} & \textsc{3sgO.pres.3sgA}}
{`Mary has read one book.'}
\item\shortex{5}
{Mirenek & liburu & bat & irakurri & du.}
{Mary.\textsc{erg.sg.foc} & book & one.\textsc{abs.sg} & read.\textsc{perf} & \textsc{3sgO.pres.3sgA}}
{`Mary has read one book.' [eus] \citep[48--49]{crowgey:bender:11}}}}


\ili{Hungarian} is a well known language as fixed \isi{focus}
position.\footnote{Some counterarguments to this generalization have
  been reported: The so-called focus position in Hungarian has been
  claimed to encode exhaustiveness rather than identificational focus
  \citep{horvath:07,fanselow:08}.} The constituent order in Hungarian
can be schematized as `(Topic*) Focus V S O' \citep{buring:10}, as
exemplified in \myref{exe:szendroi}.

\myexe{\eenumsentence{\label{exe:szendroi}
\item\shortexnt{4}
{Mari & fel & h{\'i}vta & P{\'e}tert.}
{Mary-\textsc{nom} & \textsc{vm} & rang & Peter-\textsc{acc}}
\item\shortexnt{4}
{\textsc{Mari}\mysub{F} & h{\'i}vta & fel & P{\'e}tert.}
{Mary-\textsc{nom} & rang & \textsc{vm} & Peter-\textsc{acc}}
\item\shortex{4}
{*\textsc{Mari} & fel & h{\'i}vta & P{\'e}tert.}
{Mary-\textsc{nom} & \textsc{vm} & rang & Peter-\textsc{acc}}
{`Mary rang up Peter' [hun] \citep[549]{szendroi:99}}}}

\noindent (\ref{exe:szendroi}a) is encoded as the \isi{basic word
  order}, in which a marker \textit{fel} occurs between the subject
\textit{Mari} `Mary' and the main verb \textit{h{\'i}vta} `rang'. If
\textit{Mari} is focused, the verb \textit{h{\'i}vta} should
immediately follow the focused item as given in (\ref{exe:szendroi}b),
and if not as shown in (\ref{exe:szendroi}c), it sounds bad
as. \citet{kiss:98} states that \isi{focus} in \ili{Hungarian} can appear either
in situ or immediately preverbally.\footnote{That indicates
  informational focus and identificational focus,
  respectively. According to \citet{kiss:98}, the preverbal focus in
  Hungarian (i.e.\ identificational focus) is almost the same as cleft
  constructions in \ili{English}.\is{clefting}} \citet{szendroi:01} argues
that focus in Hungarian tends not to be in situ, and that
\isi{preverbal} positioning has to be phonologically licensed (marked
with small caps above).\is{preverbal}


According to \citet{tamrazian:91}, \ili{Armenian} also places focused
constituents in the immediately \isi{preverbal} position: Both sentences
(\ref{exe:hye}a--b) sound natural in Armenian, but the first one is in
the \isi{basic word order} without a focused element. In contrast, the
preverbal item \textit{\textsc{surkin}} in (\ref{exe:hye}b) is
focused, which is signaled by the adjacent auxiliary \textit{e}.\is{auxiliary} The
auxiliary \textit{e} should immediately follow the focused item. For
instance, (\ref{exe:hye}c) in which an accent falls on
\textit{\textsc{surkin}} but \textit{e} appears after the main verb
\textit{sirum} `like' is ill-formed.


\myexe{\eenumsentence{\label{exe:hye}
\item\shortex{4}
{siran{\textschwa} & surikin & sirum & e}
{Siran(\textsc{nom}) & Surik(\textsc{acc}) & like & is}
{`Siran likes Surik'}
\item\shortex{4}
{siran{\textschwa} & \textsc{surikin} & e & sirum}
{Siran(\textsc{nom}) & Surik(\textsc{acc}) & is & like}
{`Siran likes \textsc{Surik}'}
\item\shortexnt{5} {*siran{\textschwa} & \textsc{surikin} & sirum & e & }
  {Siran(\textsc{nom}) & Surik(\textsc{acc}) & like & is & [hey]
    \citep[103]{tamrazian:91}}}}


\subsubsection{Postverbal position}
\label{4:sssec:postverbal}


Finally, the (immediate) \isi{postverbal} position is responsible for
marking \isi{narrow focus} in several languages. These include
\ili{Portuguese} \citep{ambar:99}, \ili{Toba Batak}, and
\ili{Chiche{\^w}a} \citep{van:05}.  For example, \citeauthor{ambar:99}
claims that \isi{non-contrastive focus} is preceded by the verb in
Portuguese. An example is presented below, in which the focused item
\textit{a Joana} (functioning as the subject) follows the verb
\textit{comeu} `ate'. If the subject with focus meaning precedes the
verb, the sentence sounds infelicitous in the context, as shown in
(\ref{exe:ambar:focus}A3-A4).

\myexe{\eenumsentence{\toplabel{exe:ambar:focus}
\item[Q:]\shortex{4}
{Quem & comeu & a & tarte?}
{who & ate & the & pie}
{`Who ate the pie?'}
\item[A1:]\shortexnt{3}
{Comeu & a & Joana.}
{ate & the & Joana}
\item[A2:]{A tarte comeu a Joana.}
\item[A3:]{\#A Joana comeu.}
\item[A4:]{\#A Joana comeu a tarte. [por] \citep[27]{ambar:99}}}}





\subsection{Topic position}
\label{4:ssec:topic-position}

Topic is also associated with a specific position in some languages.
For example, according to \citet{ambar:99}, topics in \ili{Portuguese}
cannot follow the verb as shown in \myref{exe:ambar:topic}.

\myexe{\eenumsentence{\label{exe:ambar:topic}
\item[Q:]{\shortex{4}
  {Que & comeu & a & Maria?}
  {what & ate & the & Mary}
  {`What did Mary eat?'}}
\item[A1:]{Comeu a tarte.}
\item[A2:]{A Maria comeu a tarte.}
\item[A3:]{\#A tarte comeu a Maria.}
\item[A4:]{\#Comeu a Maria a tarte. [por] \citep[28]{ambar:99}}}}


\noindent In \myref{exe:ambar:topic}, \textit{Maria} `Mary' plays the
\isi{topic} role in the answers. The word should either disappear (as shown
in \ref{exe:ambar:topic}A1) or precede the verb \textit{comeu} `ate'
(as presented in \ref{exe:ambar:topic}A2). The sentences in which
the topic is preceded by the verb sound infelicitous (as provided in
\ref{exe:ambar:topic}A3-A4).


\subsubsection{Topic-first restriction}
\label{4:sssec:topic-first}

Previous studies have assumed the canonical position of \isi{topic} to be
sentence-initial. In fact, quite a few languages have been reported as
having a strong tendency towards topic-fronting.  \citet{nagaya:07}
claims that topics in Tagalog canonically appear sentence-initially,
\citet{chapman:81} says topics in Paumar{\'{\i}} appear
sentence-initially, and \citet{casielles:03} states that topics should
be followed by \isi{focus} (i.e.\ \tdl{topic-focus}) in the canonical word
order in \ili{Spanish}. In \ili{Bosnian Croatian Serbian}, if a
constituent as in \textit{Olg-u} is given in the previous sentence as
the focus as shown in (\ref{exe:hbs:topic}a), it appears
sentence-initially in the following sentence such as
(\ref{exe:hbs:topic}b) when functioning as the topic.\is{clause-final}
Since focused constituents in that language appear in the clause-final
position (as mentioned in Section \ref{4:sssec:clause-final}), \textit{mi}
`we' in (\ref{exe:hbs:topic}b) is associated with focus (marked in
small caps in the translation).\footnote{In (\ref{exe:hbs:topic}b),
  \textit{i} `as well' enforces the focus effect on \textit{mi} `we'
  in the final position. That means \textit{i} in the sentence behaves
  as a focus particle, similarly to `also' in \ili{English}.} That is,
in \ili{Bosnian Croatian Serbian}, topics appear first, and foci occur
finally.


\myexe{\eenumsentence{\toplabel{exe:hbs:topic}
\item\shortex{3}
  {Slavk-o & vid-i & Olg-u}
  {Slavko.\textsc{m}-\textsc{sg}.\textsc{nom} & see-3.\textsc{sg} & Olg-3.\textsc{f}.\textsc{sg}.\textsc{acc}}
  {`Slavko sees \textsc{Olga}'}
\item\shortex{4}
  {Olg-u & vid-imo & i & mi}
  {Olg.\textsc{f}-3 & \textsc{sg}.\textsc{acc} & as well & 1.\textsc{pl}.\textsc{nom}}
  {`\textsc{We} see Olga, too' [hbs]}}}




In some languages including \ili{Japanese} and \ili{Korean}, it is the
case that (non-contrastive) topics are required to be sentence-initial
\citep{maki:etal:99,vermeulen:09}.\is{non-contrastive topic}
\citeauthor{maki:etal:99} argue
that a \wa-marked phrase can be interpreted as a \isi{topic} if and only if
it turns up in initial position. Otherwise, the \wa-marked phrase in a
clause-internal position should be evaluated as conveying a
contrastive meaning.\is{contrast}

\myexe{\eenumsentence{\toplabel{exe:maki:7-8}
\item\shortex{4}
{John-wa & kono & hon-o & yonda.}
{John-\textsc{wa} & this & book-\textsc{acc} & read}{`As for John, he read this book.'}
\item\shortex{4}
{Kono & hon-wa & John-ga & yonda.}
{this & book-\textsc{wa} & John-\textsc{nom} & read}
{`As for this book, John read it.'}
\item\shortex{4}
{John-ga & kono & hon-wa & yonda.}
{John-\textsc{nom} & this & book-\textsc{wa} & read}
{`John read this book, as opposed to some other book.' \\
`*As for this book, he read this it.' [jpn] \citep[7--8]{maki:etal:99}}}}

\noindent The same goes for \ili{Korean} in my intuition.
\citet{fery:krifka:08} provide a \textit{prima facie} counterexample
to this claim as shown in \myref{exe:fery:krifka:kor}, in which
\textit{disethu} `dessert' is combined with \nun.

\myexe{\enumsentence{\toplabel{exe:fery:krifka:kor}
\shortex{4}
  {nwukwuna-ka & disethu-nun & aiswu khwulim-ul & mek-ess-ta.}
  {everyone-\textsc{nom} & dessert-\textsc{nun} & ice.cream-\textsc{acc} & eat-\textsc{pst}-\textsc{decl}}
  {`As for dessert, everyone ate ice cream.' [kor] \citep[130]{fery:krifka:08}}}}



\noindent However, \nun is not always compatible with the information
structure meaning of \isi{topic}. That is, there is a mismatch between form
and meaning. The \onun-marked \textit{disethu} in
\myref{exe:fery:krifka:kor}, in my intuition, fills the role of
\isi{contrastive topic}, rather than aboutness topic.\is{aboutness topic} Contrastive topics
cross-linguistically have no constraint on position in word order
\citep{erteschik:07,roberts:11}.\is{contrastive topic}
In conclusion, aboutness topics in \ili{Korean} and \ili{Japanese} should be sentence-initial.




Other studies, however, indicate that topics are not necessarily
sentence-initial \citep{erteschik:07,fery:krifka:08}.  According to
\citeauthor{erteschik:07}'s analysis, topic \isi{fronting} is optional
in \ili{Danish}, and topics can be marked either in an overt way
(i.e.\ \isi{topicalization} in or
in situ as shown in \ref{exe:erteschik:topic}a--b).


\myexe{\eenumsentence{\toplabel{exe:erteschik:topic}
\item\shortexnt{9}
  {Hun & hilstep\r{a} & Ole. & \myemp{Ham} & havde & hun & ikke & m{\o}dt & f{\o}r...}
  {She & greeted & Ole. & \myemp{Him} & had & she & not & met & before}
\item\shortex{9}
  {Hun & hilstep\r{a} & Ole. & Hun & havde & ikke & m{\o}dt & \myemp{ham} & f{\o}r... }
  {She & greeted & Ole. & She & had & not & met & \myemp{him} & before}
  {\mbox{ } \mbox{ } \mbox{ } \mbox{ } \mbox{ } \mbox{ } \mbox{ } \mbox{ } \mbox{ } \mbox{ } \mbox{ }
   \mbox{ } \mbox{ } \mbox{ } \mbox{ } \mbox{ } \mbox{ } \mbox{ } \mbox{ } \mbox{ } \mbox{ } \mbox{ }
   \mbox{ } \mbox{ } \mbox{ } \mbox{ } \mbox{ }
   \mbox{ } \mbox{ } \mbox{ } \mbox{ }  [dan] \citep[7]{erteschik:07}}}}


\noindent \citeauthor{erteschik:07} asserts that so-called
topicalization in Danish, which dislocates the constituent playing the
\isi{topic} role to the left \isi{periphery}, is used only for
expressing the topic in an overt way.  In other words, topics in
Danish are not necessarily sentence-initial.




Building on the analyses presented so far, the present study argues
that the canonical position of aboutness topics is language-specific:\is{aboutness topic}
In some languages such as \ili{Japanese} and \ili{Korean} aboutness topics must
appear in the initial position, while in other languages such as
\ili{Danish} they do not.






























\subsubsection{Right dislocation}
\label{4:sssec:right-dislocation}

It is necessary to take one more non-canonical \isi{topic} position into
account. Topics can also appear sentence-finally. This phenomenon is
called \isi{right dislocation} \citep{cecchetto:99,law:03}, sentence-final
topic \citep{fery:krifka:08}, \isi{anti-topic}
\citep{chafe:76,lambrecht:96}, or \isi{postposing} \citep{kim:11}.


\myexe{\eenumsentence{\label{exe:right-dislocation:eng}
\item{Left dislocation: This book, it has the recipe in it.}
\item{Right dislocation: You should go to see it, that movie. \citep[185--186]{heycock:07}}}}
\is{dislocation} \is{left dislocation} \is{right dislocation}


\noindent \citet{gundel:88} regards this construction as a peculiar
construction within the comment-topic structure contrasting it to the
ordinary \isi{topic-comment} structure.  There must be an intonational break
(i.e.\ a prosodic phrase marked as \mysub{p}) which separates the
topic from the prior parts of the given sentence. Such constructions
exist cross-linguistically as exemplified in
(\ref{exe:right-dislocation:kor})\footnote{The suffix -\textit{n} in
  \myref{exe:right-dislocation:kor} is an allomorph of \nun, which
  mostly shows up in spoken data.} in \ili{Korean} and
(\ref{exe:anti-topic})\footnote{\citet{fery:krifka:08} state a
  boundary tone that is created by the lexical markers responsible for
  information structure meanings (e.g.\ \textit{ge} in \ili{Cantonese}
  as given in \ref{exe:anti-topic}a) allows the topic to be added
  into the final position.} in \ili{Cantonese} and
\ili{French}.\is{anti-topic}


\myexe{\eenumsentence{\label{exe:right-dislocation:kor}
\item\shortex{4}
{kumyen, & kuke-n & com & saki-nte.}
{if.so & that-\textsc{nun} & a.little & fraud-be.\textsc{sem}}
{`If so, that is a kind of fraud, I think.'}
\item\shortex{4}
{kumyen, & com & saki-nte, & kuke-n}
{if.so & a.little & fraud-be.\textsc{sem} & that-\textsc{nun}}
{`If so, that is a kind of fraud, I think.' [kor] \citep[223--224]{kim:11}}}}

\myexe{\eenumsentence{\label{exe:anti-topic}
\item\shortex{9}
{((Go & loupo)\mysub{P} & (nei & gin-gwo & gaa)\mysub{P}, & ([ni & go & namjan & ge]\mysub{T})\mysub{P})\mysub{I}.}
{\textsc{clf} & wife & 2.\textsc{sg} & see-\textsc{exp} & \textsc{dsp} & this & \textsc{clf} & man & \textsc{dsp}}
{`The wife you have seen, of this man.' [yue]}
\item\shortex{6}
{((Pierre & l' & a & mang\'ee)\mysub{P}, & ([la & pomme]\mysub{T})\mysub{P})\mysub{I}.}
{Peter & it-\textsc{acc} & has & eaten, & the & apple}
{`Peter has eaten the apple.' [fra] \citep[130]{fery:krifka:08}}}}


Despite the difference in positioning, \isi{right dislocation} has
much in common with \isi{left dislocation}. At first appearance, right
dislocation looks like a mirror image of left-dislocation, in that the
\isi{topic} is apparently separate from the main clause and it is not likely
that there is a missing function in the preceding sentence. In fact,
\citet{cecchetto:99} proposes the so-called mirror hypothesis, which
implies \isi{right dislocation} is tantamount to a mirror image of
left-dislocation.

The current study hence regards \isi{right dislocation} as a
non-canonical variant of \isi{left dislocation}.  \citet{lambrecht:96}
provides a counterargument to this hypothesis, but the difference
between left/right dislocations in \citeauthor{lambrecht:96}'s
analysis appears to be contextual, rather than the result of a
morphosyntactic operation. As the present study is not directly
concerned with pragmatic constraints, the mirror hypothesis is still
applicable to the current work. The difference between them seems to
be trivially influenced by the degree of speaker's attention to the
conversation: Left dislocation would be used for the purpose of
restricting the frame of what the speaker wants to talk about in
advance, whereas right dislocation is just an afterthought performing
almost the same function. A piece of evidence that supports this
argument is provided by a corpus study which exploits a monolingual
but fully naturally occurring text. \citet{kim:11} scrutinizes several
spoken data in \ili{Korean}, and concludes that right dislocation
(\isi{postposing}, in his terminology) such as
(\ref{exe:right-dislocation:kor}b) is largely conditioned by how
accessible and/or urgent the information is: If the information is not
uttered within several neighboring preceding sentences and is thereby
less accessible in the speaker's consciousness, it tends to be easily
postposed. These findings lend further support for the argument that
the choice between left and right dislocation is determined by only
contextual conditions.



\subsection{Contrast position}
\label{4:ssec:contrast-position}


Contrastive topics have a weaker constraint on order than
non-contrastive topics (i.e.\ aboutness topics)\is{aboutness
  topic}\is{contrast}\is{non-contrastive topic}
\citep{erteschik:07,bianchi:frascarelli:10}. Contrastive topics have a
tendency to precede aboutness topics in some languages
\citep{bianchi:frascarelli:10}, but this generalization has not been
verified in all languages.  With respect to sentence positioning of
\isi{contrastive focus}, there are two types of languages. The first,
in which contrastive focus shares the same position as non-contrastive
focus, is more common. A typical language of this type is
\ili{English}, in which contrastive focus is not distinguishable from
non-contrastive focus in terms of sentence position. The second type
of language selects two distinctive positions from among the ordinary
focus positions given earlier; (i) \isi{clause-initial}, (ii)
\isi{clause-final}, (iii) \isi{preverbal}, and (iv)
\isi{postverbal}. The languages that belong to this type include
\ili{Georgian} (preverbal \vs postverbal,
\citealt{skopeteas:fanselow:10}), \ili{Portuguese} (preverbal \vs
postverbal, \citealt{ambar:99}), \ili{Russian} (clause-initial \vs
clause-final, \citealt{neeleman:titov:09}), \ili{Ingush} (immediately
preverbal \vs clause-initial, \citealt{nichols:11}), and so on.  For
example, \myref{exe:kat:focus} shows preverbal focus and postverbal
focus in \ili{Georgian}.

\myexe{\eenumsentence{\label{exe:kat:focus}
\item\shortexnt{3}
{kal-i & kotan-s & u-q'ur-eb-s.}
{woman-\textsc{nom} & pot-\textsc{dat} & (\textsc{io}.3)\textsc{ov}-look.at-\textsc{thm-prs.s.3sg}}
\item\shortex{3}
{kal-i & u-q'ur-eb-s & kotan-s.}
{woman-\textsc{nom} & (\textsc{io}.3)\textsc{ov}-look.at-\textsc{thm-prs.s.3sg} & pot-\textsc{dat}}
{`The woman looks at the pot.' [kat] \citep[1371]{skopeteas:fanselow:10}}}}


\noindent According to \citeauthor{skopeteas:fanselow:10}, both
sentences in \myref{exe:kat:focus} are legitimate in Georgian. The
difference between them is where the narrowly focused item appears in
a sentence; either in the immediately \isi{preverbal} position or in a
postverbal position. That is, \textit{kotan-s} ` pot-\textsc{dat}' in
(\ref{exe:kat:focus}a) is a preverbal \isi{focus} (necessarily), while the
subject \textit{kal-i} `woman-\textsc{nom}' and the object
\textit{kotan-s} in (\ref{exe:kat:focus}b) can be interpreted as
preverbal focus and postverbal focus (sufficiently), respectively.
\citeauthor{skopeteas:fanselow:10} argue that focus in the preverbal
position normally bears \isi{contrastiveness} (i.e.\ \isi{contrastive focus}).
Thus, the positions that \isi{non-contrastive focus} and contrastive
focus canonically occupy are different in \ili{Georgian}.\is{contrast}\is{narrow focus}

This distinction between two types of foci requires the grammar
library for information structure to allow users to select (a) whether a
language uses the same position for both kinds of focus, and (b) if
not, which type occupies which position.


Additionally, a given language might have two (or more) ways of
expressing contrastive meaning, and this also has to be considered in
modeling information structure in a cross-linguistic perspective.  For
example, \ili{Ingush} marks contrastive focus by two means; via the
use of a clitic \textit{=m}, and via word order, as exemplified in
(\ref{exe:inh:contrast}a--b) respectively.


\myexe{\eenumsentence{\label{exe:inh:contrast}
\item\shortex{9}
{Suona=m & xoza & di & xet, & hwuona & myshta & dy & xaac & (suona).}
{1s.\textsc{dat}=\textsc{foc} & nice & day & think, & 2s.\textsc{dat} & how & D.be.\textsc{prs} & know.\textsc{prs} & (1s.\textsc{dat})}
{`I don't know what you think, but \textit{I} think it's a nice day.' \citep[721]{nichols:11}}
\item\shortex{3}
{Pacchahw & \textbf{uqazahw} & hwavoagha}
{king & here & \textsc{dx}.V.come.\textsc{prs}}
{`The king is coming \textit{here} (he was expected to go somewhere else).' [inh] \citep[690]{nichols:11}}}}



\noindent The ordinary \isi{contrastive focus}, as shown in
(\ref{exe:inh:contrast}b) where focus is in boldface, occupies the
immediate preverbal position in \ili{Ingush}, and this position is
different from the non-contrastive focus position, which occurs
clause-initially.  According to \citeauthor{nichols:11}, the use of a
clitic as given in (\ref{exe:inh:contrast}a) is motivated by the
necessity to express contrastive meaning in a more marked way.\is{contrast}



In sum, the canonical position for contrastive \isi{focus} is language
specific; contrastive focus can either share the same position with
non-contrastive focus (e.g.\ \ili{English}, \ili{Greek} \citep{gryllia:09},
etc.)  or show up in another position (e.g.\ \ili{Portuguese}
\citep{ambar:99}, \ili{Russian} \citep{neeleman:titov:09},
\ili{Georgian} \citep{skopeteas:fanselow:10}, \ili{Ingush}
\citep{nichols:11}, etc.). Contrastive topics have no rigid
restrictions on position \citep{erteschik:07,bianchi:frascarelli:10}.


\section{Summary}
\label{4:sec:summary}


There are three linguistic forms of expressing information structure:
\isi{prosody},  \isi{lexical markers}, and \isi{syntactic positioning}.\footnote{There are also 
special constructions of expressing information structure, such as clefting.
The construction will be addressed later in Chapter~\ref{chapter10-3}.} 
The use of prosody to mark \isi{topic} and
\isi{focus} is widespread but not universal. The best way to handle prosodic
marking in the current work is to allow for \isi{underspecification}
in such a way that prosodic information can be added into the formalism at
a later point.\is{adposition} Lexical markers of information structure
can be affixes, adpositions, and modifiers. 
Information-structure marking adpositions are in
complementary distribution with ordinary case-marking adpositions in a
language. With respect to sentence positioning, I argue that
information structure of sentences in the \isi{basic word order} is
necessarily underspecified. When a constituent is ex situ and
narrowly focused,\is{narrow focus} four positions can be used: \isi{clause-initial},
\isi{clause-final}, \isi{preverbal}, and \isi{postverbal}.  Topics
canonically appear sentence-initially in some languages, but the
topic-first restriction is not necessarily a property of all
languages. Contrastive focus may or may not share the same position as
non-contrastive focus (i.e.\ semantic focus).\is{contrastive focus}\is{semantic focus}
Lastly, \isi{contrastive topic} does not enforce strong 
constraints on position across languages.

