\chapter{Conclusion}
\label{chapter15}
\setcounter{enums}{0}

\section{Summary}
\label{15:sec:summary}


This dissertation begins with key motivations laid out
Chapter~\ref{chapter1} for the creation of a computational model of
information structure. Chapter~\ref{chapter2} offers background
knowledge for understanding the main proposals of this dissertation,
including an overview of HPSG (\citealt{pollard:sag:94}) and MRS
(\citealt{copestake:etal:05}).

Part~\ref{part1} scrutinizes meanings and markings of information
structure from a cross-linguistic standpoint. Information structure is
composed of four components: focus, topic, contrast, and
background. Focus identifies that which is important and/or new in an
utterance, which cannot be removed from the sentence. Topic can be
understood as what the speaker is speaking about, and does not always
appear in a sentence (unlike focus). Contrast applies to a set of
alternatives, which can be realized as either focus or
topic. Lastly. background is defined as that which is neither focus
nor topic. There are three means of expressing information structure;
prosody, lexical markers, and syntactic positioning. There are three
lexical types responsible for marking information structures;
morphemes, adpositions, and modifiers (e.g.\ clitics). Canonical
positions of focus include clause-initial, clause-final, preverbal,
and postverbal. Building upon these fundamental notions,
Chapter~\ref{chapter5} looks into several cases in which discrepancies
in form-meaning mapping of information structure happen.

Part~\ref{part2} describes data compilation for exploiting a naturally
occurring text in four languages. The annotated text consists of a
translation set of the first 100 sentences in \textit{The Adventure of
  the Speckled Band}. The annotation follows guidelines provided in
\citet{dipper:etal:07}, with the addition of dropped elements and a
distinction between inner frames and outer frames within a discourse.
From the annotated data, several intriguing properties across four
languages can be observed. These help establish the HPSG/MRS-based
formalism in Part~\ref{part3}.

Part~\ref{part3} proposes using ICONS (Individual CONStraints) for
representing information structure in MRS
(\citealt{copestake:etal:05}).  This is motivated by three factors.
First, information structure markings should be distinguished from
information structure meanings in order to solve the apparent
mismatches between them. Second, the representation of information
structure should be underspecifiable, because there are many sentences
whose information structure cannot be conclusively identified in the
context of sentence-level, text-based processing. Third, information
structure should be represented as a binary relation between an
individual and a clause. In other words, information structure roles
should be filled out as a relationship with the clause a constituent
belongs to, rather than as a property of a constituent itself. In
order to meet these requirements, three type hierarchies are
suggested; \tdl{mkg}, \tdl{sform}, and most importantly
\tdl{info-str}. In addition to them, two types of flag features, such
as L/R-PERIPH and LIGHT, are used for configuring focus and
topic. Using hierarchies and features, the remaining chapters of
Part~\ref{part3} scrutinize multiclausal utterances and specific forms
of expressing information structure and also calculate focus
projection via ICONS.


Part~\ref{part4} creates a customization system for implementing
information structure within the \lingo Grammar Matrix
(\citealt{bender:etal:10}) and examines how information structure
improves transfer-based multilingual machine translation. Building
on cross-linguistic and corpus-based findings, a large part of
HPSG/MRS-based constraints presented in Part~\ref{part3} is
implemented in TDL. A web-based questionnaire is designed in order to
allow users to implement information structure constraints within the
\texttt{choices} file. Common constraints across languages are added
into the Matrxi core (\texttt{matrix.tdl}),
and language-specific constraints which
depend on the users' choices are processed by Python scripts and
stored into the customized grammar. Evaluations of this library using
regression tests and Language 
CoLLAGE~(\myurl{http://www.delph-in.net/matrix/language-collage}) 
show that
this library works well with various types of languages. Finally, an
experiment of multilingual machine translation bears out that
information structure can be used to reduce the number of infelicitous
translations dramatically.


%% As for evaluation, three types of
%% grammars are implemented for the regression tests and the testing of
%% multilingual machine translation. This evaluation shows that information
%% structure can be used to reduce the number of infelicitous
%% translations dramatically.



\section{Contributions}
\label{15:sec:contributions}

This dissertation holds particular significance for general theoretic
studies of the grammar of information structure. First of all, quite a
few languages are surveyed to capture cross-linguistic generalizations
about information structure meanings and markings, which can serve as
an important milestone for typological research on information
structure. Second, the data collection in the corpus study, despite
its small size, offers distributional findings on information
structure. In particular, the data set presents its own implications
given that it is comprised of a set of parallel annotated sentences,
and that the set of included languages presents a diverse set of
information structure properties. Since the data set is readily
available, it is my hope that other linguists exploit the compiled
data for their own research goods.

This dissertation also makes a contribution to HPSG/MRS-based studies
by enumerating strategies for representing meanings and markings of
information structure within the formalism in a comprehensive and
fine-grained way.  Notably, this dissertation establishes a single
formalism for representation and applies this formalism to various
types of forms in a straightforward and cohesive manner. Moreover,
this dissertation addresses how information structure can be
articulated within the HPSG/MRS framework and implemented within a
computational system in the context of grammar engineering.

%% This is an achievement which
%% proves the practical feasibility and mathematical tractability of the
%% HPSG/MRS formalism via empirical test.


This dissertation also shows that information structure can be used to
produce better performance in natural language processing systems. My
firm opinion is that information structure contributes to multilingual
processing; languages differ from each other not merely in the words
and phrases employed but in the structuring information. It is my
expectation that this study will inspire future studies in
computational linguistics to pay more attention to information
structure.


Last but most importantly, this dissertation makes a contribution to
the \lingo Grammar Matrix library. The actual library makes it easy
for other developers to adopt and build on my analyses of information
structure.  Moreover, the methodology of creating libraries I take in
this study can be used for other libraries in the system.  In order to
construct the model in a fine-grained way, I collected
cross-linguistic findings about information structure markings and
exploited a multilingual parallel text in four languages. These two
methods are essential in further advancements in the \lingo framework.





\section{Future work}
\label{15:sec:future}

This dissertation closes with a brief look at directions for
improvement in the future. 

%% EMB: I think maybe you can resolve this one with the
%% cross-classification we talked about.

%% First, on the theoretical basis, the distinction of
%% \tdl{frame-setting-topic} \vs \tdl{contrast-topic} needs to be
%% reexamined. The introduction of \tdl{frame-setting-topic} as one of
%% cross-cutting categories in information structure is largely based on the
%% theory of \citet{fery:krifka:08} and \cite{lambrecht:96}, and the
%% distributional evidence of \citet{li:thompson:76} and \citet{chafe:76}. However,
%% there are several pieces of evidence showing that \tdl{frame-setting-topic} is
%% another type of \tdl{contrast-topic}, which includes left-dislocation
%% in Romance languages, and \wa and \onun-marked adjuncts in the
%% sentence-initial positions in Japanese and Korean. \citet{arregi:03}
%% particularly argues that CLLDs (CLitic Left-Dislocations) in Romance
%% languages, which are assumed as conveying \tdl{frame-setting-topic} in
%% this dissertation, are instances of contrastive topic. \textit{Wa} or \onun-marked
%% adjuncts in Japanese and Korean, even if they are sentence-initial,
%% have been evaluated as containing \tdl{contrast-topic} in this
%% dissertation, which might be controversial, too. In fact,
%% \citet{gracheva:13} defers to the distinction between
%% \tdl{frame-setting-topic} and \tdl{contrast-topic} in the
%% \tdl{info-str} hierarchy, and suggests a variant of
%% \myref{fig:info-str} in her thesis. Further studies will have to dwell on
%% the details.

%% This dissertation has not seen such
%% particles case by case, but they had better be thoroughly described to
%% check out the feasibility of the formalism.


First, it is necessary to examine other types of particles responsible
for marking information structure.  Not all focus sensitive items are
entirely implemented in TDL in this dissertation even for
English. Japanese and Korean employ a variety of lexical markers for
expressing focus and topic, which are presented in \citet{hasegawa:11}
and \citet{lee:04}. A few focus markers in some languages have a
positional restrictions. For example, as shown in
Chapter~\ref{chapter4} (Section \ref{4:sec:lexical}), the clitic \textit{tvv}
in Cherokee signals focus and the focused constituent with
\textit{tvv} should be followed by other constituents in the
sentence. That is, two means of marking information structure operate
at the same time. These kinds of additional constraints will be
interestingly investigated in the future.


%% Russian also employs clitics for expressing contrastiveness
%% \citep{gracheva:13}, which are also in necessity of more
%% research. Moreover, the lexical markers sometimes interact with
%% syntactic operations. For example, an exclusive particle
%% \textit{shika} in Japanese should co-occur with negative operators
%% \citet{hasegawa:koenig:11}, and a Russian clitic \textit{li} occurs
%% only in interrogative sentences \citep{gracheva:13}.


Second, a few more types of constructions related to information
structure will be studied in future work.  The constructions include
echo questions (Section \ref{7:ssec:focus}),
\textit{Yes}/\textit{No}-questions \citep{king:95}, coordinated
clauses \citep{heycock:07}, double nominative constructions
\citep{kim:sells:07,choi:12}, floating quantifiers
\citep{yoshimoto:etal:06,kim:11b}, pseudo clefts \citep{kim:07}, and
\textit{it}-clefts in other languages in the DELPH-IN grammars
(e.g.\ Japanese \citep{hiraiwa:ishihara:02,kizu:05} and Korean
\citep{kim:yang:09}).


Third, the method for computing focus projection in this dissertation
also needs to be more thoroughly examined. There are various
constraints on how focus can be spread to larger constituents. These
are not addressed in this dissertation, which looks at the focus
projection of only simple sentences in English. The method the present
work employs for handling focus projection could be reinforced in
further studies.


Fourth, it would be interesting for future work to delve into how
scopal interpretation can be dealt with within the framework that this
dissertation proposes.  Topic has an influence on scopal
interpretation in that topic has the widest scope in a sentence
\citep{buring:97,portner:yabushita:98,erteschik:07}.  MRS employs
HCONS (Handle CONStraints) in order to resolve scope
ambiguity. Further work can confirm whether HCONS+ICONS is able to
handle the relationship between topic and scope resolution.

%% First, the evaluation of multilingual machine translation should be
%% extended with a large number of test suites. Further, more grammatical
%% fragments related to ICONS should be incorporated into the two
%% resource grammars that this dissertation makes use of (i.e.\ ERG
%% (\citealt{flickinger:00}) and Jacy
%% (\citealt{siegel:etal:16})). 

%% Second, the discrepancies between
%% meanings and markings in information structure would be better to be
%% more investigated with various examples from a perspective of both a
%% pure grammatical theory and practical applications. 

%% Fifth, regarding implementation, it should be tested whether ICONS can
%% be straightforwardly applied into other DELPH-IN grammars, such as KRG
%% (\citealt{kim:etal:11}), GG
%% (\citealt{crysmann:03,crysmann:05a,crysmann:05b}), SRG
%% (\citealt{marimon:12}), and so forth. 


Finally, the evaluation of multilingual machine translation will be
extended with a large number of test suites. More grammatical
fragments related to ICONS will be incorporated into the
DELPH-IN resource grammars, such as ERG (English Resource Grammar,
\citealt{flickinger:00}), Jacy (\citealt{siegel:etal:16}), KRG
(Korean Resource Grammar, \citealt{kim:etal:11}), SRG (Spanish
Resource Grammar, \citealt{marimon:12}), and so forth.



%% Ending with all these "should" statements sounds a little weak.  Try
%% to rephrase more confidently --- "will be interesting" is good, but
%% you similarly shouldn't use that one in every paragraph :)

