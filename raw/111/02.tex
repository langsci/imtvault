\chapter{Preliminary notes}
\label{2:notes}
\setcounter{enums}{0}



\section{Examples}
\label{1:sec:examples}


For ease of exposition, several typeface conventions are employed in
this book to represent properties of information structure in
examples. First, if a word (or phrase) bears the accent responsible
for conveying \isi{focus}, it is marked in \textsc{small
  caps}. Second, \textbf{boldface} denotes an accent conveying
\isi{topic}.  Third, [$_{f}$] stands for \isi{focus projection}. For
example, in the \ili{English} Q/A pair in (\ref{exe:sample}),
\textsc{dog} and \textbf{Kim} bear the A and B accents
\citep{jackendoff:72}, respectively, and the focus that \textsc{dog}
(with the A-accent) conveys is projected to the VP \textit{chased the
  \textsc{dog}}.\is{A-accent}

\myexe{\enumsentence{\label{exe:sample}
\begin{tabular}[t]{ll}
Q: & {What about Kim? What did Kim do?}\\
A: & {\textbf{Kim} [$_{f}$ chased the \textsc{dog}].}\\
\end{tabular}}}

\noindent Fourth, \# means that a sentence sounds infelicitous in the
given context, though the sentence itself is syntactically
legitimate. Finally, \mysout{strike} means either a constituent is
informatively empty or the given utterance cannot be generated from
the semantic representation (i.e.\ MRS).\is{MRS}


The examples that previous studies offer, as far as possible, are
cited without any change. Thus, glossing conventions may not be
consistent across examples. For example, the past morpheme may be
glossed as \textsc{pst} in one article or \textsc{past} in another.
All the examples created by me for the present study use the gender
neutral names \textit{Kim}, \textit{Lee} and \textit{Sandy} for any
people, and the name \textit{Fido} for any dog.  When an example is
excerpted from previous literature, the proper names in the example
are not modified at all.


\begin{table} 
\caption{Catalogue of languages}
\label{tbl:catalogue}
\footnotesize
\begin{tabularx}{\textwidth}{XXX}
\lsptoprule
{\normalsize \textbf{name}} & {\normalsize \textbf{ISO 639-3}} & {\normalsize \textbf{language family}}\\
\midrule
Abma & app & Austronesian/Oceanic \\ 
Akan & aka & Niger-Congo/Kwa \\ 
Armenian & hye & Indo-European \\ 
Basque & eus & unknown \\ 
Bosnian Croatian Serbian & hbs & Indo-European/Slavic \\ 
Breton & bre & Indo-European/Celtic \\ 
Buli & bwu & Niger-Congo/Gur \\ 
Cantonese & yue & Sino-Tibetan \\ 
Catalan & cat & Indo-European/Romance \\ 
Cherokee  & chr & Iroquoian \\ 
Chiche{\^w}a & nya & Niger-Congo/Bantu \\ 
Czech & ces & Indo-European/Slavic \\ 
Danish & dan & Indo-European/Germanic \\ 
Ditammari & tbz & Niger-Congo/Gur \\ 
(Northern) Frisian & frr & Indo-European/Germanic \\ 
French & fra & Indo-European/Romance \\ 
Georgian & kat & Kartvelian \\ 
German & ger & Indo-European/Germanic \\ 
Greek & ell & Indo-European/Hellenic \\ 
Hausa & hau & Afro-Asiatic/Chadic \\ 
Hungarian & hun & Uralic \\ 
Ilonggo & hil & Austronesian/Philippine \\ 
Ingush & inh & Ingush \\ 
Italian & ita & Indo-European/Romance \\ 
Japanese & jpn & unknown \\ 
Korean & kor & unknown \\ 
Lakota & lkt & Siouan\\ 
Mandarin Chinese & cmn & Sino-Tibetan/Chinese \\ 
Miyako & mvi & Japonic\\ 
Moroccan Arabic & ary & Afro-Asiatic/Semitic \\ 
Navajo & nav & Athabaskan \\ 
Ngizim & ngi & Afro-Asiatic/Chadic \\ 
Nishnaabemwin & ojg/otw & Algic \\ 
Norwegian & nor & Indo-European/Germanic \\ 
Paumar{\'{\i}} & pad & Arauan \\ 
Portuguese & por & Indo-European/Romance \\ 
Rendile & rel & Afro-Asiatic/Cushitic \\ 
Russian & rus & Indo-European/Slavic \\ 
Spanish & spa & Indo-European/Romance \\ 
Standard Arabic & arb & Afro-Asiatic/Semitic \\ 
Tangale & tan & Afro-Asiatic/Chadic \\ 
Turkish & tur & Turkic \\ 
Vietnamese & vie & Austro-Asiatic/Vietic \\ 
Wolof & wol & Niger-Congo/Senegambian \\ 
Yiddish & ydd & Indo-European/Germanic \\ 
\lspbottomrule
\end{tabularx}
\end{table}


Where I have needed to modify an example from the source, the example
has been judged by a native speaker of the language.  Any sentences
provided by native speaker consultants have also been faithfully
reproduced.  Every example presented in the present study has been
taken from literature as is or verified by at least one native
speaker. In the cases of Korean examples (a language of which I am a
native speaker), examples were again, either taken from previous
literature or created by me and judged by another Korean native
speaker.


Lexical markers in \ili{Korean} and \ili{Japanese} have been dealt
with in different ways by previous literature. Because the current
work aims to contribute to \isi{DELPH-IN} grammars, I follow the
approaches that \isi{Jacy} (\citealt{siegel:etal:16}) and \isi{KRG}
(\citealt{kim:etal:11}) are based on.  KRG identifies the lexical
markers in Korean (e.g.\ \ika for nominatives, \lul for accusatives,
and \nun for topics) as affixes responsible for syntactic (and
sometimes semantic) functions of the phrases that they are attached
to.\is{adposition} In contrast, the lexical markers in Japanese
(e.g.\ \ga, \textit{o}, and \wa) have been treated as adpositions by
Jacy, which behave as a syntactic head.  In the literature,
postpositions in Japanese, such as \ga and \wa are sometimes attached
to NPs with a hyphen (e.g.\ \textit{inu-ga} `dog-\textsc{nom}'), and
sometimes separated by white space (e.g.\ \textit{inu ga}).  In
extracted Japanese examples the presence/absence of the hyphen
reflects its presence/absence in the original source.  In any Japanese
examples created by me, I make use of white space instead of a hyphen,
following Jacy convention. Note that, different glossing formats
notwithstanding, Japanese \isi{lexical markers} are all implemented as
adpositions (i.e.\ separate lexical items) in the current work.  In
Korean examples, following the convention in previous literature,
hyphens are made use of (e.g.\ \textit{kay-ka} `dog-\textsc{nom}')
without any white space before lexical markers. Unlike the lexical
markers in Japanese, those in Korean are dealt with and implemented as
affixes.



Lastly, note that ISO 639-3 codes, such as [spa] for Spanish, [rus]
for Russian, [eus] for Basque, [jpn] for Japanese, [kor] for Korean,
[cmn] for Mandarin Chinese, [yue] for Cantonese, etc., are attached to
all examples not in \ili{English}.  The language catalogue is provided
in Table~\ref{tbl:catalogue}.



\section{Terminology}
\label{1:sec:terminology}

In addition to differences in glossing conventions, there is also some
variation in the terminology used by previous research into
information structure. First, the distinction between \isi{focus}
\textit{vs}. \isi{topic} has sometimes been regarded as a relationship
between rheme and theme, a distinction originally conceptualized by
the Prague School.  Within this framework, theme is defined as the
element with the weakest communicative dynamism in a sentence, while
rheme is defined as the element with the strongest communicative
dynamism \citep[72]{firbas:92}.



Using slightly different terminologies, \citet{vallduvi:90} considers
\isi{focus} to be the prime factor of information structure.  A
sentence, in \citeauthor{vallduvi:90}'s schema, can be divided into
focus and ground, and \isi{ground} can be divided again into link and
tail. Link is roughly equivalent to \isi{topic} in this book, with
tail corresponding to the remaining portion of the sentence. For
example, in \myref{exe:vallduvi:ch1}, \textit{the \textsc{dog}}
functions as the focus of the sentence and \textit{\textbf{Kim}
  chased} is the ground of the sentence which comprises the link
\textit{\textbf{Kim}} and the tail \textit{chased}.


\myexe{\enumsentence{\label{exe:vallduvi:ch1}
\begin{tabular}[t]{ll}
Q: & {What about Kim? What did Kim chase?}\\ 
A: & {[[\textbf{Kim}]\mysub{LINK} chased]\mysub{GROUND} the \textsc{dog}.}\\
\end{tabular}}}


\noindent Lastly, there is also some variation in labels for denoting
contrast.  \citet{vallduvi:vilkuna:98} use the term `kontrast' in
order to emphasize a different semantic behavior from non-contrastive
focus.\is{non-contrastive focus}  Instead of using theory-specific terms (e.g.\ rheme, theme,
link, tail, kontrast), the current work, makes use of the most
widespread and common terms for referring to components of information
structure: \isi{focus}, \isi{topic}, \isi{contrast}, and
\isi{background}.


On the other hand, to avoid potential confusion, the present work
provides alternate terminology for several morphosyntactic phenomena.
First, there are the OSV constructions in \ili{English} as exemplified in
(\ref{exe:prince:213:1}b), which are sometimes cited as examples of
`topicalization' in the sense that \textit{Mary} in
(\ref{exe:prince:213:1}a) is topicalized and preposed.

\myexe{\enumsentence{\label{exe:prince:213:1}
\begin{tabular}[t]{ll}
a. & {John saw Mary yesterday.}\\
b. & {Mary, John saw yesterday. \citep[213]{prince:84}}\\
\end{tabular}}}

\noindent Instead, the present study calls such a construction
`focus/topic \isi{fronting}' taking the stance that constructions like
(\ref{exe:prince:213:1}b) are ambiguous. Because a fronted phrase such
as \textit{Mary} in (\ref{exe:prince:213:1}b) can be associated with
either \isi{focus} or topic, the term `topicalization' cannot
satisfactorily represent the linguistic properties of such a
construction.  Second, \wa in \ili{Japanese} and \nun in \ili{Korean}
have been labelled as `topic markers' by many previous
studies. However, they are not used exclusively to mark topics. They
are sometimes employed in establishing contrastive
focus.\is{contrastive focus} Thus, `topic-marker' is not an
appropriate name (see Section \ref{5:sec:lex}).  Instead, the present study
uses just \wa-marking and \onun-marking in order to avoid
confusion. In the IGT (Interlinear Glossed Text) format of Japanese
and Korean examples, even if the source of the IGT says \textsc{top},
they are glossed as \textsc{wa} and \textsc{nun} unless there is a
particular reason for saying \textsc{top}.


