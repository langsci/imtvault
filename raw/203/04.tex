\chapter{Sociolinguistic Variation in JAMPRO}\label{ch:4}
\section{JAMPRO: One site of promoting a SJE ideology}\label{sec:4.1}

In this chapter I present information on the 82 speakers, not as a sample of educated Jamaicans, but as members of a social collective,\footnote{\citet{Sealey2001} use the term to describe a grouping whose membership is indicated by an awareness of, and a kind of commitment to, the conventions that constitute the group in the first place (4).} and specifically as members of an agency of the Jamaican state, \isi{JAMPRO}.  Following on the discussion in \chapref{ch:1}, in particular on the ideology of language in Jamaica and the normative pressure exerted by institutions in their expectation and practice, I selected one government agency and the speech of its employees for analysis.  The type of agency, the requirements for employment, and the patterns that lead to success for staff, can certainly shed light on the phonology of “good” Jamaican English.  As was mentioned at the start of \chapref{ch:2}, all advertisements for employment at agencies like \isi{JAMPRO} insist on “a good command of the English language” or “excellent oral and written skills”.  But the study of staff selection and promotion can also reveal the type of employee that such an agency perceives and selects as best able to realize its own image and the image(s) that it constructs for and about Jamaica.  I have selected an agency whose function it is to market Jamaica to both local and overseas clients, with some of its staff specifically selected as representatives of the agency through the positions they hold at \isi{JAMPRO}.  

Scollon’s comments (\citeyear{Scollon1997}) are, therefore, a good starting point for this discussion:

\begin{quote} If we think of public discourse in this sense as being outputs of bureaucratic structures, then a crucial point of interest is the boundary across which the discourse is generated (...) in a business it might be an advertising department or a public relations department, in the government it might be a public affairs office.  In any case, one might always expect to find some formally constituted institutional structure along with designated members of the institution who serve to both generate and legitimate the crossing of the organizational boundary into public discourse (45).\end{quote}

Agencies of the state like \isi{JAMPRO} are, in part, engaged in creating an idea of Jamaica and the Jamaican for a client and therefore a consumer of that construct.  What the agency says or implies about a Jamaican must be reflected in the type of employee that it selects to symbolize its public face and to interact with its clients.  In its role as one of the agencies that legitimate Jamaica to others overseas, \isi{JAMPRO} puts into the arena of public discourse certain ideas, one of which concerns language.  Its own language use, as reflected by the language behaviour of its staff, therefore becomes important.  This language behaviour provides us with actual examples of how the agency’s language ideologies become realized in its allocation and selection of staff for certain positions.

A number of Jamaican government agencies provide information about the language situation in Jamaica, \isi{JAMPRO} being one of them.  I wish to discuss the description of the local sociolinguistic situation provided by these agencies of the Jamaican state, and therefore the nature of the \textit{official} public discourse about language in Jamaica.  Firstly, this discourse serves as an indicator of the linguistic expectations for staff held by these arms of the state.  But such a discussion also brings into focus the ideas held about language in Jamaica and the ways in which they (may) have changed over time.  In the previous chapters, some of the comments on the language situation found in the main local newspaper were mentioned.  They represent one aspect of public discourse, which \citet{Christie1998b} discusses in her study of this aspect of the language debate in Jamaica.  I wish to explore the institutionalised aspect of public discourse, and the image of the Jamaica(n) that can be abstracted from published official documents.  The agencies to be discussed here are a) the Jamaica Tourist Board (\isi{JTB}), b) the Jamaica Information Service (JIS) and c) \isi{JAMPRO}.  

\subsection{{Jamaican} {institutional} {discourses} {on} {language}}%4.1.1n

The marketing of Jamaica for tourism is believed to have started in 1851 \citep[62]{Taylor1971}, when a pamphlet, ascribed to someone named Anderson, was published urging Americans to come to the island for health reasons, particularly in winter.  Tourism then, as now, was geared to the needs of a largely U.S. clientele, as a great deal of initial travel to Jamaica was prompted by the United Fruit Company’s ties to Port Antonio through the banana industry.  Titchfield Hill was one of the first hotels, a “lost garden of Eden – the incomparable combination of American comfort, English cleanliness and Italian climate” (1903 pamphlet, unnamed author, cited in \citealt[99]{Taylor1971}).  

The Jamaica Tourist Association was formed in 1910, and, as Taylor remarks, scant mention was made of Jamaicans themselves, as they were viewed as more of a liability than an asset in the marketing of the island.  Even then there was concern expressed about letting visitors see all aspects of the society.  The only mention of Jamaicans cited by Taylor is in a 1914 pamphlet: “the hospitality of the people is proverbial ... demonstrated by the smiling faces and happy laughter of the natives” (4).  These are, of course, the production of an association formed during the time of a colonial government.  A comparison with post-independence publications, when the agency morphed into the \isi{JTB}, projects a similar image.  Jamaicans are described as: “gentle people named Ivy or Maud or Malcolm who will cook, tend, mend ... who will “Mister Peter, please” you all day long” (\isi{JTB} 1968, cited in \citealt[174]{Taylor1993}).  Taylor analyses this as marketing for a U.S. sensibility, a clientele more accustomed to and comfortable with the idea of “Negro servants”, and persuading them that Jamaica was safe in spite of its majority black population.  Language is not mentioned at all, perhaps because interaction was not expected, though the implication here is that English is at least understood.

More recent advertisements do discuss language.  A typical example, directed specifically at North Americans again, is contained in a \isi{JTB} pamphlet (1992) and entitled \textit{Say It Again, Mon,} “mon” being an attempt to represent the Jamaican vowel that is more backed when compared to its \isi{General American} counterpart.  I have numbered particular items for comment.

\begin{quote}
Listen to two Jamaicans talk and you’ll hear a (1) musical mix of English, \isi{patois} (a combination of English, African and Welsh (sic)) and island words and rhythms.  You might think it’s (2) impossible to make sense of it all without being a native.  Not so!  While it might be tough to wholeheartedly jump into a rocket fast discussion, (3) it’s easy to learn a few phrases that might come in handy at the local rum shack or in the marketplace.  Here is a mini-glossary to get you started.

\begin{description}
\item[Cool runnings]  (4) No bother, no fuss, it’s okay.  Also used when parting to mean “goodbye”.
\item[Gimme a chups]  Kiss me.
\item[Irie (EYE-re)]  (5) Everything is cool.
\item[Irie dawta]  A (6a) sexy, good-looking girl.  “Wat a irie dawta!”.
\item[Jah God]  as in “Praise Jah.”  From the (7) biblical name for Jehovah; this phrase is most often used by Rastafarians.
\item[Kiss me neck]  Not to be taken literally, this is an expression of surprise, incredulity or defiance.  (8) “Kiss me neck, I can’t believe you got married last night and didn’t tell anyone you were going to do it”.
\item[Soon cum]  Meaning someone or something will arrive anywhere (9) be-\linebreak tween the next five minutes and the next five hours.
\item[Tek yuh time]  Take your time; a favourite (10) in the craft market where\linebreak you're being invited to browse.  Nobody’s going to rush you here!
\item[Ting]  This is a tricky one.  First and foremost its simply (11) slang for “thing.”  However, if someone says, (6b) “Mek we do a ting” to a woman, it’s considered a proposition.  If it is said to a man, (12) the phrase means “Lets make an arrangement, do business, or have a drink.”  To make matters more (13) confusing, Ting also happens to be a popular grapefruit soda.
\item[Walk good]  A farewell meaning, “Take care, stay safe, hope all goes well with you”.\footnote{Reprinted from \textit{Bridal Guides Magazine}.}
\end{description}
\end{quote}  

Comments made about the language situation to be found readily acknowledge \isi{Creole} (Patois), suggesting it is a mix of English and other ingredients, and do so in ways that apparently justify adjectives like “musical”, “colourful” and “rhythmic”.  In addition, this mix of language varieties is confused and confusing (2, 3 and 13).  Similar terms are used to discuss regional patterns of English, in comparison with the standard varieties of English.  These imply “lack of clarity” and “the charm of the humor in confusion” (\citealt[292--293]{Silverstein1996}).  Note, however, that in spite of the idiosyncrasies and slang items, the presentation suggests that it is clear that Jamaicans speak English; though the article offers the tourist the local phrases, the representation of actual Jamaican speech is English (12 \& 8).  The tourist to Jamaica, for the most part from the U.S., needs only to learn a few simple phrases in order to communicate with the Jamaican he\slash she is likely to meet.  In addition, these phrases are most useful in the rum bar or marketplace (3), which clearly locates the contexts in which \isi{Creole} is likely to be heard and to be appropriate.

  What is also of interest are the phrases themselves that were selected for the article.  Four of them project an image of a relaxed, happy-go-lucky people (compare the 1914 pamphlet) with little concern for schedules or time, as in 4, 5 and 9, or serious business, as in 10 and 12.  The visitor is also subtly informed that the women are young and attractive (6a and b) and people, however strange looking, are non-threatening and Christian (7 and 10).  The latter piece of propaganda, the term used here in its denotative sense, is particularly important, as “visitor harassment” is said to be a problem in the resort areas on the island.  More specifically, the arguably negative notion of “defiance” in the Jamaican is reduced to personal concerns that have nothing to do with the tourist or the hardships of living in a relatively poor country.  In summary, Jamaica is said to have two language varieties, but the Jamaican is represented as English speaking (with some regional elements).  The people are carefree and laid-back.  The Jamaican woman is presented only in a sexualized way, and the wider socio-cultural context is Judaeo-Christian and therefore not unfamiliar.

  This pamphlet, produced by the \isi{JTB}, is a reproduction of a collection of articles on Jamaica that appeared in a US magazine.  In that respect, it is an \textit{American} image of Jamaica and Jamaicans that the \isi{JTB} is using to sell the destination to Americans.  Other \isi{JTB} documents, particularly more recent publications, speak of the Jamaican language situation using the same terms: “We speak English, with a few embellishments” \citep{JTB2001}.  In that respect, according to the \isi{JTB}, it is suggested that Jamaicans speak a variety of English not very different from other Englishes except for a few localisms.

A second example is taken from the Jamaica Information Service (JIS), the official information agency of the government.  Interestingly, no explicit reference is made to English.  The reader is invited to infer from the language in the electronic document that English is the \isi{official language} of the State.  Instead, the following is said:

\begin{quote}
The language “\isi{patois}” is an important part of who we are, giving the people a peculiar \isi{accent} so much so that even in countries outside the region, we are easily identified.  The Jamaican sound is so loved that even persons who do not sound like us are often quite comfortable being called Jamaican.  While there may be variations in the \isi{patois} \isi{accent} across the island, there are words and phrases that have gone beyond the boundaries of our little island (Jamaica Information Service, accessed 2002 at \url{www.jis.gov.jm}).
\end{quote}

This naming of Patois as “language” is clearly unrelated to the actual description provided.  For Patois is here considered to be only English with an \isi{accent} and some lexical\slash idiomatic regionalisms.  Moreover, this “sound” of a Jamaican is charming enough to be imitated by others.  Since, for the JIS, the difference between Patois and other Englishes is largely a matter of phonology, this may account for no direct reference to English as the language of Jamaica, just as there would be no need to make such a clarification in Australia or the UK. 

The sources of these examples represent what Scollon calls the products of formally constituted institutional structures and therefore the language ideologies of the groups that occupy such positions.  These are organizations that are staffed by “middle class” Jamaicans, in terms of education levels, income\slash occupation, residence and the like.  They naturally have particular ideologies which, according to \citet{Austin1983}, are based on

\begin{quote}the view (...) that the middle class gained and deserved their access to higher education and the professions because their socialization and Christian\linebreak ways have made them superior to the uneducated, indeed uncivilized, working class (236).\end{quote}

She cites as an example the “middle class” attitude to traditionally working class practices like Pentecostalism and Rastafarianism.  The former is seen as “degrading” (233) while the latter was coopted by the middle class and is now approved for use to sell Jamaica.  As if in confirmation of this, one member of \isi{JAMPRO} \isi{senior management} said in his interview that

\begin{quote} ...if the person don’t (sic) portray the middle class features ... am...could be kept back.  For example I know of people who are members of the conservative pentecostal and evangelical type religions who have been kept back (M20).\end{quote}

\isi{JAMPRO} was legally incorporated in April 1990 under an arrangement that merged three previously existing government bodies.\footnote{The three are \isi{JIDC} (Jamaica Industrial Development Corporation), \isi{JNIP} (Jamaica National Investment Promotions) and \isi{JNEC} (Jamaica National Export Company).}  These three bodies were created in the first half of the 1980's  “precisely in recognition of the fact that investors need to be on a fast track to speed up the investment process” \citep{JAMPRO1997}, but had a forerunner in the pre-Independence Jamaica Industrial Development Corporation of the late 1950's.  The three functioned in overlapping, but different, areas of trade\slash investment promotion.  The first, \isi{JNIP}, was charged with identifying potential investors and selling Jamaica to them as a safe and profitable location for their business, i.e. its purpose was as a national marketing agency.  The second, \isi{JNEC}, targeted both local exporters and foreign investors, and was created to smooth the way through the maze of the Jamaican trade and tariff bureaucracy.  In that respect, it was there to represent the Jamaican government to business and it promised “enough clout in the public sector...that the investor is facilitated as quickly and easily as possible” (op.cit.).  The third company, the \isi{JIDC}, was mandated to oversee the industrial development policy of the government, and therefore functioned as go-between for buyers of land, equipment and infrastructure and the holders of capital.  In 1990 government decided that one umbrella corporation would increase efficiency and inter-agency communication and created \isi{JAMPRO}.

  It is important therefore to emphasise that \isi{JAMPRO} is an arm of the Jamaican state, crucial in representing Jamaica to other individuals and governments as “the investment capital of the Caribbean” \citep{JAMPRO1996}.  It is involved in the serious enterprise of economic growth and development – without which very little else of government policy can be realized.   

\isi{JAMPRO}, to further its purpose, produces a number of publications designed to provide its clientele with information about a) the wider \isi{Jamaican social context} and b) \isi{JAMPRO} itself, including the type of employee it selects and that the investor is likely to encounter when dealing with the company.  \isi{JAMPRO} is one such “formally constituted institutional structure” that is a state agency, within which functions its own public relations department.  By examining the information put out by this department, we can get a sense of the image of Jamaica and of Jamaicans that \isi{JAMPRO} constructs for its clientele.  In constructing this image for its clientele, \isi{JAMPRO} also generates and legitimizes a particular idea of the Jamaican society and language situation.  It is this ideology that I would like to explore in detail, and how it becomes realized in the allocation of particular staff members in the agency to certain functions.

Firstly, the ideologies of language and society that \isi{JAMPRO} documents reveal are a projection of the ideas of self held by one class of Jamaicans, vis-à-vis others who occupy the same sociolinguistic space.  These ideas are promoted against the background of and in relation to their own ideas of who the client is.  To illustrate, there are a number of social and linguistic practices that could be described in \isi{JAMPRO} publications, but only some are selected.  The selection reveals not only what this class perceives itself and the ‘marketable’ Jamaica to be, but also what they believe will inspire confidence in the client.  To do this, certain assumptions must be made about themselves and the client, what they have in common – and can therefore go unexplained – and what is distinctive.  To illustrate, no \isi{JAMPRO} document tells the investor that Jamaicans live in houses, as this would be unnecessary as the class that composes the information would view this as normal for both itself and the client.  However, language in Jamaica has to be accounted for, and in ways that we can use to locate what this class perceives as normative.  

Secondly, employees of the company, and in particular those who have to interact with clients, will have to reinforce this image or the client will lose confidence in the information provided.  \isi{JAMPRO} cannot project an identity of the country and itself that is dissonant with its own behaviour, reflected in the behaviour of its staff.  Clients to the company who are encouraged to place their confidence, and ultimately their money, in this described Jamaica, do so, in part, because of the expectations created by the agency.  Its staff, therefore, must be in a position to confirm this image of Jamaica.  As I was told by two members of \isi{senior management}:

\begin{quote}
	It's not overtly stated as being important, but I think it is a factor.  I think that when you consider the sort of image that [the agency] wishes to project, certainly the thinking is, I feel, [that] people from certain backgrounds can give or protect that image, rightly or wrongly (M101).   
\end{quote}

\begin{quote}
	A job like this requires a certain type of personality as well as the qualifications you try to find a combination so its not all qualifications ... important consideration of someone here having to go abroad to represent the organization to me that is important and you’ll look for the possibility of this with somebody (F55).
\end{quote}

The published \isi{JAMPRO} information and the personnel that the agency employs steer the investor in a particular social direction when advising on where and how to interact with other Jamaicans generally.  An analysis of this advice, including some historical discussion of the type of output, can reveal what aspects of Jamaica are considered by agency staff to best conform to the norm that is created in their own publications.  

\subsection{{The JIDC\slash JAMPRO: Their construct of the Jamaican language situation}}%4.1.2n
The \isi{JIDC} was established in 1952.  I was able to find three specific references to the local language situation and the Jamaican people, which I will quote in full (my emphases).

\begin{quote}
	Jamaica’s Alert, Energetic People – The Jamaican is intelligent, self-re\-spect\-ing, and has a keen technical as well as artistic sense.  \textbf{Jamaicans speak English with a soft, gentle inflection partially reflecting the British \isi{accent}.}  The US, \isi{Canadian} or British manufacturer will not face the \textbf{problem of a foreign-language-speaking work force which he would encounter in many other Caribbean}, Latin American areas.  The \textbf{advantage of a common language} also facilitates the training of Jamaicans for responsible posts in manufacturing \citep[11--12]{JIDC1957}.  
\end{quote}

The selling points of Jamaica are that it belongs to the English language community, in common with the US, Canada and Britain, unlike many other Caribbean territories.  Moreover, all Jamaicans speak an English that is phonologically somewhat like British English, though “British” is not further defined.  It is not unreasonable to interpret this as \isi{RP}, the variety most likely to come to mind for a \isi{Canadian} or American reader.

A later publication says much the same thing:

\begin{quote}
	Jamaica is a land of soft-spoken people (...) Jamaicans speak English.  The US, \isi{Canadian} or British manufacturer arriving in Jamaica does not face a language difficulty in training and supervision. The advantage of a common language is that it speeds up training and production \citep[2--8]{JIDC1961}.
\end{quote}

The fact that Jamaicans also speak something else, \isi{Creole}\slash Patois, is not addressed.  There are a number of possible explanations for this.  Firstly, additional information in these pamphlets gives an indication of where and with whom the investor was expected to interact and the strata in the society that the \isi{JIDC} wished to own.  For example, under the heading Cultural Activities, the investor is told of “lectures on music and art sponsored by the British Council and the Institute of Jamaica” (13), despite the local popularity of mento bands and the numbers of venues at which sound systems played \citep[7--13]{Barrow1997}.  Additionally, the Social Clubs recommended by the documents were exclusive and\slash or member’s only establishments such as Jamaica Club, Liguanea Club and the Garrison Officer’s club.  Secondly, the idea of Patois as a language was not, at the time, an accepted aspect of public discourse.  Indeed, \citet{LePage1988} asserts that it was the discussion in academia, by David DeCamp, Beryl Bailey and himself, that gave some, however limited, legitimacy to the idea of something called “\isi{Jamaican Creole}” as a language in its own right.  Interestingly, both quotations are careful to suggest the gentleness of the people, perhaps to reassure the investor’s anxieties about the social context, in a similar way to Taylor’s reading of the \isi{JTB} information. 

Current \isi{JAMPRO} documents\slash pamphlets are noticeably different in some\linebreak ways, at times presenting the language situation in more measured terms:

\begin{quote}
	The \isi{official language} of Jamaica is English. However, the majority of the population speak Jamaican creole called \isi{patois}, which is a mixture of English and derivatives of various West African languages. The \isi{Jamaican Creole} has been studied by many scholars. There is a Dictionary of Jamaican English published by Cambridge University Press.
\end{quote}

There is another variety spoken in Jamaica, legitimised by being in a dictionary.  However, the magazine then immediately goes on to reassure the reader, in two other places that: “\isi{Kingston} [the capital] is the largest English speaking city a) south of Miami and b) in the Caribbean.” \citep[2--3]{JAMPRO1997b}.  This insistence that Jamaica is English speaking dominates the information available to those interested in more serious ventures -- investment, manufacturing or business.

\begin{quote}
	The \isi{official language} is English.  What we call \isi{patois}, an English-derived dialect, is widely spoken, though seldom in official or business circles.  Patois tends to prove challenging to foreigners, though a number of publications ranging from visitor guides to scholarly dissertations are available on the hotly debated language.  Local radio and television dramatic productions also have high \isi{patois} content, as do some radio call-in programmes from time to time, especially when tempers rise! \citep{JAMPRO1997}.
\end{quote}

\begin{quote}
	English-speaking, well-educated and competitively priced labour force\linebreak\citep{JAMPRO1993}. 
\end{quote}

\begin{quote}
	Jamaica’s English speaking population of approximately 2.5 million represents a diverse mix of ethnic origins \citep[15]{JAMPRO1994}.
\end{quote}

\begin{quote}
	English-speaking, non-unionised and experienced professionals (which include accredited technicians) as [film] crew \citep[5]{JAMPRO1993}.
\end{quote}

Here the investor is informed that \isi{Creole} is widely used but only in specific cultural and discourse contexts.  \isi{Creole} is also not the language of serious business, suggesting the kind of distinction the \isi{BBC} made between programmes hosted by regionally accented and \isi{RP} accented broadcasters.  Moreover, the fact that the language is “hotly debated” suggests the kind of local controversy about \isi{Creole} that is not described for English.  The investor is further reassured that the pool of labour from which comes his\slash her workforce speaks English and gets a British-style education:

\begin{quote}
	English is universally spoken.  Education is based on the British system\ldots\xspace English is the \isi{official language} of Jamaica, and this makes communication with off-shore investors and understanding new processes easier\ldots\xspace The language of the workplace being English, instruction at all levels is given in English, even where the workers speak \isi{patois} and not \isi{Standard English} \citep{JAMPRO1992}.
\end{quote}

  Two important things are being said here.  Firstly, English is \textit{universally} spoken and, it is suggested later in the passage, understood.  Even if the investor hears \textit{workers} speaking \isi{patois} they should not take this to mean that these workers cannot speak or understand English.  Secondly, a distinction is being made between English, Patois and \isi{Standard English}.  What the investor will find, and what is being implied, is that the worker’s speech, Patois, is not much more than a regional non-\isi{standard variety} of English, for what workers do not speak is \isi{Standard English}, the language of official and business circles.  

In summary, the \isi{JAMPRO} documents give the following sociolinguistic information:

\begin{itemize}
\item All Jamaicans speak English (“universally”).
\item Some Jamaicans also speak Patois (“widely spoken”).
\item Patois is non-standard English, largely distinguished by some items, possibly of West African origin.
\item Patois is used by workers, but they also speak and understand English.
\item Patois is used in specific (dramatic) contexts and when tempers flare.
\item Patois is controversial.
\item Official and business circles use (Standard) English.
\item \isi{Standard English} is only a “problem” for workers.
\end{itemize}

  Generally, we can deduce from these documents that \isi{Creole} enters the public discourse, as generated and structured by these government agencies, as the other: the non-normal, non-universal aspect of the Jamaican socio-cultural landscape.  We can further deduce that Patois, being peripheral to the official and business milieu in which \isi{JAMPRO} operates, is not going to be heard by the investor in his interactions with the agency. 

I found only one specific text reference to the staff at \isi{JAMPRO}.  The magazine in which it is contained promises “highly trained professionals [who] provide friendly, experienced assistance” \citep[10]{JAMPRO1996}.  If we relate this information to the above discussion, then we would expect \isi{JAMPRO} staff members, who do operate in official and business circles, at least to speak (Standard) English and to be very well educated.  The speech that these employees actually produce must be useful in identifying what \isi{JAMPRO} management considers to be protective of its image, and conforming to the norms they have described above for Jamaica.  This is an agency in which the investor will encounter highly trained and (Standard) English speaking professionals.

\section{Frontline staff – The public face of JAMPRO}\label{sec:4.2}

There are certain employees at \isi{JAMPRO} designated “\isi{frontline staff}”.  This label is applied to those employees who actually interact with the public, both local and foreign, in the course of their duties.  In certain departments at \isi{JAMPRO}, certain members of staff are required either to deal directly with the agency’s clients or to represent the agency in overseas departments or trade missions.  It is this subset of my informants – those in the frontline – who are the focus of this section.  

There are a number of reasons why the speech patterns of \isi{frontline staff} are crucial to the issues raised in this book, particularly in relation to the legitimizing of norms of \isi{JE} usage.  Firstly, as has already been discussed in previous chapters, selection of staff at \isi{JAMPRO} is heavily dependent on performance during the job interview.  And this is especially so for professional and (senior) management positions.  Frontline staff positions reflect an even more subjective set of selection criteria on the part of the agency, as these are employees who are the public face of \isi{JAMPRO}.  As explicitly stated by \isi{senior management}, those in frontline positions are selected because they are perceived to be the type of person who can represent both the agency and the agency’s idea of a well educated, competent and professional Jamaican.  

Secondly, in selecting some employees for frontline positions and rejecting others, \isi{JAMPRO} management is communicating the type of Jamaican they think best fits the desired image, one which may well be informed by the speaker’s use of English.  \isi{JAMPRO}, as shown at the beginning of this chapter, operates in precisely the socio-cultural milieu that it advises is reserved for (Standard) English.  Frontline positions indicate a positive judgement made by some Jamaicans, i.e. the \isi{senior management} who assign staff to jobs, on the way other Jamaicans present to them in interviews.  This forms, in large part, the basis on which certain members of staff are selected to interact directly with clients (both local and foreign) and to represent ``Jamaica" and the upper end of the construct of the local language situation described in \isi{JAMPRO} publications.  Their promotion into the frontline provides us with some means of indexing the reactions of others to the English they produce, for they have been selected from a pool of speakers who themselves had been chosen from a pool of educated Jamaicans with English language qualifications.  In that respect, the analysis of the speech of \isi{frontline staff} allows me to add an evaluative dimension, albeit an implicit one, to this study.  Within the organization, at least, the kind of behaviour that comes to be perceived as suitable and successful for many, in terms of advancement and employment prospects, can be taken to be reflected in \isi{frontline staff}.      

Thirdly, informants themselves held perceptions about the kind of person who is in the frontline.  I would like to examine these perceptions and compare them to relevant data on the sociology of the agency.  For if staff perceptions about \isi{frontline staff} are reflected in the findings, then it gives added credence to those perceptions less amenable to examination – most importantly the ones about language and speaking “good” English. 

In the course of interviews, a number of the informants made comments on their perceptions of the type of employee who gets selected for frontline positions.  An asterisk indicates that the speaker is in \isi{senior management}.    

\begin{quote}
	yes I think so, am there’s a particular look which you must have, which I think, you know, I think fair...fairer skin people tend to, especially in terms of overseas officers...it’s as if they want to portray a certain look of a Jamaican (F13).
\end{quote}

\begin{quote}
	there are some reasons for it in that you are talking about a promotional organization and sometimes you will buy into that...somebody with a particular social background, somebody who has been exposed early has the ability to present themselves and to cut a dash...but I think it happens a little bit too much in that some are excluded where they shouldn’t get (inaudible) as sophisticated they’ll step over her (F64).
\end{quote}

\begin{quote}
	I remember once a skin thing came up and it was more in relation to the film division, there was a time when the perception was that only fair skin people can deal with movie stars coming from abroad and the colouration there tended to seem to reflect that (F87).
\end{quote}

\begin{quote}
	...but for example if they want persons to go on say overseas assignment...if you’re going to Korea or one of those you know eastern countries right they prefer male to go they specifically ask for male (F71).
\end{quote}

\begin{quote}
	Yeah I think sometimes...deciding they probably notice the way you act and the way you speak if you can really represent well at a standard (F73).
\end{quote}

\begin{quote}
	I think it does I think it does because one has to have a certain amount of command a certain amount of respect from the people that you have to interact with and it [a certain background] does help it opens a lot of doors put it that way...and that’s [education] important because it it equips you to have a good command of the English language...the subject matter and it sort of helps you to be able to help the client better (F7).
\end{quote}

\begin{quote}
	\isi{race} especially pigmentation is important, the brownings, I feel so but you know that fits within the, you know, more with the profile you see some of these things can be justified on the basis that we are a promotional organization so the investors are likely to be impress (sic) by a certain type...it's just that we want to present the right image and the image would include a certain kind of, of diction and good looks too...if that person is not obviously from the middle class and if the person’s manner and, and dress don’t portray the middle class features [they] could be kept back...(M20*, Public Relations).
\end{quote}

\begin{quote}
	[language] is very important it’s very important in that in \isi{JAMPRO} you have to interface a lot with your external clients your ability to deliver is very important (F32*).
\end{quote}

\begin{quote}
	It's not overtly stated as being important, but I think it is a factor.  I think that when you consider the sort of image that [the agency] wishes to project, certainly the thinking is, I feel, [that] people from certain backgrounds can give or protect that image, rightly or wrongly (M101*)
\end{quote}

\begin{quote}
	at the stage of the interview I ask is this person somebody I could send away next week to go and talk to an investor or to go and sell Jamaica. (F55*).
\end{quote}

\begin{quote}
	This is a place where you interact a lot with the public, with international agencies...if you have a, well, you’re well educated and everything your social class tends to be middle class (F56*)
\end{quote}

Generally, these comments speak to a perception among staff, that those in the agency who interact with the public, especially foreign clients, are selected on the basis of a) certain social backgrounds, i.e. from the middle class, b) light \isi{skin colour} c) “good looks” and d) language.  The one mention of \isi{gender} is specifically related to assignment to some Asian countries where men are specifically requested by the host country.  That suggests an issue outside of \isi{JAMPRO}’s control, one which they attempt to accommodate. 

The first set of data presented discusses those social factors which proved to have a statistically significant association with having frontline status.  The data is presented in percentages, but the statistical significance was calculated on the raw data.  

\begin{table}
\resizebox{\textwidth}{!}{\begin{tabular}{l *{6}{S}}
\lsptoprule
        & \multicolumn{6}{c}{Education (\%)}\\\cmidrule(lr){2-7}
 Position &  \multicolumn{1}{c}{Primary} &  \multicolumn{1}{c}{Secondary} &  \multicolumn{1}{c}{Secondary+} &  \multicolumn{1}{c}{Tertiary} &  \multicolumn{1}{c}{Tertiary+} & \multicolumn{1}{c}{no data} \\\midrule
+Frontline              & 1.5 & 3 & 36.5\cellcolor{lsLightGray} & 27\cellcolor{lsLightGray} & 32\cellcolor{lsLightGray} & \\
\textminus Frontline    & 8 & 21.5\cellcolor{lsGuidelinesGray} & 46\cellcolor{lsGuidelinesGray} & 11 & 11 & 2.5\\
no data                 &  &  & 3 & 1 &  & \\
\lspbottomrule
\end{tabular}}
\caption{Level of education and selection to frontline duties at JAMPRO\label{tab:4.1}}
\end{table}

By inspection, it is clear that selection for a frontline position becomes more likely if one is better educated (p < .001, χ\textsuperscript{2} = 10.91 when analysed as secondary+ and tertiary+).  Here, to accept the null hypothesis (H\textsubscript{0}), we would expect to find more primary and secondary school leavers in the cadre of \isi{frontline staff} members.  What we do have is a greater proportion of university graduates in these positions (almost 60\%).  The one primary educated informant is M40, the driver\slash chauffeur for the agency.  Staff perceptions that education is important for frontline positions are confirmed by this data.  \isi{JAMPRO} does cream off its better-educated employee for interaction with the public. 

\begin{table}
\begin{tabular}{lrrr}
\lsptoprule
& \multicolumn{3}{c}{Parent’s Occupation  (\%)}\\\cmidrule(lr){2-4}
 Position & Cleaner\slash Artisan & Teacher\slash Doctor & no data\\\midrule
 +Frontline & 46 & 44 & 10\\
 \textminus Frontline & 65 & 24 & 11\\
  no data & 3 & 1 & \\
\lspbottomrule
\end{tabular}
\caption{Parent’s occupation and selection to frontline duties at JAMPRO}
\label{tab:4.2}
\end{table}

  Tests of association suggest that there is some relationship between family background and selection for a frontline position (p < .05, χ\textsuperscript{2} = 4.09), when categories are collapsed to cleaner\slash artisan and teacher\slash doctor.  Specifically, those who come from the lowest income\slash occupation bracket are under-represented in the ranks of \isi{frontline staff}; 76\% of the teacher\slash doctor group are in frontline positions.  This may be what \isi{senior management} refers to when it speaks of a preference for people of “certain backgrounds”, and the implication is that the requisite presentation for a \isi{JAMPRO} representative is least likely to be found in the employee who came from a low income\slash education household (indeed, only 8\% of \isi{frontline staff}).  Crucially, informants from the more affluent backgrounds are better educated, most likely to have tertiary level qualifications (\sectref{sec:3.3}).  It is therefore difficult to delink level of education from background and household.  “Background” is of course a loose term that attempts to capture both linguistic and cultural practice (see informant F64's comment on “sophistication”).  However, \isi{frontline staff} at \isi{JAMPRO} typically come from households of relative advantage, probably because they are, in this sample, also highly educated.\largerpage

\begin{table}
\begin{tabular}{lrrr}
\lsptoprule
 & \multicolumn{3}{c}{Transport (\%)}\\\cmidrule(lr){2-4}
 &                           Bus & Lift & Own Car\\\midrule
{+Frontline}               &   21 & 17 & 62\\
{\textminus Frontline}     &   46 & 30 & 24\\
no data                    &      &  2 & 2\\\lspbottomrule
\end{tabular}
\caption{Type of transport to work and selection to frontline duties at JAMPRO\label{tab:4.3}}
\end{table}

How one gets around in Jamaica is a fairly reliable measure of personal income and, to some extent, type of social contacts maintained.  The public transport system, certainly at the time of \isi{data collection}, was chaotic, crowded and unreliable.  Jamaicans, therefore, used the public bus system only if there was no other option, such as a lift to work.  The categories above can be used to reflect both access to income as well as interaction with other people of a certain income.  Informants who took the bus to work were not only low income earners, but they operated in networks with similarly low income earners.  Those informants who got a lift to work might not themselves have a car, but they knew people who did.  This suggests a somewhat different socio-economic situation, one with at least some personal contacts to people of a higher income.  Most \isi{frontline staff}, however, were able to buy and maintain their own car, and most of these would not have had access to \isi{JAMPRO}’s company car salary package.  This data supports the stated perceptions about the preference for “middle class” \isi{frontline staff} – they are better educated, from more affluent backgrounds and tend to have better incomes than non-\isi{frontline staff}  (p < .001, χ\textsuperscript{2} = 13.2).

\begin{table}
\begin{tabular}{lrrrr}
\lsptoprule
& \multicolumn{4}{c}{Age (\%)}\\\cmidrule(lr){2-5}
Position & 20--29 & 30--39 & 40--49 & 50--65\\\midrule
+Frontline           & 22 & 38 & 21 & 19\\
\textminus Frontline & 57 & 16 & 8  & 19\\
no data              &    & 3  &    & 1\\
\lspbottomrule
\end{tabular}
\caption{Age of informant and selection to frontline duties at JAMPRO}
\label{tab:4.4}
\end{table}

Grint (\citeyear[259]{Grint1991}, citing research by \citealt{Jenkins1985}) points out that those involved in exclusionary recruitment, which is what frontline hiring and promotion is about, tend to have a hierarchy of criteria for acceptance -- “the primary criteria involve appearance, manner (...) and maturity.  Secondary criteria relate to ‘gut feeling’ employment history (...) \isi{age}, speech style”.

But while middle-aged employees are most typical of those in the frontline (p\,<\,.001, χ\textsuperscript{2} = 14.01), this does not necessarily mean that \isi{JAMPRO} selects employees with most company experience.\largerpage

\begin{table}
\begin{tabular}{lrrr}
\lsptoprule
& \multicolumn{3}{c}{Years in \isi{JAMPRO} (\%)}\\\cmidrule(lr){2-4}
                 &  10+  & --10  &   New\\\midrule
{+Frontline}     &  33   &   38  &    29\\
{\textminus Frontline}    &  19   &   32  &    49\\
no data          &    2          &     2\\
\lspbottomrule
\end{tabular}
\caption{Years with the agency and selection to frontline duties at JAMPRO\label{tab:4.5}}
\end{table}

There is no statistically significant correlation between years in the agency and frontline status (p > .10, χ\textsuperscript{2} = 4.49), though there is a tendency to use more experienced workers.  Importantly, new recruits were hired into frontline positions, suggesting that having \isi{JAMPRO} experience is neither necessary nor sufficient for frontline employment.  Eighteen (18) members of staff were hired into posts that entail interacting with the public, notwithstanding their lack of agency experience.  Something in their interviews and qualifications prompted recruiters\slash management to do this.  Clearly, socialization to \isi{JAMPRO} linguistic or corporate norms is not a factor.  What may be at issue is socialization to educated linguistic norms, as evidenced by longer stay in the formal education system.  On closer examination, these 18 had the following profile: 

\begin{description}
\item[Education] post secondary 9 (50\%), tertiary 3 (17\%), graduate 6 (33\%);
\item[Parent’s background]  cleaner\slash artisan 6 (33\%), teacher\slash doctor 12 (67\%);
\item[Transport to work]  bus 6 (33\%), lift 2 (11\%), own car 10 (56\%);
\item[Age]  8 under 30 (44\%) and 10 over 30 (56\%);
\end{description}

As such, at least half of the new staff who entered the frontline ranks were university graduates, from relative affluence, had their own car and were over 30.  Most of them were hired into professional or managerial positions.  

Initially informants were asked to name the racial group to which they belong, an effort to collect self-ascriptive data.  For a number of reasons this proved to be problematic.  Some informants were reluctant to discuss the issue at all – typical responses were \textit{I} \textit{don’t} \textit{know} or \textit{you} \textit{put} \textit{what} \textit{you} \textit{think}.  This is not unusual, as other studies of \isi{race} in Jamaica have also recorded a similar discomfort with the subject.  As Alexander found in his survey:

\begin{quote}
	This ambivalence is expressed in the touchiness with which the subject is discussed.  Informants all framed the matter the same way: \isi{race} is a subject people do not discuss freely and openly; it remains understood (\citeyear[427]{Alexander1977}).
\end{quote}

Additionally, those that did answer did so in terms that would have been difficult for further sociolinguistic analysis.  The first 10 informants gave 5 different answers, such as \textit{Maroon,} \textit{dark,} \textit{mixed,} \textit{Jamaican} and \textit{Negro.}  Apart from an inability to correlate such a diversity of answers, it is not at all apparent that even when informants used the same term to describe themselves that they necessarily refer to the same thing.  It is entirely possible that two informants who describe themselves as “black” may have differing ideas of what [use of] that label means.  I found (\citeyear{Irvine1988}), for example, that one informant who described himself as “white” also included Chinese, Syrians and Lebanese people in the group as the label referred as much to socio-economic status as it did to \isi{race}.

In order to have some control over the number and semantic range of labels, I assigned informants to racial groups.  I am aware, however, that by doing so I am imposing my own perceptions of \isi{race} on the sample, and that this can yield misleading results.  Labov’s work (\citeyear[298]{Labov1972}) in New York City and his analysis of Italian-American vowel production can be used to illustrate the point.  Along similar lines, \citet{Horvath1987} found that refinements in method revealed a somewhat different picture of ethnic variation in Australian English than previously described.    

Informants in the \isi{JAMPRO} sample were grouped according to my perceptions of \isi{skin colour} as there is some historical and contemporary justification for believing that \isi{race} in Jamaica is essentially defined phenotypically: by colour, hair type and the like (see \citealt[20]{Lewis1968}; \citealt[24]{Nettleford1972}; \citealt{ }, 374; \citealt{Witter1980} for typical discussions).   The labels black and brown are used primarily to refer to dark brown and light brown \isi{skin colour} respectively.  

\begin{table}
\begin{tabular}{lSSS}
\lsptoprule
                      &     \multicolumn{1}{c}{Black}       &        \multicolumn{1}{c}{Brown}      &       \multicolumn{1}{c}{no data}\\\midrule
{+Frontline}           &      44.5         &        41.25        &         14.25\\
{\textminus Frontline}          &      65           &        16           &         19   \\
no data                &       1           &         3           &              \\
\lspbottomrule
\end{tabular}
\caption{Skin colour and selection to frontline duties at JAMPRO (\%)\label{tab:4.6}}
\end{table}

The data in \tabref{tab:4.6} shows that a smaller percentage of the informants who I view as “brown”, i.e. have lighter skin, are in non-frontline positions (p < .02, χ\textsuperscript{2}~=~6.52).  Put in another way, there are 32 brown informants for whom I have data on their position, 26 of them are in the frontline, and therefore more than four times the number behind the scenes at \isi{JAMPRO}.  In contrast, there are 52 black informants, with 28 in the frontline and 24 behind the scenes.  Brown people therefore are over-represented in the frontline of the agency. 

It is important to point out that staff with lighter \isi{skin colour} are not better educated; their selection cannot necessarily be explained as a preference for the highly educated in frontline positions (p > .50, χ\textsuperscript{2} = .31).   However, it is entirely possible that such a perception exists as part of the cultural traffic \citep[80]{Alvesson1993} that flows in from the attitudes held in the wider society.  In an earlier matched-guise study of 100 educated, affluent Jamaicans (\citealt[61]{Irvine1994}), I found that the guise of \isi{JE} judged to be that of a black person was also ranked lower on the intelligence\slash competence dimension.  And while not statistically significant, there seems to be some association between \isi{skin colour} and background\slash household (p > .10, χ\textsuperscript{2} = 1.68).  Brown members of staff are, for example, much less likely to come from households with parents who are “cleaners”.  

Arguably, brown members of staff are also less likely to come from monolingual JC households and consequently it is possible that an assumption is made that their use of \isi{JE} is not as dependent on formal education; moreover, they are more typically from the backgrounds that \isi{JAMPRO} seems to favour for \isi{frontline staff}. 

\begin{table}
\begin{tabular}{lrr}
\lsptoprule
   &         \multicolumn{2}{c}{Gender (\%)}\\\cmidrule(lr){2-3}
                       & Female & Male\\\midrule
          {+Frontline} & 78 &   22\\
{\textminus Frontline} & 78 &   22\\
              no data  &      4\\\lspbottomrule
\end{tabular}              
\caption{Gender and selection to frontline duties at JAMPRO\label{tab:4.7}}
\end{table}

The only specific reference to \isi{gender} and position in the company suggests that, for certain Asian markets, men are more likely to be selected as overseas officers and at the request of the host country’s agencies.  The data show men and women having virtually the same chances of being selected for the frontline, since \isi{gender} does not correlate with position in the agency (p > .95, χ\textsuperscript{2} = .004).  

The social factors that show some degree of association with membership in the ranks of \isi{frontline staff} are social class – as indexed by parent’s occupation and transport – \isi{age}, level of education attained and \isi{skin colour}.  The perceptions of staff and the statements of \isi{senior management} about the type of employee who tends to be in the frontline did mention background, class, colour and education.  I argue then that since there seems to be some validity to and empirical support for these perceptions, those that speak specifically to linguistic criteria are also to be taken as important.  Frontline staff are those employees whose English is believed to be suitable for a representative of \isi{JAMPRO}, displaying the attributes of diction, talk and English “at a standard”.

\subsection{{Group} {A} {variables} {in} {frontline} {staff}}%4.2.1n
\largerpage[1]
Along the \isi{Creole}\slash English dimension we would expect \isi{frontline staff} to use fewer \isi{Creole} forms than those not so selected.  The results for a number of Group A variables are in line with this expectation.  Frontline members of staff are much less likely to drop [h] (p < .001, χ\textsuperscript{2} = 38.54); they also produce fewer low central vowels in \textit{not} words and more of the \isi{JE} [ɔ] (p < .05, χ\textsuperscript{2} = 50.4).  For other Group A variables the results are somewhat less predictable, as some \isi{Creole} features are used more by \isi{frontline staff} or in a similar way by both classes of employee.

\begin{table}
\begin{tabular}{l *{4}{r@{ }r}}
\lsptoprule
Word Initial  &  \multicolumn{2}{c}{d} & \multicolumn{2}{c}{ð} & \multicolumn{2}{c}{t} & \multicolumn{2}{c}{θ}\\\midrule
Frontline     & 1472 & (52\%) & 1373 & (48\%) & 63 & (12\%)  &  478 & (88\%)\\
Non-frontline & 818  & (37\%) & 1369 & (63\%) & 54 & (17\%)  &  269 & (83\%)\\
\lspbottomrule
\end{tabular}
\caption{Word initial TH stopping in frontline staff\label{tab:4.8}}

\end{table}

Voiceless TH stopping, which I argue is the more indexical variable in this pair, is produced less by \isi{frontline staff}, in line with h-dropping and the \isi{low vowel} (p~<~.05, χ\textsuperscript{2} = 4.43).  The hypercorrect use of these three features in \isi{JE} speakers would suggest a sensitivity to their use and their importance to any idea of \isi{good English}.  But voiced TH stopping is more apparent in the same select speakers (p~<~.001, χ\textsuperscript{2} = 102.44), which suggests that less attention is paid to its production in a context of spoken \isi{JE} (although I would predict that in a word list test the results would be very different).  Arguably, the more speakers are cued to the variants that matter – here the [θ] – there is less of a requirement to pay attention to the ones that do not.  Certainly, voiced TH stopping does not seem to preclude promotion to the frontline.   

A similar pattern is evident in words beginning with a \isi{velar stop}.\largerpage

\begin{table}
\begin{tabular}{l *{4}{r@{ }r}}
\lsptoprule
&  \multicolumn{2}{c}{kja}   &  \multicolumn{2}{c}{k\textsuperscript{h}a}   &  \multicolumn{2}{c}{kja:}   &  \multicolumn{2}{c}{k\textsuperscript{h}a:} \\
\midrule
Frontline     & 77 & (56\%) & 61 & (44\%) & 9  & (13\%)  & 60 & (87\%)\\
Non-frontline & 42 & (59\%) & 29 & (41\%) & 10 & (37\%)  & 17 & (63\%) \\\cmidrule(lr){2-5}\cmidrule(lr){6-9}
     &  \multicolumn{4}{c}{(p > .50, χ\textsuperscript{2} = .19)} &  \multicolumn{4}{c}{(p < .01, χ\textsuperscript{2} = 6.97)}\\
\lspbottomrule
\end{tabular}
\caption{Word initial velar stops in frontline staff\label{tab:4.9}}
\end{table}

Frontline staff produce much fewer instances of [kja\textbf{:}] than do those in the background; but use of [kja] does not distinguish the two groups, and it is freely varied, as has been the case in nearly all correlations with this feature in this study.  Three possible conclusions can be drawn from this.  The first is that some \isi{Creole} features are perfectly acceptable in spoken \isi{JE} and have been normalized as such (as are [kja] or [d {\textasciitilde} ð]).  The second is that [kja] and [d] are not necessarily perceived to be \isi{Creole} features.  The third is that [kja] and [d] are peripheral to the issue of speaking \isi{JE}.  As such, their use by a speaker who consistently produces say [θ] is of little import to members of the speech community when making judgements about “good” \isi{JE}.  When speaking with another Jamaican, as was the case in their interviews both for selection to frontline duties and for this study, informants’ use suggest [kja] is an aspect of Jamaican speech attested to in spoken formal \isi{JE}.

 
\begin{table}
\begin{tabularx}{\textwidth}{l X *{2}{r@{ }r} X *{2}{r@{ }r}}
\lsptoprule
 &  & \multicolumn{2}{c}{uo}  &  \multicolumn{2}{c}{o}  & &  \multicolumn{2}{c}{ie}  &  \multicolumn{2}{c}{e} \\\midrule
Frontline     & & 87 & (8\%) &   960 & (92\%) & & 141 & (12\%) & 994 & (88\%) \\
Non-frontline & & 42 & (7\%) &   537 & (93\%) & & 114 & (18\%) & 522 & (82\%) \\\cmidrule(lr){3-6}\cmidrule(lr){8-11}
      & &  \multicolumn{4}{c}{(p > .30, χ\textsuperscript{2} = .56) }    &        &  \multicolumn{4}{c}{(p < .01, χ\textsuperscript{2} = 10.04)}\\                      
\midrule
\end{tabularx}
\resizebox{\textwidth}{!}{\begin{tabular}{l *{5}{r@{ }S}}
 &  \multicolumn{6}{c}{\textit{poor} words} &  \multicolumn{4}{c}{\textit{beer} words}\\
 &  \multicolumn{2}{c}{[ɔ]}    &  \multicolumn{2}{c}{[u\textsuperscript{o}]}   &  \multicolumn{2}{c}{[o]}  &  \multicolumn{2}{c}{[i\textsuperscript{e}r]}    &  \multicolumn{2}{c}{[er]}\\\midrule
Frontline     & 303 & (73.0\%) & 77 & (18.0\%) & 37 & (9.0\%) & 316 & (42.5\%) & 426 & (57.5\%)\\
Non-frontline & 127 & (72.5\%) & 35 & (20.0\%) & 13 & (7.5\%) & 149 & (52.0\%) & 138 & (48.0\%)\\ \cmidrule(lr){2-5}\cmidrule(lr){6-11}
      &  \multicolumn{4}{c}{(p > .80, χ\textsuperscript{2} = .42)}         &  \multicolumn{6}{c}{(p < .01, χ\textsuperscript{2} = 7.29)}\\
\lspbottomrule
\end{tabular}}
\caption{Mid vowels in frontline staff\label{tab:4.10}}
\end{table}

  \tabref{tab:4.10} shows that the \isi{back diphthong} is seldom used by any speakers either pre-consonantally or before [r].  This would confirm the theory, presented at the end of the previous chapter, that back \isi{diphthong use} is the more \textit{linguistically} \textit{indexical} variable of the pair -- a necessary element for being perceived as speaking good \isi{JE}.  The \isi{front diphthong}, however, is less likely in the speech of \isi{frontline staff}, as would be expected from previous results that also support the view that [ie] use is more \textit{socially} \textit{indexical} in \isi{JE} than [uo].   Front \isi{diphthong use} is more apparent in the young, a group under-represented in the ranks of \isi{frontline staff}.  

\subsection{{Group} {B} {variables} {in} {frontline} {staff}}%4.2.2n
\largerpage
The statistical results for \textit{butter} type words point to the difference here being who uses an [a] as opposed to an [ʌ] ending, along the \isi{Creole}\slash English dimension characteristic of Group A variables (p < .001, χ\textsuperscript{2} = 85.9).  As we would expect, \isi{frontline staff} use less of the \isi{Creole} variant.  A similar analysis can be made for that aspect of \tabref{tab:4.11}, but the [\textbf{ʃɔn} {\textasciitilde ʃǝn}] variation is much more interesting.

\begin{table}
\begin{tabular}{l *{4}{r@{ }r}}
\lsptoprule
&  \multicolumn{2}{c}{educa\textbf{[ʃan]}}  &  \multicolumn{2}{c}{educa\textbf{[ʃʌn]}}        &  \multicolumn{2}{c}{educa\textbf{[ʃɔn]}}  &  \multicolumn{2}{c}{educa\textbf{[ʃǝn]}}    \\
\midrule
Frontline     & 69 & (14\%) & 286 & (57\%) & 103 & (21\%) & 41 & (8\%)\\
Non-frontline & 80 & (38\%) & 91  & (43\%) & 17  & (8\%)  & 24 & (11\%)\\
\lspbottomrule
\end{tabular}
\caption{Articulation of final – \textit{tion} in frontline staff  (p < .001, χ\textsuperscript{2} = 61.8)\label{tab:4.11}}
\end{table}

Frontline staff are more likely to use the [\textbf{ʃɔn}] that is associated with higher education and not the schwa typical of \isi{MSE}.  Clearly here speakers are being selected for interaction with the public not only on their avoidance of stigmatized items, but also on their production of this \isi{JE} feature, given the peripheral status of schwa in this speech community.  This feature was also correlated with level of education, and was more typical of women.  Within this sample, [\textbf{ʃɔn}] is a variant in the speech of most, if not all, groups that can be used to locate standard\slash prestige forms.     

\begin{table}
\begin{tabular}{l *{4}{r@{ }r}}
\lsptoprule
&  \multicolumn{2}{c}{dj} &  \multicolumn{2}{c}{ʤ} &  \multicolumn{2}{c}{tj} &  \multicolumn{2}{c}{ʧ}  \\
\midrule
Frontline     & 25 & (76\%) & 8 & (24\%) & 49 & (32\%) & 102 & (68\%) \\
Non-frontline & 9 & (75\%)  & 3 & (25\%) & 24 & (44\%) & 31  & (56\%) \\
\lspbottomrule
\end{tabular}
\caption{Articulation of \textit{culture} type words in frontline staff (p > .10, χ\textsuperscript{2} = 2.04)\label{tab:4.12}}
\end{table}

Use of either the voiceless palatalized stop or the \isi{affricate} in \textit{culture} type words does not significantly distinguish frontline from non-\isi{frontline staff}.  Variation in this feature was correlated with \isi{gender} and level of education – the highly educated favouring the \isi{affricate} and women more likely to use the palatalized stop variant.  As frontline positions seem not to be allocated on the basis of \isi{gender}, the weak tendency for \isi{frontline staff} to use the \isi{affricate} more frequently can be explained by their higher levels of education.\largerpage

\begin{table}
\begin{tabular}{l *{2}{r@{ }S} *{2}{r@{ }r}}
\lsptoprule
&  \multicolumn{4}{c}{\textit{party} type words}    &  \multicolumn{4}{c}{\textit{forty} type words}\\\cmidrule(lr){2-5}\cmidrule(lr){6-9}
&  \multicolumn{2}{c}{[partɪ]} &  \multicolumn{2}{c}{[pa:tɪ]} &  \multicolumn{2}{c}{[fɔrtɪ]}  &  \multicolumn{2}{c}{[fɔ:tɪ]} \\
\midrule
Frontline     & 69 & (41.5\%)  & 97 & (58.5\%) & 134 & (60\%)  & 89 & (40\%)\\
Non-frontline & 39 & (46.0\%) & 46 & (54.0\%) & 67  & (79\%)  & 18 & (21\%) \\\cmidrule(lr){2-5}\cmidrule(lr){6-9}
&  \multicolumn{4}{c}{(p > .50, χ\textsuperscript{2} = .42)}  &  \multicolumn{4}{c}{(p < .01, χ\textsuperscript{2} = 9.53)}\\
\lspbottomrule
\end{tabular}
\caption{Rhoticity in frontline staff\label{tab:4.13}}
\end{table}

  While there is no statistically significant difference in post [a] \isi{rhoticity} in the two groups of employees, rhoticization of \textit{forty} type words, though clearly the norm in this total sample, is less a feature of \isi{frontline staff}’s speech.  Recall that \isi{rhoticity} after [ɔ] was less a feature of older speakers in this sample (see \tabref{tab:3.53} for example).  Moreover, this feature was seen to be normalised in \isi{JE} among younger speakers and women.  Again, persons in frontline positions tend not to be younger members of staff.  The data in total suggests an idea of a rhotic \isi{JE}, given both the quantitative and \isi{qualitative hypercorrection} identified.  But it also suggests that frequency of rhotic productions differentiate varying groups in the sample, with younger and educated female speakers generally the most rhotic of the groups in both phonetic environments and younger educated males being less rhotic after [a].     

  Frontline staff are more consistent in their use of all phonological clusters here than non-\isi{frontline staff}.  In particular, while [nt] clusters before a vowel tended to be generally used in the \isi{JAMPRO} sample, \isi{frontline staff} also produce more [st] clusters and clusters before a following consonant (p = .05, χ\textsuperscript{2} = 3.8).    Use of clusters with morphological content does not distinguish frontline and non-\isi{frontline staff}.  

\section{Discussion}\label{sec:4.3}

The patterns for \isi{frontline staff}, the selected voice(s) of \isi{JAMPRO}, suggest that certain aspects of language use are normalised as \isi{good English}.\largerpage

% \subsection{Group A Variables (Frontline speakers)}

\begin{table}[p]
\begin{tabular}{l r *{2}{rr}}
\lsptoprule
		&				   & \multicolumn{4}{c}{Variant}\\\cmidrule(lr){3-6}
Feature & Informants taped & \multicolumn{2}{c}{Never use}  & \multicolumn{2}{c}{Always use}\\
\midrule
/h/                         & 51 & 0 &       & 24\cellcolor{lsLightGray} & (47\%)\cellcolor{lsLightGray} \\\relax
{Initial[ð]}                & 51 & 1 & (2\%) & 0  &        \\\relax
Initial [θ]                 & 51 & 1 & (2\%) & 28\cellcolor{lsGuidelinesGray} & (55\%)\cellcolor{lsGuidelinesGray} \\\relax
{[ɔ}] \textbf{\textit{not}} & 51 & 0 &       & 1  &  (2\%) \\\relax
{[o] \textit{boat}}          & 51 & 0 &       & 19 & (37\%) \\\relax
{[e] \textit{face}}          & 51 & 1 & (2\%) & 20 & (39\%) \\\relax
{[ɔr] \textit{poor}}         & 50 & 2 & (4\%) & 17 & (34\%) \\\relax
{[ber] \textit{beer}}        & 51 & 1 & (2\%) & 2  &  (4\%) \\\relax
{[k\textsuperscript{h}a:]}  & 33 & 2 & (6\%) & 26\cellcolor{lsGuidelinesGray} & (79\%) \cellcolor{lsGuidelinesGray}\\
\lspbottomrule
\end{tabular}
\caption{Group A variables and distribution in frontline staff\label{tab:4.14}}
\end{table}

Significant numbers of speakers never \isi{h-drop}, use voiceless TH stopping or palatal velar stops before long vowels.  And, to a lesser extent, this also applies to diphthongs, though it appears that [ie] is more and more an aspect of younger speakers who do pass the “interview” test.  In its expectation and practice, this agency is communicating to its employees that these features of Jamaican speech [h], [θ] and [k\textsuperscript{h}a\textbf{:}] are necessary linguistic aspects of those that do well; at the same time features like [d {\textasciitilde} ð] or [kja] do not preclude advancement or employment.  Indeed they are also features of the successful employee.  In this way, by appointing some employees and excluding others, \isi{JAMPRO} is engaged in constructing and legitimizing an ideology of what are the spoken norms of Standard Jamaican English.

% \subsection{Group B Variables (Frontline speakers)}

\begin{table}[p]
\begin{tabular}{l r rr rS}
 \lsptoprule
 &				   & \multicolumn{4}{c}{Variant}\\\cmidrule(lr){3-6}
 Feature & Informants taped & \multicolumn{2}{c}{Never use}  & \multicolumn{2}{c}{Always use} \\
 \midrule\relax
{-er\textit{butter}}   & 51 & 20 & (39\%) & 0 & \\\relax
{[ʃɔn]\textit{-tion}}  & 51 & 19 & (37\%) & 0 & \\\relax
{[tj]\textit{culture}} & 41 & 17 & (41\%) & 12 & (29.0\%)\\\relax
{[r]\textit{party}}    & 49 & 19 & (39\%) & 9  & (18.0\%)\\\relax
{[r]\textit{forty}}    & 48 & 5  & (10\%) & 12 & (25.0\%)\\\relax
-nt\#\# C              & 50 & 6  & (12\%) & 8  & (16.0\%)\\
-st\#\# V              & 45 & 8  & (18\%) & 7  & (15.5\%)\\
-st\#\# C              & 50 & 25\cellcolor{lsGuidelinesGray} & (50\%)\cellcolor{lsGuidelinesGray} & 2  & (4.0\%)\\
n’t\#\# C              & 48 & 17 & (35\%) & 0 &    \\\relax
{-ed\textit{beforeC}}  & 43 & 10 & (23\%) & 6  & (14.0\%)\\
\lspbottomrule
\end{tabular}
\caption{Group B variables and distribution in frontline staff\label{tab:4.15}}
\end{table}

  Most speakers who are \isi{frontline staff} produce affricates in \textit{culture} type words, bimorphemic clusters and [nt], at least before a following vowel segment.  But the data here also suggests that saying [dõ wʌrɪ] \textit{don’t} \textit{worry} or [la:s je:r] \textit{last} \textit{year} is not unacceptable in a speaker who has been selected by the agency to represent it to the public.  Of course, a discussion in terms of  presence\slash absence of features does not reflect the complexity of the situation.  While it does show which features are generally widely used or avoided, it cannot suggest the importance of, say, using [ɔ] rather than [a] in \isi{JE}, whether in stressed or unstressed syllables, or the asymmetrical salience of one variable in a pair of related linguistic variables in the sample.\newpage 

Frontline staff are selected because they are perceived to be staff who can ``represent well at a standard", and their behaviour provides a concrete model of spoken \isi{JE}, which at least this institution considers representative of what it referred to as \isi{Standard English}.  Unlike the education system, \isi{JAMPRO} is not engaged in providing a model for imitation.  Rather, the language use of \isi{frontline staff} here, in this formal interview context, is a sampling of speech from Jamaicans who are considered by other Jamaicans to be speakers of \isi{good English}.  

The individual profiles below are examples of some frontline speakers who used the fewest \isi{Creole} forms in their interviews.

% \subsection{Group A Variables}

\begin{table}[p]
\begin{tabular}{l r S[table-format=2.1] r >{\shadecell}r}
\lsptoprule
         & \multicolumn{4}{c}{\%}\\\cmidrule(lr){2-5}
 Feature & M34 &  \multicolumn{1}{c}{F16} &  F11 &  M40 \\
\midrule
\isi{h-drop} & 0 & 3 & 0 & 0\\
Voiced TH stop & 30 & 16 & 19 & 100\\
Voiceless TH stop & 0 & 0 & 11 & 100\\
{[a] \textit{not}words} & 0 & 9 & 7 & 67\\
{[uo] \textit{boat}words} & 6 & 8 & 6 & 0\\
{[ie] \textit{face}words} & 0 & 0 & 0 & 100\\
{[\textsuperscript{u}or] \textit{poor}words} & 0 & 44.5 & 0 & 100\\
{[bier] \textit{beer}words} & 33 & 25 & 28 & 0\\
{[kja] \textit{cat}words} & 75 & 0 & 67 & no data\\
{[kja:] \textit{cart}words} & 0 & 0 & 0 & 100\\
\lspbottomrule
\end{tabular}
\caption{Group A Variables and distribution in 4 frontline speakers.\label{tab:4.16}}
\end{table}

% \subsection{Group B Variables}

\begin{table}[p]
\begin{tabular}{l *{3}{r} >{\shadecell}r}
\lsptoprule
         & \multicolumn{4}{c}{\%}\\\cmidrule(lr){2-5}
 Feature & M34 & F16 & F11 & M40 \\
\midrule\relax
{[a]\textit{butter}words} & 0 & 0 & 0 & 86\\\relax
{[ʃan]\textit{-tion}words} & 7 & 0 & 0 & 100\\\relax
{[ʃɔn]\textit{-tion}words} & 21 & 40 & 82 & 0\\\relax
{[ʃǝn]\textit{-tion}words} & 7 & 20 & 0 & 0\\\relax
{[tj]\textit{culture}words} & 100 & no data & 100 & 100\\\relax
{[r]\textit{party}words} & 20 & 0 & 100 & 0\\\relax
{[r]\textit{forty}words} & 33 & no data & 100 & no data\\
-nt\#\# before V & 33 & 67 & 100 & no data\\
-nt\#\# before C & 33 & 33 & 0 & 50\\
-st\#\# before V & 100 & 67 & 100 & no data\\
-st\#\# before C & 33 & 40 & 0 & no data\\
-n’t\#\# before V & no data & 100 & no data & no data\\
-n’t\#\# before C & 14 & 33 & 0 & 0\\
-ed before V & 100 & 100 & 100 & 100\\
-ed before C & 50 & 100 & 33 & no data\\
\lspbottomrule
\end{tabular}
\caption{Group B Variables and distribution in 4 frontline speakers.\label{tab:4.17}}
\end{table}

It would be difficult to argue that any one of the 3 unmarked speakers (M34, F16, F11) is more acrolectal than the other, or that one could be placed on a “higher” level on the continuum than some other one of these informants.  M34, for example, is unique in my sample.  He was categorical in his use of [ɔ] in \textit{not} words, he seldom produced diphthongs, in fact 2 instances in 31 tokens, articulated [st] clusters (before vowels) and so on.  Is he to be placed on a “higher” level on the continuum than F16 for example, given M34's use of [kja] or [tj]\slash [dj] or [dõ]?  If one takes \isi{MSE} as point of reference for Jamaican linguistic norms, then the answer must be no.  In F16's interview she is categorical in her use of morphological clusters and in line with other \isi{MSE} speakers in her use of phonological clusters.  But she also h-drops more and uses diphthongs more often in most environments.  Her use of variants, if one takes an \isi{endonormative approach}, may well be evaluated by others as less acrolectal than his because of the particular features that are found in her speech.  But she is a highly educated female and from a household with educated parents; he is a less educated man.  Their language use is likely to be filtered through the social perceptions that others in the society hold of members of those groups.  \citet{Kulick1998} suggests that

\begin{quote}
	...language ideologies seem never to be solely about language – they are always about entangled clusters of phenomena, and they encompass and are bound up with aspects of culture like \isi{gender}, and expression, and being “civilized” (100).
\end{quote}

H-drop in the driver M40's speech, for example, is not therefore sociolinguistically the same as it is in, say, speaker F16.  And it would be an interesting extension of this study to see if the occasional \isi{h-drop} in someone like F16 is even “heard” by the listener.  M40 is a speaker who typically uses \isi{Creole} phonology.  The exceptions are the \isi{back diphthong} [uo] and \isi{h-drop}.  These features were both infrequent in my total sample, and I have classified them both as two of the load-bearing variables in Jamaica.  M40 also uses [tj] in \textit{culture} words and past tense inflections.  In that respect, there are aspects of his speech that index “better English” and this may suggest why he has been selected for a frontline position.{\interfootnotelinepenalty=10000\footnote{In the ideology of the Jamaican middle-class \citep[236]{Austin1983} a position like driver\slash chauffeur cannot be held by someone at the professional level in the agency or a woman and therefore must be occupied by an employee like M40.}}

  The data from \isi{frontline staff}, the selected public face of \isi{JAMPRO}, is taken from speakers who are there because they have been judged suitable to represent the construct of the language situation described by the agency.  Patois and \isi{Standard English} are the varieties described in these publications, the latter, I would suggest, reflected in the formal usage of the majority of professionals appointed to the frontline ranks.  And the features present or more frequent in this majority represent what is good spoken \isi{SJE}, the \isi{acrolect} in this speech community.  In its expectation for staff and its practice of promotion, \isi{JAMPRO} is legitimizing norms of speech, necessarily mediated by the ideas of those who control it about who the speaker is and what place they are expected\slash believed to occupy in the society.  That the norm is essentially reflected in educated \isi{female speech} at \isi{JAMPRO} is also a function of the social context and the place women occupy in that agency.    
