\chapter{Phonological variation in the Jamaican acrolect}\label{ch:2}
\section{Why consider phonological variation?}%2.1

This study is concerned with the formal, spoken language of a sample of educated Jamaicans, the Jamaican \isi{acrolect} as explained in \chapref{ch:1}.  The informants used in this study all work for \isi{JAMPRO}, an agency that has explicit language requirements for staff, and this is reflected in the various language qualifications prospective employees are expected to have.  This agency, as outlined in \chapref{ch:4}, also describes in its publications a construct of the language situation and of the socio-cultural milieu in which it operates in Jamaica.  In both descriptions, the use of \isi{Standard English} is stated to be an imperative.  As such, there are a number of reasons why the analysis of language variation in the speech of \isi{JAMPRO} informants is necessarily going to focus on phonology, the essential concern of this book. 

      In general, qualifications in English are required for non-manual employment in Jamaica. Certainly, nearly all advertisements for white-collar jobs require applicants to have “a good command of the English language” or “excellent written and oral skills” (in English, by implication, since JC is not normally written) or “proficiency in English”.  Additionally, announcements of these positions are accompanied by phrases such as “the ideal candidate should have...” – typically a university degree, a credit in CXC English, or other such stipulations about educational qualifications.  These qualifications are, for the most part, locally or regionally attained and success is judged through these examinations.  

 The CXC (and the national) exams, designed to assess proficiency in English, have no equivalent test of \textit{oral} communication on leaving the school system.\footnote{The \textbf{C}aribbean \textbf{A}dvanced \textbf{P}roficiency \textbf{E}xaminations did introduce an oral English examination in 1995, for students at the sixth form level.}   Employers, therefore, rely on interviews, particularly, but not exclusively, for those candidates to supervisory or management positions.  

A number of comments by \isi{senior management} at \isi{JAMPRO} can be used to reveal this focus on the interview and its importance in selecting staff.

\begin{quote}
	...there was the question of why is it that we could not find any [men] at the interview and selection.  We always came up with women and we were doing it strictly on the criteria that we set for recruitment...one of the first criterion (sic) for employment is a first degree from CAST [now the University of Technology] and any other tertiary institution.  I think they were so much better at the interview.  The women are qualified...am...I wouldn’t say better qualified, wouldn’t say better qualified, in fact...at the stage of the interview I ask is this person somebody I could send away next week to go and talk to an investor or to go and sell Jamaica. (F55).\footnote{The code here is to be interpreted as M = male, F = female; the numbers represent the sequence of interviews.  This was then the 55\textsuperscript{th} interview.  Appendix~\ref{appendix2} has profiles of all informants.}
\end{quote}

\begin{quote}
	...the women did better, they were more reliable, more conscientious in their work.  And even in interviews the women seem to shine more than the men, so invariably they ended up employing more women than men (F87).
\end{quote}

\begin{quote}
	It seems to me that they [women] seem to do better at the interviews...(F56)
\end{quote}

These comments clearly demonstrate the perceptions held by \isi{senior management} about male and female suitability for \isi{JAMPRO} employment (see \sectref{sec:3.2}).  But they also reveal that the interview, face to face and spoken interaction with the would-be employee, is crucial.  Candidates for \isi{JAMPRO} jobs are short-listed based on their paper qualifications; but it is in the interview that the selection is made, other qualifications notwithstanding. 

In the interview, and in other contexts of spoken interaction, attention usually focusses on phonology and lexis when standards (in both the linguistic and denotative sense) are being assessed.  \citet[44--45]{Hudson1980} proposes that within communities alternatives in phonology are more likely to mark social differentiation than those in syntax or morphology, particularly as mass education has increased the spread of a (written) standard (see also \citealt{Lippi-Green1997}; \citealt{Sahgal1985}).  It is true that broadly speaking in communities like Jamaica, differences in morphology and syntax will sharply distinguish social groups across the continuum, as in \textit{im ben a sliip} {\textasciitilde} \textit{shi did a sleep} {\textasciitilde} \textit{she was sleeping.} However, Hudson’s proposition would apply in the context of a formal interview, particularly for a white-collar position, where the expected variety is \isi{SJE}.  

Salikoko Mufwene (pers. comm.) relates an example in support of this.  Students were asked to listen to two speakers, one an elderly woman and the other a middle-aged man, both of them comedians, and to judge which seemed more non-standard.  There was general agreement that the man was the more non-standard of the two, even though he used a more standard morphology and syntax.  His phonology (especially the prosody), however, seemed to inform the listener’s judgement of his standardness, as little attention was paid to his grammar until it was brought to the students’ attention. 

Further, \citegen[275]{Rickford1987} examples of \isi{Guyanese} English related in the previous chapter, involving the speech of a teacher and of a barrister, are distinctive only for the phonology.   The informants used in this book are Jamaicans who have mastered this written standard English well enough, as all but four have completed at least a secondary level of education, and many have a tertiary or graduate level education.  \isi{JAMPRO} would not have otherwise considered them for employment.     

As suggested above by \citet[370]{Gupta2001}, and as already shown in sociolinguistic research, particularly for English speaking communities (\citealt{Hewitt1986}; \citealt{Labov1972}; \citealt{Mugglestone1995}; \citealt{Trudgill1978}), there are some accents that are considered prestigious, as markers of identity, or social class membership or level of education.  While certain accents, such as \isi{RP}, may have historically been given high status socially, it is not clear that they still universally carry such associations.  \citet{Milroy2002} suggests, for example, that in the UK it is apparent that \isi{RP} “does not constitute the general model of careful or educated speech” (9) for many regions whose populations favour local patterns.  Outside of the UK, as shown in \citegen{Thaxter1977} Jamaican study, speakers also favour the local voice, the \isi{RP} \isi{accent} assessed as foreign and\slash or pretentious (239).  And \citet[47]{Nair-Venugopal2001} reports that in \isi{Malaysia} exonormative models of English might be viewed as nonconventional or unexpected in corporate business contexts.

The idea of prestige \isi{accent}(s) is therefore locally developed and particular to the specific speech community in which speakers operate.  It is within the speech community that the shared indexicality underlying communicative acts of identity and groupness evolves \citep[407]{Silverstein1998}.  This would necessarily seem to be most focussed around the phonology, especially for speakers who pass through institutions that disseminate a near universal \isi{literary standard} of English.  

There is evidence that Jamaicans perceive \isi{accent} and vocabulary also to be the primary difference between \isi{Creole} and English, at least as suggested by Beckford-Wassink’s data (\citeyear[66]{Beckford‑Wassink1999b}).  Of 51 informants in her study, only 9 (18\%) identified any aspect of morphosyntax as distinguishing \isi{Creole} and English, while 42 (82\%) mentioned either \isi{accent} (``how they sound their words") or \isi{accent} and vocabulary.  And, Beckford-Wassink adds,\largerpage[-1]

\begin{quote}
	All the professional domains about which they were questioned (those\linebreak where the auditor might be a non-familiar or subordinate, as when answering the telephone, addressing an employer, or teaching) were deemed inappropriate for Patois usage.\hbox{}\hfill\hbox{\citep[72]{Beckford‑Wassink1999b}}
\end{quote}

\citet{Muhleisen2002} reports a similar functional distribution of language use among Trinidadians.  What respondents referred to as ``Standard" (\isi{Trinidadian} English) was mostly used in formal contexts (classroom, workplace, church) and with strangers (40).  The speakers I recorded at \isi{JAMPRO} did find themselves in such a context, interacting with a stranger in a formal interview at their workplace.  There is every reason to expect them to be using the appropriately formal style, with the focussed morphosyntactic norms typical of the \isi{acrolect}.  It is their phonology, therefore, that is of primary interest here.   

\section{The phonological variables and their variants}%2.2

This study, which takes an \isi{endonormative approach} to defining the \isi{acrolect} in Jamaica, will therefore analyse use of a set of segmental phonological features in the \isi{JAMPRO} sample.  The phonological features are as follows:
\begin{enumerate}[label=\alph*)]
\item  the word initial \isi{glottal fricative} [h] 
\item  the voiced and \isi{voiceless interdental fricative}, in word initial, middle and final environments. 
\item  the low back stressed vowel, in words like \textit{not} and \textit{possible}.
\item  the mid tense vowels [e] and [o] as they occur pre-consonantally, in items like \textit{face} and \textit{goat,} and before [r] in \textit{beer} and \textit{poor} type words. 
\item  the word initial \isi{velar stop} [k] before the \isi{low central vowel} [a].
\item  post vocalic \isi{rhoticity}, before [+coronal] consonants in words like \textit{party} and \textit{forty}. 
\item  two word final phonological stop clusters, specifically [st] and [nt].
\item  two word final morphophonemic clusters, -n’t when used as not (\textit{can’t,} \textit{won’t,} \textit{don’t}); and past \isi{tense marking}.
\item  the word final unstressed vowel in words that end in \textit{-er} (\textit{butter}, \textit{teacher}).
\item  the vowel in the final syllable -tion, as in \textit{education} type words.
\item  the voiced and voiceless \isi{alveopalatal affricate}, in words like \textit{culture} and \textit{soldier}. 
\end{enumerate}


I attempted to include all features that showed phonological variation in my data.  Most of these features were present in my data in adequate frequencies.  Where they were not, the data that can be discussed is presented.  For example, \isi{rhoticity} is analysed before certain kinds of consonants, [+ coronal], because there was, in part, better data in that phonetic environment than before [\textminus coronal] consonants.  

The literature on Jamaican language varieties has identified some of these phonological features as having variants that are described as either typically, if not unequivocally, basilectal or mesolectal and has not included them as part of the \isi{acrolect}.  These are presented in Group A, i.e. features a–e above.  Additionally, there were some features that I was interested in exploring, and which I have not seen specifically discussed elsewhere in the literature.  These are categorized as Group B, i.e. f–k above.  They might prove interesting in this sociolinguistic analysis.  As a Jamaican, and as someone who considers herself a \isi{vernacular} speaker of Jamaican English, I recognize all of these features in my speech community.  While phenomena like h-dropping and voiceless TH stopping are possibly now stereotypes (in the sense used by \citealt[248]{Labov1972}) in Jamaica, pronunciations such as [kʌltjʌ] \textit{culture} are perhaps less in the consciousness of speakers.  I wish to explore how widely they are in use, particularly in the \isi{formal speech} of educated Jamaicans.  The middle three features, in particular (f–h), have been described elsewhere for Jamaica.  However, there has been little focus on their use by acrolectal speakers (but see Shields’s \citeyear{Shields1987,Shields1984} study of consonant clusters in \isi{JE}).  

The results for the total \isi{JAMPRO} sample, all 82 informants, will therefore be presented for each feature discussed, so that some comparison can be made with the descriptions in the literature and generally.   


\section{Features commonly identified in the literature (Group A)}\label{sec:2.3}%2.3

Generally, the discussion of these features across the \isi{Jamaican continuum} is typified by the following description by \citeauthor{Wells1982c}: 

\begin{quote}
At the upper (acrolectal) end of the social and linguistic scale, Jamaican consonants are phonetically much as in the standard accents (${\ldots}$) Further down the scale, in the \isi{mesolect} and \isi{basilect}, the characteristics emerge: TH stopping, Cluster reduction, avoidance of [ʒ] and some [v], H Dropping, semivowels in words such as /kjat/ \textit{cat} (${\ldots}$) \textit{face} acrolectally [fes] and basilectally [fiɛs]...
\hfill\hbox{(\citeyear[575--576]{Wells1982c})}\end{quote}

Acrolectal varieties in this description are characterized by very low frequencies, if not an absence, of certain variants; it is, in fact, this absence that distinguishes them from mesolectal and basilectal varieties.  Interestingly, Wells says that the \textit{phonetic} properties of acrolectal varieties are much as in other standard accents.  Clearly, an American, \isi{Canadian} and British speaker (or an Indian, an Australian and a Jamaican) will not sound the same.  This analysis suggests, therefore, that Wells is either speaking of an idealization of \isi{Standard English} that disregards even regional variation in spoken norms or is locating much of the distinctiveness of the \isi{acrolect} in prosody.  As discussed above, there is no standard English \isi{accent}.  Spoken English is generally outside the normative pressure exerted by the \isi{literary standard} and it is typically the phonetic characteristics of the educated speaker that distinguish varieties from different speech communities.  My concern here is then phonetic, the distribution of the variants of these variables in formal spoken \isi{JE}.  As was pointed out in the previous chapter, very little investigation of what has been called “the \isi{local standard}” or \isi{acrolect}, as used by speakers, has been done.  I will therefore use the term Putative Jamaican Standard (\isi{PJS}) to label an unverified but presumed \isi{local standard} in order to distinguish it from the \isi{SJE} which is the subject of the research in this book. 

As point of reference I include the following phonemic inventory of JC and \isi{JE} in Tables~\ref{tab:2.1}--\ref{tab:2.2}.  The JC \isi{vowel inventory} is based on \citet[35--42]{Meade2001}.  Meade’s inventory discusses a number of previous texts\footnote{Meade’s work discusses, for example, \citet{Akers1981,Alleyne1980a,Cassidy1967,Beckford-Wassink1999a,DevonishSeiler1991}.}  and his own work done on \isi{language acquisition} in Jamaica.  The \isi{JE} \isi{vowel inventory} is based on Beckford-Wassink’s data (\citeyear[150--151]{BeckfordWassink2001}), which discusses earlier work by \citet{Wells1973}.  The consonants of both systems are taken from Meade’s work.  As such these inventories can provide some indication of what phonological variation may occur in this \isi{JAMPRO} sample as well as an example of what is idealized at the polar ends of the \isi{Jamaican continuum}.  While both Meade and Beckford-Wassink use similar symbols for most of the phonetic representations of pho\-nemes, one or two differences did occur.  Meade, for example distinguishes [a] and [ɐ] in JC, which is more typically presented as [a] in the literature and by Beckford-Wassink.  I have therefore taken the common symbols from both accounts in order to facilitate comparison.  


\subsection{Basic JC and JE phoneme inventories}

\begin{table}
\begin{tabular}{lr@{ }lc}
\lsptoprule
 Item &  \multicolumn{2}{c}{JC} &  \isi{JE}\\
\midrule
bit   & /ɪ/: & [ɪ]        & /ɪ/: [ɪ]\\
beat  & /i/: & [i]        & /i/: [i]\\
bet   & /ɛ/: & [ɛ]        & /ɛ/: [ɛ]\\
bait  & /ɪɛ/:& [ɪɛ], [ie] & /e/: [e]\\
bat   & /a/: & [a], [ɐ]   & /a/: [a]\\
bath  & /a:/ & [a:]       & /a:/ [a:]\\
Bob   & /a/: & [a]        & /ɔ/: [ɔ]\\
but   & /ʌ/: & [ʌ]        & /ʌ/: [ʌ]\\
boat  & /ʊʌ/:& [ʊʌ], [uo] & /o/: [o]\\
book  & /ʊ/: & [ʊ]        & /ʊ/: [ʊ]\\
boot  & /u/: & [u]        & /u/: [u]\\
bout  & /ʌʊ/:& [ʌʊ], [au] & /aʊ/: [aʊ]\\
bite  & /aɪ/:& [ɐɪ]       & /aɪ/: [aɪ]\\
noise & /aɪ/:& [ɐɪ]       & /ɔɪ/: [ɔɪ]\\
\lspbottomrule
\end{tabular}
\caption{\label{fig:key:2}JC and JE vowel inventory}
\end{table}

Essentially, we can identify the mid-vowels and the low back vowels as areas of difference between the two systems.  JC typically does not distinguish /a/ and /ɔ/; and the \isi{JE} mid vowels /e/ and /o/ are diphthongs in JC.

\begin{table}
\begin{tabular}{lr@{ }lr@{ }lc@{\hspace{5em}}lcc}
\lsptoprule
Item & \multicolumn{2}{c}{JC} & \multicolumn{2}{c}{JE} & & Item & JC & \isi{JE}\\
\midrule
bat   & /b/:  & [b]                    & /b/: & [b]                   & & fat & /f/: [f] & /f/: [f]\\     
pat   & /p/:  & [p\textsuperscript{h}] & /p/: & [p\textsuperscript{h}]& & zap & /z/: [z] & /z/: [z]\\
dab   & /d/:  & [d]                    & /d/: & [d]                   & & sat & /s/: [s] & /s/: [s]\\
tap   & /t/:  & [t\textsuperscript{h}] & /t/: & [t\textsuperscript{h}]& & shat & /ʃ/: [ʃ] & /ʃ/: [ʃ]\\
that  & /d/:  & [d]                    & /ð/: & [ð]                   & & rouge & /ʤ/: [ʤ] & /ʒ/: [ʒ]\\
thing & /t/:  & [t\textsuperscript{h}] & /θ/: & [θ]                   & & jack & /ʤ/: [ʤ] & /ʤ/: [ʤ]\\
kit   & /k/:  & [k\textsuperscript{h}] & /k/: & [k\textsuperscript{h}]& & chat & /ʧ/: [ʧ] & /ʧ/: [ʧ]\\
give  & /ɡ/:  & [ɡ]                    & /ɡ/: & [ɡ]                   & & lap & /l/: [l] & /l/: [l]\\
cat   & /kj/: & [kj]                   & /k/: & [k\textsuperscript{h}]& & wag & /w/: [w] & /w/: [w]\\
gab   & /ɡj/: & [ɡj]                   & /ɡ/: & [ɡ]                   & & rat & /r/: [r] & /r/: [r]\\
mat   & /m/:  & [m]                    & /m/: & [m]                   & & yak & /j/: [j] & /j/: [j]\\
nab   & /n/:  & [n]                    & /n/: & [n]                   & & hat &  & /h/: [h]\\
ing   & /ŋ/:  & [ŋ]                    & /ŋ/: & [ŋ]                   & &    \\
vat   & /v/:  & [v], [b]               & /v/: & [v]\\\lspbottomrule
\caption{\label{fig:key:3}JC and JE consonant inventory}
\end{tabular}
\end{table}

Most of these consonant features are discussed below in more detail.  Briefly, \isi{JE} is typically described as having interdental fricatives, the voiced alveopalatal fricative and the \isi{glottal fricative} while these features are not usually identified for JC.  Additionally, JC has contrastive palatalization of the \isi{velar stop} in some environments.  

\subsection{{Word} {initial} {glottal} {fricative} {/h/}}\largerpage%2.3.1
/h/ is not phonemic in many varieties of \isi{Jamaican Creole} (\citealt[32]{Akers1981}; \citealt[7]{DevonishSeiler1991}; \citealt[40]{Meade2001}), though there is evidence that in Western Jamaican varieties (e.g. Trelawny, St. Elizabeth) it is \citep[12]{Wells1973}.  Hypercorrect use of the fricative is common in Jamaica \citep[lxii]{Cassidy1967}, particularly in \isi{Kingston} where my sample was recorded.  In local theatre, for example, this hypercorrect use is often played for comic effect in characters understood as ``\isi{speaky-spoky}".  According to \citet{Patrick1997} this style of Jamaican speech “is clearly recognised and labelled by Jamaicans, and is realized regularly by hypercorrect [h] and [ɔ]” (48).  He gives examples like [hɔlon] \textit{alone} and jokes such as “\textit{hemphasize} \textit{your} \textit{haitch} \textit{you} \textit{hignorant} \textit{hass}” which are heard locally.  

  In my total sample of speakers, eleven informants produced this hypercorrect pattern, as in F52 [honiŋ] \textit{owning} or M27 [haʊt] \textit{out,} and typically with single attestations.  The more general pattern was h-dropping, a term that clearly assumes that speakers are using an English phonological system, which does have /h/.  Evidence for this is furnished in the pattern of use of aitch in the texts collected.  Except for the 11 whose usage suggests non-contrastive [h], as in [haʊt {\textasciitilde} aʊt] \textit{out,} most informants produced /h/ in the same way as in \isi{IAE}, distinguishing lexical items like ‘hand’ and ‘and’, without inserting [h] where it is not lexically specified in \isi{IAE}.  It is entirely possible that some speakers have idiolects that have neither /h/ nor a hypercorrect usage of [h], while having other English phonological features.  I did not find such a speaker in my sample.\pagebreak

  Outside of Jamaica, [h]

\begin{quote}
	...has become one of the principal signals of social identity, its presence in initial positions associated almost inevitably with the ‘educated’ and ‘polite’, while its loss commonly triggers popular connotations of the ‘vulgar’, the ‘ignorant’ and the ‘lower class’ \citep[107]{Mugglestone1995}. 
\end{quote}

For some Jamaicans, h-dropping is also overtly stigmatized, though it is not at all clear that this is universally the case or that h-lessness is even perceived by the hearer.  It is a stereotype held by others in the Caribbean of bad Jamaican speech \citep[569]{Wells1982c}.  This stereotype is confirmed in \citet[90]{Roberts1988} and in Allsopp’s dictionary.  As Allsopp points out, “/h/, especially in initial position, is lost noticeably in Jamaica, even in the speech of educated persons” (\citeyear[xlvii]{Allsopp1996}).

Historically, variable use of [h] was a feature of some of the varieties of English that came to and were used in Jamaica during the early years of British settlement in the mid to late 17\textsuperscript{th} century.  H-dropping became salient and the shibboleth of vulgar speech in the metropole by the 18\textsuperscript{th} century \citep[103]{Beal2002}.  The more consistent pronunciation of [h], largely due to its presence in the \isi{orthography}, indexed literacy; and therefore it was the ``illiterate speaker'' of the 1850's in Britain who was perceived to drop [h].  \citet{Mugglestone1995} documents that:  

\begin{quote}
	While `literate speakers', at least in terms of the prevailing prescriptive ideology, thus made plain the facts of their superior education by matching grapheme with appropriate sound, or <h> with [h] in their speech, so therefore did the non-appearance of [h] take on the values (${\ldots}$) of the 'lower class' (117).  
\end{quote}

As such, hypercorrect use of [h], due to the stigma attached to [h] dropping, developed as a stable feature in sections of the British population over two centuries ago, a function of the changing prescriptions on [h] articulation and the focussing of attitudes to ``h-less" speakers.  \citet{Mugglestone1995} cites a number of examples, one of which is from Poor Letter H (1866): ``...he would persist in saying that the habbey was his 'obby'' (124).

However, \citet[432]{Wells1982b} tells us that \isi{h-drop} is not a feature of Irish English; moreover /h/ is more widely distributed in Irish than it is in English.  This suggests that variable (but not yet hypercorrect) use of [h] would probably have been more typical of founder English varieties \citep[lxii]{Cassidy1967} than an aspect of the speech of the Irish indentured servants who were also a presence in Jamaica at the time (see for example \citealt[228]{Beckles2000}).  Moreover, it is possible that given the regional associations in \isi{Jamaican Creole} with variable use of [h] (Eastern Jamaica and \isi{Kingston}), and the early history of English settlement in Eastern Jamaica \citep[85]{Sherlock1998} particularly in urban areas like Port Royal and \isi{Kingston}, the pattern of aitch distribution in Jamaica reflects to some extent the pattern of white settlement in the early years of the colony.  In that respect, the linguistic data may, as Devonish argues (\citeyear[180]{Devonish2002}), provide historians with information that points to particular demographic patterns that require exploration.         

\citet{LallaDCosta1990} cite the example from the early 19\textsuperscript{th} century of a Jamaican planter who is reported as saying “hedicating the negroes is the only way to make them 'appy (...) hedication is hall in hall” (142).  This suggests that hypercorrect use of [h], specifically h-insertion, has also been a feature in some Jamaican varieties for at least the past two centuries, no doubt also influenced by varieties of \isi{Creole} coexisting in the speech community.  It is interesting that neither \isi{h-drop} nor h-insertion has been identified in US varieties, though occasionally ``an" occurs before words beginning in aitch, suggesting possible absence (\citealt[206]{Krapp1925} gives the example \textit{an halfe} from 1653).   It has been described in some \isi{Canadian} (Newfoundland) varieties of English \citep[447]{Kirwin2001}.  It appears, therefore, that some ecologies were more favourable to the persistence of variable\slash hypercorrect [h] than others.  Certainly, speakers acquiring or learning English in Jamaica would have been exposed to and selected these hypercorrect forms, options existing in the Jamaican ``feature pool" \citep[4]{Mufwene2001}, especially if such forms were used by some speakers at the top of the society.

The \isi{JAMPRO} data show the pattern in \tabref{tab:2.1} for h-dropping word initially\footnote{This data does not include occurrences of typically unstressed items like \textit{has}, as when the auxiliary of a participle (he [hǝz] gone) or \textit{have} (we [hǝv] seen).}.

\begin{table}
\begin{tabular}{r@{ }rr@{ }r}
\lsptoprule
       \multicolumn{2}{c}{h-drop} & \multicolumn{2}{c}{h}\\
\midrule
184 & (10.4\%) & 1592  & (89.6\%)\\
\lspbottomrule
\end{tabular}
\caption{h-drop in the total sample}
\label{tab:2.1}
\end{table}

The data here points to a fairly low incidence of \isi{h-drop} in these speakers, with many not doing it at all.  The \isi{hypercorrection} discussed above does suggest a sensitivity to use of this variable as marking a speaker of “good” English.  Of the total sample, 15 of the 82 informants account for the majority of cases of \isi{h-drop} (111 of the 184 or 60\%), the more typical pattern being one or two instances of the feature in speakers’ recordings.  Of those 15 informants, 8 – two men and six women – had no productions of hypercorrect h-insertion.  A fuller discussion of \isi{hypercorrection} in this sample is presented in \sectref{sec:2.4} of this Chapter, and \chapref{ch:3} and~\ref{ch:4} present the results of a number of sociolinguistic correlations.   

\subsection{{The} {interdental} {fricatives} {/θ/} {and} {/ð/}}\label{sec:2.3.2}%2.3.2
The interdental fricatives [θ] and [ð] vary with the alveolar stops [t] and [d] in Jamaica, as in [θɪk {\textasciitilde} t\textsuperscript{h}ɪk] \textit{thick} and [ðat {\textasciitilde} dat] \textit{that}.  The \isi{alveolar stop} variants are identified as a \isi{Creole} feature in this social context, though it is found in varieties of non-standard English such as Irish or some US varieties.  The literature on Jamaica has generally suggested that the stop variants are not part of the \isi{acrolect} (\citealt[33]{Akers1981}; \citealt[23]{Meade2001}; \citealt[239]{Thaxter1977}) or the speech of the highly educated \citep[575]{Wells1982c}.  However, \citet[xlvi]{Allsopp1996} sees this as a general feature of Caribbean speech; and \citet[188]{Young1973} found relatively low frequencies of the fricative variants in even formal \isi{Belizean} speech.  Similarly, \citet{Miller1987} in her study of \isi{AJE} found that her speakers, in formal contexts, used the \isi{alveolar stop} variants about 30\% of the time (183).

Miller's analysis does not distinguish patterns of use for the voiced and voiceless stop, though the tables she presents suggest that the voiced stop is produced more frequently by speakers (57).  In \citegen{Thaxter1977} study, the “librarian’s speech”, identified by his sample of Jamaican teachers as the type of speech they would pass on to their students (249), was characterized by no voiceless TH stopping, slightly more use of [d {\textasciitilde} ð].  It seems necessary then to distinguish the patterns for the voiced and voiceless variables in our analysis.  Certainly, I have heard the use of [θ] where most speakers would have [t] in items like [θrut] \textit{truth} and [θɛrǝrɪzm] \textit{terrorism} in some Jamaican speech, including the \isi{JAMPRO} sample (F76 [θʌg] \textit{tug}).  This suggests that contrastive /θ/ is an aspect of the \isi{acrolect} and that some informants perceive the \isi{interdental fricative} to be “correct”.  A parallel use of [ð] did not occur in my sample and is rare, even though \citet[565]{Wells1982c} says that [bɛð] \textit{bed} is possible.  It is not clear from his text, however, whether he is suggesting a theoretical possibility, as his analysis does not differentiate patterns for the voiced and voiceless stop either. 

TH stopping is a general feature of most dialects of British English \citep[lviii]{Cassidy1967}, and in some with an additional variant [f].  This latter variant is also found in AAVE and Bajan but not in Jamaica.  The pronunciation of the \isi{interdental fricative}, influenced by spelling changes prior to the 15\textsuperscript{th} century \citep[168]{Pyles1993}, was normalised at the time of \isi{language contact} in Jamaica.  I find no references to hypercorrect use except in varieties of Hiberno-English.  \citet[313]{Baugh1993} cite 19\textsuperscript{th} century forms such as [θru] \textit{true} and [bʌtθǝr] \textit{butter}, pronunciations that occur today in southern Irish English \citep[105]{ Trudgill1994}.  There is evidence of an Irish population in early Jamaica, particularly in the early \isi{colonial period} before the 18\textsuperscript{th} century.  According to Cassidy \& Le Page

\begin{quote}
	Thus, whilst it is almost certain that [interdental fricatives tended to be produced as alveolar stops] due to the lack of /ð/ and /θ/ in many West African languages, the possibility remains that in the speech of [early white settlers] these processes had already begun without African influence (lvii).
\end{quote}

I find no reference to hypercorrect use of interdental fricatives in early Jamaica.  It is possible that the forms came from the speech of Irish indentured servants, but the citations I can find are all from the 1960's, when there was more widespread access to education and greater exposure to the standard usage for speakers.   Moreover, hypercorrect [θ] is not identified in American varieties, notwithstanding the Irish influence in some regions.

I will analyse use of these two variables in different positions in the word.  For one, it is possible that the most common occurrences of word initial /ð/ (words like \textit{the}, \textit{them}, \textit{that}) are less consciously monitored by speakers, even in formal contexts, because they are typically unaccented and weak in discourse (see \citealt[185]{Gimson1980}).  In addition, the type of word that will have a middle TH is, at the very least, disyllabic and the variant use here may better reflect the patterns of \isi{formal speech}, with its greater proportion of longer and more erudite words.

The data show the pattern in \tabref{tab:2.2} for the interdental fricatives and variants in all speakers.

\begin{table}

\begin{tabular}{lr@{ }Sr@{ }Sr@{ }Sr@{ }S}
\lsptoprule
      & \multicolumn{2}{c}{d} &  \multicolumn{2}{c}{ð} & \multicolumn{2}{c}{t}  & \multicolumn{2}{c}{θ}\\
\midrule
Word Initial  & 2290 & (54.3\%)        &       1924 & (45.7\%) & 117 & (14.4\%)  &   693 & (85.6\%)\\
Word Middle   & 76   & (25.0\%)          &       228  & (75.0\%)   & 22  & (18.8\%)  &    95 & (81.2\%)\\
Word Final    & 33   & (27.0\%)          &        89  & (73.0\%)   & 20  & (24.0\%)    &    63 & (76.0\%)\\
\lspbottomrule
\end{tabular}
\caption{TH stopping in the total sample}
\label{tab:2.2}
\end{table}

  The data reveal two things.  Firstly, the pattern for the word initial voiced fricative is noticeably different from all the others in the table.  It could be that contrastive /ð/ is not a feature of some idiolects in my sample, while /θ/ is; however, as I have no instances of forms like [ðɔɡ] \textit{dog}, it is more probable that it is and that its phonetic realizations are less focussed in word initial contexts.  Crucially, in the same context, speakers produce significantly fewer voiceless stops – and the \isi{hypercorrection} mentioned above again indicates that use of [θ] particularly is a salient feature of “correct” speech.  The relatively low incidence of TH stopping in all other environments also suggests that use of fricatives is the more general pattern for these speakers in formal contexts, with some variation with the \isi{alveolar stop}. 

\subsection{{The} {low} {back} {stressed} {vowel} {/ɔ}{/}  }%2.3.3
In “not” words the possible variants are [nat {\textasciitilde} nʌt {\textasciitilde} nɔt].  The “\isi{Creole}” variant in this set is the \isi{low central vowel} [a], as \isi{Jamaican Creole} has one \isi{low vowel} phoneme /a/, with [at] produced for both \textit{hat} and \textit{hot}.  \citet{Wells1982c} says that, “block is acrolectally [blɒk] (or sometimes with an unrounded back vowel [blɑk), but basilectally homophonous with black [blak]” (576).\footnote{Wells uses the symbols [ɑ] and [ɒ] to depict what I will identify as the [ɔ] sound.  I find the Jamaican vowel to be higher and rounder than he suggests (see also \citealt[142]{BeckfordWassink2001}).}  Similar analyses of the \isi{basilect} can be found in \citet[25]{Akers1981} and \citet[5]{DevonishSeiler1991}. 

Hypercorrect use of [ɔ] is the other feature that \citet{Patrick1997} associates with ``\isi{speaky-spoky}". Forms like [sɔlɛrɪ] \textit{salary} (M47), [rɪlɔks] \textit{relax} (F53) and [fɔ:mʌ] \textit{farmer} (F80) produced by some of my informants would suggest the kind of sensitivity to the feature discussed by Patrick and a belief that [ɔ] is the “correct” reflex of [a].  Like Wells, \citet[151]{BeckfordWassink2001} and \citet[42]{Meade2001} describe \isi{acrolect} (dominant) speech as clearly distinguishing /a/ and /ɔ/ in “not” words.  However, words like \textit{mother}, \textit{government} and \textit{colour} can be, and often are, pronounced as [mʌðʌ {\textasciitilde} mɔðʌ], [ɡʌvǝmɛnt {\textasciitilde} ɡɔvǝmɛnt] and [kʌlʌ {\textasciitilde} kɔlʌ] in Jamaican English.  This suggests that the relationship between /a, ʌ, ɔ/ is a complex one at the upper end of the construct called the continuum and that, for some speakers, [ɔ] is a reflex of /ʌ/ as well.  In \citegen{BeckfordWassink2001} analysis, her sample of speakers from rural St. Thomas in Jamaica did not have a distinct /ɔ/, though they did distinguish the other two sounds (150). 

Prescriptive texts for British English from 1673 are cited by \citet[356]{Freeborn1998} as listing the following as homophones: \textit{chaps} and \textit{chops}; \textit{band} and \textit{bond}; \textit{knots} and \textit{gnats}.  By the end of the 17th century there is evidence of variation between [a] and [ɔ] -- in words such as \textit{quality}, \textit{watch}, \textit{what} -- the latter variant initially regarded as vulgar but becoming more standard by the 19\textsuperscript{th} century \citep[128]{Beal2002}.  \citet{Mugglestone1995} gives an example from 1697 of an ``affected" literary character, Lord Foppington, who is depicted as speaking in this way: “Now it is nat passible far me to penetrate what species of fally it is thau art driving at (${\ldots}$) I must confess, I am nat altogether so fand of” (216). The rounding of the vowel seems to have spread from pre-rhotic environments (\textit{warm}, \textit{quart}), gradually becoming the prestige variant generally \citep[128]{Beal2002}.  Essentially, the varieties of English that came into Jamaica would have had this variation between [a {\textasciitilde} ɔ], and, like the ``problem" with [h], it persisted for some time.

A number of commentators on Jamaica remark on the language of white Creoles, in particular women's speech.  \isi{Lady Nugent} writing in the early 1800s remarked that ``...many of the ladies, who have not been educated in England, speak a sort of broken English'' \citep[98]{Wright2002}, a complaint echoed by Marly, Moreton and other contemporary writers \citep[131]{LallaDCosta1989}.  They cite the forms ``haut" (\textit{heart}) and ``knaum" (\textit{nyam}) in a 1793 publication, which I interpret as [hɔ:t] and [njɔ:m], both of which would be called hypercorrect in current \isi{JE}.  It is not clear whether they should be so labelled in the cited female speaker, but the forms have therefore been in the \isi{Jamaican speech community} at least since then.  Crucially, the speaker has taken a clearly African lexical item ([njam] \textit{eat}) and adapted the phonology to what she considers a prestige pronunciation.  Jamaican ideas of correct pronunciation must have been informed by a \isi{Creole} that did not have the /ɔ/ vowel, and it is even possible that its selection as prestigious was an autonomous Jamaican occurrence, and not one that \textit{necessarily} followed from metropolitan norms of use.  \tabref{tab:2.3} presents the \isi{JAMPRO} data.

\begin{table}
\begin{tabular}{*{3}{r@{ }r}}
\lsptoprule
\multicolumn{2}{c}{nat} & \multicolumn{2}{c}{nʌt} & \multicolumn{2}{c}{nɔt}\\
\midrule
385 & (19.8\%) & 325 & (16.8\%) & 1230 & (63.4\%)\\
\lspbottomrule
\end{tabular}
\caption{The low back vowel and variants in the total sample}
\label{tab:2.3}
\end{table}

  In these speakers there is a low incidence of the “\isi{Creole}” variant, in contrast to the relatively high incidence of [ɔ].  There is also a low occurrence of the [ʌ] variant.  However, close to a third of the time the [ɔ] variant is not produced.  Again, this is one of the variables for which we find speakers producing hypercorrect forms, which may indicate what is perceived to be correct and to be stigmatized in the general speech community.  For some speakers at least, [a] is to be avoided, even where an item is lexically specified in \isi{IAE} to have this phoneme.   

\subsection{Pre-consonantal\slash pre-rhotic mid tense vowels /e/ and /o/}%2.3.4
In words like \textit{boat} [buot {\textasciitilde} bot] and \textit{face} [fies {\textasciitilde} fes], speakers vary the mid tense vowels with a diphthong (described by \citealt[8]{DevonishSeiler1991} as a syllable nucleus preceded by either a palatalized onset in the case of [fjes] or a rounded one in the case of [gwot]).  \citet{BeckfordWassink2001} found that in her acrolect-dominant speakers, production of this pair of features was related to factors like variation in style (more informal) and \isi{gender} (male speakers).  Generally, she found that all acrolect-dominant speakers showed a predominance of monophthongal forms in all test contexts, but particularly when called on to produce \isi{careful speech} in a word list (153).  However, her informants did not use the diphthong variants with the same frequency, [uo] being less likely to occur than [ie], specifically for men (154).  \citet{Alleyne1980a} also suggests that [ie] is a more widespread feature in Jamaica (41).  \citet[113]{Meade2001} identifies the diphthongs as “variants in most Jamaican varieties, but to different degrees”.  Use of the diphthongs is described as mesolectal in Thaxter's study (\citeyear[239]{Thaxter1977}) and basilectal by \citet[25]{Akers1981} and \citet[576]{Wells1982c}. 

In my analysis, I make the distinction between the mid-tense vowels as they are produced in a pre-rhotic environment and when occurring before other consonants.  For some Jamaicans sets of words like \textit{beer}, \textit{bare} and \textit{bear} or \textit{where} and \textit{wear} are homophonous either as [be:r]\slash [we:r] or [bi\textsuperscript{e}r] / [wi\textsuperscript{e}r]; and this even in speakers who vary for example [e {\textasciitilde} ie] elsewhere before other consonants.  \citet[19]{Christie2003} describes the former pronunciations as “hypercorrect”, an overuse of the [e] vowel in order to avoid the \isi{Creole} diphthong, as a distinction between words like \textit{beer} and \textit{bear} is the pattern in \isi{MSE}.  

Historically, the diphthongs (both before other consonants and [r]) were aspects of some of the varieties of English that came to Jamaica \citep[xlvi]{Cassidy1967}.  In addition, \citet[356]{Freeborn1998} cites late 17\textsuperscript{th} century prescriptions that indicate that \textit{bare} and \textit{bear} or \textit{chair} and \textit{cheer} are to be homophones.  To label similar contemporary Jamaican pronunciations “hypercorrect” is, of course, to stipulate that the changes in the Standard in Britain are necessarily to be reflected in Jamaica.   

With the back vowel, speakers can be distinguished by the surface vowel of an underlying /o/ in \textit{poor} type words – having either the vowels [o] or [ɔ].  For some speakers the /o/ can become phonetically [o], [u\textsuperscript{o}] or [\textsuperscript{u}o]; for others the /o/ is realized as [ɔ].  Indeed, in some Jamaican speakers a word like \textit{our} can be realized as [or], and possibly as a consequence is sometimes pronounced as [ɔr] in formal contexts such as that of a radio broadcast.  Pre-rhotic back diphthongs occur in Jamaican speech therefore in variation as follows: \textit{poor}, \textit{more} or \textit{courthouse} is produced as [puor {\textasciitilde} pɔr], [mu\textsuperscript{o} r {\textasciitilde} mɔ:r], [k\textsuperscript{u}othaʊs {\textasciitilde} kɔrthaʊs]; note that \isi{JE} /ɔr/, as in \textit{forty}, \textit{horse}, \textit{north}, also varies with the expected [a:] described for the previous variable ([fa:ti], [ha:s], [na:t]).  It is possible, therefore, that the pre-rhotic context is one linguistic constraint on the type of variation identified above for the production of diphthongal allophones in speakers.   

\begin{table}

\begin{tabular}{*{4}{r@{ }r}}
\lsptoprule
         \multicolumn{2}{c}{uo} & \multicolumn{2}{c}{o} & \multicolumn{2}{c}{ie} & \multicolumn{2}{c}{e}    \\
\midrule
 246 & (11\%)  &   2041  & (89\%) & 748 & (26\%) & 2145 & (74\%)\\
\lspbottomrule
\end{tabular}
\caption{Mid tense vowels and variants (all environments) in the total sample}
\label{tab:2.4}
\end{table}

  The data here does support Alleyne’s view and Beckford-Wassink’s findings about the higher incidence of [ie] in Jamaican speech when all linguistic contexts are analysed together.  However, when pre-rhotic occurrences are excluded from the data, there is a noticeable difference in the way the diphthong patterns.  As in Beckford-Wassink’s data on \isi{careful speech}, in these \isi{JAMPRO} interviews we observe a predominance of monophthongal forms generally.  However, I find there are some speakers who do allow diphthongs, and typically only in the pre-rhotic environment.  

  The data in \tabref{tab:2.5} shows how the pre-rhotic diphthongs occur at \isi{JAMPRO}.


\begin{table}     \footnotesize
\begin{tabularx}{\textwidth}{X*{5}{r@{ }r}}
\lsptoprule
   & \multicolumn{2}{c}{[ɔ]}  &    \multicolumn{2}{c}{[u\textsuperscript{o}]}    &   \multicolumn{2}{c}{[o]} & \multicolumn{2}{c}{[\textsuperscript{i}e]} & \multicolumn{2}{c}{[e]}\\
\midrule

pre-rhotic \newline\itshape (court) -- (beer)     & 453 & (73.2\%) &   116  & (18.7\%)  &  50 & (8.1\%) & 485  & (45.5\%)     &   579 & (54.5\%)\\
pre-consonantal \itshape (coat) -- (babe) & & & 130 & (8.0\%)    &   1538 & (92.0\%)    & 263 & (14.0\%)  & 1566 & (86.0\%)\\
\lspbottomrule
\end{tabularx}
\caption{Comparison of mid-tense vowels and variants in the total sample}
\label{tab:2.5}
\end{table}

For the back vowel set, most speakers produce the [ɔr] in \textit{poor} type words, with a little over quarter of the sample using the other variants.  The pre-rhotic environment for the back vowel then seems to inhibit diphthongization in ways that it does not for the \isi{front vowel}.  And the /o/: [uo] allophone is also very infrequent pre-consonantally in my sample.  Close to half of my informants, however, use the diphthong variant of the \isi{front vowel} before [r].  Essentially these speakers tend not to produce [uo] in any contexts, selecting either forms like [bot] \textit{boat} or [pɔr] \textit{poor} in \isi{formal speech}.  In the same stylistic situation front diphthongs occur more frequently in all environments, but particularly pre-rhotically.         

\subsection{{The} {word} {initial} {velar} {stops} {/k/,} {/g/}}%2.3.5
In Jamaican speech, before the \isi{low central vowel} /a/, the /k/ and /ɡ/ have the variants [k\textsuperscript{h}] and [kj], as in \textit{cat} [k\textsuperscript{h}at {\textasciitilde} kjat] or \textit{card} [k\textsuperscript{h}a:d {\textasciitilde} kja:d]); or [ɡ] and [ɡj], as in \textit{gap} [ɡap {\textasciitilde} ɡjap] or \textit{guard} [ɡa:d {\textasciitilde} ɡja:d].  The /kj/ and /ɡj/ are contrastive in \isi{Creole} before /a/ - [ɡja:dn] \textit{garden} and [ɡa:dn] \textit{Gordon}; [kja:f] \textit{calf} and [ka:f] \textit{cough}; but *[ɡjaspl] \textit{gospel}, *[kja:k] \textit{cork} .  It is non-contrastive in \isi{JE}, as the items \textit{garden} and \textit{Gordon} are distinguished instead by the vowel a/ɔ contrast.  The \isi{palatal glide} in words such as \textit{garden} or \textit{card} was, up to the end of the 18\textsuperscript{th} century, regarded as “smooth and elegant (...) sufficient to mark the speaker as either coarse or elegant, as he adopts and neglects it” (\citealt[233]{Mugglestone1995} citing \citealt{Walker1781}).  By the middle of the next century its status had changed \citep[lviii]{Cassidy1967} and descriptions such as “affected”, “old-fashioned” and “rapidly dying out” were used for this feature, though its association with “proper” \isi{female speech} seemed to have persisted somewhat longer (Mugglestone ibid.). 

Current analyses of Jamaica tend to label [kj] a ``\isi{Creole}" or basilectal variant (\citealt[33]{Akers1981}; \citealt[89]{Miller1987}; \citealt[569]{Wells1982c}; but cf. \citealt[96]{Patrick1999} and \citeyear[11]{Patrick2000}).  However, in her \isi{AJE} sample, F. Miller found that the [kj] was produced for 44.3\% of items and more by upper middle class men than by other groups.  She states that, “no linguistic constraints on this variable were observed” (pg. 118).  Interestingly, she cites a letter to the local newspaper by a reader who wished to point out to Jamaicans that the “correct” pronunciation of a word like \textit{Canada} is [kjanɪdʌ].  This possibly indicates that some speakers hold a competing idea of the standard pronunciation that differs from modern \isi{MSE} norms.  

In a previous study (\citealt[69]{Irvine1994}), I found that educated, affluent speakers avoided the [kj] before the \isi{long vowel}, but more consistently produced /kj/ before the short [a] (see \tabref{tab:2.6}).

\begin{table}
  \begin{tabular}{*{4}{r@{ }r}}
    \lsptoprule
    \multicolumn{2}{c}{kj + a}       &     \multicolumn{2}{c}{kj + a:}   &      \multicolumn{2}{c}{k\textsuperscript{h} + a} &   \multicolumn{2}{c}{k\textsuperscript{h} + a:}\\\midrule
   61 & (94\%)  &   10 & (6\%) &      4 & (9\%)     &    41 & (91\%)\\\lspbottomrule
 \end{tabular}
\caption{The incidence of the palatal glide in eight male speakers}
\label{tab:2.6}
\end{table}

  Patrick does acknowledge the importance of vowel length for the incidence of the palatal variants, but argues instead for -\textit{ar} words as a separate class for analysis from -\textit{at} type words.  So, for example, the prestige pattern of [kj] use (found in the urban, upwardly-mobile, middle class speaker) will permit [kjat] \textit{cat} but inhibit [kja:r] \textit{car}.  I have found in my sample, however, that speakers do produce forms like [kjarɪ] \textit{carry}, [kjaraktʌ] \textit{character}, [kjarIbiǝn] \textit{Caribbean}; and there is a prominent insurance company locally that advertises its [gjaranti] \textit{guarantee} of good service.  Moreover, the -AR class he identifies (pg. 107) of lexical items like \textit{cards}, \textit{car}, \textit{guards} and \textit{garbage} all have long vowels, and it is therefore not clear why they are analysed as an exceptional word class.  \citet{DyerBeckford2001} conclude in their study that: “where the status of the upwardly mobile speaker was once marked by the absence of \isi{Creole} forms, certain \isi{Creole} forms such as (KYA) are now welcomed” (31).  They find (pg. 23) \isi{palatal glide} insertion to be constrained by the number of syllables in the word (occurring typically in mono- and disyllables).  I have also not found this pattern in my sample, as attested to above.

  The pattern of use among my informants was as described in \tabref{tab:2.7}.

\begin{table}
\begin{tabular}{*{4}{r@{ }r}}
\lsptoprule
  \multicolumn{2}{c}{kja}  & \multicolumn{2}{c}{k\textsuperscript{h}a} & \multicolumn{2}{c}{kja:} & \multicolumn{2}{c}{k\textsuperscript{h}a:}\\
\midrule
 119 & (56.9\%) & 90 & (43.1\%) & 19 & (19.8\%) & 77 & (80.2\%)\\
\lspbottomrule
\end{tabular}
\caption{Word initial velar stops and variants in the total sample}
\label{tab:2.7}
\end{table}

  In this data, \textit{can} (the modal auxiliary) was excluded in the many instances it was produced as [kɛn] and therefore was not comparable with other items in the data set.  This reduced the token count for this feature in the data.  I have therefore collapsed the voiced and voiceless \isi{velar stop} in this analysis under the general heading of /k/.  The palatal velar before the short vowel occurs frequently in the informants’ speech.  Notably, the incidence of the feature before the \isi{long vowel} is much lower, and comparable to the more stigmatized variants [nat] and voiceless TH stopping.  The pattern in the sample, which shows variation before the short vowel, suggests that [k\textsuperscript{h} {\textasciitilde} kj] is linked to /ɔ/ use in \isi{JE} and may be a phonetic feature that draws attention to possession of distinct /a/ and /ɔ/ sounds (see \citealt[xlix]{Cassidy1967}).  Arguably, use of [kj], which is prohibited before [ɔ], more sharply distinguishes [a] from [ɔ] in a sociolinguistic context where the two vary in many speakers’ production.             

The data on TH stopping, [e {\textasciitilde} ie] and [kj {\textasciitilde} k\textsuperscript{h}], raises an interesting question about sociolinguistic variation in pairs of variables.  It is possible that speakers pay “attention” to only one member of the pair of related phonological variables when prestige\slash stigma are at stake.  While the incidence of word initial voiceless TH stopping is very low in the sample with a 14\% rate of occurrence, the voiced counterpart was produced 54\% of the time for the variable.  And the incidence of the \isi{front diphthong} before /r/ is more than four times that of its counterpart elsewhere.  Here again we find one of a pair of variables being singled out and being used to index prestige/stigma, as [kja:] is seldom produced in a sample in which [k\textsuperscript{h}a {\textasciitilde} kja] freely vary.  This issue will be explored fully in the final section of \chapref{ch:3}.

\section{Features not widely discussed in the literature (Group B)}\label{sec:2.4}      

\citet{Brodber1989} identifies at least two types of speakers of Jamaican English – one group ``generally displays a command of a variety of formal and informal styles [of \isi{JE}]", while the other, an ``adherence, in speech, to \isi{spelling pronunciation}, and careful, somewhat measured, articulation" (46).  The former tend to be \isi{vernacular} speakers of English who acquire \isi{Creole} later, the latter learn English in the school system.\footnote{According to \citet{Hernández-Campoy2003} the implementation of the \isi{standard variety} usually follows the same route.  As such, the greater the frequency of standard forms in informal\slash familiar styles, the greater the degree of standardisation.}  \citet[158]{Mugglestone1995} and \citet[229]{Wells1982a} describe a similar distinction in \isi{RP} English, with speakers of \isi{adoptive} \isi{RP} i.e. ``those who did not use this \isi{accent} as children" tending to avoid features that occur naturally in the speech of ``mainstream" speakers.  I take the first three variables dealt with below to be instances of the kind of articulations Brodber discusses for \isi{adoptive} \isi{JE}.  This section on Group B variables is completed by an examination of \isi{rhoticity} and final \isi{cluster simplification} in the \isi{JAMPRO} sample.    

\subsection{{The} {word} {final} {unstressed} {vowel} {in} {words} {that} {end} {in} {-er}}%2.4.1
In some speakers, pronunciations such as [ʤʌmekʌr] \textit{Jamaica} and [bʌtʌr] \textit{butter} occur.  This may be explained in two ways.  Firstly, there seems to be a generalized sensitivity to the (perceived as) \isi{Creole} [a] discussed above, which occurs at the end of words like [mada] \textit{mother}, [tiʧa] \textit{teacher} and [tiela] \textit{tailor}.  Some speakers use a \isi{spelling pronunciation} to avoid the [a] ending, and this has been extended even into words that do potentially end in [a] and have no -\textit{er} in the \isi{orthography} (like \textit{Jamaica}).\footnote{However, the kind of \isi{hypercorrection} being discussed here is attested to elsewhere in the English language community.  \citet[100]{Mugglestone1995} cites texts from the late 19\textsuperscript{th} century which have this sort of complaint: In a young author’s first volume I found “Italy” made to rhyme with “bitterly”.  Now “Iterly”, in the mouth of a public speaker, would condemn him as a thorough Cockney \citep{Carpenter1868}.

She concludes that in England, and as suggested by the quote, there was a widespread standard ideology that “literate speech” meant a rhotic articulation, notwithstanding the reality of evolving mainstream \isi{RP} production.}   This kind of \isi{spelling pronunciation} is a consequence of avoiding a stigmatized item, and in that respect is different from the phenomenon of speakers in some (principally L\textsubscript{2}) communities saying for example [dɛbt] \textit{debt} and [θɪŋg] \textit{thing} \citep[123]{ Trudgill1994}, which does not typically appear in Jamaican English varieties.  Secondly, \isi{retroflexion} of an -\textit{er} ending is an aspect of rhotic varieties of English – for example \isi{General American} \citep[64f]{Giegerich1992} – which \isi{JE} is said to be (see \tabref{tab:2.8}).  The [ʌr] variant may reflect either this general \isi{rhoticity} or the \isi{adoptive} \isi{JE} \isi{spelling pronunciation} or both.  No one in my sample produced forms like [ʤʌmekʌr] \textit{Jamaica}, which suggests that the \isi{retroflexion} for the \isi{JAMPRO} informants is \isi{spelling pronunciation} rather than the targeting of a generalized \isi{rhoticity}.   

Here I am looking at the production of lexical items that do have a final spelled -\textit{er} in the written form (illustrated by the word \textit{butter}). See \tabref{tab:2.8} for what the informants in my sample produced.  

\begin{table}

\begin{tabular}{*{3}{r@{ }r}}
\lsptoprule
      \multicolumn{2}{c}{butt[a]} & \multicolumn{2}{c}{butt[ʌ]} & \multicolumn{2}{c}{butt[ʌ r]}\\
\midrule
 237 & (18\%) & 958 & (72.9\%) & 118 & (9.1\%)\\
\lspbottomrule
\end{tabular}

\caption{Final articulation of \textit{-er} words in the total sample of informants\label{tab:2.8}}
\end{table}

Very few occurrences of the [ʌr] variant are produced in these informants.  Interestingly, this feature (and the one in the following discussion) is one aspect of a register that has been called “teacher English” by some Jamaicans (see also \citealt[19]{Christie2003}).  The label carries with it a number of associations: overly \isi{careful speech} due to linguistic insecurity, an unvarying classroom formality, and femaleness. 

In the sample, there were six informants with a high frequency of [ʌr] in their speech, but only two speakers who used this variant more frequently than any of the other variants (M15, F52). 

\subsection{The vowel in the final syllable \textit{-tion}}%2.4.2
In words like \textit{education} four possible variants occurred in the sample for the final syllable – [ʃan] {\textasciitilde} [ʃʌn] {\textasciitilde} [ʃɔn] {\textasciitilde} [ʃǝn].  \citet[4]{Shields-Brodber1996} mentions this [ʃɔn] pronunciation in her article on the changing voice of public, formal discourse in Jamaica.  Her impression is that [ʃɔn] is also an aspect of what she calls ``\isi{adoptive} \isi{JE}'' \citep[46]{Brodber1989}, use of which indexes both the learning of English and length of stay in the school system.  I suggest that [ʃɔn] is an example of distancing from the form associated with \isi{Creole} ([ʃan]).  Just as the stigma attached to voiceless TH stopping and non-contrastive /h/ generates production which is regarded as hypercorrect by some users of \isi{JE}, the association of [a] with its “correct” reflex [ɔ] in stressed syllables has thrown up a variant which exaggerates backness and rounding in syllables that are typically unstressed in \isi{MSE}.  To avoid saying something like \textit{educa}[ʃan], or being perceived as someone who does, some speakers produce \textit{educa}[ʃɔn] even in unstressed syllables, thus over-extending the conversion of /a/ to /ɔ/.   

The \isi{JAMPRO} informants produced the data in \tabref{tab:2.9}.

\begin{table}
\begin{tabular}{*{4}{r@{ }r}}
\lsptoprule
    \multicolumn{2}{c}{educa\textbf{[ʃan]}}     &      \multicolumn{2}{c}{educa\textbf{[ʃʌ n]}}  &  \multicolumn{2}{c}{educa\textbf{[ʃɔ n]}}   &     \multicolumn{2}{c}{educa\textbf{[ʃǝn]}}  \\
\midrule
149 & (20.9\%) & 377 & (53\%) & 120 & (16.8\%) & 65 & (9.3\%)\\
\lspbottomrule
\end{tabular}
\caption{Articulation of final \textit{-tion} in the total sample}
\label{tab:2.9}
\end{table}

The [ʃɔn] variant, like the retroflex variant in \tabref{tab:2.9}, was used by a small subset of speakers (11 informants) with high frequency.  However, this group of 11 has only one speaker in common with the group of 6 who produced [ʌr] (F52).  The sociolinguistics of these sub-groupings in the \isi{JAMPRO} sample will be explored in the following chapters.  Generally, there seems to be much less focussing around a particular norm for this feature (unlike the data in \tabref{tab:2.9}), with close to half of the attestations of -\textit{tion} varying among the possible phonetic options.  What is noticeable, however, is that the [ʃɔn] variant occurs with greater frequency than the \isi{MSE} (schwa) variant.    

\subsection{{The} {alveopalatal} {affricate}}%2.4.3n
In words like \textit{creature}, \textit{culture,} \textit{soldier} and \textit{procedure}, some informants produce a palatalized \isi{alveolar stop} variant and not an \isi{affricate}, e.g. [kritjʌ], [kʌltjʌ], [soldjʌ] and [prǝsidjʌ] \textit{procedure}.\footnote{This last item is frequently pronounced with syllable initial stress, as in [prɔsɨdjʌ] \textit{procedure.}}



\begin{table}
\begin{tabular}{*{4}{r@{ }r}}
\lsptoprule
\multicolumn{2}{c}{dj}        &            \multicolumn{2}{c}{ʤ}            &     \multicolumn{2}{c}{tj}                   &       \multicolumn{2}{c}{ʧ}\\
\midrule
34 & (75.5\%)   &   11 & (24.5\%) & 73 & (35.4\%)   &     133 & (64.6\%)\\
\lspbottomrule
\end{tabular}
\caption{Articulation of \textit{culture} type words in the total sample}
\label{tab:2.10}
\end{table}

  I propose it is partially the form of the written word that influences the production of [tj] and [dj], which appears to be a fairly widely used pronunciation for \textit{culture} type words in my sample.  Speakers would seem to be using English \isi{orthography} as a guide to ``correct" pronunciation.  Words such as \textit{teacher} or \textit{bleacher}, that pattern the phonemic shape of \textit{creature} in \isi{MSE}, but with a written symbol <ch> that suggests \isi{affricate} pronunciation, never exhibit this variation between \isi{affricate} and palatalized stop.  However, the \isi{spelling pronunciation} itself may function in another way in these cases.  It also distances speakers from a stigmatized JC feature, the use of an \isi{affricate} in words such as \textit{pleasure} [plɛʤa] (F91, F23), \textit{usually} [juʤǝli] (F71) and \textit{division} [dɪvɪʤʌn] (F57).  Notably, these JC pronunciations seem to occur typically with the voiced \isi{affricate}, and this may explain the greater frequency of palatalized voiced alveolar stops used by informants in my sample.  The phoneme inventory described at the start of this chapter identifies /ʃ/ in JC but not /ʒ/.  Arguably, the stigmatized affrication in \isi{JE} items that would have the voiced fricative /ʒ/, has perhaps resulted in the higher incidence of [dj].  Moreover, two of my informants (F6, F10), on more than one occasion, produced the form [djunjʌ] j\textit{unior}, an extension of /ʤ/ → /dj/ even when the \isi{orthography} suggests the former.  Interestingly, Devonish (pers. comm.) also cites [tju:] \textit{chew} as occurring. 

Additionally, it is possible, based on Marshall's study (\citeyear{Marshall1983}), that \isi{Creole} is associated with the use of affricates generally.  He found \isi{spirantization}, in forms such as [ʒǝmeka] \textit{Jamaica} and [ʒɛnǝrʌl] \textit{general}, to be increasingly a feature of some \isi{formal speech}, an avoidance of the \isi{affricate} even when the \isi{orthography} requires it.  Another two informants at \isi{JAMPRO} (F58, F59), produced [wɪt] \textit{which} and a third item [ʧa:rd] \textit{charge} was recorded (F52).\largerpage[-2]  

In at least these speakers, I suggest, ``\isi{good English}" is not producing the \isi{affricate}, notwithstanding its use in \isi{MSE}.  (Note again that in \textit{-tion} words, the \isi{MSE} schwa variant is also not as widely used as [ʃɔn]).  For the variable in \textit{creature} type words, the \isi{Creole} form and the \isi{MSE} form are similar.  These informants, in favouring [tj]\slash [dj], are responding to local patterns of language use and to local notions of what is \isi{Creole} and what is English, not to a remote standard of English that is external to their speech community.    Discussions about the “metropolitanization” of speech in social contexts like Jamaica (\citealt[70]{Trudgill2002}, for example\footnote{\citet{Hancock1994} uses the term to mean “the replacement of creole features by those from the coexistent metropolitan language” (97) – which is initially confusing because it would be difficult to argue that \isi{Jamaican Creole} and any metropolitan variety of English coexist.  However, Hancock’s definition of metropolitan English (“any native variety of non-creole English”, 97) suggests that his use of the term \textit{metropolitanization} may be interpreted as the replacement of \isi{Creole} features by those from \isi{JE}.}) seem to disregard the local ideologies of language that must mediate any changes in language use that speakers are seen to make.  \citet{Shields1987} bears repeating,

\begin{quote}
though some phonological features are shared by mainstream \isi{RP} and \isi{Jamaican Creole} these are often eliminated from the speech of [English speaking] informants because of their obvious associations with \isi{Creole} (119--120).
\end{quote}

Historically, in \isi{MSE}, the literature does point to pronunciations such as\linebreak\relax [kritjʌ], [kʌltjʌ] and [soldjʌ] as variants of 18\textsuperscript{th} century British pronunciation, though its status in terms of prestige was unclear by then \citep[178]{Lehmann1973}.  Certainly, in the late 17\textsuperscript{th} century the recommendation was that \textit{pastor} and \textit{pasture}, \textit{pick’t} \textit{her} and \textit{picture} should be pronounced the same \citep[365]{Freeborn1998}.  \citet[146]{Beal2002} discusses a number of prescriptions on the matter from a number of contemporary sources and concludes that, for some:

\begin{quote}
	This pronunciation [as in ne:tju:r \textit{nature}] would have been beyond reproach in the late eighteenth century (...) This is almost certainly the style that Spence would have heard in the rhetorical style used by the clergymen who seem to have provided his model of ``correct'' speech.
\end{quote}

Moreover, the recommendation from Sheridan’s dictionary that favoured the \isi{affricate} was criticized as follows: “...if a foreigner or native be ambitious for passing for an English gentleman, let him avoid with utmost care, Mr. Sheridan’s -SH-” (Beal 2002: 147, citing a source from the 1790's).

This historical data may point to the reputed conservative nature of colonial Englishes (\citealt[84]{Fisher2001}, for example), so that the variation in my sample may represent continuity from the patterns discussed above for 18\textsuperscript{th} century English.  But that explanation would possibly have more traction if this kind of variation was observed in other such speech communities; I have not been able to find any reference to it, though my own impression is that it occurs in \isi{Trinidad} English as well.  Interestingly, the association of the [tj]/[dj] feature with clergymen may suggest an early model of educated \isi{JE}, given the historical connection between education and the church in Jamaica (see \citealt[262]{Campbell1996}, for example).  It is probable that this pattern in \isi{JE} is a coincidence of all three influences: an aspect of what \citet{Coye1998} calls “orthoepic piracy” – when a spelling form is more frequently used (181) – reinforced by the avoidance of the stigmatized “\isi{Creole}” \isi{affricate}, and an earlier \isi{prestige norm} that also disfavoured the \isi{affricate}.  

Nearly all my informants varied the palatalized \isi{alveolar stop} with the \isi{affricate} when voiced.  17 informants used only the [dj] variant and 22 the [tj].  Interestingly, informant F52, whose speech patterns are singled out above in the discussions of [ʃɔn] and [bʌtʌr] as one informant who produced these consistently, used only the \isi{affricate} and not the \isi{spelling pronunciation} that would perhaps have been predicted by her language use in those cases.  

\subsection{Post-vocalic rhoticity}%2.4.4n
The seminal work on \isi{rhoticity} in the \isi{Jamaican continuum} is Aker’s \isi{implicational scale} (\tabref{tab:2.11}).

\begin{table} %% This table was transposed
\begin{tabular}{l*{7}{c}}
\lsptoprule
& \multicolumn{7}{c}{Stage}\\\cmidrule(lr){2-8}
                                    & 0 & 1 & 2 & 3 & 4 & 5 & 6\\\midrule
Intervocalic                        &   & x & x & x & x & x & x\\
Word Final                          &   &   & x & x & x & x & x\\
Before Coronal liquids\slash nasals &   &   &   & x & x & x & x\\
Before other Coronals               &   &   &   &  & x & x & x\\ 
Before Consonants                   &   &   &   &  &   & x & x\\ 
After Schwa                         &   &   &   &  &   &   & x\\  
\lspbottomrule
\end{tabular}
\caption{Rhoticity in the Linguistic Continuum, adapted from \citet[73]{Akers1981}\label{tab:2.11}}
\end{table}

Aker’s description idealizes a rhotic \isi{acrolect} (stage 6), largely an extrapolation of a system maximally divergent from a non-rhotic \isi{basilect} (stage 0), and is not based on the actual production of acrolectal speakers.  There is general agreement that \isi{Jamaican Creole} varieties are generally non-rhotic pre-consonantally (\citealt[184]{Beckford-Wassink1999a}; \citealt[577]{Wells1982c}).  Non-basilectal varieties show more variation, as typified by Well’s description:

\begin{quote}
	...in \textit{start} and \textit{north} words, sporadic pre-consonantal \isi{rhoticity} is characteristic of many mesolectal and some acrolectal speakers: thus short, basilectally [ʃa:t], may be pronounced either [ʃɔ:rt] or, more commonly, [ʃɔ:t]. 
\end{quote}

Post-vocalic \isi{rhoticity} does function as a sociolinguistic marker in Jamaican speech.  \citet{Beckford-Wassink1999a} found that male informants in her acrolect-dominant urban sample were less likely to be rhotic than their female counterparts.  In an earlier study (\citealt[142]{Irvine1988}, see \sectref{sec:1.4} in this book), I found factors like \isi{parent’s background} and speaker’s level of education to also show correlations with \isi{rhoticity}.  For example, speakers were more likely to be rhotic when they were highly educated; those whose parents were relatively affluent were typically non-rhotic, unless they had spent a long time in the school system.  My conclusion then was that education in Jamaica had normalized a rhotic variety of English.  Indeed, r-insertion was observed in two informants, who produced forms like [wɔrn] \textit{one} and [ɔrpǝtjun] \textit{opportune}, which suggested that \isi{rhoticity} is perceived to be “correct” \isi{JE}.  

In the \isi{JAMPRO} sample, speakers were almost categorically rhotic word finally where the syllable is stressed (as in \textit{car} or \textit{more}), and following the vowels in words like \textit{bird} and \textit{beard}.  The data for sociolinguistic variation here concerns \isi{rhoticity} before coronal consonants and after the vowels /a/ as in \textit{party,} and /ɔ/ as in \textit{forty}.\footnote{Historically, loss of \isi{rhoticity} in this phonetic environment is first attested in 1640, but is more widely a feature of 18\textsuperscript{th} century British prescriptions (\citealt[164f]{Beal2002}; \citealt[218]{Wells1982a}).  Indeed, according to Beal, such \isi{rhoticity} became as stigmatized as h-dropping in the 19\textsuperscript{th} century.  Wells suggests that the pattern of \isi{rhoticity} in YS varieties reflects the different historical influences on \isi{American English}, with the rhotic varieties reflecting earlier contact with norms from Britain and the non-rhotic varieties later contact with new prestige pronunciations.} 

\begin{table}
\begin{tabular}{*{4}{c}}
\lsptoprule
\multicolumn{2}{c}{\textit{party} type words} & \multicolumn{2}{c}{\textit{forty} type words}\\\cmidrule(lr){1-2}\cmidrule(lr){3-4}
{[partɪ]}    &   {[pa:tɪ}] & {[fɔrtɪ}] & {[fɔ:tɪ}]\\\midrule
 114 (44.4\%)   &    143 (55.6\%) & 210 (66\%) & 108 (34\%)\\
\lspbottomrule
\end{tabular}
\caption{Rhoticity in the total sample of informants}
\label{tab:2.12}
\end{table}

r-insertion did occur in one informant (F16), who produced the forms [ɔrtɔrɪtɪ] \textit{authority} and [masarʤ] \textit{massage}; and according to F. Miller\ia{Miller, Faye@Miller, Faye} (pers. comm.) one of her informants produced the form [mɔrðʌr] \textit{mother}.  In light of the findings in the earlier study, for some \isi{JE} speakers at least, such as those with considerable formal education, “correct” English is rhotic.  Generally, however, as in Wells’ description above, speakers at \isi{JAMPRO} varied in relation to rhotic and non-rhotic post-vocalic productions after the \isi{low central vowel}.  However, there is a much higher incidence of rhotic productions after [ɔ] in the same sample.   

\subsection{Word-final clusters}%2.4.5n
In this study I wish to compare the patterns for two \isi{word-final clusters}, [st] and [nt], distinguishing for the latter between its use when only a \isi{phonological cluster} (as in \textit{government}, \textit{rent}, \textit{important}) and when it also has morphological function (as in \textit{don’t}, \textit{can’t}, \textit{aren’t}).  I am not including [nt] clusters that have past reference (as in \textit{sent}) in this set, as analysis sometimes proved problematic.  For example, in the sentence “\textit{when} \textit{I} \textit{sen’} \textit{for} \textit{the} \textit{report}...” it is not always clear whether [d] or [t] is the absent consonant in the cluster, particularly in a speaker’s longer narrative text.  I cannot necessarily assume past \isi{tense marking} as it is very possible that on some occasions speakers produced an invariant verb form as is done in Jamaican varieties.  I will also analyse separately the extent to which clusters with past tense reference (as in \textit{looked} or \textit{kept}) occur in my sample.  In my data I have collapsed regular bi-morphemic past tenses (\textit{passed}, \textit{believed}) and semi-weak forms (\textit{felt}, \textit{meant}) together to furnish enough data for analysis.  The literature would suggest that in varieties of English, the former will be more consistently marked than the latter (\citealt[5]{Guy1980}; \citealt[47]{Neu1980}, for example); but speakers of educated (standard) varieties are said to generally delete grammatical clusters less than they do purely phonological clusters.  Additionally, a preceding nasal in a \isi{phonological cluster} is one environment where it is not clear whether clusters tend to be promoted or inhibited in varieties of English (the table in \citealt[131]{Patrick1999} summarizes a number of findings in other studies).  In contrast, a preceding sibilant (as in \textit{fast} or \textit{most}) is a more favourable context for simplification.  In that respect, it will be interesting to see what occurs in the \isi{acrolect} and compare it to findings from other varieties.  

I am interested in three specific aspects of cluster production in the \isi{acrolect}: whether it is more frequent than has been found in studies of basilectal\slash mesolectal speech, as would be expected if we accept the notion of continuum; whether it is socially distributed in the \isi{acrolect} and associated with factors like education, \isi{gender} and the like; and whether it is a salient feature in identifying one as a speaker of “good” \isi{JE}.   

The literature on final clusters in varieties of English is detailed and rich, suggesting that there are two important phonological constraints on the rate of  presence\slash absence of CC\#\#:

\begin{enumerate}
	\item the following segment, so that clusters are less likely to be produced when the following segment is a consonant \citep[217]{Labov1972}; 
	\item the preceding segment, so that clusters are less likely to be produced after a sibilant (\citealt{Santa1991}, cited in \citealt[131]{Patrick1999}). 
\end{enumerate}

In addition, the data for (mesolectal) Jamaican also suggests that clusters with morphological content are least likely to be produced when they function as a negative -\textit{n’t} \citep[150]{Patrick1999}.  The data in \tabref{tab:2.13} summarizes Patrick’s findings for mesolectal\footnote{Patrick1999 defines the \isi{mesolect} as “an intermediate [between the \isi{acrolect} and \isi{basilect}] variety or range of varieties” (16); the \isi{mesolect} is therefore a necessary variety or set of varieties in a continuum, but varieties intermediate to an idealized \isi{JE} and JX do not require a continuum to exist.  Indeed, \citet[25]{Meade2001} argues for the \isi{mesolect} as a separate mixed system, given that speakers may have “a narrow linguistic range [of competence] that does not extend to include the \isi{acrolect} or \isi{basilect}”.}  Jamaican; his analysis assumes that mesolectal Jamaican is a variety of English with underlying CC\#\# that can have morphological content.
\begin{table}
\begin{tabular}{rr}
\lsptoprule
\multicolumn{1}{c}{Preceding segment} & \multicolumn{1}{c}{Absence rate}\\
\midrule  
Nasal                      &  74\%\\
Sibilant                   &  85\%\\\midrule
Grammatical function       &      \\\midrule
Regular past               &  56\%\\ 
Semi-weak                  &  59\%\\
Negative \textit{n’t}      &  87\%\\
\lspbottomrule
\end{tabular} 
\caption{Percentage of (TD)-absence in mesolect \citep[140, 150, 157]{Patrick1999}}
\label{tab:2.13}
\end{table}

  Patrick’s data on mesolectal Jamaican varieties suggest that absence of clusters is the general tendency, overall only produced on the surface about a quarter of the time.  Moreover, the cluster is typically absent in words like \textit{can’t} or \textit{don’t}, most often heard as [kjã:] or [duon].  Importantly, the rate of past \isi{tense marking} in the \isi{mesolect} is similar for regular and irregular forms, which Patrick interprets as substrate \isi{Creole} influence on the morphology.  \citegen[130,\,140]{Akers1977} results for \isi{tense marking} in Jamaica, with a 53\% absence on regular verbs in the word-list test, suggest that clusters with morphological content are more likely to show t/d absence than purely phonological (i.e. mono-morphemic) ones.

\citet{Shields1984} provides data from interviews with Jamaican teachers in formal contexts (the classroom primarily), and therefore her findings are specific to the acrolectal pattern for the general treatment of consonant clusters.  She notes that,

\begin{quote}
	[i]n \isi{SJE} (...) it is not uncommon to find [cluster] reduction [before a following vowel] at a very high rate even in very formal contexts.  In \isi{SJE}, the (...) following vowel only partially rather than totally inhibits simplification (...) resulting in acceptable variation between \isi{SJE} \textit{last evening} \& \textit{las’ evening}, with [the former] being preferred in most formal contexts (3).
\end{quote}

Additionally, speakers are conscious of the need to preserve the clusters and therefore produce exaggerated forms like [la:st\textsuperscript{h}] \textit{last} or [se:nt\textsuperscript{h}] \textit{saint} or [hand\textsuperscript{ǝ}] \textit{hand}.  Shields explains both the high rates of \isi{cluster simplification} (45\% of 1389 tokens, p.~128) and the exaggerated forms by relating them to a fundamentally \isi{Creole} open syllable structure in most of her informants, notwithstanding the relatively superficial acquisition of English forms (14).  Indeed, \citet[2]{Devonish1992} has argued that perhaps the term TD \textit{insertion} might more reflect the realities of the Jamaican situation, suggesting that speakers acquire rules that insert the final element in the cluster rather than delete (as happens in English) in the production of an underlying CC\#\#.  This would perhaps explain the following hypercorrect forms that I have heard in \isi{JE}: 1) the government put a [band] \textit{ban} on cigarettes; 2) I have a pain in my [mɪdrɪft] \textit{midriff}; 3) Chomsky discusses the under[laɪnd] \textit{underlyin}’ level of representation; 4) [and] \textit{an} analysis of population movement; and 5) the staff are trying to underm[aɪnd] \textit{undermine} me.  Such forms do not occur in a variety like AAVE \citep[217]{Labov1972}, nor have I seen them reported elsewhere.  Interestingly, these are all hypercorrections that generate well-formed English items, as I have rarely heard hypercorrect non-English forms as seen in the examples from TH stopping or avoiding affricates.  One informant, F16, did produce [ondlɪ​] \textit{only} once in her text, which may suggest the hypercorrect insertion of a stop.  If an insertion rule is operating in some speakers, then it seems to be a phonological rule that is also lexically constrained, typically substituting already heard\slash existing well-formed English morphemes like \textit{drift} and \textit{band,} and not merely converting C\#\# → CC\#\#.  With the exception of F16's [ondlɪ​] none of my informants produced any of the forms discussed above.

Notably, hypercorrect insertion of stops word finally was also a feature of some 17\textsuperscript{th}--18\textsuperscript{th} century varieties of British English, particularly in London \citep[57--73]{Pegge1814}.\footnote{\citet[11--12]{LePage1960} gives London and the ports of London as one source of early Jamaican settlement, particularly the deportation of prisoners from Newgate.}  He cites the following examples: \textit{attackted}, \textit{sermont}, \textit{drownded}, \textit{paragraft} and \textit{sinst}.  \citet[238]{Mugglestone1995} also cites the use of “oust” \textit{house} as an example of how vulgar speakers were depicted in the contemporary literature of that period.  

The data for all speakers shows the following.  I have excluded following alveolar stops, as in \textit{pay rent to}, but have not excluded interdental fricatives from the data as the typical pattern in this sample is production of [θ]. 


\begin{table}
\begin{tabular}{*{6}{c}}
\lsptoprule
\multicolumn{6}{c}{Before a following vowel C(C)\#\#V}\\\midrule
 -nt & -nt\textsuperscript{h} & -n  & -st & -st\textsuperscript{h} &-s \\\cmidrule(lr){1-1}\cmidrule(lr){2-2}\cmidrule(lr){3-3}\cmidrule(lr){4-4}\cmidrule(lr){5-5}\cmidrule(lr){6-6}
    166 (68.3\%)      &    34 (14\%)    &      43 (17.7\%) & 99 (41.5\%)   &  14 (6\%) & 125 (52.5\%)\\\tablevspace
\midrule
\multicolumn{6}{c}{Before a following consonant C(C)\#\#C}\\
\midrule
-nt & -nt\textsuperscript{h} & -n  & -st & -st\textsuperscript{h} &-s \\\cmidrule(lr){1-1}\cmidrule(lr){2-2}\cmidrule(lr){3-3}\cmidrule(lr){4-4}\cmidrule(lr){5-5}\cmidrule(lr){6-6}
 149 (45.5\%)       &    29 (9\%)   &   149 (45.5\%) & 44 (11\%)  & 12 (3\%)  &    356 (86\%)\\
\lspbottomrule
\end{tabular}

\caption{Phonological clusters in the total sample of informants}
\label{tab:2.14}
\end{table}

The data for -nt clusters in \tabref{tab:2.14} does not include -n’t negative clusters.  That data is presented separately. 

As the literature would predict, the frequency of \isi{cluster simplification} is much higher before a following consonant (e.g. χ\textsuperscript{2}\,=\,48.48, p\,<\,.001 for -nt clusters).  Speakers are more likely to produce clusters before a vowel.  Also, perhaps not unexpectedly given his sample of mesolectal speakers, Patrick’s absence rate of 74\% for [nt] is higher than the rate of 34\% in my sample.  However, for [st] clusters, the absence rate in my data is 74\% and is much closer to his mesolectal sample and lower than what Shield’s described in her teachers.  One possible explanation for these teachers’ productions is the “teacher English” referred to in an earlier section of this chapter (\sectref{sec:2.4}), characterised by its overly \isi{careful speech} and recourse to spelling for “correct” pronunciation.     For clusters with morphological function, the data for all speakers is as follows.  I will use the label \textbf{-ed} in this study to refer to the cluster final [t]\slash [d] produced for the past suffix.

\begin{table}
\begin{tabular}{*{4}{c}}
\lsptoprule
\multicolumn{4}{c}{Before a following vowel C(C)\#\#V}\\\midrule
-n’t & -n’ & -ed & Ø\\\cmidrule(lr){1-1}\cmidrule(lr){2-2}\cmidrule(lr){3-3}\cmidrule(lr){4-4}
31 (51\%) & 30 (49\%) & 156 (77\%) & 47 (23\%)\\\midrule
\multicolumn{4}{c}{Before a following consonant C(C)\#\#C}\\\midrule
-n’t & -n’ & -ed & Ø\\\cmidrule(lr){1-1}\cmidrule(lr){2-2}\cmidrule(lr){3-3}\cmidrule(lr){4-4}
89  (16\%) & 480 (84\%) & 86 (44.5\%) & 107 (55.5\%)\\
\lspbottomrule
\end{tabular}
\caption{Morphological clusters in the total sample of informants}
\label{tab:2.15}
\end{table}

Before a following vowel segment, 40 informants produced -ed categorically and 5 speakers not at all.  Before a consonant, the results were strikingly different, with only 8 informants producing -ed all the time in their interviews and 17 never doing so.  Most informants varied in their production of clusters.  Comparison of rates of deletion of bi-morphemic past tense clusters with other studies suggests that \isi{JAMPRO} informants are less likely to consistently mark past tense than speakers of \isi{Trinidad} English \citep[269]{Winford1997} or any North American varieties (\citealt[222]{Labov1972}; \citealt[43]{Neu1980}), but are more likely to do so than the mesolectal speakers in Patrick’s sample.  And the data would indicate that phonological constraints on cluster production are important even for \isi{tense marking} in \isi{JE}, a phenomenon noted for a variety of non-standard varieties of English such as Puerto Rican, Tejano, Appalachian and the like. 

        For words like \textit{can’t}, \textit{don’t} and the like, arguably the general pattern is to simplify the cluster.  Indeed, these two specific items, even in the \isi{formal speech} here, are frequently produced as [k\textsuperscript{h}ã:(n)] and [dõ(n)].  It is possible that these two items have become or are becoming lexicalized without [t].  This would explain the consistent production of non-morphological [nt] before a vowel (82.3\%) but the variation that occurs when it functions as a negative.  


\section{Discussion}%2.5
The lists below show the rates at which all features are absent\slash present in the speech of informants.  The number of informants varies according to the presence of a particular variable in the recorded interview. 


% % Group A Variables



\begin{table}
\begin{tabular}{lcrSrS}
\lsptoprule
        &                   &  \multicolumn{4}{c}{Variant}\\\cmidrule(lr){3-6}
Feature &  Informants taped &  \multicolumn{2}{c}{Never use} &  \multicolumn{2}{c}{Always use}\\
\midrule
 /h/ & 82 & 11 & (13.0\%) & 31\shadecell & (38.0\%)\shadecell\\
{Initial [ð]} & 82 & 2 & (2.0\%) & 0 &\\
 Initial [θ] & 82 & 2 & (2.0\%) & 40\shadecell & (49.0\%)\shadecell\\
{[ɔ}] {\textit{not}} & 82 & 0 & & 1 & (1.0\%)\\
{[o] \textit{boat}} & 82 & 0 & & 32\shadecell& (39.0\%)\shadecell\\
{[e] \textit{face}} & 82 & 1 & (1.0\%) & 27\shadecell & (33.0\%)\shadecell\\
{[ɔ r] \textit{poor}} & 81 & 0 & & 28\shadecell&  (34.5\%)\shadecell\\
{[ber] \textit{beer}} & 82 & 4 & (5.0\%) & 3 & (4.0\%)\\
{[k\textsuperscript{h} a]}  & 64 & 13 & (20.0\%) & 14&  (22.0\%)\\
{[k\textsuperscript{h} a:]} & 48 & 7 & (14.5\%) & 33\shadecell&  (69.0\%)\shadecell\\
\lspbottomrule
\end{tabular}
\caption{Group A variables and distribution in the sample of informants}
\label{tab:2.16}
\end{table}


% % Group B Variables

\begin{table}
\begin{tabular}{lcrSrS}
\lsptoprule
        &                   &  \multicolumn{4}{c}{Variant}\\\cmidrule(lr){3-6}
Feature &  Informants taped &  \multicolumn{2}{c}{Never use} &  \multicolumn{2}{c}{Always use}\\
\midrule
{-er \textit{butter}} & 82 & 35 \shadecell& (43.0\%) \shadecell& 0 & \\
{[ʌ] \textit{butter}} & 82 & 0 & & 7 & (8.5\%)\\
{[a] \textit{butter}} & 82 & 27 & (33.0\%) & 0 & \\
{[ʃan] \textit{-tion}} & 81 & 39 \shadecell& (48.0\%) \shadecell& 6 & (7.0\%)\\
{[ʃʌn] -\textit{tion}} & 81 & 6 & (7.0\%) & 7 & (9.0\%)\\
{[ʃɔn] \textit{-tion}} & 81 & 41 \shadecell& (51.0\%) \shadecell& 0 &\\
{[ʃǝn] \textit{-tion}} & 81 & 43 \shadecell& (53.0\%)\shadecell & 0 &\\
{[tj] \textit{culture}} & 65 & 6 & (21.0\%) & 17 & (61.0\%)\\
{[dj] \textit{soldier}} & 28 & 13 & (20.0\%) & 14 & (22.0\%)\\
{[r] \textit{party}} & 79 & 29 & (37.0\%) & 21 & (26.5\%)\\
{[r] \textit{forty}} & 78 & 8  & (10.0\%) & 29 & (37.0\%)\\
 -nt\#\# V & 71 & 3  & (4.0\%) & 41 \shadecell  & (58.0\%) \shadecell\\
 -nt\#\# C & 77 & 11 & (14.0\%) & 14 &  (18.0\%)\\
 -st\#\# V & 71 & 19 & (27.0\%) & 12 &  (17.0\%)\\
 -st\#\# C & 81 & 47 \shadecell & (58.0\%)\shadecell & 2  &  (2.0\%)\\
 n’t\#\# V & 33 & 12 & (36.0\%) & 10 &  (30.0\%)\\
 n’t\#\# C & 78 & 34 & (43.5\%) & 0& \\
{-ed \textit{before} \textit{V}} & 71 & 5  & (7.0\%) & 40\shadecell & (56.0\%)\shadecell\\
{-ed \textit{before} \textit{C}} & 62 & 17 & (27.0\%) & 8 & (13.0\%)\\
\lspbottomrule
\end{tabular}
\caption{Group B variables and distribution in the sample of informants}
\label{tab:2.17}
\end{table}

The shaded cells in Tables~\ref{tab:2.16}--\ref{tab:2.17} show where a considerable number of speakers produced a particular variant of a variable, and they suggest an idea of good Jamaican English that can be positively defined in terms of having certain features: having (voiceless) interdental fricatives, pronouncing /h/, using [k\textsuperscript{h}] before the \isi{long vowel}, articulating past tense clusters and -nt before a following word that starts with a vowel.  Notably, in spite of the reputed status of the back vowel [ɔ] and the \isi{hypercorrection} that has been identified with it, most speakers show some variation between it and [a].  

  The distribution of Group B variants is much more diffuse than for certain Group A variants and speakers cannot be said to consistently use any variant here except the two aforementioned consonant clusters.  This is well illustrated in the distribution of \textit{-tion} pronunciations in the sample.  At best, the data suggests that most speakers will vary [ʌ] with some other possible variant.  I suggest that some features seem not to be salient in producing \isi{JE} – for example [st] clusters and -n’t clusters.  More clearly, \isi{JE} speech as reflected in these informants is \textit{not} having certain JC features, such as an invariant \isi{low vowel} [a]; for other variables, and particularly [kja] and [d {\textasciitilde} ð], speakers of \isi{JE} seem to be characterised as not having \textit{only} JC pronunciation.

Most of these areas of variation were identified elsewhere in the English language world, and most from the time that English was also being established in Jamaica.  So, for example, the ‘illiterate speaker’ of the 1850's in Britain was said to use hypercorrect ‘h’ and to drop ‘h’ (\citealt{Mugglestone1995}, Ch. 2).  And the refined speaker produced [dj] in \textit{soldier} words or [kj].  Further, prescriptions on post-vocalic \isi{rhoticity} or the distinction between \textit{chap} and \textit{chop} words were changing at the time.  Such ideologies would have also arrived and evolved in Jamaica, given the phonology of \isi{Creole} varieties occupying the same social space and the desire for social differentiation.  A good illustration of this is the pronunciation of \textit{culture} type words.  The pattern of use in my sample includes the palatalized stop variant, notwithstanding the \isi{affricate} (now) in use in \isi{MSE}.  Though I do not necessarily attribute this to historical continuity, it is clear that the \isi{metropolitan norm} has changed.  The use in my sample is as much an avoidance of “bad” English, i.e. JC affricates, even when the feature is an aspect of \isi{MSE}, as it is a response to spelling and therefore a signal of being literate and educated.  Clearly, there is considerable variation for most features in this sample of speakers.  As such, the social distribution of variants must be important for any discussion of the \isi{acrolect}, as the association of variants with factors like education, or \isi{gender}, can suggest one type of normalisation of features that index “good” English among or for these speakers.  

Certainly, as \citet{Devonish2003} points out, “with reference to the Jamaican situation, in the face of two abstractions, Standard Jamaican English and ``basilectal'' \isi{Jamaican Creole}, the first is accepted (...) by the society at large as real” (164).  But this abstraction is fundamentally defined in terms of written SE, as it is elsewhere in the English language community.  Data on the spoken norms of this idealization called \isi{SJE}, as reflected in use that \citet[lvi]{Allsopp1996} calls “considered natural in formal contexts” can reveal received pronunciations of \isi{SJE}, in the sense of (an) institutionally disseminated \isi{accent} of an influential minority \citep[31]{Yallop1999}.  What is it that determines which of two prospective qualified employees will be selected during an interview?  What judgements are being reflected when someone is said to be “well spoken” or “having a good command of English”?  The abstraction that is \isi{SJE} is realized for members of the speech community in the language actually used in these formal contexts.  Recourse to external norms of speaking for a description do not reflect the reality.  In this sample, use of [h] is widespread and the norm in \isi{JE}.  But, equally, use of [d] \textit{there} and [kj] \textit{carry} is also widespread.  The [d], no doubt, would be excluded if most speakers were \textit{asked} which of the two, [d] or [ð], is standard.  It is not as clear that norms about the latter feature [kja] are as focussed.  Moreover, when others in the speech community hear successful and/or influential people using some of these features and not others; when people who use some of these features are employed and successfully so, then the idea of what is spoken \isi{SJE} must be informed by such practice.
