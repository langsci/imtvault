\chapter[Clause combining]{Clause combining}\label{chap:12}
\hypertarget{RefHeading1213321525720847}{}
In Moloko, clauses may combine in six ways: 

\begin{itemize}
\item Complement clause (\sectref{sec:12.1}). The complement clause is an argument within the matrix clause (subject, direct object, or indirect object).
\item Dependent adverbial clause (\sectref{sec:12.2}). A clause giving adverbial information concerning the verb in the matrix clause. 
\item Conjunction (\sectref{sec:12.3}). A clause linked to another clause by a conjunction . 
\item Juxtaposition (\sectref{sec:12.4}). Two independent clauses may be linked by juxtaposition.
\item Relative clause (discussed in \sectref{sec:5.4.3}). Relative clauses are clauses embedded in a noun phrase within the matrix clause. 
\item Expectation marker (discussed in \sectref{sec:11.1}). One clause may be marked with \textit{na} or other expectation marker. The \textit{na}{}-marked clause contains known or presupposed information.
\end{itemize}

\section{Complement clauses}\label{sec:12.1}
\hypertarget{RefHeading1213341525720847}{}
A complement clause is a clause that is an argument in the matrix clause. Complement clauses in Moloko can contain one of three verb forms: dependent, nominalised or finite. When the complement clause has the same subject as the main clause, the complement clause has a dependent or nominalised verb form (\sectref{sec:12.1.1}, see also \sectref{sec:7.6} and \sectref{sec:7.7}). When the subject of the complement clause is different than that in the main clause, the verb in the complement clause is finite and the clause has a noun phrase subject (\sectref{sec:12.1.2}). 

\subsection{Dependent and nominalised verb complement clauses}\label{sec:12.1.1}
\hypertarget{RefHeading1213361525720847}{}
The complement clause is embedded in the verb phrase as a subject, direct object, or indirect object within the matrix clause.  Nominalised and dependent verb forms collocate with constructions that carry different modal or aspectual\is{Tense, mood, and aspect!Aspect in complement clauses} values. The nominalised form collocates %%\is{Tense, mood, and aspect!Completion} 
with constructions that represent a finished, accomplished event,\footnote{The situation refers to something that occurred in the past with ongoing effects to the point of reference.} e.g., the nominalised form is found as a complement of \textit{ndav} ‘finish,’ (\ref{ex:12:1}, see \sectref{sec:7.6}). In contrast, the dependent form is found in constructions that refer to an event that is incomplete or unachieved at the time of the matrix situation, e.g., the dependent form is found as a complement of the verb of inception %%\is{Tense, mood, and aspect!Inception}
\textit{zlan} ‘begin’ as in (\ref{ex:12:7}) (see also \sectref{sec:7.7}). In the examples, the subject of the complement clause is indicated by Ø when it is the same as the subject of the matrix clause. The clauses are delimited by square brackets and the verb is bolded.

In \REF{ex:12:1} the nominalised form is the direct object of the matrix verb \textit{ndav} ‘finish.’ 

\ea \label{ex:12:1}
{}[Nəndavalay  [\textbf{məwəce}]].\\
\gll  [nə-ndav=alaj    [Ø \textbf{mu-wutʃ-ɛ}]]\\
      {\oneS}-finish=away    { }  {\NOM}-write-{\CL}\\
\glt  ‘I finish writing.’
\z 

A clause with the nominalised form can function as an argument of another verb. It is the subject in \REF{ex:12:2} and the direct object in (\ref{ex:12:3}--\ref{ex:12:6}). In each case, the nominalised form indicates that the event expressed by the verb is completed.

\ea \label{ex:12:2}
 [[\textbf{M}ə\textbf{mbəɗe} ahan   na],  asaw].\\
\gll  [[\textbf{mɪ-mbɪɗ-ɛ}=ahaŋ    na]   a-s=aw]\\
      {\NOM}-remain-{\CL}=\oldstylenums{3}\textsc{s}.{\POSS}  {\PSP}  \oldstylenums{3}\textsc{s}-please={\oneS}.{\IO}\\
\glt  ‘I want the leftovers.’ (lit. its remains pleases me) 
\z 

\ea \label{ex:12:3}
[Bahay  amakay  [\textbf{məzəme}  sese]].\\
\gll  [bahaj   à-mak-aj   [Ø \textbf{mɪ-ʒʊm-ɛ}  ʃɛʃɛ]]\\
      chief  \oldstylenums{3}\textsc{s}+{\PFV}-leave-{\CL}  { }  {\NOM}-eat-{\CL}  meat\\
\glt  ‘The chief stopped eating meat.’ (lit. the chief left the eating of meat) 
\z 

\clearpage
\ea \label{ex:12:4}
[Nasar  [\textbf{mədəye}  ɗaf ]  bay].\\
\gll  {}[na-sar   [Ø \textbf{mɪ-d-ijɛ}      ɗaf]  baj]\\
      {\oneS}-know  { }  {\NOM}{}-prepare-{\CL}  {millet loaf}  {\NEG}\\
\glt  ‘I don’t know how to make millet loaves.’ (lit. I don’t know the preparing of millet loaf).
\z 

\ea \label{ex:12:5}
\corpussource{Disobedient Girl, S. 4}\\
{}[Ávata  [\textbf{məvəye}  haɗa]].\\
\gll  [á-v=ata     [\textbf{mɪ-v-ijɛ}    haɗa]]\\
      \oldstylenums{3}\textsc{s}+{\IFV}-{spend time}=\oldstylenums{3}\textsc{p}.{\IO}  {\NOM}{}-{spend time}-{\CL}  many\\
\glt  ‘It would last them enough for the whole year.’  (lit. it will {spend time} for them enough time-spendings; the nominalised form of the verb ‘{spend time}’ has been lexicalized as ‘year’)
\z 

\ea \label{ex:12:6}
[Ege  [\textbf{məvəye} məko]  ehe],  [nawas  háy  əwla].\\
\gll  [ɛ-g-ɛ  [\textbf{mɪ-v-ijɛ}  mʊkʷɔ]  ɛhɛ]  [na-was    haj=uwla]\\
      \oldstylenums{3}\textsc{s}-do-{\CL}  {\NOM}{}-{spend time}-{\CL}  six  here {\oneS}-cultivate  millet={\oneS}.{\POSS}\\
\glt  ‘Six years ago (lit. it did six years here), I cultivated my millet.’ 
\z 

In contrast, the dependent form is found in clauses where the event is incomplete or unachieved idea at the time of the matrix situation, e.g., the verb of inception \textit{zlan} ‘begin’ (\ref{ex:12:7} and \ref{ex:12:8}) (see also \sectref{sec:7.7}) and \textit{sark} habitually (\ref{ex:12:9}--\ref{ex:12:10}). The writing hasn’t started in \REF{ex:12:7}; the referent isn’t necessarily eating at the moment of \REF{ex:12:9}; the fight was just beginning at the time of \REF{ex:12:8}. In each case, the dependent form is the direct object of the matrix clause.

\ea \label{ex:12:7}
[Nazlan  [\textbf{aməwəce}]].\\
\gll  [na-ɮaŋ    [Ø \textbf{amu-wutʃ-ɛ}]]\\
      {\oneS}-begin    {}  {\DEP}-write-{\CL}\\
\glt  ‘I begin to write.’ 
\z 

\ea \label{ex:12:8}
[Tazlan  aləme  [\textbf{aməzləge} va]].\\
\gll  [ta-ɮaŋ     =alɪmɛ  [Ø \textbf{amɪ-ɮɪg-ɛ}  va]]\\
      \oldstylenums{3}\textsc{p}-begin   =\oldstylenums{1}\textsc{Pex}.{\IO}   {} {\DEP}-plant-{\CL}  body\\
\glt  ‘They started to fight us.’ (lit. they started to us planting bodies) 
\z 

\ea \label{ex:12:9}
{}[Asarkva  [\textbf{aməpəɗe} sese]].\\
\gll  {}[a-sark=va  [Ø \textbf{amɪ-pɪɗ-ɛ}   ʃɛʃɛ]]\\
      \oldstylenums{3}\textsc{s}-{habitually}={\PRF}  {}  {\DEP}-crunch-{\CL}  meat\\
\glt  ‘He usually eats meat.’ (lit. He had a habit to eat meat)
\z 

\ea \label{ex:12:10}
\corpussource{Disobedient Girl, S. 17}\\
{}[Hor  na,  asərkala  afa  təta  va \textbf{na},\\  
\gll  {}[hʷɔr  na  a-sərk=ala   afa   təta=va \textbf{na}\\ 
      woman    {\PSP}  \oldstylenums{3}\textsc{s}-{habitually}=to  {at house}  \oldstylenums{3}\textsc{p}={\PRF}  {\PSP} \\
\glt ‘The woman, she was in the habit at their house,’\\
      
      \medskip
{}[\textbf{aməhaya} háy  na  gam]].\\     
\gll {}[Ø  \textbf{amə-h=aja} haj na gam]]\\
     {} {\DEP}-grind={\PLU}  millet  {\PSP}  lots\\
\glt  ‘[of] grinding a lot of millet.’
\z 

Dependent clauses are also found in complement clauses for verbs of desire. For example, the complement clause for the verb  \textit{s} ‘please’ in (\ref{ex:12:11}--\ref{ex:12:13}) expresses the unrealised object of the desire. Note that the subject of the complement clause (Ø) has the same referent as the indirect object of the matrix clause (\textit{=aw} ‘to me’) in (\ref{ex:12:11}).\footnote{Compare with \REF{ex:12:22} in \sectref{sec:12.1.2} where the subject of the complement clause is different and must be specified in the clause.} 

\ea \label{ex:12:11}
{}[Asaw  [\textbf{aməpeɗe}  sese]].\\
\gll  {}[a-s=aw [Ø  \textbf{amɪ-pɛɗ{}-}ɛ ʃɛʃɛ]]\\
      \oldstylenums{3}\textsc{s}-please={\oneS}.{\IO} {} {\DEP}-crunch-{\CL}  meat\\
\glt  ‘I want to eat meat.’ (lit. to eat meat pleases me) 
\z 

\ea \label{ex:12:12}
{}[Asan  [\textbf{amadata}  \textbf{aka  va}  azan]].\\
\gll  {}[a-s=aŋ     [Ø   \textbf{ama-d=ata}  \textbf{=aka}\textbf{=va} azaŋ]]\\
      \oldstylenums{3}\textsc{s}-please=\oldstylenums{3}\textsc{s}.{\IO}  {}  {\DEP}-prepare=\oldstylenums{3}\textsc{p}.{\IO}  =on={\PRF}  temptation\\
\glt  ‘He wanted to tempt them.’ (lit. to prepare a temptation for them pleases him)
\z 

\ea \label{ex:12:13}
\corpussource  {Race story\footnote{\citealt{Friesen2003}.}}\\
{}[A\textbf{s}aw  [\textbf{aməgəye}  ambele  mbele  nə  moktonok]].\\
\gll  {}[a-\textbf{s}=aw  [Ø \textbf{amɪ-g-ijɛ}    {ambɛlɛ mbɛlɛ}  nə  mɔkʷtɔnɔkʷ]]\\
      \oldstylenums{3}\textsc{s}-please={\oneS}.{\IO}  {}  {\DEP}-do-{\CL}  race    with  toad\\
\glt  ‘I want to race with the toad.’ (lit. to do a race with the toad pleases me)
\z 

\subsection{Finite complement clauses}\label{sec:12.1.2}
\hypertarget{RefHeading1213381525720847}{}
Finite complement clauses are used with verbs that express propositional attitude, with verbs of speech, and with verbs of desire. The complement clause is the direct object of verbs expressing propositional attitude: ‘know’ \REF{ex:12:14}, ‘think’ \REF{ex:12:15}, ‘believe’ \REF{ex:12:16}, ‘doubt’ \REF{ex:12:17}, ‘forget’ \REF{ex:12:18}, and ‘worry’ \REF{ex:12:19}. Many of the examples in this section are elicited and therefore the first clause is followed by the presupposition marker \textit{na} (\sectref{sec:11.1}). This marker indicates that the first clause contains presupposed (mutually known) information (in this case, the presupposition was established by the elicitation question).


\ea \label{ex:12:14}
{}[Nasar  na  va,  [bahay  \textbf{apaɗəva}  sese]]. \\
\gll  {}[na-sar na   =va   [bahaj   \textbf{à-paɗə}\textbf{=va}   ʃɛʃɛ]]\\
      {\oneS}-know  {\DO}  ={\PRF}  chief  \oldstylenums{3}\textsc{s}+{\PFV}-crunch={\PRF}  meat\\
\glt  ‘I know it, the chief ate meat.’ 
\z 

\ea \label{ex:12:15}
{}[Nəɗəgalay  na,  [bahay  \textbf{apaɗəva}  sese]].\\
\gll  {}[nə-ɗəgal-aj   na   [bahaj   \textbf{à-paɗə}\textbf{=va}   ʃɛʃɛ]]\\
      {\oneS}-think{}-{\CL}  {\PSP}  chief  \oldstylenums{3}\textsc{s}+{\PFV}-crunch={\PRF}  meat\\
\glt  ‘I think [that] the chief ate meat.’ 
\z 

\ea \label{ex:12:16}
{}[Nafaɗ  na,  [bahay  \textbf{apaɗəva}  sese]].\\
\gll  {}[na-faɗ   na   [bahaj   \textbf{à-paɗə}\textbf{=va}    ʃɛʃɛ]]\\
      {\oneS}-place  {\PSP}  chief  \oldstylenums{3}\textsc{s}+{\PFV}-crunch={\PRF}  meat\\
\glt  ‘I believe [that] the chief ate meat.’ (lit. I place, the chief already ate meat) 
\z 

\ea \label{ex:12:17}
{}[Nəkaɗ  waya  na,  [bahay  \textbf{apaɗəva}  sese]].\\
\gll  {}[nə-kaɗ   waja   na   [bahaj   \textbf{à-paɗə}\textbf{=va}  ʃɛʃɛ]]\\
      {\oneS}-kill because {\PSP} chief \oldstylenums{3}\textsc{s}+{\PFV}-crunch={\PRF}  meat\\
\glt  ‘I doubt [that] the chief ate meat.’  (lit. I counsel that) 
\z 

\ea \label{ex:12:18}
{}[Acəkəzlaw  a  har  ava  [bahay  \textbf{apaɗəva}  sese]].\\
\gll  {}[a-tʃəkəɮ=aw   a  har  ava  [bahaj   \textbf{à-paɗə}\textbf{=va} ʃɛʃɛ]]\\
      \oldstylenums{3}\textsc{s}-forget={\oneS}.{\IO}  at  body  in  chief  \oldstylenums{3}\textsc{s}+{\PFV}-crunch={\PRF}  meat\\
\glt  ‘I forgot [that] the chief ate meat.’  
\z 

\ea \label{ex:12:19}
{}[Nazlaway  na,  [bahay  \textbf{apaɗəva}  sese]].\\
\gll  {}[na-ɮaw-aj   na   [bahaj   \textbf{à-paɗə}\textbf{=va} ʃɛʃɛ]]\\
      {\oneS}-fear{}-{\CL}  {\PSP}  chief  \oldstylenums{3}\textsc{s}+{\PFV}-crunch={\PRF}  meat\\
\glt  ‘I am afraid [that] the chief ate meat.’
\z 

Indirect speech is often expressed using a complement clause with a finite verb (\ref{ex:12:20}--\ref{ex:12:21}).

\ea \label{ex:12:20}
{}[Ne awəy  [bahay  \textbf{apaɗəva}  sese]].\\
\gll  [nɛ  awij   [bahaj  \textbf{à-paɗə}\textbf{=va}  ʃɛʃɛ]]\\
      {\oneS}  said  chief  \oldstylenums{3}\textsc{s}+{\PFV}-crunch={\PRF}  meat\\
\glt  ‘I said [that] the chief ate meat.’
\z 

\ea \label{ex:12:21}
{}[Məloko  ahay  baba  ahay  tawəy  na,\\  
\gll  {}[Mʊlɔkʷɔ=ahaj   baba=ahaj  tawij         na \\     
      Moloko=Pl      father=Pl     \oldstylenums{3}\textsc{p}+said    {\PSP}  \\    
\glt ‘The Moloko fathers say [that]’\\
      
\medskip
[Hərmbəlom \textbf{ege} ɓərav  va  kə  war  anga  məze  dedelen  ga  aka]].\\
\gll {}[Hʊrmbʊlɔm \textbf{\`ɛ-g-ɛ}  ɓərav=va kə war aŋga mɪʒɛ dɛdɛlɛŋ ga\\ 
     \hspaceThis{[}God \oldstylenums{3}\textsc{s}+{\PFV}-do-{\CL} heart={\PRF} on child {\POSS} person black  {\ADJ}\\ 
     
     \medskip
\gll aka]]\\
     on\\
\glt  ‘God got mad at the black people.’ (lit. Creator did heart on the child that belongs to black person)
\z 

Complement clauses with irrealis verbs\is{Tense, mood, and aspect!Irrealis mood} are embedded under matrix verbs of will and desire (‘wish,’ ‘want,’ ‘hope’).  \REF{ex:12:22}  shows a complement of the verb \textit{asaw } ‘it pleases me.’ The complement shows the object of the desire expressed in the matrix clause. The complement has a different subject than the {\LOC} of the desire in the matrix clause. The verb is finite and its subject is made explicit.\footnote{Compare with \REF{ex:12:11} from \sectref{sec:12.1.1} which shows a complement with the same subject as the location of the desire in the matrix clause.}  

\ea \label{ex:12:22}
Asaw  na,  [bahay  \textbf{mapaɗay} sese].\\
\gll  a-s=aw   na   [bahaj   \textbf{mà-paɗ-aj}     ʃɛʃɛ]\\
      \oldstylenums{3}\textsc{s}-please={\oneS}.{\IO}  {\PSP}  chief  \oldstylenums{3}\textsc{s}+{\HOR}{}-crunch{}-{\CL}  meat\\
\glt  ‘I want the chief to eat meat.’ (lit. that the chief should eat meat pleases me). 
\z 

In \REF{ex:12:22} above, the complement clause is the subject of the main verb; in \REF{ex:12:23}, it is the indirect object.\footnote{In Moloko, the indirect object uniformly expresses the semantic {\LOC} (e.g., recipient or beneficiary, see \chapref{chap:9}). In this case, the metaphorical location of the imploring is its purpose -- where the speaker wants to go with his actions towards the chief. The semantic Theme (the chief) is being persuaded to the {\LOC} (eating meat).} 

\clearpage
\ea \label{ex:12:23}
Nədəbakay  bahay  na  ana  [\textbf{mazom}  sese].\\
\gll  nə-dəbak-aj   bahaj   na   ana   [\textbf{mà-zɔm}  ʃɛʃɛ]\\
      {\oneS}-implore{}-{\CL}  chief  {\PSP}  {\DAT} \oldstylenums{3}\textsc{s}+{\HOR}-eat  meat\\
\glt  ‘I persuaded the chief to eat meat.’ (lit. I implored the chief to he should eat meat)
\z 

\section{Dependent adverbial clauses}\label{sec:12.2}
\hypertarget{RefHeading1213401525720847}{}
Adverbial clauses give oblique information concerning the verb in the matrix clause. The adverbial clause containing a dependent verb is embedded in the main clause as the first or last element. Adverbial clauses before the matrix clause (\ref{ex:12:24}--\ref{ex:12:26}) function to express an event in progress at the time of the matrix event or situation. The entire adverbial clause is delimited by square brackets and the dependent verb is bolded in the examples. The subject of the dependent adverbial clause must be known in the context; in \REF{ex:12:24}, the subject is Ø  and in (\ref{ex:12:25}--\ref{ex:12:26}) the subject pronoun \textit{ndahan} precedes the dependent clause. 

\ea \label{ex:12:24}
\corpussource{Disobedient Girl, S. 5}\\
{}[[\textbf{Aməhaya}  kə  ver  aka  na],  tázaɗ  war  elé  háy  bəlen].\\
\gll  [[Ø  \textbf{amə-h=aja}  kə  vɛr  aka  na]  tá-zaɗ          war   ɛlɛ  haj     bɪlɛŋ]\\
      {} {\DEP}-grind={\PLU}  on     stone  on  {\PSP}  \oldstylenums{3}\textsc{p}+{\IFV}-take  child   eye  millet  one\\
\glt  ‘For grinding on the grinding stone, they would take one grain of millet.’ 
\z 

\ea \label{ex:12:25}
{}[[Ndahan  \textbf{aməcen} məbele  a  mbəko  ahay  dəreffefe  na],  awəy, “Almay?!”]\\
\gll  {}[[ndahaŋ  \textbf{amɪ-tʃɛŋ} mɪ-bɛl-ɛ    a  mbʊkʷɔ=ahaj  dɪrɛffɛfɛ    na]\\  
 \hspaceThis{[[}\oldstylenums{3}\textsc{s} {\DEP}-hear  {\NOM}{}-move-{\CL}  {\GEN}  Mbuko=Pl  \textsc{id}:movement  {\PSP}\\  
 \glt ‘He, hearing the movement of the Mbuko' (lit. he to hear moving of Mbukos \textit{dəreffefe}),\\
 
 \medskip
 \gll awij  almaj]\\
      {said}  what\\
\glt  ‘he said, “What?!”’ 
\z 

\ea \label{ex:12:26}
\corpussource{Disobedient Girl, S. 16}\\
{}[[Ndahan  \textbf{amandava}  ɓəl  na],  zar  ahan  olo  ametele\\  
\gll  {}[[ndahaŋ  \textbf{ama-nd=ava}  ɓəl  na]  zar=ahaŋ  \`{ɔ}-lɔ amɛ-tɛl-ɛ\\ 
    \hspaceThis{[[}\oldstylenums{3}\textsc{s}         {\DEP}-sleep=in    \textsc{id}:some  {\PSP}  man=\oldstylenums{3}\textsc{s}.{\POSS}  \oldstylenums{3}\textsc{s}+{\PFV}-go   {\DEP}-walk-{\CL}\\      
\glt ‘After she had been there for some time (lit. she, sleeping there for some time), her husband went away’\\

\medskip
kə  dəlmete  ahan  aka  a  slam  enen].\\
\gll kə dɪlmɛtɛ=ahaŋ  aka a ɬam ɛnɛŋ]\\
     on   neighbor=\oldstylenums{3}\textsc{s}.{\POSS}  on   at   place   another\\
\glt  ‘to walk in the neighborhood to some place.’
\z 

Adverbial clauses that occur after the main clause (\ref{ex:12:27}--\ref{ex:12:30}) occur with verbs of movement (\textit{lo} ‘go,’ \textit{njé} ‘leave’).\footnote{We have not found clause-final adverbial clauses with other verbs.} The dependent clause expresses the (as yet unachieved) purpose of the going; in \REF{ex:12:27} the reader does not know if the subject actually bought fish or not, although it is expected. 

\ea \label{ex:12:27}
{}[Olo  a  kosoko  ava  [\textbf{aməsəkwəme} kəlef]].\\
\gll  {}[ɔ-lɔ a   kɔsɔkʷɔ   ava   [Ø  \textbf{amɪ-søkʷøm-ɛ}   kɪlɛf]]\\
      \oldstylenums{3}\textsc{s}-go  at  market  in  {}  {\DEP}-buy/sell-{\CL}  fish\\
\glt  ‘He/she went to the market [in order] to buy fish.’
\z 

\ea \label{ex:12:28}
{}[Kəlen  zar  ahan  na,  enjé  ele  ahan  [\textbf{amətele}]].\\
\gll  {}[kɪlɛŋ  zar=ahaŋ    na  ɛ{}-nʒ-ɛ    ɛlɛ=ahaŋ      [Ø \textbf{amɪ-tɛl-ɛ}]]\\
      then    man=\oldstylenums{3}\textsc{s}.{\POSS}  {\PSP}  \oldstylenums{3}\textsc{s}-leave-{\CL}  thing=\oldstylenums{3}\textsc{s}.{\POSS} {} {\DEP}-walk-{\CL}\\
\glt  ‘Then her husband left to go walking [somewhere].’ 
\z 

Likewise, in \REF{ex:12:29}, the reader does not know if the young men actually succeed at bringing back the tree (and indeed the story reveals that they do not succeed, \sectref{sec:1.6}).

\ea \label{ex:12:29}
\corpussource  {Cicada, S. 16}\\
{}[Kəlen  albaya  ahay  tolo  [\textbf{amazala}  agwazla  na]].\\
\gll  {}[kɪlɛŋ  albaja=ahaj  tɔ-lɔ  [Ø \textbf{ama-z=ala}  agʷaɮa    na]]\\
      then    {young men}=Pl  \oldstylenums{3}\textsc{p}-go   {}  {\DEP}-take=to  {spp. of tree}  {\PSP}\\
\glt  ‘Then the young men went to bring back the tree [to the chief].’
\z 

A lengthened prefix vowel in the dependent form in an adverbial clause can also express mood\is{Tense, mood, and aspect!Mood in adverbial clauses} (the desire of the speaker, see \sectref{sec:7.4.3}). The lengthened prefix vowel, bolded in (\ref{ex:12:30}), expresses potential actions which are not yet complete or even expected, but they are desired by the speaker.

\clearpage
\ea \label{ex:12:30}
{}[Bahay  ata  ahay  ɗəw  tólo  dəren  [am\textbf{aa}kəwala  ele  məzəme]].\\
\gll  {}[bahaj=atəta=ahaj  ɗuw  t\'{ɔ}-lɔ    dɪrɛŋ      {}[Ø  am\textbf{aa}-kuw=ala ɛlɛ\\  
      chief=\oldstylenums{3}\textsc{p}.{\POSS}=Pl  also  \oldstylenums{3}\textsc{p}+{\IFV}-go  far    { }  {\DEP}+{\POT}{}-seek=to  thing\\ 
      
      \medskip
\gll mɪ-ʒʊm-ɛ]]\\
     {\NOM}{}-eat-{\CL}\\ 
\glt  ‘Their chiefs also, they will have to travel far in order to find something to eat [in a famine].’ 
\z 

\section{Clauses linked by conjunctions and conjunctive adverbs}\label{sec:12.3}
\hypertarget{RefHeading1213421525720847}{}
The verbs in clauses connected by a conjunction or conjunctive adverb are always finite. The conjunction or adverb specifies the relationship between the two linked clauses. Conjuctions can be either subordinating or coordinating. Clauses subordinated by a conjunction can be divided into two types, depending on whether the subordinate clause follows or precedes the main clause (discussed in Sections~\ref{sec:12.3.1} and~\ref{sec:12.3.2}, respectively). Coordinating conjunctions link clauses that are not syntactically dependent on one another (\sectref{sec:12.3.3}). Conjunctive adverbs also function to link clauses (\sectref{sec:12.3.4}). \tabref{tab:84} shows the characteristics of all of the conjunctions and conjunctive adverbs in Moloko.

\begin{sidewaystable}{\footnotesize
\begin{tabular}{lll}
\lsptoprule
 & \textbf{Conjunction} & \textbf{Function}\\\midrule
\textbf{Subordinate conjunction   }  & \textit{bəyna} ‘because’ & Demonstrates the proof for the statement  \\ 
Introduces an adverbial  & & in the matrix clause\\
clause following the \textbf{} &  \textit{waya} ‘because’ & Indicates the reason for something \\
matrix clause. & &  described in the previous clauses \\
 & \textit{kəwaya} ‘because’ / ‘that is’ & Introduces clauses (or noun phrases) that give \\
& & the reasoning of the speaker with regard to \\
& & the situation expressed in previous clauses. \\
& \textit{ha} ‘until’ & Expresses a literal or metaphorical boundary\\
& &  that marks the end of the activity or situation \\
& & expressed in the matrix clause.\\\midrule
\textbf{Conditional conjunction} & \textit{asa…na} ‘if…{\PSP}’ & Condition is presupposed or a real possibility \\
Introduces a conditional &  \textit{asa…ɗəw}  ‘if…{also}’ & Condition presents a new possibility.\\
clause that precedes the  &   \textit{ana}\textit{ }\textit{asa} …na ‘to if...{\PSP}’ & Condition presents a strong expectation \\
matrix clause.  & & to be fulfilled.\\
 & \textit{azana} \textit{asa} …na ‘maybe if...{\PSP}’ & Condition might be fulfilled.\\\midrule
\textbf{Coordinate conjunction } & \textit{nata} ‘and (then)’ & Marks a clause which contains the most \\
 Links independent clauses & & pivotal event in a narrative.\\
 & \textit{azləna} ‘but’ & Contains an element of counterexpectation \\
 & & with something in the previous clause.\\\midrule
\textbf{Conjunctive adverb} & \textit{kəlen} ‘next’ & Indicates the next mainline event. \\
Functions to relate clause & & Often follows a digression \\
to mainline events.  & \textit{ndɛ} ‘therefore’ & Marks conclusive statements.\\
& \textit{macəkəmbay} ‘meanwhile’ & Marks information off the main event line. \\
\lspbottomrule
\end{tabular}}
\caption{Subordinating and coordinating conjunctions\label{tab:84}}
\end{sidewaystable}

\subsection{Adverbial clauses introduced by a subordinating conjunction}\label{sec:12.3.1}
\hypertarget{RefHeading1213441525720847}{}
When an adverbial clause introduced by a subordinating conjunction follows the main clause, the adverbial clause supplies new information to the discourse. The different subordinate conjunctions specify the relationship between the new information and the matrix clause. Subordinating conjunctions include \textit{bəyna} ‘because,’ \textit{ waya} ‘because,’ and \textit{kəwaya} ‘because,’ (all involved in reason-result constructions) and \textit{ha} ‘until.’ Sentences in Moloko do not normally have multiple subordinate clauses. In the examples, each of the subordinate clauses is delimited by square brackets.  

\textit{Bəyna}\footnote{\textit{Bəyna} ‘because’ may be a compound of the negative \textit{bay} and the presupposition marker \textit{na}. }  ‘because’ (\ref{ex:12:31}--\ref{ex:12:34}) is used in result-reason clause constructions that link only two clauses. The clause subordinated by \textit{bəyna} demonstrates the proof for the statement in the matrix clause. 

\clearpage
\ea \label{ex:12:31}
\corpussource  {Snake, S. 19}\\
Nəngehe  na,   Hərmbəlom  aloko  ehɛ,  [\textbf{bəyna}  anjakay  nok\\ 
\gll  nɪŋgɛhɛ na   Hʊrmbʊlɔm=alɔkʷɔ  ɛhɛ  [\textbf{bijna} à-nzak-aj  nɔkʷ \\ 
      {\DEM}     {\PSP}       God=\oldstylenums{1}\textsc{Pin}.{\POSS}   here      because   \oldstylenums{3}\textsc{s}+{\PFV}-find{}-{\CL}  {\twoS} \\    
\glt ‘This one here, our God [is really] here [with us], because it found you’\\      
      
      \medskip
ha  a  slam  məndəye  ango  ava].\\      
\gll ha a ɬam mɪ-nd-ijɛ=aŋgʷɔ ava]\\
     until  at    place  {\NOM}{}-sleep-{\CL}={\twoS}.{\POSS}  in\\
\glt  ‘even in your bed (lit. all the way to the place of your lying).’ 
\z 

\ea \label{ex:12:32}
Náavəlaləkwəye  səloy  [\textbf{bəyna}  kogom  va  slərele  gam].\\
\gll  náá-vəl=alʊkʷøjɛ   sʊlɔj     [\textbf{bijna}  kɔ-gʷ{}-ɔm =va  ɬɪrɛlɛ  gam]\\
      {\oneS}+{\POT}-give={\twoP}.{\IO}  coin    because  2-do-{\twoP}  ={\PRF}  work  much\\
\glt  ‘I will give you money because you have done a lot of work.’
\z  

\ea \label{ex:12:33}
Nazala  məlama  əwla  a  lopəytal  ava\\  
\gll  nà-z=ala    məlama=uwla    a  lɔpijtal  ava \\ 
      {\oneS}+{\PFV}-take=to  sibling={\oneS}.{\POSS}  at  hospital  in \\ 
\glt ‘I took my brother to the hospital’\\
      
      \medskip
[\textbf{bəyna}  dəngo  awəlan].\\      
\gll {}[\textbf{bijna}  dʊŋgʷɔ   a-wəl=aŋ]\\
     because  throat  \oldstylenums{3}\textsc{s}-hurt=\oldstylenums{3}\textsc{s}.{\IO}\\
\glt  ‘because his throat was hurting.’ 
\z 

\ea \label{ex:12:34}
\corpussource{Cicada, S. 14}\\
Ɗeɗen  bahay,  agwazla  ngəndəye  ágasaka  ka  mahay  ango  aka,\\  
\gll  ɗɛɗɛŋ  bahaj  agʷaɮa  ŋgɪndijɛ  á-gas=aka  ka  mahaj=aŋgʷɔ aka\\ 
      truth  chief {spp. of tree} {\DEM} \oldstylenums{3}\textsc{s}+{\IFV}-get=on  on  door={\twoS}.{\POSS}  on \\ 
\glt ‘True, chief, it would be pleasing if that particular tree would be by your door,’\\      
      
      \medskip
 {}[\textbf{bəyna}  agwazla  ga  səlom  ga; aɓəsay  ava  bay].\\     
\gll {}[\textbf{bijna} agʷaɮa ga sʊlɔm ga aɓəsaj ava baj]\\      
     because {spp. of tree}  {\ADJ} good  {\ADJ}   blemish   {\EXT}  {\NEG}\\ 
\glt  ‘because this tree is good;  it has no faults.’
\z 

\largerpage
A clause subordinated by \textit{waya} ‘because’ (\ref{ex:12:35}{}--\ref{ex:12:36}) indicates the reason for something that is described in the previous clauses. The preceding clauses function to give a context for the statement in the \textit{waya} clause. In \REF{ex:12:35} (from S. 7--8 of the Disobedient Girl story \sectref{sec:1.5}) the \textit{waya} clause provides an explanation for the events in the preceding paragraph. S. 7 gives the result (one grain of millet would give enough food for a family) and S. 8 gives the reason behind it (because the millet multiplied while the flour was being ground).

\ea \label{ex:12:35}
\corpussource{Disobedient Girl, S. 7}\\
War  elé  háy  bəlen  fan  na,\\  
\gll  war ɛlɛ haj bɪlɛŋ faŋ na\\ 
      child   eye   millet   one    yet   {\PSP}\\     
\glt ‘Just one grain of millet,’\\

\medskip
ánjata  pew  ha  ámbaɗ  ese.\\  
\gll á-nz=ata pɛw ha á-mbaɗ ɛʃɛ \\ 
     \oldstylenums{3}\textsc{s}+{\IFV}-suffice=\oldstylenums{3}\textsc{p}.{\IO} enough   until  \oldstylenums{3}\textsc{s}+{\IFV}-remain   again \\
\glt ‘it sufficed for them, even to leaving leftovers.’\\

\medskip\corpussource{Disobedient Girl, S. 8}\\\relax
{}[\textbf{Waya}  a  məhaya  ahan  ava  na,\\  
\gll {}[\textbf{waja}  a mə-h=aja=ahaŋ ava na\\ 
     because  at   {\NOM}{}-grind={\PLU}=\oldstylenums{3}\textsc{s}.{\POSS}   in   {\PSP}  \\ 
\glt ‘Because, during its grinding,’\\

\medskip
ásak  kə  ver  aka  nə  məsəke].\\
\gll á-sak kə vɛr aka nə mɪ-ʃɪk-ɛ]\\
     \oldstylenums{3}\textsc{s}+{\IFV}-multiply   on   {grinding stone}    on    with   {\NOM}{}-multiply-{\CL}\\
\glt  ‘it would really multiply on the grinding stone.’ (lit. multiply with multiplying)  
\z 

Another result-reason construction with \textit{waya} is shown in \REF{ex:12:36} (from part of a story not illustrated in this work).  The clause subordinated by \textit{waya} explains the reason why the speaker didn’t know how to proceed. It was important in the story that the speaker had already visited the subprefect.

\ea \label{ex:12:36}
Nasar  həraf  ele  nəngehe  asabay\\  
\gll  nà-sar    həraf    ɛlɛ  nɪŋgɛhɛ  asa-baj \\
      {\oneS}+{\PFV}-know  medicine   thing  {\DEM}  again-{\NEG} \\ 
\glt ‘I didn’t know how to resolve the problem (lit. I never knew the medicine for this particular thing),’\\
\clearpage
\medskip
{}[\textbf{waya} nəlva  afa  səwpərefe].\\
\gll {}[\textbf{waja} n\`{ə}-l=va afa suwpɪrɛfɛ]\\
     because  {\oneS}+{\PFV}-go={\PRF}    {at house of}    {subprefect}\\
\glt  ‘because I had already been to the subprefect [and he didn’t help me].’ 
\z 

The demonstrative \textit{ndana} in the phrase \textit{waya ndana} refers the hearer to the previously-mentioned clauses to discover the reason behind the statement introduced by \textit{waya ndana}. In the reason-result construction shown in \REF{ex:12:37} (from the Disobedient Girl story), S. 34 states that God had gotten angry because of the girl that disobeyed. The \textit{waya ndana} clause in S. 35 identifies that the information in S.34 is the reason for the statement in S. 35; it was because of God’s anger that God took back his blessing from the Moloko. 

\ea \label{ex:12:37}
\corpussource{Disobedient Girl, S. 33}\\
Hərmbəlom  ága  ɓərav  va  kəwaya  war  dalay  na,\\  
\gll  Hʊrmbʊlɔm á-ga ɓərav=va kuwaja war dalaj na\\ 
     God \oldstylenums{3}\textsc{s}+{\IFV}-do heart={\PRF} {because of} child girl {\PSP}\\
\glt ‘God got angry because of that girl,’\\     
     
     \medskip
amecen  sləmay  baj  ngəndəye.  \\     
\gll amɛ-tʃɛŋ ɬəmaj baj ŋgɪndijɛ \\
     {\DEP}-hear ear {\NEG} {\DEM} \\
\glt  ‘that one who was disobedient.’ \\ 

\medskip
\corpussource{Disobedient Girl, S. 35}\\\relax
{}[\textbf{Waya}  ndana  Hərmbəlom  ázata  aka  barka  ahan  va].\\
\gll {}[\textbf{waja} ndana Hʊrmbʊlɔm á-z=ata=aka barka=ahaŋ=va]\\ 
     because   {\DEM} God \oldstylenums{3}\textsc{s}+{\IFV}-take=\oldstylenums{3}\textsc{p}.{\IO}=on blessing=\oldstylenums{3}\textsc{s}.{\POSS}={\PRF} \\    
\glt ‘Because of that previously-mentioned [event], God had taken back his blessing from them.’
\z 

The conjunction \textit{kəwaya}  ‘because’ / ‘that is’ (\ref{ex:12:38}--\ref{ex:12:39}, reproduced here from \ref{ex:12:37}) introduces clauses (or noun phrases) that explaination the situation expressed in previous clauses. \textit{Kəwaya} introduces the conditional construction in \REF{ex:12:38} (from the Disobedient Girl story S. 3-4) that gave the reasoning behind the blessing that the Molokos experienced in the past. 

\clearpage
\largerpage
\ea \label{ex:12:38}\corpussource{Disobedient Girl, S. 3}\\
Zlezle na, Məloko ahay na, Hərmbəlom ávəlata barka va.\\ 
\gll  ɮɛɮɛ na Mʊlɔkʷɔ=ahaj na Hʊrmbʊlɔm á-vəl=ata \\ 
     {long ago} {\PSP} Moloko=Pl  {\PSP} God \oldstylenums{3}\textsc{s}+{\IFV}-send=\oldstylenums{3}\textsc{s}.{\IO}   \\   
      
      \medskip
\gll barka=va\\
     blessing={\PRF}\\
\glt ‘Long ago, to the Moloko people, God had given his blessing.’\\

\medskip\corpussource{Disobedient Girl, S. 4}\\\relax
{}[\textbf{Kəwaya} asa  təwasva  nekwen  kəygehe  ɗəw],\\  
\gll {}[\textbf{kuwaja} asa t\`{ə}-was=va nɛkʷɛŋ kijgɛhɛ ɗuw]\\ 
     {that is}  if \oldstylenums{3}\textsc{p}+{\PFV}-cultivate={\PRF} little {like this} also\\ 
\glt ‘That is, even if they had only cultivated a little [millet] like this,’\\

\medskip
ávata  məvəye  haɗa.\\
\gll á-v=ata mɪ-v-ijɛ haɗa\\
    \oldstylenums{3}\textsc{s}+{\IFV}-{spend time}=\oldstylenums{3}\textsc{p}.{\IO} {\NOM}{}-{spend time}-{\CL} {a lot}\\
\glt  ‘it would last them enough for the whole year.’
\z  

In the conclusion of the same story (\ref{ex:12:39}), \textit{kəwaya} introduces a noun phrase with a relative clause that gives the reason for God’s anger.

\ea \label{ex:12:39}
\corpussource{Disobedient Girl, S. 33}\\
Hərmbəlom  ága  ɓərav  va\\  
\gll  Hʊrmbʊlɔm   á-ga  ɓərav=va \\   
      God    \oldstylenums{3}\textsc{s}+{\IFV}-do    heart={\PRF} \\
\glt ‘God got angry (lit. did heart)’\\      
      
      \medskip
{}[\textbf{kəwaya}  war  dalay  amecen  sləmay  bay  ngəndəye].\\      
\gll {}[\textbf{kuwaja} war dalaj amɛ-tʃɛŋ ɬəmaj baj ŋgɪndijɛ] \\
     because  child  female  {\DEP}-hear  ear  {\NEG}  that\\
\glt  ‘because of that girl, that one who was disobedient.’
\z 

The clause introduced by \textit{ha} ‘until’ expresses a literal or metaphorical boundary that marks the cessation of the activity or situation expressed by the matrix clause (\ref{ex:12:40},\footnote{From the Race story, \citealt{Friesen2003}.} \ref{ex:12:41}).

\clearpage
\ea \label{ex:12:40}
Kərcece  ahəmay  ahəmay  ahəmay\\ 
\gll  kɪrtʃɛtʃɛ a-həm-aj    a-həm-aj  a-həm-aj \\  
      giraffe  \oldstylenums{3}\textsc{s}-run-{\CL} \oldstylenums{3}\textsc{s}-run-{\CL} \oldstylenums{3}\textsc{s}-run-{\CL}\\  
\glt ‘The giraffe ran and ran and ran’\\

\medskip
{} [\textbf{ha}  ayaɗay  ndele  pəs  pəssa].\\
\gll {}[\textbf{ha}   a-jaɗ-aj     {ndɛlɛ pəs pəs}=sa]\\ 
     until  \oldstylenums{3}\textsc{s}-tire-{\CL}  {\textsc{id}:completely tired}={\ADV}\\
\glt  ‘until he was completely tired out.’ 
\z 

In \REF{ex:12:41}, the second clause begins with \textit{ha} ‘until’ and gives adverbial information to the matrix clause concerning how long that one grain of millet will satisfy their hunger. 

\ea \label{ex:12:41}
\corpussource {Disobedient Girl, S. 7}\\
War  elé  háy  bəlen  fan  na,  ánjata  pew \\ 
\gll  war    ɛlɛ   haj   bɪlɛŋ   faŋ  na      á-nz=ata  pɛw \\     
      child     eye  millet  one  already  {\PSP}  \oldstylenums{3}\textsc{s}+{\IFV}-suffice=\oldstylenums{3}\textsc{p}.{\IO}  enough \\     
\glt ‘One grain of millet, it sufficed for them’\\

\medskip
{}[\textbf{ha}  ambaɗ  ese].\\
\gll {}[\textbf{ha} a-mbaɗ ɛʃɛ]\\
     until \oldstylenums{3}\textsc{s}-remain   again\\ 
\glt  ‘even to leaving leftovers.’ (lit. until it remained again)
\z 

\subsection{Conditional construction}\label{sec:12.3.2}
\hypertarget{RefHeading1213461525720847}{}
The subordinating conjuction \textit{asa} ‘if’ intoduces a condition on the realisation of the event expressed by the main clause. The construction is \textit{asa} plus the conditional clause.  The end of the subordinate clause is delimited by the presupposition marker \textit{na} or the particle \textit{ɗəw} 'also.' Which marker is employed depends upon speaker assessment. If the presupposition marker \textit{na} delimits the condition (\ref{ex:12:42}--\ref{ex:12:44}), the clause is neutral with respect to whether the speaker expects the condition to be fulfilled or not.  In the examples of this section, both the subordinating conjunction and presupposition or ‘unexpected’ information marker are bolded, and the subordinate clause is delimited by square brackets.


\ea \label{ex:12:42}
{}[\textbf{Asa} kége  akar  \textbf{na}],  náaɓok.\\
\gll  {}[\textbf{asa}  k\'{ɛ}-g-ɛ     akar   na]  náá-ɓ=ɔkʷ\\
      if    {\twoS}+{\IFV}-do-{\CL}  theft  {\PSP}  {\oneS}+{\POT}-beat={\twoS}.{\IO}\\
\glt  ‘If you steal, I will beat you.’
\z  

\ea \label{ex:12:43}
{}[\textbf{Asa} ások njəwelek  \textbf{na}],  kándaɗay  elele  kəlen.\\
\gll  {}[\textbf{asa} á-s=ɔkʷ       nʒuwɛlɛk  na]  ká-ndaɗ-aj  ɛlɛlɛ  kɪlɛŋ\\
      if  \oldstylenums{3}\textsc{s}+{\IFV}-cut={\twoS}.{\IO}  {spp. of leaf)}  {\PSP}  {\twoS}+{\IFV}-like{}-{\CL}  sauce  then\\
\glt  ‘If you like this kind of leaf, you will like this sauce.’  
\z 

\ea \label{ex:12:44}
{} [\textbf{Asa}  taɓan  va  ana  məze  \textbf{na}], \\ 
\gll  {}[\textbf{asa}  tà-ɓ=aŋ  =va  ana  mɪʒɛ  na]\\ 
      if  \oldstylenums{3}\textsc{p}+{\PFV}-hit=\oldstylenums{3}\textsc{s}.{\IO}  ={\PRF}  {\DAT} person  {\PSP} \\     
\glt ‘If someone has gotten beaten,’\\

\medskip
ləkwəye  na,  gom  ala  sərtəfka  medekal  aləkwəye.\\
\gll lʊkʷøjɛ na g-ɔm =ala sərtfka mɛdɛkal=alʊkʷøjɛ\\
     {\twoP}    {\PSP}   do[{\IMP}]-{\twoP}   =to  certificate  medical={\twoP}.{\POSS} \\
 \glt  ‘make a medical certificate for him.’ (lit. do for him your medical certificate)
\z 

When the subordinated clause is delimited by \textit{ɗəw} ‘also' (\ref{ex:12:45}--\ref{ex:12:46}), the meaning of \textit{asa} shifts to more of a concessive idea.

\ea \label{ex:12:45}
\corpussource{Disobedient Girl, S. 4}\\
{}[\textbf{Asa} təwas  va  nekwen  kəygehe  \textbf{ɗəw}],\\  
\gll  {}[\textbf{asa}  t\`{ə}-was=va   nɛkʷɛŋ   kijgɛhɛ  \textbf{ɗuw}]\\ 
      if    \oldstylenums{3}\textsc{p}+{\PFV}-cultivate={\PRF}  little  {like this}  also \\     
\glt ‘Even if they had only cultivated a little [millet] like this,’\\

\medskip
ávata  məvəye  haɗa.\\
\gll á-v=ata mɪ-v-ijɛ haɗa\\
     \oldstylenums{3}\textsc{s}+{\IFV}-{spend time}=\oldstylenums{3}\textsc{p}.{\IO}  {\NOM}{}-{spend time}-{\CL}  many\\
\glt  ‘it would last them enough for the whole year.’
\z 

\ea \label{ex:12:46}
{}[\textbf{Asa} məze  ahay  təcahay  ele  \textbf{ɗəw}],  Hərmbəlom  ecen  asabay.\\
\gll  {}[\textbf{asa}  mɪʒɛ=ahaj   tə-tsah-aj  ɛlɛ       \textbf{ɗuw}]   Hʊrmbʊlɔm ɛ-tʃɛŋ asa-baj\\
      if       person=Pl        \oldstylenums{3}\textsc{p}-ask{}-{\CL} thing  also  God \oldstylenums{3}\textsc{s}-hear again-{\NEG}\\
\glt  ‘Even if people ask for anything, God doesn’t hear anymore.’  
\z 

Normally the subordinated clause is followed by the main clause (\ref{ex:12:42}--\ref{ex:12:47}), however the clause expressing the condition can be right-shifted in some contexts \REF{ex:12:48}. The \textit{asa} clause is always delimited by \textit{na}.

\clearpage
\ea \label{ex:12:47}
\corpussource{Disobedient Girl, S. 13}\\
{}[\textbf{Asa}  asok  aməhaya  \textbf{na}], \\ 
\gll  {}[\textbf{asa}   à-s=ɔkʷ  amə-h=aja  na]\\ 
      if  \oldstylenums{3}\textsc{s}+{\IFV}-please={\twoS}.{\IO}  {\DEP}-grind={\PLU}  {\PSP}\\ 
\glt ‘If you want to grind,’ (lit. if grinding pleases to you)\\

\medskip
kázaɗ  war  elé  háy  bəlen.\\
\gll ká-zaɗ war ɛlɛ haj bɪlɛŋ\\
     {\twoS}+{\IFV}-take  child eye  millet  one\\
\glt  ‘you take only one grain.’
\z  

\ea \label{ex:12:48}
Gəbar  anday  agaw  [\textbf{asa}  bahay  apaɗay  sese  \textbf{na}].\\
\gll  gəbar  a-ndaj    a-g=aw    [\textbf{asa}  bahaj   à-paɗ-aj  ʃɛʃɛ   na]\\
      fear    \oldstylenums{3}\textsc{s}-{\PROG}  \oldstylenums{3}\textsc{s}-do={\oneS}.{\IO}  if  chief  \oldstylenums{3}\textsc{s}+{\PFV}-crunch{}-{\CL}  meat  {\PSP}\\
\glt  ‘I am afraid that the chief ate meat.’ (lit. fear is doing me if the chief ate meat)  
\z 

Other particles co-occurring with the conjunction \textit{asa} ‘if’ can modify its force. Clauses subordinated by the dative marker plus ‘if’ \textit{ana} \textit{asa} have a strong  expectation that the condition will be fulfilled \REF{ex:12:49}, while clauses subordinated by \textit{azana} \textit{asa} ‘maybe if’ carry the expectation that the condition might be fulfilled, rendering the subordinating clause to have almost a temporal meaning \REF{ex:12:50}. 

\ea \label{ex:12:49}
 [\textbf{Ana  asa} kege  akar  bay  \textbf{na}],  náɓok bay. \\
\gll  {}[\textbf{ana} \textbf{asa} k\`{ɛ}-g-ɛ    akar  baj  na]  ná-ɓ=ɔkʷ    baj\\
      {\DAT} if  {\twoS}+{\PFV}-do-{\CL}  theft  {\NEG}  {\PSP}  {\oneS}+{\IFV}-beat={\twoS}.{\IO}  {\NEG}\\
\glt  ‘If you don’t steal [and I don’t expect you to steal], I won’t beat you.’
\z 

\ea \label{ex:12:50}
{}[\textbf{Azana  asa}  tanday  təzlaɓay  ele  memey  \textbf{na}],\\  
\gll  {}[\textbf{azana} \textbf{asa}     ta-ndaj    tə-ɮaɓ-aj    ɛlɛ    mɛmɛj   na]\\  
      maybe    if         \oldstylenums{3}\textsc{p}-{\PROG}  \oldstylenums{3}\textsc{p}-pound{}-{\CL}   thing   how      {\PSP} \\    
\glt ‘When something is being pounded,’ (lit. if perhaps they are pounding something,)\\

\medskip
t\'{ə}zləgalay  avəlo  bay.\\
\gll t\'{ə}-ɮəg=alaj avʊlɔ baj\\
      \oldstylenums{3}\textsc{p}+{\IFV}-throw=away   {high up}  {\NEG}\\
\glt  ‘the baton is not thrown too high.’ (lit. they don’t throw the baton too high)
\z 

\subsection{Coordinate constructions}\label{sec:12.3.3}
\hypertarget{RefHeading1213481525720847}{}
\largerpage
Coordinate constructions consists of two independent clauses linked by a coordinate conjunction. The coordinating conjunction specifies the way that the clauses are connected. They include \textit{nata} ‘and then’ and \textit{azləna} ‘but.’ In (\ref{ex:12:51}--\ref{ex:12:55}), the conjunction is bolded and the coordinate clause is delimited by square brackets. 

\textit{Nata} ‘and then’ marks the clauses which contain the most pivotal events in a narrative\is{Focus and prominence!Discourse peak}. Two clauses from the Cicada narrative are marked with \textit{nata} \REF{ex:12:51}. These two clauses mark the peak event of the cicada’s success at transporting the tree for the chief. One clause in the peak of the Disobedient Girl narrative is marked with \textit{nata} \REF{ex:12:52}. This marked peak event is the death of the girl, the result of her disobedience. 

\ea \label{ex:12:51}\corpussource{Cicada, S. 25}\\
{}[\textbf{Nata}  olo],\\ 
\gll  {}[\textbf{nata} \`{ɔ}-lɔ] \\ 
      {and then}     \oldstylenums{3}\textsc{s}+{\PFV}-go \\ 
\glt  ‘And then he went,’ \\

\medskip
\corpussource{Cicada, S. 26}\\
albaya  ahay  tolo  sen  na,\\ 
\gll albaja=ahaj tɔ-lɔ ʃɛŋ na\\ 
     youth=Pl \oldstylenums{3}\textsc{p}-go \textsc{id}:go {\PSP}\\ 
\glt ‘The young men went,’\\

\medskip
albaya  ahay  weley  təh  anan  dəray  na,  abay.\\  
\gll albaja=ahaj wɛlɛj təx an=aŋ dəraj na abaj\\
     youth=Pl which \textsc{id}:put   {\DAT}=\oldstylenums{3}\textsc{s}.{\IO}   head   {\PSP}   {{\EXT}+{\NEG}} \\
\glt ‘None of the young men could lift it.’ (lit. whichever young man put his head [to the tree in order to lift it], there was none) \\

\medskip  
\corpussource{Cicada, S. 27}\\
{}[\textbf{Nata}  mətəde  təh  anan  dəray  ana  agwazla  ngəndəye].\\
\gll  {}[\textbf{nata}  mɪtɪdɛ təx an=aŋ dəraj ana agʷaɮa ŋgɪndijɛ]\\
      {and then}  cicada  {\textsc{id}:put on head}  {\DAT}=\oldstylenums{3}\textsc{s}.{\IO}  head  {\DAT} {spp. of tree}  {\DEM}\\
\glt  ‘And then the cicada put his head to that tree.’ 
\z 

\clearpage
\ea \label{ex:12:52}
\corpussource{Disobedient Girl, S. 26}\\
Alala na,  ver na árah mbəf nə həmbo na,\\  
\gll  a-l=ala na vɛr na á-rax mbəf nə hʊmbɔ na\\  
      \oldstylenums{3}\textsc{s}-go=to  {\PSP}  room  {\PSP}  \oldstylenums{3}\textsc{s}+{\IFV}-fill  {\textsc{id}:up to the roof}  with  flour  {\PSP}\\            
\glt ‘Later, the room, it filled to the roof with flour,’\\      
      
\medskip 
ɗək məɗəkaka alay ana hor na,\\
\gll ɗək mə-ɗək=aka=alaj ana hʷɔr na\\
     plug  {\NOM}{}-plug=on=away  {\DAT} woman  {\PSP}\\
\glt ‘it suffocated the woman,’\\
 
\medskip
{}[\textbf{nata} ndahan dəɓəsolək məmətava alay a hoɗ a hay na ava].\\
\gll {}[\textbf{nata} ndahaŋ dʊɓʊsɔlʊkʷ mə-mət=ava=alaj a hʷɔɗ a haj \\
     then    \oldstylenums{3}\textsc{s}   \textsc{id}:collapse/die  {\NOM}{}-die=in=away  at  stomach  {\GEN}  house  \\

\medskip
\gll na ava]\\
 {\PSP}  in\\
\glt  ‘and then she collapsed \textit{dəɓəsolək}, dying inside the house.’
\z 

\textit{Azləna}\footnote{\textit{Azləna} ‘but’ may be a compound of \textit{azla} ‘now’ and the presupposition marker \textit{na}.} ‘but’ indicates that the clause that follows will contain an element of counter-expectation to something in the previous clause (\ref{ex:12:53}--\ref{ex:12:55}).

\ea \label{ex:12:53}
\corpussource{Disobedient Girl, S. 10 - 11}\\
Olo  azala  dalay.  [\textbf{Azləna}  war  dalay  ndana  cekəzlere  ga].\\
\gll  à-lɔ   à-z=ala   dalaj  [\textbf{aɮəna}  war   dalaj   ndana   tʃɛkɪɮɛrɛ\\   
      \oldstylenums{3}\textsc{s}+{\PFV}-go  \oldstylenums{3}\textsc{s}+{\PFV}-take=to  girl  but  girl  female  {\DEM}  disobedience\\ 
      
      \medskip
      ga]\\
      {\ADJ}\\
\glt  ‘He went and took a wife, but the girl [was] disobedient.’
\z 

\ea \label{ex:12:54}
Avəyon  agan  va  gəɓar  ana  Abangay.  Ahəman  alay  nekwen.\\  
\gll  avijɔŋ  a-g=aŋ       =va     gəɓar   ana   Abaŋgaj    a-həm=aŋ  =alaj   nɛkʷɛŋ \\     
      airplane  \oldstylenums{3}\textsc{s}-do=\oldstylenums{3}\textsc{s}.{\IO} ={\PRF}  fear  {\DAT} Abangay  \oldstylenums{3}\textsc{s}-run=\oldstylenums{3}\textsc{s}.{\IO}   =to    little  \\         
\glt ‘The airplane made Abangay afraid (lit. did fear to her), [so] she ran away a little.’\\

\medskip
\clearpage
{}[\textbf{Azləna}  na  me,  ləme  nata  babəza  ahay  na,\\  
\gll {}[\textbf{aɮəna} na mɛ lɪmɛ nata babəza=ahaj na\\ 
     but    {\PSP}  opinion  \oldstylenums{1}\textsc{Pex}  and  children=Pl  {\PSP} \\
\glt ‘But on the other hand, I and the children,’\\

\medskip
ko  məbele  nekwen  ɗəw,  nobəlom  bay].\\
\gll kʷɔ mɪ-bɛl-ɛ nɛkʷɛŋ ɗuw nɔ-bʊl-ɔm baj]\\
     even  {\NOM}{}-move-{\CL}  little  also  {\oneS}-move-\oldstylenums{1}\textsc{Pex}  {\NEG}\\
\glt  ‘we didn’t budge even a little (lit. even a little movement we didn’t move).’ 
\z 

\largerpage
\ea \label{ex:12:55}
Nahan  ana  hor  əwla  ne  awəy  majaw  ala  yam\\  
\gll  na-h=aŋ    ana  hʷɔr=uwla    nɛ  awij   mà-dz=aw  =ala jam\\  
      {\oneS}-tell=\oldstylenums{3}\textsc{s}.{\IO}  {\DAT} wife={\oneS}.{\POSS}  {\oneS}  said      \oldstylenums{3}\textsc{s}+{\HOR}-help={\oneS}.{\IO}    =to  water\\  
\glt ‘I told my wife to bring me water’\\     

\medskip
aməbele;  [\textbf{azləna}  acahay  bay].\\
\gll amɪ-bɛl-ɛ [\textbf{aɮəna} a-tsah-aj baj]\\
     {\DEP}-wash-{\CL}      but    \oldstylenums{3}\textsc{s}-obey-{\CL}  {\NEG}\\
\glt  ‘to wash; but she didn’t obey me.’ 
\z 

\subsection{Adverbial clauses with conjunctive adverbs}\label{sec:12.3.4}
\hypertarget{RefHeading1213501525720847}{}
Conjunctive adverbs are adverbs that function to connect clauses within a larger  discourse. They include \textit{kəlen} ‘next,’ \textit{nde} ‘therefore,’ and \textit{macəkəmbay} ‘meanwhile.’ With the exception of \textit{kəlen}, conjunctive adverbs are clause-initial. The examples give some of the surrounding context so that their function can be demonstrated. Many of the examples are from the Disobedient Girl story or the Cicada story. In order to study the larger context for the examples, the stories themselves can be found in  \sectref{sec:1.5} and \sectref{sec:1.6}, respectively. The clauses with conjunctive adverbs are delimited by square brackets.

\textit{Kəlen} ‘next’ indicates a subsequent mainline event that often follows a digression (often reported speech). This conjunction can either be clause-initial \REF{ex:12:57} or follow the first argument in the clause \REF{ex:12:56}.  

\ea \label{ex:12:56}
\corpussource  {Cicada, S. 5--6}\\
Tánday  t\'{ə}talay  a  ləhe  na,\\    
\gll  tá-ndaj t\'{ə}-tal-aj a lɪhɛ  na\\  
      \oldstylenums{3}\textsc{p}+{\IFV}-{\PROG}   \oldstylenums{3}\textsc{p}+{\IFV}-walk-{\CL}     at   bush   {\PSP} \\          
\glt ‘[As] they were walking in the bush,’\\

\medskip
\clearpage
təlo  tənjakay  agwazla  malan  ga  a  ləhe.\\
\gll t\`{ə}-lɔ t\`{ə}-nzak-aj agʷaɮa malaŋ ga a lɪhɛ \\
     \oldstylenums{3}\textsc{p}+{\PFV}-go   \oldstylenums{3}\textsc{p}+{\PFV}-find-{\CL}   {spp. of tree}    large   {\ADJ}      to   bush \\
\glt ‘they went and found a large tree (a particular species) in the bush.’\\

\medskip
{}[Albaya  ahay  ndana \textbf{kəlen} təngalala  ma  ana  bahay].\\
\gll {}[albaja=ahaj ndana \textbf{kɪlɛŋ} t\`{ə}-ŋgala=ala ma ana bahaj]\\
     youth=Pl       {\DEM}  then  \oldstylenums{3}\textsc{p}+{\PFV}-return=to    word    {\DAT}   chief\\
\glt  ‘Those young men then took the word (response) to the chief.’  
\z 

Clauses S. 7 and 8 are shown in \REF{ex:12:57}. \textit{Kəlen} functions to signal to the hearer that the events in S. 8 are part of the event line (the reported speech in S. 7 was a digression from the event line).

\ea \label{ex:12:57}
\corpussource{Cicada, S. 7}\\
Tawəy, “Bahay, mama agwazla ava a ləhe na, malan ga na,\\  
\gll  tawij  bahaj mama agʷaɮa ava a lɪhɛ na malaŋ ga na\\ 
      \oldstylenums{3}\textsc{p}+said  chief mother {spp. of tree} {\EXT} at bush {\PSP} large {\ADJ} {\PSP}\\ 
\glt ‘They said, “Chief, there is a mother-tree in the bush,  a big one,’\\

\medskip
agasaka na ka mahay ango aka aməmbese.”\\
\gll à-gas=aka na ka mahaj=aŋgʷɔ aka amɪ-mbɛʃ-ɛ\\ 
     \oldstylenums{3}\textsc{s}+{\PFV}-get=on  {\PSP}  on door={\twoS}.{\POSS} on {\DEP}-rest{}-{\CL}\\
\glt  ‘[and] it would please you to have that tree at your door, so that you could rest under it.”’\\

\medskip
\corpussource{Cicada, S. 8}\\
{}[\textbf{Kəlen} albaya ahay ndana  tolo].\\
\gll  {}[\textbf{kɪlɛŋ} albaja=ahaj ndana t\`{ɔ}-lɔ]\\
      next    youth=Pl  {\DEM}  \oldstylenums{3}\textsc{p}+{\PFV}-go\\
\glt  ‘Then those young men went.’ 
\z 

A conclusion in a discourse or a concluding remark may be introduced by the conjunctive adverb \textit{nde} ‘so.’ Example \REF{ex:12:58} shows S. 32--34 from the conclusion of the Disobedient Girl narrative.  \textit{Nde} introduces the concluding comments concerning the way that the present-day situation for the Molokos has changed from the way it was before the actions of the disobedient girl. Example \REF{ex:12:59} is from the Leopard story \citep{Friesen2003} and \textit{nde} marks the clause within the hen’s speech where she makes her concluding decision of what she should do. \textit{Nde} marks a concluding statement in an instruction in \REF{ex:12:60}. 

\ea \label{ex:12:58}
\corpussource{Disobedient Girl, S. 32--34}\\
{}[\textbf{Nde} ko  ala  a  ɗəma  ndana    ava  pew]!  Məloko  ahay  tawəy,\\  
\gll  [\textbf{ndɛ} kʷɔ=ala  a ɗəma ndana ava pɛw] Mʊlɔkʷɔ=ahaj tawij \\
      so until=to at  time {\DEM} in enough  Moloko=Pl \oldstylenums{3}\textsc{p}+said\\ 
\glt ‘So, ever since that time, it’s done!  The Molokos say,’\\      
      
\medskip  
“Hərmbəlom  ága  ɓərav  va  kəwaya  war  dalay  na,\\    
\gll Hʊrmbʊlɔm á-ga ɓərav=va kuwaja war dalaj na\\
     God \oldstylenums{3}\textsc{s}+{\IFV}-do heart={\PRF} {because of}  child girl {\PSP}\\ 
\glt ‘“God got angry because of that girl,’\\

\medskip
\largerpage
amecen  sləmay  bay  ngəndəye.\\
\gll amɛ-tʃɛŋ ɬəmaj baj ŋgɪndijɛ \\ 
     {\DEP}-hear ear {\NEG} {\DEM}\\ 
\glt ‘that one that was disobedient.’\\

\medskip
Waya  ndana  Hərmbəlom  ázata  aka  barka  ahan  va."\\
\gll waja ndana Hʊrmbʊlɔm á-z=ata  =aka \\ 
     because {\DEM} God \oldstylenums{3}\textsc{s}+{\IFV}-take=\oldstylenums{3}\textsc{p}.{\IO}  =on \\  

\medskip
\gll barka=ahaŋ=va\\
     blessing=\oldstylenums{3}\textsc{s}.{\POSS}={\PRF}\\
\glt  ‘Because of that, God had taken back his blessing from them.”’
\z 

\ea \label{ex:12:59}
Tanday  taslaw  aka  babəza  ahay  va.\\   
\gll  ta-ndaj ta-ɬ=aw =aka babəza=ahaj=va \\  
      \oldstylenums{3}\textsc{p}-{\PROG}  \oldstylenums{3}\textsc{p}-kill={\oneS}.{\IO}  =on  children=Pl={\PRF}\\   
\glt ‘They were killing more of my children.’\\

\medskip
{}[\textbf{Nde} taslaw  aka  babəza  ahay  va  na,\\ 
\gll {}[\textbf{ndɛ}  ta-ɬ=aw =aka babəza=ahaj=va na\\
     so    \oldstylenums{3}\textsc{p}-kill={\oneS}.{\IO}  =on  children=Pl={\PRF}  {\PSP}\\
\glt ‘So [since] they killed more of my children,’\\

\medskip
nəhəmay  mogo  ele  əwla].\\
\gll nə-həm-aj mɔgʷɔ ɛlɛ=uwla]\\
     {\oneS}-run-{\CL}  anger thing={\oneS}.{\POSS}\\
\glt  ‘I ran away because of my anger (lit. I ran my anger thing).’ 
\z 

\clearpage
\ea \label{ex:12:60}	
Nahok  na  va,  kége  akar  bay.\\   
\gll  nà-h=ɔkʷ na=va k\'{ɛ}-g-ɛ akar baj  \\    
      {\oneS}+{\PFV}-tell={\twoS}.{\IO}    \oldstylenums{3}\textsc{s}.{\DO}={\PRF}  {\twoS}+{\IFV}-do-{\CL}  theft  {\NEG}\\     
\glt ‘I already told you, don’t steal,’\\

\medskip
{}[Asa  bay  na],  náaɓok.\\
\gll {}[asa baj na]  náá-ɓ=ɔkʷ \\ 
     again {\NEG}  {\PSP}  {}{\oneS}+{\POT}{}-beat={\twoS}.{\IO}\\ 
\glt ‘if not, I will beat you.’\\

\medskip
{}[\textbf{Nde} azləna  kagəva  akar]  náaɓok  azla.\\
\gll {}[\textbf{ndɛ} aɮəna kà-gə=va akar] náá-ɓ=ɔkʷ  aɮa\\
     so  but {\twoS}+{\PFV}-do={\PRF}  theft  {\oneS}+{\POT}-beat={\twoS}.{\IO}  now\\
\glt  ‘But you have gone and stolen, so I will beat you now.’  
\z 

\largerpage
\textit{Macəkəmbay} ‘meanwhile’ indicates that the information in the clause marked in this way occurred off the main event line. Example \REF{ex:12:61} is from the Race story \citep{Friesen2003}. The clause with \textit{macəkəmbay} marks what the toad had done before the race -- he had secretly invited his brothers to line the race route so that there would always be a toad ahead of the giraffe. The giraffe ran faster than the toad, but when he stopped running and called out to see how far behind him the toad was, one of the toad’s friends ahead of him would call to him, making him run so hard that he collapsed, thereby losing the race. 

\ea \label{ex:12:61}
Paraw  tədəya  məhəme,  ɓərketem, ɓərketem, ɓərketem.\\ 
\gll  paraw tə-d=ija mɪ-hɪm-ɛ ɓɪrkɛtɛm ɓɪrkɛtɛm ɓɪrkɛtɛm  \\ 
      {\textsc{id}:sudden start} {\oldstylenums{3}\textsc{p}-prepare={\PLU}} {\NOM}{}-run{}-{\CL} \textsc{id}:run \textsc{id}:run \textsc{id}:run\\
\glt ‘\textit{Paraw}, they started the race, running \textit{birketem, birketem, birketem}.'\\
\medskip
Kərcece  enjé  təf  na,  awəy,  “Moktonok  nok  amta?”\\   
\gll kɪrtʃɛtʃɛ ɛ-nʒ-ɛ təf na awij~  mɔkʷtɔnɔkʷ nɔkʷ amta  \\  
     giraffe \oldstylenums{3}\textsc{s}-leave{}-{\CL} \textsc{id}:far {\PSP} said toad {\twoS} where \\
\glt 'The giraffe went far away [along the race route]. He said, “Toad, where are you?”'\\
\medskip
Moktonok awəy, “Ne  ko  ehe.”  Awəy,  “Wa  alma?!"\\
\gll mɔkʷtɔnɔkʷ  awij  nɛ  kʷɔ ɛhɛ awij wa  alma\\
       toad said {\oneS} {no matter} {here} said  {what}  {what}\\
\glt    'A toad said, “I am way over here." [The giraffe] said, “What on earth?!” (lit. he said, “What what")'\\
\medskip
\clearpage
{}[\textbf{Macəkəmbay}  moktonok  na,  abək  ta  aya  va\\  
\gll  {}[\textbf{matsəkəmbaj} mɔkʷtɔnɔkʷ na a-bək ta=aja=va\\ 
      {meanwhile}    {toad}    {\PSP}      {\oldstylenums{3}\textsc{s}-invite}     {\oldstylenums{3}\textsc{p}.{\DO}}={\PLU}={\PRF}\\  
\glt  ‘Meanwhile, the toad, he had already invited’\\     

\medskip
məlama  ahan  ahay  jəyga].\\
\gll məlama=ahaŋ=ahaj dzijga]\\
     {brother}=\oldstylenums{3}\textsc{s}.{\POSS}{=Pl}  {all}\\
\glt  ‘all his brothers.’\\
\z 

\section{Juxtaposed clauses}\label{sec:12.4}
\hypertarget{RefHeading1213521525720847}{}
Many clauses in a Moloko discourse are independent and are not linked grammatically to a preceding or following clause by a connector or by the presupposition marker \textit{na}. The semantic nature of the connection between these unmarked, \nohyphens{juxtaposed} clauses is inferred from the context.\footnote{The presupposition marker \textit{na} aids in making a connection between two clauses, because it makes it explicit that the first (\textit{na}{}-marked) clause is known information. \textit{Na} constructions have already been discussed in \sectref{sec:11.1}.}  A juxtaposed clause can simply re-express the thought in the first clause. In \REF{ex:12:62}, the second clause restates in the negative that God is near. In \REF{ex:12:63}, the second clause makes more precise the general instruction in the first clause. In \REF{ex:12:64}, the second clause expands on what the speaker sees about the chief. In the examples in this section, each clause is delimited by square brackets and the juxtaposed clause is bolded.

\ea \label{ex:12:62}
{}[Ndahan  bəfa],  [\textbf{anday  dəren  bay}]. \\
\gll  [ndahaŋ bəfa]  [\textbf{a-ndaj} \textbf{dɪrɛŋ} \textbf{baj}]\\
      he  \textsc{id}:close   {\oldstylenums{3}\textsc{s}-{\PROG}}  far  {\NEG}\\
\glt  ‘So, he was close, he was not far.’
\z 

\ea \label{ex:12:63}
{}[Makay  war];  [\textbf{mapaɗay  sese  ahan}].\\
\gll  {}[mak-aj war] [\textbf{mà-paɗ-aj}     \textbf{ʃɛʃɛ=ahaŋ}]\\
      leave[{\twoS}.{\IMP}]-{\CL}  {child}  \oldstylenums{3}\textsc{s}+{\HOR}-crunch-{\CL}  {meat}=\oldstylenums{3}\textsc{s}.{\POSS}\\
\glt  ‘Leave the child alone; let him eat his meat.’ 
\z 

\clearpage
\ea \label{ex:12:64}
{}[Nəmənjar  bahay];  [\textbf{ndahan  aka  ozom  sese}].\\
\gll  {}[nə-mənzar bahaj]  [\textbf{ndahaŋ} \textbf{aka} \textbf{á-zɔm} \textbf{ʃɛʃɛ}]\\
      {\oneS}-see  chief  \oldstylenums{3}\textsc{s}  on  \oldstylenums{3}\textsc{s}+{\IFV}-eat  meat\\
\glt  ‘I see the chief; he is eating meat.’ 
\z 

Example \REF{ex:12:65} is from S. 8--10 in the peak episode of the Snake story\is{Focus and prominence!Discourse peak}. There is a series of three juxtaposed independent clauses. The second is a restatement of the first. The third follows chronologically. 

\ea \label{ex:12:65}\corpussource{Snake, S. 8}\\
{}[Mbaɗala  ehe  na,  nabay  oko], \\
\gll  {}[mbaɗala ɛhɛ na nà-b-aj ɔkʷɔ]\\
      then     here {\PSP}  {\oneS}+{\PFV}-light{}-{\CL}     fire\\
\glt  ‘Then, I turned on a light,’\\

 \medskip\corpussource{Snake, S. 9}\\
{}[\textbf{nazala  təystəlam  əwla}], \\
\gll  {}[\textbf{nà-zaɗ=ala}  \textbf{tijstəlam=uwla}]\\
      {\oneS}+{\PFV}-take=to     torch={\oneS}.{\POSS}\\
\glt  ‘I took my flashlight,’\\

 \medskip\corpussource{Snake, S. 10}\\
{} [\textbf{nabay  cəzlarr}].\\
\gll  {}[\textbf{nà-b-aj}  \textbf{tsəɮarr}]\\
      {\oneS}+{\PFV}-light{}-{\CL}       \textsc{id}:shining the flashlight up\\
\glt  ‘I shone it up \textit{cəzlarr}.’
\z 

Two juxtaposed clauses can express a logical or chronological sequence. A temporal (or logical) sequence from the Cicada fable is seen in \REF{ex:12:66}. The two clauses are the chief’s command to bring the tree to his door. First (clause 1), the people are to bring the tree and next (clause 2), they are to place it by his door. 

\ea \label{ex:12:66}
\corpussource{Cicada, S. 9}\\
{}[Káazəɗom  anaw  ala  agwazla  ndana  ka  mahay  əwla  aka].\\  
\gll {}[káá-zʊɗ{}-ɔm an=aw  =ala  agʷaɮa ndana ka mahaj=uwla aka] \\     
      {\twoP}+{\POT}-take-{\twoP}   {\DAT}={\oneS}.{\IO}  =to   {spp. of tree}    {\DEM}      on  door={\oneS}.{\POSS}   on \\     
\glt ‘You will bring that previously mentioned tree to my door for me.’\\

\medskip
\clearpage
{}[\textbf{Káafəɗom  anaw  ka  mahay  əwla  aka}].\\
\gll {}[\textbf{káá-fʊɗ-ɔm} \textbf{an=aw}   \textbf{ka}  \textbf{mahaj}\textbf{=uwla}   \textbf{aka}]\\
     {\twoP}+{\POT}-put-{\twoP}   {\DAT}={\oneS}.{\IO}  on  door={\oneS}.{\POSS}  on\\ 
\glt  ‘You will put it down by my door.’
\z 

Example \REF{ex:12:67} is a longer temporal sequence from the peak of the Snake story (S. 13--18). S. 13 links to the preceding discourse with a \textit{na}{}-marked clause, but the rest of the clauses (S. 14--18) are juxtaposed. There are no conjunctions or discourse particles to indicate how the clauses are linked. These juxtaposed clauses are a fast-moving temporal sequence (with a narrator's interjection in S. 16): he takes his spear (S. 13), hears the penetration (S. 14--15), the snake falls (S. 17), he clubs it to death (S. 18). 

\ea \label{ex:12:67}\corpussource{Snake, S. 13}\\
{}[Ne  mbət  məmbete  oko  əwla  na]. \\ 
\gll  {}[nɛ mbət mɪ-mbɛt-ɛ ɔkʷɔ=uwla na] \\     
      {\oneS}  {turn off}  {\NOM}{}-{turn off}-{\CL}     {light}={\oneS}.{\POSS}  {\PSP}\\ 
 \glt  ‘I turned off my light.'
      
      \medskip
{}[Kaləw  nazala  ezlere  əwla]. \\      
\gll  {}[kàluw nà-zaɗ=ala ɛɮɛrɛ=uwla]\\     
     {\textsc{id}:take quickly}    {\oneS}+{\PFV}-take=to   spear={\oneS}.{\POSS} \\
\glt ‘Quickly I took my spear.’\\

\medskip
\corpussource{Snake, S. 14-15}\\
{}\textbf{[Mək  ava  alay]}.  \textbf{[Mecesle  mbəraɓ]}.\\ 
\gll  \textbf{[mək}\textbf{=ava}\textbf{=alaj]} \textbf{[mɛ-tʃɛɬ-ɛ}  \textbf{mbəraɓ]} \\ 
      {\textsc{id}:penetrate}{=in}{=to}      {\NOM}{}-penetrate-{\CL}    {\textsc{id}:penetrate} \\     
\glt  ‘Penetration \textit{mək}! It penetrated, \textit{mbəraɓ}!’\\ 

\medskip
\corpussource{Snake, S. 16}\\
{}\textbf{[Ele  a  Hərmbəlom,  ele  ga  ajənaw  ete}\\ 
\gll \textbf{[ɛlɛ} \textbf{a}   \textbf{Hʊrmbʊlɔm}   \textbf{ɛlɛ}  \textbf{ga}  \textbf{à-dzən=aw} \textbf{ɛtɛ}\\        
     {thing}   {\GEN}   {God}      {thing}  {\ADJ}   \oldstylenums{3}\textsc{s}+{\PFV}-help={\oneS}.{\IO}    {also} \\       
\glt ‘God helped me also’\\ 

\medskip
\textbf{kəl  kəl  kə  ndahan  aka]}. \\
\gll \textbf{kəl kəl} \textbf{kə}   \textbf{ndahaŋ}   \textbf{aka]} \\ 
     {exactly}    {on}    \oldstylenums{3}\textsc{s}          {on} \\
\glt ‘[that the spear] went exactly on him’  \\

\medskip
\clearpage
\corpussource{Snake, S. 17}\\
{}\textbf{[Ádəɗala  vbaɓ  a  wəyen  ava]}.\\
\gll   \textbf{[á-dəɗ}\textbf{=ala}   \textbf{\dentalflap aɓ}  \textbf{a}  \textbf{wijɛŋ}  \textbf{ava]}\\
    \oldstylenums{3}\textsc{s}+{\IFV}-fall{=to}  {\textsc{id}:falling on ground}  {at}  {ground}  {on}  \\
\glt ‘and he fell on the ground \textit{vbaɓ}.’\\

\medskip
\corpussource{Snake, S. 18}\\
{}\textbf{[Ne  dəyday  məkəɗe  na  aka]}.\\
\gll  \textbf{[nɛ}   \textbf{dijdaj}   \textbf{mɪ-kɪɗ-ɛ}  \textbf{na}\textbf{=aka]}\\
    {\oneS}  {\textsc{id}:approximately}     {\NOM}{}-kill-{\CL}   \oldstylenums{3}\textsc{s}.{\DO}{=on}\\
\glt ‘I clubbed it to death (approximately).’\\
\z  

Two clauses linked by juxtaposition can also express a comparison (\ref{ex:12:68}--\ref{ex:12:69}). The first clause is a predicate-adjective clause (see \sectref{sec:10.1.2}) including the attribute being compared. The second clause establishes the comparison by means of the verb \textit{dal} ‘pass.’

\ea \label{ex:12:68}
{}[Kəra  malan  ga],  [\textbf{adal  pataw}].\\
\gll  {}[kəra  malaŋ  ga]   [\textbf{a-dal}    \textbf{pataw}]\\
      dog    largeness  {\ADJ}      \oldstylenums{3}\textsc{s}-pass  cat\\
\glt  ‘The dog is bigger than the cat.’ (lit. the dog [is] big, it is greater than the cat)
\z 

\ea \label{ex:12:69}
{}[Ne  mədehwer  ga],  [\textbf{nadal  nok}].\\
\gll  {}[nɛ  mødœhʷœr  ga]    [\textbf{na-dal}  \textbf{nɔkʷ}]\\
      {\oneS}  {old person}  {\ADJ}    {\oneS}-pass  {\twoS}\\
\glt  ‘I am older than you.’ (lit. I old person, I surpass you)
\z 