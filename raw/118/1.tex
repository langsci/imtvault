\chapter[Introduction]{Introduction}\label{chap:1}\label{introduction}
\hypertarget{RefHeading1210241525720847}{}\hypertarget{Toc450584435}{}

Although this grammar book is currently more than 100,000 words long, it truly only scratches the surface of this beautiful language. Moloko grammar is interesting and complex; we encourage further study  to demonstrate its genius in more detail. 

The notable features of the language include the following:

\begin{itemize}
\item the simplicity of the vowel system (there is only one underlying phoneme with ten phonetic representations and 4 graphemes, see \sectref{sec:2.3}), 
\item the complexity of the verb word (\chapref{chap:7}), with information in the verb word indicating in addition to the verbal idea, subject, direct object (semantic Theme), indirect object (recipient or beneficiary), direction, location, aspect (Imperfective and Perfective), mood (realis, irrealis, iterative), and Perfect aspect, 
\item  the fact that verbs are not inherently transitive or intransitive, but rather the semantics is tied to the number and type of core grammatical relations in a clause (\chapref{chap:9}), 
\item clauses with zero transitivity; i.e., no grammatical arguments in a clause (see \sectref{sec:3.6.3} and \sectref{sec:9.4}),  
\item the presupposition construction (\chapref{chap:11}), which is the main organisational structure in Moloko discourse, 
\item interrogative formation (see \sectref{sec:10.3}), including re-arrangement of the clause so that the interrogative particle occurs clause-final,
\item the absence of adjectives as a basic word class (all adjectives are derived from nouns, \sectref{sec:5.3}),
\item ideophones (\sectref{sec:3.6}), which are lexical items that give a ``picture'' or a ``sound'' idea of the event they symbolise. Found in many African languages, they function in Moloko as adverbs, adjectives, and in particular contexts, as verbs, 
\item the consonantal skeleton of words (see \sectref{sec:6.2}), 
\item reduplication that occurs in verbs (see \sectref{sec:7.4.4}) and nouns (see \sectref{sec:3.5.2}) and can be inflectional or derivational.  
\item the fact that Moloko is a somewhat agglutinative language, since easily separable morphemes can be added to noun and verb stems,
\item cliticisation, which is productive within the language. Clitics are both inflectional and derivational, and in nouns and verbs, always follow the lexical root they modify. Cliticisation in verbs allows several layers of clitics to be added. Verbal clitics are called \textit{extensions} in this paper, following Chadic linguistic terminology.\footnote{\citet{Newman1973} noted that the term ``verbal extension'' was widely used in Chadic languages to describe “optional additions that serve to expand or modify the meaning of the basic verb (173:334). Note that the term ``extension'' for Chadic languages has a different use than for Bantu languages. Verbal ``extensions '' in languages from the Niger-Congo, Nilo-Saharan, and Khoisan families have derivational or inflectional functions \citep{Hyman2007}. } In Chadic languages, ``extension'' refers to particles or clitics in the verb word or verb phrase.
\end{itemize}

Linguistic classification, language use, and previous research are outlined in Sections \ref{sec:1.1} to \ref{sec:1.3}. The four texts that follow in Sections \ref{sec:1.4}--\ref{sec:1.7} are chosen from among many that were recorded while the first author lived in the Moloko region from 1999 to 2008. They are used with permission. These stories belong to the community because they represent their collective knowledge and culture. As such, no individual will be named as 'author' of any particular story. Many of the examples from the grammar sections in this book are taken from these stories. The sentence numbers are given in the examples so that the reader can refer to the complete texts and locate the example in its context. The first line in each sentence is the orthographic form. The second is the phonetic form (slow speech) with morpheme breaks. The third line is the gloss and the fourth is the translation.

\section{Linguistic classification}\label{sec:1.1}
\hypertarget{RefHeading1210261525720847}{}
Moloko (or Melokwo, Molkore,\footnote{Molkore is the Fulfulde\il{Fulfulde} name for Moloko.} Məloko\footnote{Məloko is the spelling for this name using the Moloko orthography. The orthography, described in \citet{Friesen2001}, is being used by the Moloko (more than a dozen titles are listed in the reference section).}) is classified by \citet{LewisSimonsFennig2009}\footnote{\citet{DieuRenaud1983} classify it as [154] Chadic family, Biu Mandara branch, center-west sub-branch, Wandala-Mafa group, Mafa-south sub-group (A5).} as Central Chadic Biu-Mandara A5, as seen in \figref{fig:1.1}. A more detailed discussion of the classification of Moloko is found in \citet{Bow1997a}. 

\begin{figure}
\begin{tabular}{llllll}
\textbf{Afro-asiatic}  &  Berber\\
&    Cushitic\\
&    Egyptian\\
&    Omotic\\
&    Semitic\\
&    \textbf{Chadic} &  East\\
 & &     Masa\\
& &      West\\
& &\textbf{Biu-Mandara}  &  \textbf{A}  &  A1\\
& & & &           A2\\
& & & &           A3\\
& & & &            A4\\
& & & &          \textbf{A5}  &  Baldemu\\
& & & & &             Cuvok\\
& & & & &              Dugwor\\
& & & & &             Giziga, North\\
& & & & &              Giziga, South\\
& & & & &            Zulgo-Gemzek\\
& & & & &             Mafa\\
& & & & &            Merey\\
& & & & &             Matal\\
& & & & &            Mefele\\
& & & & &             Mofu, North\\
& & & & &            Mofu-Gudur\\
& & & & &            Vame\\
& & & & &            \textbf{Moloko}\\
& & & & &              Mbuko\\
& & & & &     Muyang\\
& & & & &              Mada\\
& & & & &              Wuzlam\\
& & & &             A6\\
& & & &             A7\\
& & & &            A8\\
& & &           B\\
& & &           C
\end{tabular}

\caption{Classification of Moloko}
\label{fig:1.1}
\end{figure}

The Ethnologue \citep{LewisSimonsFennig2009} reports 8,500 speakers of Moloko in 1992. A survey by \citet{Starr1997} estimated 10,000--12,000 speakers. Most live near Moloko mountain, 30km north of Maroua in the district of Tokombere, department of Mayo-Sava in the Far North Province of the Republic of Cameroon. Local oral history indicates that the Moloko people actually are not a single people group historically, but that people from at least three ethnic groups sought refuge on Moloko mountain during the Fulani invasions of the 19\textsuperscript{th} century. Eventually they all came to speak the same language. 

Moloko mountain remains the center of Moloko culture. There are three villages on the summit itself. Moloko villages are organised by clan, each village being the male descendants of a particular clan and their families. Since the 1960’s, some of the Moloko language group have moved to the plains between the mountain and Maroua, and have settled in Moloko or Giziga-Moloko villages. Others have moved further away and live in small communities in and around the cities of Maroua, Garoua, Toubouro, Kousseri, and Yaounde. Minor dialectal differences exist in pronunciation and vocabulary but all speakers can understand one another without difficulty.   

\section{Language use, language contact, and multilingualism}\label{sec:1.2}
\hypertarget{RefHeading1210281525720847}{}
A minority of Moloko speakers are monolingual.  Most  speak three to five other languages.  Men and most women have at least a market level knowledge of Fulfulde\il{Fulfulde}, the language of wider communication, and also speak at least one of the neighboring languages: Giziga\il{Giziga}, Muyang\il{Muyang}, Gemzek\il{Gemzek}, Mbuko\il{Mbuko}, or Dugwor\il{Dugwor}.  Those with several years of education also speak French.  

Men often marry women from neighboring language groups, so homes can be multilingual, but the spoken language at home tends to be the language of the father.  Friends will often switch languages as they are conversing, perhaps when talking in different domains, but also simply to bond.  Dealings in the market can be done in the trade language, but people prefer to bargain in the language of the seller, if possible.  

Language viability for Moloko is only at risk in communities where Moloko is not the primary language, especially in cities like Maroua or Yaounde.  In the city, children grow up in neighborhoods where many different languages are spoken and so they tend to speak Fulfulde\il{Fulfulde} (as well as learn French at school).  In such places, Moloko is at risk to be lost in the next generation.  Otherwise, in areas where Moloko people are together, Moloko language use is strong among people of every age and in every domain of home life.

\section{Previous research}\label{sec:1.3}
\hypertarget{RefHeading1210301525720847}{}
\citet{Bradley1992} is a dialect survey of the Moloko region from Moloko mountain to Maroua. \citet{Bow1997c} is a phonological description which included some discussion on tone. \citet{Bow1999} is an M.A. thesis which further studied the vowel system.  These two documents, along with discoveries since their work form the basis of the phonology chapter and phonology sections in the verb and noun chapters. Other works consulted include the following:  \citet{Bow1997a} presents the classification of Moloko. \citet{Bow1997b} is a manuscript on labialisation and palatalisation in Moloko. \citet{StarrBoydBow2000} is a 1500 word lexicon, and \citet{Friesen2001} an orthography.  \citet{Boyd2002} analyses lexical tone in nouns.  \citet{Boyd2001}, \citet{OumarBoyd2002}, \citet{HolmakaBoyd2002}, \citet{Holmaka2002}, and \citet{Friesen2003} present interlinearised texts. \citet{Friesen2003} also presents two Moloko fables with a cultural commentary concerning each.  The Moloko translation committee has produced (among other work) two primers (\citealt{MTC2004a}, \citeyear{MTC2008}), transfer primers from French \citep{MTC2005a} and Fulfulde \citep{MTC2007a}, as well as several booklets with fables (\citealt{MTC2004b}, \citeyear{MTC2005b}, \citeyear{MTC2007a}--\citeyear{MTC2007d}).\enlargethispage{1\baselineskip}

\citet{FriesenMamalis2008} describe the Moloko verb phrase, an analysis which is reflected in this work. Prior to Friesen and Mamalis, only a few documents touched on the syntax of Moloko. The phonology statement in \citeyear{Bow1997c} explored the grammar of verbs in relation to tone, and a few comparative studies of several Chadic languages included Moloko data (\citealt{Rossing1978}, \citealt{Blama1980}, and \citealt{deColombel1982}). Rossing described Moloko noun prefixes and suffixes, plural and adjective markers, and pronouns.  He also mentioned a nominalising prefix on the verb stem that formed the nominalised form. \citet{Boyd2003} is a draft of a grammar sketch; her findings are cited where they add to this present work. 

\clearpage
\section[Snake story]{Snake story}\label{sec:1.4}
\hypertarget{RefHeading1210321525720847}{}
This true story was recorded in Lalaway, Far North Province of Cameroon, in 2007.  

\subsubsection*{Setting}
\ea
Ele  ndana  ege  na,  ne  a Kosewa.\\
\gll ɛlɛ     ndana  ɛ-g-ɛ    na    nɛ   a Kʷɔʃɛwa\\
thing {\DEM}    {\SSS-do{}-\CL}  {\PSP}     {\oneS}  at  Kossewa,\\
\glt ‘[When] this thing happened, I was [living] at Kossewa.’\\
\z

\noindent\parbox{\textwidth}{\ea
Ne məndəye  ga  elé  əwla.\\
\gll nɛ  mɪ-nd-ijɛ    ga  ɛlɛ=uwla\\
{\oneS}  {\textsc{nom}-lie down-{\CL}}   {\ADJ}   eye=\oneS.{\POSS}  \\
\glt ‘I was lying down.’
\z}

\ea 
Ne ɗəwer ga.\\
\gll nɛ   ɗuwɛr  ga\\
{\oneS}   sleep    {\ADJ} \\
\glt ‘I was sleeping.’
\z

\subsubsection*{Episode 1}
\ea 
 Alala  na, gogolvan  na, olo  alay.\\
\gll a-l=ala na gʷɔgʷɔlvaŋ   na \`ɔ{}-lɔ=alaj\\
\oldstylenums{3}\textsc{s}-go=to       {\PSP}  snake      {\PSP}   \oldstylenums{3}\textsc{s}+{\PFV}-go=away\\
\glt ‘Some time later, the snake went.’
\z

\subsubsection*{Inciting moment}
\ea
 Acar  a  hay  kəre  ava  {fo fo fo}.\\
\gll à-tsar  a  haj  kɪrɛ  ava  {fɔ fɔ fɔ}\\
\oldstylenums{3}\textsc{s}-climb at house   wood      in {\textsc{id}:sound of snake}\\
\glt ‘It climbed into the beams in the roof of the house \textit{fo fo fo}.’
\z

\newpage 
\ea 
 Sen  ala  na,  okfom  adaɗala   ɓav!\\
\gll ʃɛŋ=ala     na ɔkʷfɔm   à-dəɗ=ala   ɓav\\
\textsc{id}:go=to      {\PSP}   mouse  \oldstylenums{3}\textsc{s}+{\PFV}-fall=to     {\textsc{id}:sound of falling} \\
\glt ‘And walking, a mouse fell \textit{ɓav}!’
\z

\largerpage
\ea
 Ne awəy,  “Alma  {amədəvala}  okfom  nehe  may?”\\
\gll nɛ awij    alma   amə-dəv=ala ɔkʷfɔm   nɛhɛ    maj\\
{\oneS} said     what   {\DEP}-fall=to     mouse    {\DEM}  what \\
\glt ‘[I woke up]  I said [to myself], “What made that mouse fall?”’
\z

\subsubsection*{Peak episode}
\ea 
 Mbaɗala  ehe  na, nabay  oko,\\
\gll mbaɗala   ɛhɛ   na nà-b-aj  ɔkʷɔ\\
 then       here   {\PSP}  {\oneS}+{\PFV}-light{}-{\CL}      fire\\
\glt ‘Then, I turned on a light,’
\z

\ea 
 nazaɗala  təystəlam  əwla.\\
\gll nà-zaɗ=ala tijstəlam=uwla\\
{\oneS}+{\PFV}-take=to     torch={\oneS}.{\POSS}\\
\glt ‘I took my flashlight.’
\z 

\ea 
Nabay  cəzlar.\\
\gll nà-b-aj tsəɮar\\
{\oneS}+{\PFV}-light{}-{\CL}       {\textsc{id}:shining the flashlight up}\\
\glt ‘I shone it up \textit{cəzlar}.’
\z

\ea   Nábay  na,\\
\gll ná-b-aj     na\\ 
{\oneS}-light{}-{\CL}  {\PSP}\\
\glt ‘[As] I shone [it],’\\

\medskip
 námənjar  na,  {mbajak  mbajak  mbajak}  gogolvan!\\
\gll ná-mənzar     na     {mbadzak   mbadzak    mbadzak}  gʷɔgʷɔlvaŋ\\
{\oneS}+{\IFV}-see  {\PSP}    {\textsc{id}:something big and reflective}   snake\\
\glt ‘I was seeing it, something big and reflective, a snake!’
\z

\ea  Ne  awəy,    “A,  enen  baj  na,  memey  na!”\\
\gll nɛ awij    a          ɛnɛŋ       baj     na  mɛmɛj   na\\
{\oneS} said    interj.   snake  {\NEG}   {\PSP}  how       {\PSP}\\
\glt ‘I said to myself, “Wah! It’s a snake!”’ (lit. a snake, if not, how)
\z

\ea
 Ne  mbət  məmbete  oko  əwla  na,\\
\gll nɛ   mbət           mɪ-mbɛt-ɛ           ɔkʷɔ=uwla        na\\
{\oneS}    {\textsc{id}:turn off}  {\NOM}{}-{turn off}-{\CL}  light={\oneS}.{\POSS}  {\PSP} \\
\glt ‘I turned off my light,’\\

\medskip
 kaləw  nazaɗala  ɛɮɛrɛ=uwla.\\
\gll kàluw               nà-zaɗ=ala  ɛɮɛrɛ=uwla\\
{\textsc{id}:take quickly}  {\oneS}+{\PFV}-take=to spear={\oneS}.{\POSS} \\
\glt ‘[and] quickly took my spear.’
\z

\ea
 Mək  ava  alay,\\
\gll mək=ava=alaj\\
{\textsc{id}:positioning self for throwing}=in=to\\
\glt ‘[I] positioned [myself] \textit{mək}!’
\z

\ea
 Mecesle  mbəraɓ! \\
\gll mɛ-tʃɛɬ-ɛ            mbəraɓ\\
{\NOM}{}-penetrate-{\CL}      \textsc{id}:penetrate\\
\glt ‘It penetrated, \textit{mbəraɓ}!’
\z

\ea  
 {Ele  a}  Hərmbəlom  ele  ga  ajənaw  ete\\
\gll ɛlɛ     a        Hʊrmbʊlɔm  ɛlɛ     ga      à-dzən=aw    ɛtɛ\\
thing   {\GEN}   God    thing  {\ADJ}   \oldstylenums{3}\textsc{s}+{\PFV}-help={\oneS}.{\IO}    also \\
\glt ‘God helped me also’\\

\medskip
 kəl kəl   kə ndahan aka\\
\gll {kəl kəl}   kə ndahaŋ aka\\
exactly  on  \oldstylenums{3}\textsc{s}        on\\
\glt ‘[that the spear went] exactly on him.’
\z

\ea 
 Ádəɗala vbaɓ  a  wəyen  ava.\\
\gll á-dəɗ=ala     \dentalflap aɓ         a     wijɛŋ      ava\\
\oldstylenums{3}\textsc{s}+{\IFV}-fall=to      \textsc{id}:falling    at    ground   on\\
\glt ‘He fell on the ground \textit{vbaɓ}.’
\z

\ea
 Ne {dəyday}  məkəɗe  na  aka\\
\gll nɛ {dijdaj}              mɪ-kɪɗ-ɛ      na=aka\\
{\oneS}  approximately     {\NOM}{}-kill-{\CL}  \oldstylenums{3}\textsc{s}.{\DO}=on\\
\glt ‘I clubbed it to death (approximately).’
\z

\subsubsection*{Dénouement}
\ea Hor  əwla   olo  alay  awəy  egege,\\
\gll hʷɔr=uwla      \`ɔ{}-lɔ=alaj            awij ɛgɛgɛ\\
woman={\oneS}.{\POSS} \oldstylenums{3}\textsc{s}+{\PFV}-go=to           said that\\
\glt ‘My wife went and said,’\\

\medskip
 “{A a}  nəngehe  na,  Hərmbəlom  aloko  ehe.\\
\gll {a a}     nɪŋgɛhɛ   na   Hʊrmbʊlɔm=alɔkʷɔ        ɛhɛ\\
exclamation   {\DEM}     {\PSP}  God=\oldstylenums{2}\textsc{s}.{\POSS}   here \\
\glt ‘“Wah! This one here, our God [is] really here [with us].’\\

\medskip
 Bəyna  anjakay  nok  ha  a  slam  məndəye  ango  ava,\\
\gll bijna      à-nzak-aj          nɔkʷ   ha      a    ɬam      mɪ-nd-ijɛ=aŋgʷɔ  ava\\
because   \oldstylenums{3}\textsc{s}+{\PFV}-find{}-{\CL}  {\twoS}      until  at      place  {\NOM}{}-sleep-{\CL}={\twoS}       in\\
\glt ‘Because it found you even in your bed.’ (lit. all the way to the place of your lying)\\

\medskip
 alala  Hərmbəlom  ajənok  na,  səwse   Hərmbəlom.”\\
\gll {a-l=ala}   Hʊrmbʊlɔm    à-dzən=ɔkʷ     {na   ʃuwʃɛ}   Hʊrmbʊlɔm\\
\oldstylenums{3}\textsc{s}-go=to  God         \oldstylenums{3}\textsc{s}+{\PFV}-help={\twoS}.{\IO}  {{\PSP}  thanks}   God\\
\glt ‘And then God helped you; thanks [be to] God!”’
\z

\ea Hor  əwla  ahaw  kəygehe.\\
\gll hʷɔr=uwla     à-h=aw kijgɛhɛ\\
woman={\oneS}.{\POSS}    \oldstylenums{3}\textsc{s}+{\PFV}-tell={\oneS}.{\IO}     {like that}\\
\glt ‘My wife said it like that.’
\z

\ea  Alala,  nəzlərav  na ala  gogolvan  na   a  amata  ava. \\
\gll a-l=ala    n\`ə-ɮərav         na=ala  gʷɔgʷɔlvaŋ  na  a  amata  ava\\
\oldstylenums{3}\textsc{s}-go=to  {\oneS}+{\PFV}-exit  \oldstylenums{3}\textsc{s}.{\DO}=to  snake        {\PSP}    at   outside  in\\
\glt ‘Sometime later I took the snake outside.’ 
\z

\ea Ko  dedew  babəza  əwla  ahay  aməzləravala  amata  na, tawəy,\\
\gll kʷɔ     dɛdɛw     babəza=uwla=ahaj  amə-ɮərav=ala  amata      na    tawij\\
early   morning  child={\oneS}.{\POSS}=Pl  {\NOM}{}-exit=to     outside   {\PSP}    \oldstylenums{3}\textsc{p}+said\\
\glt ‘Early the next morning, when my children came outside, they said,’\\

\largerpage[2]
\medskip
 “Baba  ákaɗ  gogolvan,  baba   ákaɗ  gogolvan!”\\
\gll baba     á-kaɗ gʷɔgʷɔlvaŋ  baba   á-kaɗ gʷɔgʷɔlvaŋ\\
father   \oldstylenums{3}\textsc{s}+{\IFV}-kill      snake    father   \oldstylenums{3}\textsc{s}+{\IFV}-kill      snake\\
\glt  ‘“Papa killed a snake, Papa killed a snake!”’
\z

\ea Tájaka  kəygehe.\\
\gll tá-dz=aka   kijgɛhɛ\\
\oldstylenums{3}\textsc{p}+{\IFV}-say=on  {like that}\\
\glt ‘They said it like that.’
\z

\subsubsection*{Conclusion}
\ea Ka  nehe  ləbara  a  ma  ndana  ɗəwge.\\
\gll ka  nɛhɛ  ləbara  a       ma      ndana  ɗuwgɛ\\
like  {\DEM}  news    {\GEN}  word  {\DEM}      actual\\
\glt ‘And so was that story.’
\z

\section[Disobedient Girl story]{Disobedient Girl story}\setcounter{equation}{0}
\hypertarget{RefHeading1210341525720847}{}\label{sec:1.5}
This fable was recorded in Lalaway, Far North Province of Cameroon, in 2002. 

\subsubsection*{Setting}
\ea Bamba  bamba  kəlo  dərgoɗ!\\
\gll bamba   bamba   kʊlɔ     dʊrgʷɔɗ\\
story        story     under    silo\\
\glt ‘Once upon a time…’ (lit. there’s a story under the silo)
\z

\ea    Tawəy  abəya,  ma  bamba  a  war  dalay  cezlere  ga.\\
\gll tawij  abija    ma   bamba   a       war     dalaj      tʃɛɮɛrɛ        ga\\
\oldstylenums{3}\textsc{p}+said  saying   word   story   {\GEN}   child   female  disobedience   {\ADJ}\\
\glt ‘They say, the story of the disobedient girl [goes like this:]’\\
\z

\ea    Zlezle  na,  Məloko  ahay  na,  Hərmbəlom  ávəlata  barka  va.\\
\gll ɮɛɮɛ        na  Mʊlɔkʷɔ=ahaj     na  Hʊrmbʊlɔm  á-vəl=ata     \\   
{long ago}         {\PSP}  Moloko=Pl          {\PSP}  God       \oldstylenums{3}\textsc{s}+{\IFV}-send=\oldstylenums{3}\textsc{s}.{\IO}  \\ 

\medskip
 \gll barka=va\\
 blessing={\PRF}\\
 \glt ‘Long ago, to the Moloko people, God had given his blessing.’\\
\z

\clearpage
\ea Kəwaya  asa  təwasva  nekwen  kəygehe  ɗəw,\\
\gll kuwaja   asa    t\`ə-was=va  nɛkʷɛŋ    kijgɛhɛ     ɗuw\\
{that is}   if     \oldstylenums{3}\textsc{p}+{\PFV}-cultivate={\PRF}  little       {like this}    also\\
\glt ‘That is, even if they had only cultivated a little [millet] like this,’\\

\medskip
 ávata  məvəye  haɗa.  \\
\gll á-v=ata  mɪ-v-ijɛ         haɗa\\
\oldstylenums{3}\textsc{s}+{\IFV}-{spend time}=\oldstylenums{3}\textsc{p}.{\IO}   {\NOM}-{spend time}-{\CL}  {a lot}\\
\glt ‘it would last them enough for the whole year.’  \\
\z
\ea    Aməhaya  kə  ver  aka  na,  tázaɗ  war  elé  háy  bəlen.\\
\gll amə-h=aja   kə   vɛr   aka  na  tá-zaɗ       war   ɛlɛ  haj  bɪlɛŋ\\
{\DEP}-grind={\PLU}  on   stone   on    {\PSP}  \oldstylenums{3}\textsc{s}+{\IFV}-grind  child   eye   millet    one\\
\glt ‘For grinding on the grinding stone, they would take one grain of millet.’\\
\z

\ea    Nde,  asa  t\'ənday  táhaya  na  na,    \\
\gll ndɛ  asa t\'ə-ndaj    tá-h=aja      na  na\\
so  if     \oldstylenums{3}\textsc{p}+{\IFV}-{\PRG}  \oldstylenums{3}\textsc{p}+{\IFV}-grind={\PLU}  \oldstylenums{3}\textsc{s}.{\DO}  {\PSP}\\
\glt ‘So, whenever they were grinding it,’\\
\medskip

 həmbo  na, ásak  nə  məsəke.\\
\gll hʊmbɔ na á-sak nə mɪ-ʃɪk-ɛ\\
flour     {\PSP}  \oldstylenums{3}\textsc{s}+{\IFV}-multiply  with  {\NOM}{}-multiply-{\CL}\\
\glt ‘the flour, it multiplied with multiplying.’ \\
\z

\ea    War  elé  háy  bəlen  fan  na\textit,\\
\gll war     ɛlɛ   haj   bɪlɛŋ   faŋ   na\\
child   eye   millet   one    yet   {\PSP}\\
\glt ‘Just one grain of millet,’\\

\medskip
 ájata   pɛw  ha  ámbaɗ  ɛʃɛ.\\
\gll á-nz=ata                    pɛw   ha  á-mbaɗ    ɛʃɛ\\
 \oldstylenums{3}\textsc{s}+{\IFV}-suffice=\oldstylenums{3}\textsc{p}.{\IO}       enough   until     \oldstylenums{3}\textsc{s}+{\IFV}-remain   again\\
\glt ‘it sufficed for them, even to leaving leftovers.’ (lit. until it remained again)\\
\z

\ea    Waya  a  məhaya  ahan  ava  na,\\
\gll waja   a  mə-h=aja=ahaŋ      ava    na\\
because  at   {\NOM}{}-grind={\PLU}=\oldstylenums{3}\textsc{s}.{\POSS}   in   {\PSP}\\
\glt ‘Because, during its grinding,’\\

\clearpage
\medskip
 ásak  kə  ver  aka  nə  məsəke.\\
\gll á{}-sak                    kə   vɛr               aka   nə   mɪ-ʃɪk-ɛ\\
 \oldstylenums{3}\textsc{s}+{\IFV}-multiply   on     {grinding stone}     on    with   {\NOM}{}-multiply-{\CL}\\
\glt ‘it would actually multiply on the grinding stone.' (lit. multiply with multiplying)
\z

\subsubsection*{Episode 1}
\ea    Nde  ehe  na,    albaya  ava  aba.        \\
\gll ndɛ    ɛhɛ  na   albaja       ava  aba \\
so       here    {\PSP}  {young man}  {\EXT}{}+in   {\EXT}\\
\glt ‘And so, there once was a young  man.’\\
\z

\ea    Olo  azala  dalay.\\
\gll à-lɔ            à-z=ala  dalaj\\ 
\oldstylenums{3}\textsc{s}+{\PFV}-go    \oldstylenums{3}\textsc{s}+{\PFV}-take=to  girl\\
 \glt ‘He went and took a wife.’
\z

\ea   Azləna, war  dalay  ndana,  cezlere  ga.\\
\gll aɮəna  war   dalaj   ndana tʃɛɮɛrɛ          ga\\
but  child      female    {\DEM}  disobedience {\ADJ}\\
\glt ‘Now, that young girl was disobedient.’\\
\z

\ea   Sen  ala  na,  zar  ahan  na,\\
\gll ʃɛŋ=ala   na  zar=ahaŋ    na\\
\textsc{id}:go=to          {\PSP}  man=\oldstylenums{3}\textsc{s}.{\POSS}    {\PSP}\\
\glt ‘Then her husband’\\

\medskip
 dək  medakan  na,  mənjəye  ata.\\
\gll dək    mɛ-dak=aŋ            na        mɪ-nʒ-ijɛ=atəta\\ 
\textsc{id}:show   {\NOM}-show=\oldstylenums{3}\textsc{s}.{\IO}   \oldstylenums{3}\textsc{s}.{\DO}   {\NOM}{}-sit-{\CL}=\oldstylenums{3}\textsc{p}.{\POSS}\\
\glt ‘instructed her in their habits.’ (lit. instructing their sitting)
\z

\ea  Awəy, “Hor  golo,  afa  ləme  na,\\
\gll awij    hʷɔr gʷɔlɔ afa      lɪmɛ    na\\
said   woman  \textsc{hon}  {at place of}    \oldstylenums{1}\textsc{Pex}     {\PSP}\\
\glt ‘He said, “My dear wife, here at our (exclusive) place,\\

\clearpage
\medskip
 mənjəye  aləme  na,  kəygehe.\\
\gll mɪ-nʒ-ijɛ=alɪmɛ     na  kijgɛhɛ\\
{\NOM}{}-sit-{\CL}=\oldstylenums{2}\textsc{Pex}.{\POSS}    {\PSP}  {like this}\\
\glt ‘it is like this.’\\

\medskip
 Asa  asok  aməhaya  na,\\
\gll asa   à-s=ɔkʷ amə-h=aja na\\
if     \oldstylenums{3}\textsc{s}+{\PFV}-please={\twoS}.{\IO}  {\DEP}-grind={\PLU}   {\PSP}\\
\glt ‘If you want to grind’ (lit. if grinding pleases you),\\

\medskip
 kázaɗ  war  elé  háy  bəlen.\\
\gll ká{}-zaɗ  war     ɛlɛ      haj       bɪlɛŋ\\
{\twoS}+{\IFV}-take  child  eye  millet  one\\
\glt ‘you take only one grain.’\\

\medskip
 War  elé  háy  bəlen  ga  nəndəye  nok  amezəɗe  na,\\
\gll war    ɛlɛ      haj     bɪlɛŋ  ga   nɪndijɛ  nɔkʷ amɛ-ʒɪɗ-ɛ na\\
child     eye  millet  one      {\ADJ}  {\DEM}    {\twoS}    {\DEP}-take-{\CL}      {\PSP}\\
\glt ‘That one grain that you have taken,’\\

\medskip
káhaya  na  kə  ver  aka.\\
\gll ká-h=aja   na       kə  vɛr        aka\\
{\twoS}+{\IFV}-grind={\PLU}  \oldstylenums{3}\textsc{s}.{\DO}  on   {grinding stone}  on\\
\glt ‘grind it on the grinding stone,’\\

\medskip
Ánjaloko  de  pew. \\
\gll á-nz=alɔkʷɔ dɛ     pɛw\\
\oldstylenums{3}\textsc{s}+{\IFV}-suffice=\oldstylenums{1}\textsc{Pin}.{\IO}    enough  finished\\
\glt ‘It will suffice for all of us just enough.’ \\

\medskip
 Ádaloko  ha  ámbaɗ  ese.\\
\gll á-d=alɔkʷɔ ha      á-mbəɗ ɛʃɛ\\
\oldstylenums{3}\textsc{s}+{\IFV}-prepare=\oldstylenums{1}\textsc{Pin}.{\IO}  until   {\oldstylenums{3}\textsc{s}+{\IFV}-left over}  again\\ 
\glt ‘It will make food for all of us, until there is some left over.’\\

\medskip
  Waya  a  məhaya  ahan  ava  na,\\
\gll waja   a    mə-h=aja=ahaŋ ava na\\
because  at   {\NOM}{}-grind={\PLU}=\oldstylenums{3}\textsc{s}.{\POSS}   in   {\PSP}\\
\glt ‘because, during the grinding,’\\
\clearpage
\medskip
 Hərmbəlom  anday  ásakaləme  na  aka.”\\
\gll Hʊrmbʊlɔm   a-ndaj    á-sak=alɪmɛ     na=aka\\
God                \oldstylenums{3}\textsc{s}-{\PRG}  \oldstylenums{3}\textsc{s}+{\IFV}-multiply=\oldstylenums{1}\textsc{Pex}.{\IO}  \oldstylenums{3}\textsc{s}.{\DO}=on \\
\glt ‘God is multiplying it for us.”’\\
\z

\ea   Hor  na,  ambəɗan  aka,   \\
\gll hʷɔr na a-mbəɗ=aŋ =aka\\
woman {\PSP}  \oldstylenums{3}\textsc{s}-change=\oldstylenums{3}\textsc{s}.{\IO}    =on\\
\glt ‘The woman replied,’\\
\z

\ea  awəy,  “Ayokon  zar  golo.”\\
\gll awij    ajɔkʷɔŋ   zar  gʷɔlɔ\\
said   agreed man  \textsc{hon}\\
\glt ‘saying, “Yes, my dear husband.”’
\z 

\subsubsection*{Episode 2}
\ea  Ndahan  amandava  ɓəl  na, zar  ahan  olo\\
\gll ndahaŋ  ama-nd=ava   ɓəl   na  zar=ahaŋ      \`ɔ{}-lɔ\\
\oldstylenums{3}\textsc{s}         {\DEP}-sleep=in    \textsc{id}:some  {\PSP}  man=\oldstylenums{3}\textsc{s}.{\POSS}  \oldstylenums{3}\textsc{s}+{\PFV}-go  \\
\glt ‘She, sleeping there for some time, her husband went away’\\

\medskip
 ametele  kə  dəlmete  ahan  aka  a  slam  enen.\\
\gll amɛ-tɛl-ɛ     kə  dɪlmɛtɛ=ahaŋ      aka   a  ɬam  ɛnɛŋ\\
{\DEP}-walk-{\CL}  on   neighbor=\oldstylenums{3}\textsc{s}.{\POSS}  on  at place another\\
\glt ‘to walk~in the neighborhood to some place.’ \\
\z
\ea  Azləna,  hor  na,  asərkala  afa  təta  va  na,\\
 \gll aɮəna  hʷɔr   na à-sərk=ala afa təta=va na\\
 but  woman  {\PSP}  \oldstylenums{3}\textsc{s}+{\PFV}-{habitually}=to     {at house of}  \oldstylenums{3}\textsc{p}={\PRF}  {\PSP}\\
 \glt ‘Now, that woman, she was in the habit at their place’\\
 
 \medskip
  aməhaya  háy  na  gam.\\
  \gll amə-h=aja      haj        na  gam\\
 {\DEP}-grind={\PLU}  millet  {\PSP}    {a lot}\\
 \glt ‘[of] grinding a lot of millet.’ 
 \z
 
 \ea  Ndahan  jo  madala  háy  na,           \\
\gll ndahaŋ    dzɔ      ma-d=ala      háj  na\\
 \oldstylenums{3}\textsc{s}         \textsc{id}:take   {\NOM}{}-prepare=to    millet   {\PSP}\\
 \glt ‘After having gotten ready to grind (she taking millet),’\\
 
 \medskip
  ɗen  bəlen  tə  kə  ver  aka,\\
\gll ɗɛŋ        bɪlɛŋ  tə               kə  vɛr      aka,\\
 \textsc{id}:put   one      {\textsc{id}:put one}  on      stone  on\\
 \glt ‘[she put] one grain on the grinding stone.’\\
 \z
 
 \ea  Awəy,  “Gəlo  ahay  nehe  azla  na,  malmay  nəngehe  na  may?  \\
 \gll awij gʊlɔ=ahaj nɛhɛ     aɮa     na  malmaj  nɪŋgɛhɛ  na  maj\\
 said  fellow=Pl   {\DEM}    now   {\PSP}    what      {\DEM}      {\PSP}    what\\
 \glt ‘She said, “Friends, here, what is this?’\\
 
 \medskip
  Háy  bəlen  azla  na, náambəzaka  məhaya  əwla  na,\\
 \gll haj    bɪlɛŋ   aɮa   na,  náá-mbəz=aka   mə-h=aja=uwla   na\\
 millet   one        now   {\PSP}  {\oneS}+{\POT}-ruin=on  \textsc{nom}{}-grind={\PLU}={\oneS}.{\POSS}  {\PSP}\\
 \glt ‘One grain, [with it] I know I will ruin my grinding.’\\
 
 \medskip
  Meme  ege  mey?\\
\gll  mɛmɛ   ɛ{}-g-ɛ      mɛj\\
 how       \oldstylenums{3}\textsc{s}-do-{\CL}  how\\
 \glt ‘What is happening?’ (lit. how it does)\\
 
 \medskip
 Nehe  na,  məseɓete  hərav  əwla  ɗaw?\\
 \gll nɛhɛ  na mɪ-ʃɛɓɛt-ɛ    hərav=uwla    ɗaw\\
 {\DEM}             {\PSP}  {\NOM}{}-deceive-{\CL}  body={\oneS}.{\POSS}  {\textsc{q}}\\
 \glt ‘This, am I deceiving my body?’ \\
 
 \medskip
 Bəy  na,  malmay? \\
 \gll bij  na malmaj\\
 {\NEG}        {\PSP}  what\\
 \glt ‘If not, what is it then?’\\
 
 \medskip
  Aya  jen  ele  ahay  nendəye  na, nagala  kəyga  bay.”    \\
 \gll aja   dʒɛŋ  ɛlɛ=ahaj  nɛndijɛ na nà-g=ala  kijga  baj \\
 so  chance  thing=Pl       {\DEM}     {\PSP}  {\oneS}+{\PFV}-do=to  {like this}  {\NEG}           \\
 \glt ‘Above all, these things, I have never done like this.”’\\
 \z
 
\subsubsection*{Peak episode}
 \ea  Jo  madala  háy  na,  gam.  \\
\gll dzɔ    ma-d=ala  haj na gam\\
 \textsc{id}:take  {\NOM}-prepare=to  millet   {\PSP}    {a lot}\\
 \glt ‘[She] prepared lots of millet.' (lit. millet prepared, lots)
 \z
 \ea  Ndahan  bah  məbehe  háy  ahan  \\
\gll ndahaŋ  bax        mɪ-bɛh-ɛ       haj=ahaŋ         \\
 \oldstylenums{3}\textsc{s}          pour  {\NOM}{}-pour-{\CL}  millet=\oldstylenums{3}\textsc{s}.{\POSS}  \\
 \glt ‘She poured her millet’\\
 
 \medskip
 amadala  na  kə  ver  aka  azla.\\
\gll ama-d=ala     na        kə  vɛr     aka  aɮa\\
 {\DEP}-prepare=to   \oldstylenums{3}\textsc{s}.{\DO}  on   stone  on   now\\
 \glt ‘to prepare it on the grinding stone.’
 \z
 
 \ea  {Njəw  njəw  njəw}  aməhaya  azla.\\
\gll {nzuw  nzuw  nzuw}  amə-h=aja  aɮa\\
 \textsc{id}:grind                        {\DEP}-grind={\PLU}   now\\
 \glt ‘\textit{Njəw  njəw  njəw} [she] ground [the millet] now.’
 \z
 
 \ea  Həmbo  na  ɗəw,  anday  ásak   ásak  ásak.\\
\gll hʊmbɔ  na   ɗuw à-ndaj            á-sak     á-sak\\     
 flour  {\PSP}  also   \oldstylenums{3}\textsc{s}+{\PFV}-{\PRG}   \oldstylenums{3}\textsc{s}+{\IFV}-multiply  \oldstylenums{3}\textsc{s}+{\IFV}-multiply\\ 
 
 \medskip
 \gll á-sak\\
      \oldstylenums{3}\textsc{s}+{\IFV}-multiply \\
 \glt ‘The flour, it is multiplying [and] multiplying [and] multiplying.’
 \z
 
 \ea  Ndahan  na,  ndahan  aka  {njəw  njəw  njəw.}      \\
 \gll ndahaŋ   na ndahaŋ  aka       {nzuw  nzuw  nzuw}     \\
 \oldstylenums{3}\textsc{s}  {\PSP}  \oldstylenums{3}\textsc{s}       {\EXT}  \textsc{id}:grind\\
 \glt ‘And she, she is grinding some more \textit{njəw, njəw, njəw}.’
 \z
 
 \ea  Anday  ahaya  nə  məzere  ləmes  ga.\\
 \gll à-ndaj             à-h=aja   nə  mɪ-ʒɛr-ɛ         lɪmɛʃ  ga\\
 \oldstylenums{3}\textsc{s}+{\PFV}-{\PRG}  \oldstylenums{3}\textsc{s}+{\PFV}-grind={\PLU}  with  {\NOM}{}-{do well}-{\CL}  song     {\ADJ}\\
 \glt ‘She is grinding while singing well.’
 \z
 
 \ea  Alala  na, ver  na, árəh  mbaf,  nə  həmbo  na,\\
 \gll a-l=ala na vɛr na á-rəx       mbaf    nə   hʊmbɔ  na\\
 \oldstylenums{3}\textsc{s}-go=to   {\PSP}   room     {\PSP}  \oldstylenums{3}\textsc{s}+{\IFV}-fill      {up to the roof}     with    flour  {\PSP}\\
 \glt ‘After a while, the room, it filled up to the roof with the flour,’\\
 
 \medskip\largerpage
  ɗək  məɗəkaka  alay  ana  hor  na,\\
 \gll ɗək         mə-ɗək=aka=alaj   ana      hʷɔr     na\\
 plug   {\NOM}{}-plug=on=away   {\DAT}  woman   {\PSP}\\
 \glt ‘[The flour] suffocated the woman.' (lit. plugged the room for the woman [so there was no place for her to even breathe])\\
 
 \medskip
  nata  ndahan  dəɓəsolək  məmətava  alay  \\
 \gll nata   ndahaŋ  dʊɓʊsɔlʊkʷ mə-mət=ava=alaj \\
 {and then}  \oldstylenums{3}\textsc{s}           \textsc{id}:collapse/die    {\NOM}{}-die=in=away   \\
 \glt ‘and she collapsed \textit{dəɓəsolək}, dying’\\

\medskip
  a  hoɗ  a  haj  na  ava.\\
 \gll a    hʷɔɗ  a  haj  na  ava\\
 at    stomach  {\GEN}   house   {\PSP}    in\\
 \glt ‘inside the house.’
 \z
 \subsubsection*{Dénouement}
 \ea Embesen  cacapa  na,  zar  ahan  angala.\\
 \gll ɛ{}-mbɛʃɛŋ    tsatsapa na    zar=ahaŋ         à-ŋgala\\
 \oldstylenums{3}\textsc{s}-rest      {some time}     {\PSP}  man=\oldstylenums{3}\textsc{s}.{\POSS}  \oldstylenums{3}\textsc{s}+{\PFV}-return\\
 \glt ‘After a while, her husband came back.’
 \z
 \ea  Pok  mapalay  mahay  na,\\
 \gll pɔkʷ ma-p=alaj  mahaj na\\
 \textsc{id}:open  {\NOM}{}-open=away  door   {\PSP}\\
 \glt ‘Opening the door,'\\
 
 \medskip
  həmbo  árah  na  a  hoɗ  a  hay  ava.\\
 \gll hʊmbɔ  á-rax           na      a     hʷɔɗ       a        haj     ava\\
 flour     \oldstylenums{3}\textsc{s}+{\IFV}-fill   \oldstylenums{3}\textsc{s}.{\DO}   at   stomach  {\GEN}   house     in\\
 \glt ‘the flour filled the stomach (the interior) of the house.’\\
 \z
 
 \ea  Ndahan  aməmənjere   ele  nendəye  na,  awəy,\\
 \gll ndahaŋ  amɪ-mɪnʒɛr{}-ɛ   ɛlɛ    nɛndijɛ  na awij\\
 \oldstylenums{3}\textsc{s}           {\DEP}-see-{\CL}        thing     {\DEM}   {\PSP}  said\\
 \glt ‘He, seeing the things, he said,’\\
 
 \medskip
 “Aw  aw  aw,  hor  ngehe  na,  acaw  aka  va  \\
 \gll {aw aw aw}  hʷɔr      ŋgɛhɛ    na  à-ts=aw   =aka=va\\
 {cry of death}  woman  {\DEM}      {\PSP}  \oldstylenums{3}\textsc{s}+{\PFV}-understand={\oneS}.{\IO}  =on    ={\PRF}\\
 \glt ‘“Ah, this woman, today, she didn’t listen’\\
 
 \medskip
 ma  əwla  amahan  na  bay  esəmey?\\
 \gll ma=uwla      ama-h=aŋ  na  baj    ɛʃɪmɛj\\
 word={\oneS}.{\POSS}  {\DEP}-speak=\oldstylenums{3}\textsc{s}.{\IO}   \oldstylenums{3}\textsc{s}.{\DO}   {\NEG}   {not so}\\
 \glt ‘to my instructions, did she?’ \\
 
 \medskip
 Agə  na  va  ele  ne  amahan  aməjəye  \\
 \gll à-gə na=va       ɛlɛ      nɛ    ama-h=aŋ  amɪ-dʒ-ijɛ  \\
 \oldstylenums{3}\textsc{s}+{\PFV}-do  \oldstylenums{3}\textsc{s}.{\DO}={\PRF}  thing  {\oneS}  {\DEP}-say=\oldstylenums{3}\textsc{s}.{\IO}  {\DEP}-say-{\CL}      \\
 \glt ‘She has done the thing that I told her’\\
 
 \medskip
 mege  bay  na  esəmey?\\
 \gll m\`ɛ-g-ɛ            baj  na     ɛʃɪmɛj\\
 \oldstylenums{3}\textsc{s}+{\HOR}-do-{\CL}  {\NEG}   {\PSP}  {not so}\\
 \glt ‘she should not do, not so?’\\
 
 \medskip
 Nde  nége  ehe  na,  memey  gəlo  ahay?” \\
 \gll ndɛ   n\'ɛ-g-ɛ    ɛhɛ     na     mɛmɛj  gʊlɔ=ahaj\\
 so  {\oneS}+{\IFV}-do{}-{\CL}  here   {\PSP}  how     friend=Pl\\
 \glt ‘So, what can I do here, my friends?”’\\
 \z
 
 \ea  Kəlen  tazlərav   na  ala.    \\
 \gll kɪlɛŋ  tà-ɮərav          na=ala    \\
 then    \oldstylenums{3}\textsc{p}+{\PFV}-exit  \oldstylenums{3}\textsc{s}.{\DO}=to \\
 \glt ‘Then, they took her out of the house.’
 \z
 
 \ea  Babək  mələye  na.\\
\gll babək   mɪ-l-ijɛ        na\\
 \textsc{id}:bury  {\NOM}-bury-{\CL}  \oldstylenums{3}\textsc{s}.{\DO}\\
 \glt ‘[She was] buried.’
 \z
 
\subsubsection*{Conclusion}
 \ea  Nde  ko  ala  a  ɗəma  ndana  ava  pew!\\
 \gll ndɛ    kʷɔ=ala  a    ɗəma   ndana    ava      pɛw\\
 so    until=to    at    time      {\DEM}    in    enough\\
 \glt ‘So, ever since that time, it’s done!’
 \z
 \ea  Məloko  ahay  tawəy,  “Hərmbəlom  ága  ɓərav  va  \\
 \gll Mʊlɔkʷɔ=ahaj  tawij  Hʊrmbʊlɔm     á-ga      ɓərav=va  \\
 Moloko=Pl       \oldstylenums{3}\textsc{p}+said  God        \oldstylenums{3}\textsc{s}+{\IFV}-do heart={\PRF}     \\
 \glt ‘The Molokos say, “God got angry’ (lit. God did heart)\\
 
 \medskip
  kəwaya  war  dalay  na,  amecen  sləmay  bay  ngəndəye.”\\
\gll kuwaja        war    dalaj     na amɛ-tʃɛŋ      ɬəmaj  baj     ŋgɪndijɛ\\
 {because of}  child    girl    {\PSP}  {\DEP}-hear   ear      \textsc{neg}  {\DEM}\\
 \glt ‘because of that girl, that one that was disobedient.”’
 \z
 
 \ea  Waya  ndana  Hərmbəlom  ázata  aka  barka  ahan  va.\\
 \gll waja   ndana  Hʊrmbʊlɔm   á-z=ata      =aka   \\    
 because   {\DEM}   God             \oldstylenums{3}\textsc{s}+{\IFV}-take  =\oldstylenums{3}\textsc{p}.{\IO}  =on   \\  
 
 \medskip
 \gll barka=ahaŋ=va\\
 blessing=\oldstylenums{3}\textsc{s}.{\POSS}={\PRF}\\
 \glt ‘Because of that, God had taken back his blessing from them.’\\
 \z
 
 \ea  Cəcəngehe  na,  war  elé  háy  bəlen  na,  ásak  asabay.\\
 \gll tʃɪtʃɪŋgɛhɛ  na,  war  ɛlɛ  haj  bɪlɛŋ  na  á-sak asa-baj\\
 now {\PSP}  child   eye   millet   one   {\PSP}  \oldstylenums{3}\textsc{s}+{\IFV}-multiply    again-{\NEG}\\
 \glt ‘And now, one grain of millet, it doesn’t multiply anymore.’\\
 \z
 \ea  Talay  war  elé  háy  bəlen  kə  ver  aka  na,  ásak  asabay.\\
 \gll talaj     war  ɛlɛ  haj  bɪlɛŋ  kə  vɛr  aka  na  á-sak       asa-baj\\
 \textsc{id}:put  child   eye   millet    one    on    stone    on    {\PSP}  \oldstylenums{3}\textsc{s}+{\IFV}-multiply  again-\textsc{neg}\\
 \glt ‘[If] one puts one grain of millet on the grinding stone, it doesn’t multiply anymore.’\\
 \z
 
 \ea  Səy  kádəya  gobay.\\
 \gll sij ká-d=ija  gʷɔbaj\\
 only    {\twoS}+{\IFV}-prepare={\PLU}   {a lot}\\
 \glt ‘You must put on a lot.’\\
 \z
 
 \ea  Ka  nehe  tawəy,  “Metesle  anga  war  dalay  ngəndəye  \\
\gll ka  nɛhɛ  tawij  mɛ-tɛɬ-ɛ      aŋga  war    dalaj  ŋgɪndijɛ \\
 like  {\DEM}   \oldstylenums{3}\textsc{p}+said  {\NOM}{}-curse-{\CL}   {\POSS}   child  girl       {\DEM}      \\
 \glt ‘It is like this they say, “The curse [is] belonging to that young woman’\\
 
 \medskip
  amazata  aka  ala  avəya  nengehe  ana  məze  ahay  na.”\\
\gll ama-z=ata      =aka=ala      avija    nɛŋgɛhɛ  ana    mɪʒɛ=ahaj   na\\
 {\DEP}-take=\oldstylenums{3}\textsc{p}.{\IO}   =on=to  suffering  {\DEM}      {\DAT} person    =Pl  {\PSP}\\
 \glt ‘that brought this suffering to the people.”’ 
 \z

 \ea  Ka  nehe  ma  bamba  ga  andavalay.    \\
 \gll ka  nɛhɛ  ma  bamba   ga  à-ndava=alaj\\
 like  {\DEM}  word   story     {\ADJ}     \oldstylenums{3}\textsc{s}+{\PFV}-finish=away \\
 \glt ‘It is like this the story ends.’  
 \z
 
% 
\section[Cicada story]{Cicada story}\setcounter{equation}{0}\label{sec:1.6}
\hypertarget{RefHeading1210361525720847}{}
This fable was recorded in Maroua, Far North Province of Cameroon, in 2001.  
% 
\subsubsection*{Setting}
\ea   Bamba  bamba!\\
\gll bamba   bamba\\
	        story         story   \\
\glt ‘Once upon a time…’ (lit. story, story)
 \z

\ea Tawəy: \\
\gll   tawij \\
\oldstylenums{3}\textsc{p}+said  \\  
\glt ‘They say:’
  \z

\ea Albaya  ahay  aba.\\
 \gll albaja=ahaj              aba\\
        youth=Pl    {\EXT}\\
 \glt ‘There were some young men.’
  \z

\ea Tánday  t\'ətalay  a  ləhe.\\
 \gll tá{}-ndaj         t\'ə{}-tal-aj           a  lɪhɛ\\
   \oldstylenums{3}\textsc{p}+{\IFV}-{\PRG}   \oldstylenums{3}\textsc{p}+{\IFV}-walk{}-{\CL}   at    bush\\
 \glt ‘They were walking in the bush.’
 \z
 \subsubsection*{Episode 1}
\ea      Tánday  t\'ətalay  a  ləhe  na,\\
\gll tá-ndaj        t\'ə-tal-aj    a   lɪhɛ     na\\
  \oldstylenums{3}\textsc{p}+{\IFV}-{\PRG}   \oldstylenums{3}\textsc{p}-walk{}-{\CL}   at  bush   {\PSP}\\
\glt ‘[As] they were walking in the bush,’\\
         
     \medskip
təlo  tənjakay  agwazla  malan  ga  a  ləhe.  \\
\gll t\`ə-lɔ           t\`ə-nzak-aj        agʷaɮa   malaŋ  ga      a    lɪhɛ\\
 \oldstylenums{3}\textsc{p}+{\PFV}-go   \oldstylenums{3}\textsc{p}+{\PFV}-find{}-{\CL}   {spp. of tree}     large   {\ADJ}  at   bush\\
\glt `they went and found a large tree (a particular species) in the bush.’
 \z
\subsubsection*{Episode 2}
\ea Albaya  ahay  ndana  kəlen  təngalala  ma  ana  bahay.  \\
\gll albaja  =ahaj  ndana  kɪlɛŋ  t\`ə-ŋgala=ala     ma    ana      bahaj\\
          youth    =Pl       {\DEM}  then \oldstylenums{3}\textsc{p}+{\PFV}-return=to  word    {\DAT} chief\\
        \glt ‘Those young men then took the word (response) to the chief.’  
 \z

\ea     Tawəy,  “Bahay,  mama  agwazla  ava  a  ləhe  na,   \\
\gll tawij bahaj mama   agʷaɮa       ava  a   lɪhɛ  na   malaŋ ga  na\\
  \oldstylenums{3}\textsc{p}+said  chief   mother   {spp. of tree}  {\EXT}   at   bush    {\PSP}  large     {\ADJ}  {\PSP}\\
 \glt ‘They said, “Chief, there is a mother-tree in the bush, a big one,’\\
 
 \medskip
 agasaka  na  ka  mahay  ango  aka  aməmbese.”\\
\gll 	à-gas=aka   na  ka    mahaj=aŋgʷɔ aka  amɪ-mbɛʃ-ɛ\\
	\oldstylenums{3}\textsc{s}+{\PFV}-get=on  {\PSP}   on     door={\twoS}.{\POSS}  on  {\DEP}-rest-{\CL}\\
 \glt ‘[and] it would please you to have that tree at your door, so that you could rest under it.”’
 \z

\ea     Kəlen  albaya  ahay  ndana  tolo.  \\
 \gll kɪlɛŋ  albaja=ahaj  ndana  t\`ə-lɔ\\
 then   youth=Pl     {\DEM}   \oldstylenums{3}\textsc{p}+{\PFV}-go\\ 
 \glt ‘Then, those young men went.’
 \z

\ea     Nde, bahay awəy,  “Nde na,  səy  slərom  alay  war.\\
 \gll ndɛ bahaj   awij     ndɛ  na  sij  ɬər-ɔm=alaj         war\\
 so    chief  said  so  {\PSP}  only  {send[{\IMP}]-{\twoP}}=away   child\\
  \glt ‘And so the chief said, “So, you must send a child.’\\
  
  \medskip
  Káazəɗom  anaw  ala  agwazla  ndana  ka  mahay  əwla  aka.\\
  \gll káá{}-zʊɗ{}-ɔm    an=aw    =ala    agʷaɮa    ndana    ka   mahaj=uwla      aka\\
  {\twoP}+{\POT}-take-{\twoP}      {\DAT}={\oneS}.{\IO}  =to       {spp. of tree}   {\DEM}   on   door={\oneS}.{\POSS}  on\\
  \glt ‘You will bring that tree to my door for me.’\\
  
  \medskip
 Káafəɗom  anaw  ka  mahay  əwla  aka.”\\
 \gll káá{}-fʊɗ{}-ɔm          an=aw      ka   mahaj=uwla     aka\\
  {\twoP}+{\POT}-put-{\twoP}    {\DAT}={\oneS}.{\IO}   on  door={\oneS}.{\POSS}   on\\
  \glt ‘You will put it by my door.”’ 
  \z

\ea Bahay  kəlen  ede  gəzom.  \\
 \gll bahaj  kɪlɛŋ  à-d-ɛ                 gʊzɔm\\
  chief  then  \oldstylenums{3}\textsc{s}+{\PFV}-prepare{}-{\CL}  beer\\
  \glt ‘The king then made millet beer.’
  \z

\ea   Aslar  məze  ahay.  \\
 \gll à-ɬar    mɪʒɛ=ahaj\\
  \oldstylenums{3}\textsc{s}+{\PFV}-send    person=Pl\\
 \glt ‘He sent out the people.’
 \z

\ea   Təlo  tamənjar  na  ala  mama  agwazla  nəndəye.\\ 
\gll t\`ə-lɔ             tà-mənzar            na=ala  mama   agʷaɮa     nɪndijɛ\\
  \oldstylenums{3}\textsc{p}+{\PFV}-go  \oldstylenums{3}\textsc{p}+{\HOR}-see  \oldstylenums{3}\textsc{s}.{\DO}=to     mother  {spp. of tree}  {\DEM}\\
  \glt ‘They went to see the mother-tree there.’
  \z

\ea   Məze  ahay  tangala  ma  ana  bahay.\\
 \gll mɪʒɛ=ahaj   tà-ŋg=ala  ma      ana  bahaj\\
  person=Pl    \oldstylenums{3}\textsc{p}+{\PFV}-return=to   word  {\DAT}  chief\\
  \glt ‘The people brought back word to the chief.’
  \z

\ea   Tawəy,  “Ɗeɗen  bahay,  agwazla  ngəndəye \\
  \gll tawij     ɗɛɗɛŋ   bahaj    agʷaɮa  ŋgɪndijɛ \\  
  \oldstylenums{3}\textsc{p}+said  truth     chief     {spp. of tree}    {\DEM}  \\
  \glt ‘They said, “It is true, chief. That tree,'
  
  \medskip  
 ágasaka  ka  mahay  ango  aka, \\
\gll   á-gas=aka   ka    mahaj=aŋgʷɔ aka\\
  \oldstylenums{3}\textsc{s}+{\IFV}-get=on    at   door={\twoS}.{\POSS} on\\
  \glt 'It would be pleasing if it would be by your door,’\\
  
  \medskip
 bəyna  agwazla  ga  səlom  ga;  aɓəsay  ava  bay.”\\
 \gll bijna      agʷaɮa  ga    sʊlɔm ga  aɓəsaj  ava     baj\\ 
 because   {spp. of tree}  {\ADJ}   good   {\ADJ}   blemish   {\EXT}  {\NEG}\\
  \glt ‘because this tree is good;  it has no faults.”’
  \z
  
 \subsubsection*{Episode 3}
\ea   Bahay  alala  a  həlan  na,  ndahan  gədok  mədəye  gəzom.\\
\gll  bahaj  à-l=ala    a   həlaŋ   na  ndahaŋ  gʊdɔkʷ m\`ɪ-d-ijɛ\\      
 chief   \oldstylenums{3}\textsc{s}+{\PFV}-go=to   at   back   {\PSP}  \oldstylenums{3}\textsc{s}        {\textsc{id}:prepare beer}  {\NOM}-prepare-{\CL}\\ 
 
 \medskip
 \gll gʊzɔm\\
      beer \\
 \glt ‘The chief then came behind [and] he made millet beer.’
 \z
 
\ea  Kəlen  albaya  ahay  tolo  amazala  agwazla  na,\\
 \gll kɪlɛŋ  albaja=ahaj    t\`ɔ-lɔ      ama-z=ala   agʷaɮa  na\\
 then   youth=Pl      \oldstylenums{3}\textsc{p}+{\PFV}-go   {\DEP}-take=to  {spp. of tree}  {\PSP}\\
 \glt ‘And then, the young men left to bring back the tree,’\\
 
 \medskip
  taazala təta  bay.\\
 \gll tàà-z=ala  təta baj\\
 \oldstylenums{3}\textsc{p}+{\HOR}-take=to      ability  {\NEG}\\
 \glt ‘[but] they were not able to bring [it].’  
 \z
\ea   Mədəye  gəzom  makar.  \\
 \gll mɪ-d-ijɛ      gʊzɔm     makar. \\
 {\NOM}{}-prepare{}-{\CL}       beer  three\\
 \glt ‘[He] made beer for the third time.’
 \z
\ea   Bahay  alala  a  həlan  na,  awəy,\\
 \gll bahaj    à-l=ala      a   həlaŋ   na awij\\
 chief   \oldstylenums{3}\textsc{s}+{\PFV}-go=to   at   back   {\PSP}  said  \\  
 \glt ‘[And then], the chief came behind, saying,’\\
 
 \medskip
  “Náanjakay  na  wa  amazaw  ala  agwazla  ana  ne  na  way?\\
 \gll náá-nzak-aj        na       wa  ama-z=aw =ala      agʷaɮa    ana     nɛ   na  waj\\
 {\oneS}+\textsc{{\POT}}{}-find{}-{\CL}   {\PSP}   who   {\DEP}-take={\oneS}.{\IO}  =to    {spp. of tree}     {\DAT}    {\oneS}    {\PSP}  who\\
 \glt ‘“Who can I find to bring to me this tree for me?’\\
 
 \medskip
 Kə  mahay  aka  na  náambasaka  na,\\
\gll  kə   mahaj    aka  na   náá-mbas=aka   na\\
 on    door       on      {\PSP}   {\oneS}+{\POT}-rest=on      {\PSP}\\
 \glt ‘By my door I will be able to rest well.’\\

 \medskip
  Mama  agwazla  səlom  ga  lala.”\\
 \gll mama    agʷaɮa       sʊlɔm   ga     lala\\
 mother   {spp. of tree}     good     {\ADJ}   well\\
 \glt ‘The mother-tree is very good.”’
 \z
 
 \subsubsection*{Prepeak}
\ea   Kəlen  bahay  na,  olo  kə  mətəɗe  aka.  \\
\gll  kɪlɛŋ bahaj  na  \`ɔ-lɔ        kə  mɪtɪɗɛ  aka \\
 then    chief  {\PSP}  \oldstylenums{3}\textsc{s}+{\PFV}-go    on  cicada  on\\
 \glt ‘Then, the chief went to the cicada.’  
 \z
 
\ea  Mətəɗe  awəy, “Bahay,  toko! \\
\gll mɪtɪɗɛ   awij   bahaj  tɔkʷɔ\\
 cicada  said  chief   {go[{\IMP}.\oldstylenums{1}\textsc{Pin}}] \\
 \glt ‘The cicada said, “Chief, let’s go!’  \\
 
 \medskip
 Náamənjar  na  alay  memele  ga  ndana  əwɗe.”\\
 \gll náá{}-mənzar   na=alaj   mɛmɛlɛ ga   ndana  uwɗɛ\\
 {\oneS}+{\POT}-see   \oldstylenums{3}\textsc{s}.{\DO}=away   tree   {\ADJ}  {\DEM}   first\\
\glt ‘First I want to see the tree that you spoke of.”’
 \z
 
\ea  Məze  ahay  tawəy,  “A a  məze  ahay  səlom  ahay  ga  na,  \\
 \gll mɪʒɛ=ahaj  tawij   aa mɪʒɛ=ahaj   sʊlɔm=ahaj   ga   na\\
 person=Pl  \oldstylenums{3}\textsc{p}+said  ah  person=Pl     good=Pl   {\ADJ}   {\PSP}\\
 \glt ‘The people said, “O, even good people,’\\
 
 \medskip
 tázala  təta  bay  na,  \\
\gll  tá-z=ala     təta baj         na \\
 \oldstylenums{3}\textsc{p}+{\IFV}-take=to  ability  {\NEG}   {\PSP}\\
 \glt ‘they can’t bring it,’  \\
 
 \medskip
 azləna  mətəɗe  azla,  engeren  azla,   káazala  təta  na, \\
 \gll aɮəna  mɪtɪɗɛ    aɮa   ɛŋgɛrɛŋ   aɮa káá-z=ala təta          na\\
 but    cicada    now      insect     now   {\twoS}+{\POT}-take=to    ability   {\PSP}\\
 \glt ‘but you, cicada, an insect, you think you can bring it,\\
 
 \medskip
  káazala  na,  malma  ango  may?”\\
 \gll káá-z=ala     na  malma=aŋgʷɔ    maj\\
 {\twoS}+{\POT}-take=to  {\PSP}   what={\twoS}.{\POSS}  what\\
 \glt ‘[if] you do bring it,  [then] what is with you?”’
 \z
\ea  Mətəɗe  awəy,  “Náazala!”\\                   
 \gll mɪtɪɗɛ  awij náá-z=ala\\
 cicada          said    {\oneS}+{\POT}-take=to\\      
 \glt ‘The cicada said, “I will bring [it].”’
 \z
\ea  “Káazala   təta   bay!”\\
 \gll káá-z=ala     təta     baj\\
 {\twoS}+{\POT}-take=to  ability  {\NEG}\\
 \glt ‘“You can’t bring [it].”’
 \z
\ea  “Náazala!  Nde  toko  əwɗe!”\\                
 \gll náá-z=ala ndɛ tɔkʷɔ     uwɗɛ\\
 {\oneS}+{\POT}-take=to  so  go[{\IMP}.\oldstylenums{1}\textsc{Pin}]    first\\
 \glt ‘“I will bring [it], but first, let’s go!”’
 \z
 \subsubsection*{Peak}
\ea  Nata  olo.\\
 \gll nata      \`ɔ-lɔ\\
 {and then}     \oldstylenums{3}\textsc{s}+{\PFV}-go  \\
 \glt ‘And then, he went.’\\
 \z
\ea  Albaya  ahay  tolo  sen  na,    \\
\gll albaja=ahaj   tɔ-lɔ    ʃɛŋ      na\\
 youth=Pl      \oldstylenums{3}\textsc{p}-go   \textsc{id}:go   {\PSP}  \\
 \glt ‘The young men went,’\\
 
 \medskip
 albaya  ahay  weley  təh  anan  dəray  na, abay.\\
 \gll albaja=ahaj   wɛlɛj  təx     an=aŋ         dəraj   na  abaj\\
 youth=Pl    which   \textsc{id}:put   {\DAT}=\oldstylenums{3}\textsc{s}.{\IO}   head   {\PSP}   {{\EXT}+{\NEG}}\\
 \glt ‘[and] no one could lift it.’ (lit. whichever young man put his head [to the tree in order to lift it], there was none)
 \z
\ea  Nata  mətəɗe  təh  anan  dəray  ana  agwazla  ngəndəye.\\
 \gll nata  mɪtɪɗɛ  təx  an=aŋ    dəraj  ana  agʷaɮa  ŋgɪndijɛ\\
 {and then}   cicada   \textsc{id}:put    {\DAT}=\oldstylenums{3}\textsc{s}.{\IO}   head   {\DAT}  {spp. of tree}    {\DEM}\\
 \glt ‘And then,  the cicada put his head to that tree.’
 \z
 
\ea  Kəw  na! \\
 \gll kuw  na\\
 \textsc{id}:getting  \oldstylenums{3}\textsc{s}.{\DO}  \\
 \glt ‘[He] got it.’ 
 \z
\ea  Dergwecek!\\
 \gll dɛrgʷɛtʃɛk\\
 \textsc{id}:{lifting onto head}\\
 \glt ‘[He] lifted [it] onto his head.’
 \z
 \subsubsection*{Dénouement}
\ea  Amagala  ləmes,  {“Te te te te ver na tepəɗek təvəw na tambəɗek…”}\\
 \gll ama-g=ala    lɪmɛʃ  {Tɛ tɛ tɛ tɛ vɛr na tɛpɪɗɛk təvuw na tambɪɗɛk…}\\
 {{\DEP}-do=to}  song  {[words of the song]}\\
 \glt ‘He was singing (song is given), [on his way] to [the chief’s house].’\\
 \z
\ea  Sen  ala.  \\
 \gll ʃɛŋ=ala\\
 \textsc{id}:go=to\\
 \glt ‘Going, [he came to the chief’s house].’  \\
 \z

 \ea  Tahan  na.\\
 \gll tà-h=aŋ      na\\
 \oldstylenums{3}\textsc{p}+{\PFV}-greet=\oldstylenums{3}\textsc{s}.{\IO}   {\PSP}\\
 \glt ‘They greeted him.’
 \z
 
\ea Mama  agwazla  na,  ka  mahay  aka  afa  bahay  gəɗəgəzl!         \\
 \gll mama  agʷaɮa     na  ka   mahaj   aka   afa       bahaj  gəɗəgəɮ \\
 mother  {spp. of tree}   {\PSP}  on     door        on      {at place of}     chief   {\textsc{id}:put down}\\
 \glt ‘The mother-tree,  at the door of the chief’s house, [he] put [it] down.’  
 \z
\ea  Bahay  na  membese,  “Səwse,   mətəɗe   səwse, səwse, səwse!”\\
 \gll bahaj      na  mɛ-mbɛʃ-ɛ  ʃuwʃɛ   mɪtɪɗɛ   ʃuwʃɛ  ʃuwʃɛ  ʃuwʃɛ\\
 chief  {\PSP}  {\NOM}-smile{}-{\CL}  thanks   cicada   thanks   thanks    thanks\\
 \glt ‘The chief smiled, [saying] “Thank you, thank you, thank you cicada!”’
 \z
 
\ea  Mama  agwazla  na,  kə  mahay  anga  bahay  aka.\\
 \gll mama  agʷaɮa   na  kə  mahaj  aŋga  bahaj  aka\\
 mother  {spp. of tree}       {\PSP}  on     door  {\POSS}  chief  on\\
 \glt ‘The mother tree [is] by the chief's door.’
 \z

\ea  Andavalay.\\
 \gll a-ndava=alaj\\ 
 \oldstylenums{3}\textsc{s}-finish=away\\
 \glt ‘It is finished.’
\z

\clearpage
 \section[Values exhortation]{Values exhortation }\setcounter{equation}{0}\label{sec:1.7}
 \hypertarget{RefHeading1210381525720847}{}
 This exhortation was given in Lalaway, Far North Province of Cameroon, in 2002.  
 
 \subsubsection*{Setting}
\ea Səlom  ga  yawa  təde  kəyga!\\
 \gll sʊlɔm  ga  jawa   tɪdɛ   kijga\\
 goodness  {\ADJ}  well  good  {like this}\\
 \glt ‘Good, well, good, [it is] like this:’
 \z

\ea Ehe  na,  wəyen  ambaɗala  a  jere  azla.\\
 \gll ɛhɛ     na wijɛŋ   à-mbaɗ=ala   a   dʒɛrɛ   aɮa\\
 here    {\PSP}  earth  \oldstylenums{3}\textsc{s}+{\PFV}-change=to  at  truth  now\\
 \glt ‘Here, the earth has changed to truth now (sarcastic).’
 \z

\ea Səwat  na,   təta  a  məsəyon  na  ava  nəndəye  na,\\  
 \gll suwat  na təta  a  mʊsijɔŋ   na   ava  nɪndijɛ  na\\
 \textsc{id}:disperse  {\PSP}  \oldstylenums{3}\textsc{p}      at   mission  {\PSP}  in  {\DEM}  {\PSP}  \\
 \glt ‘As the people go home from church,’ (lit. disperse, they in the mission there),’ \\
 
 \medskip
 Pester  áhata,  “Ey!  Ele  nehe  na,  kógom  bay!”  \\
\gll pɛʃtɛr  á-h=ata   ɛj     ɛlɛ      nɛhɛ   na k\'ɔ-gʷ-ɔm baj\\
 pastor  \oldstylenums{3}\textsc{s}+{\IFV}-tell=\oldstylenums{3}\textsc{p}.{\IO}  hey  thing  {\DEM}  {\PSP} {\IFV}-do-{\twoP}  {\NEG}\\
 \glt ‘Pastor told them, “Hey! These things, don’t do them!”’
 \z

\ea Yawa,  war  dalay  ga ándaway  mama  ahan.\\
 \gll jawa   war   dalaj  ga  á-ndaw-aj   mama=ahaŋ\\
 well    child  female  {\ADJ}  \oldstylenums{3}\textsc{s}+{\IFV}-insult{}-{\CL}  mother=\oldstylenums{3}\textsc{s}.{\POSS}\\
 \glt ‘Well, the girls insult their mothers.’ 
 \z

\ea War  zar  ga  ándaway  baba  ahan.\\
 \gll war     zar  ga  á-ndaw-aj   baba=ahaŋ\\
 child  male  {\ADJ}  \oldstylenums{3}\textsc{s}+{\IFV}-insult{}-{\CL}  father=\oldstylenums{3}\textsc{s}.{\POSS}\\
 \glt ‘[And] the boys insult their fathers.’ 
 \z
\clearpage
\ea Yo  ele  ahay  aməgəye  bay  nəngehe    pat,\\
 \gll jɔ      ɛlɛ=ahaj  amə-g-ijɛ   baj      nɪŋgɛhɛ    pat\\
 well   thing=Pl      {\DEP}-go-{\CL}    {\NEG}    {\DEM}     all\\
 \glt ‘Well, all these particular things that we are not supposed to do,’\\
 
 \medskip
  tahata  na  va  kə  dəftere  aka.\\
 \gll tà-h=ata     na=va   kə   dɪftɛrɛ  aka\\
 \oldstylenums{3}\textsc{p}+{\PFV}-tell=\oldstylenums{3}\textsc{p}.{\IO}   \oldstylenums{3}\textsc{s}.{\DO}={\PRF}  on  book  on\\
 \glt ‘they have already told them in the book.’
 \z

\ea Hərmbəlom  awacala  kə  okor  aka.\\
 \gll Hʊrmbʊlɔm   à-wats=ala   kə   ɔkʷɔr   aka\\
 God    \oldstylenums{3}\textsc{s}+{\PFV}-write=to  on  stone  on\\
 \glt ‘God wrote them on the stone [tablet].’
 \z

\ea Álala,  asara  agas.\\
 \gll á-l=ala    asara     à-gas\\
 \oldstylenums{3}\textsc{s}+{\IFV}-go=to    {white man}  \oldstylenums{3}\textsc{s}+{\PFV}-catch\\
 \glt ‘Later, the white man accepted [it] (lit. caught).’
 \z

\ea Ege  dəftere  ahan  kə  dəwnəya  aka.\\
 \gll \`ɛ{}-g-ɛ     dɪftɛrɛ=ahaŋ          kə   duwnija    aka\\
 \oldstylenums{3}\textsc{s}+{\PFV}-do-{\CL}  book=\oldstylenums{3}\textsc{s}.{\POSS}       on  earth    on\\
 \glt ‘He made his book on the earth.’ 
 \z

\ea Ahata  na  va,  “Ele nehe na, awasl,\\
 \gll à-h=ata   na=va      ɛlɛ     nɛhɛ      na    à-waɬ\\ 
 \oldstylenums{3}\textsc{s}+{\PFV}-tell=\oldstylenums{3}\textsc{p}.{\IO}   \oldstylenums{3}\textsc{s}.{\DO}={\PRF}   thing  {\DEM}  {\PSP}  \oldstylenums{3}\textsc{s}+{\PFV}-forbid  \\
 \glt ‘He has told them already, “This thing is forbidden,’\\
 
 \medskip
  ele nehe na, awasl, ele nehe na, awasl,\\
 \gll ɛlɛ       nɛhɛ     na      à-waɬ    ɛlɛ       nɛhɛ       na      à-waɬ \\
 thing    {\DEM}     {\PSP}   \oldstylenums{3}\textsc{s}+{\PFV}-forbid  thing    {\DEM}     {\PSP}   \oldstylenums{3}\textsc{s}+{\PFV}-forbid    \\
 \glt ‘this thing is forbidden, this thing is forbidden,’\\
 
 \medskip
 ele nehe na, awasl, kəro!”\\
 \gll ɛlɛ       nɛhɛ     na     à-waɬ    kʊrɔ\\
 thing    {\DEM}     {\PSP}   \oldstylenums{3}\textsc{s}+{\PFV}-forbid    ten\\
 \glt ‘this thing is forbidden – ten [commandments]”’\\
\z

\ea   Ahata  na  cece.\\
 \gll à-h=ata   na   tʃɛtʃɛ.\\
 \oldstylenums{3}\textsc{s}+{\PFV}-tell=\oldstylenums{3}\textsc{p}.{\IO}  \oldstylenums{3}\textsc{s}.{\DO}  all\\
 \glt ‘He told all of them.’ 
 \z

\ea Yawa  nde  ele  nehe  ɗəw,  kóogəsok ma  Hərmbəlɔm.  \\
 \gll jawa  ndɛ  ɛlɛ   nɛhɛ   ɗuw k\'ɔ\'ɔ-gʷʊs-ɔkʷ ma   Hʊrmbʊlɔm\\
 well    so  thing  {\DEM}  also  {\twoS}+{\POT}-catch-{\twoP}    word  God\\
 \glt ‘So, this thing here, you should accept the word of God.’ 
 \z

\ea A  məsəyon  ava  na,  ele  ahay  aməwəsle  na,  tége  bay.\\
 \gll a  mʊsijɔŋ    ava   na  ɛlɛ=ahaj  amu-wuɬ-ɛ     na   t\'ɛ-g-ɛ     baj\\
 at  mission    in  {\PSP}  thing=Pl        {\DEP}-forbid-{\CL}   {\PSP}  \oldstylenums{3}\textsc{p}+{\IFV}-do-{\CL}  {\NEG}\\
 \glt ‘In the church, these things that are forbidden, they don’t do.’
 \z

\ea Yo, asara  ahata  na  va.    \\
 \gll jɔ     asara     à-h=ata  na=va \\
 well {white man}  \oldstylenums{3}\textsc{s}+{\PFV}-tell=\oldstylenums{3}\textsc{p}.{\IO}  \oldstylenums{3}\textsc{s}.{\DO}={\PRF}\\
 \glt ‘Well, the white man told it to them already.’ 
 \z

\ea Pester  ahata  na  va.  \\
 \gll pɛʃtɛr   à-h=ata   na=va\\
 pastor  \oldstylenums{3}\textsc{s}+{\PFV}-tell=\oldstylenums{3}\textsc{p}.{\IO}  \oldstylenums{3}\textsc{s}.{\DO}={\PRF}\\
 \glt ‘The pastor told it to them already.’ 
 \z

\ea T\'əlala a  həlan  ga  ava  ese,   \\
 \gll t\'ə-l=ala   a   həlaŋ  ga   ava   ɛʃɛ   \\
 \oldstylenums{3}\textsc{p}+{\IFV}-go=to  at  back  {\ADJ}  in  again  \\
 \glt ‘They come [home] after [church] again,’\\
 
 \medskip
  t\'əwəɗakala  har  a  məsyon  ava.\\
 \gll t\'ə-wəɗak=ala   har   a   mɪsijɔŋ   ava\\
 \oldstylenums{3}\textsc{p}+{\IFV}-divide=to  body  at  mission  in\\
 \glt ‘they disperse after church.’ (lit. they divide body in mission)
 \z

\ea Álaway  war  ahan.\\
 \gll á-law-aj      war=ahaŋ\\
 \oldstylenums{3}\textsc{s}+{\IFV}-mate{}-{\CL}  child=\oldstylenums{3}\textsc{s}.{\POSS}\\
 \glt ‘[One] sexually abuses his child.’
 \z

\ea  Ólo  áɓan  ana  baba  ahan.\\ 
 \gll \'ɔ-lɔ     á-ɓ=aŋ   ana  baba=ahaŋ\\
 \oldstylenums{3}\textsc{s}+{\IFV}-go  \oldstylenums{3}\textsc{s}+{\IFV}-hit=\oldstylenums{3}\textsc{s}.{\IO}  {\DAT}       father=\oldstylenums{3}\textsc{s}.{\POSS}\\
 \glt ‘[Another] goes and hits his father.’ 
 \z

\ea Ólo  ápaɗay  məze  nə  madan. \\
 \gll \'ɔ{}-lɔ     á-paɗ-aj       mɪʒɛ   nə   madaŋ\\
 \oldstylenums{3}\textsc{s}+{\IFV}-go  \oldstylenums{3}\textsc{s}+{\IFV}-crunch{}-{\CL}  person  with  magic  \\
 \glt ‘[Another] goes and kills someone with sorcery.’ (lit. he goes he eats a person with magic)
 \z

\ea Olo  aka  akar.\\
 \gll à-lɔ         aka   akar\\
 \oldstylenums{3}\textsc{s}+{\PFV}-go   on  theft  \\
 \glt ‘[Another] goes and steals.’ (lit. he went on theft)
 \z

\ea Ege  adama.\\
 \gll à-g-ɛ     adama\\
 \oldstylenums{3}\textsc{s}+{\PFV}-do-{\CL}  adultery\\
 \glt ‘[Another] commits adultery.’ 
 \z

\ea Təta  dəl  na  ma  Hərmbəlom  nendəye.\\
 \gll təta   dəl     na   ma   Hʊrmbʊlɔm   nɛndijɛ\\
 \oldstylenums{3}\textsc{p}    \textsc{id}:insult    \oldstylenums{3}\textsc{s}.{\DO}  word  God    {\DEM}\\
 \glt ‘They insult it, this word of God!’
 \z

\ea Nde na  cəveɗ  ahan  na,  memey?\\
 \gll ndɛ  na   tʃɪvɛɗ=ahaŋ     na   mɛmɛj\\
 so  {\PSP}  path=\oldstylenums{3}\textsc{s}.{\POSS}  {\PSP}  how\\
 \glt ‘So, what can he do?’ (lit. how [is] his pathway)
 \z

\ea Táagas  na  anga  way?\\
 \gll táá-gas     na   aŋga   waj\\
 \oldstylenums{3}\textsc{p}+{\POT}-catch  {\PSP}  {\POSS}  who\\
 \glt ‘They will accept whose word?’ (lit. they will catch it, [something] that belongs to whom?)
 \z
\clearpage
\ea  Ma   a  baba  ango  kagas  asabay.\\
 \gll ma     a  baba=aŋgʷɔ     kà-gas     asa-baj \\
 word  {\GEN}  father=\oldstylenums{3}\textsc{s}.{\POSS}  {\twoS}+{\PFV}-catch  again-{\NEG}  \\
 \glt ‘Your father’s word you no longer accept.’
\z

\ea  Ma   a  mama  ango  kagas  asabay.\\
 \gll ma     a  mama=aŋgʷɔ    kà-gas     asa-baj\\
 word  {\GEN}  mother=\oldstylenums{3}\textsc{s}.{\POSS}  {\twoS}+{\PFV}-catch  again-{\NEG}\\
 \glt ‘Your mother’s word you no longer accept.’
 \z

\ea Nde na  káagas  anga  way?\\
 \gll ndɛ  na   káá-gas       aŋga   waj\\
 so  {\PSP}  {\twoS}+{\POT}-catch    {\POSS}  who\\
 \glt ‘So, you don’t accept anyone’s word!’ (lit. you will catch [that which] belongs to whom?)
 \z

\ea Anga  Hərmbəlom  ga  kagas  asabay. \\
 \gll aŋga   Hʊrmbʊlɔm   ga     kà-gas     asa-baj\\
 {\POSS}  God    {\ADJ}    {\twoS}+{\PFV}-catch  again-{\NEG}\\
 \glt ‘The very [word] of God himself you no longer accept.’
 \z

\ea Hərmbəlom  na,  amaɗaslava  ala  məze   na,\\
 \gll Hʊrmbʊlɔm  na  ama-ɗaɬ=ava=ala  mɪʒɛ  na\\
 God    {\PSP}    {\DEP}-multiply=in=to   person   {\PSP}     \\
 \glt ‘God, who multiplied the people,’\\
 
 \medskip
  ndahan  ese  na,  kagas  ma  Hərmbəlom  na,  asabay  na,  \\
 \gll ndahaŋ  ɛʃɛ  na  ka-gas      ma  Hʊrmbʊlɔm  na       asa-baj         na\\
 \oldstylenums{3}\textsc{s}     again    {\PSP}    {\twoS}-catch    word     God          {\PSP}  again-{\NEG}  {\PSP} \\
 \glt ‘if you will never accept the word of God,’ (lit. him again, you never catch the word of God)\\
 
 \medskip
  káagas  na  anga  way?\\
 \gll káá-gas            na      aŋga     waj\\
 {\twoS}+{\POT}-catch    {\PSP}   {\POSS}  who\\
 \glt ‘whose word will you accept then?’ (lit. you will catch it, that which belongs to who)
 \z
 
\clearpage
 \ea Səlom  ga.  \\
 \gll sʊlɔm  ga \\
 goodness  {\ADJ}\\
 \glt ‘Good!’ [narrator to himself].
 \z

\ea Asara  anday  áɗakaləme  ma  a  dəwnəya.\\
 \gll asara   à-ndaj        á-ɗak=alɪmɛ               ma     a   duwnija\\
 {white man}     \oldstylenums{3}\textsc{s}+{\PFV}-{\PROG}  \oldstylenums{3}\textsc{s}+{\IFV}-show=\oldstylenums{1}\textsc{Pex}.{\IO}    word   {\GEN}    earth    \\
 \glt ‘The white man is showing us how the world is (lit. the word of the earth).’ 
 \z

\ea Anday  áɗakaləme  endeɓ.  \\
 \gll à-ndaj           á-ɗak=alɪmɛ     ɛndɛɓ \\
 \oldstylenums{3}\textsc{s}+{\PFV}-{\PROG}    \oldstylenums{3}\textsc{s}+{\IFV}-show=\oldstylenums{1}\textsc{Pex}.{\IO}  brain  \\
 \glt ‘He is showing us wisdom.’ 
 \z

\ea  Tágas  bay.\\
 \gll tá-gas     baj\\ 
 \oldstylenums{3}\textsc{p}+{\IFV}-catch  {\NEG}\\
 \glt ‘They aren’t the accepting kind.’ (lit. they don’t accept) 
 \z

\ea Ehe  na,  təta  na,  kəw  na,  bəwɗere!\\
 \gll ɛhɛ    na      təta   na       kuw      na  buwɗɛrɛ\\
 here   {\PSP}    \oldstylenums{3}\textsc{p}    {\PSP}    \textsc{id}:take   {\PSP}  \textsc{id}:foolishness\\
 \glt ‘Here, what they are taking is foolishness!’ (lit. here, they, taking, foolishness)
 \z

\ea Epəle epəle  na,  wəyen  ambaɗala  slam  a  yam  avəlo.\\
 \gll {ɛpɪlɛ ɛpɪlɛ}   na   wijɛŋ   à-mbaɗ=ala   ɬam   a   jam   avʊlɔ\\ 
 {in the future}  {\PSP}  earth  \oldstylenums{3}\textsc{s}+{\PFV}-change=to    place  {\GEN}  water  above\\
 \glt ‘Someday, the earth will change into heaven (the place of water above).’ 
 \z

\ea Nde  na,  oko  ndana  anga  way?\\
 \gll ndɛ  na   ɔkʷɔ   ndana   aŋga   waj\\
 so  {\PSP}  fire  {\DEM}  {\POSS}  who\\
 \glt ‘So who are the fires [of hell] going to strike?’ (lit. so, that fire, belonging to who)
 \z
\clearpage
\ea Cəcəngehe  na,  asa  tágalay  janga  ana  endeɓ  ango,      \\
 \gll tʃɪtʃɪŋgɛhɛ    na  asa  tá-g=alaj      dzaŋga   ana    ɛndɛɓ=aŋgʷɔ\\
 now            {\PSP}    if     \oldstylenums{3}\textsc{p}+{\IFV}-do=away  reading  {\DAT}   brain={\twoS}.{\POSS}  \\
 \glt ‘Now, if they look at your life,’ (lit. now if they do a reading to your wisdom)\\
 
 \medskip
   nafta  wəyen  aməndeve  na,  Hərmbəlom  ágok  sərəya  na,\\
 \gll nafta   wijɛŋ  amɪ-ndɛv-ɛ  na  Hʊrmbʊlɔm  á-g=ɔkʷ sərija  \\ 
 day     earth    {\DEP}-finish-{\CL}   {\PSP}    God           \oldstylenums{3}\textsc{s}+{\IFV}-do={\twoS}.{\IO}   judgement  \\  
   
 \medskip
 \gll na\\
      {\PSP} \\
 \glt ‘on the day that the earth ends, [and] God judges you [and you fail of course],’\\
% % %  \z

 \medskip
  kéege  na,  memey?\\
 \gll k\'ɛ\'ɛ-g-ɛ    na  mɛmɛj\\
 {\twoS}+{\POT}-do-{\CL}  {\PSP}  how\\
 \glt ‘what will you do [as you burn]?’
 \z


\ea Nde  ehe  kəyga.\\
 \gll ndɛ  ɛhɛ   kijga\\
 so  here  {like this}\\
 \glt ‘So, it is like this here.’ 
 \z

\ea Pepenna  na  takaɗ  sla.\\
 \gll pɛpɛŋ=ŋa   na  tà-kaɗ ɬa \\
{long ago}={\ADV}  {\PSP}  \oldstylenums{3}\textsc{p}+{\PFV}-kill  cow  \\
 \glt ‘Long ago, they killed cows.’
 \z

\ea Tége  almay?\\
 \gll t\'ɛ-g-ɛ     almaj\\
 \oldstylenums{3}\textsc{p}+{\IFV}{}-do-{\CL}  what\\
 \glt ‘What were they doing?’
 \z

\ea Məze  ákosaka  jəyga  dəres. \\
\gll mɪʒɛ   á-kʷas=aka   dʒijga   dɪrɛʃ\\
 person  \oldstylenums{3}\textsc{s}+{\IFV}-unite=on  all  \textsc{id}:many\\
 \glt ‘The people were all united together.’
\z

\ea  Tápaɗay.\\ 
 \gll tá-paɗ-aj \\
 \oldstylenums{3}\textsc{p}+{\IFV}-crunch{}-{\CL}\\
 \glt ‘They ate [the meat].’
 \z

\ea Tágaka  hərnje  bay.\\
 \gll tá-g=aka   hɪrnʒɛ   baj\\
 \oldstylenums{3}\textsc{p}+{\IFV}-do=on  hate  {\NEG}\\
 \glt ‘On top of that, they divided it without hate.’ (lit. they did no hate)
 \z

\ea Nde  ehe  na,   \\
 \gll ndɛ   ɛhɛ      na \\
 so  here   {\PSP}     \\
 \glt ‘So, here,’\\
 
 \medskip
 cəcəngehe  na  məze  ahay  tandaɗay  məze  asabay  pat.\\
 \gll tʃɪtʃɪŋgɛhɛ  na   mɪʒɛ=ahaj   ta-ndaɗ-aj     mɪʒɛ  asa-baj     pat\\
 now           {\PSP}  person=Pl  \oldstylenums{3}\textsc{p}-like-{\CL}    person  again-{\NEG}  all\\
 \glt ‘[and] now, people don’t like each other at all any more.’
 \z

\ea Se  məze  amədeɗe  məze  ehe  na, cəcəngehe  na,  se  ngomna.   \\
 \gll ʃɛ    mɪʒɛ     amɪ-dɛɗ-ɛ  mɪʒɛ  ɛhɛ     na tʃɪtʃɪŋgɛhɛ  na   ʃɛ   ŋgʷɔmna\\
 only    person  {\DEP}-like-{\CL}  person  here  {\PSP}  now    {\PSP}  only  government \\
 \glt ‘The only person that likes people now is the government.’ (sarcastic)\\
 \z
 
\ea Ngomna  na,    ele  aga  kə  wəyen  aka  na,\\
 \gll ŋgʷɔmna   na  ɛlɛ   à-ga     kə  wijɛŋ   aka  na\\
 government  {\PSP}  thing  \oldstylenums{3}\textsc{s}+{\PFV}-do  on    earth  on      {\PSP}  \\
 \glt ‘The government, [if]  there is a problem (lit. a thing does) on the earth,’\\
 
 \medskip
 ndahan  na  ágas  na  təta.\\
 \gll ndahaŋ   na  á-gas        na   təta\\
 \oldstylenums{3}\textsc{s}    {\PSP}  \oldstylenums{3}\textsc{s}+{\IFV}-catch  \oldstylenums{3}\textsc{s}.{\DO}  ability\\
 \glt ‘it (the government) will be able to take care of it.’  (lit. he, he can catch it)
 \z

\ea Waya  ləme  Məloko  ahay  na,  nəmbəɗom  a  dəray  ava  na,     \\
 \gll waja  lɪmɛ  Mʊlɔkʷɔ=ahaj    na  n\`ə-mbʊɗ-ɔm      a  dəraj  ava  na\\
 because    \oldstylenums{1}\textsc{Pex}     Moloko=Pl  {\PSP} {\PFV}-change-\oldstylenums{1}\textsc{Pex}  at   head  in    {\PSP}\\
 \glt ‘Because we the Moloko, have become' (lit. changed in our head),  \\
 
 \medskip
 ka  kərkaɗaw  ahay  nə  hərgov  ahay  ga  a  ɓərzlan  ava  na,\\
 \gll ka    kərkaɗaw=ahaj  nə   hʊrgʷɔv=ahaj ga a ɓərɮaŋ     ava    na\\
 like  monkey=Pl        with  baboon=Pl  {\ADJ}  at  mountain    in    {\PSP}\\
 \glt ‘like monkeys and baboons on the mountains,’\\
 
 \medskip
  ka ala  kəra  na,  nəsərom  dəray  bay  pat.    \\ 
 \gll ka=ala   kəra      na  n\`ə-sʊr-ɔm     dəraj   baj   pat \\
 like=to  dog   {\PSP} \oldstylenums{1}+{\PFV}-know-\oldstylenums{1}\textsc{Pex}  head  {\NEG}  all\\
 \glt ‘[and] like dogs, we don’t understand anything!’ 
 \z

\ea Kə  wəyen  aka  ehe  tezl tezlezl.\\
 \gll kə   wijɛŋ   aka   ɛhɛ  {tɛɮ tɛɮɛɮ}\\
 on  earth  on  here  \textsc{id}:hollow\\
 \glt ‘[Among the people] on earth here, [we are like] the sound of a hollow cup bouncing on the ground.’ (lit. on the earth here, hollow)
 \z

\ea Nde  məze  ahay  gogor  ahay  ga  na,  ngama.\\
 \gll ndɛ  mɪʒɛ=ahaj  gʷɔgʷɔr=ahaj ga  na   ŋgama\\
 so    person=Pl  elder=Pl   {\ADJ}   {\PSP}  better\\
\glt  ‘So, our elders [have it] better.’ 
 \z

\ea {Epele epele}  na  me,  Hərmbəlom  anday  agas  ta   \\
 \gll {ɛpɛlɛ ɛpɛlɛ}   na     mɛ        Hʊrmbʊlɔm  à-ndaj       à-gas          ta\\
 {in the future}  {\PSP}  opinion  God           \oldstylenums{3}\textsc{s}+{\PFV}-{\PROG}  \oldstylenums{3}\textsc{s}+{\PFV}-catch  \oldstylenums{3}\textsc{p}.{\DO}  \\
 \glt ‘In the future in my opinion, God is going to accept them [the elders]’\\
 
 \medskip
 a  ahar  ava  re.\\
 \gll a    ahar  ava   rɛ\\
 at   hand  in      sure\\
 \glt ‘in his hands, in spite [of what the church says].’ 
 \z

\ea  Ádal  hwəsese  ga.\\
 \gll á-dal    hʷʊʃɛʃɛ  ga\\
 \oldstylenums{3}\textsc{s}+{\IFV}-surpass  small  {\ADJ}\\
 \glt ‘He is greater than the small ones.’
 \z

\ea Nde  na,  kəygehe.\\
 \gll ndɛ    na   kijgɛhɛ\\
 so    {\PSP}  {like this}\\
 \glt ‘So [it is] like this.’
\z
