\chapter[Verb phrase]{Verb phrase}\label{chap:8}
\hypertarget{RefHeading1212501525720847}{}
The verb phrase is the third of four chapters that concern the Moloko verb. \chapref{chap:6} explores the structural features of the verb root and stem. \chapref{chap:7} discusses what we have called the verb complex, which is a phonological unit consisting of the verb stem plus the pronominal affixes and enclitics, aspect/mood markings, and verbal extensions. These components are closely phonologically bound even though they may comprise from one to three phonological words. The chapter also covers derived forms. \chapref{chap:9} describes verb types and transitivity. Moloko has a flexible valence system which allows variations in the transitivity of a given verb with no morphological marking. This chapter\footnote{This chapter is adapted from \citet{FriesenMamalis2008}.} concerns the structure and functions of the verb phrase. \sectref{sec:8.1} describes the constituents of the verb phrase and their order. \sectref{sec:8.2} shows auxiliary verb constructions where two verbs form a syntactic unit. 

\section{Verb phrase constituents}\label{sec:8.1}
\hypertarget{RefHeading1212521525720847}{}
The verb phrase in Moloko is centred around the verb complex (bolded in \figref{fig:15}, cf. \chapref{chap:7}).  Other elements are all optional and occur in the order diagrammed in \figref{fig:15}. 

\begin{figure}
\frame{\resizebox{\textwidth}{!}{\begin{tabular}{llllll}
(Auxiliary) & \textbf{Verb complex} & (Noun phrase & (Adpositional phrases) & (Adverb) & (Ideophone or\\
& & or ‘body-part’)  & & & negative)
\end{tabular}}}
\caption{Moloko verb phrase constituents\label{fig:15}}\end{figure}

The auxiliary verbs include the progressive (see \sectref{sec:8.2.1}), the verb \textit{l}\textit{o} ‘go’ when used as an auxiliary (see \sectref{sec:8.2.2}), and the verb stem or ideophone\is{Ideophone|(} in its construction (see \sectref{sec:8.2.3}). 

Direct objects follow immediately after the verb complex and are expressed as noun phrases (bolded in \ref{ex:8:1} and \ref{ex:8:5}) or ‘body-part’ incorporated nouns (bolded in \ref{ex:8:2}; see \sectref{sec:9.2.5}). Adpositional phrases (underlined in \ref{ex:8:2}--\ref{ex:8:6}, see \sectref{sec:5.6}) and then adverbs (italicised in \ref{ex:8:4} and \ref{ex:8:5}; see \sectref{sec:3.5}) or ideophone (italicised in \ref{ex:8:1} and \ref{ex:8:6}; see \sectref{sec:3.6}) follow after the direct object. The verb phrase is delimited by square brackets in the examples below.

\ea \label{ex:8:1}
Həmbo  ga  [anday  asak {\textbf{ele ahan}} \textit{wəsekeke}].\\
\gll hʊmbɔ  ga  [a-ndaj    a-sak    \textbf{ɛlɛ}\textbf{=ahaŋ}  \textit{wuʃɛkɛkɛ}]\\
      flour  {\ADJ}  \oldstylenums{3}\textsc{s}-{\PRG}    \oldstylenums{3}\textsc{s}-multiply  thing=\oldstylenums{3}\textsc{s}.{\POSS} \textsc{id}:multiply\\
\glt ‘The flour was multiplying all by itself (lit. its things), \textit{wəshekeke}.’
\z

\ea \label{ex:8:2}
{}[Tandalay  talala  təzləgə   {\textbf{va}} \underline{ana Məloko ahay}].\\
\gll  {}[ta-nd=alaj    ta-l=ala  tə-ɮəg-ə   {\textbf{va}}  \ulp{ana}{~~~~~~}  \ulp{Mʊlɔkʷɔ}{~~~~~}\uline{=ahaj}]\\
      \oldstylenums{3}\textsc{p}-{\PRG}=away   \oldstylenums{3}\textsc{p}-go=to  \oldstylenums{3}\textsc{p}-throw-{\CL} body  {\DAT} Moloko=Pl\\
\glt  ‘They were coming and fighting with the Molokos.’ (lit. they were coming they threw body to Molokos)
\z

\ea \label{ex:8:3}
{}[Enjé  \underline{kə  delmete  aka}  \underline{a  slam  enen}].\\
\gll  {}[ɛ-ndʒ{}-ɛ    \ulp{kə}{~~~}  \ulp{dɛlmɛtɛ}{~~~~}  \ule{aka} \ulp{a}{~~~}  \ulp{ɬam}{~~~~}  \uline{ɛnɛŋ}]\\
      \oldstylenums{3}\textsc{s}-leave-{\CL}    on  neighbor    on  at  place  another\\
\glt  ‘He left to go to his neighbor at some other place.’  
\z

\ea \label{ex:8:4}
{}[Názaɗ  \underline{a  dəray  ava}  \textit{sawan}].\\
\gll  {}[ná-zaɗ    \ulp{a}{~~~}  \ulp{dəraj}{~~~}  \ule{ava}   \textit{sawaŋ}]\\
      {\oneS}+{\IFV}-carry    at  head  in  {without help}\\
\glt  ‘I can carry it on my head myself!’
\z

\ea \label{ex:8:5}
{}[Nəvəlan  \textbf{yam}  \underline{ana  Mana}  \textit{zayəhha}\textbf{]}.\\
\gll \textup{[}n\'{ə}-vəl=aŋ    \textbf{jam}  \ulp{ana}{~~~}  \ule{Mana}  \textit{zajəx=xa}\textup{]}\\
{\oneS}+{\PFV}-give=\oldstylenums{3}\textsc{s}.{\IO}  water  {\DAT} Mana  care={\ADV}\\
\glt ‘I gave water to Mana carefully.’
\z

\ea \label{ex:8:6}
{}[Azləgalay \underline{a  vəlo} \textit{zor}].\\
\gll {}[a-ɮəg =alaj \ulp{a}{~~~}  \ule{vʊlɔ} \textit{zɔr}]\\
\oldstylenums{3}\textsc{s}-throw =away    at  above  \textsc{id}:throwing\\
\glt ‘She threw [the pestle] up high, \textit{zor}.’
\z

\citet[69]{Radford1981} gives diagnostic criteria for determining whether a given string of words is a sentence constituent or not. Following these criteria, all of the above elements are part of the verb phrase as a constituent of the clause. The elements of the verb phrase behave distributionally as a single structural unit that does not permit intrusion of parenthetical elements internally, but rather only at the boundaries. For Moloko, the distribution of adverbs, emphatic interrogative pronouns, ideophones, the Perfect enclitic, and the manner of fronting all attest to the unity of the verb phrase as described above. Only the presupposition marker can intrude into the verb phrase, and only in a particular construction. Each of these factors is discussed below. 

Some temporal adverbs (bolded in \ref{ex:8:7}--\ref{ex:8:8}) can occur first in the clause or last in the verb phrase , but not in the interior of the verb phrase. Likewise, emphatic interrogative pronouns (bolded in \ref{ex:8:9}) occur first or last in the clause (see \sectref{sec:10.3.5}). 

\ea \label{ex:8:7}
\textbf{Egəne}  [nólo  a  kosoko  ava].\\
\gll  \textbf{ɛgɪnɛ}  [n\'{ɔ}-lɔ     a  kɔsɔkʷɔ  ava]\\
      today  {\oneS}+{\IFV}-go  at  market  in\\
\glt  ‘Today I will go to the market.'
\z

\ea \label{ex:8:8}
{}[Nólo  a  kosoko  ava  \textbf{egəne}].\\
\gll  {}[n\'ɔ-lɔ   a  kɔsɔkʷɔ  ava  \textbf{ɛgɪnɛ}]\\
      {\oneS}+{\IFV}-go  at  market  in  today\\
\glt  ‘I will go to the market today.'
\z

\ea \label{ex:8:9}
\textbf{Wa}  [amazaw  ala  agwazla  ana  ne  na]  \textbf{way}?\\
\gll  \textbf{wa}      [ama-z=aw  =ala       agʷaɮa     ana     nɛ    na]     \textbf{waj}\\
      who   {\DEP}-take={\oneS}.{\IO}  =to   {spp. of tree}     {\DAT}    {\oneS}    {\PSP}    who\\
\glt  ‘Who can I find to bring me this tree?’
\z

Ideophones have only three slots within the clause: First in the clause \REF{ex:8:10},\footnote{Note that an ideophone that is first in the clause is sometimes delimited by \textit{na} \REF{ex:8:19}.} first in verb phrase (\ref{ex:8:11}, see \sectref{sec:8.2.3}),\footnote{When the ideophone is first in the verb phrase it necessitates the nominalised form of the verb \sectref{sec:8.2.3}.} last in verb phrase \REF{ex:8:12}. The ideophones are italicised in the examples.

\ea \label{ex:8:10}\corpussource{Snake, S. 13}\\
\textit{Kal}\textit{ə}\textit{w}  [nazala   ezlere  əwla].\\
\gll  \textit{kaluw}       [nà-z=ala    ɛɮɛrɛ=uwla]\\
      {\textsc{id}:take quickly}    {\oneS}+{\PFV}-take=to   spear={\oneS}.{\POSS}\\
\glt  ‘I quickly took my spear.’
\z

\clearpage
\ea \label{ex:8:11}\corpussource{Cicada, S. 15}\\
Ndahan  [\textit{gədok}   mədəye  gəzom].\\
\gll  ndahaŋ  [\textit{gʊdɔkʷ}     mɪ-d-ijɛ       gʊzɔm ]\\
      \oldstylenums{3}\textsc{s}        {\textsc{id}:prepare beer}  {\NOM}{}-prepare-{\CL}   beer\\
\glt  ‘He \textit{gədok} made millet beer.’
\z

\ea \label{ex:8:12}\corpussource{Snake, S. 5}\\
{}[Acar  a  hay  kəre  ava  \textit{fo fo fo}].\\
\gll  {}[à-tsar              a  haj        kɪrɛ       ava  \textit{fɔ fɔ fɔ}]\\
      \oldstylenums{3}\textsc{s}+{\PFV}-climb  at   house  beams  in          \textsc{id}:sound of snake\\
\glt  ‘[The snake] climbed into the beams in the roof \textit{fo fo fo}.’
\z

The distribution and influence of the Perfect enclitic \textit{=va}  also attests to the unity of the post-verbal elements in the verb phrase. The Perfect enclitic \textit{=va}  (bolded in \ref{ex:8:13}--\ref{ex:8:16}), can either cliticise to the end of the verb complex \REF{ex:8:13} or the end of the entire verb phrase (\ref{ex:8:14}--\ref{ex:8:16}). The phonological influence of the Perfect\is{Tense, mood, and aspect!Perfect} extends across the entire verb phrase since its presence in either post-verbal or phrase-final position causes a neutralisation of the prosody on the verb stem (see \sectref{sec:7.5.3}).

\ea \label{ex:8:13}\corpussource{Values, S. 6}\\
{}[Tahata  na \textbf{va}  kə  deftere  aka].\\
\gll  {}[tà-h=ata    na\textbf{=va}    kə  dɛftɛrɛ  aka]\\
      \oldstylenums{3}\textsc{p}+{\PFV}-tell=\oldstylenums{3}\textsc{p}.{\IO}  \oldstylenums{3}\textsc{s}.{\DO}={\PRF}    on  book  on\\
\glt  ‘They have already told them in the book.’ 
\z


\ea \label{ex:8:14}\corpussource{Disobedient Girl, S. 34}\\
Waya  ndana  Hərmbəlom  [ázata  aka  {barka  ahan} \textbf{va}].\\
\gll  waja  ndana  Hʊrmbʊlɔm   [á-z=ata=aka   {barka}{=ahaŋ}\textbf{=va}\\     
      because    {\DEM}   God             \oldstylenums{3}\textsc{s}+{\IFV}-take=\oldstylenums{3}\textsc{p}.{\IO}=on   blessing=\oldstylenums{3}\textsc{s}.{\POSS}={\PRF}\\  
\glt  ‘Because of that, God had taken back his blessing from them.’
\z

\ea \label{ex:8:15}
Baba  ango  [avəlata  {nok} \textbf{va}  {a  ahar  ata  ava}].\\
\gll  baba=aŋgʷɔ  [a-vəl=ata    {nɔkʷ}  \textbf{=va} {a}  {ahar}{=atəta}  {ava}]\\
      father={\twoS}.{\POSS}  \oldstylenums{3}\textsc{s}-give=\oldstylenums{3}\textsc{p}.{\IO}    {\twoS}  ={\PRF}  at  hand=\oldstylenums{3}\textsc{p}.{\POSS}  in\\
\glt  ‘Your father gave you into their hands [to be a wife for one of them].’
\z

\newpage
\ea \label{ex:8:16}
Nde  hor  na,  [asərkala  {afa  təta}  \textbf{va}]. . .\\
\gll  ndɛ    hʷɔr  na      [a-sərk=ala  {afa}           {təta}     =\textbf{va}]\\
      so    woman  {\PSP}    \oldstylenums{3}\textsc{s}-habitually=to   {at house of}  \oldstylenums{3}\textsc{p}.{\POSS}   ={\PRF}\\
\glt  ‘Now, that woman, she was in the habit at their house of . . .’ 
\z

Only certain elements in the verb phrase can be fronted in the clause and marked with the presupposition marker \textit{na} (see \sectref{sec:11.2} for the explanation of this construction). The fact that some elements cannot be fronted indicates that they are closely bound to the verb phrase structure. These elements include the ‘body-part’ incorporated noun (cf. \sectref{sec:9.3}),  the negative (see \sectref{sec:10.2.1}), and adverbs which are bound to the negative. Some of these elements are from the far right edge of the verb phrase. The elements that may be fronted are underlined in (\ref{ex:8:17}--\ref{ex:8:19}) and include direct object and oblique \REF{ex:8:17}, derived adverb (\ref{ex:8:18}), indirect object and ideophone\is{Ideophone|)} \REF{ex:8:19}. 


\ea \label{ex:8:17}\corpussource{Values, S. 13}\\
\underline{A  məsəyon  ava  na}  \underline{ele  ahay  aməwəsle  na},  [tége  bay].\\
\gll  \ulp{a}{~~~~~}  \ulp{mʊsijɔŋ}{~~~~~}   \ulp{ava}{~~}   \ule{na}  \ulp{ɛlɛ}{~~~~~~}\ulp{=ahaj}{~~~~~~}   \ulp{amɪ-wuɬ{}-ɛ}{~~~~~~~}     \ule{na}     [t\'{ɛ}-g-ɛ     baj]\\
      at  mission  in  {\PSP}  thing=Pl   {\DEP}-forbid-{\CL}  {\PSP}    \oldstylenums{3}\textsc{p}+{\IFV}-do-{\CL}  {\NEG}\\
\glt  ‘In the church, these things that they have forbidden, they don’t do.’
\z


\ea \label{ex:8:18}\corpussource{Values, S. 39}\\
\underline{Pepenna  na},  [takaɗ  sla].\\
\gll  \ulp{pɛpɛŋ}{~~~~~~}   \ulp{=ŋa}{~~~~}  \ule{na}    [tà-kaɗ     ɬa]\\
      {long ago} ={\ADV}  {\PSP}    \oldstylenums{3}\textsc{p}+{\PFV}-kill  cow \\
\glt  ‘Long ago, they killed cows.’
\z


\ea \label{ex:8:19}\corpussource{Values, S. 3}\\
\uline{Səwat  na},  \uline{təta  a  məsəyon  na  ava  nəndəye  na},\\ 
\gll  \ulp{suwat}{~~~~~~~~~}   \ule{na}   \ulp{təta}{~~~~~~~~}   \ule{a}   \ulp{mʊsijɔŋ}{~~~~}   \ulp{na}{~~~}  \ulp{ava}{~~~~}     \ulp{nɪndijɛ}{~~~~~}  \ule{na}\\
      \textsc{id}:disperse  {\PSP}  \oldstylenums{3}\textsc{p} at  mission  {\PSP}    in  {\DEM}  {\PSP}\\
\glt  ‘As the people go home from church,' (lit. disperse, they in the mission there),’\\
\medskip
[pester  áhata],   “Ey, ele  nehe  na,  kógom  bay!”\\
\gll  {}[{pɛʃtɛr} {á-h}{=ata}] {ɛj} {ɛlɛ} {nɛhɛ} {na} {k\'{ɔ}-gʷ{}-ɔm} {baj}\\
      pastor  \oldstylenums{3}\textsc{s}+{\IFV}-tell=\oldstylenums{3}\textsc{p}.{\IO}   hey    thing  {\DEM}  {\PSP}  \oldstylenums{2}+{\IFV}-do-{\twoP}    {\NEG}\\
\glt  ‘the Pastor told them, “Hey! These things, don’t do them!”’
\z


The only construction where an external element can appear to break up the verb phrase is the focus construction (\ref{ex:8:20}--\ref{ex:8:22}) (see \sectref{sec:11.5}) where the presupposition marker \textit{na} can appear to break up parts of the verb phrase. However the structural unity of the verb phrase unit is not challenged since \textit{na} can occur only once within the verb phrase in this construction and only in one position -- immediately before the final focussed element. \textit{Na} does not intrude into any other position. In each of (\ref{ex:8:20}--\ref{ex:8:22}), the penultimate placing of \textit{na} (bolded) functions to make the final element of the verb phrase more prominent. In each example, only the verb phrase containing \textit{na }is delimited by square brackets and the part delimited by\textit{ na}  is underlined. In \REF{ex:8:20},\textit{ na }occurs in the adverbial clause just before the verb phrase-final adverb \textit{gam} ‘much,’ making prominent the fact that the woman was going to grind \textit{a lot} of millet.  In \REF{ex:8:21}, \textit{na} occurs in the matrix clause just before the adpositional phrase (\textit{ka  mahay  ango  aka} ‘by your door’), highlighting the desire to have the tree by the chief's door. In (\ref{ex:8:22}) the final element of the verb phrase \textit{anga way} ‘belonging to whom' is highlighted in the rhetorical question which focussed on the fact that the people obeyed neither God's word nor that of the elders.  


\ea \label{ex:8:20}\corpussource{Disobedient Girl, S. 17}\\
Azləna,  hor  \textbf{na},  asərkala  afa  təta  va  \textbf{na},\\  
\gll  aɮəna  hʷɔr  \textbf{na}  [à-sərk=ala   afa            təta=va  \textbf{na}]\\
      but  woman  {\PSP}    \oldstylenums{3}\textsc{s}+{\PFV}-habitually=to    {at place of}  \oldstylenums{3}\textsc{p}={\PRF}  {\PSP}\\
\glt  ‘Now, that woman, she was in the habit at their place’\\
\medskip
[\underline{aməhaya  háy  \textbf{na}}  gam].\\
\gll  {}[\ulp{amə-h=aja}{~~~~~~~~~} \ulp{haj}{~~~~~} \ule{\textbf{na}} {gam}]\\
      \db {\DEP}-grind={\PLU}   millet  {\PSP}   {a lot}\\
\glt  ‘[of] grinding a lot of millet.’
\z

\ea \label{ex:8:21}\corpussource{Cicada, S. 7}\\
Mama  agwazla  ava  a  ləhe  \textbf{na},  malan  ga  \textbf{na}, \\ 
\gll  mama   agʷaɮa       ava       a   lɪhɛ        \textbf{na}    malaŋ   ga  \textbf{na}\\
      mother   {spp. of tree}  {\EXT}   at   bush    {\PSP}  large     {\ADJ}  {\PSP}\\
\glt  ‘There is a mother-tree in the bush,  a big one,’\\
\medskip
[\underline{agasaka  \textbf{na}}  ka  mahay  ango  aka]  aməmbese.\\
\gll  {}[\ulp{à-gas=aka}{~~~~~~~} \ule{\textbf{na}} {ka} {mahaj}{=aŋgʷɔ} {aka}] {àmɪ{}-mbɛʃ{}-ɛ}\\
      \oldstylenums{3}\textsc{s}+{\PFV}-get=on   {\PSP}   on     door={\twoS}.{\POSS}      on   {\DEP}-rest-{\CL}\\
\glt  ‘[and] it would please you to have that tree at your door, [so that you could] rest [under it].’
\z

\ea \label{ex:8:22}\corpussource{Values, S. 29} \\
Hərmbəlom  \textbf{na},  amaɗaslava  ala  məze  \textbf{na},  ndahan  ese  \textbf{na},\\  
\gll  Hʊrmbʊlɔm   \textbf{na}  ama-ɗaɬ=ava=ala  mɪʒɛ  \textbf{na}   ndahaŋ  ɛʃɛ  \textbf{na}\\
      God      {\PSP}    {\DEP}-multiply=in=to   person   {\PSP}  \oldstylenums{3}\textsc{s}     again    {\PSP}  \\
\glt ‘God, the one who mltiplied the people, him again,’ \\     
      
\medskip
{}[\underline{kagas  ma  Hərmbəlom  na}  asabay]  \textbf{na},\\  
\gll  {}[\ulp{ka-gas}{~~~~~}    \ulp{ma}{~~~~}   \ulp{Hʊrmbʊlɔm}{~~~} \ule{\textbf{na}}  asa-baj] \textbf{{na}}\\
      {\twoS}-catch   word     God  {\PSP}  again-{\NEG}  {\PSP}\\
\glt ‘[if] you no longer accept the word of God,’\\      
      
\medskip
{}[\underline{káagas  \textbf{na}}  anga  way]?\\
\gll  {}[\ulp{káá-gas}{~~~~~~~~} \ule{\textbf{na}} {anga} {waj}]\\
      {\twoS}+{\POT}-catch    {\PSP}   {\POSS}    who\\
\glt 'You won't listen to anyone.' (lit. ‘whose [word] will you accept?’) \\     
\z

\section{Auxiliary verb constructions}\label{sec:8.2}
\hypertarget{RefHeading1212541525720847}{}
In an auxiliary verb construction in Moloko, two verbs (or a verb plus an ideophone) form a syntactic unit and, consequently, have the same subject. The second verb is the main verb in the construction. Together the two verbs comprise the head of just one clause, with only one set of core participants and obliques that semantically are related to the second (main) verb. 

This section presents three auxiliary verb constructions. In the first two constructions, both main and auxiliary verbs are inflected.  These constructions express progressive aspect (\sectref{sec:8.2.1}) and movement from one place to another (\sectref{sec:8.2.2}).  The third construction consists of a verb stem or ideophone plus the main verb which is in the nominalised form (\sectref{sec:8.2.3}). We consider this third construction to be an auxiliary construction even though the verb stem/ideophone does not carry much of the inflectional information normally associated with auxiliaries (stems and ideophones carry neither subject and object agreement nor aspect and mode marking).\footnote{These criteria for verb auxiliaries are given by \citet[84]{Payne1997}.} However, the verb stem/ideophone\is{Ideophone} construction demonstrates the same structure as the progressive and movement auxiliary constructions and the stem/ideophone functions as an auxiliary in that it adds grammatical information to the main verb.  

\subsection{Progressive auxiliary}\label{sec:8.2.1}
\hypertarget{RefHeading1212561525720847}{}
\citet{FriesenMamalis2008} found that the progressive\is{Tense, mood, and aspect!Progressive|(} expresses the idea of an action in progress, an event that doesn’t take place all at once.\footnote{Note that the verb \textit{nday}  can occur alone as the main verb of a clause \sectref{sec:9.2.1}. When it does, the complement expresses the location of the subject. For example, \textit{Hawa anday a mogom} ‘Hawa is at home’.}  It is formed with \textit{nday} ‘to be’ (see \sectref{sec:9.2.1}) plus the main verb (\ref{ex:8:23}--\ref{ex:8:25}).  The auxiliary \textit{nday} occurs as the first of two verbs in a verb phrase. The main verb takes all subject affixes and also any inflections or obliques. In the examples, the progressive is bolded and the verb phrase is delimited by square brackets.

\ea \label{ex:8:23}
Mala  [\textbf{anday} ége  slərele].\\
\gll  Mala [\textbf{a-ndaj}  \'{ɛ}{}-g-ɛ  ɬɪrɛlɛ]\\
      Mala   \oldstylenums{3}\textsc{s}-{\PRG}    \oldstylenums{3}\textsc{s}+{\IFV}-do-{\CL}  work\\
\glt  ‘Mala is working (in the process of doing work).’
\z

\ea \label{ex:8:24}
Mana [\textbf{anday} ólo  a  kosoko  ava].\\
\gll Mana [\textbf{a-ndaj} \'{ɔ}{}-lɔ a kɔsɔkʷɔ ava]\\
Mana  \oldstylenums{3}\textsc{s}-{\PRG}    \oldstylenums{3}\textsc{s}+{\IFV}{}-go  at  market  in\\
\glt ‘Mana is going to the market.’  (lit. he is currently at going to the market)
\z

\ea \label{ex:8:25} 
Apazan  nanjakay  nok, [\textbf{kanday} kəhaya  háy].\\
\gll apazaŋ nà-nzak-aj nɔkʷ [\textbf{ka-ndaj} k\`{ə}-h=aja haj]\\
yesterday  {\oneS}+{\PFV}-find{}-{\CL}  {\twoS}  {\twoS}-{\PRG}    {\twoS}+{\PFV}-grind={\PLU}  millet\\
\glt ‘Yesterday when I found you, you were grinding millet.’  
\z

Both of the verbs are marked for subject. In plural forms that take subject prefix and suffix ({\oneP} and {\twoP}, \ref{ex:8:26} and \ref{ex:8:27}), \textit{nday} takes subject prefixes only.\footnote{Some Moloko say that the plural form is \textit{nondomoy}, but most use the reduced form. } 

\ea \label{ex:8:26}
{}[\textbf{Nondoy} nombosom  va].\\
\gll  {}[\textbf{nɔ-ndɔj}  nɔ-mbɔs-ɔm   va]\\
      {\oneS}-{\PRG}    \oneP-rest-\oldstylenums{1}\textsc{pex}  body\\
\glt  ‘We are resting.’ 
\z

\ea \label{ex:8:27}
{}[\textbf{Nondoy}  nodorom amsoko].\\
\gll  {}[\textbf{nɔ-ndɔj}     nɔ-dɔr-ɔm amsɔkʷɔ]\\
      {\oneS}-{\PRG}    \oneP-plant-\oldstylenums{1}\textsc{Pex}  {dry season millet}\\
\glt  ‘We (exclusive) are planting dry season millet.’
\z

The progressive auxiliary does not co-occur with the Perfect enclitic (see \sectref{sec:7.5.3}), nor does the iterative reduplicative construction (see \sectref{sec:7.4.5}) combine with the progressive auxiliary.

In discourse, progressive aspect is used to mark an event that is in progress in a Moloko text. It is not necessarily in the background, but indicates durative or ongoing dynamic events. In the Cicada setting \REF{ex:8:28}, there is a progressive in a tail-head link (see \sectref{sec:11.1.3}) showing what the young men were doing when they found the tree. 

\ea \label{ex:8:28}\corpussource{Cicada, S. 3-5}\\
Albaya  ahay  aba.\\   
\gll  albaja=ahaj    aba\\
      {young man}=Pl     {\EXT}\\
\glt  ‘There were some young men.’\\
\medskip
\textbf{Tánday} t\'{ə}talay  a  ləhe. \\ 
\gll  \textbf{tá-ndaj} t\'{ə}-tal-aj a lɪhɛ\\
      \oldstylenums{3}\textsc{p}+{\IFV}-{\PRG} \oldstylenums{3}\textsc{p}+{\IFV}-walk{}-{\CL}  at    bush\\
\glt  ‘They were walking in the bush.’\\
\medskip
\textbf{Tánday}  t\'{ə}talay  a  ləhe  na, \\
\gll  \textbf{tá-ndaj} t\'{ə}-tal-aj a lɪhɛ na\\
      \oldstylenums{3}\textsc{p}+{\IFV}-{\PRG}   \oldstylenums{3}\textsc{p}+{\IFV}-walk{}-{\CL}  at   bush   {\PSP}\\
\glt  ‘[As] they were walking in the bush,’\\
\medskip
tolo  tənjakay  agwazla  malan  ga  a  ləhe.\\
\gll  t\`{ə}-lɔ t\`{ə}-nzak-aj agʷaɮa malaŋ ga a lɪhɛ\\
      \oldstylenums{3}\textsc{p}+{\PFV}-go  \oldstylenums{3}\textsc{p}+{\PFV}-find{}-{\CL}    {spp. of tree}      large   {\ADJ}    at    bush\\
\glt  ‘they went and found a large tree (a particular species) in the bush.’
\z

Also, progressives are used in expository texts that give the ongoing state of the world and show reasons for the way things are. Example \REF{ex:8:29} from the Disobedient Girl story shows the entire reported speech when the husband explains to his wife the way things work for the Moloko. For most of the explanation, the verbs are Imperfective (see \sectref{sec:7.4.2}). However, the reason that the millet multiplied -- namely, that God used to multiply millet for the Moloko -- is given in the final line of his speech. The verb form for the reason is progressive (bolded in the example). Here, the progressive is marking an important ongoing event. 

\clearpage
\ea \label{ex:8:29}\corpussource{Disobedient Girl, S. 13}\\
Awəy,\\      
  awij\\
\glt  ‘He said,’\\
\medskip
“Hor  golo,  afa  ləme  na,  mənjəye  aləme  na,  kəyga  ehe:\\
\gll  hʷɔr gʷɔlɔ afa lɪmɛ na mɪ-nʒ-ijɛ=alɪmɛ na kijga\\ 
      woman  \textsc{hon}   {at place}   \oldstylenums{1}\textsc{Pex}   {\PSP} {\NOM}{}-sit-{\CL}=\oldstylenums{1}\textsc{Pex}.{\POSS}  {\PSP}   {like this} \\ 
      
      \medskip
\gll ɛhɛ\\
     here\\
\glt  ‘“My dear wife, here at our (exclusive) place, it is like this:’\\
\medskip
asa  asok  aməhaya  na,\\
\gll  asa à-s=ɔkʷ amə-h=aja na\\
      if   \oldstylenums{3}\textsc{s}+{\PFV}-please={\twoS}.{\IO}   {\DEP}-grind={\PLU}   {\PSP}\\
\glt ‘If you want to grind,’
      
\medskip
kázaɗ  war  elé  a  háy  bəlen.\\  
\gll  ká-zaɗ war ɛlɛ a haj  bɪlɛŋ\\
      {\twoS}+{\IFV}-take    child    eye  {\GEN}  millet  one\\
\glt  ‘you take only one grain.’  \\

\medskip
War  elé  háy  bəlen  ga  nəndəye  nok  amezəɗe  na,\\  
\gll  war ɛlɛ haj bɪlɛŋ ga nɪndijɛ nɔkʷ amɛ-ʒɪɗ{}-ɛ na\\
      child    eye  millet  one  {\ADJ}  {\DEM}    {\twoS}    {\DEP}-take-{\CL}  {\PSP} \\ 
\glt ‘That one grain that you have taken,’\\      
      
\medskip
káhaya  na  kə  ver  aka.\\   
\gll  ká-h=aja na kə vɛr aka\\
      {\twoS}+{\IFV}-grind={\PLU} \oldstylenums{3}\textsc{s}.{\DO} on {grinding stone} on\\
\glt  ‘grind it on the grinding stone.’\\

\medskip
Ánjaloko  de  pew.\\ 
\gll  á-nz=alɔkʷɔ dɛ pɛw \\
      \oldstylenums{3}\textsc{s}+{\IFV}-suffice=\oldstylenums{1}\textsc{Pin}.{\IO}  just   enough\\
\glt  ‘It will suffice for all of us just enough.’\\

\medskip
Ádaloko  ha  ámbaɗ  ese.\\ 
\gll  á-d=alɔkʷɔ  ha á-mbaɗ ɛʃɛ\\
      \oldstylenums{3}\textsc{s}+{\IFV}-prepare=\oldstylenums{1}\textsc{Pin}.{\IO}  until  {\oldstylenums{3}\textsc{s}+{\IFV}-left over}  again\\
\glt  ‘It will make food for all of us, until there is some left over.’\\

\medskip
 Waya  a  məhaya  ahan  ava  na,\\  
\gll  waja a mə-h=aja=ahaŋ ava na\\
      because  at   {\NOM}{}-grind={\PLU}=\oldstylenums{3}\textsc{s}.{\POSS}   in   {\PSP}\\
\glt  ‘Because, while you grind (lit. because in its grinding),\\

\medskip
\largerpage
Hərmbəlom  \textbf{anday}   ásakaləme  na  aka.”\\
\gll  Hʊrmbʊlɔm \textbf{a-ndaj}  á-sak=alɪmɛ na aka\\
      God            \oldstylenums{3}\textsc{s}-{\PROG}   \oldstylenums{3}\textsc{s}+{\IFV}-multiply=\oldstylenums{1}\textsc{Pex}.{\IO}   \oldstylenums{3}\textsc{s}.{\DO}   on\\
\glt  ‘God is multiplying it for us.”’
\z

Progressives are also found in the peak section\is{Focus and prominence!Discourse peak} of a narrative where they function to slow down the events and draw the reader into the action. Example \REF{ex:8:30} shows the entire peak section of the Disobedient Girl. In the story (shown in its entirety in \sectref{sec:1.5}), there is a battle between the disobedient girl and the millet itself. The millet has a supernatural ability to expand, and eventually triumphs over the girl. Verbs in the progressive form (bolded in the example) mark the ongoing multiplication of the millet (S. 23) while the girl is grinding as well as the girl’s ongoing grinding (S. 25) while the room is filling up with flour. 


\ea \label{ex:8:30}\corpussource{Disobedient Girl,  S. 20}\\
Jo  madala  háy  na,  gam.\\     
\gll  dzɔ ma-d=ala haj na gam\\
      {\textsc{id}:take}    {\NOM}-prepare=to    millet   {\PSP}   {a lot}\\
\glt  ‘[She] prepared lots of millet.’\\
\medskip
\corpussource{S. 21}\\
Ndahan  bah  məbehe  háy  ahan \\
\gll  ndahaŋ bax mɪ{}-bɛh-ɛ haj=ahaŋ\\
      \oldstylenums{3}\textsc{s} pour {\NOM}{}-pour-{\CL}  millet=\oldstylenums{3}\textsc{s}.{\POSS} \\
\glt  ‘She poured her millet’\\
\medskip
amadala  na  kə  ver  aka  azla.\\
\gll   ama-d=ala na kə vɛr aka aɮa\\
      {\DEP}-prepare=to   \oldstylenums{3}\textsc{s}.{\DO}   on   stone  on      now\\
\glt  ‘to prepare it on the grinding stone.’\\

\medskip
\corpussource{S. 22}\\
Njəw  njəw  njəw aməhaya  azla.\\  
\gll  {nzuw  nzuw  nzuw} amə-h=aja aɮa\\
      {\textsc{id}:grind}     {\DEP}-grind={\PLU}    now\\
\glt  ‘\textit{Njəw  njəw  njəw} [she]  ground [the millet] now.’ \\

\medskip
\corpussource{S. 23}\\
Həmbo  na  ɗəw  \textbf{anday}  \textbf{ásak  ásak  ásak.}\\
\gll  hʊmbɔ na ɗuw \textbf{a-ndaj} \textbf{á-sak}  \textbf{á-sak}  \textbf{á-sak}\\
      flour  {\PSP}  also  \oldstylenums{3}\textsc{s}-{\PRG} {\oldstylenums{3}\textsc{s}+{\IFV}-multiply} {\oldstylenums{3}\textsc{s}+{\IFV}-multiply} {\oldstylenums{3}\textsc{s}+{\IFV}-multiply}\\
\glt  ‘The flour, it was multiplying [and] multiplying [and] multiplying.\\

\medskip

\corpussource{S. 24}\\
 Ndahan  na,  ndahan  aka  njəw  njəw  njəw.\\   
\gll  ndahaŋ  na ndahaŋ aka {nzuw  nzuw  nzuw}\\
      \oldstylenums{3}\textsc{s}  {\PSP}  \oldstylenums{3}\textsc{s}   {\EXT}+on  {\textsc{id}:grind}\\
\glt  ‘And she, she is grinding some more \textit{njəw  njəw  njəw}.’ \\
\medskip
\corpussource{S. 25}\\
\textbf{Anday}  \textbf{ahaya}  nə  məzere  ləmes  ga.\\ 
\gll  \textbf{à-ndaj} \textbf{à-h=aja} nə mɪ{}-ʒɛr-ɛ lɪmɛʃ ga\\
      \oldstylenums{3}\textsc{s}+{\PFV}-{\PRG}  \oldstylenums{3}\textsc{s}+{\PFV}-grind={\PLU}  with   {\NOM}{}-{do well}-{\CL}  song     {\ADJ}\\
\glt  ‘She is grinding while singing well.’\\

\medskip
\corpussource{S. 26}\\
Alala  na,  ver  na  árah  mbaf,  nə  həmbo  na, \\  
\gll  a-l=ala na vɛr na á-rəx  mbaf nə  hʊmbɔ   na\\
      \oldstylenums{3}\textsc{s}-go=to   {\PSP}  room     {\PSP}  \oldstylenums{3}\textsc{s}+{\IFV}-fill  {\textsc{id}:up to the roof}  with  flour  {\PSP}\\
\glt  ‘After a while, the room, it filled up to the roof with the flour.’\\

\medskip
ɗək  məɗəkaka  alay  ana  hor  na,\\ 
\gll  ɗək mə-ɗək=aka=alaj ana hʷɔr na\\
      plug    {\NOM}{}-plug=on=away     {\DAT}  woman  {\PSP}\\
\glt  ‘[The flour] suffocated the woman (lit. plugged [the room] for the woman [so there was no place for her to even breathe].’ \\

\medskip
nata  ndahan  dəɓəsolək  məmətava  alay  a  hoɗ  a  hay  na  ava.\\
\gll  nata ndahaŋ dʊɓʊsɔlʊk mə-mət=ava=alaj a hʷɔɗ a \\ 
      {and then}  \oldstylenums{3}\textsc{s}   {\textsc{id}:collapse/die}  {\NOM}-die=in=away   at  stomach {\GEN}   \\  

\medskip     
\gll haj  na ava\\
    house  {\PSP}   in\\ 
\glt  ‘And she collapsed \textit{dəɓəsolək}, dying inside the house.’\\
\z
\is{Tense, mood, and aspect!Progressive|)}
\subsection{Movement auxiliary}\label{sec:8.2.2}
\hypertarget{RefHeading1212581525720847}{}
The verb \textit{lo} ‘go’ is often found together with a second verb within the same verb phrase to express the idea of movement from one place to another, in order to accomplish the event expressed by the main verb \citep{FriesenMamalis2008}. In (\ref{ex:8:31}--\ref{ex:8:34}), both verbs are conjugated, but only the second takes extensions or other verb phrase elements. In the examples, the verb \textit{lo} is bolded and the verb phrase is delimited by square brackets. 

\ea \label{ex:8:31}\corpussource{Cicada, S. 5}\\
{}[\textbf{Tə-lo} tənjakay  agwazla  malan  ga  a  ləhe].\\
\gll  {}[\textbf{t\`{ə}-lɔ}          t\`{ə}-nzak-aj        agʷaɮa      malaŋ   ga      a    lɪhɛ]\\
      \oldstylenums{3}\textsc{p}+{\PFV}-go   \oldstylenums{3}\textsc{p}+{\PFV}-find{}-{\CL}   {spp. of tree}    large   {\ADJ}  to   bush\\
\glt  ‘They went and found a large tree (a particular species) in the bush.’
\z

\ea \label{ex:8:32}\corpussource{Values, S. 18}\\\relax
{}[\textbf{\'{O}lo}  ában  ana  baba  ahan].\\
\gll  {}[\textbf{\'{ɔ}-lɔ}   á-b=aŋ     ana   baba=ahaŋ]\\
      \oldstylenums{3}\textsc{s}+{\IFV}-go  \oldstylenums{3}\textsc{s}+{\IFV}-hit=\oldstylenums{3}\textsc{s}.{\IO}  {\DAT} father=\oldstylenums{3}\textsc{s}.{\POSS}\\
\glt  ‘He goes and hits his father.’
\z

\ea \label{ex:8:33}\corpussource{Values, S. 19}\\
{}[\textbf{\'{O}lo}  ápaɗay  məze  nə  madan].\\
\gll  {}[\textbf{\'{ɔ}-lɔ}   á-paɗ{}-aj     mɪʒɛ   nə   madaŋ]\\
      \oldstylenums{3}\textsc{s}+{\IFV}-go  \oldstylenums{3}\textsc{s}+{\IFV}-crunch{}-{\CL}  person  with  magic\\
\glt  ‘He goes and eats someone with sorcery.’
\z

\ea \label{ex:8:34}
{}[\textbf{Lohom}  komənjɔrom  na  ala  gəvah  na].\\
\gll  {}[\textbf{lɔh-ɔm}    kɔ-mʊnzɔr-ɔm   na=ala      gəvax  na]\\
      go[{\IMP}]-{\twoP}   {\twoP}-see-{\twoP}  \oldstylenums{3}\textsc{s}.{\DO}=to  field  {\PSP}\\
\glt  ‘Go [and] you will see that field.' 
\z

\subsection{Stem plus ideophone auxiliary}\label{sec:8.2.3}\is{Focus and prominence!Stem plus ideophone auxiliary}\is{Ideophone|(}
\hypertarget{RefHeading1212601525720847}{}
\largerpage \citet{FriesenMamalis2008} discovered that pivotal events at the high points in a narrative\is{Focus and prominence!Discourse peak} may be coded with a particular verb phrase construction in which an ideophone or the uninflected stem form of a verb is followed by the main verb in its nominalised form (\ref{ex:8:35}--\ref{ex:8:37}, see \sectref{sec:7.6.2}).  In the stem plus verb construction, the stem and main verb are normally formed from the same verb root. Note that it is the stem that is used in the construction (not the root) since the /-j/ suffix is present \REF{ex:8:37}. Neither the main verb nor the auxiliary is inflected for subject, and the clause often has no noun phrase to indicate subject (\ref{ex:8:35}, \ref{ex:8:38}--\ref{ex:8:40}, \ref{ex:8:44}). When a subject noun phrase is present, it can only be a full free pronoun (\ref{ex:8:36}--\ref{ex:8:37},  \ref{ex:8:41}-- \ref{ex:8:43},  \ref{ex:8:45}). The main verb can have direct and indirect object pronominals and other extensions (\ref{ex:8:36} and \ref{ex:8:37}). In the following examples, the verb phrase is delimited by square brackets and the verb stem plus ideophone are bolded. 

\ea \label{ex:8:35}
{}[\textbf{Bah}  \textbf{məbehe}  kə  ver  aka  azla].\\
\gll  {}[\textbf{bax}    \textbf{mɪ-bɛh-ɛ}    kə  vɛr  aka  aɮa]\\
      pour   {\NOM}{}-pour-{\CL}  on  stone  on  now\\
\glt  ‘[She] poured [the grains of millet] on the grinding stone.’\textit{ }(lit. pour, pouring on the grinding stone now)
\z


\ea \label{ex:8:36}
Ndahan  [\textbf{ngah}  \textbf{mangəhata  aka  va}  yam  a  ver  ahan  ava].\\
\gll  ndahaŋ  [\textbf{ŋgax}  \textbf{ma-ŋgəh=ata=aka=va}  jam  a  vɛr=ahaŋ    ava]\\
      \oldstylenums{3}\textsc{s}    hide  {\NOM}{}-hide=\oldstylenums{3}\textsc{p}.{\IO}=on={\PRF}  water  at  room=\oldstylenums{3}\textsc{s}.{\POSS}  in\\
\glt  ‘He had hidden the water in his room.’
\z

\ea \label{ex:8:37}
Ndahan  [\textbf{ngay}  \textbf{mangaka  alay}  pərgom  ahay].\\
\gll  ndahaŋ  [\textbf{ŋg-aj}       \textbf{ma-ŋg=aka=alaj}    \\  
      \oldstylenums{3}\textsc{s}    {make with grass}-{\CL}  {{\NOM}{}-make with grass}=on=away    \\   

\medskip     
\gll pʊrgʷɔm=ahaj]\\
trap=Pl\\
\glt  ‘He made the traps out of grass.’
\z


\ea \label{ex:8:38}\corpussource{Disobedient Girl, S. 12}\\
Sen  ala  na  zar  ahan  na,\\  
\gll  ʃɛn=ala     na   zar=ahaŋ    na \\
      \textsc{id}:go=to  {\PSP}    man=\oldstylenums{3}\textsc{s}.{\POSS}    {\PSP}\\
\glt ‘Then, her husband,’\\

\medskip
[\textbf{dək} \textbf{mədəkan  na}  mənjəye  ata]. \\
\gll  {}[\textbf{dək} \textbf{mə{}-dək=aŋ}   \textbf{na}  mɪ{}-nʒ-ijɛ=atəta]\\
      instruct  {\NOM}{}-instruct=\oldstylenums{3}\textsc{s}.{\IO}   \oldstylenums{3}\textsc{s}.{\DO}   {\NOM}{}-sit-{\CL}=\oldstylenums{3}\textsc{p}.{\POSS}\\
\glt  ‘instructed her in their habits (lit. instructing their sitting).’ 
\z

In the case that there is an ideophone auxiliary (\ref{ex:8:39}--\ref{ex:8:42}), the ideophone occurs in the same slot as the verb stem auxiliary. Note that these ideophones are from entirely different roots than the verb stems. 

\clearpage
\ea \label{ex:8:39}\corpussource{Disobedient Girl, S. 20}\\
{}[\textbf{Jo madala}  háy  na  gam].\\
\gll  {}[\textbf{dzɔ}    \textbf{ma-d=ala}     haj  na      gam]\\
      \textsc{id}:take  {{\NOM}-prepare=to}  millet       {\PSP}  {a lot}\\
\glt  ‘[She] prepared lots of millet.’
\z

\ea \label{ex:8:40}\corpussource{Disobedient Girl, S. 28}\\
{}[\textbf{Pok} \textbf{mapalay}  mahay  na], \\ 
\gll  {}[\textbf{pɔk} \textbf{ma-p=alaj}      mahaj  na]\\
      \textsc{id}:open  {\NOM}{}-open=away   door     {\PSP}\\
\glt ‘[He] opened the door [and looked around];’\\

\medskip
həmbo  [árah  na  a  hoɗ  a  hay  ava].\\
\gll  hʊmbɔ [á-rax na a hʷɔɗ a haj ava]\\
      flour  \oldstylenums{3}\textsc{s}+{\IFV}-fill  \oldstylenums{3}\textsc{s}.{\DO}  at       stomach  {\GEN}   house  in\\
\glt  ‘the flour filled the house.’
\z

\ea \label{ex:8:41}
Ndahan  [\textbf{vəh}  \textbf{məngwəlva}  a  ɗəwer  ahan  ava].\\
\gll  ndahaŋ  [\textbf{vəh}    \textbf{mə-ŋgʷul=va}    a  ɗuwɛr=ahaŋ    ava]\\
      \oldstylenums{3}\textsc{s}    \textsc{id}:return    {\NOM}{}-return={\PRF}  at  sleep=\oldstylenums{3}\textsc{s}.{\POSS}  in\\
\glt  ‘He had already gone back to sleep.’
\z

\ea \label{ex:8:42}
Nata  ndahan  [\textbf{pək}  \textbf{mapata  aka  va}  pərgom  ahay  na].\\
\gll  nata    ndahaŋ  [\textbf{pək}  \textbf{ma-p=ata=aka=va}    pʊrgʷɔm=ahaj  na]\\
      also    \oldstylenums{3}\textsc{s}  \textsc{id}:open  {\NOM}{}-open=\oldstylenums{3}\textsc{p}.{\IO}=on={\PRF}  trap=Pl  {\PSP}\\
\glt  ‘He opened the traps.’ 
\z

\ea \label{ex:8:43}\corpussource{Disobedient Girl, S. 26}\\
Nata  ndahan  [\textbf{dəɓəsolək} \textbf{məmətava}  alay  a  hoɗ  a  hay  na  ava].\\
\gll  nata    ndahaŋ  [\textbf{dʊɓʊsɔlɔk}   \textbf{mə-mət=ava=alaj}   a   hʷɔɗ      a\\ 
      {and then}  \oldstylenums{3}\textsc{s}   \textsc{id}:collapse/die  {\NOM}{}-die=in=away   at   stomach  {\GEN}\\  
      
      \medskip
\gll haj na ava]\\
     house   {\PSP}    in\\
\glt  ‘And she collapsed, dying inside the house.’
\z

\ea \label{ex:8:44}\corpussource{Disobedient Girl, S. 31}\\
{}[\textbf{Babək} \textbf{mələye  na}].\\
\gll  {}[\textbf{babək} \textbf{mɪ-l-ijɛ}  \textbf{na}]\\
      \textsc{id}:bury      {\NOM}-bury-{\CL}   \oldstylenums{3}\textsc{s}.{\DO}\\
\glt  ‘She was buried.’ (lit. burying it)
\z

\ea \label{ex:8:45}\corpussource{Snake, S. 18}\\
Ne  [\textbf{dəyday} \textbf{məkəɗe  na  aka}].\\
\gll  nɛ  [\textbf{dijdaj} \textbf{mɪ-kɪɗ-ɛ}  \textbf{na=aka}]\\
      {\oneS}    \textsc{id}:approximately     {\NOM}{}-kill-{\CL}   \oldstylenums{3}\textsc{s}.{\DO}=on\\
\glt  ‘I clubbed it to death.’ (lit. I approximately killing it on)
\z

The stem or ideophone plus verb constructions mark significant events\is{Focus and prominence!Discourse peak} at the inciting moment and in the peak of a Moloko narrative. Example \REF{ex:8:38} is from the inciting moment of the Disobedient Girl story when the man instructs his wife. In the peak, the construction is seen when the woman prepares a lot of millet after having decided to disobey him \REF{ex:8:39}, when she pours a lot of millet on the grinding stone \REF{ex:8:39}, and when the millet suffocates her and she dies \REF{ex:8:43}. In the dénouement there is another ideophone plus nominalised form construction when the husband opens the door and finds her \REF{ex:8:40}. There are no other nominalised forms that fill the main verb slot in this text. 

Because the subject, direct object, and indirect object are optional for this construction, the construction can be used in Moloko discourse as a narrative device to reduce the number of explicit grammatical relations in a clause (cf. Sections~\ref{sec:3.6.3} and \ref{sec:9.4}). The participants become indefinite in the construction and must be inferred from the context. The effect is to draw the hearer into the action of the moment. In \REF{ex:8:35}, \REF{ex:8:39}, and \REF{ex:8:40}, \is{Transitivity!Clauses with zero transitivity|(}the construction is completely non-inflected for subject and has zero grammatical relations\is{Transitivity!Clauses with zero transitivity}. The narrative effect is that in \REF{ex:8:35} and \REF{ex:8:39}, the hearer only knows that someone is pouring something onto the grinding stone. In  \REF{ex:8:40}, it is as if the hearer is with the husband, looking into the house to find the woman. Likewise, in \REF{ex:8:44}, the verb \textit{mələye }‘bury’ is non-conjugated for subject, making those who buried the dead woman ‘out of sight’ in the narrative. 
\is{Ideophone|)}