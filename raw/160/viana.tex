\documentclass[output=paper]{LSP/langsci} 
\author{Fernanda Leopoldina Viana\affiliation{Instituto de Educação (CIEC), Universidade do Minho}\and 
Carla Silva\affiliation{CIEC, Universidade do Minho}\and 
Iolanda Ribeiro\affiliation{Escola de Psicologia (CIPsi), Universidade do Minho}\lastand 
Irene Cadime\affiliation{CIEC, Universidade do Minho}
}
\title{Instrumentos de avaliação da linguagem: uma perspetiva global}  
\abstract{\noabstract}
\ChapterDOI{10.5281/zenodo.889443}
\maketitle
\begin{document}
\section{Introdução}
\label{sec:viana_intro}

A linguagem é fundamental para o desenvolvimento do ser humano, uma vez que é a base da comunicação, da aprendizagem e da construção das relações interpessoais \citep{brockrankin2010,neaum2012}. A sua aquisição é um marco importante no desenvolvimento das crianças, pelo que eventuais atrasos\is{atrasos na linguagem} necessitam de identificação atempada que possibilite uma intervenção o mais precoce possível. Todavia, a enorme variabilidade interindividual que se regista quer em termos de aquisição, quer em termos de desenvolvimento, pode levar à desvalorização de atrasos reais ou à sobrevalorização de desvios que devem ser considerados “normais”. 

A linguagem é não só importante para o desenvolvimento de um modo geral, mas também para a aprendizagem da leitura. A investigação recente mostrou que a leitura (uma aquisição cultural) usa as rotas neuronais da linguagem oral \citep{dehaene2007} e que um bom desenvolvimento da linguagem é, por isso, facilitador da aprendizagem da vertente escrita da língua \citep{lee2011,locke_etal2002,mcguinness2006,neaum2012,ouellette2006}. Esta, por sua vez, é fundamental para uma verdadeira integração na sociedade. 

A evidência empírica sobre o papel da linguagem oral para a aprendizagem da leitura e da escrita teve repercussões nos currículos e/ou orientações curriculares, quer para os anos pré-escolares, quer para os anos escolares. Em Portugal, as \textit{Orientações Curriculares para Educação Pré-Escolar} (OCEPE), quer as publicadas em 1997 \citep{ministerioedu2007}, quer as publicadas em 2016 \citep{silva_etal2016}, incluem diretrizes importantes no que concerne à área da Linguagem e da Comunicação. Também os programas de Português para o Ensino Básico fazem referência, de forma mais ou menos explícita, à importância do domínio na língua – entendida como um sistema de comunicação que reúne as características específicas que a linguagem assume numa determinada comunidade linguística – como elemento central da aprendizagem escolar. No programa de 1991 \citep[97]{ministerioedu1991} pode ler-se: “Reconhece-se a Língua Materna como elemento mediador que permite a nossa identificação, a comunicação com os outros e a descoberta e compreensão do mundo que nos rodeia. Tem-se como seguro que a restrição da competência linguística impede a realização integral da pessoa, isola da comunicação, limita o acesso ao conhecimento, à criação e à fruição da cultura e reduz ou inibe a participação na praxis social”. No programa de 2007 \citep[6]{ministerioedu2007}, que assume como ponto de partida o programa de 1991, reitera-se que “a nossa língua é um instrumento fundamental de acesso a todos os saberes; e sem o seu apurado domínio, no plano oral e no da escrita, esses outros saberes não são adequadamente representados” e que “a aprendizagem da língua condiciona e favorece a relação da criança e do jovem com o mundo, bem como a progressiva afirmação de procedimentos cognitivos, de competências comunicativas e de atitudes afetivas e valorativas que são determinantes para a referida relação com o mundo e com aqueles que o povoam” (p. 12). O programa de 2009 \citep[6]{ministerioedu2009} assume estas mesmas premissas, considerando que o ensino e a aprendizagem do Português “determinam irrevogavelmente a formação das crianças e dos jovens, condicionando a sua relação com o mundo e com os outros” e que “a língua que aprendemos (e que a escola depois incorpora como matéria central) está diretamente ligada à nossa criação e ao nosso desenvolvimento como seres humanos”. No atual programa do 1.º Ciclo do Ensino Básico \citep{buescu_etal2015}, a tónica é colocada nas relações entre a linguagem oral e a aprendizagem da vertente escrita da língua.\newpage

Como se infere, a língua é, simultaneamente, conteúdo curricular e o meio através do qual grande parte do conhecimento é adquirido \citep{cazden1973}. Os currículos assumem a existência de competências linguísticas básicas, desenvolvidas através de múltiplas experiências tais como falar, ouvir, contar histórias, fazer rimas, ler e cantar \citep{brockrankin2010,whitehead2007}.

Atendendo às implicações que o desenvolvimento da linguagem tem para a aprendizagem, é importante que educadores e professores disponham de conhecimentos sobre a sua aquisição e desenvolvimento e, também, sobre a sua avaliação. Os \isi{atrasos na linguagem} podem estar associados a problemas ligeiros, muitas vezes por falta de estimulação, ou ser sinal de uma patologia mais severa, podendo, se não forem detetados precocemente, levar a dificuldades acrescidas no futuro, condicionando aprendizagens como a aprendizagem da leitura \citep{mcguinness2006}. 

\section{A avaliação da linguagem oral: perspetiva histórica}
\label{sec:viana_avaliacao_historica}

As dificuldades e/ou atrasos\is{atrasos na linguagem} na aquisição e no desenvolvimento da linguagem apresentadas pelas crianças são de tipo e de níveis muito variados, podendo estar associadas aos mais diversos fatores, nomeadamente culturais, biológicos e socioeconómicos. Podem estar relacionados, por exemplo, com condições de privação comunicacional e linguística, ou ser sinal de uma patologia severa. No entanto, independentemente da sua etiologia, é necessário que exista um diagnóstico preciso do tipo e da gravidade das dificuldades apresentadas. O diagnóstico diferencial entre crianças que apresentam atrasos de desenvolvimento globais e crianças em que os problemas se circunscrevem à área da linguagem é,\is{atrasos na linguagem} também, fundamental, nomeadamente para o desenho de intervenções precoces e ajustadas \citep{mariscallopezornat2010,mariscal_etal2007,neaum2012,westerlund_etal2006}. 

O processo de avaliação da linguagem oral passou por mudanças significativas ao longo do século XX. No início, os objetivos centravam-se essencialmente na obtenção de uma medida que permitisse a comparação com o grupo normativo e/ou a classificação num quadro, em que o propósito da avaliação foi o de obter informações relevantes para a intervenção educativa \citep{puyuelo2003}.

Nos anos 50 do século XX, a avaliação da linguagem estava centrada essencialmente nos aspetos normativos e patológicos, tendo como referência o modelo linguístico do adulto \citep{launerlahey1981}. Esta avaliação – centrada essencialmente em aspetos semânticos – era, geralmente, complementada com testes de inteligência, uma vez que o QI verbal era tido como um índice de capacidade linguística. Com o objetivo de se obterem dados de forma o mais natural possível, foram também realizadas recolhas em grande escala de discurso espontâneo produzido por crianças, tendo surgido medidas como o MLU (\textit{Mean length of utterance}),\is{mean length of utterance|see {extensão média do enunciado}} através da qual se calcula o comprimento médio dos enunciados, e o SCS (\textit{\isi{Structural Complexity Score}}), que indica a riqueza gramatical ou estrutural dos enunciados produzidos, como forma de analisar as produções obtidas. Estas medidas deram origem a diferentes \isi{escalas normativas} que indicavam a evolução em função da idade da criança. No que diz respeito a \isi{provas estandardizadas}, um dos testes criados nesta época, e ainda amplamente utilizado, é o \textit{Peabody Picture Vocabulary Test} \citep{dunndunn1959}, destinado a crianças de idades compreendidas entre os 12 meses e os 7 anos, e que avalia a linguagem recetiva (compreensão de vocabulário e de relações entre objetos e acontecimentos). Numa lógica preventiva, surgiram também estudos que abordaram a correção fonética, e, refletindo já preocupações com a intervenção, alguns autores procuraram determinar a etiologia dos atrasos\is{atrasos na linguagem} ou dificuldades ao nível da linguagem (e.g. \citealt{myklebust1954}).

Na década de 60 do século XX, a tónica foi colocada no desenho de programas de intervenção. Os índices MLU e SCS foram combinados, dando lugar ao cálculo do LCI (\textit{\isi{Length Complexity Index}}), que analisa a extensão e a complexidade dos enunciados. Como resultado da crescente preocupação com a intervenção, registou-se um crescente interesse pelas medidas estandardizadas, assistindo-se à publicação de várias provas de avaliação da linguagem. Um exemplo de prova criada nesta época, e ainda hoje muito utilizada, é o \textit{Illinois Test of Psycholinguistic Abilites} \citep{kirk_etal1961}, que avalia competências linguísticas em três dimensões: canais de comunicação (auditivo-vocal e visual-motor), processos psicolinguísticos (recetivo, organizativo e expressivo) e níveis de organização (automático e representativo). Destinado a crianças com idades compreendidas entre os 2 e os 10 anos, tem como principais objetivos o diagnóstico diferencial e o encaminhamento para programas de educação compensatória. São também desta década o \textit{Basic Language Concepts Test} \citep{engelmann_etal1966}, para crianças dos 4 aos 6.5 anos, e o \textit{Boehm Test of Basic Concepts} \citep{boehm1967}, destinado a crianças entre os 3 e os 6 anos.

Já no final desta década, assistiu-se a um aumento da preocupação com a aquisição da sintaxe, surgindo testes como o \textit{Northwestern Syntax Screening Test} \citep{lee1969}, destinado a crianças de idades compreendidas entre os 3 e os 7 anos. 

Na década de 70, mantém-se o enfoque na intervenção, aumenta o interesse por estudos normativos do desenvolvimento da linguagem e começa a observar-se uma preocupação crescente com a análise dos contextos em que a linguagem é produzida, atendendo às características socioculturais do emissor. Os contextos familiar e escolar, devido às situações naturais de diálogo que naturalmente comportam (como, por exemplo, as produções linguísticas que surgem da interação com os cuidadores durante as refeições, a higiene, o brincar e o adormecer), passam a ser privilegiados na avaliação “naturalista”, centrada na \isi{análise do discurso espontâneo} das crianças. No que concerne a provas de avaliação, regista-se a inclusão nas mesmas de dimensões da avaliação da linguagem mais correlacionadas com as aprendizagens escolares, como a consciência fonológica, como é o caso do \textit{Bankson Language Screening Test} \citep{bankson1977}, ou do \textit{Preschool Assessment Instrument} \citep{blank_etal1978}.

Nos anos 80 e 90 do século XX, a preocupação com a identificação precoce de atrasos de desenvolvimento da linguagem mantém-se, como revela, por exemplo, a publicação do\textit{ Clark-Madison Test of Oral Language} \citep{clarkmadison1981}, ou do \textit{Test of Early Language Development} \citep{hresko_etal1981}, mas regista-se o interesse pela avaliação de aspetos mais específicos do desenvolvimento da linguagem, como acontece, por exemplo, com o \textit{Test de Rélations Topologiques} \citep{deiltourhupkens1980} ou com o \textit{Test for Examining Expressive Morphology} \citep{shipley_etal1983}. Nestas décadas, com base no crescente interesse pelo estudo da linguagem como processo que tem lugar em contextos de interação, surgem, de acordo com \citet{acosta2006}, duas novas perspetivas de avaliação: uma primeira perspetiva que se centra no princípio da cooperação observado nas trocas comunicativas que surgem na conversa, e uma segunda perspetiva que se interessa essencialmente pelas rotinas ou formatos que a criança aprende para poder desenvolver as suas competências linguísticas. Em Portugal, o estudo \textit{Comunicação entre Crianças}, publicado por \citet{simoes1990}, marca precisamente esta nova tendência. 

No que concerne ao panorama nacional, é apenas na década de 80 que surgem os primeiros instrumentos de avaliação da linguagem (ver Tabela \ref{tab:viana_1}, Secção \ref{sec:viana_instrumentos_portugal}).

\section{Objetivos da avaliação: Para quê avaliar?}
\label{sec:viana_para_que}

\citet{westby_etal1996} identificaram quatro objetivos básicos – e interdependentes – que justificam a avaliação do desenvolvimento da linguagem: i) o despiste de crianças com atrasos de linguagem\is{atrasos na linguagem} (\textit{screening});\is{screening@\textit{screening}} ii) a definição da linha de base do funcionamento linguístico; iii) a definição dos objetivos e de procedimentos de intervenção; e iv) a avaliação da evolução e do impacto da intervenção.

O primeiro objetivo – \textit{screening}\is{screening@\textit{screening}} – consiste numa avaliação relativamente breve e simplificada de problemas de utilização ou de compreensão da linguagem. Este tipo de avaliação é, no geral, incluído numa avaliação mais ampla do desenvolvimento, visando a identificação precoce de atrasos de desenvolvimento gerais. Por exemplo, a \textit{Escala de Desenvolvimento} de \citet{sheridan2007}, cujo uso como instrumento de \textit{screening}\is{screening@\textit{screening}} é aconselhado pelo Ministério da Saúde\footnote{\url{https://www.dgs.pt/documentos-e-publicacoes/programa-tipo-de-atuacao-em-saude-infantil-e-juvenil.aspx}}, no âmbito do Programa Nacional de Saúde Infantil e Juvenil, é um instrumento deste tipo. Já o \textit{Rastreio de Linguagem e Fala - RALF} \citep{mendes_etal2015} é específico, visando um rastreio rápido de suporte a uma primeira tomada de decisão quanto à necessidade de proceder a uma avaliação mais aprofundada.

Quanto ao segundo objetivo, se uma criança é referenciada para uma avaliação, é importante que se proceda a uma definição da linha de base do funcionamento linguístico. No geral, tem lugar uma avaliação inicial das áreas relacionadas com a habilidade da criança para utilizar a linguagem, tais como a audição, as habilidades oral-motoras e as competências cognitivas, de modo a recolher dados que contribuam para um diagnóstico diferencial. No caso de ter sido previamente efetuada uma avaliação do tipo \textit{screening},\is{screening@\textit{screening}} no geral há já indicadores quanto às áreas que necessitam de avaliação específica (e.g. fonologia, vocabulário, morfologia, sintaxe, pragmática), o que permite a seleção dos instrumentos disponíveis mais adequados.

A utilização da expressão linha de base em vez de diagnóstico é indicadora da estreita ligação da avaliação com a intervenção. Esta linha de base, que deverá ser o mais específica possível, revelando as áreas deficitárias, mas também as áreas fortes, permitirá, por sua vez, que se desenhe o programa de intervenção mais adequado e que se tracem objetivos exequíveis a curto, a médio e a longo prazo.

 Por último, ao longo da intervenção é necessária a realização de avaliações que visem monitorizar a sua eficácia, e, eventualmente, introduzir mudanças na mesma. No final da intervenção, é importante avaliar o impacto da mesma. \citet{puyuelo2003} refere ainda um quinto objetivo da avaliação, que é o de servir a investigação. 
 
 \section{O que avaliar?}
 \label{sec:viana_o_que}
 
A definição do que se vai avaliar é fundamental e, ao nível da linguagem oral, podemos distinguir entre as dimensões da linguagem, as funções da linguagem e as áreas colaterais \citep{paul2007}.

Nas diferentes dimensões da linguagem podem avaliar-se: a sintaxe, a morfologia a fonologia, a semântica e a pragmática. As funções da linguagem englobam a compreensão e a produção linguísticas. A este nível, podemos distinguir entre linguagem recetiva e linguagem expressiva. A linguagem expressiva refere-se ao modo como os pensamentos são expressos em palavras e frases, de forma gramaticalmente adequada. Por sua vez, a linguagem recetiva refere-se à capacidade para compreender, perceber e descodificar linguagem falada e escrita.

Nas áreas colaterais podem avaliar-se aspetos das bases anatómicas e funcionais, tais como a audição e a fonação (respiração, motricidade orofacial e voz), o desenvolvimento cognitivo e as funções sociais \citep{paul2007}. 

Os procedimentos e estratégias de avaliação vão depender essencialmente dos objetivos da avaliação e daquilo que se vai avaliar \citep{acosta2006,paul2007}. 

\section{Metodologias de avaliação: Como avaliar?}
\label{sec:viana_como}

A avaliação do desenvolvimento da linguagem recorre a quatro abordagens principais: a \isi{análise do discurso espontâneo}, a \isi{análise de narrativas}, a utilização de \isi{provas estandardizadas} e os \isi{relatos parentais} \citep{acosta_etal2013,condouris_etal2003,creasey2006}. Procedemos à autonomização da \isi{análise de narrativas} em vez de incluir este tipo de abordagem dentro do discurso espontâneo (em situação semiestruturada), essencialmente porque o discurso espontâneo é preferencialmente usado até aos 24 meses de idade e a \isi{análise de narrativas} em idades posteriores.

\subsection{Análise do discurso espontâneo\is{análise do discurso espontâneo}}
\label{subsec:viana_analise_disc_espont}

A \isi{análise do discurso espontâneo} tem sido largamente utilizada na investigação, para avaliar a linguagem expressiva \citep{condouris_etal2003}, procurando-se com a mesma chegar à descrição do desenvolvimento linguístico nas suas diferentes dimensões \citep{acosta2006}.

Esta metodologia permite obter dados sobre a utilização que a criança faz da linguagem em diferentes contextos e em diferentes situações de interação \citep{condouris_etal2003}. Estes contextos e situações poderão ser mais ou menos estruturados, mas assegurando sempre que as recolhas sejam feitas em ambientes familiares às crianças, de forma a não criar constrangimentos à produção linguística; é nestes ambientes que geralmente as crianças falam de forma mais natural e espontânea, o que torna as produções mais ricas \citep{creasey2006,puyuelo2003}.

O discurso espontâneo é obtido através da gravação áudio ou vídeo de situações de interação, com o objetivo de se garantir um \textit{corpus} representativo de pelo menos 100 produções linguísticas. Tendo em conta o objetivo da avaliação, é necessário que se faça uma planificação da duração e da quantidade das recolhas, mas deverá existir sempre alguma flexibilidade, uma vez que pode ser necessário fazer a recolha em diferentes momentos, com durações de gravação também elas diferentes, para que se consiga obter os dados necessários à análise \citep{demuth1998}. Como foi referido, as situações de interação poderão ser mais ou menos estruturadas, consoante o objetivo da análise. Poderão ser utilizados diversos materiais que funcionarão como facilitadores da interação, no entanto a escolha destes materiais deverá ter em conta as características culturais dos sujeitos avaliados. 

A recolha e a \isi{análise do discurso espontâneo} requerem um grande investimento de tempo, pois mesmo amostras extensas não garantem que sejam recolhidas todas as estruturas que uma criança é capaz de produzir num dado momento do seu percurso de desenvolvimento linguístico. A riqueza da informação obtida e a possibilidade de análise de aspetos distintos da linguagem são vantagens que se destacam nesta metodologia. No entanto, dado que a recolha e a \isi{análise do discurso espontâneo} requerem um grande investimento de tempo, esta metodologia é mais utilizada em investigação do que como meio de diagnóstico. Também algumas medidas quantitativas utilizadas na análise, como o MLU, são frequentemente questionadas – nomeadamente se não são usadas em conjunto com outras medidas – dado serem consideradas desajustadas para descrever em detalhe as competências linguísticas \citep{lee1974,leecanter1971,nelson1973,owens2012}. 

A presença de observadores e de alguns recursos tecnológicos (como câmaras de vídeo ou microfones) são aspetos que suscitam várias críticas a esta metodologia. Sem eles a obtenção de dados passíveis de interpretação é mais difícil, mas a sua presença pode levar à perda de espontaneidade no discurso da criança. A (in)inteligibilidade das produções em idades em que, com frequência, as crianças recorrem ainda a um jargão característico, pode, também, ser limitadora do seu uso.

\largerpage Além dos problemas já referidos, são de salientar, ainda, questões que se prendem com a seleção dos segmentos e, consequentemente, com a representatividade das amostras recolhidas \citep{owens2012}. Esta seleção pode conduzir a uma subestimação das competências linguísticas da criança \citep{demuth1998}. A obtenção de um número suficientemente representativo de produções linguísticas junto de crianças muito novas e/ou junto de crianças com dificuldades nesta área é uma das dificuldades que importa, igualmente, salientar. Podem ainda ser referidos, como fontes de dificuldades, a falta de critérios consensuais relativamente à segmentação das produções que serão analisadas e a ausência de perfis ou índices evolutivos que sirvam como referência normativa na interpretação dos resultados obtidos por cada criança \citep{acosta2006}. 

\subsection{Análise de narrativas\is{análise de narrativas}}
\label{subsec:viana_analise_narrativas}

As situações de produção de narrativas são mais estruturadas do que as utilizadas para recolha de discurso espontâneo, havendo um maior controlo do contexto. O procedimento comum para a obtenção de um \textit{corpus} de produções de fala através de narrativas consiste em fazer uso de uma tarefa de reconto, com ou sem ajudas materiais. Quando não há ajudas materiais, a criança apenas tem de escutar uma história narrada pelo avaliador e proceder ao seu reconto. Quando é oferecida ajuda, são fornecidos cartões que ilustram a sequência de ações que constituem a narrativa. Há toda a vantagem em usar imagens, já que elas proporcionam uma importante ajuda, não sobrecarregando a memória, e oferecem um apoio importante no sentido de evitar ambiguidades e dotar de maior coesão a narrativa produzida. Um material universalmente utilizado com a finalidade de desencadear narrativas é o conto \textit{Rã, onde estás?} \citep{mayer1969}. Uma vez gravada a produção da criança, procede-se à sua transcrição literal e, posteriormente, à sua segmentação, codificação e análise. As análises incidem sobre a estrutura ou esquema narrativo, a análise linguística, a análise dos recursos de coesão e a análise das hesitações.

Dada a relativa semelhança de procedimentos, as vantagens e desvantagens da recolha de narrativas são semelhantes às referidas previamente para as recolhas de discurso espontâneo. Ambas exigem muito tempo para a recolha, transcrição e análise de dados; ambas podem não contemplar todas as estruturas linguísticas que a criança é capaz de produzir. Além disso, a recolha de narrativas não é exequível quando estamos perante crianças com idades inferiores a 24 meses, as quais podem ter dificuldades em compreender a tarefa ou em cooperar de forma consistente para que os dados recolhidos sejam fidedignos.

\section{Provas estandardizadas\is{provas estandardizadas}}
\label{subsec:viana_provas_estandard}

As \isi{provas estandardizadas} podem ser referenciadas a normas ou a critério \citep{almeidaviana2010}. No primeiro grupo encontram-se os testes em que a interpretação dos resultados é efetuada através da comparação dos resultados obtidos pelo sujeito com os resultados considerados normais para uma amostra representativa da população. No segundo grupo incluem-se os testes em que a interpretação dos resultados é efetuada tendo como referência um critério (geralmente teórico) pré-definido para um determinado grupo de sujeitos.

Segundo a \textit{American Educational Research Association} – \citep{aera2014}, as \isi{provas estandardizadas} têm como principais características a utilização de materiais especialmente desenvolvidos para o efeito e amplamente testados, procedimentos de administração claros, aplicadores treinados e regras de cotação específicas e consistentes. Para além destes aspetos, são reportados dados sobre a sua fidelidade e validade \citep{paul2007}. Estes aspetos constituem vantagens incontornáveis, às quais se alia a de exigirem muito menos tempo do que a \isi{análise do discurso espontâneo} ou de narrativas. Além disso, podem ser construídas tendo como objetivo uma avaliação de \textit{screening}\is{screening@\textit{screening}} ou uma avaliação de aspetos específicos do desenvolvimento da linguagem, tornando a avaliação menos dispendiosa em termos de recursos materiais e humanos. 

À semelhança de outras metodologias para a avaliação da linguagem, também esta apresenta desvantagens. Por um lado, a realização de provas coloca o sujeito numa situação artificial. Por outro lado, as mesmas são geralmente utilizadas por técnicos cuja familiaridade com os sujeitos pode ser reduzida e, nomeadamente quando os avaliados são crianças, a relação interpessoal avaliador/criança pode influenciar de forma determinante as respostas obtidas \citep{condouris_etal2003}. Se na avaliação da linguagem através das produções de discurso espontâneo foi referido que o \textit{corpus} recolhido seria sempre uma “amostra”, cuja representatividade era assegurada, em grande parte, pela extensão da recolha e pela variedade de contextos e situações, na avaliação com recurso a \isi{provas estandardizadas} estes constrangimentos são maiores. Cada prova oferece uma visão parcial do que o sujeito produz e/ou compreende. No entanto, quando bem construídas e aplicadas, proporcionam informação que permite comparar, rapidamente, o desempenho de uma criança com o dos seus pares \citep{condouris_etal2003,paul2007} ou o seu grau de afastamento relativamente a um determinado critério \citep{almeidaviana2010}.

\subsection{Relatos parentais}
\label{subsec:viana_relatos}
\is{relatos parentais}

Os diários, utilizados ao longo de toda a história da psicolinguística, são considerados a mais antiga forma de registo que deu, posteriormente, origem aos \isi{relatos parentais} \citep{fenson_etal2007}. 

Os \isi{relatos parentais} são uma alternativa às metodologias anteriores e, apesar de terem, também, limitações, eles são uma importante fonte de informação sobre o desenvolvimento linguístico das crianças, uma vez que os pais são observadores privilegiados de um vasto conjunto de produções e trocas linguísticas nas mais variadas situações e contextos às quais os profissionais dificilmente conseguem aceder em contexto de avaliação. A investigação tem mostrado que os dados baseados nos \isi{relatos parentais} são preditores de comportamentos comunicativos\newpage e linguísticos que ocorrem espontaneamente, permitindo a obtenção de dados mais representativos do que os obtidos por outros métodos \citep{fenson_etal2007}. 

O facto de muitos dos instrumentos de relato parental\is{relatos parentais} possuírem, tal como as \isi{provas estandardizadas}, propriedades psicométricas bem definidas \citep{paul2007}, apresentarem uma boa relação custo-eficiência e permitirem a construção de normas \citep{fenson_etal2007,simonsen2014norwegian}, justificam o incremento que se tem vindo a registar no seu uso, nomeadamente junto de crianças muito novas. Os instrumentos baseados nos \isi{relatos parentais} têm sido considerados fiáveis para a recolha de informações sobre o \textit{desenvolvimento linguístico} \citep{feldman_etal2000,fenson_etal2007,jacksonmaldonado1993,thal2000}, nomeadamente quando recorrem ao formato \textit{checklist}, no qual são incluídas produções linguísticas que ocorrem em vários contextos de vida da criança e que os pais conseguem identificar com facilidade sem necessidade de recorrer a um relato\is{relatos parentais} retrospetivo \citep{bates1993}. 

O risco de subjetividade associado à possível sobre ou subvalorização das habilidades das crianças nos \isi{relatos parentais} tem sido apontado como a principal desvantagem desta metodologia. Este risco é, no entanto, minimizado, quando o relato parental\is{relatos parentais} usa o formato de reconhecimento e se cinge a comportamentos atuais e emergentes \citep{fenson_etal2007}.

As quatro abordagens aqui descritas não são mutuamente exclusivas, sendo até recomendável que possam usar-se de forma complementar, para que os dados obtidos sejam o mais representativos, abrangentes e fidedignos possível. A opção por determinado tipo de avaliação dependerá dos sempre dos objetivos da mesma. 

\section{Instrumentos de avaliação da linguagem em Portugal}
\label{sec:viana_instrumentos_portugal}

Apesar do crescente interesse que se vem registando em Portugal pela temática da avaliação da linguagem, ainda são escassos os instrumentos destinados às idades mais precoces. A maioria dos instrumentos disponíveis apenas permitem fazer a avaliação a partir dos 30 meses, avaliam aspetos parciais da linguagem, não permitem a avaliação da linguagem emergente, que engloba a transição entre os estádios pré-linguístico e linguístico, e não são suficientes para avaliar, por exemplo, algumas áreas da sintaxe 
\citep{afonso_2011}. As exigências e problemas metodológicos associados à avaliação em idades precoces \citep{feldman_etlal2005,mariscal_etal2007} e o facto de a produção de provas ser um processo bastante moroso e complexo poderão explicar esta lacuna. Assim, em crianças com idades inferiores a 30 meses a avaliação está largamente dependente de julgamentos clínicos fortemente dependentes da experiência do avaliador e da utilização de instrumentos informais. Apenas em 2017 foi publicada a adaptação para Português Europeu dos \textit{Inventários de Desenvolvimento Comunicativo de MacArthur-Bates} \citep{viana_etal2017}.

Nas Tabelas \ref{tab:viana_a}--\ref{tab:viana_f} são apresentados os instrumentos de avaliação da linguagem disponíveis para a população portuguesa. Para a sua identificação, recorreu-se às bases de dados da Porbase e das bibliotecas universitárias associadas, ao registo de teses (Fundação para a Ciência e a Tecnologia), aos Centros de Investigação e aos catálogos de editoras de referência. Apesar dos cuidados colocados na recolha, a não disponibilização de alguns trabalhos em bases de dados ou a não utilização de palavras-chave adequadas poderá ter originado alguns lapsos na recolha, pelo que é possível que existam outros instrumentos aqui não referenciados. Desta síntese foram excluídos materiais destinados a avaliações de carácter informal. Nem todos os instrumentos podem ser considerados validados para a população portuguesa, embora alguns deles possuam alguns estudos relativos a propriedades psicométricas, pelo que as informações que facultam ao examinador devem ser analisadas de forma cuidadosa. 

\begin{sidewaystable}
\resizebox{\linewidth}{!}{
\centering
\caption{Instrumentos para avaliação da linguagem oral em português europeu\label{tab:viana_1}}
\label{tab:viana_a}
\begin{tabular}{p{0.4\textwidth}p{0.20\textwidth}p{0.1\textwidth}p{0.25\textwidth}p{0.1\textwidth}}
\lsptoprule
Nome do Instrumento                                                                                                               & Fonte                                                                                           & Idade alvo            & Áreas/competências avaliadas                                                                                                                                                                                                   & Tempo de aplicação \\
\midrule
Token Test (Versão Portuguesa)                                                                                                    & \cite{derenzivignolo1962}. Adaptado por \cite{pinto1988}                                                                     & A partir dos 6 anos   & Avalia a compreensão da linguagem recetiva complexa                                                                                                                                                                            & 10 min.            \\
A.N.A.E. Questionnaire ``Language et Comportement -3ans 1/2''. Linguagem e comportamento aos três anos e meio (Versão Portuguesa) & \cite{chevriemuller1990} Adaptado por \cite{castro_etal1994} Validado por \cite{amorim2000} & 3 anos e meio         & Avalia voz e fala; compreensão - noções cognitivas, expressão, motricidade (global e habilidade manual), memória, comportamento no jogo, comportamento nas atividades escolares, comportamento em grupo e comportamento geral. & Variável           \\
Teste da Avaliação da Linguagem Oral (ALO)                                                                                        & \cite{simsim1997}                                                                                   & 3--9 anos             & Avalia a linguagem expressiva e recetiva através de definição verbal, nomeação, compreensão de estruturas complexas, complemente de frases, reflexão morfossintática e segmentação e reconstrução segmental.                   & Variável           \\
Teste de Articulação Verbal (TAV)                                                                                                 & \cite{guimaraesgrilo1997}                                                                        & 3 - 5 anos e 11 meses & Avalia os sons produzidos pela criança, identificando possíveis perturbações articulatórias.                                                                                                                                   & 10 min.            \\
Teste de Avaliação da Produção Articulatória de Consoantes do Português Europeu (TAPAC-PE)                                        & \cite{fale_etal2001}                                                                     & a Partir dos 3 anos   & Avalia a produção articulatória de consoantes                                                                                                                                                                                  & Variável           \\
\lspbottomrule
\end{tabular}}
\end{sidewaystable}

\begin{sidewaystable}
\resizebox{\linewidth}{!}{
\centering
\caption{Instrumentos para avaliação da linguagem oral em português europeu (cont.)}
\label{tab:viana_b}
\begin{tabular}{p{0.4\textwidth}p{0.20\textwidth}p{0.1\textwidth}p{0.25\textwidth}p{0.1\textwidth}}
\lsptoprule
Nome do Instrumento                                                                                                               & Fonte                                                                                           & Idade alvo            & Áreas/competências avaliadas                                                                                                                                                                                                   & Tempo de aplicação \\
\midrule
Bateria de Provas Fonológicas & \cite{silva2002} & 5--6 anos & Avalia a capacidade para analisar as palavras aos seus componentes sonoros & Variável \\
Grelha de Observação da Linguagem – nível escolar (GOL-E) & \cite{suakaysantos2003} & 5 anos e 7 anos meses--10 anos & Avalia a capacidade linguística nas áreas da semântica, morfossintaxe e fonologia. & 30 min.\\
Teste de Identificação de Competências Linguísticas (TICL) & \cite{viana2004} & 4--6 anos & Prova de linguagem expressiva que avalia o conhecimento lexical, o conhecimento morfossintático, a memória auditiva e a capacidade de refletir sobre a língua. & 45 min. aprox.\\
Prova de Avaliação da Articulação de Sons em Contexto de Frase para o Português Europeu & \cite{vicente_etal2006} & 2--7 anos & Prova de linguagem expressiva que avalia o conhecimento lexical, o conhecimento morfossintático, a memória auditiva e a capacidade de refletir sobre a língua. & 45 min. aprox.\\
Teste de Avaliação da Linguagem na Criança (TALC) & \cite{suakaytavares2006} & 2 anos e 6 meses--6 anos & Avalia as componentes da compreensão e expressão da linguagem nas áreas da semântica, morfossintaxe e pragmática. & 30--45 min.\\
\lspbottomrule
\end{tabular}}
\end{sidewaystable}

\begin{sidewaystable}
\resizebox{\linewidth}{!}{
\centering
\caption{Instrumentos para avaliação da linguagem oral em português europeu (cont.)}
\label{tab:viana_c}
\begin{tabular}{p{0.4\textwidth}p{0.20\textwidth}p{0.1\textwidth}p{0.25\textwidth}p{0.1\textwidth}}
\lsptoprule
Nome do Instrumento                                                                                                               & Fonte                                                                                           & Idade alvo            & Áreas/competências avaliadas                                                                                                                                                                                                   & Tempo de aplicação \\
\midrule
Prova de Avaliação da Linguagem e da Afasia em Português  (PALPa-P) & \cite{kay_etal1992}. Adaptado por \cite{castro_etal2007} & A partir dos 5 anos & Bateria de avaliação neuropsicológica que avalia o processamento fonológico, a leitura e a escrita, a semântica das palavras e a compreensão de frases. & Variável\\
Prova de Avaliação Fonológica em Formatos Silábicos (PAFFS) & \cite{lima2008} & 3--7 anos & Avalia a fonologia em formatos silábicos. & Variável\\
Prova de Avaliação de Capacidades Articulatórias & \cite{baptista2009} & 3--6 anos & Avalia as capacidades articulatórias da fala. & Variável\\
Schlichting Test for Sentence Development – PT & \cite{schlichting2010schlichting}. Adaptado por \cite{vieir2011} & 3--6 anos & Avalia o comportamento linguístico sintático. & Variável\\
Prova de Repetição de Pseudopalavras (PRePP) & \cite{ribeiro2011} & 6 anos e 5 meses--10 anos e 4 meses & Avalia o desempenho na repetição de pseudopalavras. & Variável\\
Inventários do Desenvolvimento das Habilidades Comunicativas de MacArthur-Bates para o Português Europeu – forma reduzida –nível I  & \cite{fenson_etal2000}. Adaptação de \cite{frota_etal2016} & 8--18 meses & Avalia vocabulário recetivo e vocabulário expressivo. & 5--10 min.\\
Inventários do Desenvolvimento das Habilidades Comunicativas de MacArthur-Bates para o Português Europeu – forma reduzida – nível II & \cite{fenson_etal2000}. Adaptação de \cite{frota_etal2016} & 16--30 meses & Avalia vocabulário recetivo e expressivo, formação de palavras complexas e habilidade de produção de combinações de palavras. & 5--10 min.\\ 
\lspbottomrule
\end{tabular}}
\end{sidewaystable}

\begin{sidewaystable}
\resizebox{\linewidth}{!}{
\centering
\caption{Instrumentos para avaliação da linguagem oral em português europeu (cont.)}
\label{tab:viana_d}
\begin{tabular}{p{0.4\textwidth}p{0.20\textwidth}p{0.1\textwidth}p{0.25\textwidth}p{0.1\textwidth}}
\lsptoprule
Nome do Instrumento                                                                                                               & Fonte                                                                                           & Idade alvo            & Áreas/competências avaliadas                                                                                                                                                                                                   & Tempo de aplicação \\
\midrule
Teste Fonético Fonológico – Avaliação da Linguagem Pré-Escolar (TFF-ALPE) & \cite{mendes_etal2013} & 3--6 anos & Avalia a capacidade de articulação verbal, o tipo e percentagem de ocorrência de processos fonológicos e a inconsistência na repetição de palavras. & 15--20 min.\\
Modelo ASEBA – Questionário de Comportamentos da Criança (CBCL – 1.6-5) – Questionário de Desenvolvimento da Linguagem (LDS) & \cite{achenbachrescorla2001}. Adaptado por \cite{achenbach_etal2013} & 18--35 meses & Questionário de relato de problemas e competências através da informação dos pais (ou substitutos), que contém 99 descrições de comportamentos (problemas). Inclui o questionário LDS, que avalia o vocabulário expressivo. & 10 min.\\
Teste de Linguagem (TL-ALPE) & \cite{mendes_etal2014} & 3--5 anos e 12 meses & Avalia as competências de compreensão auditiva e expressão verbal oral (nos domínios semântico e morfossintático), bem como a metalinguagem (nos domínios semântico, morfossintático e fonológico). & Variável\\
\lspbottomrule
\end{tabular}}
\end{sidewaystable}

\begin{sidewaystable}
\resizebox{\linewidth}{!}{
\centering
\caption{Instrumentos para avaliação da linguagem oral em português europeu (cont.)}
\label{tab:viana_e}
\begin{tabular}{p{0.4\textwidth}p{0.20\textwidth}p{0.1\textwidth}p{0.25\textwidth}p{0.1\textwidth}}
\lsptoprule
Nome do Instrumento                                                                                                               & Fonte                                                                                           & Idade alvo            & Áreas/competências avaliadas                                                                                                                                                                                                   & Tempo de aplicação \\
\midrule
RALF– Rastreio de Linguagem e Fala & \cite{mendes_etal2015} & 3--5 anos e 11 meses & Avalia as competências de compreensão auditiva, de expressão verbal oral, de metalinguagem e as competências fonético-fonológicas. & Variável\\
TAS- Teste de Avaliação Semântica & \cite{suakay2015} & 7--13 anos e 11 meses & Avalia possíveis perturbações de linguagem, na área da semântica e é constituído por quatro provas: relações sintagmáticas, campo lexical, sinonímia e antonímia e paronímia. & 30 min.\\
Parafasia – Avaliação e Intervenção em Afasias & \cite{carreteiro2015} & Não referida & Avalia perturbações da linguagem (afasias). & Variável\\
Inventários do Desenvolvimento das Habilidades Comunicativas de MacArthur-Bates para o Português Europeu – forma reduzida –nível I  & \cite{fenson_etal2000}. Adaptado por \cite{frota_etal2016} & 8--18 meses. & Avalia vocabulário recetivo e vocabulário expressivo. & 5--10 min.\\
Inventários do Desenvolvimento das Habilidades Comunicativas de MacArthur-Bates para o Português Europeu – forma reduzida –nível II  & \cite{fenson_etal2000}. Adaptado por \cite{frota_etal2016} & 16--30 meses & Avalia vocabulário recetivo e expressivo, formação de palavras complexas e habilidade de produção de combinações de palavras. 
 & 5--10 min.\\
\lspbottomrule
\end{tabular}}
\end{sidewaystable}

\begin{sidewaystable}
\resizebox{\linewidth}{!}{
\centering
\caption{Instrumentos para avaliação da linguagem oral em português europeu (cont.)}
\label{tab:viana_f}
\begin{tabular}{p{0.4\textwidth}p{0.20\textwidth}p{0.1\textwidth}p{0.25\textwidth}p{0.1\textwidth}}
\lsptoprule
Nome do Instrumento                                                                                                               & Fonte                                                                                           & Idade alvo            & Áreas/competências avaliadas                                                                                                                                                                                                   & Tempo de aplicação \\
\midrule
Confira & \cite{castro_etalprep} & 3--9 anos & Avalia a consciência fonológica (consciência de palavra, consciência de sílaba, consciência de acento e consciência de fonema) & Aprox. 30 min.\\
Prova de Avaliação da Consciência Fonológica (PACOF) & \cite{meira_etal2017} & 5--6 anos & Avalia a consciência fonológica nas dimensões sílaba, unidade intrassilábica e fonema. & Aprox. 45 min.\\
Inventários do Desenvolvimento das Habilidades Comunicativas de MacArthur-Bates (PT-CDI-WG)  & \cite{fenson_etal2007}. Adaptado por \cite{viana_etal2017} & 8--15 meses & Avalia o desenvolvimento da comunicação e da linguagem, permitindo captar os primeiros sinais de compreensão que se manifestam através de gestos, a progressiva aquisição do vocabulário e a emergência da gramática. & Variável\\
Inventários do Desenvolvimento das Habilidades Comunicativas de MacArthur-Bates (PT-CDI-WS)  & \cite{fenson_etal2007}. Afaptado por \cite{viana_etal2017} & 16--30 mses & Avalia o desenvolvimento da linguagem expressiva, a sintaxe e a gramática. & Variável\\
\lspbottomrule
\end{tabular}}
\end{sidewaystable}

Como já foi referido, nas últimas décadas assistiu-se, em Portugal, a um crescente interesse pela área da avaliação da linguagem oral e, dentro desta, à construção de instrumentos de avaliação. Uma parte não despicienda destes instrumentos foi desenvolvida e/ou adaptada no âmbito de trabalhos académicos, cujo tempo de conclusão é relativamente curto. A produção de instrumentos de avaliação da linguagem é uma tarefa complexa, que exige a colaboração de equipas multidisciplinares, nem sempre disponíveis aquando da realização de trabalhos académicos conducentes a grau, pelo que, apesar de disponíveis, nem todos os instrumentos listados na Tabela \ref{tab:viana_1} podem ser considerados validados para a população portuguesa. Assim sendo, está em curso \citep{silva2017} uma análise das caraterísticas psicométricas destas provas. A este propósito importa referir os contributos de \citet{mccauleyswitcher1984} e de \citet{friberg2010}, que sugeriram um conjunto de parâmetros a contemplar nesta análise.

\clearpage
{\sloppy
\printbibliography[heading=subbibliography,notkeyword=this]
}
\end{document}