\documentclass[output=paper]{LSP/langsci} 
\author{Maria João Freitas\affiliation{Universidade de Lisboa, Faculdade de Letras, Centro de Linguística}
}
\title{Aquisição da fonologia em língua materna: a sílaba}  
\abstract{\noabstract}
\ChapterDOI{10.5281/zenodo.889423}
\maketitle
\begin{document}
\section{O conceito de sílaba} 
\label{sec:freitas_conceito}

Neste capítulo, centrar-nos-emos na unidade gramatical \textit{sílaba} e nas produções de crianças que nos permitem observar a aquisição gradual da sua estrutura interna, ao longo dos primeiros 5/6 anos de vida, altura em que o desenvolvimento fonológico típico se encontra, em média, concluído. Centrar-nos-emos nos dados disponibilizados por crianças portuguesas e por crianças brasileiras, o que nos permitirá dar conta da aquisição desta estrutura no português europeu e no português do Brasil. A referência a outras línguas surgirá sempre que pertinente, demonstrando a relevância de uma perspetiva comparada no contexto da análise linguística.

A sílaba está integrada na componente fonológica da gramática, cujo estudo implica, no contexto da fonologia não-linear \citep{goldsmith1995,delacy2007}, a referência a:

\begin{enumerate}[label=\alph*)]
\item \textit{constituintes de natureza prosódica} (a sílaba, o pé, a palavra prosódica, o sintagma fonológico e o sintagma entoacional;  \citealt{nesporvogel1986}).
\item \textit{constituintes de natureza segmental} (os segmentos e os traços distintivos, permitindo, os últimos, a representação das propriedades intrínsecas dos segmentos).
\end{enumerate}

A sílaba é um dos constituintes prosódicos responsáveis pela construção do ritmo nos enunciados de fala que usamos para comunicar, sendo de identificação intuitiva por parte dos falantes: ao perguntarmos, a alunos universitários sem treino em análise fonológica, quantas unidades tem a palavra \textit{livraria}, a resposta mais frequente é ``quatro'' (sílabas) e não ``oito'' (segmentos). Respostas deste tipo mostram que a unidade preferencial para a segmentação da cadeia fónica é a sílaba e não o segmento.

Contrariamente à maior parte dos constituintes fonológicos supramencionados, cujo uso decorre de propostas teóricas no contexto da fonologia não-linear,\footnote{Para uma síntese da História da Fonologia no século XX, consulte-se \citealt[capítulo 1]{mateus_etal2005}.} a sílaba surge já mencionada nas gramáticas antigas. Veja-se, a título ilustrativo, a seguinte citação de João de Barros:

\begin{quote}
O proveito de saber muitas sílabas [\ldots] E o trabalho que se néstas levar será [de] gram proveito para os mininos, ca lhe fáz a língua tam solta e costumáda a ésta generalidáde de sílabas que se nam empeça em a pronunçiaçám das dições. \citep[250]{barros1540}
\end{quote}

A sílaba goza, assim, de uma longa tradição como unidade de descrição da estrutura sonora das línguas.\footnote{Para citações retiradas de outras gramáticas do português, consulte-se \citet{freitassantos2001}.} No entanto, apesar de referida nas várias gramáticas ao longo dos séculos e nos estudos sobre a fonologia das línguas, o conceito não é muito produtivo nas descrições fonológicas dos três primeiros quartéis do século XX.  Com o advento da \isi{fonologia não-linear}, nos anos 70/80, o trabalho sobre a prosódia das línguas intensificou-se, surgindo, entre outros, o modelo de representação da estrutura silábica de \citet{selkirk1984}, o modelo de \textit{Ataque-Rima},\is{sílaba!rima}\is{sílaba!ataque} no qual a sílaba ($\sigma$) se encontra organizada em constituintes internos hierarquicamente organizados (veja-se a representação na Figura \ref{fig:freitas:ataque_rima}):\is{sílaba!rima}\footnote{Para mais informação sobre o funcionamento da sílaba em português, consulte-se \citet{mateusdandrade2000}, \citet{mateus_etal2005} e \citet{freitassantos2001} para o português europeu e \citet{bisol2005} para o português do Brasil.}

\begin{enumerate}[label=\alph*)]
\item o \textit{Ataque}\is{sílaba!ataque}\footnote{\is{sílaba!ataque}Em português do Brasil, é usada a designação sinónima \textit{Onset}. A designação \textit{ramificado/a} é normalmente sinónima de \textit{complexo/a}.} (A) domina a(s) consoante(s) na margem esquerda da sílaba; 
\item a \textit{Rima}\is{sílaba!rima} (R) domina o \textit{Núcleo}\is{sílaba!núcleo} (Nu) e a \textit{Coda} (Cd), sendo que esta última domina a(s) consoante(s) na margem direita da sílaba. 
\end{enumerate}

\begin{figure}
\begin{forest}
  [$\sigma$
    [A[X,tier=X[\textipa{[p}]]]
    [R
      [Nu[X,tier=X[\textipa{a}]]]
      [Cd[X,tier=X[\textipa{S]}]]]
] ]
\end{forest}
\quad
\begin{forest}
  [\textit{nível da sílaba}, no edge
    [\textit{nível da rima}, no edge[, no edge[\textit{nível do esqueleto}, no edge[\textit{nível segmental}, no edge]]]]
] ]
\end{forest}
\caption{Estrutura da sílaba no modelo de ‘Ataque-Rima’.}
  \label{fig:freitas:ataque_rima}
\end{figure}

Todos os constituintes terminais (ataque,\is{sílaba!ataque} núcleo,\is{sílaba!núcleo} coda)\is{sílaba!coda} estão associados a posições rítmicas, representadas no nível do esqueleto (conforme a Figura 1), as quais dominam as unidades segmentais (consoantes (C), vogais (V) ou semivogais (G)).\footnote{C representa \textit{consoante}; V representa \textit{vogal}; G representa \textit{semivogal} ou \textit{glide}.} Estas posições rítmicas permitem, por exemplo, distinguir vogais breves (uma posição de esqueleto) de vogais longas (duas posições de esqueleto), em línguas como o inglês\il{inglês} ou o holandês,\il{holandês} em que a duração da vogal implica contraste fonológico.

Todos os constituintes podem assumir os formatos \textit{não ramificado} (uma só posição de esqueleto associada ou a um só segmento ou a uma posição vazia) e \textit{ramificado} (duas posições de esqueleto associadas a dois segmentos). No português, a tipologia de constituintes silábicos é a que se apresenta na Tabela \ref{tab:freitas_tipologia}.\footnote{Algumas exceções à generalização de a Rima ramificada ser constituída por Núcleo\is{sílaba!núcleo}\is{sílaba!núcleo}+Coda\is{sílaba!coda} são referidas na literatura: vejam-se os casos de \textit{pe\underline{rs}petiva} e de \textit{so\underline{ls}tício}, com duas consoantes em Coda.\is{sílaba!coda} No caso do português do Brasil, considera-se que as sequências fonológicas \textipa{/ns/} em final de sílaba (\textit{monstro} \textipa{[‘m\~oStRu]}) constituem Codas\is{sílaba!coda} ramificadas \citep{bisol2005}. Em português europeu, esta nasalidade fonológica é representada sob a forma de um autossegmento nasal no domínio do Núcleo,\is{sílaba!núcleo}\is{sílaba!núcleo} portanto, a Coda\is{sílaba!coda} é considerada não ramificada \citep{mateusdandrade2000,mateus_etal2005}.}

\begin{table}
\begin{tabular}{lllll}
\lsptoprule
\multicolumn{4}{l}{Constituintes Silábicos}                                                          & Exemplos \\
\midrule
\multirow{3}{*}{\textbf{Ataque}} & \multirow{2}{*}{não ramificado} & \multicolumn{2}{l}{simples}              & \textit{\underline{d}á}       \\
                        &                                 & \multicolumn{2}{l}{vazio}                & \textit{\underline{~}é}       \\
                        & \multicolumn{3}{l}{ramificado}                                             & \textit{\underline{cr}uz}     \\
\multirow{3}{*}{\textbf{Rima}}   & \multirow{2}{*}{não ramificada} & \multirow{2}{*}{\textbf{Núcleo}} & não ramificado & \textit{p\underline{á}}       \\
                        &                                 &                         & ramificado     & \textit{p\underline{ai}}      \\
                        & ramificada                      & \multicolumn{2}{l}{Núcleo$+$\textbf{Coda}}          & \textit{p\underline{az}}     \\
\lspbottomrule
  \end{tabular}
  \caption{Tipologia de constituintes silábicos em português}
  \label{tab:freitas_tipologia}
\end{table}

No contexto da Fonologia Prosódica \citep{nesporvogel1986}, a sílaba é responsável pelo primeiro nível de estruturação prosódica das línguas, organizando sequências de segmentos (consoantes, vogais, semivogais) em unidades melódicas intuitivamente identificadas pelos falantes. Esta identificação intuitiva das sílabas regista-se quer em crianças quer em sujeitos não alfabetizados. Veja-se o episódio em (\ref{ex:freitas_laura}), no qual uma criança com 2 anos de idade produz a forma verbal alvo \textit{arregaçar}, identificando as suas fronteiras silábicas\largerpage[2] através do uso de silêncios, marcados na transcrição com // \citep{freitas1997,freitas_etal2007}.

Numa situação em que a mãe da Laura, falando com outro adulto, \textit{arregaça} as mangas do casaco da criança, enquanto lho veste, referindo a forma verbal que a criança usava para designar o movimento:

\begin{exe}
\protectedex{
\ex\label{ex:freitas_laura} Mãe: Institivamente, é mangar.\\Laura: Não, regaçare.\\Mãe: a//rre//ga//çar\\Laura: Não, re//ga//ça//re \jambox{(Laura, 2 anos)}
}
\end{exe}

Nas secções que se seguem, apresentaremos argumentos empíricos que mostram a adequação do modelo de \textit{Ataque-Rima}\is{sílaba!ataque} à descrição dos eventos relativos ao desenvolvimento fonológico infantil, ilustrando a relação estreita entre os segmentos e os seus estatutos silábicos e a relevância desta relação para a avaliação fonológica em contextos clínico e educacional.

\section{Aquisição da estrutura silábica}
\label{sec:freitas_aquisicao}

Quando ouvimos os primeiros enunciados de fala produzidos por crianças em processo de aquisição de diferentes línguas, verificamos que estes exibem muitas propriedades comuns, as quais nos levam a não conseguir identificar qual a língua materna da criança. Trata-se ou de produções monossilábicas articuladas isoladamente (português europeu: \textit{dá} \textipa{[\textprimstress da]} Marta: 1;2; \textit{quer} \textipa{[\textprimstress kE]} João: 1;0) ou de reduplicações,\is{reduplicação} preferencialmente dissilábicas, com ou sem contraste acentual (português europeu (Freitas 1997): \textit{Pedro} \textipa{[tete]} João: 0;11; \textit{boneca} \textipa{[ne\textprimstress nE]} Inês: 1;5). Estas primeiras produções de formas monossilábicas ou de reduplicações\is{reduplicação} apresentam, normalmente, sílaba(s) com formato CV, constituindo este o estádio inicial de produção de itens lexicais da língua alvo. Vejam-se os exemplos na Tabela \ref{tab:freitas_reduplicacoes},\footnote{As formas entre // correspondem à transcrição da forma alvo, não ao formato fonológico da palavra.} retirados de \citet[2--5]{johnsonreimers2010}.

\begin{table}[t]
\begin{tabular}{lllll}
\lsptoprule
Língua-alvo & Palavra-alvo & Produção da criança \\
\midrule
Francês\il{francês} & \textipa{/ne/} \textit{nariz} & \textipa{[nene]} \\
Árabe\il{árabe} (Jordânia) & \textipa{/bo\textlengthmark t/} \textit{sapatos} & \textipa{[bobo]} \\
Inglês\il{inglês} & \textipa{/bl\ae nk\textschwa t/} \textit{cobertor} & \textipa{[baba]} \\
Alemão\il{alemão} & \textipa{/bEr/} \textit{urso} & \textipa{[bebe]} \\
Japonês\il{japonês} & \textipa{/dZu\textlengthmark su/} \textit{sumo} & \textipa{[dZudZu]} \\
Maltês\il{maltês} & \textipa{/gazaza/} \textit{chupeta} & \textipa{[gaga]} \\
Sueco\il{sueco} & \textipa{/tak/} \textit{obrigado} & \textipa{[dada]} \\
\lspbottomrule
  \end{tabular}
  \caption{Reduplicações com formato dissilábico em diferentes línguas }
  \label{tab:freitas_reduplicacoes}
\end{table}

Os dados acima ilustram as semelhanças entre os primeiros enunciados de fala em várias línguas e são consistentes com a predição de \citet{jakobson1941}, segundo a qual a estrutura silábica universal CV é a primeira a estar disponível no sistema linguístico da criança. Vários estudos desenvolvidos para várias línguas do mundo demonstraram que assim é (para revisão bibliográfica, consulte-se \citealt{bernhardtstemberger1998,fikkert2007,johnsonreimers2010}). A presença precoce de sílabas do tipo CV nos primeiros enunciados infantis, a sua alta frequência nas línguas do mundo e o facto de haver línguas, como as da família Bantu, que usam quase exclusivamente este formato, levam os fonólogos a considerá-la o formato silábico universal, não marcado. A estrutura silábica vai\x-se tornando cada vez mais complexa ao longo do desenvolvimento infantil, sendo um dos primeiros constituintes disponíveis nos dados da produção para observar a estruturação gradual do conhecimento linguístico.

No entanto, tem sido registado um contraste na aquisição de diferentes línguas que permite questionar a exclusividade de CV nas primeiras produções: algumas línguas apenas permitem o formato CV nas primeiras produções, como é o caso do inglês\il{inglês} e do holandês,\il{holandês} enquanto outras permitem CV e V nas primeiras produções, como é o caso do português europeu, do português brasileiro, do hebraico\il{hebraico} ou do francês.\il{francês} Mais ainda, no caso do holandês,\il{holandês} por exemplo, as crianças produzem uma consoante na posição de ataque\is{sílaba!ataque} vazio na estrutura-alvo V (V $\rightarrow$CV), estratégia não ativada no português europeu. Vejam-se os exemplos em (\ref{ex:freitas_holandes}) e (\ref{ex:freitas_portugues}):\largerpage[2]

\ea\label{ex:freitas_holandes}
Dados do holandês\il{holandês} \citep{fikkert1994}
 \ea{
\begin{tabbing}
  word \quad \= phonol \enskip \= arrow \enskip \= phonet \quad \= speaker (x,x) \quad \= meaning \kill
  daar \> \textipa{/\textprimstress da\textlengthmark r/} \> $\rightarrow$ \> \textipa{[\textprimstress da]} \> (Jarmo: 1;4) \> \textit{lá}\\
  poes \> \textipa{/\textprimstress pu\textlengthmark s/} \> $\rightarrow$ \> \textipa{[\textprimstress pu]} \> (Jarmo: 1;5) \> \textit{gata}
  \end{tabbing}
}
\ex{
\begin{tabbing}
  word \quad \= phonol \enskip \= arrow \enskip \= phonet \quad \= speaker (x,x) \quad \= meaning \kill
  auto \> \textipa{/\textprimstress o\textlengthmark to\textlengthmark/} \> $\rightarrow$ \> \textipa{[\textprimstress ta\textlengthmark to\textlengthmark]} \> (Jarmo: 1;6) \> \textit{carro}\\
  apie \> \textipa{/\textprimstress a\textlengthmark pi\textlengthmark/} \> $\rightarrow$ \> \textipa{[\textprimstress ta\textlengthmark pi\textlengthmark]} \> (Jarmo: 1;7) \> \textit{macaquinho}
  \end{tabbing}
}
\z
\z

\ea\label{ex:freitas_portugues}
Dados do português europeu \citep{freitas1997}
 \ea{
\begin{tabbing}
  word \quad \= phonol \enskip \= arrow \enskip \= phonet \quad \= speaker (x,x) \kill
  papá \> \textipa{/p5\textprimstress pa/} \> $\rightarrow$ \> \textipa{[\textprimstress pa]} \> (João: 0;10)\\
  quer \> \textipa{/\textprimstress kER/} \> $\rightarrow$ \> \textipa{[\textprimstress kE]} \> (João: 1;0)
  \end{tabbing}
}
\ex{
\begin{tabbing}
  word \quad \= phonol \enskip \= arrow \enskip \= phonet \quad \= speaker (x,x) \kill
  água \> \textipa{/\textprimstress ag\super w5/} \> $\rightarrow$ \> \textipa{[\textprimstress aB5]} \> (João: 0;11)\\
  é \> \textipa{/\textprimstress E/} \> $\rightarrow$ \> \textipa{[\textprimstress E]} \> (João: 1;0)
  \end{tabbing}
}
\z
\z

Tanto no caso das línguas como o holandês\il{holandês} e o inglês\il{inglês} (exclusividade inicial de CV) como nas línguas como o português europeu e o português brasileiro (emergência\footnote{O conceito de emergência de uma estrutura na produção não é sinónimo da sua aquisição. \citet{matzenauer1990} usa a seguinte escala de aquisição, cujo uso sugerimos: até 50\%, estrutura não adquirida; entre 51\% e 75\%, estrutura em aquisição; entre 76\% e 85\%, estrutura adquirida mas não estabilizada; 86\% e 100\%, estrutura estabilizada.} simultânea de CV e V no estádio inicial), apenas formatos silábicos não ramificados estão disponíveis no estádio inicial: (i) ataque\is{sílaba!ataque} não ramificado (só simples, em línguas como o holandês;\il{holandês} simples e vazio, em línguas como o português europeu e o português brasileiro); (ii) núcleo\is{sílaba!núcleo} não ramificado; (iii) rima\is{sílaba!rima} não ramificada (ausência de coda\is{sílaba!coda} nos formatos iniciais).

A disponibilização inicial dos formatos CV e V permite propor o seu estatuto não marcado em português europeu \citep{costafreitas1999}. O primeiro argumento é o seu uso como preenchedor prosódico\is{preenchedores prosódicos} na periferia esquerda de palavras monossilábicas, que assume exclusivamente o formato V:

\ea\label{ex:freitas_preenchedores}
Preenchedores prosódicos\is{preenchedores prosódicos} (dados em \citealt{freitas1997})\\
\begin{tabbing}
  word \quad \= phonetictranscription \enskip \= speaker \kill
  \textit{pato} \> \textipa{[\textprimstress p5]} / \textipa{[5\textprimstress p5]} \> (João: 0;10)\\
    \textit{papa} \> \textipa{[\textprimstress pa]} / \textipa{[5\textprimstress p5]} / \textipa{[1\textprimstress p5]} \> (João: 0;11)\\
      \textit{Bambi} \> \textipa{[\textprimstress b5]} / \textipa{[5\textprimstress b5]} \> (Inês: 1;4)\\
        \textit{não} \> \textipa{[\textprimstress n5]} / \textipa{[5\textprimstress n5]} \> (Inês: 1;1, 1;3, 1;5)\\
  \textit{quer} \> \textipa{[\textprimstress kE]} / \textipa{[5\textprimstress kE]} / \textipa{[5\textprimstress kE]} \> (Inês 1;5)
  \end{tabbing}
\z

A presença de material segmental na periferia esquerda da palavra com diferentes formatos tem sido observada na aquisição de várias línguas do mundo \citep{peters2001}, incluindo o português brasileiro \citep{scarpa1999} e o português europeu \citep{freitas1997}. No caso do português europeu, apenas o formato V emerge nesta posição (associado a \textipa{[5]} ou a \textipa{[1]}). Tal facto argumenta a favor da natureza não marcada do formato silábico V, uma vez que estruturas epentéticas são, normalmente, estruturas não marcadas nas línguas. As interpretações mais recorrentes para estas produções iniciais na periferia esquerda da palavra são as seguintes: (i) \textit{proto-morfemas}: formas precoces que indiciam já a futura presença de determinantes naquela posição; (ii) \textit{preenchedores prosódicos}:\is{preenchedores prosódicos} produções que contribuem para a construção de um ritmo de fala que privilegia padrões iniciais dissilábicos. No caso dos dados do português europeu transcritos em \ref{ex:freitas_reconstrucao}, as duas interpretações são plausíveis para os nomes; no entanto, este material segmental ocorre também com verbos, que não co-ocorrem com determinantes à sua esquerda, e com advérbios, que não co-ocorrem com determinantes nos casos em que são o núcleo\is{sílaba!núcleo} do constituinte. Estes dados têm sido interpretados como argumentando a favor do estatuto prosódico e não morfossintático destas vogais iniciais nos dados da aquisição do português europeu. 

Por outro lado, e argumentando ainda a favor da natureza não marcada de CV e V no português europeu, estruturas não marcadas são normalmente usadas na aquisição quando a criança se confronta com uma estrutura problemática. O uso de V ocorre em português europeu nestes contextos, o que corresponde a substituição de estruturas-alvo problemáticas por ataque\is{sílaba!ataque} vazio, não sendo esta estratégia produtiva em línguas como o inglês\il{inglês} ou o holandês\il{holandês} \citep{fikkert1994,bernhardtstemberger1998}. Vejam-se os exemplos em (\ref{ex:freitas_uso_de_v}):

\ea\label{ex:freitas_uso_de_v}
Uso de V (ataque\is{sílaba!ataque} vazio) face a estruturas alvo problemáticas
 \ea{
 Fricativas\label{ex:freitas_uso_de_v_fric}
\begin{tabbing}
  wordword \quad \= phonetictrans \enskip \= speaker \kill
  \textit{vês} \> \textipa{[\textprimstress eS]} \> (Marta: 1;3)\\
  \textit{zebra} \> \textipa{[\textprimstress Eb5]} \> (Luís: 1;9)\\
  \textit{jipe} \> \textipa{[\textprimstress eS]} \> (João: 2;2)
  \end{tabbing}
}
\ex{
Líquidas\label{ex:freitas_uso_de_v_liq}
\begin{tabbing}
  wordword \quad \= phonetictrans \enskip \= speaker \kill
  \textit{lobo} \> \textipa{[\textprimstress opu]} \> (João: 2;8)\\
  \textit{mulher} \> \textipa{[mu\textprimstress E]} \> (Raquel: 2;1)\\
  \textit{tirar} \> \textipa{[ti\textprimstress a]} \> (Raquel: 2;1)
  \end{tabbing}
}
\ex{
Ataques\is{sílaba!ataque} ramificados\label{ex:freitas_uso_de_v_ataq}
\begin{tabbing}
  wordword \quad \= phonetictrans \enskip \= speaker \kill
  \textit{bicicleta} \> \textipa{[pisi\textprimstress Et5]} \> (Luís 1;11)\\
  \textit{flor} \> \textipa{[\textprimstress ol5]} \> (Inês 1;9)\\
  \textit{Pedro} \> \textipa{[\textprimstress pEu]} \> (João 2;4)
  \end{tabbing}
}
\z
\z

Os dados em \ref{ex:freitas_uso_de_v} ilustram o uso do padrão V, com Ataque\is{sílaba!ataque} vazio, face a estruturas problemáticas: no padrão V, o ataque\is{sílaba!ataque} vazio substitui ou ataques\is{sílaba!ataque} ramificados (cf. \ref{ex:freitas_uso_de_v_ataq}) ou ataques\is{sílaba!ataque} não ramificados preenchidos por fricativa ou líquida (cf. \ref{ex:freitas_uso_de_v_fric} e \ref{ex:freitas_uso_de_v_liq}), consoantes de classes naturais de aquisição tardia \citep{matzenauer1990,freitas1997,costa2010,amorim2014}.

O uso de constituintes silábicos nos estudos sobre aquisição é crucial para uma avaliação rigorosa do desenvolvimento fonológico infantil, em contextos clínico ou educacional. Sabe-se, desde os anos 90, que as crianças não adquirem os segmentos independentemente do seu estatuto prosódico. No caso da sílaba, um mesmo segmento pode já ser produzido numa dada posição silábica mas não noutra(s). No exemplo na Tabela \ref{tab:freitas_alvos},\footnote{O estatuto de Coda\is{sílaba!coda} das líquidas em final de sílaba é discutido em \citet{fikkert1994}, para o holandês,\il{holandês} em \citet{freitas1997}, para o português europeu, e em \citet{lamprecht_etal2004}, para o PB.} o Luís já produz \textipa{[R]} em Ataque\is{sílaba!ataque} não ramificado mas ainda não produz o mesmo segmento em Coda\is{sílaba!coda} nem em Ataque\is{sílaba!ataque} ramificado. Exemplos deste tipo, disponíveis para várias línguas do mundo \citep{fikkert1994,bernhardtstemberger1998,fikkert2007}, incluindo o português europeu e o português do Brasil \citep{freitas1997,lamprecht_etal2004}, mostram que não é um obstáculo articulatório que está a afetar a produção de \textipa{[R]}, uma vez que a criança já o produz em Ataque\is{sílaba!ataque} não ramificado: o que está em causa é a não disponibilização de determinadas posições silábicas (neste caso, a Coda\is{sílaba!coda} e o Ataque\is{sílaba!ataque} ramificado) no sistema fonológico da criança.

\begin{table}
\begin{tabular}{llll}
\lsptoprule
Ataque\is{sílaba!ataque} simples & \textit{amarelo} & \textipa{[m5R\textprimstress Ew]} & Luís: 1;9\\
Coda & \textit{barco} & \textipa{[\textprimstress baku]} & Luís: 1;9\\
Ataque\is{sílaba!ataque} ramificado & \textit{quatro} & \textipa{[\textprimstress k\super watu]} & Luís: 1;9\\
\lspbottomrule
  \end{tabular}
  \caption{Alvos [ɾ] em diferentes posições silábicas no português europeu \citep{freitas1997}}
  \label{tab:freitas_alvos}
\end{table}

Dados congéneres em várias línguas têm permitido argumentar a favor do processamento \textit{top-down} e não \textit{bottom-up} da informação fonológica no processo de desenvolvimento fonológico: a disponibilização das categorias prosódicas, representadas em níveis mais altos da hierarquia fonológica \citep{nesporvogel1986}, podem restringir a aquisição dos segmentos, unidades representadas em níveis mais baixos da hierarquia fonológica. No caso específico dos constituintes silábicos, sabemos que a sua (não) consideração tem impacto na avaliação fonológica infantil. Vejam-se os seguintes exemplos do Luís aos 1;9, na Tabela \ref{tab:freitas_recurso}, relativos à produção da lateral alveolar.
\begin{table}
\resizebox{\textwidth}{!}{
\begin{tabular}{lp{5cm}p{5cm}}
\lsptoprule
Produções & Avaliação sem recurso aos constituintes silábicos       & Avaliação com recurso aos constituintes silábicos                          \\
\midrule
\textit{leite} \textipa{[\textprimstress et1]}    & \multirow{4}{*}{\parbox{5cm}{50\% de sucesso na produção da lateral}} & \multirow{2}{*}{\parbox{5cm}{100\% de sucesso na produção da lateral em Ataque\is{sílaba!ataque} simples}} \\
\textit{bolacha} \textipa{[\textprimstress las5]}  &                                                         &                                                                            \\\cline{3-3}
\textit{azul} \textipa{[a\textprimstress suw1]}     &                                                         & \multirow{2}{*}{\parbox{5cm}{0\% de sucesso na produção da lateral em Coda}}             \\
\textit{Calvin} \textipa{[\textprimstress kavin]}   &                                                         &                        \\
\lspbottomrule
\end{tabular}}
  \caption{Avaliação com/sem recurso aos constituintes silábicos (Luís: 1;09)}
  \label{tab:freitas_recurso}
\end{table}

Os dados na Tabela \ref{tab:freitas_recurso} mostram que o facto de um segmento estar disponível no inventário segmental da criança não implica a sua produção em todos os contextos silábicos, sendo o desenvolvimento das posições silábicas crucial para a produção de cada consoante em cada ponto da estrutura silábica da palavra. 

Uma avaliação fonológica que se centre exclusivamente no inventário segmental penaliza a criança e pode levar à definição de estratégias de intervenção terapêutica ou educacional pouco eficazes: neste caso, os 50\% não apontam para uma estrutura específica e o terapeuta ou o professor poderão centrar-se no trabalho sobre \textipa{[l]} nos vários contextos, quando a estrutura a trabalhar preferencialmente deverá ser a Coda\is{sílaba!coda} (0\% de sucesso) e não o Ataque\is{sílaba!ataque} simples (100\% de sucesso). Uma avaliação que tenha em conta a relação entre segmento e sua localização silábica beneficiará a criança, promovendo a relação com esta e com os pais por permitir identificar aspetos positivos no desenvolvimento da criança, levando o terapeuta ou o professor a centrar o trabalho na estrutura fonológica deficitária. 

Nas secções que se seguem, focar-nos-emos na descrição do desenvolvimento fonológico infantil em função dos constituintes Ataque\is{sílaba!ataque} (Secção \ref{subsec:freitas_ataque}) e Rima\is{sílaba!rima} (Secção \ref{subsec:freitas_rima}).

\subsection{Ataque}
\label{subsec:freitas_ataque}

Os Ataques\is{sílaba!ataque} não ramificados (simples e vazios) estão disponíveis desde o estádio inicial de aquisição do português europeu e do português do Brasil. Tal não acontece no holandês\il{holandês} nem no inglês,\il{inglês} como já afirmámos acima. Quando olhamos para as classes naturais de segmentos associados ao Ataque\is{sílaba!ataque} não ramificado simples, outras assimetrias entre línguas são identificadas. 

\begin{figure}
\raggedright
\noindent\textbf{Holandês} \citep{fikkert1994}
\resizebox{\textwidth}{!}{
\begin{tabular}{ll}
\textit{Estádio I} & Ataques\is{sílaba!ataque} simples associados a oclusiva \\
\textit{Estádio II} & Ataque vazio \\
\textit{Estádio III} & outros tipos de Ataque simples \\
\textit{~~~~~~Estádio IIIa} & nasais \\
\textit{~~~~~~Estádio IIIb} & outros segmentos (fricativa $>>$líquida ou líquida$>>$fricativa)
\end{tabular}}
~\\
\noindent\textbf{Português} \citep{freitas1997,lamprecht_etal2004}
\resizebox{\textwidth}{!}{
\begin{tabular}{ll}
\textit{Estádio I} & Ataques simples associados a oclusiva e a consoante nasal \\
 & Ataque vazio \\
\textit{Estádio II} & outros tipos de Ataque simples \\
 & (fricativa$>>$líquida ou líquida$>>$fricativa)
\end{tabular}}
\caption{Ataque não ramificado}
\label{fig:freitas_ataque_nao_ram}
\end{figure}

Na Figura \ref{fig:freitas_ataque_nao_ram},\footnote{Usamos oclusiva como sinónimo de oclusiva oral e consoante nasal como sinónimo de oclusiva nasal. Em português brasileiro, o termo para \textit{oclusiva} é \textit{plosiva}.} é possível verificar que as crianças portuguesas e as brasileiras têm, desde o estádio inicial, oclusivas e consoantes nasais em ataque\is{sílaba!ataque} não ramificado, em co-ocorrência com ataques\is{sílaba!ataque} vazios. No holandês,\il{holandês} a progressão é mais lenta: no estádio inicial, apenas oclusivas são possíveis; progressivamente, emergem os ataques\is{sílaba!ataque} vazios e as consoantes nasais. Em ambas as línguas, os segmentos das classes mais problemáticas, as fricativas e as líquidas, são adquiridos posteriormente \citep{matzenauer1990,freitas1997,costa2010,amorim2014}. Esta assimetria estará certamente relacionada com diferentes propriedades do \textit{input} linguístico das crianças, tópico não integralmente explorado na literatura sobre o assunto.

Embora não seja este o único percurso possível (veja-se, para o francês,\il{francês} \citealt{rose200,almeida2011}), o Ataque\is{sílaba!ataque} tende a ser o último constituinte a ramificar no desenvolvimento silábico em várias línguas do mundo \citep{bernhardtstemberger1998,fikkert2007,demuth2009}. Trata-se, normalmente, de sequências de uma obstruinte em C\textsubscript{1} (oclusiva ou fricativa) seguida de uma líquida em C\textsubscript{2} (vibrante ou lateral) – \textit{\underline{br}uxa}, \textit{\underline{bl}usa}, \textit{li\underline{vr}o}, \textit{\underline{fl}or}. A estratégia mais comum para lidar com esta estrutura de aquisição tardia, a omissão de C\textsubscript{2}, está ilustrada em (\ref{ex:freitas_omissao_c2}) para várias línguas (dados compilados em \citealt{freitas2003}, com identificação das fontes):

\ea\label{ex:freitas_omissao_c2}
Omissão de C\textsubscript{2}
 \ea{
\begin{tabbing}
  language names \quad \= wordss \quad \= phonet \quad \= speaker \kill
  holandês\il{holandês} \> \textit{trein} \> \textipa{[tein]} \> (Leon: 1;10)
  \end{tabbing}
}
\ex{
\begin{tabbing}
  language names \quad \= wordss \quad \= phonet \quad \= speaker \kill
  inglês\il{inglês} \> \textit{please} \> \textipa{[piz]} \> (Gitanjali: 2;03--1;10)
  \end{tabbing}
}
\ex{
\begin{tabbing}
  language names \quad \= wordss \quad \= phonet \quad \= speaker \kill
  francês\il{francês} canadiano \> \textit{clé} \> \textipa{[ke]} \> (Théo: 2;04)
  \end{tabbing}
}
\ex{
\begin{tabbing}
  language names \quad \= wordss \quad \= phonet \quad \= speaker \kill
  alemão\il{alemão} \> \textit{brot} \> \textipa{[bo\textlengthmark]} \> (Naomi: 1;04)
  \end{tabbing}
}
\ex{
\begin{tabbing}
  language names \quad \= wordss \quad \= phonet \quad \= speaker \kill
  Castelhano \> \textit{truita} \> \textipa{[tutja]} \> (Laura: 3;10)
  \end{tabbing}
}
\z
\z

A estratégia ilustrada acima (C\textsubscript{1}C\textsubscript{2} $\rightarrow$ C\textsubscript{1}$\varnothing$) é também a mais usada pelas crianças portuguesas e brasileiras durante a aquisição dos ataques\is{sílaba!ataque} ramificados. No português do Brasil, corresponde a 91\% das estratégias registadas em \citet{ribas2004}. No caso do português europeu, uma outra estratégia é também comum: a inserção de vogal entre os dois membros do Ataque\is{sílaba!ataque} (C\textsubscript{1}C\textsubscript{2}V $\rightarrow$ C\textsubscript{1}VC\textsubscript{2}V). Vejam-se os dados em (\ref{ex:freitas_reconstrucao}).\newpage

\ea\label{ex:freitas_reconstrucao}
Estratégias de reconstrução\is{estratégia de reconstrução} de Ataque\is{sílaba!ataque} ramificado mais frequentes em português europeu
 \ea{
\begin{tabbing}
  omissao de blablablablablablablabla \quad \= wordss \quad \= phonet \quad \= speaker \kill
  omissão de C\textsubscript{2} (C\textsubscript{1}C\textsubscript{2} $\rightarrow$ C\textsubscript{1}$\varnothing$ \> \textit{prédio} \> \textipa{[\textprimstress pEdu]} \> (Raquel: 2;10)\\
 ~ \> \textit{abre} \> \textipa{[\textprimstress ab1]} \> (João: 2;0)
  \end{tabbing}
}
\ex{
\begin{tabbing}
  omissao de blablablablablablablabla \quad \= wordss \quad \= phonet \quad \= speaker \kill
  epêntese de vogal\is{epêntese vocálica} (C\textsubscript{1}C\textsubscript{2} $\rightarrow$ C\textsubscript{1}VC\textsubscript{2}V)\> \textit{cobra} \> \textipa{[\textprimstress kOb1R5]} \> (Pedro: 3;5)\\
 ~ \> \textit{livro} \> \textipa{[\textprimstress liv1Ru]} \> (Laura: 2;8)
  \end{tabbing}
}
\z
\z

Note-se que a epêntese de vogal\is{epêntese vocálica} na aquisição de Ataques\is{sílaba!ataque} ramificados é pouco atestada nas línguas do mundo; \citet{bernhardtstemberger1998} e \citet{johnsonreimers2010} apresentam apenas alguns exemplos da sua presença em árabe\il{árabe} da Jordânia (\textipa{/treen/} $\rightarrow$ \textipa{[tileen]}; \textipa{/klaab} $\rightarrow$ \textipa{[kalaab]}).

Numa perspetiva comparada, apresentam-se, na Figura \ref{fig:freitas_ataque_ram},\footnote{Para mais detalhe sobre o estádio II em português europeu, consulte-se \citet{freitas2003}.} os estádios de aquisição do Ataque\is{sílaba!ataque} ramificado em holandês\il{holandês} e em português europeu, sendo registada a ativação da \isi{epêntese vocálica} no português europeu (\textit{Estádio II}) mas não no holandês.\il{holandês}
\begin{figure}
\raggedright
\noindent\textbf{Holandês} \citep{fikkert1994}
\resizebox{\textwidth}{!}{
\begin{tabular}{ll}
\textit{Estádio 0}   & \isi{estratégia de seleção} (alvos lexicais com a estrutura não são possíveis)                                                                                             \\
\textit{Estádio I}   & o \isi{grupo consonântico} é reduzido ao primeiro elemento - C\textsubscript{1}C\textsubscript{2} $\rightarrow$ C\textsubscript{1}$\varnothing$                              \\
\textit{Estádio II}  & o \isi{grupo consonântico} é reduzido ao segundo elemento (líquida, nasal ou semivogal) - C\textsubscript{1}C\textsubscript{2} $\rightarrow$ $\varnothing$C\textsubscript{1} \\
\textit{Estádio III} & os dois elementos do grupo são produzidos - C\textsubscript{1}C\textsubscript{2}                                       
\end{tabular}}
~\\
\noindent\textbf{Português} \citep{freitas1997,lamprecht_etal2004}
\resizebox{\textwidth}{!}{
\begin{tabular}{ll}
\textit{Estádio 0} & \isi{estratégia de seleção} (alvos lexicais com a estrutura não são possíveis) \\
\textit{Estádio I} & o \isi{grupo consonântico} é reduzido à primeira consoante (C\textsubscript{1}C\textsubscript{2} $\rightarrow$ C\textsubscript{1}$\varnothing$) ou \\
 & a um Ataque\is{sílaba!ataque} vazio (C\textsubscript{1}C\textsubscript{2} $\rightarrow \varnothing\varnothing$) \\
\textit{Estádio II} & epêntese de vogal\is{epêntese vocálica} (C\textsubscript{1}C\textsubscript{2}V $\rightarrow$ C\textsubscript{1}VC\textsubscript{2}V) ou produção conforme ao alvo (C\textsubscript{1}C\textsubscript{2}) \\
\textit{Estádio III} & os dois elementos do grupo são produzidos - C\textsubscript{1}C\textsubscript{2}
\end{tabular}}
\caption{Ataque ramificado}
\label{fig:freitas_ataque_ram}
\end{figure}

Em português do Brasil, apenas os \textit{Estádios 0}, \textit{I} e \textit{III} se verificam \citep{ribas2004}, não sendo, portanto, produtiva a estratégia de \isi{epêntese vocálica} registada no \textit{Estádio II} do português europeu.
Note-se que o uso de \isi{epêntese vocálica} é também comum nos dados da escrita de crianças portuguesas, estando na base de erros ortográficos associados a registos de sequências consonânticas. Vejam-se os exemplos em (\ref{ex:freitas_erros}).

\protectedex{
\ea\label{ex:freitas_erros}
Erros ortográficos produzidos por crianças portuguesas \citep{santos2013}
\begin{tabbing}
  x ano de escolaridade \quad \= palavrass \quad \= ortografiasssss \quad \= criança \kill
  1º ano de escolaridade \> \textit{bruxa} \> *<berucha> \> (criança 1)\\
 ~ \> \textit{globo} \> *<gulobu> \> (criança 6)\\
 ~ \> \textit{pneu} \> *<paneo> \> (criança 2)\\
 ~ \> \textit{afta} \> *<afata> \> (criança 22)\\
 4º ano de escolaridade \> \textit{atlântico} \> *<atalântico> \> (criança 20)\\
 ~ \> \textit{astronomo} \> *<asterónemo> \> (criança 11)\\
 ~ \> \textit{Neptuno} \> *<nepetuno> \> (criança 27)\\
 ~ \> \textit{pictograma} \> *<picatograma> \> (criança 10)
\end{tabbing}
\z
}

Vários estudos têm referido que estruturas silábicas complexas estão na base da produção de taxas mais elevadas de erros ortográficos \citep{mirandamatzenauer2010,santos2013}. A \isi{epêntese vocálica} identificada nos registos acima é comum com Ataques\is{sílaba!ataque} ramificados e com grupos consonânticos problemáticos no português europeu, refletindo o comportamento verbal oral usado pelas crianças portuguesas na aquisição destas estruturas. Vejam-se os exemplos em (\ref{ex:freitas_producoes}).

\protectedex{
\ea\label{ex:freitas_producoes}
Produções orais de crianças no 1º ano de escolaridade \citep{santos2013}
\ea{
Ataques ramificados
\begin{tabbing}
palavra \quad \= transcricao \quad \= criança \kill
\textit{globo} \> \textipa{[g5\textprimstress lobu]} \> (criança 5)\\
\textit{planta} \> \textipa{[p1\textprimstress l\~5t5]} \> (criança 27)\\
\textit{flores} \> \textipa{[f1\textprimstress loR1S]} \> (criança 17)
\end{tabbing}
}
\ex{
Grupos consonânticos problemáticos 
\begin{tabbing}
palavra \quad \= transcricao \quad \= criança \kill
\textit{pneu} \> \textipa{[p1\textprimstress new]} \> (criança 14)\\
\textit{pneu} \> \textipa{[pu\textprimstress new]} \> (criança 22)\\
\textit{gnomo} \> \textipa{[g1\textprimstress nomu]} \> (criança 20)
\end{tabbing}
}
\z
\z
}

A produtividade da \isi{epêntese vocálica} em crianças portuguesas, por oposição a crianças brasileiras ou a crianças falantes de outras línguas, pode decorrer da distância entre representações fonológicas com estruturas silábicas simples e produções fonéticas com taxas elevadas de supressão vocálica (vejam-se os casos de \textit{telefone} \textipa{[t\textltilde\textprimstress fOn]} e \textit{desprevenir} \textipa{[dSpRv\textprimstress niR]}, com quatro vogais nas formas fonológicas e apenas uma nas formas fonéticas).\footnote{Exemplos retirados de \citet[44]{mateusdandrade2000}.} Esta assimetria entre representações fonológicas e formatos fonéticos das palavras, decorrente do apagamento vocálico frequente em português europeu e quase inexistente em português do Brasil, pode estar na base do uso recorrente de vogais epentéticas na aquisição do português europeu, num momento em que as crianças se encontram em processo de construção das representações fonológicas das palavras a partir dos diferentes formatos fonéticos disponíveis no \textit{input}, muitas vezes deficitários quanto à presença de vogais, cruciais para a identificação de núcleos\is{sílaba!núcleo} silábicos.

Referimos, em (\ref{ex:freitas_producoes}), dois tipos de sequências consonânticas: ataques\is{sílaba!ataque} ramificados e grupos consonânticos problemáticos. \citet{santos2013} centrou-se nestes dois tipos de estruturas, sendo os grupos consonânticos problemáticos considerados como tal por violarem princípios de boa formação silábica.\footnote{Sobre os princípios de boa formação silábica, consulte-se \citet{mateusdandrade2000}, \citet{freitassantos2001} \citet{mateus_etal2005}.} Estamos a falar de sequências como as listadas em \ref{ex:freitas_problem}.

\ea\label{ex:freitas_problem}
Grupos consonânticos problemáticos
\ea{
\begin{tabbing}
xxxxxxxxx+xxxxxxxxx \quad \= palavra\kill
oclusiva+oclusiva \> \textit{o\underline{pt}ar}
\end{tabbing}
}
\ex{
\begin{tabbing}
xxxxxxxxx+xxxxxxxxx \quad \= palavra\kill
oclusiva+nasal \> \textit{\underline{pn}eu}
\end{tabbing}
}
\ex{
\begin{tabbing}
xxxxxxxxx+xxxxxxxxx \quad \= palavra\kill
oclusiva+fricativa \> \textit{a\underline{bs}urdo}
\end{tabbing}
}
\ex{
\begin{tabbing}
xxxxxxxxx+xxxxxxxxx \quad \= palavra\kill
fricativa+oclusiva \> \textit{a\underline{ft}a}
\end{tabbing}
}
\ex{
\begin{tabbing}
xxxxxxxxx+xxxxxxxxx \quad \= palavra\kill
nasal+nasal \> \textit{a\underline{mn}istia}
\end{tabbing}
}
\z
\z

Estes grupos consonânticos permitem a \isi{epêntese vocálica} opcional no português europeu (\textit{p}\textipa{[1]}\textit{neu}), obrigatória no português do Brasil (\textit{p}\textipa{[i]}\textit{neu}). Note-se que a \isi{epêntese vocálica} não ocorre com os ataques\is{sílaba!ataque} ramificados nas produções dos adultos (*\textit{f}\textipa{[1]}\textit{ruta}) no português europeu; *\textit{f}\textipa{[i]}\textit{ruta} no português do Brasil), o que argumenta a favor da natureza silabicamente distinta dos dois grupos consonânticos. \citet{mateusdandrade2000} propõem, assim, que as duas consoantes dos grupos problemáticos sejam representadas como dois Ataques\is{sílaba!ataque} simples de sílabas adjacentes, sendo a primeira consoante Ataque\is{sílaba!ataque} de uma sílaba com núcleo\is{sílaba!núcleo} vazio e a segunda Ataque\is{sílaba!ataque} de uma sílaba com núcleo\is{sílaba!núcleo} preenchido.\footnote{Consulte-se \citet{mateusdandrade2000}, \citet{freitassantos2001} e \citet{mateus_etal2005}.}

A natureza problemática destes grupos consonânticos em português europeu é estudada em \citet{santos2013}, que testa a hipótese formulada em \citet{freitas1997} segundo a qual os ataques\is{sílaba!ataque} ramificados seriam adquiridos antes dos grupos consonânticos problemáticos. Os dados da oralidade recolhidos pela autora junto de crianças dos 1º e 4º anos do 1º Ciclo do Ensino Básico confirmaram a hipótese, sendo a ordem atestada a seguinte: \textit{Ataques ramificados}\is{sílaba!ataque} $>>$ \textit{grupos consonânticos problemáticos}. Os dados de \citet{santos2013} mostram que a aquisição destas estruturas pode não estar terminada à entrada no 1º Ciclo do Ensino Básico, com impacto significativo nos desempenhos ortográficos infantis (Tabela \ref{tab:freitas_taxas}).

\begin{table}
\begin{tabular}{lll}
\lsptoprule
\multirow{2}{*}{\textbf{Oralidade}} & Ataques\is{sílaba!ataque} ramificados                & 92\% \\
                           & Grupos Consonânticos Problemáticos & 79\% \\
\midrule
\multirow{2}{*}{\textbf{Escrita}}   & Ataques\is{sílaba!ataque} ramificados                & 32\% \\
                           & Grupos Consonânticos Problemáticos & 16\%\\
\lspbottomrule
\end{tabular}
\caption{Taxas de sucesso no 1º ano de escolaridade \citep{santos2013}}
\label{tab:freitas_taxas}
\end{table}

A formação inicial de professores de 1º Ciclo deve, assim, incorporar, entre outros aspetos fonológicos, uma reflexão sistemática sobre o efeito da estrutura silábica nos desempenhos orais e ortográficos infantis, no sentido de promover intervenções didáticas eficazes, que contribuam para a superação dos problemas dos alunos quando processam a relação entre oralidade e escrita para a construção das representações ortográficas das palavras.

\subsection{Rima}
\label{subsec:freitas_rima}

O formato silábico universal CV, disponível desde o início da produção em todas as línguas, apresenta uma Rima\is{sílaba!rima} não ramificada (só Núcleo,\is{sílaba!núcleo} sem Coda).\is{sílaba!coda} No percurso da aquisição, as estruturas silábicas passam de não ramificadas a ramificadas, como vimos na secção anterior. No caso da Rima,\is{sílaba!rima} os percursos de aquisição de diferentes sistemas linguísticos podem divergir por duas razões: a sua estrutura difere substancialmente de língua para língua; a estabilização da Rima\is{sílaba!rima} ramificada ocorre tardiamente. Dois níveis de ramificação ocorrem no domínio deste constituinte: a Rima\is{sílaba!rima} ramifica em Núcleo\is{sílaba!núcleo} e Coda\is{sílaba!coda} (\textit{p\underline{az}} \textipa{[\textprimstress pAS]}); o Núcleo\is{sílaba!núcleo} ramifica, passando a dominar duas posições de esqueleto (\textit{p\underline{ai}s} \textipa{[\textprimstress pajʃ]}).

Como referimos, as línguas diferem muito quanto à estrutura da Rima\is{sílaba!rima} e ao tipo de segmentos possíveis em Coda.\is{sílaba!coda} Línguas germânicas como o inglês\il{inglês} e o holandês\il{holandês} apresentam Rimas\is{sílaba!rima} com Codas\is{sílaba!coda} ramificadas, ou seja, Codas\is{sílaba!coda} associadas a duas consoantes (inglês:\il{inglês} \textit{fa\underline{st}}; holandês:\il{holandês} \textit{bee\underline{st}}). Nestas línguas, todos os tipos de consoantes podem ocorrer em Coda\is{sílaba!coda} (oclusivas, nasais, fricativas, líquidas), como acontece também no francês\il{francês} \citep{fikkert1994,bernhardtstemberger1998,rose200,almeida2011}.

No Português, a Coda\is{sílaba!coda} impõe fortes restrições ao nível segmental. As análises diferem do português europeu para o português do Brasil. Começaremos por apresentar os dados relativos ao português europeu, seguindo-se os relativos ao português do Brasil.\newpage

A análise proposta em \citet{mateusdandrade2000} é a de que o português europeu só possui Codas\is{sílaba!coda} não ramificadas, sendo preenchidas apenas por três segmentos fonológicos: a fricativa \textipa{/s/}, com os formatos fonéticos palatais \textipa{[S]} e \textipa{[Z]}; a lateral alveolar \textipa{/l/}, com o formato fonético velarizado \textipa{[\textltilde]}; a vibrante alveolar \textipa{/R/}, com o formato fonético \textipa{[R]}. Vejam-se os exemplos na Tabela \ref{tab:freitas_consoantes_coda}.\is{sílaba!coda} 

\begin{table}
\begin{tabular}{ll}
\lsptoprule
\textit{pa\underline{s}ta} & \textipa{/s/} $\rightarrow$ \textipa{[S]} \\
\textit{a\underline{s}ma}  & \textipa{/s/} $\rightarrow$ \textipa{[Z]} \\
\textit{ba\underline{l}de} & \textipa{/l/} $\rightarrow$ \textipa{[\textltilde]} \\
\textit{po\underline{r}ta} & \textipa{/R/} $\rightarrow$ \textipa{[R]} \\
\lspbottomrule
\end{tabular}
\caption{Consoantes em Coda (português europeu)}
\label{tab:freitas_consoantes_coda}
\end{table}

A aquisição das Codas\is{sílaba!coda} em português europeu, como no holandês\il{holandês} e no inglês,\il{inglês} é gradual: primeiro emergem as Codas\is{sílaba!coda} fricativas; mais tarde, com um intervalo temporal longo, emergem as Codas\is{sílaba!coda} líquidas. Esta ordem segue a registada para línguas como o holandês,\il{holandês} com as obstruintes em Coda\is{sílaba!coda} a emergirem antes das líquidas nesta posição. As Codas fricativas em português europeu surgem por volta dos 2;0, muito antes da emergência das Codas\is{sílaba!coda} líquidas \citep{freitas1997,correia2004,amorim2014}. A ordem relatada tem sido: \textipa{/s/} $>>$ \textipa{/R/} $>>$ \textipa{/l/}; vejam-se na Tabela \ref{tab:freitas_codas} os resultados em \citet{mendes_etal2009}.

\begin{table}
\begin{tabular}{ll}
\lsptoprule
Coda \textipa{[S]} & 3;6--4;0 \\
Coda \textipa{[R]} & 4;6--5;0 \\
Coda \textipa{[\textltilde]} & 5;0--5;6 \\
\lspbottomrule
\end{tabular}
\caption{Codas em português europeu -- idade de aquisição \citep{mendes_etal2009}}
\label{tab:freitas_codas}
\end{table}

No português do Brasil, a análise para o constituinte Coda\is{sílaba!coda} difere da proposta para o português europeu \citep{bisol2005,lamprecht_etal2004}:

\begin{enumerate}[label=(\roman*)]
\item as Codas\is{sílaba!coda} não ramificadas dominam os segmentos listados para o português europeu (fricativa \textipa{/s/}, líquidas \textipa{/l/} e \textipa{/R/)}, mas também a consoante nasal \textipa{/n/}, cuja realização fonética ocorre sob a forma de nasalidade na vogal do Núcleo\is{sílaba!núcleo} tautossilábico\is{tautossilábico}\footnote{\is{tautossilábico}O termo \textit{tautossilábico} opõe-se a \textit{heterossilábico},\is{heterossilábico} remetendo estes para estruturas dentro da mesma sílaba ou em sílabas distintas, respetivamente.}
(\textit{ponta} \textipa{[\pstr p\~ot5]});
\item na mesma linha de análise, são consideradas Codas\is{sílaba!coda} ramificadas os casos de \textit{mo\underline{ns}tro}, sendo que, uma vez mais, a consoante nasal é produzida sob a forma de nasalidade da vogal do Núcleo\is{sílaba!núcleo} tautossilábico (\textit{monstro} \textipa{[\pstr m\~oStRu]}). 
\end{enumerate}

Em \citet{wertzner2000,wertzner2003}, citado por \citet{mendes_etal2009,mendes_etal2013}, encontramos dados relativos à idade de aquisição apenas para as codas\is{sílaba!coda} \textipa{[s]} e \textipa{[R]} (Tabela \ref{tab:freitas_codas_pb}).

\begin{table}
\begin{tabular}{ll}
\lsptoprule
Coda \textipa{[s]} & 4;0 \\
Coda \textipa{[R]} & 5;06 \\
\lspbottomrule
\end{tabular}
\caption{Codas em português do Brasil – idade de aquisição \citep{wertzner2000,wertzner2003}}
\label{tab:freitas_codas_pb}
\end{table}

A referência exclusiva a estes dois segmentos decorrerá certamente de um critério fonético de identificação de consoantes em Coda,\is{sílaba!coda} uma vez que a Coda\is{sílaba!coda} \textipa{/l/} tem o formato \textipa{[w]} na maioria das variedades dialectais do português do Brasil e a Coda\is{sílaba!coda} \textipa{/n/} emerge sempre como nasalidade da vogal do Núcleo\is{sílaba!núcleo} tautossilábico.
Em \citet{mezzomo2004}, é fornecido mais detalhe sobre a ordem de aquisição das Codas\is{sílaba!coda} em português do Brasil (\ref{ex:freitas_mezzomo}): 

\ea\label{ex:freitas_mezzomo}
\textipa{/l/} final $>>$ \textipa{[n]} final $>>$ \textipa{/n/} medial $>>$ \textipa{/s/} final $>>$ \textipa{/l/} medial $>>$ \textipa{/ɾ/} final $>>$ \textipa{/s/} medial $>>$ \textipa{/ɾ/ medial}
\z

As aquisições precoces de \textipa{/l/} e \textipa{/n/} no português do Brasil, podendo ser inicialmente apresentadas como divergentes do percurso descrito para o português europeu, podem ser interpretadas de outra forma: nos dois casos, em português do Brasil, o que está em causa é a produção da forma fonética \textipa{[w]} para a coda\is{sílaba!coda} \textipa{/l/} e a produção de vogais nasais para coda \textipa{/n/}. Ora, em português europeu como em português do Brasil, a produção de \textipa{[w]} e de vogais nasais emerge precocemente. A questão que se coloca, para observação futura, é a de saber se português europeu e português do Brasil são assim tão diferentes nesta matéria ou se é a adoção de diferentes análises fonológicas para os dois sistemas que está a condicionar a descrição dos dados e a formulação de generalizações sobre os mesmos.

A escala em (\ref{ex:freitas_mezzomo}) refere contrastes de aquisição da Coda\is{sílaba!coda} no português do Brasil em função da variável posição na palavra. Também no português europeu se regista uma diferença entre aquisição de codas fricativas e de codas líquidas em posições final e medial. No caso das codas\is{sílaba!coda} fricativas, observem-se os exemplos na Figura \ref{fig:freitas_codas_ines},\is{sílaba!coda} ilustrativos do que tem sido observado em português europeu:

\begin{figure}
\begin{tabbing}
  blablablablablablablablablablablabla \quad \= palavra \quad \= transcr \quad \= idade \kill
  tónica medial (lexical) \> \textit{festa} \> \textipa{[\textprimstress tEt5]} \> 1;9 \\
  \textbf{$\surd$tónica final (\isi{marcador de plural})} \> \textit{meus} \> \textipa{[\textprimstress mewS]} \> 1;9 \\
  tónica final (lexical) \> \textit{nariz} \> \textipa{[5\textprimstress gi5]} \> 1;9 \\
  átona medial (lexical) \> \textit{buscar} \> \textipa{[Bu\textprimstress ka]} \> 1;10 \\
  \textbf{$\surd$átona final (maracador de plural)} \> \textit{bolos} \> \textipa{[\textprimstress boloS]} \> 1;9 \\
  átona final (lexical) \> \textit{lápis} \> \textipa{[\textprimstress patu]} \> 1;9 \\
  \end{tabbing}
  \caption{Codas fricativas na Inês (português europeu)}
  \label{fig:freitas_codas_ines}
  \end{figure}
  
Os dados acima mostram que a posição final de palavra promove a aquisição das Codas\is{sílaba!coda} fricativas em português europeu. O mesmo efeito é descrito em (\ref{ex:freitas_mezzomo}) para o português do Brasil \citep{mezzomo2004}. A maior parte destas sílabas finais são átonas, na sequência da preferência do português pelo padrão paroxítono (\textit{profe\underline{sso}ras}; \textit{cogu\underline{me}los}). Esperaríamos, portanto, a sua aquisição tardia, uma vez que são as estruturas tónicas que são tidas como promotoras da aquisição \citep{fikkert1994}. Neste caso, e inesperadamente, Codas\is{sílaba!coda} fricativas em sílaba final de palavra, frequentemente átona, logo, acusticamente pouco proeminente, são as primeiras a serem adquiridas. A aquisição precoce destas Codas tem sido interpretada como produto da interface fonologia-morfologia em final de palavra, contexto de ativação da morfologia flexional, portanto, gramaticalmente complexo \citep{freitas1997}. As Codas\is{sílaba!coda} fricativas finais são maioritariamente morfológicas, estando associadas ou à marcação do plural nos não-verbos (\textit{bolos}, \textit{meus}) ou à marcação de pessoa verbal (\textit{falas}, \textit{gostavas}). Por restrições associadas ao léxico infantil, as Codas fricativas lexicais até aos 5;0 são normalmente mediais (\textit{festa}, \textit{buscar}). A ordem C\textit{odas fricativas finais} $>>$ \textit{Codas fricativas mediais} \citep{freitas1997} parece, assim, decorrer da natureza gramaticalmente complexa das primeiras em final de palavra, com confluência de aspetos fonológicos e morfológicos numa mesma estrutura, o final de palavra. Neste caso, a complexidade gramatical parece promover a aquisição. O acento não parece ser relevante para a aquisição das Codas\is{sílaba!coda} fricativas finais (na Figura \ref{fig:freitas_codas_ines}, vejam-se os casos de \textit{\underline{bo}los}, com coda átona, e \textit{\underline{meus}}, coda tónica). No entanto, a variável \textit{acento de palavra} parece ter impacto na aquisição das codas\is{sílaba!coda} mediais (lexicais) em português europeu (\textit{\underline{fes}ta} $>>$ \textit{\underline{bus}car}), tendo sido observada a seguinte ordem de aquisição \textit{codas fricativas mediais tónicas} $>>$ \textit{codas fricativas mediais átonas} \citep{freitas1997}.

Casos de Codas\is{sílaba!coda} fricativas lexicais em final de palavra (\textit{lápis}, \textit{nariz}) são pouco frequentes nos dados das crianças, pelo que não é\largerpage possível, até ao momento, estabelecer generalizações sobre a aquisição das mesmas. 

A escala para o português do Brasil registada em (\ref{ex:freitas_mezzomo}) mostra o mesmo efeito da variável \textit{posição na palavra} nas Codas\is{sílaba!coda} líquidas, com cada Coda líquida final a ser adquirida antes da sua contrapartida medial. Os mesmos resultados foram observados para o português europeu \citep{freitas1997,correia2004}: as primeiras líquidas (\textipa{/l/}, \textipa{/R/}) tendem a emergir simultaneamente em Ataque\is{sílaba!ataque} simples e em Coda final, quando ainda não ocorrem em Coda\is{sílaba!coda} medial. Vejam-se os exemplos na Tabela \ref{tab:freitas_liquidas_marta}.

\begin{table}
\begin{tabular}{lll}
\lsptoprule
Alvo: Coda final & Alvo: Ataque\is{sílaba!ataque} simples & Alvo: Coda medial \\
\midrule
Natal \textipa{[t5\textprimstress tal1]} & borboleta \textipa{[bubu\textprimstress lE\textlengthmark t5]} & calçar \textipa{[k5\textprimstress s5]}           \\
sair \textipa{[sa\textprimstress iR1]}      &  girafa \textipa{[Zi\textprimstress Raf]} & urso \textipa{[\textprimstress usu]}  \\
\lspbottomrule
\end{tabular}
  \caption{Líquidas na Marta (2;2) \citep{freitas1997}}
  \label{tab:freitas_liquidas_marta}
\end{table}

Nos dados acima, a Coda\is{sílaba!coda} líquida final é obrigatoriamente produzida com epêntese de vogal\is{epêntese vocálica} em final de palavra (possível mas não obrigatória nos enunciados dos adultos: \textit{Natal} \textipa{[n5\textprimstress ta\textltilde]} / \textipa{[n5\textprimstress tal1]}; \textit{sair} \textipa{[s5\textprimstress iR]} / \textipa{[s5\textprimstress iR1]}). No mesmo estádio, são já produzidas as líquidas \textipa{/l/} e \textipa{/R/} em Ataque\is{sílaba!ataque} simples mas não em Coda medial. Tal pode levar-nos a pensar que as líquidas em final de palavra, neste estádio, não são ainda processadas como Codas\is{sílaba!coda} mas como Ataques\is{sílaba!ataque} simples. Os argumentos subjacentes a esta interpretação são: (i) a produção sistemática da \isi{epêntese vocálica} em final de palavra, à direita da líquida; (ii) a não produção de Coda medial. Tais factos denotariam a ausência da estrutura \textit{coda líquida}\is{sílaba!coda} na gramática da criança neste estádio.

Em síntese, as Codas\is{sílaba!coda} líquidas (na ordem \textipa{/R/} $>>$ \textipa{/l/}) são de aquisição tardia em português europeu: entre os 4;6 e os 5;0 para \textipa{/R/}; entre os 5;0 e os 5;6 para o \textipa{/l/} \citep{mendes_etal2009,mendes_etal2013}. No português do Brasil \citep{lamprecht_etal2004}, é registada a ordem inversa para as codas líquidas (\textipa{/l/} $>>$ \textipa{/R/}): \textipa{/l/} final está adquirido aos 1;4 e \textipa{/l/} medial aos 3;0: \textipa{/R/} está adquirido aos 3;10, em ambas as posições. Como referimos, a aquisição muito precoce de \textipa{/l/} final (aos 1;4) por crianças brasileiras pode decorrer da sua produção como \textipa{[w]} na fala dos adultos (em posição medial, só é adquirida aos 3;0), evocando diferentes interpretações: estará a Coda \textipa{/l/} final adquirida aos 1;4 ou estará, nessa fase inicial, a ser processada como semivogal?
 
Deter-nos-emos, agora na aquisição do Núcleo.\is{sílaba!núcleo} Uma vez mais, chamamos a atenção para o facto de a estrutura universal CV apresentar apenas um Núcleo\is{sílaba!núcleo} não ramificado (V). As crianças brasileiras e portuguesas terão de adquirir ditongos\is{ditongo} decrescentes fonológicos, que correspondem a estruturas de tipo VG (ou VV), representados no domínio de núcleos\is{sílaba!núcleo} ramificados (exemplos: \textit{pai} \textipa{[\textprimstress paj]}; \textit{mau} \textipa{[\textprimstress maw]}; \textit{seu} \textipa{[\textprimstress sew]}; \textit{boi} \textipa{[\textprimstress boj]}). Em português do Brasil, a sua aquisição é precoce, estando maioritariamente terminada aos 2;0. Vejam-se os dados na Tabela \ref{tab:freitas_ditongos_brasil}.

\begin{table}
\begin{tabular}{ll}
\lsptoprule
Idade      & Ditongos             \\
\midrule
1;0 -- 1;1 & \textipa{[aw]} -- \textipa{[iw]} \\
1;1 -- 1;2 & \textipa{[aj]} -- \textipa{[oj]} \\
1;3 -- 1;4 & \textipa{[ej]} -- \textipa{[ew]} \\
1;6 -- 1;7 & \textipa{[Ew]} -- \textipa{[Oj]} \\
1;8 -- 1;9 & \textipa{[Ej]} -- \textipa{[iw]} \\
2;2 -- 2;4 & \textipa{[uj]}   \\  
\lspbottomrule
\end{tabular}
  \caption{Ditongos em português do Brasil \citep{bonilha2004}}
  \label{tab:freitas_ditongos_brasil}
\end{table}

No português europeu, a emergência dos ditongos\is{ditongo} é precoce mas a sua estabilização é tardia,\footnote{Em \citet{freitas1997}, este facto é correlacionado com a aquisição tardia das líquidas em final de sílaba.} mostrando-se que a sua aquisição não está terminada aos 3;7. Veja-se a Tabela \ref{tab:freitas_ditongos_ep}, na qual se pode observar o reduzido ou nulo progresso na aquisição desta estrutura, na maior parte das crianças.

\begin{table}
\begin{tabular}{lll}
\lsptoprule
Crianças & Intervalo etário & \% de sucesso (início e fim da observação) \\
\midrule
João     & 1;11 -- 2;8      & 52.0 -- 60.9                               \\
Inês     & 1;8 -- 1;10      & 56.9 -- 63.3                               \\
Marta    & 1;2 -- 2;2       & 50.9 -- 79.4                               \\
Luís     & 1;10 -- 2;11     & 75.8 -- 65.6                               \\
Raquel   & 1;10 -- 2;10     & 51.7 -- 62.9                               \\
Laura    & 2;2 -- 3;3       & 68.6 -- 69.0                               \\
Pedro    & 2;7 -- 3;7       & 60.9 -- 75.4 \\                             
\lspbottomrule
\end{tabular}
  \caption{Ditongos do português europeu \citep{freitas1997}}
  \label{tab:freitas_ditongos_ep}
\end{table}

Regularmente, a aquisição parte de estruturas simples para chegar a estruturas complexas. No entanto, alguns relatos mostram que a complexidade pode promover a aquisição, provavelmente pelo facto de tornar determinadas estruturas proeminentes na língua. \citet{correia2004} estudou diferentes tipos de estruturas da Rima,\is{sílaba!rima} tendo verificado que a Rima\is{sílaba!rima} mais complexa em português europeu, a sequência de \isi{ditongo} nasal seguido de Coda\is{sílaba!coda} fricativa (\textit{m\underline{ãos}}) é a primeira a estabilizar. O mesmo efeito de aceleração da complexidade silábica foi relatado na comparação entre a aquisição da Rima\is{sílaba!rima} no português europeu e no holandês\il{holandês} \citep{fikkertfreitas1998}: a Rima no holandês\il{holandês} é substancialmente mais complexa do que no português europeu; no entanto, as crianças holandesas adquirem mais rapidamente os diferentes tipos de\largerpage estruturas da rima\is{sílaba!rima} do que as crianças portuguesas.

\section{Considerações finais}
\label{sec:freitas_conclusao}

Traçámos, neste capítulo, um cenário global da aquisição da estrutura silábica em português europeu e em português do Brasil, em função dos constituintes silábicos Ataque\is{sílaba!ataque} e Rima.\is{sílaba!rima} A aquisição de várias estruturas silábicas problemáticas no português não foi, deliberadamente, apresentada, dada a natureza desta publicação. As tabelas abaixo visam fornecer uma síntese dos estádios de aquisição da estrutura silábica em português europeu (Tabela \ref{tab:freitas_ordem_pe}) e em português do Brasil (Tabela \ref{tab:freitas_ordem_pb}),\footnote{Em português do Brasil, em posição final de palavra, a ordem de aquisição das Codas\is{sílaba!coda} é diferente: \textipa{/l/} ocorre aos 1;4, \textipa{/n/} aos 1;7 e \textipa{/s/} aos 2;6. A Coda\is{sílaba!coda} \textipa{/r/} estabiliza aos 3;10 nas posições medial e final \citep{lamprecht_etal2004}.} com base nos estudos citados ao longo do capítulo; considerámos, para o efeito, apenas a idade de estabilização de cada tipo de constituinte, independentemente da sua relação com a posição na palavra.

\begin{table}
\resizebox{\textwidth}{!}{
\begin{tabular}{lll}
\lsptoprule
\textbf{Estádio 1} & \multicolumn{2}{l}{Ataque não ramificado $+$ Rima\is{sílaba!rima} não ramificada: CV / V}     \\
\textbf{Estádio 2} & \multicolumn{2}{l}{Ataque não ramificado $+$ Rima\is{sílaba!rima} ramificada: (C)VC\textsubscript{fricativa}} \\
\textbf{Estádio 3} & \multicolumn{1}{l|}{Núcleo\is{sílaba!núcleo} ramificado: (C)VG / (C)VC\textsubscript{líquida}} & \multicolumn{1}{l}{Ataque ramificado CCV}                            \\
\textbf{Estádio 4} & \multicolumn{1}{l|}{Ataque ramificado: CCV}                                                        & \multicolumn{1}{l}{Núcleo ramificado (C)VC\textsubscript{líquida}} \\
\lspbottomrule
\end{tabular}}
  \caption{Ordem de estabilização dos constituintes silábicos no português europeu}
  \label{tab:freitas_ordem_pe}
\end{table}

\begin{table}
\begin{tabular}{ll}
\lsptoprule
Estádio 1 & Ataque não ramificado $+$ Rima\is{sílaba!rima} não ramificada: CV / V \\
Estádio 2 & Núcleo\is{sílaba!núcleo} ramificado: (C)VG                            \\
Estádio 3 & Rima ramificada: (C)VC\textsubscript{nasal}                         \\
Estádio 4 & Rima ramificada: (C)VC\textsubscript{fricativa} e (C)VC\textsubscript{lateral}      \\
Estádio 5 & Rima ramificada: (C)VC\textsubscript{líquida não lateral}           \\
Estádio 6 & Ataque ramificado      \\
\lspbottomrule
\end{tabular}
  \caption{Ordem de estabilização dos constituintes silábicos no português do Brasil \citep{lamprecht_etal2004}}
  \label{tab:freitas_ordem_pb}
\end{table}

Neste capítulo, pretendemos, ainda, mostrar que a relação entre desenvolvimento silábico e aquisição do inventário de segmentos é crucial para a caracterização do desenvolvimento fonológico infantil, relação essa com impacto nas áreas clínica e educacional. No caso da avaliação clínica, a maior parte dos instrumentos de avaliação fonológica disponíveis não distingue os diferentes contextos silábicos, avaliando preferencialmente o segmento em ataque\is{sílaba!ataque} não ramificado, o que pode conduzir a diagnósticos e a planificações de intervenção desajustados. Por sua vez, em contexto educacional, a consciência da relação entre estrutura silábica e segmentos permitirá aos professores avaliarem com mais rigor as produções orais e escritas dos alunos, desenhando estratégias de intervenção didática mais adequadas à superação dos obstáculos decorrentes do confronto entre oralidade e ortografia espelhado nos desempenhos académicos das crianças.

Alguns estudos têm testado, no português, a relevância da relação entre estruturas silábicas e segmentos, quer em crianças com desenvolvimento fonológico atípico (entre outros, \citealt{ramos1996}; \citealt{nogueira2007}; \citealt{baptista2015}; \citealt{ramalhoprep}), quer em crianças em processo de aprendizagem da escrita (entre outros, \citealt{mirandamatzenauer2010,santos2013}). Estes estudos são, por vezes, produzidos com base em amostras reduzidas e as hipóteses neles formuladas carecem de verificação em conjuntos de dados mais alargados. Só desta forma poderemos testar a eficácia do uso de uma abordagem fonológica não-linear\is{fonologia não-linear} \citep{bernhardtstemberger2000}, que relaciona constituintes prosódicos e segmentais, na avaliação e intervenção em contextos clínico e educacional. 



{\sloppy
\printbibliography[heading=subbibliography,notkeyword=this]
}
\end{document}