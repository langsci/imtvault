\documentclass[output=paper]{LSP/langsci} 
\author{Ana Luísa Costa\affiliation{Escola Superior de Educação do Instituto Politécnico de Setúbal e Universidade de Lisboa, Centro de Linguística}\and 
Armanda Costa\affiliation{Universidade de Lisboa, Faculdade de Letras, Centro de Linguística}\lastand 
Anabela Gonçalves\affiliation{Universidade de Lisboa, Faculdade de Letras, Centro de Linguística}
}
\title{Consciência linguística: aspetos sintáticos}
\ChapterDOI{10.5281/zenodo.889465}
\abstract{\noabstract}
\maketitle
\begin{document}
\section{Considerações iniciais}
\label{sec:costacostagoncalves_intro}

Uma das grandes conquistas genéticas do ser humano foi a linguagem. É esta faculdade mental inata, exclusiva da espécie humana, que permite a qualquer indivíduo adquirir, de forma espontânea (isto é, sem aprendizagem formal), a língua a que está exposto desde o nascimento e que virá a ser a sua língua materna.\footnote{A existência de um mecanismo inato específico para a aquisição das línguas – a linguagem – é um dos pilares do quadro teórico conhecido como Gramática Generativa.}

O conjunto de regras e princípios que os falantes adquirem de forma espontânea constitui o seu \textit{conhecimento implícito}\is{conhecimento!linguístico implícito} ou, nos termos de \citet{chomsky1986}, a sua língua internalizada (Língua-I). Trata-se, pois, de um conhecimento intuitivo das estruturas da língua que pode ser entendido como a gramática mental da língua materna, desenvolvida espontaneamente a partir da interação entre a faculdade da linguagem e o input linguístico que o meio fornece. É este sistema mental de regras e princípios que o falante usa para construir e compreender os enunciados, sem, no entanto, ter consciência das suas propriedades. Isto significa que, nas diversas situações de comunicação, os falantes se focam mais na informação a comunicar do que na forma dos enunciados (cf. \citealt{simsim1998}).

O conhecimento linguístico implícito\is{conhecimento!linguístico implícito} é um sistema altamente complexo, que envolve diversas componentes. Assim, os falantes têm de conhecer (i) as unidades lexicais e as suas propriedades (conhecimento lexical); (ii) a estrutura sonora da língua (conhecimento fonológico); (iii) os processos de formação de palavras (conhecimento morfológico); (iv) os mecanismos de combinações livres de palavras para a produção e compreensão de frases (conhecimento sintático); (v) o significado das frases e as relações semânticas entre as mesmas (conhecimento semântico). 

Por volta dos 6-7 anos, a criança já dispõe não só de um conhecimento implícito\is{conhecimento!linguístico implícito} suficientemente robusto para produzir e compreender um vasto conjunto de enunciados, mas também da capacidade de refletir sobre as estruturas da sua língua materna, nomeadamente, de identificar as unidades linguísticas e manipular, de forma deliberada, o uso das regras da gramática \citep{gombert1992,duarte2008}. Esta capacidade, designada como \textit{consciência linguística},\is{consciência!linguística} implica algum controlo por parte do sujeito sobre a forma dos seus enunciados, manifestando-se nas várias componentes da gramática. Constituem manifestações da consciência linguística\is{consciência!linguística}, por exemplo, o gosto precoce das crianças por rimas, evidenciando a sensibilidade aos sons da língua (consciência fonológica);\is{consciência!fonológica} as autocorreções das crianças em formas verbais, substituindo \textit{dizi} por \textit{disse} (consciência morfológica);\is{consciência!morfológica} a capacidade de identificar sequências agramaticais como *\textit{muro cavalo o saltou}, por não respeitarem a ordem de palavras na frase (consciência sintática)\is{consciência!sintática} – cf. \citet{goncalves_etal2011}. 

Os trabalhos sobre consciência linguística\is{consciência!linguística} têm sido desenvolvidos quer no contexto da Linguística Educacional quer no contexto da Psicologia e da Psicolinguística, tendo impacto não só sobre a literacia e a estabilização das estruturas de desenvolvimento tardio, como também sobre o diagnóstico e a intervenção terapêutica em perturbações da linguagem.\footnote{Para uma revisão detalhada sobre consciência linguística\is{consciência!linguística}, ver \citet{barbeiro1999} e \citet{castelo2012}.}

\largerpage Assim, no domínio da Educação, alguns autores defendem que a promoção da consciência linguística\is{consciência!linguística} é preditora do sucesso na aprendizagem da \isi{leitura} e da \isi{escrita} (ver, por exemplo, \citealt{hudson2001}). A língua materna pode ser um objeto de estudo importante, passível de ser descoberta e descrita a partir do conhecimento implícito\is{conhecimento!linguístico implícito} dos alunos, com o objetivo de melhorar os seus desempenhos em diferentes situações comunicativas, orais e escritas,\is{escrita} o que conduzirá a uma maior aceitação da diversidade linguística e promoverá a autoconfiança linguística \citep{duarte1998,duarte2008}. 

No contexto da Psicologia e da Psicolinguística, a investigação sobre consciência linguística\is{consciência!linguística} tem-se focado no estudo do desenvolvimento cognitivo e metacognitivo, do desenvolvimento linguístico e da aprendizagem da \isi{escrita} e da \isi{leitura}.\footnote{Como nota \citet{castelo2012}, no âmbito dos estudos em Psicologia e em Psicolinguística, o termo usado é, frequentemente, \textit{consciência metalinguística}.} Neste âmbito, são estudados, essencialmente, dois aspetos: a mobilização da atenção do sujeito, por um lado, e a língua e o seu uso enquanto objetos dessa atenção, por outro. Daqui resultam não só a reflexão do sujeito sobre as estruturas linguísticas e o seu uso em situações concretas como também o controlo deliberado sobre essas estruturas. Também nesta área são manifestações de consciência linguística \is{consciência!linguística} a capacidade de reconhecer a ambiguidade semântica dos enunciados (consciência semântica),\is{consciência!semântica} de manipular deliberadamente a estrutura sintática da frase (consciência sintática)\is{consciência!sintática} ou de segmentar as palavras nos sons que a constituem (consciência fonológica).\is{consciência!fonológica}

Vários especialistas têm defendido que o trabalho sobre consciência linguística \is{consciência!linguística} em contexto escolar permite a progressão do conhecimento linguístico implícito\is{conhecimento!linguístico implícito} para um estádio mais elevado, que se caracteriza pela capacidade de “identificar e nomear as unidades da língua (por exemplo, fonemas, sílabas, morfemas, palavras, grupos sintáticos, frases), de caracterizar as suas propriedades, as suas regras de combinação e os processos que atuam sobre as estruturas formadas.” \citep[17]{duarte2008}. Tal capacidade traduz-se na verbalização (ou explicitação) do conhecimento implícito\is{conhecimento!linguístico implícito} que o falante possui e dos princípios que regulam o uso escrito e oral desse conhecimento. Esta capacidade de explicitação do conhecimento linguístico, associada a contextos de aprendizagem formal - de reflexão sobre a língua - é gradual: em fases iniciais, os processos de identificação e de nomeação são explorados através da manipulação das unidades de língua e, após a estabilização do seu conhecimento, através do recurso progressivo a metalinguagem adequada.\footnote{Em consonância com esta perspetiva, para o ensino da língua, entende-se por \textit{conhecimento explícito} da língua o conhecimento refletido e sistematizado das unidades, regras e processos gramaticais da língua, o que implica o desenvolvimento de processos metacognitivos quase sempre dependentes da instrução formal; o desenvolvimento desta competência garante aos falantes o controlo das regras que usam e a seleção de estratégias mais adequadas à compreensão e produção em diferentes situações de comunicação \citep{simsim_etal1997}.}  

Como refere \citet{duarte2008}, para atingir elevados níveis de desempenho na competência de \isi{leitura}, de \isi{escrita} e no domínio de géneros formais e públicos do oral, é necessário que o conhecimento da língua envolvido seja, em grande medida, explícito, o que se consegue através de aprendizagens formais. O trabalho sobre o conhecimento implícito\is{conhecimento!linguístico implícito} e as atividades sobre consciência linguística\is{consciência!linguística} são determinantes não só para o desenvolvimento linguístico como também para a construção do conhecimento explícito sobre a língua no sentido de formar indivíduos proficientes no domínio do oral formal, na \isi{escrita} e na \isi{leitura}. Neste sentido, \citet[35]{simsim_etal1997} defendem que compete à escola “contribuir para o crescimento linguístico de todos os alunos, estimulando-lhes o desenvolvimento da linguagem e promovendo a aprendizagem das competências que não decorrem do processo natural de aquisição”.

O objetivo deste capítulo é descrever aspetos do conhecimento sintático (no que diz respeito quer a conhecimento linguístico quer a consciência linguística\is{consciência!linguística}) e o impacto que o trabalho sobre tal conhecimento tem na aprendizagem da \isi{escrita}.

\section{Caracterização do conhecimento sintático}
\label{sec:costacostagoncalves_caracterizacao}
O conhecimento sintático dos falantes é constituído por regras e princípios que lhes permitem produzir e compreender um conjunto potencialmente infinito de combinações livres de palavras que constituem as frases possíveis da língua. Neste sentido, a frase é entendida como uma combinação de palavras que estabelecem relações entre si, respeitando os padrões de ordenação admitidos na língua. São exemplos das relações que se estabelecem entre os elementos da frase a concordância (entre o sujeito e o verbo, por exemplo) e as relações de dependência (por exemplo, a obrigatoriedade de ocorrência de um complemento locativo no contexto do verbo \textit{ir}, o que decorre das propriedades de seleção deste verbo: \textit{A Teresa foi \textbf{a Paris}}).

O conjunto de regras e princípios que constituem o conhecimento sintático dos falantes é adquirido espontaneamente ao longo do processo de aquisição da língua, resultando da faculdade inata da linguagem e da exposição aos dados linguísticos que são fornecidos pela comunidade em que os sujeitos se encontram inseridos. Tal conhecimento caracteriza-se por intuições sobre as unidades sintáticas e as relações que se estabelecem entre as mesmas, como se descreve de seguida.\footnote{Para uma descrição mais alargada do conhecimento sintático, ver \citet{duartebrito1996}.}

Assim, os falantes possuem intuições sobre a classe a que pertencem as unidades sintáticas (que incluem as palavras), em função do contexto em que ocorrem, ou seja, tendo em conta os elementos que as precedem e/ou lhes sucedem, como ilustrado em (\ref{ex:costacostagoncalves_1}). Neste caso, as palavras \textit{adultos} e \textit{adolescentes} pertencem à mesma classe\is{classes de palavras} (nomes), sendo, por isso, precedidas de um determinante, ao contrário da palavra \textit{trabalhava} (verbo), que não pode ocorrer no mesmo contexto.

\ea\label{ex:costacostagoncalves_1}
\ea[]{\label{ex:costacostagoncalves_1a} Os \textit{adultos} não compreendem as crianças.}
\ex[]{\label{ex:costacostagoncalves_1b} Os \textit{adolescentes} não compreendem as crianças.}
\ex[*]{\label{ex:costacostagoncalves_1c} Os \textit{trabalhava} não compreendem as crianças.}
\zl

Esta capacidade de distinguir, ainda que intuitivamente, \isi{classes de palavras} permite que os falantes não considerem ambíguas as duas ocorrências da forma \textit{limpa} na frase (\ref{ex:costacostagoncalves_2}):

\ea\label{ex:costacostagoncalves_2} Pessoa limpa limpa o pó sempre que o vê.\z

Na frase (\ref{ex:costacostagoncalves_2}), a primeira ocorrência de limpa corresponde a um adjetivo, concordando em número e género com o nome que modifica (\textit{pessoa}); veja-se que sequências como *\textit{pessoa limpas} ou *\textit{pessoa limpo} seriam agramaticais. Por sua vez, a segunda ocorrência corresponde a um verbo, concordando em pessoa e número com o constituinte que desempenha a função sintática de sujeito, \textit{pessoa limpa}; a ausência de tal relação de concordância dá origem a uma sequência agramatical: *\textit{Pessoa limpa limpam o pó sempre que o vê}. 

Tendo em conta os contextos em que as palavras ocorrem, os falantes procedem a combinações das mesmas, construindo frases. Tais frases são identificadas pelos falantes não apenas como sequências lineares de palavras, mas como o resultado da combinação dessas palavras em expressões maiores – os \textit{constituintes} ou \textit{sintagmas} –, de acordo com determinados padrões de ordem. Os constituintes, como a expressão \textit{os primos da Ana}, em (\ref{ex:costacostagoncalves_3a}), são unidades sintáticas coesas, pelo que podem ser sujeitos a operações de substituição (como em (\ref{ex:costacostagoncalves_3b})) ou de deslocação (como em (\ref{ex:costacostagoncalves_3c})).

\ea\label{ex:costacostagoncalves_3}
\ea\label{ex:costacostagoncalves_3a} O João encontrou \textit{os primos da Ana} no cinema.
\ex\label{ex:costacostagoncalves_3b} O João encontrou-\textit{os} no cinema.
\ex\label{ex:costacostagoncalves_3c} \textit{Os primos da Ana}, o João encontrou(-os) no cinema.
\zl

O conhecimento sobre a organização das frases em constituintes permite que o falante reconheça e resolva ambiguidades que resultem de diferentes possibilidades de combinação desses constituintes, como se ilustra em (\ref{ex:costacostagoncalves_5}):

\ea\label{ex:costacostagoncalves_5}O Pedro trouxe computadores da China.\z


À frase (\ref{ex:costacostagoncalves_5}) podem ser atribuídas duas interpretações, uma vez que o constituinte \textit{da China} pode ser interpretado como o lugar de onde o Pedro trouxe computadores (sendo um complemento do verbo \textit{trazer}) ou como o tipo de computadores que o Pedro trouxe (correspondendo a um modificador do nome \textit{computadores}, obtendo-se a \isi{leitura} de que os computadores são \textit{chineses}). A desambiguação da frase pode ser feita através da aplicação de um teste de substituição: no caso da primeira \isi{leitura}, \textit{computadores} e \textit{da China} correspondem a dois constituintes distintos, pelo que é possível substituir o primeiro – e apenas este – por um pronome pessoal (ver (\ref{ex:costacostagoncalves_6a})); no segundo caso, computadores da China corresponde a um único constituinte, sendo, por isso, pronominalizável na sua totalidade (ver (\ref{ex:costacostagoncalves_6b})):

\ea\label{ex:costacostagoncalves_6}
\ea\label{ex:costacostagoncalves_6a} O Pedro trouxe-\textit{os} da China.
\ex\label{ex:costacostagoncalves_6b} O Pedro trouxe-\textit{os}.
\zl

Teoricamente, é possível formar um conjunto ilimitado de frases a partir de um conjunto finito de regras. Este facto decorre de uma propriedade das línguas designada como \textit{\isi{recursividade}}, termo que designa a possibilidade de aplicar sucessivamente a mesma regra na construção dos constituintes e das frases. Este processo contribui crucialmente para a \textit{criatividade linguística}. Considere-se o sintagma nominal (SN) em (\ref{ex:costacostagoncalves_7}):

\ea\label{ex:costacostagoncalves_7} a rapariga\z

A partir deste SN simples, é possível formar SNs mais complexos por encaixe sucessivo de orações \isi{relativas}, por exemplo:

\ea\label{ex:costacostagoncalves_8}
\ea\label{ex:costacostagoncalves_8a} a rapariga [que rasgou as revistas]
\ex\label{ex:costacostagoncalves_8b} a rapariga [que rasgou as revistas [que o Pedro comprou na tabacaria]]
\ex\label{ex:costacostagoncalves_8c} a rapariga [que rasgou as revistas [que o Pedro comprou na tabacaria [que fica perto do mercado]]]
\ex\label{ex:costacostagoncalves_8d} a rapariga [que rasgou as revistas [que o Pedro comprou na tabacaria [que fica perto do mercado [onde a mãe dele faz as compras]]]]
\zl

Poder-se-ia expandir sucessivamente o SN de (\ref{ex:costacostagoncalves_7}) aplicando o princípio da \isi{recursividade}. O resultado seria um SN bem formado gramaticalmente, com alto nível de informatividade, mas que colocaria problemas quer à produção quer à compreensão. Em ambas as modalidades, à medida que o SN se vai alongando poderá haver problemas de processamento por, por exemplo, haver impacto na memória de trabalho e nos mecanismos atencionais que têm de manter acessível a entidade em referência (\textit{a rapariga}). Note-se, contudo, que a manipulação das unidades sintáticas para alargamento e aumento da informação é uma estratégia muito utilizada em jogos verbais infantis, contribuindo para o desenvolvimento do comprimento médio dos enunciados de forma estruturada. Vejam-se, por exemplo, os encaixes sucessivos de frases \isi{relativas} em \textit{O Castelo de Chuchurumel}, bem como a justaposição de frases e o paralelismo estrutural em \textit{O Cuco que não Gostava de Couves} e \textit{O Macaco de Rabo Cortado} (lengalengas infantis). 

A combinação das unidades sintáticas em constituintes e em frases obedece a padrões de ordenação que restringem as possibilidades de coocorrência de tais unidades. Neste sentido, ao adquirirem a língua, os falantes passam a ter intuições sobre a ordem de palavras nas frases e no interior dos constituintes. Assim, por exemplo, um falante de português reconhece que o sujeito, nesta língua, ocupa, normalmente, uma posição pré‑verbal (a posição básica; ver (\ref{ex:costacostagoncalves_9a})), ainda que, sob determinadas circunstâncias, possa ocorrer pós‑verbalmente (ver (\ref{ex:costacostagoncalves_9b})):

\ea\label{ex:costacostagoncalves_9}
\ea\label{ex:costacostagoncalves_9a} Os amigos do Manuel adoram futebol.
\ex\label{ex:costacostagoncalves_9b} Adoram futebol, os amigos do Manuel.
\zl

Sabe, ainda, que, nos SNs, os determinantes precedem os nomes (ver (\ref{ex:costacostagoncalves_10a})), pelo que identificam (\ref{ex:costacostagoncalves_10b}) como uma sequência agramatical:
\ea\label{ex:costacostagoncalves_10}
\ea[]{\label{ex:costacostagoncalves_10a} os meninos}
\ex{*}{\label{ex:costacostagoncalves_10b} meninos os}
\zl

Os falantes possuem, ainda, intuições sobre as relações que se estabelecem entre os constituintes das frases e o seu impacto na construção do significado. Assim, são capazes de atribuir significados distintos a frases compostas pelos mesmos constituintes, mas em que se estabelecem diferentes relações sintáticas, como em (\ref{ex:costacostagoncalves_11}).

\ea\label{ex:costacostagoncalves_11}
\ea\label{ex:costacostagoncalves_11a} Os manifestantes agrediram os polícias.
\ex\label{ex:costacostagoncalves_11b} Os polícias agrediram os manifestantes.
\zl

Neste par de frases, o constituinte \textit{os manifestantes} é o sujeito em (\ref{ex:costacostagoncalves_11a}), mas o complemento direto em (\ref{ex:costacostagoncalves_11b}), verificando-se o contrário relativamente ao constituinte \textit{os polícias}. Este facto explica a diferença de interpretação entre ambas as frases: no primeiro caso, os agressores foram os manifestantes, verificando-se o contrário em (\ref{ex:costacostagoncalves_11b}).

No entanto, há casos em que a alteração das relações sintáticas não implica diferenças acentuadas de significado. É isto que acontece nos pares de frases ativa/passiva, como se mostra em (\ref{ex:costacostagoncalves_12}):

\ea\label{ex:costacostagoncalves_12}
\ea\label{ex:costacostagoncalves_12a} Os professores elogiaram todos os alunos da turma.
\ex\label{ex:costacostagoncalves_12b} Todos os alunos da turma foram elogiados pelos professores.
\zl

	O conhecimento sintático permite, ainda, que os falantes sejam capazes de estabelecer relações de dependência entre elementos que não se encontram adjacentes (designadas como \isi{dependências a longa distância}). Assim, em frases como (\ref{ex:costacostagoncalves_13}), é possível identificar uma relação entre o constituinte \textit{onde} e o verbo \textit{ir}: o primeiro instancia o complemento do segundo. Estas duas unidades não se encontram adjacentes, tendo o complemento sido movido da sua posição básica (ou seja, à direita do verbo) para a posição inicial de frase. Como resultado deste movimento, a posição à direita do verbo deixa de ter realização lexical, mas os falantes conseguem associá-la à posição final do constituinte \textit{onde} (no início da frase), o que garante o estabelecimento da dependência\is{dependências a longa distância} entre este constituinte e o verbo. 
    
\ea\label{ex:costacostagoncalves_13} Onde pensas que o João vai?\z

Nos exemplos até agora apresentados, não existem constituintes elididos. No entanto, nem sempre isso acontece. Na verdade, existem muitos casos em que nem todos os elementos estão realizados (como nas frases elíticas), o que não impede que os falantes interpretem a frase, uma vez que são capazes de recuperar a informação em falta a partir do contexto. É isso que se verifica na resposta em (\ref{ex:costacostagoncalves_14}), em que os dois complementos do verbo \textit{oferecer} – o objeto direto \textit{rosas} e o objeto indireto \textit{à Teresa} –, não estando presentes, são recuperáveis a partir da pergunta:

\ea\label{ex:costacostagoncalves_14} P: Alguém ofereceu rosas à Teresa?\\R: O Pedro ofereceu. (= O Pedro ofereceu [rosas] [à Teresa])\z

À entrada na escola, as crianças conhecem já (mesmo que apenas implicitamente) as estruturas sintáticas essenciais que lhes permitem produzir e compreender um vasto conjunto de enunciados. Assim, desde cedo que emergem nos enunciados produzidos pelas crianças, por exemplo, SNs constituídos por determinante e nome; sintagmas verbais com verbos de diferentes tipos (transitivos, intransitivos, por exemplo); a relação de concordância sujeito‑verbo; frases com ordem básica SVO (sujeito + verbo + objeto(s)) e com algumas ordens inversas (cf. (\ref{ex:costacostagoncalves_9b})). A estes aspetos, que constituem aquisições precoces, opõem-se estruturas de aquisição tardia, que, geralmente, emergem já durante o percurso escolar, como, por exemplo, as frases \isi{relativas} (em particular, as \isi{relativas} em que o pronome relativo corresponde ao objeto – \isi{relativas} de objeto; cf. \citealt{costa_etal2011}) ou as frases passivas \citep{simsim1998}. Para que a escola possa programar intervenções eficazes para o desenvolvimento das competências de língua, tem de ter em consideração estas diferenças entre aspetos do conhecimento linguístico já adquiridos e aspetos cujo domínio está ainda em estabilização. Para isso, deve promover estratégias que envolvam (i) a exploração de estruturas de aquisição precoce, estimulando a sua complexificação na produção, e (ii) uma progressiva exposição às estruturas menos frequentes e/ou mais complexas, em particular às de desenvolvimento tardio, contribuindo, assim, para o desenvolvimento linguístico das crianças quer ao nível da produção quer ao nível da compreensão.

\section{Conhecimento sintático implícito e consciência sintática\is{consciência!sintática}}
\label{sec:costacostagoncalves_conhecimento_sint_impl}

Como se afirmou na introdução a este capítulo, o \textit{conhecimento implícito}\is{conhecimento!linguístico implícito} dos falantes é constituído pelo conjunto de regras e princípios que aqueles adquirem de forma espontânea, a partir da interação entre a faculdade da linguagem e o \textit{input} linguístico que o meio fornece. O conhecimento sintático implícito,\is{conhecimento!linguístico implícito} em particular, integra o conjunto de regras e princípios de construção das frases (subjacentes aos aspetos abordados na secção anterior) que o falante usa para construir e compreender os enunciados, sem, no entanto, ter consciência das suas propriedades. 

Apesar da complexidade do sistema, o processo de aquisição da língua materna é relativamente rápido. Com efeito, à entrada no 1.º Ciclo, a criança domina de forma eficiente as estruturas sintáticas essenciais da sua língua, o que lhe permite produzir e compreender uma vasta quantidade de enunciados, que usa nas trocas conversacionais com os outros indivíduos da sua comunidade linguística. No entanto, nesse momento, o processo de aquisição não está ainda completo: vários trabalhos têm mostrado que algumas estruturas estabilizam muito cedo, como a ordem relativa entre o determinante e o nome, mas outras, como as frases \isi{relativas} ou as estruturas que envolvem contraste semântico, são de aquisição tardia (vejam-se os capítulos 6 a 11 do presente volume). Nestes casos, a escola assume um papel crucial, no sentido de promover o desenvolvimento do conhecimento implícito\is{conhecimento!linguístico implícito} das crianças. Em concreto, é o contacto com o meio linguístico típico dos contextos de escolarização que providencia a necessária exposição a estruturas linguísticas mais complexas e menos frequentes.

A par do desenvolvimento do conhecimento sintático implícito,\is{conhecimento!linguístico implícito} vai emergindo uma outra capacidade, em que se manifesta o controlo deliberado dos falantes sobre as estruturas da língua – a \textit{consciência linguística},\is{consciência!linguística} como referido na secção introdutória a este capítulo. A \textit{consciência sintática},\is{consciência!sintática} em particular, consiste na capacidade de (i) identificar e manipular as unidades sintáticas (frases, sintagmas e palavras), bem como as relações que entre elas se estabelecem (como as relações de concordância e de dependência),\is{dependências a longa distância} e (ii) elaborar juízos sobre a gramaticalidade de uma frase e proceder à sua correção \citep{simsim1998,duarte2008}. Saliente-se, uma vez mais, que o processo de escolarização, que envolve um treino progressivo de reflexão metacognitiva e metalinguística nas diferentes áreas do currículo, desempenha um papel fundamental no desenvolvimento da consciência linguística\is{consciência!linguística}.

A investigação sobre a consciência sintática\is{consciência!sintática} em português europeu é ainda escassa, sobretudo quando comparada com a quantidade de trabalhos que se centram na consciência fonológica\is{consciência!fonológica} (ver capítulo 18). No entanto, sabe-se que, muitas vezes, a capacidade para emitir juízos formais sobre a gramaticalidade das frases é fortemente limitada por fatores de natureza semântica e não sintática, o que mostra que as crianças são mais sensíveis ao conteúdo do que à forma das frases (ver, por exemplo, \citealt{gombert1992,simsim1998}).

A estimulação e a avaliação da consciência sintática\is{consciência!sintática} têm-se centrado na aplicação de tarefas orais que envolvem julgamento de frases, correção de sequências agramaticais, replicação e localização de erros, completamento de frases, categorização de palavras e analogias sintáticas. Neste domínio, salienta-se, ainda, a manipulação de estruturas que implicam operações de (i) alargamento (complexificação de constituintes, no sentido de formar enunciados progressivamente mais longos, como se ilustrou em (\ref{ex:costacostagoncalves_7}) e (\ref{ex:costacostagoncalves_8})), (ii) substituição (para a identificação de \isi{classes de palavras}, de constituintes e de funções sintáticas, como ilustrado em (\ref{ex:costacostagoncalves_3b})), (iii) redução (para a distinção entre elementos essenciais para a gramaticalidade dos enunciados – os complementos – e elementos acessórios – os modificadores), (iv) segmentação (para identificação dos constituintes da frase, operação essencial para determinar funções sintáticas), (iv) deslocação (para identificar os constituintes da frase e determinar o papel do contexto na seleção da ordem de palavras adequada, como em (\ref{ex:costacostagoncalves_3c}), e (v) complexidade sintática (para estimular a formação de frases complexas a partir de frases simples, com recurso aos \isi{conectores} adequados).\footnote{Sobre atividades de estimulação e de avaliação da consciência sintática\is{consciência!sintática} através da manipulação de estruturas que implicam estas operações, ver \citet{duarte2008}.}

Considerando que a manipulação de estruturas constitui a forma de avaliar a consciência sintática\is{consciência!sintática}, \citet{alexandre2010} e \citet{costa2010} desenvolveram, para o português europeu, dois estudos com o objetivo de trabalhar classes\is{classes de palavras} sintáticas e constituintes sintáticos, usando duas tarefas: manipulação por substituição \citep{alexandre2010} e manipulação por reconstituição, que implica a formulação de juízos de gramaticalidade e a reconstrução das sequências agramaticais \citep{costa2010}. Participaram nos estudos 84 crianças, sendo que 40, com idades entre os 6;04 e os 7;11, frequentavam o 1.º ano, e 44, com idades entre os 9;01 e os 10;05, frequentavam o 4.º ano. Os resultados mostraram que as crianças são capazes de realizar ambas as tarefas, ainda que os desempenhos tenham sido distintos consoante (i) a tarefa (melhores resultados globais na tarefa de substituição), (ii) as unidades sintáticas a manipular (nomes com melhores resultados do que verbos; sintagmas com melhores resultados do que palavras isoladas, por exemplo) e (iii) o ano de escolaridade (melhores resultados globais nas crianças do 4.º ano).

No âmbito dos estudos sobre consciência sintática\is{consciência!sintática}, destaque-se, ainda, o instrumento de avaliação da linguagem oral para o português europeu desenvolvido em \citet{simsim1997}. Neste estudo, são aplicadas duas tarefas que recrutam consciência sintática\is{consciência!sintática}: a reflexão morfossintática (julgamento de frases e, em caso de agramaticalidade, identificação e correção do erro) e completamento de frases. Os resultados são idênticos aos obtidos por \citet{alexandre2010} e \citet{costa2010}: há diferenças de desempenhos em função da idade (as crianças com 6 anos apresentam piores resultados do que as crianças com 9 anos) e da estrutura a manipular (frases simples permitem melhores desempenhos que frases complexas). Tais resultados confirmam um efeito esperado de desenvolvimento linguístico.

Resultados da investigação sobre estimulação do conhecimento sintático implícito\is{conhecimento!linguístico implícito} e da consciência linguística\is{consciência!linguística} apontam no sentido de existir uma relação entre tal trabalho e a aprendizagem da \isi{leitura} e da \isi{escrita}: no que diz respeito à \isi{leitura}, a criança pode usar pistas estruturais para a construção do significado de palavras desconhecidas, recorrendo ao seu contexto de ocorrência, e para a integração das informações lidas \citep{tunmer1990}; no que diz respeito à \isi{escrita}, a promoção de atividades de desenvolvimento do conhecimento implícito\is{conhecimento!linguístico implícito} e de estimulação da consciência linguística\is{consciência!linguística} permite que a criança use, nas suas produções, construções mais diversificadas, bem como frases e sintagmas mais extensos e complexos \citep{barbeiro1994,alcosta2010}. Na secção seguinte, apresentam-se alguns desses resultados, com especial destaque para o desenvolvimento de competências de \isi{escrita}.

\section{Manifestações do conhecimento sintático na aprendizagem da \isi{escrita}}
\label{sec:costacostagoncalves_manifestacoes}

Vários estudos têm evidenciado a relação existente entre certos géneros textuais\is{géneros de discurso} e a frequência de estruturas linguísticas, em particular de estruturas sintáticas específicas. Assim, com base em análise de \textit{corpora}, tem sido encontrada uma maior frequência de estruturas paratáticas em fala espontânea, em contraste com um maior recurso a estratégias de subordinação em registos escritos. Em concreto, na distinção de aspetos microestruturais diferenciadores de géneros textuais,\is{géneros de discurso} é possível associar o uso de marcadores fáticos a textos dialogais, a presença de formas verbais no pretérito perfeito e de expressões temporais a narrativas e a acumulação de \isi{conectores} que estruturam relações retóricas lógicas, como \isi{condicionais}, explicativas, \isi{causais} ou contrastivas, a textos de carácter explicativo e argumentativo.

Quando as crianças começam a dominar padrões de \isi{escrita} compositiva,\footnote{O domínio da \isi{escrita} compositiva corresponde à capacidade de combinar expressões linguísticas para formar um texto.} aproximadamente por volta do terceiro ano de aprendizagem formal \citep{martinsniza1998}, ainda não dispõem de um repertório suficientemente alargado de estruturas sintáticas necessárias ao uso multifuncional da \isi{escrita}, para fazer face aos desafios académicos e resolver problemas do quotidiano. Diversos aspetos do conhecimento lexical e gramatical, em particular do conhecimento sintático, estão ainda em desenvolvimento, como se afirmou anteriormente, e da sua emergência ou estabilização depende o desenvolvimento da \isi{escrita}. Contudo, importa destacar que o conhecimento implícito\is{conhecimento!linguístico implícito} de uma dada estrutura e a sua consciência linguística\is{consciência!linguística} não devem ser linearmente entendidos como condições prévias (ou como precursores) do desenvolvimento da \isi{escrita}. De facto, são alguns \isi{géneros de discurso}, requeridos no percurso de escolarização, que constituem o contexto linguístico que obriga ao aparecimento de estruturas menos frequentes e mais complexas. Como exemplo da interação entre desenvolvimento do conhecimento sintático implícito,\is{conhecimento!linguístico implícito} da consciência sintática\is{consciência!sintática} e do desenvolvimento da \isi{escrita} compositiva, apresentam-se em seguida resultados de investigação sobre a produção de construções \isi{relativas}, atendendo à necessária complexificação de SNs em sequências mais descritivas de textos narrativos, e sobre a produção de conexões proposicionais \isi{causais} e contrastivas, que sustentam a estrutura argumentativa de textos de opinião.

\subsection{Conhecimento de relativas e desenvolvimento da escrita de narrativas}
\label{subsec:costacostagoncalves_conhecimento:relativas}
\is{relativas}\is{escrita}
Resultados de estudos sobre a aquisição de algumas construções \isi{relativas} por crianças e adolescentes confirmam a ideia de que este conhecimento se estabiliza tardiamente, provavelmente a par do desenvolvimento de aprendizagens formais que acompanham o período de escolarização. Em \citet{costa_etal2011}, num estudo que compara os desempenhos de crianças entre os 3 anos e 9 meses e os 6 anos e 4 meses com os desempenhos de adultos numa tarefa de indução de produção de \isi{relativas} e numa tarefa de compreensão, verificou-se existirem assimetrias assinaláveis quer na produção quer na compreensão de \isi{relativas} de objeto, como a que se exemplifica em (\ref{ex:costacostagoncalves_15}), em contraste com \isi{relativas} em que o constituinte relativo tem a função sintática de sujeito, como em (\ref{ex:costacostagoncalves_16}). Concretamente, as crianças compreendem e produzem com mais facilidade estruturas \isi{relativas} como a presente na frase em (\ref{ex:costacostagoncalves_16}) do que \isi{relativas} como a ilustrada em (\ref{ex:costacostagoncalves_15}).

\ea\label{ex:costacostagoncalves_15} Eu vi o rapaz que a mãe abraçou.\z
\ea\label{ex:costacostagoncalves_16} Eu vi o rapaz que abraçou a mãe.\z 

Enquanto o défice de produção de \isi{relativas} de objeto pelos adultos se explica por estes optarem pelo recurso a uma estratégia alternativa legítima, as construções passivas, as crianças, incluindo as mais velhas, já em idade escolar, evidenciam uma diferença significativa entre a produção de \isi{relativas} de sujeito (com 78\%) e a produção de \isi{relativas} de objeto (com 31\%). O facto de estas últimas estruturas serem de aquisição tardia é corroborado pelos resultados do teste de compreensão aplicado às crianças, que, ao contrário dos adultos, mantêm dificuldades evidentes na compreensão de \isi{relativas} de objeto, com 49,5\% de acertos, ao passo que compreendem de forma bem-sucedida a quase totalidade de \isi{relativas} de sujeito (99,5\%).

Se estes resultados indiciam problemas na compreensão e na produção de \isi{relativas} de objeto ainda à entrada da escolaridade, outros trabalhos de investigação corroboram a necessidade de promover a consciência sintática\is{consciência!sintática} de outros aspetos do conhecimento de \isi{relativas}. Esse é o caso de \citet{fontes2008} e de \citet{valente2008}, dois estudos implementados em turmas dos 4.º, 6.º, 9.º e 12.º anos com a finalidade de descrever o conhecimento de \isi{relativas} restritivas ao longo da escolaridade. Ambos os estudos aplicam o mesmo desenho experimental, que consiste em duas tarefas escritas,\is{escrita} uma envolvendo juízos de gramaticalidade (e, consequentemente, algum grau de consciência linguística\is{consciência!linguística}), e outra, a produção induzida de frases, através da ligação de orações. 

Os resultados da tarefa de juízo de gramaticalidade mostram uma clivagem entre os desempenhos dos alunos mais novos, dos 4.º e 6.º anos, e os dos alunos dos 9.º e 12.º anos. O teste aplicado inclui quer frases \isi{relativas} canónicas (ver (\ref{ex:costacostagoncalves_17})) quer estruturas não canónicas, como as cortadoras (ver (\ref{ex:costacostagoncalves_18})) e as resuntivas (ver (\ref{ex:costacostagoncalves_19})). Os autores consideram ainda um terceiro tipo de estratégia de relativização, a pseudo-resuntiva, que se aproxima da estratégia resuntiva por apresentar a duplicação do pronome, embora o pronome introdutor da relativa\is{relativas} tenha um comportamento igual ao das \isi{relativas} canónicas (ver (\ref{ex:costacostagoncalves_20})).\footnote{Os exemplos são retirados de \citet{fontes2008}.}

\ea[]{\label{ex:costacostagoncalves_17} A Maria encontrou o livro que o professor recomendou.}\z
\ea[*]{\label{ex:costacostagoncalves_18} O presente que a Joana mais gostou foi o perfume.}\z
\ea[*]{\label{ex:costacostagoncalves_19} O Pedro ficou hospedado no hotel que eu acho que já estive lá com o meu pai.}\z
\ea[*]{\label{ex:costacostagoncalves_20} Este é o aluno a quem o professor lhe deu má nota.}\z

Até ao final do 2.º ciclo de escolaridade, a percentagem de não aceitabilidade de estratégias não canónicas de \isi{relativas} encontra-se abaixo dos 50\% \citep[156]{fontes2008}. Por outras palavras, as crianças até aos doze anos parecem aceitar estratégias de relativização não canónicas, ao contrário do que seria desejável. É preciso chegar ao final do 9.º ano para que a percentagem de rejeição de estratégias tradicionalmente excluídas da gramática do português europeu seja próxima dos 80\%, mantendo-se, porém, uma elevada aceitabilidade da estratégia cortadora \citep[81]{valente2008}. A elevada aceitabilidade desta estratégia de relativização deixa antever mais dificuldades na produção de \isi{relativas} em que o constituinte relativizado é um sintagma preposicional do que nas frases em que o constituinte relativo tem natureza nominal. Os resultados da tarefa de produção induzida de \isi{relativas} em ambos os estudos confirmam estas dificuldades de produção, através das quais se podem inferir consequências negativas na qualidade da expressão \isi{escrita}. Considerando os dados dos alunos mais novos, em \citet[159]{fontes2008}, os resultados deixam uma vez mais em evidência que a gramática correspondente ao conhecimento implícito\is{conhecimento!linguístico implícito} destes alunos ainda legitima a produção \isi{escrita} da estratégia cortadora: no 4.º ano, apenas 27,5\% dos alunos produzem constituintes relativos preposicionados e, no 6.º ano, embora se observe crescimento, ainda se observam dificuldades assinaláveis com este tipo de \isi{relativas}, havendo apenas 41,3\% de \isi{relativas} preposicionadas canónicas. Destes resultados, destaca-se, portanto, uma tendência generalizada para a preferência pela estratégia cortadora sempre que o constituinte a relativizar é um sintagma preposicional, tanto nas produções do 4.º ano (ver (\ref{ex:costacostagoncalves_21})), como nas do 6.º ano (ver (\ref{ex:costacostagoncalves_22})):

\ea\label{ex:costacostagoncalves_21} O quadro que eu colei um autocolante tinha uma moldura dourada.\\~\hfill(F.129.4:6; 4º ano)\z
\ea\label{ex:costacostagoncalves_22} O Sr. João levou o carro à oficina que o filho trabalha por causa dos travões.\\~\hfill(M.136.6:21; 6º ano)\z

À semelhança do que se observa nos resultados do teste de juízo de gramaticalidade, só no 9.º ano se alcança uma percentagem de 71,4\% de \isi{relativas} preposicionadas canónicas, que, ainda assim, contrasta com uma taxa de sucesso da ordem dos 90\% na produção de \isi{relativas} com constituinte nominal \citep[103]{valente2008}. Os dados de ambos os estudos permitem concluir que a produção de \isi{relativas} preposicionadas canónicas é outra estratégia de relativização de aquisição tardia.

Uma conclusão de ordem mais geral que se extrai dos resultados do teste de produção dos alunos mais novos consiste na verificação de que há um crescimento progressivo na produção de frases \isi{relativas} canónicas entre o 4.º e o 6.º ano, o que é corroborado em outros estudos feitos com base em produções escritas.\is{escrita} 

Em \citet{costagoncalves2010}, apresenta-se um estudo longitudinal da produção \isi{escrita} de três sujeitos, entre os 8 e os 11 anos de idade, durante quatro anos cruciais para o desenvolvimento da \isi{escrita} compositiva (entre o 3.º e o 6.º ano de escolaridade). A comparabilidade dos materiais foi garantida por se tratar de um contexto de produção \isi{escrita} em situação de avaliação. De entre as produções escritas,\is{escrita} que incluem textos de carácter narrativo e textos não narrativos, para ilustrar aspetos do desenvolvimento sintático em curso ao longo da escolarização, destaca-se o crescimento do conhecimento de \isi{relativas}, associado à progressiva complexificação de SNs, com efeitos no enriquecimento de sequências descritivas encaixadas na estrutura da narrativa.

Nos textos narrativos analisados, a ocorrência de \isi{relativas} restritivas com antecedente pode ser observada logo nas narrativas dos alunos mais novos, do 3.º ano, o que é uma constatação nada surpreendente por se tratar de uma estrutura adquirida bastante antes da entrada para a escola \citep{vasconcelos1996}. Contudo, trata-se de um uso bastante reduzido: por um lado, as frases complexas são predominantemente assentes na coordenação copulativa e, por outro, o escasso uso de \isi{relativas} está associado a estratégias mínimas e fixas de modificação nominal. Estas características mantêm-se nos textos de 4.º ano, os quais são pautados por um estilo enumerativo de sequencialização de acontecimentos, com raros momentos de descrição de personagens ou de espaços. É no decurso do 2.º ciclo do ensino básico que se observa um maior recurso a estratégias de relativização, sendo a modificação nominal dos textos do 5.º ano assegurada não só por expressões adjetivais, mas ainda por \isi{relativas} restritivas e, também, por explicativas. Neste subconjunto de textos, além da novidade destes modificadores apositivos, é possível encontrar, ainda que esporadicamente, \isi{relativas} sem antecedente. No que respeita ao uso de \isi{relativas} com antecedente frásico, de que é exemplo (\ref{ex:costacostagoncalves_23}), retirado de \citet[310]{costagoncalves2010}, estas orações surgem somente em textos de 6.º ano.\newpage

\ea\label{ex:costacostagoncalves_23} 	Em pequeno, parecia perfeitamente normal e sem defeitos com as suas escamas verdes e as suas sete cabeças, \textit{o que era normal nos dragões}\\~\hfill(6.º ano)\z

Relativas\is{relativas} como a do exemplo (\ref{ex:costacostagoncalves_23}), ainda que evidenciem crescimento de conhecimento sintático em relação ao 1.º ciclo, são escassas mesmo nos textos dos alunos mais velhos, o que poderá ser relacionado com a forma como a informação está a ser processada, implicando um maior esforço na recuperação anafórica de uma unidade proposicional (o constituinte relativo \textit{o que} recupera a informação de toda a frase anterior).

O estudo longitudinal mencionado, através da análise de parâmetros relevantes para a avaliação da progressão da complexidade da \isi{escrita}, especificamente no domínio dos processos de formação de frases complexas, permite concluir que o conhecimento implícito\is{conhecimento!linguístico implícito} sobre \isi{relativas} – manifestado quer ao nível da frequência de uso na produção \isi{escrita}, quer ao nível da diversificação de estratégias de relativização – parece ser um indicador robusto de desenvolvimento da complexidade sintática necessária à \isi{escrita} compositiva. O recurso progressivo a \isi{relativas} tem efeitos no desenvolvimento da \isi{escrita} narrativa, nomeadamente na complexificação de sequências descritivas, assegurando a expansão de constituintes por modificação (restritiva e apositiva), como descrito em 2, e garantindo uma maior informatividade textual.

Os dados empíricos resultantes da investigação sobre o conhecimento implícito de \isi{relativas},\is{conhecimento!linguístico implícito} particularmente no domínio da produção, tornam clara a inter-relação entre desenvolvimento linguístico tardio e desenvolvimento da \isi{escrita} de narrativas. Outros géneros discursivos,\is{géneros de discurso} como os que requerem a elaboração de argumentos e de explicações, propiciam o desenvolvimento do conhecimento sintático-semântico de outras estruturas sintáticas complexas, especialmente no domínio da subordinação adverbial. 

\subsection{Conhecimento de \isi{concessivas}, desenvolvimento da \isi{escrita} argumentativa e conhecimento explícito da língua}
\label{subsec:costacostagoncalves_conhecimento_concess}

Para a estruturação da \isi{escrita} de carácter argumentativo, de que são exemplo produtos textuais típicos da escolarização, como as respostas a itens de avaliação da compreensão e interpretação de uma obra ou as redações em que se pede a opinião sobre um tema, as crianças necessitam de meios linguísticos para enunciar o seu ponto de vista e para o fundamentar com argumentos; na estruturação de argumentos, torna-se essencial o recurso a conexões proposicionais \isi{causais} ou explicativas, bem como a nexos contrastivos (adversativos\is{adversativas} e concessivos\is{concessivas}) e a nexos \isi{condicionais}, de forma a contra‑argumentar, fortalecendo a perspetiva defendida. Nos enunciados de (\ref{ex:costacostagoncalves_24}) a (\ref{ex:costacostagoncalves_26}), retirados de \citet{alcosta2010}, ilustra-se o uso de conexões \isi{causais}, \isi{condicionais} e contrastivas em textos de opinião sobre experiências laboratoriais em animais, escritos por alunos do 4.º ano.\footnote{Os exemplos apresentam-se transcritos com o formato das bases de dados do CHILDES \citep{macwhinney2000}; entre parênteses indica-se a idade da criança, com anos; meses. dias.}

\ea[*]{\label{ex:costacostagoncalves_24} FD4: Eu acho que nunca devemos fazer mal aos animais porque eles são nossos amigos.\hfill(9;11.02)}\z
\ea[*]{\label{ex:costacostagoncalves_25} FI4: Se as pessoas continuarem a testar os produtos em animais, os pobrezinhos podem ficar em vias de extinção ou até desaparecer [\%spe: dezaparecer].\hfill(9;10.12)}\z
\ea[*]{\label{ex:costacostagoncalves_26} FC4: Eu também acho que as pessoas não deviam fazer experiências com os animais que provocassem [\%spe: provocasem] sofrimento aos animais [\%spe: 0,] mas eu acho que é [*] importante [*] as descobertas das curas para as doenças do Homem para os animais e dos animais para o Homem. \hfill(9;5.01)}\z

Destes exemplos, salienta-se que a estruturação de unidades textuais argumentativas é assegurada essencialmente por três \isi{conectores}, \textit{porque}, \textit{se} e \textit{mas}, os quais fazem parte do conhecimento linguístico implícito\is{conhecimento!linguístico implícito} das crianças pelo menos desde os dois anos de idade \citep{costa_etal2008}.\footnote{Em \citet{costa_etal2008}, propõe-se a seguinte escala de emergência de \isi{conectores}: mas (2;2.8) > porque (2;3.22) > se (2;5.23).} Por outras palavras, o conhecimento linguístico\is{conhecimento!linguístico implícito} com que as crianças entram para a escola parece ser suficiente para assegurar a produção de discurso argumentativo escrito com os mesmos recursos linguísticos das interações orais espontâneas. Contudo, para que a \isi{escrita} argumentativa se desenvolva, são requeridas estratégias de complexificação das relações proposicionais, evitando-se a repetição de estruturas e garantindo níveis superiores de informatividade. Na verdade, no conjunto de 24 textos de onde se retiraram os enunciados (\ref{ex:costacostagoncalves_24}) a (\ref{ex:costacostagoncalves_26}), as crianças usam 35 vezes a conjunção \textit{porque} (e apenas dois outros \isi{conectores} \isi{causais}/explicativos) e 41 vezes \textit{mas} (e apenas dois outros \isi{conectores} contrastivos), o que deixa perceber um diagnóstico de \isi{escrita} com várias repetições estruturais e com consequentes limitações expressivas. Um longo caminho há, portanto, a seguir no desenvolvimento do conhecimento linguístico, incidindo no alargamento do conhecimento lexical de conexões proposicionais, com implicações no desenvolvimento do conhecimento sintático\is{conhecimento!linguístico implícito} (Secção \ref{sec:costacostagoncalves_conhecimento_sint_impl}).

Em \citet{alcosta2010}, com a finalidade de estudar as relações entre o conhecimento de estruturas contrastivas e a \isi{escrita} argumentativa, desenvolveu-se um estudo controlado para recolha sistemática de dados de \isi{escrita} com uma fase de diagnóstico, que incluiu um teste de produção induzida, um teste de compreensão oral e um teste de produção textual, e com uma fase de intervenção educativa, com um pré-teste e um pós‑teste que avaliavam os efeitos do conhecimento explícito da língua na expressão \isi{escrita}. O diagnóstico envolveu turmas dos 4.º, 6.º e 9.º anos e, ainda, um grupo de controlo com adultos escolarizados. 

Os resultados dos testes de produção induzida de frases e de compreensão oral de texto mostram que há crescimento no conhecimento de conexões \isi{concessivas} ao longo dos três ciclos de escolaridade: as percentagens de acerto evidenciam uma curva ascendente, destacando‑se o facto de se manterem abaixo dos 75\% no 4.º e no 6.º ano.\footnote{O teste de produção induzida de frases inclui um item de construção de resposta curta a partir de um conector\is{conectores} dado e itens de transformação. No tratamento das respostas a estes itens, foi feita uma análise em termos dicotómicos (produções desviantes \textit{vs}. produções conformes com a gramática alvo), o que permitiu uma análise quantitativa. O teste de compreensão oral de texto inclui itens de seleção de paráfrases e a suas respostas são dicotómicas.} Tal como se verifica nos estudos sobre \isi{relativas} \citep{fontes2008,valente2008}, é preciso chegar ao 9.º ano para se obterem resultados semelhantes ao do padrão dos adultos. Por exemplo, a taxa de acerto do teste de produção induzida de frases alcança os 88\% entre os alunos mais velhos. O que é de certa forma surpreendente nestes dados é que o conhecimento linguístico de \isi{concessivas}, manifestado nos resultados do 9.º ano, não é mobilizado em situação de \isi{escrita}: no diagnóstico de produção de texto, há apenas 21\% de ocorrências de \isi{conectores} concessivos,\is{concessivas} em contraste com 79\% de ocorrências de \isi{conectores} adversativos\is{adversativas} (com ampla preferência por \textit{mas})\footnote{Entre os \isi{conectores} adversativos\is{adversativas} usados, há 210 ocorrências de \textit{mas} e 37 ocorrências de outros \isi{conectores}.}. O problema da ativação de conhecimento linguístico já estabilizado em situações de textualização orientou a investigação para a avaliação dos efeitos do ensino da gramática no desenvolvimento da \isi{escrita}. Os resultados do pós-teste, um texto de opinião produzido após a intervenção, revelaram uma correlação positiva entre conhecimento explícito de \isi{concessivas} e o recurso a estas estruturas na \isi{escrita} argumentativa.

Os estudos mencionados confirmam a ideia de que, ao longo da escolaridade, vários aspetos do conhecimento linguístico se encontram ainda em desenvolvimento, sendo permeáveis aos estímulos que a escolarização deve providenciar. Segundo os dados apresentados, desde a entrada na escola até cerca do final da adolescência, vários aspetos do conhecimento sintático se encontram em estabilização, interagindo com o desenvolvimento progressivo de géneros discursivos.\is{géneros de discurso} Os domínios do conhecimento sintático consolidados no final da adolescência, que correspondem a conhecimento linguístico tardio, beneficiam de aprendizagens formais promotoras de níveis superiores de mestria de língua e de competência de \isi{escrita}.

\section{Manifestações do conhecimento sintático na produção de textos}
\label{sec:costacostagoncalves_manif_textos}

\subsection{Processos cognitivos envolvidos na produção de textos}
\label{subsec:costacostagoncalves_cog_textos}

Invocando o modelo de produção de fala de \citet{levelt1989}, e ressalvando as devidas distâncias impostas por efeito de modalidade oral ou \isi{escrita}, ao pensar a \isi{escrita} de texto é crucial considerar três importantes componentes do processo de produção verbal: a conceptualização, a formulação e a produção do enunciado. Encontra-se claro paralelo destas etapas nos modelos de produção \isi{escrita} propostos por e a partir de \citet{flowerhayes1981}. Aí, o planeamento e a produção são etapas nucleares que asseguram a evocação da informação e a sua transposição para o formato de texto, dando-se grande importância a uma terceira componente que se focaliza e se demora nos processos de revisão. Pode dizer-se que \citeauthor{levelt1989} estuda mais os processos menos acessíveis à observação, ou seja os processos mentais que permitem organizar estruturas conceptuais num formato linguístico interpretável, enquanto \citeauthor{flowerhayes1981} dão uma atenção particular aos processos de produção da \isi{escrita}, observáveis na análise das reformulações do texto. Por isso, \citeauthor{levelt1989} e seguidores têm mais impacto nos fundamentos psicolinguísticos para o estudo do processamento da produção de fala, enquanto \citeauthor{flowerhayes1981} e seguidores têm tido forte impacto nos fundamentos para práticas educacionais relativas ao ensino da \isi{escrita}.

Para este capítulo, conjugando \citet{levelt1989} e \citet{flowerhayes1981}, interessa particularmente a fase de formulação (ou produção) do enunciado porque é aquela que depende crucialmente do conhecimento linguístico mentalmente representado, requisito para uma produção eficiente em termos dos intentos comunicativos. Também interessa o funcionamento do sistema cognitivo de monitorização, aquele que supervisiona a forma do discurso produzido no sentido da adequação e da correção. Este sistema apoia‑se na memória de trabalho, que permite a manutenção ativa de informação já processada para que possa ser objeto de reformulação, e num conhecimento metalinguístico, que permite comparar, substituir ou corrigir estruturas textuais, como as lexicais ou as sintáticas. 

\newpage Produzir discurso, oral ou escrito, implica ter o que dizer sobre um tópico determinado, escolhido pelo próprio ou imposto externamente, como tantas vezes acontece na escola; implica converter pensamento em representações linguísticas, gerando uma mensagem com um formato verbal, oral ou escrito; implica controlar aspetos relativos à micro e à macro-organização discursiva. 

Para planear o que vai ser dito ou escrito, o primeiro passo é ativar informação relacionada armazenada em memória semântica e selecionar o que pode ser relevante para o enunciado a produzir. Quer as operações mentais de busca de informação quer as de seleção temática e lexical vão ser reguladas internamente por decisões pessoais e por perspetivas singulares. No caso da criança que escreve na escola, é importante assegurar que seja ativada informação relevante para o tema previamente definido, que haja uma representação do destinatário a quem o texto se dirige e que o escritor antecipe o efeito que a mensagem vai produzir no seu leitor. Só assim poderá fazer ajustamentos ao texto em produção, adequando-o o mais possível às intenções comunicativas que presidem à tarefa de \isi{escrita}.

Na fase de conceptualização e de planeamento, tanto \citet{levelt1989} como \citet{flowerhayes1981} referem um contexto mental em que capacidades cognitivas gerais estão em jogo para que a produção verbal seja bem-sucedida. Como se afirmou anteriormente, a ativação de informação relevante requer a existência de representações mentais do público a quem se dirige o discurso a ser produzido e também capacidades de antecipação do efeito desse discurso em quem o ouve ou lê. Essas capacidades são do domínio da Teoria da Mente, que defende que, para que haja sucesso na comunicação, é preciso que o locutor seja capaz de se imaginar na mente do outro com quem vai comunicar, adivinhando-lhe os pensamentos, sentimentos, conhecimentos e desejos, e, assim, melhor poder planear o seu próprio discurso \citep{miller2006}. Esta capacidade depende de um desenvolvimento sociocognitivo normal, que, por vezes, não está presente em perfis com défices pragmáticos, como acontece, por exemplo, no autismo.

\largerpage Por outro lado, produzir um texto adequado requer que haja acesso a formatos de textos específicos para dar corpo às ideias a transmitir da forma mais conveniente ao público, ou mais adequada à encomenda: modelos de narrativa, descrições mais ou menos detalhadas e ordenadas, ou estruturas mais ou menos padronizadas de textos explicativos (ver secção \ref{sec:costacostagoncalves_manifestacoes}). Esta é uma fase em que são usadas estruturas de pensamento que se podem traduzir em conceitos e relações lexicais, mas que não está ainda obrigatoriamente vinculada a uma forma gramatical. Depende de capacidades cognitivas gerais, da memória e do raciocínio, de focalização da atenção para selecionar, de experiência do mundo que incorpore informação sobre o discurso e da forma como ele é ligado ao contexto. 

A fase de conceptualização permite estabelecer a ponte entre uma intenção comunicativa e uma mensagem que tem de adquirir uma forma inteligível para aquele a quem se dirige. Ora, isso só é possível se essa mensagem mentalmente gerada for codificada gramaticalmente (operações de codificação gramatical como referidas em \citealt{bocklevelt2002}). São estas operações e os correlatos visíveis em forma de texto produzido que mais interessam nesta secção. 

A codificação gramatical das estruturas conceptuais precisa de suporte linguístico, de um conhecimento específico que esteja acessível e que, por vezes, para além de se constituir como um conhecimento intuitivo\is{conhecimento!linguístico implícito} que se usa sem esforço (ver Secção \ref{sec:costacostagoncalves_conhecimento_sint_impl}), precisa de ser um conhecimento consciente e explícito que se usa estrategicamente para resolver problemas de expressão (como explicado na Secção \ref{sec:costacostagoncalves_manifestacoes}).

Adotando a perspetiva de uma representação do conhecimento linguístico baseada nas hipóteses teóricas da gramática de base cognitivista \citep{chomsky1986}, e também com base em dados empíricos que sustentam os modelos de organização cerebral de base neurolinguística \citep{ullman2001}, aceita-se que, na fase de codificação gramatical, o sujeito tem acesso a duas componentes principais: uma lexical e outra computacional. Na primeira, está armazenada informação relativa ao conjunto de itens lexicais e respetivas propriedades que se encontra representado na mente dos falantes – o léxico mental. Da segunda, faz parte um conjunto de princípios que permitem combinar tais itens lexicais e construir sequências estruturais com significado interpretável (como referido na Secção \ref{sec:costacostagoncalves_conhecimento_sint_impl}).

\largerpage Acede-se ao léxico para nomear conceitos, propriedades e relações. Segundo \citet{levelt1989}, há duas fases a considerar. Num primeiro momento, haverá acesso a itens lexicais com informação relativa à sua categoria sintática e ao seu significado, mas ainda independente de propriedades impostas por um contexto sintático. Estas formas mentalmente representadas são designadas por \textit{lemas}, próximas da forma de citação que encontramos no dicionário. Mas aceder a itens lexicais não chega para transmitir ideias organizadas para comunicar, como o texto requer. A fase seguinte será a de colocar essas palavras em sequências linguísticas (contextos sintagmáticos) e estabelecer relações entre elas, o que obriga a um conhecimento para além do léxico. Esse conhecimento gramatical requer o uso de princípios de ordenação de palavras e de criação de relações de dependência\is{dependências a longa distância} entre essas palavras através de processos morfossintáticos. Só na sequência da combinação destes processos, sustentados pelo conhecimento linguístico implícito\is{conhecimento!linguístico implícito}, as estruturas conceptuais geradas serão inteligíveis para os interlocutores que partilham a mesma língua e a mesma gramática. Sendo isto verdade para a produção oral, mais notório se torna na \isi{escrita}, em que as convenções a respeitar são ainda mais evidentes. 

\subsection{Conhecimento sintático envolvido na produção de texto}
\label{subsec:costacostagoncalves_conh_sint_text}

Para iniciar a observação de dados linguísticos relativos à produção de texto, veja\x-se um caso interessante de produção \isi{escrita} de uma criança de 6 anos, no 1.º ano de escolaridade (retirado de \citealt{costa2008}; a barra (/) assinala final de linha). 

\begin{center}
\fbox{%
  \parbox{0.5\textwidth}{%
    \begin{flushleft}
      Eu gosto da minha professora /\\ e gosto da minha amiga Eunice e /\\gosto da minha amiga Carla /\\e eu gosto da minha amiga Isabel /\\e da minha amiga /\\Patrícia Carla
    \end{flushleft}
  }%
}\\\vspace{1em}Texto 1
\end{center}

O Texto 1 foi produzido por uma criança que começou a escrever apenas quando entrou na escola, a meio do seu 1.º ano de escolaridade. É uma manifestação de uma tentativa bem-sucedida de comunicar, por escrito, a expressão de sentimentos a partir de estruturas linguísticas básicas. Possivelmente, não porque não pudesse exprimir-se oralmente de forma bem mais complexa, mas porque, atendendo à sua representação do que é um texto e às ferramentas de \isi{escrita} disponíveis, foi o que esteve ao seu alcance. Usam-se itens lexicais para referir entidades (p. e., \textit{professora}, \textit{amiga Eunice}) e estabelecer relações entre entidades (\textit{gostar de}), que se combinam de acordo com princípios básicos de ordenação de palavras (neste caso, adota-se sempre a ordem SVO). Note-se que a primeira frase é usada como matriz que vai servir para ir encaixando mais informação, o que se faz pela simples substituição de núcleos nominais. Provavelmente por razões cognitivas de parcimónia e de tentativa de adaptação aos requisitos do que é um texto, há estratégias linguísticas que favorecem a coesão textual: para não ser uma lista de frases, há recurso à coordenação nominal e oracional, e à elipse do sujeito. Resulta num texto repetitivo, mas funcional, onde se adivinham estratégias de treino a partir do que já se domina.

Veja-se agora numa produção do 5.º ano (retirada de \citealt{costagoncalves2010}) como as mesmas estratégias permanecem ao serviço de uma competência de \isi{escrita} mais elaborada. 

\begin{center}
\fbox{%
  \parbox{\textwidth}{%
    \begin{flushleft}
      Era uma vez (\ldots)\\
      \textit{uma rainha, um rei, um príncipe, uma princesa, uma fada, um mordomo e, por fim, uma cozinheira.\\
      A rainha chamava-se Maria, o rei, João, o príncipe, também, João, a princesa, Mariana, a fada Marisa, o Mordomo, Frederico, e a cozinheira Catarina.\\
      (\ldots) Fantasmas, bruxas, esqueletos}, o que será?\\
      (\ldots) \textit{O rei, o príncipe, a princesa e a rainha} foram-se deitar.\\
      Os outros: \textit{o mordomo, a fada e a cozinheira, fizeram uma longa volta à casa}
    \end{flushleft}
  }%
}\\\vspace{1em}Texto 2
\end{center}

Neste caso, o aluno usa recorrentemente SNs complexos com a mesma estrutura, obtida por justaposição. Sendo um aluno do 5.º ano, poderá argumentar-se que já não se trata só de uma estratégia de aprendizagem de uma \isi{escrita} inicial, mas sim uma estratégia retórica para a construção do texto, designada por \textit{paralelismo estrutural} (utilização de sequências com a mesma estrutura em fragmentos textuais contíguos que pode explicar-se pela atuação de processos cognitivos que agilizam a produção). Quer uma quer outra estratégia são plausíveis, sendo as duas indicadoras da mobilização de um conhecimento sintático que, no início, apoia e, mais tarde, facilita a expressão de ideias na \isi{escrita}. Note-se que este uso de estruturas paralelas dá evidências de um conhecimento consciente da estrutura possível do SN e, mais ainda, evidencia o uso de recursos deliberados para criar textos coesos, o que parece ser um expoente do designado conhecimento transformado ou formatado (\textit{knowledge transforming}) como referem \cite{scardamaliabereiter1987}. Deste modo, com uma atenção focada na micro e na macroestrutura textual, potencia-se o efeito comunicativo que se quer atingir e isso faz‑se com base num conhecimento linguístico aprendido que é usado estrategicamente.

Compare-se agora o Texto 2 com o Texto 3. Este parece poder ser entendido como uma manifestação de uma outra relação com o processo de \isi{escrita}. Foi produzido por uma criança em meados do 1.º ano, que começou a escrever antes de ir para a escola. No modelo de \cite{scardamaliabereiter1987}, esta forma de relatar parece ser uma instância do que se designa por conhecimento relatado sem transformação (\textit{knowledge tellling}), por oposição ao conhecimento transformado já acima referido. O autor do Texto 3 tem que contar, e muito, mas não dispõe ainda de recursos cognitivos e linguísticos que lhe permitam codificar o texto de forma a torná-lo legível a quem o lê. Assim, limita-se a dar livre fluxo às suas ideias e a linearizá-las sem qualquer preocupação com a legibilidade do que fica escrito.

  
\begin{center}
\fbox{%
  \parbox{\textwidth}{%
    \begin{flushleft}
      A rua 4278 se estiver enh prigo pode / causar muita pouca sorte em todas as partes / do mundo e secalhar no dezerto podem todas as / pesoas por-causa do vento morem e no antartico / vai secalhar estara tão fiu doque todos os dias / e as pesoas secalhar vão comjelar e o tempo pode- / se confundir e parar e em povoa de Varzim pode aver muitas esplozões todo por causa da rua 4278 estar em preigo e a rua 4278 é a parte do mundo que vai estar mais em prigo (\ldots)
    \end{flushleft}
  }%
}\\\vspace{1em}Texto 3
\end{center}

Para lá das representações ortográficas não canónicas, que revelam o acesso a fontes fonológicas ainda não mapeadas em representações ortográficas convencionais, o mais interessante é o fluxo imparável do pensamento, que se revela num encadeamento em que não há qualquer codificação de fronteiras sintáticas principais, seja por pontuação, quebra de linha ou outras. A informação ativada em memória é a informação que é transposta diretamente para a \isi{escrita}, sem qualquer transformação decorrente de um planeamento prévio ou de revisões de texto, ou que tenha em consideração a adequação ao leitor ou preocupações com as normas da \isi{escrita}. As operações de moldagem do texto - tendo em visita a sua legibilidade para um público e um contexto particulares, a sua organização interna, ou as convenções da \isi{escrita} - são próprias de estádios de desenvolvimento de competência da \isi{escrita} mais avançados. São também efeito de aprendizagem e de uma maturação cognitiva que dá acesso a um conhecimento consciente da forma como as estruturas linguísticas funcionam e de como podem ser manipuladas a favor da eficácia da comunicação \isi{escrita}. 

Observem-se, de seguida, duas aberturas de um reconto escrito de um conto ouvido (retiradas de \citealt{costagoncalves2010}) de alunos de 4.º ano. Ambas partilham a mesma informação, mas o que reproduzem varia em função da quantidade da informação retida e reutilizada, assim como da forma gramatical escolhida para a apresentar por escrito.

\begin{center}
\fbox{%
  \parbox{\textwidth}{%
    \begin{flushleft}
R1 - Era uma vez um monstro que estava a tomar banho no mar, e como era um  mar cheio de peixe, os pescadores estavam a pescar com a rede. De repente, apanharam um monstro, e ficaram assustados, por isso mandaram-no para o Jardim Zoológico.\\R2 - Num dia um monstro foi pescado por pescadores. Os mesmos levaram-no ao Jardim Zoológico. Lá prenderam-no numa jaula.
    \end{flushleft}
  }%
}\\\vspace{1em}Texto 4 
\end{center}

As duas aberturas preservam a informação essencial, mas é evidente que R1 é mais informativo do que R2, o que, em parte, decorre do uso de estratégias gramaticais que permitem veicular mais informação: (i) modificadores nominais oracionais (monstro \textit{que estava a tomar banho}) e adjetivais (mar \textit{cheio de peixe}; \textit{banho sossegado}); (ii) predicativo do sujeito (ficaram \textit{assustados}); (iii) predicativo do objeto direto (acharam-no \textit{muito estranho}). R2 é constituído por 3 períodos simples, ou seja, formados por uma única oração. O que pode não ser tão evidente é que R2 é mais complexo do que R1. Essa complexidade decorre de várias estratégias gramaticais de organização sintática. Ao começar com uma frase passiva, o autor consegue colocar o tópico discursivo (\textit{o monstro} de que fala a história) como tópico frásico, enquanto o SN \textit{os pescadores}, que estava gramaticalmente ao mesmo nível que o monstro em R1, passa para agente da passiva, perdendo proeminência sintática e discursiva. Na segunda frase, retoma as entidades referidas na primeira através de pronomes distintos e inequívocos em termos da sua ligação ao antecedente (\textit{os mesmos} levaram-\textit{no}). Retoma anaforicamente o espaço referido (\textit{Lá}), topicalizando-o, e retoma a primeira referência ao monstro, mantendo-o como tópico discursivo ao usar um pronome nulo na referência à entidade competitiva (os pescadores). 

Assim, embora os dois excertos sejam bem construídos e equivalentes em informação essencial, parece que o autor de R2 tem acesso a estruturas gramaticais mais complexas, que servem melhor a expressão coesa das ideias e que são reveladoras de uma competência de \isi{escrita} mais avançada. 

Considerem-se, finalmente, os Textos 5 e 6, que correspondem a dois fragmentos de texto narrativo produzido por duas crianças de 4.ºano. 

\begin{center}
\fbox{%
  \parbox{\textwidth}{%
    \begin{flushleft}
O gato a ver a bruxa a dormir pensou ``O que é que posso fazer? Depois teve uma ideia, tirou a varinha mágica à bruxa, \sout{mas a bruxa não sentiu nada} e então fez uma magia para o telefone e o jarro de flores \sout{fosse} se tornassem em comida. O jarro de flores ficou \sout{em com uma} numa bebida e o telefone ficou \sout{em um} hambúrguer. Depois começou a comer\textsuperscript{[muito feliz]} e achou delicioso (\ldots)
    \end{flushleft}
  }%
}\\\vspace{1em}Texto 5
\end{center}

\begin{center}
\fbox{%
  \parbox{\textwidth}{%
    \begin{flushleft}
Quando a bruxinha Natália entrou no hotel encontrou um gato e chamou-lhe Traquinas e levou-o para \sout{um} o quarto de hotel. \sout{DE} O \sout{hotel t} quarto onde a bruxinha entrou tinha um sofá, uma almofada, \sout{uma est }um móvel com uma gaveta e por cima do móvel estava um \sout{relógio} telemóvel e um jarro com uma flor.
\end{flushleft}
  }%
}\\\vspace{1em}Texto 6
\end{center}

Em ambos os fragmentos ocorrem reformulações do que vai sendo escrito. Há palavras ou expressões rasuradas e substituídas por outras, inserções de expressões já depois de uma frase construída. A substituição de palavras é indicadora de processos de seleção lexical no sentido de se precisarem as ideias a transmitir. O facto de se substituírem porções de texto, como orações inteiras (\textit{mas a bruxa não sentiu nada}), é sinal de tentativa de melhor expressar o pensamento, mas também sinal evidente de que está a ser construída uma representação do texto enquanto objeto que requer uma sequência interna, a expressão de relações lógicas através da ligação de orações. A substituição da sequência descontínua de preposição e determinante por uma só palavra (\textit{em uma}/\textit{numa}) é sinal de controlo de aspetos microestruturais que têm mais a ver com a forma do enunciado. A substituição do determinante indefinido pelo definido (\textit{um}/\textit{o} quarto) é sinal de atenção ao facto de a entidade que já teria sido referida anteriormente agora ter de ser retomada de forma mais definida (\textit{\sout{um} o quarto do hotel}).

Com o desenvolvimento e a aprendizagem, as capacidades de \isi{escrita} evoluem no sentido de um maior controlo metacognitivo do texto produzido e de um progressivo ajustamento à tarefa de \isi{escrita}. As capacidades de codificação gramatical vão evoluindo em função da idade, da escolarização, da exposição à \isi{escrita}, da prática de \isi{escrita}. E, necessariamente, do conhecimento linguístico tornado cada vez mais acessível e, por isso, mais manipulável. 

\section{Em síntese}
\label{sec:costacostagoncalves_conclusao}

Mesmo o sujeito mais experiente tem consciência de que, quando escreve, há uma tensão entre o que quer dizer e o que consegue escrever; muitas vezes, há a perceção de que o texto produzido fica aquém do que se quer transmitir. O sucesso na \isi{escrita} depende globalmente da forma textual que se adota para transpor as ideias em estruturas linguísticas. Mas, a par disso, depende da seleção e do uso de unidades linguísticas menores e estruturantes do texto: das palavras, que são necessárias para nomear entidades, propriedades das entidades e relações entre estas; das estruturas sintáticas, que são necessárias para ordenar e relacionar, para criar unidades de significado coerentes e coesas. No caso da criança que aprende a escrever ou que está em fase de desenvolvimento da sua competência em lidar com a língua \isi{escrita}, o conhecimento intuitivo que possui sobre o léxico e as estruturas sintáticas da sua língua é posto à prova, mais ainda quando a sua \isi{escrita} é produzida em sala de aula e se destina a ser avaliada pelo professor. É fundamental que a criança consiga ativar o conhecimento linguístico intuitivo, servindo uma \isi{escrita} mais espontânea, mas, sobretudo, que possa aceder a esse conhecimento de forma consciente, para melhor atingir os seus objetivos comunicativos. É importante que a criança possa escolher entre estruturas alternativas, que à sintaxe se associe o significado; que possa escolher entre ordens de palavras que melhor se adequem à informação a veicular, que possa estabelecer nexos frásicos que traduzam relações lógicas que se quer pôr em evidência, que lhe sejam acessíveis estruturas mais ou menos complexas para, assim, dosear a quantidade de informação e a sua distribuição ao longo do texto, de forma a ser eficaz na comunicação. 


O foco deste texto foi o conhecimento sintático,\largerpage[2] que subjaz à competência de \isi{escrita}. Tendo em atenção estruturas sintáticas específicas, mostrou-se como as mesmas são progressivamente dominadas com o desenvolvimento e como são usadas na construção de textos de formatos variados. A componente sintática foi perspetivada pelo contributo particular que tem na produção \isi{escrita}, devendo ser articulada com outros tipos de conhecimento linguístico, com estratégias discursivas e com conhecimentos não linguísticos relevantes para a comunicação \isi{escrita}, abordados em outros capítulos deste livro. 
\sloppy
\printbibliography[heading=subbibliography,notkeyword=this]

\end{document}