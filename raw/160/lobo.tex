\documentclass[output=paper]{LSP/langsci} 
\author{Maria Lobo\affiliation{Centro de Linguística da Universidade Nova de Lisboa \& Faculdade de Ciências Sociais e Humanas, Universidade Nova de Lisboa}\lastand 
Carla Soares-Jesel\affiliation{Laboratoire de Linguistique Formelle - Université Paris Diderot}
}
\title{Interrogativas, relativas e clivadas}  
\abstract{\noabstract}
\ChapterDOI{10.5281/zenodo.889435}
\maketitle
\begin{document}
\section{Interrogativas, relativas e clivadas: o que têm as crianças de aprender} 
\label{sec:lobo_intro}

Neste capítulo começaremos por evocar as propriedades fundamentais das frases interrogativas (\ref{ex:lobo_1a}), relativas (\ref{ex:lobo_1b}) e \isi{clivadas} (\ref{ex:lobo_1c}), que apresentam como característica essencial comum o facto de implicarem a utilização da \isi{periferia} esquerda da frase, ou seja, um conjunto de posições sintáticas que, no domínio frásico em que ocorrem, precedem o sujeito:

\ea\label{ex:lobo_1}
\ea\label{ex:lobo_1a} [\textbf{A quem} [\textbf{ofereceste} [tu \sout{\textbf{ofereceste}} o livro \sout{\textbf{a quem}} ontem]]]?
\ex\label{ex:lobo_1b} A Maria saiu com o amigo [\textbf{a quem} [ela ofereceu o livro \sout{\textbf{a quem}}]].
\ex\label{ex:lobo_1c} Foi [\textbf{ao amigo} [que a Maria ofereceu o livro \sout{\textbf{ao amigo}}]]
\zl

Em (\ref{ex:lobo_1a}) apresentamos uma frase interrogativa parcial,\is{interrogativas!interrogativas parciais} que corresponde a um pedido de informação e que, como abaixo veremos, tem a particularidade de ser introduzida por um constituinte interrogativo.\is{constituintes interrogativos} Em (\ref{ex:lobo_1b}), a frase que modifica o elemento nominal \textit{amigo} é uma frase relativa que é introduzida por um constituinte idêntico ao que encontramos na frase interrogativa. Finalmente, em (\ref{ex:lobo_1c}), apresentamos uma frase clivada, que é utilizada em contextos nos quais se pretende enfatizar um constituinte. Em (\ref{ex:lobo_1c}), o constituinte \textit{ao amigo}, tal como o constituinte interrogativo\is{constituintes interrogativos} em (\ref{ex:lobo_1a}) e tal como o constituinte que introduz a relativa em (\ref{ex:lobo_1b}), tem uma relação próxima com o verbo \textit{ofereceu}, visto que se trata de um dos seus complementos, o objeto indireto. Assim, há uma relação de dependência sintática e semântica entre o objeto indireto e o verbo, embora aquele não apareça na sua posição canónica, mas sim numa posição periférica da frase.

Depois de revermos as propriedades mais importantes destas construções, examinaremos o modo como são adquiridas pelas crianças e como se desenvolvem na linguagem infantil.

\subsection{Interrogativas}
\label{subsec:lobo_interrogativas}

Uma das características distintivas das interrogativas parciais é o facto de apresentarem um constituinte interrogativo\is{constituintes interrogativos} que marca o foco da interrogação. Os \isi{constituintes interrogativos}\footnote{Pronomes, advérbios ou adjetivos interrogativos, na gramática tradicional.} apresentam, na sua maior parte, um morfema qu- (\textit{qu}em, \textit{qu}ando, \textit{qu}e...), em português, ou um morfema wh- (\textit{wh}o (‘quem’), \textit{wh}at (‘o que’), \textit{wh}ere (‘onde’), \ldots), em inglês.\il{inglês} Por esta razão, são designados por cons\-ti\-tu\-in\-tes-wh. Estas estruturas têm a particularidade de apresentarem alterações à ordem básica de palavras. Em (\ref{ex:lobo_1a}), acima, o constituinte interrogativo,\is{constituintes interrogativos} que corresponde ao objeto indireto da frase, aparece em posição inicial. Ora, numa declarativa simples e sendo o português uma língua SVO, a posição canónica do objeto indireto situa-se após a do verbo e a do \isi{objeto direto}. Ainda em (\ref{ex:lobo_1a}), o verbo, por seu lado, precede o sujeito sintático, sendo esta uma configuração típica deste tipo de interrogativas. Nas subsecções que se seguem, estudaremos as propriedades centrais destas interrogativas que acabámos de mencionar.

\subsubsection{Os constituintes interrogativos}
\label{subsubsec:lobo_constituintes_int}

Os \isi{constituintes interrogativos} podem corresponder a \isi{palavras-wh} que são expressões nominais, como em (\ref{ex:lobo_2}), ou que têm um valor adverbial (\ref{ex:lobo_3}):

\ea\label{ex:lobo_2}
\ea\label{ex:lobo_2a} \textbf{Quem} ofereceu o livro?
\ex\label{ex:lobo_2b} \textbf{O que}/\textbf{que}{\footnotemark} ofereceu a Maria ao amigo?\footnotetext{Para \citet{mateus_etal2003}, \textit{o que} é a forma mais produtiva.}
\zl
\largerpage
\ea\label{ex:lobo_3}
\ea\label{ex:lobo_3a} \textbf{Onde} comprou a Maria o livro?
\ex\label{ex:lobo_3b} \textbf{Quando} comprou a Maria o livro?
\ex\label{ex:lobo_3c} \textbf{Porque} comprou a Maria o livro?
\ex\label{ex:lobo_3d} \textbf{Como} viajaste para Itália?
\ex\label{ex:lobo_3e} \textbf{Aonde} vais depois do jantar?
\zl

Os \isi{constituintes interrogativos} também podem corresponder a sintagmas que incluem uma palavra-wh\is{palavras-wh} e um nome:

\ea\label{ex:lobo_4}
\ea\label{ex:lobo_4a} \textbf{Que livro} comprou a Maria?
\ex\label{ex:lobo_4b} \textbf{Quantos livros} comprou a Maria?
\ex\label{ex:lobo_4c} \textbf{Qual dos livros} escolheu a Maria?
\zl

Finalmente, a palavra-wh\is{palavras-wh} pode integrar um \isi{sintagma preposicional}, sendo o constituinte interrogativo,\is{constituintes interrogativos} neste caso, todo o \isi{sintagma preposicional}. Assim, se a palavra-wh\is{palavras-wh} fizer parte de um constituinte deste tipo e se ocorrer numa posição inicial de interrogativa, todo o \isi{sintagma preposicional} tem de ocorrer em posição inicial. Veja-se o contraste entre (\ref{ex:lobo_5}) e (\ref{ex:lobo_6}):

\ea\label{ex:lobo_5}
\ea\label{ex:lobo_5a} \textbf{A quem} ofereceu a Maria o livro?
\ex\label{ex:lobo_5b} \textbf{Com quem} falou a Maria?
\ex\label{ex:lobo_5c} \textbf{Em que cidade} vive a Maria?
\zl

\ea\label{ex:lobo_6}
\ea[*]{\label{ex:lobo_6a} \textbf{Quem} ofereceu a Maria o livro \textbf{a}?}
\ex[*]{\label{ex:lobo_6b} \textbf{Quem} falou a Maria \textbf{com}?}
\ex[*]{\label{ex:lobo_6c} \textbf{Que cidade} vive a Maria \textbf{em}?}
\zl
\subsubsection{A posição do constituinte-wh}
\label{subsubsec:lobo_posicao_wh}

Nas interrogativas parciais do português europeu, o constituinte-wh ocorre tipicamente numa posição inicial de interrogativa, como vimos em todos os exemplos até agora apresentados. A estas estruturas tem-se atribuído uma análise em termos de movimento sintático, que permite dar conta do facto de um constituinte ter uma relação à distância com uma posição sintática que é distinta daquela na qual é pronunciado. O movimento responsável pela anteposição de \isi{constituintes interrogativos} é designado por movimento-wh\is{movimento!movimento-wh}. Assim, considera-se que o constituinte interrogativo\is{constituintes interrogativos} é engendrado na sua posição canónica, i.e., numa posição interna à frase, na qual é determinada a sua função sintática e o seu papel semântico. Em seguida é movido para uma posição periférica da frase, à esquerda \citep{bresnan1970,chomsky1986,ambar1988}, i.e., a posição inicial da interrogativa, à qual não está associada uma função sintática específica.\footnote{Sobre a análise das interrogativas-wh\is{interrogativas!interrogativas-wh} em português, veja-se \citet{mateus_etal2003,ambar1988,ambar2003,duarte2000,soares2006}.} Considera-se igualmente que este domínio periférico para o qual o constituinte interrogativo\is{constituintes interrogativos} se desloca é o domínio do \isi{complementador} (domínio-C, que aqui designaremos por \isi{CP}, \textit{Complementizer Phrase}, em inglês),\il{inglês} uma vez que se trata de um espaço sintático que acolhe igualmente os complementadores\is{complementador} que introduzem as frases encaixadas:

\ea\label{ex:lobo_7}
\ea\label{ex:lobo_7a} O João disse \textbf{que} a Maria ofereceu o livro ao amigo.
\ex\label{ex:lobo_7b} O João perguntou \textbf{se} a Maria ofereceu o livro ao amigo.
\zl

Em português europeu, é igualmente possível não realizar a anteposição do constituinte interrogativo\is{constituintes interrogativos} e pronunciá-lo na posição que lhe é canonicamente reservada, i.e., \textit{in situ}. Assim, as interrogativas \textit{in situ}\is{interrogativas!interrogativas \emph{in situ}}\footnote{Há várias línguas naturais em cujas interrogativas-wh\is{interrogativas!interrogativas-wh} não é visível o movimento-wh\is{movimento!movimento-wh}. O chinês,\il{chinês} o japonês\il{japonês} e o coreano\il{coreano} são um exemplo. Nestas línguas o constituinte interrogativo\is{constituintes interrogativos} ocorre sempre \textit{in situ}.} não apresentam um movimento visível do constituinte interrogativo\is{constituintes interrogativos} para o domínio \isi{CP}:

\ea\label{ex:lobo_8}
\ea\label{ex:lobo_8a} A Maria ofereceu o livro \textbf{a quem}?
\ex\label{ex:lobo_8b} A Maria comprou \textbf{o quê}?
\ex\label{ex:lobo_8c} A Maria falou \textbf{com quem}?
\zl

É possível interpretar as interrogativas \textit{in situ}\is{interrogativas!interrogativas \emph{in situ}} de dois modos distintos em português europeu. Podem corresponder a um pedido de informação, tal como as interrogativas que apresentam um constituinte anteposto, ou podem ter uma interpretação de uma interrogativa eco.\is{interrogativas!interrogativas eco} Neste segundo caso, a interrogativa é utilizada para completar uma informação que não foi integralmente compreendida ou para manifestar a admiração do locutor em relação a uma informação que acabou de ser veiculada (ver \citealt{mateus_etal2003}).\largerpage[1]

\ea\label{ex:lobo_9} A: Ontem comprei uma saia nova.\\B: Compraste o quê?\z

A análise das interrogativas \textit{in situ}\is{interrogativas!interrogativas \emph{in situ}} não é consensual. Alguns autores consideram que não envolvem movimento-wh\is{movimento!movimento-wh} mas que estão sujeitas a restrições pragmáticas específicas \citep{pirestaylor2007}. Outros autores consideram que nas interrogativas \textit{in situ}\is{interrogativas!interrogativas \emph{in situ}} também se manifesta o movimento do constituinte interrogativo\is{constituintes interrogativos} para a \isi{periferia} da frase, sendo esta operação seguida de um movimento da proposição para uma posição periférica mais alta \citep{ambar2003,kato2013}. \citet{huang1982} propôs que, mesmo nas interrogativas do chinês,\il{chinês} nas quais o constituinte interrogativo\is{constituintes interrogativos} aparece sempre \textit{in situ}, o movimento-wh\is{movimento!movimento-wh} ocorre. Para \citet{huang1982}, trata-se simplesmente de um movimento que não é visível, visto não ter repercussões sobre o modo como se pronuncia a interrogativa. Um dos seus argumentos centrais é o facto de, em chinês,\il{chinês} não ser possível encontrar um constituinte interrogativo\is{constituintes interrogativos} em ilhas sintáticas.\is{ilha} Uma \isi{ilha} é um domínio do qual não se pode extrair um constituinte, incluindo, por exemplo, orações subordinadas adverbiais e relativas. Assim, se um constituinte-wh não pode ocorrer numa \isi{ilha}, isso sugere que tal não é possível porque precisa de ser extraído. Em português europeu a estratégia \textit{in situ} é possível dentro de ilhas\is{ilha} sintáticas, o que sugere que se trata de uma estratégia que, na verdade, não implica movimento nesta língua. Em (\ref{ex:lobo_10}) apresentamos um exemplo de um constituinte interrogativo\is{constituintes interrogativos} que ocorre numa frase subordinada com um valor adverbial, considerada habitualmente uma \isi{ilha} sintática:

\ea\label{ex:lobo_10} A Maria comprou o livro [quando o Pedro telefonou \textbf{a quem}]?\z

Assim, adotaremos aqui a análise segundo a qual as interrogativas \textit{in situ}\is{interrogativas!interrogativas \emph{in situ}} não implicam movimento do constituinte-wh, ocupando este uma posição canónica.


\subsubsection{A inversão sujeito-verbo e a estratégia \textit{é que}}
\label{subsubsec:lobo_inversao_suj_v}

Em português europeu, as interrogativas-wh\is{interrogativas!interrogativas-wh} nas quais figura uma palavra-wh\is{palavras-wh} em posição inicial apresentam a ordem VS, sendo a ordem SV excluída, o que é ilustrado pelos exemplos em (\ref{ex:lobo_11}):

\ea\label{ex:lobo_11}
\ea[]{\label{ex:lobo_11a} O que compraste tu?}
\ex[*]{\label{ex:lobo_11b} O que tu compraste?}
\ex[]{\label{ex:lobo_11c} Onde compraste tu o livro?}
\ex[*]{\label{ex:lobo_11d} Onde tu compraste o livro?}
\zl

A impossibilidade de ocorrência da ordem SV sugere que o verbo, nas interrogativas gramaticais de (\ref{ex:lobo_11}), se encontra numa posição estrutural mais alta do que aquela que ocupa habitualmente nas frases declarativas. Vários autores propuseram que o verbo se move para o domínio \isi{CP} nas frases interrogativas-wh\is{interrogativas!interrogativas-wh} \citep{brito1982,rouveret1992,ambar1988,soares2006}, explicando-se deste modo o facto de preceder o sujeito frásico. Assim, no domínio \isi{CP} estão pelo menos disponíveis duas posições: i) uma posição inicial que acolhe os constituintes-wh, ii) uma posição nuclear que recebe o verbo flexionado. Em português europeu, as interrogativas-wh\is{interrogativas!interrogativas-wh} que apresentam um constituinte-wh anteposto e sem um N lexicalmente realizado\largerpage[2]\footnote{
Para alguns falantes, interrogativas que apresentam um constituinte que inclui um N realizado lexicalmente não requerem a subida do verbo para o domínio \isi{CP} \citep{ambar2003,mateus_etal2003}:
\ea\label{ex:lobo_i}
\ea\label{ex:lobo_ia} Que livro a Maria comprou?
\ex\label{ex:lobo_ib} Quantos livros a Maria comprou?
\zl} (cf. (\ref{ex:lobo_11})) só são bem formadas se a segunda posição que acabámos de referir também estiver preenchida. O movimento do verbo flexionado para uma posição periférica é uma das estratégias possíveis para o preenchimento dessa posição. Uma estratégia alternativa é a formação de interrogativas com \textit{é que}. Quando \textit{é que} ocorre na interrogativa, a ordem SV pode ser mantida:

\ea\label{ex:lobo_12}
\ea\label{ex:lobo_12a} O que \textbf{é que} tu compraste?
\ex\label{ex:lobo_12b} Onde \textbf{é que} tu compraste o livro?
\ex\label{ex:lobo_12c} Quando \textbf{é que} a Maria oferecerá o livro ao amigo?
\zl

\textit{É que} é uma expressão invariável que combina o \isi{complementador} \textit{que} e a cópula \textit{é}. Por um lado, não se verifica concordância temporal entre a cópula e o verbo principal que ocorre na interrogativa (\ref{ex:lobo_12}). Por outro lado, \textit{que} e a cópula têm de estar adjacentes. Um advérbio não pode ocorrer entre estes dois elementos, por exemplo:

\ea[*]{\label{ex:lobo_13} O que \textbf{é} ontem \textbf{que} tu compraste?}\z

Assim, assumiremos que, no português europeu, \textit{é que} é uma expressão gramaticalizada que preenche precisamente a posição nuclear do \isi{CP}, legitimando a ordem SV. No entanto, note-se que as interrogativas-wh\is{interrogativas!interrogativas-wh} formadas com a estratégia \textit{é que} podem também apresentar a inversão sujeito-verbo:\is{inversão sujeito-verbo}

\ea\label{ex:lobo_14} O que é que comprou a Maria?\z

\citet{soares2006} propôs que (\ref{ex:lobo_14}) ilustra uma estratégia de focalização do sujeito disponível em português europeu: em vez de se deslocar para uma posição pré-verbal, o sujeito permanece numa posição mais baixa, no domínio verbal. 

Em resumo, no português europeu há varias estratégias de formação de in\-ter\-ro\-ga\-ti\-vas-wh:\is{interrogativas!interrogativas-wh}

\begin{enumerate}[label=\roman*)]
\item interrogativas sem movimento do constituinte-wh, i.e. interrogativas \textit{in situ};\is{interrogativas!interrogativas \emph{in situ}}
\item interrogativas com anteposição do constituinte-wh e movimento do verbo para o núcleo do \isi{CP};
\item interrogativas com anteposição do constituinte-wh, com a expressão é\textit{ que} e com a ordem SV;
\item interrogativas com anteposição do constituinte-wh, com a expressão \textit{é que} e com a ordem VS.
\end{enumerate}

\subsection{Relativas}
\label{subsec:lobo_relativas}

As orações relativas são introduzidas por um constituinte relativo,\is{constituintes relativos}\footnote{Na gramática tradicional, designados por pronomes, advérbios ou adjetivos relativos.
} muitas vezes idêntico aos constituintes que introduzem interrogativas parciais. Tal como nestas estruturas, nas relativas também ocorre movimento-wh\is{movimento!movimento-wh} de um constituinte para uma posição da \isi{periferia} esquerda da oração. As orações relativas podem modificar uma expressão nominal (\ref{ex:lobo_15a}), designada por antecedente, ou não ter antecedente expresso. As relativas sem antecedente expresso designam-se por relativas livres\is{relativas!relativas livres} (\ref{ex:lobo_15b}).\footnote{São também chamadas orações substantivas relativas na classificação tradicional.}

\ea\label{ex:lobo_15}
\ea\label{ex:lobo_15a} A pessoa [a quem eu telefonei] ficou surpreendida.
\ex\label{ex:lobo_15b} Escrevi [a quem eu quis].
\zl

\subsubsection{Os constituintes relativos}
\label{subsubsec:lobo_cobstituintes_rel}

Apesar de se encontrar numa posição inicial de oração, o constituinte relativo\is{constituintes relativos} tem uma função sintática específica no interior da oração relativa em que ocorre e a sua forma depende precisamente da relação gramatical que mantém com os outros constituintes da relativa. Assim, \textit{que} introduz relativas nas quais tem a função sintática de sujeito\is{sujeito}\is{relativas!relativas de sujeito} (\ref{ex:lobo_16a}) ou de \isi{objeto direto}\is{relativas!relativas de objeto} (\ref{ex:lobo_16b}):

\ea\label{ex:lobo_16}
\ea\label{ex:lobo_16a} O ator [\underline{que} \underline{\hspace{1em}} ganhou o prémio] organizou uma festa para os amigos.
\ex\label{ex:lobo_16b} O livro [\underline{que} a Maria comprou \underline{\hspace{1em}}] é excelente.
\zl\largerpage[2]

O constituinte relativo\is{constituintes relativos} pode corresponder a um \isi{sintagma preposicional}. Neste caso, pode ter a função sintática de objeto indireto (\ref{ex:lobo_17a}), de \isi{oblíquo} (\ref{ex:lobo_17b}) ou corresponder a um adjunto (\ref{ex:lobo_17c}, \ref{ex:lobo_17d}) ou a um genitivo (\ref{ex:lobo_17e}).\footnote{O elemento que surge no interior do \isi{sintagma preposicional} varia em função dos traços do antecedente – \textit{quem} emprega-se com um antecedente humano, \textit{que} com um antecedente [- humano], \textit{o qual} com os dois tipos de antecedente. Para a caracterização detalhada dos contextos nos quais ocorrem estes elementos, ver \citealt{mateus_etal2003}.}\textsuperscript{,}\footnote{O genitivo pode igualmente ser marcado por \textit{cujo} \citep{mateus_etal2003}.}

\ea\label{ex:lobo_17}
\ea\label{ex:lobo_17a} O amigo [\underline{a quem} a Maria ofereceu o livro \underline{\hspace{1em}}] ficou feliz.
\ex\label{ex:lobo_17b} O Manuel conhece a pessoa [\underline{sobre a qual} escreveram um romance apaixonante \underline{\hspace{1em}}].
\ex\label{ex:lobo_17c} O Manuel vai sempre à livraria [\underline{na qual} costumamos comprar os nossos livros \underline{\hspace{1em}}].
\ex\label{ex:lobo_17d} Já não me lembro do dia [\underline{em que} o Manuel fez anos \underline{\hspace{1em}}].
\ex\label{ex:lobo_17e} A Maria ofereceu um romance [\underline{do qual} leu algumas páginas \underline{\hspace{1em}}].
\zl

Finalmente, as orações relativas podem ser introduzidas por \textit{onde}, que é sempre um complemento ou adjunto com um valor de locativo (\ref{ex:lobo_18}):

\ea\label{ex:lobo_18} Comprei o livro na livraria [\underline{onde} foi apresentado \underline{\hspace{1em}}].
\z

Note-se que, tal como os \isi{constituintes interrogativos}, os \isi{constituintes relativos} apresentam usualmente uma palavra-wh.\is{palavras-wh}



\subsubsection{A formação de relativas}
\label{subsubsec:lobo_formacao_rel}

Como referido acima, os \isi{constituintes relativos} introduzem a relativa, aparecendo assim numa posição inicial de oração. Tal facto é o resultado do movimento-wh\is{movimento!movimento-wh} do constituinte relativo\is{constituintes relativos} para o domínio \isi{CP}. Nesta posição, o constituinte relativo\is{constituintes relativos} estabelece uma relação entre o antecedente da relativa e a posição na qual se encontrava inicialmente, i.e., antes de o movimento-wh\is{movimento!movimento-wh} ter operado. Deste modo, a relativa pode ser interpretada como predicado. Como vimos na Secção \ref{subsubsec:lobo_posicao_wh}, a posição inicial do domínio \isi{CP} acolhe os constituintes-wh das interrogativas parciais. Supõe-se que esta posição acolhe igualmente os constituintes que introduzem orações relativas:

\ea\label{ex:lobo_19} O \textbf{amigo} [\textsubscript{\isi{CP}} \textbf{a quem} [a Maria ofereceu o livro \sout{\textbf{a quem}}]] ficou feliz.\z

Note-se que, em (\ref{ex:lobo_19}), o constituinte relativo\is{constituintes relativos} é um \isi{sintagma preposicional} e que a preposição, neste caso, tem igualmente de ser movida para uma posição periférica. Assim, a estrutura em (\ref{ex:lobo_20}) é agramatical:

\ea\label{ex:lobo_20} O \textbf{amigo} [\textsubscript{\isi{CP}} \textbf{quem} [a Maria ofereceu o livro a \sout{\textbf{quem}}]] ficou feliz.\z

Tal como acontece noutras línguas,\footnote{Tal é o caso do inglês,\il{inglês} do francês,\il{francês} do espanhol.}\il{espanhol} as relativas sujeito\is{relativas!relativas de sujeito} e objeto direto do português europeu apresentam a particularidade de serem introduzidas por um \isi{complementador} e não por um sintagma-wh \citep{brito1991}. Vários linguistas têm assumido que, nestes casos, a relação entre o antecedente e a posição argumental vazia na oração relativa é estabelecida por um elemento-wh designado habitualmente por operador-wh, que é nulo e que foi movido para o domínio \isi{CP}. O \isi{complementador} \textit{que}, por seu lado, é engendrado diretamente neste domínio. Admite-se assim a possibilidade de o movimento-wh\is{movimento!movimento-wh} também se poder aplicar a elementos nulos \citep{chomsky1977}:

%%%draw arrow from blank to wh
\ea\label{ex:lobo_21} O \textbf{livro} [\textsubscript{\isi{CP}} \linkto \textbf{wh} \under que [A Maria comprou \from \underline{\hspace{1em}}]] é excelente.\z
%%%draw arrow from blank to wh
\ea\label{ex:lobo_22} A \textbf{pessoa} [\textsubscript{\isi{CP}} \linkto \textbf{wh} \under que [\from \underline{\hspace{1em}} comprou o livro]] gosta muito de ler.\z

Desta forma, os diferentes tipos de relativas são analisados de um modo uniforme.\footnote{
Outros autores propuseram que o antecedente, ou núcleo da relativa, é engendrado numa posição interna à oração relativa (\citealt{brame1968,schachter1973,vergnaud1974}; ver também \citealt{kayne1994}) sendo depois movido para o domínio \isi{CP}:
\ea\label{ex:lobo_i2}
\ea\label{ex:lobo_i2a} O [\textsubscript{\isi{CP}} \textbf{livro} que [a Maria comprou \sout{\textbf{livro}}]] é excelente.
\ex\label{ex:lobo_i2b} O [\textsubscript{\isi{CP}} \textbf{escritor} que [\sout{\textbf{escritor}} escreveu o romance]] é excelente.
\zl

Assim, só as relativas cujo constituinte relativo\is{constituintes relativos} é um \isi{sintagma preposicional} implicariam o movimento-wh\is{movimento!movimento-wh} deste último para a \isi{periferia}.}

\subsection{Clivadas}
\label{subsec:lobo_clivadas}
\is{clivadas}
As construções de clivagem permitem destacar ou focalizar um constituinte da frase. Trata-se de estruturas que envolvem todas uma forma do verbo SER e em que um dos constituintes da frase é destacado. Podem ser usadas para contrastar um constituinte (\ref{ex:lobo_23}), para o marcar como informação nova (\ref{ex:lobo_24}) ou para o identificar relativamente a um conjunto de alternativas (\ref{ex:lobo_25}):

\ea\label{ex:lobo_23} O Rui não fez nada. \underline{O Pedro} é que tomou conta de tudo.\z
\ea\label{ex:lobo_24} Quem telefonou?/Quem telefonou foi \underline{o teu irmão}.\z
\ea\label{ex:lobo_25} Todos os alunos tiveram boas notas, mas quem ficou em primeiro lugar foi \underline{o Pedro}.\z


\subsubsection{Estratégias de formação de clivadas}
\label{subsubsec:lobo_estrategias_form_cliv}
\is{clivadas}
Em português europeu existem seis tipos diferentes de clivadas:  i) as clivadas canónicas\is{clivadas!clivadas canónicas} (\ref{ex:lobo_26}); ii) as clivadas-wh\is{clivadas!clivadas-wh} (\ref{ex:lobo_27}); iii) as clivadas de \textit{é que}\is{clivadas!clivadas de \emph{é que}} (\ref{ex:lobo_28}); iv) as pseudoclivadas básicas\is{clivadas!pseudoclivadas} (\ref{ex:lobo_29}); v) as pseudoclivadas invertidas (\ref{ex:lobo_30}); vi) as clivadas de \textit{ser}\is{clivadas!clivadas de \emph{ser}} (\ref{ex:lobo_31}) (\citealt{malacacasteleiro1979}, \citealt{costaduarte2001}, e.o.).

\ea\label{ex:lobo_26} Foi a Maria que comprou o livro.\z
\ea\label{ex:lobo_27} Foi a Maria quem comprou o livro.\z
\ea\label{ex:lobo_28} A Maria é que comprou o livro.\z
\ea\label{ex:lobo_29} Quem comprou o livro foi a Maria.\z
\ea\label{ex:lobo_30} A Maria foi quem comprou o livro.\z
\ea\label{ex:lobo_31} A Maria comprou foi o livro.\z

As análises que têm sido propostas para as clivadas apresentam algumas divergências. No entanto, é possível destacarmos algumas propriedades fundamentais destas construções. As clivadas de \textit{é que},\is{clivadas!clivadas de \emph{é que}} contrariamente às clivadas canónicas,\is{clivadas!clivadas canónicas} clivadas-wh\is{clivadas!clivadas-wh} e pseudoclivadas, são frases simples, nas quais \textit{é que} corresponde a uma expressão lexicalizada que preenche uma posição da \isi{periferia} esquerda da frase \citep{soares2006,lobo2006}. Por outro lado, as clivadas canónicas,\is{clivadas!clivadas canónicas} as clivadas-wh\is{clivadas!clivadas-wh} e as pseudoclivadas\is{clivadas!pseudoclivadas} são frases bioracionais.

Nas clivadas de \textit{é que},\is{clivadas!clivadas de \emph{é que}} o constituinte clivado desloca-se para uma posição na \isi{periferia} da frase (\ref{ex:lobo_32}). O mesmo acontece nas clivadas canónicas.\is{clivadas!clivadas canónicas}\footnote{Ver \citet{vercauteren2015} para uma análise diferente.} Contudo, neste caso, o constituinte clivado desloca-se para a \isi{periferia} da frase encaixada (\ref{ex:lobo_33}):

\ea\label{ex:lobo_32} \textbf{Este livro} é que [eu escrevi \sout{\textbf{este livro}}].\z

\ea\label{ex:lobo_33} Foi \textbf{este livro} [que eu escrevi \sout{\textbf{este livro}}].\z

As clivadas-wh\is{clivadas!clivadas-wh} (\ref{ex:lobo_34}), as pseudoclivadas\is{clivadas!pseudoclivadas} básicas (\ref{ex:lobo_35}) e as pseudoclivadas invertidas (\ref{ex:lobo_36}) têm a particularidade de apresentar um constituinte-wh. Assim, o movimento-wh\is{movimento!movimento-wh} só opera nestas clivadas, mais precisamente, no interior da frase subordinada que é um dos constituintes da clivada:

\ea\label{ex:lobo_34} Foi este livro [\textbf{o que} eu escrevi \sout{\textbf{o que}}].\z

\ea\label{ex:lobo_35} [\textbf{O que} eu escrevi \sout{\textbf{o que}}] foi este livro.\z

\ea\label{ex:lobo_36} Este livro foi [\textbf{o que} eu escrevi \sout{\textbf{o que}}].\z

As clivadas em (\ref{ex:lobo_34})--(\ref{ex:lobo_36}) requerem o estabelecimento de uma relação anafórica entre o constituinte \textit{este livro} e o constituinte-wh da oração subordinada. Essa relação é a base para determinar o constituinte focalizado.

Finalmente, as clivadas de \textit{ser}\is{clivadas!clivadas de \emph{ser}} são frases simples nas quais o verbo \textit{ser} foi reanalisado como marcador de foco contrastivo \citep{santos_etal2013,vercauteren2015}.


\subsection{Relação entre sintaxe e discurso}
\label{subsec:lobo_relacao_sint_disc}

Para empregar adequadamente interrogativas, relativas e clivadas, é necessário estabelecer relações entre sintaxe e discurso.

Uma interrogativa é tipicamente usada para obter informação, embora haja interrogativas que podem ser usadas para fazer pedidos (\textit{Quem me pode dizer as horas?} \textit{Não se importa de fechar a porta?}) ou para expressar atitudes (\textit{Quem não gosta de elogios?}). Assim, para além do domínio das alterações de ordem de palavras associadas às estruturas interrogativas, é necessário saber em que contextos é adequado formular diferentes tipos de interrogativas.

Relativamente às estruturas clivadas, é necessário dominar os diferentes contextos discursivos em que cada tipo de estrutura clivada é adequada. De facto, nem todos os tipos de estruturas clivadas são igualmente adequados em todos os contextos. As clivadas de \textit{ser}\is{clivadas!clivadas de \emph{ser}} e as clivadas de \textit{é que}\is{clivadas!clivadas de \emph{é que}} são normalmente usadas em contextos em que se quer estabelecer um contraste entre o constituinte clivado e outro constituinte. Já as clivadas canónicas\is{clivadas!clivadas canónicas} e as pseudoclivadas\is{clivadas!pseudoclivadas} podem aparecer também em contextos em que o constituinte clivado corresponde a informação nova. Assim, as diferentes estruturas sintáticas terão de ser associadas a contextos discursivos específicos.

Finalmente, a diferença entre relativas restritivas e apositivas implica saber que as primeiras contribuem para a identificação do referente, ao passo que as segundas não são necessárias para a identificação do referente.

\section{A aquisição de interrogativas}
\label{sec:lobo_aquisicao_interrogativas}

Apesar de as interrogativas-wh\is{interrogativas!interrogativas-wh} emergirem cedo na produção espontânea, como veremos, as propriedades destas construções são adquiridas gradualmente e levam algum tempo a consolidar-se. Nesta secção apresentaremos os resultados sobre a aquisição de interrogativas-wh\is{interrogativas!interrogativas-wh} em português europeu disponíveis atualmente: \citet{soares2003,soares2004,soares2006}, estudo baseado na produção espontânea de três crianças monolingues com idades compreendidas entre 1;2 e 4;6 anos, \citet{cerejeira2009,cerejeira2010}, \citet{baiao2013} e \citet{baiaolobo2014}, cujos resultados foram obtidos a partir da realização de testes de compreensão e de produção induzida.


\subsection{Desenvolvimento de estruturas interrogativas em dados de produção espontânea}
\label{subsec:lobo_desenvolvimento}

Por volta dos dois anos, as crianças já produzem interrogativas-wh\is{interrogativas!interrogativas-wh} mas estas não são frequentes no período inicial da aquisição da linguagem \citep{soares2006}.\footnote{ A variação individual é significativa. Assim, o momento em que estas interrogativas emergem na produção espontânea é variável.} As primeiras interrogativas atestadas apresentam sempre um sintagma interrogativo em posição inicial e correspondem a dois tipos de estruturas: i) estruturas identificacionais com a cópula \textit{ser} (\ref{ex:lobo_37}), ii) interrogativas introduzidas por \textit{onde} (\ref{ex:lobo_38}).

\ea\label{ex:lobo_37}
\ea\label{ex:lobo_37a} O que é isto?\jambox{(1;10)}
\ex\label{ex:lobo_37b} Quem é?
\zl

\ea\label{ex:lobo_38}
\ea\label{ex:lobo_38a} Onde está o gato?\jambox{(1;11)}
\ex\label{ex:lobo_38b} Onde (es)tá mão?
\zl

As interrogativas identificacionais são as mais frequentes. Por exemplo, na produção da criança mais nova do corpus de \citet{soares2006}, 72\% das interrogativas são deste tipo. As interrogativas de sujeito (\ref{ex:lobo_39}) e de objeto (\ref{ex:lobo_40}) são bastante menos frequentes:

\ea\label{ex:lobo_39} Quem mora aqui?\jambox{(2;8)}\z
\ea\label{ex:lobo_40} O que tem?\jambox{(2;1)}\z

Note-se que (\ref{ex:lobo_40}) não apresenta um \isi{sujeito} lexical. Trata-se, contudo, de uma interrogativa gramatical visto que, em português europeu, não é necessário que o \isi{sujeito} seja realizado foneticamente em cada frase, podendo ser nulo. A ausência de sujeitos lexicais nas interrogativas de objeto iniciais e certos erros produzidos pelas crianças (cf. \ref{ex:lobo_41}) permitiram concluir que as crianças não realizam o movimento do verbo para o núcleo do domínio \isi{CP}, nas interrogativas-wh\is{interrogativas!interrogativas-wh} que produzem \citep{soares2003,soares2004,soares2006}:

\ea\label{ex:lobo_41} O que tu tens aqui na mala?\jambox{(2;6)}\z

Finalmente, note-se também que as crianças constroem as interrogativas-wh\is{interrogativas!interrogativas-wh} iniciais sem recorrerem à estratégia \textit{é que}. Este conjunto de factos levou \citet{soares2006} a propor que, numa fase inicial, as crianças utilizam um domínio \isi{CP} simplificado para construírem as primeiras interrogativas-wh.\is{interrogativas!interrogativas-wh} Na verdade, algumas propriedades relacionadas com \isi{CP} não parecem estar ativas no sistema inicial da criança, o que fez com que alguns investigadores propusessem que tal domínio não está disponível inicialmente. Este tem sido um tópico amplamente debatido na bibliografia. No que diz respeito ao português, \citet{soares2006} mostra que há argumentos empíricos que permitem concluir que, por volta dos 2 anos de idade, este domínio está incluído na gramática das crianças.

No corpus de \citet{soares2006}, as interrogativas com \textit{é que} começam a ser produzidas aos 2;6 anos e tornam-se sistemáticas mais tarde, a partir dos 3;4 anos. A emergência das interrogativas com \textit{é que} marca uma fase de desenvolvimento distinta. As interrogativas que apresentam \textit{é que} implicam, como vimos, a utilização de duas posições sintáticas diferentes, na \isi{periferia} esquerda da frase. Progressivamente, a estratégia \textit{é que} (\ref{ex:lobo_42}) é adotada pela criança,\footnote{No corpus de \citet{soares2006}, a produção de interrogativas sem \textit{é que} é rara aos 3;6 anos e inexistente a partir dos 4 anos.} sendo preferida à inversão sujeito-verbo,\is{inversão sujeito-verbo} não atestada nos dados infantis. Esta preferência explica-se pelo facto de a inserção da expressão gramaticalizada \textit{é que} na estrutura ser mais acessível do que a realização de uma operação suplementar com o verbo \citep{soares2003,soares2004,soares2006}.

\ea\label{ex:lobo_42} Qual é que eu vou fazer?\jambox{(3;3)}\z

O conjunto de interrogativas-wh\is{interrogativas!interrogativas-wh} produzidas pelas crianças alarga-se de modo gradual. Interrogativas nas quais o constituinte interrogado corresponde a um \isi{oblíquo} ou a um \isi{adjunto}  emergem mais tardiamente, depois dos três anos, no corpus de \citet{soares2006}. Nesta idade, a criança utiliza um conjunto mais vasto de palavras interrogativas (\textit{quem}, \textit{o que}, \textit{onde}, \textit{qual}, \textit{como}, \textit{quando}, \textit{porque}).

\subsection{Interrogativas de sujeito versus interrogativas de objeto em tarefas de produção induzida e de compreensão}
\label{subsec:interrogativas_sujeito}

\citet{cerejeira2009,cerejeira2010} testou a produção e a compreensão de interrogativas-wh\is{interrogativas!interrogativas-wh} por crianças com uma média de idades compreendida entre 3;6 e 5;6 anos, com o objetivo de comparar interrogativas de sujeito e interrogativas de objeto. 

Na produção, através de uma tarefa de produção induzida, \citet{cerejeira2009,cerejeira2010} mostra que as crianças diferem dos adultos nas estratégias de formação de interrogativas: tal como \citet{soares2006}, Cerejeira constata que só as crianças mais velhas (grupos com a idade média de 4;6 e 5;6 anos) adotam de forma sistemática a estratégia \textit{é que} para construírem interrogativas-wh.\is{interrogativas!interrogativas-wh}  Por outro lado, a produção de interrogativas de objeto revelou-se mais difícil para as crianças do que a produção de interrogativas de sujeito.

Na compreensão, testada através de uma tarefa de escolha de imagens, as crianças mostraram uma boa capacidade de compreensão destes dois tipos de interrogativas sempre que o sintagma interrogativo correspondia a uma palavra-wh\is{palavras-wh} simples (\textit{quem}). No entanto, quando o constituinte interrogativo\is{constituintes interrogativos} integrava um elemento lexical (\textit{que menino}), as crianças revelaram dificuldade na compreensão de interrogativas de objeto. \citet{cerejeira2009,cerejeira2010}, adotando uma hipótese defendida por \citet{friedmann_etal2009}, atribui esta assimetria ao facto de, nas interrogativas-wh\is{interrogativas!interrogativas-wh} de objeto,\is{interrogativas!interrogativas-wh de objeto} se construir uma configuração na qual o \isi{sujeito} da frase se constitui como elemento interveniente na relação entre o constituinte interrogativo\is{constituintes interrogativos} em posição inicial e a sua posição de origem. Estes efeitos de \isi{intervenção} são mais fortes quando o constituinte interrogativo\is{constituintes interrogativos} integra um elemento lexical (um nome) com traços gramaticais semelhantes aos do sintagma nominal na posição de \isi{sujeito}:

\ea\label{ex:lobo_43}
\ea\label{ex:lobo_43a} \textbf{Quem} é que a avó está a abraçar \sout{\textbf{quem}}?
\ex\label{ex:lobo_43b}\textbf{Que menina} é que a \fbox{avó} está a abraçar \sout{\textbf{que menina}}?
\zl

Tal tipo de \isi{intervenção} não se verifica nas interrogativas de sujeito:

\ea\label{ex:lobo_44}
\ea\label{ex:lobo_44a} \textbf{Quem} é que \sout{\textbf{quem}} está a abraçar a avó?
\ex\label{ex:lobo_44b} \textbf{Que menina} é que \sout{\textbf{que menina}} está a abraçar a avó?
\zl

Assim, apesar de concluir que o domínio \isi{CP} está ativo no sistema da criança, \citet{cerejeira2009,cerejeira2010} sublinha o facto de as crianças alcançarem primeiro uma competência adulta nas interrogativas de sujeito. Por seu lado, as interrogativas-wh\is{interrogativas!interrogativas-wh} de objeto\is{interrogativas!interrogativas-wh de objeto} também se desenvolvem de modo progressivo: aquelas que apresentam verbos irreversíveis\footnote{Nas interrogativas com verbo semanticamente irreversível, o \isi{sujeito} é [+ animado] e o objeto é [- animado], sendo impossível inverter a função semântica destes constituintes. É o caso da interrogativa em (\ref{ex:lobo_i3}), produzida num contexto em que a menina está a comer um bolo:

\ea\label{ex:lobo_i3} O que é que a menina está a comer?\z
}
são produzidas de acordo com a gramática alvo mais cedo. E as interrogativas-wh\is{interrogativas!interrogativas-wh} de objeto\is{interrogativas!interrogativas-wh de objeto} com uma palavra-wh\is{palavras-wh} (cf. (\ref{ex:lobo_43a})) são mais facilmente compreendidas pelas crianças do que aquelas nas quais figura um constituinte interrogativo\is{constituintes interrogativos} com palavra-wh\is{palavras-wh} e nome (cf. (\ref{ex:lobo_43b})).


\subsection{Interrogativas preposicionadas}
\label{subsec:lobo_interrogativas_prepo}
\is{interrogativas!interrogativas preposicionadas}
\citet{baiao2013} e \citet{baiaolobo2014} testaram a produção e a compreensão de interrogativas preposicionadas,\is{interrogativas!interrogativas preposicionadas} comparando-as com as de interrogativas não preposicionadas de sujeito e de objeto em crianças com idades compreendidas entre os 3;0 e os 5;11 anos. A metodologia usada foi semelhante à de \citet{cerejeira2010}.

Na produção, as autoras confirmam os dados de \citet{cerejeira2010}, mostrando que a produção de interrogativas de sujeito é mais fácil e mais precoce do que a de interrogativas de objeto. Não encontraram, contudo, diferenças assinaláveis entre taxas de produção de interrogativas de objeto não preposicionado (\textit{Quem é que o gato mordeu?}) e de interrogativas de objeto preposicionado (\textit{De quem é que o gato fugiu?}). Acontece, porém, que nas interrogativas de objeto preposicionado, por vezes as crianças optam por uma estratégia de omissão da preposição (\textit{Quem é que o gato fugiu?}) e fazem mais frequentemente interrogativas \textit{in situ}\is{interrogativas! interrogativas \emph{in situ}} (\textit{O gato fugiu de quem?}), o que é compatível com a ideia de que o movimento de constituintes-wh preposicionados envolve mais custos.

Na compreensão, não houve dificuldades com as interrogativas sem nome realizado (\textit{Quem é que a avó está a abraçar?}, \textit{Para quem é que o menino está a olhar?}). Nas interrogativas com nome realizado, encontram-se assimetrias entre interrogativas de sujeito (\textit{Que menina é que a avó está a abraçar?}), que não levantam dificuldades, e interrogativas de objeto, quer preposicionadas\is{interrogativas!interrogativas preposicionadas} (\textit{De que cão é que o menino está a fugir?}), quer não preposicionadas (\textit{Que menina é que a avó está a abraçar?}). Isto mostra que a presença da preposição não facilita a compreensão e que os efeitos de \isi{intervenção} se devem sobretudo à presença de um nome realizado:

\ea\label{ex:lobo_45} \textbf{De que menina} é que a \fbox{avó} está a tratar \sout{\textbf{de que menina}}?\z

\section{A aquisição de relativas}
\label{sec:lobo_aquisicao_rel}

As frases relativas não são frequentes na produção inicial das crianças e emergem bastante mais tarde do que as interrogativas-wh,\is{interrogativas!interrogativas-wh} na produção espontânea \citep{soares1998}. O constituinte que introduz a relativa é por vezes omitido, o que sugere que há dificuldades associadas à sua realização \citep{soares1998}:

\ea\label{ex:lobo_46} Isto (es)tá (a)qui é o rato.\jambox{{1;10.4}}\z

Ainda há poucos dados sobre a aquisição e desenvolvimento das frases relativas na produção espontânea do português europeu. \citet{duarte_etal2011,duarte_etal2015}, comparando a emergência de relativas finitas\is{relativas!relativas finitas} com relativas infinitivas finais\is{relativas!relativas infinitas finais} (e.g. \textit{uma manta para tapar}) mostram que as relativas finitas\is{relativas!relativas finitas} emergem mais tarde do que as relativas finais.\is{relativas!relativas infinitas finais} Mais numerosos são os estudos experimentais sobre a aquisição destas estruturas \citep{vasconcelos1993,costa_etal2011,costa_etal2012,costa_etal2014,costa_etal2015,costasilva2014}, cujos resultados apresentaremos na próxima secção. 

\subsection{Relativas de sujeito vs. relativas de objeto\is{relativas!relativas de objeto} em tarefas de produção induzida e de compreensão}

Como vimos na Secção \ref{subsec:lobo_relativas}, as frases relativas implicam o estabelecimento de uma relação entre uma posição argumental no interior da oração relativa e um operador relativo no domínio \isi{CP}. Considera-se, assim, que entre estas duas posições existe uma relação de dependência abstrata. Note-se que, no caso das relativas de objeto,\is{relativas!relativas de objeto} o sujeito da relativa intervém nesta relação de dependência (\ref{ex:lobo_48}). Tal não acontece numa relativa de sujeito\is{relativas!relativas de sujeito} (\ref{ex:lobo_47}):

\ea\label{ex:lobo_47} Eu vi a menina [\textsubscript{\isi{CP}} OP que [\underline{\hspace{1em}} comeu o bolo]].\z

\ea\label{ex:lobo_48} Eu vi o bolo [\textsubscript{\isi{CP}} OP que [\fbox{a menina} comeu \underline{\hspace{1em}}]].\z

Diversos resultados apresentados para outras línguas têm mostrado que as crianças têm mais dificuldades em compreender e em produzir relativas de objeto\is{relativas!relativas de objeto} do que relativas de sujeito,\is{relativas!relativas de sujeito} tendo a \isi{intervenção} sido apontada como o fator que está na origem dessas dificuldades.
Usando uma tarefa de manipulação de objetos, \citet{vasconcelos1993} testa a compreensão, por crianças entre 3;6 e 8;6, de relativas de sujeito\is{relativas!relativas de sujeito} e de relativas de objeto\is{relativas!relativas de objeto} encaixadas em diferentes posições na frase: à direita, modificando o complemento direto (\ref{ex:lobo_49c}, \ref{ex:lobo_49d}), e ao centro (\ref{ex:lobo_49a}, \ref{ex:lobo_49b}), modificando o sujeito, como nas frases seguintes:

\ea\label{ex:lobo_49}
\ea\label{ex:lobo_49a} O cão que pisou o gato mordeu a galinha.
\ex\label{ex:lobo_49b} O cão que o gato pisou mordeu a galinha.
\ex\label{ex:lobo_49c} O cão mordeu a galinha que pisou o gato.
\ex\label{ex:lobo_49d} O cão mordeu a galinha que o gato pisou.
\zl

Apesar de a metodologia usada ser controversa, trata-se do primeiro estudo que considera de forma sistemática a compreensão de relativas no português europeu. A autora conclui que as crianças têm melhores desempenhos nas relativas de sujeito\is{relativas!relativas de sujeito} e que, em geral, têm melhores desempenhos nas relativas encaixadas à direita.

\citet{costa_etal2011} testaram a produção e a compreensão de relativas de sujeito\is{relativas!relativas de sujeito} e de objeto\is{relativas!relativas de objeto} por crianças com idades compreendidas entre os 3;9 e os 6;2 anos e confirmaram a existência de uma assimetria entre os dois tipos de relativas. \citet{costa_etal2011} utilizaram uma tarefa de produção induzida para testarem a produção de relativas. Verificou-se que as crianças produziram relativas de sujeito\is{relativas!relativas de sujeito} em 78\% dos contextos e relativas de objeto\is{relativas!relativas de objeto} apenas em 31\% dos contextos requeridos. Contudo, também os adultos mostraram alguma assimetria na produção de cada um destes tipos de relativas: em vez de relativas de objeto,\is{relativas!relativas de objeto} optaram por vezes por produzir relativas de sujeito\is{relativas!relativas de sujeito} com passivas. Uma análise qualitativa das respostas mostra que existem diferenças entre adultos e crianças: os primeiros nunca produzem estruturas agramaticais ou incongruentes com a tarefa; nas crianças, em contrapartida, podem encontrar-se estratégias agramaticais. Para testarem a compreensão de relativas, \citet{costa_etal2011} aplicaram um teste de escolha de imagens (para os detalhes, ver \citealt{costa_etal2011}). Constatou-se que as relativas de sujeito\is{relativas!relativas de sujeito} foram compreendidas pelas crianças em 96\% dos casos, enquanto as relativas de objeto\is{relativas!relativas de objeto} foram interpretadas corretamente em 68\% dos contextos. É importante referir que, neste caso, a assimetria encontrada nas crianças não caracteriza os adultos, uma vez que estes tiveram uma performance de 100\% na compreensão dos dois tipos de relativas. \citet{costa_etal2011} atribuem as dificuldades associadas à produção e compreensão de frases relativas de objeto\is{relativas!relativas de objeto} à configuração de \isi{intervenção} que estas implicam. Na verdade, quando um constituinte intervém entre a posição argumental vazia no interior da relativa e o operador relativo, como no caso das relativas de objeto,\is{relativas!relativas de objeto} pode ser mais problemático para a criança determinar a interpretação de todos os argumentos envolvidos e, mais concretamente, a interpretação da posição de objeto no seio da relativa.\footnote{É possível também uma análise em que é o próprio nome antecedente que se desloca do interior da oração relativa. Para um aprofundamento desta análise, ver \citealt{kayne1994,friedmann_etal2009}.} Os erros realizados pelas crianças quando incitadas a produzir relativas de objeto\is{relativas!relativas de objeto} foram considerados um argumento a favor desta ideia:

\ea\label{ex:lobo_50} Alvo: Gostava de ser o menino que o avô visita.
\ea\label{ex:lobo_50a} Inversão de papéis semânticos:\\Gostava de ser o menino que visita o avô.
\ex\label{ex:lobo_50b} Inserção de um clítico \isi{resumptivo}:\\Gostava de ser o menino que o avô o visita.
\zl

Em (\ref{ex:lobo_50a}) as crianças eliminam a configuração de \isi{intervenção}, transformando uma relativa de objeto\is{relativas!relativas de objeto} numa relativa de sujeito,\is{relativas!relativas de sujeito} e em (\ref{ex:lobo_50b}) inserem um argumento suplementar (o clítico \isi{resumptivo}, neste caso, \textit{o}) na posição de objeto, preenchendo a posição que está normalmente vazia.

Em \citet{costa_etal2012}, compararam-se quatro tipos de relativas: relativas de sujeito\is{relativas!relativas de sujeito} com antecedente,\is{relativas!relativas com antecedente} relativas de sujeito\is{relativas!relativas de sujeito} livres,\is{relativas!relativas livres} relativas de objeto\is{relativas!relativas de objeto} com antecedente\is{relativas!relativas com antecedente} e relativas de objeto\is{relativas!relativas de objeto} livres.\is{relativas!relativas livres} As relativas livres\is{relativas!relativas livres} têm a particularidade de não terem um antecedente expresso, como referido na Secção \ref{subsec:lobo_relativas}. Os autores realizaram um teste de compreensão (escolha de imagens) a crianças com idades compreendidas entre os 4;0 e os 5;11 anos e uma tarefa de leitura automonitorizada (\textit{self paced reading-task})\footnote{Os participantes leem palavras, sintagmas, orações, frases ou textos, no ecrã do computador, pressionando uma tecla que faz aparecer cada unidade, em função do que está a ser testado.} a adultos. Os resultados do teste de compreensão permitiram confirmar a assimetria encontrada anteriormente entre relativas de objeto\is{relativas!relativas de objeto} e relativas de sujeito:\is{relativas!relativas de sujeito} as crianças revelaram uma melhor compreensão de relativas de sujeito.\is{relativas!relativas de sujeito} Por outro lado, verificou-se globalmente a inexistência de uma preferência clara por relativas livres,\is{relativas!relativas livres} quando comparadas com relativas com antecedente:\is{relativas!relativas com antecedente} as crianças tiveram piores desempenhos nas relativas livres\is{relativas!relativas livres} de sujeito\is{relativas!relativas de sujeito} do que nas relativas de sujeito\is{relativas!relativas de sujeito} com antecedente.\is{relativas!relativas com antecedente} Apenas as crianças de 4 anos mostraram uma maior facilidade na compreensão de relativas de objeto\is{relativas!relativas de objeto} livres,\is{relativas!relativas livres} quando comparadas com relativas de objeto\is{relativas!relativas de objeto} com antecedente.\is{relativas!relativas com antecedente} Assim, os autores concluíram que, nas relativas de objeto,\is{relativas!relativas de objeto} os efeitos de \isi{intervenção} do sujeito que figura na oração relativa se mantêm igualmente nas relativas livres,\is{relativas!relativas livres} apesar de serem mais fracos.

A tarefa de leitura automonitorizada aplicada aos adultos permitiu verificar que estes efetuaram uma leitura mais rápida de relativas de sujeito.\is{relativas!relativas de sujeito} Além disso, revelaram uma melhor compreensão de relativas livres\is{relativas!relativas livres} e realizaram igualmente uma leitura mais rápida deste tipo de frases. Verificou-se que o efeito de \isi{intervenção} não desaparece quando o operador relativo que não tem restrição lexical cruza o sujeito, apesar de este efeito ser atenuado, o que pode ser explicado pelo facto de o operador relativo \textit{quem} ter traços de \isi{pessoa} e \isi{animacidade}. Os autores sugeriram que o facto de os adultos terem uma melhor capacidade para lidar com relativas de objeto\is{relativas!relativas de objeto} com antecedente\is{relativas!relativas com antecedente} do que as crianças pode ser explicado pelo desenvolvimento de uma capacidade de processamento.\largerpage[2]

Em \citet{costa_etal2014,costa_etal2013} a compreensão de relativas de sujeito\is{relativas!relativas de sujeito} (\ref{ex:lobo_51a}) foi comparada com a compreensão de relativas de objeto\is{relativas!relativas de objeto} com preposição (\ref{ex:lobo_51c}) ou sem preposição (\ref{ex:lobo_51b}):

\ea\label{ex:lobo_51}
\ea\label{ex:lobo_51a} Mostra-me o menino que toca no macaco.
\ex\label{ex:lobo_51b} Mostra-me a menina que a enfermeira trata.
\ex\label{ex:lobo_51c} Mostra-me o menino em que o macaco toca.
\zl

A performance nas relativas de sujeito\is{relativas!relativas de sujeito} foi melhor do que a performance em ambos os tipos de relativas de objeto.\is{relativas!relativas de objeto} Assim, não foi encontrada uma diferença entre as relativas de objeto\is{relativas!relativas de objeto} direto e as relativas de objeto\is{relativas!relativas de objeto} preposicionado, uma vez que a performance das crianças relativamente a estes dois tipos de relativas é idêntica. Assim, a natureza categorial do objeto (sintagma nominal vs. \isi{sintagma preposicional}) não parece ser determinante para efeitos de \isi{intervenção}, tal como acontece com as interrogativas. Para além disso, não houve diferenças significativas entre relativas em que o constituinte preposicionado é complemento e relativas em que o constituinte preposicionado é um modificador.


Na produção, o estudo de \citet{costasilva2014} mostra que a estratégia de formação de relativas com preposição parece, contudo, ser custosa, o que será um aspeto independente do efeito de \isi{intervenção}. Duas das estratégias desviantes encontradas com maior frequência foram a omissão da preposição, dando origem a uma relativa cortadora (\ref{ex:lobo_52a}), ou o preenchimento da lacuna com um pronome (\ref{ex:lobo_52b}) ou com uma expressão nominal (\ref{ex:lobo_52c}).\footnote{Note-se que as estratégias não padrão (\ref{ex:lobo_52a}) e (\ref{ex:lobo_52b}) podem ser encontradas quer nas produções das crianças, quer nas produções dos adultos, embora sejam mais frequentes nas crianças.}

\ea\label{ex:lobo_52}
\ea\label{ex:lobo_52a} Gostava de ser o menino que o avô gosta. (vs. de que o avô gosta)
\ex\label{ex:lobo_52b} Gostava de ser o menino que a borboleta toca nele. (vs. em que a borboleta toca)
\ex\label{ex:lobo_52c} Gostava de ser o menino que o avô gosta do menino.
\zl

\section{A aquisição de clivadas}
\label{sec:lobo_aquisicao_clivadas}
\subsection{Tipos de clivadas: desenvolvimento de estruturas clivadas na produção espontânea e na produção induzida}
\label{subsec:lobo_tipos_clivadas}
As estruturas clivadas surgem na produção espontânea entre os 2 e os 3 anos e não são muito frequentes \citep{soares2006}. As crianças começam por produzir clivadas canónicas,\is{clivadas!clivadas canónicas} clivadas de \textit{é que}\is{clivadas!clivadas de \emph{é que}} e clivadas de \textit{ser}\is{clivadas!clivadas de \emph{ser}} e, tal como no caso das relativas, nos períodos iniciais de produção de clivadas estão atestadas clivadas com omissão do \isi{complementador}, como em (\ref{ex:lobo_53}):

\ea\label{ex:lobo_53} Era este (es)tava aqui!\jambox{(2;7)}\z

Para além disso, verifica-se que há uma diferença significativa entre a produção de clivadas de sujeito\is{clivadas!clivadas de sujeito} e a produção de clivadas de outros constituintes: as clivadas de sujeito\is{clivadas!clivadas de sujeito} são muito mais frequentes \citep{lobo_etal2016}. Numa tarefa que, a partir de imagens, induzia a produção de estruturas de foco contrastivo, em que estão incluídas as clivadas, aplicada a crianças entre 3;2 e 6;2 anos \citep{lobo_etal2016}, verificou-se que a assimetria entre as clivadas de sujeito\is{clivadas!clivadas de sujeito} e de outros constituintes se mantinha e que, apesar de as estruturas clivadas não serem frequentes e de a criança optar por muitas outras estratégias, há tipos de clivadas mais frequentes do que outros: encontra-se uma maior produção de clivadas de \textit{é que}\is{clivadas!clivadas de \emph{é que}} e de clivadas canónicas\is{clivadas!clivadas canónicas} comparativamente com outros tipos de estruturas clivadas, praticamente inexistentes. Assim, as pseudoclivadas\is{clivadas!pseudoclivadas} e as clivadas-wh,\is{clivadas!clivadas-wh} que envolvem o estabelecimento de uma dependência não gerada por movimento entre o constituinte-wh incluído numa oração não selecionada e o constituinte clivado, são de aquisição tardia. As crianças produziram muitas outras estruturas que não correspondem a clivadas, incluindo estruturas elípticas. Estas incluíam quer fragmentos\is{fragmento} que correspondem a constituintes (\ref{ex:lobo_54a}), quer fragmentos\is{fragmento} com o verbo copulativo SER (\ref{ex:lobo_54b}), que podem ser analisadas como clivadas elípticas (veja-se \ref{ex:lobo_54c}):

\ea\label{ex:lobo_54}
\ea\label{ex:lobo_54a} A menina.
\ex\label{ex:lobo_54b} Foi a menina.
\ex\label{ex:lobo_54c} Foi a menina \sout{que partiu o vidro}
\zl

Assim, verificou-se que, apesar de as crianças dominarem o contexto pragmático associado a foco contrastivo, há uma preferência maior das crianças por estruturas elípticas, que são também estratégias legítimas, embora possam ser fonte de maior ambiguidade. 


\subsection{Compreensão de estruturas clivadas}
\label{subsec:lobo_compreensao_clivadas}

Em \citet{lobo_etal2014}, testou-se, através de uma tarefa de avaliação da adequação de uma frase a uma imagem, a compreensão de três diferentes tipos de estruturas clivadas de sujeito\is{clivadas!clivadas de sujeito} e de objeto:\is{clivadas!clivadas de objeto} clivadas de \textit{é que},\is{clivadas!clivadas de \emph{é que}} clivadas canónicas\is{clivadas!clivadas canónicas} e pseudoclivadas.\is{clivadas!pseudoclivadas} Foram testadas 40 crianças de 4 e 5 anos, da zona da grande Lisboa. Verificou-se que, nas clivadas de \textit{é que}\is{clivadas!clivadas de \emph{é que}} (\ref{ex:lobo_55a}) e nas clivadas canónicas\is{clivadas!clivadas canónicas} (\ref{ex:lobo_55b}), há uma assimetria clara entre clivadas de sujeito\is{clivadas!clivadas de sujeito} e clivadas de objeto:\is{clivadas!clivadas de objeto} os desempenhos são muito inferiores nas clivadas de objeto.\is{clivadas!clivadas de objeto} Contudo, nas pseudoclivadas\is{clivadas!pseudoclivadas} (\ref{ex:lobo_55c}), não há diferenças significativas entre a compreensão de estruturas com movimento do sujeito e estruturas com movimento do objeto. Se, como propõem \citet{lobo_etal2014}, apenas nas clivadas de \textit{é que}\is{clivadas!clivadas de \emph{é que}} e nas clivadas canónicas\is{clivadas!clivadas canónicas} houver um constituinte movido com restrição lexical, estes resultados são esperados:

\ea\label{ex:lobo_55}
\ea\label{ex:lobo_55a} \textbf{A vaca} é que [\elip{a girafa} lambeu \sout{\textbf{a vaca}}].
\ex\label{ex:lobo_55b} Foi \textbf{a vaca} que [\elip{a girafa} lambeu \sout{\textbf{a vaca}}].
\ex\label{ex:lobo_55c} [\textbf{Quem} \elip{a girafa} lambeu \sout{\textbf{quem}}] foi a vaca.
\zl


\section{Conclusão}
\label{sec:lobo_conc}

Os estudos relatados neste capítulo mostram que grande parte das propriedades associadas ao movimento-wh\is{movimento!movimento-wh} e ao contexto pragmático de uso de interrogativas, relativas e clivadas\is{clivadas} são adquiridas em fases precoces. Contudo, determinados tipos de estruturas são de desenvolvimento mais tardio, podendo causar problemas ainda aos 5 anos de idade, nomeadamente: i) as que envolvem movimento-wh\is{movimento!movimento-wh} de um objeto (direto ou preposicionado) e \isi{intervenção} de um sujeito; e ii) as estruturas pseudoclivadas\is{clivadas!pseudoclivadas} e clivadas-wh,\is{clivadas!clivadas-wh} que envolvem uma dependência não gerada por movimento entre o constituinte-wh incluído numa oração não selecionada e o constituinte clivado. Um aspeto a reter é que os efeitos de \isi{intervenção} são comuns a relativas, interrogativas e \isi{clivadas}, estruturas que, como vimos, partilham a propriedade de envolverem movimento para uma posição na \isi{periferia} esquerda da frase.

Relativamente a propriedades como o preenchimento da posição de \isi{complementador} (com \textit{é que} ou com o verbo) e ao movimento de constituintes-wh preposicionados, a gramática da criança vai-se aproximando progressivamente da gramática adulta, favorecendo, em fases iniciais, estruturas mais económicas.

{\sloppy
\printbibliography[heading=subbibliography,notkeyword=this]
}
\end{document}