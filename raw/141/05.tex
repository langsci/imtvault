\chapter{Verb categorization}\label{AspectualClassification}
\is{aspect!Aristotelian|(}
This chapter deals with the classification of verbal expressions according to their inherent aspectual potential (Aristotelian aspect). The focus is on lexical verbs, although some observations on the phrasal level are also included. After a short introduction to the distinction between inchoative and non-inchoative verbs (\sectref{InchoativeNonInchoative}), the language internal diagnostics for distinguishing verb classes will be presented (\sectref{CriteriaClassificationOfVerbs}), followed by a discussion of each Nyakyusa verb class (\sectref{VerbClasses}).
\section{Introduction}\label{InchoativeNonInchoative}\is{inchoative verbs|(}\is{aspect!perfective|(}\is{aspect!persistive|(}
As mentioned in \sectref{AristotelianAspect}, an essential lexical distinction in many, if not all, Bantu languages, is the one between inchoative and non-inchoative verbs, i.e. those verbs that encode a change-of-state as well as a resultant state on the one hand, and those that do not on the other hand. A prototypical case of an inchoative verb in Nyakyusa is \textit{kalala} \lq be(come) angry'. With the aspectually imperfective\is{aspect!imperfective} \isi{simple present} (\sectref{Present}) this verb denotes an ongoing change-of-state (\ref{exInchoativeKalalaPRS}) (habitual/generic\is{aspect!habitual}\is{aspect!generic} and \isi{futurate} readings aside). When used with the present perfective (\sectref{PresentPerfective}), \textit{kalala} typically has a stative meaning of being angry (\ref{exInchoativeKalalaPRSPFV}).

\begin{exe}
\ex \label{exInchoativeKalalaPRS}\gll i-kʊ-kalal-a\\
1-\textsc{prs}-be(come)\_angry-\textsc{fv}\\
\glt `S/he is becoming angry.'
\ex \label{exInchoativeKalalaPRSPFV} \gll a-kaleele\\
1-be(come)\_angry.\textsc{pfv}\\
\glt (Default reading:) \lq S/he is angry.'
\end{exe}

At first sight, this might seem like the result of a conversational implicature: stating that one has become angry normally serves to indicate one's state of anger. That this resultant state is in fact part of the verb's lexical meaning becomes clear by contrasting it with a non-inchoative verb like \textit{fika} \lq arrive'. Like \textit{kalala}, this verb has a coming-to-be reading in the \isi{simple present} (\ref{exNonInchoativePRS}). However, when used in the present perfective it does not denote a state (\ref{exNonInchoativePFV}).

\begin{exe}
\ex \label{exNonInchoativePRS}\gll i-kʊ-fik-a\\
1-\textsc{prs}-arrive-\textsc{fv}\\
\glt `S/he is arriving.'
\ex \label{exNonInchoativePFV} \gll a-fik-ile\\
1-arrive-\textsc{pfv}\\
\glt `S/he has arrived.'
\end{exe}

The distinction between inchoative and non-inchoative verbs is not a mere question of translation, but has morphosyntactic repercussions. An inchoative like \textit{kalala} `be(come) angry' that is inflected for perfective aspect can stand as the complement of the persistive auxiliary (\sectref{Persistive}), where it denotes a continuing state (\ref{exInchoativePersIle}). A non-inchoative verb like \textit{fika} `arrive' is not licensed in this construction (\ref{exNonInchoativePersIle}), as it does not have a resultant state as part of its lexical meaning. Instead, to denote a persistent state, it is necessary to resort to other grammatical devices, in this case the existential construction (ex. \ref{exNonInchoativePersExistential}; see \sectref{Existentials}).

\begin{exe}
\ex[]{\gll a-kaalɪ a-kaleele\\
1-\textsc{pers} 1-be(come)\_angry.\textsc{pfv}\\
\glt `S/he is still angry.'}\label{exInchoativePersIle}
\ex[*]{\label{exNonInchoativePersIle}\gll a-kaalɪ a-fik-ile\\
1-\textsc{pers} 1-arrive-\textsc{pfv}\\
\glt(intended: \lq S/he is still present due to his/her arrival.')}
\ex[]{\label{exNonInchoativePersExistential}\gll a-kaalɪ a-li=po\\
1-\textsc{pers} 1-\textsc{cop}=16\\
\glt \lq S/he is still present.'}
\end{exe}

\citet[165]{BotneRKershnerT2000}, in a discusssion of \ili{Zulu} S42, summarize inchoative verbs as expressing ``a change of condition or location of the experiencer or patient''. For a verb denoting a mental state such as anger, this is fairly transparent from the perspective of English as the metalanguage. It is important to notice, however, that inchoative verbs are not limited to typical experiencer verbs.\is{semantic roles} For instance, `to carry', an activity in English,\il{English} is expressed by an inchoative verb in Nyakyusa (\ref{exInchoativeCarryTwala}). The distinction between inchoative and non-inchoative verbs is thus central to tense and aspect inflection in Nyakyusa.

\begin{exe}
\ex \label{exInchoativeCarryTwala}\gll \textup{(}a-kaalɪ\textup{)} a-twele ɪ-kɪ-kapʊ\\
(1-\textsc{pers}) 1-carry.\textsc{pfv} \textsc{aug}-7-basket\\
\glt `S/he is (still) carrying a/the basket.'
\end{exe}
\is{aspect!persistive|)}\is{aspect!perfective|)}

The basic opposition between inchoative and non-inchoative verbs will serve as a guiding line through most parts of the description of tense and grammatical aspect in Nyakyusa. A closer look, however, reveals more fine-grained distinctions and patternings of verbs. This is the topic of the following sections, which are intended as a first systematic approach towards Aristotelian aspect in Nyakyusa, ultimately to be enhanced by further research, and which is among the first such analyses for Bantu languages.\footnote{\label{FootnoteCriticismLusekelo2013}\citet{LusekeloA2013} includes a short discussion of the topic in Nyakyusa, which unfortunately is highly unsystematic and does not include any diagnostic criteria. Concerning other Bantu languages see \citet{MretaA1998} on \ili{Chasu} G20, who applies a variation of Breu and Sasse's framework (\sectref{AristotelianAspect}), \citet{FleischA2000} on \ili{Luchazi} K13, also within the Breu-Sasse framework and \citet{KershnerT2002}, \citet{BotneROchwadaHMarloM2006}, \citet{SeidelF2008}, \citet{BotneR2008} on \ili{Sukwa} M301, \ili{Saamia} JE34, \ili{Yeyi} R41, \ili{Ndali} M301, respectively. The latter authors apply Botne and Kershner's approach to Aristotelian aspect.} The analysis presented here is based on some 50 verbs which have been tested in targeted elicitation with at least two speakers each. Where available, uses in texts were also considered.\is{inchoative verbs|)}
\section{Diagnostic criteria}\label{CriteriaClassificationOfVerbs}
\is{phase|(}
Based on the tenets of radical selection theories of aspect (\sectref{AristotelianAspect}), a combination of various language-specific semantic and syntactic\is{syntax} diagnostic criteria have been applied to determine the lexicalized phasal structure of the verbs in question. Some of the diagnostics are adapted from \citeauthor{BotneR2008}'s (\citeyear{BotneR2008}) and \citeauthor{KershnerT2002}'s (\citeyear{KershnerT2002}) work on \ili{Ndali} and Sukwa,\il{Sukwa} others are taken from the typological literature or have emerged during the course of the present study.

As indicated in the previous section, only inchoative verbs,\is{inchoative verbs} that is those verbs that lexicalize a Coda state,\is{phase!Coda phase} are compatible with the syntactic\is{syntax} frame of the persistive aspect\is{aspect!persistive} auxiliary plus a complement inflected for perfective aspect.\is{aspect!perfective} Another diagnostic criterion is the possible readings with the imperfective\is{aspect!imperfective} \isi{simple present} (\sectref{Present}), or, more precisely, whether a progressive reading is available. As an extension of this, this study tested whether the verbs in question allow for a single event reading in the syntactic frame of the persistive aspect\is{aspect!persistive} \isi{auxiliary} plus the simple present,\is{simple present} and if so, which phase of the eventuality this denotes: A persistent process is taken as an indication of an extended Nucleus phase,\is{phase!Nucleus phase} whereas the lack of such a reading, but the possibility to coerce a reading of a persistently sustained result state is taken as secondary evidence for the lexical encoding of the result state. Related to the possible readings in the \isi{simple present} is the verb's behaviour with the periphrastic progressive\is{aspect!progressive} (\sectref{Progressive}), which shows mostly similar, but slightly different selectional properties.

A further diagnostic criterion for the classification of a given verb is its compatibility with the time-span phrase \lq take X time'. Only telic verbs, that is, verbs that encode an inherent endpoint or change-of-state, are predicted to appear readily in this construction, although repair readings are available for other types of verbs (see e.g. \citealt[57]{DowtyD1979}).

Further indications of the phasal structure of lexical verbs are found in their behaviour with \isi{phasal verbs} (\lq Aktionsart verbs' or \lq aspectualizers'), which patterns in significant ways with the tense-aspect constructions presented so far. The ingressive \isi{auxiliary} \textit{anda} \lq begin, start' denotes the beginning of the state-of-affairs encoded in the lexical verb. This can be a subphase of a single occurrence as well as the beginning of multiple occurrences. In the latter sense it can also be used to single out the first of various occurrences (\ref{exAndaVagheit}). Similarly, the auxiliaries\is{auxiliary} \textit{mala} \lq finish' (\ref{exMalaVagheit}) and \textit{leka} \lq cease, stop' (\ref{exLekaVagheit}) can refer to the termination or cessation of either a single or multiple occurrences.

\begin{exe}
\ex \label{exAndaVagheit} \gll tw-and-ile ʊ-kʊ-mog-a\\
\textsc{1pl}-begin-\textsc{pfv} \textsc{aug}-15-dance-\textsc{fv}\\
\glt 1. \lq We have (just) started dancing.'\\
2. \lq We have begun to dance (e.g. repeatedly or as a new habit).'\\
3. \lq We have begun (were the first) to dance.' 
\ex\label{exMalaVagheit} \gll tʊ-mal-ile ʊ-kʊ-mog-a\\
\textsc{1pl}-finish-\textsc{pfv} \textsc{aug}-15-dance-\textsc{fv}\\
\glt 1. \lq We have (just) finished dancing.'\\
2. \lq We are done dancing (multiple times).'

\ex\label{exLekaVagheit}\gll tʊ-lek-ile ʊ-kʊ-mog-a\\
\textsc{1pl}-cease-\textsc{pfv} \textsc{aug}-15-dance-\textsc{fv}\\
\glt 1. \lq We have (just) stopped dancing.'\\
2. \lq We have given up dancing.'
\end{exe}

What is relevant as a diagnostic of the aspectual potential encoded in the lexical verb is the possibility of a single event reading, together with the specific phase of the eventuality that is selected: is there a pre-culmination phase (either Onset\is{phase!Onset phase} or Coda)\is{phase!Coda phase} that can be said to start? If not, is there a resultant state whose early stages \textit{anda} can refer to? Likewise, is there a process (i.e. an extended Nucleus)\is{phase!Nucleus phase} that can be said to cease (\textit{leka}) or finish (\textit{mala})? If not, does the behaviour with these \isi{phasal verbs} give support for the phase structure diagnosed by means of the basic verb inflections? In the following discussion of aspectual classes, it is thus the single event reading that is referred to, without excluding further readings (unless stated otherwise).\label{SingleEventReading}

\tabref{TableOverviewVerbClasses} gives an overview of the verb classes identified and their behaviour in the respective constructions. The labels for the individual classes follow \citet{BotneR2003}. Values in brackets refer either to criteria that are not directly applicable, but for which the semantic clash can be resolved through repair readings, or to specific readings that are conditioned by semantic factors outside of aspectuality. In both cases, these are discussed in more detail in the sections on the individual verb classes.
\is{aspect!perfective|(}
\is{aspect!imperfective|(}
\is{aspect!progressive|(}
\is{aspect!persistive|(}
\is{auxiliary|(}
\is{phasal verbs|(}
\is{simple present|(}
\is{phase!Onset phase|(}
\is{phase!Nucleus phase|(}
\is{phase!Coda phase|(}
\is{inchoative verbs}

\begin{table}
\setlength{\tabcolsep}{4pt}
\fittable{
\begin{tabular}{>{\raggedright}m{2cm}lllll>{\columncolor[gray]{.6}}l>{\columncolor[gray]{.6}}l}
\lsptoprule
 & {Activity} & {Simple} & {Transitional} & {Transitional} & {Resultative} & {Inceptive} & {Acute}\\
& & {accomplishment} & {accomplishment} & {achievement} & {achievement} & {achievement} & {achievement}\\
\midrule
{Persistive of \textsc{pfv}} & no & no & yes & yes & yes & no & no\\
\tablevspace
{\textsc{prs} as progressive} & yes & yes & yes & yes & no & yes & no\\
\tablevspace
{Persistive of \textsc{prs}} & ongoing & ongoing & ongoing & (result state) & (result state) & n/a & n/a\\
\tablevspace
{Periphrastic \textsc{prog}} & ongoing & ongoing & ongoing & ongoing & result state & ongoing & n/a\\
\tablevspace
{\lq take X time'} refers to & (n/a) & culmination & culmination & culmination & culmination & culmination & culmination\\
\tablevspace
{\textit{anda} \lq start'} refers to & process & process & process & pre-change & result state & pre-change & n/a\\
\tablevspace
{\textit{mala} \lq finish'} refers to & process & process & process & (result state) & (result state) & n/a & n/a\\
\tablevspace
{\textit{leka} \lq cease'} refers to & process & process & process & (result state) & (result state) & n/a & n/a\\
\lspbottomrule
\end{tabular}
}
\caption{\label{TableOverviewVerbClasses}Overview of aspectual classes. Columns with a grey background designate tentative classes; see \sectref{VerbalClassOtherAchievements}. The criteria \lq \textsc{prs} as progressive', \lq Persistive of \textsc{prs}' and \textit{anda}/\textit{mala}/\textit{leka} refer to a possible single event reading only.}
\end{table}
 
\newpage
\section{Verb classes}\label{VerbClasses}
\subsection{Activities}\label{VerbalClassActivities}
Activity verbs encode a durative nuclear phase. Their phasal structure can be schematized as in \figref{FigureActivity} for \textit{moga} `dance'. 

\begin{figure}[h]
\begin{center}
\includegraphics{figures/GrafikActivity.eps}
\end{center}
\caption{Phasal structure of activity}
\label{FigureActivity}
\end{figure}
The traditional label \textit{activity} has been adopted for reasons of familiarity. It is important to notice that in Nyakyusa this class of verbs not only encompasses actions performed by a volitional agent,\is{semantic roles} such as \textit{kama} `milk', \textit{keeta} `look', \textit{lɪma} `cultivate' or \textit{lya} `eat', but also dynamisms such as \textit{bala} `shine (of sun)', \textit{tima} `rain' and verbs traditionally subsumed under states, such as \textit{swiga} `wonder' and \textit{tiila} `fear, obey, respect'. All of these pattern together in their syntactic\is{syntax} and semantic behaviour. \citet[271f]{SeidelF2008} makes a similar observation for \ili{Yeyi} R41.  Semelfactives such as \textit{kema} \lq bark' or \textit{kosomola} \lq cough' may also be subsumed under the category of activities. These verbs normally give a series reading and otherwise pattern with activities in the relevant diagnostic criteria.
 
In the simple present, activity verbs denote an ongoing activity or process (\ref{exActivityPRS}), although they may also have a habitual/generic\is{aspect!habitual}\is{aspect!generic} or \isi{futurate} reading. Similarly, the periphrastic progressive gives an ongoing reading (\ref{exActivityPROG}).

\begin{exe}
\ex \label{exActivityPRS}
\begin{xlist}
\ex \gll i-kʊ-mog-a\\
 1-\textsc{prs}-dance-\textsc{fv}\\
\glt `S/he is dancing.'
\ex \gll ɪɪ-fula jɪ-kʊ-tim-a\\
\textsc{aug}-rain(9) 9-\textsc{prs}-rain-\textsc{fv}\\
\glt `It is raining.'
\ex \gll ɪ-m-bwa jɪ-kʊ-kem-a\\
\textsc{aug}-9-dog 9-\textsc{prs}-bark-\textsc{fv}\\
\glt `A/the dog is barking.'
\end{xlist}

\ex \label{exActivityPROG}
\begin{xlist}
\ex \gll a-lɪ pa-kʊ-mog-a\\
1-\textsc{cop} 16-15-dance-\textsc{fv}\\
\glt `S/he is dancing.'
\ex \gll ɪɪ-fula jɪ-lɪ pa-kʊ-tim-a\\
\textsc{aug}-rain(9) 9-\textsc{cop} 16-15-rain-\textsc{fv}\\
\glt \lq It is raining.'
\ex \gll ɪ-m-bwa jɪ-lɪ pa-kʊ-kem-a\\
\textsc{aug}-9-dog 9-\textsc{cop} 16-15-bark-\textsc{fv}\\
\glt `A/the dog is barking.'
\end{xlist}
\end{exe}

The simple present as the complement of the persistive aspect auxiliary has a reading of a continuing process and can also have a persistent habitual/generic\is{aspect!habitual}\is{aspect!generic} reading:
\begin{exe}
\ex \label{exActivityPersPRS}
\begin{xlist}
\ex \gll a-kaalɪ i-kʊ-mog-a\\
1-\textsc{pers} 1-\textsc{prs}-dance-\textsc{fv}\\
\glt 1. \lq S/he is still dancing.' \\ 2. \lq S/he still dances.'

\ex \gll ɪɪ-fula jɪ-kaalɪ jɪ-kʊ-tim-a\\
\textsc{aug}-rain(9) 9-\textsc{pers} 9-\textsc{prs}-rain-\textsc{fv}\\
\glt 1. \lq It is still raining.' \\ 2. \lq It still rains.'
\ex \gll ɪ-m-bwa jɪ-kaalɪ jɪ-kʊ-kem-a\\
\textsc{aug}-9-dog 9-\textsc{pers} 9-\textsc{prs}-bark-\textsc{fv}\\
\glt 1. \lq A/the dog is still barking.' \\ 2. \lq A/the dog still barks.'
\end{xlist}
\end{exe}

In the present perfective, activity verbs denote a past eventuality (\ref{exActivityPFV}). Note that this also holds for state-like verbs such as \textit{tiila} \lq  fear' (\ref{exTiilaPfv}). As the class of activities does not encode a resultant state, perfective aspect is not licensed in the complement of the persistive (\ref{exActivityNotPersPFV}).

\begin{exe}
\begin{multicols}{2}
\ex \label{exActivityPFV}
\begin{xlist}
\ex \gll a-mog-ile\\
1-dance-\textsc{pfv}\\
\glt `S/he has danced.'
\ex \gll ɪɪ-fula jɪ-tim-ile\\
\textsc{aug}-rain(9) 9-rain-\textsc{pfv}\\
\glt `It has rained.'
\ex \gll ɪ-m-bwa jɪ-kem-ile\\
\textsc{aug}-9-dog 9-bark-\textsc{pfv}\\
\glt `A/the dog has barked.'
\ex \label{exTiilaPfv}\gll a-kʊ-tiil-ile\\
1-\textsc{2sg}-fear-\textsc{pfv}\\
\glt \lq S/he has feared you.'
\end{xlist}
\end{multicols}
\clearpage
\ex\label{exActivityNotPersPFV} \begin{xlist}
\ex[*]{\gll a-kaalɪ a-mog-ile\\ 
1-\textsc{pers} 1-dance-\textsc{pfv}\\}
\ex[*]{\gll ɪɪ-fula jɪ-kaalɪ jɪ-tim-ile\\
\textsc{aug}-rain(9) 9-\textsc{pers} 9-rain-\textsc{pfv}\\}
\ex[*]{\gll ɪ-m-bwa jɪ-kaalɪ jɪ-kem-ile\\
\textsc{aug}-9-dog 9-\textsc{pers} 9-bark-\textsc{pfv}\\}
\ex[*]{\gll a-kaalɪ a-kʊ-tiil-ile\\
1-\textsc{pers} 1-\textsc{2sg}-fear-\textsc{pfv}\\}
\end{xlist}
\end{exe}

As activity verbs do not encode an inherent endpoint, they are not directly compatible with the time-span verb phrase \lq take X time'. Two repair readings are available, however, and most activity verbs allow for at least one of the two. The first is a conative reading denoting the time that elapses before the beginning of the lexical act (see \citealt[57]{DowtyD1979} for a similar observation for English).\il{English} The second repair reading is that of a quasi-accomplishment; see p. \pageref{QuasiAccomplishment} below for discussion.

\begin{exe}
\ex\label{exActivityTakeXTime}
\begin{xlist}
\ex \gll eeg-ile a-ka-balɪlo a-ka-tali ʊ-kʊ-mog-a\\
1.take-\textsc{pfv} \textsc{aug}-12-time \textsc{aug}-12-long \textsc{aug}-15-dance-\textsc{fv}\\
\glt 1. \lq S/he took a long time to (begin to) dance.'\\
2. \lq S/he took a long time to finish dancing (i.e. at a social event).'

\ex \gll eeg-ile a-ka-balɪlo a-ka-tali ʊ-kʊ-ly-a\\
1.take-\textsc{pfv} \textsc{aug}-12-time \textsc{aug}-12-long \textsc{aug}-15-eat-\textsc{fv}\\
\glt 1. \lq  S/he took a long time to begin to eat.'\\
2. \lq  S/he took a long time to eat (i.e. finish the meal).' 

\ex \gll eeg-ile a-ka-balɪlo a-ka-pimba ʊ-kʊ-kosomol-a\\
1.take-\textsc{pfv} \textsc{aug}-12-time \textsc{aug}-12-short \textsc{aug}-15-cough-\textsc{fv}\\
\glt \lq S/he took a short time to finish coughing (i.e. overcome illness).'
\end{xlist}
\end{exe}

The auxiliary \textit{anda} `begin, start' refers to the beginning of the activity (\ref{exActivityAnda1}, \ref{exActivityAnda2}), or in the case of semelfactives the beginning of the series (\ref{exActivitySemelfactiveAnda}).

\begin{exe}
\ex \begin{xlist}
\ex \label{exActivityAnda1}\gll and-ile ʊ-kʊ-mog-a\\
1.begin-\textsc{pfv} \textsc{aug}-15-dance-\textsc{fv}\\
\glt `S/he has started to dance.' 
\ex \label{exActivityAnda2} \gll ɪɪ-fula j-and-ile ʊ-kʊ-tim-a\\
\textsc{aug}-rain(9) 9-begin-\textsc{pfv} \textsc{aug}-15-rain-\textsc{fv}\\
\glt `It has started to rain.'
\ex \label{exActivitySemelfactiveAnda}\gll ɪ-m-bwa j-and-ile ʊ-kʊ-kem-a\\
\textsc{aug}-9-dog 9-begin-\textsc{pfv} \textsc{aug}-15-bark-\textsc{fv}\\
\glt `A/the dog has started to bark.'
\end{xlist}
\end{exe}

\label{QuasiAccomplishment} The terminative auxiliary \textit{mala} can be used with some but not all activity verbs. This is apparently dependent on two factors. First, a \lq\lq quasi-accomplishment sense'' \citep[176]{BinnickR1991} needs to be available. Binnick, on the basis of \citet[61]{DowtyD1979}, observes that this is the case when one speaks about an activity that forms part of or constitutes a specific task or habit. The second requirement is not one of phasal structure, but of thematic relations, namely that the subject be an agent or force.\is{semantic roles} The latter is in agreement with the findings on simple and transitional accomplishments (\sectref{VerbalClassSimpleAccomplishment}, \ref{VerbalClassTransitionalAccomplishment}, respectively); see also \citet[135]{FreedA1979} on \ili{English} \textit{finish}. Thus compare (\ref{exActivityMalaOK1}, \ref{exActivityMalaOK2}) to (\ref{exActivityMalaBad1}, \ref{exActivityMalaBad2}).
\begin{exe}
\ex\begin{xlist}
\ex[]{\label{exActivityMalaOK1}\gll a-mal-ile ʊ-kʊ-mog-a\\
1-finish-\textsc{pfv} \textsc{aug}-15-dance-\textsc{fv}\\
\glt \lq S/he has finished dancing.'}
\ex[]{\label{exActivityMalaOK2}\gll ɪ-m-bwa jɪ-mal-ile ʊ-kʊ-kem-a\\
\textsc{aug}-9-dog 9-finish-\textsc{pfv} \textsc{aug}-15-bark-\textsc{fv}\\
\glt \lq The dog has finished barking.'}
\ex[*]{\label{exActivityMalaBad1}\gll ɪɪ-fula jɪ-mal-ile ʊ-kʊ-tim-a\\
\textsc{aug}-rain(9) 9-finish-\textsc{pfv} \textsc{aug}-15-rain-\textsc{fv}\\
\glt (intended: \lq The rain has finished.')}
\ex[*]{\label{exActivityMalaBad2}\gll a-mal-ile ʊ-kʊ-n-diil-a\\
1-finish-\textsc{pfv} \textsc{aug}-15-\textsc{1sg}-fear-\textsc{fv}\\
\glt (intended: \lq S/he is done fearing me.')}
\end{xlist}
\end{exe}

Lastly, the egressive auxiliary \textit{leka} denotes a cessation or interruption:
\begin{exe}
\ex
\begin{xlist}
\ex \gll a-lek-ile ʊ-kʊ-mog-a\\
1-cease-\textsc{pfv} \textsc{aug}-15-dance-\textsc{pfv}\\
\glt \lq S/he has stopped dancing.'
\ex \gll ɪɪ-fula jɪ-lek-ile ʊ-kʊ-tim-a\\
\textsc{aug}-rain(9) 9-cease-\textsc{pfv} \textsc{aug}-15-rain-\textsc{pfv}\\
\glt \lq It has stopped raining.'
\ex \gll ɪ-m-bwa jɪ-lek-ile ʊ-kʊ-kem-a\\
\textsc{aug}-9-dog 9-cease-\textsc{pfv} \textsc{aug}-15-bark-\textsc{fv}\\
\glt \lq A/the dog has stopped barking.'
\end{xlist}
\end{exe}

\subsection{Simple accomplishments}\label{VerbalClassSimpleAccomplishment}
Simple accomplishments encode an activity that is delimited by an endpoint. That is, they correspond to \citeauthor{VendlerZ1957}'s (\citeyear{VendlerZ1957}) \textit{accomplishments}. Following \citet{BotneR2008}, the qualification \textit{simple} has been adopted to distinguish them from their transitional counterpart (\sectref{VerbalClassTransitionalAccomplishment}). The phasal structure of simple accomplishments can be schematized as in \figref{FigureSimpleAccomplishment} for \textit{pona} \lq recover'. Other lexical verbs of this class are \textit{bɪfwa} \lq ripen', \textit{lembʊka} \lq wake up, get up' and \textit{talalɪla} \lq cool (intr.)'. Accomplishments can also be derived from activities, e.g. by means of a quantized primary object, as in \textit{lya ɪngʊkʊ joosa} \lq eat a whole chicken' or \textit{kama ɪɪng'ombe syosa} \lq milk all cows' (see \citealt{VerkuylH1972}; \citealt{DowtyD1979}). The derivation of accomplishments through other means, such as measure (\lq walk a mile') or goal noun phrases (\lq walk to the park') is open to further research. Note that in Nyakyusa, objects, once they have been introduced into discourse, are often understood from context without repetition or cross-referencing. Lastly, it should be noted that not all simple accomplishments constitute actions performed by a volitional agent.\is{semantic roles} The same is true for activity verbs.

\begin{figure}[h]
\begin{center}
\includegraphics{figures/GrafikSimpleAccomplishment.eps}
\end{center}
\caption{Phasal structure of simple accomplishment}
\label{FigureSimpleAccomplishment}
\end{figure}
Like activity verbs and transitional accomplishments, simple accomplishments denote an ongoing process in the simple present (\ref{exSimpleAccomplishmentPRS}), although they may also have a futurate and habitual or generic reading.\is{aspect!habitual}\is{aspect!generic} In the same vein, the periphrastic progressive refers to the process of change (\ref{exSimpleAccomplishmentPROG}).

\begin{exe}
\ex \label{exSimpleAccomplishmentPRS}
\begin{xlist}
\ex \gll ʊ-m̩-bine i-kʊ-pon-a\\
\textsc{aug}-1-ill 1-\textsc{prs}-recover-\textsc{fv}\\
\glt \lq A/the sick person is recovering.'

\ex \gll i-kʊ-lembʊk-a\\
1-\textsc{prs}-awake-\textsc{fv}\\
\glt \lq S/he is waking up.'

\ex \gll i-kʊ-ly-a ɪ-n-gʊkʊ j-oosa\\
1-\textsc{prs}-eat-\textsc{fv} \textsc{aug}-9-chicken 9-all\\
\glt \lq S/he is eating a whole chicken.'
\end{xlist}
\ex \label{exSimpleAccomplishmentPROG}
\begin{xlist}
\ex \gll ʊ-m̩-bine a-lɪ pa-kʊ-pon-a\\
\textsc{aug}-1-ill 1-\textsc{cop} 16-15-recover-\textsc{fv}\\
\glt \lq A/the sick person is recovering.'

\ex \gll a-lɪ pa-kʊ-lembʊk-a\\
1-\textsc{cop} 16-15-awake-\textsc{fv}\\
\glt \lq S/he is waking up.'

\ex \gll a-lɪ pa-kʊ-ly-a ɪ-n-gʊkʊ j-oosa\\
1-\textsc{cop} 16-15-eat-\textsc{fv} \textsc{aug}-9-chicken 9-all\\
\glt \lq S/he is eating a whole chicken.'
\end{xlist}
\end{exe}

As expected, the combination of persistive aspect and the simple present denotes the continuation of the process and further allows for a persistent habitual/generic reading:\is{aspect!habitual}\is{aspect!generic}

\begin{exe}
\ex
\begin{xlist}
\ex \gll ʊ-m̩-bine a-kaalɪ i-kʊ-pon-a\\
\textsc{aug}-1-ill 1-\textsc{pers} 1-\textsc{prs}-recover-\textsc{fv}\\
\glt 1. \lq  A/the sick person is still recovering.'\\ 2. \lq  A/the sick person still recovers (frequently).'

\ex \gll a-kaalɪ i-kʊ-lembʊk-a\\
1-\textsc{pers} 1-\textsc{prs}-awake-\textsc{fv}\\
\glt 1. \lq S/he is still waking up.'\\ 2. \lq S/he still wakes up.'

\ex \gll a-kaalɪ i-kʊ-ly-a ɪ-n-gʊkʊ j-oosa\\
1-\textsc{pers} 1-\textsc{prs}-eat-\textsc{fv} \textsc{aug}-9-chicken 9-all\\
\glt 1. \lq S/he is still [occupied with] eating a whole chicken.'\\
2. \lq S/he still eats whole chickens.'
\end{xlist}
\end{exe}

In the present perfective, simple accomplishments denote that the eventuality has passed (\ref{exSimpleAccomplishmentPFV}). As they do not encode a resultant state, perfective aspect is not licensed in the complement of the persistive aspect auxiliary (\ref{exSimpleAccomplishmentNoPERSPFV}).

\begin{exe}
%keine multicols weil akaali aliile ... zu lang
\ex\label{exSimpleAccomplishmentPFV}
\begin{xlist}
\ex[]{ \gll ʊ-m̩-bine a-pon-ile\\
\textsc{aug}-1-ill 1-recover-\textsc{pfv}\\
\glt \lq A/the sick person has recovered.'}

\ex[]{ \gll a-lembwike\\
1-awake.\textsc{pfv}\\
\glt \lq S/he has woken up.'}

\ex[]{ \gll a-l-iile ɪ-n-gʊkʊ j-oosa\\
1-eat-\textsc{pfv} \textsc{aug}-9-chicken 9-all\\
\glt \lq S/he has eaten a whole chicken.'}
\end{xlist}
\ex\label{exSimpleAccomplishmentNoPERSPFV} \begin{xlist}
\ex[*]{\gll a-kaalɪ a-pon-ile\\
1-\textsc{pers} 1-recover-\textsc{pfv}\\
\glt (intended: \lq S/he is still healed.')}
\ex[*]{\gll a-kaalɪ a-lembwike\\
1-\textsc{pers} 1-awake.\textsc{pfv}\\
\glt (intended: \lq S/he is still awake.')}
\ex[*]{\gll a-kaalɪ a-l-iile ɪ-n-gʊkʊ j-oosa\\
1-\textsc{pers} 1-eat-\textsc{pfv} \textsc{aug}-9-chicken 9-all\\
\glt (intended: \lq S/he is still full from eating a whole chicken.')}
\end{xlist}
\end{exe}%keine multicols, weil zu lang

With simple accomplishments, the time-span verb phrase \lq take X time' unambiguously refers to the time that elapses before the culmination of the process.

\begin{exe}
\ex
\begin{xlist}
\ex \gll ʊ-m̩-bine eeg-ile a-ka-balɪlo a-ka-tali ʊ-kʊ-pon-a\\
\textsc{aug}-1-ill 1.take-\textsc{pfv} \textsc{aug}-12-time \textsc{aug}-12-long \textsc{aug}-15-recover-\textsc{fv}\\
\glt \lq A/the sick person has taken a long time to recover.'

\ex \gll eeg-ile ɪɪ-sala j-oosa ʊ-kʊ-lembʊk-a\\
1.take-\textsc{pfv} \textsc{aug}-hour(9) 9-all \textsc{aug}-15-awake-\textsc{fv}\\
\glt \lq S/he has taken a whole hour to wake up.'

\ex \gll eeg-ile a-ka-balɪlo a-ka-pimba ʊ-kʊ-ly-a ɪ-n-gʊkʊ j-oosa\\
1.take-\textsc{pfv} \textsc{aug}-12-time \textsc{aug}-12-short \textsc{aug}-15-eat-\textsc{fv} \textsc{aug}-9-chicken 9-all\\
\glt \lq S/he has taken a short time to eat a whole chicken.'
\end{xlist}
\end{exe}

The ingressive auxiliary \textit{anda} \lq start, begin' refers to the beginning of the process:
\begin{exe}
\ex\begin{xlist}
\ex \gll ʊ-m̩-bine and-ile ʊ-kʊ-pon-a\\
\textsc{aug}-1-ill 1.begin-\textsc{pfv} \textsc{aug}-15-recover-\textsc{fv}\\
\glt \lq A/the sick person has begun to recover.'

\ex \gll and-ile ʊ-kʊ-lembʊk-a\\
1.begin-\textsc{pfv} \textsc{aug}-15-awake-\textsc{fv}\\
\glt \lq S/he has begun to wake up.'

\ex \gll and-ile ʊ-kʊ-ly-a ɪ-n-gʊkʊ j-oosa\\
1.begin-\textsc{pfv} \textsc{aug}-15-eat-\textsc{fv} \textsc{aug}-9-chicken 9-all\\
\glt \lq S/he has started to eat a whole chicken.'
\end{xlist}
\end{exe}
The terminative auxiliary \textit{mala} is compatible with some, but not all, accomplishments and denotes that the process has been completed. As with activity verbs in their quasi-accomplishment reading, and as will be seen for transitional accomplishments (\sectref{VerbalClassTransitionalAccomplishment}), \textit{mala} requires a subject with the semantic role of agent or force:\is{semantic roles}

\begin{exe}
\ex\begin{xlist}
\ex[]{\label{exSimpleAccomplishmentMalaOK1}\gll a-mal-ile ʊ-kʊ-lembʊk-a\\
1-finish-\textsc{pfv} \textsc{aug}-15-awake-\textsc{fv}\\
\glt \lq S/he has finished getting up.'}
\ex[]{\label{exSimpleAccomplishmentMalaOK2}\gll a-mal-ile ʊ-kʊ-ly-a ɪ-n-gʊkʊ j-oosa\\
1-finish-\textsc{pfv} \textsc{aug}-15-eat-\textsc{fv} \textsc{aug}-9-chicken 9-all\\
\glt \lq S/he has finished eating a whole chicken.'}
\ex[*]{\label{exSimpleAccomplishmentMalaBad1}\gll ʊ-m̩-bine a-mal-ile ʊ-kʊ-pon-a\\
\textsc{aug}-1-ill 1-finish-\textsc{pfv} \textsc{aug}-15-recover-\textsc{fv}\\
\glt (intended: \lq A/the sick person has accomplished recovery.')}
\ex[*]{\label{exSimpleAccomplishmentMalaBad2}\gll a-ma-tooki ga-mal-ile ʊ-kʊ-bɪfw-a\\
\textsc{aug}-6-banana 6-finish-\textsc{pfv} \textsc{aug}-15-ripen-\textsc{fv}\\
\glt (intended: \lq The bananas have become completely ripe.')}
\end{xlist}
\end{exe}

Lastly, \textit{leka} \lq cease, stop' denotes a cessation or interruption of the process:
\begin{exe}
\ex\begin{xlist}
\ex \gll ʊ-m̩-bine a-lek-ile ʊ-kʊ-pon-a\\
\textsc{aug}-1-ill 1-cease-\textsc{pfv} \textsc{aug}-15-recover-\textsc{fv}\\
\glt \lq A/the sick person has ceased to recover.'

\ex \gll a-lek-ile ʊ-kʊ-lembʊk-a\\
1-cease-\textsc{pfv} \textsc{aug}-15-awake-\textsc{fv}\\
\glt \lq S/he has ceased to wake up (viz. fallen asleep again).'

\ex \gll a-lek-ile ʊ-kʊ-ly-a ɪ-n-gʊkʊ j-oosa\\
1-cease-\textsc{pfv} \textsc{aug}-15-eat-\textsc{fv} \textsc{aug}-9-chicken 9-all\\
\glt \lq S/he has ceased eating a whole chicken.'
\end{xlist}
\end{exe} 

\subsection{Transitional accomplishments}\label{VerbalClassTransitionalAccomplishment}
\is{inchoative verbs|(}Transitional accomplishments encode a process that leads to a new state. They thus share characteristics of activity verbs and simple accomplishments on the one hand and of inchoative achievement verbs (\sectref{VerbalClassTransitionalAchievement}, \ref{VerbalClassResultativeAchievement}) on the other. Note that a distinction between transitional achievements and transitional accomplishments has so far only been observed for neighbouring \ili{Ndali} \citep{BotneR2008}. Its importance for a theory of Aristotelian aspect is highlighted in \citet{PersohnB2017b}. More systematic research on Aristotelian aspect in other Bantu languages might bring to light similar distinctions.

The phasal structure of transitional accomplishments can be schematized as in \figref{FigureTransitionalAccomplishment} for \textit{gaala} \lq get drunk, be drunk'. Other verbs of this class include \textit{fwala} `dress, wear', \textit{isʊla} \lq swell, be full', and \textit{onangɪka} \lq be(come) spoiled'. As becomes most clear with the last two verbs, and as is the case with activity verbs and simple accomplishments, the lexicalized process need not be dependent on a volitional agent.\is{semantic roles}

\begin{figure}[h]
\begin{center}
\includegraphics{figures/GrafikTransitionalAccomplishment.eps}
\end{center}
\caption{Phasal structure of transitional accomplishment}
\label{FigureTransitionalAccomplishment}
\end{figure}

Like activity verbs and simple accomplishments, transitional accomplishments denote an ongoing process in the simple present (\ref{exTransitionalAccomplishmentPRS}), as well as a possible futurate or habitual/generic reading.\is{aspect!habitual}\is{aspect!generic} In the same vein, the periphrastic progressive refers to the process of change (\ref{exTransitionalAccomplishmentPROG}).
\begin{exe}
\ex \label{exTransitionalAccomplishmentPRS}
\begin{xlist}
\ex\gll i-kʊ-gaal-a\\
1-\textsc{prs}-be(come)\_drunk-\textsc{fv}\\
\glt `S/he is getting drunk.'
\ex\gll i-kʊ-fwal-a \textup{(}ii-koti\textup{)}\\
1-\textsc{prs}-dress/wear-\textsc{fv} (5-coat<SWA)\\
\glt `S/he is dressing (putting on a/the coat).'
\end{xlist}
\ex \label{exTransitionalAccomplishmentPROG}
\begin{xlist}
\ex\gll a-lɪ pa-kʊ-gaal-a\\
1-\textsc{cop} 16-15-be(come)\_drunk-\textsc{fv}\\
\glt `S/he is getting drunk.'
\ex\gll a-lɪ pa-kʊ-fwal-a \textup{(}ii-koti\textup{)}\\
1-\textsc{cop} 16-15-dress/wear-\textsc{fv} (5-coat)\\
\glt `S/he is dressing (putting on a/the coat).'
\end{xlist}
\end{exe}

\largerpage
Also like activities and simple accomplishments, but unlike transitional achievements, the simple present as the complement of the persistive aspect auxiliary denotes the continuation of the pre-culmination process:
\begin{exe}
\ex\label{exTransitionalAccomplishmentPERSPRS}
\begin{xlist}
\ex\gll a-kaalɪ i-kʊ-gaal-a\\
1-\textsc{pers} 1-\textsc{prs}-be(come)\_drunk-\textsc{fv}\\
\glt 1. \lq S/he still gets drunk (regularly).'\\
2. \lq S/he is still getting drunk.'
\ex\gll a-kaalɪ i-kʊ-fwal-a \textup{(}ii-koti\textup{)}\\
1-\textsc{pers} 1-\textsc{prs}-dress/wear-\textsc{fv} (5-coat)\\
\glt 1. \lq S/he still dresses (puts on a/the coat).'\\
2. \lq S/he is still dressing (putting on a/the coat).'
\end{xlist}
\end{exe}

As with all inchoative verbs, but unlike activities and simple accomplishments, the perfective of transitional accomplishments is licensed as the complement of the persistive aspect auxiliary (\ref{exTransitionalAccomplishmentPERSPFV}), a combination that denotes a persistent resultant state.
\begin{exe}
\ex \label{exTransitionalAccomplishmentPERSPFV}\begin{xlist}
\ex\gll a-kaalɪ a-gaal-ile\\
1-\textsc{pers} 1-be(come)\_drunk-\textsc{pfv}\\
\glt `S/he is still drunk.'
\ex \gll a-kaalɪ a-fweele \textup{(}ii-koti\textup{)}\\
1-\textsc{pers} 1-dress/wear.\textsc{pfv} (5-coat)\\
\glt `S/he is still dressed (with a/the coat).'
\end{xlist}
\end{exe}

The time-span phrase \lq take X time', as with simple accomplishments, refers to the time elapsing before the culmination:
\begin{exe}
\ex \begin{xlist}
\ex \gll eeg-ile a-ka-balɪlo a-ka-tali ʊ-kʊ-gaal-a\\
1.take-\textsc{pfv} \textsc{aug}-12-time \textsc{aug}-12-long \textsc{aug}-15-be(come)\_drunk-\textsc{fv}\\
\glt \lq S/he took a long time to get drunk.'
\ex \gll eeg-ile a-ka-balɪlo a-ka-tali ʊ-kʊ-fwal-a \textup{(}ii-koti\textup{)}\\
1.take-\textsc{pfv} \textsc{aug}-12-time \textsc{aug}-12-long \textsc{aug}-15-dress/wear-\textsc{fv} (5-coat)\\
\glt \lq S/he took a long time to dress (put on a/the coat).'
\end{xlist}
\end{exe}

The auxiliary \textit{anda} \lq begin, start' in the single event reading denotes the beginning of the process of change:
\begin{exe}
\ex\begin{xlist}
\ex \gll and-ile ʊ-kʊ-gaal-a\\
1.begin-\textsc{pfv} \textsc{aug}-15-be(come)\_drunk-\textsc{fv}\\
\glt \lq S/he has begun to get drunk.'
\ex \gll and-ile ʊ-kʊ-fwal-a \textup{(}ii-koti\textup{)}\\
1.begin-\textsc{pfv} \textsc{aug}-15-dress/wear-\textsc{fv} (5-coat)\\
\glt \lq S/he has started to dress (to put on a/the coat).'
\end{xlist}
\end{exe}

The auxiliary \textit{mala} `finish' with transitional accomplishments can have a single event reading, in which case it refers to the culmination of the process. This behaviour is shared with activities and simple accomplishments, but not with transitional achievements. To be compatible with \textit{mala} requires the subject to have the semantic role of agent or force,\is{semantic roles} as in (\ref{exTransitionalAccomplishmentMalaOK1}, \ref{exTransitionalAccomplishmentMalaOK2}) but not (\ref{exTransitionalAccomplishmentMalaBad}).
\begin{exe}
\ex\begin{xlist}
\ex[]{\label{exTransitionalAccomplishmentMalaOK1}\gll a-mal-ile ʊ-kʊ-gaal-a\\
1-finish-\textsc{pfv} \textsc{aug}-15-be(come)\_drunk-\textsc{fv}\\
\glt \lq S/he has finished getting drunk (purposefully).'}
\ex[]{\label{exTransitionalAccomplishmentMalaOK2}\gll a-mal-ile ʊ-kʊ-fwal-a \textup{(}ii-koti\textup{)}\\
1-finish-\textsc{pfv} \textsc{aug}-15-dress/wear-\textsc{fv} (5-coat)\\
\glt \lq S/he has finished dressing (putting on a/the coat).'}
\ex[*]{\label{exTransitionalAccomplishmentMalaBad}\gll ii-galɪ lɪ-mal-ile ʊ-k-oonangɪk-a\\
5-car 5-finish-\textsc{pfv} \textsc{aug}-15-be(come)\_spoiled-\textsc{fv}\\
\glt (intended: \lq A/the car has broken down completely.')}
\end{xlist}
\end{exe}

Lastly, \textit{leka} \lq cease, stop' in a single event reading refers to a cessation or interruption of the process:
\begin{exe}
\ex \begin{xlist}
\ex \gll a-lek-ile ʊ-kʊ-gaal-a\\
1-cease-\textsc{pfv} \textsc{aug}-15-be(come)\_drunk-\textsc{fv}\\
\glt `S/he has ceased to get drunk (e.g. stopped drinking).'
\ex \gll a-lek-ile ʊ-kʊ-fwal-a \textup{(}ii-koti\textup{)}\\
1-cease-\textsc{pfv} \textsc{aug}-15-dress/wear-\textsc{fv} (5-coat)\\
\glt `S/he has stopped dressing (putting on a/the coat).'
\end{xlist}
\end{exe}
\is{inchoative verbs|)}
\subsection{Transitional achievements}\label{VerbalClassTransitionalAchievement}
\is{inchoative verbs|(}
Transitional achievements encode a change-of-state as well a pre-culmination state and a resultant state. Their phasal structure can thus be schematized as in \figref{FigureTransitionalAchievement} for \textit{kalala} `be(come) angry'. This class of verbs makes up the vast majority of achievements in the sample. Other examples include \textit{fugama} \lq kneel', \textit{fwa} `die', \textit{gwa paasi} \lq fall down', \textit{katala} `be(come) tired', \textit{kola} \lq grasp, hold' and \textit{nyala} \lq be(come) dirty'.

\begin{figure}[h]
\begin{center}
\includegraphics{figures/GrafikTransitionalAchievement.eps}
\caption{Phasal structure of transitional achievement}
\label{FigureTransitionalAchievement}
\end{center}
\end{figure}

In the simple present, transitional achievements have a coming-to-be reading (\ref{exTransitionalAchievementPRS}), as well as a habitual/generic\is{aspect!habitual}\is{aspect!generic} and a \isi{futurate} one. Likewise, the periphrastic progressive refers to the coming-to-be. Typically this is understood as being close to a change-of-state (\ref{exTransitionalAchievementPROG}).
\begin{exe}
\begin{multicols}{2}
\ex\label{exTransitionalAchievementPRS} \begin{xlist}
\ex \gll i-kʊ-kalal-a\\
1-\textsc{prs}-be(come)\_angry-\textsc{fv} \phantom{16-15}\\
\glt `S/he is becoming angry.'
\ex \gll i-kʊ-fw-a\\
1-\textsc{prs}-die-\textsc{fv}\\
\glt `S/he is dying.'
\end{xlist}
\columnbreak
\ex\label{exTransitionalAchievementPROG} \begin{xlist}
\ex \gll a-lɪ pa-kʊ-kalal-a\\
1-\textsc{cop} 16-15-be(come)\_angry-\textsc{fv}\\
\glt `S/he is about to be angry.'
\ex \gll a-lɪ pa-kʊ-fw-a\\
1-\textsc{cop} 16-15-die-\textsc{fv}\\
\glt `S/he is about to die.'
\end{xlist}
\end{multicols} 
\end{exe}%hier muss vllt getrickst werden mit \vspace*{\fill} nach jeweils dem ersten bsp o.ae.

Some, but not all, transitional achievements can be used with the simple present as the complement of the persistive aspect auxiliary (\ref{exTransitionalAchievementPERSPRS}). In this case, what is referred to is the continuation of the resultant state. This behaviour is shared with resultative achievements (\sectref{VerbalClassResultativeAchievement}) but not with transitional accomplishments (\sectref{VerbalClassTransitionalAccomplishment}). However, those language assistants that accepted these readings were either hesitant at first or considered these readings to be less natural than the use of the persistive plus perfective aspect, which makes this look like a clear case of coercion; see \citet{MichaelisL2004} for a theory of coercion.

\begin{exe}
\ex \label{exTransitionalAchievementPERSPRS}\begin{xlist}
\ex[]{\gll a-kaalɪ i-kʊ-fugam-a\\
1-\textsc{pers} 1-\textsc{prs}-kneel-\textsc{fv}\\
\glt 1. \lq S/he still kneels.'\\
2. \lq  S/he is still kneeling.'}
\ex[*]{\gll a-kaalɪ i-kʊ-fw-a\\
1-\textsc{pers} 1-\textsc{prs}-die-\textsc{fv}\\
\glt (intended: \lq S/he is still being dead.')}
\end{xlist}
\end{exe}

As with all inchoative verbs, the perfective form of transitional achievements is licensed as the complement of the persistive aspect auxiliary (\ref{exTransitionalAchievementPersPFV}).
\begin{exe}
\ex \label{exTransitionalAchievementPersPFV}\begin{xlist}
\ex\gll a-kaalɪ a-kaleele\\
1-\textsc{pers} 1-be(come)\_angry.\textsc{pfv}\\
\glt `S/he is still angry.'
\ex \gll ii-lʊʊka lɪ-kaalɪ lɪ-fw-ile\\
5-store 5-\textsc{pers} 5-die-\textsc{pfv}\\
\glt `The store is still dead (viz. closed).'
\end{xlist}
\end{exe}

The ingressive auxiliary \textit{anda} `begin, start' refers to the beginning of the development:
\begin{exe}
\ex \begin{xlist}
\ex \gll and-ile ʊ-kʊ-kalal-a\\
1.begin-\textsc{pfv} \textsc{aug}-15-be(come)\_angry-\textsc{fv}\\
\glt `S/he has started to become angry.'
\ex \gll and-ile ʊ-kʊ-fw-a\\
1.begin-\textsc{pfv} \textsc{aug}-15-die-\textsc{fv}\\
\glt `S/he has started to die.'
\end{xlist}
\end{exe}
Some transitional achievements are compatible with \textit{mala} \lq finish', in which case the reference is to the eventuality as a whole. As observed in \sectref{VerbalClassActivities}--\ref{VerbalClassTransitionalAccomplishment}, this requires the subject to have the semantic role\is{semantic roles} of agent or force, or to be construable as such. Thus compare (\ref{exTransitionalAchievementMala1}, \ref{exTransitionalAchievementMala2}) to (\ref{exTransitionalAchievementMala3}, \ref{exTransitionalAchievementMala4}):

\begin{exe}
\ex
\begin{xlist}
\ex[]{\label{exTransitionalAchievementMala1}\gll a-mal-ile ʊ-kʊ-kol-a ii-bwe\\
1-finish-\textsc{pfv} \textsc{aug}-15-grasp/hold-\textsc{fv} 5-stone\\
\glt \lq S/he has finished holding a/the stone.'
}
\ex[]{\label{exTransitionalAchievementMala2}\gll a-mal-ile ʊ-kʊ-fugam-a\\
1-finish-\textsc{pfv} \textsc{aug}-15-kneel-\textsc{fv}\\
\glt \lq S/he has finished kneeling.'}

\ex[?]{\label{exTransitionalAchievementMala3}\gll a-mal-ile ʊ-kʊ-kalal-a\\
1-finish-\textsc{pfv} \textsc{aug}-15-be(come)\_angry-\textsc{fv}\\
\glt \lq S/he has finished being angry.'}

\ex[*]{\label{exTransitionalAchievementMala4}\gll ii-lʊʊka lɪ-mal-ile ʊ-kʊ-fw-a\\
5-store 5-finish-\textsc{pfv} \textsc{aug}-15-die-\textsc{fv}\\
\glt (intended: \lq The store is not closed anymore (i.e. has opened again).')}
\end{xlist}
\end{exe}

Likewise, a single event reading with the egressive auxiliary \textit{leka} is available for some, but not all, transitional achievements. In accordance with their meaningful focus on the resultant state, this denotes an interruption or cessation of the latter. It is not entirely clear what the determining semantic factors are. The following examples suggest that at least world knowledge comes into play:

\begin{exe}
\ex\begin{xlist}
\ex[]{\label{exTransAchievementLeka1}\gll a-lek-ile ʊ-kʊ-fugam-a\\
1-cease-\textsc{pfv} \textsc{aug}-15-kneel-\textsc{fv}\\
\glt \lq S/he has ceased to be kneeling.'}
\ex[]{\label{exTransAchievementLeka2}\gll a-lek-ile ʊ-kʊ-kalal-a\\
1-cease-\textsc{pfv} \textsc{aug}-15-be(come)\_angry-\textsc{fv}\\
\glt \lq S/he has ceased to be angry.'}
\ex[?]{\gll ɪɪ-nyumba jɪ-lek-ile ʊ-kʊ-nyala-a\\
\textsc{aug}-house(9) 9-cease-\textsc{pfv} \textsc{aug}-15-be(come)\_dirty-\textsc{fv}\\
\glt \lq A/the house has ceased to be dirty.'}
\ex[*]{\gll ii-lʊʊka ly-a-fw-ile looli lɪ-lek-ile ʊ-kʊ-fw-a\\
5-store 5-\textsc{pst}-die-\textsc{pfv} but 5-cease-\textsc{pfv} \textsc{aug}-15-die-\textsc{fv}\\
\glt (intended: \lq The store was closed but it is not closed anymore.')}
\end{xlist}
\end{exe}

The language assistants commented that examples such as (\ref{exTransAchievementLeka1}, \ref{exTransAchievementLeka2}) are acceptable but not very natural. A more common way to refer to a state, e.g. of anger, that has come to an end would be the following:

\begin{exe}
\ex 
\begin{xlist}
\gll a-a-kaleele, looli si-mal-iike\\
1-\textsc{pst}-be(come)\_angry.\textsc{pfv} now/but 10-finish-\textsc{neut.pfv}\\
\glt \lq S/he was angry, but now it is over.'
\end{xlist}
\end{exe}

Last, the time-span phrase \lq take X time' refers to the time elapsing before the change-of-state:

\begin{exe}
\ex  \gll eeg-ile a-ka-balɪlo a-ka-pimba fiijo ʊ-kʊ-kalal-a\\
1.take-\textsc{pfv} \textsc{aug}-12-time \textsc{aug}-12-short \textsc{intens} \textsc{aug}-15-be(come)\_angry-\textsc{fv}\\
\glt \lq S/he got angry in a very short time.'
\ex  \gll eeg-ile a-ka-balɪlo a-ka-tali ʊ-kʊ-fugam-a\\
1.take-\textsc{pfv} \textsc{aug}-12-time \textsc{aug}-12-short \textsc{aug}-15-kneel-\textsc{fv}\\
\glt \lq S/he took a lot of time to get on his/her knees.'
\end{exe}
\is{inchoative verbs|)}
\subsection{Resultative achievements}\label{VerbalClassResultativeAchievement}
\is{inchoative verbs|(}
Resultative achievements encode a change-of-state together with the resultant state. Their phasal structure can thus be schematized as in \figref{FigureResultativeAchievement} for \textit{hoboka} `be(come) happy'. Other examples are \textit{benga} \lq hate', \textit{gana} \lq like, love', \textit{gona ʊtʊlo} \lq sleep' and \textit{twala} \lq carry, bring'.

\begin{figure}[h]
\begin{center}
\includegraphics{figures/GrafikResultativeAchievement.eps}
\caption{Phasal structure of resultative achievement}
\label{FigureResultativeAchievement}
\end{center}
\end{figure}

In the simple present, resultative achievements have a \isi{futurate} reading as well as a habitual/generic one,\is{aspect!habitual}\is{aspect!generic} but no progressive reading (\ref{exResultiveAchievementPRS}).

\begin{exe}
\ex \label{exResultiveAchievementPRS} \begin{xlist}
\ex \gll i-kʊ-hobok-a\\
1-\textsc{prs}-be(come)\_happy-\textsc{fv}\\
\glt 1. `S/he will become happy.'\\2. `S/he becomes happy (e.g. on each particular occasion).' 

\ex \gll i-kʊ-m-beng-a\\
1-\textsc{prs}-\textsc{1sg}-hate-\textsc{fv}\\
\glt 1. \lq S/he will hate me.'\\
2. \lq S/he hates me (e.g. shows it every time we meet).'

\end{xlist}
\end{exe}

The periphrastic progressive construction with resultative achievements refers to the resultant state (\ref{exResultativeAchievementPROG}). This is unlike transitional achivements (\sectref{VerbalClassTransitionalAchievement}), which do encode an Onset state. 
\begin{exe}
\ex \label{exResultativeAchievementPROG}\begin{xlist}
\ex\gll a-lɪ pa-kʊ-hobok-a\\
1-\textsc{cop} 16-15-be(come)\_happy-\textsc{fv}\\
\glt `S/he is happy.'
\ex \gll a-lɪ pa-kʊ-m-beng-a\\
1-\textsc{cop} 16-15-\textsc{1sg}-hate-\textsc{fv}\\
\glt \lq S/he hates me (e.g. is acting hostile).'
\end{xlist}
\end{exe}

Some, but not all, resultative achievements can be coerced into a progressive reading with the simple present as the complement of the persistive aspect auxiliary (\ref{exResultativeAchievementPERSPRS}). As with transitional achievements, what is referred to in this case is the resultant state.

\begin{exe}
\ex \label{exResultativeAchievementPERSPRS}
\begin{xlist}
\ex\gll a-kaalɪ i-kʊ-hobok-a\\
1-\textsc{pers} 1-\textsc{prs}-be(come)\_happy-\textsc{fv}\\
\glt 1. \lq S/he still becomes happy.'\\
2. \lq  S/he is still being happy.'

\ex\gll a-kaalɪ i-kʊ-m-beng-a\\
1-\textsc{pers} 1-\textsc{prs}-\textsc{1sg}-hate-\textsc{fv}\\
\glt 1. \lq S/he still hates me (e.g. still shows it every time we meet).'\\
2. \lq  S/he is still hating me (e.g. acting hostile).'
\ex \gll a-kaalɪ i-kʊ-twal-a ɪ-kɪ-kapʊ\\
1-\textsc{pers} 1-\textsc{prs}-carry-\textsc{fv} \textsc{aug}-7-basket\\
\glt \lq S/he still carries a/the basket (regularly).'\\
not: \lq S/he is still carrying a/the basket.'
\end{xlist}
\end{exe}

As with all inchoative verbs, the common way to refer to the resultant state is with the use of the perfective aspect, which is licensed as the complement of the persistive aspect auxiliary (\ref{exResultativeAchievementPERSFV}), a combination that denotes the persistence of the resultant state.
\begin{exe}
\ex \label{exResultativeAchievementPERSFV}\begin{xlist}
\ex\gll a-kaalɪ a-hobwike\\
1-\textsc{pers} 1-be(come)\_happy.\textsc{pfv}\\
\glt `S/he is still happy.'

\ex \gll a-kaalɪ a-m-beng-ile\\
1-\textsc{pers} 1-\textsc{1sg}-hate-\textsc{pfv}\\
\glt \lq S/he still hates me.'
\end{xlist}
\end{exe}

The time-span phrase \lq take X time' refers to the time elapsing before the entry into the new state:

\begin{exe}
\ex \begin{xlist}
\ex \gll eeg-ile a-ka-balɪlo a-ka-tali ʊ-kʊ-hobok-a\\
1.take-\textsc{pfv} \textsc{aug}-12-time \textsc{aug}-12-long \textsc{aug}-15-be(come)\_happy-\textsc{fv}\\
\glt \lq S/he took a long time to become happy.'

\ex \gll eeg-ile a-ka-balɪlo a-ka-pimba ʊ-kʊ-m-beng-a\\
1.take-\textsc{pfv} \textsc{aug}-12-time \textsc{aug}-12-short \textsc{aug}-15-\textsc{1sg}-hate-\textsc{fv}\\
\glt \lq S/he came to hate me within a short time.'
\end{xlist}
\end{exe}

Related to the behaviour of resultative achievements with the simple present and periphrastic progressive, the auxiliary \textit{anda} `begin, start' in the single event reading refers to an initial subphase of the resultant state:
\begin{exe}
\ex \begin{xlist}
\ex \gll and-ile ʊ-kʊ-hobok-a\\
1.begin-\textsc{pfv} \textsc{aug}-15-be(come)\_happy-\textsc{fv}\\
\glt `S/he has begun to be happy.'
\ex\gll and-ile ʊ-kʊ-m-beng-a\\
1.begin-\textsc{pfv} \textsc{aug}-15-\textsc{1sg}-hate-\textsc{fv}\\
\glt \lq S/he has begun to hate me.'
\end{xlist}
\end{exe}

Parallel to what has been observed for transitional achievements, at least some resultative achievements can be coerced into a progressive reading of the resultant state in the syntactic frame of the simple present as the complement of the persistive aspect auxiliary. 

\begin{exe}
\ex[]{\gll a-kaalɪ i-kʊ-hobok-a\\
1-\textsc{pers} 1-\textsc{prs}-be(come)\_happy-\textsc{fv}\\
\glt 1. \lq S/he still becomes happy (e.g. on each certain occasion).'\\
2. \lq S/he is still being (behaving) happy.'}
\ex[]{\gll a-kaalɪ i-kʊ-m-beng-a\\
1-\textsc{pers} 1-\textsc{prs}-\textsc{1sg}-hate-\textsc{fv}\\
\glt 1. \lq S/he still hates me (generally speaking).'\\
2. \lq S/he is still hating me (i.e. acting hostile).'}
\end{exe}

The auxiliary \textit{mala} \lq finish' refers to the resultant state. As has been observed in the preceding sections, \textit{mala} requires its subject to have the semantic role of agent or force:\is{semantic roles}

\begin{exe}
\ex[]{\gll a-mal-ile ʊ-kʊ-twal-a ɪ-kɪ-kapʊ\\
1-finish-\textsc{pfv} \textsc{aug}-15-cary-\textsc{fv} \textsc{aug}-7-basket\\
\glt \lq S/he has finished carrying a/the basket.'}
\ex[]{\gll a-mal-ile ʊ-kʊ-gon-a ʊ-tʊ-lo\\
1-finish-\textsc{pfv} \textsc{aug}-15-rest-\textsc{fv} \textsc{aug}-12-sleep\\
\glt \lq S/he has finished sleeping.'}
\ex[*]{\gll a-mal-ile ʊ-kʊ-hobok-a\\
1-finish-\textsc{pfv} \textsc{aug}-15-be(come)\_happy-\textsc{fv}\\
\glt (intended: \lq S/he is not happy anymore.')}
\ex[*]{\gll a-mal-ile ʊ-kʊ-m-beng-a\\
1-finish-\textsc{pfv} \textsc{aug}-15-\textsc{1sg}-hate-\textsc{fv}\\
\glt (intended: \lq S/he does not hate me anymore.')}
\end{exe}

As is the case with transitional achievements, a single event reading with the egressive auxiliary \textit{leka} is available for at least some resultative achievements:

\begin{exe}
\ex \begin{xlist}
\ex[]{\gll a-lek-ile ʊ-kʊ-gon-a ʊ-tʊ-lo\\
1-cease-\textsc{pfv} \textsc{aug}-15-rest-\textsc{fv} \textsc{aug}-12-sleep\\
\glt \lq S/he has ceased to be sleeping.'}
\ex[]{\gll a-lek-ile ʊ-kʊ-n̩-gan-a\\
1-cease-\textsc{pfv} \textsc{aug}-15-1-like/love-\textsc{fv}\\
\glt \lq S/he has ceased to love him/her.'}
\ex[?]{\gll a-lek-ile ʊ-kʊ-hobok-a\\
1-cease-\textsc{pfv} \textsc{aug}-15-be(come)\_happy-\textsc{fv}\\
\glt \lq S/he has ceased to be happy.'}
\end{xlist}
\end{exe}

One verb in the sample, \textit{manya} \lq know', patterns to a large extent with resultative achievements. Unlike the latter, however, \textit{manya} is incompatible with the periphrastic progressive:
\begin{exe}
\ex[*]{\label{exManyaProgressive} \gll a-lɪ pa-kʊ-many-a ɪ-kɪ-ngelesa\\
1-\textsc{cop} 16-15-know-\textsc{fv} \textsc{aug}-7-English\\
\glt (intended: \lq S/he is learning English.' or \lq S/he knows English.')}
\end{exe}

In all other respects, \textit{manya} \lq know' behaves no differently from the verbs discussed so far in this section. Thus in the simple present it has a generic (\ref{exManyaGeneric}) as well as a futurate reading (\ref{exManyaFuturate}). To refer to the state of having knowledge, the perfective aspect is employed (\ref{exManyaPFV}) and its compatibility with persistive aspect shows that this state forms part of its lexical meaning (\ref{exManyaPERSPFV}).

\begin{exe}
\ex \label{exManyaGeneric}
\gll a-baa-sukuulu bi-kʊ-many-a ɪ-kɪ-ngelesa\\
\textsc{aug}-2-student 2-\textsc{prs}-know-\textsc{fv} \textsc{aug}-7-English\\
\glt \lq Students know English.'
\ex \label{exManyaFuturate}
\gll lɪlɪno kʊʊ-many-a, fiki ʊ-ti-kw-amul-a bo n-gʊ-kʊ-laalʊʊsy-a?\\
now/today \textsc{2sg.prs.1sg}-know-\textsc{fv} why \textsc{2sg}-\textsc{neg}-\textsc{prs}-answer-\textsc{fv} as \textsc{1sg}-\textsc{prs}-\textsc{2sg}-ask-\textsc{fv}\\
\glt \lq Now you'll get to know me, why don't you answer when I'm asking you?' [Saliki and Hare]
\ex \label{exManyaPFV} \gll a-meenye ɪ-kɪ-ngelesa\\
1-know.\textsc{pfv} \textsc{aug}-7-English(<SWA)\\
\glt \lq S/he knows English.'
\ex \label{exManyaPERSPFV}\gll a-kaalɪ a-meenye ɪ-kɪ-ngelesa\\
1-\textsc{pers} 1-know.\textsc{pfv} \textsc{aug}-7-English\\
\glt \lq S/he still knows English.'
\end{exe}

The time-span verb phrase \lq take X time' refers to the time elapsing before entering into the state of knowledge (\ref{exManyaTakeXTime}). Also note that \textit{manya} in the \isi{narrative tense} -- see \sectref{NarrativeTense} -- normally yields a change-of-state reading (\ref{exManyaNarrative}).

\begin{exe}
\ex \label{exManyaTakeXTime} \gll eeg-ile ɪ-fy-ɪnja f-ingi ʊ-kʊ-many-a ɪ-kɪ-ngelesa\\
1.take-\textsc{pfv} \textsc{aug}-8-year 8-many \textsc{aug}-15-know-\textsc{fv} \textsc{aug}-7-English\\
\glt \lq S/he took many years to (get to) know English.' [ET] 

\ex \label{exManyaNarrative} \gll ʊ-mw-ene Jesu nakalɪnga a-lɪnkʊ-\textbf{many}-a mu-n-dumbula j-aake ʊkʊtɪ bi-kw-inogon-a bo ɪ-si mu-n-dumbula sy-abo, a-lɪnkʊ-ba-laalʊʊsy-a, a-lɪnkʊ-tɪ \ldots\\
\textsc{aug}-1-self J. immediately 1-\textsc{narr}-know-\textsc{fv} 18-9-heart 9-\textsc{poss.sg} \textsc{comp} 2-\textsc{prs}-think-\textsc{fv} as \textsc{aug}-\textsc{prox.10} 18-10-heart 10-\textsc{poss.pl} 1-\textsc{narr}-2-ask-\textsc{fv} 1-\textsc{narr}-say {}\\
\glt \lq And immediately when Jesus perceived in his spirit that they so reasoned within themselves, he said unto them \ldots [J. immediately understood in his heart \ldots he asked them \ldots]' (Mark 2: 8)
\end{exe}

In their discussion of neighbouring \ili{Ndali} and Sukwa,\il{Sukwa} \citet{BotneR2008} and \citet{KershnerT2002}, respectively, recognize a separate group of purely static verbs, which among others includes the cognates of Nyakyusa \textit{benga} \lq hate', \textit{gana} \lq love', and, in the case of Sukwa, \textit{manya} \lq know'. As seen above, in Nyakyusa the first two pattern with other verbs such as \textit{hoboka} \lq be(come) happy' as resultative achievements. It is noteworthy that these putatively stative verbs in \ili{Ndali} and \ili{Sukwa} pattern with the other classes of inchoatives with respect to their behaviour with perfective aspect (\lq\lq completive'' in Botne \& Kershner's terms; see \sectref{PerfectivityCompletion} for discussion), which is unfortunately not further discussed by these authors. The validity of a separate class of states in Nyakyusa remains open to further research. Note that \citet{SeidelF2008} does not recognize such a class for \ili{Yeyi} R41. Concerning the broader Niger-Congo context, \citet[ch. 5.4]{ToewsC2015} finds that \ili{Siamou} (Kru) entirely lacks state verbs. In order to describe stative situations, other strategies are evoked, namely non-verbal predicates, a stativizing verbal suffix, imperfective aspect with certain non-inchoatives (often as the result of a figurative reading) and perfective aspect with inchoative verbs.
\is{inchoative verbs|)}
\subsection{Other achievement classes}\label{VerbalClassOtherAchievements}
Two verbs in the sample, \textit{fika} \lq arrive' and \textit{aga} \lq find', classify as achievements, but they both differ from the achievements classes discussed in the preceding sections in important ways. These two verbs can be taken as representatives of the classes of inceptive and acute achievements (Kershner's \lq inceptive punctives' and \lq achievement punctives'), which are well-established classes in neighbouring \ili{Ndali} and Sukwa.\il{Sukwa} Their scarcity in the sample is most likely due to the limited sample of verbs. As a single verb each, however, is insufficient to justify an achievement class of its own, this classification remains tentative. The two verbs are discussed jointly in this section.

To begin with, \textit{fika} in the simple present has a coming-to-be reading (\ref{exInceptiveAchievementPRS}) as well as a habitual/generic\is{aspect!habitual}\is{aspect!generic} and a \isi{futurate} one. Likewise, the periphrastic progressive refers to the coming-to-be (\ref{exInceptiveAchievementProg}). This indicates a lexical Onset phase and parallels the transitional achievements (\sectref{VerbalClassTransitionalAchievement}).

\begin{exe}
\begin{multicols}{2}
\ex \label{exInceptiveAchievementPRS}

\gll i-kʊ-fik-a\\
1-\textsc{prs}-arrive-\textsc{fv}\\
\glt \lq S/he is arriving.'
\columnbreak
\ex \label{exInceptiveAchievementProg}
\gll a-lɪ pa-kʊ-fik-a\\
1-\textsc{cop} 16-15-arrive-\textsc{fv}\\
\glt \lq S/he is arriving.'
\end{multicols}
\end{exe}

\textit{Fika} also resembles transitional achievements in that the simple present as the complement of the persistive aspect auxiliary has a habitual/generic\is{aspect!habitual}\is{aspect!generic} reading, but not one of a progressive change-of-state:

\begin{exe}
\ex \gll a-kaalɪ i-kʊ-fik-a\\
1-\textsc{pers} 1-\textsc{prs}-arrive-\textsc{fv}\\
\glt \lq S/he still arrives (regularly).'\\
not: \lq S/he is still arriving.'
\end{exe}

Further proof of a lexicalized Onset phase is found in the behaviour of \textit{fika} with \textit{anda} \lq begin, start'. This auxiliary has a habitual/generic\is{aspect!habitual}\is{aspect!generic} reading and 
can also refer to the preliminary phase of a single eventuality with \textit{fika} (\ref
{exInceptiveAchievementsAnda}).

\begin{exe}
\ex\label{exInceptiveAchievementsAnda}
\gll and-ile ʊ-kʊ-fik-a\\
1.begin-\textsc{pfv} \textsc{aug}-15-arrive-\textsc{fv}\\
\glt 1. \lq S/he has begun to arrive (e.g. get to a place regularly).'\\
2. \lq She has begun to arrive (right now).'
\end{exe}

As expected, the time-span verb phrase \lq take X time' with \textit{fika} refers to the time elapsing before the change-of-state:

\begin{exe}
\ex \gll eeg-ile a-ka-balɪlo a-ka-tali ʊ-kʊ-fik-a\\
1.take-\textsc{pfv} \textsc{aug}-12-time \textsc{aug}-12-long \textsc{aug}-15-arrive-\textsc{fv}\\
\glt \lq S/he took a long time to arrive.'
\end{exe}

The perfective aspect with \textit{fika} denotes that the eventuality has passed (\ref
{exInceptiveAchievementPerfective}). The fact that perfective aspect is not licensed in the complement of the persistive aspect auxiliary (\ref{exInceptiveAchievementNoPersPerfective}) provides proof that, unlike transitional and resultative achievements, no Coda state is lexically encoded.
\begin{exe}
\ex\label{exInceptiveAchievementPerfective}
\begin{multicols}{2}
\gll a-fik-ile\\
1-arrive-\textsc{pfv}\\
\glt \lq S/he has arrived.'
\columnbreak
\ex[*]{\label{exInceptiveAchievementNoPersPerfective}
\gll a-kaalɪ a-fik-ile\\
1-\textsc{pers} 1-arrive-\textsc{pfv}\\}
\end{multicols}
\end{exe}

Lastly, \textit{mala} \lq finish' is not compatible with \textit{fika} (\ref{exInceptiveAchievementNoMala}), while \textit{leka} \lq cease, stop' denotes the cessation or interruption of a series or habit, but does not have a single event reading with this verb (\ref{exInceptiveAchievementLeka}).

\begin{exe}
\ex[*]{\label{exInceptiveAchievementNoMala}\gll a-mal-ile ʊ-kʊ-fik-a\\
1-finish-\textsc{pfv} \textsc{aug}-15-arrive-\textsc{fv}\\
\glt (intended: \lq S/he has arrived completely.')}
\ex[]{\label{exInceptiveAchievementLeka}\gll a-lek-ile ʊ-kʊ-fik-a a-pa\\
1-cease-\textsc{pfv} \textsc{aug}-15-arrive-\textsc{fv} \textsc{aug}-\textsc
{prox.16}\\
\glt \lq S/he no longer gets here.'\\
not: \lq S/he has ceased to arrive here.'}
\end{exe}

To summarize, \textit{fika} differs from transitional and resultative achievements in that it does not lexicalize a Coda state. Like transitional, but unlike resultative achievements, it does, however, encode an Onset phase. Its phasal structure can thus be schematized as in 
\figref{FigureInceptiveAchievement}.
\begin{figure}[h]
\begin{center}
\includegraphics{figures/GrafikInceptiveAchievement.eps}
\caption{Phasal structure of \textit{fika}}
\label{FigureInceptiveAchievement}
\end{center}
\end{figure}

As for \textit{aga} \lq find', in the simple present this verb has a habitual/generic\is{aspect!habitual}\is{aspect!generic} and a \isi{futurate} reading, but no progressive one (\ref{exAcuteAchievementPRS}), which indicates the lack of a lexical Onset phase. Accordingly, the simple present in the complement of the persistive aspect auxiliary does not have a single event reading (\ref{exAcuteAchievementPersPRS}).

\begin{exe}
\ex \label{exAcuteAchievementPRS}
 \gll tʊ-kw-ag-a ɪɪ-fungulo jɪ-lɪ paa-meesa\\
\textsc{1pl}-\textsc{prs}-find-\textsc{fv} \textsc{aug}-key(9)(<SWA) 9-\textsc{cop} 16-table(9)(<SWA)\\
\glt 1. \lq We will find that a/the key is on a/the table (e.g. thus we have 
been informed).'\\
2. \lq We find that a/the key is on a/the table (e.g. each time we search for it).'

\ex \label{exAcuteAchievementPersPRS}\gll tʊ-kaalɪ tʊ-kw-ag-a bi-kʊ-ly-a\\
\textsc{1pl}-\textsc{pers} \textsc{1pl}-\textsc{prs}-find-\textsc{fv} 
2-\textsc{prs}-eat-\textsc{fv}\\
\glt \lq We still find them eating (frequently).'\\
not: \lq We are still finding them eating (sic!).'
\end{exe}

Further proof of the lack of an Onset phase is found in the facts that \textit{aga} is incompatible with the periphrastic progressive (\ref{exAcuteAchievementProg}) and that it does not have a single event reading with the ingressive \textit{anda} (\ref{exAcuteAchievementAnda}):
\begin{exe}
\ex[*]{\label{exAcuteAchievementProg}\gll tʊ-lɪ pa-kw-ag-a bi-kʊ-ly-a\\
\textsc{1pl}-\textsc{cop} 16-15-find-\textsc{fv} 2-\textsc{prs}-eat-\textsc
{fv}\\
\glt (intended: \lq We are about to find them eating.')}

\ex[]{\label{exAcuteAchievementAnda}\gll tw-and-ile ʊ-kw-ag-a bi-kʊ-ly-a\\
\textsc{1pl}-start-\textsc{pfv} \textsc{aug}-15-find-\textsc{fv} 2-\textsc
{prs}-eat-\textsc{fv}\\
\glt \lq We have begun to find them eating (e.g. each time we pass).'\\
not: \lq We have begun to find them eating (right now).'}
\end{exe}

The perfective aspect with \textit{aga }denotes that the eventuality has passed (\ref
{exAcuteAchievementPFV}). The incompatibility of perfective aspect with persistive aspect shows that no Coda phase is encoded (\ref{exAcuteAchievementPersPFV}).

\begin{exe}
\ex[]{\label{exAcuteAchievementPFV} \gll tw-ag-ile bi-kʊ-ly-a\\
\textsc{1pl}-find-\textsc{pfv} 2-\textsc{prs}-eat-\textsc{fv}\\
\glt \lq We have found them eating.'}

\ex[*]{\label{exAcuteAchievementPersPFV}\gll tʊ-kaalɪ tw-ag-ile bi-kʊ-ly-a\\
\textsc{1pl}-\textsc{pers} \textsc{1pl}-find-\textsc{pfv} 2-\textsc{prs}-eat-
\textsc{fv}\\
\glt (intended: \lq We are still informed that they are eating.')}
\end{exe}

The time-span phrase \lq take X time' denotes the time elapsing before the 
change-of-state:
\begin{exe}
\ex \gll tw-eg-ile a-ka-balɪlo a-ka-tali ʊ-kw-ag-a ɪɪ-fungulo jɪ-lɪ paa-meesa
\\
\textsc{1pl}-take-\textsc{pfv} \textsc{aug}-12-time \textsc{aug}-12-long  \textsc{aug}-15-find-\textsc{fv} \textsc{aug}-key(9)(<SWA) 9-\textsc{cop} 16-table(9)(<SWA)\\
\glt \lq We took a long time to find that a/the key is on a/the table.'
\end{exe}

Lastly, \textit{mala} \lq finish' cannot be used with \textit{fika} (\ref{exAcuteAchievementsMala}) and \textit{leka} \lq cease, stop' does not have a single event reading (\ref{exAcuteAchievementsLeka}).
\begin{exe}
\ex[*]{\label{exAcuteAchievementsMala}\gll tʊ-mal-ile ʊ-kw-ag-a ɪɪ-fungulo 
jɪ-lɪ paa-meesa\\
\textsc{1pl}-finish-\textsc{pfv} \textsc{aug}-15-find-\textsc{fv}  \textsc
{aug}-key(9) 9-\textsc{cop} 16-table(9)\\
\glt (intended: \lq We're done finding that a/the key is on the table (sic!).')}
\ex[]{\label{exAcuteAchievementsLeka}\gll tʊ-lek-ile ʊ-kw-ag-a ɪɪ-fungulo jɪ-lɪ paa-meesa\\
\textsc{1pl}-cease-\textsc{pfv} \textsc{aug}-15-find-\textsc{fv} \textsc{aug}-key(9) 9-\textsc{cop} 16-table(9)\\
\glt \lq We no longer find that a/the key is on the table'\\
not: \lq We have ceased to be finding that a/the key is on the table (sic!).'}
\end{exe}

To summarize, \textit{aga} differs from the other achievements in that it encodes neither an Onset nor a Coda phase, but only a punctual change-of-state. That is, it corresponds to the Vendlerian\ia{Vendler, Zeno} definition of achievements; see \sectref{AristotelianAspect} for discussion. Its phasal structure can be schematized as in \figref{FigureAcuteAchievement}.

\begin{figure}[h]
\begin{center}
\includegraphics{figures/GrafikAcuteAchievement.eps}
\caption{Phasal structure of \textit{aga}}
\label{FigureAcuteAchievement}
\end{center}
\end{figure}

\is{aspect!Aristotelian|)}
\is{aspect!perfective|)}\is{aspect!imperfective|)}\is{aspect!progressive|)}\is{aspect!persistive|)}\is{auxiliary|)}\is{phasal verbs|)}\is{simple present|)}\is{phase!Onset phase|)}\is{phase!Nucleus phase|)}\is{phase!Coda phase|)}