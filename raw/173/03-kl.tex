\documentclass[output=paper]{LSP/langsci}
\ChapterDOI{10.5281/zenodo.1228247} 

\author{Marian Klamer\affiliation{Leiden University}\lastand František Kratochvíl\affiliation{Palacký University, Olomouc}
}

\title{The evolution of differential object marking in Alor-Pantar languages}

\abstract{This paper investigates the evolution of Differential Object Marking (DOM) in Abui and Teiwa, two Papuan languages of the Alor-Pantar family in Eastern Indonesia. In both languages, reflexes of the same proto-morpheme are used in the differential marking of P (the non-agentive argument in transitive constructions), but the languages contrast in the way Ps are differentiated. We compare the synchronic DOM patterns of Abui and Teiwa with each other as well as with the DOM patterns we reconstruct for their shared ancestor. We establish how different patterns of DOM in this family have evolved over time, and which semantic and morphological changes occurred in the process. 

In their morphological expression, there are two strategies by which P’s are differentiated: (i) the asymmetrical strategy involves an opposition between P as either a verbal prefix or a free nominal, and (ii) the symmetrical strategy where the choice of a P-prefix is variable depending on the semantics of P. Both strategies are used in both Teiwa and Abui, but the symmetrical strategy involves a choice between two different prefixes in Teiwa and five different prefixes in Abui. 
 
Different factors trigger DOM in both languages: in Teiwa it is mostly based on the inherent properties (animacy) of P, while in Abui there are many other triggers besides the animacy of P, including the affectedness relation between the action and the P referent and the inflectional class of the verb. Furthermore, Abui has developed an extra, third, formal strategy to differentiate human Ps from non-human ones in a serial verb construction.

The alignment system we reconstruct for the proto-language was semantic. It evolved into an accusative alignment system in Teiwa, but was retained and further complexified in Abui. Alignment systems are not static: their forms and triggers may be modified and complexified over time.
}
% \keywords{Timor-AP languages, diachronic change in DOM, DOM, triggers, DOM in verbal agreement}
\maketitle

\begin{document}

\section{Introduction}
\largerpage
\label{03-kl-sec:1}
This paper describes and compares the differential object marking in \ili{Teiwa} \citep{Klamer2010Grammar} and \ili{Abui} (\citealt{Kratochvil2007Grammar,Kratochvil2014Differential,Kratochviletal2015Degrees}), two members of the AP language family of Papuan\footnote{Note that the term ‘Papuan’ is not a genealogical term, but rather refers to a cluster of several dozens of unrelated language families that are spoken on or close to the Papuan mainland, and are not Austronesian.} languages spoken in eastern Indonesia (\figref{03-kl-fig:1}--\ref{03-kl-fig:3}). We show that different members of a language family may show different patterns of Differential Object Marking (DOM) that are triggered by different factors and involve different forms, and that the evolutionary path of DOM has both stable and unstable features.

\begin{figure}
\includegraphics[width=\textwidth]{figures/03-kl-fig1a.pdf}
\caption{The islands of Timor, Alor and Pantar in Indonesia}
\label{03-kl-fig:1}
\end{figure}

\begin{figure}
\includegraphics[width=\textwidth]{figures/03-kl-fig2a.pdf}
\caption{The Papuan languages of Timor (in the areas that are left white, Austronesian languages are spoken)}
\label{03-kl-fig:2}
\end{figure}

\begin{figure}
\includegraphics[width=\textwidth]{figures/03-kl-fig3a.pdf}
\caption{The languages of Alor and Pantar}
\label{03-kl-fig:3}
\end{figure} %TODO [cmld] What about arrow from Fig 2 to Fig 3?

After an introduction to the history and typology of the \ili{Alor-Pantar} (AP) language family (\sectref{03-kl-sec:1}), we present evidence that Proto-AP (the ancestor language of \ili{Teiwa} and \ili{Abui}) treated both transitive objects (P) and \isi{intransitive} subjects (S) in a split fashion, and we list the morphological forms involved in the proto-splits (\sectref{03-kl-sec:2}). In \sectref{03-kl-sec:3}, we describe the formal and semantic characteristics of DOM in \ili{Teiwa}, pointing out the elements of the proto-DOM system that have been retained, changed and lost in \ili{Teiwa}. In \sectref{03-kl-sec:4}, we similarly describe DOM in \ili{Abui} and compare it to the proto-system.

By studying patterns of DOM in these two related languages and comparing them with their shared ancestor, we can establish how different patterns of DOM evolve over time, and which semantic and morphological changes occur in the process. For the descriptive data presented in this paper, we build on our own publications on \ili{Teiwa} and \ili{Abui}, as well as unpublished fieldwork data included in the respective corpora of \ili{Teiwa} and \ili{Abui}.\footnote{These corpora are available as part of the Laiseang corpus in The Language Archive (TLA) at the Max Planck Institute for Psycholinguistics in Nijmegen \url{http://tla.mpi.nl.}} For the typological component of the paper, we have used information on argument encoding in the AP languages that has been published elsewhere (\eg \citealt{Klamer2010Ditransitive,
Klamer2010One},\citeyear*{Klamer2017Alor-Pantar};
\citeauthor{Kratochvil2011Transitivity} \citeyear*{Kratochvil2011Transitivity},
\citeyear*{Kratochvil2014Differential};
\citealt{Klameretal2012Role,
Klameretal2012Give,
Feddenetal2013Feddenetal,
Feddenetal2014Variation},
\citeauthor{Kratochviletal2015Definiteness} \citeyear*{Kratochviletal2015Definiteness,Kratochviletal2015Degrees}). 
For the historical reconstruction of the DOM system in Proto-AP, we draw on published historical reconstruction work on the AP family \citep{Holtonetal2012Historical,Holtonetal2017Linguistic,Holtonetal2017Internal,}.

\section{Introduction to the history and typology of Alor-Pantar languages}
\label{03-kl-sec:2}

Together with the Papuan languages spoken on the neighbouring island of Timor, the AP sub-family constitute the larger Timor-\ili{Alor-Pantar} family counting about 30 languages (\figref{03-kl-fig:2}--\ref{03-kl-fig:4})
\citep{Holtonetal2012Historical,Holtonetal2017Linguistic,Holtonetal2017Internal,Robinsonetal2014Reconstructing,Schapper2014Kamang,Schapperetal2017Relatedness}. 
An indication of the position of \ili{Teiwa} and \ili{Abui} in the Timor-\ili{Alor-Pantar} family tree is shown in \figref{03-kl-fig:4}. 
Based on phonological innovations \citep{Holtonetal2012Historical}, we assert that \ili{Teiwa} and \ili{Abui} share a common ancestor, Proto-AP, but are not direct sister languages, as it is possible to construct an intermediate node (labelled \ili{Proto-Alor} in \figref{03-kl-fig:4}) between \ili{Teiwa} and \ili{Abui}.

\begin{figure}[h]
	\caption{The position of Teiwa and Abui in the Timor-Alor-Pantar family tree (derived from \citealt[114]{Holtonetal2012Historical}, Fig.\,2).}\label{03-kl-fig:4} 
\begin{forest} for tree={fit=band}
[Proto-Timor-\ili{Alor-Pantar}
 [Proto-\ili{Alor-Pantar}
 [\ili{Teiwa},tier=word]
 [\ldots,tier=word] [\ldots,tier=word] [\ldots,tier=word]
 [\ili{Proto-Alor} [\ldots~\ili{Abui}~\ldots,roof,tier=word]]
 ] 
 [Proto-Timor]
]
\end{forest}	
\end{figure}
	
Basic (pragmatically unmarked, declarative) transitive clauses in the AP languages are verb-final, and Agent-Patient-Verb (APV) and Subject-Verb (SV) is the basic constituent order attested in all the modern languages.\footnote{The notions A, S and P are used here as comparative concepts, where A is the most Agent-like argument of a transitive clause and P the least Agent-like, while S is the single argument of an intransitive verb \citep{Comrie1989Language}.} Objects in AP languages are expressed with free nominal constituents (NPs or pronouns), which exist alongside verbal affixes that index person and number of verbal arguments. The AP languages are all head-marking and show a preponderance to index P over S/A \citep[20]{Klamer2017Alor-Pantar}. This pattern is typologically extremely rare, occurring in only 7\% of the 378 languages surveyed by \citep{Siewierska2013Verbal}, yet it is universally found in the AP family. In other words, in AP, a person-number prefix on a verb typically indexes the object (P), while subjects (S/A) may also be indexed but are more typically expressed as free forms (pronouns or NPs).

Differential Object Marking (DOM) is seen here as `the non-uniform grammatical marking of objects which occurs within one and the same language, with objects of one and the same verb' \citep[1]{Dalrympleetal2011Objects}. The grammatical marking of objects in AP languages involves differential patterns of object indexing on verbs \citep{Iemmolo2011Towards}, and in this respect is crucially different from differential marking of arguments by \isi{case marking} on the noun phrase. In the AP family, nouns are never marked for case, and alignment is always defined relative to the pronominal indexing of the verb.

Other crucial differences between the AP languages and the well-known European languages include the following. First, AP languages have few, if any, tri-valent (\isi{ditransitive}) verbs. Instead of having a predicate with three arguments, two of which are object-like, the languages use a strategy where serial verb constructions express events which involve more than two participants. Second, the object (P) of a bi-valent verb in AP languages can express a multitude of semantic roles: a P may be a semantic patient, recipient, goal, benefactive, or source. This is illustrated for \ili{Teiwa} in \REF{03-kl-ex:1},\footnote{Orthographic conventions used in this article: \textit{x} = /ħ/, \textit{q} = /q/, \textit{’} =/Ɂ/, and a double vowel symbol stands for a long vowel.} where P
is a patient; in \REF{03-kl-ex:2}, where the P of bi-valent -\textit{an} is a
recipient, in \REF{03-kl-ex:3}, where the P of -\textit{mian} ‘put at’ is a goal;
in \REF{03-kl-ex:4}, where the P of -\textit{lal} ‘show’ is a benefactive, and in
\REF{03-kl-ex:5}, where the P of -\textit{umbangan} ‘ask (something) from someone’
is a source. Similar observations can be made for \ili{Abui}, see \REF{03-kl-ex:17}--\REF{03-kl-ex:21} below.

\ea \langinfo{Teiwa}{}{\citealt[114, 169, 334–335]{Klamer2010Grammar}, fieldnotes, TSS: 001}\\
\ea\label{03-kl-ex:1}
\gll Sematar na \textbf{h-ua’}.\\
in.a.moment(\textsc{ind}) 1\textsc{sg} 2\textsc{sg}-hit\\
\glt ‘I’ll hit you!’

\ex\label{03-kl-ex:2}
\gll Uy ga’an u sen ma n-oma’ \textbf{g-an}.\\
person \textsc{dem} \textsc{dist} money come 1\textsc{sg}.\textsc{poss}-father 3\textsc{sg}-give\\
\glt ‘That person gives my father money.’

\ex
\label{03-kl-ex:3}
\gll Jadi hala biar kriman la pin aria’ ma \textbf{ni-mian}\ldots\\
so others children small \textsc{foc} hold arrive come 1\textsc{pl}-put.at\\
\glt ‘So other people brought some small children here and gave them to us\ldots ’

\ex
\label{03-kl-ex:4}
\gll Yitar ga-qau ma \textbf{na-lal-an}.\\
road 3\textsc{sg}.\textsc{poss}-good come 1\textsc{sg}-show-\textsc{real}\\
\glt ‘[You] show me the right way.’

\ex
\label{03-kl-ex:5}
\gll A daa \textbf{n-um-bangan}.\\
3\textsc{sg} ascend 1\textsc{sg}-\textsc{appl}-ask.for\\
\glt ‘He comes up to ask [sth.] from me’ or ‘He comes up to ask me [for/about sth.].’ 
\z
\z

Note that in (\ref{03-kl-ex:2}), (\ref{03-kl-ex:3}) and (\ref{03-kl-ex:4}) the theme participants (\textit{sen} ‘money’, \textit{biar kriman} ‘small children’, \textit{yitar gaqau}
‘right way’) are introduced with a separate verb (\textit{ma}
‘come’).\footnote{This function of \ili{Teiwa} \textit{ma} is further
 described in \citet{Klamer2010Grammar,Klamer2010Ditransitive}.} This verb occurs in a serial verb construction with a second verb in clause final
position.\footnote{Example (\ref{03-kl-ex:3}) involves another serial verb
 (\textit{pin aria} 'arrive holding something’). We will not discuss
 serialization in \ili{Teiwa} or \ili{Abui} here; see the respective grammars for
 further information.} The second verb carries the P-prefix. Homologous affixes combine with nouns to index possessors: examples include \textit{n-oma} ‘1\textsc{sg}.\textsc{poss}-father’ in (\ref{03-kl-ex:2}) and
\textit{ga-qau} ‘3\textsc{sg}.\textsc{poss}-good’ in (\ref{03-kl-ex:4}).


\section{Differential object marking in Proto-Alor-Pantar} \label{03-kl-sec:3}

Pronouns and pronominal indexes are known to belong to the most stable and archaic part of the lexicon (\citealt{Filimonova2005Noun,Heineetal2011Genesis,Heineetal2011Grammaticalization}).
 Given their stability, pronouns have been used to
suggest deep genetic relationships \citep{Nicholsetal2013M}. The
morpho-syntactic patterns attested in the modern AP languages
regularly involve morphemes reflecting forms that are reconstructable
up to the ancestor language of the family, Proto-AP.

\begin{table}[b]
\caption{Reconstructed forms for A, P, and Possessor in Proto-Alor-Pantar}
\begin{tabularx}{\textwidth}{ Q Q Q Q }
\lsptoprule & \textbf{A free pronoun} & \textbf{P prefix} & \textbf{Possessor prefix}\\\midrule
1\textsc{sg} & \textit{*na(N)}\footnote{\textit{N} represents a nasal unspecified for place.} & \textit{*na-}\\
2\textsc{sg} & \textit{*a(N)} & \textit{*(h)a-}\\
3 & \textit{*ga(N)} & \textit{*ga-}\footnote{\citet{Holtonetal2017Internal} reconstruct two separate third person prefixes, of which the singular is \textit{*ga-} and the plural \textit{*gi-.}} & \textit{*ge-}\footnote{Proto-AP may also have had possessor prefixes for other persons but only the third person form is reconstructed so far. Possible reconstructed forms would be \textit{*ne-} `\textsc{1sg}', \textit{*(h)e-} `\textsc{2sg}', \textit{*te-} `\textsc{Distr}'. In the plural, the vowel distinction was likely neutralized.}\\
\textsc{distr} & &
\textit{*ta-}\\
\textsc{1pl.inc} & \textit{*pi(N)} & \textit{*pi-}\\
\textsc{1pl.exc} & \textit{*ni(N)} & \textit{*ni-}\\
2\textsc{pl} & \textit{*i(N)} & \textit{*(h)i-}\footnote{\citet{Robinsonetal2014Reconstructing} do not reconstruct the initial consonant of this prefix as optional, because of the regular reflex of Proto-AP *h in Western Pantar and Sar.}\\
\lspbottomrule
\end{tabularx}
\label{03-kl-tab:1}
\end{table}

\largerpage[-2]
\tabref{03-kl-tab:1} lists the reconstructed pronoun forms (\citealt{Holtonetal2012Historical}; \citealt{Robinsonetal2014Reconstructing}; \citealt[170]{Holtonetal2017Internal}). In AP pronouns, initial consonants encode person features, while theme vowels encode number features (/a/ singular, /i/ plural) and possession (/e/).\footnote{Proto-AP \textit{*ta}- is grouped
 with the singular forms in \tabref{03-kl-tab:1} because it carries the singular theme vowel /a/. \textit{*ta} has a common or impersonal referent (cf. \textit{one} in English '\textit{One} should consider this'), and its reading is often distributive or reflexive (`each one', `each other').}

 
In addition to reconstructing the form of the Proto-AP prefixes we can also reconstruct some of the Proto-AP bi-valent verbs as bound forms, and others as unbound. We reconstruct a verb as bound when that has a P-prefix in daughter languages across the family, while a verb is reconstructed as unbound when all its modern reflexes lack a P-prefix. The reconstructed verbs are given in \tabref{03-kl-tab:2}. 

\begin{table}[h]
\caption{Reconstructed bi-valent verbs in Proto-Alor-Pantar \citep{Holtonetal2012Historical,Holtonetal2017Internal,Schapperetal2017Relatedness,Klamerforthcoming}.}\label{03-kl-tab:2}
\begin{tabularx}{\textwidth}{ Q l Q l}
\lsptoprule
\multicolumn{2}{c}{\textbf{With P-prefix}} & \multicolumn{2}{c}{\textbf{Without P-prefix}}\\
\textbf{Proto-AP verb} & \textbf{Meaning} & \textbf{Proto-AP verb} & \textbf{Meaning}\\\midrule
\textit{*-ten} & wake up someone & \textit{*tapai} & pound, pierce\\
\textit{*-wel} & bathe someone & \textit{*mi} & be in, be at\\
\textit{*-ena} & give to someone & \textit{*magi} & hear\\
*-\textit{asi} & bite someone (of dogs) & \textit{*(ta)ki} & bite (food?)\\ 
& & \textit{*nai} & eat\\ 
& & \textit{*med} & take\\ 
& & \textit{*kabar} & scratch\\ 
& & \textit{*tiari(n)} & close\footnote{\citet[75]{Holtonetal2017Internal} reconstruct ‘close’ with a prefix. We find no evidence for this in a larger dataset.}\\
\lspbottomrule
\end{tabularx}
\end{table}

In other words, Proto-AP encoded its Ps in a split fashion: certain verbs indexed P using a pronominal prefix, other verbs used (only) a free form to express P. Even with the limited evidence these verbs provide us with, it is already possible to see that this split in P-marking probably had a semantic motivation. For the reconstructed verbs with a P-prefix, the prefix likely indexed a human/\isi{animate} referent, as waking up and bathing someone are activities applied to a human object. Also, across the AP family, the (single) object of the verb `give' is always a human referent (the P-prefix always indexes a recipient), while the theme (=the thing given) is encoded as either a separate oblique constituent or with its own predicate, using a serialization strategy \citep{Klameretal2012Give}.

\largerpage[-2]
In contrast, the verbs that are reconstructed without a P-prefix such as `be in, be at', `eat', and `take' seem to typically have an \isi{inanimate} P. The object of the verb `scratch' is typically a surface (which may or may not be a human skin). The verb `pound' typically refers to pounding food objects (\eg rice or corn). The two verbs for `bite' may have been split in use depending on the \isi{animacy} of the object. And in the AP languages, the verb `hear' does not typically take a personal object (as in \textit{I heard your father sing}) but rather a sound or a sound-producing event (\eg \textit{Your father's singing, I heard it}). In sum, \ili{Proto-Alor} Pantar had a split in the marking of P, and this split was probably motivated by the distinction between human/\isi{animate} objects (which were indexed with a verbal prefix) versus \isi{inanimate} objects (which were expressed as free constituents). The fact that the feature `human/\isi{animacy}' triggers the indexing of Ps is cross-linguistically not unusual: agreement is often sensitive to the discourse salience of arguments, and since humans/animates have more \isi{discourse prominence} than inanimates they are more eligible to be indexed on verbs (\cf \citealt{Dalrympleetal2011Objects}).

In addition to a split P-marking, the proto-language may also have had a split in the marking of \isi{intransitive} subjects (S) that was based on semantics (\citealt{Klamer2012Dressing,Robinsonetal2014Reconstructing}); a system
referred to in the literature as ‘\isi{semantic alignment}’ \citep{Mithun1991Active,Donohueetal2008Typology}, in contrast to `\isi{accusative alignment}' or `ergative alignment'. Languages with \isi{accusative alignment} treat S and A alike, as opposed to P; languages with \isi{semantic alignment} encode S sometimes like P (by prefixing it to the verb, as in the AP languages), and sometimes like A (eg by expressing it as a free pronoun, as in the AP languages). The variable encoding of S is motivated by the semantics of the verb and its argument, but the lexical sub-categorisation characteristics of verbs also play a role \citep[cf.][]{Feddenetal2013Feddenetal,Feddenetal2014Variation}.

The hypothesis that Proto-AP had \isi{semantic alignment} is based on the following observations.\footnote{To reconstruct the \isi{alignment system} of \ili{Proto-Alor} Pantar with confidence, comparative data from cognate sets of a sizable number of verbs across a wide range of Alor Pantar languages need to be collected and their alignment patterns compared, work that yet needs to be done.} First, AP languages with \isi{semantic alignment} are found across the region, while languages with \isi{accusative alignment} are confined to a region in the centre, as shown in \figref{03-kl-fig:5}. This geographical spread suggests that \isi{semantic alignment} was the original pattern from which the accusatively aligning languages diverged. 

\begin{figure}
\includegraphics[width=\textwidth]{figures/03-kl-fig5.pdf}
\caption{Semantic (green) and accusative (red) alignment in Alor-Pantar languages. (For the language areas left white, information on alignment is lacking).}
\label{03-kl-fig:5}
\end{figure}

Second, some languages that today have \isi{accusative alignment} show morphological traces of \isi{semantic alignment}. An example is Kaera (Pantar), which encodes the S of certain \isi{intransitive} verbs with a prefix otherwise typically used to index P arguments \citep[135--136]{Klamer2014Kaera}. This Kaera class of verbs includes verbs such as `live', `be silent', `jump up', `faint, be unconscious', `think', `give birth'.\footnote{Although the coverage of our comparative database is currently insufficient to determine whether the Kaera forms are regularly inherited from the Proto-AP lexicon, verbs with the similar senses regularly either allow or require S-indexing in semantically aligned languages such as Western Pantar \citep{Holton2014Western}, 
 \ili{Klon }\citep{Baird2008Grammar}, \ili{Abui} \citep{Kratochvil2007Grammar,Kratochvil2011Transitivity}, \ili{Kamang}
 \citep{Schapper2014Kamang}, \ili{Sawila} \citep{Kratochvil2014Sawila}, and \ili{Wersing} \citep{Schapperetal2017Relatedness}.} The presence of such morphological fossils suggests that there may have been an earlier historical stage with \isi{semantic alignment} from which modern Kaera with \isi{accusative alignment} has developed.

Third, some languages that are accusatively aligning today are still attuned to semantic factors in the alignment of P. Examples are \ili{Adang} \citep{Haan2001Grammar,Robinsonetal2014Adang} or \ili{Blagar}  \citep{Steinhauer2014Blagar}. This sensitivity to semantics in an otherwise accusative \isi{alignment system} suggests that the language developed from an earlier language with \isi{semantic alignment}.\footnote{In \ili{Adang}, objects are either indexed by prefixes on the verb or expressed by free object pronouns. There is a tendency for verbs with \isi{animate} objects to be prefixing \citep{Feddenetal2013Feddenetal}. In \ili{Blagar}, various degrees of \isi{affectedness} can be
 distinguished using object pronoun, possessive pronouns, or a prefix
 \citep[167, 189]{Steinhauer2014Blagar}.}

If \ili{Proto-Alor} Pantar indeed had \isi{semantic alignment}, then it must have expressed \isi{intransitive} S sometimes like A, using a free form, and sometimes like P, using a verbal prefix (compare \tabref{03-kl-tab:2}). Some examples of reconstructed mono-valent verbs in Proto-AP are presented
in \tabref{03-kl-tab:3}.\footnote{\citet[75]{Holtonetal2017Internal} reconstruct ‘close’ with a prefix. We find no evidence for this in a larger dataset.}

\begin{table}[t]
\caption{Reconstructed mono-valent verbs in Proto-Alor-Pantar \citep{Holtonetal2012Historical,Holtonetal2017Internal,Schapperetal2017Relatedness,Klamerforthcoming}.}
\begin{tabularx}{.5\textwidth}{ Q Q Q }
\lsptoprule

\textbf{Proto-AP verb} & \textbf{Meaning} \\\midrule
\textit{*tas} & stand \\
\textit{*tia}& sleep\\
\textit{*purVn}& spit\\
\textit{*jagir}& laugh\\
\textit{*luk(V)} & crouch\\
\textit{*mai} & come (here)\\
\textit{*kabar}& scratch\\
\textit{*tiari(n)} & close\\

\lspbottomrule
\end{tabularx}
\label{03-kl-tab:3}
\end{table}

We have not, or not yet, been able to reconstruct bound mono-valent verbs, \ie verbs that encode their S argument with a prefix in their modern reflexes across the AP family. The evidence for the \isi{semantic alignment} of Proto-AP is thus circumstantial. 

To summarize, the following grammatical information about Proto-AP, the ancestor language of \ili{Teiwa} and \ili{Abui} has been presented:

\begin{enumerate}
\item The reconstructed pronouns include free and bound forms that are formally clearly related (\cf \tabref{03-kl-tab:1}). 
\item In Proto-AP, free pronouns express A while bound pronouns typically express P and Possessor. 
\item Proto-AP has some kind of DOM, as Ps are expressed in a split fashion: some bi-valent verbs take a P-prefix, other bi-valent verbs express P with a free form. 
\item The P-split is likely based on the distinction between human/\isi{animate} and \isi{inanimate} referents, where human/\isi{animate} Ps are indexed on the verb and \isi{inanimate} Ps are not. 
\item Proto-AP likely has \isi{semantic alignment}, encoding the S of certain \isi{intransitive} verbs with a prefix otherwise typically used to index P arguments. However, so far we have only been able to reconstruct mono-valent verbs with a free-standing S.
\end{enumerate}

\section{Differential object marking in Teiwa} \label{03-kl-sec:4}

In \ili{Teiwa}, some of the Proto-AP properties listed above were retained, while others were lost. \ili{Teiwa} retained both the proto-prefix for P (and some S) and the free proto-pronoun that encoded A (and some S). The full set of \ili{Teiwa} pronouns and person prefixes encoding A, P, S, and the possessor is given in \tabref{03-kl-tab:4}. (Using a long rather than a short free pronoun encodes \isi{contrastive focus} of A and S in \ili{Teiwa}.) As in Proto-AP, free pronouns express A while bound pronouns typically express P and Possessor. Unlike Proto-AP, \ili{Teiwa} has no \isi{semantic alignment} where S can be marked like P: \ili{Teiwa} is completely accusative.

\begin{table}[t]
\caption{Teiwa pronouns (S, A, P) and prefixes (P and possessor) \citep[77--78]{Klamer2010Grammar}}
\begin{tabularx}{\textwidth}{ l Q Q Q Q Q }
\lsptoprule
& A, S long pronoun & A, S short pronoun & P free pronoun & P prefix & Possessor prefix\footnote{Possessors can also be expressed with short and long forms of free pronouns, see \citep[79]{Klamer2010Grammar}. \ili{Teiwa} possessive prefixes contain the theme vowel /a/ just like the prefixes that index P. Alienable and inalienable possession are distinguished by the optional versus obligatory use of the possessive prefix \textit{na-yaf} ‘1\textsc{sg}.\textsc{poss}-house’ ‘my house(s)’ \vs \textit{yaf} ‘a house, house(s)’; \textit{na-tan} ‘1\textsc{sg}.\textsc{poss}-hand’ ‘my hand(s)’ \vs \textit{*-tan} (intended reading ‘a hand, hand(s)’).}

\\\midrule
1\textsc{sg} & \textit{na’an} & \textit{na} & \textit{na’an} & \textit{n(a)-} & \textit{n(a)-}\\
2\textsc{sg} & \textit{ha’an} & \textit{ha} & \textit{ha’an} & \textit{h(a)-} & \textit{h(a)-}\\
3\textsc{sg} & \textit{a’an} & \textit{a} & \textit{ga’an} & \textit{g(a)-, gə-} & \textit{g(a)-, a-}\\
\textsc{distr} & \textit{ta’an} & \textit{ta} & \textit{ta’an} & \textit{t(a)-} & \textit{t(a)-}\\
1\textsc{pl}.\textsc{inc} & \textit{ni’in} & \textit{ni} & \textit{ni’in} & \textit{n(i)-} & \textit{n(i)-}\\
1\textsc{pl}.\textsc{exc} & \textit{pi’in} & \textit{pi} & \textit{pi’in} & \textit{p(i)-} & \textit{p(i)-}\\
2\textsc{pl} & \textit{yi’in} & \textit{yi} & \textit{yi’in} & \textit{y(i)-} & \textit{y(i)-}\\
3\textsc{pl} & \textit{iman, i’in} & \textit{i, a} & \textit{iman, gi’in} & \textit{g(i)-, ga-} & \textit{g(i)-, a-, ga-}\\
\lspbottomrule
\end{tabularx}
\label{03-kl-tab:4}
\end{table}

As in Proto-AP, some bi-valent verbs in \ili{Teiwa} take a P-prefix, while other such verbs express P with a free form. \ili{Teiwa} bi-valent verbs typically use a prefix to index an \isi{animate} P, while a free form (pronoun or NP) expresses an \isi{inanimate} P. This is illustrated in \REF{03-kl-ex:6}. In \REF{03-kl-ex:6a},\footnote{Compare \textit{Xa’a ma} \textbf{\textit{na-}}\textit{mai} ‘this come 1\textsc{sg}-keep.for’ ‘Keep this for me.’ [constructed example].} the object of \textit{mai} is \textit{ha-gas qai} ‘your younger sister’, an \isi{animate} referent that is indexed on the verb. In \REF{03-kl-ex:6b} the verbs \textit{mai} ‘keep’ and \textit{usan} ‘lift’ share a single object \textit{aga’} ‘all [of it]’, which is not indexed on the verb because the referent is \isi{inanimate}.

\protectedex{
\ea \label{03-kl-ex:6}
\langinfo{Teiwa}{}{Klamer fieldnotes TAS:0055; TAS2012:001}
\ea \label{03-kl-ex:6a}
\gll Xa’a ma {ha-gas qai} ga-mai.\\
this come 2\textsc{sg}.\textsc{poss}-younger.sister 3\textsc{sg}-keep.for\\
\glt ‘Keep this for your younger sister.’

\ex\label{03-kl-ex:6b}
\gll Aga’ usan kamar gom ma mai.\\
all lift room(\textsc{ind}) inside come keep\\
\glt ‘Pick up all (of it) and keep (it) inside the room.'
\z
\z
}

Another example of an \isi{animate} P that is indexed on the verb is given
in \REF{03-kl-ex:7a}. It contrasts with the P in \REF{03-kl-ex:7b}, which is \isi{inanimate} and not indexed. A similar contrast is shown in (\ref{03-kl-ex:8}), but here the free form is a pronoun rather than a lexical constituent.

\ea \label{03-kl-ex:7}
\langinfo{Teiwa}{}{Klamer fieldnotes TAS2011:138; TPV2011\_2:016}
\ea
\label{03-kl-ex:7a}
\gll Bif g-oqai sen ma ga-mian.\\
child 3\textsc{sg}.\textsc{poss}-child money come 3\textsc{sg}-put.at\\
\glt ‘His child gave him money.’

\ex
\label{03-kl-ex:7b}
\gll In qap ii’ kalax gom mian.\\
thing cut red basket inside put.at\\
\glt ‘A red cloth is put inside a basket.’
\z
\z

\protectedex{
\ea\label{03-kl-ex:8}
\langinfo{Teiwa}{}{\citealt[91]{Klamer2010Grammar}}
\ea
\label{03-kl-ex:8a}
\gll Na ga-mar.\\
1\textsc{sg} 3\textsc{sg}-take\\
\glt ‘I follow him/her.’

\ex
\label{03-kl-ex:8b}
\gll Na ga’an mar.\\
1\textsc{sg} 3\textsc{sg} take\\
\glt ‘I take it.’
\z
\z
}

Some additional illustrations of \ili{Teiwa} verbs that show DOM based on
\isi{animacy} are given in \REF{03-kl-ex:9}. These verbs are attested with both an
\isi{animate} and \isi{inanimate} object in the \ili{Teiwa} corpus.

\ea
 Illustrations of Teiwa transitive verbs showing DOM\\ \label{03-kl-ex:9}
\begin{tabularx}{115mm}{ Q l Q l }
\multicolumn{2}{l}{With P-prefix} & \multicolumn{2}{l}{Without P-prefix}\\
\textit{ga-mar} & ‘follow someone’ & \textit{mar} & ‘take (something)’\\
\textit{ga-sii} & ‘bite someone’ & \textit{sii} & ‘bite (into) (something)’\\
\textit{ga-dee} & ‘burn someone’ & \textit{dee} & ‘burn (something)’\\
\textit{ga-sar} & ‘notice, find someone’ & \textit{sar} & ‘notice, find (something)’\\
\end{tabularx}
\z

However, DOM in \ili{Teiwa} is not completely predictable and regular, as there are also some verbs that index Ps which are not \isi{animate}. First, the \ili{Teiwa} corpus contains some examples of verbs whose prefix optionally indexes an \isi{animate} or an \isi{inanimate} referent. An example is \textit{uyan} ‘search for’ in \REF{03-kl-ex:10}. Both \REF{03-kl-ex:10a} and \REF{03-kl-ex:10b} are grammatical, but in \REF{03-kl-ex:10b} the indexed P has an \isi{inanimate} referent (\textit{wat} ‘coconut(s)’). In examples \REF{03-kl-ex:11}–\REF{03-kl-ex:13} the prefix on other verbs from the same class indexes \isi{inanimate} referents: a
tree, a coconut, and a spoon.
 
 \largerpage
 
\ea \label{03-kl-ex:10}
\langinfo{Teiwa}{}{Klamer, fieldnotes TAS:0628, TC:025a, TTR2010:024; \citealt[307]{Klamer2010Grammar}}
\ea
\label{03-kl-ex:10a}
\gll Na n-ogai ga-uyan.\\
1\textsc{sg} 1\textsc{sg}.\textsc{poss}-child 3\textsc{sg}-search\\
\glt ‘I’m looking for my child.’

\ex \label{03-kl-ex:10b}
\gll Na wat ga-uyan.\\
1\textsc{sg} coconut 3\textsc{sg}-search\\
\glt ‘I’m looking for coconut(s).’

\ex \label{03-kl-ex:11}
\gll Burilak ga’an ma Sibari \textit{heer} \textit{nuk} \textit{ga-sar.}\\
clan.name 3\textsc{sg} come k.o.tree stem one 3\textsc{sg}-notice\\
\glt ‘The Burilaks noticed a Sibari tree’

\ex \label{03-kl-ex:12}
\gll \ldots uy quaf eran ta om qalixil ta\footnotemark {} a-fat mat ma, wat u ga-tane’ si\ldots \\
person grandmother that \textsc{top} inside angry \textsc{top} 3\textsc{sg}-foot take come coconut \textsc{dist} 3\textsc{sg}-kick \textsc{sim}\\
\footnotetext{\textit{Ta} marks switched topics, but here it functions as a clause-linking device. Its interclausal function may be characterized as marking the discontinuity or asymmetry of events in discourse \citep[Sec.\,11.4]{Klamer2010Grammar}.}
\clearpage
\glt ‘\ldots that grandmother was angry and with (lit.\,taking) her foot kicked that coconut, then\ldots’

\ex \label{03-kl-ex:13}
\gll Sii’ ga’an in qap \textit{ga-tiri} \textit{ba} \textit{ga-wa’} \textit{la} \textit{a’an} \textit{dagar.}\\
spoon \textsc{dem} thing cut 3\textsc{sg}-float \textsc{seq} 3\textsc{sg}.\textsc{poss}-leaf \textsc{foc} 3\textsc{sg} be.visible\\
\glt ‘That spoon is covered by a cloth so that [only] its round part is visible.’\footnote{A more literal translation of this sentence is ‘That spoon, a cut thing floats on [it] so that only its leaf is visible’.}
\z
\z

Second, there is a set of verbs that take alternating prefixes to index animates and inanimates: the ‘normal’ prefix \textit{ga-} encodes \isi{inanimate} Ps, while an ‘augmented’ prefix \textit{ga’-} (pronounced as [gaʔ]) encodes \isi{animate} Ps. 
Illustrations are given in \REF{03-kl-ex:14}. To distinguish \isi{animate} and \isi{inanimate} objects by choosing a different prefix seems to be a minority pattern in \ili{Teiwa}, attested at least for the verbs listed in \REF{03-kl-ex:15}.

\ea \label{03-kl-ex:14}
\ea \label{03-kl-ex:14a}
\langinfo{Teiwa}{}{\citealt[92]{Klamer2010Grammar}}\\
\gll Na gi ga’-tad.\\
1\textsc{sg} go 3\textsc{anim}-strike\\
\glt ‘I go hit him/her.’

\ex
\label{03-kl-ex:14b}
\gll Na gi ga-tad.\\
1\textsc{sg} go 3\textsc{sg}-strike\\
\glt ‘I go hit it.’
\z
\z

\ea \label{03-kl-ex:15}
\ili{Teiwa} transitive verbs with alternating prefixes \citep[91--92]{Klamer2010Grammar}\\
\begin{tabularx}{\textwidth}{ l Q l Q }
\multicolumn{2}{l}{With \textit{ga’}-prefix} & \multicolumn{2}{l}{With \textit{ga}-prefix}\\
\textit{ga’-wulul} & ‘talk with s.o., tell s.o.’ & \textit{ga-wulul} & ‘talk about sth., tell sth.’\\ 
\textit{ga’-wultag} & ‘talk to/about s.o., tell s.o.’ & \textit{ga-wultag} & ‘talk about sth.’\\
\textit{ga’-tewar} & ‘go/walk together with s.o.’ & \textit{ga-tewar} & ‘his (manner of) \\
		& 	 & \textit{ga-tewar} &	 walking’\\

\textit{ga’-tad} & ‘hit, strike, touch s.o.’ & \textit{ga-tad} & ‘hit, strike at sth.’\\
\end{tabularx}
\z

Note that \isi{definiteness} does not play a role in the distinction as both \isi{definite} and \isi{indefinite} Ps can be indexed. An example of a \isi{definite} Ps that is indexed is \textit{wat u} ‘that coconut’ in \REF{03-kl-ex:12}, while \textit{wat} ‘coconut(s)’ in \REF{03-kl-ex:10b} is an indexed \isi{indefinite} P.

The distinction between free and bound pronouns (person prefixes) is not uniquely reserved for marking the \isi{animacy} of a referent but is also used to encode contrastive or identificational focus in \ili{Teiwa}.\footnote{New information focus (\citealt{Lambrecht1994Information}; \citealt[47--48]{Dalrympleetal2011Objects}) is marked in \ili{Teiwa} with a dedicated focus particle \textit{la} and is not further discussed here, see \citet[Ch.\,11]{Klamer2010Grammar}.} This is illustrated in \REF{03-kl-ex:16}, where the \isi{animate} P is indexed on the verb with a prefix in \REF{03-kl-ex:16a}, but is expressed as a free form in \REF{03-kl-ex:16b}, where it encodes a focused constituent.

\newpage 
\ea \label{03-kl-ex:16}
\langinfo{Teiwa}{}{\citealt[407]{Klamer2010Grammar}}
\ea 
\label{03-kl-ex:16a}
\gll Miaag yivar ga-sii.\\
yesterday dog 3\textsc{sg}-bite\\
\glt ‘Yesterday a dog bit him.’

\ex
\label{03-kl-ex:16b}
\gll Miaag yivar ga’an sii.\\
yesterday dog 3\textsc{sg} bite\\
\glt ‘Yesterday a dog bit HIM (not me).’
\z
\z

In sum, the Proto-AP split marking of P plus its \isi{semantic alignment} system developed into an accusative system with DOM in \ili{Teiwa}. The distribution of the person prefix paradigms is lexicalized (normal \vs ‘augmented’). The person prefix that was used for human/\isi{animate} Ps (and some S) in Proto-AP is used in \ili{Teiwa} to index mostly \isi{animate} Ps. A small class of verbs lexicalized the prefix, and indexes both \isi{animate} and \isi{inanimate} Ps. The original free pronouns that were used to express A (and some S) in Proto-AP function in modern \ili{Teiwa} to express both A and S (in an accusative system), and also as a marker of \isi{contrastive focus} of P.

\begin{figure}
	\caption{The historical relation between forms encoding P in Proto-Alor-Pantar and in Teiwa}\label{03-kl-fig:6}
 \begin{tikzpicture}[every node/.style={anchor=base west}]
	\matrix (proto) [anchor=base,draw=lsMidDarkBlue,matrix of nodes,nodes in empty cells=true,column sep=2cm]{
		& \bfseries Proto-AP & \\
	\textsc{a}, \textsc{s} \color{lsMidDarkBlue}{\textit{*ga(N)}} & \textsc{p}, \textsc{s} \color{lsMidDarkBlue}{\textit{*ga-}} & \textsc{poss} \color{lsMidDarkBlue}{\textit{*ge-}}\\
	};
	\matrix (p) [draw=lsMidDarkBlue,matrix of nodes,nodes in empty cells=true,below=5\baselineskip of proto.center]{
	\scshape p & \\
	\color{lsRed}{Focus:} & \itshape \color{lsMidDarkBlue}{gaʔan}\\
	\color{lsRed}{Animate:} & \itshape \color{lsMidDarkBlue}{ga-, gaʔ-}\\
	\color{lsRed}{Inanimate:} & \textsc{np}, \itshape \color{lsMidDarkBlue}{ga-}\\		
	};
	\matrix (poss) [right=of p,draw,matrix of nodes] {
	\textsc{poss}\\
	\textit{\color{lsMidDarkBlue}{ga-}}\\
	};

\draw[dashed,{-Triangle[]}] (proto-2-1) -- (p-2-1.west);
\draw[dashed,{-Triangle[]}] (proto-2-2) -- (p-2-2);
\draw[dashed,{-Triangle[]}] (proto-2-2) -- (poss);
\end{tikzpicture} 
\end{figure}
 
 \largerpage 
\section{Differential object marking in Abui} 
\label{03-kl-sec:5}

Reflexes of the Proto-AP pronouns in \tabref{03-kl-tab:1} are attested in \ili{Abui},
both in free and bound forms, as shown in \tabref{03-kl-tab:5}. Taking the theme vowel /a/, the first (\textsc{pat}) paradigm reflects the Proto-AP prefixes that encoded Ps in the proto-language. The additional paradigms, distinguished by vowel grading and vowel lengthening, elaborated the proto-system.\footnote{Most languages of the Alor branch have expanded their verbal prefix paradigms in a similar way as \ili{Abui}; with a prefix containing an /o/ and/or an /e/. \ili{Sawila} has two verbal prefix paradigms \citep{Kratochvil2014Sawila}, \ili{Adang}, \ili{Klon}, and \ili{Wersing} have three paradigms \citep{Haan2001Grammar,Baird2008Grammar,Robinsonetal2014Adang,Schapperetal2017Relatedness}, and \ili{Kamang} has seven paradigms \citep{Schapper2014Kamang}. This suggests that \ili{Proto-Alor} may already have had two verbal prefixes.} Each of the five prefix paradigms may be used to index Ps, and a vague connection may be seen between a particular paradigm and the \isi{semantic role} of the P it encodes, as indicated by the \isi{semantic role} given in brackets in the column header.

\begin{table}[t]
\caption{Abui pronominals (\citealt[78]{Kratochvil2007Grammar}, \citeyear[591]{Kratochvil2011Transitivity}, \citeyear[555]{Kratochvil2014Differential})}
\begin{tabularx}{\textwidth}{ l Q l l Q Q l }
\lsptoprule
& free pronoun & I (\textsc{pat}) & II (\textsc{loc}) & III (\textsc{rec}) & IV (\textsc{ben}) & V (\textsc{goal})\\\midrule

1\textsc{sg} & \textit{na} & \textit{na-} & \textit{ne-} & \textit{no-} & \textit{nee-} & \textit{noo-}\\

2\textsc{sg} & \textit{a} & \textit{a-} & \textit{e-} & \textit{o-} & \textit{ee-} & \textit{oo-}\\

3 & -- & \textit{ha-} & \textit{he-} & \textit{ho-} & \textit{hee-} & \textit{hoo-}\\

\textsc{distr} & -- & \textit{ta-} & \textit{te-} & \textit{to-} & \textit{tee-} & \textit{too-}\\

1\textsc{pl.exc} & \textit{ni} & \textit{ni-} & \textit{ni-} & \textit{nu-} & \textit{nii-} & \textit{nuu-}\\

1\textsc{pl.inc} & \textit{pi} & \textit{pi-} & \textit{pi-} & \textit{pu-} & \textit{pii-} & \textit{puu-}\\

2\textsc{pl} & \textit{ri} & \textit{ri-} & \textit{ri-} & \textit{ru-} & \textit{rii-} & \textit{ruu-}\\
\lspbottomrule
\end{tabularx}
\label{03-kl-tab:5}
\end{table}

The second (\textsc{loc}) paradigm has the theme vowel /e/, and is a reflex of the \ili{Proto-Alor} Pantar possessive prefix \textit{*ge-} ‘\textsc{3gen}’. It has often been noted that location and possession are semantically related notions: an item is typically located at or near the person that possesses it. \ili{Abui} has drawn on this relation to recruit the possessor prefix of Proto-AP as a locative person index.\footnote{In \ili{Abui} possessive constructions, the Proto-AP possessor prefix (with theme vowel /e/) is used to express alienable possession, while the
 Proto-AP P-prefix (with theme vowel /a/) is used for inalienable possession \citep{Kratochvil2007Grammar}: \textit{ne-fala} ‘1\textsc{sg}.\textsc{poss}.\textsc{al}-house’ ‘my house’ versus \textit{na-min} ‘1\textsc{sg}.\textsc{poss}.\textsc{inal}-nose’ ‘my nose’. Other Alor languages share this innovation.} Paradigm four (\textsc{ben}) elaborates on the locative paradigm by lengthening the theme vowel /e/. Vowel lengthening is a strategy to create new forms in \ili{Abui}, and is also used to create a separate set of goal prefixes on the basis of the Recipient paradigm. The recipient (\textsc{rec}) paradigm itself contains the theme vowel /o/. While a prefix with this vowel cannot be reconstructed to the level of Proto-AP, it may have been present in \ili{Proto-Alor} as similar forms are found in other languages of Alor, \eg \ili{Adang} \textit{Ɂo} \citep{Haan2001Grammar}, \ili{Klon} \textit{go-} \citep{Baird2008Grammar}, and \ili{Kamang} \textit{wo-} \citep{Schapper2014Kamang}, where they have a locative
function. Prefixes with /o/ might have evolved from a word that was
originally locative postposition or verb, and became reanalyzed as a
verbal prefix.

In \REF{03-kl-ex:17}--\REF{03-kl-ex:21} it is illustrated how the different \ili{Abui} prefixes roughly correspond to semantically different Ps. The prefix expresses, respectively: a patient \REF{03-kl-ex:17}, a location \REF{03-kl-ex:18}, a recipient/benefactive \REF{03-kl-ex:19}, a benefactive \REF{03-kl-ex:20}, or a goal \REF{03-kl-ex:21}. Note also that some of the predicates are complex, consisting of two or more verbs forming a single phonological word, as in \emph{-l=bol} `give=hit' in \REF{03-kl-ex:18} and \emph{-k=yai} `throw=laugh' in \REF{03-kl-ex:19} (\cf \citealt{Klameretal2010Abui}).

\newpage  
\ea \langinfo{Abui}{}{\citealt[592]{Kratochvil2007Grammar}}\\
\ea \label{03-kl-ex:17}
\gll Na a-ruidi.\\
1\textsc{sg}.\textsc{agt} 2\textsc{sg}.\textsc{pat}-wake.up.\textsc{pfv}\\
\glt ‘I woke you up.’

\ex \label{03-kl-ex:18}
\gll Di palootang mi ne-l=bol.\\
3\textsc{agt} rattan take 1\textsc{sg}.\textsc{loc}-give=hit\\
\glt ‘He hit me with a rattan (stick).’

\ex \label{03-kl-ex:19}
\gll Fanmalei no-k=yai.\\
Fanmalei 1\textsc{sg}.\textsc{rec}-throw=laugh\\
\glt ‘Fanmalei laughed at me.’

\ex\label{03-kl-ex:20}
\gll Ma na ee-bol.\\
be.\textsc{prox} 1\textsc{sg}.\textsc{agt} 2\textsc{sg}.\textsc{ben}-hit\\
\glt ‘Let me hit [it] for you.’

\ex \label{03-kl-ex:21}
\gll Simon di noo-dik.\\
Simon 3\textsc{agt} 1\textsc{sg}.\textsc{goal}-prick\\
\glt ‘Simon is poking me.’
\z
\z


Although the above examples show rather transparent relations between the prefix and the \isi{semantic role} of the argument it encodes, in most instances where prefixes are used in \ili{Abui}, the relation between form and semantics is either vague, or absent. This is because in \ili{Abui}, P-indexing is also heavily determined by inflectional classes of verbs, and \isi{inflectional class} assignments are mostly idiosyncratic (see below). 

In \ili{Abui}, the different semantic types of transitive verbs (\eg verbs
of perception, cognition, speech, or transfer) encode their P in
various ways. Here we will not describe all the possible patterns, as
that would amount to writing another article \citep[see][]{Kratochvil2007Grammar,Kratochvil2011Transitivity,Kratochvil2014Differential,Kratochviletal2015Degrees}. Rather, we focus here on the differential marking of the P of so-called ‘typical transitive’ (\citealt[111]{Comrie1989Language}; \citealt[545]{Haspelmath2011S}) verbs only. Such verbs convey the most typical transitive activities, such as kill, hit, kick, carry, search for, take, and hold, which have a highly agentive A and a highly patientive P. In \ili{Abui}, even this restricted class of typical transitive verbs shows significant differentiation in the marking of P, as we will discuss now.

In \ili{Abui}, as in \ili{Teiwa}, \isi{animacy} determines whether or not a prefix is
used on the verb. This is illustrated in \REF{03-kl-ex:22}--\REF{03-kl-ex:23}: the \isi{inanimate} P \textit{kanai do} ‘these pili nut(s)’ is not indexed on the verb \textit{bol} ‘to hit’ in (\ref{03-kl-ex:22}), while the human body part \textit{netoku} ‘my leg(s)’ is prefixed on \textit{bol} in (\ref{03-kl-ex:23}).\footnote{Unlike English ‘hit’ and many other verbs, \ili{Abui} bi-valent \textit{bol} can take different prefixes, indicating arguments with different semantic roles and often somewhat different senses: *\textsc{pat}-\textit{bol}, \textsc{rec}-\textit{bol} ‘hit at s.o.’, \textsc{ben}-\textit{bol} ‘hit for/instead of s.o.’, \textsc{goal}-\textit{bol} ‘dust off s.o.’.} Note that both NPs are \isi{definite}: possessives and NPs marked with the demonstrative \textit{do} are \isi{definite} in \ili{Abui} \citep{Kratochviletal2015Definiteness}.

\protectedex{
\ea \langinfo{Abui}{}{\citealt[566]{Kratochvil2014Differential}}\\
\ea \label{03-kl-ex:22}
\gll Di kanai do bol took.\\
3\textsc{agt} pili.nut \textsc{prox} hit drop\\
\glt ‘He was hitting pili nuts (and) dropping [them].’

\ex\label{03-kl-ex:23}
\gll Baloka ne-toku he-bol he-balasi ba\ldots\\
k.o.grass 1\textsc{sg.poss}-leg 3.\textsc{loc}-hit 3.\textsc{loc}-beat.\textsc{pfv} \textsc{sim}\\
\glt ‘The \textit{baloka} grass hit my legs slashing them\ldots’
\z
\z
}

The variation in \REF{03-kl-ex:22} and \REF{03-kl-ex:23} is an instance of asymmetric morphological alternation between a nominal P and a P indexed on the verb with an overt morphological exponent \citetv{Witzlacketal2017Differential}.  It is parallel to the reconstructed Proto-AP pattern and to the pattern in \ili{Teiwa}, illustrated in (\ref{03-kl-ex:7})--(\ref{03-kl-ex:8}) above. In addition, \isi{animacy} also determines marking of P in \ili{Abui} following a symmetric system, where both alternatives are morphologically marked. In \REF{03-kl-ex:24}, the \isi{inanimate} P of \emph{puna} `hold' is encoded with a \textsc{loc} prefix, while in \REF{03-kl-ex:25}, the same verb takes an \isi{animate} P which is indexed with a \textsc{goal} prefix. This type of DOM marking in \ili{Abui} is analogous to the symmetrical pattern in \ili{Teiwa}, illustrated in (\ref{03-kl-ex:14})--(\ref{03-kl-ex:15}) above.


\ea \langinfo{Abui}{}{\ili{Abui} corpus: E15BD071, E15BD072}\\
\ea \label{03-kl-ex:24}
\gll Maama, na mahiting he-puna yo!\\
father 1\textsc{sg}.\textsc{agt} meat 3.\textsc{loc}-hold.\textsc{ipfv} \textsc{md}.\textsc{ad}\\
\glt ‘Father, I will hold the meat (while you slice it)!’

\ex \label{03-kl-ex:25}
\gll Di noo-puna!\\
3.\textsc{agt} 1\textsc{sg}.\textsc{goal}-hold.\textsc{ipfv}\\
\glt ‘He is grabbing (groping) me!'
\z
\z

In addition, \ili{Abui} P-marking is also sensitive to the semantically more narrow distinction between human and non-human referents. When the referent of P is human, the main transitive verb combines with another (generic) verb in a \isi{complex predicate} where the P-prefix attaches to the generic verb, as illustrated in (\ref{03-kl-ex:26}). The semantic contribution of the generic verb ‘give’ in (\ref{03-kl-ex:26}) is to flag the presence of a human P. In \REF{03-kl-ex:26} we illustrate
two such serial constructions: \textit{-l=bol} ‘give hit’ and
\textit{-l=balasa} ‘give beat’. In both cases, the referent is human,
therefore must prefix to -\textit{l} ‘give’. When a referent is not
human, the prefix is not expressed in such a serial construction with
-\textit{l}, but rather attached directly to the main verb, as was
illustrated in \REF{03-kl-ex:22} and \REF{03-kl-ex:23} above. \citet[567--569]{Kratochvil2014Differential} provides further examples of this asymmetrical DOM pattern, which is sensitive to the distinction [+/- human]. This pattern is quite frequent in \ili{Abui} and typical for verbs of change (impingement, locomotion, search verbs) and spreading into emotion and cognition verbs.

\ea \label{03-kl-ex:26}
\langinfo{Abui}{}{\ili{Abui} corpus: N12.070}\\
\gll Markus di ne-l=bol ne-l=balasa.\\
M. 3\textsc{sg} 1\textsc{sg}.\textsc{loc}-give=hit 1\textsc{sg}.\textsc{loc}-give=beat.\textsc{ipfv}\\
\glt ‘Markus gives me a beating (lit. hits me (and) beats me).' 
\z

Furthermore, besides \isi{animacy} and humanness, the \isi{affectedness} of P also plays a role in the choice of prefix. This DOM type is the topic of \citet{Kratochviletal2015Degrees}. \ili{Abui} systematically encodes the degree of \isi{affectedness} for predicates that describe change (observable change, (loco)motion, physical impingement, and going out of or coming into existence).\footnote{Other AP languages have been described having DOM systems where  ‘\isi{affectedness}’ is one of the trigger features: \ili{Blagar}  \citep[188--189]{Steinhauer2014Blagar}; \ili{Kamang} \citep[64--66]{Feddenetal2014Variation}; \ili{Klon}  \citep{Baird2008Grammar}; \ili{Sawila} \citep{Kratochvil2014Sawila}; \ili{Kula} \citep{Williams2016Kula}.} In terms of \citegen{Beavers2011Affectedness} account of \isi{affectedness}, the \ili{Abui} \textsc{pat}-indexed verbs indicate a maximum degree of \isi{affectedness} while the \textsc{loc}-indexed verbs shift one degree lower \citep[232]{Kratochviletal2015Degrees}. The alternation of the degree of \isi{affectedness} can be tested with entailments, as shown in \REF{03-kl-ex:27}--\REF{03-kl-ex:28}. The \textsc{pat}-indexed verb entails a maximal change to the effect described by the verb and this change cannot be negated by the entailment \REF{03-kl-ex:27}, but this is possible with \textsc{loc}-indexed verbs, as shown in \REF{03-kl-ex:28}.

\ea\label{03-kl-ex:27-28}
\langinfo{Abui}{}{\ili{Abui} corpus: E15BD51, E15BD52}\\
	\ea\label{03-kl-ex:27}
	\gll di	kawen	ha-komangdii	\#haba	de-i=bula	\\ 
		 	3.\textsc{agt}	machete	3.\textsc{loc}-make.blunt.\textsc{pfv}	but	3\textsc{i.loc}-have=be.sharp\\ 
		\glt `He made the knife blunt, \#but it’s still sharp.'

	\ex\label{03-kl-ex:28}
	\gll di	kawen	he-komangdii	haba		de-i=bula\\ 
	\textsc{3.agt}	machete	\textsc{3.loc}-make.blunt.\textsc{pfv}	but	\textsc{3i.loc}-have=be.sharp\\
	\glt `He made the knife blunter, but it’s still sharp.'
	\z
\z

A number of verbs of change participate in this DOM pattern in \ili{Abui}, with some examples given in \REF{03-kl-ex:29}--\REF{03-kl-ex:32}. The entailments work in the same way as for the verb \mbox{\textit{-komangdii}} ‘make blunt’ above. It should be noted that the \ili{Abui} senses may map sometimes onto different verbs in English, underlining the semantic distinctions invoked by this DOM pattern.

\ea \label{03-kl-ex:29}
Observable Change verbs \citep[222]{Kratochviletal2015Degrees}
\ea \label{03-kl-ex:29b}
+Affected: \textsc{pat} \textit{ha-}\\
\begin{tabularx}{\textwidth}{ll}
\textit{ha-lilri} & ‘boil it’\\
\textit{ha-siki} & ‘separate it’\\
\textit{ha-kol} & ‘tie it up’\\
\textit{ha-kuya} & ‘expose it’\\
\end{tabularx}

\ex \label{03-kl-ex:29a}
--Affected: \textsc{loc} \textit{he-}\\
\begin{tabularx}{\textwidth}{ll }
\textit{he-lilri} & ‘warm it up’\\
\textit{he-siki} & ‘split it’\\
\textit{he-kol} & ‘tie it’\\
\textit{he-kuya} & ‘peel it’\\
\end{tabularx}
\z
\z

\ea
\label{03-kl-ex:30}
Move and Stay at Some Location verbs \citep[227]{Kratochviletal2015Degrees}
\ea \label{03-kl-ex:30b}
+Affected: \textsc{pat} \textit{ha-}\\
\begin{tabularx}{\textwidth}{ll}
\textit{ha-taang} & ‘give it away’\\
\textit{ha-fil} & ‘pull it’\\
\textit{ha-bel} & ‘pull it out’\\
\textit{ha-baang} & ‘put on (its lid)’\\
\textit{ha-kil} & ‘turn it upside down’\\
\end{tabularx}

\ex \label{03-kl-ex:30a}
--Affected: \textsc{loc} \textit{he-}\\
\begin{tabularx}{\textwidth}{ll}
\textit{he-taang} & ‘pass it along’\\
\textit{he-fil} & ‘pull on it’\\
\textit{he-bel} & ‘pluck it’\\
\textit{he-baang} & ‘put on shoulder’\\
\textit{he-kil} & ‘put it out’\\
\end{tabularx}
\z
\z

\ea \label{03-kl-ex:31}
Physical impingement verbs \citep[227]{Kratochviletal2015Degrees}
\ea \label{03-kl-ex:31b}
+Affected: \textsc{pat} \textit{ha-}\\
\begin{tabularx}{\textwidth}{ll}
\textit{ha-dik} & ‘pierce it, stab through it’\\
\textit{ha-ril} & ‘ram it in’\\
\textit{ha-taakda}& ‘stab to death’\\
\textit{ha-keila} & ‘plug it’\\
\textit{h-afuui} & ‘scoop it up’\\
\textit{h-ahii} & ‘remove it’\\
\textit{ha-fuuidi} & ‘flatten it’\\
\end{tabularx}

\ex \label{03-kl-ex:31a}
--Affected: \textsc{loc} \textit{he-}\\
\begin{tabularx}{\textwidth}{ll}
\textit{he-dik} & ‘stab at it’\\
\textit{he-ril} & ‘plant it in’\\
\textit{he-taakda} & ‘skewer it’\\
\textit{he-keila} & ‘block it’\\
\textit{he-afui} & ‘scoop it’\\
\textit{he-ahii} & ‘select it, pick it’\\
\textit{he-fuuidi} & ‘made it flatter’\\
\end{tabularx}
\z
\z

\ea \label{03-kl-ex:32}
Go Out of Existence verbs \citep[228]{Kratochviletal2015Degrees}
\ea \label{03-kl-ex:32b}
+Affected: \textsc{pat} \textit{ha-}\\
\begin{tabularx}{\textwidth}{ll}
\textit{ha-lak} & ‘destroy it’\\
\textit{h-akung} & ‘extinguish it’\\
\end{tabularx}

\ex \label{03-kl-ex:32a}
--Affected: \textsc{loc} \textit{he-}\\
\begin{tabularx}{\textwidth}{ ll }
\textit{he-lak} & ‘demolish it’\\
\textit{he-akung} & ‘shade it’\\
\end{tabularx}
\z
\z


And finally, P-indexing is also restricted by \ili{Abui} verbal inflectional classes, which in some cases stipulate the P-index type as \textsc{pat}, irrespective of the semantics of the event expressed by the verb, as described in \cite{Feddenetal2013Feddenetal,Feddenetal2014Variation,FeddenBrown2017}. In these studies the prefixing behaviour of \ili{Abui} verbs was examined. About 10\% of the verbs always index the P with the \textsc{pat} prefix and do not allow any symmetrical DOM. This particular \isi{inflectional class} includes both typical transitive verbs, describing events of observable change \REF{03-kl-ex:33}, (loco)motion \REF{03-kl-ex:34}, physical impingement \REF{03-kl-ex:35}, and going out of or coming into existence \REF{03-kl-ex:36}) (e.g., \textit{-balak} ‘to hit, punch s.o./sth.’ and \textit{-basa} ‘to brush off sth.’), but also verbs of speech, cognition and transfer, as well as verbs of perception, posture, placement and sound \citep{Feddenetal2014Variation,Kratochviletal2015Degrees}. It is possible that these verbs represent an older layer of the \ili{Abui} lexicon, and reflect an older stage of its grammar, before the systematic DOM alternation between \textsc{pat}- and \textsc{loc}-indexed verb was fully grammaticalized.

\ea \label{03-kl-ex:33}
Observable Change verbs \citep[222]{Kratochviletal2015Degrees}\\
\begin{tabularx}{\textwidth}{ll ll}
\textit{ha-basa} & ‘brush him off, dust it’ & \textit{ha-weel} & ‘wash him, bathe him’\\
\textit{ha-kuol} & ‘shave it’ & \textit{h-iel} & ‘roast it’\\
\textit{ha-tamadia} & ‘repair it’\\
\end{tabularx}
\z

\ea \label{03-kl-ex:34}
Move and Stay at Some Location verbs \citep[223]{Kratochviletal2015Degrees}\\
\begin{tabularx}{\textwidth}{ll ll}
\textit{ha-fik} & ‘pull it, pull him’ & \textit{ha-kuoila} & ‘topple it’\\
\textit{ha-ai} & ‘add it’ & \textit{ha-bi} & ‘lean against it’\\
\textit{ha-suonra} & ‘push it’ & \textit{ha-kai} & ‘drop it, trip him’\\
\textit{ha-reng} & ‘turn to it’\\
\end{tabularx}
\z

\ea \label{03-kl-ex:35}
Physical impingement verbs \citep[224]{Kratochviletal2015Degrees}\\
\begin{tabularx}{\textwidth}{llll}
\textit{ha-balak} & ‘punch him’ & \textit{h-uol} & ‘hit/strike him’\\
\textit{ha-laanga} & ‘grope him’ & \textit{ha-paakda} & ‘slap him’\\
\textit{ha-taak} & ‘shoot him’\\
\end{tabularx}
\z

\ea \label{03-kl-ex:36}
Go Out of Existence verbs \citep[224]{Kratochviletal2015Degrees}\\
\begin{tabularx}{\textwidth}{llll}
\textit{ha-al} & ‘burn it’ & \textit{ha-pok} & ‘cover it’\\
\textit{ha-fuul} & ‘swallow it’ & \textit{ha-yol} & ‘bury it’\\
\end{tabularx}
\z

The inflectional verb class illustrated in \REF{03-kl-ex:33}--\REF{03-kl-ex:36} contrasts with the \textsc{pat}{$\sim$}\textsc{loc} alternating verbs in \REF{03-kl-ex:29}--\REF{03-kl-ex:32} in that the degree of \isi{affectedness} of their P is not fixed. This can be seen when the entailment is a `failed' reading, as shown in \REF{03-kl-ex:37}--\REF{03-kl-ex:39}, something not possible for the \textsc{pat}-indexed verbs that participate in the symmetrical DOM discussed above. For more details, see \citet{Kratochviletal2015Degrees}.


 



\ea \langinfo{Abui}{}{\ili{Abui} corpus: E15BD34, E15BD35, E15BD36}\\
\ea \label{03-kl-ex:37}
\gll na ha-fik-i haba burook naha\\
1\textsc{sg}.\textsc{agt} 3.\textsc{pat}-pull-\textsc{pfv} but but not\\
\glt ‘I pulled it but it didn’t move.’ 

\ex \label{03-kl-ex:38}
\gll na	ha-fik-i	haba	sik	naha.\\
1\textsc{sg}.\textsc{agt}	3.\textsc{pat}-pull-\textsc{pfv}	but	snap	not\\
\glt ‘I pulled it but it didn’t snap.’

\ex \label{03-kl-ex:39}
\gll na	ha-fik-i	haba	dara	de-yal	mia.\\
1\textsc{sg}.\textsc{agt}	3.\textsc{pat}-pull-\textsc{pfv}	but	still	3\textsc{i.al}-place	be.in\\
\glt ‘I pulled it but it is in its place (it’s too heavy).’ 
\z
\z
 
 
Clearly, this class does not show any evidence of symmetrical DOM as it marks P always in the same way (with a \textsc{pat} prefix). Yet it is important to mention it in the context of the current paper, because it shows that while \ili{Abui} differentiates Ps in symmetric and asymmetric ways, along a number of different semantic dimensions, the language also has a reasonably large class of bivalent verbs that do not take part in symmetrical differential marking of P at all.

The DOM pattern of alternation between \textsc{loc}- and \textsc{pat}-indexed verbs is attested with  22\% of the sample investigated by Fedden et al. \citeyear{Feddenetal2013Feddenetal,Feddenetal2014Variation}. Furthermore, verbs in this class can also combine with other series (\textsc{ben}, \textsc{rec}, or \textsc{goal}), \ie alternate symmetrically. At the same time, verbs in this class can also occur without a prefix and alternate asymmetrically in complex predicates (see \citealt[Table~5]{Feddenetal2014Variation}). In general, the three additional series (\textsc{ben}, \textsc{rec}, and \textsc{goal}) are less restricted and combine on average with about 87\% of the roots. This is expected, given their later development and greater productivity.

In sum, there is a variety of factors involved in the marking of the objects of the typical transitive verbs in \ili{Abui}. These include: 

\begin{itemize}
\item the \isi{semantic role} of P (where Ps that are semantically patient, locative, benefactive or goal can be marked differently); 
\item the inherent semantic properties of the argument (whether P is \isi{animate} or not, whether P is human or not);
\item the relation between the verb and its argument (whether P is affected or not and to what degree); 
\item the inflectional verb class (which determines whether or not P is marked differentially, and how it is marked differentially, \ie using a symmetrical or asymmetrical pattern).
\end{itemize}

The \ili{Abui} data clearly show that in a single language, DOM can have multiple triggers, involving inherent lexical argument properties, inflectional classes, and event semantics; and combine symmetrical and asymmetrical morphological alternations. In a language family such as the AP family, which tends to index P over S/A, languages may develop in a direction where they elaborate on the encodings of P in new ways, as \ili{Abui} demonstrates. \figref{03-kl-fig:7} shows how the modern \ili{Abui} morphemes used for DOM relate to the reconstructed forms in Proto-AP.

Unlike in \ili{Teiwa}, \ili{Abui} retained the \isi{semantic alignment} of Proto-AP, where S could sometimes be marked as P. In numerous cases, S arguments can be indexed on verbs as if they are Ps. In general, such S arguments have a more affected, and less volitional, semantics than free-standing S arguments \citep{Kratochvil2007Grammar,Kratochvil2011Transitivity,Kratochvil2014Differential,Feddenetal2013Feddenetal,Feddenetal2014Variation,FeddenBrown2017}.

\clearpage 
\begin{figure}
	\caption{The historical relation between forms encoding P in Proto-Alor-Pantar and Abui.}\label{03-kl-fig:7} 
	\begin{center}\begin{tikzpicture}[every node/.style={anchor=base west}]
		\matrix (proto) [anchor=base,draw=lsMidDarkBlue,matrix of nodes,nodes in empty cells=true,column sep=2cm]{
			& \bfseries Proto-AP & \\
			\textsc{a}, \textsc{s} \color{lsMidDarkBlue}{\textit{*ga(N)}} & \textsc{p}, \textsc{s} \color{lsMidDarkBlue}{\textit{*ga-}} & \textsc{poss} \color{lsMidDarkBlue}{\textit{*ge-}}\\
		};
		\matrix (a) [draw, matrix of nodes,below=5\baselineskip of proto-2-1.base] {A\\};
		\matrix (p) [draw=lsRed,matrix of nodes,nodes in empty cells=true,below=5\baselineskip of proto-2-2.base]{
			\scshape p, s & \\
			\itshape \color{lsMidDarkBlue}{ha-} & \color{lsRed}{+Affected}\\
			\itshape \color{lsMidDarkBlue}{he-} & \color{lsRed}{-Affected}\\
			\itshape \color{lsRichGreen}{hee-}\\		
			\itshape \color{lsRichGreen}{ho-}\\
			\itshape \color{lsRichGreen}{hoo-}\\			
		};
		\matrix (poss) [right=of p,draw,matrix of nodes] {
			    & \color{lsMidDarkBlue}{\textit{he-} (AL)}\\
			\textsc{poss} & \color{lsMidDarkBlue}{\textit{ha-} (INAL)}\\
			\\
		};
		
		\draw[dashed,{-Triangle[]}] (proto-2-1) -- (a);
		\draw[dashed,{-Triangle[]}] (proto-2-2) -- (p-2-2);
		\draw[dashed,{-Triangle[]}] (proto-2-2) -- (poss-2-2.north west);
		\draw[dashed,{-Triangle[]}] (proto-2-3) -- (poss);
		\draw[dashed,{-Triangle[]}] (proto-2-3) -- (p-3-2.east);
		\end{tikzpicture}\end{center}
\end{figure}	
		

\section{Conclusions}
\label{03-kl-sec:6}
Sharing a common ancestor that had DOM, \ili{Teiwa} and \ili{Abui} still mark objects differentially, and in both languages, reflexes of the same proto-morpheme are used in the differential marking of P. Yet, there are many differences between the two languages in the proto-forms that have been retained and innovated, and in the way DOM is applied.

In their morphological expression, there are two dimensions in which Ps are differentiated in both \ili{Teiwa} and \ili{Abui}. The first is asymmetrical: either P is expressed as a verbal prefix (with an optional co-referent pronoun or NP in the clause), or P is expressed as a free pronoun or nominal phrase. Second, Ps may be differentiated symmetrically, by the variable choice of a P-prefix depending on the semantics of P. Both strategies are used in both languages, but the symmetrical strategy involves two prefixes in \ili{Teiwa} and five prefixes in \ili{Abui}. The DOM patterns are summarized below (the \isi{information structure} uses are not included).


Also, the factors triggering DOM are different: in \ili{Teiwa} it is mostly based on the inherent properties (\isi{animacy}) of P, while in \ili{Abui} there are many other triggers besides the \isi{animacy} of P, including the \isi{affectedness} relation between the action and the P referent and the \isi{inflectional class} of the verb. Furthermore, \ili{Abui} has developed an extra, third, formal strategy to differentiate human Ps from non-human ones in a serial verb construction. 

The reconstructed \isi{alignment system} of Proto-AP was semantic. In \ili{Teiwa}, this system has evolved into an accusative \isi{alignment system}, but the original system was retained and further complexified in \ili{Abui}. This indicates that alignment systems are not static and can be modified and complexified over time by putting morphemes of an ancestor language into new uses and creating new forms, e.g. by adding symmetrical paradigms of person-indexing prefixes.

An interesting comparison can be made with the \isi{semantic alignment} systems of the Papuan languages of North Halmahera discussed in \citet{Holton2008}. While there is evidence for syntactic alignment in proto-North Halmaheran, many of the modern North-Halmaheran languages have innovated \isi{semantic alignment} \citep[274--275]{Holton2008}. In the AP languages, the situation is the opposite: the \isi{semantic alignment} is reconstructed for the proto-language, and the syntactic alignment in \ili{Teiwa} is an innovation. The path of historical evolution of \isi{alignment system} can therefore not be unidirectional (from syntactic to \isi{semantic alignment}), but the evolution in both directions is possible, and facilitated by DOM (in Alor Pantar) and optional or pleonastic marking (in North Halmahera).

It seems that languages that have \isi{semantic alignment} (or differential S marking) alongside DOM, such as \ili{Abui}, tend to develop more complex systems of DOM than languages with \isi{accusative alignment}, such as \ili{Teiwa}.

In the development of their respective DOM systems, \ili{Teiwa} and \ili{Abui} underwent different morphological changes. The Proto-AP prefix \textit{*ga-} is reflected in the \ili{Teiwa} prefix that encodes topics and \isi{animate} Ps, as well as in \ili{Teiwa} possessors. In \ili{Abui}, this prefix is reflected as the \textsc{pat} prefix and could be the source of the innovated prefixes as well \citep{Klameretal2012Role}. The \textsc{pat} prefix is the most semantically bleached prefix of all five of the \ili{Abui} P-prefixes, as it is obligatory for a semantically diverse class of verbs that makes up 10\% of the total number of verbs investigated in \ili{Abui}. Most of these verbs encode events describing various types of change (observable change, (loco)motion, physical impingement, going out of and coming into existence) – suggesting a relationship with \isi{affectedness}. The \ili{Abui} \textsc{loc} and \textsc{ben} prefixes feature the theme vowel /e/, reflecting the \ili{Proto-Alor} Pantar possessive prefix \textit{*ge-} ‘\textsc{3gen}’, but in \ili{Teiwa} no reflex of this prefix has been retained.

The proto-pronoun \textit{*ga(N)} that was used to encode A and S in Proto-AP is reflected in modern \ili{Teiwa} as the free pronoun \textit{ga’an}, but in \ili{Teiwa} it encodes contrastive focused Ps. In \ili{Abui} it encodes A, but the final nasal has been lost. \ili{Abui} has also innovated a new prefix paradigm with a theme vowel /o/, and two additional paradigms by lengthening the vowel of existing paradigms. Apart from the use of reflexes of the Proto-AP object prefix \textit{*ga-}, very few similarities remain between the morphemes that are used in \ili{Teiwa} and \ili{Abui} DOM. 

In sum, this study has shown that the evolutionary path of DOM from \ili{Proto-Alor} Pantar into its daughter languages has both stable and unstable features. Stable features are the inherent semantic feature of humanness/\isi{animacy} of P that is being coded, and the shape of the person prefix that is used in the coding. However, the \isi{semantic alignment} system of \ili{Proto-Alor} Pantar appears to be volatile, as it changed to accusative in \ili{Teiwa}. This is not an unexpected result since alignment patterns are sensitive to morphological and phonological changes. Also, a language can develop additional triggers for DOM as well as the additional person markers that it needs to encode these additional types of Ps alongside inflectional verb classes, as has happened in \ili{Abui}. In general, the DOM triggers in \ili{Abui} shifted away from being purely participant-related, to include event-related features (degree of \isi{affectedness}) as well.  The Alor Pantar languages show that alignment systems are not static: their forms and triggers may be modified and complexified quite substantially over time. 

\section*{Acknowledgements} 
We would like to thank the editors of this volume and an anonymous reviewer for their comments on previous versions of this paper. Both authors gratefully acknowledge the hospitality of the Abui and  Teiwa communities, as well as research funding for the VICI research project 'Reconstructing the past through languages of the present: the Lesser Sunda Islands', funded by the Netherlands Organisation for Scientific Research, project number 277-70-012 (Klamer), and Nanyang Technological University (Singapore) through a Singapore Ministry of Education Tier 2 Grant MOE2013-T2-1-016 (Kratochvíl). Both authors also gratefully acknowledge the generous support of the Royal Netherlands Academy of Arts and Sciences through its NIAS Fellow-in-Residence scheme (2015).

\section*{Abbreviations}
\begin{tabularx}{.45\textwidth}{lQ}
1 & person markers\\
2 & second person\\
3 & third person\\
\textsc{ad} & addressee-perspective\\
\textsc{agt} & agentive pronoun\\
\textsc{anim} & animate\\
\textsc{ap} & Alor Pantar\\
\textsc{appl} & applicative\\
\textsc{assoc} & associative\\
\textsc{ben} & undergoer prefix paradigm (benefactive-like)\\
\textsc{cont} & continuative\\
\textsc{dem} & demonstrative\\
\textsc{dist} & distal\\
\textsc{distr} & distributive\\
\textsc{dom}& differential object marking\\
\textsc{evid} & evidential\\
\textsc{foc} & focus\\
\\
\end{tabularx}
\begin{tabularx}{.45\textwidth}{lQ}
\textsc{goal} & undergoer prefix paradigm (goal-like)\\
\textsc{inc} & inchoative\\
\textsc{ind} & Indonesian\\
\textsc{lnk} & linker\\
\textsc{loc} & undergoer prefix paradigm (location-like)\\
\textsc{md} & medial\\
\textsc{mod} & modal\\
\textsc{pat} & undergoer prefix paradigm (patient-like)\\
\textsc{prior} & priorative\\
\textsc{prox} & proximate\\
\textsc{rec} & undergoer prefix paradigm (recipient-like)\\
\textsc{seq} & sequential\\
\textsc{sim} & simultaneous\\
\textsc{spc} & specific determiner\\
\textsc{top} & topic\\
\end{tabularx}


{\sloppy
\printbibliography[heading=subbibliography,notkeyword=this] }
\end{document}