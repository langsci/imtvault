\documentclass[output=paper]{LSP/langsci}
\ChapterDOI{10.5281/zenodo.1228253} 
\author{Åke Engsheden\affiliation{Stockholm University, Department of Archaeology and Classical Studies}}
\title{Verbal semantics and differential object marking in Lycopolitan Coptic}
%\epigram{Change epigram in chapters/03.tex or remove it there}
\abstract{This paper seeks to clarify the role of affectedness for the marking of direct objects through an analysis of a corpus of Lycopolitan Coptic texts (4th to 5th centuries AD). Whereas previous research has shown the importance of definiteness for the use of the direct object marker \textit{n} with the so-called imperfective tenses (present and imperfect), it has proven more difficult to establish why it alternates in the non-imperfective with a zero marker. An attempt is made here to correlate the two different object constructions to Tsunoda’s verb-type hierarchy, which was conceived to capture the degree of affectedness. It appears that the more affected a direct object is, the more likely it is to receive the direct object marker; whenever the object is little affected or unaffected, the zero-marked construction is preferred.}

%\keywords{Differential Object Marking, Lycopolitan Coptic, Affectedness, Transitivity, Historical grammar}
 
\maketitle

\begin{document}

\section{Introduction}
\label{06-en-sec:1}

Most works that have tried to explain Differential Object Marking (DOM) have focused on the semantic and information-structural properties of the \isi{direct object} (\isi{animacy}, \isi{definiteness}, specificity, or \isi{topicality}). There are a few languages for which the identification of the triggering factor behind DOM may be quite straightforward, such as \isi{definiteness} in Modern \ili{Hebrew} \citep{Danon2001Syntactic} or specificity in \ili{Turkish} \citep{Enc1991Semantics}, but more commonly a multidimensional DOM system results not from a single factor, but from several interacting factors %interacting. 

One language with a multidimensional DOM is \ili{Coptic} (Afro-Asiatic, Egyptian branch, now extinct).\footnote{Egyptian is divided into the following language stages: Old Egyptian (c.\, 3100–2000 BC), Middle Egyptian (2000–1350 BC), \ili{Late Egyptian} (1350–700 BC), \ili{Demotic} (700 BC–AD 452) and \ili{Coptic} (AD 200–1400). For a useful grammatical overview see \citet{Haspelmath2015Egyptian}. For a detailed diachronic description aimed at a linguistic readership, see \citet{Loprieno1995Egyptian}.} \ili{Coptic} DOM has received far less attention one might expect, given that \ili{Coptic} has a long tradition in academic studies. Indeed, it is still unclear what exact factors are operative and how they relate to each other. The present study aims to show how the verb type, which is defined through the degree of \isi{affectedness} found with the object, influences whether the object is marked as such or receives no marking. 
This will be done through a corpus-based study of Lycopolitan, an early literary variety (traditionally and henceforth “dialect”) of \ili{Coptic} that was prevalent in the 4th and 5th centuries AD. 
The analysis indicates that the overtly marked construction is favoured by the presence of a highly affected object, whereas the zero-marked construction is favoured whenever the object is little affected or unaffected by the verbal action. 
Beside the value of such a study for our understanding of argument marking in \ili{Coptic} itself, a wider knowledge of \ili{Coptic} data should be of interest to linguistics because \ili{Coptic} presents a system that is markedly different from better-explored patterns of DOM. 

This paper is structured as follows. \sectref{06-en-sec:2} provides a synthesis of \ili{Coptic} object marking, including a summary of previous research. \sectref{06-en-sec:3} contains a short description of the corpus of Lycopolitan texts and presents some background data. \sectref{06-en-sec:4} presents an overview of the role of verbal semantics in research into DOM, and introduces some theoretical work on how verbs can be arranged on semantic grounds in broad verb-type categories. In \sectref{06-en-sec:5}, statistics are provided for the realisation of the object in Lycopolitan \ili{Coptic} in relation to the verb types. The analysis suggests that the distribution of two alternating object constructions depends on the degree of \isi{affectedness} of the object. In \sectref{06-en-sec:6}, the relationship between \isi{affectedness} and other factors is discussed. Finally, \sectref{06-en-sec:7} contains a summary and preliminary conclusions. 

\section{Argument realisation in Coptic}\label{06-en-sec:2}

\ili{Coptic} DOM is of the asymmetric type (\citealt{Hoopetal2008Case-marking}; \citealt{Iemmolo2013Symmetric}), in which the \isi{direct object} is either overtly marked with a preposition or zero-marked. The marker before NPs is a preposition, \textit{n} (before labials \textit{m}), the origin of which is ultimately locative. A longer form, \textit{mma}, is used preceding the clitic person markers.\footnote{In \ili{Sahidic}, the supra-regional dialect of the south, the equivalent forms are \textit{n} and \textit{mmo}. Both forms derive from the preposition  \textit{m}, used in older Egyptian for location in something (‘essive’) as well as for motion away from something (‘elative’), from whence derives the partitive meaning that seems to have given rise to object marking \citep{Winand2015Expression}.} Both are subsumed in the following under the term \textit{n-}marking. Note that one often has a double marking of the transitive construction, because many verbs have separate allomorphs depending on which object construction is used. The verbal allomorphs are, by and large, distinguished by different vowels because of the shape of the syllable and stress rules. The \textit{n-}marked object appears only after the regular stem of the verb with a full vowel carrying stress (\eg \textit{nouje} ‘to throw’, see \tabref{tab-06-en:1}). A zero-marked NP, on the other hand, can appear both after the regular stem and as an allomorph of the verb with a reduced vowel.\footnote{Only the latter is possible in many other dialects. I have deliberately not distinguished these two cases in the counts in the tables, because I wish to avoid a digression on the morphology of the verb classes.} For some morphological classes of the verb only one allomorph is used before zero-marked NPs and personal pronouns. Thus, the verb ‘to throw’ can assume the form \textit{naj} before NPs and personal clitics (exemplified in \tabref{tab-06-en:1}, through the 3 msg. pronoun \textit{f}). 

\begin{table}
\begin{tabular}{lcc}
\lsptoprule
\textit{n-}marked O & \textit{nouje n} NP /\textit{nouje mma-f} & \textit{—}\\
zero-marked O & \textit{nouj} NP / \textit{—} & \textit{naj-}NP / \textit{naj-f}\\
\lspbottomrule
\end{tabular}
\caption{Verb allomorphs and object marking in Lycopolitan}\label{tab-06-en:1}
\end{table}

There are also verbs that have different allomorphs with zero-marked objects, depending on what follows them. For instance, the verb \textit{eire} ‘to make’ or ‘to do’ assumes the form \textit{r} in front of NPs, while it becomes \textit{ee} alt. \textit{eit} in front of personal clitics. 

The rules governing the selection between the \textit{n-}marked form and the zero-marked form are far from clear. A few important observations that have been made in the past are summarised here and in the following subsections (\sectref{06-en-sec:2-1}--\sectref{06-en-sec:2-2}). 

Case marking occurs only in the post-verbal position (this also applies to subject marking, see \citealt{Grossman2015Verb}). When an object is fronted, a familiar strategy for topicalisation, it is then not case-marked but is resumed postverbally through the appropriate person marker. Both \textit{n-}marked and zero-marked objects occur~(\ref{06-en:ex:1a}--\ref{06-en:ex:1b}):

\begin{exe}
\ex 
\begin{xlist}
\ex \label{06-en:ex:1a}
\gll t-mnt-lilou a-i-t\textsuperscript{h}bio mma-s\\
	\textsc{def.f-abst-}youth \textsc{pst-1sg-}subdue \textsc{acc-3f.sg}\\
\glt ‘Youth I subdued’ (Psalm-book 88, 27)

\ex\label{06-en:ex:1b}
\gll eis p-kah m-p-keke a-n-šab-f\\
	\textsc{ptcl} \textsc{def.m-}land \textsc{gen-def.m-}darkness \textsc{pst-1pl-}devastate\textsc{-3m.sg}\\
\glt ‘Look, the Land of Darkness we devastated’ (Psalm-book 201, 23)

\end{xlist}
\end{exe}


Object marking with \textit{n}/\textit{mma} is also found in some non-differential contexts. For example, \textit{n-}marking must be used whenever the \isi{direct object} is separated from the verb by any element. In the following example~(\ref{06-en:ex:2}), the object (\textit{tef-hikôn}) is preceded by \textit{n} due to the placement of the verbal particle \textit{abal}. Zero-marked objects are only allowed when the object directly follows the verb with no intervening element.

\begin{exe}
\ex \label{06-en:ex:2}
\gll ša-p-sêu  etere  p-iôt  na-côlp  abal n-tef-hikôn  n-t-pe  mma-u / (*côlp abal tef-hikôn)\\
	until-\textsc{def.m-}time \textsc{rel} \textsc{def.m-}father \textsc{fut-}reveal \textsc{ptcl} \textsc{acc-}\textsc{poss.3m.sg-}image \textsc{adv-}\textsc{def.f-}sky \textsc{prep-}them / *reveal \textsc{ptcl} \textsc{poss.3m.sg-}image\\
\glt ‘until the time when the Father will reveal his image above them’ (Kephalaia 103, 6)

\end{exe}

Furthermore, the majority of verbs borrowed from \ili{Greek} have their objects introduced with \textit{n}/\textit{mma}~(\ref{06-en:ex:3}). This is determined by the valency of the verb and is not a differential environment and, consequently, \ili{Greek} loan-verbs are not treated in this paper. 

\begin{exe}
\ex \label{06-en:ex:3}
\gll a-s-panhoplize mma-f\\
	\textsc{pst-3f.sg-}arm \textsc{acc-3m.sg}\\
\glt ‘She armed him’ (Kephalaia 39, 4)
\end{exe}

It has long been recognised that \textit{n-}marking is used with an NP only when the latter is determined by any of the articles (\isi{definite} or \isi{indefinite}), the possessive determiner, or a demonstrative. The \textit{n-}marking is not used with a bare noun, which signals a generic and \isi{indefinite} sense. It would thus seem as if \ili{Coptic} DOM conforms to the \isi{definiteness} hierarchy: personal pronoun > proper noun > \isi{definite} NP > \isi{indefinite} specific NP > non-specific NP (\eg \citealt{Aissen2003Differential}: 437). The cut-off point along this scale differs between the main two TAM categories (imperfective \vs non-imperfective), but the lowest ranked category (non-specific NPs) is excluded in both. As \isi{definiteness} is an all-pervasive feature (irrespective of TAM), it can be said to be the single most important factor for the selection of \textit{n-}marking in \ili{Coptic} (\cf \citealt[309]{Sinnemki2014Typological}).


\subsection{Imperfective tenses}
\label{06-en-sec:2-1}

There is a TAM-based split in the distribution of object marking, to the effect that the \textit{n-}marked form is obligatory with the imperfective tenses (present and imperfect) when the object is grammatically \isi{definite}, and optional, it seems, with all other tenses (see \sectref{06-en-sec:2-2}).\footnote{The rules governing object marking with the imperfective tenses were first described by Ludwig \citet{Stern1880Koptische} before being elaborated by Pëtr Viktorovič Ernštedt \citep{Jernstedt1927Koptische}, for which reason they are known as the Stern-Jernstedt rule in Coptological jargon.} This means that the \textit{n-}marked form was used with personal pronouns~(\ref{06-en:ex:4}), demonstratives~(\ref{06-en:ex:5}), and NPs preceded either by the \isi{definite} article~(\ref{06-en:ex:6}) or the \isi{indefinite} article~(\ref{06-en:ex:7}) whenever the verb is in the present or the imperfect:

\begin{exe}
\ex \label{06-en:ex:4}
\gll etbe peei pa-eiôt maeie mma-i / (*merit-${\emptyset}$) \\
	because \textsc{dem.m} \textsc{poss.1sg-}father love \textsc{acc-}1\textsc{sg} / *love-1\textsc{sg}\\
\glt ‘Because of this my father loves me’ (John 10: 17)
\end{exe}

\begin{exe}
\ex \label{06-en:ex:5}
\gll auô tes-ke-meeu ne-s-jou n-neei / (*je-neei)\\
	and \textsc{poss.3f.sg-}also-mother \textsc{impf-}3\textsc{fsg-}say \textsc{acc-}this.\textsc{n} / *say-this.\textsc{n}\\
\glt ‘And also her mother was saying this’ (Acts of Paul 11, 25)
\end{exe}

\begin{exe}
\ex \label{06-en:ex:6}
\gll anak ti-saune m-pa-eiôt / (*souôn-pa-eiôt)\\
	\textsc{1sg} \textsc{1sg-}know \textsc{acc-}\textsc{poss.1sg-}father / *know-\textsc{poss.1sg-}father\textsc{}\\
\glt ‘I know my father’ (John 10: 15)
\end{exe}

\begin{exe}
\ex \label{06-en:ex:7}
\gll p-et-šôl n-ou-ônh abal / (*šal-ou-ônh)\\
	\textsc{def.m-}\textsc{rel}shed \textsc{acc-indf-}life out / *shed\textsc{-indf-}life\\
\glt ‘He who sheds a life’ (Psalm-book 39, 26)
\end{exe}

The rule of obligatory marking also holds true for the possessive determiner~(\ref{06-en:ex:8}) that is formed from the \isi{definite} article marking the gender and number of the possessee, to which the appropriate personal marker for the possessor is affixed.

\begin{exe}
\ex \label{06-en:ex:8}
\gll hama=nde an ne-s-maeie n-tes-šêre 	phalkônilla / (*meri-tes-šêre)\\
	at.the.same.time=but also \textsc{impf-3f.sg-}love \textsc{acc-poss.3f.sg-}daughter Falconilla / *love-\textsc{poss.3f.sg-}daughter\\
\glt ‘at the same time she also loved her daughter Falconilla’ (Acts of Paul 22, 17)
\end{exe}

Grammatically \isi{definite} objects are marked irrespective of specificity. In general, both specific and non-specific NPs are \textit{n-}marked. Exceptions to this occur whenever a light verb forms a verbal expression together with its syntactical object, as in the following example~(\ref{06-en:ex:9}), with \textit{r-p-meeue} ‘to remember’ (lit. ‘to do the remembrance’). \textit{N-}marking is attested with light verbs in other dialects and texts \citep[133]{Layton2000Coptic}. 

\begin{exe}
\ex \label{06-en:ex:9}
\gll ntaf 	n-šarp 		p-et-hn-plêrouma 		p-et-ah-tôbh 	mmaf	auô 	e-f-r-p-meeue\\
\textsc{3m.sg} \textsc{adv-}first \textsc{def.m-}\textsc{rel-}in-Pleroma \textsc{def.m-rel-pst-}pray \textsc{acc-3m.sg} and \textsc{circ-3m.sg-}do-\textsc{def.m-}memory\\
\glt ‘The one who is in the Pleroma was what he first prayed to and remembered’ (Tripartite tractatate 81, 30–32)
\end{exe}

There is one lexical exception to this pattern, where the \isi{definite} article has no influence on object marking with the imperfective tenses. The verb \textit{ouôš} ‘to want’ is always used with a zero-marked \isi{definite} object, as seen in~(\ref{06-en:ex:10}):\footnote{In accordance with the Leiden Conventions for Papyrology, I use square brackets for restorations, and angled brackets for text omitted by the ancient scribe.} 

\begin{exe}
\ex \label{06-en:ex:10}
\gll e-u-jpo m-p-et-〈ou〉-ouaš-f / (*ouôš mma-f)\\
	\textsc{circ-3pl-}give.birth \textsc{acc-def-3pl-}wish-3\textsc{m.sg} / *wish \textsc{acc-}3\textsc{m.sg}\\
\glt ‘they begetting what they wish’ (Tripartite tractate 64, 15)
\end{exe}

Language history has been evoked to explain this exception. It has been suggested that the distinction between the two different frames – \textit{wḫȝ n} O ‘to look for’, contrasting with \textit{wḫȝ} O, ‘to wish’ – was made at the earlier stage of the language (\ili{Demotic}), and is preserved here \citep{Depuydt1993Sake}. In \sectref{06-en-sec:5} I will offer an alternative functional explanation, which is based on an observation of \ili{Coptic} data.

When no determiner is present, the object is zero-marked~(\ref{06-en:ex:11}). In such a case the noun is non-referential and non-specific, and does not reappear in the discourse. Zero-marking usually applies to \isi{indefinite} pronouns as objects, but there are counter-examples, such as the one found in the first part of the sentence quoted in~(\ref{06-en:ex:12}).\footnote{One may try to attribute a specific reading to the object in~(\ref{06-en:ex:12}), which would be awkward, or else one can explain the use of \textit{n-}marking with \textit{sa}\textit{une} ‘to know’ in morphological terms (see \sectref{06-en-sec:5-3}).}

\begin{exe}
\ex \label{06-en:ex:11}
\gll ti-šp-hmat 				n-toot-k / (*šôp n-hmat)\\
 \textsc{1sg-}receive-grace from-hand-\textsc{poss.2msg } / *receive \textsc{acc-}grace\\\\
\glt ‘I receive grace from your hand [\ie ‘I thank you’] (John 11: 41)
\end{exe}

\begin{exe}
\ex \label{06-en:ex:12}
\gll f-saune 		n-laue 		en / (?f-senouôn-laue en)	oude 	f-r-laue 		n-hôf 	en 		an \\
\textsc{3m.sg-}know \textsc{acc-}something \textsc{neg} / *\textsc{3m.sg-}know-something \textsc{neg} nor \textsc{3m.sg-}do-something \textsc{gen-}thing \textsc{neg} also\\
\glt ‘It [sc. the fruit] knows nothing, nor does it do anything’ (Gospel of Truth 28, 9–10)
\end{exe}

I have not found in my corpus of Lycopolitan \ili{Coptic} any example of a proper noun as an object with the imperfective tenses, but data from other dialects show that \textit{n-}marking must be used in such cases. As is apparent from the above, semantic and morphological \isi{definiteness} triggers the marking of the object. 

Note that object marking is an innovation in the evolution of the Egyptian language. Afroasiatic case has not left any indisputable traces. Differential marking with the preposition \textit{n} started to appear around 1000 BC, first in the imperfective as a marker of the unbounded aspect (\citealt{Winand2015Expression}; \cf \citealt{Engsheden2006Koptischen}: 218–219), but it spread to the non-imperfective tenses in the first millennium AD.

\subsection{Non-imperfective tenses}
\label{06-en-sec:2-2}

The rationale behind the alternating use of \textit{n}/${\emptyset}$ with non-imperfective tenses is less clear. \ili{Coptic} is rich in various TAM forms that are often labelled in an idiosyncratic way. What I call non-imperfective TAM forms covers every verbal form other than the present and the imperfect.\footnote{The group comprises past, future, optative, jussive, aorist, conditional, imperative, and a verb form called conjunctive that is used for subsequent action, etc.} 

The non-imperfective is a negatively-defined term that is used here as a label only: it encompasses the perfective as well as aspectually neutral forms. I include the future among the non-imperfective tenses. This differs from the tradition in \ili{Coptic} linguistics to include the future, which is characterised by the infix \textit{na-} (traditionally known as the ‘first future’), along with the present and the imperfect, among the imperfective tenses.\footnote{There are historical reasons for dividing \ili{Coptic} TAM forms into two groups: the so-called adverbial/bipartite/durative pattern (\ie my imperfective) \vs the verbal/tripartite/non-durative pattern (my non-imperfective). As the future tense form mostly appears in non-imperfective contexts, and shares its argument realisation strategies with non-imperfectives, I believe that the \ili{Coptic} future is better classified among the non-imperfective tenses (following \citealt{Alvarez2001futuro}). For this reason, counts for the future are included among the non-imperfective tenses in this article.} 

With the non-imperfective tenses (including the future), \textit{n-}marking appears optionally with personal pronouns and NPs that have any of the three determiners: the \isi{indefinite}, \isi{definite}, or possessive articles. The common view of non-imperfective tenses among Coptologists is that “non-zero objects fluctuate (by speaker’s stylistic choice)” \citep[132]{Layton2000Coptic}. One leading Coptologist has even stated that \textit{n-}marking and zero-marking of the object “are generally understood to be functionally equivalent” \citep[41]{Emmel2006Coptic}. At first glance, this appears to be true, because both constructions are found in more or less identical contexts, as in \REF{06-en:ex:13a}–\REF{06-en:ex:13b}, where both phrases have the same verb in a terminative subordinate clause:

\begin{exe}
\ex \label{06-en:ex:13}
\begin{xlist}
\ex \label{06-en:ex:13a}
\gll šant-i-jak-pa-agôn\\
	until-\textsc{1sg-}complete-\textsc{poss.1sg-}struggle\\
\glt ‘until I complete my struggle’ (Psalm-book 93, 9)

\ex\label{06-en:ex:13b}
\gll šant-i-jôk 				m-pa-agôn\\
	until-\textsc{1sg-}complete \textsc{acc-}\textsc{poss.1sg-}struggle\\
\glt ‘until I complete my struggle’ (Psalm-book 149, 19)
\end{xlist}
\end{exe}

It should be noted that employing DOM with non-imperfective tenses is a relatively late phenomenon. There are no unequivocal examples of it from \ili{Demotic}, the language stage that immediately preceded \ili{Coptic}. Object marking in \ili{Demotic} is restricted to the imperfective tenses, so that the extension of DOM into the non-imperfective tenses must be considered as being only a little older than the oldest texts in \ili{Coptic}.

\subsection{Previous research on DOM with the non-imperfective tenses}
\label{06-en-sec:2-3}

To find out whether the two alternating constructions really are functionally equivalent, it is best to undertake a corpus-based statistical investigation. 
I have in two previous papers \citep{Engsheden2006Koptischen,Engsheden2008Differential} analysed the canonical gospels in \ili{Sahidic} \ili{Coptic} (the supra-regional dialect of the south). I argued that \ili{Coptic} can indeed be analysed as an example of a language with DOM, and that the selection of the \textit{n-}marked form was determined by both \isi{referentiality} (or specificity) and \isi{topicality} (\citealt[209–212]{Engsheden2006Koptischen}; \citealt[329–335]{Engsheden2008Differential}), while further possible factors included semantic features such as degree of \isi{affectedness} and causation. No evidence was found for \ili{Coptic} DOM being sensitive to \isi{animacy}.

A pertinent example for demonstrating that the marked form corresponds to the topic is found in the story of John the Baptist, whose head is what the story is about. Here, as elsewhere in this study, I mean by topic an aboutness topic,  \ie “the presupposed part of which pieces of information are conveyed” \citep[262]{Iemmolo2010Topicality}, operating on sentence level. I cite here my original \ili{Sahidic} example since the Gospel of Matthew is not preserved in Lycopolitan. Immediately before this passage, Salome has asked her stepfather the king to give her the head of John the Baptist (\ref{06-en:ex:14a}--\ref{06-en:ex:14g}): 

\begin{exe}
\ex \label{06-en:ex:14}
\begin{xlist}
\ex \label{06-en:ex:14a}
\gll a-f-lupei 			nci-p-rro 			emate\\
	\textsc{pst-}\textsc{3m.sg-}grieve \textsc{agt-def.m-}king much \\

\ex \label{06-en:ex:14b}
\gll etbe 		n-anauš=de 		mn 		n-et-nêj 		nmma-f\\
	because \textsc{def.pl-}oath.\textsc{pl}=\textsc{ptcl} with \textsc{def.pl-}\textsc{rel-}recline.\textsc{state} with-\textsc{3m.sg}\\
\ex \label{06-en:ex:14c}
\gll a-f-ouehsahne 		e-ti 		mmo-s 		na-s 		a-f-joou\\
	\textsc{pst-}\textsc{3m.sg-}command to-give \textsc{acc-}\textsc{3f.sg} to-\textsc{3f.sg} \textsc{pst-}\textsc{3m.sg-}send\\

\ex \label{06-en:ex:14d}
\gll a-f-fi 				n-t-ape 		n-iôhannês 	hm-pe-šteko\\
	\textsc{pst-}\textsc{3m.sg-}carry \textsc{acc-}\textsc{def.f-}head \textsc{gen-}John in-\textsc{def.m-}prison\\

\ex \label{06-en:ex:14e}
\gll a-u-eine 		mmo-s 		hijm 	p-pinaks\\
	\textsc{pst-}\textsc{3pl-}bring \textsc{acc-}\textsc{3f.sg} on \textsc{def.m-}platter\\

\ex \label{06-en:ex:14f}
\gll a-u-taa-s 				n-t-šeere 		šêm\\
	\textsc{pst-}\textsc{3pl-}give-3\textsc{f.sg} to-\textsc{def.f-}girl little\\
				
\ex\label{06-en:ex:14g}
\gll a-s-eine 			mmo-s 		n-tes-maau\\
	\textsc{pst-}\textsc{3f.sg-}bring \textsc{acc-}\textsc{3f.sg} to-\textsc{poss.3m.sg-}mother\\
\glt ‘The king grieved much. Because of the oaths and those who lay at table with him, he commanded to give it (sc. the head) to her, (and) he sent and beheaded John in the prison. It was brought on a platter and given to the little girl, (and) she brought it to her mother’ (Matthew 14: 9–11)

\end{xlist}
\end{exe}


The head is reactualised in~(\ref{06-en:ex:14c}) through an \textit{n-}marked pronoun. In~(\ref{06-en:ex:14d}) it is referred to by means of the repetition of the NP, and mentioned next in~(\ref{06-en:ex:14e}) with an \textit{n-}marked pronoun before it appears in~(\ref{06-en:ex:14g}), once more with an \textit{n-}marked pronoun. Note that the original \ili{Greek} text here does not have any object pronoun, so there is no influence from the original on the use of \textit{n-}marking. The omission of pronouns for the object in Ancient \ili{Greek} correlates to high \isi{topicality} \citep{Luraghi2003Greek}, which lends support to my analysis.

The identification of \isi{topicality} as a factor for the marking of the \isi{direct object} was made by observing pronominal anaphora, and how they contribute to the discourse coherence. It is more difficult to demonstrate a similar topical function for full NPs. As with extinct languages in general, it is often difficult to investigate discourse-pragmatic features because the competence of native speakers is replaced by a closed corpus of texts. It is however not surprising to discover that \isi{topicality} is a factor for DOM, because it has been recognised as such in a wide range of languages (\citealt[125–139]{Dalrympleetal2011Objects}; \citealt{Escandell-Vidal2009Differential,Iemmolo2010Topicality, Shainetal2010Synchrony}). Accordingly, I posit that the identification of \isi{topicality} as a factor in DOM, as suggested for \ili{Sahidic} \ili{Coptic} in my previous articles, is also relevant for Lycopolitan \ili{Coptic}.\footnote{Lycopolitan had a closer relationship to \ili{Sahidic} than to any other \ili{Coptic} dialect (\citealt{Funk1988Dialects}; \citealt[343]{Kasser2002Lyco-diospolitain}).}

Topicality relates to \isi{definiteness} in such a way that topics are mostly \isi{definite}, whereas it is less likely that indefinites appear as topics in discourse. It is often taken for granted that topics are specific, even though this is not a necessary condition, at least in Romance languages (\citealt[138–140]{Leonetti2013Plurals}). The idea that \isi{topicality} is \textit{the} trigger for DOM in the non-imperfective tenses is made problematic because marking varies in frequency depending on the semantic verb type, as will be illustrated below in \sectref{06-en-sec:5}. Topicality cannot account fully for the variation \textit{n}/${\emptyset}$, since there is no reason for some verbs to never be followed by a topical object. Of those verb types that disprefer \textit{n-}marking with non-imperfective tenses, simple zero-marked nouns must also be able to function as topics as one would not expect to encounter any lexical restrictions on verb, depending on the topical function of the object. A similar uneven distribution of the marker \textit{a} in \ili{Spanish} led Delbecque to state “if the discourse function were the \textit{raison d’être} of the prepositional frame, then, the preposition should be able to appear after any transitive verb” \citep[85]{Delbecque2002Construction}. Consequently, \isi{topicality} must work in conjunction with other factors in order to produce DOM in \ili{Coptic}.

Specificity also plays an important role for \textit{n-}marking in the non-imperfective tenses. In the example in~(\ref{06-en:ex:15}), the \isi{definite} article is used in a generic sense without reference to any specific individuals and, hence, there is no marking on the object:

\begin{exe}
\ex \label{06-en:ex:15}
\gll tehm-n-hêke 			mn-n-et-mokh 		mn-n-cale  mn 		n-blle\\
	invite-\textsc{def.pl-}poor and-\textsc{def.pl-rel-}afflicted and-\textsc{def.pl-}lame  and \textsc{def.pl-}blind \\
\glt ‘Invite the poor, the afflicted, the lame and the blind’ (Luke 14: 13)
\end{exe}

This example is from \ili{Sahidic}, but it is not difficult to find examples also in Lycopolitan \ili{Coptic} (see~\ref{06-en:ex:20}). 

\section{Data and methodology}
\label{06-en-sec:3}

Lycopolitan \ili{Coptic} \citep{Nagel1991Lycopolitan} was rediscovered at the beginning of the 20th century through the discovery of manuscripts from Middle Egypt that date to the 4th and 5th centuries AD. Lycopolitan can be divided into the following subdialects (\tabref{06-en-tab:2}), for which conventional labels are used (\cf \citealt{Kasser2006Aspe}: 418–420). 

\begin{table}
\begin{tabularx}{\textwidth}{lQ}
\lsptoprule

L4 & Manichaean texts from Medinet Madi (including Homilies; Kephalaia; Psalm-book)\footnotemark{}\\
L5 & Gospel of John (only Chapters 2–20)\\
L6 & Gnostic texts from Nag Hammadi; Acts of Paul \\
L9 & Manichaean texts from the Dakhla oasis\\
\lspbottomrule
\end{tabularx}
\caption{Lycopolitan subdialects}\label{06-en-tab:2}
\end{table}

\footnotetext{ I have used the older editions (Allberry, Böhlig, Polotsky), but these differ little with regard to objects from the still-incomplete re-edition in \textit{Corpus fontium manichaeorum}.} 

Orthographical/phonological criteria form the basis of these subdivisions, with less attention being paid to grammatical features. L4 is the most important subdialect by size, and makes up almost two-thirds of the entire Lycopolitan text corpus. It is expected to grow as there is still unpublished material. The main representative of L5 is largely derived from a \ili{Sahidic} \textit{Vorlage} of the Gospel of John \citep[195–207]{Askeland2012Gospel}. L9, known from texts discovered as late as in the 1980s, is the only subdialect to include original documentary material, whereas all preserved texts from the other subdialects seem to be translations from \ili{Greek}, even though a translation directly from \ili{Syriac} is sometimes invoked for some of the L4 texts. I have deliberately omitted two fragmentary leaves of the Pauline epistles, which have been classified as L3 \citep[419]{Kasser2006Aspe}. Not only is the dialectal identification controversial, but the texts offer too little in matters of object marking to warrant their inclusion in this study. It should be noted that the internal relationships of the Lycopolitan varieties and their background are still a matter of discussion. Some commentators have even questioned whether they should be classified as a discrete group among the \ili{Coptic} dialects (\citealt{Funk1985Subakhmimic}, \cf \citealt{Kasser2002Lyco-diospolitain}).

To undertake a quantitative analysis of this corpus, I have built a relational database that includes all instances of the \textit{n}/${\emptyset}$ variation from published Lycopolitan texts (with the exception of L3),\footnote{Lycopolitan texts make up only a tiny fraction of all existing \ili{Coptic} texts; only 2.5\% according to one estimate \citep[59]{Diebneretal1989Papyrus}.} which contains 7244 entries. The database contains only those syntactic contexts that would potentially allow DOM marking, so cases where the \textit{n-}marking is part of the valency, such as \textit{amahte} ‘to seize’ or loans from \ili{Greek} (see \ref{06-en:ex:3} above), are not included in the counts in \tabref{06-en-tab:3}–\tabref{06-en-tab:6}. Heavily restored passages have been omitted. The fact that the corpus comes from a limited period and is relatively large, including several longer texts, makes Lycopolitan appealing for the study of \ili{Coptic} DOM.

\tabref{06-en-tab:3} illustrates the difference between the number of attestations of \textit{n-}marked constructions in imperfective (present and imperfect) and non-imperfective tenses. As noted above, the \textit{n-}marked construction is obligatory when the verb is imperfective (\cf \sectref{06-en-sec:2-1}) with personal pronouns, proper nouns, and grammatically \isi{definite} nouns.\footnote{The few exceptions include cases such as those mentioned in respect to example~(\ref{06-en:ex:9}), involving light verbs, but likely also include simple errors in textual transmission.} The number of \textit{n-}marked objects \vs the total number of occurrences is given in parentheses. 

\begin{table}

\fittable{
\begin{tabular}{lr@{~}rr@{~}rr@{~}rr@{~}rr@{~}rr@{~}r}
\lsptoprule
& 
\multicolumn{2}{c}{Personal pronoun} &
\multicolumn{2}{c}{Proper noun} & 
\multicolumn{2}{c}{Poss. det. + NP} &
\multicolumn{2}{c}{Def. art. + NP} &
\multicolumn{2}{c}{Idf. art. + NP}\\
\midrule
Imperfective     & 100\% & (640/642) & {}-& {}-          & 97\%& (63/65)  & 96\%& (100/104)  & 96\%& (22/23)\\
Non-imperfective & 5\% & (206/3793)  & 54\%& (21/39) & 37\%& (162/442) & 36\%& (321/889) & 32\%& (89/282)\\
\lspbottomrule
\end{tabular}
}
\caption{ Percentage of marked objects in Lycopolitan in DOM-sensitive contexts (affirmative sentences only)}\label{06-en-tab:3}
\end{table}

The low figure for \textit{n-}marked non-imperfective pronouns is a result of the preference for direct affixation of the clitic pronoun to the verb. The \textit{n-}marking clearly dominates among proper nouns, whereas the zero-marked construction dominates among markers of \isi{definiteness}. The proportion of \textit{n-}marked constructions lessens slightly between \isi{definite} and \isi{indefinite} articles, but it is unclear whether any significance should be attributed to this. It is questionable whether these categories should even be arranged in a hierarchy.

When the data is broken down into Lycopolitan subdialects (\tabref{06-en-tab:4}) substantial differences become apparent, not only between the subdialects themselves, but also between texts and even within texts. For example, only 29\% of direct objects in the Manichaean \textit{Kephalaia} (L4) that are preceded by a determiner (\isi{indefinite}, \isi{definite}, or possessive) have \textit{n-}marking, whereas 75\% are so marked in the \textit{Tripartite Tractate} (L6). One should note that the mean for L9 is negatively influenced by the very low number of \textit{n-}marked objects in non-literary texts.

\begin{table}
%\begin{tabularx}{\textwidth}{XXXX{\hsize=.14\hsize}} 
\begin{tabular}{lrrrrrrr}
\lsptoprule
&
\multicolumn{2}{l}{Poss. det. + NP} &
\multicolumn{2}{l}{Def. art. + NP} &
\multicolumn{2}{l}{Idf. art.+ NP} & Mean\\
\midrule
L4 Homilies  & 37\% & (15/40) & 31\%& (20/65)   & 33\%& (5/15) & 34\%\\
L4 Kephalaia & 30\% & (32/107) & 35\%& (110/311) & 22\%& (25/114) & 29\%\\
L4 Psalm-book& 41\% & (68/166) & 31\%& (82/264)  & 30\%& (20/66) & 34\%\\
L5           & 29\%& (14/49) & 39\%& (22/56)     & 52\%& (13/25) & 40\%\\
L6           & 65\%& (26/40) & 67\%& (73/109)    & 74\%& (23/31) & 69\%\\
L9           & 18\%& (7/38) & 18\%& (18/98)       & 10\%& (3/29) & 15\%\\
\lspbottomrule
\end{tabular}
\caption{Frequency of \textit{n-}marked construction with NP determined by (in)definite article or possessive determiner in non-imperfective contexts in Lycopolitan subdialects (affirmative sentences)}\label{06-en-tab:4}
\end{table}


The reason for the differences in marking between the various subdialects is currently unclear, but see the discussion in \sectref{06-en-sec:6} for the possibility of a diachronic explanation.

\section{Semantic verb categories and DOM}
\label{06-en-sec:4}

From the discussion above, it is clear that no single factor determines DOM in \ili{Coptic}. Despite the general importance of \isi{definiteness} and \isi{topicality} in the non-imperfective domain, neither is able to account for the phenomenon, and one is left with a great many \textit{n-}marked direct objects for which an interpretation as a topic seems unwarranted. One way out of this dilemma is to extend the analysis to the immediate environment of the object, to inquire whether there was any lexical preference for one construction or the other, and whether such preferences had any semantic motivation. One should bear in mind that in the event of a disorderly spread of the \textit{n-}marked construction from the imperfective tenses into the non-imperfective tenses, there should be no significant differences in frequency of \textit{n-}marked \vs zero-marked constructions between the various verb types. As will be seen in \sectref{06-en-sec:5}, however, such differences are precisely what are observed in the corpus. The two constructions of the object are unevenly distributed and largely in agreement with the degree of \isi{affectedness} in correlation to the verb types, which demonstrates that DOM in \ili{Coptic} cannot be interpreted as a matter of style, as mentioned in \sectref{06-en-sec:2-2}. Similarly, in a discussion on object marking in \ili{Hindi} and \ili{Ostyak}, \citet[13]{Dalrympleetal2011Objects} reached the conclusion that the degree of \isi{affectedness} does not play a role in DOM in those languages because “[o]ptionality is observed with exactly the same subjects and exactly the same verbs.” Nor would one expect there to be lexical restrictions for the use of the marked construction in the optional marking of objects. Note that \isi{optionality} does not mean free variation, and it is doubtful whether any free variation involving case-marking \vs zero-form really exists (\cf \citealt[1615]{McGregor2010Optional}). \ili{Coptic} is an example of what has been termed “semantically enabled \isi{optionality}” \citep[505]{Kittila2005Optional}.

\largerpage
The degree to which the semantic relationship between the verb and its arguments can contribute to the understanding of DOM has been shown in several studies of \ili{Spanish}. It is generally held that \isi{animacy} in conjunction with specificity triggers the use of the prepositional accusative \textit{a} before the \isi{direct object} in standard European \ili{Spanish}. This explains the different object encoding in \ili{Spanish} sentences where an \isi{animate} \isi{definite} object is preceded by \textit{a}~(\ref{06-en:ex:16a}), and an \isi{inanimate} \isi{definite} object is not~(\ref{06-en:ex:16b}).

\begin{exe}
\ex 
\begin{xlist}
\ex \label{06-en:ex:16a}
\gll Vi 				a 		la 		mujer. \\
	see.\textsc{pst.1sg} \textsc{acc} \textsc{def.f} woman\\
\glt ‘I saw the woman.’

\ex\label{06-en:ex:16b}
\gll Vi 				la 		mesa. \\
	see.\textsc{pst.1sg} \textsc{def.f} table \\
\glt ‘I saw the table.’ \citep[41]{vonHeusingeretal2003Interaction}

\end{xlist}
\end{exe}

This traditional approach does not adequately explain the not-infrequent use of \textit{a} before \isi{inanimate} objects (\cf \citealt[51]{vonHeusingeretal2003Interaction}). One way to explain such irregularities is to employ a model that takes account of the whole predicate frame, including the relationship between subject and object \citep{Delbecque2002Construction,Garcia2007Inanimate}. Thus, in case of a dynamic verb that is used transitively, one can note a two-sided approach in which the agentive subject is conceived as reacting to the object, not only acting upon it \citep[103]{Delbecque2002Construction}. Marking \vs non-marking constructions represent different event structures. Differences in meaning can be approximated through translation, as illustrated by \textit{abandonar} ${\emptyset}$ DO ‘to desert, drop, give up’ \vs \textit{abandonar a} DO ‘to leave behind, abandon’ \citep[93]{Delbecque2002Construction}. 

In their now-classic study, Hopper and \citet{Hopperetal1980Transitivity} described \isi{transitivity} as a scalar concept consisting of different parameters that can be arranged from high to low. Thus, telic action characterises a transitive clause more than an \isi{atelic} action does, a volitional agent is more typical for \isi{transitivity} than a non-volitional one, affirmative sentences are more likely to be transitive than negative sentences, and so forth. Another component in the original model was ‘\isi{affectedness} of O’, which is characterised as total \vs partial \isi{affectedness}. The idea of \isi{transitivity} as a scalar concept was elaborated in a study by \citet{Tsunoda1985Remarks}, in which he arranged verbs in seven categories, and correlated these with case-frames from many unrelated languages and the degree of \isi{affectedness}. The hierarchy can be reformulated as a scale: effective action > perception > pursuit > knowledge > feeling > relationship > ability. Verbs of effective action can be further divided into subtypes, depending on whether the verb is resultative (‘to kill’, ‘to break’, ‘to bend’) or non-resultative (‘to hit’, ‘to shoot’, ‘to kick’, ‘to eat’). Perception verbs can likewise be divided into two subtypes, one more attained by verbal action, and the other less attained: ‘to see’, ‘to hear’, and ‘to find’ are considered more attained; ‘to listen’ and ‘to look’ as less attained. The model predicts that any category will be considered for object marking if any higher ranked (to the left in the scale) category is marked for \isi{transitivity}. It has been said that the hierarchy correlates with both control and \isi{affectedness} \citep{Testelec1998Transitivity}. These parameters were further studied by \citet{Malchukov2005Case}, who deconstructed Tsunoda’s original hierarchy in two dimensions. The first (sub\nobreakdash-)hierarchy notes decreased patienthood (break > hit > look for > search > go) and the second (sub\nobreakdash-)hierarchy decreasing agenthood (break > see/know > like/fear > freeze/be cold). Such divisions of verb types following semantic principles are of interest for the present paper because they provide points of comparison for testing, to see whether the statistical arrangement in~\tabref{06-en-tab:5} can be matched to semantic features. 

It is probable that one can correlate DOM with the \isi{verb-type hierarchy}. Some languages for which \isi{affectedness} has been claimed as an important factor for DOM are: \ili{Abui} \citep{Kratochvil2014Differential}, Ancient \ili{Greek} \citep{Riano2014Differential}, Djapu \citep[205]{Naess2007Prototypical}, Mongolian \citep[64–65]{Guntsetseg2008Differential}, and \ili{Spanish} \citep{vonHeusingeretal2003Interaction,vonHeusingeretal2011Affectedness}. 
The difference between the partitive and the accusative in \ili{Finnish} has also been explained in terms of partly affected \vs highly affected object (\citealt[262]{Hopperetal1980Transitivity}; \citealt[1203]{Naess2004What}; critically \citealt[381]{Iemmolo2013Symmetric}). 
A practical application of verb-type hierarchies in relation to the argument realisation strategies in DOM can be found in a study by \citet{vonHeusingeretal2007Differential}, in which the authors were able to show how the frequency of the prepositional accusative increased over time, from Old \ili{Spanish} up to modern \ili{Spanish}, based on an analysis of successive translations of the Bible. 
In that article, only three verbal prototypes were chosen for analysis: (a) to hurt/kill, (b) to see/find, (c) to put/take. The authors found it plausible that the lexical semantics of the verb were a driving force in the \isi{diachronic development} of \ili{Spanish} DOM, and they carried the analysis a step further in a subsequent study of twelve verbs, which represented the first five verb types from Tsunoda’s \isi{verb-type hierarchy} \citep{vonHeusingeretal2011Affectedness}. 
They discovered that the \ili{Spanish} data did not entirely agree with the hierarchy, inasmuch as verbs of feeling (\textit{querer} ‘to love’, \textit{temer} ‘to fear’), contrary to expectation, take a more transitive case-frame than verbs of perception, such as \textit{ver} ‘to see’ or \textit{mirar} ‘to look at’ \citep[612]{vonHeusingeretal2011Affectedness}. 
The competition for \isi{agentivity} between the participants in the event was mentioned as a possible cause for this \citep[613]{vonHeusingeretal2011Affectedness}.

In \ili{Coptic}, the object of perception verbs is typically introduced by a preposition, mostly \textit{a} (\ili{Sahidic}\textit{ e}), which also has a directional meaning ‘to’. This explains why verbs of perception are poorly represented in the material analysed in \sectref{06-en-sec:5-3}. However, verbs of feeling are lower ranked than verbs of perception, and take a zero-marked object. This disagrees with \citeauthor{Malchukov2005Case}’s two-dimensional model (\citeyear[81]{Malchukov2005Case}), 
which predicts that any intermediate verb-type category will display the same case-frame if both higher- and lower-ranked categories do so. Among the verbs of perception are \textit{neu} ‘to see’ (\eg  \ref{06-en:ex:17}) and \textit{sôtme} ‘to hear’.\footnote{A thorough study of the valency of this verb is found in \citet{Emmel2006Coptic}. The occasional alternation between \textit{e} and the usual construction with \textit{n}/\textit{mma} is different from the \textit{n-}marked \vs zero-marked construction, and is not pertinent to the present study.} There is no TAM-based split for perception verbs, as can be seen in a comparison between \REF{06-en:ex:17a}, which has a verb in the imperfective tense, and \REF{06-en:ex:17b}, which has a non-imperfective verb. 

\begin{exe}
\ex \label{06-en:ex:17}
\begin{xlist}
\ex \label{06-en:ex:17a}
\gll tn-neu ara-k tinou p-makarios\\
	\textsc{1pl-}see at-2\textsc{m.sg} now \textsc{def.m-}blessed\\
\glt ‘We see you now, o blessed one’ (Psalm-book 26, 12)

\ex\label{06-en:ex:17b}
\gll a-u-neu		 a-u-ouaine 	n-brre\\
	\textsc{pst-3pl-}see at-\textsc{indf-}light \textsc{gen-}new\\
\glt ‘They saw a new light’ (Psalm-book 196, 18)

\end{xlist}
\end{exe}

The preposition \textit{a} also occurs before the object with some speech verbs, such as \textit{smou}, ‘to bless’ or \textit{moute} ‘to call’. This correlation of argument realisation and verb type is so strong that is also used with loan verbs from \ili{Greek}, such as the mental verb \textit{pisteue} ‘to believe’.\footnote{In \ili{Greek}, the object takes the dative and so cannot be explained as a calque of the source language.}

\begin{exe}
\ex \label{06-en:ex:18}
\gll ari-pisteue 		a-p-ou{\ob}aein{\cb}\\
 do.\textsc{imp-}believe to-\textsc{def.m-}light\\
\glt ‘Believe in the light’ (John 12: 36)
\end{exe}

The government of perception verbs has historical roots in earlier phases of the Egyptian language. Indeed, the \ili{Coptic} verbs introducing their object with the preposition \textit{a} originally had diverse object marking strategies. It might be that the semantic development of late hieroglyphic \textit{nw} ‘to look at’ (\citealt{Depuydt1988Verben}: 6–7), which later became a neutral verb of vision in Lycopolitan \textit{neu} (\ili{Sahidic} \textit{nau}) ‘to see’, had operated on other perception verbs while retaining its government \textit{r} (older Egyptian) > \textit{a} (\ili{Coptic}).

\largerpage
\section{Analysis}
\label{06-en-sec:5}

In this section, I will review the ways that verb types relate to DOM in Lycopolitan \ili{Coptic}. The data, which is drawn from the texts discussed in \sectref{06-en-sec:3}, is presented as a simple frequency list of verbs with \textit{n-}marked and determined NPs, for which see \tabref{06-en-tab:5}. Very few proper nouns as objects with non-imperfective forms are attested, so conclusive results can rarely be obtained from them, so proper nouns have been omitted from the analysis and the discussion. For convenience, I use the \isi{verb-type hierarchy} proposed by Tsunoda (\sectref{06-en-sec:4}). Tsunoda’s division of verbs of effective action into two sub-categories, of resultative and non-resultative action, has also been expanded to include other categories, although for the purposes of this paper I refer mainly to change-of-state verbs. I also use Tsunoda’s classification of verbs as a heuristic tool without any attempt to refine the \isi{verb-type hierarchy} itself. Of course, it is an oversimplification to make verbs fit into a single category without paying attention to how the presence of other arguments in the sentence can lead to recategorisation. 

All the transitive verbs listed in~\tabref{06-en-tab:5} are attested at least ten times in affirmative sentences of the Lycopolitan \ili{Coptic} corpus. I define a verb as transitive (bi- or trivalent) if its object can be coded with the imperfective tenses in at least some contexts through \textit{n-}marking or zero-marking. The table therefore only lists those verbs that participate in the \textit{n}/${\emptyset}$ variation. As noted above, verbs of perception code their objects through the preposition \textit{a}, and are therefore omitted. The object NP is always preceded by one of the determiners (\isi{definite} article, \isi{indefinite} article, or possessive determiner). The lemmas are listed in the first column in their Lycopolitan form (which differs only slightly from \ili{Sahidic}). The second column shows the number of the morphological verb class according to a modern standard grammar (\citealt{Layton2000Coptic}: 153–157). Where no number is provided, it means that the verb should be considered irregular. The third column presents a standard translation. The fourth column contains the percentage of \textit{n-}marked constructions out of the total number of occurrences, and the ratio between \textit{n} \vs ${\emptyset}$ is shown in parentheses. The fifth column provides the subdialect from which the attestations come; the dominance of L4 (Manichaean texts) is evident. The final column lists the subsection in this paper where examples of the verb in question may be found. 

\begin{table}[b]
%\begin{tabularx}{\textwidth}{X{\hsize=.1\hsize}X{\hsize=.1\hsize}XXXX{\hsize=.1\hsize}}
\fittable{
\begin{tabular}{lclrrlr}
\lsptoprule
Verb & Class & Translation & \multicolumn{2}{c}{Percentage (ratio)} & Subdialect & Section\\
\midrule
\textit{tôhme} & 1 & to call & 92\% &  (11/12) & L4 & 5.1.\\
\textit{hôtbe} & 1 & to kill & 91\% &  (10/11) & L4 & 5.1.\\
\textit{t}\textit{\textsuperscript{h}}\textit{bio} & 5 & to humiliate (subdue) & 75\% &  (21/28) & L4, L9 & 5.1.\\
\textit{kôt} & 2 & to build & 70\% &  (7/10) & L4 & 5.1.\\
\textit{sôtp} & 1 & to choose & 67\% &  (6/9) & L4 & 5.2.\\
\textit{jôk} & 2 & to complete, finish & 64\% &  (16/25) & L4, L5, L6, L9 & 5.1.\\
\textit{pôrš} & 1 & to spread out & 60\% &  (6/10) & L4 & 5.2.\\
\textit{teko} & 5 & to destroy & 54\% &  (7/13) & L4, L6 & 5.1.\\
\textit{jpo} & 5 & to beget & 53\% &  (8/15) & L4, L5, L6 & 5.1.\\
\textit{smine} & 7 & to establish & 48\% &  (12/25) & L4, L6, L9 & 5.1.\\
\textit{tôbh} & 1 & to implore, pray & 44\% &  (8/18) & L4, L5, L6 & 5.2.\\
\textit{ji} & {}- & to take & 41\% &  (74/179) & L4, L5, L6, L9 & 5.2.\\
\textit{teho} & 5 & to reach, set up & 36\% &  (8/22) & L4, L6 & 5.2.\\
\textit{nouje} & 2 & to throw & 36\% &  (4/11) & L4, L5, L6 & 5.2.\\
\textit{tnnau} & 5 & to send & 28\% &  (6/21) & L4, L5, L6, L9 & 5.2.\\
\textit{saune} & {}- & to know & 27\% &  (5/21) & L4, L6, L9 & 5.3.\\
\textit{mour} & 2 & to bind & 22\% &  (4/18) & L 4, L6 & 5.2.\\
\textit{eire} & {}- & to do & 22\% &  (37/168) & L4, L5, L6, L9 & 5.2.\\
\textit{ti} & {}- & to give & 21\% &  (30/141) & L4, L5, L6, L9 & 5.2.\\
\textit{cine} & 7 & to find & 21\% &  (19/91) & L4, L5, L6, L9 & 5.3.\\
\textit{shei} & {}- & to write & 20\% &  (3/18) & L4, L9 & 5.1.\\
\textit{teouo} & 5 & to send, produce, utter & 20\% &  (11/58) & L4, L5, L6, L9 & 5.2.\\
\textit{kô} & 2 & to put, leave & 17\% &  (12/71) & L4, L5, L6, L9 & 5.2.\\
\textit{fi} & {}- & to bear, carry & 14\% &  (10/69) & L4, L6 & 5.2.\\
\textit{eine} & 7 & to bring & 3\% &  (1/37) & L4, L5, L6, L9 & 5.2.\\
\textit{šine} & 7 & to seek, ask & 0\% & (0/24) & L4 & 5.4.\\
\textit{meie} & {}- & to love & 0\%  & (0/20) & L4, L5, L6 & 5.5.\\
\lspbottomrule
\end{tabular}
}
\caption{Distribution of \textit{n-}marking with non-imperfective tenses and determined NPs for most common transitive verbs in Lycopolitan Coptic (affirmative sentence only)}\label{06-en-tab:5}
\end{table}

\tabref{06-en-tab:5} shows that the range of \textit{n-}marking with determined NPs that take non-im\-per\-fective tenses ranges from 0\% to 92\%, with a median of 36\%. Even a quick perusal reveals that the distribution of verbs shows agreement with semantically-defined verb types, ordered according to the \isi{affectedness} hierarchy, especially at the upper and lower ends. There are considerable differences between the individual verbs with some (\eg \textit{hôtbe} ‘to kill’) predominately having the \textit{n-}marked construction, while others (\eg \textit{meie} ‘to love’) exclusively take the zero-marked construction. In the imperfective tenses, all the listed verbs must take the \textit{n-}marked construction with determined NPs (see \sectref{06-en-sec:2-1}). Morphology does not trigger the selection of the object construction. Most importantly, it contradicts the idea, first expressed by \citet[165]{Steindorff1894Koptische}, that the zero-marked construction is typical for the category of the fifth class, which contains etymological causatives. The Lycopolitan data show that verbs belonging to this class are subject to the effect of lexical semantics to the same degree as other verb types.

\subsection{Verbs of effective action (resultative)}
\label{06-en-sec:5-1}

There is a strong correlation between marking and verbs of effective action. The subject is highly agentive and volitional, exercising full control over the action. The object is fully affected and undergoes a change of state. Among them one finds \textit{hôtbe} ‘to kill’, \textit{jôk} ‘to complete’, and \textit{teko} ‘to destroy’, but verbs of creation are included in this group as well. The median for \textit{n-}marked verbs of effective action is 64\%, which means that resultative verbs of effective action are predominantly \textit{n-}marked. A representative example of \textit{n-}marking with a verb of effective action is~(\ref{06-en:ex:19}):

\begin{exe}
\ex \label{06-en:ex:19}
\gll a-u-hôtbe 		n-n-sabeue\\
	\textsc{pst-}\textsc{3pl-}kill \textsc{acc-}\textsc{def.pl-}wise.\textsc{pl}\\
\glt ‘They killed the wise men’ (Homilies 80, 30)
\end{exe}

It seems significant that the only example of a zero-marked object with \textit{hôtbe} ‘to kill’, which is quoted in~(\ref{06-en:ex:20}), has a generic referent. As noted at the end of \sectref{06-en-sec:2}, non-specific objects take the zero-marked construction.

\begin{exe}
\ex \label{06-en:ex:20}
\gll n-t-he 					n-hn-rôme 		e-u-na-hatbe-hn-moui\\
	\textsc{adv-}\textsc{def.f-}manner \textsc{gen-indf.pl-}man \textsc{circ-3pl-fut-}kill-\textsc{indf.pl-}lion\\
\glt ‘in the manner of men who are about to kill lions’ (Psalm-book 205, 30) 
\end{exe}

At first sight, the verb\textit{ tôhme} ‘to call’, which scores highest in selecting the \textit{n-}marking in~\tabref{06-en-tab:5}, does not seem to be an ideal candidate for demonstrating the relevance of \isi{affectedness}; under normal circumstances the object of ‘to call’ is not affected by the verb action. However, Manichaean cosmogony provides a likely explanation for this deviance from the expected pattern. The call directed at the various Aeons is a metaphor for them being called into existence as a counter-measure against the approaching advent of Evil. This creational aspect can be highlighted by the translation ‘call forth’ (\cf \citealt{Kasser1991Premier}). 

\begin{exe}
\ex \label{06-en:ex:21}
\gll pa-iôt 	p-ouaine 	et-talêl... a-f-tôhme 	n-n-aiôn 	m-p-ouaine...
	a-f-tôhme 	n-n-aiôn 	n-t-eirênê... 	a-f-tôhme 	n-n-aiôn 	m-p-scraht... 	se-[h]atp 		têr-ou 	se-ti-mete\\
	\textsc{poss.1sg-}father \textsc{def.m-}light \textsc{rel-}be.glad \textsc{pst-3m.sg-}call \textsc{acc-def.pl-}aeon \textsc{gen-}\textsc{def.f-}light \textsc{pst-3m.sg-}call \textsc{acc-def.pl-}aeon \textsc{gen-}\textsc{def.f-}peace \textsc{pst-3m.sg-}call \textsc{acc-def.pl-}aeon \textsc{gen-}\textsc{def.m-}rest  \textsc{3pl-}be.in.peace.\textsc{state} all-\textsc{3pl} \textsc{3pl-}give-satisfaction\\
\glt ‘My father, the glad Light... He called forth the Aeons of the Light into existence... He called forth the Aeons of the Peace... He called forth the Aeons of the Rest... They are all in peace and satisfied’ (Psalm-book 203, 3–23) 
\end{exe}

Note that the sequence in~(\ref{06-en:ex:21}) contains \isi{inanimate} objects, whereas animates are normally expected with this verb, but it is unclear whether \isi{animacy} has any significance for the selection of the object construction. This cannot easily be resolved because the boundaries between \isi{animate} and \isi{inanimate} were not sharp in the Manichaean universe. One might also wish to consider the topical status of the objects, which is was announced in the title of the psalm: “Concerning the Father and all his Aeons and the Stirring of the Enemy.” The Aeons appear again in the discourse as a plural subject in the last line quoted above.

Related to this are examples of \textit{n-}marking with the verb \textit{t}\textit{\textsuperscript{h}}\textit{bio}, the usual translation of which is ‘to humiliate’, but are better translated in examples such as~(\ref{06-en:ex:22}) as ‘to subdue’ (\cf 1a) when followed by an \textit{n-}marked NP, to signal a higher degree of \isi{affectedness}. 

\begin{exe}
\ex \label{06-en:ex:22}
\gll a-u-tʰbio 			m-p-keke\\
	\textsc{pst-3pl-}subdue \textsc{acc-def.m-}darkness\\
\glt ‘They have subdued the darkness’ (Kephalaia 35, 5)
\end{exe}

This does not prevent zero-marked constructions from appearing with similar meanings~(\ref{06-en:ex:23}): one cannot readily identify the two \ili{Coptic} predicate frames with different verbs in translation, as one might by using Delbecque’s model for \ili{Spanish} (see \sectref{06-en-sec:4}). 

\begin{exe}
\ex \label{06-en:ex:23}
\gll n-t-he 				je 		a-p-šarp 			n-rôme 	tʰbio-p-keke\\
	\textsc{adv-}\textsc{def.f-}way \textsc{ptcl} \textsc{pst-def.m-}first \textsc{gen-}man subdue-\textsc{def.m-}darkness\\
\glt ‘in the way the first man subdued the darkness’ (Kephalaia 49, 4)
\end{exe}

In some cases, such as in~(\ref{06-en:ex:24}), not even the combination of a \isi{definite} reference and an affected object produces \textit{n-}marking. The object \textit{hôb} (lit. ‘thing’), which does not recur in the discourse, can be regarded as synonymous to the head of an \isi{indefinite} relative clause “that which [lit. the thing] you have given to me to do.” What the work consisted of is not explained. Whether discourse factors play a role is unclear. The object is in this case non-topical.

\begin{exe}
\ex \label{06-en:ex:24}
\gll a-ei-jak-p-hôb 				abal  nt-a-k-tee-f 					nêi 		a-tr-a-ee-f\\
     \textsc{pst-}1\textsc{sg-}finish-\textsc{def.m-}thing \textsc{ptcl} \textsc{rel-pst-2msg-}give-\textsc{3m.sg} to.\textsc{1sg} to-\textsc{caus-}\textsc{1sg-}do\textsc{-3m.sg}\\
\glt ‘I have finished the thing you have given me to do’ (John 17: 4)
\end{exe}

Even though the general trend is clear, there are significant differences between the verbs in this group that require explanation. It is difficult to see any reason why \textit{hôtbe} ‘to kill’ and \textit{teko} ‘to destroy’ would take 91\% and 54\% of \textit{n-}marked objects respectively. 

\subsection{Verbs of effective action (non-resultative)}
\label{06-en-sec:5-2}

Many of the verbs listed in~\tabref{06-en-tab:5} are action verbs where the actor retains control of the action expressed by the verb. The object is little affected, but may undergo limited physical movement (\eg ‘to spread out’, ‘to take’, ‘to throw’, ‘to set up’, ‘to bring’). The median percentage of \textit{n-}marking in this group is 25\%, so \textit{n-}marking is clearly the exception. This group could be further divided into subcategories on semantic grounds, but this would obscure the relevant point, which is the overall dependency of object marking on the \isi{affectedness} hierarchy. 

It can be difficult to identify where the difference between the \textit{n-}marked construction \vs the zero-marked construction lies. These difficulties are illustrated in (\ref{06-en:ex:25a}--\ref{06-en:ex:25b}). 

\begin{exe}
\ex 
\begin{xlist}
\ex \label{06-en:ex:25a}
\gll a-f-nouje 			n-hn-jôr{\ob}me{\cb} 			ha			te-f-staurôsis\\
	\textsc{pst-3m.sg-}throw \textsc{acc-indf.pl-}allusion concerning \textsc{poss.3m.sg-}crucifixion\\%TODO@SN fix overflow here
\glt ‘He made allusions to his crucifixion’ (Homilies 44, 17)

\ex\label{06-en:ex:25b}
\gll a-u-nouj-ou-halu{\ob}sis{\cb} 	a-pe-f-mout\\
	\textsc{pst-3pl-}throw-\textsc{indf-}chain to-\textsc{poss.3m.sg-}neck\\
\glt ‘They put a chain around his neck’ (Homilies 48, 21)

\end{xlist}
\end{exe}


Both examples are from the same text, which tells how Mani, the founder of the religion named after him, suffered martyrdom in AD 277. In the first example \REF{06-en:ex:25a}, the \textit{n-}marked object (‘allusions’) is \isi{inanimate} and \isi{indefinite}, and does not seem to be more affected than the object in  \REF{06-en:ex:25b}. Pragmatic factors may be relevant, because the ‘allusions’ in  \REF{06-en:ex:25a} may be a reference to Mani’s comments on his martyrdom in the following line. The marking would then indicate that the \isi{indefinite} noun ‘allusions’ should be regarded as specific, serving as a referential anchor (\cf \citealt{vonHeusinger2002Specificity}). In  \REF{06-en:ex:25b}, on the other hand, the ‘chain’ does not appear again in the following discourse. To understand the importance of extent discourse-informational factors for \textit{n}/${\emptyset}$ variation, one would need to explore discourse persistence in a systematic fashion, which would require time-consuming manual processing. Due to the non-narrative character of most texts in the corpus, there is little referential persistence with regard to the \isi{direct object}, so that referential tracking becomes difficult.

Within this group are a few verbs for which the subject exercises full control over the action, and that have a non-affected object, such as \textit{sôtp} ‘to choose’, \textit{tôbh} ‘to implore’, and \textit{teouo} ‘to utter’. In this context, the verb \textit{tôbh} ‘to implore’ has different case-frames depending on the \isi{animacy} of the object. On the one hand, when the object is \isi{inanimate}, such as in~(\ref{06-en:ex:26}), \textit{n-}marking is clearly preferred. On the other hand, zero-marking is used with \isi{animate} objects, as in~(\ref{06-en:ex:27}), in a way that is reminiscent of other speech verbs (\cf \textit{teouo} ‘to utter’ and \textit{šine} ‘to ask’).

\begin{exe}
\ex \label{06-en:ex:26}
\gll e-u-a-tôbh 		m-p-kô 			abal 	n-n{\ob}ou-nabe{\cb} n-tot-f 				m-p-noute\\
	\textsc{fut-3pl-fut-}ask \textsc{acc-def.m-}give away \textsc{gen-poss.3pl-}sin from-hand-\textsc{3m.sg} \textsc{gen-def.m-}god\\
\glt ‘They will ask for the forgiveness of their sins from God’ (Homilies 23, 8)
\end{exe}

\begin{exe}
\ex \label{06-en:ex:27}
\gll a-k-tabh-pek-iôt\\
	\textsc{pst-2msg-}ask-\textsc{poss.2msg-}father\\
\glt ‘You have asked your father’ (Psalm-book 44, 11)
\end{exe}

It is also worth mentioning the verb \textit{eire} ‘to do’. It frequently occurs as a light verb in a few common expressions, such as \textit{r-p-meue} ‘to remember’ (lit.\,‘to do the remembrance’), where the incorporation of the object underlines its low referential status (see \ref{06-en:ex:9}). The compound is understood synchronically as a verb, and can even be followed by an \textit{n-}marked object, as in \REF{06-en:ex:28}. 

\begin{exe}
\ex \label{06-en:ex:28}
\gll e-s-na-r-p-meue 		mma-ou 	nci-ti-ekklêsia\\
	\textsc{foc-3f.sg-fut-}do-\textsc{def}.\textsc{m-}memory \textsc{acc-3pl} \textsc{agt def.f-}church\\
\glt ‘the Church will remember them’ (Tripartite Tractate 135, 25)
\end{exe}

The presence of an object changes the \isi{telicity} of the verb, which is another factor for high \isi{transitivity} according to \citet{Hopperetal1980Transitivity}. One may contrast examples with a nuance ‘to do s.o.’s wish’, \ie  to fulfill it~(\ref{06-en:ex:29}), against other examples where the activity is unbounded, as is the case of ‘spending (lit. doing) time’~(\ref{06-en:ex:30}).

\begin{exe}
\ex \label{06-en:ex:29}
\gll hn 	ou-špnšôpe 	mn-ou-{\ob}c{\cb}lam  ša-u-eire 		m-p-ôk 		n-hêt 		m-pou-jais\\
	in \textsc{indf-}sudden and-\textsc{indf-}rapid \textsc{aor-}3\textsc{pl-}do \textsc{acc-def.m-}delight \textsc{gen-}heart \textsc{gen-poss.3pl-}lord\\
\glt  ‘Suddenly and rapidly they fulfill the desire of their lord’ (Kephalaia 51, 16–17)
\end{exe}

\begin{exe}
\ex \label{06-en:ex:30}
\gll alla 	a-u-r-pou-kairos 		têr-f 		e-u-šôpe 		hn-〈ou-thli〉psis\\
	but \textsc{pst-}3\textsc{pl-}do-\textsc{poss.3pl-}time all-\textsc{3m.sg} \textsc{circ-3pl-}become in-\textsc{indf-}distress\\
\glt ‘But they spent all their time falling in distress’ (Kephalaia 150, 29)
\end{exe}

Although the general trend between resultative and non-resultative action seems clear, there are considerable differences between the Lycopolitan subdialects regarding the frequency of the \textit{n-}marked construction, as seen for a selection of verbs in~\tabref{06-en-tab:6}.

\begin{table}
\begin{tabular}{l r@{\,}r r@{\,}r r@{\,}r r@{\,}r}
\lsptoprule  
		  & 
\multicolumn{2}{c}{L4} &
\multicolumn{2}{c}{L5} &
\multicolumn{2}{c}{L6} &
\multicolumn{2}{c}{L9}\\
\midrule
\textit{ji} ‘take’ & 30\%& (29/98) & 72\%& (13/18) & 88\%& (29/33) & 10\%& (3/29)\\
\textit{eire} ‘do’ & 23\%& (26/115) & 50\%& (4/8) & 25\%& (6/24) & 5\%& (1/18)\\
\textit{ti} ‘give’ & 10\%& (9/92) & 50\%& (2/4) & 82\%& (18/22) & 5\%& (1/21)\\
\textit{kô} ‘put’ & 6\%& (3/53) & 44\%& (4/9) & 83\%& (5/6) & 0\%& (0/3)\\
\textit{fi} ‘carry’ & 7\%& (3/46) & 10\%& (1/10) & 86\%& (6/7) & 0\%& (0/5)\\
\lspbottomrule
\end{tabular}
\caption{Subdialectal variation in \textit{n-}marking for a selection of verbs}\label{06-en-tab:6}
\end{table}

This reveals very different proportions of \textit{n-}marking in the various subdialects of Lycopolitan \ili{Coptic}. It is notable that \textit{n-}marking is virtually non-existent in L9, especially in non-literary texts. It is difficult to tell what this signifies. There are also remarkably low \textit{n-}marking percentages for several verbs in L4. In this subdialect, zero-marking appears to constitute the normal transitive construction for verbs of non-resultative action. It seems as if \textit{n-}marking is more common in L5, but the totals are rather low for that subdialect. By contrast, the percentage of \textit{n-}marking is high in L6, with no real distinction in treatment between resultative (as in \sectref{06-en-sec:5-1}) and non-resultative action verbs. Thus, \textit{n-}marking is clearly the norm in the L6 dataset, the only clear deviation from the trend being \textit{eire} ‘to do’, partly due to its frequent use as a light verb. If one omits objects as complements in complex predicates, which are zero-marked (as in 28), one still does not arrive at more than 46\% \textit{n-}marking with \textit{eire}. 

\largerpage
\subsection{Verbs of perception/cognition}
\label{06-en-sec:5-3}

As stated above (end of \sectref{06-en-sec:4}), the object of perception verbs is mostly introduced by the preposition \textit{a}. This explains why \tabref{06-en-tab:5} only lists two examples of perception verbs participating in the \textit{n}/${\emptyset}$ variation (\textit{saune} ‘to know’, \textit{cine} ‘to find’). The agent exerts no control on the action and the object is unaffected.

The behaviour of \textit{saune}, which has 27\% \textit{n-}marking, is unique to Lycopolitan: I do not know any examples of a \textit{n-}marked object together with this verb in any dialect other than Lycopolitan. In the imperfective~(\ref{06-en:ex:31}) the stem is \textit{saune} (\ili{Sahidic} \textit{sooun}):

\begin{exe}
\ex \label{06-en:ex:31}
\gll {\ob}e{\cb}peidê 	f-saune 		n-t-gnôsis\\
	since \textsc{3m.sg-}know \textsc{acc-def.f-}gnosis\\
\glt ‘Since he knows the gnosis’ (Kephalaia 233, 26) 
\end{exe}

The verb \textit{saune} is evidently a secondary form, having developed out of a verb form often called the stative, which expresses a resultative state \citep[163]{Peust2013Etymologie}. The morphology of this verb is quite complex and presents many variants (overview in \citealt{Vycichl1983Dictionnaire}: 202). The verb \textit{saune} itself, like similar verbs expressing knowledge in earlier Egyptian dialects, was originally an inchoative mental verb, not a verb of state, that had the basic meaning ‘get to know’. It is only through the spread of the stative form that the verb evolved into a verb of state, similar to one meaning of the English ‘to know’. In dialects other than Lycopolitan, \textit{saune} (and predictable variants thereof) is used indiscriminately with imperfective and non-imperfective tenses. With non-imperfective TAM forms, NPs as direct objects are almost invariably zero-marked (and thus different from Lycopolitan). Originally, the stem may have been \textit{souôn}/\textit{snouôn}, and it appears as such in Lycopolitan with non-imperfective tenses, with either \textit{n-}marking~(\ref{06-en:ex:32}) or zero-marking~(\ref{06-en:ex:33}).\footnote{\textit{S}\textit{aune} is also possible with a non-imperfective, when there is no object.} In other dialects, this allomorph is used with zero-marking.

\begin{exe}
\ex \label{06-en:ex:32}
\gll a-i-snouôn 		n-ta-psuchê\\
	\textsc{pst-}1\textsc{sg-}know \textsc{acc-poss.1sg-}soul\\
\glt ‘I have known my soul’ (Psalm-book 56, 26)
\end{exe}

\begin{exe}
\ex \label{06-en:ex:33}
\gll tote 	e-u-šan-souôn-p-iôt\\
	then \textsc{cond-}3\textsc{pl-}\textsc{cond-}know-\textsc{def.m-}father\\
\glt ‘Then if they know the father’ (Gospel of Truth 24, 31)
\end{exe}

A comparatively high percentage (21\%) of \textit{n-}marked objects are found with \textit{cine} ‘to find’. In Tsunoda’s original model, the verb ‘to find’ was listed among perception verbs based on the argument realisation of ‘to find’ in, for example, North Caucasian languages (\cf \citealt{Ganenkov2006Experiencer}), though ‘to find’ can also be a verb of perception in \ili{English} \citep[423]{Simon-Vandenbergen1999Thinking}. It is not easy to see which semantic reason could favour either one object marking strategy over another for this verb. Compare the following, where the objects are near synonyms expanded through a genitive adjunct, and both times have a verb in the past tense.

\begin{exe}
\ex 
\begin{xlist}
\ex \label{06-en:ex:34a}
\gll {\ob}...{\cb} a-f-cine 		n-t-šbiô 				m-pf-his{\ob}e{\cb}\\
  { } 	\textsc{pst-3m.sg-}find \textsc{acc-def}.\textsc{f-}requital \textsc{gen-poss.3m.sg-}toil\\
\glt ‘... he has found the requital for his toil’ (Homilies 83, 19)

\ex\label{06-en:ex:34b}
\gll je 	a-i-cn-p-beke 				m-pa-hise \\
	for \textsc{pst-}I-find-\textsc{def.m-}reward \textsc{gen-}\textsc{poss}.1\textsc{sg-}toil\\
\glt ‘for I have found the reward of my toil’ (Psalm-book 93, 30)

\end{xlist}
\end{exe}

The fragmentary context of~(\ref{06-en:ex:34a}) makes it impossible to observe anaphoric behaviour. The selection seems to be truly optional. 

Four instances where the object of \textit{cine} is \textit{n-}marked can be interpreted as being topical. This interpretation follows from the repetition of the object each time in short, explanatory nominal sentences.

\begin{exe}
\ex \label{06-en:ex:35}
\gll a-i-cine 		n-t-mrô 				t-mrô 	te 		t-entolê... a-i-cine 		n-n-ejêu 			n-ejêu 	ne 		p-rê 		mn-p-ooh a-i-cine 		n-ou-hêu 		e-mn-ase 		{\ob}nhêt-f{\cb}\\
	\textsc{pst-}\textsc{1sg-}find \textsc{acc-def.f-}harbour \textsc{def.f-}harbour \textsc{cop} \textsc{def.f-}command \textsc{pst-}\textsc{1sg-}find \textsc{acc-def.pl-}ship \textsc{def.pl-}ship \textsc{cop.pl} \textsc{def.m-}sun and-\textsc{def.m-}moon \textsc{pst-}\textsc{1sg-}find \textsc{acc-indf-}gain \textsc{circ-neg-}loss in-\textsc{3m.sg}\\
\glt ‘I found the harbour. The harbour is the Commandment... I found the ships. The ships are the sun and the moon... I found a gain wherein there is no loss...’ (Psalm-book 168, 1–9)
\end{exe}

There is a further consideration, because the second consonant in \textit{cine} is identical to the \isi{object marker} \textit{n}, and this could play a role for the common use of zero-marking. It is true that phonology can sometimes override semantic-pragmatic parameters, as happens sometimes with the \ili{Spanish} \textit{a} \citep[108]{Kliffer1995Kinesis}, in order to promote the zero-marked form. But the percentage of attestations for the \textit{n-}marked construction differs between \textit{cine} (21\%) and \textit{šine} (0\%), which has the same rhyming pattern, so the idea of phonological influence is unlikely. 

\subsection{Verbs of pursuit}
\label{06-en-sec:5-4}

In this category the subject has a low degree of control and the object is unaffected. The list comprises a single verb of pursuit, \textit{š}\textit{ine} ‘to ask’, which here is zero-marked~(\ref{06-en:ex:36}). In other dialects (Akhmimic, Mesokemic), where the percentage of \textit{n-}marking is higher, the object of this verb can be \textit{n-}marked.

\begin{exe}
\ex \label{06-en:ex:36}
\gll a-ke-mathêtês 			šn-p-apostolos \\
	\textsc{pst-}other-disciple ask-\textsc{def.m-}apostle\\
\glt ‘Another disciple asked the apostle’ (Kephalaia 208, 15)
\end{exe}


\subsection{Verbs of feeling}
\label{06-en-sec:5-5}

Here the subject lacks control, the object is not affected, and the verb expresses a state. The verb \textit{meie} (\ili{Sahidic} \textit{me}) ‘to love’ is incompatible with \textit{n-}marking in the non-imperfec\-tive tenses, a feature that appears to be shared by all \ili{Coptic} dialects.\footnote{I know of only one possible example of this verb with a marked \isi{direct object}: \textit{p-e-}\textit{š}\textit{a-u-ka-ou-koui de na-f ebol e-}\textit{ša-f-me n-ou-koui} ‘The one to whom little is forgiven, he loves only a little’ (\ili{Sahidic} Luke 7: 57), in which the object is focalised by means of the preposition. It therefore does not seem to be an example of a differential context.} See~(\ref{06-en:ex:4}) and~(\ref{06-en:ex:8}) for examples with the imperfective. Its antonym \textit{maste} ‘to hate’, not included in the list above, also avoids \textit{n-}marking in the non-imperfective.

\begin{exe}
\ex \label{06-en:ex:37}
\gll a-u-mrre-p-eau				gar 	n-n-rôme\\
	\textsc{pst-3pl-}love-\textsc{def.m-}glory for \textsc{gen-def.pl-}man\\
\glt ‘for they loved the glory of men’ (John 12: 43)
\end{exe}

In this context, it is appropriate to consider \textit{ouôš} ‘to want’, ‘to wish’. As mentioned at the end of \sectref{06-en-sec:2-1}, this verb is the sole exception to the rule that \isi{definite} objects must be \textit{n-}marked with the imperfective tenses. A problem for the historical explanation referred to earlier is that the difference between \textit{wḫȝ n} O ‘to look for’ and \textit{wḫȝ} O ‘to wish’ is found only in \ili{Demotic} \citep{Depuydt1993Sake}, meaning that it had disappeared before the spread of \textit{n-}marking into the non-imperfective. Once the former expression had disappeared, it would have been possible for \textit{ouôš} to have taken part in the expansion of object marking. A semantic analysis based on \isi{affectedness} offers an alternative, functional explanation, which holds true synchronically. Thus, semantics may have blocked \textit{ouôš} from acquiring object marking in the non-imperfective, and it may have had a similar effect on the imperfective.

\section{Discussion}
\label{06-en-sec:6}

The foregoing section lends support to the idea that \ili{Coptic} DOM can be successfully analysed, based upon a view of \isi{transitivity} as a scalar concept involving several semantic features \citep{Hopperetal1980Transitivity}. In \ili{Coptic}, \isi{definiteness}, specificity, \isi{topicality}, and \isi{affectedness} seem to act together to create a high degree of \isi{transitivity}, and interact in triggering \textit{n-}marking. How the various factors contributing to DOM in \ili{Coptic} relate to each other is open to question. The study of the development of DOM in \ili{Coptic} is still in its formative stages, and the following remarks are therefore preliminary, and have no immediate bearing on \ili{Coptic} dialects other than Lycopolitan. 

Definiteness is a factor for object marking with all TAM forms, although in the non-imperfective tenses it leads only to optional DOM (\cf \sectref{06-en-sec:4}). I posit that marking spread across \isi{definite} NPs more-or-less simultaneously, and not stepwise from one \isi{definite} category to the next, because the difference in percentages of \textit{n-}marked nouns seems negligible when compared to determined NPs (see \tabref{06-en-tab:3}). This last fact speaks against a spread along the \isi{definiteness} hierarchy scale as claimed, \textit{inter alia}, for the \ili{Spanish} prepositional accusative \citep{Aissen2003Differential}. The topical status of the marked objects may have been a secondary development, which followed from semantic \isi{definiteness}. A topical function is best visible in the phonologically heavier form \textit{mma}, which was used for pronouns (see \ref{06-en:ex:14}) that are semantically \isi{definite}. The \textit{n-}marked object would receive separate stress from the verb, and thus in an iconic way reflect the saliency of the object. If so, \textit{n-}marking might be described as a topicalisation strategy through right-dislocation, even though the right periphery is not recognised as a position for topics in \ili{Coptic}. It is, however, difficult to identify \isi{topicality} in NPs as objects by studying referential coherence, because the non-narrative character of most Lycopolitan texts is such that objects, once mentioned, do not commonly persist over several sentences, and their behaviour cannot be observed. Substitution or question tests for \isi{topicality} are difficult to apply without a native speaker’s intuition. It can be expected that the effect of \isi{topicality} for overruling the expected selection of \textit{n} \vs ${\emptyset}$ would be greatest for non-effective action verbs (see \sectref{06-en-sec:5-2}), because this is the only group in which one notes significant differences between the subdialects (see \tabref{06-en-tab:6}). These differences, ultimately affecting the percentage and their placement in the list in~\tabref{06-en-tab:5}, indicate that not all factors operated in an identical manner in all subdialects.

The frequency list of Lycopolitan transitive verbs and their construction with non-imperfective tenses, in~\tabref{06-en-tab:5}, shows that object marking was generally in agreement with Tsunoda’s \isi{affectedness} hierarchy, particularly at the upper and lower ends. Over 90\% of examples of a typical action verb with an affected object (\sectref{06-en-sec:5-1}), such as ‘to kill’, take\textit{ n-}marked objects, while a typical verb of feeling (\sectref{06-en-sec:5-5}), ‘to love’, takes 0\%. The more the object is affected, the more likely it is to receive \textit{n-}marking. It is more difficult to assess the large group of non-effective action verbs (\sectref{06-en-sec:5-2}). 

The correlation between marking, which is an innovation of Egyptian-\ili{Coptic} language history, and the \isi{affectedness} hierarchy with the non-imperfective, must reflect synchronic priorities. It is conceivable, a priori, that the marking spread randomly from the imperfective to the non-imperfective without any functional basis. However, the difference in marking frequency by verb type suggests that this was not the case. If it was, one would be at a loss to explain why some verbs do not have the marker with the non-imperfective tenses, but uniformly do with the imperfective ones. Note that my interpretation of Lycopolitan DOM is a counter-example against the generalisation that asymmetric DOM systems are not regulated by \isi{affectedness} \citep{Iemmolo2013Symmetric}. The TAM-based split that has differing rules for the imperfective and non-imperfective tenses under similar syntactic conditions (obligatory \vs pragmatic-semantically determined DOM) already speaks against the general validity of this hypothesis. 

At first glance, there seems to be no particular information-structural reason why the Manichaean texts (L4) should have far fewer \textit{n-}marked direct objects than the Gnostic texts (L6). The difference between L4 and L6 is significant, as indicated by a chi-square test with Yates’ correction that yields a statistical significance at \textit{p} < 0.001. Since the \textit{n-}marked construction was an innovation, one may feel inclined to assume that the difference between the percentages in L4 and L6 would reflect an ongoing spread of the marker into the non-imperfective tenses. This would, in principle, mean that texts with a low incidence of the \textit{n-}marked construction are from an older stage of language development, and texts with a high incidence of the \textit{n-}marked construction are from a more recent stage. It is plausible to conceive that the use of \textit{n} as a topic-marker was extended to non-topical contexts, so that more and more determined and specific expressions would ultimately receive the marker within the non-imperfective domain (\cf \citealt{Dalrympleetal2011Objects}: 208). Affectedness may have been the path along which the construction spread. It might be argued, on the basis of the more frequent use of \textit{n-}marking in L6, that the role of \isi{affectedness} was then gradually diminished as \isi{definiteness} alone, irrespective of any eventual topical role of the object, would often trigger marking. This seems to move towards a clearer separation of a group of verbs (action verbs) that favoured \textit{n-}marking from verbs of feeling that favoured zero-marking, indicating a lexically-based selection of object construction (\cf \citealt{Iemmolo2013Symmetric}: 390). 

It is difficult to offer support for such an assumed diachronic scenario, or to refute it through independent criteria, since the dating of manuscripts, let alone of the texts themselves, is very insecure. 
But diachronic studies on DOM in \ili{Spanish} show a similar span in object marking as that observed between the Lycopolitan subdialects, and these appear to have evolved over two centuries. 
Thus, in \textit{El Cantar de mio Cid} from the 13th century, only 36\% of \isi{animate} direct objects are overtly marked (data from Brenda Laca, quoted in \citealt[602]{vonHeusingeretal2011Affectedness}, yet two centuries later objects are marked under identical conditions at 70\%–90\% \citep[610]{vonHeusingeretal2011Affectedness}. 
Conversely, such variation does not need to be understood as a reflex of language \isi{diachrony}. This can be seen in \ili{Old Japanese}, where NPs from contemporary prose texts of 10th century are marked at 44\%–72\% \citep[248]{Sadler2002Pragmatic}. 
Data from \ili{Portuguese} also show that there can be substantial quantitative differences between contemporary texts \citep[85, 119–120]{Delille1970Entwicklung}. 
Furthermore, the letters from L9, in which object marking is sparingly attested, are originals and can be securely dated to the latter half of the 4th century AD. 
This makes them, for all practical purposes, contemporary with the text copies of L6, in which \textit{n-}marking is the dominant pattern. 
Thus, variation in object marking was acceptable concurrently. Such cases are a reminder that differences between subdialects should not necessarily be interpreted as a reflex of \isi{diachronic development}. Despite this, the blurry picture of \ili{Coptic} DOM is likely to reflect an evolving DOM system. 

It is worth reasserting the lack of any role for \isi{animacy} in \ili{Coptic} DOM, to judge from the Lycopolitan corpus used in this paper. 
It is not possible to find any parallel alignment between verb hierarchy and \isi{animacy} that is in a way similar to what von \citet{vonHeusingeretal2007Differential} suggested in their analysis of \ili{Spanish}. They observed a decrease in object marking from the verbs ‘to kill’, ‘to see’, ‘to consider’, and ‘to have’, which were analysed as representatives of different verb classes. Their conclusion that “the particular ranking depends on the \isi{animacy} requirement imposed by the verb on the \isi{direct object}” \citep[605]{vonHeusingeretal2011Affectedness} is not cogent because it was based on a study of no more than four to six verbs. 
Searching the \isi{animate} \vs \isi{inanimate} objects listed in this database reveals no such \isi{animacy} ranking. Rather, the \ili{Coptic} data indicate that the \isi{affectedness scale} is parallel to the decrease of control by the actor on the process of the verb. 
Furthermore, \ili{Coptic} DOM calls into question the general validity of any theory that relies on the need for disambiguation, on syntactic or semantic grounds, between the agent and object as a motivation for DOM (\eg \citealt{Aissen2003Differential,deSwart2005Resolution,Primus2012Animacy}). The \isi{word order} SVO means that there was no need for disambiguation of the core participants.

\section{Conclusion}\label{06-en-sec:7}

The present study supports the claim that \ili{Coptic} DOM in the non-imperfective domain has a functional motivationand is not arbitrary. I do not claim to have formulated a set of inviolable rules. Instead, I have shown tendencies that seem to be shared by all Lycopolitan subdialects (except for L9), for which the \textit{n-}marking number is too low to permit any satisfactory conclusions. The clear differences in \textit{n-}marking percentages between the Lycopolitan subdialects does, however, confirm their relative independence. It is apparent from the analysis that semantic factors act in conjunction with discourse-structural factors in Lycopolitan \ili{Coptic}. The quantitative analysis in \sectref{06-en-sec:5}, on the alternation of marking of NPs as objects through \textit{n}/${\emptyset}$ with non-imperfective tenses, has revealed striking differences in marking between the semantic verb categories. There is an overall agreement with Tsunoda’s \isi{verb-type hierarchy}: a highly-affected object with a dynamic action verb (\eg \textit{hôtbe} ‘to kill’) is likely to receive \textit{n-}marking; a little-affected object is less likely to receive \textit{n-}marking (\eg \textit{nouje} ‘to throw’). A low \textit{n-}marking percentage is found for the few verbs of perception/cognition that take the \textit{n}/${\emptyset}$ variation (\textit{saune} ‘to (get to) know’, \textit{cine} ‘to find’). Verbs of feeling (\eg \textit{meie} ‘to love’) uniformly have a zero-marked construction. 

Although generalised findings from an analysis of Lycopolitan cannot be extended to \ili{Coptic} as a whole, it should be apparent that it is relevant to examine the semantics of verb types is a relevant subject in future studies of DOM in that language. 

\section*{Acknowledgements}

I acknowledge the support of the Swedish Research Council for my research. The views expressed herein are not necessarily shared by those I would like to thank for their valuable comments and remarks: Natalie Bosson (who read a first version), Ivan Miroshnikov, and Jean Winand. The text was greatly improved thanks to the proofreading of R. Gareth Roberts.

\section*{Abbreviations}
\begin{tabularx}{.45\textwidth}{lQ}
\textsc{abst} & Abstract noun formative\\
\textsc{acc} & Object marker\\
\textsc{adv} & Adverb\\
\textsc{agt} & Agent preposition\\
\textsc{aor} & Aorist\\
\textsc{caus} & Causative\\
\textsc{circ} & Circumstantial clause marker\\
\textsc{cond} & Conditional\\
\textsc{cop} & Copula\\
\textsc{def} & Definite article\\ 
\textsc{dem} & Demonstrative pronoun\\
\textsc{f} & Feminine\\
\textsc{foc} & Focalizer\\
\end{tabularx}
\begin{tabularx}{.45\textwidth}{lQ}
\textsc{fut} & Future\\
\textsc{gen} & Genitive\\
\textsc{imp} & Imperative\\
\textsc{impf} & Imperfect\\
\textsc{indf} & Indefinite\\
\textsc{m} & Masculine\\
\textsc{neg} & Negative\\
\textsc{pl} & Plural\\
\textsc{poss} & Possessive article/pronoun\\
\textsc{prep} & Preposition\\
\textsc{pst} & Past\\
\textsc{ptcl} & Particle\\
\textsc{rel} & Relative\\
\textsc{state} & Resultative state\\
\end{tabularx}



{\sloppy
\printbibliography[heading=subbibliography,notkeyword=this] }
\end{document}
