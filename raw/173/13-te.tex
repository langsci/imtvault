\documentclass[output=paper]{LSP/langsci}
\ChapterDOI{10.5281/zenodo.1228267} 
\author{Amos Teo\affiliation{University of Oregon}}
\title{Differential A and S marking in Sumi (Naga): Synchronic and diachronic considerations}
\shorttitlerunninghead{Differential A and S marking in Sumi (Naga)}
\abstract{This paper presents data on the argument marking system of Sumi, a Tibeto-Burman language of Nagaland, and examines the possible diachronic sources of differential A and S marking in the language. In Sumi, there is a two-way contrast for A arguments (\textit{=no} vs. \textit{=ye}) and a three-way contrast for S arguments (\textit{=no} vs. \textit{=ye} vs. unmarked). I examine the triggers of such differential marking, looking at semantic factors associated with transitivity, as well as pragmatic factors associated with information structure.

In transitive clauses, \textit{=no} is more commonly found on A arguments, where it marks a semantic agent, while \textit{=ye} on A arguments signals a lack of agentivity. In intransitive clauses, \textit{=no} on S arguments marks contrastive focus, while \textit{=ye} marks a contrastive topic, or sometimes continuing reference.

Based on available synchronic data from Sumi and related languages, I examine the possibility that one source for the marker \textit{=ye} is an old locative marker. I also examine potential sources for the marker \textit{=no}, which has cognates across the language family that function as agentives or ergatives, as well as instrumentals and ablatives. 
% \keywords{differential marking, ergativity, Nagaland, Sumi, Tibeto-Burman}
}
\maketitle

\begin{document}


\section{Introduction}
\label{13-sec:1}

\ili{Sumi}, also known as Sema or Simi, is a \ili{Tibeto-Burman} language spoken in Nagaland, North-East India. Like many other \ili{Tibeto-Burman} languages of the area, \ili{Sumi} displays semantically and pragmatically motivated differential A and S argument marking.\footnote{In this paper, I follow \citet{Dixon1994Ergativity}’s use of the terms A, S and O to refer to: the subject of transitive clause, the subject of an \isi{intransitive} clause; and the object of a transitive clause respectively.} This type of \isi{differential argument marking} is not unusual for the area, where it appears that semantic and pragmatic factors play a major role in the distribution of what is sometimes described as the ‘ergative’ or the ‘agentive’ in these languages.\footnote{\citet{LaPolla1995Ergative} distinguishes ‘ergative’ from ‘agentive’ marking thus: the former is ‘systematic’ (in others words, the A argument is consistently marked); while the latter is ‘non-systematic’ (in other words, what one might call ‘differential A marking’ or ‘optional ergativity’, \eg  \citealt{Chelliahetal2012Optional,McGregor2010Optional}). In this paper, I use the terms ‘ergative’ and ‘agentive’ in a similar fashion to \citeauthor{LaPolla1995Ergative}.} 
Similar patterns of argument marking are found in other languages of Nagaland, including \ili{Mongsen Ao} \citep{Coupe2007Grammar,Coupe2011Pragmatic}, and more generally across \ili{Tibeto-Burman} (see \citealt{DeLancey2011Optional,Chelliahetal2012Optional}).



What is unusual about \ili{Sumi}, at least for a \ili{Tibeto-Burman} language of the area, is that I find a two-way distinction with A marking: the choice of two enclitics: \textit{=ye} and \textit{=no}; but a three-way distinction with S marking: the choice of the two enclitics: \textit{=ye} and \textit{=no}, and no overt morphological marking. In addition, O arguments are unmarked. By comparison, two closely related languages \ili{Khezha} and \ili{Mao} display the more typical ‘optional ergative’ system, where A and S may take an overt ‘ergative’ marker vs. no overt morphological marking; in addition to differential O argument marking. For instance, \ili{Khezha} has an ‘optional’ ergative marker \textit{nü} (glossed ‘nominative’ by \citealt{Kapfo2005Kheza}) on A arguments, as well as an ‘optional’ patientive / locative marker \textit{eh} /è/ on O arguments. 

Traditional accounts of \isi{differential argument marking} have focused on differential object marking and the role of \isi{animacy} and \isi{definiteness} (\eg \citealt{Bossong1983Animacy,Bossong1985Differentielle,Aissen2003Differential}). More recent work on \isi{differential argument marking} has also looked at the role of \isi{information structure} (\eg \citealt{Dalrympleetal2011Objects,Iemmoloetal2014Introduction}). Comparatively fewer studies have examined differential subject marking / differential agent marking / optional ergativity, with notable exceptions such as \citet{deHoopetal2008Differential} and  \citet{McGregoretal2010Optional}. Although differential subject marking has been assumed to be the mirror counterpart of differential object marking, there is evidence suggesting that the triggers of both are not identical (\citealt{Malchukov2008Animacy}; \citealt{Fauconnier2011Differential}).


\citet{Malchukov2008Animacy} also argues that the indexation of \isi{animacy} is simply an epiphenomenon associated with the expression of two potentially competing functions: the indexation of semantic roles, and the differentiation of subjects from objects. \citet{Fauconnier2011Differential} considers the role of \isi{animacy}, but rejects the notion that ‘Agents’ (defined as participants in the A role) and ‘Objects’ can be placed on a single \isi{animacy hierarchy} (as per \citealt{Silverstein1976Hierarchy}). Rather, she suggests that \isi{unexpectedness} plays a crucial role in differential agent marking, where for instance, inanimates are not expected to act as Agents and may receive special morphological marking or be restricted from appearing as Agents. Similarly, \citet{McGregor2010Optional} shows that in Gooniyandi and \ili{Warrwa}, the absence of an ergative morpheme on an A argument marks an unusual or unexpected A.

\largerpage
Similarly, it will be shown that in \ili{Sumi}, differential A and S marking is not triggered by some inherent \isi{animacy} of the referent, but by the interaction between situational factors such as \isi{agentivity}, defined by the degree of volition, control and purpose associated with a referent in a particular situation; and discourse pragmatic functions, including the marking of contrastiveness and \isi{unexpectedness}. However, this notion of ‘\isi{unexpectedness}’ is primarily about the management of listener-based and/or speaker-based expectations.

In~\sectref{13-te-sec:2}, I first give some background on \ili{Sumi} and describe the circumstances under which argument marking is obligatory in the language. In~\sectref{13-te-sec:3}, I then describe some of the triggers of differential A and S marking. In~\sectref{13-te-sec:4}, I consider the diachronic origins of these markers by presenting both language-internal and cross-linguistic evidence. Finally, in~\sectref{13-te-sec:5}, I summarize the findings and consider future avenues of inquiry.


\section{Language background}\label{13-te-sec:2}

\ili{Sumi}, also known as Sema, is a \ili{Tibeto-Burman} language spoken by an estimated 104,000 speakers \citep{Lewisetal2013Ethnologue} mainly in Nagaland, North-East India. \citet{Burling2003Tibeto} classifies \ili{Sumi} as a member of the \ili{Angami-Pochuri} group, along with Angami, \ili{Khezha} and \ili{Mao}. Many \ili{Sumi} speakers also speak English, as well as \ili{Nagamese}, an \ili{Assamese}-based creole. The canonical \isi{word order} of \ili{Sumi} is AOV / SV, like other \ili{Tibeto-Burman} languages of the region. In \ili{Sumi} transitive clauses, A arguments must be marked by either \textit{=ye} or \textit{=no},\footnote{There is in fact a third option, the additive \textit{ghi} ‘also’. Additionally, speakers may choose to omit an NP altogether. However, these will not be discussed in this paper.} while O arguments are typically unmarked morphologically when they occur right before the verb, as seen in \REF{13-te-ex:1}--\REF{13-te-ex:3}.


\ea\label{13-te-ex:1}
\ili{Sumi} (\ili{Tibeto-Burman}; Kutili\_bird\_short, Line 23){}{\rmfnm}\\

\gll ... [\textbf{a-zü=no}]\textbf{\textsubscript{A}} [\textbf{küma}]\textbf{\textsubscript{O}} yipesü-u-ve.{\rmfnm}\\
{ } \textsc{nrl}-water\textit{=no}]\textsubscript{A} [\textsc{3du}\textsc{\textsubscript{O}}\textsc{\-} sweep-go-\textsc{vm}\\
\glt ‘... the water swept them both away.’
\z
\footnotetext{Examples from texts can be found at \url{http://catalog.paradisec.org.au/repository/ABT1}. 
Cite as: Amos Teo (collector). 2008. \emph{\ili{Sumi} (India)} (ABT1), Digital collection managed by PARADISEC.  DOI: 10.4225/72/56E7A73CE9FA7}
\footnotetext{In this paper, examples are given in the working orthography, which does not consistently mark tones. The graph \textit{ü} represents a high central unrounded vowel /ɨ/.}

\ea\label{13-te-ex:2}
 \textit{* a-zü küma yipesü-u-ve}
\z

\ea\label{13-te-ex:3}
\langinfo{Sumi}{Tibeto-Burman}{Origin\_of\_axone, line 5}\\
\gll {…} [\textbf{{küma=ye}}]\textbf{\textsubscript{A}} [\textbf{{a-kishina}}]\textbf{\textsubscript{O}} {chu-kha-mo-ve=ke=hu} {...}\\
{ } \textsc{3du}={ye}]\textsubscript{A} [\textsc{nrl}-lunch]\textsubscript{O} eat-\textsc{ncpl}-\textsc{neg}-\textsc{vm}=\textsc{nzr}=\textsc{dist}\\
\glt ‘... they were unable to finish their lunch ...’  %[Origin\_of\_axone, line 4]
\z

First and second person singular pronominal O arguments are realized as proclitics on verbs, as in \REF{13-te-ex:4} and~\REF{13-te-ex:5}.\footnote{Note that these pronominal proclitics are identical in form to the possessive pronominal prefixes.}

\ea\label{13-te-ex:4}
\langinfo{Sumi}{Tibeto-Burman}{elicited}\\
\gll  Pa=no \textbf{o=}he.\\
\textsc{3sg=}\textit{no} \textsc{2sg}=hit\\
\glt ‘He hit you.’  %[elicited]
\z

\ea\label{13-te-ex:5}
\langinfo{Sumi}{Tibeto-Burman}{elicited}\\
\gll  No=no \textbf{i=}he.\\
\textsc{2sg=}\textit{no} \textsc{1sg}=hit\\
\glt ‘You hit me.’ %[elicited]
\z

In \isi{intransitive} clauses, S arguments can be morphologically unmarked, as in \REF{13-te-ex:6}, or marked with either \textit{=ye} or \textit{=no}, as in \REF{13-te-ex:7} and~\REF{13-te-ex:8} respectively.

\ea\label{13-te-ex:6}
\langinfo{Sumi}{Tibeto-Burman}{Origin\_of\_axone, line 3}\\
\gll  \textbf{Küma} a-lu=lo hu-niye=ke=lo {...}\\
3\textsc{du} \textsc{nrl}-field=\textsc{loc} go.field-\textsc{pros=nzr=loc}\\
\glt ‘While the two were about to go to the field ...’ %[Origin\_of\_axone, line 3]
\z

\ea\label{13-te-ex:7}
\langinfo{Sumi}{Tibeto-Burman}{Telephone\_conversation01, line 4}\\
\gll  O \textbf{Kivi=ye} zü a-phi.\\
\textsc{excl} Kivi\textit{=ye} sleep \textsc{prog}-\textsc{cont}\\
\glt ‘Oh, Kivi is still sleeping.’ (2\textsuperscript{nd} mention) %[Phone\_call\_01, line 6]
\z

\ea\label{13-te-ex:8}
\langinfo{Sumi}{Tibeto-Burman}{elicited}\\
\gll  \textbf{Kivi=no} zü a-ni.\\
Kivi=\textit{no} sleep \textsc{prog}-\textsc{npst}\\
\glt ‘Kivi (not someone else) is sleeping.’ %[elicited]
\z

The obligatoriness of argument marking therefore depends largely on clause type: A arguments (and as I shall demonstrate, the first NPs in non-verbal clauses) must be accompanied by either \textit{=ye} or \textit{=no}, while S arguments may be marked by \textit{=ye}, \textit{=no} or be morphologically unmarked. 
In all cases, the choice of marking, or lack thereof, depends largely on semantic and pragmatic factors. 
These will be examined in the next section.


\section{Triggers of differential A and S marking} \label{13-te-sec:3}

In this section, I describe some of the triggers of differential A and S marking in \ili{Sumi}. The analysis presented here is a summary of the one presented in \citet{Teo.2012}. Generally, in transitive clauses, situational characteristics of arguments like control and volition largely determine the choice of \textit{=no} or \textit{=ye}. In \isi{intransitive} and non-verbal clause types, it seems that discourse characteristics like \isi{topicality}, contrastiveness, focus, and perhaps \isi{unexpectedness} are the main triggers. However, there are cases where such discourse characteristics appear to also influence differential A marking in transitive clauses, while certain situational characteristics of arguments may also be relevant for differential S marking.


\subsection{Transitive clauses}\label{13-te-sec:3.1}

As mentioned in the previous section, A arguments in clauses with two or more core arguments must take either \textit{=no} or \textit{=ye}. The use of \textit{=no} in such clauses is often associated with an agent that has a high degree of volition, control and purpose. For instance, in \REF{13-te-ex:9}--\REF{13-te-ex:11}, \textit{=no} marks a volitional and purposeful A that is able to effect a change in the world. Note that in \REF{13-te-ex:10}, the river is regarded as a supernatural force that has been actively preventing a mother from making a crossing with her baby, and eventually sweeps them away when they attempt to cross. In contrast, \textit{=ye} often marks experiencers, which are characterized by having a low degree of volition and control over an action, as in \REF{13-te-ex:12} and~\REF{13-te-ex:13}.\footnote{In \REF{13-te-ex:13}, the argument marked by \textit{=ye} would not be considered to be an A argument, but rather an \isi{experiencer}/locative subject.}
These features of the A argument: volition, control and purpose, are in line with some of the components proposed by  \citet{Hopperetal1980Transitivity} in their analysis of semantic \isi{transitivity}.

\ea\label{13-te-ex:9}
\langinfo{Sumi}{Tibeto-Burman}{elicited}\\
\gll  \textbf{I=no} a-lhache he-qhi-ve. \\
\textsc{1sg}\textit{=no}\textsc{} \textsc{nrl-}ant\textsc{\-} hit-kill-\textsc{vm}\\
\glt ‘I killed an ant.’ %[elicited]
\z

\ea\label{13-te-ex:10}
\langinfo{Sumi}{Tibeto-Burman}{Kutili\_bird\_short, line 23}\\
\gll {...} \textbf{a-zü=no} küma yipesü-u-ve.\\
{ } \textsc{nrl}-water\textit{=no}\textsc{} \textsc{3du\-} sweep-go-\textsc{vm}\\
\glt ‘... the water swept them both away.’ %[Kutili\_bird, line 9]
\z

\ea\label{13-te-ex:11}
\langinfo{Sumi}{Tibeto-Burman}{elicited}\\
\gll  \textbf{Ni-nga=no} kuu shi-va kea?\\
1\textsc{pl}-child\textit{=no} what do-\textsc{prf} \textsc{q}\\
\glt ‘What has our daughter done?’ %[elicited]
\z

\ea\label{13-te-ex:12}
\langinfo{Sumi}{Tibeto-Burman}{Kutili\_bird\_short, line 27}\\
\gll  \textbf{Ni=ye} ni-nga=sütsa chu-mla-va-i.\\
1\textsc{sg}\textit{=ye} 1\textsc{pl}-daughter=voice hear-\textsc{ncap-prf-emph}\\
\glt ‘I no longer hear any news from our daughter.’ %[Kutili\_bird, line 11]
\z

\ea\label{13-te-ex:13}
\langinfo{Sumi}{Tibeto-Burman}{Kutili\_bird\_short, line 26}\\
\gll  \textbf{Ni-nga=ye} kuu shi-va kea?\\
1\textsc{pl}-child\textit{=ye} what do-\textsc{prf} \textsc{q}\\
\glt ‘What has happened to our daughter?’ %[Kutili\_bird, line 10]
\z

Certainly, in most of these examples, the degree of \isi{agentivity} of the A is closely linked to the lexical verb: \textit{=no} is preferred on the A argument with canonical transitive verbs like ‘kill’, as in \REF{13-te-ex:9}, where A has a higher degree of \isi{agentivity}; while \textit{=ye} is preferred on A with verbs of passive perception like ‘see’ or ‘hear’, as in \REF{13-te-ex:12}, where A has a low degree of \isi{agentivity}.

With some verbal predicates, the choice of \textit{=no} or \textit{=ye} on A corresponds to a specific sense of the verb. For example, \REF{13-te-ex:14}, where A is marked with \textit{=ye}, describes a scene where the referent is not in control of the action. One could interpret \textit{pele} as ‘to spill’ or ‘to bleed’. In contrast, in \REF{13-te-ex:15}, where A is marked with \textit{=no}, the verb \textit{pele} has more of a causative interpretation: ‘cause to spill’.

\ea\label{13-te-ex:14}
\langinfo{Sumi}{Tibeto-Burman}{elicited}\\
\gll  \textbf{Pa=ye} a-ji pele-ve.\\
3\textsc{sg}\textit{=ye} \textsc{nrl}-blood spill-\textsc{vm}\\
\glt ‘He was bleeding.’ %[elicited]
\z

\ea\label{13-te-ex:15}
\langinfo{Sumi}{Tibeto-Burman}{elicited}\\
\gll  \textbf{Pa=no} a-ji pele-ve. \\
3\textsc{sg}\textit{=no} \textsc{nrl}-blood spill-\textsc{vm}\\
\glt ‘He threw away blood.’ %[elicited]
\z

With some verbal predicates, as in \REF{13-te-ex:16}, \textit{=no} is the expected marker on A if one assumes that the chief should have authority among his people. In comparison, in one possible interpretation of \REF{13-te-ex:17}, the use of \textit{=ye} suggests that A is a less effective agent, \ie a chief who cannot make his people obey him even though he gave an explicit command. Note that \isi{animacy} and \isi{definiteness} do not appear to affect the choice of \textit{=no} or \textit{=ye} in these examples.

\ea\label{13-te-ex:16}
\langinfo{Sumi}{Tibeto-Burman}{elicited}\\
\gll \textbf{A-kü-ka-u=no} a-zah tsü-ve.\\
\textsc{nrl-nzp}-rule-\textsc{def}\textit{=no} \textsc{nrl}-command give-\textsc{vm}\\
\glt ‘The chief gave a command.’ %[elicited]
\z

\ea\label{13-te-ex:17}
\langinfo{Sumi}{Tibeto-Burman}{elicited}\\
\gll  \textbf{A-kü-ka-u=ye} a-zah tsü-ve.\\
\textsc{nrl-nzp}-rule-\textsc{def}\textit{=ye} \textsc{nrl}-command give-\textsc{vm}\\
\glt ‘The chief gave a command.’ (One interpretation: has a sarcastic reading and implies no one obeyed him.) %[elicited]
\z

Transitive clauses show a split-A system, where \textit{=no} typically correlates with higher \isi{agentivity} and \textit{=ye} with lower \isi{agentivity}, such as experiencers with a low degree of volition and control.\footnote{A prototypical “\isi{experiencer}”, as per \citet[50]{Payne1997Describing}, is “an entity that receives a sensory impression, or in some other way is the locus of some event or activity that involves neither volition nor a change of state.”} However, the \isi{agentivity} of the A referent cannot always explain the distribution of the morphemes \textit{=no} and \textit{=ye}. It is important to note that the sentence in \REF{13-te-ex:17} could also be interpreted without sarcasm as ‘(Someone else did something), as for the chief, (he) gave a command.’ This, as well as evidence from \isi{intransitive} clauses (see next section), suggests that \textit{=ye} can also function as a kind of topic marker in some transitive clauses.

In narratives, it is not always easy to tease apart the various functions of \textit{=ye}. For example, in \REF{13-te-ex:18},\footnote{It should be noted that \textit{=no} and \textit{=ye} can also occur on adverbial adjuncts. This will be discussed further in~\sectref{13-te-sec:4.4}.} \textit{=no} occurs in the first clause, which describes how two sisters made \textit{axone}, a popular \ili{Sumi} dish of fermented soya beans, for the very first time. In contrast, \textit{=ye} is found in the second clause, which describes how Sumis then habitually cooked the dish. Although the use of \textit{=no} and \textit{=ye} does not appear to be motivated by situational characteristics relating to volition and control of the participants, one might still argue that according to \citegen{Hopperetal1980Transitivity} criteria, the first clause displays a higher degree of \isi{transitivity} than the second, since the former refers to the first (telic and punctual) instance of an event, while the latter refers to a repeated event that is \isi{atelic} and non-punctual. 
On the other hand, in an alternative analysis that assigns greater importance to discourse factors, \textit{=no} highlights that this was a newsworthy event, and that it was this pair of sisters, not anyone else, who instigated the first instance of the event; while \textit{=ye} is used in the second clause to set up a change in A argument from the two sisters to Sumis in general.

\protectedex{
\ea\label{13-te-ex:18}
\langinfo{Sumi}{Tibeto-Burman}{Origin\_of\_axone, lines 17--20}\\
\gll  Tishi=no [\textbf{küma=no} a-xone lho-chu-phe=püzü=no]  tingu=no a-la-u=ye \textbf{Sümi=qo=ye} a-xone lho-chu-u-ve].\\
like.that\textit{=no} [3\textsc{du}\textit{=no} \textsc{nrl}-ferm.soya.beans cook-eat-start=\textsc{conj=conn}] because.of.that\textit{=no} \textsc{nrl}-path-\textsc{def}\textit{=ye} [\ili{Sumi}=\textsc{pl}\textit{=ye} \textsc{nrl}-ferm.soya.beans cook-eat-go-\textsc{vm}]\\
\glt `Henceforth, the two (sisters) started to cook and eat axone (a fermented soya bean dish) and consequently from then on, the Sumis have cooked and eaten axone.’
\z
}

In some cases, it may be difficult to tell if \textit{=no} is marking an agent or some kind of contrast. For example, in \REF{13-te-ex:19}, the A argument has volition and control, which may explain the appearance of \textit{=no}. However, it is also possible that the use of \textit{=no} is associated with counter-expectation, \ie the event that is instigated by A is not expected given the known circumstances, if one assumes that having children gives a husband less reason to abuse his wife.

\ea\label{13-te-ex:19}
\langinfo{Sumi}{Tibeto-Burman}{Kutili\_bird\_short, lines 6--7}\\
 \gll {...} a-tianu a-u-ve=mu [\textbf{a-kimi=no} li=sapüsa]\\
{ } \textsc{nrl}-children \textsc{exist}-go-\textsc{vm=neg} \textsc{nrl-}husband\textit{=no} 3\textsc{sg.f=}mistreat]\\
\glt ‘... despite having children, the husband mistreated her.’ 
\z

In general, the degree of \isi{agentivity} of A seems to be the more important factor in the choice of \textit{=no} or \textit{=ye}. A corpus study is currently being done to investigate the extent to which the choice of =no or =ye is determined by the number of core arguments licensed by a verb, the semantic roles assigned by a verb, and the \isi{animacy} of A.


\subsection{Intransitive clauses} \label{13-te-sec:3.2}

In \isi{intransitive} clauses, the first time an argument is mentioned in discourse, it can be morphologically unmarked, as in \REF{13-te-ex:20}. However, if an S is being contrasted with another S, it takes \textit{=ye}, which marks it as a contrastive topic, i.e. ‘as for this S, S did something', as in \REF{13-te-ex:21}.

\ea\label{13-te-ex:20}
\langinfo{Sumi}{Tibeto-Burman}{Telephone\_conversation01, line 4}\\
\gll \textbf{Kivi} zü a-ni.\\
Kivi sleep \textsc{prog}-\textsc{npst}\\
\glt ‘Kivi is sleeping.’ (1\textsuperscript{st} mention) 
\z

\ea\label{13-te-ex:21}
\langinfo{Sumi}{Tibeto-Burman}{Telephone\_conversation01, line 7}\\
\gll O \textbf{Kivi=ye} zü a-phi.\\
\textsc{excl} Kivi\textit{=ye} sleep \textsc{prog}-\textsc{cont}\\
\glt ‘Oh, Kivi is still sleeping.’ (2\textsuperscript{nd} mention) (Kivi was previously mentioned, but the speaker then switched to talking about her other son, before switching back to talking about Kivi) 
\z

S is unmarked after having been introduced in a previous presentational clause, as in \REF{13-te-ex:22}, which follows the opening line: ‘Once upon there were two sisters’. Here, the S argument \textit{küma} is not marked with \textit{=ye} because the two sisters are not being contrasted with anyone else in the story.

\ea\label{13-te-ex:22}
\langinfo{Sumi}{Tibeto-Burman}{Origin\_of\_axone, line 3}\\
\gll \textbf{Küma} a-lu=lo hu-niye=ke=lo {...}\\
3\textsc{du} \textsc{nrl}-field=\textsc{loc} go.field-\textsc{pros}=\textsc{nzr}=\textsc{loc}\\
\glt ‘While the two were about to go to the field ...’  %[Origin\_of\_axone, line 3]
\z

Importantly, S is always marked with \textit{=ye} in elicited sentences, such as~\REF{13-te-ex:23}.

\ea\label{13-te-ex:23}
\langinfo{Sumi}{Tibeto-Burman}{elicited}\\
\gll \textbf{A-kulu=ye} ighi=va.\\
\textsc{nrl-}light\textit{=ye} come-\textsc{prf}\\
\glt ‘The power has come back.’ %[elicited]
\z

This illustrates how only in data collected in more naturalistic contexts, \ie from conversations and narratives, can S be morphologically unmarked. When working with recorded texts, if speakers are asked to repeat sentences produced in such texts, they will sometimes add \textit{=ye} to S arguments, even in cases where \textit{=ye} was not found with S in the original text. This suggests that the use of \textit{=ye} in \isi{intransitive} clauses is associated with some discourse pragmatic function, such as continuing topic, than with the marking of the \isi{semantic role} of \isi{experiencer}, as was described for transitive clauses.

In addition, S arguments can be marked by \textit{=no}. The use of \textit{=no} here, rather than marking the \isi{semantic role} of agent, typically marks some kind of focus on the argument. For example, in \REF{13-te-ex:24}, \textit{=no} is used when S is the answer to a question. It can also be used to highlight \isi{contrastive focus}, \ie ‘\textit{this} S, not any other one, as well as corrective focus, \ie ‘\textit{this} S, not the one you think it is’.

\ea\label{13-te-ex:24}
\langinfo{Sumi}{Tibeto-Burman}{elicited}\\
\gll \textbf{Pa=no} nu-va.\\
3\textsc{sg}\textit{=no} laugh-\textsc{prf}\\
\glt ‘He laughed (not anyone else).’ (answers the question: “Who laughed?”) %[elicited]
\z

In some situations, S is marked with \textit{=no}, with no obvious \isi{contrastive focus} reading. An example is given in \REF{13-te-ex:25}, which describes God’s descent to Earth in the biblical story of the Tower of Babel. 

\ea\label{13-te-ex:25}
\langinfo{Sumi}{Tibeto-Burman}{\ili{Sumi} Bibel Genesis 11:5}\\
\gll {…} \textbf{ A-mpeu=no} iqi-e.\\
{} \textsc{nrl}-lord\textit{=no} descend-\textsc{emph}\\
\glt ‘... the Lord came down.’
\z

The ongoing corpus study will also look at how frequently \textit{=no} occurs with S and what factors best account for its occurrence in \isi{intransitive} clauses, since it is unclear whether \textit{=no} is used in examples \REF{13-te-ex:25} because: (a) it signals a high degree of volition, control and purpose associated with the referent, \ie an omnipotent being; or (b) it marks some degree of surprise or counter-expectation for the action performed by S; or (c) it is a combination of these two and other factors.

\subsection{Non-verbal clauses} \label{13-te-sec:3.3}

Non-verbal clauses are also worth mentioning in a discussion of \isi{differential argument marking} in \ili{Sumi}. 
There is no copula verb in the affirmative present tense and in such clauses, the first NP is obligatorily marked by either \textit{=ye} or \textit{=no}. In pragmatically unmarked statements, the subject requires \textit{=ye}, \cf \REF{13-te-ex:26} and~\REF{13-te-ex:27}. 
If the first NP is marked with \textit{=no}, as in \REF{13-te-ex:28}, corrective focus or \isi{contrastive focus} reading  is obtained, similar to the use of \textit{=no} with S arguments in \isi{intransitive} clauses. 
This particular example came about when a speaker corrected the researcher for assuming that the father of a person of mixed ancestry in the town was \ili{Sumi} – in fact, it was the mother who was \ili{Sumi}.

\ea\label{13-te-ex:26}
\langinfo{Sumi}{Tibeto-Burman}{elicited}\\
\gll  \textbf{Pa-za=ye} Sümi.\\
3\textsc{sg}-mother\textit{=ye} \ili{Sumi}\\
\glt ‘His mother is \ili{Sumi}.’ %[elicited]
\z

\ea\label{13-te-ex:27}
 \textit{*}\textbf{\textit{Pa-za}} \textit{Sümi.}
\z

\ea\label{13-te-ex:28}
\langinfo{Sumi}{Tibeto-Burman}{natural conversation, unrecorded}\\
\gll \textbf{Pa-za=no} Sümi.\\
3\textsc{sg}-mother\textit{=no} \ili{Sumi}\\
\glt ‘His mother is \ili{Sumi}.’ (\ie not his father, not anyone else)\z

Unlike in the previously discussed clause types, the choice between \textit{=no} and \textit{=ye} in \isi{equative} clauses cannot be attributed to differences in the semantic \isi{transitivity} of the clause. 
Rather, it is discourse pragmatic factors that seem to condition the distribution of \textit{=no} and \textit{=ye}, with the former used to mark contrastive or corrective argument focus while the latter is used to mark either a new, contrastive or continuing topic.


\subsection{Summary of triggers of differential argument marking}\label{13-te-sec:3.4}

A summary of the functions of \textit{=no} and \textit{=ye} by clause type is given in \tabref{13-te-tab:1}.

\begin{table}
\caption{Summary of functions of \textit{=no} and \textit{=ye} by clause type}\label{13-te-tab:1}
\begin{tabularx}{\textwidth}{LLLL}
\lsptoprule

Clause type & \textbf{\textit{=no}} & \textbf{\textit{=ye}} & \textbf{unmarked}\\
\midrule
Transitive clauses & ‘agent’ – high degree of control, volition, purpose etc. & ‘\isi{experiencer}’ – low degree of control, volition, purpose etc. & [not possible]\\
Intransitive clauses & ‘focus’ – contrastive / corrective & ‘topic’ – contrastive, continuing & first mention of referent\\
Non-verbal clauses & ‘focus’ – contrastive / corrective & ‘topic’ – new, contrastive, continuing & [not possible]\\
\lspbottomrule
\end{tabularx}
\end{table}

It appears that situational characteristics of arguments like control and volition play a large role in differential A marking in transitive clauses, while discourse characteristics like \isi{topicality} and contrastiveness play a large role in \isi{differential argument marking} in \isi{intransitive} and \isi{equative} clauses. Nevertheless, it is important to note that this distinction is not as clear-cut as it appears in~\tabref{13-te-tab:1}. As previously shown, there are examples that suggest that discourse characteristics like focus and \isi{unexpectedness} may play a role in determining differential A marking even in transitive clauses, while situational characteristics like volitionality and control may also determine differential S marking in some \isi{intransitive} clauses. 
Crucially, it should be noted that certain features of referents like \isi{animacy} and \isi{definiteness} do not seem to play a large role in \isi{differential argument marking} in \ili{Sumi}. 
Certainly, such features interact with notions of \isi{discourse prominence} and expectedness, but any apparent indexation of these features could simply be regarded as epiphenomenal.


\section{Origins of differential A and S marking}\label{13-te-sec:4}

Having looked at some factors governing the synchronic pattern of \isi{differential argument marking} in \ili{Sumi}, let us now consider the diachronic origins of the relevant markers. 
Given that the primary functions of \textit{=no} and \textit{=ye} differ by clause type, and that different clause types differ in terms of the obligatoriness of argument marking, it would be prudent to consider the origin of the \textit{=ye} and \textit{=no} markers in each clause type separately.


\subsection{Origins of \textit{=ye} in transitive clauses}\label{13-te-sec:4.1}

It was shown earlier that experiencers in transitive clauses are typically marked by \textit{=ye,} as in \REF{13-te-ex:29}.

\ea\label{13-te-ex:29}
\langinfo{Sumi}{Tibeto-Burman}{Kutili\_bird\_short, line 27}\\
\gll \textbf{Ni=ye} ni-nga=sütsa chu-mla-va-i.\\
1\textsc{sg}=\textit{ye} 1\textsc{pl}-daughter=voice hear-\textsc{ncap}-\textsc{prf}-\textsc{emph}\\
\glt ‘I no longer hear any news from our daughter.’ %[Kutili\_bird, line 11]
\z

There is some language-internal evidence that points to a locative as the source of this marker, even though the synchronic locative marker in \ili{Sumi} is \textit{lo}. In predicate possession clauses, such as~\REF{13-te-ex:30}, the possessor is marked with \textit{=ye}. The possessor is then followed by the possessee and an existential verb \textit{ani} or \textit{ache}. The structure of such predicate possession clauses parallels that of existential clauses, as in \REF{13-te-ex:31}, where the location \textit{aghuloki lakhi lo} is marked with the synchronic locative \textit{lo}, followed by the entity in question and an existential verb. 

\ea\label{13-te-ex:30}
\langinfo{Sumi}{Tibeto-Burman}{elicited}\\
\gll \textbf{Ni=ye} a-tsü a-ni.\\
1\textsc{sg}\textit{=ye} \textsc{nrl}-dog \textsc{exist}-\textsc{npst}\\
\glt ‘I have a dog.’
\z

\ea\label{13-te-ex:31}
\langinfo{Sumi}{Tibeto-Burman}{Origin\_of\_axone, line 2}\\
\gll Khaghi \textbf{a-ghuloki} \textbf{lakhi=lo} a-tsünipu kini a-che=ke=ti~…\\
long.ago \textsc{nrl}-time.period one=\textsc{loc} \textsc{nrl}-sister two \textsc{exist}-\textsc{pst}=\textsc{nzr}=\textsc{med}\\
\glt ‘Once upon a time, there were two sisters …’ 
\z

Given that \ili{Sumi} does not appear to have a separate verb meaning ‘to possess’, but rather the same existential verb root \textit{a-} in both clause types, this suggests that the \textit{=ye} that is found on the possessor was once used to indicate location. 
The use of a locative subject in possessive constructions are common in \ili{Tibeto-Burman}, but are also found in other languages of the world (see \citealt{Clark1978Locationals}; \citealt{Stassen2013Predicative}).

Similarly, in constructions that express ‘to like’, as in \REF{13-te-ex:32}, the liker is typically marked with \textit{=ye}. What looks like a verb meaning ‘to like’ \textit{alo} has the internal structure of a noun meaning ‘goodness’ or ‘good’, and has the same nominal prefix \textit{a-} found in the citation form of most nouns in \ili{Sumi}. This would suggest that the origin of this construction is possibly a locative construction that may be translated literally as ‘At you, \textit{axone} is (usually) good?’ The verb \textit{cheni} marks the existence of a habitual state and in some contexts can be used interchangeably with the existential verb \textit{ani}.

\ea\label{13-te-ex:32}
\langinfo{Sumi}{Tibeto-Burman}{natural conversation, unrecorded}\\
\gll \textbf{No=ye} a-xone a-lo che-ni kea?\\
\textsc{2sg}\textit{=ye} \textsc{nrl}-fermented.soya.beans \textsc{nrl}-good \textsc{hab}-\textsc{npst} \textsc{q}\\
\glt ‘Do you like axone (fermented soya bean dish)?’
\z

The use of locative constructions to code \isi{experiencer} “subjects” is well attested in the languages of South Asia (see \citealt{Vermaetal1990Experiencer}), including \ili{Tibeto-Burman} languages of the area, such as \ili{Meithei} \citep[108]{Chelliah1997Grammar} and \ili{Tshangla} \citep[142]{Andvik2010Grammar}. In these languages, locative (as well as dative) \isi{case marking} is also found on possessor subjects in copular clauses. The second argument in these clauses is usually in the absolutive case, which is typically morphologically unmarked.

Preliminary comparative data from other \ili{Angami-Pochuri} languages further suggest that \ili{Sumi} \textit{=ye} derives from an old locative marker. In \ili{Khezha}, the locative marker is \textit{eh} /è/,\footnote{The grave accent marks low tone in \ili{Khezha}.} as seen in \REF{13-te-ex:33}, while in \ili{Mao}, the locative marker is \textit{yi}, as seen in \REF{13-te-ex:34}.\footnote{\citet{Giridhar1994Mao} does not provide morpheme-by-morpheme glosses for his examples. All glosses for examples from \ili{Mao} have been added based on his grammatical description and examples given in the grammar.}
 It is possible that both these markers are cognates with \ili{Sumi} \textit{=ye}, although more work is to be done to establish their cognacy by examining regular sound correspondences between these languages.

\ea 
\langinfo{Khezha}{Tibeto-Burman}{\citealt[286]{Kapfo2005Kheza}}\\
\label{13-te-ex:33}

\gll Mary nü \textbf{ketsü} \textbf{eh} beh a.\\
Mary \textsc{nom} garden \textsc{loc} \textsc{exist} \textsc{part}\\
\glt `Mary is in the garden.’
\z

\protectedex{
\ea 
\langinfo{Mao}{Tibeto-Burman}{based on \citealt[185]{Giridhar1994Mao}}\\
\label{13-te-ex:34}

\gll Athikho \textbf{Lokho-yi} kahie.\\
Athikho Lokho-\textsc{loc} be.close?\\
\glt ‘Athikho is close (in spatial distance) to Lokho.' 
\z
}
Given the above evidence, it would therefore be reasonable to hypothesize that an old \ili{Angami-Pochuri} locative is the origin of \ili{Sumi} \textit{=ye}, at least in transitive clauses.

However, it should also be noted that in \ili{Khezha} and \ili{Mao}, O arguments appear to optionally take locative markers, \ie there is a contrast between an overt marker and a lack of marking, though the triggers for such differential marking are not well described. Examples where O arguments are overtly marked are given in \REF{13-te-ex:35}--\REF{13-te-ex:37}. 
It is unclear if these markers really do mark semantic patients / grammatical objects vs. semantic locations, since they usually occur with contact verbs, \eg \textit{meke} ‘to bite’ or a compound based on a contact verb, \eg \textit{meke-thru} ‘to kill by biting’. However, in \ili{Mao} at least, the locative with O is also used with the verb ‘to love’, as in \REF{13-te-ex:37}, suggesting it has started to mark O arguments more generally.

\ea\label{13-te-ex:35}
\langinfo{Khezha}{Tibeto-Burman}{\citealt[288]{Kapfo2005Kheza}}\\
\gll Cotsü nü \textbf{coha} \textbf{eh} meke-thru dah.\\
black.ant \textsc{nom} red.ant \textsc{loc} bite-kill \textsc{part}\\
\glt `A black ant has killed a red ant.'
\z

\ea\label{13-te-ex:36}
 \langinfo{Mao}{Tibeto-Burman}{based on \citealt[180]{Giridhar1994Mao}}\\
\gll Nili-no \textbf{Nisa-yi} da pie.\\
Nili-\textsc{erg} Nisa-\textsc{loc} beat give\\
\glt `Nili beats Nisa.'
\z

\ea\label{13-te-ex:37}
\langinfo{Mao}{Tibeto-Burman}{based on \citealt[184]{Giridhar1994Mao}}\\
\gll Ai \textbf{Athia-yi} le shüe.\\
\textsc{1sg.nom} Athia-\textsc{loc} love\\
\glt  `I love Athia.'
\z

While it is still uncertain what the exact triggers for such differential O marking in \ili{Khezha} and \ili{Mao} are, for the purposes of this paper, it is simply important to note the shift from a locative to what is starting to look like a patientive marker. 
Similar patterns have been noted in other \ili{Tibeto-Burman} languages of South Asia, including \ili{Tshangla} \citep[156]{Andvik2010Grammar}, where the locative / dative \textit{ga} may occur on an \isi{experiencer} or goal patient.

It therefore appears that one source for \textit{=ye} on A arguments in transitive clauses is the reanalysis of locative experiencers/patients as \isi{experiencer} As. The function of \textit{=ye} was then extended to non-agent-like As, possibly because it was then in contrast with the agentive marker \textit{=no}. This would be a \ili{Sumi}-specific innovation not found in other \ili{Angami-Pochuri} languages where the locative optionally marks O arguments.\footnote{In new data collected by the author, it turns out that there are some \ili{Sumi} speakers who can optionally mark O arguments with the synchronic \ili{Sumi} locative \textit{=lo}. Little is known of the triggers of differential O marking, and preliminary data and speaker judgements shows much variation across the community: some speakers reject any marking on O arguments; some accept O marking only with verbs of contact; and others accept optional marking on O arguments in general.}


\subsection{Origins of \textit{=ye} in intransitive and non-verbal clauses}\label{13-te-sec:4.2}

In the previous section, we saw how a locative might have developed into an \isi{experiencer} A marker. 
In \isi{intransitive} clauses, the same locative marker might have developed into a topic marker. 
However, the latter is not a widely attested \isi{grammaticalization} pathway and without sufficient language-internal and comparative evidence, I am left to speculate on the origins of \textit{=ye} in \isi{intransitive} clauses. 

One possible clue to the origins of \textit{=ye} marking on S arguments may come from no-verbal clauses. 
As previously shown, the first NP in such clauses obligatorily takes \textit{=ye} or \textit{=no}\textstyleannotationreference{.} Synchronically, there is no copular verb in such clauses in the present affirmative.\footnote{However, in other tenses, \ili{Sumi} does use copulas derived from the verb \textit{shi} ‘to do’.} 

In contrast, in the related language \ili{Mao}, \citet{Giridhar1994Mao} gives examples of \isi{equative} clauses (which he calls "predicate phrases") where -\textit{ko-e} is added to the second argument in the clause, as in \REF{13-te-ex:38}. 
What is represented as the suffix -\textit{ko} is identical in form to a verbal nominalizing prefix in the language. This would suggest that -\textit{e} has a verbal origin – more specifically, a copular verb.

\ea
\langinfo{Mao}{Tibeto-Burman}{based on \citealt[456]{Giridhar1994Mao}}\\
\label{13-te-ex:38}

\gll hihi a zhu-ko-e\\
\textsc{prx} \textsc{1sg} name-\textit{ko-e}\\
\glt `This is my name.'
\z

This may lead one to wonder if \ili{Sumi} also once used a copular verb in \isi{equative} clauses.\footnote{One also wonders if the \ili{Sumi} post-verbal emphatic suffix \textit{-e {\textasciitilde} -i}, as seen below, is a reflex of an older copula. 
\ea
\langinfo{Sumi}{Tibeto-Burman}{Kutili\_bird\_short, line 33} \\
\gll Pa=ye khaghi=no o=pütsa-ni pi u-va-\textbf{e}.\\
3\textsc{sg}\textit{=ye} long.time\textit{=no} 2\textsc{sg}=talk.to-pros say go-\textsc{prf}-\textsc{emph}\\
\glt ‘She said a long time ago that she was going to see you and left.’
\z
} The pathway from copula to topic marker is not common, but it is attested. \citet[165--166]{Harrisetal1995Historical} give examples of copulas being reanalyzed as topic markers, in what they term “anti-cleft” constructions.

Alternatively, it is not uncommon for \isi{equative} copulas to develop into focus markers, typically through cleft constructions  \citep[95]{Heineetal2002World}. 
One could speculate that an old \ili{Sumi} \isi{equative} copula was reanalyzed as a \isi{focus marker} via a cleft construction, which has been extended to mark new and continuing topics. 
This pathway is attested – for instance, \citet{Ueno1987Functions} uses historical textual data to show that the \ili{Japanese} topic marker \textit{wa} originated as a contrastive marker \textit{ha} used for “emphasis” before it developed the function of marking topic differentiation, and eventually topic continuity.

In the case of \ili{Sumi}, good historical data is not available and what has been presented here is still speculation. Furthermore, it is still unclear how \textit{=ye} would have spread from \isi{equative} clauses to \isi{intransitive} ones. Perhaps, if the time depth for such \isi{grammaticalization} processes in \ili{Sumi} is shallow, it might even be helpful to look at differences in the distribution of \textit{=ye} in the speech of older vs. younger speakers or between villages which are said to speak more ‘conservative’ varieties of \ili{Sumi} vs. other villages.


\subsection{Origin of \textit{=no} in transitive and intransitive clauses}\label{13-te-sec:4.3}

There is evidence pointing to an instrumental origin for the agentive \textit{=no},  which then was extended to mark constrastive focus. However, positing a \ili{Sumi}-specific origin for \textit{=no} in transitive and \isi{intransitive} clauses is somewhat problematic. The instrumental marker in \ili{Sumi} is \textit{pesü}, derived from a verb meaning ‘take’, but there is evidence of a rarer instrumental \textit{=no} that is homophonous with the agentive \textit{=no}, as in \REF{13-te-ex:39}. This rarer \textit{=no} is likely an older instrumental marker that is being replaced by a more recently innovated and morphologically transparent \textit{pesü}.

\ea\label{13-te-ex:39}
\langinfo{Sumi}{Tibeto-Burman}{elicited}\\
\gll \textbf{Pa-puh=no} \textbf{a-ngu=no} a-chequ qhi-ve.\\
3\textsc{sg}-father\textit{=no} \textsc{nrl}-spear\textit{=no} \textsc{nrl}-porcupine pierce-\textsc{vm}\\
\glt ‘His father impaled the porcupine with the spear.’  %[elicited]
\z

Syncretism between the agentive and instrumental (and sometimes the ablative) is widespread across languages \citep{Garrett1990Origin} and found throughout \ili{Tibeto-Burman} (\citealt{DeLancey1984Etymological,LaPolla1995Ergative,Noonan2009Patterns}). In the family, one finds numerous morphemes with the form \textit{nV} (where V is a vowel) that have been glossed as ‘ergative’, ‘instrumental’ or ‘ablative’. Consequently, this makes it difficult to determine whether the agentive function of \ili{Sumi} \textit{=no} is inherited from an earlier proto-language, or if it is an example of parallel \isi{grammaticalization} across languages of the family, as per \citet{LaPolla1995Ergative}.

In terms of directionality, the development of ergative / agentive markers from instrumental markers is well attested, \eg \citet{Garrett1990Origin}.\footnote{Note that \citet{Garrett1990Origin} does not rule out the possibility that some instrumental markers may reflect older ergatives.} However, \citet{Coupe2011Pragmatic} questions this particular pathway for the Ao languages (\ili{Tibeto-Burman}), which often display syncretism between the agentive, instrumental and allative. Rather, he posits a proto-Ao \textit{*na} which was a “semantically underspecified marker of location” and that it was pragmatic context that determined the “precise” \isi{semantic role} it marked, such as agent, instrument, goal, source etc.\footnote{\citet{Coupe2011Pragmatic} also shows that most synchronic ablatives in Ao languages are compounds of the locative + agentive / instrumental, and suggests that the original ablative in these languages was syncretic with the agentive and instrumental, as well as the allative.} 

In addition, in many \ili{Tibeto-Burman} languages, the agentive / ergative, like \ili{Sumi} \textit{=no}, does not simply mark \isi{agentivity}, but has been extended to other functions, including discourse pragmatic functions like contrastiveness and \isi{unexpectedness}. For example, in \ili{Lhasa} Tibetan, the ergative marker -\textit{s}/\textit{-gis} on an argument in certain monovalent clauses can give a \isi{contrastive focus} reading, \ie ‘this S, not someone else’, when accompanied by the “proper intonation” \citep{Tournadre1991Rhetorical}. In \ili{Mongsen Ao}, the agentive \textit{nə} can be used to indicate willfulness, in addition to intentionality \citep[157]{Coupe2007Grammar}. In terms of directionality, it has been demonstrated in some languages, the discourse pragmatic morphemes have developed from the \isi{semantic role} markers (\eg \citealt{Chelliah2009Semantic} for \ili{Meithei}), following the expected path from more concrete to more abstract meaning (\citealt{Heineetal2002World}, \textit{inter alia}).

However, once again given the presence of numerous potential \textit{nV} cognates across the family, it is difficult to use cross-linguistic data to determine the extent to which the functions of \textit{=no} in \ili{Sumi} as both an agent marker and a \isi{focus marker} is something that was inherited from an ancestor language, or is an example of parallel drift within the \ili{Tibeto-Burman} family. It would perhaps be useful to look beyond the marking of A and S and examine morphological marking in other parts of the grammar.

\subsection{ Morphological marking of adverbial adjuncts}\label{13-te-sec:4.4}

To further understand the historical development of \textit{=ye} and \textit{=no}, one area for further research is the marking of adverbial adjuncts in \ili{Sumi}. Like S arguments, these adjuncts show a three-way opposition in morphological marking. In \REF{13-te-ex:40}--\REF{13-te-ex:42}, there are examples of adjuncts marked by \textit{=ye, =no} or by neither enclitic, respectively. These examples are important to consider, since they appear to have similar discourse pragmatic functions, \eg \isi{contrastive focus}, to what have been described for S argument markers.

\ea\label{13-te-ex:40}
\langinfo{Sumi}{Tibeto-Burman}{Kutili\_bird\_short, line 8}\\
\gll  Ishi=ke=hu pa=ye \textbf{ghulo} \textbf{lakhi=ye}, “Ei {...}”\\
like.this=\textsc{nzr=dist} \textsc{3sg}\textit{=ye} day one\textit{=ye} \textsc{excl}\\
\glt ‘So one day she thought to herself, “Oh ...”’
\z

\ea\label{13-te-ex:41}
\langinfo{Sumi}{Tibeto-Burman}{Kutili\_bird\_short, line 33}\\
\gll Pa=ye \textbf{khaghi=no} o=pütsa-ni pi u-va-e.\\
3\textsc{sg}\textit{=ye} long.time\textit{=no} 2\textsc{sg}=talk.to-\textsc{pros} say go-\textsc{prf}-\textsc{emph}\\
\glt ‘She said a long time ago that she was going to see you and left.’
\z

\ea\label{13-te-ex:42}
\langinfo{Sumi}{Tibeto-Burman}{Origin\_of\_axone, line 7}\\
\gll  \textbf{A-tsala} \textbf{a-küthü-ni-u} a-lu=lo ilesü hu-ghi=no\\
\textsc{nrl}-day \textsc{nrl}-three-\textsc{ord}-\textsc{def} \textsc{nrl}-field=\textsc{loc} return go.field-come=\textsc{conn}\\
\glt ‘On the third day, they returned to the field’
\z

The question here is: did such marking on adjuncts arise prior to, parallel to, or even after differential A and S marking? For example, one  might posit a locative function and origin for \textit{=ye} in \REF{13-te-ex:40}, but it cannot be assumed that its development followed the same diachronic pathway as \textit{=ye} in \textit{pa=ye}, in the same example. One also cannot easily posit an origin for \textit{=no} in \REF{13-te-ex:41}.

If the development of differential A and S marking has been driven to some extent by \isi{information structure}, it is important to understand how pragmatic discourse factors have influenced other aspects of the grammar, including cleft / cleft-like constructions and the marking of relative clauses. Such work would benefit from the use of experimental methods typically used to study the role of prosody in \isi{information structure}, including questionnaires and other tasks designed to elicit semi-spontaneous speech (\eg \citealt{Skopeteasetal2006Questionnaire,Hellmuthetal2007Intonational}).


\section{Summary and further questions}\label{13-te-sec:5}

In this paper, I first looked at the distribution of A and S marking in \ili{Sumi}, and showed that \ili{Sumi} has a two-way contrast for A and a three-way contrast for S, but no morphological marking of non-pronominal O. This is markedly different from  closely related languages such as \ili{Mao} and \ili{Khezha} that show a two-way opposition for O, in addition to a two-way opposition on S and A arguments.

Next, I examined some of the triggers of differential A and S marking in \ili{Sumi}. It was shown that in transitive clauses, differential A marking is determined largely by the \isi{agentivity} of the A argument, \ie the degree of volition, control and purpose of the A argument. In \isi{intransitive} clauses, it was shown that differential S marking was determined mainly by discourse pragmatic functions such as continuing reference, \isi{contrastive focus}, and the marking of \isi{unexpectedness}. Furthermore, some of these functions seem to influence differential A marking even in transitive clauses, although the extent to which this is the case remains a topic for further investigation.

I then considered the origins of such differential markers in \ili{Sumi}. It was hypothesized that \textit{=ye} in transitive clauses developed from an old locative marker. It was further speculated that \textit{=ye} in \isi{intransitive} and \isi{equative} clauses may have developed from an old copula.\footnote{To account for the same form \textit{=ye} used in transitive, \isi{intransitive} and \isi{equative} clauses, one might have to further speculate that the old locative marker and \isi{equative} copula both derive from an older locative copula.} No clear \ili{Sumi}-specific origin could be presented for the agentive / \isi{focus marker} \textit{=no}, given that cognates of \textit{=no} are found throughout the \ili{Tibeto-Burman} family – these typically function as agentives or ergatives, but also instrumentals and ablatives, and can have discourse pragmatic functions like marking \isi{contrastive focus}.

There are still many questions to be answered regarding the distribution of \textit{=ye} and \textit{=no} in \ili{Sumi}, as well as their diachronic origins. Future research will also need to look at the morphological marking of adjuncts and relative clauses. Such work would benefit from corpus studies based on naturalistic data, as well as the use of experimental tasks designed to elicit and identify \isi{information structure} categories. 

\section*{Acknowledgements}
I am indebted to my Sumi language consultants and friends, especially Inotoli Zhimomi, Abokali Jimomi, Canato Jimomi and Zhekügha Assümi, I am also grateful for comments from the editors and an anonymous reviewer.


\section*{Abbreviations}

\begin{tabularx}{.45\textwidth}{lQ}
1 &  1\textsuperscript{st} person\\
2  & 2\textsuperscript{nd}  person\\
3 &  3\textsuperscript{rd}  person\\
\textsc{conj} & conjunction\\
\textsc{conn} & connective\\
\textsc{cont} & continuative aspect\\
\textsc{def} & definite\\
\textsc{dist} & distal\\
\textsc{du} & dual\\
\textsc{emph} & emphatic\\
\textsc{erg} & ergative\\
\textsc{excl} & exclamation\\
\textsc{exist} & existential verb\\
\textsc{hab} & habitual aspect\\
\textsc{loc} & locative\\
\textsc{med} & medial\\
\textsc{ncap} & non-capability\\
\textsc{ncpl} & non-completive\\
\end{tabularx}
\begin{tabularx}{.54\textwidth}{lQ}
\textsc{neg} & negative\\
\textsc{nom} & nominative\\
\textsc{npst} & non-past tense\\
\textsc{nrl} & non-relational (unpossessed noun)\\
\textsc{nzp} & nominalizing prefix\\
\textsc{nzr} & nominalizing enclitic\\
\textsc{ord} & ordinal number\\
\textsc{part} & particle\\
\textsc{pl} & plural\\
\textsc{prf} & perfect aspect\\
\textsc{prx} & proximal\\
\textsc{pst} & past tense\\
\textsc{prog} & progressive aspect\\
\textsc{pros} & prospective aspect\\
\textsc{q} & question particle\\
\textsc{quot} & quotative\\
\textsc{sg} & singular\\
\textsc{vm} & verbal marker\\
\end{tabularx}

{\sloppy
\printbibliography[heading=subbibliography,notkeyword=this] 
}

\end{document}
