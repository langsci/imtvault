\documentclass[output=paper]{LSP/langsci}
\ChapterDOI{10.5281/zenodo.1228249} 
\author{Chantal Melis\affiliation{Universidad Nacional Autónoma de México}}
\title{Spanish indexing DOM, topicality, and the case hierarchy}
%\epigram{Change epigram in chapters/03.tex or remove it there}
\abstract{The Spanish language is known for its widespread phenomenon of \textit{Differential Object Marking} (DOM). A particularly interesting feature of DOM in contemporary Spanish relates to the obligatory use of a double system of marking – a “flagging” preposition and an “indexing” clitic \citep{Haspelmath2005Argument} – in the domain of the full personal pronouns. The prepositional marker goes back to the very beginnings of the language, whereas the cross-referencing strategy, also called \textit{clitic doubling}, is the product of a much later development, which joining forces with the existing older form gave rise to the twice-marked pronouns. In this paper I focus on the origin of Spanish indexing DOM and, through a careful examination of the first contexts of use, I propose that the relevant notion of “topicality” implicated in the evolution of indexing DOM is not animacy, but has to do with the role participants play in the event structure and the organization of these roles into a topical case hierarchy \citep{Givon1976Topic}.

% \keywords{ clitic doubling, DOM, personal pronouns, Spanish, topicality}
}
\maketitle

\begin{document}

\section{Introduction}\label{04-me-sec:1}

It is known that \ili{Spanish} has a robust system of \textit{Differential Object Marking} \citep{Bossong1991Differential,Bossong1998Marquage}. A particularly interesting feature of \ili{Spanish} DOM is that some direct objects require double marking. This phenomenon characterizes the stressed object personal pronouns, which in present day \ili{Spanish} impose the use of the preposition \textit{a} along with the presence of an unstressed person form – a verbal clitic – showing the relevant agreement features with the pronominal object phrase: 

\begin{exe}
\ex%1
\label{04-me-ex:1}
%\langinfo{lg}{fam}{src}\\
\gll Porque tú \textbf{me} am-as \textbf{a} \textbf{mí}  ¿no es cierto?\\
 because you.\textsc{nom} I.\textsc{acc} love-\textsc{prs.2sg} \textsc{acc} I not is certain\\
\glt ‘Because you love me, right?’ (1999, Jorge Volpi, En busca de Klingsor, \textsc{crea})
\end{exe}

 The preposition \textit{a} represents the older and more basic instrument of \ili{Spanish} DOM, traceable to the earliest texts. According to the hypothesis outlined in \citet{Pensado1995Creacion}, and now commonly accepted (\citealt{Torrego1999Gramatica}; \citealt{Leonetti2004Specificity}; \citealt{Iemmolo2010Topicality}), the development of \textit{a} into a DOM marker has its roots in contexts where \ili{Latin} \textit{ad}, meaning ‘with regard to, as to’, indicated a shift of topic.\footnote{The topicalizing function of \ili{Latin} \textit{ad} manifests itself in examples where the topic shifter forms part of a larger phrase (\textit{quod ad me attinet, X} ‘as far as I am concerned, X’, \textit{quod ad Xenonem, X} ‘as for Xenon, X’), and in contexts where it is used alone (\textit{ad ea autem, quae scribis de testamento, X} ‘with regard to what you write about the will, X’) \citep{Pensado1995Creacion}.} The topicalizing function of \textit{ad} was passed on to various Romances via vulgar \ili{Latin} – initially confined to the personal pronouns of first and second person in a dative or accusative role – and from there evolved towards the grammaticalized use of a differential object marking in \ili{Spanish} \citep{Pensado1995Creacion}.\footnote{Pensado reconstructs the historical path of \ili{Spanish} \textit{a} on the basis of a careful examination of Vulgar \ili{Latin} and early Romance data, which support her hypothesis that the origin of \textit{a} as a topic marker goes back to a construction of Vulgar \ili{Latin} restricted to the personal pronouns of first and second person both dative and accusative; \cf \textit{Ad mihi, (mihi) dixit} ‘To me, he told (me)’ or \textit{Ad mihi, (me) amat} ‘Me, he loves (me)’ \citep[203]{Pensado1995Creacion}. Her proposal to situate the beginnings of Romance DOM in the area of the personal pronouns, as she notes, ties in with what other authors had pointed out in the past (\citealt{Meier1948Acusativo}; \citealt{Rohlfs1971Autour}).}

 More specifically, the history of \ili{Spanish} \textit{a} shows the evolution of a DOM marker extending gradually downwards along the \isi{animacy hierarchy}, in close interaction with a parameter of \isi{definiteness} (\citealt{Garciaetal1995Palabra}; \citealt{Melis1995Objetodirecto}; \citealt{Aissen2003Differential}; \citealt{vonHeusingeretal2005Evolution}; \citealt{Laca2006Objeto}). The evolutionary path emerges from comparing the situation reflected by the earliest available text (\textit{Cantar de mio Cid}, dating most probably from around the turn of the 13th century; \cf \citealt[8]{Montaner1993Cantar}) with that of contemporary \ili{Spanish}. At the beginning, one observes a compulsory use of \textit{a} with both the stressed personal pronouns and the human-referring proper names, as opposed to the incipient and optional marking of the common nouns indicating \isi{definite} (sets of) individuals. In today’s \ili{Spanish}, on the other hand, following the progressive descent of the preposition from \isi{definite} to non-specific \isi{indefinite} persons, \textit{a} introduces nearly all human objects, while the \isi{inanimate} objects are usually left unmarked. To illustrate the prevailing situation in contemporary \ili{Spanish}, \citet[1781–1782]{Torrego1999Gramatica} offers this contrast:

\begin{exe}
\ex%2
\label{04-me-ex:2}
\begin{xlist}
\ex%2a
\label{04-me-ex:2a}
%\langinfo{lg}{fam}{src}\\
\gll Traj-eron \textbf{a} un amigo con ellos.\\
 bring-\textsc{pfv.3pl} \textsc{acc} a friend with them\\
\glt ‘They brought a friend with them.’ 
\ex%2b
\label{04-me-ex:2b}
%\langinfo{lg}{fam}{src}\\
\gll Traj-eron una maleta con ellos.\\
bring-\textsc{pfv.3pl} a suitcase with them\\
\glt ‘They brought a suitcase with them.’ 
\end{xlist}
\end{exe} 

 \ili{Spanish} \textit{a} thus profiles a well-attested path of evolution for DOM markers. Descending from an \textit{as-for} topic expression of late \ili{Latin}, the preposition is originally used to signal the promotion of a salient pronominal object referent to the status of clausal topic in a pragmatically marked construction. With the passage of time, however, as suggested by the earliest records of the \ili{Spanish} language, the preposition evolves into a differential marking device extended to direct objects which are no longer topics, but which conserve features of topicworthiness such as \isi{animacy} and \isi{definiteness} \citep{Iemmolo2010Topicality}.\footnote{I must add that even though the split between human and non-human objects is held to define \ili{Spanish} DOM \citep[82]{Leonetti2004Specificity} – the status of the non-human \isi{animate} objects being unclear \citep[4]{vonHeusinger2008Verbal} –, the system is actually much more complex owing to its sensitivity to various factors beyond \isi{animacy}, such as the properties of \isi{definiteness} and/or specificity of the object referents, the aspectual features of the predicate, the semantics of the event denoted by the verb, as well as the relation holding between the subject and the object. The interaction of \isi{animacy} with these additional parameters, accounting for the appearance of unmarked human objects and \textit{a}-marked \isi{inanimate} objects in specific discourse contexts, has been examined in a number of studies with important contributions to our understanding of the intricacies of \ili{Spanish} DOM (\citealt{Kliffer1984Personal,Pensado1995Complemento,Torrego1999Gramatica,Company2002Grammaticalization,Delbecque2002Construction,Aissen2003Differential,vonHeusingeretal2003Interaction,Leonetti2004Specificity,Garcia2007Inanimate,GarciaGarcia2014Objektmarkierung,vonHeusinger2008Verbal}; among others).} 

 The second instrument of \ili{Spanish} DOM, on the other hand, is the product of a more recent development, which the obligatorily \textit{a}-marked strong personal pronouns begin to undergo around the turn of the 16th century, that is to say, during the transition period between medieval and renaissance \ili{Spanish} (\citealt[83]{Keniston1937Syntax}; \citealt{Silva-Corvalan1984Semantic}; \citealt{Rini1991Redundant}; \citealt{Gabrieletal2010Information}). As shown in \REF{04-me-ex:1} above, the new device consists of a coreferential clitic pronoun, morphologically bound (but not attached) to the verb. This phenomenon is known as \textit{clitic doubling} (for the definition of the coreferential forms in terms of clitics, see \sectref{04-me-ex:2}), and the relation of \isi{clitic doubling} to DOM has been acknowledged \citep[221–224]{Bossong1998Marquage}. Indeed, the use of the coreferential pronoun in \ili{Spanish} separates the higher-ranked pronouns, which display the clitic along with \textit{a}, from the lower-ranked nominal objects, marked with \textit{a} alone (human) or taking no marking (\isi{inanimate}).

 In \citeauthor{Haspelmath2005Argument}'s (\citeyear{Haspelmath2005Argument}: 2) 
 terminology, \ili{Spanish} \textit{a} instantiates the “flagging” type of argument marking (= coding by case affixes and adpositions), whereas the clitic corresponds to the “indexing” type (= cross-referencing or agreement). From this perspective, the peculiarity of the stressed object personal pronouns of \ili{Spanish} resides in conjoining two kinds of DOM: flagging DOM (\textit{a}) and indexing DOM (clitic). Another way of referring to the double marking of the \ili{Spanish} pronouns is proposed by \citet{Iemmolo2014Dislocated}, who reserves the label DOM for the flagging type of marking and calls the other mechanism DOI (Differential Object Indexation).

 The \isi{clitic doubling} strategy employed for the purpose of \ili{Spanish} DOM is the central topic of the present paper. Scholars have been interested in the question of why the personal pronouns became subjected to the new type of marking, and different hypotheses have been put forward (\citealt{Silva-Corvalan1984Semantic}; \citealt{Rini1991Redundant}; \citealt{Gabrieletal2010Information}). None, however, as we shall see, manages to satisfactorily account for the change associated with the turn of the 16th century. This leaves room for a new attempt at explaining how the change came about. The way I intend to approach the topic at hand is through a careful examination of textual sources in which the new type of marking has the character of an incipient phenomenon, affecting a few pronouns and leaving the rest untouched, under the assumption that defining what the selected few share in common may shed light on the original motivating force behind the change. 

   
 As background to the analysis I will present, two important facts have to be mentioned. First, it should be pointed out that the introduction of \isi{clitic doubling} into the pronominal domain did not imply a creation in the strict sense of the word; the coreferential form had a long history of appearing in topicalizing constructions, known as left- and right-dislocations, where it was used to bind a detached object constituent to the core clause. So it will be necessary to look at these structures in order to understand how they prepared the way for the development of indexing DOM with the pronouns.

 Second, it has to be borne in mind that the stressed object pronouns affected by the change have been emphatic forms throughout the history of \ili{Spanish}. Their selection in specific discourse contexts always signals a deliberate intention on the part of the speaker to highlight something about the referent of the pronoun. More will be said below on the split between stressed and unstressed forms within the \ili{Spanish} personal pronoun system. For the moment, the fitting observation is that the development of \isi{clitic doubling} as a device for DOM cannot be explained without taking into account the crucial emphatic value of the targeted pronominal items.

 Anticipating the results of my analysis, I will argue that the emergence of indexing DOM in \ili{Spanish} appears to have involved a notion of \isi{topicality}, but not one in which \isi{animacy} was the relevant feature, in contrast to \textit{a}. As suggested by \citet[152]{Givon1976Topic}, \isi{topicality} should be visualized as encompassing a number of binary hierarchic relations, among which the author includes one that concerns the \textit{role} of participants in the \isi{event structure}. On the role dimension, entities are ranked according to the degree to which their participation contributes to the coming about of the event (more involved participant > less involved participant). This binary relation is assumed to underlie the \isi{case hierarchy} (agent > dative > accusative), in which the more “topical” participants, in addition to being typically human and \isi{definite}, rank above the accusative object from the point of view of their higher degree of involvement in the action. My aim is to show that the \isi{grammaticalization} of indexing DOM in \ili{Spanish} closely interacted with this specific dimension of \isi{topicality}. The primary evidence for this proposal is that \ili{Spanish} indexing DOM will be seen to favor the dative pronouns before generalizing to all personal object pronouns (indirect and direct). Further support comes from the later extension of indexing DOM to the indirect (not direct) object noun phrases. 

 The interaction between flagging and indexing DOM in \ili{Spanish} thus offers a complex panorama of historical developments, which can be divided in three major stages:


\begin{itemize}
\item \begin{styleListParagraph}
throughout medieval \ili{Spanish}, flagging DOM and the indexing device (in topic constructions) operate independently from one another (see  \sectref{04-me-sec:3});
\end{styleListParagraph}
\item \begin{styleListParagraph}
in renaissance \ili{Spanish}, indexing DOM becomes a grammaticalized feature of the personal object pronouns, both dative (marked by a homophonous \textit{a} form) and accusative (obligatorily DOM flagged). This is the period in which the two types of DOM meet, and their convergence is the focus of the present paper;
\end{styleListParagraph}
\item \begin{styleListParagraph}
in modern \ili{Spanish}, indexing DOM spreads to the dative noun phrases, whereas the nominal direct objects only show flagging DOM or are left unmarked (\citealt{Melisetal2009Interplay}, and see below  \sectref{04-me-sec:4.1}).
\end{styleListParagraph}
\end{itemize}

 The paper is organized as follows. \sectref{04-me-sec:2} provides a brief overview of the object personal pronouns of \ili{Spanish}. In  \sectref{04-me-sec:3}  the older use of the coreferential pronoun with dislocated object phrases is examined. \sectref{04-me-sec:4} is dedicated to the development of \ili{Spanish} indexing DOM: The general properties of the diachronic change are sketched in \sectref{04-me-sec:4.1}; previous approaches are discussed in \sectref{04-me-sec:4.2}; the hypothesis set forth in this paper is outlined in \sectref{04-me-sec:4.3}; the corpus of data is described in \sectref{04-me-sec:4.4}; and the analysis of the data is carried out in \sectref{04-me-sec:4.5}. \sectref{04-me-sec:5} concludes with a summary of the paper. 

\section{The Spanish object person forms}
\label{04-me-sec:2}

For the purpose of this paper, a brief introduction to the \ili{Spanish} personal pronoun system will be helpful. Of specific interest are the object pronouns, which show a division into stressed and unstressed forms. The former are referred to in terms of “full”, “strong” or “tonic” pronouns, whereas the latter are called “weak” pronouns or “clitics”. In \tabref{04-me-tab:1}, a simplified picture of the object paradigm based on \citet[119]{Pennyetal1991History} is presented. It is important to observe (for the change to be discussed) that across the paradigm, with a few exceptions in the third person area, identical forms cover both the accusative and dative realizations of the pronouns.\footnote{The segments in parenthesis indicate changes that took place in late Old \ili{Spanish} (end of 15th cent.), namely, the reduction \textit{vos} > \textit{os} and the expansion \textit{vos} > \textit{vosotros}, followed at a later stage by the analogical expansion \textit{nos} > \textit{nosotros}. I have excluded the contemporary deferential forms of address \textit{usted} and \textit{ustedes}. Nor does my overview mirror the early phenomenon of \textit{leísmo} (which continues in standard Peninsular \ili{Spanish}), whereby the dative form \textit{le} is used as a \isi{direct object} form with masculine referents.} 

\begin{table}
\caption{The Spanish object person forms}
\label{04-me-tab:1}
\begin{tabular}{llllll}
\lsptoprule
& & \multicolumn{2}{c}{\textsc{accusative}} & \multicolumn{2}{c}{\textsc{dative}}\\\cmidrule(lr){3-4}\cmidrule(lr){5-6}
& & stressed & unstressed & stressed & unstressed\\\midrule
\oldstylenums{1} \textsc{sg} &        & \textit{mí}         & \textit{me}    & \textit{mí}         & \textit{me}\\
\oldstylenums{2} \textsc{sg} &        & \textit{ti}         & \textit{te}    & \textit{ti}         & \textit{te}\\
\oldstylenums{3} \textsc{sg} & masc.  & \textit{él}         & \textit{lo}    & \textit{él}         & \textit{le}\\ 
     & fem.   & \textit{ella}       & \textit{la}    & \textit{ella}       & \textit{le}\\ 
     & neuter & \textit{ello}       & \textit{lo}    & \textit{ello}       & \textit{le}\\ 
\oldstylenums{1} \textsc{pl} &        & \textit{nos(otros)} & \textit{nos}   & \textit{nos(otros)} & \textit{nos}\\
\oldstylenums{2} \textsc{pl} &        & \textit{vos(otros)} & \textit{(v)os} & \textit{vos(otros)} & \textit{(v)os}\\
\oldstylenums{3} \textsc{pl} & masc.  & \textit{ellos}      & \textit{los}   & \textit{ellos}      & \textit{les}\\
     & fem.   & \textit{ellas}      & \textit{las}   & \textit{ellas}      & \textit{les}\\
\lspbottomrule
\end{tabular}
\end{table}

 When a language possesses a pronominal system with a similar division, it is usually the case that the unstressed, that is, phonologically attenuated, forms encode highly topical and cognitively accessible referents \citep[174]{Siewierska2004Person}. This tendency is confirmed by \ili{Spanish}, where the weak object pronouns are, and always have been, the canonical forms used to refer to the participants that are deictically or anaphorically anchored in the discourse (\cf \textit{me} \textit{vio} ‘(s)he saw me’; \textit{lo} \textit{vi} ‘I saw him’).

 What did change in the course of time is the grammatical status of the weak object pronouns. These began as phonologically bound forms, which had to “lean” on a preceding or following word for accentual reasons, but enjoyed a certain degree of independence from a syntactic point of view. Over time, however, the weak object pronouns were led to transform into elements definable as clitics on the basis of their morphological binding to the verb (immediately preceding the finite verb or attached at the end of imperative and non-finite verbal forms). As discussed in the literature, the cliticization of the weak pronouns – product of a gradual loss of positional and combinatory options on the syntactic level – was completed by the early 17th century (\citealt{Rivero1986Parameters}; \citealt{Rini1990Dating}; \citealt{Fontana1993Phrase}; \citealt{FernandezSoriano1999Pronombre}; \citealt{Nieuwenhuijsen2006Cambios}).

 It is worth noting that the period during which the weak pronouns were evolving into clitics (15th–16th century) more or less coincides with that of the rise of indexing DOM. A relation between the two phenomena has to be established, since the development of a type of object-verb agreement in the area of the strong personal pronouns was no doubt facilitated by the newly acquired clitic status of the weak person forms (\citealt{Rini1990Dating}; \citealt{Enrique-Arias2003Clitics}).

 The strong object personal pronouns, on the other hand, behave like (prosodically and morphologically) independent noun phrases, associated with one peculiar feature: they are emphatic. So when a strong pronoun surfaces in discourse, some kind of special effect is intended, typically, a contrast: the individual encoded by the pronoun is compared or opposed to other referents, whether explicitly or implicitly \citep{Lujan1999Expresion}. 

 To illustrate, consider this pair of late medieval examples:

\begin{exe}
\ex%3
\label{04-me-ex:3}
\begin{xlist}
\ex%3a
\label{04-me-ex:3a}
%\langinfo{lg}{fam}{src}\\
\gll pues diz-es que \textbf{me} am-as\\
 since say-\textsc{prs.2sg} \textsc{comp} I.\textsc{acc} love-\textsc{prs.2sg}\\
\glt ‘since you say that you love me’ (15th c., Bursario, \textsc{corde})
\ex%3b
\label{04-me-ex:3b}
%\langinfo{lg}{fam}{src}\\
\gll miémbr-a-te que por am-ar \textbf{a} \textbf{mí} […] mat-aste a tres hermanos míos\\
remember-\textsc{imp-2sg.refl} \textsc{comp} for love-\textsc{inf} \textsc{acc} I { } kill-\textsc{pfv.2sg} \textsc{acc} three brothers mine\\
\glt ‘remember that for the sake of loving me […] you killed three of my brothers’ (15th c., Bursario, \textsc{corde})
\end{xlist}
\end{exe}

In \REF{04-me-ex:3a} the weak form \textit{me} corresponds to the way in which a first person functioning as direct or \isi{indirect object} is expected to appear in most discourse contexts. But on occasion, as in \REF{04-me-ex:3b}, the speaker chooses the tonic instead (accusative \textit{mí} preceded by obligatory \textit{a}), the emphatic force of which is in this case called upon to underscore the explicit contrast established between an act of love and a triple murder. The pronouns that will undergo \isi{clitic doubling} are these emphatic forms.

\section{Coreferential weak pronouns in left- and right-\-dislocations}\label{04-me-sec:3}

In this section, left- and right-dislocated sentences motivating the occurrence of a coreferential pronoun are examined. They constitute a very old phenomenon that is present in the earliest \ili{Spanish} texts and indeed goes back to \ili{Latin}, where topicalized constituents were often accompanied by a resumptive pronoun \citep[198]{Pensado1995Creacion}. From the point of view of this study, the dislocations in question are of particular significance because scholars have relied on these pragmatically marked structures to explain the origin of \ili{Spanish} indexing DOM. At the end of the section, the plausibility of tracing the indexing device back to the dislocations will be evaluated.

The most suitable text to examine the older function of the coreferential pronoun is the epic poem \textit{Cantar de mio Cid}, especially rich in examples \citep[323]{MenendezPidal1964Cantar}. These are built with different kinds of object phrases. Their common property lies in the peripheral position the object occupies on the left or right end, along with the occurrence of a coreferential pronoun in the core clause. For example, in \REF{04-me-ex:4}, the DOM flagged \isi{direct object} (\textit{a las sus fijas} ‘his daughters’) has been detached to the left periphery and is resumed by the weak form \textit{las}, which reproduces the case, gender and number features of the detached noun:\footnote{The examples of the \textit{Cid} are cited from Montaner's (1993) edition.}

\begin{exe}
\ex%4
\label{04-me-ex:4}
%\langinfo{lg}{fam}{src}\\
\gll \textbf{a} \textbf{las} \textbf{sus} \textbf{fijas} {\textbar} en braços \textbf{las} prend-ía\\
\textsc{acc} the his daughters { } in arms they.\textsc{ acc.fem} take-\textsc{ipfv.3sg}\\
\glt ‘his daughters, he embraced them’ (v 275)
\end{exe}
 

The vertical bar in \REF{04-me-ex:4} symbolizes the caesura, indicative of an intonation break in the recitation (\citealt[71]{Gabrieletal2010Information}, with a reference to \citealt[263]{Fontana1993Phrase}), and in this sense helpful for the recognition of a dislocated structure.\footnote{\citet[400]{MenendezPidal1964Cantar} discusses another diagnostic for the identification of pragmatically marked structures in the \textit{Cantar}, related to the position of the coreferential pronoun in the structure of the verse.} With respect to left dislocations, \citet[183]{Lambrecht1994Information} points out that they are often used “to mark a shift in attention from one to another of two or more already activated topic referents”. The \textit{Cantar de mio Cid} illustrates this nicely insofar as many of its left dislocations involve central figures of the poem, for instance, the Cid’s daughters, as in \REF{04-me-ex:4}.

 In \REF{04-me-ex:5}, a DOM-flagged strong personal pronoun (\textit{a vós} ‘you’) occupies the right periphery and is accompanied by the coreferential form \textit{vos}:\footnote{In \REF{04-me-ex:5} the plural form \textit{vos} is used as a deferential form of address.}

\begin{exe}
\ex%5
\label{04-me-ex:5}
%\langinfo{lg}{fam}{src}\\
\gll aquéllas \textbf{vos} acomiend-o {\textbar} \textbf{a} \textbf{vós}, abbat don Sancho\\
those.\textsc{acc.fem} you.\textsc{dat} entrust-\textsc{prs.1sg} {} \textsc{dat} you abbot don Sancho\\
\glt ‘I now entrust those [girls] to you, you abbot don Sancho’ (v 256)
\end{exe}

Right dislocations, also called afterthought-topics, are less easily recognizable in \ili{Spanish}, because the detached object may appear as if it were occupying the canonical postverbal position (of direct and most types of indirect objects). In this example, however, the correct analysis gains support from the presence of the caesura.\footnote{In modern spoken language, right-detached constituents are characterized by a number of defining prosodic features (\citealt[765]{Anagnostopoulou1999Conditions}; \citealt[852]{Escandell-Vidal2009Differential}; \citealt[64–65]{Gabrieletal2010Information}).} Right dislocations pose an additional challenge to the extent that their function in discourse continues to be a matter of some dispute. Broadly speaking, they are supposed to bear on the identity of the referent of the coreferential form in the core clause, adding explicitness for the benefit of the addressee \citep[2003]{Lambrecht1994Information}, or, from a wider perspective, providing an informational “update” that is meant to replace, correct or partially adjust elements contained in the core clause (\citealt[856–859]{Escandell-Vidal2009Differential}, following \citealt{Vallduvi1992Informational}). The right dislocations of the \textit{Cid} would have to be examined in detail in order to verify these proposals. 

Of greater interest to us is the fact that the detached objects in \REF{04-me-ex:4} and \REF{04-me-ex:5} are both marked with \textit{a.} The co-occurrence of \textit{a} and the coreferential pronoun in the dislocations of the \textit{Cid} explains why it has been claimed that the two devices have a long history of working jointly in the service of \ili{Spanish} DOM (\citealt{Laca1995Acusativo}; \citealt{Melis1995Objetodirecto}; \citealt{Leonetti2004Specificity,Leonetti2008Specificity}).\footnote{Some decades ago, a strong hypothesis regarding the interaction between \isi{clitic doubling} and prepositional DOM received expression in what came to be known as “Kayne´s generalization”, which stated that for an object noun phrase to be doubled by a clitic it had to be preceded by a preposition (\citealt{Kayne1975French}; \citealt{Jaeggli1982Topics}). The hypothesis has been refuted on the basis of empirical data – doubling clitics appear alone (\citealt{Suner1988Agreement}; \citealt{Anagnostopoulou1999Conditions}; \citealt{Leonetti2008Specificity}) – but it continues to raise expectations about potential co-occurrences of the two marking mechanisms.} Yet the truth is that flagging \textit{a} and the resumptive weak form operate independently from one another, as the following data show. Indeed, in the poem, \textit{a} marks the stressed personal pronouns and the human-referring proper names obligatorily and is used optionally with human \isi{definite} nouns, but none of these objects are cross-referenced if they appear in a sentence bearing no sign of dislocation:

\begin{exe}
\ex%6
\label{04-me-ex:6}
%\langinfo{lg}{fam}{src}\\
\gll Oí-d \textbf{a} \textbf{mí}, Álbar Fáñez {\textbar} e todos los cavalleros.\\
 listen-\textsc{imp} \textsc{acc} I Alvar Fañez { } and all the knights\\
\glt ‘Alvar Fañez, and all the knights, listen to me.’ (v 616)
\end{exe}
 
And inversely, a resumptive pronoun tends to show up in dislocated structures, but if the topicalized element does not pertain to the class of direct objects that impose or attract flagging DOM, we find a coreferential pronoun without \textit{a}, as exemplified by the non-specific human referents in \REF{04-me-ex:7a} and the \isi{inanimate} entity in \REF{04-me-ex:7b}:

\begin{exe}
\ex%7
\label{04-me-ex:7}
\begin{xlist}
\ex%7a
\label{04-me-ex:7a}
%%\langinfo{lg}{fam}{src}\\
\gll \textbf{los} \textbf{moros} \textbf{e} \textbf{las} \textbf{moras} {\textbar} vend-er non \textbf{los} pod-remos\\
 the moorish.men and the moorish.women { } sell-\textsc{inf} not they.\textsc{acc.masc} be.able-\textsc{fut.1pl}\\
\glt ‘the moorish people, we won’t be able to sell them’ (v 619)
\ex%7b
\label{04-me-ex:7b}
%%\langinfo{lg}{fam}{src}\\
\gll mas \textbf{el} \textbf{castiello} {\textbar} non \textbf{lo} quier-o herm-ar\\
 but the castle { } not he.\textsc{acc} want-\textsc{prs.1sg} destroy-\textsc{inf}\\
\glt ‘but the castle, I don’t want to destroy it’ (v 534)
\end{xlist}
\end{exe}

 Furthermore, cross-referencing pronouns are also found to interact, as in \REF{04-me-ex:8}, with clausal object complements, which are never subjected to flagging DOM:

\begin{exe}
\ex%8
\label{04-me-ex:8}
%\langinfo{lg}{fam}{src}\\
\gll Ya \textbf{lo} ve-e el Cid {\textbar} \textbf{que} \textbf{d-el} \textbf{rey} \textbf{non} \textbf{av-ié} gracia.\\
now it.\textsc{acc} see-\textsc{prs.3sg} the Cid { } \textsc{comp} of-the king not have-\textsc{ipfv.3sg} grace\\
\glt ‘Cid now knew it, that he was out of favor with the king.’ (v 50)
\end{exe}

 In light of these data, one is able to conclude that \textit{a} and the coreferential pronoun in the \textit{Cid} have different functions. Whereas prepositional DOM marks the higher-ranked objects with human reference, the pronoun, unrelated to DOM, appears in pragmatically marked topic constructions, where its principal function is to bind the dislocated object constituent to the core clause (\citealt[84]{Keniston1937Syntax}; \citealt[109]{Nocentini2003Object}; \citealt[757]{RealAcademia2010Espanola}).\footnote{Worthy of note is the fact that the human \isi{definite} objects, which in the \textit{Cid} are optionally flagged, often occur in (left) dislocation structures accompanied both by \textit{a} and the coreferential pronoun. When they are not topicalized, besides lacking the pronoun, of course, these objects are also more likely to appear without \textit{a} \citep{Melis1995Objetodirecto}. A plausible explanation for this phenomenon is that the incipient use of \textit{a} with these objects still depends, in some measure, on the establishment of the referent of the object as the pragmatic sentence “topic”, and may therefore be viewed as a vestige of the beginnings of flagging DOM in the Romances (see  \sectref{04-me-ex:1}). The topicalizing phenomenon with the human \isi{definite} objects in the \textit{Cid} has been instrumental in creating the wrong impression that the preposition and the weak pronoun have worked jointly for DOM throughout the history of \ili{Spanish}.}

 It is possible that the widely used dislocations in the \textit{Cid} should be viewed as lingering traces of the oral tradition that is assumed to have given shape to the epic poem. What is certain is that the medieval texts posterior to the poem display an extremely scanty use of topicalizing constructions, and as a direct consequence of this decline in frequency coreferential pronouns become equally rare. That is to say, \textit{a} continues to extend downwards along the \isi{animacy hierarchy}, but the objects are not cross-referenced since they are not dislocated. The perception that coreferential pronouns with (\textit{a}-marked or unmarked) object constituents were not common during the post-\textit{Cid} medieval period of \ili{Spanish} is shared by all scholars who have dealt with this issue, in relation to flagging DOM (\citealt{Laca2006Objeto}; \citealt{vonHeusingeretal2005Evolution}), or from the angle of indexing DOM (\citealt{Silva-Corvalan1984Semantic}; \citealt{Rini1991Redundant}; \citealt{Fontana1993Phrase}; \citealt{Eberenz2000Espanol}; \citealt{Gabrieletal2010Information}; \citealt{Rozasetal2012Object}). This does not mean that topic constructions with coreferential pronouns died out. Actually, they continue to be in use today, but they appear as infrequently as in the medieval texts (for some quantitative data on fronted objects in contemporary \ili{Spanish}, see \citealt[215–216]{Garcia2015Variable}).

 So the question is whether the use of the coreferential pronoun in the examined constructions paved the way for the rise of indexing DOM, associated with the second historical period of the \ili{Spanish} language. It is tempting to motivate a link between the older use of the pronoun and the later development, in view of \citegen{Givon1976Topic} hypothesis about the rise of (subject and) object agreement markers as being due to a reanalysis of anaphoric pronouns in “over-used” topic-shift constructions, meaning, in constructions where the pragmatic motivation for the marked \isi{word order} had lost transparency. Under \citeauthor{Givon1976Topic}'s (\citeyear[156–157]{Givon1976Topic})  proposal, afterthought-topic constructions (right dislocations) are likely to be particularly relevant to the development of object agreement (\cf \textit{I saw} \textit{him}\textit{,} \textit{the man} > \textit{I saw-}\textit{him} \textit{the man}).

 It will be seen below (\sectref{04-me-sec:4.2}) that Givón’s hypothesis underlies some of the proposals that have been put forward to explain the rise of indexing DOM in \ili{Spanish}. Nevertheless, as argued in  \citet[279]{Rozasetal2012Object}, the envisaged scenario cannot be made to fit the \ili{Spanish} data with ease, considering that the poorly documented topic constructions of medieval times do not evoke the “overuse” established as a condition for the reanalysis of the anaphoric pronouns. Additionally, it turns out that the examples appearing in the medieval sources (\citealt[279]{Rozasetal2012Object}, with a reference to \citealt{Riiho1988Redundancia}) often display a dislocated subordinate clause cross-referenced with the neuter pronoun \textit{lo}, as in \REF{04-me-ex:8} above. An object of this nature is not what we think of when defining the notion of \isi{topicality}, nor does it in any way resemble the strong personal pronouns that will eventually attract the coindexing strategy.

 In short, the difficulty of tying indexing DOM immediately to the medieval dislocations is real, and everything seems to point in the direction of an innovative process of change, whereby the function of an available form – a coreferential pronoun – was expanded to satisfy a different purpose. 

\section{The grammaticalization of Spanish indexing DOM}\label{04-me-sec:4}

\subsection{Preliminaries}\label{04-me-sec:4.1}

The rise of \ili{Spanish} indexing DOM can be traced back to the turn of the 16th century, when an increase in the use of a weak pronoun with a (necessarily) DOM-flagged strong object personal pronoun becomes noticeable (\citealt[83]{Keniston1937Syntax}; \citealt{Silva-Corvalan1984Semantic}; \citealt{Rini1990Dating}; \citealt{Gabrieletal2010Information}). This increase is the signal of a change in process that would culminate with the \isi{grammaticalization} of object agreement in the pronominal domain, completed more or less by the end of the 17th century.\footnote{The proposed date of completion varies. Some authors associate it with the end of the 16th century, while others detect non-doubled pronouns until the 18th century (\citealt{Silva-Corvalan1984Semantic}; \citealt{Rini1990Dating}; \citealt{Gabrieletal2010Information}; \citealt{Rozasetal2012Object}). The discrepancy hinges on the nature of the data. As we will see below, the textual sources suggest that doubling was used with variable frequency depending on the individual speakers/writers. What is clear is that \isi{clitic doubling} grammaticalized a bit more slowly with the third person pronouns than with the speech act participants (see \sectref{04-me-sec:4.4} and footnote 17). 
My own perusal of data of the CORDE motivates my stating that instances of non-doubled pronouns are extremely rare after the 17th century.} 

 The examples of doubled pronouns in \REF{04-me-ex:9} come from the CORDE materials examined for the purpose of this study, on which more will be said below (\sectref{04-me-sec:4.4}):

\protectedex{
\begin{exe}
\ex%9
\label{04-me-ex:9}
\begin{xlist}
\ex%9a
\label{04-me-ex:9a}
%\langinfo{lg}{fam}{src}\\
\gll Por cierto, que \textbf{a} \textbf{mí} \textbf{me} pes-a mucho de su muerte.\\
of course that \textsc{dat} I I.\textsc{dat} grieve-\textsc{prs.3sg} much of his death\\
\glt ‘Of course, I very much lament his death.’ [lit. ‘it grieves me of his death’] (1555, Espejo, \textsc{corde})
\ex%9b
\label{04-me-ex:9b}
%\langinfo{lg}{fam}{src}\\
\gll Señor, {¿por qué} \textbf{me} d-ais cargo \textbf{a} \textbf{mí}?\\
sir why I.\textsc{dat} give-\textsc{prs.2pl} charge \textsc{dat} I\\
\glt ‘Sir, why do you accuse me?’ (1517, Arderique, \textsc{corde})
\ex%9c
\label{04-me-ex:9c}
%\langinfo{lg}{fam}{src}\\
\gll El que \textbf{a} \textbf{mí} aquí \textbf{me} trux-o no es el diablo  que {diz-es […]}\\
The who \textsc{acc} I here I.\textsc{acc} bring-\textsc{pvf.3sg} not is the devil that say-\textsc{prs.2sg}\\
\glt ‘The one who brought me here is not the devil as you say’ (1504, Esplandián, \textsc{corde})
\end{xlist}
\end{exe}
}

The examples exhibit strong pronouns that are collocated in different positions, without suggesting the presence of a recognizable dislocation. I found this to be true in the majority of cases.\footnote{Some dislocated structures did show up. In the following example, \textit{a mí} is detached to the left and the core clause begins with the subject pronoun \textit{uno}: \textit{Por cierto a mí uno solo me perdió, mas yo he perdido a muchos} ‘Actually, as far as I am concerned, only one fellow ruined me, whereas I ruined many’ (1520, \textit{Ysopo,}\textsc{ corde}). For a right dislocation, see  footnote 13 below. The dislocations were eliminated from my analysis.}

 Certainly, fronted pronouns, as in \REF{04-me-ex:9a}, are common, but they are selected by verbs with special characteristics like \textit{pesar} ‘to grieve, to lament’, whose \isi{experiencer} argument has always had a tendency to favor the sentence initial position \citep{Melisetal2013Historical}. During the period under study, the pronominal \isi{experiencer} of these verbs is frequently preverbal including when it is not doubled (see example \REF{04-me-ex:18a} below). Other pronouns occupy the sentence final position, as in \REF{04-me-ex:9b}, and may provoke ambiguity (perhaps a right dislocation), although nothing in their behavior differentiates them from the non-doubled tonic phrases which likewise appear at the right-most end (see example \REF{04-me-ex:14a} below).\footnote{A right dislocation was detected in a few examples, as in this one: \textit{porque Cortés me mostró la misma carta a mí y a otros conquistadores} ‘because Cortés showed me the same letter, to me and to other conquistadors’ (1568–75, \textit{Historia}, \textsc{corde}). The detached segment shows a coordinated structure in which the tonic pronoun appears together with a reference to another participant not evoked by the weak pronoun. This may be interpreted as an “update” of referential character aiming at rectifying – completing – the information given in the core clause, which is characteristic of right dislocations as discussed above.} The remaining pronouns occur in the middle of the sentence, as in \REF{04-me-ex:9c}, and are impossible to confuse with a detached constituent.

 The doubled pronouns also differ as to their status in the \isi{information structure} of the sentence. From the cognitive point of view, the entities coded in the form of personal pronouns are “prominent”, in the sense that the referent of the pronoun, beyond its presupposed condition of familiarity, is also the current center of attention of the speech participants (\citealt[770]{Anagnostopoulou1999Conditions}; \cf \citealt[94]{Lambrecht1994Information}). This explains the common assumption that pronouns are topics. However, with respect to the \isi{information structure} of a clause, pronouns may stand in different relations to the proposition, and may appear in the focus domain of an utterance, as part of the comment or as the sole constituent in focus \citep[128–130]{Lambrecht1994Information}.

 In fact, since we are dealing with stressed pronouns, a focus status would typically be expected \citep[183]{Siewierska2004Person}. But in \ili{Spanish} the strong personal pronouns are not necessarily focal \citep{Lujan1999Expresion}, and in the data I examined, as it turns out, a clear focal interpretation imposed itself in a few cases only. The pronouns were always contrastive foci. \REF{04-me-ex:10} may serve as an example:

\begin{exe}
\ex%10
\label{04-me-ex:10}
%\langinfo{lg}{fam}{src}\\
\gll y el visorey respond-ió: ``Matar-me h-an si salg-o." Aliaga dij-o: ``Primero \textbf{me} matarán \textbf{a} \textbf{mí}."\\
and the viceroy answer-\textsc{pvf.3sg} kill.\textsc{inf}-I.\textsc{acc} have-\textsc{prs.3pl} if leave-\textsc{prs.1sg} Aliaga say-\textsc{pvf.3sg} first I.\textsc{acc} kill-\textsc{fut.3pl} \textsc{acc} I\\

\glt ‘and the viceroy answered: “They will kill me if I come out”. Aliaga said: “They will kill me first”.’ (1555--84, Guerras, \textsc{corde})
\end{exe}

 In most cases, the doubled pronouns from my textual sources function as pragmatic topics. Some of them invite to be characterized in terms of “secondary topics” \citep{Nikolaeva2001Secondary}. This analysis is suggested for pronouns occurring in a clause which “in addition to conveying information about the topic referents conveys information \textit{about the relation} that holds between them as arguments in the proposition” \citep[148]{Lambrecht1994Information}. Consider \REF{04-me-ex:11}:

\begin{exe}
\ex%11
\label{04-me-ex:11}
%\langinfo{lg}{fam}{src}\\
\gll A esta Luscinda am-é, qu-ise y ador-é desde mis tiernos y primeros años, y ella \textbf{me} qu-iso \textbf{a} \textbf{mí}, con aquella sencillez y buen ánimo que su poca edad permitía.\\
\textsc{acc} this Lucinda love-\textsc{pfv.1sg} like-\textsc{pfv.1sg} and adore-\textsc{pfv.1sg} since my tender and first years and she.\textsc{nom} I.\textsc{acc} like-\textsc{pfv.3sg} \textsc{acc} I with that simplicity and good heart that her little age allowed\\
\glt ‘I loved, cherished and adored Lucinda since my early tender years, and she loved me with the simplicity and noble heart of her youth.’ (1605, Quijote, \textsc{corde})
\end{exe}

Other pronouns, in spite of being objects, must be viewed as encoding the entity the proposition expressed by the sentence is primarily about. Pronouns with a primary topic role, as in \REF{04-me-ex:12}, are very common in my data and will be discussed below (\sectref{04-me-sec:4.5}):

\protectedex{
\begin{exe}
\ex%12
\label{04-me-ex:12}
%\langinfo{lg}{fam}{src}\\
\gll que nunca sent-imos rumor de gente y \textbf{a} \textbf{mí} \textbf{me} parec-ió que deb-íamos sal-ir d-el pueblo de aquella manera\\
because never feel-\textsc{pfv.1pl} murmur of people and \textsc{dat} I I.\textsc{dat} seem-\textsc{pfv.3sg} that have-\textsc{ipfv.1pl} leave-\textsc{inf} from-the village of that manner\\
\glt ‘because we never heard voices and I thought [lit. ‘it seemed to me’] that we had to leave the village that way’ (1519–26, Cartas, \textsc{corde}) 
\end{exe}
}

Space limitations prevent me from showing that the strong pronouns which do not undergo doubling while the change is in process display a similar panorama of distribution between topic and focus relations. The motivation for indexing DOM, in other words, does not seem to have depended on the pragmatic structuring of the utterances. 

 Let us stop one moment to consider the functional shift which the coreferential pronoun is experiencing in examples like the ones shown in \REF{04-me-ex:9} to \REF{04-me-ex:12}. It is clear that binding a dislocated object phrase to the core clause no longer corresponds to the function it fulfills. The coreferential person form now co-occurs with a strong pronoun within one and the same syntactic domain, as opposed to the situation of medieval times when in this type of pragmatically unmarked or “neutral” environments the weak object pronoun and the strong personal pronoun were in complementary distribution (\textit{me} or \textit{a mí}, not both). 

 The functional shift should be thought of in terms of a gradual process. In the initial phase, the doubling pronoun must have been felt as redundant. Indeed, redundancy will play a crucial role in the hypothesis I will outline in \sectref{04-me-sec:4.3}. But the doubling pronoun eventually becomes categorical and gives rise to a phenomenon of coindexing on the verb which has come to be viewed as an instance of object-verb agreement (\citealt{Suner1988Agreement}; \citealt{Garcia-Miguel1991Duplicacion}; \citealt{Bogard1992Estatus}; \citealt{FernandezSoriano1999Pronombre}; \citealt{Franco2000Agreement}). Let us recall in this regard (\sectref{04-me-sec:2}) that the \isi{grammaticalization} of indexing DOM in the pronominal area unfolded in parallel to the cliticization of the \ili{Spanish} weak pronouns, a development which makes it easy to defend the agreement analysis: the weak pronoun is morphologically bound to the verb, forming with the “target” \citep{Corbett1983Hierarchies} of the agreement relationship a phonological unit, without being attached to it like an affix.\footnote{As already indicated, in the exceptional case of the imperative and non-finite verbal forms the \ili{Spanish} clitics are suffixed; \cf \textit{hábla}\textit{le} ‘talk to him’, \textit{quiere ver}\textit{me} \textit{a mí} ‘he wants to see me’. In fact, arguments have been advanced to justify the view that the \ili{Spanish} clitics behave like inflectional affixes, on a par with the subject agreement suffixes appearing on the verb (\citealt{AlarcosLlorach1980Pronombres}; \citealt{Bogard1992Estatus}; \citealt{Fontana1993Phrase}; \citealt{Enrique-Arias2003Clitics}), but not everyone agrees with this analysis (\citealt{Olivaetal2013variacion}, among others). The lack of consensus has much to do with the fact that the weak forms, in the majority of their occurrences, function as anaphoric pronouns encoding syntactic arguments. They are “ambiguous” \citep[126]{Siewierska2004Person} agreement markers in this sense.} 

 In addition to enabling the \isi{grammaticalization} of object agreement in \ili{Spanish}, the conversion of the weak pronouns into clitics may have contributed to the relative swiftness with which coindexed strong personal pronouns became the norm. The time span reflected by written materials, as mentioned, covers a period of more or less two centuries. We will also see that \isi{clitic doubling} was generalized at different rates depending on the individual speakers/authors, and in some cases very quickly (\sectref{04-me-sec:4.5}).

 Two essential facts have to be kept in mind for a thorough understanding of \ili{Spanish} indexing DOM, namely, that \isi{clitic doubling} spread to accusative and dative tonics alike, and that once it became established with the personal pronouns it continued to evolve towards the nominal \textit{indirect} object.

 Disregarding the second phenomenon, one could argue that indexing DOM was extended to the dative object pronouns owing to the absence of formal case distinctions within the personal pronoun paradigm, as seen above (\sectref{04-me-sec:2}). The neutralization of the split between accusative and dative would have taken place in accordance with the general properties of \ili{Spanish}, a language in which the boundary between the two grammatical object functions is not sharp \citep{Garcia2015Variable}. From this perspective, one could then sustain that indexing DOM, irrespective of case considerations, and much like flagging DOM, began with the personal pronouns because the participants encoded by these forms are human, \isi{definite}, and moreover highly prominent in discourse, all of which justifies their superior ranking in the universal hierarchy of \isi{topicality}. As to the question of why a new device was recruited to signal differential object features already marked with \textit{a}, one could invoke the need for a “renewal” of DOM, in the sense that the clitic helped reestablish the original distinction between pronouns and non-pronouns which had existed before \textit{a} was extended to some of the accusative nouns (\sectref{04-me-sec:1}).

 The second phenomenon, however, forces us to modify these assumptions. Indeed, if the new marking had been fundamentally motivated by the pronominal features of \isi{animacy}, \isi{definiteness} and \isi{discourse prominence}, one would have expected a development more in line with that of flagging DOM. Recall that \textit{a} began as a topicalizer which did not differentiate either between accusative and dative pronouns. In its descent towards the non-pronouns, along the \isi{animacy hierarchy}, \textit{a} was directed at the more topical human and \isi{definite} accusatives, and was not extended to the typically human and \isi{definite} datives, because the \ili{Spanish} datives were already case-marked with a homonym of flagging DOM (deriving from the locative uses of \ili{Latin} \textit{ad}). In the case of indexing DOM, however, nothing prevented the clitic from moving along the same hierarchy to the whole range of more topical nouns, which would have included the datives and also many of the by then \textit{a}-flagged accusatives. Instead, the clitic proceeded selectively, picking out the dative nominal with which it came to establish a systematic relation in the course of time (\citealt{Silva-Corvalan1984Semantic}; \citealt{Rini1991Redundant}; \citealt{Melisetal2009Interplay}; \citealt{Rozasetal2012Object}). As a result of this expansion, \isi{clitic doubling} is at present obligatory or strongly preferred in most dative contexts in all varieties of \ili{Spanish} \citep[1250]{FernandezSoriano1999Pronombre}, thus functioning as a perfectly entrenched \isi{indirect object} agreement marker in the opinion of most scholars working on \ili{Spanish}.\footnote{I have to mention that in some dialects, most notably in Argentinian \ili{Spanish}, the doubling clitic is sometimes used with nominal direct objects. There have been different attempts at explaining the triggering conditions for the optional use of the clitic in these contexts, but the proposals suggest a total lack of agreement (see \citealt{Belloro2007Spanish} for a good overview of the divergent hypotheses, ranging from “topic” to “focus”, and from “presupposed” to “new” entities, among other claims; \cf \citealt{Sanchezetal2013Restricciones} for additional references). What seems clear is that the regional phenomenon obeys principles of its own, different from those underlying standard flagging and indexing DOM, since the doubling clitic occurs rather freely with \isi{inanimate} entities and allows for these to be devoid of \textit{a}.} An example is given in \REF{04-me-ex:13}:

\protectedex{
\begin{exe}
\ex%13
\label{04-me-ex:13}
%\langinfo{lg}{fam}{src}\\
\gll El padre Miguel \textbf{le} entreg-ó \textbf{a} \textbf{Sole} una pequeña campana de bronce.\\
the father Miguel she.\textsc{dat} give-\textsc{pfv.3sg} \textsc{dat} Sole a small bell of bronze\\
\glt ‘Father Miguel gave Sole a small bronze bell.’ (1999, González, Quién como Dios, \textsc{crea})
\end{exe}
}

In this example, the recipient argument \textit{Sole} is introduced by the case-marking preposition \textit{a}, and the product of the modern extension of indexing DOM to the dative nominals is seen in the cross-referencing dative clitic \textit{le}. Hence, within the nominal area, \isi{clitic doubling} today also functions as a case marker, opposing datives (\textit{a} + clitic) to accusatives (\textit{a} or Ø).

 In his paper on the rise of object agreement, \citet[165]{Givon1976Topic} remarks on the tendency for dative to take precedence over accusative agreement in languages in which the accusative and dative objects are equally case-marked (or unmarked). And the author also notes \citep[169]{Givon1976Topic} that if the agreement system is allowed to mature, “the agreement primacy of one (mostly the dative) over the other becomes effectively the signal differentiating the object cases from each other”. On this view, \ili{Spanish} would have evolved following a universal tendency.

 However, typological research carried out during the last few years has demonstrated that languages with indirective alignment like \ili{Spanish} do not illustrate a situation in which the dative is indexed and the accusative is not \citep[12]{Haspelmath2005Argument}. So weighed against this new piece of evidence, the \isi{dative case} marker \ili{Spanish} developed through indexing represents “a typologically anomalous fact” \citep[232]{Garcia2015Variable}.

 To account for this anomaly, different explanations have been proposed. It is possible that dative indexing in \ili{Spanish} arose as a means to promote oblique-like arguments to the level of core participants \citep[232–233]{Garcia2015Variable}. It has also been suggested that through the dative clitic a case distinction was reinforced in a language which has met with difficulty in keeping its two object categories apart \citep{Melisetal2009Interplay}. Whatever the explanation, the point of major interest for this study is that the dative orientation of the subsequent development of indexing DOM in \ili{Spanish} implies a distinction with strong ties to a concept of grammatical case functions. This property cannot be ignored when one tries to account for the emergence of \ili{Spanish} indexing DOM. 

\subsection{Previous approaches}\label{04-me-sec:4.2}

Before I present my analysis, a brief review of previous approaches to \ili{Spanish} indexing DOM is in order. Under \citeauthor{Silva-Corvalan1984Semantic}'s (\citeyear{Silva-Corvalan1984Semantic}) proposal, based on \citet{Givon1976Topic}, \ili{Spanish} \isi{clitic doubling} shows a phenomenon of topic-verb agreement evolving into an object agreement marker, sensitive to the relative \isi{topicality} of the object phrases. To sustain her proposal, the author observes that in the medieval texts, where – she acknowledges – doubling is scant, the objects that favor the occurrence of the clitic are fronted. From these topicalizing structures, the clitic spreads to the personal pronouns, located in the upper region of the universal hierarchy of \isi{topicality}, and later moves on to the nominal \isi{indirect object}, overwhelmingly human and \isi{definite}, and in this sense more topical than the \isi{direct object}, which, as \citeauthor{Silva-Corvalan1984Semantic} (\citeyear{Silva-Corvalan1984Semantic}) argues, tends to be non-human and \isi{indefinite}.

 Although the proposal is attractive, I have already commented (\sectref{04-me-sec:3}) on the difficulty of establishing a direct connection between indexing DOM and the sparingly used topic-shift constructions of earlier times. Another problem lies in the crucial dependence of this account on the topical features of \isi{animacy} and \isi{definiteness}. As indicated in the preceding section, if these features had been the driving force one would have expected \isi{clitic doubling} to spread not just to the dative nouns but also to the similarly human and \isi{definite} \textit{a-}marked accusatives. Finally, \citeauthor{Silva-Corvalan1984Semantic}’s (\citeyear{Silva-Corvalan1984Semantic})  hypothesis disregards the fact that \isi{clitic doubling} in the pronominal area begins as a highly selective process that picks out a few personal pronouns only (see \sectref{04-me-sec:4.5}), thus making it evident that some additional factor beyond the shared \isi{topicality} of the pronouns is at play.

\citeauthor{Gabrieletal2010Information}'s (\citeyear{Gabrieletal2010Information}) thesis – along the lines suggested by \citet{Givon1976Topic} – is that object agreement in \ili{Spanish} derives from a reanalysis of the right-dislocated topic structures of the medieval era. The authors work under the assumption that the coindexed objects of modern \ili{Spanish} are “preferably construed as belonging to the focus domain” \citep[62]{Gabrieletal2010Information}, and argue that the proposed reanalysis is able to explain why topical participants, such as the personal pronouns and the human/\isi{definite} dative nouns, occur in \isi{clitic doubling} constructions in which they are assigned focus status.

 One weak point of this thesis relates to the presupposed focus interpretation of the doubled objects, which suits the dative nouns far better than it does the personal pronouns. Indeed, datives in \ili{Spanish} are typically coded in the form of clitics, because their referents are prominent in the discourse, and when they appear as noun phrases they tend to (re)introduce “new” entities which are likely to form part of the comment (\citealt[286–287]{Rozasetal2012Object}), but the personal pronouns, as seen in \sectref{04-me-sec:4.1}, cannot be assumed to function as foci on a regular basis. The more important objection to be raised, however, has to do with the choice of the historical data, strongly influenced by the focus thesis and represented by \textit{postverbal} objects only \citep[75]{Gabrieletal2010Information}. It is clear that the skewed character of the sample must have seemed to lend support to the reanalysis of right dislocations, yet this was done at the expense of doubled objects occurring in other positions, which were simply left out of the study. 

 Approaching \ili{Spanish} indexing DOM from a different perspective, \citet{Rini1991Redundant} takes the emphatic/contrastive property of the strong personal pronouns as his point of departure, and proposes that the use of the doubling clitic was developed as a means to compensate for the gradual loss of emphasis which the tonics had suffered over time. On this view, the duplicating element was recruited to form a construction with patently redundant attributes, the effect of which would be able to ensure the emphatic value of the strong pronominal object. As supporting evidence for the hypothetical loss of emphasis, Rini mentions the growing tendency for the dative tonics to occupy the preverbal position starting from the 14th century \citep[277]{Rini1991Redundant}. The preverbal datives are assumed to represent left dislocations (but recall example \REF{04-me-ex:9a} above), and are regarded by the author as instantiating an alternative strategy to \isi{clitic doubling}, deployed for the same purpose of reinforcing the weakened emphasis of the strong personal pronouns.

 It will be seen below that I agree with Rini in giving importance to a notion of redundancy as a way of explaining the origin of \ili{Spanish} indexing DOM. But I do not believe that a loss of emphasis was at issue. The gradual process oriented towards certain types of case roles, which will be analyzed below, suggests otherwise. The major problem here, however, has to do with the expansion of \isi{clitic doubling} to the dative nouns. The author himself recognizes the challenge the diachronic evolution of the clitic poses to his thesis “since NP duplication cannot be assumed to have ever been emphatic” \citep[282]{Rini1991Redundant}, and he is forced to speculate that the nominal indirect objects were submitted to doubling “by analogy” \citep[282]{Rini1991Redundant}.

 To sum up, our survey of previous approaches to \ili{Spanish} indexing DOM leads us to conclude that there remain important aspects linked to the development of this phenomenon which have not been fully elucidated.

\subsection{Hypothesis: the topic case hierarchy}\label{04-me-sec:4.3}

My analysis starts from the observation that the stressed object personal pronouns with which \ili{Spanish} indexing DOM arises around the turn of the 16th century encode discourse prominent referents – all are highly topical in this sense – and in addition share the same emphatic form that is indicative of the presence of a contrast drawn between the referent of the pronoun and other individuals.

 To answer the question of what may have triggered the innovative use of indexing DOM with the tonic personal pronouns, we first have to define the value the coreferential pronoun supplies to the construction. As discussed above in \sectref{04-me-sec:4.1}, the use of the coreferential pronoun in the innovative contexts no longer hinges on the occurrence of dislocated objects, and yet, at the same time, we are aware of the fact that the pragmatically marked structures of medieval times provide the single source of environments in which a prior use of coreferential pronouns was found. This gives us reason to turn to these structures to ask if something in the behavior of the coreferential pronoun might explain the new function it came to develop with the strong object personal pronouns. One good justification for this is that processes of change have been described as being “economical”, to the extent that diachronic changes have a tendency to seize upon existing forms in a language, which are reused for new purposes \citep[73]{Hopperetal2003Grammaticalization}.

 My claim is that the relevant property we are looking for lies in the fact that the dislocated sentences of medieval times are characterized by the \textit{double mention} of one and the same referent. Although it is evident that the double mention is syntactically motivated, it is no less true that it has the effect of enhancing the prominence of the topic participant in question. From this perspective, one may therefore suggest that the doubling clitic was introduced in the area of the strong personal pronouns as a means of drawing special attention to a referent by mentioning it twice, i.e. through redundancy. As \citet[41]{Pulgram1983Reduction} %AW the draft incorrectly had 1963
cited by \citet[281]{Rini1991Redundant} points out, in any of its forms a redundant construction “aims at a kind of greater explicitness, emphasis, preciseness”, without necessarily providing a clue as to what is being emphasized. 

 Defining the communicative intention of a redundant construction calls for a detailed examination of the contexts in which it occurs, to be compared with the use of analogous expressions lacking the redundant element. The comparative analysis of doubled and non-doubled pronouns to be presented in \sectref{04-me-sec:4.5} will put in evidence that the “redundant” coreferential pronoun was initially used in specific discourse contexts where it served the purpose of emphasizing the subjective involvement of the personal pronoun referent in the designated verbal situation.

This original bias towards the \textit{role} of the object participant in the \isi{event structure} establishes the scenario for the development of indexing DOM in \ili{Spanish}, from the emergence of its use with the personal pronouns until its conversion into a case marker with the lexical \isi{indirect object}. As mentioned in \sectref{04-me-sec:1}, the development at issue directs us to a specific dimension of \isi{topicality}, according to which the property of being “topical” is evaluated with regard to the degree of involvement of a participant (more involved participant > less involved participant) and functions as the parameter governing the \isi{case hierarchy} (agent > dative > accusative) \citep{Givon1976Topic}.


\subsection{The data}\label{04-me-sec:4.4}

For the purpose of my study, a corpus of data was formed with examples retrieved from the electronic data base CORDE. Since the rise of \ili{Spanish} indexing DOM is associated with the transition decades between the 15th and the 16th century, the first materials I reviewed were several works of the late 15th century. The sample included the play \textit{La Celestina. Tragicomedia de Calisto y Melibea} by Fernando de Rojas (1499–1502) and a number of narrative and historical texts produced between the years 1480 and 1499. The comparison between textual sources made plain that doubling pronouns in the \textit{Celestina} were more frequent than in the materials which did not intend to reflect “oral” productions to the same degree. To give an example, the first person pronoun \textit{a mí} ‘me’ in the \textit{Celestina} was doubled in almost 60\% of its occurrences (26/44), as opposed to 26\% of duplications (25/97) in the narrative texts and 16\% of doubled \textit{a mí} (11/67) in the historical works. These results suggested that the innovative use of the coreferential pronoun – like most changes – got a firm footing in spoken language before finding its way into writings. On the basis of these results, and given my interest in exploring the beginnings of indexing DOM, the decision was made to keep the more conservative texts for this research. Confronting theater plays with variable doubling would have been another possibility, of course, but those available from the CORDE for the period under study showed indexes of frequency similar to, or higher than, the percentages of the \textit{Celestina.}

 The doubling data from the \textit{Celestina} also seemed to suggest that indexing DOM with the pronouns began as a type of \textit{ego}-centric strategy (or one involving the speaker/hearer dyad, but second person tonics were too scant to appreciate this). Thus, in comparison to \textit{a mí}, doubled in 60\% of the examples, the third person singular pronoun (\textit{a él} ‘him’ / \textit{a ella} ‘her’) only yielded 30\% of duplications (7/23). These results motivated the decision to organize the study around the first person pronoun which had played a leading role in the process of change. Dealing with a single form, to my mind, offered the additional advantage that a more homogeneous picture of the development of indexing DOM could be obtained, by excluding potential variables connected to the distinct persons.

 In this way, the definitive corpus of data came to consist of 794 tokens of (doubled and non-doubled) \textit{a mí}, extracted from 14 narrative and historical works covering the period between 1482 and 1605. In \tabref{04-me-tab:2} I show the distribution of \isi{clitic doubling} associated with each of the texts I examined. It needs to be noted that the quantitative profiles vary considerably in spite of the temporal proximity of the texts. We may interpret the differences as reflecting individual preferences, which are not unusual when a change is in progress.\footnote{Additionally, as suggested by the anonymous reviewer, the variation in terms of doubling frequencies may also be due to the involvement of distinct textual traditions in the examined sources. This is a question of great interest, which, unfortunately, lies beyond the scope of this paper.} At the same time, we cannot ignore the fact that, if viewed as a whole, the texts project the image of a rather quickly unfolding change. 

 For the sake of my analysis, the textual sources were divided in three sets according to the indexes of duplication: no more than 30\% of duplication, around 50\%, and a near categorical phenomenon of \isi{clitic doubling} with the first person. 

\begin{table}
\caption{The distribution of clitic doubling with the personal pronoun \textit{a mí}}
\label{04-me-tab:2}
\begin{tabular}{llrrr}
\lsptoprule
& & & \multicolumn{2}{c}{\textsc{doubling}}\\\cmidrule(lr){4-5}
\textsc{date} & \textsc{text} & \textsc{registered tokens} & \textsc{number} & \textsc{percent}\\\midrule
\multicolumn{2}{l}{\textsc{low index of doubling}}\\
1482--92 & \textit{Amadís}   & 91 & 27 & 30\\ 
1495     & \textit{Grimalte} & 58 & 6  & 10\\ 
1501     & \textit{Tristán}  & 73 & 7  & 10\\ 
1520     & \textit{Ysopo}    & 47 & 12 & 26\\ 
\cmidrule(lr){3-5}
& \textsc{average} &269& 52 & \textbf{19}\\\midrule
\multicolumn{2}{l}{\textsc{intense competition}}\\
1504     & \textit{Esplandián} & 39 & 18 & 46\\ 
1516     & \textit{Floriseo}   & 46 & 25 & 54\\ 
1517     & \textit{Arderique}  & 47 & 21 & 45\\ 
1555     & \textit{Espejo}     & 99 & 59 & 60\\ 
1560     & \textit{Crónica}    & 23 & 14 & 61\\ 
\cmidrule(lr){3-5}
& \textsc{average} & 254 & 137 & \textbf{54}\\\midrule
\multicolumn{2}{l}{\textsc{generalized doubling}}\\
1519--26 & \textit{Cartas}   & 41 & 37 & 90\\ 
1553--84 & \textit{Guerras}  & 33 & 30 & 91\\ 
1568--75 & \textit{Historia} & 74 & 68 & 92\\ 
1595     & \textit{Granada}  & 27 & 24 & 89\\ 
1605     & \textit{Quijote}  & 96 & 92 & 96\\ 
\cmidrule(lr){3-5}
& \textsc{average} & 271 & 251& \textbf{93}\\
\lspbottomrule
\end{tabular}
\end{table}


\subsection{Analysis and discussion}
\label{04-me-sec:4.5}

I have advanced the hypothesis that indexing DOM in \ili{Spanish} originated as a means to give the highest degree of prominence to a referent’s subjective involvement in the action. This is achieved through a strategy of double mention, whose redundant value is exploited to create the desired emphasis. In order to verify the hypothesis, the textual sources belonging to the first set will be examined. They can help us track the beginnings of indexing DOM since doubling in these works is still exceptional. We will proceed by having a look at several pairs of examples.

The first pair is shown in \REF{04-me-ex:14}:

\ea \label{04-me-ex:14}
\ea \label{04-me-ex:14a}
\textit{Y si culpa tiene Fortuna, no la pongas \textbf{a mí}.}\\
\glt And if Fortune is to blame, don’t put the blame on me.’ (1495, Grimalte, \textsc{corde})

\ex \label{04-me-ex:14b}
\textit{Porque entonces era enemigo queriendo cobrar de ti aquello que ya cobré, cuya causa \textbf{a mí me} puso descanso y a ti estos sospiros que tienes. Y si lloras lo que conmigo perdiste, yo asimesmo lo que contigo gané.}\\
\glt ‘Because at the time I was your enemy, wanting to get from you that which I finally got, an outcome that gave me peace [lit. ‘put peace on/to me’] but left you with these sighs. And if you bemoan what you lost with me, I cry all the same over what I won with you.’ (1495, Grimalte, \textsc{corde})
\z
\z

In both examples \textit{a mí} functions as the dative argument of \textit{poner} ‘to put something on someone’, and in both cases the choice of the strong pronominal has been motivated by the expression of a contrast (Fortune vs. me, me vs. you). The non-doubled use in \REF{04-me-ex:14a} represents the normal way of encoding the object pronoun at the time. By comparison, the context in which \REF{04-me-ex:14b} is inserted contains a far more elaborate opposition between the speaker’s personal memories of a bygone love and the experience of the beloved one. In this context, the redundant function of coreferential \textit{me} is called upon to center the attention on the subjective experience of the speaker. 

\largerpage
 The utterances in \REF{04-me-ex:15} are produced by the same character of the textual source, a rejected lover. 

\ea \label{04-me-ex:15}
\ea \label{04-me-ex:15a}
\textit{Mas esto} \textbf{\textit{a mí}} \textit{acaescer no puede, segunt el precio que ya me costaes y aún} \textit{no sois mía}.
\glt ‘But this cannot happen to me, since you’ve already costed me a fortune and you are still not mine.’ (1495, Grimalte, \textsc{corde})

\newpage 
\ex \label{04-me-ex:15b}
\textit{y así como aquellos que por faltas suyas vergonçosos buelven a sus tierras,} \textit{tal} \textbf{\textit{a mí me}} \textit{acaesció, que con menos favor que partí me buelvo a los reinos} \textit{dEspaña y castellana tierra donde yo natural era.}
\glt ‘and like those who due to errors of their own return to their homeland with shame, so it happened to me, who returns to the kingdom of Spain and my native Castile having much less in my favour than when I left.’ (1495, Grimalte, \textsc{corde}) 
\z
\z

\REF{04-me-ex:15a} follows a statement as to the fact that people easily let go of things that were easily obtained, and opposes the situation of the speaker, who cannot give up something that is still not his. The event alluded to in \REF{04-me-ex:15b} is more tragic: The speaker returns from a failed mission knowing that the woman who rejects his advances has conditioned a potential change in her attitude on the successful outcome of the assignment she herself imposed. The double-mention strategy in this example serves to emphasize the feelings of shame and despair which underlie the comparison with other defeated individuals.

 Now consider \REF{04-me-ex:16}:

\ea\label{04-me-ex:16}
\ea \label{04-me-ex:16a}
\textit{Suplico ante tu excelente majestad que otorgues} \textbf{\textit{a mí}}\textit{, tu servidora, esta gran} \textit{merced} 
\glt ‘Appearing before your excellent majesty I beg you to grant me, your servant, this great favour’ (1520, Ysopo, \textsc{corde)} 

\ex%16b
\label{04-me-ex:16b}
\textit{Que si Dios} \textbf{\textit{a mí}} \textit{de sus gracias alguna parte} \textbf{\textit{me}} \textit{diera, yo soy cierto que vos} \textit{ya fuérades mía} 
\glt ‘If God had given me a fraction of her [Fiammetta’s] talent, I am certain that you would be mine by now’ (1495, Grimalte, \textsc{corde}) 
\z
\z

The non-doubled tonic occurs in a petition addressed to Jupiter, where the contrastive value of the pronoun is used to emphasize the distance that separates the humble petitioner from the king of gods. In the emotionally charged context of \REF{04-me-ex:16b}, on the other hand, a doubled tonic surfaces. The speaker is the rejected lover of \REF{04-me-ex:15}, who in this passage laments his not having been blessed with the gift of eloquence, another condition imposed by the beloved for her to yield to his advances. This explains both the comparison with Fiammetta, who does possess the gift, and the use of the redundant construction as a means of underscoring the fatal shortcoming that condemns the speaker to a life away from the woman he loves.

 The verb \textit{parecer} ‘to seem’ is involved in the following choice between uses:

\begin{exe}
\ex%17
\label{04-me-ex:17}
\begin{xlist}
\ex%17a
\label{04-me-ex:17a}
%\langinfo{lg}{fam}{src}\\
\textit{¡Por Dios -dixo Gorvalán-,} \textbf{\textit{a mí}}\textit{paresce locura en querer probar todas las} \textit{aventuras!}
\glt ‘For God’s sake -Gorvalan said- it seems madness to me wanting to have a taste of any kind of adventure!’ (1501, Tristán, \textsc{corde})
\ex%17b
\label{04-me-ex:17b}
%\langinfo{lg}{fam}{src}\\
\textit{En el nombre de Dios -dixo el Cavallero de la Verde Spada-, ésse} \textbf{\textit{me}} \textit{pareçe} \textbf{\textit{a mí}} \textit{el mejor acuerdo, porque, ahunque el Emperador sea mayor que vos,} \textit{y tenga más gentes, para doze cavalleros tan buenos se fallarán en vuestra} \textit{casa como en la suya}.
\glt ‘In the name of God –the Knight of the Green Sword said- this seems to me the best resolution, because, although the Emperor is older than you, and has more troops, for a fight with twelve knights you’ll find as good ones among yours as he among his.’ (1482–92, Amadís, \textsc{corde)} 
\end{xlist}
\end{exe}

\REF{04-me-ex:17a} and \REF{04-me-ex:17b} communicate a personal state of mind with respect to a proposal set forth by the interlocutor. In the lines preceding \REF{04-me-ex:17a} Tristan expresses his desire to go and rescue a noblewoman in distress, to which the speaker opposes his contrasting view on the matter with a simple \textit{a mí}. In \REF{04-me-ex:17b}, the king’s project to war against the twelve knights of the emperor motivates a fully supportive (“the best resolution”) and elaborated upon (“because…”) response, in which the doubling form brings additional emphasis to the degree to which the speaker approves of the decision for war.

 My last examples are constructed with the verb \textit{placer} ‘to please, to like’, which in certain types of contexts comes closer to expressing a notion of will. This is especially true in dialogues where \textit{placer} communicates the speaker’s consent to a request or agreement with a proposal, and where, depending on the case, slightly different shades of meaning may emerge (‘it pleases me’, ‘I want to’, ‘it is my will’, ‘I agree’, etc.). In such environments the stimulus argument is often omitted, being recoverable from the context:

\begin{exe}
\ex%18
\label{04-me-ex:18}
\begin{xlist}
\ex%18a
\label{04-me-ex:18a}
%\langinfo{lg}{fam}{src}\\
\textit{E dixo Tristán: -}\textbf{\textit{A mí}} \textit{plaze}.
\glt ‘And Tristan said: “It pleases me”.’ (1501, Tristán, \textsc{corde)}
\ex%18b
\label{04-me-ex:18b}
%\langinfo{lg}{fam}{src}\\
\textit{E el rey dixo: -}\textbf{\textit{A mí me}} \textit{plaze, e fago gracias a Dios de tamaña merced} \textit{como me á fecho}. 
\glt ‘And the king said: “It pleases me, and I thank God for doing me this great favour”.’ (1501, Tristán, \textsc{corde)}
\end{xlist}
\end{exe}

The sentence with the non-doubled pronoun is an expression of agreement with a travel mate’s proposal to split up and go separate ways. \REF{04-me-ex:18b} is the king’s response to a request for his daughter’s hand, occurring at the end of a dialogue in which the father reiterates his consent, as well as his delight in the thought that his daughter will marry Tristan. The redundant construction contained in the response is a way of emphasizing the speaker’s internal state of profound happiness.

The examined pairs of examples have given us insight into the communicative strategy of redundancy which lies at the root of \ili{Spanish} indexing DOM. As is expected to happen at the early stage of a \isi{grammaticalization} process, the innovative function of the doubling clitic is appealed to in specific discourse contexts, here suggestive of a search for greater expressivity or emphasis regarding the involvement of a participant in the denoted event. Following \citet[1057]{Haspelmath1999Why}, we could say that the emergence of \ili{Spanish} indexing DOM illustrates the “extravagance maxim” characteristic of the actions of speakers who “want their utterance to be imaginative and vivid”. 
What is easier to understand after the examination of the examples is why the strong personal pronouns were good candidates to trigger the new strategy. They were indeed emphatic forms, which in themselves implied that a personal attitude or behavior would be brought to stand out through the means of a contrast, and this is precisely what made them eligible to become the targets of some additional emphasis. So even though one can never explain why a change takes place, it is possible to state that \ili{Spanish} indexing DOM arose in contexts where the contrastive value of the strong pronouns and the emphatic aim of the redundant construction fused in a natural and harmonious way. 

 If my proposal is on the right track, it should receive support from the evolutionary path of the clitic. As a change progresses, an increase in the frequency of the new form is detected, and coupled with this increase certain patterns of use become visible. The choice of the new form over the older one loses its dependency on specific discourse contexts and acquires some systematicity, meaning that certain types of contexts now motivate the appearance of the new form on a regular basis. In order to verify this, the corpus texts pertaining to the second set may prove useful, since the extension of \isi{clitic doubling} to one half of the registered examples profiles a movement towards the consolidation of indexing DOM.

 As it happens, the distribution between doubled and non-doubled \textit{a mí} in the texts under discussion affords a clear pattern, which resides in the near obligatoriness of the clitic with one particular verbal class, namely, mental predicates specialized in denoting a subjective attitude, whether intellectual or emotional. Thus, the tonic pronoun with \textit{parecer} (‘it seems to me, I think’) is doubled in almost all of its occurrences (28/29 = 96.5\%), while \textit{placer} (‘it pleases me, I like’) and its antonym \textit{pesar} (it grieves me, I lament’) motivate the duplication of \textit{a mí} in 83\% (15/18) of the registered examples. There are also complex predicates that convey similar meanings (\textit{ser oscuro} ‘it is obscure to me, I don’t understand’, \textit{causar pena} ‘it causes me grief, I am sorry’, \textit{dar contento} ‘it gives me happiness, I am happy’, \textit{caer en gracia} ‘it strikes me as funny, I am amused’, etc.), and they too trigger doubling with high frequency (16/20 = 80\%).

 All these mental predicates take a dative \isi{experiencer} argument, and are construed, as is usual, with a stimulus of \isi{inanimate} reference, coded in the form of a noun phrase when designating some object (\cf something pleases me) or appearing as a clausal complement when expressing a situation (\cf it pleases me that…). In this way, the sole human participant to go on stage is the dative \isi{experiencer} (\textit{a mí}), highly salient, whose subjective attitude with regard to some entity or event is the focus of the utterance. \ili{Spanish} experiencers of this type are associated with a series of peculiar features that have prompted their analysis in terms of “dative subjects” (see \citealt{Melisetal2013Historical}, and references therein). Their subject-like behavior comes as no surprise considering that mental meanings of analogous nature are often expressed, in \ili{Spanish} and in other languages, with nominative-\isi{experiencer} predicates.

 It makes sense that indexing DOM grammaticalized first with these mental predicates, having moved along a path that leads from a redundant emphasis on one’s subjective involvement in a situation to a class of verbs specialized in the description of one’s subjective mental state. The predicates in question also confirm that the doubling clitic was tied to a notion of participant roles ever since it was introduced into the domain of the strong personal pronouns. This can be inferred from the character of the predicates' \isi{experiencer} argument. Experiencers never perform like volitional agents. Yet mental experiences can be construed from different vantage points, and in some of these construals the internal process appears to be under the control of the \isi{experiencer}. The mental predicates under discussion are of this type: they do not express the reaction of an \isi{experiencer} to the impact of a stimulus, but portray a subject-like dative \isi{experiencer} as being in a state with respect to a given object. Hence, in the \isi{case hierarchy} (agent > dative > accusative) proposed by \citet[152]{Givon1976Topic}, the dative \isi{experiencer} of these predicates would be placed near the top-end (no agent but subject-like). And in light of this, one is able to argue that \ili{Spanish} indexing DOM first spread to these experiencers because they were more “topical” than all the other object pronouns implicated in the change. 

 It is now worth examining the behavior of the less topical objects in the texts of the second set. These objects occur in sentences containing another human participant who realizes the action and functions as the topical subject. So the \isi{case hierarchy} predicts that doubling with these “less involved” object participants should lag somewhat behind, as the data corroborate. Additionally, the \isi{case hierarchy} leads us to expect that the higher-ranked datives should motivate the use of the clitic more often than the accusative pronouns. But the data are less transparent in this regard for one obvious reason: the distinction between more and less involved participants was neutralized due to the formal identity of the pronouns. 

 The less topical objects were found to display percentages of doubling hovering around 50\%, irrespective of the dative/accusative distinction. To investigate the dative function, I gathered the verbs of “giving” (primarily \textit{dar} ‘to give’, but also \textit{otorgar} ‘to grant’, \textit{ofrecer} ‘to offer’, \textit{encomendar} ‘to entrust’, etc.) and the verbs of “saying” (\textit{decir} ‘to say’, \textit{contar} ‘to tell’, \textit{pedir} ‘to ask’, \textit{prometer} ‘to promise’, \textit{mandar} ‘to order’, etc.), with which the referent of \textit{a mí} is semantically speaking a “recipient”. Taken together, these verbs yielded duplicated tokens of \textit{a mí} in 52\% of the examples (28/54). Curiously, when viewed as separate verb types, a striking disparity as to their behavior emerged: 79\% of duplications (11/14) with verbs of “saying”, against 42.5\% (17/40) with verbs of “giving”. The elevated percentage in the former case would probably need some tuning given the numerical poverty of the sample. In the latter case, the low percentage may be related to the fact that some of the sentences built with a verb of “giving” (\textit{dar la muerte} ‘to kill’, lit. ‘to give death’, \textit{atribuir la culpa} ‘to blame’, lit. ‘to attribute a fault’, etc.) have a dative coded argument whose \isi{semantic role} comes closer to that of a patient. This does not happen with the verbs of “saying”, always accompanied by a dative who participates in the completion of the event by processing the received message. So it is possible after all that the discrepancy between “saying” and “giving” verbs with respect to the frequency of doubling may reflect the operation of an underlying scale of degrees of involvement.

 The accusative population of \textit{a mí}, on the other hand, is associated with a rather heterogeneous set of verbs (\textit{ver} ‘to see’, \textit{engañar} ‘to deceive’, \textit{matar} ‘to kill’, \textit{librar} ‘to free’, \textit{traer} ‘to bring’, \textit{buscar} ‘to look for’, etc.), which does not offer the opportunity of inspecting the behavior of particular subclasses given the meager representation of the distinct event types. Globally, the accusative pronouns attract \isi{clitic doubling} in 45.5\% of the registered examples (36/79). A more fine-grained contextual analysis would be necessary to uncover why some patients were judged to be better candidates for doubling than others. 

 In the next step of the \isi{grammaticalization} process, the distinction between more topical and less topical pronouns becomes obliterated, allowing for the spread of the clitic to all tokens of \textit{a mí} as a near to obligatory object agreement marker. This is the situation which the textual sources of the third set bring to view. Eventually, indexing DOM will be extended to the entire category of the strong personal pronouns, marking datives and accusatives alike.\footnote{The historical data make clear that the development of the clitic into a near categorical object agreement marker took some more time with the third person pronouns. For example, in Hernán Cortés’ \textit{Cartas}, where \textit{a mí} is accompanied by a doubling clitic in 90\% of the examples (\tabref{04-me-tab:2}),  the third person pronouns show 57\% of duplications (38/67), and in Cervantes’ \textit{Quijote}, one century later, \textit{a mí} yields 96\% of agreement (\tabref{04-me-tab:2}), against a frequency index of 77\% (44/57) in the third person area. In order to verify the later entrenchment of the third person clitic, I reviewed a sample of narrative and historical texts, dating from the years 1660 to 1699. My sample showed 98\% of \isi{clitic doubling} with \textit{a mí} (130/133) and 79\% with the third person pronouns (162/205), thus confirming that these were lagging slightly behind. Curiously, “us”, “you” and “you all” were found to behave much like the third persons (66/86 of doubling = 77\%). So it appears that the \isi{grammaticalization} process of indexing DOM was from beginning to end somewhat biased towards the highest-ranked entity on the topic person hierarchy (\textit{ego}).} 

 The lack of formal case distinctions within the domain of the \ili{Spanish} personal pronouns has to be viewed as the principal reason for why the accusative pronouns were drawn into the orbit of the \isi{grammaticalization} process. If we understand this, the following historical events related to \isi{clitic doubling} in \ili{Spanish} fall into place: The control exercised by the topic \isi{case hierarchy} over the progression of the clitic recovers visibility and propitiates the development of the object agreement device into a case marker reserved for the \textit{dative} lexical nouns. The datives are the obvious targets, because they rank above the accusative objects in the case role hierarchy.

 From this point of view, the question of how the strong object personal pronouns became subjected to a second type of marking can also be resolved. Although the co-occurrence of two mechanisms, on first sight, might suggest a case of useless overlapping, the truth is that \textit{a} and the doubling clitic complement each other. Both have been motivated by a factor of \isi{topicality}, but the dimensions involved are not the same. Old flagging DOM signals the prominence of the personal pronouns on the \isi{animacy scale}; it is sensitive to their semantic properties. Newer indexing DOM is concerned with degrees of involvement in relation to the \isi{case hierarchy}; it evaluates a participant’s role in the \isi{event structure}. This justifies the association of the \ili{Spanish} personal pronouns with two types of DOM.\footnote{The problem of defining the semantic import of \ili{Spanish} indexing DOM has been addressed in the literature. On the whole, scholars have been especially concerned with offering an account that may serve to differentiate the contribution of the clitic from that of the animacy-related preposition \textit{a}. But no agreement has been reached. Thus, for some, the clitic is supposed to encode the semantic feature of “specificity” \citep{Suner1988Agreement} or “\isi{definiteness}” \citep{vonHeusingeretal2003Interaction,Leonetti2008Specificity}. From another perspective, the doubling form is associated with a condition of discourse “prominence”, for which the notions of both familiarity and activation are relevant (\citealt{Anagnostopoulou1999Conditions}; \cf \citealt{vonHeusingeretal2008Triggering}). And it is also viewed as a mechanism that simply serves to emphasize the heightened \isi{topicality} of DOM marked objects \citep{Escandell-Vidal2009Differential}. 
My proposal seeks to throw new light on this question.} 

\largerpage
\section{Conclusions}
\label{04-me-sec:5}

The present study has dealt with a DOM language whose strong object personal pronouns bear two obligatory markings: they are flagged with the preposition \textit{a} and are indexed on the verb by means of a clitic pronoun. \ili{Spanish} flagging DOM, which goes back to the recorded beginnings of the language, has been thoroughly investigated in both diachronic and synchronic works. Indexing DOM, also known as \isi{clitic doubling} (and DOI under \citeauthor{Iemmolo2014Dislocated}'s (\citeyear{Iemmolo2014Dislocated}) proposal), is the product of a later development traced to Renaissance \ili{Spanish}. It has received less attention in the literature and has been the focus of this paper.

 One common assumption underlying the approaches to phenomena of differential object marking in the languages of the world is the idea that the development of these marking systems proceeds under the guidance of a handful of universally operating hierarchies. However, this assumption has recently been challenged by \citet{Bickeletal2008Referential}, who invite us to consider the possibility that different systems of DOM might originate from individual, highly specific, and non-comparable diachronic changes. What the history of the two types of \ili{Spanish} DOM suggests, as I want to show in my conclusions, is that room should be allowed for both scenarios.

 Thus, starting with the origin of the two marking devices, it is clear that we are being directed to familiar discourse-pragmatic strategies of cross-linguistic character. Flagging \textit{a} begins as a topicalizer, while the (future) indexing clitic of DOM, in the form of a more independent coreferential pronoun, emerges in topic-shift constructions where it binds dislocated objects to the core clause. 

 Both trajectories are also closely tied to the personal pronouns at the beginning stage. A distinction between pronouns and non-pronouns is a well-attested tendency in differential marking systems \citep[195]{Comrie1989Language}. It reflects the way in which language users tend to conceive of the participants coded in the form of personal pronouns as more worthy of being talked “about”, so that the pronouns naturally come to occupy the upper regions of the universal hierarchy of \isi{topicality}. In the case of flagging DOM \citep{Pensado1995Creacion}, the pronominal connection is visible at the onset (late \ili{Latin} and early Romance), when \textit{a} topicalizes the object pronouns of first and second person. With indexing DOM, the connection is established as soon as the clitic starts to develop its differential marking function in Renaissance \ili{Spanish}. 

 How the clitic acquires this function is the result of a particular diachronic change, not susceptible of being cross-linguistically generalized or at least not expected to allow for such enterprise. Without entering into the details of the study presented in this paper, suffice it to say that the functional shift experienced by the coreferential pronoun is achieved through the means of a purposefully redundant construction, used to emphasize the subjective involvement of the pronominal referent in the denoted situation.

 Beyond the peculiarity of this change, the evolutionary paths of both types of DOM bring us back to hierarchies of universal scope. On one side, flagging \textit{a,} linking up with the human and \isi{definite} features of the topical pronouns, begins its descent along the \isi{animacy hierarchy} and grammaticalizes into a nearly obligatory marker with all direct objects of human reference. 

 The \isi{grammaticalization} process of indexing DOM, on the other side, evidences the influence of one of the hierarchic relations involved in the definition of what it means to be topicworthy. Topicworthiness in this case hinges on an underlying concept of \isi{agentivity} and ranks the discourse participants along the hierarchy of semantic case roles in accordance with the degree to which the participants contribute to the event. Observe that the specific evolution of indexing DOM has been anticipated in the use of pragmatic redundancy for the purpose of highlighting the subjective involvement of the twice-mentioned participant. It is this original concern with role issues that predisposes the doubling clitic to become sensitive to the \isi{case hierarchy}. The control exercised by the \isi{case hierarchy} on \ili{Spanish} indexing DOM is perceived during the expansion period of the \isi{grammaticalization} process, via the early entrenchment of the clitic with the more topical subject-like datives; it loses transparency with the extension of the clitic to all the strong object pronouns regardless of their dative or accusative role (propitiated, as I suggested, by the lack of formal case distinctions within the \ili{Spanish} pronominal system); and it again becomes visible when the clitic is introduced into the nominal area of the more topical datives to develop a case-marking function that separates the higher-ranked dative participants [+ clitic] from the lower-ranked accusative object nouns [– clitic].

 From this perspective, it is easier to understand why the strong object personal pronouns carry double marking. Flagging DOM interacts with the semantic properties of \isi{animacy} and \isi{definiteness}, whereas the relevant criterion for indexing DOM is the role of the participant in the \isi{event structure}. The topicworthiness of the personal pronouns is thus simultaneously evaluated on two separate dimensions.

\section*{Acknowledgements}

I am indebted to an anonymous reviewer and to Ilja Seržant, most especially, for critical and helpful comments on earlier drafts of this paper.

\section*{Data sources}

\begin{tabularx}{\textwidth}{lL}
CREA & Electronic data base \textit{Corpus de referencia del español actual} of the Royal Academy of Spanish, accessible through, \url{http://www.rae.es}\\

CORDE & Electronic data base \textit{Corpus diacrónico del español} of the Royal Academy of Spanish, accessible through, \url{http://www.rae.es}\\

\textit{Amadís} &Garci Rodríguez de Montalvo,\textit{ Amadís} \textit{de Gaula}, 1482–92 \\
\textit{Arderique} &Juan de Molina, \textit{Libro del esforzado caballero Arderique}, 1517\\
\textit{Cartas} &Hernán Cortés, \textit{Cartas de relación}, 1519–26 \\
\textit{Crónica} &Francisco Cervantes de Salazar, \textit{Crónica de la Nueva España}, 1560 \\
\textit{Espejo} &Diego Ortúñez de Calahorra, \textit{Espejo de príncipes y caballeros}, 1555\\
\textit{Esplandián}& Garci Rodríguez de Montalvo, \textit{Las sergas del virtuoso caballero Esplandián}, 1504 \\
\end{tabularx}
\newpage 
\begin{tabularx}{\textwidth}{lL}
\textit{Floriseo} &Fernando Bernal, \textit{Floriseo}, 1516\\
\textit{Granada} &Ginés Pérez de Hita, \textit{Guerras civiles de Granada}, 1595 \\
\textit{Grimalte} &Juan de Flores, \textit{Grimalte y Gradisa}, 1495\\
\textit{Guerras} &Pedro Cieza de León, \textit{Las guerras civiles peruanas}, 1553–84\\
\textit{Historia} &Bernal Díaz del Castillo, \textit{Historia verdadera de la conquista de la N. España}, 1568–75\\
\textit{Quijote} &Miguel de Cervantes Saavedra, \textit{El ingenioso hidalgo don Quijote de la Mancha}, 1605\\
\textit{Tristán} &Anonymous, \textit{Tristán} \textit{de Leonís}, 1501 \\
\textit{Ysopo} &Anonymous, \textit{Vida de Ysopo}, 1520\\
\end{tabularx}


\section*{Abbreviations}
\begin{tabularx}{.45\textwidth}{lQ}
1 & first person\\
2 & second person\\
3 & third person\\
\textsc{acc} & accusative\\
\textsc{comp} & complementizer\\
\textsc{dat} & dative\\
\textsc{fem} & feminine\\
\textsc{fut} & future\\
\textsc{imp} & imperative\\
\end{tabularx}
\begin{tabularx}{.45\textwidth}{lQ}
\textsc{inf} & infinitive\\
\textsc{ipfv} & imperfective\\
\textsc{masc} & masculine\\
\textsc{nom} & nominative\\
\textsc{pfv} & perfective\\
\textsc{pl} & plural\\
\textsc{prs} & present\\
\textsc{refl} & reflexive\\
\textsc{sg} & singular\\
\end{tabularx}


{\sloppy
\printbibliography[heading=subbibliography,notkeyword=this] }
\end{document}
