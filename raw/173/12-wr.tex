\documentclass[output=paper]{LSP/langsci}
\ChapterDOI{10.5281/zenodo.1228265} 
\author{Melani Wratil\affiliation{University of Düsseldorf}}
\title{Structural case and objective conjugation in Northern Samoyedic}
%\epigram{Change epigram in chapters/03.tex or remove it there }
\abstract{In Samoyedic syntactic objects and, to a much lesser extent, syntactic subjects are morphologically marked in some way if they pragmatically deviate from the prototypical grammatical relation they represent. The present paper focuses on the Northern Samoyedic branch in this respect, where morphological case and possessive marking, the selection of conjugational patterns and even argument drop is employed to a variable extent in order to assign grammatical functions and to distinguish between the involved arguments and their semantic and pragmatic characteristics. It provides evidence for the fact that the synchronic variation in the manifestation and application of these means in the Northern Samoyedic languages Nganasan, Tundra Nenets and Forest Enets can be explained by the interrelation between the individual developmental paths that specific nominal, pronominal and verbal markers have followed. Whereas in Nganasan the morphophonemic change of number and accusative case markers in conjunction with possessive morphemes and moreover the grammaticalization of the latter to definiteness markers has resulted in a system of differential object marking (DOM) that exclusively applies to nouns, in Tundra Nenets and Forest Enets DOM is implemented by the verbal morphology. This variation in differential marking is attributable to the fact that the agreement suffixes of the objective conjugation in Tundra Nenets and in Forest Enets -- but not in Nganasan -- have adopted substantial functional features of ambiguous object agreement suffixes and at the same time of topic markers. An instance of differential subject marking (DSM) only exists in Nganasan. In contrast to Tundra Nenets and Forest Enets where the paradigm of personal pronouns has been enriched by suppletive accusative forms, Nganasan relies on morphological realization and non-realization in order to mark subject pronouns whose referents do not exhibit the topic- and agent-worthiness of prototypical actor subjects but rather combine specific semantic and pragmatic features of undergoer objects. 

% \keywords{ Samoyedic; grammaticalization; pragmaticalization; conjugation type; agreement}
}
\maketitle

\begin{document}


\section{Introduction}
\label{12-wr-sec:1}

\ili{Samoyedic}, the eastern principal branch of the \ili{Uralic} family, nowadays consists of four still living language groups: \ili{Nganasan} with its dialects Vadey and Avam \citep[480--482]{Helimski1998Nganasan}, the Nenets sub-branch, which is split up into \ili{Tundra Nenets} and Forest Nenets (\citealt[13--14]{Salminen1997Tundra}; \citealt[1--2]{Nikolaeva2014Grammar}), Enets with its sub-languages Tundra Enets and \ili{Forest Enets} \citep[45]{Siegl2013Materials} and finally \ili{Selkup}, which forms a broad dialect continuum \citep[549--550]{Helimski1998Selkup}. According to the classical taxonomy, which is illustrated in Figure~\ref{12-wr-fig:1}, the former three language groups constitute the Northern \ili{Samoyedic} branch, the language area of which is located in North West Siberia and extends from the White Sea region in the West to the Khatanga gulf in the East. \ili{Selkup} is the last survivor of the Southern \ili{Samoyedic} group, which also encompassed the by now extinct Sayan or Mountain \ili{Samoyedic} languages \ili{Kamas} and \ili{Mator} \citep[457--458]{Janhunen1998Samoyedic}. \ili{Selkup} is still sparsely spoken in the West Siberian taiga region enclosed by the Ob and the Yenisei River in the west and the east and by the Turukhan and the Chulym River in the north and the south. More recent approaches interlink \ili{Nganasan} and \ili{Mator} due to their affiliation to the supposedly more archaic, eastern part of \ili{Samoyedic} by separating the former from Nenets and Enets and the latter from \ili{Kamas} and \ili{Selkup} (\cf \citealt[458--459]{Janhunen1998Samoyedic}; \citealt[35--36]{Siegl2013Materials}). 

\begin{figure}
	\caption{Taxonomy of the Uralic languages with localization of structural case/definiteness markers and conjugational splits}\label{12-wr-fig:1}
\fittable{
\begin{forest} 
[\textsc{\ili{Uralic} languages}, s sep-=2mm
  [\textsc{Finno-Ugric},name=finno
    [\textsc{Ugric}
      [\textsc{Ob-Ugric} [Khanty,draw] [\bfseries Mansi, draw]]
      [\textit{Hungarian},draw]
    ]
    [\textsc{Finno-Permic}, s sep-=2mm
      [\textsc{Permic}\\Komi\\Udmurt]
      [\textsc{Volgaic},
        [\textsc{Saamic}-\\\textsc{Fennic}, l sep+=\baselineskip
          [\textsc{Saamic}
            [\begin{tabular}{l@{~}l}
            	South & Ume\\
            	Pite & Lule\\
            	North & Inari\\
            	Akkala & Skolt\\
            	Kildin & Ter
            \end{tabular}, no edge, before drawing tree={y+=1.5\baselineskip}]
          ]
          [\textsc{Baltic-Fennic}
            [\begin{tabular}{l}
            	Livonian\\
	            \textit{Estonian}\\
	            Votic\\
	            Ingrian\\
	            \textit{Finnish}\\
	            Karelian\\
	            Veps
            \end{tabular}, no edge, before drawing tree={y+=1.5\baselineskip}]
          ]
        ]
        [\textsc{Finno-Volgaic}\\\textbf{Mari}, l sep=-2ex[Mordva,draw,no edge]]
      ]
    ]
  ]	
  [\textsc{Samoyedic},name=samo
    [\textsc{Northern} [\textit{\textbf{Nganasan}}\\\textbf{Nenets}\\Enets,draw, no edge, before drawing tree={y+=\baselineskip}] ]
    [\textsc{Southern} [\textit{\textbf{Selkup}}\\Kamass (✝)\\\ili{Mator} (✝),draw,no edge,before drawing tree={y+=\baselineskip}] ]
  ]		
]
\node[below=13\baselineskip of samo,draw] {%
\begin{tabular}{@{}l@{~}l@{~}p{3.5cm}@{}}
	\bfseries bold & = & accusative \textit{-m} still exists\\
	\itshape italic & = & plural \textit{-j} still exists\\
	\frame{framed} & = & differentiation between subjective and objective inflection on verbs
\end{tabular}
};
\end{forest}	
}
\end{figure}

\newpage
\ili{Samoyedic} generally employs \isi{differential argument marking} (DAM). More precisely, syntactic objects and, to a lesser extent, syntactic subjects are morphologically marked in some way if they pragmatically or semantically deviate from the prototypical grammatical relation they represent. Like certain languages of the \ili{Finno-Ugric} branches \ili{Ob-Ugric} and Volgaic, \ili{Samoyedic}
has partially preserved the original Proto-\ili{Uralic} \isi{object marker} \textit{*-m} (\cf Figure~\ref{12-wr-fig:1}). The plural suffix \textit{*-j}, which is still present in the Baltic-Fennic languages \ili{Estonian} and \ili{Finnish} and in \ili{Hungarian} (\cf Figure~\ref{12-wr-fig:1}), has a differentiating function,  especially in \ili{Nganasan}. Like the entire Ugric branch and the \ili{Finno-Volgaic} language Mordva,  \ili{Samoyedic} exhibits an essential \isi{conjugational split} between the subjective or “indeterminative” inflection and the objective or “determinative” inflection.\footnote{In classical Uralistics the \isi{subjective conjugation} is often called “\isi{indefinite}” conjugation whereas the \isi{objective conjugation} is referred to as “\isi{definite}” conjugation.} Especially in the Northern \ili{Samoyedic} languages finite verbs that inflect in the \isi{objective conjugation} agree not only with the syntactic subject in person and number but also with the \isi{direct object} in number \citep[27--30]{Abondolo1998Uralic}. Since the \ili{Samoyedic} number category is subdivided into the values singular, plural and moreover dual, there are three agreement paradigms within the \isi{objective conjugation} of \ili{Nganasan}, Nenets and Enets.

Northern \ili{Samoyedic} makes use of morphological \isi{case marking}, the selection of conjugation types and even argument drop to a variable extent in order to distinguish between arguments and their semantic and pragmatic properties and in order to establish grammatical relations. On the basis of modern \ili{Nganasan}, \ili{Tundra Nenets} and \ili{Forest Enets} data that have been made available by the universities of Moscow and Vienna in the context of their research projects “LangueDOC” and “Negation in \ili{Ob-Ugric} and \ili{Samoyedic} Languages (NOS)” (\cf “Data sources”)\footnote{The corresponding online corpora consist of various annotated narrative texts and comprise 905 \ili{Nganasan}, 260 \ili{Tundra Nenets} and 229 \ili{Forest Enets} sentences in total.} on the one hand and by \citet{Siegl2013Materials} and \citet{Nikolaeva2014Grammar} in the data sections of their \ili{Forest Enets} and \ili{Tundra Nenets} grammar books\footnote{\citegen{Siegl2013Materials} grammar of \ili{Forest Enets} contains various narrative texts that consist of 254 \ili{Forest Enets} sentences in total. \citegen{Nikolaeva2014Grammar} grammar of \ili{Tundra Nenets} contains the edited versions of two Nenets narrations (comprising 482 sentences) that were recorded by Labanauskas in the early 1990s (\cf \citealt{Labanauskas1995Neneckij}).} on the other hand it will be shown in this paper that they represent different intermediate stages in the rise and loss of structural \isi{case marking} and the development of objective suffixes on verbs. 

While \sectref{12-wr-sec:2} presents a cursory overview of argument marking and DAM in Early \ili{Uralic}, \sectref{12-wr-sec:3} is dedicated to the mechanisms of DAM in \ili{Nganasan}. It will turn out that \ili{Nganasan} employs differential object case markers on nouns but does not yet feature any distinct structural \isi{case marking} on personal pronouns. It is argued in \sectref{12-wr-sec:3-2} that the case syncretism of the latter is resolved by specific restrictions on their morphological realization or non-realization, respectively. As shown in \sectref{12-wr-sec:3-3}, the agreement suffixes of the \isi{objective conjugation} have not yet adopted any characteristics of grammatical object agreement markers in \ili{Nganasan}. They incorporate anaphoric third person object arguments by themselves and co-occur with lexical objects only if they are bound as resumptive pronouns in a typical left-dislocation construction. \sectref{12-wr-sec:4} and \sectref{12-wr-sec:5} illustrate that in \ili{Tundra Nenets} and \ili{Forest Enets} differential object \isi{case marking} (DOC) on nouns does not exist. However, whereas \ili{Tundra Nenets} exhibits uniform accusative \isi{case marking} in its nominal declension, \ili{Forest Enets} has lost structural case markers on nouns almost entirely. However, as elucidated in \sectref{12-wr-sec:4-2} and \sectref{12-wr-sec:5-2}, by now their paradigm of personal pronouns has been enriched with distinct accusative forms. Their third person forms are mostly dropped in favor of an objective suffix on the corresponding verbal head. However, in contrast to the agreement morphology of the \ili{Nganasan} objective verb forms, the agreement morphemes of the \ili{Tundra Nenets} and \ili{Forest Enets} objective inflection have gained essential properties of ambiguous object agreement markers. They are no longer simply hosts of the selected object argument. That is why they co-occur with clause-mate objects to a variable extent. In \ili{Tundra Nenets}, as illustrated in \sectref{12-wr-sec:4-3}, they predominantly specify relevant pragmatic properties of these objects while in \ili{Forest Enets}, as shown in \sectref{12-wr-sec:5-3}, they have a discriminatory function. 


\section{Differential argument marking in Early Uralic}\label{12-wr-sec:2}

The main strategies of Northern \ili{Samoyedic} DAM have their roots in Proto-\ili{Uralic}. This pertains to differential \isi{case marking} as well as to the \isi{conjugational split}. Both emerged or were already present in some way in the earliest \ili{Uralic} language periods. 

\subsection{The nominal suffixes \textit{*-m} and \textit{*-j} in Early Samoyedic}\label{12-wr-sec:2-1}

According to \citet[34--35]{Kunnap2008Finnic} Proto-\ili{Uralic} subject and object nouns were distinctively marked with respect to the categories of number and \isi{definiteness} but lacked any case distinctions. \citet{Kunnap2008Finnic} identifies the singular \isi{definiteness} marker \textit{*-m} for Proto-\ili{Uralic}. \citet[172--175]{Katz1979Beitrag}, \citet[29--31]{Janhunen1982Structure} and \citet[65--67]{Honti1995Morphotaktik} postulate the existence of the plural morphemes \textit{*-t} and \textit{*-i} in Proto-\ili{Uralic}. Following \citet[238–239]{Mikola1988Geschichte} \textit{*-i} corresponds to the glided semi-vowel \textit{*-j}, which as inflectional marking derived from an early general augmentative suffix and later functionally contrasted with the other plural marker \textit{*-t.} \citet{Katz1979Beitrag} argues that \textit{*-t} performed the function of \isi{definiteness} marking in the Proto-\ili{Samoyedic} plural paradigm. The suffix \textit{*-j}, however, not only encoded plurality and the absence of \isi{definiteness} but also indicated \isi{accusative case} in his opinion. \citet[21]{Abondolo1998Uralic} agrees with \citet{Katz1979Beitrag} regarding the number and \isi{case marking} function of \textit{*-j}. Like \citet[27]{Salminen1996Comments} and \citeauthor{Janhunen1998Samoyedic} (\citeyear[469]{Janhunen1998Samoyedic}; \citeyear[63]{Janhunen2009Proto-Uralic}), he also defines the Proto-\ili{Uralic} \textit{*-m} as a full-fledged object case marker. 
But he additionally points out that \textit{*-m} originally only attached onto \isi{definite} nouns. 
Thus, \citet{Abondolo1998Uralic} only partially disagrees with \citet[35]{Kunnap2008Finnic} who takes the view that marking by \textit{*-m} was generally applied in order to morphologically indicate \isi{definiteness} in unexpected cases. 
While \isi{definiteness}, which is connected to the topic-worthiness and \isi{animacy} of the referent, is a prototypical feature of agents, it is highly atypical for patient arguments (\cf \citealt[212--214]{Kuno1987Functional}; \citealt[149--158]{Payne1997Describing}; \citealt{Aissen2003Differential}). 
Since \ili{Uralic} employs \isi{accusative alignment} with respect to its case and agreement marking, \citet{Kunnap2008Finnic} infers that singular objects but not singular subjects of Proto-\ili{Uralic} were provided with \textit{*-m} when \isi{definite}. 

Hence, there are different approaches to the Early \ili{Uralic} object and \isi{definiteness} marking, as well as to the Early \ili{Uralic} DAM. 
At least \citet{Katz1979Beitrag}, \citet{Abondolo1998Uralic} and \citet{Kunnap2008Finnic} belong to those Uralists who assume that Early \ili{Uralic} in some sense exhibited DOC conditioned by the \isi{definiteness} and indefiniteness of the lexical nouns involved. 
A definitive rejection or a definitive support of \citegen{Katz1979Beitrag}, \citegen{Abondolo1998Uralic} and \citegen{Kunnap2008Finnic} account have not yet been brought forward. Also the question of whether \ili{Samoyedic} has unalteredly inherited the Early \ili{Uralic} nominal markers or not, is still a matter of debate (\cf \citealt[237]{Mikola1988Geschichte}; \citealt[66]{Salminen1996Comments}; \citealt[36]{Kunnap2008Finnic}). 
Since the above mentioned subject and object markers or traces of them are visible in recent \ili{Samoyedic}, it seems plausible to reconstruct them into Proto-\ili{Samoyedic}. 
Under the premise that they were assigned a differentiating function, Early \ili{Samoyedic} employed differential object marking (DOM) by differential \isi{case marking}. 
More precisely, Early \ili{Samoyedic} \isi{definite} singular objects differed from their \isi{indefinite} counterparts and from singular subjects in that they assumed the \ili{Uralic} \textit{*-m}-suffix. 
Definite plural objects differed from \isi{indefinite} plural objects and also from \isi{indefinite} plural subjects in that they exhibited the plural \textit{*-t-}marker. Indefinite plural objects differed from their \isi{definite} counterparts and, moreover, from \isi{indefinite} plural subjects in that they exhibited the \textit{*-i} or \textit{*-j}-suffix. Exactly this is schematically summed up in the following table:

\begin{table}
\begin{tabularx}{\textwidth}{X cc X cc}
\lsptoprule
& 
\multicolumn{2}{c}{singular} & & 
\multicolumn{2}{c}{plural}\\
\midrule
& \multicolumn{1}{l}{definite} & \isi{indefinite} & & \multicolumn{1}{l}{definite} & \isi{indefinite}\\
nominative & \multicolumn{2}{l}{-} & & \multicolumn{2}{l}{\textit{*-t}}\\
accusative & \multicolumn{1}{l}{\textit{*-m}} & - & & \multicolumn{1}{l}{\textit{*-t}} & \textit{*-j}\\
\lspbottomrule
\end{tabularx}
\caption{Case/definiteness markers on nouns in Early Samoyedic}
\end{table}


\subsection{The conjugational split in Early Samoyedic}\label{12-wr-sec:2-2}

According to \citet{Gulya1995Transitivitat,Honti1995Morphotaktik,Honti2009Tunodesek,Abondolo1998Uralic,Havas2004Objective,Kortvely2005Verb,Kunnap2008Person} and \citet{Kiss2010Magyar}, 
to mention just a few, the \isi{conjugational split} between the subjective and the \isi{objective conjugation} is also nascent in some of the earliest \ili{Uralic} language periods. Honti (\citeyear[59]{Honti1995Morphotaktik}, \citeyear[136--143]{Honti2009Tunodesek}), \citet[119--138]{Havas2004Objective} and \citet[70--88]{Kortvely2005Verb} among others assume that the objective pattern descends from \isi{definite} third person pronouns that encliticized onto finite verbs of transitive clauses. They argue that the \ili{Uralic} third person singular verb forms were the first finite verbs that exhibited the \isi{conjugational split}. In \citegen{Havas2004Objective} and \citegen{Kortvely2005Verb} opinion, this is because only the third person singular verb form of the Early \ili{Uralic} general conjugation lacked an agreement suffix and therefore allowed for an analysis of the object enclitic as an inflectional ending of a special conjugation type. \citet{Havas2004Objective} takes the view that the first and second person objective verb forms emerged much later, after the division into the separate \ili{Uralic} branches. In his opinion the \ili{Hungarian} first and second person objective verb forms displaying a \textit{(V)m}- or \textit{(V)d}-suffix used to belong to the common \ili{Uralic} verbal subject agreement paradigm. 
He argues that they were re-interpreted as first and second person finite verbs that include a \isi{definite} third person pronominal object, while finite third person singular verb forms that were followed by a third person object clitic prevailed as regular agreeing verb forms. \citegen{Mikola1988Geschichte} and \citegen{Kortvely2005Verb} investigations suggest a similar development for \ili{Samoyedic}. They point out that the recent \ili{Samoyedic} first and second person singular subjective verb forms  came into being  later than the corresponding first and second person singular objective verb forms. 
Hence, following \citet{Havas2004Objective} and \citet{Kortvely2005Verb}, the \ili{Uralic} third person singular subjective form is the only subjective form that is of earlier origin than its objective counterpart. This, however, is not in line with Honti’s (\citeyear{Honti1995Morphotaktik,Honti2009Tunodesek}) considerations. Honti (\citeyear{Honti1995Morphotaktik}; \citeyear{Honti2009Tunodesek}) argues for a scenario where the \ili{Uralic} first and second person objective verb forms were analogously created on the basis of verb forms that later made up the \isi{subjective conjugation} or, at least, where these forms arose in tandem with specialized subjective forms. 

\citet[191--196]{Kunnap2008Person} agrees with the approaches by  Honti (\citeyear{Honti1995Morphotaktik,Honti2009Tunodesek}), \citet{Havas2004Objective}, and \citet{Kortvely2005Verb}  with respect to the role of the third person singular verb form. 
In other words, he also assumes that the development of the \ili{Uralic} \isi{objective conjugation} started with third person singular verb forms that indicated the presence of third person objects. 
But, similarly to \citet{Redei1962Targyas}, he formulates the hypothesis that demonstrative suffixes are the source of the verbal objective suffixes. Since, in his view, especially third person possessor agreement affixes generally represent such demonstrative meanings, they attached to the corresponding third person verb forms in the beginning. 
With that \citet{Kunnap2008Person} is able to explain the match between the \ili{Uralic} third person possessor agreement markers on nominal and pronominal categories and the corresponding third person agreement markers on objective verb forms. 

Others, for example \citet{Gulya1995Transitivitat} and \citet{Kiss2010Magyar}, assume that there were various conjugation types already in the early language periods of \ili{Uralic}. 
Whereas \citet[99]{Gulya1995Transitivitat} argues for the existence of an intransitive-transitive split in Proto-\ili{Uralic}, \citet[140–145]{Kiss2010Magyar} traces at least the \ili{Hungarian} \isi{conjugational split} back to three separate verbal paradigms. In her opinion, these paradigms were a reflex of topic agreement. In the presence of a subject topic the clausal main verb agreed with the subject, in the presence of an additional object topic it agreed with the subject and the object and in the absence of any topic it lacked agreement markers. These three agreement patterns melted into two in \ili{Hungarian}. Especially the objective pattern was composed partly of forms agreeing with the subject and partly of forms simultaneously agreeing with the subject and the \isi{direct object}. 
According to \citet{Kiss2010Magyar}, it used to indicate the topichood of the clausal object. 

Hence, whether the \isi{conjugational split} had a \isi{differential argument marking} function already before the separation of the various \ili{Uralic} branches is still a matter of debate. \citet{Honti1995Morphotaktik,Honti2009Tunodesek,Havas2004Objective} and \citet{Kortvely2005Verb} among others are contesting this. They hold the view that the \isi{conjugational split} had nothing to do with DAM in Early \ili{Uralic}. They argue that the objective marker, which exclusively appeared on certain third person verb forms in the beginning, represented a third person pronominal argument by itself. \citet{Kunnap2008Person} and \citet{Kiss2010Magyar}, however, relate the earliest objective suffixes or their immediate predecessors, respectively, to the \isi{information structure} of the corresponding clauses. More precisely, in their view these suffixes indicated a non-prototypical pragmatic status of objects and were therefore responsible for DOM in some sense. 

\section{Nganasan: Differential argument marking on nouns and pronouns}
\label{12-wr-sec:3}

Together with \ili{Mator}, which is extinct probably since the early 19th century, \ili{Nganasan} forms the eastern tract of the \ili{Samoyedic} language area. As depicted in Figure~\ref{12-wr-fig:1} above, \ili{Nganasan} has preserved the \ili{Uralic} accusative marker \textit{-m} as well as the plural morpheme \textit{-j}. These markers are dealt with in \sectref{12-wr-sec:3-1}. 
It is shown that they can be defined as differential object case markers in some sense. 
In \sectref{12-wr-sec:3-2} it is elucidated that the \ili{Nganasan} paradigm of personal pronouns has not yet developed any structural case markers. 
Argument drop on the one hand and morphological realization on the other hand specify the corresponding syntactic functions. The agreement suffixes of the \ili{Nganasan} \isi{objective conjugation} are, as shown in \sectref{12-wr-sec:3-3}, still at the outset of their \isi{grammaticalization} to differential object markers. 


\subsection{Differential object marking on nouns}\label{12-wr-sec:3-1}

The \ili{Uralic} case and number markers \textit{-m} and \textit{-j} are involved in DOM in \ili{Nganasan}. The morpheme \textit{-m} nowadays suffixes to \ili{Nganasan} singular accusative nouns only in case they are \isi{definite} (\cf \REF{12-wr-ex:1}).\footnote{The spelling of the example sentences cited in this article largely complies with the spelling of the corresponding data in the corpora (but see footnote 9). 
Consequently, the spelling of data originating from different corpora may vary slightly even if they document one and the same language.} 
The \isi{definiteness} of these objects is always additionally marked by a possessor agreement marker. Even if there is no potential possessor that has been introduced in the preceding context or discourse, the accusative marker \textit{-m} precedes such a morpheme. 

\ea  \label{12-wr-ex:1}
\langinfo{Nganasan (Avam)}{Northern Samoyedic}{NOS. mou djamezi.134, 313}\\
\ea \label{12-wr-ex:1a}
\gll Təti-rə merigi͡ai-ʔ t’entïrï-ʔi-ðə n’enat’ə-ʔa \textbf{bakəə-ʔa-m-tï} n’akəl’i-ʔe …\\
 that-\textsc{2sg(poss)} quick-\textsc{gen.pl} make-\textsc{pf-3sg.rc} huge-\textsc{augm} scraper-\textsc{augm-acc-3sg(poss)} take-\textsc{pf(3sg.sc)}\\
\glt ‘He prepared everything and took the big scraper …’  %(NOS. mou djamezi.134)

\ex \label{12-wr-ex:1b}
\gll Tahari͡aa \textbf{ŋoj-mə} təbə-l’i-ʔe-ŋ.\\
 now leg-\textsc{acc.1sg(poss)} press-\textsc{inch-pf-2sg.sc}\\
\glt ‘You are squeezing my leg now.’ %(NOS. mou djamezi.313)
\z
\z
 
Especially the third person possessive suffixes, such as \textit{-tï} in \REF{12-wr-ex:1a}, have meanwhile entered the \isi{grammaticalization path} to nominal \isi{definiteness} markers on objects.\footnote{In accordance with \citet[2]{Hopperetal1993Grammaticalization} I define all diachronic processes where a specific lexeme or discourse structure receives a grammatical function or where a function word or a functional morpheme becomes more functionalized through time as instances of \isi{grammaticalization}. For the sake of simplicity I do not draw a distinction between ‘primary’ and ‘secondary’ (\cf \citealt{Traugott2004Exaptation}) \isi{grammaticalization}.} They have lost their specific reference to any possessing entity via semantic bleaching. As shown by \citet{Gerland2014Possessive}, nowadays they  indicate general belonging and thus a certain degree of specificity. 
That is why they are used for expressing prominence or simply \isi{definiteness} in contexts that lack any available possessor.\footnote{\citet{Toivonen1998Lexical}, \citet{Bartos1999Morfoszintaxis} and \citet{Dekany2015Syntax} among others have observed a similar distribution of third person possessor agreement affixes in some Ugric and Saamic-Fennic varieties. According to them, these affixes have lost their person specification. They are suitable for speech act participant (SAP) as well as for non-SAP possessors. They merely indicate that the referent of the nominal expression they are attached to is in some possessive relationship.} 

Since accusative \textit{-m} has degeminated in conjunction with the first person singular, dual and plural possessive affixes \textit{-mǝ}, \textit{-mi\textsuperscript{c}} and \textit{-muʔ}, the accusative possessum nouns agreeing with any first person possessor are homonymous with the corresponding nominative forms \citep{Salminen1996Comments}. Hence the object \textit{ŋojmə} ‘my leg’ of \REF{12-wr-ex:1b}, which was presumably pronounced with a gemination of the bilabial nasal \textit{-m} (*\textit{ŋojmmə}) in earlier language periods, formally coincides with the corresponding nominative noun.

The absence of the accusative \textit{-m} suffix on \isi{indefinite} singular objects like \textit{sənəhu͡aa} ‘a larch’ and \textit{kubaʔa} ‘a huge skin’ in \REF{12-wr-ex:2} is not a reflex of the Early \ili{Uralic} DOM. 
Rather, it has to do with a quite innovative phonological change that has resulted in a regressive assimilation ensuing from the word final accusative \textit{-m} and its subsequent apocope. 
Morphophonemic influences of an erstwhile \textit{-m} morpheme, which was the obligatory accusative marker probably till the 19th century (\cf \citealt[156]{Castren1845Grammatik}), can be observed on \isi{indefinite} accusative non-possessum nouns until today (\citealt[71--89]{WagnerNagy2002Chrestomathia}; \citealt[357--365]{Katzschmann2008Chrestomathia}). 

\ea%2
\label{12-wr-ex:2}
\langinfo{Nganasan  (Avam)}{Northern Samoyedic}{NOS. kehy luu.114, NOS. mou djamezi.110}\\
\ea\label{12-wr-ex:2a}
\gll \textbf{Sənəhu͡aa} ŋəði-ʔə.\\
 larch\textsc{(acc)} find-\textsc{pf(3sg.sc)}\\
\glt ‘He found a larch.’

\ex\label{12-wr-ex:2b}
\gll … bintiʔs’i \textbf{nenat’a-ʔa} \textbf{kuba-ʔa} təða-ʔa\\
 {} wolverine huge\textsc{(acc)}-\textsc{augm} skin\textsc{.acc}-\textsc{augm} bring-\textsc{pf(3sg.sc)}\\
\glt ‘… he brought a huge skin of the wolverine.’
\z
\z

Plural \isi{definite} object nouns like \textit{s’iərt’i} ‘the news’ in \REF{12-wr-ex:3a} match the corresponding possessum nominative nouns. Like the latter they undergo a stem alternation and display a possessor agreement affix, which is phonologically shaped by the formerly preceding connective morpheme \textit{*-j} \citep[84]{WagnerNagy2002Chrestomathia}. According to \citet[29--32]{Janhunen1982Structure} exactly this \ili{Uralic} connective *\textit{-j} has become the plural accusative marker in Early \ili{Samoyedic}. In recent \ili{Nganasan} it suffixes to all \isi{indefinite} plural objects. This is shown in \REF{12-wr-ex:3b} where the \isi{indefinite} object \textit{latəəj} ‘bones’ exhibits a final \textit{-j} morpheme. With that the \isi{indefinite} objects morphologically differ not only from their \isi{definite} counterparts but also from the non-possessum plural subjects, which exhibit the plural marker \textit{-ʔ} like \textit{mirəimaʔ} (‘the steps’) in \REF{12-wr-ex:3c}. As shown by \citet[238]{Mikola1988Geschichte}, \textit{-ʔ} is an immediate descendant of the Proto-\ili{Uralic} plural marker \textit{*-t}. 

\ea \label{12-wr-ex:3}
\langinfo{Nganasan  (Avam)}{Northern Samoyedic}{NOS. mou djamezi. 173, 062, 130}\\

\ea \label{12-wr-ex:3a}
\gll Bəńd’ə təniʔia \textbf{sʼiər-tʼi} dʼebta-ʔa.\\
 all\textsc{(acc)} so affair\textsc{(acc)}-\textsc{pl.3sg(poss)} tell-\textsc{pf(3sg.sc)}\\
\glt  ‘He told all the news.’ %(NOS. mou djamezi.173)

\ex \label{12-wr-ex:3b}
\gll Tahari͡aa satəra-ŋku maa-güǝ hünʼsʼərəədʼəə \textbf{latəə-j} ŋonəi-ʔ təða-ʔa.\\
 now polar.fox-\textsc{dim} what-\textsc{cl} ancient(\textsc{acc)} bone-\textsc{acc.pl}  one.more-\textsc{gen.pl} bring-\textsc{pf(3sg.sc)}\\
\glt  ‘Then the little polar fox brings some old bones.’ %(NOS. mou djamezi.062)

\ex \label{12-wr-ex:3c}
\gll … \textbf{mirəima-ʔ} sojbu-ʔə-ʔ nʼenama-gitə.\\
 { } step-\textsc{nom.pl} begin.to.sound-\textsc{pf-3pl.sc} neighbour-\textsc{abl.pl}\\
\glt  ‘The steps of the neighbour resounded.’ %(NOS. mou djamezi.130)
\z
\z

Dual objects are exempted from DOM. On the one hand, this is because there is no specific agglutinative accusative morpheme in the dual number. 
On the other hand, duality is in some sense associated with the cohesiveness of the involved participants anyhow. As a consequence, dual objects normally display a possessor agreement affix in \ili{Nganasan} like in all other \ili{Samoyedic} languages -- irrespective of how \isi{definite} they are. 
Thus, they are naturally syncretic with the corresponding nominative dual possessum nouns.

Consequently, there is DOM only on singular and plural nouns in contemporary Nga\-na\-san. The accusative marker \textit{-m} suffixes to singular \isi{definite} objects and is always accompanied by a possessor agreement affix. In this way \ili{Nganasan} \isi{definite} singular objects differ from their \isi{indefinite} counterparts, whose accusative marker has demorphologized and which moreover lack any possessor agreement suffix. The accusative marker \textit{-j}, however, suffixes to \isi{indefinite} plural objects. 
Accordingly, \ili{Nganasan} \isi{indefinite} plural objects differ from their \isi{definite} counterparts, whose former number marker and predecessor of the accusative \textit{-j} has demorphologized and which moreover take a possessor agreement affix. Exactly this is summed up in \tabref{12-wr-tab:2}.\footnote{\textsc{sa =} stem alternation; \textsc{poss =} possessor agreement morpheme}

\begin{table}
\begin{tabularx}{\textwidth}{XXXXX}
\lsptoprule
& \multicolumn{2}{c}{singular} & \multicolumn{2}{c}{plural}\\
& \isi{definite} & \isi{indefinite} & \isi{definite} & \isi{indefinite}\\
\midrule
nominative & - & - & -\textit{Ɂ} & -\textit{Ɂ}\\
accusative & \textsc{(sa)}-\textit{m-}\textsc{poss} & \textsc{(sa)} & \textsc{(sa)}-\textsc{poss} & \textsc{(sa)}-\textit{j}\\

\lspbottomrule
\end{tabularx}
\caption{Structural case/definiteness markers on nouns in Nganasan}\label{12-wr-tab:2}
\end{table}


\subsection{Differential argument marking on personal pronouns}\label{12-wr-sec:3-2}

\tabref{12-wr-tab:3} illustrates that \ili{Nganasan} personal pronouns do not show any morphological distinction between their structural case forms (\cf \citealt[93]{WagnerNagy2002Chrestomathia}). 


\begin{table}
\begin{tabularx}{\textwidth}{XXXX} 
\lsptoprule
& nominative & accusative & genitive\\
\midrule
\textsc{1sg} & \textit{mənə} & \textit{mənə} & \textit{mənə}\\
\textsc{2sg} & \textit{tənə} & \textit{tənə} & \textit{tənə}\\
\textsc{3sg} & \textit{sïtï} & \textit{sïtï} & \textit{sïtï}\\
\textsc{1dual} & \textit{mi} & \textit{mi} & \textit{mi}\\
\textsc{2dual} & \textit{ti} & \textit{ti} & \textit{ti}\\
\textsc{3dual} & \textit{sïtï} & \textit{sïtï} & \textit{sïtï}\\
\textsc{1pl} & \textit{mïŋ} & \textit{mïŋ} & \textit{mïŋ}\\
\textsc{2pl} & \textit{tïŋ} & \textit{tïŋ} & \textit{tïŋ}\\
\textsc{3pl} & \textit{sïtïŋ} & \textit{sïtïŋ} & \textit{sïtïŋ}\\
\lspbottomrule
\end{tabularx}
\caption{Structural case paradigm of Nganasan personal pronouns \citep{WagnerNagy2002Chrestomathia}} \label{12-wr-tab:3}
\end{table}


Thus, \ili{Nganasan} personal pronouns are at first glance inconsistent with the common \isi{markedness} hierarchies of DOM, which predict that pronouns are generally more likely to be case marked than lexical nominal expressions (\citealt{Bossong1985Differentielle}; \citealt{Croft1988Agreement}; \citealt{Aissen2003Differential}). However, it has been shown in \citet{Wratil2013Distribution} that, although the \ili{Nganasan} system of personal pronouns does not employ any overt \isi{case marking} of direct objects, it does not constitute a categorical exception to these hierarchies. 
This is because the individual grammatical function of its pronominal items is determined on the basis of their morphological realization and non-realization. Whether and in which way personal pronouns appear is constrained by the ranking of their thematic roles in the actor and \isi{undergoer} hierarchies as well as by their person feature value. Following \citet[53--72]{VanValin2001Syntax} the actor and the \isi{undergoer} hierarchy can be outlined as follows:

\ea
\textbf{\textit{Actor Hierarchy}}\\ 
Agent > Instrument > Experiencer > Recipient
\z

\protectedex{
\ea
\textbf{\textit{Undergoer Hierarchy}}\\ 
Patient > Theme > Stimulus > Experiencer > Recipient / Goal / Source / Location
\z}

According to the actor hierarchy, the agent role has the most actor-like properties. It is the prototypical \isi{thematic role} of all arguments that refer to acting, initiating, willing and mostly human entities. According to the \isi{undergoer} hierarchy the patient role has the most undergoer-like properties. It is the prototypical \isi{thematic role} of all arguments that refer to undergoing, passive and often non-human entities that are affected by an event or action. Experiencer and recipient roles combine actor and \isi{undergoer} properties. They are low in the actor hierarchy as well as in the \isi{undergoer} hierarchy. The corresponding referents are affected by conditions, situations, impressions or actions but are not completely passive and powerless. In most cases they are \isi{animate} and willful entities.

In \ili{Nganasan} the realization of subject pronouns is constrained by the \isi{thematic role} they bear \citep[248--262]{Wratil2013Distribution}. The more actor-like the \isi{thematic role} of a subject personal pronoun is, the more likely it is unmarked, hence, the less likely it is to be realized as a free pronoun. On the other hand, the more undergoer-like its \isi{thematic role} is, the more likely it is to have a morphological representation as one of the pronominal items illustrated in the first column of \tabref{12-wr-tab:2}. This is illustrated in examples \REF{12-wr-ex:4} and \REF{12-wr-ex:5}. 

\ea\label{12-wr-ex:4}
\langinfo{Nganasan  (Avam)}{Northern Samoyedic}{NOS. mou djamezi.022, NOS. kehy luu.021}\\
 
\ea \label{12-wr-ex:4a}
\gll (*Sïtïŋ) tahari͡aa maara-j kotə-kə-ndu-ʔ.\\
 (*they) now any-\textsc{acc.pl} destroy\textsc{-iter-aor-3pl.sc}\\
\glt  ‘They kill everything.’ 

\ex \label{12-wr-ex:4b}
\gll Maa-ðə (*tənə) mənə muaʔkuj-ŋu-əu-ŋ?\\
what-\textsc{abl.adv} (*you) I torment-\textsc{interr-excl-2sg.sc}\\
\glt ‘Why are you tormenting me?’
\z
\z

\ea \label{12-wr-ex:5}
\langinfo{Nganasan  (Avam)}{Northern Samoyedic}{NOS. kehy luu.036, Languedoc. dva čuma.023, Languedoc. škola.024}\\

\ea \label{12-wr-ex:5a}
\gll N’enatʼə-ʔa hu͡aa-ʔa katʼəmi-ʔə.\\
 huge\textsc{(acc)}-\textsc{augm} tree\textsc{(acc)}-\textsc{augm} see-\textsc{aor.3sg.sc}\\
\glt ‘He noticed a tall tree.’

\ex \label{12-wr-ex:5b}
\gll \textbf{Mənə} təəśəðə	təʔ               ŋəmnam-suə-m.\\
I         totally	you.know  be.hungry-\textsc{pst-1sg.sc}\\
\glt ‘I was totally hungry.’

\ex \label{12-wr-ex:5c}%adjusted the example slightly to make it more readable
\gll *\textbf{(Mi)}   təndə       śiəðïr-mənï        ŋimi-ľə-rɨ-ʔi-niʔ\\
we.\textsc{du} that.\textsc{gen} window-\textsc{prol} drag-\textsc{inch-pass-aor-1du.rc}\\
\glt  ‘We were dragged through the window.’
\z
\z

In \REF{12-wr-ex:4a} and \REF{12-wr-ex:4b} the finite lexical verb selects a subject that features most characteristics of a prototypical agent. Its referent is acting, initiating, willing and \isi{animate}. Consequently, it is not morphologically realized as a personal pronoun. Its person and number features are specified by the inflectional morphology of the corresponding verb. In \REF{12-wr-ex:4a} the subjective subject agreement suffix of the main verb indicates that the clausal subject is a third person plural subject. In example \REF{12-wr-ex:4b} it identifies a second person singular subject. By contrast, \REF{12-wr-ex:5a} and \REF{12-wr-ex:5b} contain a main verb that assigns its subject an \isi{experiencer} role. Since the \isi{experiencer} role is quite low on the actor as well as on the \isi{undergoer} hierarchy, the corresponding pronominal subject may be omitted like in \REF{12-wr-ex:5a} or morphologically realized like in \REF{12-wr-ex:5b}. As shown by \citet[257--261]{Wratil2013Distribution}, verbs that do not assign any specific \isi{thematic role} like copulas or that withdraw role assignment in some sense like negation auxiliaries are also quite liberal with respect to the (non-)realization of their pronominal subjects. The same holds true for verbs that background their agent argument due to a specific valence or aspect marker. In passive clauses like \REF{12-wr-ex:5c}, however, the subject combines all properties of a typical patient. It is therefore necessarily realized as overt personal pronoun. 

Direct object personal pronouns, which are normally assigned the undergoer-like roles patient and theme, are always overt. Thus, their grammatical relation already determines their morphological manifestation as overt free personal pronouns. As illustrated by \REF{12-wr-ex:6} and \REF{12-wr-ex:4b} above, this holds true at least for the speech act participant (SAP) objects, \ie for all singular, dual and plural object personal pronouns with a first or second person specification. 
In \REF{12-wr-ex:4b}, for example, the transitive main verb takes a first person singular object and in \REF{12-wr-ex:6a} a second person plural object, which is morphologically realized as \textit{tïŋ}. The finite verb of \REF{12-wr-ex:6b} follows its first person dual object \textit{mi}. 

\ea \label{12-wr-ex:6}
\langinfo{Nganasan  (Avam)}{Northern Samoyedic}{NOS. mou djamezi.223, Languedoc. škola.034}\\

\ea \label{12-wr-ex:6a} 
\gll tahari͡aa tïmini͡a \textbf{tïŋ}  ŋəðə-ʔki-ʔə-m\\
 now now you.\textsc{pl(acc)} examine-\textsc{res-aor-1sg.sc}\\
\glt  ‘Now I will search you.’

\ex \label{12-wr-ex:6b} 
\gll Bejkiʔmi͡aʔku tʼüü-tʼü kunsï-mənï \textbf{mi}  mütəmi-ʔə\\
 Beikimyaku sleeping.bag-\textsc{gen.3sg(poss)} inside-\textsc{prol} we.\textsc{du(acc)} put-\textsc{aor(3sg.sc)}\\

\glt ‘Bejkimjaku puts us in her sleeping bag.’
\z
\z

Accordingly, the quite unusual lack of structural \isi{case marking} within the \ili{Nganasan} paradigm of personal pronouns is compensated for by a system of realization and omission. Whereas SAP objects are always realized by overt free personal pronouns, subject personal pronouns are morphologically realized only if their \isi{thematic role} deviates from the \isi{thematic role} prototypical subjects are assigned to. Consequently, \ili{Nganasan} employs a strategy of DSM that is mainly conditioned by semantic roles. 
Thus, it is an atypical instances of DSM. But in some sense it is also a reflex of the topic-worthiness of referents. More precisely, only \ili{Nganasan} subjects that bear properties of high topic-worthiness such as \isi{definiteness} and/or \isi{animacy} and moreover adopt a \isi{thematic role} that is extremely high on the actor hierarchy are completely unmarked, hence, lack any morphological representation. 


\subsection{Argument incorporation and objective conjugation}\label{12-wr-sec:3-3}

The number of the third person personal pronouns \textit{sïtï} and \textit{sïtïŋ} that occur as direct objects in the finite clauses of the accessible corpora is vanishingly small. Nevertheless there are numerous two-or more-participant clauses whose finite verb takes a third person \isi{direct object} that is \isi{definite} and anaphoric. However, these clauses as, for example, \REF{12-wr-ex:7a} and \REF{12-wr-ex:7b}, differ from the other two-or more-participant finite clauses not only in that they lack any free object but also in that their main verb is inflected in the \isi{objective conjugation}. The respective agreement suffixes are given in \tabref{12-wr-tab:4} below.\footnote{\tabref{12-wr-tab:4} only contains the basic morphs of these suffixes. Note that there is a wide range of phonologically conditioned allomorphy within the \ili{Nganasan} agreement paradigms.} %TODO this is footnote 9, reference might be working with package footmisc, will try later

\protectedex{
\ea \label{12-wr-ex:7} 
\langinfo{Nganasan  (Avam)}{Northern Samoyedic}{NOS. mou djamezi.153, 241}\\

\ea  \label{12-wr-ex:7a} 
\gll \textbf{Ka’təmi-ʔe-ðu.}\\
 look-\textsc{pf}-\textsc{3sg.oc}\\
\glt  ‘He has looked at it.’

\ex \label{12-wr-ex:7b} 
\gll kuni-ðe \textbf{ŋətə-d’üəd’əə-ðuŋ}?\\
 where-\textsc{abl} find-\textsc{pstpf-3pl.oc}\\
\glt  ‘Where did they find it?’ 
\z
\z
}

\begin{table}
\begin{tabularx}{\textwidth}{X  X X@{\,}X@{\,}X X} 
\lsptoprule
& subjective & & objective & & reflexive\\
& & singular & {dual} & plural &\\
\midrule
\textsc{1sg} & \textit{-m} & \textit{-mə} & \textit{-kəi-j-n}\textit{ə} & \textit{-j-n}\textit{ə} & \textit{-nə}\\
\textsc{2sg} & \textit{-ŋ} & \textit{-rə} & \textit{-kəi-j-t}\textit{ə} & \textit{-j-t}\textit{ə} & \textit{-ŋ}\\
\textsc{3sg} & \textit{${\varnothing}$} & \textit{-tu} & \textit{-kəi-j-tu} & \textit{-j-tu} & \textit{-t}\textit{ə}\textit{\textsuperscript{c}}\\
\textsc{1dual} & \textit{-mi}\textit{\textsuperscript{c}} & \textit{-mi}\textit{\textsuperscript{c}} & \textit{-kəi-j-ni}\textit{\textsuperscript{c}} & \textit{-j-ni}\textit{\textsuperscript{c}} & \textit{-ni}\textit{\textsuperscript{c}}\\
\textsc{2dual} & \textit{-ri}\textit{\textsuperscript{c}} & \textit{-ri}\textit{\textsuperscript{c}} & \textit{-kəi-j-ti}\textit{\textsuperscript{c}} & \textit{-j-ti}\textit{\textsuperscript{c}} & \textit{-nti}\textit{\textsuperscript{c}}\\
\textsc{3dual} & \textit{-kəj} & \textit{-ti}\textit{\textsuperscript{c}} & \textit{-kəi-j-ti}\textit{\textsuperscript{c}} & \textit{-j-ti}\textit{\textsuperscript{c}} & \textit{-nti}\textit{\textsuperscript{c}}\\
\textsc{1pl} & \textit{-mu}\textit{ʔ} & \textit{-mu}\textit{ʔ} & \textit{-kəi-j-nuʔ} & \textit{-j-nuʔ} & \textit{-nuʔ}\\
\textsc{2pl} & \textit{-ru}\textit{ʔ} & \textit{-ru}\textit{ʔ} & \textit{-kəi-j-tuʔ} & \textit{-j-tuʔ} & \textit{-ntuʔ}\\
\textsc{3pl} & \textit{-ʔ} & \textit{-tuŋ} & \textit{-kəi-j-tu}\textit{ŋ} & \textit{-j-tu}\textit{ŋ} & \textit{-ntəʔ}\\
\lspbottomrule
\end{tabularx}
\caption{Verbal suffixes of the subjective, objective and reflexive conjugation in Nganasan \citep{WagnerNagy2002Chrestomathia}}\label{12-wr-tab:4}
\end{table}

As soon as any free pronominal \isi{direct object} appears within a minimal clause, the corresponding main verb inflects in the \isi{subjective conjugation} the inflectional pattern of which is listed in the first column of \tabref{12-wr-tab:4}. This holds true for all \isi{definite} object pronouns as for example for the personal pronouns including all SAP and third person pronouns and for all \isi{indefinite} object pronouns. The sentences of \REF{12-wr-ex:6} in \sectref{12-wr-sec:3-2} illustrate the co-occurrence of SAP objects and finite verbs with subjective patterns. Example \REF{12-wr-ex:8a} belongs to the extremely rare clauses that contain a third person object personal pronoun while \REF{12-wr-ex:8b} and \REF{12-wr-ex:8c} exhibit \isi{indefinite} pronominal objects. As can be observed, each of these third person objects precedes a subjective verb form. 

\protectedex{
\ea\label{12-wr-ex:8}%8
\langinfo{Nganasan  (Avam)}{Northern Samoyedic}{NOS. kehy luu.196, NOS. mou djamezi.027, 022}\\

\ea \label{12-wr-ex:8a}
\gll  Bəńd’ə-ʔ \textbf{sïtï} \textbf{n’üəsïj-t’i-ʔ} təndə kobtu͡a-m-tuŋ n’üəsï-ndï-ʔ.\\
 all-\textsc{pl} she kiss\textsc{-prs-3pl.sc} there girl-\textsc{acc.sg-3pl(poss)} kiss-\textsc{prs-3pl.sc}\\
\glt  ‘All people kissed her, they kissed their girl there.’

\ex \label{12-wr-ex:8b}
\gll maa \textbf{ńakələ-tə-ŋɨ}\\
 what(\textsc{acc}) take-\textsc{fut-inter(3sg.sc)}\\
\glt  ‘What does it take?’

\ex \label{12-wr-ex:8c}
\gll tahari͡aa maara-j \textbf{kotə-kə-ntu-ʔ}\\
now any-\textsc{acc.pl} bag-\textsc{iter-prs-3pl.sc}\\
\glt ‘They kill everything.’
\z
\z
}

The vast majority of clauses that display a non-pronominal \isi{direct object} are also headed by a finite verb inflected in the \isi{subjective conjugation}. None of the minimal clauses containing a non-pronominal object constituent mentioned in \sectref{12-wr-sec:3-1} exhibits a verbal head that bears an objective suffix -- irrespective of whether this object constituent is \isi{definite} or \isi{indefinite}. The example clauses \REF{12-wr-ex:9a} with a \isi{definite} object and \REF{12-wr-ex:9b} with an \isi{indefinite} object are further examples that illustrate the subjective inflection due to the presence of any free object.

\ea \label{12-wr-ex:9}%9
\langinfo{Nganasan  (Avam)}{Northern Samoyedic}{NOS. kehy luu.149, Languedoc. koujkia.006}\\

\ea \label{12-wr-ex:9a}
\gll ŋonəi-ʔ \textbf{śigiʔi-ʔ}  \textbf{luu-ʔə-m-tu}  \textbf{śeri-ʔə}\\
 one.more\textsc{-adv} ogre-\textsc{gen.pl} parka-\textsc{augm-acc-sg.3sg(poss)} put.on-\textsc{pf(3sg.sc)}\\
\glt  ‘He has put on once more the ogre’s parka.’

\ex \label{12-wr-ex:9b}
\gll Ta-gǝtǝ lakarїari͡aʔ     \textbf{maagüə}  \textbf{saü}  \textbf{d’indi- ʔə-gǝj}.\\
 that-\textsc{abl} suddenly somewhat noise(\textsc{acc)} hear-\textsc{pf-3du.sc}\\
\glt ‘Then they suddenly heard some noise.’
\z
\z

In turn, constructions whose main verb exhibits an objective suffix alongside a non-pronominal accusative object constituent are extremely rare. As has been elucidated in \citet[251--257]{Wratil2013Distribution}, these object phrases have in common that they refer to topic entities. They represent old or contextually presupposed information and are marked as being \isi{definite} by an appropriate possessor agreement morpheme. Moreover, they appear in the left-peripheral position. This is illustrated by \REF{12-wr-ex:10b}. The unambiguously accusative noun \textit{banəmtu} ‘dog’ establishes the white dog, which has been introduced earlier in the discourse (\cf \REF{12-wr-ex:10}), as the  primary topic. 

\ea \label{12-wr-ex:10}%10
\langinfo{Nganasan  (Avam)}{Northern Samoyedic}{Languedoc. rebjata. 031, 033}\\
\ea \label{12-wr-ex:10a}
\gll tə-tə tahari͡abiɁ /ńerəbtukuɁ/ \textbf{ńimə} \textbf{hon-tïə} \textbf{ban-tu} \textbf{təi-śüə}  ďeŋkuə banu-Ɂə təti bəjkaɁa\\
well now /at.first/ name(\textsc{acc)} have-\textsc{ptcp} dog-\textsc{3sg(poss)} be.available-\textsc{pst(3sg.sc)} white dog-\textsc{augm} that old.man\\
\glt ‘The famous white dog originally belonged to the old man.’

\ex \label{12-wr-ex:10b}
\gll \textbf{ban-əm-tu} \textbf{śüküða-Ɂa-ðu}  tahari͡aa buəgəlïðə-j ŋantəmə-gə-śa  \textbf{ban-əm-tu} \textbf{mütəmi-Ɂə} ďebakuə turka-Ɂa ńa-ntə i-śa\\
dog-\textsc{acc-3sg(poss)} strangle-\textsc{aor-3sg.oc} now good.words-\textsc{acc.pl} pray-\textsc{iter-inf} dog-\textsc{acc.3sg(poss)} send-\textsc{aor(3sg.sc)} red.\textsc{gen} lake-\textsc{augm.gen} friend-\textsc{lat} be-\textsc{inf}\\
\glt ‘The dog he strangled, praying good words he sent the dog to the ground of the red lake.’
\z
\end{exe}

Hence, it is at least debatable whether the accusative noun \textit{banəmtu} is part of the minimal clause containing the finite main verb inflected in the \isi{objective conjugation} at all. It is conceivable that \textit{banəmtu} is a left dislocated topic constituent that is referentially associated with a clause internal resumptive pronoun or clitic. The agreement suffix of the following objective verb form would represent the clause internal resumptive element in this case. The fact that \textit{banəmtu} precedes a finite verb inflected in the \isi{subjective conjugation} in the subsequent asyndetical conjunct \REF{12-wr-ex:10b}, corroborates this analysis. Since the discourse properties of the mentioned referent are fully defined by a left dislocation procedure in the first conjunct, it behaves like a canonical object in the second conjunct. 

The distribution of objective verb forms described in this section allows to conclude that \ili{Nganasan} is situated on an early stage in the development of the \isi{conjugational split}. Especially the data of \REF{12-wr-ex:7} and \REF{12-wr-ex:8} suggest that the suffixes of objective verb forms still include pronominal third person object arguments by themselves. Note that this incorporation hypothesis complies with \citegen{Havas2004Objective} and \citegen{Kortvely2005Verb} assumptions about the roots of the \ili{Uralic} \isi{objective conjugation}. According to these considerations their incompatibility with free clause-mate accusative pronouns can be quite convincingly explained. Since pronominal clitics may be bound as resumptive elements by a topicalized object phrase in clitic left-dislocation constructions, sentences like \REF{12-wr-ex:10b} also fit this analysis. But \REF{12-wr-ex:10b} supports \citegen{Kiss2010Magyar} topic agreement approach to the evolvement of the \isi{objective conjugation} as well. 
This is because the objective verb form \textit{śüküðaɁaðu} ‘strangled’ in some sense points to the special topic status of the sentence initial object constituent. 


\section{Tundra Nenets: Information structuring and the objective inflection }
\label{12-wr-sec:4}

\ili{Tundra Nenets} is the language spoken by the westernmost speech community of the Northern \ili{Samoyedic} region (\cf \citealt[iv]{Abondolo1998Uralic}; \citealt{Nikolaeva2014Grammar} among others). In contrast to \ili{Nganasan}, \ili{Tundra Nenets} does not exhibit DOM on nouns. This is shown in \sectref{12-wr-sec:4-1}. Since, as pointed out in \sectref{12-wr-sec:4-2}, its paradigm of personal pronouns has been enriched with distinct accusative forms, \ili{Tundra Nenets} also lacks DSM within its pronominal system. Nevertheless \ili{Tundra Nenets} employs DAM in some sense. 
This is because, as elucidated in \sectref{12-wr-sec:4-3}, the \ili{Tundra Nenets} objective suffixes have acquired the essential features of ambiguous verbal agreement markers in the sense of \citet[225--331]{Siewierska1999Anaphoric} and at the same time assumed an information structuring function. 


\subsection{Uniform structural case marking on nouns}\label{12-wr-sec:4-1}

DOC does not apply to \ili{Tundra Nenets} nouns. 
Uniform accusative \isi{case marking} prevails instead. In the singular number this is attributable to the analogical extension of the \ili{Uralic} nominal marker \textit{*-m} to all kinds of lexical objects. 
Therefore contrary to \ili{Nganasan}, which has retained \textit{*-m} merely in connection with possessor agreement markers, \ili{Tundra Nenets}  lacks differential accusative marking on object nouns in the singular number. 
This is illustrated by the example sentences of \REF{12-wr-ex:11}.\footnote{There is a phonemic difference between the nasalizable and the non-nasalizable glottal stop. 
The former is marked by \textit{h} and the latter by \textit{q} in a number of treatments of Nenets phonology and morphology (\cf \ie \citealt[522--523]{Salminen1998Nenets}; \citealt[18--19]{Nikolaeva2014Grammar}). 
For the sake of simplicity, I follow \citet{Hajdu1988Samojedischen} in not drawing a graphemic distinction between the nasalizable and the non-nasalizable glottal stop. 
This pertains to the following example sentences and tables, where \textit{ʔ} covers both kinds of glottal stop.
}

\ea\label{12-wr-ex:11}%11
\langinfo{Tundra Nenets}{Northern Samoyedic}{NOS. tesjada nisjami.058, 023, NOS. tet weli teta.105}\\

\ea \label{12-wr-ex:11a}
\gll \textbf{Nʼe}  tarʼem ma: …\\
 woman so say\textsc{(3sg.sc)}\\
\glt  ‘The woman said: ...’ 

\ex \label{12-wr-ex:11b}
\gll  Tʼi Tes’ada n’is’e-mi \textbf{m’apoj-m} pod’erŋa.\\
 so Tesjada father-\textsc{1sg(poss)} small.reindeer.caravan-\textsc{acc} harness\textsc{(3sg.sc)}\\
\glt  ‘So, my father Tesjada harnessed a small reindeer caravan.’

\ex  \label{12-wr-ex:11c}
\gll Ŋarka Wel’i teta xasawa \textbf{n’u-m} malca-xa-danta nixibta-da, man-ma:\\
 big Welji farmer man child\textsc{-acc} malice-\textsc{dat-3sg.dat} pull-\textsc{3sg.oc} pull-\textsc{narr}\\
\glt  ‘He caught hold of the malice of the son of the old Weli-farmer and said:’ 
\z
\z

The \isi{indefinite} singular \isi{direct object} in \REF{12-wr-ex:11b} as well as the \isi{definite} singular \isi{direct object} in \REF{12-wr-ex:11c} displays the \isi{accusative case} marker \textit{-m}. Due to this marker the singular objects of \ili{Tundra Nenets} uniformly differ from the corresponding syntactic subjects, which are not case marked at all, such as \textit{n’e} ‘woman’ in \REF{12-wr-ex:11a}.

Leaving aside the dual object forms, which do not exhibit any specific case morpheme (\citealt[538]{Salminen1998Nenets}; \citealt[57--58]{Nikolaeva2014Grammar}), the uniform object \isi{case marking} on \ili{Tundra Nenets} object nouns in the plural number is simply due to the regular suffixation of the accusative plural marker \textit{-j}. Nowadays \textit{-j} has undergone a process of de-morphologization. As a result, the recent \ili{Tundra Nenets} accusative plural objects are subject to a stem alternation \citep[238]{Mikola1988Geschichte}. Examples are given in \REF{12-wr-ex:12}, where \REF{12-wr-ex:12a} displays the \isi{indefinite} plural object noun \textit{t}\textit{\=\i} ‘reindeers’ and \REF{12-wr-ex:12b} the \isi{definite} plural object \textit{p’ib’i} ‘boots’. Both of them have undergone a vowel change. 

\ea \label{12-wr-ex:12} 
\langinfo{Tundra Nenets}{Northern Samoyedic}{\citealt[472]{Nikolaeva2014Grammar}, NOS. tesjada nisjami.037}\\

\ea \label{12-wr-ex:12a} 
\gll Tǝd°xǝw°ʔ yur° m’an° \textbf{tī }  nikelŋa.\\
 now(\textsc{aff}) hundred about reindeer\textsc{(pl+acc)} set.apart\textsc{(3sg.sc)}\\
\glt  ‘It split up about a hundred reindeer (from the herd).’ 

\ex\label{12-wr-ex:12b} 
\gll P’i sawo jern’a \textbf{p’ib’i}  s’era-dm, wen’eko-dar’em p’in n’alkara-dm.\\
 night good in.the.middle.of boot\textsc{(pl+acc)} put.on-\textsc{1sg} dog-\textsc{equ} out slink\textsc{-1sg}\\
\glt ‘In the middle of the night I put on my boots and slipped out of the tent like a dog.’
\z
\z

The latter sentence as well as \REF{12-wr-ex:11b} shows that \isi{definiteness} is not a sufficient condition for the suffixation of possessive markers in \ili{Tundra Nenets}. In \REF{12-wr-ex:11c} \textit{n’um} ‘child’ is \isi{definite} not only because of its thematic status in this part of the narration but also because of its close affiliation to \textit{Wel’i}, who is one of the protagonists of the story. The \isi{definiteness} of \textit{p’ib’i} ‘boots’ in \REF{12-wr-ex:12b} is due to its immediate associative relation to the first-person narrator. Nevertheless, neither \textit{n’um} nor \textit{p’ib’i} displays any possessive suffix. 
This is because the \ili{Tundra Nenets} nominal possessor agreement markers predominantly specify possessivity relations between possessum nouns and possessors. They do not function as object \isi{definiteness} markers and let alone as differential object markers. 

Plural object nouns displaying a possessor agreement marker are completely homonymous with the corresponding nominative possessive forms \citep[59]{Nikolaeva2014Grammar}. Since possessum subjects formerly also exhibited the suffix \textit{-j} as a connective morpheme, they feature the same alternation as the plural accusative forms. This is shown in \REF{12-wr-ex:13a} and \REF{12-wr-ex:13b}. The nominal stem \textit{te} ‘reindeer’ has undergone vowel change owing to the former suffixation and subsequent de-morphologiza\-tion of \textit{-j} in its accusative and in its nominative form. It cannot unambiguously be identified as subject or as object on the morphological level. 

\protectedex{
\ea \label{12-wr-ex:13}
\langinfo{Tundra Nenets}{Northern Samoyedic}{NOS. tet weli teta.020, 022}\\
\ea \label{12-wr-ex:13a}
\gll  Tʼet jonarʔ \textbf{tí-da} ŋob-t mandalʼa-dʔ.\\
 four thousand reindeer(\textsc{nom+pl)-3sg(poss)} one-\textsc{dat} assemble-\textsc{3pl.rc}\\
\glt  ‘His four thousand reindeers assembled in one group.’

\ex \label{12-wr-ex:13b}
\gll Tʼiki \textbf{tí-da} jarka, podʼer-ja-da.\\
 that reindeer(\textsc{acc+pl)-3sg(poss)} catch(\textsc{3sg.subj)} harness-\textsc{pl.o-3sg.oc}\\
\glt  ‘He caught and harnessed these reindeers.’
\z
\z
}

The non-possessive plural subject forms, as illustrated in \REF{12-wr-ex:14}, however, differ from the corresponding non-possessed objects in that they are provided with the plural suffix \textit{-ʔ}.

\ea \label{12-wr-ex:14}%14
\langinfo{Tundra Nenets}{Northern Samoyedic}{NOS. tet weli teta.094, 141}\\
\ea  \label{12-wr-ex:14a}
\gll … \textbf{n’enaca-ʔ} jab’el-mi-d\\
{ } man-\textsc{pl(nom)} make.drunk-\textsc{ptcp.pass-3pl.rc}\\
\glt  ‘… the people get drunk.’

\ex \label{12-wr-ex:14b}
\gll  … \textbf{Welʼi teta-ʔ} jamdaj-dʔ.\\
{ } Weli.land.owner-\textsc{pl(nom)} leave-\textsc{3pl.rc}\\
\glt ‘… the Weli-farmers left.’
\z
\z

\hspace*{-0.7pt}Thus, \ili{Tundra Nenets} employs DOM neither on singular nor on plural accusative nouns. It exhibits uniform structural \isi{case marking} instead. Exactly this is outlined in \tabref{12-wr-tab:5}.

\begin{table}
\begin{tabularx}{\textwidth}{X ll X ll} 
\lsptoprule
& \multicolumn{2}{c}{ singular} & & \multicolumn{2}{c}{ plural}\\
\midrule
& \isi{definite} & \isi{indefinite} & & \isi{definite} & \isi{indefinite}\\
nominative & - & - & & \textit{-Ɂ} & \textit{-Ɂ}\\
accusative & \textit{-m} & \textsc{-}\textit{m} & & \textsc{(sa)} & \textsc{(sa)}\\
\lspbottomrule
\end{tabularx}
\caption{Structural case markers on nouns in Tundra Nenets \citep[61]{Nikolaeva2014Grammar}}\label{12-wr-tab:5}
\end{table}


\subsection{Suppletion in the paradigm of personal pronouns}\label{12-wr-sec:4-2}

In contrast to the \ili{Nganasan} paradigm of personal pronouns, the \ili{Tundra Nenets} set of personal pronouns morphologically differentiates between subject and object personal pronouns by means of suppletion. As \citet[14--15]{Hajdu1988Samojedischen} points out, this is due to the \isi{grammaticalization} of the \ili{Uralic} lexeme \textit{ś}\textit{i}\textit{ʔ} ‘shape’. Owing to semantic bleaching \textit{ś}\textit{i}\textit{ʔ} has become a pronominal stem that currently represents the basis of the accusative and genitive personal pronouns. The individual person and number specifications of these forms are indicated by accusative and genitive possessor agreement suffixes (\cf \tabref{12-wr-tab:6}).\footnote{See footnote 9.} 

\begin{table}
\begin{tabularx}{\textwidth}{XXXX} 
\lsptoprule
& nominative & accusative & genitive\\
\midrule
\textsc{1sg} & \textit{mań} & \textit{śiʔm‘i} & \textit{śiʔn}\\
\textsc{2sg} & \textit{pidar} & \textit{śit} & \textit{śit°}\\
\textsc{3sg} & \textit{pida} & \textit{śita} & \textit{śita}\\
\textsc{1dual} & \textit{mańiʔ} & \textit{śid°n‘iʔ} & \textit{śid°n‘iʔ}\\
\textsc{2dual} & \textit{pidaŕiʔ} & \textit{śid°d‘iʔ} & \textit{śid°t‘iʔ}\\
\textsc{3dual} & \textit{pid‘iʔ} & \textit{śid°d‘iʔ} & \textit{śid°t‘iʔ}\\
\textsc{1pl} & \textit{mańaʔ} & \textit{śid°naʔ} & \textit{śid°ʔnaʔ}\\
\textsc{2pl} & \textit{pidaraʔ} & \textit{śid°daʔ} & \textit{śid°taʔ}\\
\textsc{3pl} & \textit{pidoʔ} & \textit{śid°doʔ} & \textit{śid°toʔ}\\
\lspbottomrule
\end{tabularx}
\caption{Structural case paradigm of the Tundra Nenets 
personal pronouns (\citealt[14--15]{Hajdu1988Samojedischen}; \citealt{Nikolaeva2014Grammar})}\label{12-wr-tab:6}
\end{table}


Moreover there is suppletion for person in the nominative array of the \ili{Tundra Nenets} system of personal pronouns. The first person forms exhibit the stem \textit{man}, the second and third person forms, however, the stem \textit{pi.} As hypothesized by \citet{Castren1845Grammatik}, \citet{Lehtisalo1939Refleksiivipronominista}, \citet{Hajdu1953Samojeden} and \citet{Siegl2009Personal} \textit{pi} does not descend from the Proto-\ili{Uralic} or Proto-\ili{Samoyedic} pronoun system. Whereas \citet[342]{Castren1845Grammatik} assumed that the stem of the second and third person pronouns is originally \ili{Turkish}, \citet{Hajdu1953Samojeden} proposes a contact-induced transfer from \ili{Ket}. \citet[120--121]{Siegl2009Personal} finally supports \citegen{Lehtisalo1939Refleksiivipronominista} hypothesis. He argues that the \ili{Tundra Nenets} second and third person subject personal pronouns result from the \isi{grammaticalization} of the \ili{Samoyedic} lexeme \textit{pixid} ‘body’. 

Regardless of which of these accounts proves right, the \ili{Tundra Nenets} set of personal pronouns has obviously undergone diachronic processes that are not evidenced within the corresponding \ili{Nganasan} system. Because of the exclusively Proto-\ili{Samoyedic}/\ili{Uralic} origin of its pronominal items, the latter is often conceived of as the most archaic pronominal system of the Northern \ili{Samoyedic} languages \citep[120]{Siegl2009Personal}. Contrary to \ili{Nganasan}, \ili{Tundra Nenets} therefore behaves in quite an ordinary way with respect to the morphological realization of its pronominal subjects and objects. Owing to the dimensional progression described above, its subject personal pronouns are realized as overt free pronominal items only if they are used for emphasis \citep[540]{Salminen1998Nenets} and object pronouns are always overt and free. This applies to the SAP object pronouns. Their third person forms are different. As will be shown in the following section, they are neither canonical free pronouns nor incorporated objects. 


\subsection{Object topic marking on finite verbs}\label{12-wr-sec:4-3}

The agreement markers of the \ili{Tundra Nenets} \isi{objective conjugation} listed in  \tabref{12-wr-tab:7}\footnote{See footnote 9.} do not simply incorporate the \isi{direct object} of a clause. Although they exhibit some essential properties of anaphoric third person objects, they belong to ambiguous verbal agreement markers in some sense. 

\begin{table}
\begin{tabularx}{\textwidth}{X X X@{\,}X@{\,}X X} 
\lsptoprule
& subjective & & objective & & reflexive\\
& & singular & dual & plural &\\
\midrule
\textsc{1sg} & \textit{-(d°)m}\textit{ʔ} & \textit{-w°} & \textit{-xǝyu-n°} & \textit{-yǝ-n°} & \textit{-w°ʔ}\\
\textsc{2sg} & \textit{-n°} & \textit{-r°} & \textit{-xǝyu-d°} & \textit{-yǝ-d°} & \textit{-n°}\\
\textsc{3sg} & \textit{${\varnothing}$} & \textit{-da} & \textit{-xǝyu-da} & \textit{-y°-da} & \textit{-ʔ}\\
\textsc{1dual} & \textit{-ńi}\textit{ʔ} & \textit{-ḿi}\textit{ʔ} & \textit{-xǝyu-}\textit{ńi}\textit{ʔ} & \textit{-y°-}\textit{ńi}\textit{ʔ} & \textit{-ńi}\textit{ʔ}\\
\textsc{2dual} & \textit{-d‘i}\textit{ʔ} & \textit{-ŕi}\textit{ʔ} & \textit{-xǝyu-}\textit{d‘i}\textit{ʔ} & \textit{-y°-}\textit{d‘i}\textit{ʔ} & \textit{-d‘i}\textit{ʔ}\\
\textsc{3dual} & \textit{-x(V°)}\textit{ʔ} & \textit{-d‘i}\textit{ʔ} & \textit{-xǝyu-}\textit{d‘i}\textit{ʔ} & \textit{-y°-}\textit{d‘i}\textit{ʔ} & \textit{-x(V°)}\textit{ʔ}\\
\textsc{1pl} & \textit{-wa}\textit{ʔ} & \textit{-wa}\textit{ʔ} & \textit{-xǝyu-}\textit{na}\textit{ʔ} & \textit{-y°-}\textit{na}\textit{ʔ} & \textit{-naʔ}\\
\textsc{2pl} & \textit{-da}\textit{ʔ} & \textit{-da}\textit{ʔ} & \textit{-xǝyu-}\textit{da}\textit{ʔ} & \textit{-y°}\textit{-da}\textit{ʔ} & \textit{-daʔ}\\
\textsc{3pl} & \textit{-ʔ} & \textit{-do}\textit{ʔ} & \textit{-xǝyu}\textit{-do}\textit{ʔ} & \textit{-y°}\textit{-do}\textit{ʔ} & \textit{-d°ʔ}\\
\lspbottomrule
\end{tabularx}
\caption{Verbal suffixes of the subjective, objective and reflexive conjugation in Nenets  (\citealt[16--17]{Hajdu1988Samojedischen}; \citealt[78--80]{Nikolaeva2014Grammar})} \label{12-wr-tab:7}
\end{table}

They are not completely incompatible with free clause-mate direct objects. However, due to their residual pronominal features they impose restrictive requirements on such complements. Above all, their third person specification excludes the insertion of SAP direct objects. As shown below, first \REF{12-wr-ex:15a} and second \REF{12-wr-ex:15b} person objects always precede a finite verb inflected in the \isi{subjective conjugation}, the agreement suffixes of which are listed in the first column of \tabref{12-wr-tab:7}.

\ea \label{12-wr-ex:15}%15
\langinfo{Tundra Nenets}{Northern Samoyedic}{NOS. tesjada nisjami.060, \citealt[447]{Nikolaeva2014Grammar}}\\
\ea \label{12-wr-ex:15a}
\gll Tʼiki pu-d \textbf{s’imʼi} \textbf{ŋawla.}\\
 that behind-\textsc{abl} me(\textsc{acc)} feed\textsc{(3sg.sc)}\\
\glt  ‘After that she gave me some food.’

\ex  \label{12-wr-ex:15b}
\gll Xumpa°nc’iʔ \textbf{s’it} \textbf{ŋædara-dəm-c‘°}.\\
 in.vain you\textsc{(acc)} send-\textsc{1sg.sc-pst}\\
\glt ‘In vain I let you go.’
\z
\z

Moreover, their extant characteristics of \isi{definiteness} cause a feature conflict with \isi{indefinite} objects. Accordingly, as illustrated in the following examples, pronominal \REF{12-wr-ex:16a} as well as non-pronominal \REF{12-wr-ex:16b} \isi{indefinite} objects obligatorily co-occur with subjective verb forms.

\ea\label{12-wr-ex:16}
\langinfo{Tundra Nenets}{Northern Samoyedic}{\citealt[436]{Nikolaeva2014Grammar}, NOS. tesjada nisjami.076}\\
\ea\label{12-wr-ex:16a}
\gll  Yebtow°ʔ, \textbf{ŋəmke-m} \textbf{mǝneʔŋa-ney°ʔ}?\\
 darling(\textsc{foc)} what-\textsc{acc} see-\textsc{2sg.sc.foc}\\
\glt ‘Darling, what can you see?’

\ex\label{12-wr-ex:16b}
\gll  Jaxa xara tʼa-xana \textbf{ŋob} \textbf{mʼadʼiko-m} \textbf{xo-dmʔ}.\\
 river curve there-\textsc{loc} one small.tent\textsc{-acc} find-\textsc{1sg.sc}\\
\glt  ‘After the bend of the river I found a small tent.’
\z
\z

The combination of both their third person specification and the \isi{definiteness} limitation, finally blocks the appearance of free \isi{definite} third person pronouns due to redundancy. 
This is why finite verbs with objective suffixes identify the referents of unmarked non-SAP object personal pronouns exclusively by themselves, \cf \REF{12-wr-ex:17a} and \REF{12-wr-ex:17b}. 

\ea\label{12-wr-ex:17}
\langinfo{Tundra Nenets}{Northern Samoyedic}{NOS. tet weli teta.066, 035}\\

\ea\label{12-wr-ex:17a}
\gll Xuc'erʔ \textbf{miŋku-da}?\\
 how marry-\textsc{3sg.oc}\\
\glt  ‘How could he marry her?’\\

%\todo[inline]{What does CO stand for?}

\ex\label{12-wr-ex:17b}
\gll  Tad \textbf{maneŋa-da}.\\
 then behold-\textsc{3sg.oc}\\
\glt ‘Then he realized it.’
\z
\z

Definite free-standing accusative third person pronouns are allowed to appear as soon as they are emphasized (\citealt[386--389]{Nikolaeva2014Grammar}) or belong to the non-determinative demonstrative pronouns. Like the \ili{Nganasan} free \isi{definite} pronominal objects they usually complement a verb inflected in the \isi{subjective conjugation}. During her colloquial elicitations \citet[201--210]{Nikolaeva2014Grammar} recorded a clause like \REF{12-wr-ex:18a}, where the free third person singular object personal pronoun \textit{s’ita} ‘him’ receives contrastive stress.\footnote{\citet[203]{Nikolaeva2014Grammar} points out that some speakers of the Western Nenets dialect group sometimes allow the co-occurrence of free third person object personal pronouns and objective verb forms.} The narrative texts of the \ili{Tundra Nenets} data base also contain clauses like \REF{12-wr-ex:18b} the pronominal object of which is a demonstrative pronoun bearing a possessor agreement affix. 

\ea \label{12-wr-ex:18}
\langinfo{Tundra Nenets}{Northern Samoyedic}{\citealt[203, 439]{Nikolaeva2014Grammar}}\\
\ea\label{12-wr-ex:18a}
\gll  N’is’a-da \textbf{s’ita} \textbf{ladə}.\\
 father\textsc{-3sg(poss)} him-(\textsc{acc)} hit(\textsc{3sg.sc)}\\
\glt  ‘His father hit him.’ 

\ex \label{12-wr-ex:18b}
\gll \textbf{T’ika-xeyu-da} \textbf{pod’erŋa}.\\
 this-\textsc{acc.du-3sg(poss)} harness(\textsc{3sg.sc)}\\
\glt  ‘He harnessed those two.’
\z
\z

The only stressed object pronouns that optionally take an objective verb form are reflexive pronominal expressions with the stem \textit{pixdə} \citep[203]{Nikolaeva2014Grammar}, \cf \REF{12-wr-ex:19a} and \REF{12-wr-ex:19b}. This extraordinary facultative co-occurrence may be due some non-functional residue that  \textit{pixd} `body' still bears as a lexical category.

\ea \label{12-wr-ex:19} 
\langinfo{Tundra Nenets}{Northern Samoyedic}{\citealt[203]{Nikolaeva2014Grammar}}\\

\ea\label{12-wr-ex:19a} 
\gll \textbf{pix°də-m’i} \textbf{lad°ə-d°m}.\\
\textsc{refl-1sg} hit-1\textsc{sg.sc}\\
\glt  ‘I hit myself.’

\ex\label{12-wr-ex:19b} 
\gll \textbf{pix°də-m’i} \textbf{lad°ə-w°.}\\
\textsc{refl-1sg} hit-1\textsc{sg.oc}\\
\glt  ‘I hit myself.’
\z
\z

This at least approximately conforms to the fact that the overwhelming majority of the free direct objects that are accompanied by a verb inflected in the \isi{objective conjugation} in \ili{Tundra Nenets} are non-pronominal anyhow \citep[122]{Kortvely2005Verb}. If, however, a non-pronominal complement appears in a \ili{Tundra Nenets} clause headed by an objective verb form, it is \isi{definite} and refers to an individuated and highly topical entity \citep[125--139]{Dalrympleetal2011Objects}. 
On the morphosyntactic level this is reflected by the suffixation of an appropriate possessor agreement morpheme on the one hand and on the other hand by its appearance in the left periphery or second position of the clause.
%\todo[inline]{the left AREA of the clause sounds strange. Could you please paraphrase?}
Usually, such non-pronominal complements immediately follow the syntactic subject like in \REF{12-wr-ex:20d} or even appear  sentence initially. 
The latter is illustrated in \REF{12-wr-ex:20b} and \REF{12-wr-ex:21b}. 

\ea \label{12-wr-ex:20}
\langinfo{Tundra Nenets}{Northern Samoyedic}{NOS. tesjada nisjami.003, 006, 009, 086}\\

\ea \label{12-wr-ex:20a}
\gll N'is'a-m'i tan’a n’eb’a-m’i tan’a n’ud’a papa-ko-m’i tan’a.\\
father-\textsc{1sg(poss)} exist(\textsc{3sg.sc)} mother-\textsc{1sg(poss)} exist(\textsc{3sg.sc)} young brother-\textsc{dim-1sg(poss)} exist(\textsc{3sg.sc)}\\
\glt ‘There is my father, my mother and my little brother.’

\ex  \label{12-wr-ex:20b}
\gll \textbf{N'is'a-m'i} Tes’ada-ŋæ \textbf{pær-c’eti-da}.\\
father-\textsc{acc+1sg(poss)} Tesjada-\textsc{ess} call-\textsc{hab-3sg.oc}\\
\glt ‘My father is called Tesjada.’

\ex \label{12-wr-ex:20c}
\gll Ŋobŋ-kuna n’is’a-m’i n’eb’a-xa-n’i ma:\\
one-\textsc{loc} father-\textsc{1sg(poss)} mother-\textsc{dat-1sg(poss)} say\textsc{(3sg.sc)}\\
\glt ‘Once, my father told my mother:’

\ex \label{12-wr-ex:20d}
\gll N'is'a-m'i \textbf{jil'e-m'a-m-ta} s'eroku-ta s'er \textbf{wad'eŋa-da}.\\
father-\textsc{1sg(poss)} live-\textsc{nmlz-acc-3sg(poss)} separate-\textsc{3sg(poss)} affair tell\textsc{-3sg.oc}\\
\glt ‘My father told me what he lived through in detail’
\z
\z

\ea \label{12-wr-ex:21} 
\langinfo{Tundra Nenets}{Northern Samoyedic}{\citealt[452–453]{Nikolaeva2014Grammar}}\\
\ea\label{12-wr-ex:21a} 

\gll N’ud’a Way° xada-wi° n’e°ka-xǝnta tæwi°-ʔ.\\
little Waya kill-\textsc{pstpf.ptcp} elder.brother-\textsc{dat.3sg(poss)} arrive-\textsc{3sg.rc}\\
\glt ‘Younger Waya reached the place where his murdered brother lay.’ %\citep[452]{Nikolaeva2014Grammar}

\ex\label{12-wr-ex:21b} 

\gll \textbf{Xalm’era-m-toʔ} s'id’a xoba-ʔ n’iʔ \textbf{peŋa-doʔ}.\\
dead.body-\textsc{acc-3pl(poss)} two skin\textsc{-gen} onto put\textsc{-3pl.oc}\\
\glt ‘They put the dead body (of their brother) onto two skins.’ %\citep[453]{Nikolaeva2014Grammar}

\ex\label{12-wr-ex:21c} 
\gll Lobeku-ʔ n’eb’a ma: “Ŋemc’i-da temna səwa-ʔ.\\
Lobeku-\textsc{gen} mother say\textsc{(3sg.sc)} flesh.\textsc{3pl-3sg(poss)} still good-\textsc{3pl}\\
\glt ‘Lobeku’s mother said: “His muscles are still good.”’ %\citep[453]{Nikolaeva2014Grammar}

\ex\label{12-wr-ex:21d} 
\gll Xəd°riʔ yil’e-bt’e-° xorta-nakew°.\\
of.course live-\textsc{caus-mod} try-\textsc{prob.1sg.oc}\\
\glt ‘I might try and revive him.”’ %\citep[453]{Nikolaeva2014Grammar}
\z
\z

The boldfaced \isi{direct object} nouns in \REF{12-wr-ex:20b} and \REF{12-wr-ex:21b} are separated from the sentence-final objective verb form by at least one constituent. 
In \REF{12-wr-ex:20b} \textit{n’is’am’i} ‘my father’ converts its referent introduced before (\cf \ref{12-wr-ex:20a}) into the main \isi{discourse topic} and designates with that the protagonist (\cf \REF{12-wr-ex:20d}) at the very beginning of the story. 
In \REF{12-wr-ex:21b} \textit{xalm’eramtoʔ} ‘dead body’, which refers to Waya’s murdered brother and belongs to the old information (\cf \REF{12-wr-ex:21a}), announces the main topic of the following direct speech (\cf \REF{12-wr-ex:21c}, \REF{12-wr-ex:21d}). 

Thus, the relation between accusative complements and objective verb forms in \ili{Tundra Nenets} is reminiscent of the distribution of objective affixes in \ili{Nganasan}. The \ili{Tundra Nenets} objective markers indicate that the \isi{direct object} phrase they co-occur with is or becomes the main topic of the following discourse. However, in \ili{Tundra Nenets} left-dislocation into any pre-sentential position is no longer an indispensable operation that non-pronominal objects must undergo in order to be compatible with an objective verb form (\cf \REF{12-wr-ex:21d}). This implies that the objective affixes on \ili{Tundra Nenets} finite verbs have acquired some relevant properties of grammatical agreement markers. The development of such functional features can presumably be described as a \isi{grammaticalization} process that started with the loss of stylistic force which left-dislocated constituents originally exerted. As a consequence of this loss the formerly left dislocated constituents were reanalyzed as clause-internal topic constituents and the formerly bound resumptive clitics as agreement markers attaching to the respective verb under certain conditions. Since only non-pronominal constituents underwent topicalization by clitic left-dislocation the third person specification of the former resumptive elements has been preserved. And since, moreover, the conditions under which these elements appeared in the presence of object constituents has always been defined by the pragmatic status of the latter, the newly emerged agreement markers unfolded information structuring functions of topic markers by the process of pragmaticalization (\cf \citealt{Diewald2011Pragmaticalization}). 

It is conceivable that exactly this diachronic process is responsible for the mechanism of DOM that nowadays holds in \ili{Tundra Nenets}. Its objective agreement suffixes on the finite verb indicate that the non-SAP object deviates from the prototypical \isi{patient argument} in that it is \isi{definite} and establishes the actual \isi{discourse topic}. Thus, \ili{Tundra Nenets} differentially marks object topics by means of differential object indexing (DOI). 


\section{Forest Enets: Differential object marking on finite verbs}\label{12-wr-sec:5}

The Enets language area is located in the lower Yenisei region \citep[457]{Janhunen1998Samoyedic}, which extends to the Kara Sea in the North. In the west it borders on the Nenets and in the east on the \ili{Nganasan} language area. Its southernmost \ili{Samoyedic} neighbor is the \ili{Selkup} region. There are two Enets dialects: Forest (Bai) Enets and Tundra (Maddu) Enets, the predominant of which, \ili{Forest Enets}, is considered in the following. 

\ili{Forest Enets} is in a much more moribund state than \ili{Nganasan} and Nenets \citep[30--57]{Siegl2013Materials}. It features a number of morphosyntactic characteristics that have to be seen as an advancement of the diachronic processes that are attested for the other Northern \ili{Samoyedic} languages. While the distinct morphology of structural \isi{case marking} on its nouns is progressively eroding, as shown in \sectref{12-wr-sec:5-1}, the suffixes of the \isi{objective conjugation} gain more and more weight in the relational assignment of arguments, which is elucidated in \sectref{12-wr-sec:5-3}. The \ili{Forest Enets} personal pronouns are not affected by the loss of specific morphology. On the contrary, similarly to the \ili{Tundra Nenets} personal pronouns, they have established a structural case distinction by the adoption of supplementary forms. This is illustrated in \sectref{12-wr-sec:5-2}.

\subsection{The erosion of structural case marking on nouns}\label{12-wr-sec:5-1}

In \ili{Forest Enets} the \ili{Uralic} nominal accusative marker \textit{*-m} has vanished almost entirely \citep[13–14]{Kunnap1999Enets}. With the only exception of a few nouns that belong to a subgroup of the second \isi{inflectional class} and undergo stem alternation in the accusative paradigm \citep[121--124]{Siegl2013Materials},  singular direct objects morphologically conform to the corresponding singular subject nouns in that they are not case marked at all. As shown in \REF{12-wr-ex:22}, \textit{te} ‘reindeer’ gets along without any specific case marker regardless of whether it is selected as syntactic subject (\cf \REF{12-wr-ex:22a}) or object (\cf \REF{12-wr-ex:22b}). 

\ea  \label{12-wr-ex:22} 
\langinfo{Forest Enets}{Northern Samoyedic}{NOS. text 39.015, text 39.030}\\

\ea  \label{12-wr-ex:22a} 
\gll \textbf{Te } nebr-ið…\\
 reindeer run.away-\textsc{3sg.rc}\\
\glt  ‘The reindeer runs away.’

\ex  \label{12-wr-ex:22b} 
\gll   … to ar \textbf{te} kaða-ð\\
{} such size reindeer(\textsc{acc)} kill\textsc{-1sg.sc}\\
\glt  ‘I have killed such a big reindeer.’
\z
\z

This holds true at least for all non-possessive forms. Their possessive counterparts still bear traces of the suffix \textit{*-m} \citep[242]{Mikola1988Geschichte}. Owing to its coalescence with  the respective adjoining possessive affixes, they exhibit portmanteau morphs encoding case and possessor agreement that are -- at least in the case of a second or third person possessor specification -- morphologically distinct from the respective possessor agreement morphemes attached to subject nouns. This is shown in \REF{12-wr-ex:23} where the accusative third person dual possessor agreement suffix of \REF{12-wr-ex:23b} deviates from its nominative counterpart in \REF{12-wr-ex:23a} due to its previous fusion with \textit{*-m}. 

\ea %23
\label{12-wr-ex:23}
\langinfo{Forest Enets}{Northern Samoyedic}{\citealt[479--480]{Siegl2013Materials}}\\
\ea\label{12-wr-ex:23a}
\gll  Kiuða šer to-sau-jet \textbf{sama-ðiʔ}.\\
morning(\textsc{gen)} before come-\textsc{prob+pst(3sg.sc)-emph} beast-\textsc{3du(poss)}\\
\glt ‘But in the morning their bear apparently came.’%AW changed the translation to match the cited source \citealt[480]{Siegl2013Materials}

\ex \label{12-wr-ex:23b}
\gll  Oti-ðiʔ oti-ðiʔ \textbf{bogl’a-diʔ}.\\
wait-\textsc{3du.oc} wait-\textsc{3du.oc} bear-\textsc{acc+3du(poss)}\\
\glt  ‘They waited for their bear.’
\z
\z

Like the dual subject and object forms the plural non-possessum subject (\cf 24a) and object (\cf 24b) forms are subject to a natural syncretism. This is due to the fact that after the de-morphologization and \isi{definite} loss of the plural marker \textit{*-j}, the former subject plural marker \textit{-ʔ} has entered the paradigm of plural non-possessive objects \citep[238]{Mikola1988Geschichte}. 

\ea\label{12-wr-ex:24}
\langinfo{Forest Enets}{Northern Samoyedic}{\citealt[477, 479]{Siegl2013Materials}}\\
\ea \label{12-wr-ex:24a}
\gll  čan-da mi-n \textbf{ kari-ʔ} tonä-bi-č\\
 tub-\textsc{gen+3sg(poss)} in-\textsc{loc} fish-\textsc{pl} exist\textsc{-prf-pst+3pl.sc}\\
\glt  ‘(and) in a tub there were fishes’

\ex \label{12-wr-ex:24b}
\gll  Salba ne-on \textbf{kari-ʔ} noo-bi-š.\\
 ice\textsc{(gen)} on-\textsc{prol} fish-\textsc{pl(acc)} take-\textsc{prf-pst+3sg.sc}\\
\glt ‘Along the ice, the bear took fishes along.’
\z
\z

Since in Northern \ili{Samoyedic} the possessor agreement affixes on plural nouns do not show any distinction with regard to the subject or object function of the corresponding arguments, the paradigm of the possessive plural nouns also lacks any nominative-accusative distinction. That is why the object \textit{kasiðu} ‘men’ in \REF{12-wr-ex:25a} exactly matches the corresponding subject form in \REF{12-wr-ex:25b}. 

\ea\label{12-wr-ex:25}
\langinfo{Forest Enets}{Northern Samoyedic}{NOS. text.39.043, Languedoc. otpusk.029}\\
\ea\label{12-wr-ex:25a}
\gll \textbf{Kasi-ðu} d’oxara-ʔ\\
 man-\textsc{pl+3pl(poss)} not.know-\textsc{3pl.sc}\\
\glt  ‘The men do not know each other.’

\ex\label{12-wr-ex:25b}
\gll Kutui-ðu \textbf{kasi-ðu }  paroxodo-xoðo karaa-t’i…\\
 some\textsc{(acc)}-\textsc{pl+3pl(poss)} man\textsc{(acc)}-\textsc{pl+3pl(poss)} steamer-\textsc{abl} take.along\textsc{-3pl.sc+pst}\\
\glt ‘They took along some of their fellows with the steamer.’
\z
\z

Thus, with the only exception of a number of non-possessum singular nouns belonging to the second declensional class and of all singular accusative nouns displaying a second or third person possessor agreement affix in the singular number, objects are not distinguishable from subjects on the basis of their inflectional morphology. Like \ili{Tundra Nenets}, \ili{Forest Enets} dispenses with DOM on nouns entirely. Neither \isi{definiteness} nor indefiniteness of direct objects is indicated by any special case marker or obligatorily associated with the presence or absence of any possessor agreement suffix. Exactly this is sketched in \tabref{12-wr-tab:8}.

\begin{table}
\begin{tabularx}{\textwidth}{Xcc Xcc} 
\lsptoprule
& \multicolumn{2}{c}{ singular} & & \multicolumn{2}{c}{ plural}\\
& \isi{definite} & \isi{indefinite} & & \isi{definite} & \isi{indefinite}\\
\midrule
nominative & - & - & & \textit{-Ɂ} & \textit{-Ɂ}\\
accusative & \textsc{(sa)} & \textsc{(sa)} & & \textit{-Ɂ} & \textit{-Ɂ}\\
\lspbottomrule
\end{tabularx}
\caption{Structural case/definiteness markers on nouns in Forest Enets}\label{12-wr-tab:8}
\end{table}


\subsection{Hybrid forms in the paradigm of personal pronouns}\label{12-wr-sec:5-2}

One thing that the \ili{Forest Enets} pronominal system has in common with the \ili{Tundra Nenets} pronominal system is that the introduction of the grammaticalized morpheme \textit{śiɁ} has resulted in the removal of the structural case syncretism from the paradigm of personal pronouns. 
However, it differs from the \ili{Tundra Nenets} system in that the new inflected forms of  \textit{śiɁ} do not always simply replace the original syncretic pronouns. 
Rather they form an optional part of complex pronouns that also consist of the respective unmarked singular, dual and plural personal pronouns \citep[20–22]{Kunnap1999Enets}. 
The corresponding paradigms of the structural cases are given in \tabref{12-wr-tab:9}\footnote{Note that \citet[186–187]{Siegl2013Materials} -- in contrast to \citet[20--21]{Kunnap1999Enets} and \citet[227–229]{Sorokina2010Eneckij} among others -- denies the existence of genitive personal pronouns in \ili{Forest Enets}.} the last two columns of which contain bipartite forms headed by a form of \textit{śiɁ}. 

\begin{table}
\begin{tabularx}{\textwidth}{XXXX} 
& nominative & accusative & genitive\\
\lsptoprule
\textsc{1sg} & \textit{mod' (mud‘)} & \textit{(mod') ši(j)Ɂ} & \textit{(mod') siń}\\
\textsc{2sg} & \textit{uu} & \textit{(uu) šit} & \textit{(\=u) sit}\\
\textsc{3sg} & \textit{bu} & \textit{(bu) šita} & \textit{(bu) sita}\\
\textsc{1dual} & \textit{mod'ińɁ} & \textit{(mod‘ińɁ) si}\textit{ð}\textit{ińʔ} & \textit{(modińɁ) si}\textit{ð}\textit{iń}\\
\textsc{2dual} & \textit{uudiɁ} & \textit{(uudiɁ) ši}\textit{ðð}\textit{iɁ} & \textit{(\=udiɁ) si}\textit{ð}\textit{tiɁ}\\
\textsc{3dual} & \textit{bud'iɁ} & \textit{(bud‘iɁ) šiðid‘i} & \textit{(budiɁ) si}\textit{ðð}\textit{i}\\
\textsc{1pl} & \textit{mod'naɁ} & \textit{(mod‘naɁ) ši}\textit{ð}\textit{naɁ} & \textit{(modinaɁ) si}\textit{ð}\textit{naɁ}\\
\textsc{2pl} & \textit{uudaɁ} & \textit{(uudaɁ) ši}\textit{ðð}\textit{aɁ} & \textit{(\=udaɁ) si}\textit{ð}\textit{taɁ}\\
\textsc{3pl} & \textit{buduɁ} & \textit{(buduɁ) ši}\textit{dd}\textit{uɁ} & \textit{(buduɁ) si}\textit{ð}\textit{tuɁ}\\
\lspbottomrule
\end{tabularx} 

\caption{Structural case paradigm of the Forest Enets personal pronouns 
(\citealt[21]{Kunnap1999Enets}; \citealt[186–187]{Siegl2013Materials})}\label{12-wr-tab:9}
\end{table}

\citet[76]{Prokovjev1937Eneckij} was the first who noticed the divergence of a number of \ili{Forest Enets} personal pronouns from the corresponding genuine \ili{Uralic} and \ili{Samoyedic} pronominal items and their resemblance to personal pronouns used in the \ili{Yeniseian} languages. Nowadays Uralists by and large agree that their second and third person nominative singular forms have been directly borrowed from the \ili{Yeniseian} language \ili{Ket} (\citealt[456]{Terescenko1966Eneckij}; \citealt[ 119--121]{Siegl2009Personal}). Their dual and plural forms are, like the corresponding first person forms, provided with common \ili{Uralic} number markers \citep[124--127]{Siegl2009Personal}. Till this day they encode the person and number specification of the respective accusative forms as soon as they are not omitted. Consequently, with the exception of the second and third person singular and all first person and non-complex forms, the \ili{Forest Enets} personal pronouns are hybrid forms. They are composed of hereditary \ili{Uralic} and borrowed \ili{Ket} morphemes. Accordingly, through borrowing and \isi{grammaticalization} \ili{Forest Enets} has developed a suppletive paradigm of personal pronouns that, like the corresponding Nenets paradigm, features a morpheme-based distinction between the structural cases. 

In discourse situations the \ili{Forest Enets} subject pronouns are optionally omitted in case they are not emphasized \citep[37]{Kunnap1999Enets}. The corresponding object pronouns, however, are always overt with the partial exception of the third person forms. Like their \ili{Tundra Nenets} counterparts, these pronouns are no longer fully realized as clausal arguments by the agreement morphology of finite verbs inflected in the \isi{objective conjugation}. Although the \ili{Forest Enets} objective affixes still retain some essential properties of anaphoric third person objects, they have  already gone one step further on the developmental path to grammatical object agreement morphemes than the Nenets objective affixes. This is elucidated in the following section. 

\subsection{Object definiteness marking on finite verbs}
\label{12-wr-sec:5-3}

The agreement markers of the three \ili{Forest Enets} conjugation types are compiled in \tabref{12-wr-tab:10}.

\begin{table}
\begin{tabularx}{\textwidth}{X X lll X} 
\lsptoprule
& subjective & \multicolumn{3}{c}{ objective} & reflexive\\
& & singular & dual & plural &\\
\midrule 
\textsc{1sg} & \textit{-ð}\textit{ʔ} & \textit{-a, -u, -b} & \textit{-xu-n} & \textit{-i}\textit{-n} & \textit{-i}\textit{ -j}\textit{ʔ, -bʔ}\\
\textsc{2sg} & \textit{-d} & \textit{-r} & \textit{-xu-ð} & \textit{-i-ð} & \textit{-i-d‘}\\
\textsc{3sg} & \textit{${\varnothing}$} & \textit{-ða} & \textit{-xu-ða} & \textit{-i-da} & \textit{-i-ðʔ}\\
\textsc{1dual} & \textit{-j}\textit{ʔ, -bʔ} & \textit{-j}\textit{ʔ, bʔ} & \textit{-xu-}\textit{ń}\textit{ʔ} & \textit{-i-}\textit{ń}\textit{ʔ} & \textit{-i-}\textit{b}\textit{ʔ}\\
\textsc{2dual} & \textit{-ri}\textit{ʔ} & \textit{-ri}\textit{ʔ} & \textit{-xu-}\textit{ði}\textit{ʔ} & \textit{-i-}\textit{ði}\textit{ʔ} & \textit{-i-}\textit{ði}\textit{ʔ}\\
\textsc{3dual} & \textit{-xi}\textit{ʔ} & \textit{-ði}\textit{ʔ} & \textit{-xu-}\textit{ði}\textit{ʔ} & \textit{-i-}\textit{ði}\textit{ʔ} & \textit{-i-}\textit{xi}\textit{ʔ}\\
\textsc{1pl} & \textit{-a}\textit{ʔ,baʔ} & \textit{-a}\textit{ʔ, baʔ} & \textit{-xu-}\textit{na}\textit{ʔ} & \textit{-i-}\textit{na}\textit{ʔ} & \textit{-i-}\textit{na}\textit{ʔ}\\
\textsc{2pl} & \textit{-ra}\textit{ʔ} & \textit{-ra}\textit{ʔ} & \textit{-xu-}\textit{ða}\textit{ʔ} & \textit{-i-}\textit{ða}\textit{ʔ} & \textit{-i-}\textit{ða}\textit{ʔ}\\
\textsc{3pl} & \textit{-ʔ} & \textit{-ðu}\textit{ʔ} & \textit{-xu}\textit{-ðu}\textit{ʔ} & \textit{-i-}\textit{ðu}\textit{ʔ} & \textit{-i-}\textit{ð}\textit{ʔ}\\
\lspbottomrule
\end{tabularx}
\caption{Verbal suffixes of the subjective, objective and reflexive conjugation in Enets \citep[247--260]{Siegl2013Materials}}\label{12-wr-tab:10}
\end{table}


With respect to the choice between the subjective and the objective inflection in the presence of pronominal direct objects Enets slightly deviates from Nenets. Like in \ili{Tundra Nenets}, in \ili{Forest Enets} SAP object pronouns, for example, the second person singular accusative personal pronoun \textit{s’it} ‘you’ in \REF{12-wr-ex:26a}, as well as \isi{indefinite} third person pronouns, like the interrogative pronoun \textit{obu} ‘what’ in \REF{12-wr-ex:26b}, are accompanied by finite verbs inflected in the \isi{subjective conjugation}. 

\protectedex{
\ea \label{12-wr-ex:26}
\langinfo{Forest Enets}{Northern Samoyedic}{NOS. text 39.017, text 39.004}\\
\ea \label{12-wr-ex:26a}
\gll modʼ \textbf{sʼi -t} \textbf{kojta-da-ð}\\
I you\textsc{-acc.sg} set.up-\textsc{fut-1sg.sc}\\
\glt ‘I will trick you.’

\ex \label{12-wr-ex:26b}
\gll \textbf{obu} ɛke-n \textbf{ponʼi-ŋa-d}\\
 what this-\textsc{loc.adv} do-\textsc{freq}-\textsc{2sg.sc}\\
\glt  ‘What are you doing here?’
\z
\z
}

Likewise, non-pronominal objects that are \isi{indefinite} like \textit{ŋubai} ‘a mat’ in \REF{12-wr-ex:27a} and \textit{kobaʔ} ‘skins’ in \REF{12-wr-ex:27b} require a finite verb form of the subjective paradigm.

\ea \label{12-wr-ex:27}
\langinfo{Forest Enets}{Northern Samoyedic}{\citealt[47]{Siegl2013Materials}}\\
\ea \label{12-wr-ex:27a}
\gll Točgoð čiki kaði läxäči ne-on \textbf{ŋubai} \textbf{pu-da-ʔ}.\\
 then this fur(\textsc{gen)} twig\textsc{(acc)} on-\textsc{prol} mat(\textsc{acc)} lay\textsc{-fut-3pl.sc}\\
\glt  ‘Then they will lay a mat on the fur twigs.’

\ex\label{12-wr-ex:27b}
\gll Ŋubai ne-on ańʔ čiki mu \textbf{koba-ʔ} \textbf{läxta-da-ʔ}\\
 mat(\textsc{gen)} on-\textsc{prol} \textsc{foc} this so skin(\textsc{pl)} spread-\textsc{fut-3pl.sc}\\
\glt  ‘Over the mat, they will spread out skins.’
\z
\z

Also, like in \ili{Tundra Nenets}, finite verbs inflected in the \isi{objective conjugation} only co-occur with \isi{definite} third person objects. But in \ili{Forest Enets}, unlike in \ili{Tundra Nenets} free \isi{definite} third person object pronouns are not exempt from this. More precisely, if a  third person \isi{definite} pronoun, as for example any strong third person personal or any demonstrative pronoun, is inserted into a clause, the corresponding finite verb normally inflects in the \isi{objective conjugation}. This is illustrated in \REF{12-wr-ex:28a} and \REF{12-wr-ex:28b}. 

\ea\label{12-wr-ex:28}
\langinfo{Forest Enets}{Northern Samoyedic}{\citealt[252, 468]{Siegl2013Materials}}\\
\ea\label{12-wr-ex:28a}
\gll Mud’ \textbf{s’ita} soiða-n \textbf{täne-u.}\\
 I he(\textsc{acc)} good-\textsc{prol} know-\textsc{1sg.oc}\\
 \glt  ‘I know him well.’

\ex\label{12-wr-ex:28b}
\gll \textbf{Čiki-ru-}\textbf{$\delta $}\textbf{a} \textbf{oo-ma-ða.}\\
 this-\textsc{lim-3sg} eat-\textsc{res}-\textsc{3sg.oc}\\
\glt  ‘Only this it had eaten.’
\z
\z
Nevertheless, the objective affixes of the \ili{Forest Enets} verbal inflection are still able to represent anaphoric third person objects by the person features of their pronominal predecessors. Accordingly, they block the appearance of non-emphatic anaphoric third person personal pronouns for reasons of redundancy. 
Clauses, in which the third person \isi{definite} pronominal object is not independently realized as in \REF{12-wr-ex:29a} and \REF{12-wr-ex:29b}, are therefore much more frequent than clauses like \REF{12-wr-ex:28a}. 

\ea \label{12-wr-ex:29}
\langinfo{Forest Enets}{Northern Samoyedic}{NOS. text 01.009, \citealt[269]{Siegl2013Materials}}\\
\ea\label{12-wr-ex:29a}
\gll  Mod' nas'il \textbf{tuda-a-b-o-s'}.\\
 I not.easily recognize-\textsc{prs}-\textsc{1sg.oc-ep-pst}\\
\glt  ‘I hardly recognized him.’

\ex\label{12-wr-ex:29b}
\gll \textbf{Sirta-b-i-ða} bočka mi-ʔ …\\
 salt-\textsc{prf-obj.pl-3sg.oc} barrel\textsc{(gen)} in-\textsc{lat}\\
\glt ‘They salted them into a barrel.’%I adjust the translation according to the source
\z
\z

\ili{Forest Enets} furthermore differs from \ili{Tundra Nenets} in that the non-pronominal complements of objective predicates need not reside in the left area of the clause and do not even obligatorily refer to the \isi{discourse topic}. 

In most cases the referent of lexical direct objects that complement an objective verb form is \isi{definite} and at the same time topical insofar as it has been introduced in the preceding context. In \REF{12-wr-ex:30} for example the reindeer and the mouse are established as protagonists at the beginning of the story \REF{12-wr-ex:30a}. In its conclusive statement \REF{12-wr-ex:30b} the direct objects \textit{te} ‘reindeer’ and \textit{tobik} ‘mouse’ therefore belong to the old information. They are \isi{definite} and their referents are highly topical. That is why \textit{te} ‘reindeer’ and \textit{tobik} ‘mouse’ obligatorily co-occur with an objective verb form in \REF{12-wr-ex:30b}.

\ea \label{12-wr-ex:30}
\langinfo{Forest Enets}{Northern Samoyedic}{NOS. text 39.001, \citealt[269]{Siegl2013Materials}}\\
\ea\label{12-wr-ex:30a}
\gll dʼiri-bi ŋo-lʼu dʼa-xan \textbf{tobik} anʼ \textbf{te}\\
live\textsc{-narr(3sg.sc)} one-\textsc{lim} earth\textsc{-loc.sg} mouse and reindeer\\
\glt ‘There lived on the earth a mouse and a reindeer.’

\ex\label{12-wr-ex:30b}
\gll te \textbf{d'oxara-ða} \textbf{tobik}, tobik \textbf{d'oxara-ða} \textbf{te}\\
reindeer not.know\textsc{-3sg.oc} mouse(\textsc{acc)} mouse not.know-\textsc{3sg.oc} reindeer(\textsc{acc)}\\
\glt ‘The reindeer does not know the mouse and the mouse does not know the reindeer.’
\z
\z

However, the \isi{definiteness} of non-pronominal objects accompanied by a finite verb inflected in the \isi{objective conjugation} is not necessarily pragmatically motivated. Semantic \isi{definiteness} is a sufficient criterion for direct objects to become a complement of an objective verb form in \ili{Forest Enets}. \textit{D’urak baða} ‘Nenets language’ in \REF{12-wr-ex:31b} and \textit{nu} ‘door’ in \REF{12-wr-ex:32b}\footnote{\citet[490]{Siegl2013Materials} himself points out that the combination of a future and a past tense marker is semantically unexpected.}
 for example are part of the new information (\cf \REF{12-wr-ex:31a}, \REF{12-wr-ex:32a}). 

\ea \label{12-wr-ex:31}
\langinfo{Forest Enets}{Northern Samoyedic}{NOS. text 01.016, NOS. text 01.017}\\

\ea \label{12-wr-ex:31a}
\gll Mod’ onaj baða-an sujða-an d’uri-ŋa-ð.\\
 I true language-\textsc{prol.sg} good\textsc{-prol.sg} say\textsc{-prs-1sg.sc}\\
\glt  ‘I speak Enets well.’

\ex  \label{12-wr-ex:31b}
\gll \textbf{D'urak}  \textbf{baða} ŋubtoreɁ sujða-an \textbf{tɛnɛɛ-w}\\
 Nenets language(\textsc{acc)} also good\textsc{-prol.sg} know\textsc{-1sg.oc}\\
\glt  ‘I also speak Nenets well.’
\z
\z

\ea\label{12-wr-ex:32}
\langinfo{Forest Enets}{Northern Samoyedic}{\citealt[489–490]{Siegl2013Materials}}\\
\ea\label{12-wr-ex:32a}
\gll Mud’na okružkom aga bem äsi mäku-xuð-da mosa-xa-da kada-bi-ða.\\
 we(\textsc{pl)} party.committee(\textsc{gen)} big boss father(\textsc{acc)} house-\textsc{abl.sg-3sg(poss)} work-\textsc{lat.sg-3sg(poss)} take\textsc{-prf-3sg.oc}\\
\glt ‘An official from our party committee came to take father from his house to work.’

\ex\label{12-wr-ex:32b}
\gll Äsi-j pe-t käni-ta-š \textbf{nu} lokri \textbf{toru-ða}\\
 father-\textsc{1sg(poss)} street-\textsc{lat} go\textsc{-fut-pst(3sg.sc)} door suddenly close\textsc{-3sg.oc}\\

\glt ‘My father went out on the street and suddenly closed the door.’
\z
\z

Owing to the uniqueness of the referent in the case of \textit{d’urak baða} and due to the evident associative relation of the object referent in the case of \textit{nu} to an already implemented referent (here: the house of the father (\cf \ref{12-wr-ex:32a})) they are \isi{definite} as a result of the encyclopedic knowledge of the discourse participants. Their \isi{definiteness} is therefore semantically motivated and triggers agreement in the \isi{objective conjugation}, as can be observed in \REF{12-wr-ex:31} and \REF{12-wr-ex:32}. 

Hence, the \ili{Forest Enets} objective affixes, like the \ili{Tundra Nenets} objective affixes, indicate specific properties of selected object arguments via a grammatical agreement relation. 
The \ili{Forest Enets} verb takes an agreement suffix of the \isi{objective conjugation} if its third person \isi{direct object} deviates from the prototypical \isi{patient argument} in being \isi{definite}. 
Supported by its object number specification it establishes the basic syntactic function of the occurring nominal expressions, which, by and large, have lost their structural case morphology. Accordingly, the relation between the \ili{Tundra Nenets} and the \ili{Forest Enets} objective suffixes is characterized by an increase of syntactic obligatoriness and the \isi{grammaticalization} from pragmatic \isi{definiteness} to semantic \isi{definiteness} marking (\cf \citealt[57]{Lehmann1982Thoughts}; \citealt[39]{Himmelmann1997Deiktikon}). 
That is why \ili{Forest Enets} DOI does not merely reflect pragmatic characteristics of the selected third person objects like the Nenets objective agreement marking. Rather it also fulfills a discriminatory function in that it distinguishes between arguments and their roles. 

\section{Conclusion}\label{12-wr-sec:6}

It has been shown in this paper that in the Northern \ili{Samoyedic} languages \ili{Nganasan}, \ili{Tundra Nenets} and \ili{Forest Enets} the \isi{grammaticalization} of objective agreement markers on verbs goes hand in hand with the specific development of \isi{accusative case} and \isi{definiteness} markers on nouns. 

The north eastern language \ili{Nganasan} has brought forth a system of DOM that exclusively applies to nouns. 
This is due to various phonological processes that have affected \isi{accusative case} markers and to the \isi{grammaticalization} of possessor agreement affixes to \isi{definiteness} markers. 
The agreement markers on \ili{Nganasan} finite verbs do not yet serve as DOM in the proper sense. 
The objective affixes of them incorporate anaphoric third person object arguments. They only co-occur with free object constituents if they are bound by the latter in a typical clitic left-dislocation construction. 
In the north western language \ili{Tundra Nenets} DOM of nouns does not exist. 
Uniform accusative \isi{case marking} prevails instead and nominal possessor agreement markers predominantly specify possessivity relations between possessum nouns and possessors. 
However, the agreement morphemes of the \ili{Tundra Nenets} \isi{objective conjugation} have adopted functional features of object agreement markers that enable them to reflect the non-typical behavior of syntactic objects in information structuring. 
In this way, the inflectional system of the \ili{Tundra Nenets} finite verbs has acquired the function of DOI by a process of \isi{grammaticalization}. In \ili{Forest Enets}, the central Northern \ili{Samoyedic} language, the agreement morphemes of the \isi{objective conjugation} already exhibit evident features of full-fledged head-marking verb suffixes. They indicate the presence of a \isi{definite} third person \isi{direct object}. Since \ili{Forest Enets} differs from \ili{Tundra Nenets} in that the mere structural \isi{case marking} on its nouns is becoming extinct, the choice of the respective verbal agreement allomorph in \ili{Forest Enets} serves to distinguish between clausal arguments and their roles. 

Since the \ili{Uralic} SAP pronouns are neither immediately affected by the emergence and loss of nominal differential object markers nor involved in the \isi{grammaticalization} of the objective agreement suffixes on verbs, the Northern \ili{Samoyedic} system of personal pronouns has developed independently. In \ili{Tundra Nenets} and \ili{Forest Enets} it has undergone a significant dimensional progression. In contrast to \ili{Nganasan}, which employs a system of morphological realization and non-realization drawing a distinction between pronominal agent and patient arguments, \ili{Tundra Nenets} and \ili{Forest Enets} have grammaticalized the morpheme \textit{śiʔ}, which nowadays represents the \isi{direct object} forms by suppletion. 
This is summarized in Figure~\ref{12-wr-fig:2}: 

\begin{figure}[!h] %ex40
	\centering
	\includegraphics[width=80ex]{figures/12-wr-fig2.jpg} %\vspace{-10ex}
	\caption{The development of structural case marking on nouns and pronouns and of the objective conjugation in Northern Samoyedic}\label{12-wr-fig:2}
\end{figure}


\section*{Abbreviations}

\begin{tabularx}{.45\textwidth}{lQ}
1 &first person\\
2 &second person\\
3 &third person\\
\textsc{abl} & ablative\\
\textsc{acc} & accusative\\
\textsc{adv} & adverbial suffix\\
\textsc{aor} & aorist\\
\textsc{augm} & augmentative\\
\textsc{car} & caritative\\
\textsc{caus} & causative\\
\textsc{cneg} & connegative\\
\textsc{dat} & dative\\
\textsc{dest} & destinative\\
\textsc{dim} & diminutive\\
\end{tabularx}
\begin{tabularx}{.49\textwidth}{ll}
\textsc{doc} & differential object case marking\\
\textsc{du} & dual\\
\textsc{dur} & durative\\
\textsc{emph} & emphasis\\
\textsc{ep} & epenthetic vowel\\
\textsc{ess} & essive\\
\textsc{excl} & exclamative\\
\textsc{foc} & focus marker\\
\textsc{freq} & frequentative\\
\textsc{fut} & future\\
\textsc{gen} & genitive\\
\textsc{hab} & habituative\\
\textsc{impfut} & imperative future\\
\textsc{inch} & inchoative\\
\end{tabularx}

\begin{tabularx}{.45\textwidth}{lQ}
\textsc{inf} & infinitive\\
\textsc{inter} & interrogative marker\\
\textsc{ipfv} & imperfective\\
\textsc{iter} & iterative\\
\textsc{lat} & lative\\
\textsc{lim} & limitative\\
\textsc{loc} & locative\\
\textsc{mod} & modal gerund\\
\textsc{narr} & narrative\\
\textsc{negaux} & negation auxiliary\\
\textsc{nmlz} & nominalizer\\
\textsc{nom} & nominative\\
\textsc{o} & object\\
\textsc{oc} & objective conjugation\\
\textsc{pass} & passive\\
\end{tabularx}
\begin{tabularx}{.45\textwidth}{lQ}
\textsc{pst} & past tense\\
\textsc{pstpf} & past perfect\\
\textsc{pf} & present perfect\\
\textsc{pl} & plural\\
\textsc{poss} & possessive\\
\textsc{prf} & perfect\\
\textsc{prob} & probabilitative\\
\textsc{prol} & prolative\\
\textsc{prs} & present continuous\\
\textsc{ptcp} & participle\\
\textsc{rc} & reflexive conjugation\\
\textsc{res} & resultative\\
\textsc{sc} & subjective conjugation\\
\textsc{sg} & singular\\
\textsc{sup} & supine\\
\end{tabularx}


\section*{Data sources:}%TODO incorporate all sources into the regular bibliography

\begin{itemize}
\item Stories \textit{\textcyrillic{Два чума}} (Languedoc. dva čuma), \textit{\textcyrillic{Как утонули ребята}} (Languedoc. rebjata), \textit{\textcyrillic{Как сгорела наша школа}} (Languedoc. škola) from the online corpus of the project “Languedoc”, available at \url{http://www.philol.msu.ru/~languedoc/rus/ngan/corpus.php} [accessed on June, 24, 2017]

\item Story \textit{\textcyrillic{Отпуск}} (Languedoc. otpusk) from the online corpus of the project “Languedoc”, available at 
\url{http://www.philol.msu.ru/~languedoc/rus/enets/corpus.php} [accessed on June, 24, 2017]

\item Stories \textit{Kehy Luu} (NOS. kehy luu), \textit{Mou Djamezi} (NOS. mou djamezi) from the online corpus of the project “Negation in \ili{Ob-Ugric} and \ili{Samoyedic} Languages”, University of Vienna, available at\\ \url{http://www.univie.ac.at/negation/sprachen/nganasanischa.html} [accessed on June, 24, 2017]

\item Stories \textit{Tesjada Nisjami} (NOS. tesjada nisjami), \textit{Tet Weli Teta} (NOS. tet weli teta) from the online corpus of the project “Negation in \ili{Ob-Ugric} and \ili{Samoyedic} Languages”, University of Vienna, available at\\
\url{http://www.univie.ac.at/negation/sprachen/nenzischa.html} [accessed on June, 24, 2017]

\item Stories \textit{Text 1} (NOS. text 01), \textit{Text 39} (NOS. text 39) from the online corpus of the project “Negation in \ili{Ob-Ugric} and \ili{Samoyedic} Languages”, University of Vienna, available at \url{http://www.univie.ac.at/negation/sprachen/enzischa.html }  [accessed on June, 24, 2017]

%\item Narrative texts recorded by \textit{Kazimir Labanauskas} (Nikolaeva. Labanauskas) available in: Nikolaeva, Irina. 2014. \textit{A Grammar of \ili{Tundra Nenets}.} Berlin / New York: De Gruyter, 432-495.%added as the usual reference to Nikolaeva

\end{itemize}
\newpage 
{\sloppy
\printbibliography[heading=subbibliography,notkeyword=this] }
\end{document}
