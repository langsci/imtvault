\documentclass[output=paper]{LSP/langsci}
\ChapterDOI{10.5281/zenodo.1228251} 
\author{Eitan Grossman\affiliation{The Hebrew University of Jerusalem }} %TODO [cmld] Full name and affiliation?
\title{From suffix to prefix to interposition via Differential Object Marking in Egyptian-Coptic}
\shorttitlerunninghead{From suffix to prefix to interposition via DOM in Egyptian-Coptic}
\abstract{This article argues that Differential Argument Indexing (DOI) and Differential Argument Marking (DOM) constructions in Coptic (Afroasiatic, Egypt) are reanalyzed, resulting in a set of verbs with interposed P-indexes within bipartite stems \citep{DeLancey1996Penutian,Nichols2003Bipartite}. Basically, incorporated noun phrases with prefixed possessor indexes become parts of derived verbs with unpredictable lexical semantics, and their erstwhile possessor prefixes, entrapped within the derived verb, are reanalyzed as P-interpositions. Since this possessor prefix ultimately developed from an earlier possessor suffix, the pathway documented here, stripped down to its essentials, is \textsc{suffix} → \textsc{prefix} → \textsc{interposition}, and erstwhile complex construction → \textsc{bipartite stem}. Finally, an overt genitive prefix that marks lexical possessors of incorporated noun phrases is reanalyzed as an accusative case prefix. These changes introduce new complexity into Coptic Differential Argument Marking: not only are P arguments either indexed as suffixes, case marked, or incorporated for the majority of verbs, they can be indexed as interpositions for a lexically determined set of verbs.

% \keywords{DOM, DOI, noun incorporation, affix ordering, language change, Coptic}
}
\maketitle

% TODO REFs REFS DONE
\begin{document}


\section{Introduction}
\label{05-sec:1}

In recent years, Differential Object Marking (DOM) has been distinguished from Differential Object Indexing (DOI) \citep{Iemmolo2011Towards}, but both fall under the generalized definition of Differential Argument Marking proposed by Witzlack-Makarevich \& Seržant (this volume), \ie “Any kind of situation where an argument of a predicate bearing the same generalized \isi{semantic role} (or macrorole) may be coded in different ways, depending on factors other than the argument role itself.” Under this definition, as Witzlack-Makarevich \& Seržant point out, “DAM is not restricted to \isi{case marking} (also called dependent marking or flagging […] but also includes differential agreement (or head marking or indexing).” However, since some languages have both DOM and DOI, the two can interact, sometimes in complex ways.

 The aim of this article is to show one way that DOM and DOI can interact in language change. It is argued that for a number of verbs, the specific constructions implicated in both DOM and DOI in \ili{Coptic} (Afroasiatic, Egypt) are reanalyzed, resulting in the reanalysis of a prefixed possessor index as an interposed P-index within a bipartite stem.\footnote{In this article, I follow the Comrian approach to \isi{transitivity} and argument roles articulated in \citet{Comrie1981Language}, \citet{Lazard2002Transitivity} and \citet{Haspelmath2011S}. Basically, transitive clauses are those with A and P as core arguments. A and P arguments are those that are coded like the arguments of a prototypical biactant clause in which the predicate expresses an action, \eg ‘kill.’} Bipartite stems, described by \citet{Jacobsen1980Washo}, \citet{DeLancey1996Penutian}, and \citet{Nichols2003Bipartite} for some North American and Nakh-Daghestanian Caucasian languages,\footnote{I would like to thank Alena Witzlack-Makarevich for drawing my attention to this similarity.} are defined by \citet[321]{Nichols2003Bipartite} as “a segmentable simplex stem; or a stem with inflection positioned so as to split the stem into two parts.” The term \textit{interposition} is used to characterize the person index that occurs between the two pieces of a bipartite stem, and the term “interposed” is used to describe its position of occurrence. Interpositions are distinguished from infixes, which “occur inside of a simple stem, where [their] position is usually defined phonologically” \citep[321]{Nichols2003Bipartite}.

The coding of P arguments in \ili{Coptic} involves both DOM and DOI, since a lexical P argument can be either overtly case marked or incorporated into the verb (DOM), or can be indexed on the verb as a suffix (DOI). For the vast majority of transitive verbs, both \isi{case marking} and incorporation of P are in complementary distribution with P-indexing. However, for a lexically-determined set of verbs, incorporated noun phrases with prefixed possessor indexes become parts of derived verbs with unpredictable lexical semantics, and their erstwhile possessor prefixes, entrapped within the derived verb, are reanalyzed as P-interpositions. Since this possessor \textit{prefix} ultimately developed from an earlier possessor \textit{suffix}, the pathway documented here, stripped down to its essentials, is \textsc{suffix} → \textsc{prefix} → \textsc{interposition}, and erstwhile complex construction → \textsc{bipartite stem}. Finally, an overt genitive prefix that marks lexical possessors of incorporated noun phrases is reanalyzed as an \isi{accusative case} prefix. 

 All in all, these changes introduce new complexity into \ili{Coptic} DAM: not only are P arguments either indexed as suffixes, case marked, or incorporated for the majority of verbs, they can be indexed as interpositions for a lexically determined set of verbs. 

 The structure of this article is as follows. \sectref{05-sec:2} presents the basic problem dealt with here. \sectref{05-sec:3} describes some background about the marking of grammatical relations in \ili{Coptic}. \sectref{05-sec:4} presents some basic facts about the synchrony and \isi{diachrony} of possessive phrases in \ili{Ancient Egyptian}-\ili{Coptic}, tracing the replacement of suffixed possessor indexes by prefixed possessor indexes. \sectref{05-sec:5} shows how prefixed possessor indexes are reanalyzed as infixed P indexes. \sectref{05-sec:6} suggests that this process, alongside the well-known ‘\textsc{have}-drift’ (\citealt{Comrie1981Language}, \citealt{Stassen2009Predicative}), is yet another type of ‘P-drift,’ in which non-P arguments are reanalyzed as P arguments. \sectref{05-sec:7} concludes and sketches what an explanation for P-drift might look like.

\section{The problem: P infixes within a lexical verb}
\label{05-sec:2}

In \ili{Coptic}, P indexes are bound to the rightmost edge of the lexical verb in monotransitive clauses.\footnote{The examples presented here are glossed according to the Leipzig Glossing Rules, and are transliterated according to the Leipzig-Jerusalem system \citep{Grossmanetal2015Leipzig-Jerusalem}. Abbreviations used in the glosses in this article, beyond those found in the Leipzig Glossing Rules list, are: \textsc{aor} – aorist, basically a habitual verb form; \textsc{bg} – backgrounder, prefix that marks the verb as topical and, in the present case, an adjunct as focal; \textsc{mod} – modifier marker, \textsc{seq} – sequential verb form; \textsc{stat} – stative verb form. The glossing convention of a space followed by a hyphen indicates that the morpheme following the hyphen is part of the same morphological – but not phonological – word.} In \REF{05-gr-ex:1},\footnote{Examples are cited as found in easily accessible secondary sources, such as \citet{Layton2004Coptic}, an excellent descriptive grammar, or \citet{Shisha-Halevy1988Coptic}, a learner’s chrestomathy based on authentic \ili{Coptic} texts. This is because \ili{Coptic} texts are usually published in text editions that are not easily available to non-specialists. In some cases, I have cited examples from the letters and sermons of Besa, a \ili{Coptic} abbot. The references are to page and line number of Kuhn’s edition \citep{Kuhn1956Letters}.} the \textsc{3sg.m} index is \textit{-f}. 


\begin{exe}
\ex%1
\label{05-gr-ex:1}
\langinfo{Coptic}{}{\citealt[138]{Layton2004Coptic}}\\
\gll a-f-kaa-f e-f-onh\\
\textsc{pst}-\textsc{3sg.m}A-leave-\textsc{3sg.m}P \textsc{cvb}-\textsc{3sg.m}-live{\textbackslash}\textsc{stat}\\
\glt ‘He left him alive.’ 
\end{exe}

Exceptionally, however, a small number of verbs have interposed P-indexes, which occur within the lexical verb.

\begin{exe}
\ex%2
\label{05-gr-ex:2}
\langinfo{Coptic}{}{2 Timothy 2:14}\\
\gll mar-ou-rpe<u>meeue\\
\textsc{juss}-\textsc{3pl}A-remember<\textsc{3pl}P>remember\\
\glt ‘Let them\textsubscript{i} remember them\textsubscript{j}.’  %(2 Timothy 2:14)
\end{exe}

\begin{exe}
\ex%3
\label{05-gr-ex:3}
\langinfo{Coptic}{}{Besa 4:17}\\
\gll n-se-tm-rpe<u>ôbš\\
\textsc{seq-3pl}a\textsc{-neg}-forget<\textsc{3pl}P>forget\\
\glt ‘that they\textsubscript{i} not forget them\textsubscript{j}.’ %(Besa 4:17)
\end{exe}

\begin{exe}
\ex%4
\label{05-gr-ex:4}
\langinfo{Coptic}{}{Matthew 25:36}\\ %cited in \citealt[142]{Layton2004Coptic}
\gll a-tetn-cmp<a>šine\\
\textsc{pst-2pl}A-visit<\textsc{1sg}P>visit\\
\glt ‘You visited me.’ 
\end{exe}

When the P argument of these verbs is a lexical noun phrase, on the other hand, it is not indexed on the verb at all. Rather, it is marked by an overt case prefix \textit{n}- (\textit{m}- before labials), glossed here as accusative (\textsc{acc}).

\begin{exe}
\ex%5
\label{05-gr-ex:5}
\langinfo{Coptic}{}{Luke 22:61}\\
\gll a-petros -rpmeeue m-p-šače m-p-čoeis\\
\textsc{pst}-Peter -remember \textsc{acc}-\textsc{def.m.sg}-word \textsc{gen-def.m.sg}-lord\\
\glt ‘Peter remembered the word of the Lord.’ %(Luke 22:61)
\end{exe}

\begin{exe}
\ex%6
\label{05-gr-ex:6}
\langinfo{Coptic}{}{Luke 7:16}\\
\gll a-p-noute -cmpšine m-pe-f-laos\\
\textsc{pst-def.m.sg}-God -visit \textsc{acc-poss.m.sg-3sg.m}-people\\
\glt ‘God visited his people.’ %(Luke 7:16)
\end{exe}

From a synchronic point of view, this is a curious fact: for a small list of verbs, the P argument is indexed \textit{within} the lexical verb. However, from the point of view of language change, this unusual feature has a clear explanation. 

In short, it is argued that these verbs are derived, via noun phrase incorporation,\footnote{The notion ‘incorporation’ is usually not extended to constructions in which nominals with phrasal properties (\eg determination, possessor marking, etc.) are attached to verbs. However, some accounts of incorporation do indeed recognize that such nominals may be incorporated (\eg \citealt{Aikhenvald2007Typological,Grossman2016Noun}), and some languages are described in a straightforward way as incorporating determiners and other items typically associated with noun phrases, \eg \citegen{Donohue1999Warembori} description of \ili{Warembori}.} from the compounding of a verbal root and a possessive noun phrase, in which the possessor index is prefixed to a lexical noun. Returning to example \REF{05-gr-ex:4} above, the original structure is as follows:

\begin{exe}
\ex%7
\label{05-gr-ex:7}
\langinfo{Coptic}{}{Matthew 25:36}\\
\gll a-tetn-cm-p-a-šine\\
\textsc{pst-2pl}-find-\textsc{poss.m.sg-1sg}-report\\
\glt ‘You visited me’ (lit.\,‘you found my report’). %(Matthew 25:36)
\end{exe}

The possessor prefix (\textit{pa}-) originates from a construction in which an even earlier possessor index (corresponding to \ili{Coptic} -\textit{a} \textsc{1sg}) is suffixed to a demonstrative base (corresponding to \ili{Coptic} \textit{p}- \textsc{poss.m.sg}). However, in order to demonstrate that the synchronic structure involves an interposed P-index, rather than a prefixed possessor index, I show that the meanings of the verbs derived via incorporation of these noun phrases are not completely predictable. In short, the lexical verb ‘visit’ in \ili{Coptic} is \textit{cmp}\textit{šine}, and <\textit{a}> is interposed in a position that is synchronically arbitrary but historically explicable.

\section{The background}
\label{05-sec:3}

\subsection{Ancient Egyptian-Coptic}
\label{05-subsec:3-1}

\ili{Ancient Egyptian}-\ili{Coptic}, the indigenous language of Egypt, is an independent branch of the Afroasiatic phylum. It is documented from around the turn of the 3\textsuperscript{rd} millennium BCE up until the 13\textsuperscript{th} or 14\textsuperscript{th} century CE, when its last speakers shifted to Arabic; for overviews of \ili{Ancient Egyptian}, see \citet{Loprieno1995Egyptian,Loprienoetal2012Ancient,Grossmanetal2015Egyptian-Coptic}, or \citet{Haspelmath2015Egyptian}. \ili{Coptic}, the latest stage of the language, is documented in a dozen or so literary dialects, as well as a range of less standardized language varieties attested in non-literary texts, such as private letters, legal documents, and financial records. The main literary dialects are \ili{Sahidic} and Bohairic. The data for the present article are taken from the \ili{Sahidic} dialect, which is the best described (\citealt{Layton2004Coptic,Reintges2004Coptic,Shisha-Halevy1986Coptic}). 

\subsection{Grammatical relations in Coptic: a brief overview}
\label{05-subsec:3-2}

Due to the complexity of \ili{Coptic} grammatical relations, I focus on the coding properties of \isi{intransitive} and monotransitive verbal clauses, \ie those with S or A and P as arguments of the predicate, leaving out \isi{ditransitive} clauses and clauses with non-accusative objects. \sectref{05-sec:3-2-1} deals with argument indexing, \sectref{05-sec:3-2-2} with \isi{case marking}, and \sectref{05-sec:3-2-3} with incorporation. Before proceeding to the presentation of grammatical relations, it is important to briefly describe two basic facts of \ili{Coptic} transitive verbs. First, each lexical verb occurs in up to three distinct allomorphs, the conditioning factor being the encoding of P (see \tabref{05-gr-tab:1}). The allomorphs are labelled here as distinct stems (represented as Σ with superscript numerals, borrowing a practice of \ili{Sino-Tibetan} linguistics).\footnote{Guillaume Jacques (p.c.) informs me that it was Georg van Driem who originated this practice.}


\begin{enumerate}
\item The free form of the verb (Σ\textsuperscript{1}) occurs when no P is present or when P is overtly case marked. It is also the citation form.

\item A second allomorph (Σ\textsuperscript{2}) occurs when lexical P is incorporated.

\item A third allomorph (Σ\textsuperscript{3}) occurs when P is indexed on the verb.\footnote{This presentation is more convenient than precise. Actually, the choice of bound verb stem is conditioned by phonological considerations: phonologically light elements condition Σ\textsuperscript{2}, while phonologically heavy elements condition Σ\textsuperscript{3}. However, almost all person indexes are phonologically light. I would like to thank Matthias Müller (p.c.) for reminding me of this.}
\end{enumerate}

\begin{table}
\caption{Allomorphs of the Coptic verb}
\label{05-gr-tab:1}

\begin{tabularx}{\textwidth}{Xlll} 
\lsptoprule
& \textsc{free form}

(Σ\textsuperscript{1}) & \textsc{with incorporated p}

(Σ\textsuperscript{2}) & \textsc{with p index}

(Σ\textsuperscript{3})\\
\midrule
‘draw (sword)’ & \textit{tôkm} & \textit{tekm-} & \textit{tokm-}\\
‘drink’ & \textit{sô} & \textit{se-} & \textit{soo-}\\
‘find’ & \textit{cine} & \textit{cn-} & \textit{cnt-}\\
\lspbottomrule
\end{tabularx}
\end{table}

The allomorphs that occur with incorporated P (Σ\textsuperscript{2}) or indexed P (Σ\textsuperscript{3}) are bound forms, \ie they cannot occur as free forms.

 Second, \ili{Coptic} verbs occur in two main constructions, which will be treated here as templates. The first is the Present tense (see \tabref{05-gr-tab:2}), which comprises two main slots \citep{Polotsky1960Coptic}. The first slot, for the A argument,\footnote{This slot is also the one in which S arguments occur, but since they are not the focus of this article, I ignore them here. \ili{Coptic} argument indexing is nominative-accusative (S=A${\neq}$P) in terms of linear order.} is occupied either by a lexical noun phrase or a prefixed person index. The second slot is occupied by the lexical verb or by a locative expression. P cannot be indexed on the verb, but is rather overtly case-marked.

\begin{table}
\begin{tabular}{lll}
\lsptoprule
A/S & Lexical verb & (P)\\
\midrule
\textit{t}\textit{\textsuperscript{i}}\textit{-} & \textit{-sops} & \textit{(mmo-k)}\\
\textsc{1sg} & entreat & \textsc{(acc-2m.sg)}\\
‘I entreat you.’ & & \\
\lspbottomrule
\end{tabular}
\caption{The structure of the Present tense verb}\label{05-gr-tab:2}
\end{table}


The second construction is for all verbal templates other than the Present tense. It comprises three obligatory slots. The first is occupied by a TAM/Polarity prefix, the second by an A index, and the third by a lexical verb. P-indexes occur in an optional fourth slot, suffixed to the lexical verb. As is discussed in the following section (\sectref{05-sec:3-2-1}), P-indexes and case-marked P are largely in complementary distribution.

\begin{table}
\begin{tabular}{llll}
\lsptoprule
TAM/Polarity & A/S & Lexical verb & (P)\\\midrule
\textit{a-} & \textit{-f-} & \textit{-tamio-} & \textit{-ou}\\
\textsc{past} & \textsc{3sg.m} & create & \textsc{3pl}\\
\multicolumn{4}{l}{‘He created them’ (Shenoute, cited in \citealt[34]{Shisha-Halevy1988Coptic}).}\\
\lspbottomrule
\end{tabular}
\caption{The structure of non-Present tense verbs}\label{05-gr-tab:3}
\end{table}

\subsubsection{Indexing} \label{05-sec:3-2-1}

In monotransitive clauses, A and P can be indexed on the verb. Argument indexing is not obligatory. A given monotransitive verb can occur with an A index \REF{05-gr-ex:8}, a P index \REF{05-gr-ex:9}, both \REF{05-gr-ex:10}, or neither \REF{05-gr-ex:11}. 
Bound A indexes are prefixed to the lexical verb (or an auxiliary verb), and if an overt TAM/Polarity prefix is present, the latter precedes the person index. 
In order to simplify the presentation, the following examples are taken from the Past tense, whose basic structure is presented in \tabref{05-gr-tab:3} above):

\begin{exe}
\ex%8
\label{05-gr-ex:8}
\langinfo{Coptic}{}{Besa 46:26}\\
\gll a-u-sôtp n-ne-u-hiooue\\
\textsc{pst-3pl}A-choose \textsc{acc-poss.pl-3pl}-ways\\
\glt ‘They have chosen their ways.’ %Besa 46:26
\end{exe}

\begin{exe}
\ex%9
\label{05-gr-ex:9}
\langinfo{Coptic}{}{Shenoute, cited in \citealt[34]{Shisha-Halevy1988Coptic}}\\
\gll a-p-čoeis -tsto-ou ebol\\
\textsc{pst-def.m.sg}-lord -reject-\textsc{3pl}P out\\
\glt ‘The Lord rejected them.’ %(Shenoute, cited in \citealt[34]{Shisha-Halevy1988Coptic})
\end{exe}

\begin{exe}
\ex%10
\label{05-gr-ex:10}
\langinfo{Coptic}{}{Besa 45:32}\\
\gll a-u-tamo-n\\
\textsc{pst-3pl}A-inform-\textsc{1pl}P\\
\glt ‘They informed us.’  %Besa 45:32
\end{exe}

\begin{exe}
\ex%11
\label{05-gr-ex:11}
\langinfo{Coptic}{}{Shenoute, cited in \citealt[35]{Shisha-Halevy1988Coptic}}\\
\gll a-n-daimonion -soun-p-čoeis\\
\textsc{pst-def.pl}-demon -know-\textsc{def.m.sg}-lord\\
\glt ‘The demons knew the Lord.’  %(Shenoute, cited in \citealt[35]{Shisha-Halevy1988Coptic})
\end{exe}


\ili{Coptic} has DOI, since person indexes can either be suffixed to the verb \REF{05-gr-ex:12} or case-marked \REF{05-gr-ex:13}, even for one and the same verb in the same verbal construction, \eg the Past tense:

\begin{exe}
\ex%12
\label{05-gr-ex:12}
\langinfo{Coptic}{}{Matthew 13:48, cited in \citealt[132]{Layton2004Coptic}}\\
\gll n-et-hoou=de a-f-noč-ou ebol\\
\textsc{def.pl-rel}-bad{\textbackslash}\textsc{stat=ptcl} \textsc{pst-3sg.m}-cast-\textsc{3pl} out\\
\glt ‘The bad ones, he cast them out.’  %(Matthew 13:48, cited in \citealt{Layton2004Coptic}: 132)
\end{exe}

\begin{exe}
\ex%13
\label{05-gr-ex:13}
\langinfo{Coptic}{}{Luke 4:35, cited in \citealt[132]{Layton2004Coptic}}\\
\gll a-f-nouče=de mmo-f\\
\textsc{pst-3sg.m}-cast=\textsc{ptcl} \textsc{acc-3sg.m}\\
\glt ‘And he threw him down.’  %(Luke 4:35, cited in \citealt{Layton2004Coptic}: 132)
\end{exe}



At present, there is no account of \ili{Coptic} DOI, so I will not speculate on the functions associated with it. What is important to establish in the present context is that P-indexes are suffixed to the lexical verb, and cannot occur elsewhere within the verbal bound group.

\subsubsection{Case marking}
\label{05-sec:3-2-2}

\ili{Coptic} has a cross-linguistically unusual case-marking system: both the Nominative \REF{05-gr-ex:14} and the Accusative \REF{05-gr-ex:15} are overtly marked by prefixed case markers, but neither of these is the citation form. The citation form is the bare noun form, which is simply a nominal stem without case markers or other inflectional material, such as (in)\isi{definiteness} or number-gender markers. Such case-marking systems have been called ‘marked A/S vs. marked P’ by \citealt{Creissels2009Uncommon} (see also \citealt{Grossman2015Verb}). Moreover, noun phrases are overtly case-marked only if they are postverbal; if they are preverbal or incorporated into the verb, they are not case-marked.

\begin{exe}
\ex%14
\label{05-gr-ex:14}
\langinfo{Coptic}{}{Luke 1:12}\\
\gll a-f-štortr=de nci-zakharias\\
\textsc{pst-3sg.m}-be.troubled=\textsc{ptcl} \textsc{nom}-Zacharias\\
\glt ‘But Zacharias was troubled.’ %Luke 1:12
\end{exe}

\begin{exe}
\ex%15
\label{05-gr-ex:15}
\langinfo{Coptic}{}{Luke 1:36}\\
\gll a-s-ô n-ou-šêre\\
\textsc{pst-3sg.f}-conceive \textsc{acc-indef}-son\\
\glt ‘She conceived a son.’ %(Luke 1:36)
\end{exe}

Examples \REF{05-gr-ex:16} and \REF{05-gr-ex:17} show the main constructions involved in \ili{Coptic} DOM: lexical P must be either incorporated or overtly case-marked. 

\begin{exe}
\ex%16
\label{05-gr-ex:16}
\langinfo{Coptic}{}{Mark 4:36}\\
\gll a-f-ka-p-mêêše\\
\textsc{pst-3sg.m-}leave\textsc{-def.m.sg}-multitude\\
\glt ‘He left the multitude.’ %(Mark 4:36)
\end{exe}

\begin{exe}
\ex%17
\label{05-gr-ex:17}
\langinfo{Coptic}{}{Matthew 13:36}\\
\gll a-f-kô m-p-mêêše\\
\textsc{pst-3g.m}-leave \textsc{ acc-def.m.sg}-multitude\\
\glt ‘He left the multitude.’ %(Matthew 13:36)
\end{exe}

The conditions regulating \ili{Coptic} DOM are complex, and involve both an aspectual split and discourse conditions that are still poorly understood and may vary from dialect to dialect and even from corpus to corpus \citep{Engsheden2008Differential}. However, there are some broad regularities.

 First of all, in the Present tense and in verbal constructions built on the Present tense (\eg the Imperfect), DOM is strictly regulated by what is traditionally seen as \isi{definiteness}, but which could also be seen as a matter of \isi{referentiality}: bare nouns stems, which tend to have non-referential semantics, are obligatorily incorporated into the verb; in \REF{05-gr-ex:18}, for example, the noun stem \textit{daimonion} ‘demon(s)’ is non-referential. On the other hand, referential noun phrases of any sort are obligatorily case marked, as in \REF{05-gr-ex:19}, in which \textit{daimonion} is referential and bears an indefiniteness prefix.

\begin{exe}
\ex%18
\label{05-gr-ex:18}
\langinfo{Coptic}{}{Luke 11:15, cited in \citealt[132]{Layton2004Coptic}}\\
\gll e-f-neč-daimonion ebol hn-beelzeboul\\
\textsc{bg.prs-3sg.m}-cast-demon out in-Beelzebul\\
\glt ‘He casts out demons by means of Beelzebul.’  %(Luke 11:15, cited in \citealt[132]{Layton2004Coptic})
\end{exe}

\begin{exe}\ex%19
\label{05-gr-ex:19}
\langinfo{Coptic}{}{Luke 11:14, cited in \citealt[132]{Layton2004Coptic}}\\
\gll ne-f-nouče=de ebol n-ou-daimonion\\
\textsc{impf-3sg.m}-cast=\textsc{ptcl} out \textsc{acc-indef.sg}-demon\\
\glt ‘He cast out a demon.’ %(Luke 11:14, cited in \citealt[132]{Layton2004Coptic})
\end{exe}

This extends to bound person markers as well: since person markers are referential by nature, they cannot be indexed on the verb and must receive overt \isi{case marking}, as in \REF{05-gr-ex:20}.

\begin{exe}
\ex%20
\label{05-gr-ex:20}
\langinfo{Coptic}{}{Acts 13:46, cited in \citealt[236]{Layton2004Coptic}}\\
\gll tetn-nouče mmo-f ebol\\
\textsc{2pl.prs}-cast \textsc{acc-3sg.m} out\\
\glt ‘You cast it out.’ %(Acts 13:46, cited in \citealt{Layton2004Coptic})
\end{exe}

Outside of the Present tense and related constructions (\eg the Imperfect), it is still the case that bare noun stems are obligatorily incorporated into the verb, \ie they cannot bear overt \isi{accusative case} \REF{05-gr-ex:21}. On the other hand, noun phrases can either be case-marked \REF{05-gr-ex:22} or incorporated \REF{05-gr-ex:23}, the conditioning factors governing the alternation still being unclear.

\begin{exe}
\ex%21
\label{05-gr-ex:21}
\langinfo{Coptic}{}{1 Timothy 5:23}\\
\gll mpr-se-moou\\
\textsc{proh}-drink-water\\
\glt ‘Don’t drink water!’  %(1 Timothy 5:23)
\end{exe}

\begin{exe}\ex%22
\label{05-gr-ex:22}
\langinfo{Coptic}{}{Matthew 26:51}\\
\gll a-f-tôkm n-te-f-sêfe\\
\textsc{pst-3sg.m}-draw \textsc{acc-poss.f.sg-3sg.m}-sword\\
\glt ‘He drew his sword.’  %(Matthew 26:51)
\end{exe}

\begin{exe}\ex%23
\label{05-gr-ex:23}
\langinfo{Coptic}{}{Mark 14:47}\\
\gll a-f-tekm-te-f-sêfe\\
\textsc{pst-3sg.m}-draw-\textsc{poss.f.sg-3sg.m}-sword\\
\glt ‘He drew his sword.’ %(Mark 14:47)
\end{exe}

\subsubsection{Incorporation}
\label{05-sec:3-2-3}

As discussed in \sectref{05-sec:3-2-2} above, lexical A, S, or P can be incorporated into the verb. A/S incorporation is unexpected from a cross-linguistic view\footnote{An anonymous reviewer has drawn my attention to \citet{Zavala2000Inversion}, which argues that \ili{Olutec} (Mixean) allows the incorporation of A.} but it is unimportant for the present discussion; I will focus here on P-incorporation, which is highly productive in \ili{Coptic}. 

 Nouns referring to body parts are often incorporated in \ili{Coptic}, as in other languages (\citealt{Mithun1984Evolution,Mithun1986Nature,Mithunetal1999Effect}), and these body part terms often bear possessor indexes, as in \REF{05-gr-ex:24} and \REF{05-gr-ex:25}.

\begin{exe}
\ex%24
\label{05-gr-ex:24}
\langinfo{Coptic}{}{Besa 10:24}\\
\gll a-f-ka-toot-f\\
\textsc{pst-3sg.m}-put-hand-\textsc{3sg.m}\\
\glt ‘He ceased.' (lit.\,‘he put his hand’) 
\end{exe}

\begin{exe}
\ex%25
\label{05-gr-ex:25}
\langinfo{Coptic}{}{Besa 3:30}\\
\gll n-tn-smn-toot-n\\
\textsc{seq-1pl}-establish-hand-\textsc{1pl}\\
\glt ‘And let us agree.’ (lit.\,‘let us establish our hand’) %(Besa 3:30)
\end{exe}

The free forms (Σ\textsuperscript{1}) of these verbs are, respectively, \textit{kô} ‘put’ and \textit{smine} ‘establish.’ What is noteworthy in these constructions is that the possessive suffixes can be analyzed as P-indexes, since incorporation of body parts produces new verbs whose meaning is not transparently predictable from the sum of the verbal and nominal roots. In other words, \REF{05-gr-ex:24} and \REF{05-gr-ex:25} above could be analyzed as follows in \REF{05-gr-ex:26} and \REF{05-gr-ex:27}, with A and P being coreferential, and the construction as a whole being reflexive.

\begin{exe}
\ex%26
\label{05-gr-ex:26}
\langinfo{Coptic}{}{Besa 10:24}\\
\gll a-f-katoot-f\\
\textsc{pst-3sg.m}-cease-\textsc{3sg.m}\\
\glt ‘He ceased’ (lit.\,‘he put his hand’). %(Besa 10:24)
\end{exe}

\begin{exe}
\ex%27
\label{05-gr-ex:27}
\langinfo{Coptic}{}{Besa 3:30}\\
\gll n-tn-smntoot-n\\
\textsc{seq-1pl}-agree-\textsc{1pl}\\
\glt ‘And let us agree’ (lit.\,‘let us establish our hand’). %(Besa 3:30)
\end{exe}

It is important to note that this reanalysis is plausible, since these possessive suffixes are a relic of an earlier head-marking possessive construction, in which possessor indexes are suffixed directly to the possessum, as in \REF{05-gr-ex:28} (\citealt{Egedi2010Possessive}; \citealt{Haspelmath2015Three}).

\begin{exe}
\ex % Earlier Egyptian%28
\label{05-gr-ex:28}
\langinfo{Earlier Egyptian}{}{\citealt[102, 124]{Allen2013Ancient}}\\
\gll rn-k pr-k\\
 name-\textsc{2m.sg} house-\textsc{2m.sg}\\
\glt ‘your name’ { }{ }{ }‘your house’
\end{exe}

In \ili{Coptic}, however, these suffixes are nearly obsolete, and occur only on a small list of body parts and other inalienable nouns. The most frequent – and the only productive – possessive construction in \ili{Coptic} comprises a possessive prefix, which in turn comprises a pronominal base that shows number (singular vs. plural), and gender (masculine vs. feminine) distinction in the singular, to which a possessor index attaches, as in \REF{05-gr-ex:29}.

\begin{exe}
\ex%29
\label{05-gr-ex:29}
\langinfo{Coptic}{}{Matthew 7:22, John 10:3}\\
\gll pe-k-ran ne-u-ran\\
\textsc{poss.m.sg-2m.sg}-name \textsc{poss.pl-3pl}-name\\
\glt ‘your name’ { }{ }{ }{ }{ }{ }{ }{ }{ }{ }{ }{ }{ }{ }{ }{ }{ }{ }{ }{ }‘their names’
\end{exe}

Moreover, many of the nouns denoting body parts in the incorporation construction are themselves obsolete as independent lexical items, and they occur almost exclusively as parts of noun-verb compounds such as those in \REF{05-gr-ex:24} and \REF{05-gr-ex:25}, or as parts of prepositions, as in \REF{05-gr-ex:30}. As such, they can be treated as ‘obligatorily possessed nouns’ \citep{Nicholsetal2005Possessive}. A short list of forms used as bound roots are compared with the free forms in \tabref{05-gr-tab:4} (for a full list, see \citealt[102--104]{Layton2004Coptic}).

\begin{exe}
\ex%30
\label{05-gr-ex:30}
\langinfo{Coptic}{}{Matthew 5:25}\\
\gll etoot-f (<e-toot-f)\\
 to-\textsc{3sg.m} to-hand-\textsc{3sg.m}\\
\glt ‘to him’{ }{ }{ }{ } ‘to his hand’ 
\end{exe}

%%please move \begin{table} just above \begin{tabular
\begin{table}
\begin{tabular}{lll}
\lsptoprule
\textsc{meaning} & \textsc{bound form} & \textsc{free form}\\
\midrule
‘hand’ & \textit{toot-} & \textit{ci}\textit{č}\\
‘foot’ & \textit{rat-} & \textit{ouerête}\\
‘eye’ & \textit{eiat-} & \textit{bal}\\
‘head’ & \textit{čô-} & \textit{ape}\\
\lspbottomrule
\end{tabular}
\caption{Bound forms and free forms of nouns denoting body parts}\label{05-gr-tab:4}
\end{table}

One can assume that at least arguably, the erstwhile possessor indexes have been reanalyzed as P suffixes, due to the following reasons: (a) the possessor suffixes are not a productive strategy for marking the possessor on nouns, (b) the noun roots to which they attach are not identifiable as free forms with a lexical meaning, and (c) the meaning of the incorporated constructions are not transparent. This is further corroborated by the fact that P-indexes of underived verbs are also suffixed to the lexical verb, which plausibly would have enhanced the likelihood of possessor indexes being reanalyzed as P-indexes.

These facts about \ili{Coptic} will be used to explain the origin of infixed P-indexes that occur within the lexical verb. In the next section, it is shown that the \ili{Coptic} possessor prefix developed, in part, from an earlier possessor suffix.


\section{From suffix to prefix in the coding of possessors}\label{05-sec:4}

The diachronic relationship between the two ways of indexing the possessor in a possessive phrase, \ie via possessor suffixes \REF{05-gr-ex:28} or possessor prefixes \REF{05-gr-ex:29} is well-documented in the history of \ili{Ancient Egyptian}. The head-marking construction with a possessor index suffixed to the noun denoting the possessum \REF{05-gr-ex:28} is, historically speaking, the older construction, attested from the very beginning of the textual record.

 A competing construction, which emerged relatively early in the textual record, comprises a demonstrative pronoun (\textit{p\texttt{Ꜣ}y}), to which the possessor index (\eg -\textit{f}) was suffixed. One of the earliest examples documented is shown in \REF{05-gr-ex:31}.

\begin{exe}
\ex
\label{05-gr-ex:31}
\langinfo{Old Egyptian}{}{cited in \citealt{SojicEmergence}}\\
\gll p\texttt{Ꜣ}y-f hrw\\
\textsc{dem}-\textsc{3sg.m} day\\
\glt ‘his day’
\end{exe}

This newer construction rose in frequency over the course of \ili{Ancient Egyptian} \isi{diachrony}, but remained in variation with the older construction until thousands of years after the new construction is first documented (\citealt{GardinerSpread}; \citealt{SojicEmergence}; \citealt{WinandPronominal}). For example, in the 14\textsuperscript{th} century BCE, we find the two constructions as variants at the same time in the same type of text. The earlier construction is found in \REF{05-gr-ex:32}, the innovative one in \REF{05-gr-ex:33}.

\begin{exe}
\ex%32
\label{05-gr-ex:32}
\langinfo{Late Egyptian}{}{cited in \citealt{SojicEmergence}}\\
\gll mš\texttt{Ꜥ}-f\\
 army-\textsc{3sg.m}\\
\glt ‘his army’
\end{exe}

\begin{exe}
\ex%33
\label{05-gr-ex:33}
\langinfo{Late Egyptian}{}{cited in \citealt{SojicEmergence}}\\
\gll p\texttt{Ꜣ}y-f mš\texttt{Ꜥ}\\
\textsc{poss.m.sg-3sg.m} army\\
\glt ‘his army’
\end{exe}


By the time of \ili{Coptic}, the latest stage of the language, the new construction has become bound to the possessum, becoming in effect a prefixed possessor index \citep{Grossman2016Noun}, as in \REF{05-gr-ex:34}:\footnote{For a full account of the \isi{diachrony} of the two possessive constructions in the history of Egyptian, see \citet{GardinerSpread}, \citet{SojicEmergence} and \citet{WinandPronominal}, as well as \citet{Haspelmath2015Three} and \citet{Kammerzell2000Egyptian}, which are typologically-oriented.} 

\begin{exe}
\ex%34
\label{05-gr-ex:34}
\langinfo{Coptic}{}{Matthew 1:23}\\
\gll pe-f-ran\\
\textsc{poss.m.sg-3sg.m-}name\\
\glt ‘his name’
\end{exe}

In brief, the diachronic change observed here can be represented schematically as in \tabref{05-gr:tab:5}.


\begin{table}
\begin{tabularx}{\textwidth}{lQ} 
\lsptoprule
& Possessor index\\
\midrule
Stage 1 & suffix only (\textit{rn-k} name-\textsc{2m.sg} ‘your name’)\\
Stage 2 & suffix productive (\textit{rn-k}), preposed possessor index (\textit{p\texttt{Ꜣ}y-k rn}) begins to emerge\\
Stage 3 & suffix and preposed possessor index in variation (\textit{rn-k} vs. \textit{p\texttt{Ꜣ}y-k rn})\\
Stage 4 & (a) preposed possessor index becomes prefixed to noun (\textit{pe-k-ran})
(b) prefix productive, suffix limited to a small set of nouns\\
\lspbottomrule
\end{tabularx}
\caption{The diachrony of possessor infixes in Ancient Egyptian-Coptic}\label{05-gr:tab:5}
\end{table}

We now turn to the development of an interposed P-index from the prefixed possessor index in \REF{05-gr-ex:34}. 


\section{From prefix to infix in the coding of P}\label{05-sec:5}

As mentioned above in \sectref{05-sec:3}, \ili{Coptic} has a productive noun incorporation construction, in which nouns in P role are attached to a bound form of the verb. Unusually from a cross-linguistic point of view, not only bare noun roots but also referential noun \textit{phrases} can be incorporated in tenses other than the present.

For one thing, incorporated nouns can bear overt (in)\isi{definiteness} marking, as in \REF{05-gr-ex:11} above, repeated here as \REF{05-gr-ex:35} for convenience.

\begin{exe}
\ex%35
\label{05-gr-ex:35}
\langinfo{Coptic}{}{Shenoute, cited in \citealt[35]{Shisha-Halevy1988Coptic}}\\
\gll a-n-daimonion -soun-p-čoeis\\
\textsc{pst-def.pl}-demon -know-\textsc{def.m.sg}-lord\\
\glt ‘The demons knew the Lord.’
\end{exe}

Moreover, incorporated nouns can be quantified \REF{05-gr-ex:36} or modified adjectivally \REF{05-gr-ex:37}.

\begin{exe}
\ex%36
\label{05-gr-ex:36}
\langinfo{Coptic}{}{Shenoute, cited in \citealt[37]{Shisha-Halevy1988Coptic}}\\
\gll mp-f-ka-ce-hôb\\
\textsc{pst.neg-3sg.m}-put-another-thing\\
\glt ‘He did not leave another thing.’ 
\end{exe}


\begin{exe}
\ex%37
\label{05-gr-ex:37}
\langinfo{Coptic}{}{Mark 2:22, cited in \citealt[132]{Layton2004Coptic}}\\
\gll mere-laau -neč-êrp b-brre e-hôt n-as\\
\textsc{aor.neg}-anyone -throw-wine \textsc{mod}-new to-wineskin \textsc{mod}-old\\
\glt ‘No one puts new wine into old wineskins.’  %(Mark 2:22, cited in \citealt[132]{Layton2004Coptic})
\end{exe}

Incorporated noun phrases can be referred to anaphorically, as in \REF{05-gr-ex:38}.

\begin{exe}
\ex%38
\label{05-gr-ex:38}
\langinfo{Coptic}{}{Besa 9:31}\\
\gll mp-ou-oueš-pe-smou a-f-pôt ebol mmo-ou\\
\textsc{pst.neg-3pl}-love-\textsc{def.m.sg}-blessing \textsc{pst-3sg.m}-flee out \textsc{obl}-\textsc{3pl}\\
\glt ‘They did not love the blessing, and it fled away from them.’
\end{exe}

Crucially, incorporated nouns can be marked as possessed in at least three ways. The first is when erstwhile possessive suffixes attach to incorporated body parts, as in \REF{05-gr-ex:24}--\REF{05-gr-ex:25}  above. The second way is when the possessor is a lexical noun phrase, which follows the incorporated noun and is marked as dependent by the Genitive prefix \textit{n}-, as in \REF{05-gr-ex:39} and \REF{05-gr-ex:40}. 

\begin{exe}
\ex%39
\label{05-gr-ex:39}
\langinfo{Coptic}{}{Besa 2:23}\\
\gll mar-n-r-p-meeue n-ne-nt-a-pe-n-eiôt -čoo-u\\
\textsc{juss-1pl}-do-\textsc{def.m.sg}-thought{\rmfnm} \textsc{gen}-\textsc{def.pl-rel-pst-poss.m.sg-1pl}-father -say-\textsc{3pl}\\
\glt ‘Let us remember those things that our father has said’ (lit.\, ‘Let us do the thought of the things that our father has said’).
\end{exe}

\footnotetext{The lexical noun \textit{meeue} means ‘thought,’ but the derived verb \textit{rpmeeue} (lit.\,‘do the thought’ means ‘remember.’}

\begin{exe}
\ex%40
\label{05-gr-ex:40}
\langinfo{Coptic}{}{Besa 4:19}\\
\gll e-r-p-ôbš n-n-entolê m-p-noute\\
\textsc{inf}-do-\textsc{def.m.sg}-forget \textsc{gen}-\textsc{def.pl}-commandment \textsc{gen-def.m.sg}-god\\
\glt ‘to forget the commandments of God’ (lit.\,`to do the forgetting of the commandments of God') 
\end{exe}

The third way is by means of the possessor prefix described in \sectref{05-sec:4}. In \REF{05-gr-ex:41}, the possessor prefix \textit{pes}- is part of the incorporated nominal.

\begin{exe}
\ex%41
\label{05-gr-ex:41}
\langinfo{Coptic}{}{Hebrews 13:2, cited in \citealt[142]{Layton2004Coptic}}\\
\gll t-mntmaišmmo mpr-r-pe-s-ôbš\\
\textsc{def.f.sg-}hospitality \textsc{proh}-do-\textsc{poss.m.sg-3sg.f}-forget\\
\glt ‘As for hospitality, do not forget it.’  %(Hebrews 13:2, cited in \citealt{Layton2004Coptic}:142)
\end{exe}

The question is whether the verbs \textit{rpmeeue} ‘remember,’ \textit{rpôbš} ‘forget,’ and \textit{cmpšine} ‘visit’ are synchronically analyzable as compositionally derived from a verb root and a possessive noun phrase, or whether they are better treated as distinct lexical items with no internal structure.

A point in favor of the former analysis is the fact that their derivational history is clear, and their component parts all exist as independent lexical items in \ili{Coptic}. On the other hand, in favor of the latter is the fact that they have a distinct lexical meaning that is unpredictable from the original components. For example, the bound verb form (Σ\textsuperscript{2}) \textit{r}- ‘do’\footnote{The corresponding free form (Σ\textsuperscript{1}) is \textit{eire}.} is commonly used to derive verbs from nouns, \eg \textit{nobe} ‘sin’ vs. \textit{r-nobe} ‘to sin.’ In the case of \textit{rpmeeue}, it does not derive a verb from \textit{meeue}, which means ‘think, thought, opinion,’ but rather from \textit{pmeeue}, which means ‘remembrance,’ and \textit{rpmeeue} means ‘to remember, to be mindful of.’

 Similarly, \textit{cmpšine} is the result of the compounding of the verb ‘\textit{cn}- (free form \textit{cine}) ‘find’ and \textit{pšine} ‘visit,’ itself derived from \textit{šine}, which means ‘to ask, to inquire, to visit,’ or, ‘inquiry, news, report.’ In this case, the derived noun lexicalizes only a narrow part of the polysemy network of the underived noun. If \textit{šine} means ‘to ask, to inquire, to visit,’ \textit{pšine} lexicalizes only ‘visit,’ and the derived verb \textit{cmpšine} lexicalizes this meaning. I take this as evidence that the meaning of the verbs derived via incorporation is not fully predictable from its components, and as such, that verbs like \textit{rpmeeue} or \textit{cmpšine} are synchronically distinct form-meaning pairings. This is typical of some types of incorporation \citep{Mithunetal1999Effect}.

 Another argument in favor of analyzing these derived verbs as synchronically simple verbs is that the genitive prefix that marks lexical noun possessors of the incorporated noun phrase is homonymous with the \isi{accusative case} prefix. Compare the genitive prefix in \REF{05-gr-ex:47} with the accusative prefix in \REF{05-gr-ex:48}. In \REF{05-gr-ex:47}, the original structure of the construction can be glossed as ‘let us do the thought of those things that our father has said,’ with the incorporation of \textit{p-meeue} ‘the-thought.’ The genitive prefix \textit{n}- marks the determined relative clause (‘those things that our father has said’). In \REF{05-gr-ex:48}, the accusative prefix \textit{n}- simply marks the P argument.

\begin{exe}
\ex%47
\label{05-gr-ex:47}
\langinfo{Coptic}{}{Besa 2:23}\\
\gll mar-n-r-p-meeue \textbf{n}-ne-nt-a-pe-n-eiôt -čoo-u\\
\textsc{juss-1pl}-do-\textsc{def.m.sg}-thought \textbf{\textsc{gen}}-\textsc{def.pl-rel-pst-poss.m.sg-1pl}-father -say-\textsc{3pl}\\
\glt ‘Let us remember those things that our father has said’ (lit.\,‘Let us do the remembrance of those things that our father has said’).
\end{exe}

\begin{exe}
\ex%48
\label{05-gr-ex:48}
\langinfo{Coptic}{}{Besa 46:26}\\
\gll a-u-sôtp \textbf{n}-ne-u-hiooue\\
\textsc{pst-3pl}A-choose \textbf{\textsc{acc}}\textsc{-poss.pl-3pl}-ways\\
\glt ‘They have chosen their ways.’ %Besa 46:26
\end{exe}

These prefixes are diachronically distinct \citep{Winand2015Expression}, but in this particular environment, they are homonymous. This homonymy would plausibly lead to the reanalysis of the genitive prefix in this context as the accusative prefix, i.e.:

\begin{exe}
\ex%49
\label{05-gr-ex:49}
\langinfo{Coptic}{}{Besa 2:23}\\
\gll mar-n-rpmeeue  n-ne-nt-a-pe-n-eiôt -čoo-u\\
\textsc{juss-1pl}-remember \textbf{\textsc{acc}}-\textsc{def.pl-rel-pst-poss.m.sg-1pl}-father -say-\textsc{3pl}\\
\glt ‘Let us remember those things that our father has said.’
\end{exe}

If the verbs discussed here are analyzed as distinct lexical items, the person indexes in \REF{05-gr-ex:50}--\REF{05-gr-ex:52} are interpositions, occurring synchronically at an arbitrary position. Diachronically, however, they are simply in the position of earlier possessor indexes, which were prefixed to incorporated possessed nouns. For example, in \REF{05-gr-ex:50}--\REF{05-gr-ex:52}, the P interposition is in the position of the earlier possessor index, which occurred between the earlier lexical verb and the possessed noun. 

\begin{exe}
\ex%50
\label{05-gr-ex:50}
\langinfo{Coptic}{}{2 Timothy 2:14}\\
\gll mar-ou-rpe<u>meeue\\
\textsc{juss}-\textsc{3pl}A-remember<\textsc{3pl}P>remember\\
\glt ‘Let them\textsubscript{i} remember them\textsubscript{j}.’ 
\end{exe}

\begin{exe}
\ex%51
\label{05-gr-ex:51}
\langinfo{Coptic}{}{Besa 4:17}\\
\gll n-se-tm-rpe<u>ôbš\\
\textsc{seq-3pl}A\textsc{-neg}-forget<\textsc{3pl}P>forget\\
\glt ‘that they\textsubscript{i} not forget them\textsubscript{j}.’
\end{exe}

\begin{exe}
\ex%52
\label{05-gr-ex:52}
\langinfo{Coptic}{}{Matthew 25:36}\\
\gll a-tetn-cmp<a>šine\\
\textsc{pst-2pl}A-visit<\textsc{1sg}P>visit\\
\glt ‘You visited me.’
\end{exe}

The pathway of change sketched in this article shows one way that an affix can move without moving. The constellation of changes involved is complex, and involves the interaction of multiple grammatical systems. To summarize, I have argued that the following changes led to a suffix becoming a prefix, and this prefix becoming an infix, or more properly, an interposition:

\begin{enumerate}
\item 
First, an old head-marking possessive construction involving suffixed possessor indexes is superseded by a newer construction in which the possessor index is suffixed to a demonstrative, the entire construction grammaticalizing into a possessive prefix with the possessor index prefixed to the possessum noun. 

\item 
Later on, noun phrases comprising the newer possessor prefix undergo incorporation, with the resulting derived verb being a synchronically distinct form-function pairing whose meaning is not fully predictable from its component parts. 

\item 
Once incorporated, the possessor index is reanalyzed as a P-index, which is infixed, or more properly, interposed, within the lexical verb. The process of reanalysis was facilitated by the homonymy of the prefix \textit{n}-, which marks both lexical possessors (\textsc{gen}) and lexical P arguments (\textsc{acc}). As such, the postverbal possessor of the incorporated noun was reanalyzed as a postverbal P. 

\end{enumerate}

This complex series of changes is represented schematically, and with much flattening out of actual \isi{diachrony}, in \figref{05-gr-fig:1}:


\begin{figure}
\begin{tabularx}{\textwidth}{Qcl}
\lsptoprule
 Construction && \\
 \midrule
possessor index suffixed to noun && \textit{X-f} ‘his X’\\
 &↓ & \\
development of new preposed possessive article\textsc{} from\textsc{ demonstrative+possessor suffix} && p\texttt{Ꜣ}y-f \textit{X} ‘his X’\\
 &↓ & \\
possessive article becomes bound to noun, possessor index becomes prefix on noun && \textit{pef-X} ‘his X’\\
 &↓ & \\
possessed nouns incorporated into verbs && \textit{V-pef-X} ‘to V his X’\\
 &↓ & \\
loss of compositional semantics, reanalysis of genitive as accusative && \\
  &↓ & \\
reanalysis of verb as bipartite stem, reanalysis of possessor prefix as interposed P index && \textit{V}\textit{\textsubscript{1}}\textit{-f-V}\textit{\textsubscript{2}} ‘to V him’\\
\lspbottomrule
\end{tabularx}
\caption{Schematic representation of the change from suffix to prefix to interposition}
\label{05-gr-fig:1}
\end{figure}


\section{A broader view: P-drift?}\label{05-sec:6}

Taking a broader view of the complex change here, it might be possible to speak of \textit{P-drift} or \textit{\isi{direct object} drift}, in which certain non-P clause participants, given the right circumstances, are preferentially reanalyzed as P. This is, in a sense, inverting – but also broadening – the phenomena associated with “\textsc{have}-drift” \citep{Comrie1981Language,Stassen2009Predicative}, in which \isi{intransitive} predicative possession constructions gradually acquire properties associated with \isi{transitivity}. Such a process also occurred in \ili{Ancient Egyptian}-\ili{Coptic}, in which existential-locative constructions gradually acquired DOM properties, \ie the alternation between possessum incorporation and overt accusative marking. 

 In the first stage, the possessum noun occurred between a clause-initial existential marker and a clause-final locative preposition, as in \REF{05-gr-ex:53}.

\begin{exe}
\ex%53
\label{05-gr-ex:53}
\langinfo{Late Egyptian}{}{Late Ramesside Letters 19:15}\\
\gll wn \texttt{ḥ}mt \texttt{ỉ}m m-d\texttt{ỉ}-k\\
\textsc{exist} copper there \textsc{loc-}hand\textsc{-2m.sg}\\
\glt ‘You have copper’ (lit.\,‘there is copper in your hand’).  %(Late Ramesside Letters 19:15)
\end{exe}

The existential marker \textit{wn} and the locative preposition \textit{m-d\texttt{ỉ}}- (‘in-hand\_of’) underwent univerbation, with the loss of the locative preposition, which left the possessum after the bound person marker, resulting in structures like that in \REF{05-gr-ex:54}.

\begin{exe}
\ex%54
\label{05-gr-ex:54}
\langinfo{Late Egyptian}{}{P. Moscow 120, 1,58}\\ %TODO check reference
\gll \texttt{ỉ}n wn-d\texttt{ỉ}-f \texttt{ỉ}s.t \texttt{ḫꜢ}rw\\
\textsc{int} \textsc{exist}-in\_hand=\textsc{3sg.m} crew Syrian\\
\glt ‘Does he have a Syrian crew?’
\end{exe}


By the time of \ili{Coptic}, the possessor is bound to the possessive predicate \textit{ounta}-, and the lexical possessum can be marked by the accusative prefix \textit{n}-, as in \REF{05-gr-ex:55}.

\begin{exe}
\ex%55
\label{05-gr-ex:55}
\langinfo{Coptic}{}{John 10:16, cited in \citealt[306]{Layton2004Coptic}}\\
\gll ounta-i=on mmau n-hen-ke-esoou\\
\textsc{poss-1sg}=also there \textsc{acc-indef.pl}-other-sheep\\
\glt ‘I have other sheep too.’
\end{exe}

The possessum can also be incorporated, as in \REF{05-gr-ex:56}.

\begin{exe}
\ex%56
\label{05-gr-ex:56}
\langinfo{Coptic}{}{Matthew 8:20, cited in \citealt[308]{Layton2004Coptic}}\\
\gll n-bašor ounta-u-ne-u-bêb\\
\textsc{def.pl}-fox \textsc{poss-3pl-poss.pl}-\textsc{3pl}-hole\\
\glt ‘As for foxes, they have their holes.’
\end{exe}

In \ili{Coptic}, these constructions also acquired the DSM properties of transitive clauses in \ili{Coptic} \citep{Grossman2015Verb}, with lexical possessor incorporation \REF{05-gr-ex:57} alternating with overt nominative marking on the lexical possessor \REF{05-gr-ex:58}. In \REF{05-gr-ex:57}, the noun phrase referring to the possessor (‘the servant’) is incorporated into the possessive predicate \textit{ounte}-, while in \REF{05-gr-ex:58}, the lexical possessor (‘the son’) occurs after the possessive predicate, which bears a person marker (-\textit{f}) that indexes the possessor. 

\begin{exe}
\ex%57
\label{05-gr-ex:57}
\langinfo{Coptic}{}{Luke 17:9, cited in \citealt[307]{Layton2004Coptic}}\\
\gll mê ounte-p-hmhal hmot\\
\textsc{q} \textsc{poss}-\textsc{def.m.sg}-servant thanks\\
\glt `Does the servant have any thanks?'
\end{exe}

\begin{exe}
\ex%58
\label{05-gr-ex:58}
\langinfo{Coptic}{}{Mark 2:10, cited in \citealt[308]{Layton2004Coptic}}\\
\gll ount-f-eksousia mmau nci-p-šêre m-p-rôme e-ka-nobe ebol\\
\textsc{poss-3sg.m}-authority there \textsc{nom-def.m.sg}-son \textsc{gen-def.m.sg}-man \textsc{inf}-put-sin out\\
\glt ‘The son of man has authority to forgive sins.’ 
\end{exe}

Compare with A/S-incorporation \REF{05-gr-ex:59} vs. \isi{nominative case} marking \REF{05-gr-ex:60} in monotransitive verbal clauses:

\begin{exe}
\ex%59
\label{05-gr-ex:59}
\langinfo{Coptic}{}{Mark 15:2}\\
\gll a-pilatos -čnou-f\\
\textsc{pst}-Pilate ask-\textsc{3sg.m}\\
\glt ‘Pilate asked him.’ 
\end{exe}

\begin{exe}
\ex%60
\label{05-gr-ex:60}
\langinfo{Coptic}{}{Mark 13:3}\\
\gll a-f-čnou-f nci-petros\\
\textsc{pst-3sg.m}-ask-\textsc{3sg.m} \textsc{nom}-Peter\\
\glt ‘Peter asked him.’  %(Mark 13:3)
\end{exe}

In other words, in terms of indexing and case-marking, \ili{Coptic} possessors behave like A and possessums behave like P.

 While the examples of \ili{Ancient Egyptian}-\ili{Coptic} ‘\textsc{have}-drift’ sketched above provide additional data for an already established pathway, the present study shows yet another pathway in which possessors are reinterpreted as A and possessums as P, namely, via the incorporation of body parts with possessor indexes in the same position as P indexes in underived verbs. This in turn provides evidence that transitivization is not a single pathway, especially if we take into account pathways like those described in \citet{Gildea1998Reconstructing} for Cariban languages, \eg \textsc{possessor > nominative}, and \textsc{possessor} > \textsc{ergative}. These changes, interestingly, involve nominalizations being reinterpreted as main clauses, which is strikingly different from what we find in \ili{Coptic}. 

 However, since synchronic polysemies of case-markers as well as diachronic evidence indicate that other pathways are possible, \textsc{possessor} > \textsc{accusative} (also in \citealt{Gildea1998Reconstructing}), the motivations and mechanisms of P-drift still remain in need of clarification. A possible explanation might be found in \citet[303]{Serzant2013Rise}, which explains the development of canonical subject coding, \eg \isi{nominative case} marking, by appealing to semantics, arguing that “the consistent endowment of a constituent with some functional properties of a prototypical subject is the main catalyst for the (re)assignment of subject coding and behavioral properties to that constituent; it is an adjustment of grammatical properties to function.” \citeauthor{Serzant2013Rise} formulates the diachronic universal as follows (\citeyear[303]{Serzant2013Rise}): 

\begin{modquote}
Consistent functional-semantic overlap of an oblique case-marked constituent with the prototypical subject may trigger the (re)assignment of the subject coding and behavioral properties to that constituent if there are no other constituents in the construction that would show even greater overlap.
\end{modquote}

Since possessors often have the semantic and discourse properties of prototypical subjects (\eg \isi{animacy}, \isi{topicality}), and possessums often have the semantic and discourse properties of prototypical objects (\eg inanimacy, focality), the way is paved for the morphosyntactic coding properties of the possession construction to be ‘adjusted’ to fit its semantics. In the case of \ili{Coptic}, these coding properties mainly involve the participation in DSM (the alternation between nominative marking and incorporation) and DOM (the alternation between accusative marking and incorporation).

\section{Conclusions}
\label{05-sec:7}

The phenomenon of bipartite stems with person interpositions seems to be quite rare, cross-linguistically. Bipartite stems with person interpositions have been documented only in several language families spoken in a fairly small number of areas \citep{Bickeletal2007Inflectional,Hildebrandt2005Himalayan}. The diachronic pathways through which bipartite stems develop are assumed to include relics of derivational morphology or compounding, or infixation that has become morphologized (\citealt[199]{Bickeletal2007Inflectional}, \citealt{DeLancey1996Penutian}), the movement and entrapment of clitics \citep{Nichols2003Bipartite}, or the copying of affixes from another construction type, \eg head class markers from nouns to verbs \citep{Nichols2003Bipartite}. \ili{Ancient Egyptian}-\ili{Coptic} presents us with a particular pathway of development that is close to the reanalysis of compounding, since compounding and incorporation are related morphological processes, and in some views, incorporation is a particular type of compounding \citep{Mithunetal1999Effect}. 

 However, actual diachronic studies – in documented historical corpora – of the development of bipartite stems and interpositions are few and far between; previous research on bipartite stems has leaned heavily on reconstruction. The present case study shows how complex the development of bipartite stems and interpositions can be, since it is the specific interaction of Differential Object Marking – the alternation between overt accusative \isi{case marking} vs. incorporation of possessed nouns – and Differential Object Indexing – the complementary distribution between object marking and object indexing, that led to the reanalysis of possessor indexes as P indexes, and more specifically, to the reanalysis of possessor prefixes as P indexes interposed within a simplex verb stem. 

\section*{Abbreviations}
\begin{tabularx}{.45\textwidth}{lQ}
1 & first person\\
2 & second person\\
3 & third person\\
\textsc{a} & agent-like argument of canonical transitive verb\\
\textsc{acc}	& accusative\\
\textsc{aor} & aorist (habitual verb form)\\
\textsc{art} & article\\
\textsc{bg} & backgrounder, prefix that marks the verb as topical/an adjunct as focal\\
\textsc{cnvb} & converb\\
\textsc{dat} & dative\\
\textsc{def} & definite\\
\textsc{dem} & demonstrative\\
\textsc{det} & determiner\\
\textsc{exist} & existential\\
\textsc{f} & feminine\\
\textsc{foc} & focus\\
\textsc{gen}	& genitive\\
\textsc{indef} & indefinite\\
\textsc{inf} & infinitive\\
\end{tabularx}
\begin{tabularx}{.45\textwidth}{lQ}
\textsc{int} & intransitive\\
\textsc{impf} & imperfect\\
\textsc{juss} & jussive\\
\textsc{loc} & locative\\
\textsc{m} & masculine\\
\textsc{mod} & modifier marker\\
\textsc{neg} & negation, negative\\
\textsc{nom} & nominative\\
\textsc{p} & patient-like argument of canonical transitive verb\\
\textsc{pl} & plural\\
\textsc{poss} & possessive\\
\textsc{prs} & present\\
\textsc{proh} & prohibitive\\
\textsc{pst} & past\\
\textsc{ptcl} & particle\\
\textsc{ptcp} & participle\\
\textsc{q} & question particle/marker\\
\textsc{rel} & relative\\
\textsc{sg} & singular\\
\textsc{seq} & sequential verb form\\
\textsc{stat} & stative verb form\\
\end{tabularx}


 
{\sloppy
\printbibliography[heading=subbibliography,notkeyword=this] }
\end{document}
