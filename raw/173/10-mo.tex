%TODO ask the author
\ChapterDOI{10.5281/zenodo.1228261} 
% The author lists 
%Mohanan, Tara. 1994. Argument Structure in Hindi. CSLI Publications.
%Mohanan, Tara. 1994. Case OCP: a constraint on Word Order in Hindi. In Theoretical Perspectives on Word Order in Indian Languages. Butt, Mariam, King, Tracy & Ramchand, Gilian, 185-216. Stanford: CSLI.
% Unclear which Mohanan 1994 is refered to in the text
\documentclass[output=paper]{LSP/langsci}
\author{Annie Montaut\affiliation{INALCO, Paris (SEDYL UMR 8002, CNRS/INALCO/IRD), Labex 083 (Empirical Foundations of Linguistics)}}
\title{The rise of differential object marking in Hindi and related languages}
%\epigram{Change epigram in chapters/03.tex or remove it there}
\abstract{Differential object marking (DOM), which involves a contrast between zero marking and accusative marking by means of an originally dative postposition, appeared in Indo-Aryan languages only a few centuries ago as opposed to Dravidian languages which had it right from the earlier attested stage (1\textsuperscript{st} century) and have a specific accusative marker. Hindi like other Indo-Aryan languages uses the dative postposition to mark this specific accusative, a postposition which appeared at around the same period for marking experiencers. It is now required with human objects with very few exceptions, and optional with inanimate objects even when definite and individuated. But the historical evolution of the marking shows that the prevalence of animacy over definiteness is quite recent. The paper is an attempt to find explanations for this evolution, which only partly corresponds to the scenario put forward by \citet{Aissen2003Differential}, according to which the obligatoriness of marking develops by extension from an initial kernel of marked objects. The paper will first analyze the properties and range of DOM in Modern Standard Hindi (semantic, discourse related, particularly topic related, and syntactic ones; \sectref{10-mo-sec:2} and \sectref{10-mo-sec:3}), a fairly well explored topic. I will then inquire into the historical emergence of DOM (\sectref{10-mo-sec:4}), and its presence in non-standard varieties or “dialects” (\sectref{10-mo-sec:5}), both topics far less studied. Finally it will suggest some hypotheses on the emergence and grammaticalization of the marked accusative in Hindi and related dialects (\sectref{10-mo-sec:6}). 
}
% \keywords{animacy – specificity - obligatoriness – individuation - Hindi dialects}
\maketitle

\begin{document}
\section{Introduction}
\label{10-mo-sec:1}

Differential object marking (DOM), which involves a contrast between zero marking and accusative marking by means of an originally dative postposition, is a relatively new phenomenon in \ili{Indo-Aryan} languages \citep{Masica1982Identified} as is the rise of dative \isi{experiencer} subjects, both expressed with the dative marker. This contrasts with \ili{Dravidian} languages where DOM is attested since the earliest texts, with a specific accusative marker. It is obligatory in \ili{Hindi} only with human individuated objects, and optional with \isi{inanimate} objects even when individuated. However, an inquiry in the historical evolution of the marking shows that the supposed prevalence of \isi{animacy} over \isi{definiteness} is quite recent. The aim of this paper is to attempt to find explanations for this evolution, which only partly corresponds to the scenario put forward by \citet{Aissen2003Differential}, according to which the obligatoriness of marking develops by extension from an initial kernel of marked objects. The paper will first analyze the properties and range of DOM in Modern Standard \ili{Hindi} (semantic, discourse related, particularly topic related, and syntactic ones; \sectref{10-mo-sec:2} and \sectref{10-mo-sec:3}), before looking at the historical emergence of DOM (\sectref{10-mo-sec:4}) and its presence in non-standard varieties or “dialects” (\sectref{10-mo-sec:5}), and suggesting some hypotheses on its emergence and \isi{grammaticalization} (\sectref{10-mo-sec:6}). 

\section{Basic facts in modern Hindi DOM}
\label{10-mo-sec:2}

DOM is largely grammaticalized in Modern Standard \ili{Hindi}, where identified objects are both case marked and can trigger a change in verb agreement: in ergative constructions,\footnote{\ili{Hindi} is a language with (aspectually) split ergativity: \textit{lar̩ke ne film dekhī} [boy.\textsc{m.sg.obl erg} film.\textsc{f.sg} see.\textsc{f.sg}] ‘The boy saw the film’ vs. \textit{lar̩kā āyā} [boy.\textsc{m.sg} come.\textsc{3sg}] ‘The boy came’, \textit{lar̩kā āegā} [boy.\textsc{m.sg} come.\textsc{fut.3m.sg}] ‘The boy will come’. Examples are from everyday exchanges or my own when not otherwise indicated.} as well as in passive constructions, the verb agrees with an unmarked patient, but not with a marked patient. DOM is constrained first by the semantic or inherent properties of the argument (obligatory overt marking), and secondarily by discourse related properties (optional marking). DOM occurs only with formally transitive verbs and formal \isi{transitivity} is found only with verbs high on the \isi{transitivity} hierarchy \citep{Hopperetal1980Transitivity,Tsunoda1985Remarks}, involving a binary relation between real agent and real patient. It follows that DOM occurs only with typical agents. In turn, marked objects are more sensible to \isi{topicality} \citep{Dalrympleetal2011Objects} than to, as suggested by \citet{Naess2004What}, \isi{affectedness}. As for what is often analyzed as syntactic properties of marked objects, they ultimately can also be accounted for in terms of discourse related properties, such as \isi{topicality} or saliency. 

\subsection{Morphological properties: flagging and indexation}
\label{10-mo-sec:2-1}

The case marker is the postposition \textit{ko} (suffixed to pronouns), the same which is also used for beneficiaries or experiencers, a syncretic case for dative/accusative. Example \REF{10-mo-ex:1a} illustrates the obligatory marking of human objects (particularly proper nouns and personal pronouns) with no effect on agreement in the present, whereas \REF{10-mo-ex:1b} illustrates the same marking with a verb showing default agreement (masculine singular) in ergative constructions (past transitive clauses), and in the non-promotional passive \REF{10-mo-ex:1c}. 
The contrast between agreement with unmarked objects \REF{10-mo-ex:2b} and default agreement \REF{10-mo-ex:2a} is found with \isi{inanimate} objects:

\protectedex{
\ea Modern Standard Hindi (own data) \label{10-mo-ex:1}

\ea \label{10-mo-ex:1a}
\gll {maĩ} \textbf{tumko} / \textbf{Rām} \textbf{ko} / apnī \textbf{bet̩ī} \textbf{ko} dekh rahā hū̃\\
\textsc{1sg} \textsc{2.acc} { } Ram \textsc{acc} { } \textsc{refl} daughter \textsc{acc} see \textsc{prog.m.sg} \textsc{prs.1sg}\\
\glt ‘I am looking at you /Ram / my daughter.’


\ex \label{10-mo-ex:1b}
\gll {maĩne} \textbf{tumko} / \textbf{Rām} \textbf{ko} / apnī \textbf{bet̩ī} \textbf{ko} kal 	nahī̃ dekhā\\
\textsc{1sg.erg} \textsc{2.acc} { } Ram \textsc{acc}  { }  \textsc{refl} daughter \textsc{acc} yesterday \textsc{neg} see.\textsc{pfv.3m.sg}\\
\glt ‘I did not see you /Ram / my daughter yesterday.’

\ex \label{10-mo-ex:1c}
\gll {donõ} \textbf{ādmiyõ} \textbf{ko} dekhā gayā\\
 the.two man.\textsc{m.pl} \textsc{acc} see \textsc{pass.pst.m.sg}\\
\glt ‘Both men were seen’
\z
\z
}
\ea Modern Standard Hindi (own data)  \label{10-mo-ex:2}
\ea  \label{10-mo-ex:2a}
\gll {maĩne} \textbf{is} \textbf{film} \textbf{ko} dekhā\\
\textsc{1sg.erg} \textsc{dem} movie.\textsc{f.sg} \textsc{acc} see.\textsc{pfv.3m.sg}\\

\ex \label{10-mo-ex:2b}
\gll {maĩne}  \textbf{yah} \textbf{film} dekhī\\
\textsc{1sg.erg}  \textsc{dem} movie.\textsc{f.sg} see.\textsc{pfv.3f.sg}\\
\glt `I have seen this film.'
\z
\z

\subsection{Type of arguments: Animacy, definiteness, specificity}
\label{10-mo-sec:2-2}

Since the role played by the semantics of the verb as suggested in \citet[81]{Mohanan1994Argument} can be seriously questioned (cf., \textit{inter alia}, \citealt{Self2012Differential} for an overview), and given the limitations of this study, it will not be treated here.

As in many languages, the \isi{animacy} (human > \isi{animate} > \isi{inanimate}) and \isi{definiteness} scales, into which specificity can be integrated \citep[132]{Croft2003Typology} (Personal pronoun / Proper name > Definite NP > Indefinite specific NP > Non-specific NP) overlap, with an apparent prevalence of \isi{animacy}: \REF{10-mo-ex:3a} with an \isi{indefinite} human object is obligatorily marked, and so are proper nouns referring to human objects, in contrast with those referring to \isi{inanimate} objects \REF{10-mo-ex:3b}. 
Pronominalized \isi{inanimate} objects are more often marked than the corresponding nouns \REF{10-mo-ex:3c}. Example \REF{10-mo-ex:3d} shows that the pronominalization of ‘the note’ does trigger the accusative marking, whereas the same noun (‘the note’) occurs thereafter in the unmarked form:

\ea Modern Standard Hindi (own data)\label{10-mo-ex:3}

\ea \label{10-mo-ex:3a}
\gll {kisī} \textbf{ko} bulāo!\\
\textsc{indef} \textsc{acc} call.\textsc{imp}\\
\glt ‘Call somebody!’

\ex \label{10-mo-ex:3b}
\gll mai͂ne Dilīp ko (*Dilīp) dekhā / mai͂ne Kalkattā dekhā\\
\textsc{1sg.erg} Dilip \textsc{acc} (Dilip) see.\textsc{pfv} { } \textsc{1sg.erg} Calcutta see.\textsc{pfv}\\
\glt ‘I saw Dilip / I saw Calcutta’

\ex \label{10-mo-ex:3c}
\gll koī pitā bhī \textbf{ise} (is bāt ko / yah bāt) bardaśt nahī̃  kar saktā\\
\textsc{indef} father even \textsc{3sg.acc} this thing \textsc{acc} { }  this thing tolerate \textsc{neg} do can.\textsc{m.sg}\\
\glt ‘No father at all could tolerate this (this thing).’

\ex \label{10-mo-ex:3d}
\gll jeb se do rupae kā noṭ nikālā “Jivrākhan, \textbf{ise} rakh-lo”  Jivrākhan ne not̩ mānī beg mẽ rakh-liyā\\
 pocket from two rupies of note took.out Jivrakhan \textsc{3sg.acc} put-take.\textsc{imp} Jivrakhan \textsc{erg} note money bag in place-took\\
\glt ‘He took a two rupee note out of his pocket, “Jivrakhan, take it”. Jivrakhan put the note into his purse.’
\z
\z

Animacy seems at first sight to be the prevalent trigger for accusative marking, while \isi{definiteness} and specificity seem to act as an optional trigger only, as summarized in \citet[469]{Aissen2003Differential} on the basis of the dominant view in \ili{Hindi} linguistics. However, the deviant cases can be better explained in terms of specificity or saliency as will be argued below.


\subsubsection{Deranking}\label{10-mo-sec:2-2-1}

Human animates can, exceptionally, remain unmarked, a case of “deranking” in \citeauthor{Aissen2003Differential}’s \citeyear{Aissen2003Differential} terms: for example, variation is found with NPs that are used to refer to the function their referents are associated with, and not to the respective individuals \REF{10-mo-ex:4a}--\REF{10-mo-ex:4b}, NPs with collective reference \REF{10-mo-ex:5a}--\REF{10-mo-ex:5b}, and NPs used in comparisons decreasing the \isi{referentiality} of the NP \REF{10-mo-ex:6a}:

\ea Modern Standard Hindi (own data)  \label{10-mo-ex:4}
\ea \label{10-mo-ex:4a}
\gll merī sahelī ne nayā naukar rakhā\\
 my friend.\textsc{f.sg} \textsc{erg} new servant.\textsc{m.sg} place.\textsc{pfv.m.sg}\\
\glt ‘My friend took a new servant.’

\ex  \label{10-mo-ex:4b}
\gll ve laṛkā dekh rahe haĩ\\
 \textsc{3pl} boy.\textsc{m.sg} look \textsc{prog} \textsc{prs.3m.pl}\\
\glt ‘They are visiting a boy (a suitable groom).’
\z
\z

\ea Modern Standard Hindi (own data)  \label{10-mo-ex:5}
\ea \label{10-mo-ex:5a}
\gll maĩne bahut log dekhe, bahut gār̩iyā̃ 	dekhī̃,  bahut gandagī dekhī\\
\textsc{1sg.erg} many people.\textsc{m.pl} see.\textsc{m.pl} many car.\textsc{f.pl} see.\textsc{f.pl} much dirt.\textsc{f.sg} see.\textsc{f.sg}\\
\glt ‘I saw a lot of people, a lot of cars, much dirt.’

\ex  \label{10-mo-ex:5b}
\gll maĩne bahut \textbf{logõ} \textbf{ko} dekhā\\
\textsc{1sg.erg} many people.\textsc{m.pl} \textsc{acc} see.\textsc{pfv.3m.sg}\\
\glt ‘I saw many people.’ 
\z
\z

\ea Modern Standard \ili{Hindi} (own data) 
\label{10-mo-ex:6}
\ea  \label{10-mo-ex:6a}
\gll tum jaisā koī nahī dekhā\\
 2 like.\textsc{m.sg} \textsc{indef} \textsc{neg} saw.\textsc{pfv.m.sg}\\
\glt ‘I didn’t see anybody like you’ (movie title)

\ex  \label{10-mo-ex:6b}
\gll mai͂ne kisī ko nahī͂ dekhā\\
 \textsc{1sg.erg} \textsc{indef} \textsc{acc} \textsc{neg} saw\\
\glt ‘I didn’t see anybody.’
\z
\z

Examples such as \REF{10-mo-ex:4} have been well discussed in the literature (\citealt{Mohanan1994Argument}; \citealt{Dayal2011Hindi}) and are analyzed in \citet{Self2012Differential} as an illustration of what he calls the specificity requirement, which, according to him, may be the main and only constraint. This constraint requires the object NP to be specific in order for it to be marked. Examples such as \REF{10-mo-ex:5} and \REF{10-mo-ex:6} are less frequently discussed, but also show that human non-specific objects can be unmarked, when they involve a collectivity considered as an indivisible whole \REF{10-mo-ex:5b} rather than a set of individuals \REF{10-mo-ex:5a} or decrease in \isi{referentiality} by a comparison in a negative context \REF{10-mo-ex:6a}.


\subsubsection{Upranking}\label{10-mo-sec:2-2-2}
Certain inanimates and abstract nouns in the object position are very frequently marked: this type of upranked objects have been noted for nouns with unique referents such as ‘moon’, ‘sun’, ‘earth’ or ‘ocean’, whose reference can be identified on the basis of shared knowledge. Abstract nouns such as ‘death’ or ‘time’, which belong to a different class and are not referential, are in fact quite frequently marked:

\ea Modern Standard Hindi (own data)\\ %TODO add source  \label{10-mo-ex:7}
\gll cā̃d  \textbf{ko} dekho!\\
 time \textsc{acc} look\textsc{.imp}\\
\glt ‘Look at the moon!’
\z 


\ea Modern Standard \ili{Hindi} (Agyeya, 1951, \emph{Nadi ke dvip})\\ %TODO add source
\gll ham kyā samay \textbf{ko} / mrityu \textbf{ko} rok sakte haĩ?\\
\textsc{1pl} \textsc{int} time \textsc{acc} { } death \textsc{acc} stop can \textsc{prs.1m.pl}\\
\glt ‘Can we stop time, death?’
\z


In \ili{Spanish} abstract nouns are far more often marked than concrete inanimates, since 79\% occur with the preposition \textit{a,} whereas only 21\% concrete inanimates occur with the preposition \textit{a} \citep[209]{Company2002Grammaticalization}. In \ili{Hindi}, non-referential abstract nouns can be marked, such as ‘glass’, ‘darkness’, ‘outside’:

\ea Modern Standard Hindi (\citealt{Self2012Differential}, from \citealt{Burton1957Compound})  \label{10-mo-ex:8}

\gll lohā sīse ko kāt̩tā hai\\
 iron glass \textsc{acc} cut \textsc{prs.3m.sg}\\
\glt ’Iron cuts glass.’
\z

\ea Modern Standard \ili{Hindi} (Vinod Kumar Shukla, 1996, \emph{Khilega to dekhenge}) %TODO add source
\label{10-mo-ex:9}
\ea 
\gll ham \textbf{andhere} \textbf{ko} rok dete\\
\textsc{1pl} darkness \textsc{acc} stop give\textsc{.cond.1m.pl}\\
\glt ‘We would stop the darkness.’

\ex
\gll hamne \textbf{pūre} \textbf{bāhar} \textbf{ko} band kar diyā hai\\
\textsc{1pl.erg} whole outside \textsc{acc} closed make give \textsc{prf.3m.sg}\\
\glt ‘We have locked up all the outside.’ 
\z
\z

One might think that the whole series displays nouns like mass nouns such as ‘glass’ in~\REF{10-mo-ex:8}, which are according to \citet{Self2012Differential} similar to natural kind terms, and natural kind terms may have the properties of \isi{definite} NPs \citep{Gross2009Natural}. However, the fact that they are more often marked than other inanimates (as in \ili{Spanish}), which are marked only when specific, both in \ili{Hindi} and \ili{Spanish}, requires a different explanation. The reason, not explored to my knowledge, maybe because such abstract nouns, with semantic rigidity, are not liable to variations of \isi{definiteness}/specificity – except when they change status and become discrete (‘a specific blue’, ‘the very same sadness’) they tend to be marked for their semantic rigidity. Hypotheses along these lines should be checked in a distinct study.

\subsection{Syntactic properties of the object with attribute}
\label{10-mo-sec:2-3}

It has been argued that marked objects have differential control properties: no unmarked object can control a non-finite adjunct \citep{Bhatt2007Unaccusativity}, whereas propositional adjuncts are commonly controlled by marked objects, particularly after main verbs of perception. \citeauthor{Bhatt2007Unaccusativity}’s \citeyear{Bhatt2007Unaccusativity} examples are the following: 

\ea Modern Standard Hindi \citep[17]{Bhatt2007Unaccusativity} \label{10-mo-ex:10}
\ea \label{10-mo-ex:10a}
\gll Minā\textsubscript{i} ne bāzār mẽ ek sailānī\textsubscript{j} ko nācte hue\textsubscript{j} dekhā.\\
 Mina \textsc{erg} market in a tourist \textsc{acc} dancing being see.\textsc{pfv}\\
\glt ‘Mina\textsubscript{i} saw a tourist\textsubscript{j} dancing\textsubscript{j} in the bazar.’ 

\ex \label{10-mo-ex:10b}
\gll Minā\textsubscript{i} ne bāzār mẽ ek sailānī\textsubscript{j} nācte hue\textsubscript{ i/*j} dekhā.\\
 Mina \textsc{erg} market in a tourist dancing being see.\textsc{pfv}\\
\glt ‘In the market Mina\textsubscript{i} saw a tourist\textsubscript{j} when she\textsubscript{i/*j} was dancing.’ (*a tourist dancing) 
\z
\z

According to \citeauthor{Bhatt2007Unaccusativity} \citet{Bhatt2007Unaccusativity}, the non-finite adjunct ‘dancing’ in~\REF{10-mo-ex:10b} can only be controlled by the subject of the matrix clause \textit{Mina}, not by the unmarked object, whereas the same, when marked, controls the adjunct. However, unmarked objects are commonly used with an adjunct that they control, although they are in this case typically \isi{inanimate}. In \REF{10-mo-ex:11a}, the implicit subject of the participial clause ‘having come back / be back’ (state, past) is controlled by the unmarked object \textit{gāṛī} ‘car’, and in~\REF{10-mo-ex:11b},\footnotetext{The \isi{complex predicate} \textit{gāyab honā} ‘to disappear’ is formed with the adjectival unit \textit{gāyab} and light verb ho ‘be’, here in the present participle form.} the participial clause (dynamic event, present) is controlled by the unmarked object \textit{jāmun} ‘Java plums’. Both sentences involve a coverb, whose subject is controlled by the main verb’s subject, and the same control rule within the participial clause apply as in~\REF{10-mo-ex:10a}:

\ea Modern Standard \ili{Hindi} (Krishna Baldev Vaid, \emph{Dusra na koi}) %TODO add source
\label{10-mo-ex:11}

\ea \label{10-mo-ex:11a}
\gll {gār̩ī vāpas āī huī dekhkar maĩne socā…}\\
 car.\textsc{f.sg} back come be.\textsc{ptcp.f.sg} see.\textsc{cv} \textsc{1sg.erg} think.\textsc{pfv.m.sg}\\
\glt ‘I saw the car having come back and thought…’ 
 (not `Having come back/I came back and I saw the car.') 

\ex \label{10-mo-ex:11b}
\gll kāle-kāle jāmun gāyab hote dekhkar  uske mũh se zāl t̩apakne lagī\\
 black-black jamun vanished being see.\textsc{cv} his mouth from salive.\textsc{f}\textsc{.sg} drip start.\textsc{pfv.f.sg}\\
\glt ‘Seeing the black Java plums disappearing his mouth started watering.’/ ‘He saw the black Java plums disappearing and he started salivating’ 
\z
\z

Both participles \textit{āī huī} past participle of verb \textit{ānā} ‘to come’ in~\REF{10-mo-ex:11a} and \textit{gāyab hote}, present participle of verb \textit{gāyab honā} ‘to disappear’, are clearly controlled by the object of the coverb. In other words, a small clause complement of a matrix verb may license an unmarked noun only if it is \isi{inanimate} and accompanied by an attributive participle, as in~\REF{10-mo-ex:11}, not when the noun is \isi{animate}.

The differential behavior of \REF{10-mo-ex:10b} and \REF{10-mo-ex:11}, both with unmarked object, can be explained by the fact that in~\REF{10-mo-ex:10b} the unmarked object is a human being in the singular, which makes its unmarkedness highly atypical: a tourist in the market, as an unmarked human patient, must be totally devoid of individuation (like ‘people’ in example \ref{10-mo-ex:5a}), treated as a mere element of the bazaar. Therefore. its individuation by means of a striking event (dancing in the bazaar) contradicts its implicit characterization as non salient. The ‘car’ or the ‘Java plums’ in~\REF{10-mo-ex:11} in contrast are \isi{definite} inanimates, but their unmarkedness conforms to the tendency for inanimates to remain unmarked if devoid of \isi{discourse prominence} (\cf below). What is centre-staged in~\REF{10-mo-ex:11} is not the entity (‘plum’ or ‘car’) but the global scenario of the disappearance or re-appearance respectively. The objects are not described for their own sake since what prevails for the speaker is the event in which the object is involved, not the object itself.

Similar reasons account for the systematic marking of all objects with nominal or adjectival attributes, whether human or \isi{inanimate} and non-specific, a fact which remains unnoticed in the literature on \ili{Hindi} DOM. The following series \REF{10-mo-ex:12} involves verbs with two objects such as ‘judge’ / ‘consider’ / ‘call’ / ‘make’ (X Y), a main object and its attribute:

\ea \ili{Hindi} (own data) 
\label{10-mo-ex:12}
\ea
\gll maĩ cor \textbf{ko} / *Ø cor kahtā hū̃\\
\textsc{1sg} thief \textsc{acc} { } { } thief say \textsc{prs.1sg}\\
\glt ‘I call a thief a thief.’

\ex
\gll maĩ billī ko apnā duśman / beimān māntā hū̃\\
\textsc{1sg} cat \textsc{acc} \textsc{refl} enemy { } disloyal consider \textsc{prs.1sg}\\
\glt ‘I consider cats as my personal enemies/disloyal.’

\ex
\gll ve rassī \textbf{ko} / *Ø sā̃p	 samajhte haĩ\\
\textsc{3pl} rope \textsc{acc} { } { }  snake understand \textsc{prs.m.pl}\\
\glt ‘They mistake a rope for a snake’ (or {‘}ropes for snakes’).

\ex
\gll ve pun̩ya \textbf{ko} / *Ø pāp banāte haĩ\\
\textsc{3pl} virtue \textsc{acc} { } { } sin make \textsc{prs.m.pl}\\
\glt ‘They transform virtue into sin{.’}
\z
\z

The marking is obligatory even for non-specific \isi{indefinite} \isi{inanimate} objects. Here the attributive adjunct, noun or adjective, does not describe an event in which the object could in principle be a simple element less salient than the process itself as in~\REF{10-mo-ex:11}, where the adjunct is a mere qualification. The sentence amounts to attributing a property to the noun, and this attribution itself makes the noun centre-staged and not secondary to the property or part of it.

\subsection{Information structure}
\label{10-mo-sec:2-4}

The above examples \REF{10-mo-ex:10}--\REF{10-mo-ex:12} corroborate a major principle of differential object marking that \citet{Dalrympleetal2011Objects} as well as \citet{Iemmolo2010Topicality} have captured with the relevance of \isi{information structure} and the notion of \isi{topicality} \citep{Iemmolo2010Topicality} or secondary \isi{topicality} \citep{Dalrympleetal2011Objects}. The syntactic properties analyzed in \sectref{10-mo-sec:2-3} are in conformity with a more general tendency which holds also in the absence of syntactic constraints. \citet{Dalrympleetal2011Objects} assume that topical objects are marked, while narrow focused objects – even if \isi{definite} specific – are obligatorily unmarked, giving the following \ili{Hindi} example:\footnote{Wide focused objects are preferably unmarked, narrow focused objects are obligatorily unmarked as opposed to topicalized objects, which are marked. For a definition of wide \textit{vs.} narrow focus, see \citet[215]{Rebuschietal1999Grammar}. Wide focus sentences felicitously answer “out of the blue” questions such as “What happened?”, whereas in narrow focus at least one of the participants is given or known, such as “What did X do?, What did X do with Y?”.} 

\ea \ili{Hindi} \citep[167]{Dalrympleetal2011Objects}
\label{10-mo-ex:13} 

\ea \label{10-mo-ex:13a} 
\gll ham mez paũchẽge\\
\textsc{1pl} table wipe.\textsc{fut.m.pl}\\

\ex \label{10-mo-ex:13b} 
\gll ham mez ko paũchẽge\\
1pl table \textsc{acc} wipe.\textsc{fut.m.pl}\\
\glt ‘We will wipe the table.’ 
\z
\z

In \REF{10-mo-ex:13a} “the object is construed as part of the event and is not individuated as a pragmatically salient element: informationally, it is part of wide focus” \citep[167]{Dalrympleetal2011Objects}, whereas in~\REF{10-mo-ex:13b} the ‘table’ was already the centre of attention.

However, topics can remain unmarked in \ili{Hindi}, either by simple fronting \REF{10-mo-ex:14a} or fronting with topic particle \REF{10-mo-ex:14b}, which suggests that \isi{topicality}, whether secondary or primary, is not in itself responsible for the marking of objects.

\ea \ili{Hindi} (own data) 
\label{10-mo-ex:14}
\ea \label{10-mo-ex:14a}
\gll yah film kisne dekhī?\\
 this film.\textsc{f.sg} who.\textsc{erg} see.\textsc{pfv.f.sg}\\
\glt ‘This film, who saw it?’

\ex 
\label{10-mo-ex:14b}
\gll  yah bāt to ham sab jānte haĩ\\
 this thing \textsc{top} \textsc{1pl} all know \textsc{prs.pl}\\
\glt ‘This thing, we all know it.’
\z
\z

Besides, internal objects, which are, by nature, very low in \isi{topicality}, may be marked and statements such as \REF{10-mo-ex:15} are in no way exceptional: 

\ea  \ili{Hindi} (Vinod Kumar Shukla, 1996, \emph{Khilega to dekhenge})\\ %TODO add source
\label{10-mo-ex:15}
\gll  zindagī \textbf{ko} jīnā sthagit maut \textbf{ko} jīnā hai\\
 life \textsc{acc} live postponed death \textsc{acc} live is\\
\glt ‘To live life is to live a deferred death.’
\z

The reason why some topics remain unmarked whereas some internal objects are marked is, again, related to how the speaker wishes to represent the situation involving the object: even a topicalized object may be deprived of saliency in comparison with the process that it is part of (knowing in~\REF{10-mo-ex:14b}) or with the focus in~\REF{10-mo-ex:14a}, and thus can remain unmarked, since it is the event or the focus, and not the object, that is discursively salient. An initial sentence like \REF{10-mo-ex:14b} can be followed by a proposition discussing its whole content (“but we don’t care”), but not bearing on the topicalized notion (“this thing is most important or interesting”).\footnote{For instance \textit{phir bhī log is saccāī se dūr bhāgte hai͂} ‘however people run away from this truth’ [that \textit{jisne is dharti par janm liyā hai use mrityu prāpt hogī} ‘whoever was born on this earth will die’] (Bollywoodtadka). A continuation bearing on the topicalized notion requires an initial sentence with a marked object (\textit{is bāt ko}). One may hypothesize that both sentences in~\REF{10-mo-ex:14} have a focused constituent, which makes \isi{topicality} less prominent.} In contrast, internal objects, if emphasized for the purpose of parallel contrast as is the case in~\REF{10-mo-ex:15}, acquire sufficient saliency to be marked: this is not really life that we are living, it is rather like living death. Semantically the added meaning to ‘life’ is its opposite (‘death’), hence the marking. Without marking, the object comes back to its ordinary status as an internal, non-individuated object, which is part of a process from which it cannot be dissociated.

In a discourse with no particular constraints, the same reasons account for the marking of the vast class of optionally marked \isi{inanimate} objects. In \REF{10-mo-ex:16} for instance, the same object ‘door’ occurs first as marked and then as unmarked, although the first occurrence refers to an \isi{indefinite}, and the second has more specifying properties since it does not refer to just any ‘door’, but to ‘our own’ door. 

\ea  \ili{Hindi} (Vinod Kumar Shukla, 1996, \emph{Khilega to dekhenge})\\
\label{10-mo-ex:16}
\gll  \textbf{ek} \textbf{darvāze} \textbf{ko} band kar, hamne pūre bāhar ko band kar diyā hai. … apne kamre kā darvāzā band kar sārī duniyā ko bāhar band kar diyā.\\
 one door \textsc{acc} close do.\textsc{cv} \textsc{1pl} all outside \textsc{acc} close do %\textsc{prf.3m.sg}  
 give.\textsc{pfv.3m.sg}  \textsc{prt} { } \textsc{refl}  room of door closed do.\textsc{cv} all world \textsc{acc} outside closed do give.\textsc{pfv.3m.sg}\\
\glt ‘By closing a (mere) door, we have locked up the whole outside. (…) By closing the door of our room, we have locked outside the whole world.’
\z

The door in the first sequence, although appearing as new information and not specific, is singled out as responsible for huge consequences, in contrast with its triviality: hence the marking. In the second occurrence, this disparity is already given, and it is the event as a whole (to lock oneself in one’s room) that is emphasized: hence the absence of marking. 

In \REF{10-mo-ex:17}, this object is already present in the anterior context, where the village head asked the master, Guruji, to open a lock on a door. In \REF{10-mo-ex:17a}, lock, the object, is topicalized by its position and it is \isi{definite}, however it is not marked: what is emphasized is the inference of the speaker’s ability of the speaker to do the unlocking, since he had locked the door himself. Besides, the subject is focalized (preverbal position). In contrast, in the very next sequence, the same lock, again in a topic position \REF{10-mo-ex:17b}, is given centre stage because the protagonist is confronting it for itself (testing its solidity), and since, in segment \REF{10-mo-ex:17c} as well, the process singles out the lock (and key) as the centre of everybody’s attention, although it is non-topicalized. When the protagonist goes to open the lock, everybody’s attention shifts from the lock to the process of opening the lock:\footnote{As confirmed by his wife’s insistence on the act of opening, totally backgrounding (omitting) the object: {\textit{bahār}}{} ‘\textit{khar̩ī uskī strī ne kahā ‘maĩ khol d\~{ū}?’} ‘His wife, who stood outside said ‘Shall I open it (myself)?’}

\ea Hindi (Vinod Kumar Shukla, 1996, \emph{Khilega to dekhenge}) \label{10-mo-ex:17}\\
\ea \label{10-mo-ex:17a}
\gll  ‘Yah tālā maĩne xud lagāya hai’,\\
this lock \textsc{1sg.erg} \textsc{refl} put.\textsc{prf} \textsc{3m.sg}\\
\glt ‘This lock, I put it myself,’

\ex \label{10-mo-ex:17b}
\gll \textbf{tāle} \textbf{ko} Gurūjī ne jhanjhanāyā. (…)\\
lock \textsc{acc} Guruji \textsc{erg} shake.\textsc{pfv.3m.sg}\\
\glt `Guruji shaked \textbf{the} \textbf{lock} noisily.'

\ex \label{10-mo-ex:17c}
\gll ‘Maĩ \textbf{tāle} \textbf{ko} khol saktā 	hū̃, cābī mere pās hai’.\\
 \textsc{1sg} lock \textsc{acc} open can \textsc{prs.1m.sg} key \textsc{1sg} near be.\textsc{prs.3sg}\\
\glt ‘I can open \textbf{the} \textbf{lock}, I have the key with me’.

\ex \label{10-mo-ex:17d}
\gll Unhõne cābī t̩ẽt̩ se nikālī. Ve \textbf{tālā} kholne jā rahe the.\\
 3.\textsc{hon.erg} key belt \textsc{abl} take.out.\textsc{pfv.f.sg} \textsc{3hon} lock open go \textsc{prog} \textsc{pst.3hon}\\
\glt `He took the key from his belt. He was going to open the lock.’
\z
\z

What such examples highlight with marked objects is their saliency (\citealt[155]{Croft1991Syntactic}; \citealt{Haudeetal2012Saillance}), a notion I am invoking in the sense of \citet[14–15, 57]{Dalrympleetal2011Objects} on the role played by a referent in the pragmatic structure of the proposition, rather than \citeauthor{Naess2004What}’s (\citeyear{Naess2004What}) more general interpretation of the term (which focuses on the question as to which entities are of greater interest for human perception in general).


\section{Particular clause types in Hindi} \label{10-mo-sec:3}

\subsection{The case of non-promotional passive} \label{10-mo-sec:3.1}

A characteristic of the \ili{Hindi} passive, apart from the fact that it applies equally to intransitives, is that it is very frequently non-promotional, and retains the \isi{object marker} \textit{ko} for the noun which is the corresponding object in the equivalent \isi{active} clause, with the result of blocking the agreement (\cf example~\REF{10-mo-ex:1c} above). The conditions for marking the ex-object are not the same as those form marking the object in an \isi{active} sentence and an attempt is made below to define them better. Given the fact that promotional passive is also frequent, and consequently marked objects in the passive are less frequent than in the \isi{active}, one would expect that the obligatorily marked objects of an \isi{active} sentence such as a human referential object is better retained in the passive sentence than \isi{inanimate} objects, which are only optionally marked in the \isi{active} sentence.\footnote{In keeping with \citegen[468]{Aissen2003Differential} “basic hypothesis: if overt marking is possible with direct objects with property $\alpha $, then it is possible with direct objects with property $\beta $, where $\beta $ dominates $\alpha $”.} But this is not the case. Unmarked human patients which are absolutely compulsory in \isi{active} sentences, such as first person pronouns \REF{10-mo-ex:18b} or proper nouns \REF{10-mo-ex:19}, are quite frequent, as are marked inanimates in~\REF{10-mo-ex:20} and \REF{10-mo-ex:21}:

\ea Modern Standard \ili{Hindi} (own data)
\label{10-mo-ex:18}

\ea \label{10-mo-ex:18a}
\gll \textbf{mujhe} aspat̩āl le jāyā gayā\\
\textsc{1sg.acc/dat} hospital take go \textsc{pass.pfv.3m.sg}\\
\glt ‘I was taken to the hospital.’

\ex \label{10-mo-ex:18b}
\gll \textbf{maĩ} aspat̩āl le jāyī gayī\\
\textsc{1sg} hospital take go \textsc{pass.pfv.f.sg}\\
\glt ‘I was taken to the hospital.’ (feminine speaker)
\z
\z

\protectedex{
\ea Modern Standard \ili{Hindi} (\textit{Times of India}, January 2013)\\
\label{10-mo-ex:19}
\gll śef Hemant Oberāy apne das sahyogiyõ kesāth vahā̃ bheje gae the\\
 Chef Hemant Oberoy \textsc{refl} ten helper.\textsc{m.pl} with there send \textsc{pass} \textsc{pprf.m.pl}\\
\glt ‘The chef Hemant Oberoi had been sent there with ten of his helpers.’ 
\z
}

\ea Modern Standard \ili{Hindi} (\textit{Times of India}, January 2013)\\
\label{10-mo-ex:20}
 \gll \textbf{mere} \textbf{hazārõ} \textbf{samarthakõ} \textbf{ko} Madurai ikāī se nikāl diyā gayā hai\\
 my thousand supporter.\textsc{m.pl} \textsc{acc} Madurai unit from expel give \textsc{pass} \textsc{prf.m.sg}\\
\glt ‘Thousands of my supporters have been ousted from the Madurai unit.’
\z


\ea Modern Standard \ili{Hindi}
\label{10-mo-ex:21}
\ea
\gll \textbf{mr̥tyu} \textbf{ko} / \textbf{samay} \textbf{ko} rokā 	nahī̃ 	jā saktā\\
 death \textsc{acc} { } time \textsc{acc} stop \textsc{neg} \textsc{psv} can.\textsc{prs.3m.sg}\\
\glt ‘Death / time cannot be stopped.’ (single entities, common knowledge) (own data)

\ex
\gll par bahut dinõ tak sthagit \textbf{maut} \textbf{ko} bhī nahī̃ jiyā jā saktā\\
 but many days till postponed death \textsc{acc} even \textsc{neg} live \textsc{pass} can.\textsc{prs.3m.sg}\\
\glt ‘But one cannot live even a deferred death for very long.’ (Vinod Kumar Shukla, 1996, \emph{Khilega to dekhenge})

\ex
\gll \textbf{unke} \textbf{vilamban} \textbf{ko} 24 janvarī kī subah us vaqt kiyā gayā jab…\\
\textsc{3pl.gen} suspension \textsc{acc} 24 January of morning that time do \textsc{pass.m.sg} when\\
\glt ‘Their suspension occurred on the morning of January 24 when…’ (\emph{Times of India} 13/1/2015)
\z
\z

The marking of such inanimates, which are essentially compact abstract nouns, is common to \isi{active} and passive sentences. The non-marking of human patient in contrast is possible only in passive sentences. The fact that the marking of abstract nouns such as in series \REF{10-mo-ex:21}, is maintained irrespective of the construction, whether \isi{active} or passive, seems to suggest that this category may be deemed as ranking high in the hierarchy of markable objects.

\subsection{Reduced passive clauses} \label{10-mo-sec:3.2}
Passive nominalizations do not confirm this equal frequency of marked human and inanimates, since human objects behave quite differently from inanimates in reduced passive clauses, and there is a triple distinction for inanimates. In \ili{Hindi}, the nominal or adverbial reduction of a clause, whether \isi{active} or passive, requires the genitive marking of its subject when distinct from the main subject (\REF{10-mo-ex:22a} and \REF{10-mo-ex:22b}), with a few exceptions \REF{10-mo-ex:22c} corresponding to nouns analyzed as pseudo-incorporated \citep{Dayal2011Hindi} and analyzed in \citet{Montaut2012Saillance} as anti-salient, or as having extremely low individuation.\footnote{The way \citet{Dayal2011Hindi} and \citet{Mohanan1994Argument} define incorporation excludes the morphophonological features usually associated with the notion, hence the suggested appellation of “semantic incorporation” \citep{Dayal2011Hindi}.} 

\ea Modern Standard \ili{Hindi} (own data)
\label{10-mo-ex:22}
\ea \label{10-mo-ex:22a}
\gll āpkā yahā͂ ānā mujhe bilkul acchā nahī͂ lagā\\
 2H.\textsc{gen} here come.\textsc{inf} \textsc{1sg.dat} really really \textsc{neg} seem.\textsc{pfv.m.sg}\\
\glt ‘I did not like at all (the fact) that you came here.’
\ex \label{10-mo-ex:22b}
\gll rām ke āte (*rām āte) hī sab gāyab ho gae the\\
 Ram \textsc{gen} coming Ram coming just all.\textsc{m.pl} disappeared be go \textsc{pprf.m.pl}\\
\glt ‘Right after Ram came, all had disappeared.’

\ex \label{10-mo-ex:22c}
\gll andherā (*?ke) hote hī sab gāyab ho gae the\\
 darkness \textsc{gen} being just all.\textsc{m.pl} disappeared be go \textsc{pprf.m.pl}\\
\glt ‘Right after the coming of darkness all had disappeared.’
\z
\z

In the nominalized passive clause, the patient is in the subject position and can either be marked by accusative \textit{ko}\textit{,} by the genitive or unmarked, depending on the type of passive (promotional or not) and on the type of (promoted) object (\isi{animate} vs. \isi{inanimate} referent, individuation). While a human patient is obligatorily marked in the \isi{active} and optionally marked in the passive, the nominalized clause echoes both possibilities with the \isi{optionality} of a regular subject marking in the genitive and a retention of the accusative marking, but it cannot remain unmarked \REF{10-mo-ex:23}: 

\ea Modern Standard \ili{Hindi} \citep[9]{Bhatt2007Unaccusativity}

\label{10-mo-ex:23}
\gll  Rina kā / ko / *Ø bāzār mẽ dekhā jānā śaram kī bāt hai.\\
 Rina \textsc{gen} / \textsc{dat} { } { } market in see \textsc{pass.inf} shame of thing is\\
\glt ‘For Rina to be seen in the market is a matter of shame.’ 
\z

In contrast, \isi{inanimate} nouns may either be marked as ordinary subjects, retain their object marking or have no marking at all like the so-called incorporated objects:

\protectedex{
\ea Modern Standard \ili{Hindi}  (\citealt[9]{Bhatt2007Unaccusativity}; author's translation) \\
\label{10-mo-ex:24}
\gll Per̩ kā / ko / Ø is tarah kāt̩ā jānā śaram kī bāt hai.\\
 tree \textsc{gen} { } \textsc{dat} { } { } this way cut \textsc{pass.inf} shame of thing is\\
\glt ‘The fact that the/a tree was cut in this way/this kind of tree cutting is a matter of shame.’ 
\z
}

\subsection{The opposite type of noun-verb relation: “Incorporated” objects} \label{10-mo-sec:3.3}

Example \REF{10-mo-ex:24} shows a distinct meaning of the unmarked noun, devoid of any individuation to the point of being incorporated. The notion of (semantic) incorporation in \ili{Hindi} was elaborated by \citet{Dayal2011Hindi} to account for a type of bare nominals with special behavior, particularly in disallowing pronominal anaphorization. Such objects fail to control agreement in sentences ordinarily constraining object agreement, namely ergative sentences involving a complement infinitive \REF{10-mo-ex:25}, and abilitative or obligative sentences with transitive main verb in the infinitive \REF{10-mo-ex:26}. The standard Object-Verb agreement occurs in~\REF{10-mo-ex:25b} and \REF{10-mo-ex:26b}, where the feminine object \textit{sāikil} ‘bike’ controls the agreement of the matrix verbs ‘do’ and ‘come’ as well as the infinitive ‘drive’, which in \ili{Hindi}, may vary in gender. In \REF{10-mo-ex:25a} and \REF{10-mo-ex:26a}, on the contrary, it does not vary, and the infinitive remains in the masculine form, controlling the agreement of the matrix verb, as do \isi{intransitive} verbs \REF{10-mo-ex:25c}:

\ea Modern Standard \ili{Hindi}  (own data)
\label{10-mo-ex:25}

\ea \label{10-mo-ex:25a}
\gll bacce ne sāikil \textbf{calānā} śurū kiyā\\
 child.\textsc{m.sg} \textsc{erg} bike.\textsc{f.sg} drive.\textsc{inf.m.sg} beginning do.\textsc{pfv.3m.sg}\\
\glt ‘The boy started to ride a bicycle.’ (has started bicycle riding)
\ex \label{10-mo-ex:25b} 
\gll bacce ne sāikil \textbf{calānī} śurū kī\\
 child.\textsc{m.sg} \textsc{erg} bike.\textsc{f.sg} drive.\textsc{inf.f.sg} beginning do.\textsc{pfv.3f.sg}\\
\glt ‘The boy started to ride a bicycle.’
\ex \label{10-mo-ex:25c}
\gll baccõ ne skūl \textbf{jānā} śurū kiyā\\
 child.\textsc{m.pl} \textsc{erg} school go.\textsc{inf.m.sg} beginning do.\textsc{pfv.3m.sg}\\
\glt ‘The children started going to school.’
\z
\z

\ea Modern Standard \ili{Hindi}  (own data)
\label{10-mo-ex:26}
\ea \label{10-mo-ex:26a}
\gll mujhe sāikil \textbf{calānā} ātā hai\\
\textsc{1sg} bike.\textsc{f.gs} drive.\textsc{inf.m.sg} come \textsc{prs.3m.sg}\\
\glt ‘I know how to ride a bicycle (how to cycle).’

\ex\label{10-mo-ex:26b}
\gll mujhe \textbf{sāikil} calānī ātī hai\\
 \textsc{1m.sg} bike.\textsc{f.sg} drive.\textsc{inf.f.sg} come \textsc{prs.3f.sg}\\
\glt ‘I know how to ride a bicycle.’
\z
\z

In \REF{10-mo-ex:25a} and \REF{10-mo-ex:26a} the constituent triggering agreement is the whole infinitival clause, sometimes considered to be an instance of incorporation of the object into the verb since \textit{sāikil calānā} “bicycle drive” it behaves in this respect like an \isi{intransitive} verb. 

Although both alternating constructions can be used in similar unmarked contexts, there is a preference for the non-agreeing type, with some conventional object-verb expressions like ‘drink tea’ or ‘buy vegetable’.\footnote{Similarly, in ergative sentences, like ‘I began/wanted to drink tea’ or ‘I wanted to buy vegetables’ minimal individuation is required for the object of the complement infinitive to trigger agreement, and agreement with the object is highly improbable with the bare noun (as opposed to ‘I wanted to buy various vegetable’ or ‘drink this excellent tea’. More examples in \citet{Montaut2012Saillance}.} Here, the infinitive triggers agreement on the matrix verb:

\ea Modern Standard \ili{Hindi}  (own data)\\
\label{10-mo-ex:27}
\gll {mujhe} sabzī \textbf{kharīdnā} / ?? kharīdnī hai\\
\textsc{1sg} vegetable.\textsc{f.sg} buy.\textsc{inf.m.sg} { } { } buy.\textsc{inf.f.sg} be.\textsc{prs.3.sg}\\
\glt ‘I have to buy vegetable.’
\z

To summarize, only “incorporated” objects with very low individuation can dispense with indexation on the verb in the relevant clause types. Marked objects pattern at the opposite side of the following hierarchy of objects: incorporated (unmarked) > unmarked (non incorporated) > marked.

The triggering feature for this triple syntactic differentiation is individuation. It is not, directly, \isi{topicality}, nor is it the role played within the focus, although of course, the semantic feature individuation is also relevant in \isi{information structure}.

\section{The emergence of object marking}
\label{10-mo-sec:4}
Most scholars do not date the emergence of Modern standard \ili{Hindi} before the 18\textsuperscript{th} century. Previous to this stage, the language, although it is systematically called medieval or ancient \ili{Hindi}, is expectedly not standardized, and as such it is much closer to some of the regional varieties today analyzed as independent languages. What is generally called “Old \ili{Hindi}” is the so-called \textit{sant bhasha}, a poetic language forged by the first mystic poets who expressed their religious opposition to the brahmanic world order by using popular vernacular speech instead of \ili{Sanskrit}. This language, which was first used by the devotional mystic Kabir (14\textsuperscript{th} c.), and later by Mira Bai (16\textsuperscript{th} c.), has been fairly well studied and shown to display various regional features, taken more from the Eastern languages in Kabir, and more from the Western varieties in Mira, but fused in what will become the literary \textit{koine} of medieval Northern India. In what follows I will discuss the three main stages of the DOM evolution in pre-modern “\ili{Hindi}”.


\subsection{First New Indo-Aryan stage: 14\textsuperscript{th} century} \label{10-mo-sec:4.1}

During the first stages of \ili{Hindi} and of other New \ili{Indo-Aryan} languages (NIA), the inflectional system of \ili{Sanskrit} is in the process of being replaced by adpositions (nominal category) and auxiliaries (verbal category). Yet this process is far from being completed and the absence of clear relators is a common feature of ordinary discourse, the few oblique cases maintained in the language being used for various syntactic purposes: the -\textit{i} locative for the agent in past transitive processes, and a syncretic oblique -\textit{hi} (derived from the fusion of the old dative/instrumental already achieved in Middle \ili{Indo-Aryan}) for all kinds of obliques. Most of the time, nouns remain unmarked, the meaning being easily recoverable from the context and sequence since sentences are usually minimal. This -\textit{hi} ending is the most frequent marker of objects in Kabir, whereas the postpositional marking (\textit{kū͂/kau}) is just starting to appear \citep[325]{Strnad2013Morphology}, but in both inflectional or adpositional cases, the marking is far from systematic. 

Human objects, including proper names, are either marked \REF{10-mo-ex:28b} or unmarked \REF{10-mo-ex:28a}, and sometimes in the same sequence both marked and unmarked proper nouns occur~\REF{10-mo-ex:28b}:

\ea 
\label{10-mo-ex:28}
\ea  \label{10-mo-ex:28a}
\gll Hiranākasa māryau.\\
 Hiranakashyapu kill.\textsc{pfv.m.sg}\\
\glt ‘[He] killed Hiranyakashyapu.’ 

\ex \label{10-mo-ex:28b}
\gll Rāma\textbf{hi} janai janai Rahimā̃na.\\
 Ram.\textsc{acc} know.\textsc{prs.3sg} know.\textsc{prs.3sg} Merciful.Ø\\
\glt ‘[He] knows Ram, he knows the Merciful.’ (Kabir, verse 302)
\z
\z

Even a proper name, if occurring with a predicative adjective, can be unmarked \REF{10-mo-ex:29a}, whereas other human referents can be marked \REF{10-mo-ex:29b}:
\ea
\label{10-mo-ex:29}
\ea \label{10-mo-ex:29a}
\gll Rāma.Ø kari sanehī.\\
 Ram make.\textsc{cv} dear\\
\glt ‘Making Ram your dear.’ (Kabir, verse  381)

\ex \label{10-mo-ex:29b}
\gll āpana ãdha \textbf{aura} \textbf{kū̃} kahai kanā̃nā̃\\
 self blind other \textsc{acc} say.\textsc{prs.3sg} one-eyed\\
\glt ‘[Being] himself blind, he will call others one-eyed.’ (Kabir, verse 149)
\z
\z

The only category which is systematically marked is the personal pronoun (1\textsuperscript{st} and 2\textsuperscript{nd} person), and occurrences of the 3\textsuperscript{rd} person are frequently unmarked even when referring to a human entity. 

\ea
\label{10-mo-ex:30}
\gll  jaga.Ø maĩ des̩ū̃	 jaga na des̩i \textbf{mohi}\\
 world \textsc{1sg} see.\textsc{prs.1sg} world \textsc{neg} see.\textsc{prs.3sg} \textsc{1sg.acc}\\
\glt ‘I see the world, the world does not see me.’ (Kabir, verse 76.3)
\z

Given the fact that humanity, which is today the main (compulsory) trigger for object marking, does not apply, we would expect that \isi{inanimate} objects are systematically unmarked, but this is not the case, and the marking of inanimates seem to be as random as the marking of human objects. Example \REF{10-mo-ex:31} for instance displays two parallel clauses patterning identically, with the same construction, the same semantic class of objects (the so-called class of single entities), the same relation between predicate and object and the same ordering of both sequences. Yet ‘ocean’ is marked and ‘sun’ is unmarked:

\ea
\label{10-mo-ex:31}
\gll ulat̩ī Gangā sāmudra-\textbf{hi} sosai, sasihara sūra.Ø grāsai\\
 reversed Ganga ocean-\textsc{acc} dry.up.\textsc{prs} moon sun swallow.\textsc{prs}\\
\glt ‘The reversed Ganga dries up the ocean, the moon swallows the sky.’
\z

The adpositional marking by means of \textit{kū/kau}, infrequent and more recent, occurs in similar conditions, and most often without apparent reason. In \REF{10-mo-ex:32}, we may hypothesise that the relative pronoun is topicalized since the \ili{Hindi} correlative system amounts to topicalizing the relative clauses (\citealt{Gupta1986Discourse,Montaut2012Saillance}) in the same way as conditionals \citep{Haiman1978Conditionals}, but in~\REF{10-mo-ex:33}, the noun \textit{pada} ‘word, line’, which is a marked object, is not the head of the relativized expression:

\ea
\label{10-mo-ex:32}
\gll 	jākū̃ yahu jaga ghini kari cālai\\
\textsc{rel.acc} \textsc{dem} world horrible do.\textsc{cv} go.\textsc{prs.3sg}\\
\glt ‘That which this world avoids with disgust.’  (\textit{lit}. that which considering horrible the world goes by) (Kabir, verse 185.4)
\z

\ea
\label{10-mo-ex:33}
\gll  yā 		pada kū̃ 	bujhai 	tākū̃ 		tinyū̃ trībhuvana	sūjhai 	\\
\textsc{dem} verse \textsc{acc} understand.\textsc{prs.3sg} \textsc{3sg.dat} three world think.\textsc{prs.3sg}\\
\glt ‘[Who] understands this pada, he knows the three worlds.’ 
\z

In \REF{10-mo-ex:33} the reason why the \isi{inanimate} is marked is probably, apart from \isi{definiteness} (not in itself a triggering factor as shown by \REF{10-mo-ex:28}), the intrinsic importance of the word ‘word/verse’ in the ideological context of the time: for a devotional mystic nothing is more central and more emphasized than the deity’s speech, or the word pointing to the deity. What is also noticeable is the parallel marking of the marked object (\textit{jākū͂, pada kū͂}) and the dative subject (\textit{tākū͂}) by the same postposition in~\REF{10-mo-ex:32}. 

\subsection{Second stage: 16\textsuperscript{th} century}
\label{10-mo-sec:4.2}

In 16\textsuperscript{th} century classical texts like Tulsidas Ramayana (T), the inflectional marking (-\textit{hi}) is maintained yet the postpositional marking occurs more often, in conditions similar to the ones in stage 1: pronouns for 1\textsuperscript{st} and 2\textsuperscript{nd} person are consistently in the oblique, \REF{10-mo-ex:34} and~\REF{10-mo-ex:35}, as in the stage 1, and the same oblique form is also used for oblique subjects~\REF{10-mo-ex:35}. But unlike the earlier period, human objects are systematically marked (\REF{10-mo-ex:34} and \REF{10-mo-ex:36}), and only exceptionally unmarked, either as proper nouns or pronouns \REF{10-mo-ex:37}:

\ea
\label{10-mo-ex:34}
\gll \textbf{tehi} na jānā nr̥pa.Ø, nr̥pa-\textbf{hi} so jānā\\
\textsc{3sg.obl} \textsc{neg} know.\textsc{pfv} king king-\textsc{obl} \textsc{3sg} know.\textsc{pfv}\\
\glt ‘The king did not recognize him, he recognized the king.’ (T 140)
\z

\ea
\label{10-mo-ex:35}
\gll  kahā tapas nr̥pati.Ø jānaū̃ \textbf{tohi} … lāg bhala mohi\\
 said hermit king know.\textsc{1sg} \textsc{2.obl} {}  seem good \textsc{1sg.obl}\\
\glt ‘Said the hermit: “I know you as the king [this move] pleased me/I liked”’ (T 160)
\z

\ea\label{10-mo-ex:36}

\ea \label{10-mo-ex:36a}
\gll Raghupati-\textbf{hi} nihāri prabhū-\textbf{hi} citaï.\\
 Sun.lord-\textsc{obl} look.\textsc{cv} Lord-\textsc{obl} look.\textsc{cv}\\
\glt ‘[Sita] seeing Rama (king of sun linage).’ ‘[Sita] looking at the Lord.’ (T 140)

\ex \label{10-mo-ex:36b}
\gll Sīya-\textbf{hi} biloki.\\
 Sita-\textsc{obl} see.\textsc{cv}\\
\glt  [Ram] ‘Looking at Sita.’ (T 250)
 \z
 \z
 
\ea
\label{10-mo-ex:37}

\ea \label{10-mo-ex:37a}
\gll Rām biloke log […] citaï Sīya kr̥pāyatan jāni vikal bises̩i.\\
 Ram see.\textsc{pfv} people { } look.\textsc{cv} Sita gracefully knew worried special\\
\glt ‘Ram saw the folk, [...] looking at Sita with mercy he perceived her great distress.’ (T 251)

\ex \label{10-mo-ex:37b}
\gll rāu tr̥śit nahĩ so Ø pahicānā\\
 prince thirsty \textsc{neg} \textsc{3sg} { } recognize.\textsc{pfv}\\
\glt ‘The king, overcome by thirst, did not recognize him.’ (T 158)
\z
\z

Whereas the unmarkedness of the collective \textit{log} ‘people’ is still possible (\cf \sectref{10-mo-sec:1}), the zero marking of the proper noun Sita is no longer grammatical, even though it was quite usual two centuries earlier. Indeed, all instances of X looks at/sees Y exhibit marking of proper names in~\REF{10-mo-ex:36}: whether Ram looks at Sita or Sita at Ram, whatever verb is used (\textit{cita} ‘look at/gaze’, \textit{nihār} ‘see/look’, \textit{bilok} ‘see/look’). 

Another difference with the previous stage it seems to be a more frequent marking of nouns in small clauses \REF{10-mo-ex:38} – which however is still not systematic \REF{10-mo-ex:39} – even when the small clause includes a participle \REF{10-mo-ex:40}: 

\ea
\label{10-mo-ex:38}
\gll  bhale-\textbf{hi} manda manda-\textbf{hi} bhale 	karahū̃\\
 good-\textsc{obl} vile vile-\textsc{obl} good do.\textsc{prs.2sg}\\
\glt ‘You debase the good man (make vile the good), you praise the vile.’
\z

\ea
\label{10-mo-ex:39}
\gll  kol biloki bhūpa bar̩a dhīrā bhāgi pait̩h giriguhā̃ 	 gabhīra\\
 boar see.\textsc{cv} king much determined flee enter mountain.cave deep\\
\glt ‘Seeing the king so much determined the boar entered a deep cave.’
\z

\ea
\label{10-mo-ex:40}
\gll  jo prabhū tumah bipin phirat dekhā\\
\textsc{rel} Lord \textsc{2pl} forest roaming see.\textsc{pfv}\\
\glt ‘The Lord whom you saw roaming in the forest.’ %(T 140)
\z

Examples \REF{10-mo-ex:36} to \REF{10-mo-ex:40} are from Tulsidas \textit{Ramayana}, in an Eastern variety (\ili{Awadhi}), but in the Western dialects the situation was similarly unconstrained. Even proper nouns can remain unmarked, as was the case in the first stage:

\ea
\label{10-mo-ex:41}
\gll māī rī mhā̃ liyā Govinda mol\\
 sister \textsc{interj} \textsc{1sg} take.\textsc{pfv} Govinda buy.\textsc{cv}\\
\glt ‘Sister, I bought (and took) Govinda [a name for Krishna].’ (Mira Bai, 16\textsuperscript{th} c.)
\z

\subsection{Third stage: 17\textsuperscript{th}–18\textsuperscript{th} centuries: the modern system}
\label{10-mo-sec:4.3}

There is not much to comment after the 17\textsuperscript{th} century since the system does not present noticeable differences with the modern system. The literature available during this period makes a more liberal use of \ili{Persian} idioms and structures (particularly ezafe for determination of nouns) than in earlier \ili{Hindi} and today standard \ili{Hindi}. Ezafe specified objects can be either marked (in~\REF{10-mo-ex:42a} ‘fire of torment’) or unmarked (in~\REF{10-mo-ex:42b} ‘heat’): what prevails is the degree of \isi{topicality} in the discourse:

\ea\label{10-mo-ex:42}
\ea \label{10-mo-ex:42a}
\gll wafādārī ne dilbar kī bujhāyā \textbf{ātiś-e-gam} \textbf{ko}\\
 faithfulness.\textsc{f.sg} \textsc{erg} lover of extinguish.\textsc{pfv.3m.sg} fire-of-torment \textsc{acc}\\
 \ex \label{10-mo-ex:42b}
 \gll ke \textbf{garmī} dafā kartā hai gulāb āhistā āhistā\\
 that/as heat off make \textsc{prs.3m.sg} rose slowly slowly\\
\glt ‘My faithful love has quenched the fire of my love (\textsc{foc}), as rose dispels the effect of heat, step by step.’ (Wali, mid. 17\textsuperscript{th} c.)
\z
\z

\ea
\label{10-mo-ex:43}
\gll  jab sõ dekhā nahī̃ nazar-bhar kākul-e-muśkin-e-yār\\
 when from see.\textsc{pfv} \textsc{neg} glance-full locks-\textsc{ez}-scented-\textsc{ez}-beloved\\
\glt ‘Since I did not see fully her [my love’s] scented locks (\textsc{foc}).’ (Wali, mid 17\textsuperscript{th} c.)
\z

In the two parallel constructions (X diminishes Y) of \REF{10-mo-ex:42}, the first object, an abstract NP, is extracted and put in a postverbal position at the rime, in conformity with its discourse function, since love torment is the main topos of the poems. It is marked. In turn, the second object, also an abstract noun, remains preverbal as an ordinary part of the wider focus and is unmarked. However, in~\REF{10-mo-ex:43}, an ezafe-specified object similar to \REF{10-mo-ex:42a}, ‘scented locks of the beloved’, remains unmarked although concrete and in a postverbal position; even though it is strongly emphasized by its position, it is not given centre stage. Discourse saliency is the triggering factor, as it is today for inanimates.

Objects are always marked when controlling nominal or adjectival adjuncts, either relative pronouns with \isi{inanimate} reference (whereas relative pronoun with human referent could be unmarked in the earlier period) or nouns, \isi{inanimate} as well as \isi{animate}:

\ea
\label{10-mo-ex:44}
\gll  ke \textbf{jisko} kasīne kabhī vā na dekhā\\
 that \textsc{rel.acc} \textsc{indef.erg} ever open \textsc{neg} see.\textsc{pfv.m.sg}\\
\glt ‘That which (\textsc{acc}) nobody has seen bloom.’ (‘which’ = merā dil ‘my heart’) 
\z

\ea
\label{10-mo-ex:45}
\gll kiyā mujh īśq ne \textbf{zālim} \textbf{ko} āb āhistā āhistā\\
 do\textsc{.pfv.3sg} my love \textsc{erg} despot \textsc{acc} water slowly slowly\\
 
\gll  ke ātiś \textbf{gul} \textbf{ko} kartī hai gulāb āhistā āhistā\\
 that/as fire flower \textsc{acc} do \textsc{prs.3fsg} rose slowly slowly\\
\glt ‘My love has melted the despot (made this despot water), step by step, as fire distil (make the flower rose-perfume) the essence of rose, step by step.’ (Walid, mid 17\textsuperscript{th} c.)
\z


The following tables provide an overview of the different referent types according to the \isi{animacy} and \isi{definiteness} scales (1), and the syntactic constraints (2).\footnote{\tabref{10-mo-tab:1} does not take into account the cases of deranking. In \tabref{10-mo-tab:2} data is lacking for passive transformation in the earlier stages of the language since passive was rare and always with a modal meaning of incapacity.} 

\begin{table}
\begin{tabularx}{\textwidth}{X l cccc}
\lsptoprule
 && Stage 1:& Stage 2:& Stage 3:& Modern \\
 && 14 c. & 16 c. & 17–18 cc. &  \ili{Hindi}\\
\midrule
Human  & SAP pronouns & always & always & always & always\\
objects& Proper noun & optional & frequent & always & always\\
 & Third person pronoun & optional & frequent & always & always\\
 & Other human nouns & optional & frequent & always & always\\
In\isi{animate}  & Specific nouns & optional & optional & optional & optional\\
nouns& Abstract (compact) nouns & optional & optional & frequent & frequent\\
\lspbottomrule
\end{tabularx}
\caption{Animacy and definiteness constraints on DOM}\label{10-mo-tab:1}
\end{table}

\begin{table}
\begin{tabularx}{\textwidth}{X ccccc}
\lsptoprule
 & Stage 1:& Stage 2:& Stage 3:& Modern \\
 & 14 c. & 16 c. & 17–18 cc. &  \ili{Hindi}\\
\midrule
Human noun in small clause (human referents) & optional & frequent & always & always\\
In\isi{animate} noun in small clause with participle & optional & optional & frequent & very frequent\\
Passive finite clauses	&no data	& no data & optional& optional \\
				& 		&		& (human, & (human,\\
				&		&		& \isi{inanimate}) &  \isi{inanimate})\\
\lspbottomrule
\end{tabularx}
\caption{Syntactic constraints on object marking} \label{10-mo-tab:2}
\end{table}

The only objects obligatorily marked in stage 1 and 2 are first and second person pronouns, whereas neither person names, nor titles and nouns referring to culturally prominent persons are consistently marked. Objects controlling adjectival / nominal adjuncts are still only optionally marked before stage 3 (17\textsuperscript{th} c.). Inanimates are optionally marked right from stage 1 as well as animates other than first person pronouns. At no stage was marking used as a distinguishing device, contrary to \citegen{Comrie1979Definite} or \citegen{Croft1988Agreement} hypothesis, and in accordance with the observations made by \citet[213]{Malchukov2008Animacy} that the discriminatory function is quite rare across languages, by \citet{Arkadiev2009Differential} that it is not relevant for \ili{Indo-Iranian} languages, and by \citet{deHoopetal2005Differential} that it is absent in \ili{Hindi}.

Regarding pronouns, the consistent marking from the very beginning of 1\textsuperscript{st} and 2\textsuperscript{nd} person pronouns should not be over-emphasized, since they retained the accusative inflection till the late Middle \ili{Indo-Aryan} stage as opposed to all other nominal categories (including 3\textsuperscript{rd} person pronoun: unmarked in~\REF{10-mo-ex:37b} and marked in~\REF{10-mo-ex:34}). \tabref{10-mo-tab:2a} is according to \citet{Bubenik2006Cases} the table of pronominal forms in the late Apabhramsha stage (10–11\textsuperscript{th} c.): 1\textsuperscript{st} and 2\textsuperscript{nd} persons retain an accusative, which is distinct both from nominative and from dative/ablative, whereas accusative is fused with nominative for the 3\textsuperscript{rd} person:

\begin{table}
\begin{tabularx}{\textwidth}{l ll l}
\lsptoprule
1\textsuperscript{st} & \textsc{nom} \textit{hau/haũ} (Sk \textit{aham}) & \textsc{acc} \textit{maï, maĩ, ma\~{ï}, me} (Sk \textit{mām}) &  \textsc{dat} \textit{mujjh}\\
2\textsuperscript{nd} &  \textsc{nom} \textit{tuhu/tuhũ} (Sk \textit{tvam}) & \textsc{acc} \textit{paï, paĩ, taĩ} (SK \textit{tvām})  & \textsc{dat} \textit{tujjh}\\
3\textsuperscript{rd} & \textsc{nom}/\textsc{acc} \textit{so} (\textsc{m.sg}), \textit{sa} (f.\textsc{sg}) & & \\
\lspbottomrule
\end{tabularx}
\caption{Syntactic constraints on object marking} \label{10-mo-tab:2a}
\end{table}

%%\todo[inline]{@Montaut do you mean restructuring in the next sentence? restructuration is not an English word, I think. [AM response:] Restructuring 'restructuration' is an English word: http://www.linguee.com/english-french/translation/restructuration.html [AW: the link you provide gives a French translation of the word "restructuring"]}

A considerable morphological restructuring of the system occurred between this stage and the first stages of New \ili{Indo-Aryan}, with the genitive in \textit{-r-} in most regional varieties, and various oblique forms depending on the region, which came to be used both for marked accusative and dative \REF{10-mo-ex:35}, before adpositions substituted for inflectional morphemes with the same bi-functional use as the old dative/accusative. Remarkably, modern standard \ili{Hindi} maintained the oblique form \textit{mujh} and \textit{tujh} before postpositions and the old inflectional forms \textit{mujhe} and \textit{tujhe} for dative/accusative, in alternation with the adpositional forms \textit{mujhko} and \textit{tujhko}, and it extended this system to the third person: the direct (\textit{vah}) and oblique (\textit{us}) cases are distinct, and in the dative accusative there are two alternate forms, one inflectional (\textit{use}) and one adpositional (\textit{usko}). Yet the fact that the distinctive accusative was retained throughout Middle \ili{Indo-Aryan} certainly played an important role in the marking of 1\textsuperscript{st} and 2\textsuperscript{nd} person pronouns, in contrast with third person pronoun and other nouns.

\section{Object marking in the “dialects” of Hindi}
\label{10-mo-sec:5}

A good deal of ambiguity prevails in the field of language description since these language varieties are considered, administratively and politically, as dialects of \ili{Hindi}, with various names and inner variation. Yet linguistically the variations in comparison with modern standard \ili{Hindi} are so important that many regard them as distinct languages: nominal and verbal flexions are different, some languages (like \ili{Bhojpuri}, \ili{Awadhi}, \ili{Maithili}) ignore grammatical gender and ergative alignment, others have three grammatical genders or had them until recently (\ili{Western Rajasthani}, or close to the three genders, languages like \ili{Gujarati} and \ili{Marathi}). The modern stage of these “dialects” itself displays great variations regarding the object marking, some maintaining the old situation as sketched above, some closer to the system of standard \ili{Hindi}. A comprehensive representation of the whole picture, that involves 331 distinct varieties, out of which there are at least a dozen distinct languages, is obviously outside the limits of this study. I will therefore limit myself to the presentation of a few features that are distinct from modern standard \ili{Hindi}, and that may help explain the general trends in the evolution of object marking. 

Let us begin with a step ahead of the evolution of standard \ili{Hindi}, if one takes agreement to be a reliable marker of the integration into the grammatical system. In \ili{Hindi} the verb never agrees with a marked object, and indexation on the verb is only by default. In \ili{Marwari}, a Western dialect of Rajasthan, like in \ili{Gujarati}, on the contrary, the marked object is indexed on the verb (gender number agreement) as shown by \REF{10-mo-ex:46}. This is also the case in \ili{Magahi}, an Eastern dialect (Bihari) that shows agreement with marked objects, though somewhat differently, since all \isi{animate} participants are indexed in the verb \REF{10-mo-ex:47}:


\ea \ili{Marwari} \citep{Khokholova2001Ergativity}

\label{10-mo-ex:46}
\gll mhaĩ śaraṇ naĩ dekhā\\
\textsc{1sg} Sharan.\textsc{f.sg} \textsc{acc} see.\textsc{f.sg}\\
\glt ‘I have seen Sharan.’ 
\z

\ea \ili{Magahi} \citep{Verma1991Exploring}

\label{10-mo-ex:47}
\gll ham dekh-l-i ham dekh-l-i-a ham dekh-l-i-ain\\
 1\textsc{sg} see-\textsc{pst}-1 1\textsc{sg} see-\textsc{pst}-1-3non\textsc{h} 1\textsc{sg} see-\textsc{pst}-1-3\textsc{h}\\
\glt ‘I saw it.’ ‘I saw him (servant).’ ‘I saw him (guru).’ 
\z

One could also argue that indexing the marked object the same way as the unmarked object is not a step further if it is expected that marked objects should also be indexed in a marked way. Yet no example in the various stages of object marking in \ili{Indo-Aryan} displays an agreement with marked objects prior to agreement with unmarked objects (the ergative pattern precedes the emergence of DOM by far), whereas all other examples point to the agreement \isi{blocking effect} of DOM. 

Another factor observed in certain regional varieties which is at discrepancy with standard \ili{Hindi} and its historical emergence is the correlation in object and subject marking: in 19\textsuperscript{th} century \ili{Kumaoni} for instance, no marked object occurs with an ergative agent, as in~\REF{10-mo-ex:48}, even when controlling an adjunct, as in~\REF{10-mo-ex:49}, whereas with a nominative subject, objects are marked when human or specific centre-staged inanimates, as in~\REF{10-mo-ex:50}:

\protectedex{\ea Kumaoni (\citealt{Grierson1903Linguistic}: IX-4) \label{10-mo-ex:48}

\gll myārā dagar̩iyana.le ek bāman pakar̩o\\
 my companion.\textsc{m.pl.erg} one Brahmin seize.\textsc{pfv.m.sg}\\
\glt ‘My companions captured a Brahmin.’
\z
}

\ea Kumaoni (\citealt{Grierson1903Linguistic}: IX-4) \label{10-mo-ex:49}

\gll prithvi.m lag yo pahār̩ hamārī thātī raci dev.le\\
 earth.on too this mountain our place.to.live make.\textsc{pfv} god.\textsc{erg}\\
\glt ‘God made this mountain a place to live for us on earth too.’
\z

\ea Kumaoni (\citealt{Grierson1903Linguistic}: IX-4) \label{10-mo-ex:50}

\gll tab ū wī bwaj kan̩i apan ghar huni lī āy\\
 then \textsc{3sg} \textsc{dem} load \textsc{acc} \textsc{refl} home \textsc{all} take come.\textsc{pfv}\\
\glt ‘Then he brought this load to his house.’
\z

According to \citet{Stronski2013Evolution} and \citet{Sharma1987Formation}, object marking has now in some varieties started to extend to ergative sentences: a modern development, still unknown in standard \ili{Kumaoni}, as in~\REF{10-mo-ex:51a}, which contrasts with Garhwali, very closely related to \ili{Kumaoni} and part of the same sub-group of Pahari languages, which allows the object to be marked in presence of an ergative agent, as in~\REF{10-mo-ex:51b}: 

\ea Kumaoni (Krzysztof Stroński p.c.) \label{10-mo-ex:51}

\ea \label{10-mo-ex:51a}
\gll {mī.le naunai *san̩ī baĩt̩ le māre}\\
\textsc{1sg.erg} child \textsc{acc} cane \textsc{ins} strike.\textsc{pfv}\\
\glt ‘I hit the boy with a cane.’ 
\z

\ea Garhwali (personal field work) \label{10-mo-ex:51b}

\gll {mī.na naunai ta\~{ī} / san̩ī (nauno.Ø) baĩt̩ na māri}\\
\textsc{1sg.erg} child \textsc{acc} { } \textsc{acc}  child cane \textsc{ins} strike.\textsc{pfv}\\
\glt ‘I hit the boy with a cane.’
\z
\z

In Garhwali, the marked object is allowed in a sentence displaying an ergative agent right from the first attested texts collected by \citet{Grierson1903Linguistic} \REF{10-mo-ex:52}, whether the object is \isi{inanimate} or human, but it is not compulsory even today, except for proper names \REF{10-mo-ex:53}. Its \isi{optionality} is not constrained by the presence vs. absence of an ergative agent. In folk songs, which are linguistically archaic, it is optional, and prosodic considerations, for instance, may possibly apply, as in~\REF{10-mo-ex:54} where a married girl is not allowed to visit her family. The objects nouns consisting of one long syllable and one short, ‘mother’, ‘brother’ are marked, while nouns with two long syllables ‘father’, ‘sister in law’, ‘sister’ remain unmarked.

\protectedex{
\ea Garhwali (\citealt{Grierson1903Linguistic}: IX-4)

\label{10-mo-ex:52}
\gll ve-na sattu san̩ī ve talau mā d̩āl dini/dine\\
\textsc{3sg-erg} sattu \textsc{acc} \textsc{dem} lake in throw gave\\
\glt ‘He threw the sattu (a sort of cereal) in the lake.’ 
\z
}
\ea
\label{10-mo-ex:53}
\gll Tā Anīl Rawat tai͂/san̩i/kū jāndi cha?\\
 \textsc{2sg} Anil Rawat \textsc{acc}/\textsc{acc}/\textsc{acc} know \textsc{prs}\\
 
\glt ‘(Do) you know Anil Rawat?’
\z

\ea \label{10-mo-ex:54}

\ea \label{10-mo-ex:54a}
\gll chorie, mu jān̩ na deule\\
 girl \textsc{1sg} go \textsc{neg} give.\textsc{fut.1sg}\\
 \glt ‘girl, I won’t let you go.’  (GCT 124)

\ex \label{10-mo-ex:54b}
\gll tero bāpū yakhī bulaulo, mu jān̩ na deule\\
 your father here call.\textsc{fut} \textsc{1sg} go \textsc{neg} give.\textsc{fut.1sg}\\
 \glt ‘I will invite your father, I won’t let you go.’

\ex \label{10-mo-ex:54c}
\gll terī amī ku yakhī bulaulo, mu jān̩ na deule  …\\
 your mother \textsc{acc} here call.\textsc{fut}, \textsc{1sg} go \textsc{neg} give.\textsc{fut.1sg}\\

\ex \label{10-mo-ex:54d}
\gll tere bhāi ku yakhī bulaulo, mu jān̩ na deule …\\
 your brother \textsc{acc} here call.\textsc{fut} \textsc{1sg} go \textsc{neg} give.\textsc{fut.1sg}\\

\ex \label{10-mo-ex:54e}
\gll terī bhābhī yakhī bulaulū, mu jān̩ na deule …\\
 your sister.in.law here call.\textsc{fut} \textsc{1sg} go \textsc{neg} give.\textsc{fut.1sg}\\

\ex \label{10-mo-ex:54f}
\gll terī dīdī yakhī bulaula, mu jān̩ na deule …\\
 your sister here call.\textsc{fut} \textsc{1sg} go \textsc{neg} give.\textsc{fut.1sg}\\
\glt ‘I will invite your mother, I won’t let you go. I will invite your brother, I won’t let you go. I will invite your sister-in-law, I won’t let you go. I will invite your elder sister, I won’t let you go.’
\z
\z


Similarly in \ili{Bhojpuri}, which is not a language as closely related as Garhwali and \ili{Hindi}, popular songs display unmarked human objects such as ‘my child’ in~\REF{10-mo-ex:55a}, whereas in modern speech a similar object ‘my son’ is obligatorily marked with the dative/accusative marker \textit{ke} in~\REF{10-mo-ex:55b}:

\ea \ili{Bhojpuri} \citep{Saxena1937Evolution}\\
\label{10-mo-ex:55}
\ea \label{10-mo-ex:55a}
\gll apnā bālaka mohi dīte, apnā bālaka nahī̃ debo\\
\textsc{refl} male-child \textsc{1sg.dat} give.\textsc{cond.1sg} \textsc{refl} male.child \textsc{neg} give.\textsc{fut.1sg}\\
\glt ‘If you give me your son – I will not give [you] my son.’

\ex  \label{10-mo-ex:55b}

\gll tū apnā laïkā ke bhejā\\
\textsc{2sg} \textsc{refl} boy \textsc{acc} send.\textsc{imp}\\
\glt ‘Send your son.’ 
\z
\z

Obligatoriness in flagging human objects giving the priority to human referents over inanimates is clearly a recent phenomenon, in \ili{Bhojpuri} as well as in \ili{Hindi}, and it is limited to certain dialects. Discourse related triggers are \isi{active} everywhere, and \isi{affectedness} does not play a noticeable role (\cf \REF{10-mo-ex:1} and \REF{10-mo-ex:33}). As for agreement, it is exceptionally present in the \ili{Hindi} belt and has been attributed to contact with in the case of \ili{Magahi} (see example \REF{10-mo-ex:47}): \citet{Verma1991Exploring} suggests that this peculiarity of the language, which also presents numerous cases of double agreement, results from contact with \ili{Mundari}, an \ili{Austro-Asiatic} tribal language spoken in central-eastern India. With few exceptions, DOM can co-occur with differential agent marking when the subject is an ergative agent, which is a clear indication that the discriminatory function is weakly relevant. It never co-occurs with an \isi{experiencer} subject in the \isi{dative case}, nor did it in stage 1 \REF{10-mo-ex:33}, because DOM is strictly restricted to formally transitive clauses, while experiential clauses, even with two arguments, are not transitive \textit{sensu stricto}. 

The fact that the accusative marker is morphologically identical to the dative marker, whatever the form of the marker in the various dialects, also accounts for this situation. In \ili{Dravidian} languages, where the accusative is distinct from the dative, in \ili{Tamil} for instance, \textit{-ai/e} (\textsc{acc}) vs. \textit{-akku} (\textsc{dat}), such a constraint does not hold:

\ea Tamil (own data)\label{10-mo-ex:56a}\\
\gll enakku avar.ai pidikkum / teriyum\\
\textsc{1sg.dat} \textsc{3m.sg.acc} like { } know\\
\glt ‘I like / know him.’
\z

\ea Modern Standard Hindi (own data)\label{10-mo-ex:56b}\\
\gll {mujhe vah *usko acchā lagtā hai / milā}\\
\textsc{1sg.dat} \textsc{3sg} \textsc{3sg.acc} good seem \textsc{prs.3sg} { } meet.\textsc{pfv}\\
\glt ‘I like / met him.’ 
\z

\section{Some hypotheses regarding the origin of the marking and the markers}
\label{10-mo-sec:6}

\subsection{Contact with substratum, adstratum and prestige language}
\label{10-mo-sec:6.1}

As already mentioned, DOM is part of the dozen features that are systematically considered to define South Asia as a linguistic area, along with dative subjects, prevalence of complex predicates, coverbs, causative derivation, lack of ‘have’ verb, head final order, reduplication, etc. (\citealt{Masica1976Defining}; \citealt{Emeneau1980Language}). Its appearance in \ili{Indo-Aryan} is more or less contemporary with the rise of dative subjects: it has not been inherited from \ili{Sanskrit}, a inflectional language where accusative is a structural case (all objects are case-marked, a purely syntactic phenomenon). On the contrary, the agglutinative \ili{Dravidian} languages had, right from the first attested texts (slightly before the Christian era), a DOM marking for human objects (suffix {}-\textit{ai}), while it developed the Dative Subject pattern much later with a distinct suffix \citep{Murugaiyan2004Note}, only slightly before \ili{Indo-Aryan} languages. Given the importance of structural borrowings from \ili{Dravidian} in IA, such as the use of coverb and quotative, and the evidence of a \ili{Dravidian} substratum in the area now occupied by \ili{Indo-Aryan} speakers \citep{Witzel1995Early}, \ili{Dravidian} could be a plausible source for the IA marking. The behavior of the “accusative” suffix \nobreakdash-\textit{ai} in modern \ili{Tamil}, however, is not constrained by \isi{transitivity} since it can occur with dative subjects, as in~\REF{10-mo-ex:56a}, unlike in \ili{Hindi}, as in~\REF{10-mo-ex:56b-rep}:\footnote{Note that \ili{Bengali} also allows the accusative marker, even if the same as the dative marker, in experiential sentences such as \ili{Tamil} \REF{10-mo-ex:56a}, because experiential subjects are in the genitive in \ili{Bengali}.} 

\ea Tamil (own data)\label{10-mo-ex:56}\\
\gll enakku avar.ai pidikkum / teriyum\\
\textsc{1.sg.dat} \textsc{3m.sg.acc} like { }  know\\
\glt ‘I like / know him.’
\z

\ea  Modern Standard Hindi (own data) \label{10-mo-ex:56b-rep}\\
\gll mujhe vah *usko acchā lagtā hai / milā\\
\textsc{1.sg.dat} \textsc{3sg} \textsc{3sg.acc} good seem \textsc{prs.3sg} { } meet.\textsc{pfv}\\
\glt ‘I like / met him.’ 
\z

Moreover, the wide time gap observed before the borrowing makes the hypothesis of a structural borrowing dubious. Similarly, \ili{Austro-Asiatic} languages which also played a non-trivial role in the evolution of early \ili{Indo-Aryan} \citep{Witzel1995Early}, have always been around so that a sudden borrowing in the second millennium is little convincing. 
They consistently index human objects as well as beneficiaries on the predicate, but do not index inanimates, whatever their syntactical function, since indexing is constrained by semantics, particularly the \isi{animacy} and activity, and by the general grammatical structure as in semantically aligned languages. 
Moreover, they do not have differentially marked objects:

\ea
\label{10-mo-ex:57}
\ea \label{10-mo-ex:57a}
\gll (in) lel-jad-in-a-e\\
 (\textsc{1sg}) see-\textsc{pst-1sg-v-3sg} \\
 \glt ‘He saw me’ (V marks the predicative function, in a language with no noun-verb polarity)
 
 \ex \label{10-mo-ex:57b}
 \gll (in) om-am-tan-a-in~\\
 (\textsc{1sg}) give-\textsc{2sg-prs-v-1sg}\\
\glt ‘I give (it/them) to you’
\z
\z

Such features can only very indirectly be deemed responsible for new features in IA, whether DSM \citep{Montaut2013Rise} or DOM, yet they may have acted as favoring factor. 

The other possible source in terms of contact is \ili{Persian}, which came to be the dominant cultural and administrative language at the time when DOM became systematic in \ili{Hindi} (16\textsuperscript{th} c.\, onwards). Extremely influential in the renewal of the predicate lexicon by means of complex predicates \citep{Montaut2015Noun-Verb}, \ili{Persian}, which extensively uses a marker (\textit{râ}) (originally a topic marker) for specific objects, is also sometimes credited to have triggered DOM in \ili{Hindi}/Urdu. \citet[143]{Krishnamurtietal1986South} observe that the development of DOM is more developed in the North Western IA languages than in central and Eastern ones, and conclude on a probable influence of \ili{Persian} and more generally of central Asian languages.\footnote{Eastern IA has other devices for marking specificity such as the so-called “article” or “classifier” -\textit{ṭa}, which does not co-occur with the accusative marker as shown by \citet{Dasgupta2015Basic}. Besides, all Dardic languages, spoken in the North West of the South Asian area, have always shared features with Iranian languages, before the Mughal Empire which marked the entrance of \ili{Persian} as a cultural language in Central India.} 

While none of these hypotheses fully explains the rise of DOM in \ili{Indo-Aryan} – as expected in keeping with its interpretation as a mainly discourse factor – the latter, allowing for a possible convergence with other substrata in the sub-continent, must definitely be taken into account. The origin of the new case markers has in contrast nothing to do with contact.

\subsection{The origin of the case markers}
\label{10-mo-subsec:6.2}

Since the function is anterior to the morphological renewal of markers as seen in examples \REF{10-mo-ex:28b}, \REF{10-mo-ex:30} and \REF{10-mo-ex:31} with inflectional forms in -\textit{hī} (\sectref{10-mo-sec:3.1}), one can expect that some other case marker, already present in the language, extends its range of functions to the marking of certain objects, and that the dative is chosen for such an extension as for instance the \ili{Spanish} preposition \textit{a}. But the new \ili{Hindi} marker appeared at the same time as the other case markers, continuing the oblique flexion of the earlier language, which was largely syncretic and not restricted to goals. It is obvious, however, that in all IA languages, although they display several distinct forms of markers for accusative, the same marker is now used for dative (including DSM) and marked accusative \citep{Krishnamurtietal1986South}: the case meaning specialization (its syntactic function) came later than the marking itself of DOM, and the double use of a single marker as a dative and an accusative has a logic \textit{per se,} which is found in too many languages in the world to be specific to the area. 

Now the question remains: why are there so many morphologically unrelated markers for dative/\isi{accusative case} in languages which are so closely related, in contrast with \ili{Dravidian} languages, which all exhibit related forms? \ili{Marathi} for instance has \textit{lā}, \ili{Gujarati} has \textit{ne}, Konkani, until recently considered a dialect of \ili{Marathi} has -\textit{k}; \ili{Hindi}/Urdu has \textit{ko}, Punjabi which is structurally extremely close to \ili{Hindi}/Urdu and established a distinct identity after the 16\textsuperscript{th} c. has \textit{nū͂}, \ili{Hindi} “dialects” such as central Paharis (Garhwali, \ili{Kumaoni}) have \textit{saṇī}, Eastern Pahari, such as \ili{Nepali}, has \textit{lāi.}

The basis used most extensively is \textit{lā} (\textit{le, lāi, lai}), \textit{ko} (\textit{kau, kū, kū͂}) or \textit{ne} (\textit{nai, ne͂, nū͂}), and neither of them, except \textit{lā} to a certain degree, derives from a clearly allative notion. The base for \textit{lā} and its reflexes for instance is generally derived since \citet{Beames1970Comparative} from the verbal root \textit{lag}, meaning ‘touch’, ‘be stuck to’ (although some scholars have suggested the verb \textit{labh} ‘to get, obtain’ as an alternative derivation \citep{Tiwari1955Hindi}. The regular path is as follows; \textit{lagya} ‘having come in touch with’ > \textit{lage} >\textit{laï}, \textit{lai (le)} ‘for the sake of’, ‘with the object of’ \citep{Juyal1976Madhya}. As for \textit{ko} and its reflexes, it comes from the \ili{Sanskrit} noun \textit{kakṣa} ‘side, place’, with intermediate forms closer to the original in certain Pahari varieties (\textit{kakh, kākh, kakhā̃}), initially a locative, which further developed a directional meaning, then became dative/accusative marker \citep[325]{Strnad2013Morphology}. Similarly \textit{ne} and its reflexes were initially locatives derived from a \ili{Sanskrit} noun meaning ‘ear’: a shortened form of \textit{kan̩haĩ} according to \citet{Tessitori1914Notes}, 
from *\textit{karn̩asmin} (itself a reconstructed analogical locative of \ili{Sanskrit} \textit{karn̩e}, the \isi{locative case} of the noun ‘ear’), which is attested in Apabhramsha as \textit{kan̩n̩ahī} and developed the meaning ‘aside, near’, then ‘towards, to’. \citet[401]{Trumpp1872Grammar} also gives the original meaning ‘near’ for \textit{naï/ne}, a derivation accepted by \citet{Tiwari1955Hindi,Tiwari1966Origin} and by \citet{Chatak1980Madhya} who relates to it the alternate form \textit{kuṇī,} frequent in Garwhali (central Pahari). The originally locative meaning is very clear in~\REF{10-mo-ex:58}:

\protectedex{
\ea  Old Rajasthani \citep[68--70]{Tessitori1914Notes} \label{10-mo-ex:58}\\

\ea \label{10-mo-ex:58a}
\gll cārāï naï nirmala nīra\\
 road \textsc{loc} pure water\\
\glt ‘A limpid lake close by the road’
\z

\ea \label{10-mo-ex:58b}
\gll ā̃vyā 	rā 	kan̩hai\\
 come.\textsc{m.pl} king \textsc{loc/all}\\
\glt ‘[They] went to the Raja (king)’
\z

\ea \label{10-mo-ex:58c}
\gll  te 	savihū̃  	naï 	karaũ 	paranām \\
 \textsc{3pl} all.\textsc{obl} \textsc{loc/all} do.\textsc{prs.1sg} salutation\\
\glt ‘I bow to all of them (in front of/ for)’ 
\z
\z
}

The adessive/locative meaning still visible in~\REF{10-mo-ex:58a}, is also the original meaning of the ‘side’ base (at the origin of \textit{ko}), and the main meaning of the ‘touch/be in contact’ base (origin of \textit{lai}). As a matter of fact, the word ‘ear’, is according to \citet[121]{Heineetal2002World}, a very infrequent source for dative, and mentioned only as a source for locative.

Other sources for DOM markers are even farther from a goal source or they are semantically totally empty: not common but not rare either (it is present in \ili{Sinitic} languages, \cf \citealt{Chappell2014Sources}), is the comitative source, which is found in markers such as Garhwali/\ili{Kumaoni} \textit{saṇī (haṇī}), from the \ili{Sanskrit} noun \textit{sanga} ‘society, company’, then ‘with’, now the dative/accusative most usual marker. Other unusual markers, also used in Pahari languages, are \textit{taĩ, tai}, derived from the locative of the \isi{indefinite} \textit{tavati} (\textit{tāvah\~\i, tāmh\~\i *taa\~\i, *tann\~\i, tā\~\i}) ‘so long, so far, up to, till’, \textit{thaĩ} from the existential verb \textit{sthā} ‘stand’, ‘exist’ and \textit{te/tī}, from the present participle of the verb ‘be’ in the locative (Sk \textit{bhavati} > \textit{hontai}, \textit{hunti}). One marker has not yet been convincingly traced to a reliable origin, \textit{baĩ, be}, dative/accusative marker in modern Kullui (Western Himalaya) as well as in Bundeli (South Madhya Pradesh).

This \textit{be} is perhaps related to the Garhwali/\ili{Kumaoni} \textit{bāt̩i,} used in these languages as an ablative, and derived from the verbal noun \textit{vartamāna} (from \ili{Sanskrit} \textit{vr̥t} ‘turn, expand’, then ‘what happens’, ‘present’). Ablative and goal obviously encode with opposite semantic meanings, but similar “opposite” uses of case markers are extremely common across IA: \textit{te/tī} is also used as an ablative in other northern dialects, \textit{ne} is a frequent marker for ergative (\ili{Hindi}/Urdu, Panjabi, \ili{Marathi}) and \textit{le} (a reflex of \textit{lai}) is the \ili{Nepali} and \ili{Kumaoni} ergative marker.

Even more striking is the fact that, in the very same language, the same marker may work as an ergative, an instrumental/ablative, and a dative/accusative, as is the case in \ili{Bangaru} in both southern \REF{10-mo-ex:59a} or northern \REF{10-mo-ex:59b}--\REF{10-mo-ex:59c} varieties: 

\ea \ili{Bangaru}
\label{10-mo-ex:59}
\ea \label{10-mo-ex:59a}
\gll rupay tī us-tī le lo\\
 money \textsc{acc} \textsc{3sg-abl} take take.\textsc{imp}\\
\glt ‘Take the money from him’ \citep[177]{Tiwari1955Hindi}

\ex \label{10-mo-ex:59b}
\gll kutte nae dande nae mārya\\
 dog \textsc{acc} stick \textsc{ins} strike.\textsc{ins}\\
 \glt ‘Strike the dog with the stick’ \citep{Singh1970Descriptive}

\ex \label{10-mo-ex:59c}
\gll balkā nae tor̩iyā honge\\
 child.\textsc{m.pl} \textsc{erg} break \textsc{prsumpt.3m.pl}\\
\glt ‘The children have probably broken [it]’ \citep{Singh1970Descriptive}
 \z
 \z

All the IA case markers are derived from words with such a vague semantic content that they are able to fulfill all casual functions, with the exception of the new locative \textit{me͂/mā}, even if in most languages they are now more or less specialized into broad functions. New functions (DOM, EXP) as well as inherited ones (ERG, DAT, INS) selected any of the available markers when \isi{case marking} shifted from the old inflections, by then much eroded and syncretic, to the new postpositional system during the first part of second millennium. But, interestingly, none developed a specific marker on the \ili{Dravidian} or \ili{Persian} model, and none selected a DOM marker distinct from the \textsc{dat} one.

\section{Conclusions}

As a result of the identical case-marking for dative and accusative, experiential subjects and marked objects are similarly encoded, and the rise of DOM and DSM is chronologically very comparable: starting with only sporadic non-consistent occurrences during the 14\textsuperscript{th} c. and getting systematic and consistent after the 17\textsuperscript{th} c. Is it an argument for making both processes complementary as suggested by \citet{Aissen2003Differential}? This is highly controversial since experiential subjects are strictly constrained by the lexical semantics of the predicate (and to a certain degree by its morphology since it occurs almost exclusively in \ili{Hindi} with complex predicates), whereas marked objects obey discourse constraints. Specificity can be considered the more important triggering factor for DOM, yet in order to account for those alternations which at first glance seem to be syntactically constrained (\sectref{10-mo-sec:2-3} and \sectref{10-mo-sec:3}) another factor is required, namely discourse saliency. This is not incompatible with \citegen{Dalrympleetal2011Objects} notion of secondary \isi{topicality}, nor with the prominence involved in the twin scales of \isi{animacy} and specificity, yet it also allows us to account for examples where unmarked objects are in a topicalized position and vice-versa. Not surprisingly, the first constraints which emerged during the diachronic evolution of the structure are neither \isi{animacy} nor specificity but \isi{discourse prominence}, of which prosodic requirements can be considered an auxiliary. Besides, the existence of a threefold distinction between objects (‘incorporated’, unmarked and accusative-marked) in nominalizations, depending on their individuation, has no equivalent for subjects.


\section*{Abbreviations}
\begin{tabularx}{.45\textwidth}{lQ}
\textsc{1} & first person\\
\textsc{2} & second person\\
\textsc{3} & third person\\
\textsc{acc} & accusative\\
\textsc{all} & allative\\
\textsc{cond} & conditional\\
\textsc{cv} & coverb\\
\textsc{dat} & dative\\
\textsc{def} & definite\\
\textsc{dem} & demonstrative\\
\textsc{det} & determiner\\
\textsc{erg} & ergative\\
\textsc{ez} & ezafe\\
\textsc{f} & feminine\\
\textsc{foc} & focus\\
\textsc{fut} & future\\
\textsc{gen} & genitive\\
\textsc{h} & human\\
\textsc{hon} & honorific\\ 
\textsc{imp} & imperative\\
\textsc{indef} & indefinite\\
\textsc{inf} & infinitive\\
\end{tabularx}
\begin{tabularx}{.45\textwidth}{lQ}
\textsc{ins} & instrumental\\
\textsc{intr} & intransitive\\
\textsc{m} & masculine\\
\textsc{neg} & negation, negative\\
\textsc{non} & non-\\
\textsc{nom} & nominative\\
\textsc{obl} & oblique\\
\textsc{pass} & passive\\
\textsc{pfv} & perfective\\
\textsc{pl} & plural\\
\textsc{poss} & possessive\\
\textsc{pprf} & pluperfect\\
\textsc{prf} & perfect\\
\textsc{prog} & progressive\\
\textsc{prs} & present\\
\textsc{prsumpt} & presumptive\\
\textsc{pst} & past\\
\textsc{refl} & reflexive\\
\textsc{rel} & relative\\
\textsc{sg} & singular\\
\textsc{top} & topic\\
\textsc{v} & predicative function\\
\end{tabularx} 

{\sloppy
\printbibliography[heading=subbibliography,notkeyword=this] }
\end{document}