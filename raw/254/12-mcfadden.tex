\documentclass[output=paper, modfonts, nonflat]{langsci/langscibook} 
\ChapterDOI{10.5281/zenodo.3541765}

\title{The morphosyntax of allocutive agreement in Tamil} 
\author{Thomas McFadden\affiliation{Leibniz-Zentrum Allgemeine Sprachwissenschaft}}


\abstract{In this paper I examine allocutive agreement in Tamil,
  a phenomenon in which an agreement suffix attached to the verb or
  other clause-final element indexes features not of any argument, but
  of the addressee of the speech act. I report in detail on the
  morphophonology, syntactic distribution and discourse use of this
  agreement, supplementing the basic facts reported by
  \citet{amrit:1991} with several additional crucial details, and
  compare the Tamil data with what has been reported for other
  languages, especially Basque and Japanese. I then discuss the
  consequences of Tamil allocutive agreement for the theoretical
  treatment of how discourse information interacts with the
  morphosyntax, leading to a preliminary analysis of the patterns I
  find. The Tamil data presented in the paper provide interesting insights into the structural representation of the addressee and into how
  allocutive agreement is derived, in particular from how the relevant
  suffix is ordered relative to other verbal material.}

\begin{document}

\maketitle
\section{Background}

\subsection{Introduction by example}

In many colloquial varieties of Tamil (Dravidian; South Asia), one
commonly comes across utterances of the following kind:

\ea\label{alloc}
\gll Naan ʤaangiri vaang-in-een-ŋgæ.\\
I Jangri buy-\textsc{pst}{}-1\textsc{sg.sbj}-\textsc{alloc}\\
\glt `I bought Jangri.'\footnote{Jangri is a delicious sweet made by
  deep-frying a type of lentil batter in flower shapes and then
  soaking them in sugar syrup.} 
\z 
%
Aside from the good news it brings to the hearer, (\ref{alloc}) is of
interest because it contains two different types of agreement stacked
on top of each other. First, there is the suffix \textit{-een}, which
marks the unremarkable agreement of a finite verb with its subject
that is found in a significant portion of the languages of the
world. Second, there is the suffix \textit{-ŋgæ}, glossed here as
\textsc{alloc}, which marks a rather different kind of agreement that
is far less widely attested. Specifically, rather than
cross-referencing properties of one of the arguments of the verb, it
provides information about the addressee, specifically that this
sentence is addressed to either a group or an individual with whom the
speaker would use polite forms. If addressed instead to a single
person with whom the speaker would use familiar forms, this suffix is
simply lacking, as in (\ref{noalloc}):

\ea\label{noalloc}
\gll Naan ʤaangiri vaang-in-een.\\
I Jangri buy-\textsc{pst}{}-1\textsc{sg.sbj}\\
\glt `I bought Jangri.'
\z 

\noindent As we will see directly, similar types of agreement with the addressee
have been described for a number of other languages. It is most
famously found in Basque \citep{oyharcabal:1993}, where it is referred
to as \textit{allocutive agreement}, a term I will adopt here. Allocutive
agreement is of considerable interest, both for the theory and
typology of agreement systems, and for what it can tell us about the
grammatical representation of speech acts and their participants. As
such, looking into the precise distribution of and constraints on this
kind of agreement will hopefully shed some light on currently ongoing
discussion and controversy over the extent to which certain phenomena
normally associated with discourse and semantics may actually have a
morphological and syntactic side.

In this paper, I will present newly collected, detailed data on
allocutive agreement in Tamil and compare it with what has been
reported for other languages, especially Basque and Japanese. We will
see that the Tamil facts resemble those from the other languages in
the broad strokes, but that there are a number of interesting points
of detail where Tamil differs in ways that are relevant for our
theoretical understanding. I will discuss the data before the
background of a recent body of work on the syntactic side of the
representation of speech acts. Of particular interest here will be how
allocutive agreement interacts with the phenomenon of monstrous
agreement in the language discovered by \citet{sundaresan:diss}, where
speech act participants also play a crucial role. I will conclude the
paper with a preliminary analysis of the patterns and some discussion
of how it can serve as a starting point for future investigations.


\subsection{On allocutive agreement}
\label{sec:backallagr}

Allocutive agreement (henceforth abbreviated as \allagr), while far
from common, has been identified in a number of languages from a wide
selection of families (see \citealt{antonov:2015} for an initial typological
overview). Adapting \citet{antonov:2015} slightly, we
can identify something as \allagr{} if it has the following
properties. First, it marks properties of the addressee of the current
speech context, i.e.\ it provides information about the gender, number
or politeness status of the person or persons to whom the utterance is
directed. Second, it is not limited to cases where the addressee is an
argument of the local predicate, so it is to be clearly distinguished
from subject and object agreement, even instances where the 2nd person
might behave in a special way. Third, it involves the use of
grammaticalized morphological markers in the verbal or clausal
inflectional system. This is meant to exclude e.g.\ special vocative
forms like `madam', `sir' or `captain' which may serve similar
functions but are not grammaticalized in the same way and show
different morpho-syntactic behaviors than true \allagr.

As already noted, the classic example of \allagr{} comes from Basque,
for which the term was first introduced by \citet{bonaparte:1862}. In
Basque, the use of \allagr{} depends, in ways that vary across
dialects, on the politeness relationship between the speaker and
addressee as well as the number of the addressee, with the form
reflecting the gender of the addressee \citep[see also][ch.\
5]{oyharcabal:1993, alcazarsaltarelli:2014}. In Standard Basque e.g.,
the agreement only crops up when the speaker and addressee would use
the highly familiar form of address, and then only when the addressee
is singular and is not additionally an argument of the verb. The
examples from the Souletin dialect given in (\ref{basque})
\citep[reported by][]{antonov:2015} illustrate the phenomenon:

\ea\label{basque}
  \ea\label{nobasque}\gll etʃe-a banu\\
  house-\textsc{all}{} 1\textsc{sg}.go\\
  \glt `I am going to the house.'
  \ex\label{mbasque}\gll etʃe-a banu-k\\
  house-\textsc{all}{} 1\textsc{sg}.go-\textsc{alloc}.\textsc{m}\\
  \glt `I am going to the house.' (familiar \textsc{male} addressee)
  \ex\label{fbasque}\gll etʃe-a banu-n\\
  house-\textsc{all}{} 1\textsc{sg}.go-\textsc{alloc}.\textsc{f}\\
  \glt `I am going to the house.' (familiar \textsc{female} addressee)
  \ex\label{rbasque}\gll etʃe-a banu-sy\\
  house-\textsc{all}{} 1\textsc{sg}.go-\textsc{alloc}.\textsc{rsp}\\
  \glt `I am going to the house.' (respected addressee)
  \z
\z
%
Example (\ref{nobasque}) gives the baseline, where the verb only shows 1.\textsc{sg}
agreement with the subject. The remaining examples all have the same
basic meaning, but add allocutive suffixes to this verb form
cross-referencing the addressee, in (\ref{mbasque}) a familiar \textsc{acc}
addressee, in (\ref{fbasque}) a familiar female, and in
(\ref{rbasque}) an addressee with whom the speaker would use the
polite form.\footnote{This last form is only in use in certain
  dialects. In others, including the standard, the form with no
  \allagr{} in (\ref{nobasque}) would be used with a formal
  addressee.} These suffixes indicate information about the addressee
independent of it being an argument. Furthermore, they are fully
grammaticalized verbal inflection forms, appearing in the normal
position for agreement in the language and involving (nearly) the same
forms as those used to agree with a 2nd familiar ergative argument
\citep[see][66f. for discussion of the forms]{antonov:2015}. What
we have here thus clearly meets our criteria for \allagr{}.

% \todo[inline]{R1: make sure that it's clear that the following is not
%   definitional for allocutive agreement in general, but describes its
%   properties in Basque. Reviewer 1 takes issue with the obligatoriness
%   business, but this is not a proposal about what should be considered
%   as AllAgr, rather a description of how it is in Basque}
There are some additional interesting properties of Basque \allagr{}
-- not necessarily exhibited by the phenomenon in other languages, as
we will see, and thus not definitional of \allagr{} in general --
that should be noted here. First, the appearance of the allocutive
suffixes is not actually \emph{independent} of the addressee being an
argument, but rather requires that it is \emph{not}. If the addressee
\emph{is} one of the arguments, it will be coindexed with the
appropriate (ergative, absolutive or dative) 2nd person argument
agreement, and \allagr{} will not appear.\footnote{This plausibly
  reduces to the fact that Basque independently blocks a single
  referent from being coindexed with multiple agreements (e.g.\ in
  reflexives), and argument agreement is obligatory
  \citep{antonov:2015}.} Second, in contexts where the conditions for
it are met, \allagr{} is obligatory, i.e.\ we are dealing with a fully
grammaticalized system, not optional marking of familiarity or respect
(at least in Basque). Third, \allagr{} is generally restricted to root
clauses and, at least in many dialects, is not possible in questions.

\citet{miyagawa:2017} has argued that Japanese politeness marking
should also be analyzed as a type of \allagr{}. Japanese has a range
of constructions and markers belonging to its system of ``honorifics'',
which encode various types of social relationships between the speech
act participants and different nominal arguments in a given
clause. These include lexical choices and verbal affixes that reflect
honorification toward the subject or the object which will not be of
direct concern to us here, because they crucially involve arguments
and need not relate to the addressee. There is also, however, verbal
marking used to indicate politeness or honorification from the speaker
toward the addressee, as in (\ref{japanese}), from\largerpage
\citet{miyagawa:2017}:

\ea\label{japanese}
  \ea\label{fjap}\gll Watasi-wa piza-o tabe-mas-u.\\
  I-\textsc{top} pizza-\textsc{acc}{} eat-\textsc{alloc}-\textsc{prs}\\
  \glt `I will eat pizza.' (formal)
  \ex\label{cjap}\gll Watasi-wa piza-o tabe-ru.\\
  I-\textsc{top} pizza-\textsc{acc}{} eat-\textsc{prs}\\
  \glt `I will eat pizza.' (colloquial)
  \z
\z
%
%/* check out Sawara, former UConn student, mentioned by Jonathan Bobaljik?
Here again, the marker is clearly giving information about the
addressee, independent of what the arguments of the verb are, and
furthermore it is a clearly grammaticalized part of the verbal
inflectional system, appearing as a suffix on the verb, inside of a
tense suffix. 

What makes the case here a bit trickier than the Basque one is that
Japanese doesn't have straightforward argument agreement, never
seeming to indicate standard $\phi$-featural information (i.e.\
person, number or gender) about subjects or objects. Indeed, Japanese
is usually regarded as an agreementless language. However, as noted
above, the language \emph{does} indicate honorification towards the
subject or object in certain constructions, at least some of which
have been argued to involve a type of agreement \citep[see
e.g.][]{boeckxniinuma:2004}. Furthermore, \citet{miyagawa:2017}
argues that the lack of prototypical agreement elsewhere in the
language should not at all dissuade us from recognizing the politeness
marking as a type of agreement. If we assume that there is a universal
set of grammatical features, which are overtly manifested in all
languages \citep[his principle of Strong Uniformity,
from][]{miyagawa:2010}, then the lack of $\phi$-agreement on T
actually leads us to \emph{expect} $\phi$-agreement on C, i.e.\
something like \allagr{}. It just happens to be the case that the type
of $\phi$-features overtly manifested in the language have to do with
honorification rather than person, number or
gender. % Beyond this, there are some additional
% differences of note with respect to Basque \allagr{}. 

\allagr{} patterns have also been reported for Pum\'e (isolate;\largerpage
Venezuela), Nambikware (isolate; Brazil), Mandan (Siouan; North
America) and Beja (Cushitic; Northeast Africa), as summarized by
\citet{antonov:2015}. Beyond the criteria for identifying \allagr{} we
have already discussed, Antonov notes several points about the
typology of the phenomenon. First, languages differ in what
information about the addressee they encode, with gender and varying
types of familiarity or politeness being perhaps most common, and
number being rather less common. Indeed, in his sample it seems to be
found only in Basque, and even here it is limited in most dialects to
the fact that \allagr{} only appears when the addressee is
singular.\footnote{That is, one can infer a partial singular/plural
  contrast from the fact that allocutive marking is always lacking
  when the addressee is plural, though the lack of marking may have
  other causes as well. There are, however, apparently some dialects
  that allow explicit marking of a plural addressee. See
  \citet{antonov:2015} for brief discussion and references.} Second,
languages also differ in how \allagr{} interacts with the argument
status of the addressee. As noted above, it is ruled out in Basque
when one of the arguments is 2nd person, but this restriction does not
seem to apply in any of the other languages surveyed. Finally, while
there is some variation in the distribution of \allagr{} across clause
types, there are clear generalizations to be made. The core
environment, where \allagr{} is found in all of the languages
considered, is root declarative clauses. There is then a fair amount
of variation across the languages in whether it is also found in other
types of root clauses, i.e.\ interrogatives, exclamatives and
imperatives. Basque, for example, excludes it in all of these, Beja
allows it in all of them, and Japanese allows it in interrogatives and
exclamatives, but not imperatives. Finally, in all of these languages,
\allagr{} is heavily restricted or entirely ruled out in embedded
clauses. The details about embedding have been, as far as I am aware,
most carefully examined for Japanese, and it is perhaps no surprise
then that it is here that some embedded environments have been
reported to allow the phenomenon. I will return to this issue once I
have presented the relevant data from Tamil.

From a theoretical perspective, \allagr{} is highly intriguing because
it seems to involve an active role for information about the speech
act in the morphosyntax. It is clear that the identity of the author
and addressee of an utterance, as well as its time and location, play
a role in the semantic and pragmatic interpretation, and thus must be
encoded somehow in the discourse context. This is necessary, among
other things, for the appropriate interpretation of so-called
indexical items, like 1st and 2nd person pronouns and expressions like
`here' and `now'. What is less obvious is whether we need to assume
that a representation of such information is accessible in the
syntax. \allagr{} potentially offers evidence that we do. One could
argue that, if it really is an instance of morphosyntactic agreement,
then there must be some representation of the addressee in the syntax
that it is agreeing \emph{with}. As we will discuss in
\sectref{sec:theoprelim}, this has led to the idea in much recent work
that \allagr{} targets the syntactic representation of the discourse
context.

% conclude that \allagr{} provides evidence for a literal
% syntactic representation of the discourse context, including
% information about the speech-act participants
% \citep{haegemanhill:2011, miyagawa:2012, zu:2015,
%   haegemanmiyagawa:2016, miyagawa:2017}. Indeed, this fits into a
% broader trend of adopting what is sometimes called the
% neo-performative hypothesis, i.e.\ \citet{speastenny:2003}'s
% reinterpretation of \citet{ross:1970}'s idea that the speech-act
% participants are represented not just in the semantics and pragmatics,
% but also in the syntax. Of note here is e.g.\
% \citet{sundaresan:diss}'s use of this approach to model indexical
% shift. 

The only prior work on \allagr{} Tamil that I am aware of is
\citet{amrit:1991}. That (unfortunately all too brief) article reports
the central data, including a number of insightful observations, and
compares the Tamil facts with those in the closely related languages
Kannada and Telugu. However, (as a contribution to a festschrift) it
does not have the space to explore the data in detail, and there are a
number of important points that it does not touch upon. The current
paper will attempt to fill this gap by providing a careful and
extensive description of the empirical situation with respect to
\allagr{} in Tamil. We will see that it displays a number of
properties in the language that are of theoretical interest, some of
which distinguish it crucially from what has been reported for other
languages.


\subsection{Some relevant properties of Tamil}

%\subsection{The sociolinguistic situation and the sources of my data}

Tamil is a Southern Dravidian language, spoken by approximately 70
million people, primarily in southern India and Sri Lanka, as well as
a significant diaspora, e.g.\ in Malaysia, Singapore, Mauritius and
South Africa. As the second classical language of India after
Sanskrit, it has a written tradition going back over two thousand
years. For better or worse, the written standard is extremely
conservative, approximating a rather archaic variety of the language,
and differs significantly in all aspects of grammar and lexicon from
contemporary spoken varieties. This leads to a marked diglossia, such
that there is even a spoken version of the literary language used
e.g.\ for newscasts and political speeches. While there is arguably a
contemporary standard version of the spoken language, used e.g.\ in
films and television talk shows \citep[see e.g.][]{schiffman:1999,
  asherannamalai:2002}, there is no generally agreed-upon written form
for this variety, nor is there any standard romanization. The
\allagr{} that is of interest here is very much a phenomenon of the
colloquial language, not the written standard, and the form of the
data presented will reflect this fact. I adopt essentially the
transliteration used by \citet{sundaresan:diss}, which attempts to
reflect the phonology of the standard colloquial variety, without
going into too much phonetic detail.

Contemporary Tamil is also characterized by extensive dialectal
variation, reflecting communities defined by geography as well as
socio-economic and religious factors. Here as well, the variation is
relevant for our considerations of \allagr. For one thing, the
phenomenon is largely restricted to non-Brahmin dialects, plausibly
related to the fact that Brahmin dialects use different forms for 2nd
plural and polite agreement than the one that is involved in
\allagr. I conducted sessions with three speakers of Iyer Brahmin
Tamil from Chennai and Tiruchirapalli, in which I was able to confirm
that they make little or no use of \allagr{} in their native
dialect.\footnote{However, they generally use a hybrid of Brahmin and
  non-Brahmin colloquial Tamil when speaking with non-Brahmins. In
  such circumstances they do make use of \allagr, though to a more
  limited extent. Being regularly exposed to other spoken varieties in
  the media and daily interactions, they also have passive command of
  its use by other speakers. I found in my sessions that they have
  clear intuitions about core uses of \allagr, but are less certain
  about points of detail. I will refer to this dialect as \textit{Central
  Iyer} henceforth, and will note at certain points where these
  speakers have intuitions that differ in an interesting way from
  those of my primary informant.}  Furthermore, while the basic
\allagr{} patterns are found in a wide array of colloquial varieties,
there is variation -- at least partly geographic -- in the frequency
with which it is used. My primary informant is from Pollachi, in the
Kongu Nadu region surrounding Coimbatore, which is reputed to be an
area that makes particularly heavy use of \allagr. This has the
advantage that he has quite robust intuitions about the phenomenon. I
will follow local practice and refer to his dialect as \textit{Kongu
Tamil}. I also collected preliminary data from two speakers of
Singapore Tamil, which suggest that the phenomenon is more restricted
there and also subject to different constraints regarding the ordering
of affixes, as will be briefly mentioned below.  Given all of these
factors, I have chosen to focus here on the patterns found in the
speech of my Kongu Tamil informant, as he was able to provide the most
extensive and consistent data on \allagr. Thus unless otherwise
indicated, the examples provided here come from my sessions with
him. Note that I have made no attempt to systematically investigate
the dialectal distribution of the phenomenon, but am simply
registering here that relevant differences do exist.  Speakers of
other dialects of the language should thus not be expected to agree
with all of the judgments reported.

Regarding the research methodology, with my Kongu Tamil and Central
Iyer informants, I used a questionnaire of pre-constructed sentences,
combined with elicitation based on translation of English examples for
some of the more complex structures. The questionnaire was based on a
combination of my own prior observations of the phenomenon in
naturally occurring speech, data from \citet{amrit:1991} and
additional sentences constructed based on patterns reported for other
languages in the literature. With the two informants from Singapore, I
collected basic judgment data on the core patterns in brief, informal
interviews.

%\subsection{A primer on relevant aspects of Tamil morphosyntax}

Tamil is a highly inflecting language with a strongly agglutinative
character, though it shows some fusional tendencies, and is almost
exclusively suffixing. Syntactically speaking, it is SOV and indeed
quite generally head-final, allows pro-drop of all arguments and has
long-distance anaphors. The language has a nominative-accusative case
system with differential object marking, and distinguishes a total of
six or seven cases marked by suffixes on nouns and pronouns in
addition to the unmarked nominative. The pronominal system includes an
inclusive/exclusive distinction in the first person and a local/distal
distinction in the third person forms. Politeness is indicated by
plural forms in the 2nd person and by distinct pronominal forms in the
3rd person (historically related to older plural forms), and
occasionally also with plural marking on nouns. Three genders are
distinguished -- masculine, feminine and neuter -- corresponding
essentially to the notional status of the referent, and play a role in
both the pronominal and verbal agreement systems. The language has a
range of complex predication constructions, and its verbs display an
impressive array of participial and nominalized forms, many of which
can head particular types of non-finite clauses, and often include
aspectual marking. Finite verbs can be marked for transitivity,
aspect, passive and middle voice, mood, negation, tense and agreement
\citep[]{sundaresanmcfadden:tamtrans}. There are, however, interesting
restrictions on co-occurrence, as e.g.\ mood, negation and agreement
are essentially in complementary distribution
\citep[][]{amritjay:2005}.

Let us focus then on agreement. Standard verbal agreement targets the
highest nominative argument in the clause, which is typically the
subject, but may also be an object if the subject is marked with a
(quirky) dative or locative case, as in (\ref{objagr}) from
\citet{baker:casepp}.\footnote{Transliteration and formatting have
  been modified to fit the system used elsewhere in the
  paper.}

\ea\label{objagr}\gll En-ækk\U{} andæ{} poɳɳ\U{} teeve-ppaɖ-r-aa\\ 
I-\textsc{dat}{} that girl.\textsc{nom}{} need-suffer-\textsc{prs}-3\textsc{sg.f}\\ 
\glt `I need the girl.'
\z
%
The form of the agreement reflects person and number, as well as
gender in the 3rd person and politeness in the 2nd and 3rd
persons. The inclusive/exclusive distinction is \emph{not} reflected
by agreement, there being a single 1st plural form used for
both. \tabref{tab:agrforms} shows the regular agreement paradigms
for the simple present tense and imperative forms of \textit{ooɖ\U}
`run'.\footnote{The \textit{-r-} suffix found before the agreement
  suffixes outside of the 3rd neuter forms marks the present
  tense. Tense marking generally interacts in odd ways with neuter
  agreement.}
  
\begin{table}
  \caption{Regular verb agreement in Tamil\label{tab:agrforms}}
  \begin{tabular}{lll} 
    \lsptoprule
    & \textsc{sg} & \textsc{pl}\\ 
    \midrule
    1 & ooɖ\U-r-een & ooɖ\U-r-oom \\
    2 & ooɖ\U-r-æ & ooɖ\U-r-iiŋgæ\\
    3\textsc{f} & ooɖ\U-r-aa & ooɖ\U-r-aaŋgæ\\
    3\textsc{m}& ooɖ\U-r-aan & ooɖ\U-r-aaŋgæ\\
    3\textsc{pol}& ooɖ\U-r-aar\U & ooɖ\U-r-aaŋgæ\\
    3\textsc{n} & ooɖ\U-d\U & ooɖ\U-d\U \\
    \textsc{imp} & ooɖ\U & ooɖ\U-ŋgæ\\
    \lspbottomrule
  \end{tabular}
\end{table}\noindent The agreement suffix follows all aspect, tense and voice markers.
We can see an example of a moderately complex, fully inflected finite
verb in (\ref{prog}), where the combination of an aspectual marker
followed by the suffix \textit{kiʈʈ\U} and a form of `be' forms a
progressive, to which tense and agreement are further suffixed.

\ea\label{prog}\gll Kausalya paɖi-\v{c}\v{c}\U-kiʈʈ\U-ru-nd-aa\\
Kausalya study-\textsc{asp}-\textsc{nom}-\textsc{be}-\textsc{pst}-3\textsc{sg.f}\\
\glt `Kausalya was studying.'
\z
%
The finite verb, terminated by the agreement suffix, is typically the
final element in a root declarative clause (aside from extraposed
material), but it can be followed by further suffixes that we might
expect to be in the C domain, e.g.\ the complementizer \textit{-nn\U}
as in (\ref{nnu}) or the polarity question particle \textit{-aa} as in
(\ref{yn}):

\ea
  \ea\label{nnu}\gll Venkaʈ [Kausalya paɖi-\v{c}\v{c}\U-kiʈʈ\U-ru-nd-aa]-nn\U{} so-nn-aan\\
  Venkat [Kausalya  study-\textsc{asp}-\textsc{nom}-\textsc{be}-\textsc{pst}-3\textsc{sg.f}]-\textsc{comp}{} say-\textsc{pst}-3\textsc{sg.m}\\
  \glt `Venkat said that Kausalya was studying.'
  \ex\label{yn}\gll Kausalya paɖi-\v{c}\v{c}\U-kiʈʈ\U-ru-nd-aaɭ-aa?\\
  Kausalya study-\textsc{asp}-\textsc{nom}-\textsc{be}-\textsc{pst}-3\textsc{sg.f}-\textsc{q}\\
  \glt `Was Kausalya studying?'
  \z
\z
%
The \textit{-ɭ} that suddenly appears before the question particle in
(\ref{yn}) is part of the underlying form of the agreement suffix,
which is deleted in coda position, but surfaces when a vowel-initial
suffix immediately follows within the same word. This is a common
phenomenon in the morphophonology of Tamil, and crops up also in the
various plural agreement forms ending in \textit{-ŋgæ}, which surface
as \textit{-ŋgæɭ-} before vowel-initial subjects. As we will see,
this includes \allagr{} marker.


\section{The core data}

In this section I will present the empirical details on Tamil \allagr,
again based on a combination of what was already reported by
\citet{amrit:1991} and what I have collected in the work with my
informants. We will see that it meets all the criteria to be
considered genuine allocutive agreement, but that it also shows
interesting details in its behavior that distinguish it from what has
been reported for Basque, Japanese and other languages.

\subsection{The morphophonology of the suffix}

A central part of the argument that the Tamil phenomenon of interest
here really is a type of agreement, rather than e.g.\ a specialized
vocative (along the lines of English \textit{sir/ma'am} or certain
uses of \textit{guys}) or a speech act particle (like those discussed
for Romanian and West Flemish by \citealp{haegemanhill:2011}), comes
from the form and position of the actual marker. Let us take the basic
example we started with in (\ref{alloc}), repeated here in
(\ref{alloc2}), as a basis for the discussion:

\ea\label{alloc2}
\gll Naan ʤaangiri vaang-in-een-ŋgæ.\\
I Jangri buy-\textsc{pst}{}-1\textsc{sg}.\textsc{sbj}-\textsc{alloc}\\
\glt `I bought Jangri.'
\z
%
As we see, the shape of the allocutive suffix is \textit{-ŋgæ}. It
turns out that this serves as a rather general \emph{plural} marker
throughout the language. Looking back at \tabref{tab:agrforms}, we
see that it is the final component of all of the 2nd and 3rd person
plural agreement markers (setting aside the 3rd neuter marker, which
simply doesn't distinguish number), and it is the marker attached to
the verb root to form (2nd) plural imperatives. Furthermore, it, or
the related form \textit{-gæ}, is used as the plural marker in a
number of nominal categories, as shown by the examples in \tabref{tab:plnom}.

\begin{table}
  \caption{Number marking in Tamil nominals\label{tab:plnom}}
  \begin{tabular}{lll} 
    \lsptoprule
    & \textsc{sg} & \textsc{pl}\\ 
    \midrule
    1\textsc{excl} & naan & naaŋgæ\\
    2 & nii & niiŋgæ\\
    3\textsc{m} & avan & avaŋgæ\\
%    3\textsc{n} & adu & aduŋgæ\\
    `girl' & poɳɳ\U & poɳɳ\U gæ\\
    `tree' & maram & maraŋgæ\\
    \lspbottomrule
  \end{tabular}
\end{table}
In the (rather common) case that a noun or pronoun stem ends in a
nasal, it is impossible to tell whether the plural suffix is
\textit{-gæ} or \textit{-ŋgæ}.\footnote{Note that in all of these
  cases, whether with \textit{-gæ} or \textit{-ŋgæ}, there is an
  underlying final \textit{-ɭ}, which surfaces when any vowel-initial
  suffix follows.} But even with vowel-final nouns, there is a fair
amount of variation between the two. Interestingly enough, the two
main instances where the plural ending is unambigously \textit{-ŋgæ}
are both in the 2nd person, in the 2nd person pronoun itself (since
2nd singular \textit{nii} ends in a vowel) and in the plural
imperative suffix, which is \textit{-ŋgæ} regardless of what the verb
root ends in. To summarize all of this we can say that \textit{-ŋgæ}
is a plural ending which always occurs in the 2nd person and variably
occurs elsewhere. Note again that the 2nd plural forms are also used
for politeness with singular addressees. As we will see below, this
will allow us to understand its use in AllAgr as also involving
plural marking. 

Example (\ref{alloc2}) also demonstrates that the allocutive marker
attaches to the clause-final verb, after all of the other inflectional
suffixes that might precede it, including tense, aspect, voice and
argument agreement. This is also true when the verb has a modal or
negative suffix rather than agreement. Again, the allocutive marker
follows at the very end of the verb form, as demonstrated by the
sentences in (\ref{modneg}), based on examples from
\citet{amrit:1991}.

\ea\label{modneg}
 \ea\label{modal}\gll koɻandæ{} ippaɖi sejjæ-kkuuɖaad\U-ŋgæ\\ 
 child {like.this} do-{must.not}-\textsc{alloc}\\
 \glt `The child should not act in such a way.'
 \ex\label{neg}\gll Venkaʈ varæ-læ-ŋgæ\\
 Venkat come-\textsc{neg}-\textsc{alloc}\\
 \glt `Venkat didn't come.'
 \z
\z
%
As we will see in more detail below, the marker can also appear in
clauses without a verb, like (\ref{rajni}), and even in fragmentary or
elliptical utterances that aren't even clauses, as in (\ref{boy}) and
(\ref{yes}):

\ea\label{short}
 \ea\label{rajni}\gll naan aaʈʈookkaaran-ŋgæ\\
 I automan-\textsc{alloc}\\
 \glt `I am an auto rickshaw driver.'
 \ex\label{boy}\gll indæ{} payyan-ŋgæ\\
 this boy-\textsc{alloc}\\
 \glt `this boy' (e.g.\ as answer to `Who's next?')
 \ex\label{yes}\gll illæ-ŋgæ\\
 no-\textsc{alloc}\\
 \glt `No' (as answer to polar question)
 \z
\z
%
It is important to note that the \allagr{} marker can also co-occur
with unambiguous vocatives. It occurs strictly attached to the verb,
with the vocative obligatorily coming outside (typically extraposed
past the end of the clause), which confirms that \textit{-ŋgæ} itself
cannot be a vocative:

\ea\label{vocative}
 \ea[]{
   \gll naan va-r-een-ŋgæ{} saar\\
   I come-\textsc{prs}-\textsc{1sg}.\textsc{sbj}-\textsc{alloc}{} sir\\
   \glt `I'll take my leave, sir.'}
 \ex[*]{
   \gll naan va-r-een saar-ŋgæ{}\\
   I come-\textsc{prs}-\textsc{1sg}.\textsc{sbj}{} sir-\textsc{alloc}\\}
 \z
\z
%
The basic generalization is that the marker attaches to whatever is
final in the clause or sub-clausal utterance (again, ignoring
extraposed material), regardless of what category that might be. Note,
though, that it is clearly a bound form, not an independent word or
particle. It never appears alone, or after a pause, always being
attached to a preceding word. Indeed, the sequence \textit{ŋg-} is
not licit word-initially in Tamil phonotactics.\footnote{I will not
  attempt to determine here whether it should be considered a suffix
  or an enclitic, in part because I am not familiar with any arguments
  about whether this is actually a meaningful distinction in the
  language.}

\subsection{Distribution of allocutive agreement}

Let us now turn to the conditions under which allocutive agreement
appears and does not appear in Tamil.
%\subsubsection{Used when addresse is plural or polite}
The central determining factor is the identity of the addressee and
their relationship with the speaker. There is only one allocutive
suffix in the language -- unlike e.g.\ Basque, which distinguishes
two or three forms depending on the dialect -- thus the number of
distinctions that can be made is minimal.\footnote{The language does
  additionally have two particles, feminine \textit{-ɖii} and
  masculine \textit{-ɖaa}, which have a similar function in marking
  properties of the addressee -- specifically gender and intimacy --
  but their morphosyntactic behavior is somewhat different (e.g.\ they
  can co-occur with the \textit{-ŋgæ} suffix, strictly ordered after
  it, and show different ordering relative to the polar question
  particle to be discussed below). How exactly they fit into the
  overall picture presented here is a matter of ongoing research.}
Quite simply, the agreement is found whenever \textit{niiŋgæ}
would be the appropriate 2nd person pronoun, i.e.\ when the addressee
is plural or is a singular individual with whom the speaker would use
the polite form of address. Thus an utterance like (\ref{dunno}) would
be appropriate when addressed to a group of friends or to an adult
stranger, but not to an individual friend.\footnote{For my Central
  Iyer speakers, the allocutive suffix is only used to reflect
  politeness, not plural, i.e.\ for them (\ref{dunno}) could not be
  used with a group of friends.} 

\ea\label{dunno}\gll enæ-kk\U{} teri-læ-ŋgæ\\
me-\textsc{dat}{} know-\textsc{neg}-\textsc{alloc}\\
\glt `I don't know.'
\z
%
Note again that the addressee is not in any way an argument of `know',
or of any other overtly expressed predicate in the sentence, nor does
the addressee figure in as an adjunct in any way to the eventuality
described here. The only role for the 2nd person here is as the
addressee of the speech act. This again makes it clear that what we
are seeing is not any kind of argument agreement, or even something
like an ``ethical dative'', but rather true \allagr.

%\subsubsection{Interaction with 2nd person arguments}

A question we might ask then is what happens with \allagr{} when the
2nd person \emph{is} an argument of some predicate in the
utterance. As we noted above, \allagr{} is blocked in Basque in such
circumstances, while some of the other languages discussed by
\citet{antonov:2015} allow it. Tamil shows a somewhat mixed behavior,
which is quite instructive. When the \emph{subject} is 2nd person and
it triggers regular argument agreement on the verb, \allagr{} is ruled
out, as shown in (\ref{2ndagr}):\footnote{At least one of my Central
  Iyer speakers accepts examples like these where \allagr{} appears on
  top of 2nd person subject agreement.}

\ea\label{2ndagr}
  \ea[*]{
    \label{howare}\gll eppaɖi iru-kk-iiŋgæ-ŋgæ?\\
    how be-\textsc{prs}-2\textsc{pl}-\textsc{alloc}\\
    \glt `How are you?'}
  \ex[*]{
    \label{smart}\gll niiŋgæ{} rombaa smart-aa iru-kk-iiŋgæ-ŋgæ\\
    you.\textsc{pl} very smart-\textsc{pred}{} be-\textsc{prs}-2\textsc{pl}-\textsc{alloc}\\
    \glt `You're very smart.'}
  \z
\z
%
The question is whether the problem here is the fact that the
addressee is an argument at all, or that it triggers
agreement. % \footnote{The \textit{\U} at the
  % beginning of the \textsc{alloc}{} suffix in (\ref{2ndobj}) is epenthetic,
  % inserted whenever the preceding element is consonant-final.} 
This is resolved by the examples in (\ref{2ndarg}), all of which
involve 2nd person arguments combined with \allagr:

\ea\label{2ndarg}
  \ea[]{
    \label{2ndobj}\gll naan ongaɭ-æ{} paɖatt-læ{} paa-tt-een-ŋgæ\\
    I you.\textsc{pl}.\textsc{obl}-\textsc{acc}{} film-\textsc{loc}{} see-\textsc{pst}-1\textsc{sg}.\textsc{sbj}-\textsc{alloc}\\
    \glt `I saw you in a film.'}
  \ex[]{
    \label{2ndquirk}\gll ongaɭ-\U kk\U{} coffee veeɳum-aa-ŋgæ?\\
    you.\textsc{pl}.\textsc{obl}-\textsc{dat}{} coffee want-\textsc{q}-\textsc{alloc}\\
    \glt `Do you want coffee?'}
  \ex[]{
    \label{2ndnoagr}\gll niiŋgæ{} saap-ʈ-aach-aa-ŋgæ?\\
    you.\textsc{pl}{} eat-\textsc{asp}-\textsc{res}-\textsc{q}-\textsc{alloc}\\
    \glt `Have you eaten?'}
  \ex[*]{
    \label{2ndpst}\gll niiŋgæ{} saap-ʈ-iiŋgæɭ-aa-ŋgæ?\\
    you.\textsc{pl}{} eat-\textsc{pst}-2\textsc{pl}-\textsc{q}-\textsc{alloc}\\
    \glt `Did you eat?'}
  \z
\z
%
(\ref{2ndobj}) shows that \allagr{} is perfectly fine with a 2nd
person direct object, and (\ref{2ndquirk}) shows the same with a
quirky dative subject. Datives never trigger agreement in the
language, so here the \allagr{} is the only agreement with the
addressee. We see something similar in (\ref{2ndnoagr}), where the
main predicate of the clause is in a resultative participial form
which doesn't host argument agreement. \allagr{} only fails in cases
like (\ref{2ndpst}), where there is a (finite, non-negative,
non-modal, non-participial) verb form capable of bearing argument
agreement, with a 2nd person subject in the nominative case, which
thus triggers that agreement. At least descriptively then, it seems
that double expression of agreement with the addressee -- both
argument agreement and \allagr{} -- is ruled out.% \todo{S: suggests
  % that -ngae is allways AllAgr for Jegan but may be either AllAgr or
  % vocative for her.}

%\subsubsection{Utterances, sentences, obligatoriness}

Now let us consider the further conditions on the appearance of
\allagr, once we've restricted our attention to utterance contexts
with the right kind of addressee and no 2nd person argument agreement.
We've already seen that \allagr{} \emph{can} appear in root
declaratives and various fragmentary utterances. Furthermore, unlike
in at least some dialects of Basque, it can appear in root
interrogatives. (\ref{2ndquirk}) and (\ref{2ndnoagr}) above show it in
polar questions, and (\ref{whquestion}) demonstrates its use in a
\textit{wh}-question:


\ea\label{whquestion}\gll evɭav\U{} aag-um-ŋgæ?\\
{how.much} become-\textsc{fut}-\textsc{alloc}\\
\glt `How much will it come to?' (i.e.\ `How much does it cost?')
\z
%
One crucial point in all of this is that, when its conditions are met,
\allagr{} is obligatory, at least for my Kongu Tamil informant.  I.e.\
when one would use \textit{niiŋgæ} with the addressee, only something
like (\ref{thanksnga}) is possible. Leaving off the \textit{-ŋgæ}
signals non-politeness, and thus (\ref{thanks}) is ill-formed in such
a context.

\ea\label{oblig}
 \ea[]{
   \label{thanksnga}\gll rombaa thanks-ŋgæ\\
   very thanks-\textsc{alloc}{}\\
   \glt `Thanks a lot'}
 \ex[*]{
   \label{thanks}\gll rombaa thanks (to a polite addressee)\\
   very thanks\\}
 \z
\z
%
This is strong evidence that this use of \textit{-ŋgæ} is fully
grammaticalized agreement.

%\subsubsection{Embedding}

Next, we must consider embedded environments, where \allagr{} has been
reported to be blocked or at least heavily restricted in other
languages. Interestingly enough, Tamil seems to be more permissive
here, though there are some complications in the judgments. As a
starting point, examples like (\ref{comp1}) are grammatical. Note that
the \textit{-ŋgæ} suffix is showing up outside of the embedded
argument agreement \textit{-aa-}, but inside of the complementizer
\textit{-nn\U}, thus clearly inside a clause that is the complement of
a verb meaning `say'.  

\ea\label{comp1}\gll Maya [avæ pooʈʈ-læ ʤejkkæ-poo-r-aa-ŋgæ-nn\U] so-nn-aa\\  
Maya [she contest-\textsc{loc}{} win-go-\textsc{prs}-3\textsc{sg.f}-\textsc{alloc}-\textsc{comp}]
say-\textsc{pst}-3\textsc{sg.f}\\ 
\glt `Maya said that she would win the contest.'
\z
%
Exactly which conditions must be satisfied to allow such embedded
\allagr{} is still under investigation, as the empirical situation and
the relevant intuitions quickly get rather tricky. One recurring issue
is that, when the \allagr{} suffix immediately follows the argument
agreement, as in (\ref{comp1}), there are some processing difficulties
that arise, in particular the tendency to interpret the two suffixes
as a single plural argument agreement suffix.\footnote{Recall from
  \tabref{tab:agrforms} that \textit{-ŋgæ} is the second component
  of several plural agreement forms, where the first component marks
  gender and/or person.} It thus takes some care to ensure that
judgments of ungrammaticality do not reflect an unintended parse.

A more interesting complication with embedding, especially unter
attitude predicates, is that there is often some flexibility or
ambiguity as to whether the embedded clause is interpreted for various
purposes relative to the utterance speech act or relative to the
speech act or attitude expressed by the matrix attitude
predicate. This is relevant of course for \allagr, because it
expresses information about the addressee and potentially also the
relationship between that addressee and the author (i.e.\ whether the
latter would use the familiar or formal form of adress with the
former). Consider example (\ref{comp1}) in this light, assuming that
the entire sentence has been uttered by Tom to Venkat, and that it is
reporting on Maya saying the equivalent of `I'm going to win' to
Kausalya. We can reasonably ask now whether the embedded \allagr{}
reflects Tom showing respect to Venkat (the utterance speech act) or
Maya showing respect to Kausalya (the embedded speech act). In this
case, my informant reports that it can only reflect respect being
shown by Tom in the utterance speech act toward Venkat. There are
other cases where inuitions are more uncertain, and what we observe is
clearly affected by things like the type of the matrix attitude
predicate and the plausibility of the various scenarios in a given
context. These concerns make it especially difficult to determine the
constraints on when \allagr{} is possible in embedding contexts, which
readings are available, and what a speaker's rejection of a particular
example should be attributed to. For now I will thus simply report
that \allagr{} is possible in some complement clauses, but probably
not in all, and leave a more complete investigation of the facts for
future work.\largerpage

There is, however, one point on which I will already say more, because
the judgments here are relatively clear, and the pattern is extremely
interesting and highly relevant for the theoretical treatment of
\allagr. As background, consider the pattern of ``monstrous agreement''
investigated in detail by \citet{sundaresan:diss}:

\ea
  \label{monstrous}\gll Maya$_i$ [taan$_{i,*j}$ pooʈʈi-læ{} ʤejkkæ-poo-r-een-nn\U{} ] so-nn-aa\\
  Maya \textsc{anaph} contest-\textsc{loc}{}  win-go-\textsc{prs}-\textsc{1sg}-\textsc{comp}{}  {} say-\textsc{pst}-\textsc{3sg.f}.\textsc{sbj}\\
  \glt `Maya$_i$ said that she$_i$ would win the contest.'
\z 
%
Sentences like (\ref{monstrous}) have a matrix speech verb embedding a
clause where the subject, expressed as an anaphor, is co-referent with
the matrix subject.  What is interesting is that the argument
agreement on the embedded verb in cases like this can be
\textsc{1sg}. But this indicates not the actual speaker of the
utterance, i.e.\ not Tom in the example we discussed above, but Maya,
the author of the speech act described by the matrix speech
predicate. This should make it clear that the representation of speech
act participants will be highly relevant for the derivation and
interpretation of such sentences, and indeed, \citet{sundaresan:diss}
analyzes monstrous agreement in terms of indexical shift, with the
syntactic representation of the embedded speech act playing a crucial
role.

Now, if both monstrous agreement and \allagr{} imply the involvement
of information about speech act participants in the morphosyntax, we
might expect interesting things to happen if we can manage to get them
to co-occur. Fortunately, we can, and the results do not
disappoint. Consider (\ref{monstrousallagr}):

\ea\label{monstrousallagr}\gll Maya$_i$ [taan$_{i,*j}$ pooʈʈ-læ ʤejkkæ-poo-r-een-ŋgæ-nn\U] so-nn-aa\\  
Maya [\textsc{anaph} contest-\textsc{loc}{} win-go-\textsc{prs}-1\textsc{sg}-\textsc{alloc}-\textsc{comp}] say-\textsc{pst}-3\textsc{sg.f}\\ 
\glt `Maya$_i$ said that she$_i$ would win the contest.'
\z
%
The combination of the two interesting types of agreement does indeed
seem to be possible, as we have the monstrous \textsc{1s} suffix
\textit{-een} immediately followed by the allocutive
\textit{-ŋgæ}. The reading is similar to what we saw in
(\ref{comp1}), but with two important differences. First, whereas in
(\ref{comp1}) the embedded subject \textit{avæ} could be either
coreferent with the matrix subject \textit{Maya} or not, here the
coreference is obligatory, as \textit{taan} is a long-distance
anaphor, and \textit{Maya} is the only appropriate binder
around.\footnote{Actually, monstrous agreement structures place very
  specific restrictions, such that the long-distance anaphor in
  subject position must be bound by the attitude holder of the
  immediately embedding predicate. Thus, while in other contexts
  \textit{taan} has some more flexibility in the choice of its
  antecedent, in (\ref{monstrousallagr}) it would be quite fixed to
  \textit{Maya} even if some additional potential perspective holder
  were present elsewhere in the sentence (see \citealt{sundaresan:diss} for details).} 
Second, in this case the politeness
implied by the \allagr{} is reported as having been shown by Maya to
the addressee of the embedded speech act (in our scenario above this
would be Kausalya). In other words, when we have \allagr{} in an
embedded clause that also contains monstrous agreement, it seems to
have to make reference to the embedded speech act. In the absence of
monstrous agreement, reference is apparently made instead to the
utterance speech act, even when the suffix shows up in the embedded
clause as we saw in (\ref{comp1}). These facts will need to be
investigated in more exhaustive detail in future work, but the
preliminary picture they paint is highly suggestive about the kind of
analysis that we should pursue for \allagr, as we will see in Section
\ref{sec:towards-an-account}.

Moving away from complement clauses, Tamil also allows \allagr{} in
some other types of embedding. For example, it can be found in certain
adverbial clauses, like the temporal adjunct built on a completive
participle in (\ref{adverbial}):

\ea\label{adverbial}
\gll [ naan viʈʈ-ukk\U{} poo-ji-ʈʈ\U-ŋgæ~], call paɳɖ-r-een-ŋgæ\\ 
{} I house-\textsc{dat}{} go-\textsc{ptcp}-\textsc{compl}-\textsc{alloc} call do-\textsc{prs}-\textsc{1sg}.\textsc{sbj}-\textsc{alloc}\\ 
\glt `When I get home, I'll call.'
\z
%
Perhaps relatedly, it is perfectly fine on a very common kind of
hanging topic construction, built by following the topic itself with a
participial form of the verb meaning `come', as we seen in
(\ref{hanging}):

\ea\label{hanging}
\gll [ naan va-nd\U-ŋgæ~], naaɭækki Coimbatore-ukk\U{} poo-v-een-ŋgæ\\
{} I come-\textsc{ptcp}-\textsc{alloc}, tomorrow Coimbatore-\textsc{dat}{}
go-\textsc{fut}-\textsc{1sg}.\textsc{sbj}-\textsc{alloc}\\ 
\glt `As for me, I'm going to Coimbatore tomorrow.'
\z
%
Notice incidentally that \textsc{alloc}{} marking actually appears twice in
(\ref{adverbial}) and (\ref{hanging}) -- once on the embedded part
and once on the root clause. This doubling seems to be optional. 


\subsection{Affix ordering and doubling}\label{sec:orderdoub}
Recall that in Basque, \allagr{} is ruled out in questions in addition
to embedded clauses. It has been proposed that this is because
\allagr{} realizes C and is thus in competion with question particles
and with the complementizers found in embedding.  Japanese shows that
this can't be a general property of \allagr, since it does allow the
marking in questions, as we see in (\ref{japQ}) from
\citet{miyagawa:2017}:

\ea\label{japQ}\gll Dare-ga ki-mas-u ka?\\
who-\textsc{nom}{} come-\textsc{alloc}-\textsc{prs}{} \textsc{q}\\
\glt `Who will come?'
\z
%
Note then that the \allagr{} marker \textit{-mas} appears below not
just the question particle \textit{ka}, but also the tense suffix
\textit{-u}. This leads \citet{miyagawa:2017} to argue that, while
\allagr{} involves the C domain, its morphological realization in
Japanese is lower in the structure, near T.

Consider now what happens in similar cases in Tamil.  We have seen
above that Tamil is also perfectly happy to have its \allagr{} marker
\textit{-ŋgæ} appear on a \textit{wh}- or polar-interrogative.
Indeed, it is actually quite common on short fragment and
tag-question-like utterances, which are marked by the polar question
particle \textit{-aa}.  What is potentially odd is how \textit{-ŋgæ}
is ordered relative to this particle.  Consider a minimal pair
building on examples from above:\largerpage[2]

\ea
 % \ea[]{
 %   \label{2ndquirk2}\gll ongaɭ-\U kk\U{} coffee veeŋ um-aa-ŋgæ?\\
 %   you.\textsc{pl}-\textsc{obl}-\textsc{dat}{} coffee want-\textsc{q}-\textsc{alloc}\\
 %   \glt `Do you want coffee?'
 % }
 \ea{
   \label{2ndnoagr2}\gll niiŋgæ{} saap-ʈ-aa\v{c}\v{c}-aa-ŋgæ?\\
   you.\textsc{pl}{} eat-\textsc{asp}-\textsc{res}-\textsc{q}-\textsc{alloc}\\
   \glt `Have you eaten?'
 }
 \ex{
   \label{direct2}\gll niiŋgæ{} saap-ʈ-aa\v{c}\v{c}\U-ŋgæɭ-aa?\\
   you.\textsc{pl}{} eat-\textsc{asp}-\textsc{res}-\textsc{alloc}-\textsc{q}\\ 
   \glt `Have you eaten?'
 }
 \z
\z
%
(\ref{2ndnoagr2}) and (\ref{direct2}) differ only in the order of the
\allagr{} marker and the question particle.  In (\ref{2ndnoagr2}), the
\allagr{} suffix comes at the end, outside of the question particle,
while in (\ref{direct2}) it comes before it.\footnote{The other minor
  differences we see are the result of regular morphophonology. In
  (\ref{2ndnoagr2}), the final \textit{\U} of the resultative suffix
  is deleted before a vowel-initial suffix, and the final \textit{ɭ}
  of the allocuative suffix is deleted in coda position.} In other
words, both orders of the two suffixes are possible. More examples
show that this ordering alternation (again accompanied by predictable
morphophonological effects) is fairly general:\footnote{The first
  version of (\ref{child}) is from \citet{amrit:1991}. She did not
  discuss the other order, with \textsc{alloc}{} before \textsc{q}, in her
  paper.}


\ea
 \ea\label{illiya}\gll illij-aa-ŋgæ? $\sim$ illi-ŋgæɭ-aa?\\
 no-\textsc{q}-\textsc{alloc}{} {} no-\textsc{alloc}-\textsc{q}\\
 \glt various uses, e.g.\ `Isn't it?', `No?', tag question
 \ex\label{apdiya}\gll appaɖij-aa-ŋgæ? $\sim$ appaɖi-ŋgæɭ-aa?\\
 like.that-\textsc{q}-\textsc{alloc}{} {} like.that-\textsc{alloc}-\textsc{q}\\
 \glt `Oh really?', `Is that so?' 
 \ex\label{child}\gll koɻandæ{} ippaɖi sejji-laam-aa-ŋgæ? $\sim$
 koɻandæ{} ippaɖi sejji-laam-ŋgæɭ-aa?\\ 
 child like.this do-\textsc{sbj}v-\textsc{q}-\textsc{alloc}{} {} child like.this do-\textsc{sbj}v-\textsc{alloc}-\textsc{q}\\ 
 \glt `Is it right for the child to do this?'
 \ex\label{winning}\gll Naan ʤej-\v{c}\v{c}-een-aa-ŋgæ? $\sim$ Naan ʤej-\v{c}\v{c}-een-ŋgæɭ-aa?\\
 I win-\textsc{pst}-1\textsc{sg}.\textsc{sbj}-\textsc{q}-\textsc{alloc}{} {} I win-\textsc{pst}-1\textsc{sg}.\textsc{sbj}-\textsc{alloc}-\textsc{q}\\
 \glt `Did I win?'
 \z
\z
%
Especially with the fragment utterances, the order with the \allagr{}
preceding the question particle is the prefered one, but both are
entirely possible under the right circumstances.\footnote{This seems
  to be a point of dialectal variation. The preference described in
  the main text holds for all of my informants from India, but my
  Singapore informants reported that they would only use the
  \allagr-\textsc{q} order, and that they identify the other order
  with speakers from India.} This variation in the order of the
affixes is surprising, and is not generally found in the inflectional
morphology of the language. That is, the various temporal, aspectual,
voice-related and other suffixes that can appear on verb forms are
rigidly ordered relative to each other, and two given suffixes
generally cannot have their order reversed \citep[for extended
discussion on this point,
see][]{sundaresanmcfadden:tamtrans}.\footnote{As pointed out by an
  anonymous reviewer, the fact that the ordering of tense-aspect-voice
  morphology is rigid while that involving apparent C elements is more
  varied is in line with proposals of \citet{aboh:2015}, according to
  which structural variation is found at phase edges.}

Indeed, it gets even more interesting. In the cases where the
\allagr{} suffix can appear either before or after the \textsc{q}
particle, it is actually possible for it to be doubled, appearing
simultaneously in both positions:\footnote{Note that this is distinct
  from the kind of ``doubling'' in (\ref{2ndagr}) above that was
  impossible for my Kongu Tamil informant. There we had 2nd plural
  argument agreement plus \allagr, whereas here we have two instances
  of \allagr.}

\ea\label{double}
 \ea\label{apdidouble}\gll appaɖi-ŋgæɭ-aa-ŋgæ?\\
 like.that-\textsc{alloc}-\textsc{q}-\textsc{alloc}\\
 \glt `Oh really?'
 \ex\label{saapdouble}\gll niiŋgæ{} saapʈ-aa\v{c}\v{c}\U-ŋgæɭ-aa-ŋgæ?\\
 you.\textsc{pl} eat-\textsc{res}-\textsc{alloc}-\textsc{q}-\textsc{alloc}\\
 \glt `Have you eaten?'
 \ex\label{coffeedouble}\gll ongaɭ-ukk\U{} coffee veɳum-ŋgæɭ-aa-ŋgæ?\\
 you.\textsc{pl}-\textsc{dat}{} coffee want-\textsc{alloc}-\textsc{q}-\textsc{alloc}\\
 \glt `Would you like coffee?'
 \z
\z
%
It should be noted that, at least for my primary Kongu Tamil
informant, such structures are not particularly marked, nor do they
correspond to elevated or exaggerated politeness.  Doubling of this
kind is quite unexpected, and again, I am aware of no other piece of
grammaticalized morphology in the language that behaves this
way.

An obvious question to ask then is whether the different ordering and
doubling possibilities are associated with interpretive
differences. We might expect, e.g., that the two orders would
correspond somehow to distinct scope readings of some kind. The data
here are tricky on this point, and it will ultimately require more
careful empirical work, ideally with multiple informants who speak
relevant dialects. However, to a first approximation, the two orders
seem to differ in how the question is biased. Consider the following
pairs, with the distinct translations offered by my Kongu Tamil
informant:

\ea
 \ea\gll illij-aa-ŋgæ? {vs.} illi-ŋgæɭ-aa?\\
 no-\textsc{q}-\textsc{alloc}{} {} no-\textsc{alloc}-\textsc{q}\\
 \glt `It's not, is it?' ~ `Isn't it?'
 \ex\gll appaɖij-aa-ŋgæ? {vs.} appaɖi-ŋgæɭ-aa?\label{apDiaa}\\
 like.that-\textsc{q}-\textsc{alloc}{} {} like.that-\textsc{alloc}-\textsc{q}\\
 \glt `So, it is the case?' ~~ `Is that the case?'
 \ex\gll Naan ʤej-\v{c}\v{c}-een-aa-ŋgæ? {vs.} Naan ʤej-\v{c}\v{c}-een-ŋgæɭ-aa?\\
 I win-\textsc{pst}-1\textsc{sg}.\textsc{sbj}-\textsc{q}-\textsc{alloc}{} {} I win-\textsc{pst}-1\textsc{sg}.\textsc{sbj}-\textsc{alloc}-\textsc{q}\\
 \glt `I won, didn't I?' ~~~~~~~~~~~~~~~~~~~~~~~~~~~ `Did I win?'
 \z
\z
%
The \textsc{q}-\textsc{alloc}{} order seems to be biased towards confirmation,
whereas the \textsc{alloc}{}-\textsc{q} order seems unbiased, a genuine
request for information. So in the first variant in (\ref{apDiaa})
with \textsc{q}-\textsc{alloc}{} order, the speaker is expecting that the
answer will be `yes', and is just asking for confirmation -- e.g.\ in
order to get the addressee to admit something or just to be absolutely
sure of something. But in the second variant with \textsc{alloc}{}-\textsc{q}
order, the speaker genuinely doesn't know what the answer will be, and
is asking in order to find out. 

We can ask then what happens to the interpretation in cases of
doubling. Here the judgments are subtle, and not all of the examples I
checked seem to behave the same, but in one case where my informant
did have a clear intuition, the doubling pattern goes together with
the \textsc{q}-\textsc{alloc}{} order in being biased towards
confirmation. Consider the following minimal triplet of examples
repeated from above, now with information added about the bias on the
question: 

\ea
 \ea{
   \label{youveeaten}\gll niiŋgæ{} saap-ʈ-aa\v{c}\v{c}-aa-ŋgæ?\\
   you.\textsc{pl}{} eat-\textsc{asp}-\textsc{res}-\textsc{q}-\textsc{alloc}\\
   \glt `Have you eaten?' (speaker expects that addresse has)
 }
 \ex{
   \label{haveyoue}\gll niiŋgæ{} saap-ʈ-aa\v{c}\v{c}\U-ŋgæɭ-aa?\\
   you.\textsc{pl}{} eat-\textsc{asp}-\textsc{res}-\textsc{alloc}-\textsc{q}\\ 
   \glt `Have you eaten?' (speaker doesn't know)
 }
 \ex\label{youveeatenhaveyou}\gll niiŋgæ{}
 saapʈ-aa\v{c}\v{c}\U-ŋgæɭ-aa-ŋgæ?\\ 
 you.\textsc{pl} eat-\textsc{res}-\textsc{alloc}-\textsc{q}-\textsc{alloc}\\
 \glt `Have you eaten?' (speaker expects that addressee has)
 \z
\z
%
It should be stressed at this point that the description of the
interpretations here, especially in the doubling case, is highly
preliminary. Note for example that it is a bit difficult to lock down
exactly what the bias is. In the pair in (\ref{coffee}) below (again
based on examples from above, but now with additional interpretive
information), it is still the \textsc{alloc}{}-\textsc{q} order that comes
with a bias, and the \textsc{q}-\textsc{alloc}{} order that is neutral, but
the direction of the bias is difficult to pin down. My informant
reports that this might be how you ask someone as part of a routine,
where you can anticipate what the answer will be based on your
familiarity with their coffee drinking habits. But it does not seem to
be restricted to either an expectation that they will say yes or an
expectation that they will say no. The \textsc{q}-\textsc{alloc}{} order in
(\ref{coffeeAQ}) again shows no bias and is a genuine request for
information, while the doubling in (\ref{coffeeAQA}) again patterns
with the \textsc{alloc}{}-\textsc{q} in showing a bias that is difficult to
pin down.

\ea\label{coffee}
  \ea{
    \label{coffeeQA}\gll ongaɭ-\U kk\U{} coffee veeɳum-aa-ŋgæ?\\
    you.\textsc{pl}.\textsc{obl}-\textsc{dat}{} coffee want-\textsc{q}-\textsc{alloc}\\
    \glt `Do you want coffee?' (some bias, direction not clear)
  }
  \ex{
    \label{coffeeAQ}\gll ongaɭ-\U kk\U{} coffee veeɳum-ŋgæɭ-aa?\\
    you.\textsc{pl}-\textsc{obl}-\textsc{dat}{} coffee want-\textsc{alloc}-\textsc{q}\\
    \glt `Do you want coffee?' (no bias)
  }
  \ex{
    \label{coffeeAQA}\gll ongaɭ-\U kk\U{} coffee veeɳum-ŋgæɭ-aa-ŋgæ?\\
    you.\textsc{pl}-\textsc{obl}-\textsc{dat}{} coffee want-\textsc{alloc}-\textsc{q}-\textsc{alloc}\\
    \glt `Do you want coffee?' (some bias, direction not clear)
  }
  \z
\z
%
Wide-reaching generalizations about the affects of \textsc{alloc}-affix
ordering beyond the specific examples discussed are thus not yet
supported.

\section{Towards an account}
\label{sec:towards-an-account}

\subsection{Theoretical preliminaries}
\label{sec:theoprelim}

The phenomenon of \allagr{} is clearly of great theoretical
interest. As already noted in \sectref{sec:backallagr}, some basic
information about each utterance -- including the identity of the
author and the addressee, as well as the time, location and other
similar parameters -- is obviously relevant for semantic and
pragmatic interpretation, and so it is uncontroversial that such
information must be encoded in some way in the representation of the
discourse context. The question is whether such information is already
represented in some form in the syntax. For indexical pronouns, for
example, it is clear that we must assume a morphosyntactic status for
person features, as they play a role in various agreement and binding
operations, but it is entirely plausible that their relationship with
the actual author and addressee of a given utterance is only
established in the semantics. That is, a feature like [2] or
[$+$participant, $-$author] would be treated no differently in the
morphosyntax than a feature like [\textsc{pl}], and the association with the
utterance context -- and in particular the identity of the discourse
participants -- would only play a role in the interpretive component
when reference is determined.

However, it is more difficult to imagine how an approach like this
would work in the case of \allagr, for two reasons. First, \allagr{}
encodes information about addressees beyond just their role in the
discourse context, including their gender, number or status relative
to the speaker, depending on the specific language. This seems to
imply sensitivity to the actual identity of the addressee of a given
utterance, as opposed to the simple fact that there \emph{is} an
addressee for every typical utterance. Second, while indexical
pronouns appear as syntactic arguments or adjuncts, and clearly
contribute to the asserted meaning of the sentences where they are
found, \allagr{} is again orthogonal to whether the addressee plays a
role as an argument or adjunct, and clearly does not contribute
anything to the assertion. That is, a Basque sentence like
(\ref{fbasque2}), repeated from above, does \emph{not} assert
something like `The speaker is going to the house, and the addressee
is a female individual familiar to the speaker'.

\ea\label{fbasque2}\gll etʃe-a banu-n\\
  house-\textsc{all}{} 1.\textsc{sg}.go-\textsc{alloc}.\textsc{f}\\
  \glt `I am going to the house.' (familiar female addressee)
\z
%
Rather, it asserts something more like `The speaker is going to the
house', and comes with something like a presupposition, such that it
can only be uttered felicitously when the addressee is a female
individual familiar with the speaker. Thus we have morphosyntactic
agreement, which by the nature of what ``agreement'' means must be
\emph{with} something, but that something is not an argument or
adjunct that we would normally expect to be part of the syntactic
representation.

This has led a number of recent authors to conclude that \allagr{}
provides evidence for a literal syntactic representation of the
discourse context, including information about the speech-act
participants \citep{haegemanhill:2011, miyagawa:2012, miyagawa:2017, zu:2015,
  haegemanmiyagawa:2016}. This work generally builds on
and adapts what is sometimes called the neo-performative hypothesis of
\citet{speastenny:2003}, which is in turn a reinterpretation of
proposals by \citet{ross:1970}. The basic idea is that the speech-act
participants are represented not just in the semantics and pragmatics,
but also in the syntax, by quite normal syntactic material that
happens (generally) not to be pronounced. Setting aside a number of
important details, I will follow \citet{hill:2007} and
\citet{miyagawa:2017} in assuming that the left periphery of the
relevant clause types includes a (potentially internally complex)
Speech Act Phrase (SAP).\footnote{For \citet{speastenny:2003}, who
  adopt assumptions about phrase structure from \citet{larson:1988,
    halekeyser:1993}; etc., this SAP has a shell structure including
  two head positions, two specifier positions and a
  complement. \citet{hill:2007, haegemanhill:2011}; and
  \citet{miyagawa:2012, miyagawa:2017} update this by splitting it up
  into two phrases, SAP for the speaker and saP for the addressee.}
The \textsc{speaker} and \textsc{hearer} (or \textsc{author} and
\textsc{addressee}) of the speech act are then directly represented by
(silent, essentially pronominal) elements introduced in specifiers of
this functional structure, as in (\ref{miyatree}):

\ea\label{miyatree}\Tree [.SAP \textsc{speaker} [.SA' SA [.saP \textsc{hearer} [.sa'
sa \qroof{\ldots}.CP ] ] ] ]
\z

\noindent Following \citet{miyagawa:2017} in particular, \allagr{} then
represents straightforward agreement with the \textsc{hearer} in
Spec-saP. Since the \textsc{hearer} is actually represented
syntactically, its various $\phi$-features will be available and thus
can be reflected in the form of the allocutive suffixes on the
verb.\footnote{An anonymous reviewer is concerned that treating Tamil
  \allagr{} as agreement with the representation of the addressee
  neglects the fact that it reflects politeness, which characterizes
  not just the addressee, but also the speaker's relationship with the
  addressee. While this is acurate as a description of the pragmatics
  of the situation, the \emph{grammatical} situation is correctly
  treated by the description in the main text. In Tamil (and many
  other languages), politeness is treated grammatically as a property
  inherent to referents and behaves essentially like another
  $\phi$-feature, which cross-classifies with 2nd and 3rd person as
  well as gender and number. The undeniable relation to the speaker
  seems to come in pragmatically, in the sense that an individual will
  be associated with a polite feature from the perspective of a
  particular speaker.}  The usual assumption is that the probe for
\allagr{} is located somewhere in the C domain. This helps to explain
why the agreement targets the speech act domain, whereas classic
subject agreement in T targets something lower down in the argument
domain. It also provides an approach to restrictions on \allagr{} in
things like interrogatives in certain languages, under the idea that a
question particle competes with the \allagr{} probe to realize
C. Finally, the impossibility of \allagr{} in (most) embedding
contexts can be attributed to the fact that SAP and saP are only
projected in root clauses. Miyagawa thus argues in detail \citep[see
especially][]{miyagawa:2012}, that \allagr{} is a ``root phenomenon'' in
the sense of \citet{emonds:1970}.

In fact, these attempts to analyze \allagr{} fit into a broader trend
of arguing for an expanded left periphery containing a syntactic
representation of the speech act and its
participants. \citet{sundaresan:diss} argues that the (limited)
possibility of projecting a SpeechActP in the complement of certain
attitude predicates (primarily speech predicates) is responsible for
the phenomenon of indexical shift. \citet{haegemanhill:2011} make
crucial use of SAP in their analysis of a series of verbal particles
in Romanian and West Flemish, which serve to ``signal the speaker's
attitude or his/her commitment towards the content of the utterance
and/or of his relation towards the interlocutor'' (p.~9). 
\citet{zu:2015} uses the SAP to analyze speaker-related allocutive
agreement in Jingpo and conjunct marking in Newari, which relates the
subject of a clause either to a preceding subject or to the speech act
participants.  Sundaresan's work here is especially relevant because,
as discussed above, it investigates the monstrous agreement pattern in
Tamil.

Her analysis makes crucial use of an SAP in the embedded clause,
containing a representation of the speech act associated with matrix
`say', which then plays an important role in the determination of
argument agreement in the embedded clause. The anaphoric subject
\textit{taan} is unable to trigger agreement, thus agreement is
instead with a coreferent 1st person form, the interpretation of which
is ``shifted'' by the embedded SAP to be relative to the author of the
matrix speech predicate rather than the author of the utterance
context. This thus provides evidence, independent of \allagr, for the
syntactic representation of information about speech-act participants
in Tamil.

\subsection{The proposal}
\label{sec:account}

I will assume to begin with that the work mentioned in the previous
section is on the right track. In particular, there is a syntactic
representation of the speech act in the left periphery of the clause
which includes information about the speech act participants, and
\allagr{} is a case of the addressee playing an active role in the
morphosyntax.\footnote{For simplicity, from here on out I will speak
  in terms of a single SAP rather than distinguishing the
  speaker-introducing SAP from the hearer-introducing saP.} But this
still leaves a number of interesting issues open. First, where exactly
is the SAP located, and how does it interact with other elements in
the left periphery, in particular the material relevant for forming
interrogatives? Second, how does the overt \allagr{} morphology
actually relate to that syntactic representation?  Third, what is
behind the facts we observed in \sectref{sec:orderdoub} involving
variation in the ordering and even doubling of the allocutive suffix?

The simplest analysis would be that \textit{-ŋgæ} directly realizes
the \textsc{hearer} in the SAP.  In other words, the Tamil phenomenon
wouldn't really be allocutive \textit{agreement} per se, but rather a
direct spell out of (at least one of the $\phi$-features of) the
otherwise silent \textsc{hearer} in Spec-saP in Miyagawa's tree in
(\ref{miyatree}). This has some initial plausibility given the facts
about where the \textit{-ŋgæ} suffix occurs, at the end of the clause
after all of the other inflectional suffixes on the clause-final verb.
Given that Tamil is a strictly head-final language, this is where we
expect something near the top of the functional sequence, like the
SAP, to show up.

There are some issues with this idea, however.  First, the fact that
it shows up as a suffix on the clause-final verbal material suggests
that it realizes a head in the functional sequence, not a specifier,
i.e.\ not the actual representation of the addressee. Second, it runs
into trouble with the variable ordering of the \textit{-ŋgæ} suffix
and the polar question particle \textit{-aa}, a subset of which are
repeated in (\ref{order}).

\ea\label{order}
 \ea\label{illiya2}\gll illij-aa-ŋgæ? $\sim$ illi-ŋgæɭ-aa?\\
 no-\textsc{q}-\textsc{alloc} {} no-\textsc{alloc}-\textsc{q}\\
 \glt various uses, e.g.\ `Isn't it?', `No?', tag question
 \ex\label{apdiya2}\gll appaɖij-aa-ŋgæ? $\sim$ appaɖi-ŋgæɭ-aa?\\
 like.that-\textsc{q}-\textsc{alloc} {} like.that-\textsc{alloc}-\textsc{q}\\
 \glt `Oh really?', `Is that so?' 
  \z
\z
%
The question is which of these two orders we should actually expect
if \textit{-ŋgæ} is realizing something in SAP.  Both the question
particle and the SAP should be somewhere in the C domain, i.e.\ in the
left periphery of the clause higher than T, and it is that idea that
has been employed to explain why \allagr{} is incompatible with
questions in Basque.  But since at least \citet{rizzi:1997} we are
generally willing to recognize a richer structure in this region of
the clause, involving a series of (more or less strictly ordered)
heads. We can assume that there is a head responsible for indicating
whether a clause is interrogative, declarative etc.\ -- let's adopt
Rizzi's Force, though this may be an oversimplification -- in
addition to the SAP heads. The trees in (\ref{forcesaptrees}) give the
two obvious logical options for how these two heads could be ordered
with respect to each other above the rest of the clause, and we must
decide which is more likely to be correct.\pagebreak% \footnote{There is of course

\ea\label{forcesaptrees}
\begin{multicols}{2}
\ea \begin{forest}
[ForceP [Force] [SAP [SA] [...]]]
\end{forest}
\ex\begin{forest}
[SAP [SA] [ForceP [Force] [...]]]
\end{forest}
% \ex\Tree [.{a. ~~~~~~~~~~} Force [.SAP SA
% \ldots{} ] ] ~~~~~\Tree [.{b. ~~~SAP~~~~~~~} SA [.ForceP Force \ldots{} ] ]
\z\end{multicols}
\z
%
It seems to me that, semantically speaking, only
(\ref{forcesaptrees}b) is plausible. If SAP really introduces the
representation of the speech act participants and related information,
then it is setting the stage for the entire sentence. It provides the
background against which a question is asked, including who is asking
and answering, and the order in (\ref{forcesaptrees}b) seems to best
reflect this. The order in (\ref{forcesaptrees}a), on the other hand,
would seem to imply that the contents of the SAP are part of what the
question in Force is being asked about. 
If SAP comes above Force, and both are realized by overt affixes, then
by the Mirror Principle, SAP should come after Force in linear
order. So if we assume that allocutive \textit{-ŋgæ} realizes
something in SAP, and the question particle \textit{-aa} realizes
Force, then we should get the order in (\ref{2ndnoagr3}):

\ea
 \ea\label{2ndnoagr3}\gll niiŋgæ{} saapʈ-aa\v{c}\v{c}-aa-ŋgæ?\\
 you.\textsc{pl}{} eat-\textsc{res}-\textsc{q}-\textsc{alloc}\\
 \glt `Have you eaten?'
 \ex\label{direct3}\gll niiŋgæ{} saapʈ-aa\v{c}\v{c}\U-ŋgæɭ-aa?\\
 you.\textsc{pl}{} eat-\textsc{res}-\textsc{alloc}-\textsc{q}\\ 
 \glt `Have you eaten?' 
 \z
\z
%
Indeed we do get this order, but of course the whole point is that we
\emph{also} get the reverse order shown in (\ref{direct3}). Now, we
could conclude that this is simply a morphological quirk. We could say
that the syntax really corresponds to something like (\ref{2ndnoagr3})
with the expected ordering of \allagr{} outside of the question
particle, but that there is then a post-syntactic process that
optionally flips their order.  Again however, this has some
problems. For one thing, it doesn't have a good way of dealing with
doubling, i.e.\ the fact that the \allagr{} marker can simultaneously
show up in both positions relative to the question particle. For
another, if the two orderings are identical in the syntax, being
differentiated only in the morphological portion on the PF branch the
derivation, it predicts that there should be no meaning difference
between them, under the standard assumption that PF doesn't feed into
LF. But this is incorrect -- as we saw in \sectref{sec:orderdoub}, 
the \textsc{q}-\textsc{alloc}{} order seems to be biased
towards a particular answer, while the \textsc{alloc}{}-\textsc{q} seems to be
a more neutral request for information. I hesitate to draw any firm
analytical conclusions from this, again because the empirical
situation is unsettled, but what it suggests at least is that the
ordering difference involves something more substantial than just
post-syntactic morphology.\footnote{As an anonymous reviewer notes,
  there are a number of ways one could attempt to derive the two
  surface orders from a single underlying structure by purely
  syntactic means, e.g.\ if SAP and Force were just distinct (bundles
  of) features on a single head, with some version of equidistance
  allowing them (or elements related to them) to be realized in either
  order, or if the entire ForceP could optionally move into
  Spec-SAP. However, such approaches suffer from the same problems in
  dealing with the interpretation and especially the doubling facts
  and thus can be set aside, at least in the absence of some theory
  that could tie them to bias in question interpretations.}

Therefore, I'd like to propose something (slightly) more
interesting. First, for the \textsc{alloc}{}-\textsc{q} order we need to
reaffirm the idea that \textit{-ŋgæ} really is agreement. That is, it
does not realize anything in SAP directly, but rather the features of
something in SAP being reflected elsewhere in the structure. This lets
us put the question particle in Force, below SAP, getting the broad
semantics right, with \allagr{} realizing an even lower head that
agrees with the addressee in SAP. This has to be distinct from the
head that realizes argument agreement, because we get both types of
agreement simultaneously, even when both are below the question
particle, as in the second variant in (\ref{winning}), repeated in
(\ref{winning3}).

\ea\label{winning3}\gll Naan ʤej-\v{c}\v{c}-een-ŋgæɭ-aa?\\
 I win-\textsc{pst}-1\textsc{sg}.\textsc{sbj}-\textsc{alloc}-\textsc{q}\\
 \glt `Did I win?'
\z
%
Given the ordering facts, though, the locus of low \allagr{} should be
very close to, and just a bit higher than, that of argument agreement,
itself just above T.\largerpage[-1]

For the order where \textit{-ŋgæ} shows up after the question
particle, something different must be going on. Again by the Mirror
Principle, it must be realizing a higher position in the left
periphery, which as far as I have found only comes below the (rather
high) complementizer \textit{-nn\U}. We can thus potentially place it
even as high as SAP itself. This opens up the possibility that
\emph{this} instance of \textit{-ŋgæ} isn't agreement in the T
region, but is more directly spelling out something relating to the
representation of the addressee. Perhaps the simplest assumption is
that it is the sa head itself, which agrees in $\phi$-features with
the representation of the addressee in its specifier, but it could
potentially even be a clitic spelling out that addressee directly. The
presently available data do not put me in a position to defend any
specific proposal. The following structure gives an idea of how this
might look:\footnote{One should not take the label ``AllAgr'' too
  seriously, and of course this leaves open how exactly the higher
  \textit{-ŋgæ} relates to the actual representation of the
  addressee. As an anonymous reviewer points out, the relationship
  between AllAgr and SAP could constitute evidence for the possibility
  of upward Agree \citep[e.g.][]{zeijlstra:2012} though see also \textcitetv{chapters/11-diercks-etal}.}

\ea\Tree [.SAP [.ForceP [.AllAgrP \qroof{\ldots}.TP [.AllAgr
\textit{-ŋgæ} ] ] [.Force \textit{-aa} ] ] [.SA \textit{-ŋgæ} ] ] \z
%

\noindent This approach has the clear advantantage that it provides two distinct
structural positions for the allocutive suffix, and thus will allow a
natural account of the doubling data.\footnote{As pointed out by a
  reviewer, for the various types of fragmentary utterances discussed
  above, where both orders and doubling of \allagr{} are possible, we
  can imagine something like a sluicing analysis. I.e.\ there is a
  full clause structure going up to SAP, with remnant material moved
  into a left-peripheral position, followed by ellipsis of TP.} It
does not explain the subtly different readings available with the two
orders, but it does at least allow an account to be formulated, once
the facts are better understood, because the two positions for
allocutive suffixes have different sources. The idea is basically as
follows. The baseline is that an Agree relationship is established
between the representation of the addressee and the AllAgr head above
T, which itself has no semantic consequences. An additional
relationship can be established with a higher head in the SAP region,
but if this is done, it has the semantic consequence of introducing a
bias with respect to a polar question. The results of this higher
Agree relation are always pronounced when they obtain. The lower one
is usually also pronounced, but can be optionally left off when the
higher one is pronounced. When only the lower is pronounced, we get
the \textsc{alloc}-\textsc{q} order and no bias, because the higher Agree
operation has not occurred. Anytime the higher is pronounced, we get
the bias, because this is derived by the higher Agree operation. But
in this case pronunciation of the lower Agr is optional, so we get
both doubing and the \textsc{q}-\textsc{alloc}{} order, both with the
semantics of bias.

We can also take some first steps towards an analysis of the embedding
facts and the interaction of \allagr{} with monstrous agreement. As
noted above, \citet{sundaresan:diss} analyzes monstrous agreement in
terms that require an SAP in the embedded clause which encodes
information about the embedded speech act rather than the matrix
one. Among other things, the contexts in which monstrous agreement is
possible are constrained by the conditions on when such an embedded
SAP can be projected, and the optionality of monstrous agreement in
certain speech complements amounts to optionality in whether the SAP
will be present. Now, under the proposal being considered here, at
least the lower \allagr{} probe is not itself in the SAP region, thus
we do not necessarily predict that \allagr{} will only be available in
embedded clauses that contain their own SAP -- a lower \allagr{}
above T could at least potentially Agree with some higher
representation of an addressee.\footnote{This of course raises
  important and complicated questions of mechanics. In particular, how
  would such an Agree relation be constrained to probe specifically
  for the $\phi$-features of addressee representations, and what are
  the locality implications? Consideration of such issues must wait
  until we have a better grasp on the empirical situation. Especially relevant here will be the comparison with other types of agreement which involve the C domain, in particular upward and downward complementizer agreement, as discussed by \textcitetv{chapters/11-diercks-etal}.} We do,
however, have clear expectations about how the presence of an embedded
SAP should be relevant for the \emph{interpretation} of \allagr. If
there is an embedded SAP, we expect on minimality grounds that an
embedded \textit{-ŋgæ} suffix must be Agreeing with that, and not
with the matrix SAP associated with the utterance speech act. If
monstrous agreement implies an embedded SAP, then we predict that an
accompanying embedded \allagr{} will reflect properties of the
addressee of the embedded speech act (and her relationship with the
embedded author), not of the addressee of the utterance speech
act. This is precisely what we observed in the contrast between
(\ref{comp1}) and (\ref{monstrousallagr}) above, repeated in
(\ref{monnomon}):\largerpage

\ea\label{monnomon}
 \ea\label{comp2}
 \gll Maya [avæ{} pooʈʈ-læ ʤejkkæ-poo-r-aa-ŋgæ-nn\U] so-nn-aa\\  
 Maya [she contest-\textsc{loc}{} win-go-\textsc{prs}-3\textsc{sg.f}-\textsc{alloc}-\textsc{comp}] say-\textsc{pst}-3\textsc{sg.f}\\ 
 \glt `Maya said that she would win the contest.' (speaker being polite)
 \ex\label{monstrousallagr2}\gll Maya$_i$ [taan$_{i,*j}$ pooʈʈ-læ ʤejkkæ-poo-r-een-ŋgæ-nn\U] so-nn-aa\\  
 Maya [\textsc{anaph} contest-\textsc{loc}{} win-go-\textsc{prs}-1\textsc{sg}-\textsc{alloc}-\textsc{comp}] say-\textsc{pst}-3\textsc{sg.f}\\
 \glt `Maya$_i$ said that she$_i$ would win the contest.' (Maya being polite)
 \z
\z
%
In (\ref{comp2}) without monstrous agreement, we can presume that
there is no embedded SAP, thus the embedded \allagr{} is interpreted
relative to the utterance speech act, and so it is the utterance
speaker, Tom in our scenario from above, who is showing politeness to
the addressee, Venkat. In (\ref{monstrousallagr2}), on the other hand,
monstrous agreement establishes the presence of an embedded SAP,
relative to which the embedded \allagr{} must be intepreted. Thus it
is the author of the embedded speech act Maya who is showing
politeness towards her addressee, Kausalya.

Of course, this account clearly still leaves a number of open
questions, both large and small, many of them resulting from the
preliminary state of our understanding of the empirical situation
surrounding \allagr{} in Tamil and in general. I have already
discussed the uncertainty with regard to the precise nature of the two
positions where agreement can be realized, the relationship between
the orderings with the question particle and the bias interpretations,
and the mechanics of the apparent long-distance agreement when
\allagr{} appears in non-monstrous embedded clauses. To this we can
add the precise nature of the ban on \allagr{} when there is 2nd
person argument agreement,\footnote{Note that in the analysis proposed
  here, \allagr{} and 2nd person argument agreement involve distinct
  probes Agreeing with distinct goals, so it's difficult to see what
  could prevent them from both applying in the same clause. Comparison
  with other languages as well as the dialectal variation on this
  point might suggest that the ban does not reflect anything deeply
  syntactic but rather something about surface realizations.} the
issue of why Tamil \allagr{} only marks number (and related
politeness), but not gender or person distinctions, as well as a host
of comparative questions, e.g.\ why Tamil permits \allagr{} more
readily in embedding than Basque does. Nonetheless, I hope to have
shown on a general level that the Tamil data add further support to a
framework where information about the discourse participants is
represented and active in the syntax, and on a more specific level
that \allagr{} interacts in interesting ways with question formation
and indexical shift, and that we should recognize two distinct
positions for its realization.

\pagebreak
\section*{Abbreviations}
\begin{multicols}{2}
	\begin{tabbing}
\textsc{alloc}\hspace{5mm} \= allocutive marker\kill
	\textsc{acc} \> accusative\\
		\textsc{all} \> allative\\
  \textsc{alloc} \> allocutive marker\\
  \textsc{anaph} \> anaphor\\
  \textsc{asp} \> aspect\\
  	\textsc{comp} \> complementizer\\
  \textsc{dat} \> dative\\
  \textsc{excl} \> exclusive\\
  	\textsc{f} \> feminine\\
  	\textsc{fut} \> future\\
  	\textsc{loc} \> locative\\
  		\textsc{m} \> masculine\\
  		\textsc{neg} \> negation\\
  			\textsc{nom} \> nominative\\
  			\textsc{obl} \> oblique\\
  			\textsc{pl} \> plural\\
  			\textsc{pred} \> predicative\\
  			\textsc{prs} \> present\\
  \textsc{pst} \> past\\
  \textsc{ptcp} \> participle\\
  \textsc{q} \> question particle/marker\\
  \textsc{res} \> resultative\\
  \textsc{rsp} \> respected addressee\\
  	\textsc{sbj} \> subject\\
  		\textsc{sg} \> singular\\
  			\textsc{top} \> topic\\
	\end{tabbing} 
\end{multicols}

\section*{Acknowledgements}

I am grateful first of all to my informants, Jegan Murugesan, Champa
Sundaresan, Subramania Sundaresan and Sandhya Sundaresan, for
patiently exploring this particular detail of their language with
me. Thanks also to Sandhya and Jegan, as well as the audiences at
TripleA 4 in Göteborg, BCGL 10 in Brussels and the 2017 MaTüBe
workshop in Mannheim, for very helpful discussion of both empirical
and theoretical issues. Finally, I am endebted to Rahul Balusu for not
only directing my attention to, but also sending me a copy of, the
very important (but not so easy to obtain) \citet{amrit:1991}.

{\sloppy\printbibliography[heading=subbibliography,notkeyword=this]}
\end{document}
