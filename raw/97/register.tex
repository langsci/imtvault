\chead{Register.}
\pdfbookmark[0]{Register.}{register}
\chapter*{Register.}\label{Register}
\fed{{\textbar}467{\textbar}} \sed{{\textbar}{\textbar}487{\textbar}{\textbar}}
\begin{register}
\subsection*{A.}\label{reg.A}\pdfbookmark[1]{A.}{reg.A}

\so{A} in der \inlineupdate{indogerman.}{indogerm.} Ursprache \pageref{sp.188}.

\so{Abchasisch}, dessen Laute \pageref{sp.34}. Conjugation \pageref{sp.423}.

\so{Abel}, Carl \sed{\pageref{sp.229}}, \pageref{sp.244}, \pageref{sp.380}.

\so{Abendmahl} \pageref{sp.245}.

\so{Abfragen} fremder Sprachen \pageref{sp.68}–\pageref{sp.69}.

\sed{\so{Abgrenzung} der Sprachgemeinschaften \pageref{sp.56}.}

\so{Ablativus} absolutus \pageref{sp.467}.

\so{Ablaut} \pageref{sp.200}.

\so{Ableitung}. Abgeleitete Wörter und verschiedene grammatische Formen desselben Wortes \pageref{sp.121}–\pageref{sp.122}.

\so{Abschaffen} von Ausdrücken \pageref{sp.45}.

\so{Abschattungen}, dialektische, siehe \so{Dialekt}. – In den Bedeutungen \pageref{sp.100}.

\so{Absicht}. Adverbialsätze der Absicht \pageref{sp.104}.

\so{Absolute Redeform} \pageref{sp.322}.

\so{Abstracta}. Bildliche Benennung derselben \pageref{sp.42}–\pageref{sp.43}.

\so{Abstraction}, unbewusste, bei der Spracherlernung \pageref{sp.63}, \sed{\pageref{sp.210}}.

\so{Abstumpfung} des sprachlichen Gewissens \pageref{sp.276}.

\so{Accent}. Die Lehre von ihm in der Grammatik \pageref{sp.87}. \sed{Wort- und Satzaccent, dessen Einfluss auf die Euphonik \pageref{sp.198}.} Geschliffener \pageref{sp.200}.

\so{Accent} (französisch ausgesprochen) \pageref{sp.34}.

\sed{\so{Accentwandel}, Beispiele dafür \pageref{sp.212}.}

\sed{\so{Accusativ} cum Infinitiv \pageref{sp.467}.}

\so{Accusativus} \pageref{sp.102}.

\sed{\so{Accusativus} interjectionis \pageref{sp.321}.}

\so{Achsel} \pageref{sp.236}.

\so{Activum} \pageref{sp.102}\sed{, \pageref{sp.459}}.

\so{Activus}, casus \pageref{sp.102}.

\so{Adam}, Lucien \pageref{sp.31}. Seine grammatischen Extracte \pageref{sp.52}. Du parler des hommes etc. dans la langue caraïbe \pageref{sp.248}. Hybridologie, Creolensprachen \pageref{sp.279}. Über das grammatische Geschlecht \pageref{sp.481}.

\so{Adaptationstheorie} \pageref{sp.180}.

\so{Adelung}, Joh. Christoph, Mithridates \pageref{sp.28}, \pageref{sp.31}.

\so{Adessivus} \pageref{sp.114}.

\so{Adjectivum} \pageref{sp.101}. Werth der Adjectiva für den Verwandtschaftsnachweis \pageref{sp.152}. A. – Substantivum – Verbum \pageref{sp.382}–\pageref{sp.385}. Stellung des attributiven A. \pageref{sp.401}.

\sed{\so{Adnominalcasus} \pageref{sp.115}.}

\so{Adnominale Attribute} \pageref{sp.101}.

\so{Adnominalsätze} \pageref{sp.104}.

\so{Adverbiale Attribute} \pageref{sp.101}; – und Objecte \pageref{sp.102}. A. A. und psychologisches Subject \pageref{sp.371}–\pageref{sp.372}.

\so{Adverbialsätze} \pageref{sp.104}.

\so{Adverbien} \pageref{sp.101}. Deutende und fragende \pageref{sp.385}.

\so{Aelius Herodianus} \pageref{sp.21}.

\so{Aes}, Neutrum, nachwirkend im grammatischen Geschlechte der übrigen lateinischen Metallnamen \pageref{sp.237}.

\so{Affixe}, vergl. \so{Formativlaute}. Prae-, Sub- und Infixe \pageref{sp.349}. Lautlicher Gehalt, Articulation \pageref{sp.436}.

\sed{\so{Afformative} \pageref{sp.256}.}

\so{Afrika}. Lepsius’ Hypothese über die genealogischen Beziehungen der Sprachen Afrikas. \pageref{sp.282}. Anscheinende Mischsprachen \pageref{sp.406}.

\so{Agau} \pageref{sp.160}, \pageref{sp.282}.

\so{Agglutination}. Etymologisches und morphologisches Bewusstsein \pageref{sp.124}, \pageref{sp.384}. Begriff der A. nach Steinthal \pageref{sp.337}. Wesen der A. \pageref{sp.345}. Unterabtheilungen \pageref{sp.349} flg. Vorwurf der Formlosigkeit \pageref{sp.396}–\pageref{sp.397}. Werth für die Beurtheilung der Sprachen \pageref{sp.403}–\pageref{sp.404}.

\fed{{\textbar}468{\textbar}}

\so{Agglutinationstheorie} l\pageref{sp.80}\sed{, \pageref{sp.255}–\pageref{sp.258}}. Erklärung der Unregelmässigkeiten von ihr aus \pageref{sp.211}. Spirallauf \pageref{sp.255}–\pageref{sp.258}. Vorgeschichtliche Wortstellungsgesetze \pageref{sp.365}. Innere Articulation \pageref{sp.434}–\pageref{sp.436}.

\so{Agglutinirende Sprachen} \pageref{sp.122}\sed{, \pageref{sp.349}}.

\so{Aegypter}, Erfinder der Buchstabenschrift \pageref{sp.19}.

\so{Aegyptisch} \pageref{sp.142}. Als Koptisch noch Kirchen\sed{{\textbar}{\textbar}488{\textbar}{\textbar}}sprache \pageref{sp.146}. Pronomina, Zahlwörter \pageref{sp.161}. Aeg. als Formsprache (nach Steinthal) \pageref{sp.336}. \sed{Femininbildung \pageref{sp.240}. „Nichtwortige“ Sprache \pageref{sp.342}.} Gegensinn der Wörter? \pageref{sp.380}. Höhere Würdigung der Sprache \pageref{sp.389}.

\so{Aham} (sanskrit) – ego \pageref{sp.214}.

\so{Ahn}. Dessen grammatische Methode \pageref{sp.110}.

\so{Aehnlichkeit}, Wie, als grammatische Kategorie \pageref{sp.104}. Ae. als Merkmal des Verwandtschaftsgrades \sed{\pageref{sp.148}–\pageref{sp.149},} \pageref{sp.159}. Ae. der Laute, der Bedeutung als Grund falscher Analogien \pageref{sp.211}–\pageref{sp.212}. Ae. der Vorstellungen den Bedeutungswandel fördernd \pageref{sp.232}–\pageref{sp.234}.

\so{Ahom} \pageref{sp.149}\sed{, \pageref{sp.251}, \pageref{sp.426}}.

\texttt{Ꜣ}\so{Ain} in k verwandelt \pageref{sp.314}.

\so{Aino, Ainu}. Isolierte Sprache \pageref{sp.147}. Im Aussterben \pageref{sp.261}. \sed{Vocalismus \pageref{sp.408}.}

\sed{\so{Aisin} \pageref{sp.131}.}

\so{Aitom}, \pageref{sp.149}, \sed{\pageref{sp.257}, \pageref{sp.426}.}

\so{Akademien}, die Sprachen regelnd \pageref{sp.125}–\pageref{sp.126}. 

\so{Akka} \pageref{sp.177}.

\so{Akkader} \pageref{sp.18}, \pageref{sp.389}.

\so{Akra} \pageref{sp.150}, \pageref{sp.282}.

\so{Akustische Mittel} der Gedankenmittheilung, warum tauglicher als die optischen? \pageref{sp.311}–\pageref{sp.312}.

\sed{\so{Akvas} avis ka \pageref{sp.170}.}

\so{Al}– Arabische Lehnwörter mit al– im Spanischen \pageref{sp.266}.

\so{Alalen} \pageref{sp.303}.

\so{Alas}! (englisch) \pageref{sp.360}.

\so{Albanesen}, \so{Albanesisch}, s. \so{Arnauten}.

\so{Aleutisch} \pageref{sp.147}, \pageref{sp.383}.

\so{Alexandriner}, Philologen \pageref{sp.21}.

\so{Algonkin} \pageref{sp.162}.

\so{Algonkin-Sprachen}. Namen des Pferdes \pageref{sp.41}. Verwandtschaft mit dem Nahuatl? \pageref{sp.147}, \sed{\pageref{sp.152},} \pageref{sp.172}–\pageref{sp.173}. \sed{Polysynthetischer Bau \pageref{sp.423}.} Substantiva für Theile oder Beziehungen \pageref{sp.441}. Congruenz der Tempora und Modi \pageref{sp.466}.

\so{Alif prostheticum}, arabisches \pageref{sp.157}.

\so{Alifuru}, s. Toumpakewa.

\so{Allgemeine Sprachwissenschaft}. Ihre Aufgabe \pageref{sp.11}–\pageref{sp.12}, \pageref{sp.302}–\pageref{sp.303}. Bücher über sie \pageref{sp.48}–\pageref{sp.49}, \pageref{sp.52}, Anm.

\so{Allgemeinheit}. Ausdrucksformen dafür \pageref{sp.95}. A. und Unbestimmtheit \pageref{sp.326}. Fähigkeit, die A. zum Ausdrucke zu bringen \pageref{sp.446}–\pageref{sp.447}.

\so{Allheit}, Ausdrucksformen dafür \pageref{sp.95}.

\so{Alliteration}, lautsymbolisch empfunden \pageref{sp.220} flg.

„\so{Alpdrücken}“, Albdrücken \pageref{sp.127}.

\sed{\so{Alphabetische Anordnung} der Wörterbücher \pageref{sp.123}–\pageref{sp.124}.}

\so{Altbaktrisch}, Schrift und Transscription \pageref{sp.134}. Epenthese \pageref{sp.199}, \pageref{sp.401}.

\sed{\so{Altbulgarisch} \pageref{sp.254}.}

\so{Altersstufen}. Die Sprache verschiedener A. \pageref{sp.258}. Verschiedene Stimmlagen der Menschen je nach den A. \pageref{sp.310}–\pageref{sp.311}.

\sed{\so{Althebräisch} \pageref{sp.111}.}

\sed{\so{Altfranzösisch}, Lautwandel \pageref{sp.192}.}

\sed{\so{Altirisch} \pageref{sp.111}.}

\so{Altitalische Sprachen} \pageref{sp.114}.

\so{Altlibysch} \pageref{sp.160}.

\so{Altnordisch}. Das Passivum auf \mbox{–sk,} \mbox{–st} \pageref{sp.241}. Der Umlaut \pageref{sp.401}.

\so{Altpreussisch} \pageref{sp.114}. Ausgestorbene Sprache \pageref{sp.146}.

\so{Altslavisch}. Lautwesen \pageref{sp.34}.

\so{Alvarez}. Dessen lateinische Grammatik \pageref{sp.106}.

\so{Ameisen}, ob sprachbegabt? \pageref{sp.4}.

\so{Amerika}. Sprachen der Ureinwohner, s. \so{Indianersprachen}.

\sed{\so{Amharisch} \pageref{sp.282}.}

\sed{\so{Amtlicher Geschäftsstil} \pageref{sp.246}.}

„\so{Ana}“\sed{, Nachsilbe,} \pageref{sp.215}.

\so{Anähnlichung} der Laute \pageref{sp.200}.

\so{Analogie}, Wirkung unbewusster Abstraction \pageref{sp.63}–\pageref{sp.64}. In der neueren Indogermanistik \pageref{sp.137}. Lautgesetz und A. \pageref{sp.174}, \pageref{sp.186}–\pageref{sp.187}. A. als sprachgeschichtliche Macht \pageref{sp.210}–\pageref{sp.212}\sed{, \pageref{sp.211}}. \sed{Falsche A. in der Sprachgeschichte \pageref{sp.251}.}

\so{Analyse} des Gedankens in der Sprache \sed{\pageref{sp.3},} \pageref{sp.81}. Grammatische Analyse der Sprache \pageref{sp.92} (vergl. auch \so{analytisches System}). A. der grammatischen Beispiele \pageref{sp.116}.

Analytische Sprachen, neu-europäische \pageref{sp.393}.

\fed{{\textbar}469{\textbar}}

\so{Analytisches System} der Grammatik \pageref{sp.85}, \pageref{sp.88}. Hat dem synthetischen voranzugehen \pageref{sp.86}. Inhalt und Einrichtung \pageref{sp.88}–\pageref{sp.93}. Vergleich mit dem synthetischen \pageref{sp.93}–\pageref{sp.94}.

\so{Anatomie} und Sprachwissenschaft \pageref{sp.15}.

\so{Anbildung} = Flexion, nach Steinthal \pageref{sp.337}, A. und Agglutination \pageref{sp.351}.

\so{Andamanisch}. Classen der Substantiva \pageref{sp.442}.

\so{Andree}, Richard, Ethnographische Parallelen und Vergleiche \pageref{sp.148}.

\so{Andrews}, L., Grammar of the Hawaiian Language \pageref{sp.463}.

\so{Aneinanderfügung} als Verbindung des Stoffes \pageref{sp.324}.

\sed{{\textbar}{\textbar}489{\textbar}{\textbar}}

\so{Aneiteum, Aneityum, s. Annatom-Sprache}.

\so{Anfügende Sprachen, s. Agglutinirende Sprachen}.

\so{Annamitisch}. Schrift \pageref{sp.129}. Wortaccent \pageref{sp.148}. Verwandtschaft mit den kolarischen Sprachen; Bau \pageref{sp.150}. Composita auf iek, iet \pageref{sp.223}. \sed{Doppelungsformen \pageref{sp.224}.} Einsilbigkeit und Isolirung nicht ursprünglich \pageref{sp.255}. Chinesischer Einfluss \pageref{sp.271}. \sed{Den kolarischen Sprachen verwandt \pageref{sp.273}.} Schwierigkeit der Beurtheilung \pageref{sp.426}. 

\so{Annatom-Sprache}. Lautwesen \pageref{sp.34}. Innere Sprachform, Satzbau, conjugirtes Fürwort \pageref{sp.151}. \sed{Lautcongruenz \pageref{sp.214}.}

\so{Anomala, s. Unregelmässige Bildungen.}

\sed{\so{Anregung} zur Sprachwissenschaft \pageref{sp.17}.}

\so{Anschaulichkeit}, ein Vorzug der neuromanischen Sprachen \pageref{sp.183}. \sed{Eine sprachgeschichtliche Macht \pageref{sp.185}.} Streben nach A., zu periphrastischen Formen führend \pageref{sp.239}–\pageref{sp.241}.

\so{Anschauung} der Anschauung, innere Sprachform, Vorstellungen \pageref{sp.335}.

\sed{\so{Ansprache}, Gegensatz zur Deklamation \pageref{sp.318}.}

\sed{\so{Anstandswerte} der Ausdrücke, Einfluss auf den Bedeutungswandel \pageref{sp.235}.}

\so{Ansteckung}, sprachliche, s. \so{Sprachmischung}.

„\so{Anstehen}“ \pageref{sp.232}.

\so{Anstrengung} der Sprachorgane, besondere, die Laute umgestaltend \pageref{sp.183}.

\so{Anthropologie} und Sprachwissenschaft \pageref{sp.13}. Inwieweit Leibes- und Geistesart auf Sprachverwandtschaft schliessen lasse? \pageref{sp.147}–\pageref{sp.148}. Einfluss der amerikanischen Schule auf die Sprachwissenschaft \pageref{sp.159}.

\so{Anthropomorphisirung} in der Sprache \pageref{sp.2}.

\sed{\so{Anti} (Campa). Unsichere Articulation \pageref{sp.194}.}

\so{Anticipation} im Lautwesen, Congruenz u.~s.~w. \pageref{sp.401}–\pageref{sp.403}.

\so{Antithese, s. Gegensatz}.

\so{Anubandhas} \pageref{sp.119}.

\sed{\so{Aorist} \pageref{sp.114}, A. im Germanischen verloren gegangen \pageref{sp.253}.}

\sed{\so{Apellativum} \pageref{sp.307}.}

\so{Aphaeresis} \pageref{sp.201}.

\sed{\so{Aphoristische Redeweise} \pageref{sp.464}–\pageref{sp.465}.}

\so{Apokope} \pageref{sp.201}.

\so{Apollonios Dyskolos} \pageref{sp.21}.

\so{Appendices} der Grammatik \pageref{sp.106}–\pageref{sp.107}.

\so{Apposition} \pageref{sp.101}. Ergänzung der elliptischen Rede durch eine Art A. \pageref{sp.375}.

\so{Arabisch}. Vocalismus und Rhythmus \pageref{sp.92}–\pageref{sp.93}. Die fünfzehn Conjugationen \pageref{sp.116}. Das typische fa\texttt{Ꜣ}ala \pageref{sp.117}. Die a. Schrift bei Türken, Persern und Malaien \pageref{sp.129}. Das Alif prostheticum \pageref{sp.157}. Gutturale, Abneigung gegen Consonantenhäufungen \pageref{sp.197}. \sed{Femininbildung \pageref{sp.240}.} Einfluss auf andere Sprachen \pageref{sp.271}. Aegyptisches Vulgär-A. \pageref{sp.276}. \sed{Genealogie \pageref{sp.282}. Consonantismus \pageref{sp.409}.} Wörter für Farben und Gebrechen \pageref{sp.441}. Objectivconjugation \pageref{sp.460}. Congruenz der Tempora und Modi \pageref{sp.466}.

\so{Arakanisch} \pageref{sp.149}, \pageref{sp.257}.

\so{Araukanisch}. Ausdrücke für „hungern“ \pageref{sp.482}.

\so{Arawakisch} \pageref{sp.248}.

\sed{\so{Arbeit}, gemeinsame Arbeit und Sprachvermögen \pageref{sp.309}.}

\so{Archaismen} \pageref{sp.107}. Berücksichtigung in der einzelsprachlichen Forschung \pageref{sp.125}–\pageref{sp.127}.

\so{Argot} \pageref{sp.45}.

\so{Arische Sprachen} im engeren Sinne, s. \so{Indisch-iranische Sprachen}.

\so{Aristokratie} in der \inlineupdate{Sprache}{Sprachgeschichte} \pageref{sp.249}–\pageref{sp.250}.

\so{Aristoteles}\sed{, sein Verdienst um die Grammatik} \pageref{sp.20}.

\so{Armbrust} \pageref{sp.267}.

\so{Armenisch} \pageref{sp.256}.

\so{Armuth} der Sprachen, scheinbare und wirkliche \pageref{sp.406} flg.

\so{Arnauten}, Nachkommen der Pelasger? \pageref{sp.146}.

\so{Arnautisch} \pageref{sp.256}. \sed{Suffigirte Artikel \pageref{sp.273}.}

\sed{\so{Arowaken} \pageref{sp.390}.}

\so{Art und Weise}. Adverbiale Bestimmungen der – \pageref{sp.101}.

\so{Artentheilung}, sprachliche \pageref{sp.15}.

\fed{{\textbar}470{\textbar}}

\so{Articulation}, deren Begriff \pageref{sp.4} flg. Schärfe der A., verschiedenes Verhalten der Sprachen in dieser Hinsicht \pageref{sp.34}. Mangelhafte A. aus Bequemlichkeit \pageref{sp.181}–\pageref{sp.182}. Normale, flüchtige, übertreibende, Einfluss auf die Gestaltung des Lautwesens \pageref{sp.183}. \sed{Bevorzugung und Verwahrlosung in derselben \pageref{sp.205}.} Einfluss der Stimmung, Stimmungsmimik \pageref{sp.376}–\pageref{sp.380}. Innere A. \pageref{sp.432}–\pageref{sp.437}.

\so{Artikel}, bestimmter, der sinnlichen Anschaulichkeit dienend \pageref{sp.415}–\pageref{sp.416}.

\so{Aschanti} \pageref{sp.150}, \pageref{sp.282}.

\so{Assimilation} der Laute \pageref{sp.37}–\pageref{sp.38}.

\so{Association}, vgl. \so{Abstraction}, \so{Analogie}. – A. der Vorstellungen \pageref{sp.43}–\pageref{sp.44}.

\so{Assonanz}, lautsymbolisch empfunden \pageref{sp.220} flg.

\sed{{\textbar}{\textbar}490{\textbar}{\textbar}}

\so{Assyrer}, ihre sprachkundlichen Arbeiten \pageref{sp.18}.

\so{Aesthetik}. Ihr Einfluss auf die Grammatik \pageref{sp.95}.

\so{Aesthetisches Verhalten} der zu erlernenden fremden Sprache gegenüber \pageref{sp.83}–\pageref{sp.84}.

\so{Athapaskische Sprachen} \pageref{sp.405}, \pageref{sp.423}.

\so{Aethiopisch}. \sed{Hamitische Sprache \pageref{sp.160}.} Griechische Einflüsse \pageref{sp.272}. Possessive Conjugation \pageref{sp.391}.

\sed{\so{Athmung} und Tonerzeugung \pageref{sp.225}.}

\so{Attribute}, adnominale und adverbiale \pageref{sp.101}, \pageref{sp.462}. Stellung voran oder nach \pageref{sp.149}, A. und Prädicate \pageref{sp.451}–\pageref{sp.459}.

\so{Attributivprädicate} \pageref{sp.458}.

\so{Auch} \pageref{sp.103}.

\so{Auer}, J. G., Elements of the Gedebo Language \pageref{sp.379}.

\so{Auge}, geistiges, Uebung desselben \pageref{sp.32}. \inlineupdate{Äusserung}{Aeusserung} desselben in der Sprache \pageref{sp.325}.

\sed{\so{Ausdrucksfähigkeit} der Sprache und Ausdrucksbedürfniss \pageref{sp.394}.}

\sed{\so{Ausführlichkeit} der Grammatik \pageref{sp.112}.}

\so{Ausnahmen}. Unregelmässige Formen \pageref{sp.64}. A. gegen die Lautgesetze \pageref{sp.141}, \pageref{sp.187}–\pageref{sp.209}.

\so{Ausruf}, ausrufende Rede \pageref{sp.319} flg. Formloser \pageref{sp.345}, \pageref{sp.360}. Inhalt des A. in der Ursprache und den jetzigen Sprachen \pageref{sp.349}.

\so{Ausrufsätze} \pageref{sp.103}, \pageref{sp.321}.

\so{Ausschliesslichkeit}, s. Nur.

\so{Aussprache} (vgl. \so{Laute}), Schwankungen \pageref{sp.33}–\pageref{sp.34}. Gleichberechtigte mundartliche \pageref{sp.125}–\pageref{sp.126}. \inlineupdate{Änderungen}{Aenderungen} in der A., gleiche und verschiedene A. \pageref{sp.184}, \pageref{sp.245}. Gezierte \pageref{sp.187}. Einathmende A. \pageref{sp.225}.

\so{Ausspracheweise}, Stimmungsmimik \pageref{sp.376}–\pageref{sp.380}.

\so{Aussterben} von Sprachen und Mundarten \pageref{sp.146}, \pageref{sp.260}–\pageref{sp.261}.

\so{Austausch} der Sprache = Ausgleichung \pageref{sp.259}. Vgl. \so{Sprachmischung}.

\so{Australische Sprachen} \pageref{sp.142}. Den kolarischen verwandt? \pageref{sp.147}, \sed{\pageref{sp.152},} \pageref{sp.281}. Casus activo-instrumentalis und neutro-passivus \pageref{sp.151}. Unsichere Articulation \pageref{sp.194}. Suffigirender Bau \pageref{sp.349}. \sed{Sprachverwandtschaft? \pageref{sp.280}.} Schwierigkeit, ihren Werth zu bestimmen \pageref{sp.425}. Die verheiratheten Weiber behalten auch im fremden Volke ihre Muttersprache bei \pageref{sp.428}.

\so{Auszug} aus der Grammatik \pageref{sp.108}.

\so{Autochthonen} und ihre Sprachen \pageref{sp.143}.

\sed{\so{Aymará} \pageref{sp.389}.}

\sed{\so{Azteken} \pageref{sp.389}; ihre Schrift \pageref{sp.131}.}

\subsection*{B.}\label{reg.B}\pdfbookmark[1]{B.}{reg.B}

\sed{\so{Baarda}, M. J. van, \pageref{sp.475}.}

\sed{\so{Babylonischer Thurm} \pageref{sp.386}.}

\so{Bachstelze} \pageref{sp.216}.

\so{Bacon}, Francis \sed{\pageref{sp.15},} \pageref{sp.158}.

\so{Bagrima}, \so{Baghirmi} \sed{\pageref{sp.208},} \pageref{sp.282}.

\so{Bahuvrîhi-Composita} \pageref{sp.357}.

\sed{\so{Bakaïrí-Sprache}, Unsichere Articulation \pageref{sp.194}.}

\so{Bambara} \pageref{sp.150}, \pageref{sp.282}.

\so{Bantusprachen}. Lautwesen \pageref{sp.34}. Verbreitung \pageref{sp.142}. Prä- und suffigirender Bau \pageref{sp.149}. Entferntere Verwandte? \pageref{sp.150}. Substantivische Classen \pageref{sp.150}. Lepsius’ Hypothese \pageref{sp.282}. Gleiche Denkgewohnheiten \pageref{sp.293}. \sed{„Nichtwortige“ Sprachen \pageref{sp.342}.} Congruenzgesetz \pageref{sp.390}. \sed{Sprachbau und Volkscharakter \pageref{sp.420}.} Unterscheidung der Redetheile \pageref{sp.440}. Prädicativer Satzbau \pageref{sp.453}. Satzverknüpfung \pageref{sp.465}. Congruenz der Tempora und Modi \pageref{sp.466}.

–\so{bar} \pageref{sp.122}.

\so{Barea} \pageref{sp.160}, \pageref{sp.282}.

\so{Bari} \pageref{sp.282}.

\so{Barmanisch}, dessen Bau \pageref{sp.149}, \pageref{sp.257}. Innere Sprachform \pageref{sp.151}. Attributiver Satzbau \pageref{sp.453}.

\so{Barth}, Heinr. \pageref{sp.69}.

\sed{\so{Barth}, J. \pageref{sp.160}, \pageref{sp.411}.}

\so{Baskisch} \pageref{sp.54}, \sed{\pageref{sp.102}, \pageref{sp.299}, \pageref{sp.301}, \pageref{sp.383}. Lautgesetze \pageref{sp.190}, \pageref{sp.193}. Euphonische Einschiebsel \pageref{sp.202}. Höflichkeitsformen \pageref{sp.246}.} Einengung des Gebietes \pageref{sp.146}. \sed{Genitiv und Adjectivum \pageref{sp.455}. Baskische Wurzeln im Französischen \pageref{sp.251}.} Conjugation \pageref{sp.423}. Schwierigkeit der Beurtheilung \pageref{sp.426}.

\so{Bataver}, ein fränkischer Stamm? \pageref{sp.159}. 

\so{Batta-Sprache}. Zauberformeln \pageref{sp.107}. Schrift \pageref{sp.131}. Unsichere Articulation \pageref{sp.193}. Sandhi \pageref{sp.199}. Lautsymbolik \pageref{sp.223}, \pageref{sp.379}.

\fed{{\textbar}471{\textbar}}

\sed{\so{Baure} \pageref{sp.390}.}

\sed{\so{Becker}, K. F. \pageref{sp.15}.}

\so{Bedeutungen}, verschiedene desselben Wortes in verwandten Sprachen \pageref{sp.76}. B. und Gebrauch der Wörter bedingen einander \pageref{sp.125}. Woran erkennt man die ursprünglichsten \pageref{sp.157}.

\so{Bedeutungswandel} durch das lautsymbolische Gefühl \sed{\pageref{sp.214},} \pageref{sp.222}. Die Lehre vom B. \pageref{sp.227}–\pageref{sp.229}. Classification \pageref{sp.229}–\pageref{sp.231}. Bewegende Mächte \pageref{sp.231} flg.

\so{Bedingung}. Ausdrucksformen dafür \pageref{sp.95}, \pageref{sp.98}. Stellung im synthetischen Systeme der Grammatik \pageref{sp.104}.

\sed{{\textbar}{\textbar}491{\textbar}{\textbar}}

\so{Bedscha} \sed{\pageref{sp.240},} \pageref{sp.282}.

\so{Befehl} \pageref{sp.103}, \pageref{sp.322}. Verschiedene Formen \pageref{sp.472}, \pageref{sp.473}.

\so{Begriff}. Der neue B. verlangt einen Ausdruck. Woher dieser? \pageref{sp.230}.

„\so{Bei Muttern}“ \pageref{sp.45}.

\so{Beichten} \pageref{sp.231}.

\so{Beispiele}, grammatische, zum Beweise, zur Verdeutlichung und zur Uebung \pageref{sp.116}.

\so{Belebung} des Leblosen \pageref{sp.315}–\pageref{sp.317}.

\sed{\so{Bellen}, Sprachversuch? \pageref{sp.304}. Als Symbol für Hund \pageref{sp.310}.}

\so{Belgien}, dessen Sprachen \pageref{sp.54}.

\so{Benedictivus}, modus \pageref{sp.114}.

\so{Benennung} der Dinge. Vergleiche \pageref{sp.40}–\pageref{sp.43}. Elliptisches Verfahren \pageref{sp.366}. Nach hervorragenden Eigenschaften \pageref{sp.381}.

\so{Benfey}, Th. \sed{\pageref{sp.252},} \pageref{sp.295}.

\sed{\so{Bentley}, W. H. \pageref{sp.421}.}

\sed{\so{Beobachtungsgabe} des Grammatikers \pageref{sp.104}–\pageref{sp.105}.}

\so{Bequemlichkeit} als sprachgeschichtliche Macht \pageref{sp.181}–\pageref{sp.185}\sed{, \pageref{sp.256}}. Annahme der bequemeren Sprache \pageref{sp.262}. Zweierlei Arten der B. \pageref{sp.429}.

\so{Berberisch}. Schrift \pageref{sp.131}.

\so{Berber-Sprachen} \pageref{sp.142}, \pageref{sp.160}\sed{, \pageref{sp.299}}. \sed{Das pleonastische feminine t \pageref{sp.214}. Geschlechtszeichen t \pageref{sp.240}. Casussuffixe \pageref{sp.353}.}

\sed{\so{Bernhardi}, A. F. \pageref{sp.15}.}

\sed{\so{Le Berre} \pageref{sp.421}.}

\so{Berufsarten} bestimmen die Denkgewohnheiten \pageref{sp.44}. – auf niederer Culturstufe völkerweise vertheilt \pageref{sp.44}.

\so{Beseelung} des Unbeseelten \pageref{sp.315}–\pageref{sp.317}.

\so{Bestimmungen}, nähere, s. \so{Attribute}.

\so{Betonung}, die Worttrennung fördernd \pageref{sp.132}. Unregelmässige, der Antithese zuliebe \pageref{sp.227}, \pageref{sp.375}. Ihr Wesen; ob aus ihr die Stellungserscheinungen zu erklären? \pageref{sp.373}–\pageref{sp.376}. B. als Mittel der grammatischen Formung \pageref{sp.451}.

\sed{\so{Bevorzugung} in der Articulation \pageref{sp.205}.}

\so{Beweis}, inductiver, der grammatischen Lehren, überall nothwendig \pageref{sp.91}–\pageref{sp.92}, \pageref{sp.114}.

\so{Beziehungen} der Satzglieder und Sätze zu einander und des Sprechenden zur Rede \pageref{sp.448} flg.

\so{Bgai} – Karen \pageref{sp.201}.

\so{Bibelübersetzungen} als Texte zur Spracherlernung \pageref{sp.51}, \pageref{sp.73}.

\so{Bickell}, G., Outlines of Hebrew Grammar \pageref{sp.113}.

\so{Bicol}. Bildsamkeit \pageref{sp.349}.

\so{Bilabiale} \pageref{sp.36}.

\so{Bild} als Vorläufer der Schrift \pageref{sp.127}.

\sed{\so{Bildliche} Ausdrücke, psychologische Erklärung \pageref{sp.42}–\pageref{sp.43}.}

\so{Bildungsmittel} der Sprache, s. \so{Formenmittel}.

\so{Bilin} \pageref{sp.160}, \pageref{sp.307}. \sed{Femininbildung. \pageref{sp.240}.}

\so{Bindemittel} als Stoffe und als formende Kräfte \pageref{sp.325}.

\so{Biologie} der Sprache \pageref{sp.17}.

\so{Bisaya}. Bildsamkeit \pageref{sp.349}.

\so{Bischari} \pageref{sp.160}.

\so{Biscuit}, holländ. beschuit, westphäl. Beschütchen \inlineupdate{\pageref{fp.216}.}{\pageref{sp.267}.}

\so{Bismarck}. „Wurschtigkeit“ \pageref{sp.45}.

\so{Bitte} \pageref{sp.103}, \pageref{sp.318} flg.

\so{Blau}, flavus \pageref{sp.153}.

\so{Bleek}, W. H. J. \pageref{sp.281}\sed{, \pageref{sp.420}}.

\so{Blondhaarigkeit} der Germanen, Kelten und Finnen \pageref{sp.293}.

\so{Boccaccio}. Seine Sprache und das heutige Italienisch \pageref{sp.139}.

\so{Böhmisch}, s. Czechisch.

\so{Böhtlingk}, Otto v., Die Sprache der Jakuten \pageref{sp.52}.

\sed{\so{Boilat} \pageref{sp.408}.}

\so{Bongo} \pageref{sp.282}.

\so{Bopp}, Franz, Conjugationssystem \pageref{sp.26}. Vergleichende Grammatik \pageref{sp.26}, \pageref{sp.31}. Philologische Nebenbeschäftigungen \pageref{sp.136}. B. weist die malaischen und kaukasischen Sprachen dem indogermanischen Stamme zu \sed{\pageref{sp.144},} \pageref{sp.155}, \pageref{sp.156}, \pageref{sp.266}. \sed{Sanskritwurzeln \pageref{sp.252}. Vergleich mit Schleicher \pageref{sp.164},\pageref{sp.172}.} Etymologische Richtung seines Forschens \pageref{sp.169}–\pageref{sp.170}, \pageref{sp.179}.

\so{Born} – Brunnen \pageref{sp.267}.

\so{Borneo} \pageref{sp.147}.

\so{Bornu-Sprache}, s. Kanuri.

\sed{\so{Bourdon}, B. \pageref{sp.313}.}

\so{Brasilien}. Vereinzelte Jägerstämme \pageref{sp.177}.

\so{Bribri} \pageref{sp.423}.

\sed{\so{Bruchmann}, Curt \pageref{sp.229}.}

\so{Brugmann}, Karl. \fed{Griechische Gram}{\textbar}472{\textbar}\-\fed{matik \pageref{fp.119}.} Grundriss der vergleichenden Grammatik der indogermanischen Sprachen \pageref{sp.169}–\pageref{sp.170}. Das grammatische Geschlecht \pageref{sp.255}. \sed{Urformen für „sechs“ \pageref{sp.279}.}

\sed{\so{Bücherverbrennung} \pageref{sp.19}.}

\so{Buchstabenschrift} \pageref{sp.131}. \sed{Deren Erfinder \pageref{sp.18}.}

\inlineupdate{\so{Buddhismus}}{\so{Buddhismus,}} \sed{Einfluss auf das Japanische \pageref{sp.24}. Buddhistischer Einfluss auf die koreanische Schrift \pageref{sp.130}. B.} verbreitet indische Fremdwörter \sed{\pageref{sp.148},} \pageref{sp.231}. Sonstiger sprachlicher Einfluss \pageref{sp.271}.

\sed{{\textbar}{\textbar}492{\textbar}{\textbar}}

\sed{\so{budh} \pageref{sp.200}.}

\so{Bühnenmässige Sprache} \pageref{sp.127}.

\sed{\so{Bulgarisch}. Suffigirte Artikel.} \edins{\pageref{sp.273}.}

\so{Bullom} \pageref{sp.282}.

\so{Burjätisch} \pageref{sp.383}. Vocalharmonie \pageref{sp.403}.

\so{Burnouf}, Eugen \sed{\pageref{sp.28},} \pageref{sp.31}.

\so{Burzen} in Siebenbürgen, deren Dialekt \pageref{sp.190}.

\so{Buschmannsprachen.} Schnalzlaute \pageref{sp.34}\sed{, \pageref{sp.269}}. Kein grammatisches Geschlecht \pageref{sp.150}. Lepsius’ Hypothese \pageref{sp.282}–\pageref{sp.283}.

\sed{\so{Byren} \pageref{sp.313}.}

\so{Byrne}, Principles of the Structure of Language. Urtheil über die amerikanischen Sprachen \pageref{sp.424}. Sein Werk \pageref{sp.426}–\pageref{sp.427}, \pageref{sp.459}.

\so{Byzantiner} Philologen \pageref{sp.21}.

\subsection*{C.}\label{reg.C}\pdfbookmark[1]{C.}{reg.C}

\so{Cacare} \pageref{sp.314}.

\so{Caldwell}, R., Compar. Grammar of the Dravidian Languages \pageref{sp.344}.

\so{Cambodjanisch} hat keinen Wortaccent \pageref{sp.148}. Schwierigkeit der Beurtheilung \pageref{sp.426}.

\sed{\so{Campa}, siehe Anti.}

\so{Canaresisch} (Karnatta, Kannadi). Positive und negative Conjugationsformen \pageref{sp.344}.

\so{Caraibisch}. Männer- und Weibersprache \pageref{sp.248}–\pageref{sp.249}, \pageref{sp.428}. Formenbildung \pageref{sp.328}.

\so{Caramba} (spanisch) \pageref{sp.322}.

\so{Carey}, Sanskrit-Grammatik \pageref{sp.26}.

\so{Caricatur} = übertreibende Nachahmung \pageref{sp.104}.

\sed{\so{Caspari}, Gramm. arabe \pageref{sp.372}.}

\sed{\so{Castilianisch}, herrschende Mundart in den oberen Klassen Spaniens \pageref{sp.260}.}

\so{Castrén}, Alexander \pageref{sp.81}, \pageref{sp.69}\sed{, \pageref{sp.171}.}

\so{Casus}. Die Lehre von ihnen im synthetischen Systeme \pageref{sp.102}. C. oder Postpositionen? \pageref{sp.115}. Verluste in indogerman. Vorzeit und später \pageref{sp.253}–\pageref{sp.254}, \pageref{sp.354}. In den uralaltaischen und den malaischen Sprachen \pageref{sp.415}\sed{, \pageref{sp.420}}. Werth der C. \pageref{sp.461}–\pageref{sp.463}.

\sed{\so{Casus Activus} \pageref{sp.102}.}

\so{Catalonisch} \pageref{sp.54}.

\so{Causalität}. Ausdrucksformen dafür \pageref{sp.95}.

\so{Causalsätze} \pageref{sp.104}.

\so{Causativum} \pageref{sp.102}.

\sed{\so{Celebes} \pageref{sp.350}.}

\so{Cerebrale} s. \so{Velare}.

\sed{\so{Champollion-Figeac},}\edins{{\textbar}{\textbar}\so{Champollin-Fignac},} \sed{J.Fr., Hieroglyphen-Untersuchungen \pageref{sp.28}.}

\so{Charakter der Sprachen} \pageref{sp.178}.

\sed{\so{Charencey} \pageref{sp.442}.}

\sed{\so{Chibcha} \pageref{sp.423}.}

\so{Chilenisch}, \so{Chilidügü}. Unsichere Articulation \pageref{sp.194}. Diminutiva \pageref{sp.379}.

\so{Chinesen}, ihre Arbeiten zur Sprachkunde \pageref{sp.19}. Knotenschnüre als Vorläufer der Schrift \pageref{sp.128}.

\so{Chinesisch}. Wortbildung durch Übertragung \pageref{sp.42}. Dialekte \pageref{sp.58}. Einrichtung der Grammatik: analytisches System \pageref{sp.90}–\pageref{sp.92}; synthetisches System \pageref{sp.101}–\pageref{sp.104}. \sed{Modalausdrücke \pageref{sp.103}.} Neutra transitiva \pageref{sp.102}. Stilistik \pageref{sp.105}–\pageref{sp.106}. Syntaktische Lehrsätze in Paradigmen und Formeln \pageref{sp.114}–\pageref{sp.118}. Wortschrift \pageref{sp.128}. Ideographische und phonetische Bestandtheile der Schriftzeichen \pageref{sp.129}. Scheidung zwischen attributiver und prädicativer Beziehung \pageref{sp.151}. \sed{Euphonik \pageref{sp.199}.} Rhythmik \pageref{sp.197}. Yaó – wünschen \pageref{sp.222}. Buddhistisches Ch. \pageref{sp.231}. Composita \pageref{sp.242}, \pageref{sp.243}. \sed{Rhetorische Frage \pageref{sp.244}.} Einsylbigkeit und Isolirung nicht ursprünglich \pageref{sp.255}, \pageref{sp.257}. šing = Wahrheit und Freiheit, ngí = Rechtlichkeit und Ehre \pageref{sp.265}, Einfluss auf andere Sprachen \pageref{sp.271}. Auslaut r oder l \pageref{sp.290}–\pageref{sp.291}. Wurzelsprache? \pageref{sp.296}. Weib und Du \pageref{sp.306}. \sed{Besitzt kein Verbum und keine Conjugation \pageref{sp.340}. Wurzelisolirende Sprache \pageref{sp.342}.} Isolirend, dabei hoch entwickelt \pageref{sp.346}, \pageref{sp.362}. \sed{tsái kia \pageref{sp.345}.} Hülfswörter \pageref{sp.347}. Stellungsgesetze und Inversionen \pageref{sp.372}. Betonungen \pageref{sp.377}, \pageref{sp.379}. Höhere Würdigung der Sprache \pageref{sp.389}. Genialität \pageref{sp.399}. Mangel grammatischer Redetheile \pageref{sp.440}. Ch. und Englisch \pageref{sp.440}, Gegensatz zu den europäischen Sprachen \pageref{sp.447}. Secundäre Prädicate \pageref{sp.459}. Der Objectscasus \pageref{sp.461}. Die Attributsverhältnisse \pageref{sp.462}. Aneinanderreihung und Verbindung der Sätze \pageref{sp.465}, \pageref{sp.469}. Ausdrücke für Können u.~s.~w. \pageref{sp.471}. Desgl. für Gewissheit \pageref{sp.471}. Sociale Modalität \pageref{sp.474}.

\so{Chiquito} \sed{\pageref{sp.390}.} \inlineupdate{hat}{Hat} keine Zahlwörter \pageref{sp.152}. \sed{Männer- und Weibersprache \pageref{sp.248}.}

\so{Choctaw} \pageref{sp.358}.

\fed{{\textbar}473{\textbar}}

\so{Christaller}, A. G. \pageref{sp.379}.

\so{Christenthum} und Sprachenkunde \pageref{sp.21}. Die Missionare \pageref{sp.25}. Hebräische und griechische Fremdwörter in Sprachen christlicher Völker \pageref{sp.230}–\pageref{sp.231}.

\so{Citate}, Stellenangaben, in den Collectaneen \pageref{sp.80}.

\so{Clans}, Stämme, Horden \pageref{sp.307}.

\sed{\so{Classicität} \pageref{sp.25}, \pageref{sp.139}.}

\so{Classification} der Welt in der Sprache nach der gemüthlichen Perspective \pageref{sp.307}. Morphologische \inlineupdate{Class.}{Classification} der Sprachen \pageref{sp.345}–\pageref{sp.360}.

\sed{{\textbar}{\textbar}493{\textbar}{\textbar}}

\sed{\so{Cochinchinesisch} (Khmêr) \pageref{sp.273}.}

\so{Codrington}, R.~H., The Melanesian Languages \pageref{sp.141}, \pageref{sp.280}.

\so{Colebrooke}, Sanskrit-Grammatik \pageref{sp.26}.

\so{Collectaneen} \pageref{sp.68}. Deren Anlegung und Führung \pageref{sp.77}–\pageref{sp.79}. Deren Prüfung und Ordnung \pageref{sp.79}–\pageref{sp.80}. C. zur Sprachenvergleichung: lexikalische \pageref{sp.166}–\pageref{sp.167}, phonologische, grammatische \pageref{sp.168}.

\so{Comparativ} \pageref{sp.103}.

\so{Composita} \sed{\pageref{sp.122}}, \so{Composition}, s. \so{Zusammensetzung}. \sed{Fähigkeit einzelner Sprachen zu deren Bildung \pageref{sp.236}.} Syntaktische \pageref{sp.359}.

\sed{\so{Composition} und Isolation \pageref{sp.346}.}

\so{Concessivsätze}. Verschiedene Formen derselben im Deutschen \pageref{sp.98}. \fed{(Sie sind auf S.~\pageref{fp.108} zwischen f) und g) Causalverhältnisse und Fortsetzung, Steigerung, einzuschalten.)}

\sed{\so{Concessivverhältnisse} \pageref{sp.104}.}

\sed{\so{Confucius}, seine Lehre in Japan \pageref{sp.24}.}

\so{Congenialität} einer fremden Sprache gegenüber \pageref{sp.84}.

\so{Congruenz}, falsche \pageref{sp.214}. Bedeutsamkeit für die \inlineupdate{Werthsbestimmung}{Werthbestimmung} der Sprachen \pageref{sp.390}. Vorgreifende C. \pageref{sp.402}. C. in den Bantusprachen \pageref{sp.420}–\pageref{sp.421}. C. der Tempora und Modi im zusammengesetzten Satze \pageref{sp.465}.

\so{Conjugation}, possessive und prädicative \pageref{sp.391}, \pageref{sp.460}. \sed{Bedeutung für die Sprachform \pageref{sp.339}.}

\so{Conjunctionen} \pageref{sp.104}. C. und Satzbau \pageref{sp.418}. Satzverknüpfende C. \pageref{sp.465} f\pageref{sp.1}g. 

\so{Construction}, Fallen aus der C. \pageref{sp.43}, \pageref{sp.182}. Einfluss der C. auf den Bedeutungswandel \pageref{sp.234}–\pageref{sp.236}.

\so{Coordination} \pageref{sp.101}. C. der Sätze \pageref{sp.104}. Übliche C., den Bedeutungswandel beeinflussend \pageref{sp.234}–\pageref{sp.236}.

\so{Copula} \pageref{sp.102}.

\sed{\so{cor} \pageref{sp.186}.}

\sed{\so{Cordova}, J. de, \pageref{sp.194}.}

\so{Cornisch}, ausgestorben \pageref{sp.146}.

\sed{\so{Cornwall}, keltische Sprachinsel \pageref{sp.146}.}

\so{Correctheit}, s. Richtigkeit.

\so{Creolensprache} \pageref{sp.158}, \pageref{sp.182}, \pageref{sp.279}.

\sed{\so{Culturerwerb} veranlasst Neubildungen von Begriffen \pageref{sp.231}.}

\so{Culturgewächse}. Deren Namen in den europäischen Sprachen \pageref{sp.297}.

\so{Cultursprachen}, junge, durch Akademien geregelt \pageref{sp.126}. C. scheiden sich in Standessprachen \pageref{sp.288}. Welche sind es? \pageref{sp.388}–\pageref{sp.389}.

\so{Culturvölker} erweitern den Begriff der Sprachgemeinschaft \pageref{sp.57}–\pageref{sp.58}.

\so{Culturwerth} der Sprachen und Völker \pageref{sp.388}–\pageref{sp.389}.

\so{Cultus}. Ausdrücke des C. \pageref{sp.107}.

\so{Cursorische Lectüre} bei der Spracherlernung \pageref{sp.78}.

\so{Cursus}, grammatische \pageref{sp.111}.

\so{Curtius}, Georg \pageref{sp.80}. \sed{Wurzeletymologie \pageref{sp.252}.}

\sed{\so{Cushing}, J. N., \pageref{sp.224}.}

\so{Czechisch}. Tonfall \pageref{sp.34}, \pageref{sp.431}. Mangel des dicken l \pageref{sp.269}. Beständigkeit der Sprache \pageref{sp.428}.

\subsection*{D.}\label{reg.D}\pdfbookmark[1]{D.}{reg.D}

\so{Dame} im Schachspiel, deren Name \pageref{sp.268}.

\so{Dänisch}. Graasmyge \pageref{sp.216}.

\so{Dankali} \pageref{sp.160}, \pageref{sp.282}, \pageref{sp.307}.

\so{Darwin}, Charles \pageref{sp.15}.

\so{Dativus} \pageref{sp.102}. Partikeln und Affixe mit d oder t \pageref{sp.153}.

\so{Dayak-Sprache}. Archaismen in den Dichtungen \pageref{sp.107}. Unsichere Articulation \pageref{sp.194}. \sed{„Nichtwortige“ Sprache \pageref{sp.342}.} Einfachheit des Baues \pageref{sp.349}. Verlegung des Plurals in das Verbum \pageref{sp.446}.

\so{Declamation} \pageref{sp.315}.

\so{Defectivsystem} in den indogermanischen und anderen Sprachen \pageref{sp.352}, \pageref{sp.398}, \pageref{sp.400}. In den semitischen Sprachen \pageref{sp.420}.

\so{Definition}, grammatische, Wort- und Sachdef. \pageref{sp.1}. – der Wörter und grammatischen Formen im Gegensatze zu Übersetzungen \pageref{sp.47}–\pageref{sp.48}.

\so{Defter} - διφθέρα \pageref{sp.264}.

\so{Delaware}, s. \so{Lenape}.

\so{Delbrück}, Einleitung in das Sprachstudium \pageref{sp.171}\sed{, \pageref{sp.180}}. Die indogerman. Verwandtschaftsnamen \pageref{sp.294}.

\fed{{\textbar}474{\textbar}}

\so{Denken}, Begriff desselben \pageref{sp.6} flg. – und Sprechen, gleichzeitiges Verhalten Beider \pageref{sp.43}–\pageref{sp.44}. Gegenseitige Förderung Beider \pageref{sp.312}. \sed{D. als Stoff der Rede \pageref{sp.324}.}

\so{Denkgewohnheiten}, individuell und national verschieden \pageref{sp.44}, individuelle, äussern sich in Sprachgewohnheiten \pageref{sp.98}–\pageref{sp.99}, \pageref{sp.150}. D. und grammatische Kategorien \pageref{sp.253}. Rassenweise gleichmässige \pageref{sp.293}. Einfluss auf die Wortstellungsgesetze \pageref{sp.372}; – auf die Betonung \pageref{sp.376}; – auf die Bildung grammatischer Redetheile \pageref{sp.382}. Worauf können sich die D. richten? \pageref{sp.428}.

\sed{\so{Denkvermögen} verleiht der Sprache Formalität \pageref{sp.329}.}

\sed{{\textbar}{\textbar}494{\textbar}{\textbar}}

\so{Denominatio} fit a potiori \pageref{sp.56}.

\so{Denominativum} \pageref{sp.102}.

\so{Dentale} \pageref{sp.36}. Unorganische im Deutschen, Holländischen, Malaischen, Mafoor \pageref{sp.201}.

\so{Deprecativus}, modus \pageref{sp.114}.

\so{Deutlichkeit} als sprachgeschichtliche Macht \pageref{sp.181}–\pageref{sp.185}. Verdeutlichungen, den Bedeutungswandel beeinflussend \pageref{sp.239}–\pageref{sp.243}.

\so{Deutsche Dialekte} \pageref{sp.284} flg. Baltisch, dessen Tonfall \pageref{sp.34}. Thüringisch, das Singen \pageref{sp.34}. Niederrheinisch: Lautverschiebung \pageref{sp.189}. „derschlagen, derzählen“ \pageref{sp.201}. Conjugirte Conjunctionen \pageref{sp.214}, \pageref{sp.398}. Ostseeprovinzen, Einfluss des Esthnischen und Lettischen \pageref{sp.269}. Ostpreussischer: Diminutiva \pageref{sp.218}. Altenburgischer: „meech, halch“ \pageref{sp.284}. Siebenbürgische \pageref{sp.286}. Obersächsisch: „lernen“ \pageref{sp.316}. Dgl. Stimmungsmimik in der Lautbildung \pageref{sp.378}. Tempora und Modalausdrücke in den süddeutschen D. \pageref{sp.431}.

\so{Deutsche Sprache}. Vorherrschen von e und n \pageref{sp.34}. Wortstellungsgesetze \pageref{sp.63}. Gleichberechtigte Ausspracheweisen, Provinzialismen und Archaismen \pageref{sp.125}–\pageref{sp.127}. Die Orthographie \pageref{sp.132}–\pageref{sp.133}. Umlaut \pageref{sp.199}. Pluraldoubletten \pageref{sp.254}. In Amerika \pageref{sp.261}–\pageref{sp.262}. Französische und englische Lehnwörter \pageref{sp.264}. Kosenamen \pageref{sp.277}–\pageref{sp.278}. Der Umlaut \pageref{sp.401}. Wörter mit zweifachen Pluralen \pageref{sp.444}. Stellung des Verbum finitum \pageref{sp.429}. Bildung zusammengesetzter Sätze, Einschachtelung \pageref{sp.467}–\pageref{sp.469}.

\so{Deutschrussen}, ihre Mehrsprachigkeit \pageref{sp.70}.

\sed{\so{deva} \pageref{sp.186}.}

Διάβολος \pageref{sp.230}.

\so{Dialekt} und Sprache, Haupt- und Unterdialekte \pageref{sp.54}–\pageref{sp.58}. Mischungen der Dialekte \pageref{sp.177}, \pageref{sp.273}–\pageref{sp.277}. Schichtung verschiedener Dialekte in einem Individuum \pageref{sp.269}. Dialektforschung \pageref{sp.283}–\pageref{sp.287}.

\so{Dicere}, zeigen \pageref{sp.163}.

\sed{\so{Didaktische} Grammatiken \pageref{sp.113}.}

\so{Diez}, F , Grammatik der romanischen Sprachen \pageref{sp.173}.

\so{Differenzirung} der Bedeutung \pageref{sp.100}. Bei Formdoubletten \pageref{sp.254}. Vgl. Entähnlichung.

\so{Diminutiva}, die ursprünglichen Namen der Dinge verdrängend \pageref{sp.242}. D. als Formungen \pageref{sp.324}. Im Italienischen und Spanischen \pageref{sp.445}.

\so{Ding} – Eigenschaft, Zustand, Bethätigung \pageref{sp.381} flg., \pageref{sp.453} flg.

\so{Dinka} \pageref{sp.282}.

\so{Donner}, O. \pageref{sp.294}.

\so{Doppelformen}, s. \so{Doubletten}.

\so{Doppelsinn}, Euphemismus und Zote \pageref{sp.248}.

\so{Doppelung} in. Onomatopöien der menschlichen Ursprache \pageref{sp.255}\sed{, \pageref{sp.314}}. D. als ursprüngliche Form \pageref{sp.326}.

\so{Doubletten}. \sed{Entähnlichung der Bedeutung bei Doubletten \pageref{sp.238}. Gleichwerthige Wörter infolge unsicherer Articulation \pageref{sp.195}.} Gleichwerthige grammatische Formen \pageref{sp.254}. Wörter gleicher Herkunft, aber verschiedener Lautform in derselben Sprache \pageref{sp.266}–\pageref{sp.267}, \pageref{sp.276}. Dialektische \pageref{sp.285}.

\so{Drâvidische Sprachen}, suffigirend \inlineupdate{\pageref{sp.29};}{\pageref{sp.29},} \pageref{sp.142}, \pageref{sp.149}, \pageref{sp.349}. Dentale und Cerebrale \pageref{sp.269}. \sed{Suffigirender Bau \pageref{sp.349}.} Vocalharmonie \pageref{sp.402}. Schwierigkeit der Beurtheilung \sed{\pageref{sp.281},} \pageref{sp.426}.

\so{Du}. Ich und Du \pageref{sp.2}. In der Ehe \pageref{sp.306}. Du und Weib durch die gleichen Laute ausgedrückt \pageref{sp.306}. Das Du bei der Rede \pageref{sp.318}.

\so{Dualis} im Schwinden begriffen \pageref{sp.254}\sed{, \pageref{sp.393}}.

\so{Dubec}. Ursprung des Wortes \pageref{sp.41}.

\fed{{\textbar}475{\textbar}}

\so{Ducken} – tauchen \pageref{sp.267}.

\so{Duke of York Island} \pageref{sp.165}.

\so{Durchdringung}, gegenseitige der Stoffe, als Verbindung \pageref{sp.324}.

\so{Duret}, Claude, Thresor des langues etc. \pageref{sp.27}.

\so{Dvandva-Composita} \pageref{sp.114}, \pageref{sp.359}.

\subsection*{E.}\label{reg.E}\pdfbookmark[1]{E.}{reg.E}

Ĕ in der indogermanischen Ursprache \pageref{sp.186}.

\so{Efik} \pageref{sp.282}.

\sed{\so{Egede}, P., Grammatik \pageref{sp.248}.}

\so{Ego} – aham \pageref{sp.214}.

\so{Eigennamen} \pageref{sp.107}.

\so{Eigenschaft} und Ding \pageref{sp.381}, \pageref{sp.453} flg.

\so{Ein} Tager vierzehn, ein Stücker zwanzig u.~s.~w. \pageref{sp.60}.

\so{Einathmen} im Sprechen, einathmende Aussprache \pageref{sp.225}.

\sed{\so{Eindringlichkeit} als sprachgeschichtliche Macht \pageref{sp.185}.}

\sed{\so{Einfälle}, als Grundlage sprachgeschichtlicher Forschung \pageref{sp.289}.}

\so{Einheit} oder Mehrheit des Menschengeschlechts und der Sprachen \sed{\pageref{sp.62},} \pageref{sp.143}, \pageref{sp.395}.

\so{Einschachtelung} im deutschen Satzbaue \pageref{sp.468} flg.

\sed{\so{Einschiebsel}, euphonische \pageref{sp.202}.}

\so{Einschliesslichkeit}, s. \so{Auch}, \so{Noch}.

\so{Einsylbigkeit} schlechthin, nicht noth\-wendiger Urzustand der menschlichen Sprache \pageref{sp.255}, \pageref{sp.314}.

\sed{{\textbar}{\textbar}495{\textbar}{\textbar}}

\so{Einverleibende Sprachen} \pageref{sp.354}–\pageref{sp.359}, \pageref{sp.460}.

\so{Einzahl}, Singular \pageref{sp.101}.

\so{Einzelsprachen}, als Gegenstand der Sprachwissenschaft \pageref{sp.8}. \sed{E. u. Sprachwissenschaft \pageref{sp.49}–\pageref{sp.50}.} Ihre Zahl und ihr Umfang \pageref{sp.54}–\pageref{sp.58}. \sed{Deren Werthbestimmung \pageref{sp.428}. Ihre Erforschung \pageref{sp.75}–\pageref{sp.80}. Ziel der Erforschung \pageref{sp.76}. Grammatik \pageref{sp.81}–\pageref{sp.82}.}

\so{Einzelsprachliche Forschung}. Ihre Aufgaben und Grenzen \pageref{sp.8}–\pageref{sp.9}; \pageref{sp.58}–\pageref{sp.61}. Wichtigkeit der Sprachgeschichte in ihr \pageref{sp.60}. \sed{Verhältnis zur Sprachgeschichte \pageref{sp.140}.} Neue Fassung ihres Problems \pageref{sp.209}.

\so{Einzelvorstellung} \pageref{sp.325}.

\so{Eitelkeit} und Sprachschöpfung \sed{\pageref{sp.245},} \pageref{sp.309}.

\so{Elativus} \pageref{sp.114}.

Ἑλεημοσύνη \pageref{sp.231}.

\so{Elementarbücher}, grammatische \pageref{sp.108}–\pageref{sp.109}.

\sed{\so{Elementarmethode} in der Grammatik \pageref{sp.112}.}

\so{Elephant}, bos libycus \pageref{sp.41}.

\so{Ellipsen} \pageref{sp.101}. In der Wortschöpfung und Bedeutungsänderung \pageref{sp.238}. In der Classification der Redeformen \pageref{sp.321}. Voraussetzungen der E. \pageref{sp.367}–\pageref{sp.368}.

\sed{\so{Empfindungen}. E. als erste Anlässe zur Stimmäusserung \pageref{sp.312}.}

\so{Empfindungslaute} als Ausnahmen von den Lautgesetzen \pageref{sp.208}–\pageref{sp.209}. In der menschlichen Ursprache und den jetzigen Sprachen \pageref{sp.365}.

\sed{\so{Encounter Bay}, Sprache \pageref{sp.102}.}

\so{Encyklopädie}. Encyklopädisches Wörterbuch einer Sprache \pageref{sp.123}–\pageref{sp.124}.

\so{Englisch}. Laute und Stellung der Sprachorgane \pageref{sp.34}, \pageref{sp.36}. Orthographie \pageref{sp.133}. Schwund des pron. \pageref{sp.2}. pers. sing. \pageref{sp.152}. Germanische Grundlage \pageref{sp.158}. Neigung zur Lautverflüchtigung \pageref{sp.207}. \inlineupdate{–s}{S} statt –th als Endung der \pageref{sp.3}. pers. sing. \pageref{sp.213}. To want = wünschen \pageref{sp.222}. Periphrastisches Verbum mit to do \pageref{sp.239}. Prüderie \pageref{sp.249}. Das grammatische Geschlecht \pageref{sp.254}. Analytischer Bau \pageref{sp.257}. Hülfsverbum to will \pageref{sp.316}. \sed{Angelsächsische Suffixformen \pageref{sp.348}.} Betonung, Verba und Nomina unterscheidend \pageref{sp.379}. To be doing \pageref{sp.384}. Urtheile über das E. \pageref{sp.393}. Aneignungskraft \pageref{sp.429}. \sed{Mischsprache \pageref{sp.406}.} Wortverstümmelungen \pageref{sp.433}. Genitiv des Subjects mit Verbalnomen auf –ing \pageref{sp.467}. \inlineupdate{Social-Modalität}{Sozial-Modalität} \pageref{sp.474}–\pageref{sp.475}.

\so{Enklitische Wörter} \pageref{sp.348}–\pageref{sp.349}.

\so{Entähnlichung} der Laute \pageref{sp.200}. E. der Bedeutungen \pageref{sp.238}–\pageref{sp.239}.

\so{Entdeckung} von Sprachverwandtschaften. Zufall und Methode \pageref{sp.144}–\pageref{sp.145}.

\sloppypar{
\so{Entlehnung}. \inlineupdate{Übertragung}{Uebertra\-gung} von Namen auf neue Vorstellungen \pageref{sp.40}–\pageref{sp.43}. \sed{E. in der Lautgeschichte \pageref{sp.186}.} E. von Wörtern \pageref{sp.262}–\pageref{sp.268}. E. von Redeweisen und Stilformen \pageref{sp.270}–\pageref{sp.273}.
}

\so{Entwickelung}, \sed{sprachgeschichtliche, deren Erkenntnisquellen \pageref{sp.175}. E.,} freie, im Gegensatze zu äusseren Beeinflussungen \pageref{sp.178}. Angeblich aufsteigende \pageref{sp.415}.

\so{Entzifferung} \sed{\pageref{sp.28},} \pageref{sp.74}.

\so{Epenthese} im Zend \pageref{sp.199}\sed{, \pageref{sp.401}.}

\so{Epidemien}, sprachliche \pageref{sp.274}.

Ἐπίσκοπος \pageref{sp.231}.

\so{Erfahrungswissenschaft}, die Sprach\-wissenschaft als solche \pageref{sp.10}–\pageref{sp.11}.

\so{Erforschung} und Erlernung \pageref{sp.75}.

\so{Erlernung} der Sprachen zu praktischen Zwecken \pageref{sp.7}, \pageref{sp.16}. \sed{Deren Nothwendigkeit für den Sprachforscher \pageref{sp.50} ff.} Jede Sprache wird durch E. angeeignet \pageref{sp.61}.

\sed{\so{Erman}, Ad. \pageref{sp.160} Anm.}

\so{Eroberer} und Unterjochte, ihr sprachlicher Verkehr \pageref{sp.182}\sed{, \pageref{sp.260}.}

\fed{{\textbar}476{\textbar}}

\so{Ersatzdehnung} \pageref{sp.200}.

\so{Ersatzwörter}: Pronomina, Proadverbien, Proverba u.~s.~w. \pageref{sp.101}.

\so{Erscheinung}. Die Sprache als zu deutende E. \pageref{sp.84}–\pageref{sp.85} (vergl. \so{analytisches System}).

\so{Erweiterung} der Bedeutungen, s. \so{Bedeutungswandel}.

\sed{\so{Erweiterung} der Satztheile \pageref{sp.101}.}

\so{Erziehung}, zwei- oder mehrsprachliche der Kinder \pageref{sp.70}.

\so{Eskimo-Sprache} \pageref{sp.426}.

\so{Esthen}, ihre Rasse \pageref{sp.147}. Geistige Begabung \pageref{sp.389}.

\so{Esthnisch}, gepflegt und doch gefährdet \pageref{sp.146}. \sed{Vokalharmonie \pageref{sp.403}.}

\so{Ethik}, ihr Antheil an der Formung der Sprache \pageref{sp.95}.

\so{Ethnographie} und Sprachwissenschaft \pageref{sp.14}. Inwieweit sie die Vermuthung einer Sprachverwandtschaft begründen könne? \pageref{sp.148}.

\so{Etiquette}, gesellschaftliche, ihr Einfluss auf die Grammatik \pageref{sp.95}, \pageref{sp.246}.

\so{Etruskisch}, ausgestorbene Sprache \pageref{sp.146}.

\sed{\so{Etruskische} Wurzeln im Latein \pageref{sp.251}.}

\so{Etymologie} kann im Sprachbewusstsein schwinden oder verschoben werden \pageref{sp.60}–\pageref{sp.61}. E. als Erkenntnissgrund der Synonymik \pageref{sp.100}. E. als Grundlage eines Wörterbuches \pageref{sp.123}–\pageref{sp.124}\sed{, {\textbar}{\textbar}496{\textbar}{\textbar} \pageref{sp.179}}. E. als Theil der geschichtlichen Sprachforschung \pageref{sp.179}–\pageref{sp.181}. Bewusstsein der E. wirkt laut-erhaltend \pageref{sp.203}. Etymologisches Bedürfniss \pageref{sp.214}–\pageref{sp.218}. Die E. und das lautsymbolische Gefühl \pageref{sp.218}–\pageref{sp.225}. \sed{Falsche E., Einfluss auf die grammatische Behandlung der Wörter \pageref{sp.225}.} Wurzeln \pageref{sp.295}–\pageref{sp.297}. E. und innere Sprachform \pageref{sp.328}–\pageref{sp.334}\sed{, \pageref{sp.336}, \pageref{sp.344}}. Bedeutsamkeit für die Werthbestimmung der Sprachen \pageref{sp.395}–\pageref{sp.398}, \pageref{sp.437}. Wichtigkeit für eine allgemeine Wortschatzkunde \pageref{sp.482}–\pageref{sp.483}.

\so{Etymologisches Bedürfniss} als sprachgeschichtliche Macht \pageref{sp.215}–\pageref{sp.218}. In den indogermanischen und in agglutinirenden Sprachen \pageref{sp.403}.

\so{Etymologisches Bewusstsein} \pageref{sp.404}.

\so{Etymology}, Begriff des Wortes bei den Engländern \pageref{sp.179}.

\so{Euphemismen} und Zoten \pageref{sp.248}–\pageref{sp.249}.

\so{Euphonik}, \inlineupdate{sandhi}{Sandhi} \pageref{sp.196}–\pageref{sp.205}.

\so{Euphonischer Zuwachs} der Wörter \pageref{sp.157}.

\so{Evolutionstheorie} \pageref{sp.216}.

\so{Ewhe, Ewe} \pageref{sp.150}, \pageref{sp.282}. Genitiv und Adjectivum \pageref{sp.455}.

\so{Exclusivität}, s. Nur.

\so{Exercitien} im Sprachunterrichte \pageref{sp.72}.

\so{Explosivlaute}, vocallose, von der Natur vorgebildet \pageref{sp.314}.

\so{Extemporalien} im Sprachunterrichte \pageref{sp.72}.

\sed{\so{Eys}, J. W. van, \pageref{sp.193}, \pageref{sp.202}.}

\subsection*{F.}\label{reg.F}\pdfbookmark[1]{F.}{reg.F}

F, bilabiales und labiodentales im Deutschen \pageref{sp.188}.

\sed{\so{Fabricius}, O. Grammatik \pageref{sp.248}.}

\so{Factivum} \pageref{sp.102}.

\sed{\so{Fady} \pageref{sp.245}.}

\so{Faidherbe}, General \pageref{sp.32}, \pageref{sp.69}.

\so{Fa}\texttt{Ꜣ}\so{ala}, in der arabischen Grammatik paradigmatisch verwendet \pageref{sp.117}.

\so{Falascha} \pageref{sp.282}.

\so{Familie}, Familienleben. Einfluss auf die Sprachschöpfung \pageref{sp.306}. F. und Clan \pageref{sp.307}.

\sed{\so{Fehlerlose} Sprache im Sinne des Sprachforschers \pageref{sp.62}.}

\so{Fetisch} \pageref{sp.315}.

\so{Feuerländer} \pageref{sp.187}.

\so{Fick}, A., Etymologisches Wörterbuch \pageref{sp.293}. Die ehemalige Spracheinheit etc. \pageref{sp.294}. Seine Etymologien \pageref{sp.295}.

\so{Fidschi}: \sed{\pageref{sp.280}} faka Praefix und selbständiges Verbum \pageref{sp.350}.

\so{Finnen}, ihre Rasse \pageref{sp.147}. Blondes Haar \pageref{sp.293}. Geistige Begabung \pageref{sp.389}.

\so{Finnisch}, gepflegt und doch gefährdet \pageref{sp.146}. \sed{Finnische Wurzeln im Germanischen \pageref{sp.251}.} Germanische Lehnwörter \pageref{sp.266}. \sed{Agglutinirende oder scheinwortige Sprache \pageref{sp.342}.}

\so{Finnische Sprachen}. Tonfall \pageref{sp.34}. Steinthal’s Urtheil über sie \pageref{sp.336}. \sed{Epenthese \pageref{sp.402}. Wandel im Stammvokalismus \pageref{sp.434}.} Adverbiale und adnominale Casus \pageref{sp.462}.

\so{Finnisch-ugrische Sprachen}. Vocalharmonie \pageref{sp.148}–\pageref{sp.149}. \sed{Accent \pageref{sp.212}. Schärfung \pageref{sp.352}.}

\so{Finno-tatarische Sprachen}, s. \so{Ural-altaische Sprachen}.

\sed{\so{Fiotessprache} \pageref{sp.397}.}

\sed{\so{Firlinger}, v. \pageref{sp.196}.}

\sed{\so{Flectirende} Sprachenklasse \pageref{sp.354}.}

\so{Flexion}, Anbildung nach Steinthal \pageref{sp.337}. Die s.~g. Flexion der indogermanischen Sprachen \pageref{sp.351}–\pageref{sp.352}, \pageref{sp.389}, \pageref{sp.398}–\pageref{sp.401}.

\sed{\so{Florentiner} Mundart und Lateinisch \pageref{sp.192}.}

\so{Flüche}, als Ausdruck der Versicherung \pageref{sp.183}. Euphemistisch abgeändert \pageref{sp.249}.

\fed{{\textbar}477{\textbar}}

\sed{\so{Flüchtigkeit} als sprachgeschichtliche Macht \pageref{sp.183}.}

„\so{Flügel}“ \pageref{sp.232}.

\sed{\so{Folgerichtigkeit} in der Lautverschiebung \pageref{sp.191}.}

\so{Form} der Rede, Sprachform \pageref{sp.327} flg. Innere \pageref{sp.327}–\pageref{sp.345}. Aeussere \pageref{sp.345} flg.

\so{Formativlaute}. \sed{Deren Stellung in der Grammatik \pageref{sp.88}.} Ihr Werth für den Verwandtschaftsnachweis \pageref{sp.153}. Umladung der F. \pageref{sp.214}. Ihre Entstehung aus selbständigen Wörtern, s. Agglutinationstheorie. – Schwund in Contactsprachen \pageref{sp.407}.

\so{Formdoubletten} sich in der Bedeutung differenzirend \pageref{sp.254}. Ursprüngliche \pageref{sp.361}.

\so{Formen}, grammatische, periphrastisch durch   Zusammensetzung entstanden (vgl. auch \so{Agglutinationstheorie}) \pageref{sp.241}. F. der Rede \pageref{sp.320}–\pageref{sp.323}. Ursprüngliche F. \pageref{sp.326}. Verschliff \pageref{sp.437}.

\so{Formenmittel} \sed{\pageref{sp.104}.} F. begreifen Mittel der Wort- und der Formenbildung in sich \pageref{sp.122}. Tonhöhe, Tonbiegung, Rhythmus, Pausen als grammatische F. \pageref{sp.450}–\pageref{sp.451}.

\sed{\so{Formensinn} der Sprachen \pageref{sp.394}.}

\so{Formeln}, graphische, zur Darstellung grammatischer Lehrsätze \pageref{sp.116}–\pageref{sp.119}. Religiöse, rechtliche und sonstige solenne, den Sprachgebrauch beeinflussend \pageref{sp.245}.

\sed{{\textbar}{\textbar}497{\textbar}{\textbar}}

\so{Formenbildung}, \so{Formenlehre}, ob im Sprachbewusstsein von der Wortbildung unterschieden? \pageref{sp.122}.

\sed{\so{Formkategorien} \pageref{sp.352}.}

\so{Formlosigkeit} der Sprachen (Steinthal) \pageref{sp.337} flg. (Fr. Müller) \pageref{sp.338}. Vorwurf der F. \pageref{sp.364}. Angebliche F. \pageref{sp.393}.

\so{Formosa} \pageref{sp.147}.

\so{Formprinzip} der Sprache \pageref{sp.17}.

\so{Formsprachen} \pageref{sp.342}, \pageref{sp.388}.

\so{Formung}, Verbindung, Gliederung \pageref{sp.324}.

\so{Formungstrieb} in der Sprache \pageref{sp.360}–\pageref{sp.365}\sed{, \pageref{sp.394}.}

\sed{\so{Formwurzeln} nach Fr. Müller \pageref{sp.338}.}

\sed{\so{Forschungsreisende}, linguistische \pageref{sp.69}.}

\so{Forster}, Sanskrit-Grammatik \pageref{sp.26}.

\so{Fortsetzung} als syntaktische Kategorie \pageref{sp.104}.

\so{Frage}. Formen der Fr. im Deutschen \pageref{sp.96}. Rhetorische Fr., statt der versichernden Rede \pageref{sp.183}, \pageref{sp.244}. Seelischer Thatbestand der Fr. \pageref{sp.310}. Fragende Rede \pageref{sp.319}.

\so{Fragesätze} \pageref{sp.103}. Die zwei Arten derselben \pageref{sp.468}–\pageref{sp.469}.

\so{Fränkisch} (Dialekt) und Holländisch \pageref{sp.159}.

\so{Französisch}. Inversionen \pageref{sp.103}, \pageref{sp.370}, \pageref{sp.371}. Die Akademie und die Belletristik \pageref{sp.126}. Orthographie \pageref{sp.132}. Fr. auf Hayti \pageref{sp.147}. Nasale \pageref{sp.148}. \sed{Neigung zur Lautverflüchtigung \pageref{sp.207}.} Lautsymbolisch anmuthende Wörter \pageref{sp.219}. \inlineupdate{Übertreibende}{Uebertreibende} Ausdrücke \pageref{sp.243}. Deutsche Lehnwörter \pageref{sp.264}. Doubletten von verschiedener Lautform \pageref{sp.267}. Die Dame im Schachspiele \pageref{sp.268}. Kosenamen \pageref{sp.278}. Fr. auf Hayti \pageref{sp.293}. \sed{cinq. \pageref{sp.298}. Lateinische Suffixformen \pageref{sp.348}.} Tempora der Vergangenheit \pageref{sp.430}. Eine \inlineupdate{Übereinstimmung}{Uebereinstimmung} mit dem Arabischen \pageref{sp.441}. \sed{Ausdrucksfähigkeit \pageref{sp.446}.} Stellung der attributiven Adjectiva \pageref{sp.457}. \sed{Soziale Modalität \pageref{sp.475}.}

\so{Freiheit} im Gebrauche der Sprache \pageref{sp.234}–\pageref{sp.239}\sed{, \pageref{sp.386}.}

\so{Fremdwörter}, s. \so{Lehnwörter}.

„Frug“ statt: fragte \pageref{sp.186}.

\so{Fulbe}, s. \so{Pul.}

\so{Functionslehre} als Gegenstand der allgemeinen Grammatik \pageref{sp.480}.

\so{Furcht}, das Leblose belebend und beseelend \pageref{sp.315}.

\so{Fürwörter}, Ersatzwörter \pageref{sp.101}.

\so{Futurum}, vgl. \so{Vorhaben} \pageref{sp.103}. F. der neuromanischen Sprachen \sed{\pageref{sp.159},} \pageref{sp.348}.

\subsection*{G.}\label{reg.G}\pdfbookmark[1]{G.}{reg.G}

\so{Gabelentz}, Hans Conon von der \pageref{sp.30}, \pageref{sp.49}. „Die melanesischen Sprachen“ \pageref{sp.74}, \sed{\pageref{sp.151},} \pageref{sp.280}, \pageref{sp.406}. Seine grammatische Kunst \pageref{sp.112}. Tscherokesische Grammatik \pageref{sp.358}. \sed{„Ueber das Passivum“ \pageref{sp.327}.} Beurtheilung von Sprachen \pageref{sp.427}. \inlineupdate{Über}{Ueber} das Passivum \pageref{sp.481}.

\sed{\so{Galela} \pageref{sp.391}, soziale Modalität \pageref{sp.475}.}

\so{Galibi} \pageref{sp.248}.

\so{Galla-Sprache} \pageref{sp.142}, \pageref{sp.282}, \pageref{sp.307}. Pronomina, Zahlwörter \pageref{sp.160}–\pageref{sp.161}.

\so{Gallicismen}, syntaktische, im Deutschen \sed{\pageref{sp.243},} \pageref{sp.272}.

\so{Gato} (ital., span.) – catus \pageref{sp.190}.

\sed{\so{Gatschet}, A. S.~\pageref{sp.194}, \pageref{sp.423}.}

\so{Gauch} – Kukuk \pageref{sp.208}.

\so{Gaunersprachen} \pageref{sp.45}. Hebräische Elemente in der deutschen, deutsche in der spanischen \pageref{sp.288}, \pageref{sp.289}.

\so{Gaussin}, P. L. J. B., Du dialecte de Tahiti \pageref{sp.463}.

\so{Gazelle-Halbinsel}, Neu-Pommern \pageref{sp.165}.

\so{Geberdensprache} \pageref{sp.2}, \pageref{sp.311}.

\fed{{\textbar}478{\textbar}}

\so{Gebietende Rede} \pageref{sp.319} flg.

\so{Gebrauch} wirkt erhaltend aber auch abnutzend \pageref{sp.182}.

\so{Gedächtniss}, dessen Antheil bei der Spracherlernung \pageref{sp.63}–\pageref{sp.64}.

\so{Gedanke}, s. \so{Denken.}

\sed{\so{Gedankenverkehr} als Zweck der Sprache \pageref{sp.55}.}

\so{Gedebo}, s. \so{Grebo.}

\so{Ge}\texttt{Ꜣ}\so{ez} \sed{\pageref{sp.282},} s. \so{Aethiopisch.}

\so{Geflügelte Worte} \pageref{sp.45}.

\so{Gegensatz} bei der sprachwissenschaftlichen Induction \pageref{sp.48}. G. in der Synonymik \pageref{sp.100}.

\so{Gegensinn} \pageref{sp.244}, \pageref{sp.381}.

\so{Gehör} und Gesicht. Warum Ersteres besser für die sprachliche Mittheilung geeignet? \pageref{sp.312}.

\sed{\so{Gehörseindrücke}, alphabetisch wiedergegeben \pageref{sp.299}.}

\so{Geist}, naiver – logisch geschulter \pageref{sp.39}. Geistesart der Sprachen und Völker, s. \so{Sprachwürderung}.

\so{Gemeinsinn} \pageref{sp.307}.

\so{Gemüth}, dessen Perspective entscheidend bei der Classification der Aussenwelt \pageref{sp.307}. \sed{G. und Stimmungsmimik \pageref{sp.378}.} Antheil an der Formung der Rede, psychologische Modalität \pageref{sp.472}–\pageref{sp.474}.

\so{Gemüthszustände}. Bildliche Benennung derselben \pageref{sp.43}.

\sed{{\textbar}{\textbar}498{\textbar}{\textbar}}

\so{Genealogisch-historische Sprachwissenschaft} (vgl. Sprachgeschichte). Ihre Aufgaben \pageref{sp.9}–\pageref{sp.10}. Ob = Sprachwissenschaft überhaupt? \pageref{sp.11}. Ihr Gegenstand \pageref{sp.135}–\pageref{sp.146}. Unterschied von der einzelsprachlichen Forschung \pageref{sp.138}–\pageref{sp.142}.

\so{Generationen}. Die Sprache verschiedener G. \pageref{sp.258}, \pageref{sp.284}.

\so{Genitivus}  \pageref{sp.101}. Lateinischer, \inlineupdate{griechischer}{griechischer}, deutscher \pageref{sp.115}. Partikeln und Affixe mit n \pageref{sp.153}. G. partitivus als Objectscasus \pageref{sp.462}. Zweierlei G. in den polynesischen Sprachen \pageref{sp.463}. G. absolutus \pageref{sp.467}.

\so{Genus verbi}. Lehre davon im synthetischen Systeme \pageref{sp.101}.

\so{Geographie}. Inwieweit sie eine Vermuthung der Sprachverwandtschaft begründe? \pageref{sp.146}–\pageref{sp.147}.

\so{Georgisch}. Lautwesen \pageref{sp.34}.

\so{Geräthe} nach Thieren benannt \pageref{sp.41}.

\so{Gerben}. Ursprüngliche Bedeutung des Wortes \pageref{sp.229}.

\so{Gerland}, Gg. Intensiva und Iterativa \pageref{sp.481}.

\so{Germania}, spanische Gaunersprache \pageref{sp.288}.

\so{Germanische Sprachen}. \sed{Lautwandelgesetze \pageref{sp.192}.} Eigenthümlichkeit ihrer Lautentwickelung: übertriebene Articulation \pageref{sp.183}. \sed{Deren Charakter \pageref{sp.178}.} Spaltung des th in ein hartes und ein weiches \pageref{sp.190}. Das –st der \pageref{sp.2}. pers. sing. \pageref{sp.203}. Das Imperfectum der schwachen Verba \pageref{sp.241}. Verlust von Praeteritformen \pageref{sp.253}–\pageref{sp.254}. Infix –r–, –l–, ein Irrlicht \pageref{sp.292}. Stellung des Adjectivums hinter, statt vor das Substantivum \pageref{sp.457}.

\so{Gerundien}. Satzverknüpfung durch G. \pageref{sp.466}.

\so{Gesammtvorstellung} \pageref{sp.325}.

\so{Gesang} und Sprache \pageref{sp.309}–\pageref{sp.310}, \pageref{sp.311}.

\sed{\so{Geschäftssprache} \pageref{sp.184}.}

\so{Geschichte} und Sprachwissenschaft \pageref{sp.13}. Geschichtliche Einflüsse auf den Culturwerth der Völker und Sprachen \pageref{sp.395}.

\so{Geschlecht}, grammatisches \pageref{sp.160}\sed{, \pageref{sp.390}.} Deutet auf das zu ergänzende Substantivum \pageref{sp.237}. Schwund des grammat. G. \pageref{sp.254}. \sed{Natürliches G. \pageref{sp.331}.} Zähigkeit, mit der es sich behauptet \pageref{sp.364}. Werth desselben \pageref{sp.391}.

\so{Geschlechtstrieb} des Menschen, an keine bestimmten Zeiten gebunden \pageref{sp.306}.

\so{Geschmack}, nationaler, und Stil \pageref{sp.105}–\pageref{sp.106}.

\so{Geschöpf} \pageref{sp.235}.

\so{Geschwätzigkeit} \pageref{sp.309}, \pageref{sp.472}.

\sed{\so{Gesenius}, Wörterbuch \pageref{sp.410}.}

\so{Gesetz} in der Sprache. Möglichkeit – Regel – G. \pageref{sp.385}–\pageref{sp.387}.

\so{Gesichtskreis}, geistiger \pageref{sp.325}.

\so{Gesittung}, nationale, und Sprache \pageref{sp.17}. Inwieweit sie eine Vermuthung für Sprachverwandtschaft begründen könne? \pageref{sp.148}.

\so{Gesprächigkeit} \pageref{sp.473}.

\sed{\so{Gesten}, als Verständigungsmittel \pageref{sp.67}.}

\so{Gewichtswesen} \pageref{sp.107}.

\sed{\so{Gewissen}, Sprachliches G. \pageref{sp.248}. Dessen Abstumpfung \pageref{sp.276}.}

\so{Gewohnheiten} in der Sprache \pageref{sp.39}, \pageref{sp.182}. G. werden zu Regeln \pageref{sp.382} flg.

\so{Ghat} \pageref{sp.160}.

\so{Gickel} – Kikeriki \pageref{sp.208}.

\so{Giljäkisch} \pageref{sp.147}.

\fed{{\textbar}479{\textbar}}

\so{Gleichartigkeit} der grammatischen Erscheinungen im Sinne der Einzelsprache \pageref{sp.90}.

\so{Gleichheit} als grammatische Kategorie: „wie“ \pageref{sp.103}.

\sed{\so{Gleichklanggefühl} \pageref{sp.227}.}

\so{Gleichmass}, lautliches, das etymologische Gefühl beeinflussend \pageref{sp.214}–\pageref{sp.216}.

\so{Gleichniss} s. \so{Vergleich}. Gl. vom verlorenen Sohne, als Probetext \pageref{sp.106}. \sed{Einfluss des Gl. auf den Bedeutungswandel der Wörter \pageref{sp.234}.}

\so{Gliederung} als Merkmal der menschlichen Sprache \pageref{sp.3}, \pageref{sp.5}, \sed{\pageref{sp.310},} \pageref{sp.346}. Gl. des Stoffes in der Rede \pageref{sp.324}. Gröbere oder feinere im Satze \sed{\pageref{sp.451}–\pageref{sp.446}.}\edins{{\textbar}{\textbar}\pageref{sp.451}–\pageref{sp.456}.}

\sed{\so{Glossarien}, \pageref{sp.111} flg.}

\so{Glottik}, s. \so{Sprachwissenschaft}.

\sed{\so{Goajiren} \pageref{sp.390}.}

\so{Gotisch} \sed{\pageref{sp.111}.} G. in der Krim, ausgestorben \pageref{sp.146}.

\inlineupdate{\so{Gottscheer}}{\so{Gottscher}} Mundart im Aussterben \pageref{sp.146}.

\so{Grade} der Sprachverwandtschaft, s. \so{Verwandtschaftsgrade}.

\so{Grammatik} = Lehre vom Sprachbaue \pageref{sp.81}. Einzelsprachliche \pageref{sp.81}–\pageref{sp.121}. Selbstschilderung. \inlineupdate{Selbstanalyse}{Selbsanalyse} \pageref{sp.82}–\pageref{sp.83}. Die beiden Systeme, das analytische und das synthetische \pageref{sp.84}–\pageref{sp.86}. Prolegomena \pageref{sp.86}–\pageref{sp.88}. Das Minimum einer Gr. \pageref{sp.385}. Allgemeine Gr. \pageref{sp.479}–\pageref{sp.482}.

\so{Grammatiken}, philosophische oder allgemeine \pageref{sp.10}–\pageref{sp.11}. Missbräuchliche Zugrundelegung der lateinischen \pageref{sp.25}, \pageref{sp.52}, \sed{\pageref{sp.91},} \pageref{sp.105}. Schulgrammatiken, deren Verstösse gegen System und Methode \pageref{sp.81}–\pageref{sp.82}. \sed{Keine Sprache ohne Gr. \pageref{sp.84}.} Gemischte Systeme \pageref{sp.91}–\pageref{sp.92}. \sed{{\textbar}{\textbar}499{\textbar}{\textbar}} Systematische – methodische \pageref{sp.109}–\pageref{sp.110}. Vollständige Gr. – Elementarbücher \pageref{sp.110}–\pageref{sp.113}. Kritische – didaktische \pageref{sp.113}–\pageref{sp.114}. Die Beispiele \pageref{sp.116}. Paradigmen und Formeln \pageref{sp.116}–\pageref{sp.119}. \sed{Verhältniss zum Wörterbuch \pageref{sp.122}–\pageref{sp.123}.} Neuere indogermanische, vernachlässigen die Syntax \pageref{sp.137}–\pageref{sp.138}. Vergleichende der indogermanischen Sprachen von Bopp, Schleicher und Brugmann \pageref{sp.170}–\pageref{sp.173}. Gr. als Grundlagen zur Beurtheilung der Sprachen \pageref{sp.406}, \pageref{sp.450}, \pageref{sp.471}.

\so{Grammatiker}. Arabische \pageref{sp.21}. Indische \pageref{sp.22}–\pageref{sp.23}. Japanische \pageref{sp.24}. Erfordernisse eines Gr. \pageref{sp.81}–\pageref{sp.82}. Sprache des Gr. eine Zufälligkeit \pageref{sp.88}, \pageref{sp.120}–\pageref{sp.121}.

\fed{\so{Gras} (französisch) – crassus \pageref{fp.223}.}

\so{Grasmücke} \pageref{sp.216}.

\so{Grasserie}, R. de la \pageref{sp.31}, Études de grammaire comparée \pageref{sp.481}.

\so{Grebo} \pageref{sp.150}, \pageref{sp.282}\sed{, \pageref{sp.326}.} Unterscheidung der Pronomina \pageref{sp.1}. und \pageref{sp.2}. Person durch die Betonung \pageref{sp.379}. Conjugation \pageref{sp.391}. \sed{Innere Flexion \pageref{sp.434}.} Genitiv und Adjectivum \pageref{sp.455}.

\so{Gregorio}, A. de, Cenni di glottologia bantu \pageref{sp.283}.

\so{Grenzen} der Sprachgemeinschaften \pageref{sp.56}–\pageref{sp.58}.

\so{Greve} (italienisch) – grave \pageref{sp.222}.

\so{Grézel}, Dictionnaire futunien-français \pageref{sp.463}.

\so{Griechen}, fehlende Neigung zu sprachwissenschaftlicher Arbeit \pageref{sp.20}. Entdeckung \inlineupdate{grammat.}{grammatischer} Kategorien \pageref{sp.20}, \pageref{sp.113}.

\so{Griechisch}, zur Bildung neuer Benennungen verwendet \pageref{sp.230}. Νόμος, πιττάκιον, διφθέρα in andere Sprachen übergegangen \pageref{sp.264}. Betonung \pageref{sp.379}.

\so{Grimm}, Jacob, Deutsche Grammatik \pageref{sp.27}, \pageref{sp.31}, \pageref{sp.122}. Jacob und Wilhelm, ihr Wirken \pageref{sp.173}. Urtheil über das Englische \pageref{sp.393}.

\inlineupdate{\so{Grönländisch}, suffigirend}{\so{Grönländisch}. Weibersprache \pageref{sp.248}. Einverleibende oder satzwortige Sprache \pageref{sp.342}. Suffigirend} \pageref{sp.349}. \sed{Genitiv und Adjectivum \pageref{sp.455}.}

\inlineupdate{\so{Grossstädte,}}{\so{Grosstädte,}} Brutstätten neuer Ausdrücke \pageref{sp.45}.

\sed{\so{Grotefend}, G. Fr. Keilschriftentzifferung \pageref{sp.28}.}

\sed{\so{Grouth}, L., Grammar of the Zulu \pageref{sp.421}.}

\so{„Grün“} \pageref{sp.232}.

\so{Grund}. Adverbialsätze des Gr. \pageref{sp.104}.

\so{Grusinisch}, s. \so{Georgisch}.

\so{Guarani} (= Tupi). Formenbildung \pageref{sp.328}, \pageref{sp.423}.

\so{Guineaküste}. Sprachen \pageref{sp.150}.

\so{Gutturale} \pageref{sp.36}, \sed{\pageref{sp.148},} \pageref{sp.197}.

\so{Guyard} über die sumero-akkadische Sprache \pageref{sp.389}.

\so{Gyarmathy}, S., seine grammatische Sprachvergleichung \pageref{sp.26}.

\so{Gyarung} \pageref{sp.157}.

\so{Gymnastik} der Sprachorgane \edins{\pageref{sp.35}–\pageref{sp.38}.{\textbar}{\textbar}\textsuperscript{1891:} \pageref{fp.35}–\pageref{fp.38}.{\textbar}\textsuperscript{1901:} \pageref{sp.35}–\pageref{sp.88}.}

\subsection*{H.}\label{reg.H}\pdfbookmark[1]{H.}{reg.H}

\sed{H aspiré \pageref{sp.196}.}

\so{Halévy} über die sumero-akkadische Sprache \pageref{sp.389}.

\inlineupdate{\so{Halmahera}}{\so{Halmaheras}} \pageref{sp.246}.

\so{Hamitische Sprachen} \pageref{sp.142}. Verwandtschaft mit den semitischen \pageref{sp.160}–\fed{{\textbar}480{\textbar}}\pageref{sp.162}, \pageref{sp.281}–\pageref{sp.282}. \sed{Lautschwankungen \pageref{sp.300}.} \inlineupdate{Übereinstimmung}{Uebereinstimmung} der Conjugationsformen für die \pageref{sp.2}. pers. sing. und die \pageref{sp.3}. pers. fem. sing. \pageref{sp.307}.

\so{Hamito-semitische Sprachen}. Ihre Verbreitung \pageref{sp.142}–\pageref{sp.143}. Grammatisches Geschlecht \pageref{sp.160}, \pageref{sp.254}. Zum Verwandtschaftsnachweise \pageref{sp.160}–\pageref{sp.162}.

\so{Han-iü}, dessen Ansicht vom Ursprunge der Sprache \pageref{sp.19}.

\so{Hand}. Wörter dafür in den indogermanischen Sprachen \pageref{sp.153}. Die H. und das Sprachvermögen \pageref{sp.305}.

\so{Handbücher} – Lehrbücher, grammatische \pageref{sp.111}.

\so{Hanxleden}, seine Sanskritgrammatik \pageref{sp.26}.

\sed{\so{Harari} \pageref{sp.282}.}

\so{Hardeland}, A. \pageref{sp.193}, \pageref{sp.411}.

\sed{\so{Harris}, J. \pageref{sp.15}.}

van \so{Hasselt} \pageref{sp.444}.

\so{Hauptdialekte}, s. \so{Dialekte}.

\so{Hausa}. Lautwesen \pageref{sp.34}. Anklänge an die hamitischen Sprachen \pageref{sp.161}, \pageref{sp.282}. Vocalismus der Personalpronomina \pageref{sp.408}.

\so{Havestadt}. Chilidugu \pageref{sp.194}.

\so{Hawaiisch}. Zweierlei Genitiv \pageref{sp.463}.

\so{Hayti}, Negerstaat, französische Sprache \sed{\pageref{sp.147},} \pageref{sp.293}.

\sed{\so{Hebräische} Grammatiker \pageref{sp.22}.}

\sed{\so{Hegel}. Stilistische Ungeheuerlichkeiten \pageref{sp.275}, \pageref{sp.456}, \pageref{sp.469}.}

\so{Hélas} (französ.) \pageref{sp.360}.

\sed{van \so{Helmont} \pageref{sp.251}.}

\sed{\so{Henry}, V. Arte y Vocab. \pageref{sp.248}.}

\fed{\so{Herberge} \pageref{fp.263}.}

\so{Herodot} \pageref{sp.20}.

\so{Hervás}, Lor., Catálogo de las lenguas \pageref{sp.27}.

\sed{\so{Heumann}, E. \pageref{sp.390}.}

\sed{\so{Hiatus} führt zur Entähnlichung der Laute \pageref{sp.200}.}

\so{Hidatsa}-Sprache. Unsichere Articulation \pageref{sp.194}.

\so{Hieroglyphen}, ägyptische, stilisiren die Bilder \pageref{sp.129}. Ihr System \pageref{sp.130}–\pageref{sp.131}.

\so{Himâlaya} \pageref{sp.261}.

\sed{{\textbar}{\textbar}500{\textbar}{\textbar}}

\so{Hindu}, arische, ihre Rasse \pageref{sp.147}.

\inlineupdate{\so{Hindustani}.}{\so{Hindustanisch}.} Arabische Einflüsse \pageref{sp.271}.

\so{Hinterindien}. \pageref{sp.261}.

\sed{\so{Historisches} Element in der Sprachwissenschaft \pageref{sp.14}.}

\sed{\so{Historische} Sprachforschung, deren Aufgabe \pageref{sp.136} flg.}

\so{Hm}! – lat. hem, franz. hein, heim \pageref{sp.208}–\pageref{sp.209}.

\so{Hochdeutsch} gegenüber den Dialekten und dem Plattdeutschen \pageref{sp.55}.

\so{Hodgson}, B. H. \pageref{sp.69}.

\so{Höflichkeit}. Ausdrücke der H. \pageref{sp.107}. \sed{Als sprachgeschichtliche Macht \pageref{sp.248}.} Ihr Wesen und ihre Wirkungen auf die Sprachform \pageref{sp.474}–\pageref{sp.475}.

\sed{\so{Hold} \pageref{sp.236}.}

\so{Hollander}, J. J. de, \pageref{sp.193}.

\so{Holländisch}. Gutturale Lautbildung \pageref{sp.34}. Besondere Sprache \pageref{sp.55}, \pageref{sp.57}. Verlust des pron. \pageref{sp.2}. pers. sing. \pageref{sp.152}. H. und Fränkisch \pageref{sp.159}. Bilabiales w, \pageref{sp.188}. \fed{tachtig = achtzig \pageref{fp.207}. beschuit,} schandaal \pageref{sp.217}. Fremdwörter durch Nachbildungen ersetzt \pageref{sp.262}–\pageref{sp.263}.

\so{Homophonen} als Wirkungen des Lautverschliffes \pageref{sp.178}. Sie können zu verdeutlichenden Ausdrücken veranlassen \pageref{sp.243}.

\so{Horden} \pageref{sp.307}.

\so{Horizont}, geistiger \pageref{sp.325}.

\sed{\so{Hörnle} \pageref{sp.241}.}

\so{Horpa} \pageref{sp.157}.

\inlineupdate{\so{Hottentottisch.}}{\so{Hottentotisch.}} Schnalzlaute \pageref{sp.34}, \pageref{sp.269}. Grammatisches Geschlecht \pageref{sp.150}, \pageref{sp.374}. Lepsius’ Hypothese \pageref{sp.282}. \sed{Suffigirender Bau \pageref{sp.349}.}

\fed{\so{Hovelacque}, A., La linguistique \pageref{fp.29}.}

\sed{\so{Howell} Grammar Arabic \pageref{sp.409}.}

H\c{r}d (sanskrit) – cor, καρδία, Herz \pageref{sp.186}, \pageref{sp.217}.

\sed{\so{Huazteken} \pageref{sp.389}.}

\sed{\so{Hugo}, Victor, seine Sprache \pageref{sp.275}.}

\so{Hulda} – Frau Holle \pageref{sp.230}.

\so{Huldvoll} \pageref{sp.236}.

\so{Hülfswörter} als Formenmittel \pageref{sp.347}.

\so{Humanismus}, Studium der classischen Sprachen \pageref{sp.25}.

\so{Humboldt}, Wilhelm von, als Sprachphilosoph \sed{\pageref{sp.16},} \pageref{sp.28}, \pageref{sp.31}. \sed{Analogie als sprachgeschichtliche Macht \pageref{sp.210}.} \inlineupdate{Über}{Ueber} „symbolische Bezeichnung“ \pageref{sp.221}. \inlineupdate{Über}{Ueber} die innere Sprachform \pageref{sp.327}–\pageref{sp.330}. Einverleibender Sprachbau \pageref{sp.354}–\pageref{sp.359}. Werthsbestimmung der Sprachen \pageref{sp.388} flg., \pageref{sp.426}. \sed{Beurtheilung der Sprachen \pageref{sp.394}. Kawi Sprache \pageref{sp.411}, \pageref{sp.413}. Nominales und verbales Prädikat \pageref{sp.439}.} Ueber den Dualis \pageref{sp.481}.

\so{Hundert}. Die Endung nicht ursprünglich \pageref{sp.167}.

\so{Hürkanisch} \pageref{sp.423}.

\so{Hyperboräer} \pageref{sp.177}.

\subsection*{I.}\label{reg.I}\pdfbookmark[1]{I.}{reg.I}

\fed{I als zweiter Laut eines Diphthongs, ein s ersetzend.}

\so{Ibo} \pageref{sp.150}, \pageref{sp.282}.

\so{Ideen}, angeborene \pageref{sp.381}.

\so{Ideenassociationen} \pageref{sp.43}–\pageref{sp.44}.

\so{Illativus} \pageref{sp.114}.

\so{Iloca}. Bildsamkeit \pageref{sp.349}.

\sed{\so{Imperfectum} futuri \pageref{sp.253}.}

\so{Inclusivität} s. \so{Auch}, \so{Noch}.

\so{Incorporation} \pageref{sp.354}–\pageref{sp.359}, \pageref{sp.459}.

\so{Inder}, als Grammatiker \pageref{sp.22}–\pageref{sp.23}.

\fed{{\textbar}481{\textbar}}

\so{Indianer}, amerikanische, ihre Pictographien \pageref{sp.128}.

\so{Indianersprachen Amerikas}. Lautwesen \pageref{sp.34}. Geistige und leibliche Verwandtschaft \pageref{sp.150}–\pageref{sp.151}. \sed{Unsichere Artikulation \pageref{sp.194}.} Agglutination, Polysynthetismus \pageref{sp.257}. Gleiche Denkgewohnheiten \pageref{sp.293}. \inlineupdate{Incorporation}{Incorporationen} \pageref{sp.354}–\pageref{sp.359}\sed{, \pageref{sp.423} flg.} Charakter der Sprachen und der Völker \pageref{sp.423}–\pageref{sp.425}.

\so{Indianisten} \pageref{sp.173}.

\so{Indicativ} \pageref{sp.310}. I. statt des Imperativs \pageref{sp.473}.

\sed{\so{Indisch}, verwandt mit Griechisch \pageref{sp.173}.}

\so{Indische Schriften} \pageref{sp.129}.

\so{Indisch-iranische Sprachen}. \inlineupdate{Ähnlichkeit}{Aehnlichkeit} in der Entwickelung der Gutturale mit den litu-slavischen \pageref{sp.159}, \pageref{sp.163}.

\so{Individualsprachen}, ihre Umgrenzung \pageref{sp.55}–\pageref{sp.56}. Gegenseitige Beeinflussung \pageref{sp.273}–\pageref{sp.277}.

\so{Indochinesische Sprachen}. \inlineupdate{Mannichfaltigkeit}{Mannigfaltigkeit} im Baue \pageref{sp.149}, \pageref{sp.257}, \pageref{sp.293}. Wörter für Acht und Hundert \pageref{sp.157}. Desgl. für Ich, Fünf, Fisch, – Du, zwei, Ohr, – Auge, Feuer \pageref{sp.158}. Unregelmässigkeiten in den Zahlwörtern \pageref{sp.226}. Einsilbigkeit und Isolirung nicht ursprünglich \pageref{sp.257}. Schwierigkeit der Beurtheilung \pageref{sp.426}.

\so{Indogermanischer Sprachstamm}. Seine Verbreitung \pageref{sp.142}. Grammatisches Geschlecht \pageref{sp.150}. Verfrühte Vergleichungen mit dem semitischen \pageref{sp.162}. Stammbaum und Wellentheorie \pageref{sp.163}–\pageref{sp.165}. \sed{Lautgesetze, unsichere Articulation \pageref{sp.195}.} Sandhi \pageref{sp.198}, \pageref{sp.401}–\pageref{sp.402}. Schwinden des Dualis \pageref{sp.253}. Flexion \pageref{sp.256}–\pageref{sp.257}, \sed{\pageref{sp.251}}\edins{{\textbar}{\textbar}\pageref{sp.351}}–\pageref{sp.354}, \pageref{sp.398}–\pageref{sp.401}, \pageref{sp.435}–\pageref{sp.436}. Mundartliche Spaltungen in der Ursprache \pageref{sp.284}, \pageref{sp.285}. Schwierigkeit, das Altgemeinsame zu erkennen \pageref{sp.293}–\pageref{sp.294}. Defectivsystem \pageref{sp.352}. Vocalsteigerung und -schwächung \pageref{sp.352}. Pronominalsuffixe in der Conjugation \pageref{sp.372}. \sed{{\textbar}{\textbar}501{\textbar}{\textbar}} Waren diese possessiv oder prädicativ? \pageref{sp.391}. Etymologie, deren Bedeutsamkeit \pageref{sp.395}–\pageref{sp.398}. Etymologisches Bewusstsein \pageref{sp.403}–\pageref{sp.404}. \sed{Wandel im Stammvocalismus \pageref{sp.434}.} Einfluss der Mythologie \pageref{sp.447}. Stellung des Attributes, Composita \pageref{sp.457}. Die Casus \pageref{sp.462}.

\so{Indogermanistik} \pageref{sp.29} flg. Ihre Wichtigkeit für die allgem. Sprachwissenschaft \pageref{sp.52}. Die Schreibweise in ihren Lehr- und Handbüchern \pageref{sp.108}. Ihre Geschichte vorbildlich \pageref{sp.137}. Ob sie alle Theile der inneren Sprachgeschichte erklären kann? \pageref{sp.170}. Zunehmende Zurückhaltung den etymologischen Fragen gegenüber \pageref{sp.179}–\pageref{sp.180}. \sed{I. und Lautwandel \pageref{sp.185}.} Einführung der Analogie \pageref{sp.210}. Berücksichtigung möglicher dialektischer Nebenformen \pageref{sp.285}.

\so{Induction}, grammatische \pageref{sp.89}, \pageref{sp.91}–\pageref{sp.92}. Schematismus \pageref{sp.93}. Zur Entdeckung der sprachgeschichtlichen Gesetze \pageref{sp.170}.

\so{Inessivus} \pageref{sp.114}.

\so{Infinitiv}, historischer im Latein. \pageref{sp.473}.

\so{Infixe} \pageref{sp.348}.

\so{Innuit} s. \so{Eskimo}.

\so{Instrumentalis} \pageref{sp.102}\sed{, \pageref{sp.241}}.

\so{Intensität} der Stimme \pageref{sp.314}.

\so{Interjectionen} \pageref{sp.321}.

\so{Inversion} \pageref{sp.101}–\pageref{sp.102}\sed{, \pageref{sp.321}.}

\sed{\so{Irânisch} \pageref{sp.159}.}

\so{Irokesische Sprachen} \sed{\pageref{sp.390},} \pageref{sp.423}.

\so{Ironie}. Einfluss auf den Bedeutungswandel \pageref{sp.244}.

\so{Islâm} und arabische Philologie\sed{, Wirkung auf die Sprachenkunde} \pageref{sp.21}. Culturausgleichende Wirkung \pageref{sp.148}. Verbreitung arabischer Fremdwörter \pageref{sp.231}.

„\so{Ismen}“ \pageref{sp.215}.

\sed{\so{Isolation} und Polysynthetismus \pageref{sp.257}.}

\so{Isolirende Sprachen} \sed{\pageref{sp.257}}. Etymologie und Morphologie in ihnen \pageref{sp.123}. Chinesisch, barmanisch und siamesisch in Rücksicht auf attributive und prädicative Anschauung \sed{\pageref{sp.346},} \pageref{sp.456}.

\so{Isolirte Sprachen} \inlineupdate{als}{\pageref{sp.177} flg. A}ls Gegenstände der einzelsprachlichen Forschung \pageref{sp.61}. Aufzählung einiger \pageref{sp.146}–\pageref{sp.147}. \inlineupdate{Schwierigkeit,}{Schwierigkeit} sie auf ihren Werth zu beurtheilen \pageref{sp.426}.

\so{Italienisch}. Lautwesen \pageref{sp.34}. \fed{Auslautendes s in i verwandelt \pageref{sp.201}.} eglino, elleno \pageref{sp.214}. Greve – grave \pageref{sp.222}. Einfluss auf die kaufmännische Sprache \pageref{sp.265}\sed{, \pageref{sp.289}}. Endung der \pageref{sp.3}. Pers. Pluralis \pageref{sp.435}. Diminutiva, Augmentativa u.~s.~w. \pageref{sp.445}.

\so{Itelmenisch} \pageref{sp.147}.

\so{Iterativa} \pageref{sp.445}.

\fed{{\textbar}482{\textbar}}

\subsection*{J.}\label{reg.J}\pdfbookmark[1]{J.}{reg.J}

\so{Ja} und Nein \pageref{sp.103}. Ja und jo \pageref{sp.233}, \pageref{sp.285}.

\so{Jakutisch} \pageref{sp.271}. \sed{Agglutinirende oder scheinwortige Sprache \pageref{sp.342}.} Vocalharmonie \pageref{sp.402}.

\so{Japaner}, ihre Sprachforschung \pageref{sp.24}. Geistige Begabung \pageref{sp.389}.

\so{Japanisch}. Lautwesen der alten Sprache \pageref{sp.34}. Einfluss der Etiquette auf die Grammatik \pageref{sp.95}, \pageref{sp.246}. \inlineupdate{Sylbenschrift}{Silbenschrift} \pageref{sp.129}. Isolirte Sprache \pageref{sp.147}. Lautverschiebung \pageref{sp.190}–\pageref{sp.191}. J + a wird e \pageref{sp.199}. \sed{Höflichkeit in der Sprache \pageref{sp.246}.} Chinesische Lehnwörter \pageref{sp.266}. Chinesischer Einfluss \pageref{sp.271}, \pageref{sp.428}. Vergleichung mit dem Mandschu \pageref{sp.289}–\pageref{sp.290}. Empfindungslaute und grammatische Hülfswörter \pageref{sp.347}. Composita \fed{\pageref{fp.341}; deren Betonung} \sed{\pageref{sp.359}. Adjektivische Conjugation \pageref{sp.384}, \pageref{sp.440}. Betonung der Composita,} \pageref{sp.379}. Vocalharmonie \pageref{sp.402}. Vocalismus der Zahlwörter \pageref{sp.408}. Sociale Modalität, Gebrauch des Passivums und Causativums \pageref{sp.474}.

\so{Japhetiten} \pageref{sp.162}, \pageref{sp.282}.

\so{Jargon} \pageref{sp.126}–\pageref{sp.126}, \pageref{sp.289}.

\so{Javanisch}: Kråmå, mådyå, ngoko \pageref{sp.246}. Sociale Modalität \pageref{sp.475}.

\so{Jemine} \pageref{sp.322}.

\so{Jihvā} (sanskrit) – Zunge \pageref{sp.185}, \pageref{sp.217}.

\so{Jodirung} s. \so{Palatalisirung}.

\so{Juden}, ihre philologischen Arbeiten \pageref{sp.22}. Dauerhaftigkeit der Rassemerkmale \pageref{sp.147}. \sed{Beibehalten vererbter Lautgewohnheiten \pageref{sp.270}.}

\so{Junggrammatiker}. Die „falschen Analogien“ \pageref{sp.137}. Unterschied zwischen den J. und ihren Gegnern \pageref{sp.181}.

\so{Jünglingsalter} der Völker und Sprachen \pageref{sp.399}.

\subsection*{K. }\label{reg.K}\pdfbookmark[1]{K.}{reg.K}

K (deutsch) = lateinisch c in Lehnwörtern \pageref{sp.186}.

\so{Kabakada} und Duke of York-Sprache \pageref{sp.165}–\pageref{sp.166}. \sed{Hülfswörter \pageref{sp.453}.} Relativwörter na, a \pageref{sp.457}.

\so{Kabylisch} \pageref{sp.160}\sed{, \pageref{sp.299}, \pageref{sp.301}. Femininbildung \pageref{sp.240}. Lautschwankungen \pageref{sp.299}.}

\so{Kaffernsprachen}. Schnalzlaute \pageref{sp.34}, \sed{\pageref{sp.199},} \pageref{sp.269}.

\so{Kaffernvölker}. Merkzeichen für die Boten \pageref{sp.127}.

\so{Kallispel} = \so{Selish}. Lautwesen \pageref{sp.34}.

\inlineupdate{\so{Kalmükisch}}{\so{Kalmückisch}} \sed{\pageref{sp.349}.} Schrift \pageref{sp.129}, \sed{\pageref{sp.417}}. Vocalharmonie \pageref{sp.403}.

\sed{{\textbar}{\textbar}502{\textbar}{\textbar}}

\so{Kampf} \inlineupdate{\so{um’s}}{\so{ums}} \so{Dasein} in der Sprachgeschichte \pageref{sp.17}, \sed{\pageref{sp.143},} \pageref{sp.261}–\pageref{sp.262}.

\so{Kamtschatka}, s. \so{Itelmenisch}.

\so{Kannadi}, s. \so{Canaresisch}.

\so{Kanuri} \pageref{sp.282}.

κάπρος – Eber \pageref{sp.224}.

\so{Karen}-Sprache. Anlaute \pageref{sp.201}.

\so{Karl} – Kerl \pageref{sp.230}.

\so{Karnatha} s. \so{Canaresisch}.

\so{Karnickel} – caniculus \pageref{sp.435}.

\inlineupdate{\so{Kasikumükisch}}{\so{Kasikumückisch.}} \inlineupdate{Reichthum}{Reichtum} an örtlichen Casus \pageref{sp.462}.

\so{Kassia}. Grammatisches Geschlecht \pageref{sp.390}. Vocalismus der Personalpronomina \pageref{sp.409}.

\so{Katechismen} als Texte zur Spracherlernung \pageref{sp.73}–\pageref{sp.75}.

\so{Kategorien}, sprachliche, deren Aneignung \pageref{sp.63}–\pageref{sp.64}. Deren Benennung, Terminologie \pageref{sp.114}–\pageref{sp.116}. Grammatische, deren Schwund und Entstehen neuer \pageref{sp.253}–\pageref{sp.255}. Grammatische Redetheile \pageref{sp.381}–\pageref{sp.385}. Schwierigkeit ihrer Beurtheilung, Missgriffe \pageref{sp.405}–\pageref{sp.408}. K. der Gedankenverbindung, allen Völkern gemeinsame \pageref{sp.464}. Logische Modalität \pageref{sp.470}–\pageref{sp.472}.

\so{Katschari} \pageref{sp.257}.

\sed{\so{Kaufmännischer Stil} \pageref{sp.184}.}

\so{Kaukasische Sprachen}. Lautwesen \pageref{sp.34}, \pageref{sp.197}. Nördlicher und südlicher Stamm \pageref{sp.142}. Defectivsystem \pageref{sp.352}. Schwierigkeit der Beurtheilung \pageref{sp.426}. \inlineupdate{Örtliche}{Oertliche} Casus \pageref{sp.463}.

\so{Kaukasus}. Völker des K. vielleicht Nachkommen alter kleinasiatischer Völker? \pageref{sp.146}.

\so{Keilschriften}. \sed{Erste Untersuchungen \pageref{sp.28}.} Sprachen, die uns in solchen erhalten sind (Altpersisch, Assyrisch, Medisch, Sumerisch u.~s.~w.) \pageref{sp.114}. Stilisirung \pageref{sp.129}. Hieroglyphisches System \pageref{sp.130}–\pageref{sp.131}.

–\so{keit} –heit \pageref{sp.216}.

\so{Kelten, keltische Sprachen}. Einengung ihres Gebietes \pageref{sp.146}. \sed{Keine Agglutination \pageref{sp.256}.}

\sed{\so{Kerbholz} \pageref{sp.128}.}

\sed{\so{Kerenzer} Mundart \pageref{sp.33}, \pageref{sp.298}.}

\so{Kerl} – Karl \pageref{sp.230}.

\fed{{\textbar}483{\textbar}}

\sed{\so{Kern}, H. \pageref{sp.245}.}

\so{Ketschua} \sed{\pageref{sp.389}.} Gutturale, Abneigung gegen Consonantenhäufungen \pageref{sp.194}. Lautsymbolisch anmuthende Wörter \pageref{sp.219}. \sed{Agglutinirend-suffigirend \pageref{sp.423}.}

\so{Keuschheit}. Ihr Einfluss auf den Sprachgebrauch \pageref{sp.248}–\pageref{sp.249}.

\so{Khamti} \pageref{sp.149}\sed{, \pageref{sp.257}, \pageref{sp.426}}.

\so{Khmêr} s. \so{Cambodjanisch}.

\so{Kietze} = Kätzchen \pageref{sp.324}.

\so{Kikeriki} – Gickel \pageref{sp.208}.

\so{Kinder}. Spracherlernung und Sprachschöpfungen derselben \pageref{sp.65}–\pageref{sp.67}. Aufnahme ihrer Naturlaute in den Sprachschatz \pageref{sp.154}. Ihre Hülfsbedürftigkeit fördert das Familienleben \pageref{sp.306}. \sed{Sprachspielerei und Sprachvermögen \pageref{sp.311}.} Einfluss auf die Sprache der Erwachsenen \pageref{sp.445}.

\so{Kindersprache}. Ihre Eigenthümlichkeiten und ihr Einfluss auf die Sprache der Erwachsenen \pageref{sp.277}–\pageref{sp.278}. \sed{Isolirende Sprachform \pageref{sp.346}.}

\so{Kindertausch} \pageref{sp.70}.

\so{Kirânti-Sprachen} \pageref{sp.149}, \pageref{sp.257}.

\so{Kirchenslavisch}, s. \so{Altslavisch}.

\sed{\so{Kitan} \pageref{sp.131}.}

\sed{\so{Klamath-Sprache}. Unsichere Articulation \pageref{sp.194}.}

\sed{\so{Klangähnlichkeit} und Lautgesetze \pageref{sp.223}.}

\so{Klaproth}, Julius \pageref{sp.31}.

\so{Kleinasiatische Sprachen}, alte \pageref{sp.114}.

\so{Kleinasien}. Alte Völker \pageref{sp.146}.

\sed{\so{Kleinschmidt}, S., Grammatik \pageref{sp.248}, \pageref{sp.350}.}

\sed{\so{Kluge}, Etymol. Wtb. \pageref{sp.190}, \pageref{sp.236}.}

\so{Knoten}, \so{Knotenschnüre}, als Merkzeichen, Surrogate der Schrift \pageref{sp.128}.

\sed{\so{Koch}, griechische Grammatik \pageref{sp.112}.}

\sed{\so{Köhler}, Aug. \pageref{sp.270}.}

\so{Kolarische Sprachen} \pageref{sp.142}. Verwandtschaft mit den australischen? \pageref{sp.147}, \pageref{sp.152}. Haben keinen Wortaccent \pageref{sp.148}\sed{, \pageref{sp.150}}. Einfluss der Sitte auf die Syntax \pageref{sp.249}. Prae- und suffigirend \pageref{sp.348}. Schwierigkeit der Beurtheilung \pageref{sp.426}.

\so{Kolaro-australischer Sprachstamm?} \pageref{sp.281}.

\so{Kolh}. Merkwürdiger Gebrauch des Duals \pageref{sp.249}. Accusativ \pageref{sp.358}.

\so{Kondschara} \pageref{sp.282}.

\so{Kongo-kaffrische Sprachen} s. \so{Bantusprachen}.

\so{Können}. Eine Sprache k. oder beherrschen, nothwendig um sie zu begreifen \pageref{sp.61}, \pageref{sp.82}. Was es besagt \pageref{sp.84}, \pageref{sp.89}. Als grammatische Kategorie \pageref{sp.103}.

\so{Kopf}: Haupt, caput, capo, chef, – testa, tête \pageref{sp.232}.

\so{Koptisch} ausgestorben, nur noch Kirchensprache \pageref{sp.147}. Hamitische Sprache \pageref{sp.160}\sed{, \pageref{sp.282}}. Griechische Einflüsse \pageref{sp.272}. \sed{Präfigirende Conjugation \pageref{sp.240}.} Vocalwandel in der Formenbildung \pageref{sp.354}. \sed{Innere Flexion \pageref{sp.434}. Hilfswörter \pageref{sp.452}.}

\sed{{\textbar}{\textbar}503{\textbar}{\textbar}}

\so{Koreanisch}. Scheinbare Unregelmässigkeiten in der Declination und Conjugation \pageref{sp.87}. Einfluss der Etiquette auf die Grammatik \pageref{sp.95}, \pageref{sp.246}. \sed{Orthographie \pageref{sp.134}.} Schrift \pageref{sp.129}. \sed{Frag- und Verneinungswörter \pageref{sp.244}.} Chinesische Lehnwörter \pageref{sp.266}. \sed{Chinesische Schriftsprache \pageref{sp.260}.} Chinesischer Einfluss \pageref{sp.271}. Vergleichung mit Chinesisch, Mandschu, Aino \pageref{sp.290}–\pageref{sp.291}. \sed{Fünf Declinationen \pageref{sp.350}. Adjectivische Conjugation \pageref{sp.384}.}

\so{Korjäkisch} \pageref{sp.147}.

\so{Körpertheile}, deren Namen in übertragener Anwendung \pageref{sp.40}–\pageref{sp.41}.

\so{Kosenamen} \pageref{sp.277}–\pageref{sp.278}.

\so{Kottisch} \pageref{sp.383}.

\so{Kräfte} der Sprachgeschichte, erhaltende \pageref{sp.46}. Vergl. \so{Mächte}.

\so{Kraftaufwand} einer Sprache zum Ausdrucke eines Gedankens scheint sich im Wesentlichen gleich zu bleiben \pageref{sp.243}.

\so{Kraftersparniss}, Bequemlichkeit \pageref{sp.182}–\pageref{sp.185}.

\so{Kral} (slavisch) – Karl \pageref{sp.230}.

\so{Kretzscham}, \so{Kretzschmar} \pageref{sp.265}.

\so{Kri} \pageref{sp.163}. \sed{Höflichkeitsformen gegenüber der angeredeten Person \pageref{sp.246}.} Incorporation \pageref{sp.358}. Der Conjunctiv \pageref{sp.359}. \sed{Wandel im Vocalismus \pageref{sp.391}. Suffixe \pageref{sp.424}.}

\so{Krim}. Gotische Gemeinden \pageref{sp.146}.

\sed{\so{Kritische Grammatiken} \pageref{sp.113}.}

\so{Kru} \pageref{sp.150}, \pageref{sp.282}. Unterscheidung der Pronomina \pageref{sp.1}. und \pageref{sp.2}. Person durch die Betonung \pageref{sp.379}.

\so{Kuhn}, Adalbert, Zur ältesten Geschichte der indogermanischen Völker \pageref{sp.294}.

\sed{\so{Kuhn}, Ernst \pageref{sp.150}, \pageref{sp.281}.}

\so{Kühner}, Raphael, lateinische und griechische Grammatiken \pageref{sp.111}, \pageref{sp.112}.

\so{Kuki-Sprachen} \pageref{sp.149}, \pageref{sp.257}. Demonstrativpronomina \pageref{sp.408}.

\so{Kukuk} – Gauch \pageref{sp.208}.

\so{Kumanisch}, ausgestorben \pageref{sp.146}.

\so{Kunama} \pageref{sp.160}. Die Personalpronomina \pageref{sp.379}, \pageref{sp.408}.

\so{Kunstschriften} \pageref{sp.130}.

\so{Kürzungen} \pageref{sp.101}. K. der Composita \pageref{sp.235}–\pageref{sp.236}. 

\so{Kuschitische} Sprachen \pageref{sp.282}.

\subsection*{L.}\label{reg.L}\pdfbookmark[1]{L.}{reg.L}

L, tönendes \pageref{sp.186}. Dickes, dessen Verbreitung in den slavischen, türkischen und mongolischen Sprachen u.~s.~w. \pageref{sp.269}.

\fed{{\textbar}484{\textbar}}

\so{Labialisirung} der Vocale \pageref{sp.37}.

\so{Labio-Dentale} \pageref{sp.36}.

\sed{\so{Lacombe} \pageref{sp.424}.}

\sed{\so{Lagarde}, Paul de \pageref{sp.160}, \pageref{sp.411}.}

\sed{\so{Landsknechtsprache} \pageref{sp.288}.}

\so{Langeweile}, Spieltrieb und Sprachschöpfung \pageref{sp.308} flg.

\so{Lao} \pageref{sp.149}\sed{, \pageref{sp.257}, \pageref{sp.426}}.

\so{Lappisch}. Declination und Conjugation \pageref{sp.350}. \sed{Keine Vocalharmonie \pageref{sp.350}, \pageref{sp.351}.} Der Nominativ \pageref{sp.391}.

\so{Lapsus} linguae, calami, mentis \pageref{sp.43}.

\so{Lateinische Sprache}. Vorherrschen von i und s \pageref{sp.34}. Formen der erzählenden Rede \pageref{sp.99}. Nur schwache Spuren der Vocalabstufung \pageref{sp.149}. Entartung in der Lingua rustica \pageref{sp.182}–\pageref{sp.183}. \sed{Lautwandel \pageref{sp.195}.} Die Metallnamen Neutra \pageref{sp.237}. Imperfectum auf –bam, Futurum auf –bo \pageref{sp.241}. Verbale Composita \pageref{sp.243}. Carmen, germen, terminus \pageref{sp.292}. –mini (amamini) \pageref{sp.384}. Duodeviginti, undeviginti \pageref{sp.401}. Tempora der Vergangenheit \pageref{sp.409}. Anschaulich, aber ungemüthlich \pageref{sp.473}. Sociale Modalität wenig vertreten \pageref{sp.474}.

\so{Laterallaute} \pageref{sp.187}.

\so{Laut}, Articulation \pageref{sp.4} flg. Nachahmung fremder, Uebung darin \pageref{sp.33}–\pageref{sp.39}. Verschiedene Laute im Sinne der Phonetik und in jenem der Sprachwissenschaft \pageref{sp.33}. Einfache und Doppellaute im Sinne der Sprachwissenschaft \pageref{sp.135}. \sed{Laut und Sinn \pageref{sp.220}. L. als ständige Symbole im Sprachvermögen \pageref{sp.310}.} Laute in der menschlichen Ursprache \pageref{sp.313}–\pageref{sp.315}.

\sed{\so{Lautanalyse}, Voraussetzung für die Buchstabenschrift \pageref{sp.131}.}

\sed{\so{Lautbilder}, unsichere \pageref{sp.300}.}

\sed{\so{Lautbildung}, Einflüsse auf dieselbe \pageref{sp.187}.}

\sed{\so{Lautformen}. Die ältesten für die Sprachvergleichung wichtig \pageref{sp.156}–\pageref{sp.157}.}

\so{Lautgefühl}. Grade seiner Empfindlichkeit \pageref{sp.188}.

\so{Lautgesetze} und Ausnahmen \pageref{sp.137}, \pageref{sp.185}–\pageref{sp.196}. Bedeutsamkeit in psychologischer Hinsicht \pageref{sp.401}–\pageref{sp.403}.

\so{Lautkörper} = Wörter und lautliche Formenelemente \pageref{sp.64}. \sed{Deren älteste Gestalt \pageref{sp.255}.}

\so{Lautlehre}. Deren Platz in der Grammatik \pageref{sp.87}.

\so{Lautphysiologie}, s. \so{Phonetik}.

\sed{\so{Lautschwund} \pageref{sp.207}.}

\so{Lautsprache} \pageref{sp.4} flg.

\so{Lautsymbolik} im Sprachgefühle \pageref{sp.124}, \pageref{sp.218}–\pageref{sp.225}\sed{, \pageref{sp.379}.}

\sed{\so{Lautunterscheidung} als phonetische Schulung \pageref{sp.35}–\pageref{sp.37}.}

{\textbar}{\textbar}504{\textbar}{\textbar}

\sed{\so{Lautverflüchtigung} \pageref{sp.207}.}

\so{Lautvergleichung} bei der Sprachvergleichung unerlässlich \pageref{sp.158}.

\sed{\so{Lautvermischung} \pageref{sp.301}.}

\so{Lautverschiebung}, \inlineupdate{hochdeutsche}{deren Entstehung \pageref{sp.37}. Hochdeutsche} \pageref{sp.159}. Wie L. geschieht \pageref{sp.187}–\pageref{sp.188}. Allmähliches Umsichgreifen, Stockungen \pageref{sp.190}–\pageref{sp.191}. \sed{Beispiele dafür \pageref{sp.193}.}

\sed{\so{Lautverschiebungsregel} \pageref{sp.191}.}

\sed{\so{Lautvertheilung} in der Sprache \pageref{sp.34}.}

\sed{\so{Lautvertretung}, scheinbare \pageref{sp.290}.}

\sed{\so{Lautvorstellung} und Sachvorstellung \pageref{sp.297}–\pageref{sp.301}.}

\sed{\so{Lautverwischung} \pageref{sp.301}.}

\so{Lautwandel} \inlineupdate{durch}{\pageref{sp.38}. Durch} Benachbarung, s. \so{Sandhi}. Gesetze des L., ob und inwiefern ausnahmslos? \pageref{sp.185}–\pageref{sp.196}.

\so{Lautwesen}. Aehnlichkeiten im L., inwieweit sie Vermuthung der Verwandtschaft begründen \pageref{sp.148}–\pageref{sp.149}. Aehnlichkeiten in seiner Entwickelung als Beweis für nähere Verwandtschaft \pageref{sp.159}–\pageref{sp.160}, \pageref{sp.164}–\pageref{sp.165}. Gleiches L. bei grosser Verschiedenheit im Wortschatze \pageref{sp.165}–\pageref{sp.166}. Verschliff des L., dessen weitere Wirkungen \pageref{sp.178}. Einfluss der flüchtigeren oder intensiven Articulation \pageref{sp.183}. Inwieweit brauchbar zur Beurtheilung der Sprachen \pageref{sp.431}–\pageref{sp.432}. Problem der allgemeinen Grammatik \pageref{sp.479}.

\sed{\so{Leben} der Sprache \pageref{sp.8}–\pageref{sp.9}, \pageref{sp.15}.}

„\so{Le style c’est l'homme}“ \pageref{sp.98}, \pageref{sp.105}.

\so{Leber} – jecur u.~s.~w. \pageref{sp.192}, \pageref{sp.217}.

\so{Lehnwörter} beim Verwandtschaftsnachweise \pageref{sp.154}. Anpassung an das heimische Lautwesen \pageref{sp.186}–\pageref{sp.187}. \sed{Umgestaltung im Volksmunde \pageref{sp.217}.} Religiöse \pageref{sp.231}. Kampf um’s Dasein \pageref{sp.238}, \pageref{sp.262}. L. und Nachbildungen \pageref{sp.261}–\pageref{sp.268}.

\so{Lehrbücher} – Lehrer \pageref{sp.71}. L. und Handbücher \pageref{sp.111}.

\so{Lehrer} und Lehrbücher \pageref{sp.71}. L. und Gelehrte \pageref{sp.110}.

\so{Lehrgänge}, Cursus der Grammatik \pageref{sp.111}.

\so{Lehrsätze}, grammatische, ihr Ausdruck durch Paradigmen und graphische Formeln \pageref{sp.116}–\pageref{sp.119}. Zusammenwirken verschiedener \pageref{sp.118}–\pageref{sp.119}.

\so{Leibniz} und die Sprachwissenschaft \pageref{sp.27}.

\so{Leichdorn} als \inlineupdate{Hühnerauge,}{Hühneruage,} Elsterauge, Fischauge benannt \pageref{sp.41}.

\so{Leitfäden}, grammatische \pageref{sp.109}.

\so{Lenape} \pageref{sp.163}.

\so{Lepsius}, Richard, Standard-Alphabet \pageref{sp.38}\sed{, \pageref{sp.69}}. Nubische Grammatik \pageref{sp.161}. \inlineupdate{Über}{Ueber} die genealogischen Verhältnisse der Sprachen Afrikas \pageref{sp.282}, \pageref{sp.406}.

„\so{Lernen}“ im obersächsischen Dialekte \pageref{sp.316}.

\fed{{\textbar}485{\textbar}}

\so{Lesen, Lesbarkeit}, als Kennzeichen der Schrift im Gegensatze zum Bilde oder Symbole \pageref{sp.128}–\pageref{sp.129}. Das Durchfliegen und Ueberfliegen \pageref{sp.433}.

\so{Leskien}, Aug. \pageref{sp.254}.

\so{Lessing}. Volksthümliche Sprache \pageref{sp.46}\sed{, \pageref{sp.438}}.

\so{Lettisch} droht auszusterben \pageref{sp.146}.

\sed{\so{Lexikographie}, deren Zweitheilung \pageref{sp.85}.}

\so{Lexikon} s. \so{Wörterbuch}.

\so{Libysche Sprachen}, Berbersprachen \pageref{sp.282}.

–\so{lich} \pageref{sp.122}, \pageref{sp.437}.

\so{Liebe}, \inlineupdate{Erweiterung}{Erweiteruug} des Ich \pageref{sp.307}.

\so{Lifu} \pageref{sp.280}.

\so{Ligurer}. Ihre Sprache ausgestorben \pageref{sp.146}.

\so{Limes sorabicus} \pageref{sp.287}.

\so{Lingua} – jihvā, Zunge \pageref{sp.186}, \pageref{sp.217}.

\so{Lingua rustica latina} \pageref{sp.174}, \pageref{sp.183}.

\sed{\so{Lingua toscana} \pageref{sp.139}.}

\so{Linguistik} s. \so{Sprachwissenschaft}.

\so{Linien} der Sprachverwandtschaft \pageref{sp.145}.

\so{Litauisch} droht auszusterben \pageref{sp.146}.

\so{Litauisch-slavische Sprachen}, \inlineupdate{Ahnlichkeit}{Aehnlichkeit} in der Entwickelung der Gutturale mit den indisch-iranischen \pageref{sp.159}, \pageref{sp.163}. Die Zahlwörter Neun und Zehn \pageref{sp.401}.

\so{Literatur}. \inlineupdate{Übersichten}{Uebersichten} der L. als Anhang zur Grammatik \pageref{sp.107}. Einfluss fremder Literaturen auf die Muttersprache \pageref{sp.274}–\pageref{sp.275}.

\sed{\so{Lituslavisch}, Verwandtschaft mit dem Arischen? \pageref{sp.164}.}

\so{Logik} und Sprachwissenschaft \pageref{sp.14}. Logische Schulung des Sprachforschers \pageref{sp.47}–\pageref{sp.48}. Die L. stellt der Sprache Aufgaben \pageref{sp.48}, \pageref{sp.95}. L. und Grammatik \pageref{sp.448}.

\so{Logone} \pageref{sp.282}.

\so{Lotze}, Definition des Denkens \pageref{sp.6}.

\so{Ludwig}, Alfred, Adaptationstheorie \pageref{sp.180}.

\so{Lüge} und Sprache \pageref{sp.309}, \pageref{sp.310}.

\sed{\so{Lule} \pageref{sp.328}.}

\so{Lüneburg}. Sprache der Wendländer \pageref{sp.146}.

\so{Lushai} \pageref{sp.257}.

\so{Luther} \pageref{sp.46}. Seine Sprache und die heutige \pageref{sp.139}, \pageref{sp.271}.

\so{Luxus} in der Sprache als Aeusserung des Formungstriebes \pageref{sp.361}–\pageref{sp.364}.

\sed{{\textbar}{\textbar}505{\textbar}{\textbar}}

\subsection*{M.}\label{reg.M}\pdfbookmark[1]{M.}{reg.M}

M, tönendes \pageref{sp.186}.

\so{Maba} \pageref{sp.282}.

\so{Macdonald}, D., Oceania, Linguistic and Anthropological \pageref{sp.162}.

\so{Mächte} der Sprachgeschichte. Ihr Wirken im Allgemeinen \pageref{sp.168}–\pageref{sp.169}. Einheimische und fremde \pageref{sp.177}. Die einzelnen M. \pageref{sp.181} flg. – M. des Bedeutungswandels \pageref{sp.232}. Hemmende und beschleunigende \pageref{sp.258}. Die alltäg\pageref{sp.1}ichsten sind die wirksamsten \pageref{sp.259}.

\so{Maclay-Küste} von Neu-Guinea, Sprachen \pageref{sp.280}.

\so{Madagaskar}, von Malaien besiedelt \pageref{sp.147}.

\so{Madagassisch, Madegassisch}, zum malaischen Sprachstamme gehörig \pageref{sp.26}. Euphonisches –ă \pageref{sp.157}, \pageref{sp.435}. \sed{Lautwandel \pageref{sp.192}.} Sandhi \pageref{sp.198}.

\so{Mädchen}. Wörter dafür in den romanischen und germanischen Sprachen \pageref{sp.153}.

\so{Mafoor} und Malaisch: Lautvertretungen \pageref{sp.158}. Unorganisches, euphonisches d \pageref{sp.201}. Malaio-polynesische Elemente \pageref{sp.280}. Prädicative Conjugation \pageref{sp.391}. Dual der Personalpronomina \pageref{sp.409}. Namen für Körpertheile und Verwandtschaftsgrade \pageref{sp.441}. Innerer Wandel der Wörter \pageref{sp.443}.

\sed{\so{Magio}, A. \pageref{sp.424}.}

\so{Magyaren}, ihre Rasse \pageref{sp.147}. \sed{Nicht mit Chinesisch verwandt \pageref{sp.156}. Indogermanischer Einfluss \pageref{sp.273}. Agglutinirende oder scheinwortige Sprache \pageref{sp.342}.}

\so{Magyarisch}. \inlineupdate{Ähnlichkeiten}{Aehnlichkeiten} in der Conjugation mit den indogermanischen Sprachen \pageref{sp.153}. Vocalharmonie \pageref{sp.350}, \pageref{sp.403}. Objectivconjugation \pageref{sp.384}, \pageref{sp.391}. \fed{Bestimmter Artikel \pageref{fp.394}.}

\sed{\so{Mähre} \pageref{sp.228}.}

\so{Malaisch}. Tonfall \pageref{sp.34}. Wortschöpfungen \pageref{sp.42}. Arabische Schrift \pageref{sp.129}. Ersatz für die \inlineupdate{Pronn.}{Pron.} \pageref{sp.1}. und \pageref{sp.2}. pers. \pageref{sp.152}. M. und Mafoor: Lautvertretungen \pageref{sp.158}. Unsichere Articulation \pageref{sp.193}. Unorganischer Dentalvorschlag \pageref{sp.201}. Lautsymbolik in den Wortstämmen \pageref{sp.223}. Präposition akan, Suffix –kan \pageref{sp.348}. Präfixe me-. pe-, Accent- und Quantitätswandel \pageref{sp.350}. Relativwort yaṅ \pageref{sp.457}.

\so{Malaische Sprachen}, consonantische Auslaute \pageref{sp.29}. Wohlklang \pageref{sp.34}. \fed{Passiva des Ortes und des Werkzeugs \pageref{fp.106}.} \sed{Verhältniss zu den melanesischen \pageref{sp.141}.} Prä- und suffigirender Bau \pageref{sp.149}. Innere Sprachform \pageref{sp.150}. Den semitischen ver\fed{{\textbar}486{\textbar}}wandt? \pageref{sp.162}. Sandhi \pageref{sp.199}. Zweisylbige Wortstämme, ob zerlegbar? \pageref{sp.242}. Indische und arabische Lehnwörter \pageref{sp.266}. Gleiche Denkgewohnheiten \pageref{sp.293}. Verschiedengradige Bildsamkeit \pageref{sp.349}. \inlineupdate{Possessivconjugation}{Possesivconjugation} \pageref{sp.384}. Wiederholung der Formenelemente in der Coordination \pageref{sp.400}. Sprache und Rasse; Vergleichung mit den semitischen \pageref{sp.411}–\pageref{sp.415}. Desgl. mit den uralaltaischen \pageref{sp.415}–\pageref{sp.420}. Satzverknüpfung \pageref{sp.465}.

\so{Malaio-polynesische Sprachen}. Ihre Verbreitung \pageref{sp.142}, \pageref{sp.147}. Zweifelhaft, welche Stufe die ursprünglichere sei, die einfachere oder die reich agglutinirende \pageref{sp.257}. Bopp’s Versuch, sie dem indogermanischen Stamme zuzuweisen \pageref{sp.144}, \pageref{sp.155}, \pageref{sp.266}.

\sed{\so{Malayâlam} \pageref{sp.344}.}

\so{Malediven}, \so{Maledivisch} \pageref{sp.147}.

\so{Man}, E. H. The Andaman Islanders \pageref{sp.442}. 

\so{Mande}, \so{Mandingo} \pageref{sp.150}, \pageref{sp.282}.

\so{Mandschu}. Lautwesen \pageref{sp.34}\sed{, \pageref{sp.314}.} Name des Hühnerauges \pageref{sp.41}. Schrift \pageref{sp.129}, \pageref{sp.131}. Onomatopöien \pageref{sp.154}. Vocalharmonie \pageref{sp.199}. Accentuation \pageref{sp.212}. Nomun, bitχe, debtelin, \inlineupdate{tumen}{tumen,} \pageref{sp.264}. Hosihon, Kesike \pageref{sp.267}. \inlineupdate{Ähnlichkeiten}{Aehnlichkeiten} mit Japanisch und Koreanisch \pageref{sp.289}–\pageref{sp.290}. Gleiches Hülfswort für den Genitiv und den Fragesatz \pageref{sp.347}. Vocalismus der Suffixe \pageref{sp.352}. Syntaktische Composita \pageref{sp.359}. Begabung des Volkes \pageref{sp.389}. Gebrauch der Formenelemente \pageref{sp.381}. Vocalsymbolik \pageref{sp.408}. Satzverknüpfung \pageref{sp.418}. Eine in ihrer Ausbildung gestörte Sprache \pageref{sp.427}. Pause hinter dem Subjecte \pageref{sp.455}.

\so{Manieren}, sprachliche \pageref{sp.275}.

\so{Manipuri} \pageref{sp.257}.

\so{Männer-} und \so{Weibersprachen} \pageref{sp.248}–\pageref{sp.249}.

\so{Maori}. Zweierlei Genitiv \pageref{sp.463}.

\so{Maré} \pageref{sp.280}.

\so{Mariveles} \pageref{sp.280}.

\so{Marquesas}. Zweierlei Genitiv \pageref{sp.463}.

\sed{\so{Marschall} \pageref{sp.228}.}

\so{Marshman}, chines. Grammatik \pageref{sp.25}.

\so{Massoreten} \pageref{sp.22}.

\so{Masswesen} \pageref{sp.107}.

\so{Materialismus}, Einfluss auf die Sprachwissenschaft \pageref{sp.15}.

\so{Matthews}, W. \pageref{sp.194}.

\so{Maya}\sed{, Schrift \pageref{sp.131},} \pageref{sp.383}\sed{, \pageref{sp.389}}.

\so{Maya-Sprachen} \pageref{sp.258}. \sed{Verbindungs-Hilfswörter zwischen Zahlwort und Substantiv \pageref{sp.442}.}

\so{Mediae}, \sed{Fehlen derselben in den polynesischen Sprachen \pageref{sp.148}. M.,} in manchen Sprachen nicht von den Tenues unterschieden \pageref{sp.188}, \pageref{sp.194}–\pageref{sp.195}. Media und Tenuis aufeinanderfolgend \pageref{sp.201}.

\sed{{\textbar}{\textbar}506{\textbar}{\textbar}}

\so{Mehrsylbigkeit} in Naturlauten \pageref{sp.314}.

\so{Mehrzahl} s. \so{Plural}.

\so{Melanesien} \pageref{sp.147}.

\so{Melanesier}. \sed{Verhältniss zu den Malaien \pageref{sp.141}.} Rasse und Sprachen \pageref{sp.147}. Sprachverderb durch fremden Einfluss \pageref{sp.155}–\pageref{sp.166}, \pageref{sp.406}, \pageref{sp.428}. Kannibalismus \pageref{sp.177}. Ihre Sprachen \pageref{sp.280}–\pageref{sp.283}.

\so{Mende} \pageref{sp.160}.

\so{Mensch}, das M. \pageref{sp.236}. M. und Thier in Rücksicht auf das Sprachvermögen \pageref{sp.303} flg.

\so{Menschenkenntniss} und Urtheil in sprachlichen Dingen \pageref{sp.47}.

\so{–mente}, \so{–ment} Adverbialendung in den romanischen Sprachen \pageref{sp.316}, \pageref{sp.348}\sed{, \pageref{sp.437}}.

\so{Merkzeichen}, Vorläufer der Schrift \pageref{sp.127}–\pageref{sp.128}.

\so{Messapier} und ihre Sprache verschwunden \pageref{sp.146}.

\so{Metaphysik} und Sprachwissenschaft \pageref{sp.14}.

\so{Metathesis} der Laute \pageref{sp.200}.

\so{Methode} des Sprachunterrichts \pageref{sp.71}–\pageref{sp.75}. M. und System der Grammatik \pageref{sp.109}–\pageref{sp.110}.

\so{Mexikanisch}. Schrift \pageref{sp.130}. Sprache dem Algonkinstamme verwandt? \pageref{sp.147}, \sed{\pageref{sp.152},} \pageref{sp.162}–\pageref{sp.163}. Formenbildung, Incorporation \pageref{sp.328}, \pageref{sp.354}–\pageref{sp.357}. \sed{Einverleibende oder satzwortige Sprache \pageref{sp.342}.} Conjugation \pageref{sp.383}.

\so{Meyer}, Adolf Bernhard \pageref{sp.280}.

\sed{\so{Meyer}, H. A. E., Vocabulary \pageref{sp.102}.}

\sed{{\so{Mezzofonti,}}\edins{{\textbar}{\textbar}\so{Mezzofanti,}} Cardinal, Sprachvirtuos \pageref{sp.48}.}

\so{–mi}, Conjugation auf –mi im Sanskrit und in einigen slavischen Sprachen \pageref{sp.186}.

\so{Miao-tse} \pageref{sp.257}.

\so{Miauen} \pageref{sp.322}.

\so{Miklosich}, Franz, Vergleichende Grammatik der slavischen Sprachen \pageref{sp.173}.

\so{Mikmak} \pageref{sp.163}.

\so{Mikronesien} \pageref{sp.147}.

\sed{\so{Minahasa} \pageref{sp.245}.}

\so{„Mir“} statt: wir \pageref{sp.202}.

\fed{{\textbar}487{\textbar}}

\so{Mischsprachen}. Früheres Verhalten der Wissenschaft ihnen gegenüber \pageref{sp.158}. Eigentliche M. \pageref{sp.278}–\pageref{sp.283}. Geschäftlicher Zweck, keine Entfaltung des Formungstriebes \pageref{sp.366}. Einfachheit und Armuth, Rohheit \pageref{sp.406}–\pageref{sp.407}.

\so{Missionare}, ihre Verdienste um Sprachenkunde und Sprachwissenschaft \pageref{sp.25}. Ihr Verhalten bei der Erlernung fremder Sprachen \pageref{sp.68}.

\sed{\so{Mistéli}, Franz \pageref{sp.296}. Beurtheilung der Sprachen \pageref{sp.394}. Innere Sprachform. \pageref{sp.338} flg.}

\so{Mittel}. Die Sprache als M. des Ausdruckes \pageref{sp.84}–\pageref{sp.86} (vergl. \so{synthetisches System}).

\fed{\so{Mitterrutzner} \pageref{fp.70}.}

\so{Mittheilende Rede} \pageref{sp.318} flg.

\so{Mittheilsamkeit}, sich in der psychologischen Modalität äussernd \pageref{sp.473}.

\so{Mittheilung} der Gedanken, der geschäftliche Zweck der Sprache, aber nicht der einzige \pageref{sp.362}.

\sed{\so{Mixteken} \pageref{sp.389}.}

\so{Modalität}, logische und psychologische \pageref{sp.103}. Logische \pageref{sp.470}–\pageref{sp.472}. Psychologische \pageref{sp.472}–\pageref{sp.474}. Sociale \pageref{sp.474}–\pageref{sp.475}.

\so{Mode} in der Sprache \pageref{sp.126}, \pageref{sp.249}–\pageref{sp.250}.

\so{Modulationen} der \inlineupdate{Stimme,}{Stimme} als Ausdruck der Stimmungen \pageref{sp.376}–\pageref{sp.380}.

\so{Modus}. Bezeichnung durch Verbalformen \pageref{sp.383}. Modi der Haupt- und Nebensätze einander bedingend \pageref{sp.465}.

\so{Möglichkeit}. Ausdrucksformen dafür \pageref{sp.95}. Die Kategorie des Könnens \pageref{sp.103}. M. – Regel – Gesetz \pageref{sp.385}–\pageref{sp.387}.

\so{Momoro Dualu Bukere}, Schrifterfinder \pageref{sp.131}.

\so{Môn}, s. \so{Peguanisch}. \sed{Annam-Sprachfamilie \pageref{sp.281}.}

\so{Mond.} Namen in verschiedenen indogermanischen Sprachen \pageref{sp.154}.

\so{Mongolisch}. Tonfall \pageref{sp.34}. Schrift \pageref{sp.131}. Nom, bičik \pageref{sp.264}. Indische Einflüsse \pageref{sp.271}. Vocalharmonie \pageref{sp.402}.

\so{Monogamie} \pageref{sp.306}.

\so{Morphologie} der Sprache \pageref{sp.17}. Bildungsweise der Wörter als lexikalischer Eintheilungsgrund \pageref{sp.123}. M. des Satzes \pageref{sp.149}.

\sed{\so{Mouboddo},\edins{{\textbar}{\textbar}\so{Monboddo},} Lord \pageref{sp.15}.}

\sed{\so{Moxa} \pageref{sp.390}.}

\so{Muffeln} \pageref{sp.36}, \pageref{sp.182}.

\so{Muhammedanismus}, s. \so{Islâm.}

\sed{\so{Mulattensprache} eine Mischsprache \pageref{sp.159}.}

\sed{\so{Müller}, A., Semitische Nomina \pageref{sp.411}.}

\so{Müller}, Friedrich, Grundriss der Sprachwissenschaft \pageref{sp.28}, \pageref{sp.51}. \inlineupdate{Über}{Ueber} hamitische Anklänge in anderen afrikanischen Sprachen \pageref{sp.161}. \sed{Innere Sprachform \pageref{sp.337} flg.} \inlineupdate{Über}{Ueber} das grammatische Geschlecht \pageref{sp.481}.

\so{Müller}, Max, Vorlesungen über die Wissenschaft der Sprache \pageref{sp.52} Anm. Die turanischen Sprachen \pageref{sp.155}. Religionsgeschichte \pageref{sp.294}. \sed{Unsichere Lautbilder \pageref{sp.301}.}

\so{Munda-Sprachen}, \inlineupdate{s.}{siehe} \so{Kolarische Sprachen.}

\so{Mundart}, s. \so{Dialekt.}

\so{Münzwesen} \pageref{sp.107}.

\so{Murmeln} \pageref{sp.182}.

\sed{\so{Murray’s Island} \pageref{sp.446}.}

{\textbar}{\textbar}507{\textbar}{\textbar}

\sed{\so{Musikinstrumente}, Onomatopoetischer Name \pageref{sp.225}.}

\so{Müssen} \pageref{sp.103}.

\so{Mutsun} \pageref{sp.358}, \pageref{sp.423}.

\so{Mutter}. Wörter dafür \pageref{sp.153}.

\so{Muttersprache}, angelernt, nicht angeboren \pageref{sp.61}. Begriff \pageref{sp.62}–\pageref{sp.63}. \sed{Deren Erlernung \pageref{sp.65} flg.} Ursache von Fehlneigungen bei der Erlernung und Handhabung fremder Sprachen \pageref{sp.71}\sed{, \pageref{sp.120}}. Naives Verhalten zu ihr; lautsymbolisches Gefühl \pageref{sp.218}–\pageref{sp.225}. Sprachmischung innerhalb der M. \pageref{sp.273}–\pageref{sp.277}.

\so{Mythologie}. Neigung der Indogermanen zur M. \pageref{sp.447}.

\subsection*{N.}\label{reg.N}\pdfbookmark[1]{N.}{reg.N}

N, tönendes \pageref{sp.186}.

\so{Nachahmung} von Lauten \pageref{sp.36} flg. Der Sprachanwendung überhaupt \pageref{sp.104}–\pageref{sp.105}. In der \inlineupdate{Übersetzungsliteratur}{Uebersetzungsliteratur} \pageref{sp.105}. N. fremder Sprachformen und Redensarten in der Muttersprache \pageref{sp.270}–\pageref{sp.272}.

\so{Nachahmungstrieb} und Sprachschöpfung \pageref{sp.308} flg.

\sed{\so{Nachbarsprachen} und -Dialekte. Einfluss auf das Lautwesen \pageref{sp.269}.}

\so{Nachbildungen} statt der Fremdwörter \pageref{sp.262} flg.

\inlineupdate{\so{Naga-sprachen}}{\so{Naga-Sprachen}} \pageref{sp.149}, \pageref{sp.257}.

\so{Nahrung} des Urmenschen und sein Sprachvermögen \pageref{sp.304}.

\so{Nahuatl} s. \so{Mexikanisch.}

\sed{\so{Naivität} gegenüber der Muttersprache \pageref{sp.218}. Naive Beseelung des Leblosen \pageref{sp.316}.}

\sed{\so{Nama-Hottentotisch} \pageref{sp.214}.}

\so{Namen} der Vorstellungen, woher entlehnt \pageref{sp.40}–\pageref{sp.42}.

\sed{\so{Nancowry-nicobarische Sprache} \pageref{sp.281}.}

\so{Napoleon} I. Sein Stil \pageref{sp.465}.

\so{Napoleon} III. Nationalitätsprinzip \pageref{sp.261}.

\fed{{\textbar}488{\textbar}}

\so{Nasale}, sich dem folgenden Consonanten anähnlichend \pageref{sp.201}. \sed{Im Auslaut \pageref{sp.200}.}

\so{Nasalis sonans} \pageref{sp.186}.

\so{Näseln} \pageref{sp.36}, \pageref{sp.184}.

\sed{\so{Nationalität} im Stil \pageref{sp.105}–\pageref{sp.106}.}

\so{Nationalitätsprinzip} \pageref{sp.55}, \pageref{sp.261}.

\so{Naturlaute} als Ausnahmen von den Lautgesetzen \pageref{sp.208}–\pageref{sp.210}. Vorbildlich für die Vocalsymbolik \pageref{sp.255}. \sed{Deren Nachahmungen in der Ursprache \pageref{sp.314}.}

\so{Naturvölker} \pageref{sp.387}.

\so{Naturwissenschaft} und Sprachwissenschaft \pageref{sp.15}.

\so{Nebenbeschäftigungen}, wissenschaftliche, des Sprachforschers \pageref{sp.53}.

\so{Nebenformen}, s. \so{Doubletten}, \so{Formdoubletten}.

\so{Nebrixa}. Dessen lateinische Grammatik \pageref{sp.106}.

\so{Negation} \pageref{sp.103}.

\so{Negritos} der Philippinen \pageref{sp.280}.

\so{Neid} \pageref{sp.307}–\pageref{sp.308}.

\so{Nengone} \pageref{sp.280}.

\so{Neubildungen}, gemeinsame, deuten auf engere Verwandtschaft \pageref{sp.159}. N. von Wortstämmen, lautsymbolische \pageref{sp.223}.

\so{Neu-Britannien} \pageref{sp.165}.

\so{Neuerungen} in der Sprache \pageref{sp.126}–\pageref{sp.127}. Von Hause aus Fehler \pageref{sp.184}. Verbreiten sich von einem Punkte aus \pageref{sp.229}. \sed{N. in der Wortbedeutung \pageref{sp.228}.} Abänderungen und constitutionelle Weiterbildungen \pageref{sp.428}–\pageref{sp.429}.

\so{Neugier} und wissenschaftlicher Trieb \pageref{sp.17}. N. und Sprachschöpfung \pageref{sp.309}.

\so{Neugriechisch}. Ortsnamen mit Resten von Präpositionen im Anlaute \pageref{sp.203}.

\so{Neu-Guinea} \pageref{sp.147}. Sprachen von N. \pageref{sp.280}, \pageref{sp.282}.

\so{Neuindische Sprachen}, arische. Casussuffixe \pageref{sp.159}. Agglutination \pageref{sp.257}. Dentale und Cerebrale \pageref{sp.269}.

\so{Neu-Lauenburg} \pageref{sp.165}.

\sed{\so{Neupersisch}, eine arische Sprache \pageref{sp.158}.}

\so{Neu-Pommern} \pageref{sp.165}.

\so{Neu-seeländisch} (Maori). Zweierlei Genitiv \pageref{sp.463}.

\so{Neutro-passivus}, casus \pageref{sp.102}, \pageref{sp.151}.

\so{Neutrum}, vergl. \so{verbum neutrum}.

\so{Nevome}-Sprache in Sonora. Unsichere Articulation \pageref{sp.194}.

\so{Newar} \pageref{sp.157}.

\so{Niederrheinländer}. Ihr Dialekt \pageref{sp.190}.

\so{Nigritier}. Rasse und Sprachen \pageref{sp.147}, \pageref{sp.177}.

\so{Noch} \pageref{sp.103}.

\so{Nomen-Verbum} \pageref{sp.115}.

\so{Nomina}. N. propria s. \so{Eigennamen}.

\so{Nominativus} \pageref{sp.102}, \sed{\pageref{sp.115},} \pageref{sp.354}. Sein angeblicher Werth \pageref{sp.392}.

\so{Noten}, musikalische, zur Bezeichnung des Redetones \pageref{sp.377}.

\so{Nothwendigkeit}, vergl. \so{Müssen}, \so{Sollen} \pageref{sp.103}.

\so{Nuba} \pageref{sp.282}\sed{, \pageref{sp.426}}.

\so{Nufoor}, s. \so{Mafoor}.

\so{Numerirung} als Mittel der grammatischen Terminologie \pageref{sp.116}.

\so{Numerus}, s. \so{Zahl}.

\sed{{\textbar}{\textbar}508{\textbar}{\textbar}}

\so{Nupe} \pageref{sp.150}.

\so{Nur} \pageref{sp.103}.

\so{Nuscheln} \pageref{sp.36}, \pageref{sp.182}.

\sed{νῦ ὲφελκυστικόν \pageref{sp.198}.}

\subsection*{O.}\label{reg.O}\pdfbookmark[1]{O.}{reg.O}

Ŏ in der indogermanischen Ursprache \pageref{sp.186}.

\fed{„\so{Ob}“ \pageref{sp.230}.}

\inlineupdate{\so{Object,}}{\so{Objekt,}} logisches: Activum und Passivum \pageref{sp.95}. O. und adverbiales Attribut \pageref{sp.102}, \pageref{sp.461}.

\so{Objectivität} und Subjectivität als Gegenstände des sprachlichen Ausdruckes \pageref{sp.438}. Die O. \pageref{sp.438} flg.

\inlineupdate{\so{Objectssatz}}{\so{Objectsatz}} \pageref{sp.104}.

\so{Odschi} \pageref{sp.282}.

\so{Odschibwe}, \so{Otschipwe} \pageref{sp.163}\sed{, \pageref{sp.314}.}

\so{Oekonomie} in der Sprache: kein \inlineupdate{Überfluss}{Ueberfluss} \pageref{sp.100}. Möglichst geringer Aufwand für den Zweck \pageref{sp.182}.

\so{Officiere}. Schnarren und Näseln \pageref{sp.184}.

\so{Oigob} \pageref{sp.282}.

\sed{\so{Oldenberg}, H. \pageref{sp.190}, \pageref{sp.196}.}

\so{Ollendorf}. Grammatische Methode \pageref{sp.110}.

\sed{\so{Olymp} \pageref{sp.316}.}

\so{Onomatopöie}, ähnliche in stammverschiedenen Sprachen \pageref{sp.154}. Die O. als Ausnahme von den Lautgesetzen \pageref{sp.208}. \sed{O. und Sprachgefühl \pageref{sp.222}–\pageref{sp.223}.} Vocalsymbolik \pageref{sp.255}. \sed{Symbolisirende O. \pageref{sp.312}.} O. und Stimmungsmimik \pageref{sp.378}.

\so{Optativ}, vgl. \so{Wollen}, \so{Wünschen} \pageref{sp.103}.

\fed{{\textbar}489{\textbar}}

\so{Optische Mittel} der Gedankenmittheilung, warum weniger tauglich als die akustischen? \pageref{sp.311}–\pageref{sp.312}.

\so{Organismus} der Sprache \pageref{sp.17}. Merkmale des höheren oder niederen \pageref{sp.389}–\pageref{sp.394}.

\so{Ort}. Adverbiale Bestimmungen des O. \pageref{sp.101}. Passivum des O. \pageref{sp.102}. Kategorien des O. \pageref{sp.446}.

\so{Orthographie}. Privatorthographien \pageref{sp.108}, \pageref{sp.115}–\pageref{sp.116}, \pageref{sp.133}. Historische und phonetische \pageref{sp.132}–\pageref{sp.133}. Wissenschaftlicher Werth der historischen \pageref{sp.175}.

\so{Ortsnamen} aus Sprachen der früheren Landesbewohner \pageref{sp.265}, \pageref{sp.286}.

\sed{\so{Osmanli} Türkisch. Einwirkung des Islâm \pageref{sp.271}.}

\so{Ostasien}, reich an isolirten Sprachen \pageref{sp.147}.

\so{Osterinsel}, Rapa-nui \pageref{sp.142}, \pageref{sp.147}, \pageref{sp.177}.

\sed{\so{Ostmongolisch} \pageref{sp.349}.}

\so{Othomi} \pageref{sp.257}, \pageref{sp.423}.

\so{Otschipwe} \pageref{sp.163}.

\subsection*{P.}\label{reg.P}\pdfbookmark[1]{P.}{reg.P}

P, deutsch, in Lehnwörtern \pageref{sp.186}.

\so{Paarungszeiten}. Der Mensch hat keine bestimmten \pageref{sp.306}.

\so{Pahi}, \so{Pahri} \pageref{sp.157}.

\so{Palatisirung} \pageref{sp.37}, \pageref{sp.201}.

\sed{\so{Pali} \pageref{sp.245}.}

\sed{\so{Pallegoix} \pageref{sp.224}.}

\so{Pampanga}. Bildsamkeit \pageref{sp.349}.

\sed{\so{Pandosy}, M. C. \pageref{sp.424}.}

\so{Pangolat} in der Batta-Schrift \pageref{sp.131}.

\so{Pânini}, indischer Grammatiker, sein Werk \pageref{sp.22}–\pageref{sp.23}, \pageref{sp.111}, \pageref{sp.113}.

\sed{\so{Papuas} \pageref{sp.280}.}

\so{Paradigmen}, grammatische \pageref{sp.116}–\pageref{sp.119}.

\so{Parsi}, ihre philologischen Arbeiten \pageref{sp.22}.

\sed{\so{Partikeln}, Bedeutung für die Sprachform \pageref{sp.339}.}

\so{Participium}, adjectivisches \pageref{sp.101}. Participiale Verbindungen \pageref{sp.465}.

\so{Pasigraphie}. Die chinesische Schrift als solche für einen Theil Ostasiens \pageref{sp.129}–\pageref{sp.130}.

\so{Passé antérieur} \pageref{sp.253}.

\so{Passivum} \pageref{sp.102}. Die drei Passiva der philippinischen Sprachen \pageref{sp.363}. Vorliebe der malaischen Sprachen für das P. \pageref{sp.415}, \pageref{sp.419}–\pageref{sp.420}.

\so{Pathologie} der Sprache \pageref{sp.17}.

\so{Patois} \pageref{sp.126}.

\so{Paul}, Prinzipien der Sprachgeschichte \pageref{sp.136}.

\so{Paulinus} a. S.~\so{Bartholomaeo}, s. \so{Wesdin}.

\so{Pause} in der Rede \pageref{sp.225}. P. als grammatisches Formenmittel \pageref{sp.451}.

\so{Payne}, J., Grebo Grammar \pageref{sp.379}.

\so{Peguanisch} \sed{\pageref{sp.269}, \pageref{sp.273}}. Schwierigkeit der Beurtheilung \pageref{sp.426}.

\so{Pelasger} und Arnauten \pageref{sp.146}.

\sed{\so{Pepet} im Malaischen \pageref{sp.206}.}

\so{Perfectum} \pageref{sp.103}. Das P. als erzählendes Präteritum in oberdeutschen Dialekten \pageref{sp.253}.

\so{Periphrastische Formen} \pageref{sp.183}. Der Anschaulichkeit dienend, ältere Formen verdrängend \pageref{sp.241}, \pageref{sp.256}.

\so{Perser}, ihre Verdienste um die arabische Grammatik \pageref{sp.22}.

\so{Persisch}. Arabische Schrift \pageref{sp.129}. Indogermanische Grundlage, semitische Beimischungen \pageref{sp.158}. tuman in’s Mandschu übergegangen; defter = διφθέρα \pageref{sp.264}. Arabische Einflüsse \pageref{sp.271}. Adjectivische und genitivische Attribute \pageref{sp.457}.

\sed{{\textbar}{\textbar}509{\textbar}{\textbar}}

\so{Person} = Frauenzimmer \pageref{sp.235}. Dritte P. in der Conjugation ohne Formzeichen \pageref{sp.383}, \pageref{sp.391}–\pageref{sp.392}, \pageref{sp.460}.

\inlineupdate{\so{Personificirung}}{\so{Personifizirung}} als Factor der Sprachbildung \pageref{sp.315}–\pageref{sp.317}.

\so{Perspective} des Gemüths \pageref{sp.307}. Geistige P. in der Sprache \pageref{sp.325}.

\so{Peruaner}. Ihre Quipus \pageref{sp.128}.

\sed{\so{Petitot}, \inlineupdate{E.}{E,} \pageref{sp.404}, \pageref{sp.424}.}

\so{Pf} in deutschen Lehnwörtern \pageref{sp.186}.

\so{Pfeifen} – piepen \pageref{sp.208}.

\so{Pferd}, von den Algonkinvölkern „grosser Hund“ genannt \pageref{sp.41}.

\so{Pgo} – Karen \pageref{sp.201}.

\so{Philippinen-Inseln} \pageref{sp.147}. Deren Sprachen besonders bildsam \pageref{sp.349}. Drei Passiva \pageref{sp.363}. Der malaische Sprachtypus \pageref{sp.415}.

\so{„Philister“} \pageref{sp.233}.

\so{Philologie}, Wissenschaft der Epigonen \pageref{sp.21}, \pageref{sp.24}–\pageref{sp.25}; verlangt „dramatischen Instinct“ \pageref{sp.52}.

\so{Philosophie} und Sprachwissenschaft \pageref{sp.14}.

\so{Phonetik} (Lautphysiologie) \sed{\pageref{sp.5}}, ein Zweig der Naturwissenschaft \pageref{sp.14}. Kein Theil der Sprachwissenschaft \pageref{sp.33}. Ihr Gegenstand \pageref{sp.33}. Phonetische Schriftsysteme \pageref{sp.38}. \sed{Phonetische Orthographie \pageref{sp.132}–\pageref{sp.133}.}

\so{Phonograph} \pageref{sp.130}.

\fed{{\textbar}490{\textbar}}

\so{Phraseologie} im Wörterbuche \pageref{sp.124}. \sed{Phr. und Sandhi \pageref{sp.213}.} Vgl. \so{Redensarten.}

\so{Physiologie} der Sprache \pageref{sp.17}. \sed{Ph. und Shandhi\edins{{\textbar}{\textbar}Sandhi} \pageref{sp.201}.}

\sed{\so{Pictet}, A. \pageref{sp.294}.}

\sed{\so{Pictographien} \pageref{sp.128}.}

\so{Piepen} – pfeifen \pageref{sp.208}.

\so{Pima}-Sprache in Sonora. Unsichere Articulation \pageref{sp.194}.

\so{Piṭaka} \pageref{sp.264}.

\so{Pitchen-Englisch} \pageref{sp.435}.

Platon: φύσει oder θέσει? \pageref{sp.179}.

\inlineupdate{\so{Plattdeutch}}{\so{Plattdeutsch}} \pageref{sp.55}, \pageref{sp.57}. P. und Holländisch \pageref{sp.159}.

\so{Platzwechsel} der Laute \pageref{sp.200}.

\sed{\so{Pleonasmus} \pageref{sp.239}.}

\so{Plinse} \pageref{sp.265}.

\so{Plural} \pageref{sp.101}. Collectiver und individualisirender im Deutschen \pageref{sp.254}.

\so{Poesie} in der Wortschöpfung \pageref{sp.42}.

\so{Poetik} als Anhang zur Grammatik \pageref{sp.107}.

\so{Polarvölker} \pageref{sp.177}.

\so{Polnisch}. Tonfall \pageref{sp.34}, \pageref{sp.431}. \sed{Accent \pageref{sp.212}.}

\so{Polyglotten} \pageref{sp.27}–\pageref{sp.28}. Probetexte in P. \pageref{sp.106}.

\so{Polynesien} \pageref{sp.147}.

\so{Polynesische Sprachen}. Lautwesen \pageref{sp.34}, \pageref{sp.149}, \pageref{sp.197}. Das Tapu \fed{\pageref{fp.46},} \pageref{sp.245}. Sprachen oder Dialekte? \pageref{sp.54}. Auftauchen alter Auslaute in der Passivbildung \pageref{sp.87}. Nicht typisch für den malaio-polynesischen Sprachbau \pageref{sp.415}. Schwund der Auslautsconsonanten \pageref{sp.435}. Armuth an \inlineupdate{Afformativen}{Afformationen} und Formwörtern, doppelter Genitiv \pageref{sp.463}.

\so{Polysynthetismus} \pageref{sp.354}–\pageref{sp.359}. \sed{P. und Isolation \pageref{sp.257}.}

\so{Pomai-Bog} \pageref{sp.265}.

\so{Portugiesisch} \pageref{sp.54}.

\sed{\so{Posan} \pageref{sp.245}.}

\sed{\so{Possessivbegriff} \pageref{sp.341}.}

\sed{\so{Possessivconjugation} \pageref{sp.384}.}

\so{Postpositionen} \pageref{sp.101}. P. oder Casus? \pageref{sp.115}. Ihr Werth \pageref{sp.461}–\pageref{sp.463}.

\so{Pott}, Aug. Friedr., Etymologische Forschungen \pageref{sp.27}, \sed{\pageref{sp.176},} \pageref{sp.293}. P. und Humboldt \pageref{sp.29}\sed{, \pageref{sp.137}}; seine Vielseitigkeit \pageref{sp.30}, \fed{\pageref{fp.146}.} P. führt die Lautgesetze in die Indogermanistik ein \pageref{sp.170}. Zerlegung indogermanischer Wurzeln \pageref{sp.180}, \pageref{sp.242}. \sed{Wurzeletymologie \pageref{sp.252}.} Definition der Wurzel \pageref{sp.296}. Innere Sprachform \pageref{sp.327}, \pageref{sp.334}. Schriften über Theile der allgemeinen Grammatik \pageref{sp.481}.

\sed{\so{Prâkrit-Dialekte}, dessen Wurzeln im Sanskrit \pageref{sp.251}.}

\so{Prädicat}, grammatisches \pageref{sp.102}. Pr. des Seins, possessives, ursächliches \pageref{sp.103}. Psychologisches \pageref{sp.365}–\pageref{sp.373}. Nominales und verbales \pageref{sp.339}, \pageref{sp.460}. P. und Attribut \pageref{sp.451}–\pageref{sp.459}. Secundäres P. \pageref{sp.458}–\pageref{sp.459}.

\so{Prädicativattribute} \pageref{sp.456}–\pageref{sp.458}.

\so{Prädicatsprädicate} \pageref{sp.458}–\pageref{sp.459}.

\so{Prädicatscasus} in finnischen Sprachen \pageref{sp.392}.

\so{Prädicatsnomina} \pageref{sp.115}.

\so{Prädicatssatz} \pageref{sp.104}.

\so{Prä- und suffigirende Sprachen} \pageref{sp.149}, \pageref{sp.349}.

\so{Präfixe} \pageref{sp.348}.

\inlineupdate{\so{Präpositionen}}{\so{Präpositonen}} \pageref{sp.101}, \pageref{sp.461}–\pageref{sp.463}.

\so{Prärogativinstanzen} bei der Sprachvergleichung \pageref{sp.158}.

\so{Präsens}, historisches im Lateinischen \pageref{sp.473}.

\so{Pratt}, G. \pageref{sp.193}, \pageref{sp.463}.

\so{Prémare}, chines. Grammatiker \pageref{sp.25}.

\sed{\so{Presse}. Ihre Antheilnahme an der Sprachbildung \pageref{sp.288}.}

\so{Preussisch}, s. \so{Altpreussisch}.

\so{Priscianus}, Institutiones grammaticae \pageref{sp.21}.

\so{Proadverbien} \pageref{sp.101}.

\so{Proklitische Wörter} \pageref{sp.348}.

\so{Prolegomena} der Grammatik \pageref{sp.86}–\pageref{sp.88}.

\sed{{\textbar}{\textbar}510{\textbar}{\textbar}}

\so{Pronomina} \pageref{sp.101}. – personalia, conjugirbar im Anatom \pageref{sp.151}. Beweiswerth der P. für die Verwandtschaft \pageref{sp.152} flg. Die Personalpronomina und das Familienleben \pageref{sp.306}–\pageref{sp.307}. \sed{Bedeutung für die Sprachform \pageref{sp.339}.}

\so{Prosa}. Einfluss der griechisch-römischen und der französischen auf andere Sprachen \pageref{sp.271}. Griechische und chinesische \pageref{sp.414}.

\so{Prosecutivus} \pageref{sp.114}.

\sed{\so{Protagoras} als Sprachphilosoph \pageref{sp.20}.}

\so{Proverba} \pageref{sp.101}.

\so{Provinzialismen}, fehlerhafte \pageref{sp.44}–\pageref{sp.45}. Einmischung in die Schriftsprache \pageref{sp.62}. Berücksichtigung in der einzelsprachlichen Forschung \pageref{sp.125}–\pageref{sp.127}.

\so{Prüderie} und Zote \pageref{sp.248}–\pageref{sp.249}.

\so{Psychologie} und Sprachwissenschaft \pageref{sp.14}. Psychologische Schulung des Sprachforschers \pageref{sp.39}–\pageref{sp.47}. Praktische Ps. \pageref{sp.47}. Antheil an der Formung der Sprache \pageref{sp.95}–\pageref{sp.96}. \sed{P. und Sandhi \pageref{sp.201}, \pageref{sp.205}.}

\so{Psychologisches Prädicat} \pageref{sp.365}–\pageref{sp.373}.

\fed{{\textbar}491{\textbar}}

\so{Psychologisches Subject} \pageref{sp.365}–\pageref{sp.373}. Bei den Semiten und Malaien \pageref{sp.391}.

\so{Pul} \pageref{sp.282}. Lautwechsel am Substantivum und Adjectivum \pageref{sp.391}.

\so{Punctatoren} \pageref{sp.22}.

\so{Pyrenäische Halbinsel}, deren Sprachen \pageref{sp.54}.

\subsection*{Q.}\label{reg.Q}\pdfbookmark[1]{Q.}{reg.Q}

\so{Quasi}-Wörter \pageref{sp.466} flg.

\sed{\so{Quechua} \pageref{sp.314}.}

\so{Queen} (englisch) \pageref{sp.230}.

\so{Quipus}, \so{Quipos} der Peruaner \pageref{sp.128}.

\subsection*{R.}\label{reg.R}\pdfbookmark[1]{R.}{reg.R}

R. Verschiedene Aussprachen dieses Lautes \pageref{sp.36}, \pageref{sp.37}. Tönendes r \pageref{sp.186}.

\so{Radloff}, L. und W. \pageref{sp.69}.

\so{Rapa-nui} \pageref{sp.147}.

\so{Rask}, Rasmus Christian, entdeckt das germanische Lautverschiebungsgesetz \pageref{sp.26}–\pageref{sp.27}.

\so{Rassen}. Gleichheit und Ungleichheit der R. und Sprachverwandtschaft \pageref{sp.147}–\pageref{sp.148}, \pageref{sp.178}. Verschiedene geistige Begabung \pageref{sp.395}. Geistesanlagen der Rassen und Bauart der Sprachstämme \pageref{sp.407}. Semiten \pageref{sp.408}–\pageref{sp.411}. Malaien und Semiten \pageref{sp.411}–\pageref{sp.415}. Malaien und Uralaltaier \pageref{sp.415}–\pageref{sp.420}. Die Bantuvölker \pageref{sp.420}–\pageref{sp.423}. Indianer Amerikas \pageref{sp.423}–\pageref{sp.425}. Andere Völker \pageref{sp.425}–\pageref{sp.426}. Byrne’s Beurtheilung \pageref{sp.426}–\pageref{sp.427}.

\so{Realien} als Anhang zur Grammatik \pageref{sp.107}.

\so{Recht} und Sprache \pageref{sp.17}.

\so{Reciprocum} \pageref{sp.102}.

\sed{\so{Reconstruction} alter Sprachen \pageref{sp.175}.}

\so{Rede}. Sprache = Rede \pageref{sp.3}\sed{, \pageref{sp.59}}. R. in abgerissenen Worten \pageref{sp.182}. Gebundene \pageref{sp.225}–\pageref{sp.227}. Inhalt und Form, Classification darnach \pageref{sp.317}–\pageref{sp.324}. Elliptische R. \pageref{sp.367}–\pageref{sp.368}. Polemische: lebhafte Betonung \pageref{sp.374}.

\sed{\so{Redeformen} \pageref{sp.318} flg.}

\so{Redensarten}, vergleichende \pageref{sp.42}. Werth der R. für die Synonymik \pageref{sp.100}. Alltägliche R. werden unvollkommen articulirt \pageref{sp.182}. Gleichklang in R. \pageref{sp.222}. Einfluss auf den Bedeutungswandel \pageref{sp.234}. Entlehnte R. \pageref{sp.270}–\pageref{sp.273}.

\so{Redetheile}, grammatische, deren Entstehung \pageref{sp.381}–\pageref{sp.385}. Deren Werth \pageref{sp.438}–\pageref{sp.442}.

\so{Reflexivum} \pageref{sp.102}.

\sed{\so{Reflexion} als Anlass zur Ausbildung neuer Begriffe \pageref{sp.308}.}

\so{Reflexlaute} \pageref{sp.309}.

\so{Reformation}, Bibelforsch., Hebräisch \pageref{sp.25}.

\so{Regel}. Das Gewohnte wird R. \pageref{sp.382}. Möglichkeit – R. – Gesetz \pageref{sp.385}–\pageref{sp.387}.

\so{Reim}, lautsymbolisch empfunden \pageref{sp.220} flg.

\sed{\so{Reimarus} \pageref{sp.263}.}

\so{Reinisch}, Leo \pageref{sp.69}. Arbeiten zur Kunde und Vergleichung der nordostafrikanischen Sprachen \pageref{sp.162}. Die Kunama-Sprache \pageref{sp.379}.

\so{Reisende}. Ihr Verhalten bei der Erlernung fremder Sprachen; sprachkundliche Sammlungen \pageref{sp.68}–\pageref{sp.69}.

\so{Relandus}, Hadr., entdeckt Lautvertretungsgesetze \pageref{sp.26}.

\so{Relativsätze}. Adjectivische = Adnominalsätze. Substantivische \pageref{sp.104}. Zwischenprädicate \pageref{sp.457}–\pageref{sp.459}.

\so{Religionen} und Sprachen \pageref{sp.17}.

\so{Rémusat}, Jean-Pierre Abel – \pageref{sp.81}.

\so{Repetitorien}, grammatische \pageref{sp.109}.

\sed{\so{Rhetorische} Fragen als verstärkte Versicherung \pageref{sp.183}. Rh. Fr., Einfluss auf den Bedeutungswandel \pageref{sp.244}.}

\so{Rhythmus}, Bestandtheil der Sprache \pageref{sp.34}. \sed{Rh. als grammatischer Faktor \pageref{sp.147}.} Gefallen der Menschen an rhythmischem Thun \pageref{sp.226}. Rh. als Mittel der Stimmungsmimik \pageref{sp.380}. Rh. als grammatisches Formenmittel \pageref{sp.461}.

\so{Richtigkeit}. Richtige Handhabung der Muttersprache \pageref{sp.61}–\pageref{sp.63}.

\sed{{\textbar}{\textbar}511{\textbar}{\textbar}}

\so{Ridley}, W. \pageref{sp.193}.

\sed{\so{Riggs}, S.~R. \pageref{sp.424}.}

\so{Rohheit} der Sprachen \pageref{sp.381}. Angebliche \pageref{sp.396}.

\so{Romanische Sprachen.} Augmentativa, Diminutiva, kosende und schmähende Wortbildungen \pageref{sp.95}. Bildung des Futurums \pageref{sp.159}, \pageref{sp.348}. Verhältniss zum Lateinischen und der Lingua rustica \pageref{sp.183}. Freiheit der Composition mangelt \pageref{sp.236}. Das Futurum \pageref{sp.241}. Plusquamperfectum und Passé antérieur \pageref{sp.253}. Schwund des Neutrums \pageref{sp.254}. Analytische, periphrastische Formen \pageref{sp.257}, \pageref{sp.349}. Deutsche Lehnwörter \pageref{sp.264}.  \sed{Futurgebilde mit habere \pageref{sp.348}.} Schwund der Casusformen \pageref{sp.354}.

\fed{{\textbar}492{\textbar}}

\so{Römer}, \fed{fehlende Neigung zu sprachwissenschaftlicher Arbeit \pageref{fp.18}–\pageref{fp.19}.} \sed{als Grammatiker \pageref{sp.21}.}

\so{Rufnamen}. Kindliche Verstümmelungen derselben \sed{\pageref{sp.207},} \pageref{sp.277}. R., Familiennamen, Appellativa \pageref{sp.307}.

\so{Rukheng} \pageref{sp.149}.

\so{Rumänisch} und Italienisch \pageref{sp.74}. \sed{Suffigirte Artikel \pageref{sp.273}.}

\so{Russisch}. Grammatische Fremdwörter durch Nachbildungen ersetzt \pageref{sp.263}. Betonung \pageref{sp.431}.

\subsection*{S.}\label{reg.S}\pdfbookmark[1]{S.}{reg.S}

S, laterales im Arabischen \pageref{sp.187}. \sed{Als Polsterlaut im Patois \pageref{sp.198}.} Auslautendes s in i verwandelt im Italienischen; – ob auch anderwärts? \pageref{sp.191}.

\sed{\so{Saali} \pageref{sp.245}.}

\so{Sacchetti}, Franco, sein Toscanisch \pageref{sp.139}.

\so{Sachsen}, Königreich, dessen Sprachen \pageref{sp.54}. Siebenbürgische S., deren Dialekt \pageref{sp.190}.

\so{Sacy}, Sylvestre de \pageref{sp.15}, \pageref{sp.31}. Grammaire arabe \pageref{sp.111}.

\so{Saho} \pageref{sp.160}, \pageref{sp.307}.

\so{Sajnovics}, J. \pageref{sp.26}.

\so{Salzburger} in Litauen \pageref{sp.287}.

\so{Sammlungen} zur Erforschung von Sprachen, s. \so{Collectaneen}. Vgl. \so{Reisende}.

\so{Samoanisch}. Unsichere Articulation \pageref{sp.193}. Zweierlei Genitiv \pageref{sp.463}.

\so{Sandhi}. Die Lehre vom S.~in der Grammatik \pageref{sp.87}. \sed{Sandhigesetze \pageref{sp.114}.} Der S.~als sprachgeschichtliche Macht \pageref{sp.196}–\pageref{sp.205}. Bedeutsamkeit in psychologischer Hinsicht \pageref{sp.401}–\pageref{sp.403}.

\so{Sanguinisches Temperament} des Urmenschen \pageref{sp.309}.

\so{Sanskrit}-Sprache\sed{. Charakteristik derselben} \pageref{sp.22}. Studium in Europa \pageref{sp.26}. Ihre Laute\pageref{sp.34}. Die zehn Conjugationen \pageref{sp.116}. Die Anubandhas der einheimischen Grammatiken \pageref{sp.119}. Suffix \inlineupdate{–in}{in} \pageref{sp.123}. Virâma; mangelnde Worttrennung in der Schrift \pageref{sp.131}–\pageref{sp.132}. Zugrundelegung der S.~bei der \inlineupdate{indogerm.}{indogerman.} Sprachvergleichung \sed{\pageref{sp.141},} \pageref{sp.186}. Die Conjugation auf –mi \pageref{sp.186}. Sandhi \pageref{sp.199}–\pageref{sp.200}. Composita \pageref{sp.236}, \pageref{sp.466}. \sed{Periphrastisches Futur \pageref{sp.241}, \pageref{sp.360}.} Acht Casus \pageref{sp.253}–\pageref{sp.254}. Dentale und Cerebrale \pageref{sp.269}. Guna und Vriddhi \pageref{sp.352}. Bh\=os, bh\=o \pageref{sp.360}. \fed{Periphrastisches Futurum \pageref{fp.369}.} Passivum mittels –yá– \pageref{sp.397}. ûnavĩçati \pageref{sp.402}. Nominale Ausdrucksweise \pageref{sp.425}. Die Präterita \pageref{sp.446}.

\so{Santal} \pageref{sp.249}\sed{, \pageref{sp.390}}. Einverleibende Conjugation \pageref{sp.358}.

\so{Sassetti}, Filippo, vergleicht Indisch und Italienisch \pageref{sp.25}.

\so{Satz}, Ausgangspunkt für die grammatische Analyse \pageref{sp.86}, \pageref{sp.89}. – ist erste eigenlebige Einheit der Sprache \pageref{sp.88}. Zusammengesetzter S.~\pageref{sp.103}–\pageref{sp.104}. S.~als Form der Rede \pageref{sp.322}. S.~und Satztheil \pageref{sp.451}–\pageref{sp.456}. Verwandlung der Sätze in Satztheile \pageref{sp.463}–\pageref{sp.470}.

\so{Satzbau} als Sprachbau \pageref{sp.81}. S.~der malaischen und der semitischen Sprachen \pageref{sp.413}–\pageref{sp.414}. Der uralaltaischen Sprachen \pageref{sp.418}–\pageref{sp.419}.

\so{Satzlehre}, s. \so{Satzbau}, \so{Syntax}.

\so{Satztheile.} Ihre Bildung, Erweiterung, Ersetzung \pageref{sp.101}; ihre Weglassung (Ellipse) \pageref{sp.101}. Verwandlung der Sätze in Satztheile \pageref{sp.104}. Zu ergänzende S.~\pageref{sp.366}. S.~und Sätze \pageref{sp.451}–\pageref{sp.456}. \inlineupdate{Verwandelung}{Verwandlung} der Sätze in S.~\pageref{sp.463}–\pageref{sp.470}.

\so{Satzung}, den Sprachgebrauch beeinflussend \pageref{sp.245}–\pageref{sp.250}.

\so{Satzverbindungen} \pageref{sp.103}–\pageref{sp.104}.

\so{Satzwörter} \pageref{sp.81}. Ungeformte \pageref{sp.345}.

\so{Saugen,} Wörter dafür \pageref{sp.153}.

\so{Sayce,} A. H., Principles of Comparative Philology; Introduction to the Science of Language \pageref{sp.52} Anm.\sed{, \pageref{sp.304}.}

\so{Schachspiel}. Die Dame (Königin) statt des Wessiers \pageref{sp.268}.

\so{Schacht} – Schaft \pageref{sp.267}.

\so{Schallnachahmung} s. \so{Onomatopöie}.

\sed{\so{Schan,} \pageref{sp.257}, \pageref{sp.426}. Sch. in Berma, lautspielerische Gebilde \pageref{sp.223}. Doppelungsformen \pageref{sp.224}.}

\so{Scheidekunst,} Etymologie \pageref{sp.179}–\pageref{sp.181}.

\so{Schematismus} bei der grammatischen Induction \pageref{sp.92}–\pageref{sp.93}.

\so{Schi-hoang-ti,} chinesischer Kaiser, Bücher\sed{{\textbar}{\textbar}512{\textbar}{\textbar}}verbrennung \pageref{sp.19}. Anmassung eines besonderen pron. \pageref{sp.1}. pers. \pageref{sp.230}.

\so{Schilluck} \pageref{sp.282}.

\so{Schlaf.} Geistesleben im Schlafe \pageref{sp.275}.

\fed{{\textbar}493{\textbar}}

\so{Schlecht} \pageref{sp.234}–\pageref{sp.235}.

\so{Schlegel,} Friedrich \pageref{sp.26}. Beide Brüder \pageref{sp.31}.

\so{Schlegel,} J. B., die Ewe-Sprache \pageref{sp.52}.

\so{Schleicher,} Aug., die Sprachwissenschaft als Theil der Naturwissenschaft \pageref{sp.15}. Philologische Nebenbeschäftigungen \pageref{sp.136}. Stammbaumtheorie \pageref{sp.163}. Aufstellung der indogermanischen Ursprache, Vergleichung mit Bopp \sed{\pageref{sp.164},} \pageref{sp.172}. Lautgesetz und Analogie \pageref{sp.173}. Etymologie \pageref{sp.180}. Verhalten zu den Lautgesetzen \pageref{sp.185}\sed{, \pageref{sp.197}}. Der Zetacismus \pageref{sp.201}.

\sed{\so{Schliemann,} Heinrich \pageref{sp.75}.}

\so{Schmidt,} Johannes. Wellentheorie \pageref{sp.164}–\pageref{sp.165}\sed{, \pageref{sp.192}}. \sed{Die Urheimath der Indogermanen \pageref{sp.294}.}

\so{Schnalzlaute} \pageref{sp.34}. \sed{Sch. bei den Nama Hottentotten \pageref{sp.270}.} Vorbilder in der Natur \pageref{sp.314}.

\so{Schnarren,} schnarrende Sprache \pageref{sp.184}.

\so{Schoho} \pageref{sp.282}.

\so{Schott,} Wilhelm \pageref{sp.31}. Chinesische Sprachlehre \pageref{sp.113}.

\so{Schottland.} Clan, loch, glen \pageref{sp.265}.

\so{Schrader,} O., Sprachvergleichung und Urgeschichte \pageref{sp.294}.

\sed{\so{Schreibung} fremder Sprachen \pageref{sp.38}–\pageref{sp.39}.}

\so{Schreibweise} der Grammatiken \pageref{sp.107}–\pageref{sp.109}.

\so{Schreien.} Einfluss auf die Lautbildung \pageref{sp.444}–\pageref{sp.445}.

\so{Schreiten} \pageref{sp.234}.

\so{Schrift.} Phonetische Systeme \pageref{sp.38}. Schrift und Sprache \pageref{sp.127}–\pageref{sp.135}. Erfindung der S.~\pageref{sp.127}–\pageref{sp.129}. Abkürzungen \pageref{sp.433}. Das Schmieren oder Fledern \pageref{sp.433}.

\sed{\so{Schriftenkunde} und Sprachwissenschaft \pageref{sp.127}.}

\so{Schriftsprache} als Ursache der Sprachgemeinschaft \pageref{sp.55}, \pageref{sp.57}–\pageref{sp.58}. Einmischung von Provinzialismen \pageref{sp.62}. Starrheit \pageref{sp.132}. Die S.~ist eine zweite Volkssprache \pageref{sp.133}\sed{, \pageref{sp.276}.}

\so{Schriftsteller,} angesehene, deren sprachliche Eigenthümlichkeiten \pageref{sp.126}.

\sed{\so{Schriftsysteme,} phonetische \pageref{sp.38}.}

\so{Schuchardt,} Hugo \pageref{sp.279}.

\sed{\so{Schulenburg,} A. C. Grf. v. d., Grammatik d. Zimschian-Sprache \pageref{sp.442}. Murray’s Island, Grammatik \pageref{sp.446}.}

\so{Schulgrammatiken,} lateinische und griechische \pageref{sp.112}, \pageref{sp.113}.

\so{Schulung} des Sprachforschers \pageref{sp.31}–\pageref{sp.53}. Phonetische \pageref{sp.32}–\pageref{sp.39}. Psychologische \pageref{sp.39}–\pageref{sp.46}. Logische \pageref{sp.47}–\pageref{sp.48}. \inlineupdate{Allgemein}{Allgemeine} sprachwissenschaftliche \pageref{sp.48}–\pageref{sp.53}.

\so{Schwankungen} in Form und Gebrauch der Sprache \pageref{sp.97}–\pageref{sp.98}.

\so{Schwedisch.} Orthographie \pageref{sp.133}. Bilabiales w \pageref{sp.188}.

\so{Schweigen.} Onomatopöie tsib, sipp dafür \pageref{sp.154}.

\so{Schweinfurth,} Georg \pageref{sp.69}.

\so{Schweizer-Deutsch.} Lautbildung \pageref{sp.34}.

\so{Schwund} der Laute \pageref{sp.201}.

\so{Schwüre,} als Ausdruck der Versicherung \pageref{sp.183}.

\sed{\so{Sechs} \pageref{sp.279}.}

\so{Sehen,} sequi \pageref{sp.153}. Gucken, schauen, lugen, blicken \pageref{sp.238}.

\so{Sehr,} englisch sore \pageref{sp.153}.

\so{Sein} (esse) \pageref{sp.103}. „Stehen“ \pageref{sp.326}.

\so{Selbstprüfung} des Grammatikers \pageref{sp.83}–\pageref{sp.84}.

\so{Selbstunterricht} in fremden Sprachen nach Lehrbüchern \pageref{sp.73}–\pageref{sp.74}.

\so{Selbstverständlich} ist in der Wissenschaft nichts \pageref{sp.120}.

\so{Seler,} Eduard, Das Conjugationssystem der Mayasprachen \pageref{sp.257}.

\sed{\so{Seltene} Laute und Lautverbindungen, ihre Tendenz zu verschwinden \pageref{sp.192}.}

\so{Selish.} Lautwesen \pageref{sp.34}. Die \pageref{sp.3}. Person in der \inlineupdate{Conjugation}{Conjugatien} ohne Lautzeichen \pageref{sp.393}.

\so{Semitische Sprachen.} Lautwesen \pageref{sp.34}. Sprachen oder Dialekte? \pageref{sp.54}. Vocalismus (Triconsonantismus) \pageref{sp.122}, \pageref{sp.148}–\pageref{sp.149}. Die Schriften \pageref{sp.129}–\pageref{sp.130}. Transscription \pageref{sp.134}. \sed{Verwandtschaft derselben unter einander \pageref{sp.160}.} Freiheit der Composition mangelt \pageref{sp.236}. Zweisylbige Wortstämme, ob zerlegbar? \pageref{sp.242}. Frage- und Verneinungswörter \pageref{sp.244}. Schwinden des Dualis \pageref{sp.254}. Gleiche Denkgewohnheiten \pageref{sp.293}. Die Wurzeln \pageref{sp.295}. \sed{Lautschwankungen \pageref{sp.300}.} Die \pageref{sp.2}. pers. \inlineupdate{sg.}{sing.} und die \pageref{sp.3}. pers. fem. sing. \pageref{sp.307}. \sed{Ursprüngliche Formen \pageref{sp.326}. Worteinheit \pageref{sp.340}. Flektirende oder ächtwortige Sprachen \pageref{sp.342}. Verwandtschaft mit den indogermanischen Sprachen? \pageref{sp.353}. Symbolisation im Vocalismus \pageref{sp.353}.} Perfectische und imperfectische Conjugation \pageref{sp.372}, \pageref{sp.391}. Der Vocalismus stammbildend \pageref{sp.391}. Die \pageref{sp.3}. Pers. masc. in der perfectischen Conjugation \pageref{sp.392}. \sed{Dreiconsonantismus \pageref{sp.410}.} Die Sprachen und die Geistesart der Rasse \pageref{sp.408}–\pageref{sp.411}. Malaien und Semiten \pageref{sp.411}–\pageref{sp.415}. Objectivconjugation \pageref{sp.462}. Satzverknüpfung \pageref{sp.465}.

\sed{{\textbar}{\textbar}513{\textbar}{\textbar}}

\so{Senegambien.} Sprachen von S.~\pageref{sp.150}.

\so{Sequoyah,} Schrifterfinder \pageref{sp.131}.

\so{Serbisch.} Betonung \pageref{sp.431}.

\so{Sērpa} \pageref{sp.157}.

\fed{{\textbar}494{\textbar}}

\so{Sette communi.} Ihre Mundart droht auszusterben \pageref{sp.146}.

\so{Shan} \pageref{sp.149}.

\so{Siamesisch,} s. Thai.

\so{Sibilanten} s. \so{Zischlaute}.

\so{Siebenbürgen.} Mehrsprachigkeit der Sachsen \pageref{sp.70}.

\so{Sievers,} Phonetik \pageref{sp.38}, Anm.

\so{Singen,} singendes Sprechen \pageref{sp.33}, \pageref{sp.34}.

\so{Singpho} \pageref{sp.157}.

\so{Singstimme} \pageref{sp.314}.

\so{Singular} \pageref{sp.101}.

\so{Sinnlichkeit} in der Sprache ausgeprägt \pageref{sp.413}.

\so{Sitte.} Ihr Einfluss auf den Bedeutungswandel \pageref{sp.245}–\pageref{sp.250}.

\sed{\so{Skandinavier,} Rasse \pageref{sp.147}.}

\so{Slang} \pageref{sp.45}.

\so{Slavische Sprachen.} Lautwesen \pageref{sp.34}. Sprachen oder Dialekte? \pageref{sp.54}. Zischlaute \pageref{sp.148}. Präterita auf lu, la, lo \pageref{sp.159}, \pageref{sp.384}. Entwickelung der Gutturalen im Vergleich mit der indoiranischen \pageref{sp.159}, \pageref{sp.164}–\pageref{sp.165}. Der Dualis \pageref{sp.254}. Ein neues grammatisches Geschlecht \pageref{sp.254}. Diminutiva \pageref{sp.277}. \sed{Particip \pageref{sp.341}.} Die Zahlwörter Neun und Zehn \pageref{sp.402}. Betonung \pageref{sp.431}. Aphoristische Redeweise \pageref{sp.465}.

\so{Slavo-lettische Sprachen} s. \so{Litauisch-slavische Sprachen.}

„\so{So}“ als Relativpronomen und = wenn \pageref{sp.100}.

\sed{\so{Sollen} \pageref{sp.103}.}

\so{Somali} \pageref{sp.160}, \pageref{sp.282}, \pageref{sp.307}.

\so{Sonanten} \pageref{sp.186}.

\so{Sonrhai} \pageref{sp.282}.

\so{Soso} \pageref{sp.282}.

\so{Spaltungen} der Sprachen \pageref{sp.138}. Vgl. \so{Dialekte, Sprachstämme.}

\so{Spanisch} \pageref{sp.8}, \pageref{sp.54}. Euphonisches e vor s + Consonanten \pageref{sp.157}. Der Dativ statt des Accusativs \pageref{sp.255}. Arabische Lehnwörter \pageref{sp.266}. Sp. von Nachkommen baskischer, maurischer und germanischer Völker gesprochen \pageref{sp.293}. Diminutiva, Augmentativa u.~s.~w. \pageref{sp.445}.

\so{Spielkarten,} deren Farben und ihre Namen \pageref{sp.233}–\pageref{sp.234}.

\so{Spieltrieb} und Sprachschöpfung \pageref{sp.308} flg., \pageref{sp.361}.

\so{Spirallauf} der Sprachgeschichte \pageref{sp.255}–\pageref{sp.258}.

\so{Spitta Bey} \pageref{sp.276}.

\sed{\so{Sprachaneignung der Kinder} \pageref{sp.67}.}

\so{Sprachbau.} Kenntniss seiner verschiedenen Formen nothwendig zur allgemeinen Sprachwissenschaft \pageref{sp.50}. = Ausdruck der inneren Sprachform \pageref{sp.81}. = Satzbau \pageref{sp.81}–\pageref{sp.82}. Grundgesetze des Spr. in der Grammatik \pageref{sp.87}. \inlineupdate{Ähnlichkeiten}{Aehnlichkeiten} im Spr. als Anzeichen der Verwandtschaft \pageref{sp.149}–\pageref{sp.150}. Der Spr. als Werthmesser \pageref{sp.437} flg. Desgl. als Object der allgemeinen Grammatik \pageref{sp.479}–\pageref{sp.482}.

\so{Sprachbewusstsein, Sprachgefühl} löst zuweilen alte Verbindungen und knüpft neue an \pageref{sp.60}–\pageref{sp.61}; ist bei allen Sprachgenossen im Wesentlichen dasselbe \pageref{sp.64}. Rechtfertigung des synthetischen Systemes der Grammatik durch das Spr. \pageref{sp.97}. Ob das Spr. zwischen Wort- und Formenbildung unterscheide? \pageref{sp.122}. Verhalten gegenüber der Etymologie und Morphologie \pageref{sp.123}. Lautsymbolisches \pageref{sp.123}–\pageref{sp.124}. Gegenüber Archaismen und. Provinzialismen \pageref{sp.125}–\pageref{sp.127}. Gegenüber der Wortabtheilung \pageref{sp.132}. \sed{Spr. und Sandhi \pageref{sp.204}.} Abstumpfung des Spr. \pageref{sp.275}.

\so{Sprache} der Natur, der Steine u.~s.~w. \pageref{sp.2}; der Thiere \pageref{sp.2}; menschliche, deren Begriff \pageref{sp.3} flg. = \inlineupdate{Rede,}{Rede} \sed{\pageref{sp.315} flg.} = Einzelsprache \pageref{sp.3}\sed{, \pageref{sp.8}}; = Sprachvermögen \pageref{sp.4}. – des gemeinen Mannes \pageref{sp.45}. \sed{Deren Zahl \pageref{sp.54}.} Logische, psychologische und räumlich-zeitliche \inlineupdate{Factoren}{Faktoren} \pageref{sp.81}. Die Spr. als Erscheinung und als Mittel \pageref{sp.84}–\pageref{sp.85} (vgl. \so{analytisches} und \so{synthetisches System}). In der Grammatik zugleich Gegenstand und Mittel der Darstellung \pageref{sp.84}–\pageref{sp.85}. Einschmuggelungen, Neuerungen, Einbussen \pageref{sp.126}–\pageref{sp.127}. Spr. und Schrift \pageref{sp.127}–\pageref{sp.135}. Alles dem Wandel unterworfen \pageref{sp.168}–\pageref{sp.169}. \sed{Deren Klang \pageref{sp.192}.} Menschliches Erzeugniss oder göttliches Geschenk? \pageref{sp.303}–\pageref{sp.304}. Die Sprache als Erzeugniss und Erzieherin des Volksgeistes \pageref{sp.387} flg. \inlineupdate{Dgl.}{Desgl.} als Mittel des Gedankenausdruckes \pageref{sp.429}. Voraussetzungen des sprachlichen Ausdrucksbedürfnisses \pageref{sp.437}–\pageref{sp.438}. Arbeit des \fed{{\textbar}495{\textbar}} ganzen Volkes, verhältnissmässiger Reichthum \pageref{sp.470}.

\sed{\so{Sprachengenealogie,} deren Aufgabe \pageref{sp.142} flg.}

\sed{\so{Spracherlernung} \pageref{sp.65}.}

\so{Spracheninseln} \pageref{sp.146}. Vergl. \so{Isolirte Sprachen.}

\sed{\so{Sprachenmischung} \pageref{sp.169}.}

\so{Sprachfamilien} als Gegenstand der Sprachwissenschaft \pageref{sp.9}–\pageref{sp.10}. \sed{Jetziger Stand unseres Wissens darüber \pageref{sp.142}–\pageref{sp.143}.}

\so{Sprachfehler,} auf Denkfehlern beruhend \sed{{\textbar}{\textbar}514{\textbar}{\textbar}} \pageref{sp.44}–\pageref{sp.45}. Ursache von Neuerungen in der Sprache \pageref{sp.45}, \sed{\pageref{sp.184},} \pageref{sp.428}.

\so{Sprachform,} innere = Bildungsprinzip = Sprachgeist \pageref{sp.63}. = Auffassung der logischen, psychologischen und räumlich-zeitlichen Beziehungsarten \pageref{sp.82}. Inwieweit die Vermuthung der Verwandtschaft begründend? \pageref{sp.150}–\pageref{sp.151}. Innere Spr. \pageref{sp.327}–\pageref{sp.348}. \inlineupdate{Äussere}{Aeussere} Spr. \pageref{sp.345}–\pageref{sp.360}. Die Redetheile \pageref{sp.381}–\pageref{sp.385}.

\so{Sprachforscher.} Einige namhafte, deren sonstiger Beruf \pageref{sp.31}. Unterschied zwischen Spr. und Sprachkenner \pageref{sp.88}.

\sed{\so{Sprachforschung,} Verhältniss zur Sprachphilosophie \pageref{sp.11}–\pageref{sp.12}. Schulung zur Sprachforschung \pageref{sp.31}. Einzelsprachliche Spr. \pageref{sp.54} flg. Genealogisch-historistorische Spr. \pageref{sp.136} flg.}

\so{Sprachgefühl,} \sed{Dessen Empfindlichkeit \pageref{sp.298}. Spr.,} s. \so{Sprachbewusstsein.}

\so{Sprachgeist} \pageref{sp.9}. = System der Sprachgesetze \pageref{sp.63}.

\so{Sprachgemeinde, Sprachgemeinschaft} \pageref{sp.8}, \pageref{sp.55}–\pageref{sp.58}.

\so{Sprachgeschichte,} als Aufgabe der Sprachwissenschaft \pageref{sp.9}–\pageref{sp.10}; ergänzt die einzelsprachliche Forschung \pageref{sp.59}–\pageref{sp.60}. \sed{Verhältniss zur einzelsprachlichen Forschung \pageref{sp.140}.} \inlineupdate{Äussere}{Aeussere} und innere \pageref{sp.141}–\pageref{sp.142}. Die äussere \pageref{sp.142}–\pageref{sp.168}. Die innere; deren Aufgaben \pageref{sp.168}–\pageref{sp.171}. Spirallauf \pageref{sp.255}–\pageref{sp.258}. Die Arbeit des Forschers ist mikroskopisch \pageref{sp.283}. Anregungen \inlineupdate{und}{u.} Irrlichter \pageref{sp.289}–\pageref{sp.293}. Allgemeine Spr. \pageref{sp.479}, \pageref{sp.481}.

\so{Sprachgesetze,} bilden ein System; dessen Aneignung \pageref{sp.63}–\pageref{sp.64}.

\so{Sprachgewohnheiten,} individuelle \pageref{sp.97}–\pageref{sp.100}. Beeinflusst durch fremde \pageref{sp.259}–\pageref{sp.260}. Solche der Lautbildung \pageref{sp.270}–\pageref{sp.271}.

\so{Sprachkenntniss} und Sprachwissenschaft \pageref{sp.29}–\pageref{sp.30}, \pageref{sp.50}. Grundsätze \pageref{sp.61}–\pageref{sp.64}.

\so{Sprachkünstler,} Gegensatz zum Sprachforscher \pageref{sp.32}, \pageref{sp.49}.

\so{Sprachlehre,} vgl. \so{Grammatik.} Vollständige \pageref{sp.111}.

\so{Sprachmischung} \fed{\pageref{fp.254}–\pageref{fp.277},} \pageref{sp.259} \sed{flg.}, \pageref{sp.406}–\pageref{sp.408}.

\so{Sprachorgane,} deren Haltung \pageref{sp.34}–\pageref{sp.36}. Auch die Bewegungen der Spr., nicht nur die akustischen Wirkungen, bedingen die Richtigkeit der Aussprache \pageref{sp.187}. Verhalten der Spr. in Ansehung der Euphonik \pageref{sp.197}.

\so{Sprachphilosophie,} deductive, apriorische \pageref{sp.11}–\pageref{sp.12}, \pageref{sp.28}; bei den Chinesen \pageref{sp.19}; bei den Griechen \pageref{sp.20}.

\so{Sprachschatz} = Vorrath an Wörtern und Redensarten, ein Ganzes; Wechselbeziehung zum Sprachbaue \pageref{sp.121}. Wechselvolle Schicksale \pageref{sp.152}. \inlineupdate{Übereinstimmung}{Uebereinstimmungen} bedeutsam für die Urgeschichte \pageref{sp.293}–\pageref{sp.294}. Allgemeine Wortschatzkunde \pageref{sp.481}–\pageref{sp.483}.

\so{Sprachschilderung} \pageref{sp.476}–\pageref{sp.479}.

\sed{\so{Sprachschöpfung} \pageref{sp.303}.}

\sed{\so{Sprachsinn} \pageref{sp.331}.}

\sed{\so{Sprachspielerei} und Sprachvermögen \pageref{sp.311}.}

\so{Sprachstämme,} als Gegenstand der Sprachwissenschaft \pageref{sp.9}–\pageref{sp.10}. Einige bekannte \pageref{sp.142}–\pageref{sp.148}. Erweiterung \pageref{sp.143}–\pageref{sp.144}. Methode der Entdeckung und Erweiterung \pageref{sp.145}–\pageref{sp.154}.

\sed{\so{Sprachstoff} \pageref{sp.63}.}

\so{Sprachtalent} (vgl. \so{Sprachkünstler}) und Sprachwissenschaft \pageref{sp.48}. Steigerung durch \inlineupdate{Übung}{Uebung} \pageref{sp.50}–\pageref{sp.52}, durch Spracherlernung aus Texten \pageref{sp.73}.

\so{Sprachvergleichung,} unmethodische \pageref{sp.154}–\pageref{sp.156}. Technik, Collectaneen \pageref{sp.166}–\pageref{sp.168}. Spr. und Urgeschichte \pageref{sp.293}–\pageref{sp.294}.

\so{Sprachvermögen,} menschliches, als Gegenstand der Sprachwissenschaft \pageref{sp.10}–\pageref{sp.11}. Das Spr. ist, insoweit es sich gleich bleibt, Gegenstand der einzelsprachlichen, insoweit es sich verändert, Gegenstand der sprachgeschichtlichen Forschung \pageref{sp.138}–\pageref{sp.139}. \sed{Phonetische\footnote{ \edins{[\textit{Das Adjektiv zeigt mit seiner femininen Form, dass} Spr. \textit{in} Sprachvergleichung \textit{aufzulösen ist und der Verweis hier falsch eingeordnet wurde}]}} Spr. versagt gegenüber den Lautschwankungen \pageref{sp.300}.} Grundlagen des menschlichen Spr. \pageref{sp.303}–\pageref{sp.317}. Physische Grundlagen \pageref{sp.304}–\pageref{sp.307}. Psychische Grundlagen \pageref{sp.307}–\pageref{sp.313}.

\sed{\so{Sprachverwandtschaft} \pageref{sp.9}.}

\so{Sprachwissenschaft.} Begriff \pageref{sp.1} flg. Aufgaben \inlineupdate{\pageref{fp.2}–\pageref{fp.12}.}{\pageref{sp.7}–\pageref{sp.13}.} Nicht durch die genealogisch-historische Forschung erschöpft \pageref{sp.8}–\pageref{sp.9}, \pageref{sp.11}. Ihre Stellung \pageref{sp.13}–\pageref{sp.17}. Anregungen zur Spr., Geschichtliches \pageref{sp.17}–\pageref{sp.30}. Verschiedene Richtungen \pageref{sp.29}–\pageref{sp.31}. \sed{Spr. und Einzelsprachen \pageref{sp.49}–\pageref{sp.50}. Geschichte der Spr. \pageref{sp.137}–\pageref{sp.138}.}

\so{Sprachwürderung} \inlineupdate{\pageref{fp.371}–\pageref{fp.457}.}{\pageref{sp.387} flg.}

\fed{{\textbar}496{\textbar}}

\so{–st} Endung der \pageref{sp.2}. pers. sing. in den germanischen Sprachen \pageref{sp.203}.

\so{Staat.} Frühe Anfänge des staatlichen Lebens \pageref{sp.307}.

\so{Stammbaumtheorie} \pageref{sp.163}–\pageref{sp.165}.

\so{Standard-Alphabet,} Lepsius’sches \pageref{sp.38}, \pageref{sp.69}.

\so{Standessprachen} \pageref{sp.45}, \sed{\pageref{sp.237},} \pageref{sp.288}–\pageref{sp.289}. Der \inlineupdate{Offficiere}{Offiziere} \pageref{sp.184}.

\so{Standesverhältnisse.} Sociale \inlineupdate{Modalität}{Modalität,} \pageref{sp.474}–\pageref{sp.475}.

\so{Standpunkt,} geistiger in der Spr. \pageref{sp.325}.

\so{Starke,} die \pageref{sp.237}.

\sed{{\textbar}{\textbar}515{\textbar}{\textbar}}

\so{Status constructus} \pageref{sp.114}.

\so{Steigerung,} als syntaktische Kategorie \pageref{sp.104}.

\sed{\so{Steinen,} K. von den \pageref{sp.194}.}

\so{Steinhaufen} der Steppennomaden \pageref{sp.127}.

\so{Steinthal,} H., Definition des Denkens \pageref{sp.6}. Wissenschaftlicher Psycholog \pageref{sp.46}. „Die \inlineupdate{Mande-Negersprachen“,}{Mande-Negersprachen“} \sed{\pageref{sp.451}.} Eintheilung dieses Buches \pageref{sp.86}. Die innere Sprachform \pageref{sp.150}, \pageref{sp.327}, \pageref{sp.335} flg., \pageref{sp.343}. \sed{„Der Ursprung der Sprache“ \pageref{sp.304}. „Sprachphil. Werke W. v. H.“ \pageref{sp.330}.} Beurtheilung des geistigen Werthes der Sprachen \pageref{sp.388}\sed{, \pageref{sp.394}.} Beurtheilung der uralaltaischen Vocalharmonie \pageref{sp.402}. Schilderung des Arabischen \pageref{sp.409}. De pronomine relativo \pageref{sp.481}.

\so{Stellungsgesetze,} inwieweit in den Sprachfamilien typisch? \pageref{sp.149}. Kennzeichnend für die äussere und innere Sprachform \pageref{sp.359}–\pageref{sp.360}. Freiheiten \pageref{sp.363}. Psychologisches Subject und Prädicat \pageref{sp.373}–\pageref{sp.376}. St. und die Stimmungsmimik \pageref{sp.380}–\pageref{sp.381}.

\so{Stil,} individueller, = Sprach- und Denkgewohnheiten des Einzelnen \pageref{sp.98}, \pageref{sp.105}; nationaler \pageref{sp.105}–\pageref{sp.106}. \sed{St. und Werthbestimmung der Sprache \pageref{sp.475}–\pageref{sp.476}.}

\so{Stilisirung} als Merkmal der Schrift \pageref{sp.128}–\pageref{sp.129}. Desgl. der Sprache \pageref{sp.315}.

\so{Stilistik} als Theil der Grammatik \pageref{sp.104}–\pageref{sp.106}. \sed{Einfluss des Flexionssystems auf die St. \pageref{sp.400}–\pageref{sp.401}.}

\so{Stimmlage,} verschieden nach Alterund Geschlecht \pageref{sp.311}.

\so{Stimmung} des Redenden im grammatischen Ausdrucke \pageref{sp.95}–\pageref{sp.96}.

\so{Stimmungsmimik,} Ausspracheweise \pageref{sp.376}–\pageref{sp.380}. St. und Stellungsgesetz \pageref{sp.380}–\pageref{sp.381}.

\so{Stocken} in der Rede und Sandhi \pageref{sp.203}.

\so{Stoff} und Form in der Sprache \pageref{sp.122}–\pageref{sp.123}, \pageref{sp.324}–\pageref{sp.360}. Der Stoff \pageref{sp.324}–\pageref{sp.327}.

\sed{\so{Stoffwurzeln} nach Fr. Müller \pageref{sp.338}.}

\so{Stoiker,} ihr Verdienst um die Grammatik \pageref{sp.20}.

\so{Stoll,} Otto \pageref{sp.31}, dessen Arbeiten über die Maya-Sprachen \pageref{sp.52}, \pageref{sp.257}.

\so{Streit} \pageref{sp.307}, \pageref{sp.310}. Schreiendes Sprechen dabei \pageref{sp.374}.

\sed{\so{Studentenjargon} \pageref{sp.289}.}

\so{Subject,} Prädicat, Object; die Lehre davon im synthetischen Systeme \pageref{sp.101}–\pageref{sp.102}. Psychologisches S.~\pageref{sp.102}–\pageref{sp.103}, \pageref{sp.365}–\pageref{sp.373}.

\so{Subjectivität} und Objectivität als Gegenstände des sprachlichen Ausdruckes \pageref{sp.438}. Die S.~\pageref{sp.472}–\pageref{sp.475}.

\so{Subjectssatz} \pageref{sp.104}.

\so{Substantiva.} Ihr Werth für den Verwandtschaftsnachweis \pageref{sp.153}–\pageref{sp.154}. Als logische und grammatische Kategorie, im \inlineupdate{Gegensatze}{Gegensatz} zu Adjectivum und Verbum \pageref{sp.381}–\pageref{sp.385}. Namen für Theile oder Beziehungen, einen ergänzenden Genitiv verlangend \pageref{sp.441}.

\so{Sudân.} Sprachen des westlichen S.~\pageref{sp.150}.

\so{Suffigirende Sprachen} \pageref{sp.149}, \pageref{sp.349}.

\so{Suffixe} \pageref{sp.348}.

\so{Sumatra} \pageref{sp.147}.

\so{Sumerier} \pageref{sp.18}, \sed{\pageref{sp.294},} \pageref{sp.389}.

\sed{\so{Suomi} \pageref{sp.271}.}

\so{Superlativ} \pageref{sp.103}.

\so{Suppe} \pageref{sp.264}.

\so{Susu} \pageref{sp.150}.

\so{Sylbenschrift,} japanische \pageref{sp.130}. Die altsemitische \pageref{sp.131}. Die der Tscheroki und der Vei \pageref{sp.131}.

\so{Symbole,} Bilder und Schrift, ihr Unterschied \pageref{sp.128}.

\so{Symbolisation} \pageref{sp.353}.

\sed{Synonima,\edins{{\textbar}{\textbar}Synonyma,} deren Entähnlichung \pageref{sp.238}.}

\so{Synonymik,} grammatische = synthetisches System \pageref{sp.94}; wurzelt im Sprachgefühle \pageref{sp.97}–\pageref{sp.98}. Wornach dieses unterscheidet \pageref{sp.98}–\pageref{sp.99}. Grund sätze \pageref{sp.99}–\pageref{sp.100}. Lexikalische, synonymisches Wörterbuch \pageref{sp.123}–\pageref{sp.124}\sed{, \pageref{sp.179}.}

\so{Syntax.} \sed{Keine Grammatik ohne Syntax \pageref{sp.84}.} Ihre Darstellung in Paradigmen und Formeln \pageref{sp.117}–\pageref{sp.118}. Vergleichende in der historischen Sprachforschung \pageref{sp.137}–\pageref{sp.138}.

\so{Synthese} in der Sprache \pageref{sp.81}.

\fed{{\textbar}497{\textbar}}

\so{Synthetisches System} der Grammatik \pageref{sp.85}–\pageref{sp.86}; hat auf das analytische zu folgen \pageref{sp.86}. Inhalt und Anordnung \pageref{sp.94}–\pageref{sp.104}.

\so{Syrjänisch} hat keine Vocalharmonie \pageref{sp.149}, \pageref{sp.403}. Russische Hülfswörter \pageref{sp.273}.

Σῦς – ὗς \pageref{sp.202}.

\so{System} der Grammatik: Analytisches \pageref{sp.85}–\pageref{sp.86}, \pageref{sp.88}–\pageref{sp.93}; synthetisches \pageref{sp.85}–\pageref{sp.86}, \pageref{sp.94}–\pageref{sp.104}; gemischtes \pageref{sp.91}–\pageref{sp.92}. S.~und Methode in der Grammatik \pageref{sp.109}–\pageref{sp.110}.

\subsection*{T.}\label{reg.T}\pdfbookmark[1]{T.}{reg.T}

\so{Tabelle.} Tabellarische Form, die ideale Form einer Grammatik \pageref{sp.85}. In grammatischen Auszügen \pageref{sp.108}.

\so{Tabuwesen} (Tapu) l\pageref{sp.66}, \pageref{sp.245}.

\so{Tacitus} \pageref{sp.20}.

\so{Tagalisch} \pageref{sp.257}, \pageref{sp.266}\sed{, \pageref{sp.363}}. Bildsamkeit \pageref{sp.349}.

\sed{{\textbar}{\textbar}516{\textbar}{\textbar}}

\so{Tahitisch.} Zweierlei Genitiv \pageref{sp.463}.

\so{Talaing,} s. \so{Peguanisch.}

\so{Tamaschek} \pageref{sp.160}.

\so{Tamulisch} \sed{\pageref{sp.344}}. Attributiver Satzbau \pageref{sp.453}.

\so{Tapu} \fed{\pageref{fp.46}.}. Vgl. \so{Tabuwesen.}

\sed{\so{Taraskisch} \pageref{sp.390}.}

\so{Tasmanier} \pageref{sp.280}.

\so{Taufen} \pageref{sp.231}, \pageref{sp.245}.

\so{Techmer,} Fr., seine Definition der Articulation und des Lautes \pageref{sp.4} flg. Phonetik \pageref{sp.38}.

\so{Teda} \pageref{sp.282}.

\so{Teichelmann,} C. G. und C. W. Schürmann \pageref{sp.194}.

\sed{\so{Telugu} \pageref{sp.344}.}

\so{Temne} \pageref{sp.150}, \pageref{sp.282}.

\so{Temporalformen} \pageref{sp.383}.

\so{Tendenzen} in der Entwickelung der Sprachen \pageref{sp.178}.

\so{Tenues} in manchen Sprachen nicht von den Mediis unterschieden \pageref{sp.188}, \pageref{sp.194}\sed{, \pageref{sp.314}}. Tenuis und Media aufeinanderfolgend \pageref{sp.201}\sed{, \pageref{sp.298}}.

\so{Terminologie,} grammatische \pageref{sp.114}–\pageref{sp.116}.

\so{Texte} zur praktischen Spracherlernung \pageref{sp.51}. Erlernung und Erforschung der Sprachen aus solchen \pageref{sp.73}–\pageref{sp.75}. Besonders geeignete \pageref{sp.106}.

\so{Th} (θ) im Germanischen in ein hartes und weiches gespalten \pageref{sp.190}.

\so{Thai,} Wortschöpfungen \pageref{sp.42}. \sed{„Nichtwortige“ Sprache \pageref{sp.342}.} Prädicativer Bau \sed{\pageref{sp.149},} \pageref{sp.151}. \sed{Euphonik \pageref{sp.199}. Doppelungsformen \pageref{sp.224}.} \inlineupdate{Einsylbigkeit}{Einsilbigkeit} und Isolirung nicht ursprünglich \pageref{sp.257}. \sed{Phonetische Gestalt \pageref{sp.135}.} Prädicativer Satzbau \pageref{sp.452}.

\inlineupdate{\so{Thai-sprachen,}}{\so{Thai-Sprachen,}} Familie der –; deren Bau \pageref{sp.149}, \pageref{sp.257}. Innere Sprachform \pageref{sp.150}. Me = Du und Mutter \pageref{sp.306}.

\so{Thaksya} \pageref{sp.157}.

\largerpage[1]\so{Thätigkeit} – Eigenschaft – Ding, vorbildlich für grammatische Kategorien \pageref{sp.381}–\pageref{sp.385}.

\so{Theater.} Bühnenmässige Sprache \pageref{sp.127}.

\sed{\so{Théâtre français,} Einfluss auf die Sprachbildung \pageref{sp.127}.}

\sed{\so{Thematische Wurzeln} \pageref{sp.296}.}

θεός – deva, deus \pageref{sp.186}, \pageref{sp.217}.

\so{Thiere.} Deren Sprache \pageref{sp.3}. Th. und Theile solcher, deren Namen in übertragener, vergleichsweiser Anwendung \pageref{sp.40}–\pageref{sp.42}. \inlineupdate{Onomatopoetische}{Onomatopoëtische} Namen von Th. \pageref{sp.208}. Th., welche menschliche Sprachlaute nachahmen \pageref{sp.304}.

\so{Thierfabel} \pageref{sp.315}.

\so{Thiersprachen} \pageref{sp.3}.

\so{Tibetisch.} Lautwesen \pageref{sp.34}. Casus activus und neutro-passivus \pageref{sp.102}, \pageref{sp.151}. \sed{Hilfsmittel für die indochinesische Sprachvergleichung \pageref{sp.141}.} Wortaccent in einem Dialekte \pageref{sp.148}. Bau \pageref{sp.150}. Innere Sprachform \pageref{sp.151}. Alterthümlichkeit des Lautwesens \pageref{sp.157}. Anlaut dp \pageref{sp.199}. Der Formativlaut s \pageref{sp.214}. Nachbildungen indisch-buddhistischer Wörter \pageref{sp.231}, \pageref{sp.271}. Wandelbare Verbalstämme \pageref{sp.257}, \pageref{sp.391}. \sed{Vocalwandel in der Formenbildung \pageref{sp.354}. Innere Flexion \pageref{sp.484}.}

\so{Tifinagh,} berberische Schrift \pageref{sp.131}.

\sed{\so{Tigre} \pageref{sp.282}.}

\so{Tiki-tiki} \pageref{sp.187}.

\so{Titanen.} „Sprachstamm der T.“ \pageref{sp.144}.

\sed{\so{Titel.} T. und Höflichkeitsformen, Stellung in der Grammatik \pageref{sp.107}. Neue Worthildungen durch Verstümmelung von T. \pageref{sp.207}.}

\so{Toba,} s. \so{Batta.}

\sed{\so{Tolteken} \pageref{sp.389}.}

\so{Töne} in der menschlichen Ursprache \pageref{sp.313}–\pageref{sp.315}. T. in der Sprache als Ausdruck der Stimmung und als lexikalische und grammatische Factoren \pageref{sp.376}–\pageref{sp.380}.

\so{Tonfall,} verschiedener in verschiedenen Sprachen \pageref{sp.33}. Bestandtheil der Sprache \pageref{sp.34}.

\sed{\so{Torrend,} J. \pageref{sp.420}.}

\sed{\so{Totonakisch} \pageref{sp.390}.}

\so{Toumpakewa-Alifurisch.} Bildsamkeit \pageref{sp.350}.

\so{Toussaint-Langenscheidt,} Methode \pageref{sp.110}, \pageref{sp.111}.

\so{Trägheit,} s. \so{Bequemlichkeit.}

\so{Transscription} \pageref{sp.38}–\pageref{sp.39}; \pageref{sp.134}.

\fed{{\textbar}498{\textbar}}

\so{Tredeci communi.} Ihre Mundart droht auszusterben \pageref{sp.146}.

\fed{Treitschke, Heinr. von \pageref{sp.258}.}

\sed{\so{Trüöng-Vinh-Ky,} Gramm. Annamite \pageref{sp.223}.}

\sed{\so{Tscham,} Lautwesen \pageref{sp.269}.}

\sed{\so{Tschechisch,} Accent \pageref{sp.212}.}

\so{Tscheroki.} Syllabar \pageref{sp.131}. Beispiel der Conjugation \pageref{sp.357}. Gesittungsfähigkeiten des Volkes \pageref{sp.428}.

\so{Tschi} s. \so{Aschanti.}

\so{Tschuktschisch} \pageref{sp.147}.

\so{Tshahta} (Choctaw) \pageref{sp.358}.

\so{Tuareg} \pageref{sp.160}\sed{, \pageref{sp.301}, \pageref{sp.306}.}

\sed{\so{Tungusen} (Mandschu) \pageref{sp.283}.}

\sed{\so{Tungusisch} \pageref{sp.349}.}

\so{Tupi.} Formenbildung \pageref{sp.328}.

\so{Turanische Sprachen} \sed{\pageref{sp.144},} \pageref{sp.155}.

\so{Türken,} kaukasischer Rassetypus \pageref{sp.147}.

\so{Türkisch.} Tonfall \pageref{sp.34}. Die Casus \pageref{sp.115}. Die \sed{{\textbar}{\textbar}517{\textbar}{\textbar}} arabische Schrift \pageref{sp.129}, \pageref{sp.131}. \sed{Accent \pageref{sp.212}.} Sonstige arabische Einflüsse. \pageref{sp.271}. Agglutinative Bildsamkeit \pageref{sp.337}. Die \pageref{sp.3}. Person in der Conjugation ohne Lautzeichen \pageref{sp.383}.

\so{Tuuk,} H. N. van der, \pageref{sp.193}, \pageref{sp.379}.

\so{Tyrrhenier} s. \so{Messapier.}

\subsection*{U.}\label{reg.U}\pdfbookmark[1]{U.}{reg.U}

\so{Ueberflüssiges.} Abneigung der Sprachen dagegen \pageref{sp.100}–\pageref{sp.101}, \pageref{sp.238}. Scheinbar Ü., \inlineupdate{Äusserung}{Aeusserung} des Formungstriebes \pageref{sp.361}–\pageref{sp.365}.

\so{Uebergangsformen,} sprachliche \pageref{sp.14}.

\so{Uebersetzung.} Missbrauch damit im Sprachunterrichte \pageref{sp.71}–\pageref{sp.72}. Bibelübersetzungen als Unterlagen zur Spracherforschung \pageref{sp.73}. Bedenkliche Unterlagen \pageref{sp.105}. Treffende \inlineupdate{Uebers.}{Uebersetzung} als grammatisches Darstellungsmittel \pageref{sp.120}. Zwischenzeilige \pageref{sp.405}.

\so{Uebertragung} als Mittel der Benennung \pageref{sp.40}–\pageref{sp.42}\sed{, \pageref{sp.233}}. \sed{Ü. von Eigenschaften lebender Wesen auf Lebloses \pageref{sp.316}.}

\so{Uebertreibungen} in der Articulation \pageref{sp.183}. \sed{Anlass zum Lautwandel \pageref{sp.191}.} In der Wahl der Ausdrücke, wenn üblich, drücken die Bedeutung herunter \pageref{sp.243}.

\so{Ueberzeugung} \pageref{sp.263}.

\so{Uebungsstücke} in der Grammatik \fed{\pageref{sp.115}, \pageref{sp.126}.} \sed{\pageref{sp.111} flg.}

\so{Uigurisch.} Schrift \pageref{sp.129}.

\so{Umale} \pageref{sp.282}.

\sed{\so{Umgang,} mündlicher, als Mittel zur Spracherlernung \pageref{sp.65} flg.}

\so{Umladung} der Formativa \pageref{sp.214}.

\so{Umlaut} \pageref{sp.200}, \pageref{sp.401}.

\so{Umschreibungen,} umschreibende Ausdrucksweisen \pageref{sp.183}.

\so{Umstand.} Umstandssätze \pageref{sp.104}.

\so{Unbestimmtheit} und Allgemeinheit \pageref{sp.326}.

\so{–ung} \pageref{sp.122}.

\so{Ungarisch} s. \so{Magyarisch.}

\so{Ungebildete.} Ihre Handhabung der Muttersprache \pageref{sp.45}–\pageref{sp.46}. Sie lernen leicht fremde Sprachen \pageref{sp.67}.

\sed{\so{Ungeformtes} in der Sprache \pageref{sp.331}.}

\sed{\so{Ungeformte} Sprachäusserungen \pageref{sp.321}-\pageref{sp.322}.}

\so{Universalia sunt nomina.} Ob auf die Einzelsprachen. anwendbar? \pageref{sp.58}.

„\so{Unpass, unpässlich}“ \pageref{sp.126}–\pageref{sp.127}.

\so{Unregelmässige Bildungen.} Deren gedächtnissmässige Aneignung \pageref{sp.64}. Wie sie möglich waren \pageref{sp.211}–\pageref{sp.212}. \sed{U. in der Lautbildung \pageref{sp.187}.}

\so{Unterdialekt} s. \so{Dialekt.}

\so{Unterricht,} methodischer in fremden Sprachen \pageref{sp.71}–\pageref{sp.75}.

\so{Unveränderlichkeit} nicht nothwendigerweise der Zustand der ältesten Wörter menschlicher Sprache \pageref{sp.255}.

\sed{\so{Unverbrüchlichkeit} der Lautgesetze \pageref{sp.186}.}

\so{Ural-altaische Sprachen.} Vocalharmonie \pageref{sp.132}, \pageref{sp.199}, \fed{\pageref{sp.382}, \pageref{sp.383}.} \sed{\pageref{sp.402}}. Ihre Verbreitung \pageref{sp.142}. Suffigirender Bau \pageref{sp.149}. Freiheit der Wortzusammensetzung mangelt \pageref{sp.236}. Gleiche Denkgewohnheiten \pageref{sp.293}. \sed{Suffigirender Bau \pageref{sp.349}. Harmoniegesetze \pageref{sp.350}.} Die Conjugation prädicativ empfunden \pageref{sp.391}. \sed{Vocalharmonie \pageref{sp.402}.} Vergleichung mit den malaischen \pageref{sp.415}–\pageref{sp.420}. Satzverknüpfung \pageref{sp.465}.

\so{Urdialekt} \pageref{sp.141}.

\sed{\so{Ureinheit} der Sprachen (deren Möglichkeit) \pageref{sp.143}–\pageref{sp.144}.}

\so{Urdu} s. \so{Hindustani.}

\so{Urgeschichte} der Volksstämme, durch Sprachvergleichung ermittelt \pageref{sp.293}–\pageref{sp.294}.

\so{Urheber} eines passiven Verbums \pageref{sp.102}.

\so{Ursprache,} menschliche, Versuche sie zu reconstruiren \pageref{sp.67}. U. eines Stammes \pageref{sp.141}. Indogermanische \pageref{sp.184}. Deren Lautwesen \pageref{sp.186}. \sed{Deren Sandhi \pageref{sp.202}.} Menschliche U., nicht schlechthin einsylbig und isolirend \pageref{sp.255}\sed{, \pageref{sp.314}}. Ihre Laute und Töne \pageref{sp.313}–\pageref{sp.315}.

\sed{\so{Urverwandtschaft} der Sprachen \pageref{sp.143}–\pageref{sp.144}.}

\so{Urtheil,} vollständiges oder unvollständiges \pageref{sp.320} flg.

\sed{\so{Urvocale} \pageref{sp.314}.}

\sed{\so{Urvolk,} dessen Sitz \pageref{sp.163}.}

\so{Uslar,} P. Freiherr von \pageref{sp.69}.

\sed{\so{Ussel} \pageref{sp.397}.}

\fed{{\textbar}499{\textbar}}

\subsection*{V.}\label{reg.V}\pdfbookmark[1]{V.}{reg.V}

\so{Varo,} Franc., chinesischer Grammatiker \pageref{sp.25}.

\so{Varro} \pageref{sp.21}.

\so{Vater.} Wörter dafür \pageref{sp.153}.

\so{Vater,} Joh. Severin \sed{\pageref{sp.15},} \pageref{sp.28}, \pageref{sp.31}.

\so{Veden} \pageref{sp.22}\sed{, \pageref{sp.361}}.

\so{Vei.} Syllabar \pageref{sp.131}. Sprache \pageref{sp.150}, \pageref{sp.282}.

\so{Velare} \pageref{sp.36}.

\so{Veneter,} verschwunden \pageref{sp.146}.

\so{Verallgemeinerung,} voreilige \pageref{sp.77}.

\so{Veränderungen} in der Sprache \sed{\pageref{sp.58}–\pageref{sp.59},} \pageref{sp.168}–\pageref{sp.172}. Vergl. \so{Sprachgeschichte.}

\so{Veranschaulichung.} Periphrastische Ausdrucksweisen \pageref{sp.239}–\pageref{sp.243}.

\sed{{\textbar}{\textbar}518{\textbar}{\textbar}}

\so{Verba.} Ihr Werth für den Verwandtschaftsnachweis \pageref{sp.152}. Incorporation \pageref{sp.354}–\pageref{sp.359}. V. – Adjectiva – Substantiva als logische und grammatische Kategorien \pageref{sp.381}–\pageref{sp.385}. Doppelte Auffassung des Verbalbegriffes \pageref{sp.382}–\pageref{sp.385}.

\so{Verbalnomina} \pageref{sp.466}, \pageref{sp.467}.

\so{Verbindungen,} etymologische, alte gelöst und neue angeknüpft \pageref{sp.61}. Verbindung des Stoffes in der formenden Rede \pageref{sp.324}.

\so{Verbum} neutrum \pageref{sp.102}. V. illativum, locativum, instrumentale \pageref{sp.416}. V. substantivum in den uralaltaischen Sprachen \pageref{sp.416}.

\so{Verdeutlichung,} den Bedeutungswandel beeinflussend \pageref{sp.239}–\pageref{sp.243}.

\so{Verdichtungsprocess} in der Sprache \pageref{sp.433}, \pageref{sp.436}.

\so{Vereinfachung} der Sprache durch falsche Analogien \pageref{sp.211}–\pageref{sp.212}.

\so{Vereinsamung, Vereinzelung} der Völker und Sprachen \pageref{sp.176}–\pageref{sp.179}.

\so{Vereinzelte} Sprache \pageref{sp.176}–\pageref{sp.179}.

\so{Verengung} der Bedeutung, s. \so{Bedeutungswandel.}

\so{Vererblichkeit} der Beanlagung zu einer gewissen Form des Sprachbaues? \pageref{sp.61}.

\sed{\so{Verflüchtigung} im Lautwandel \pageref{sp.191}. V. im Laut \pageref{sp.206}.}

\so{Vergeistigung} der Bedeutung von Wörtern und Wortformen \pageref{sp.403}.

\so{Vergleich,} Gleichniss, als Mittel der Benennung \pageref{sp.40}–\pageref{sp.43}. Gleichheit, \inlineupdate{Ähnlichkeit}{Aehnlichkeit} als grammatische Kategorie \pageref{sp.103}. Einfluss auf den Bedeutungswandel \pageref{sp.234}. Worauf beruht die Volksthümlichkeit und Aufnahme in’s Sprachgut? \pageref{sp.314}.

\inlineupdate{\so{Vergleichnng,}}{\so{Vergleichung,}} \so{vergleichende Wissenschaften} \pageref{sp.10}.

\sed{\so{Verhältnisswörter,} Bedeutung für die Sprachform \pageref{sp.339}.}

\so{Verkehr.} Die Sprache dient ihm \fed{\pageref{sp.191}.} \sed{\pageref{sp.381}, \pageref{sp.319}. V. bedingt Neuschöpfungen von Begriffen \pageref{sp.230}. V. als sprachgeschichtliche Macht \pageref{sp.259} flg.}

\inlineupdate{\so{Verkehrscentren.}}{\so{Verkehrszentren.}} Verschliff der Sprache in solchen \pageref{sp.184}.

\sed{\so{Verkürzungen} der Worte \pageref{sp.207}.}

\so{Verlust} von Wörtern, Redensarten oder grammatischen \inlineupdate{Formenmitteln}{Formenmittel} \sed{\pageref{sp.230},} \pageref{sp.227}–\pageref{sp.258}. V. grammatischer Kategorien \pageref{sp.253}–\pageref{sp.255}.

\so{Verneinung} \pageref{sp.103}.

\so{Verner’sches Gesetz} \pageref{sp.171}, \pageref{sp.186}.

\sed{\so{Vernier} gramm. arabe \pageref{sp.223}.}

\so{Verschiebungen} der Völker- und Sprachgrenzen \pageref{sp.293}.

\so{Verschliff} des Lautwesens, dessen weitere Wirkungen \pageref{sp.178}.

\so{Verschmelzung} der Laute \pageref{sp.200}, \pageref{sp.351}.

\sed{\so{Verschwinden} von Lauten \pageref{sp.200}.}

\so{Versmass,} die Stimmung symbolisirend \pageref{sp.227}.

\so{Verstand} und Gedächtniss beim grammatischen Unterrichte \pageref{sp.110}.

\so{Verständigung, Verständniss.} Die Sprache als Mittel zur ersteren, die Sprachgemeinschaft durch letzteres bedingt \pageref{sp.54}–\pageref{sp.58}. Stumme Verständigung \pageref{sp.67}. Als Zweck der Sprache, deren Entwickelung beeinflussend \pageref{sp.182}\sed{, \pageref{sp.360},} vergl. Verdeutlichung.

\so{Verstärkung} des Ausdruckes, den Bedeutungswandel beeinflussend \pageref{sp.239}–\pageref{sp.243}.

\sed{\so{Verstümmelung} der Worte u. Articulation \pageref{sp.207}.}

\sed{\so{Verwahrlosung} in der Articulation \pageref{sp.205}. V. der Sprache \pageref{sp.260}.}

\so{Verwandtschaft} der Sprachen. Ihr Nachweis \pageref{sp.142}–\pageref{sp.168}. Bedeutet theilweise \inlineupdate{Ähnlichkeit}{Aehnlichkeit} und theilweise Verschiedenheit \pageref{sp.148}. Voll- und Halbbürtige \pageref{sp.158}–\pageref{sp.159}, \pageref{sp.278}–\pageref{sp.283}.

\sed{\so{Verwandtschaftsbeweis} der Sprachen \pageref{sp.154}–\pageref{sp.156}.}

\so{Verwandtschaftsgrade} der Sprachen \pageref{sp.9}. Namen der V. in übertragener Bedeutung \pageref{sp.41}–\pageref{sp.42}, \pageref{sp.145}.

\so{Verwunderung} als Ursache des wissenschaftlichen Strebens \pageref{sp.17}.

\so{Vielheit} als grammatische Kategorie \pageref{sp.101}.

\fed{{\textbar}500{\textbar}}

\so{Violette,} L., Dictionnaire Samoa-français-anglais \pageref{sp.463}.

\so{Virâma} in der indischen Schrift \pageref{sp.131}.

\so{Viti} s. \so{Fidschi.}

\so{Vlämisch} \pageref{sp.54}.

\sed{\so{Vocabularien} \pageref{sp.107} ff.}

\so{Vocabularium Catharinae} \pageref{sp.25}.

\so{Vocale.} Vocalreihen \pageref{sp.36}. Dumpfere und hellere in Onomatopöien nach dem Vorbilde der Naturlaute \pageref{sp.255}. \sed{V. in der Usprache \pageref{sp.313}–\pageref{sp.314}.} Symbolische Bedeutsamkeit \pageref{sp.379}.

\so{Vocalharmonie,} dem Gefühle für die Worttrennung förderlich \pageref{sp.132}. Im Mandschu \pageref{sp.199}. Beeinflussung der Suffixe \pageref{sp.350}, \pageref{sp.352}.

\so{Vocalismus,} symbolischer in einer Kindersprache \pageref{sp.65}. Desgl. im Deutschen und im Batta \pageref{sp.379}.

\so{Vocalsymbolik,} vergl. \so{Vocalismus.} – Beispiel aus verschiedenen Sprachen \pageref{sp.408}–\pageref{sp.409}.

\sed{{\textbar}{\textbar}519{\textbar}{\textbar}}

\so{Vocativ} \pageref{sp.321}.

\so{Voces verbi} s. \so{genus verbi.}

\so{Vögel,} sprechende \pageref{sp.304}.

\so{Völkerkunde} \inlineupdate{und}{u.} Sprachwissenschaft \pageref{sp.13}–\pageref{sp.14}.

\so{Völkermischungen} \inlineupdate{siehe}{s.} \so{Sprachmischung, Aussterben der Sprachen, Entlehnungen, Mischsprachen.}

\so{Volksetymologie} \pageref{sp.216}, \pageref{sp.267}.

\sed{\so{Volksgeist und Sprache} \pageref{sp.387}.}

\sed{\so{Volksindividualität,} deren Einfluss auf die Sprache \pageref{sp.121}.}

\sed{\so{Volkssprache} \pageref{sp.55}–\pageref{sp.56}, \pageref{sp.276}.}

\sed{\so{Volksthum,} dessen Einfluss auf d. Sprache \pageref{sp.44}.}

\sed{\so{Vollkommenheit} der Sprache \pageref{sp.393}–\pageref{sp.394}.}

\sed{\so{Vollständigkeit} der Grammatik \pageref{sp.111}.}

\so{Voreiligkeit, Vorgreiflichkeit} in den indogermanischen Sprachen \pageref{sp.401}.

\so{Vorfahren,} \inlineupdate{deren}{der} Sprache \pageref{sp.66}.

\so{Vorgeschichte} der Sprachen, \inlineupdate{siehe}{s.} \so{Sprachgeschichte.}

\sed{\so{Vorhaben} \pageref{sp.103}.}

\sed{\so{Vorherrschende} Vocale \pageref{sp.34}.}

\so{Vorschulen,} grammatische \pageref{sp.112}.

\so{Vorstellungen,} neue, verlangen Benennung. Woher diese entlehnt? \pageref{sp.40}–\pageref{sp.42}. Die Welt von V. ausgedrückt im Sprachschatze \pageref{sp.121}, \pageref{sp.124}–\pageref{sp.125}. \sed{Aehnlichkeit der V., Bedeutung für den Bedeutungswandel \pageref{sp.232}.} Lückenhafte V. bei der Frage \pageref{sp.310}–\pageref{sp.311}. \sed{V. als Stoff der Rede \pageref{sp.325}.}

\subsection*{W.}\label{reg.W}\pdfbookmark[1]{W.}{reg.W}

W, bilabiales und labiondentales im Deutschen \pageref{sp.188}.

\sed{\so{Wagap} \pageref{sp.390}.}

\sed{\so{Wagner,} Richard, Sprachspielereien \pageref{sp.275}.}

\sed{\so{Wahlverwandtschaften} in der Wortbildung \pageref{sp.222}.}

\sed{\so{Wallmann,} J. C., Namaquasprache \pageref{sp.214}.}

\so{Wallonisch} \pageref{sp.64}.

\so{Wandala} \pageref{sp.282}.

\so{Wanderungen} \inlineupdate{der}{deren} Völker \pageref{sp.147}.

„\so{Ward}“ und „\so{wurde}“ \pageref{sp.100}.

\sed{\so{Wegener,} Ph. \pageref{sp.46}.}

„\so{Weib}“ und „\so{Du}“ \pageref{sp.306}.

\so{Weiber-} und \so{Männersprachen} \pageref{sp.248}–\pageref{sp.249}.

\so{Wellentheorie} \pageref{sp.163}–\pageref{sp.165}.

\so{Welt} = Gesammtheit der Einzelvorstellungen \pageref{sp.325}. Welt und Weltordnung, subjective \pageref{sp.344}–\pageref{sp.345}.

\so{Weltanschauung} in der Sprache \pageref{sp.40}\sed{, \pageref{sp.76}}. \sed{W. u. Rede \pageref{sp.331}. W. u. Sprachbildung \pageref{sp.386}.}

\so{Wendisch} \pageref{sp.54}. Im Lüneburgischen \pageref{sp.146}. Ober- und Niederlausitzer W. im Aussterben \pageref{sp.146}. Germanismen \pageref{sp.270}.

\sed{\so{Wer} = \so{Mann} \pageref{sp.351}.}

\so{Werden} \pageref{sp.103}. \so{Vertere} \pageref{sp.318}.

\so{Werkzeug.} Passivum des W. \pageref{sp.102}.

\so{Werthbestimmung} der Sprachen \inlineupdate{\pageref{sp.371}–\pageref{sp.457}.}{\pageref{sp.371} flg.}

\so{Wesdin,} Joh. Phil. \pageref{sp.26}.

\sed{\so{Weske} \pageref{sp.402}.}

\so{Westphal.} Evolutionstheorie \pageref{sp.216}.

\so{Whitmee} \pageref{sp.193}.

\so{Whitney,} William Dwight, Sanskrit-Grammatik \pageref{sp.34}. Seine Bücher über allgemeine Sprachwissenschaft \pageref{sp.52} Anm. Life and Growth of Language \pageref{sp.136}.

\so{Wie} \pageref{sp.103}.

\sed{\so{Wiederholungen} als Verdeutlichungen \pageref{sp.239}.}

\so{Wilkins,} Sanskrit-Grammatik \pageref{sp.26}.

\so{Williams,} W. C., First Lessons in the Maori Language \pageref{sp.463}.

\so{Wilson,} Indianist \pageref{sp.31}.

\so{Winteler,} „Die Kerenzer Mundart“ \pageref{sp.33}.

„\so{Wir}“. Erweiterung des Ich in der Familie \pageref{sp.306}.

\so{Wissen,} grammatisches; dessen Bestandtheile \pageref{sp.84}–\pageref{sp.86}.

\so{Wissenschaft,} Gegensatz zum Erlernen für praktische Zwecke \pageref{sp.7}–\pageref{sp.8}, \pageref{sp.17}–\pageref{sp.18}. Gemeinsamer Endzweck aller W. \pageref{sp.13}.

\sed{\so{Wissenschaftlichkeit} der Grammatik \pageref{sp.110}.}

\sed{\so{Witz} in der Wortschöpfung \pageref{sp.42}.}

\so{Wohllaut,} Euphonik, Sandhi \pageref{sp.197}–\pageref{sp.205}.

\so{Wollen} \pageref{sp.103}.

\so{Woloffisch} \pageref{sp.150}, \pageref{sp.282}. \sed{Vocalismus \pageref{sp.408}.}

\so{Wort} und Satz \pageref{sp.88}. Selbständigkeit der Wörter und Sandhi \pageref{sp.210}. \inlineupdate{Älteste}{Aelteste} Wörter der menschlichen Sprache \pageref{sp.255}.

\sed{\so{Wortabtheilung} in der Schrift \pageref{sp.132}.}

\sed{\so{Wortbedeutung,} deren Veränderung in der Sprachgeschichte \pageref{sp.229}–\pageref{sp.230}.}

\so{Wortbildung} u. Formenbildung, ob im Sprachbewusstsein unterschieden? \pageref{sp.122}. Die Lehre von der W. gehört \fed{{\textbar}501{\textbar}} in die Grammatik \pageref{sp.122}. Wortformen \inlineupdate{von}{v.} absoluter Bedeutung \pageref{sp.443}–\pageref{sp.448}.

\so{Wörterbuch.} Collectaneen zu einem solchen \pageref{sp.78}. Herkömmliche Zweitheilung \pageref{sp.85}. Einzelsprachliches als Nachschlagebuch \pageref{sp.121}. Grundlagen und Erfordernisse eines wissenschaftlichen \pageref{sp.121}–\pageref{sp.123}. Mögliche Theile und Arten eines solchen \pageref{sp.123}–\pageref{sp.125}. W. zur Sprachvergleichung, wie einzurichten? \pageref{sp.166}–\pageref{sp.167}. Allgemeine Wortschatzkunde \pageref{sp.482}–\pageref{sp.483}.

\sed{{\textbar}{\textbar}520{\textbar}{\textbar}}

\so{Wortfolge.} \sed{In der deutschen Sprache \pageref{sp.63}.} Inwieweit ihre Gesetze in den Sprachfamilien typisch sind? \pageref{sp.149}–\pageref{sp.150}. \inlineupdate{Äussere}{Aeussere} und innere Sprachform durch die W. gekennzeichnet \pageref{sp.359}–\pageref{sp.360}. Psychologisches Subject und Prädicat \pageref{sp.365}–\pageref{sp.373}. W. und Betonung \pageref{sp.373}–\pageref{sp.376}. W. und Stimmungsmimik \pageref{sp.380}–\pageref{sp.381}. W. in den malaischen und den uralaltaischen Sprachen \pageref{sp.415}–\pageref{sp.416}, \pageref{sp.418}. In den Bantusprachen \pageref{sp.420}.

\so{Wortschatz} \inlineupdate{siehe}{s.} \so{Sprachschatz, Wörterbuch.}

\so{Wortschatzkunde,} allgemeine \pageref{sp.482}–\pageref{sp.483}.

\so{Wortschöpfung.} Vergleiche \pageref{sp.40}–\pageref{sp.43}. Elliptisches Verfahren \pageref{sp.366}. Benennung nach einer hervorragenden Eigenschaft \pageref{sp.381}. Vgl. Etymologie.

\so{Wortschrift,} chinesische, annamitische \inlineupdate{\pageref{fp.138}, \pageref{fp.139}.}{\pageref{sp.130}.}

\sed{\so{Wortstämme} \pageref{sp.115}. Nach- und Neuschöpfungen von W. \pageref{sp.250}–\pageref{sp.252}.}

\so{Wortstellung} s. \so{Wortfolge.}

\so{Wortstreit,} Streit über Sinngleichheit und -verschiedenheit der Wörter, bedeutsam für das Sprachbewusstsein \pageref{sp.97}.

\so{Worttrennung} in der Schrift \pageref{sp.131}.

\sed{\so{Wright,} Arabic grammar \pageref{sp.372}.}

\so{Wunsch} und Ausruf \pageref{sp.322}.

\so{Wünschen} \pageref{sp.103}.

\so{Wurzeln.} Innere Veränderungen derselben in einer Kindersprache \pageref{sp.65}. Letzte, nicht weiter lösbare Bestandtheile \pageref{sp.179}. Ursprüngliche Lautsymbolik \pageref{sp.225}. \sed{Nach- und Neuschöpfungen von W. \pageref{sp.250}–\pageref{sp.252}.} Indogermanische W. geben kein Bild der Urwörter \pageref{sp.255}. Begriff der W. \pageref{sp.295}–\pageref{sp.297}. Verbale oder stoffliche und pronominale oder Formwurzeln \pageref{sp.395}–\pageref{sp.396}.

\sed{\so{Wurzellaute.} Auch W, sind Rede \pageref{sp.331}.}

\sed{\so{Wurzelwörter} \pageref{sp.380}.}

\subsection*{X.}\label{reg.X}\pdfbookmark[1]{X.}{reg.X}

\so{Xenophon} \pageref{sp.20}.

\subsection*{Y.}\label{reg.Y}\pdfbookmark[1]{Y.}{reg.Y}

\so{Yakama} \pageref{sp.358}\sed{, \pageref{sp.424}.}

\so{Yeux,} französisch: zieux \pageref{sp.203}.

\so{Yoloff} s. \so{Woloffisch.}

\so{Yoruba} \pageref{sp.282}.

\so{Yucatekisch,} Schrift \pageref{sp.130}. Vgl. \so{Maya.}

\so{Yunga} \pageref{sp.153}.

\subsection*{Z.}\label{reg.Z}\pdfbookmark[1]{Z.}{reg.Z}

\so{Zahl,} numerus, im synthetischen Systeme der Grammatik \pageref{sp.101}. Die Kategorie \inlineupdate{der}{d.} Z. \pageref{sp.445}–\pageref{sp.446}.

\so{Zahlwörter} als Beweis für die Verwandtschaft \pageref{sp.152} flg. \sed{Bedeutung für die Sprachform \pageref{sp.339}.} Zweifelhafte Herkunft \pageref{sp.384}.

\so{Zambales} \pageref{sp.280}.

\so{Zank} \pageref{sp.310}–\pageref{sp.311}.

\sed{\so{Zapoteken} \pageref{sp.389}.}

\sed{\so{Zapotekische} Sprache, Unsichere Articulation \pageref{sp.194}.}

\so{Zauberei.} Formeln und Ausdrücke \pageref{sp.107}.

\so{Zebu.} Bildsamkeit \pageref{sp.349}.

\so{Zeichensprache} im internationalen Verkehre \pageref{sp.17}–\pageref{sp.18}.

\so{Zeit.} Adverbialsätze der Z. \pageref{sp.104}. Z. des Verbums \pageref{sp.101}. Die Kategorie der Z. \pageref{sp.446}.

\so{Zeitrechnung} als Anhang zur Grammatik \pageref{sp.107}.

\so{Zeitungen} als sprachliche Macht \pageref{sp.126}.

\so{Zend,} s. \so{Altbaktrisch.}

\so{Zetacismus} s. \so{Palatalisirung.}

\sed{\so{Zettelcollectaneen} \pageref{sp.79}.}

\so{Zeuss,} Joh. Kaspar, Grammatica celtica \pageref{sp.173}.

\so{Zicke} – kleine Ziege \pageref{sp.324}.

\so{Ziererei.} Annahme einer fremden Standessprache \pageref{sp.184}.

\so{Ziffern,} Zahlzeichen, ob gelesen oder gedeutet? \pageref{sp.128}.

\so{Zigeunerisch} \pageref{sp.54}. Dickes l in der russischen Mundart \pageref{sp.269}.

\sed{\so{Zimschian} \pageref{sp.442}.}

\so{Zischlaute} \pageref{sp.36}. \sed{Deren Vorwiegen im Slavischen \pageref{sp.148}.} Z. in der Natur \pageref{sp.314}.

\so{Zitze.} Wörter dafür \pageref{sp.153}–\pageref{sp.154}, – τίτθη \pageref{sp.208}.

\so{Žo} s. \so{Kukisprachen.}

\fed{{\textbar}502{\textbar}}

ζῶον πολιτικόν \pageref{sp.307}.

\so{Zoroastrier:} daêva = Dämon \pageref{sp.231}.

\so{Zote} und Euphemismus \pageref{sp.249}.

\sed{\so{Zuchtwahl} im Lautwesen der Ursprachen \pageref{sp.314}.}

\so{Zulu.} Schnalzlaute \pageref{sp.269}.

\sed{\so{Zumpt,} Grammatiker \pageref{sp.49}.}

\sed{\so{Zunge,} deren Taktgefühl \pageref{sp.313}.}

\so{Zungenbrechende Sätze} \pageref{sp.37}–\pageref{sp.38}.

\so{Zusammensetzung} als Mittel der Wortschöpfung \pageref{sp.41}, \pageref{sp.346}. Gleichklang, Alliteration, Reim \pageref{sp.222}. Einfluss der Z. auf den Bedeutungswandel \pageref{sp.234}–\pageref{sp.237}. Z. und Anbildung nach Steinthal \pageref{sp.337}. Syntaktische Composition \pageref{sp.359}. Z. als Mittel, um Sätze in Wörter zu verwandeln \pageref{sp.466}.

\sed{\so{Zweck} der Sprache \pageref{sp.181}.}

\sed{\so{Zwehle} \pageref{sp.190}.}

\sed{\so{Zwerch} \pageref{sp.190}.}

\so{Zwischenprädicate} \pageref{sp.456}–\pageref{sp.458}.
\end{register}
\clearpage
\chead{}
 