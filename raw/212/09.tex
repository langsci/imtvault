%!TEX root = ../main.tex

\chapter{Complex syntax} \label{cha:interclausalsyntax}

\section{Introduction}\label{interclausintro}

This section describes the combination of two or more predicates. There are three parameters involving the coding of complex clauses. The first parameter is the verb inflection. Are both predicates fully inflected or is one of them nominalised? The second parameter is the way, how an interclausal relationship between two fully inflected predicates is marked. This often involves demonstratives marked for case. The third parameter are syntactic restrictions in one of the two clauses. These parameters allow us to decide whether a particular clause combination should be analysed as \isi{coordination} or \isi{subordination}. Note that the first parameter supersedes the other two, in that nominalised predicates are always analysed as subordinate clauses, and the other two parameters do not apply. Only if two clauses contain inflected verbs, these two parameters help to identify the relationship between them. For example, relative clauses are structurally similar to content questions, but they differ in two points. First, they are usually headed by the relativised element, which is in some sense the answer to the question posed by the \isi{relative clause}. Secondly, relative clauses have a more rigid structure than questions. Hence, they are analysed as a type of \isi{subordination}. On the other hand, complements of knowledge consist of one \isi{clause} with a predicative \isi{nominal} (\emph{miyatha} `knowledge') and the copula. The epistemic content can be expressed by a separate \isi{clause}, which shows no syntactic dependency to the first. It follows that in some cases these parameters fail and only semantic criteria can be applied.

I want to give a few examples, to show that there is a cline of syntactic integration between two clauses. Givón provides a functional explanation to the various degrees of syntactic integration: ``the stronger the semantic bond between two events, the more extensive will be the syntactic integration of the two clauses into a single though complex clause'' (\citeyear[41]{Givon:2001syntax}). As we will see, Komnzo supports this observation to some extent. I choose the domain of `cause' to illustrate this below. The clearest way to mark a causer is by putting the element in the \isi{ergative} case. In Komnzo nominalised verbs can be used in this way (\ref{ex662}). In the example, a \ili{Marind} headhunter tries to distract his victims by imitating the sound that dogs make when chewing bones, but he ends up only distracting himself. The phrase \emph{ane wäsifnzo} `only that cracking' functions as a clausal \isi{subject}. The event `crack' and the event `close' are tightly integrated. They occur simultaneously and they stand in direct causal relation.

\begin{exe}
	\ex \emph{bäne thuwänzrm fof ... zarfa surmänwrm \textbf{ane wäsifnzo}.}\\
	\gll bäne thu\stem{wä}nzrm fof (.) zarfa su\stem{rmän}wrm ane wä-si=f=nzo\\
	\Dem:\Med{} \Stsg:\Sbj>\Stpl:\Obj:\Pst:\Dur/crack fof (.) ear \Stsg:\Sbj>\Tsg.\Masc:\Obj:\Pst:\Dur/close {\Dem} crack-\Nmlz=\Erg=\Only\\
	\trans `He was cracking those (coconut shells) ... This cracking was blocking his ears.'\\\Corpus{tci20120818}{ABB \#67-68}
	\label{ex662}
\end{exe}

The \isi{characteristic} \isi{case} is used for adverbial adjuncts marking origin and cause. In example (\ref{ex663}), \emph{mni frazsi} functions as an adverbial \isi{clause}. The predicate `be weak' and the event `extinguish' occurred at different times, but they stand in a causal relation.

\begin{exe}
	\ex \emph{komnzo tayo zwrä \textbf{mni frazsima}.}\\
	\gll komnzo tayo z=wo\stem{rä} mni fraz-si=ma\\
	only weak \Prox=\Fsg:\Sbj:\Nonpast:\Ipfv/be fire extinguish-\Nmlz={\Char}\\
	\trans `I am just weak here from extinguishing the fire.'\Corpus{tci20120922-24}{STL \#21}
	\label{ex663}
\end{exe}

Komnzo has a \isi{recognitional} \isi{demonstrative} \isi{pronoun}, which can function in a number of ways ({\S}\ref{recognitional-pronoun-subsec}). It is frequently used in `tip-of-the-tongue' situations. Example (\ref{ex664}) explains why a particular woman in Rouku grew very old, while her friends and some of her children have passed away already. The structure is the same as (\ref{ex663}). The only difference is that the speaker uses the \isi{recognitional} inflected with the \isi{characteristic} \isi{case} (`because of that one'). After a short pause, he fills in the referent \emph{rirksima} `because she respected'. The event `survive' (lit. `jump') and the event `respect' occurred in different times, but they stand in a causal relation.

\begin{exe}
	\ex \emph{watik, fi komnzo zathfär \textbf{bänema} fof ... nima \textbf{rirksima} brä.}\\
	\gll watik fi komnzo za\stem{thfär} bäne=ma fof (.) nima rirk-si=ma b=\stem{rä}\\
	then \Third.{\Abs} only \Tsg.\F:\Sbj:\Rpst:\Pfv/junp \Recog={\Char} {\Emph} (.) {like.this} respect-\Nmlz={\Char} \Med=\Tsg.\F:\Sbj:\Nonpast:\Ipfv/be\\
	\trans `Well, she just lived on because of that ... because respected (the taboos). There she is.'\Corpus{tci20120922-26}{DAK \#22-23}
	\label{ex664}
\end{exe}

In discourse, the use of the \isi{recognitional} creates some kind of expectation that something should follow. This something can remain empty, for example when the referent is common ground between the speaker and the addressee, but it can also be `filled in' (\ref{ex664}). This latent expectation explains why the \isi{recognitional} is employed to introduce another clause (\ref{ex665}). The function of that clause is determined by the \isi{case} marker on the \isi{recognitional}. In (\ref{ex665}) it is the \isi{characteristic} \isi{case}, and consequently the function of the following \isi{clause} is to mark a reason, in other words \emph{bänema} can be translated with `because'. The event in the first clause `exit' and the event in the second clause `close' stand in a causal relationship. However, the causal chain of events involves a number of steps.

\begin{exe}
	\ex \emph{keke kwamätrakwrm \textbf{bänema} ... \textbf{fam z zwärmänth}.}\\
	\gll keke kwa\stem{mätrak}wrm bäne=ma (.) fam z zwä\stem{rmän}th\\
	{\Neg} \Sg:\Sbj:\Pst:\Dur/exit \Recog={\Char} (.) thought {\Iam} \Stsg:\Sbj>\Tsg:.\F:\Obj:\Rpst:\Pfv/close\\
	\trans `She did not come outside because they had already closed her thoughts (with magic).'\Corpus{tci20120901-01}{MAK \#148-149}
	\label{ex665}
\end{exe}

Lastly, I want to contrast the use of the \isi{recognitional} from other demonstratives. Consider example (\ref{ex666}), which includes the general \isi{demonstrative} \emph{ane} in the \isi{characteristic} \isi{case} in the second clause. The \isi{demonstrative} \emph{ane} functions anaphorically, and in that sense it is the mirror image of the \isi{recognitional}. The events `disturb' and `submerge' stand in a causal relationship, but the components are reversed. We can translate it to \ili{English} with `therefore' or `that's why'. The two clauses are otherwise independent. This is also supported by the paragraph marker \emph{watik} `well, then' which occurs at the beginning of the second clause, but this is optional.

\begin{exe}
	\ex \emph{ane ŋatha bä nzwathofikwr ... watik \textbf{anema} nzibrüzé bobo.}\\
	\gll ane ŋatha bä nzwa\stem{thofik}wr (.) watik ane=ma nz=ybrüzé bobo\\
	{\Dem} dog \Med{} \Stsg:\Sbj>\Fsg:\Obj:\Rpst:\Ipfv/disturb (.) then \Dem={\Char} \Immpst=\Fsg:\Sbj>\Stsg.\Masc:\Obj:\Nonpast:\Ipfv/submerge \Med.{\All}\\
	\trans `That dog disturbed me there. Well, that's why I pushed him into the water.'\\\Corpus{tci20130903-03}{MKW \#188}
	\label{ex666}
\end{exe}

The examples illustrate, that there is a cline between syntactically integrated clauses, i.e. subordinated clauses, and independent clauses. While both ends of the cline are relatively easy to identify, the middle is a grey zone. It is clear that examples like (\ref{ex666}) consists of two independent clauses. Likewise, the nominalised predicates in (\ref{ex662} - \ref{ex664}) are clear cases of \isi{subordination}. But examples like (\ref{ex665}) are somewhat indeterminate. On the one hand, the \isi{recognitional} \isi{pronoun} creates a gap that needs to be filled. In other words, semantically, the second \isi{clause} is subordinated to the first \isi{clause}. On the other hand, the second clause is syntactically independent. Therefore, I refrain from analysing the \isi{recognitional} as a subordinator, but rather as having a connecting function.\footnote{Note that the recognitional demonstrative can be inflected for following cases: characteristic \emph{=ma} `because', instrumental \emph{=me} `thereby' and purposive \emph{=mr} `in order to, until'.}

The following description is functionally motivated, that is subsections are sorted thematically. For example, a subsection on \isi{purposive} clauses will include clear cases of \isi{subordination}, but also constructions where the purpose is expressed in an independent clause connected with the \isi{recognitional}. I will describe coordinated clauses ({\S}\ref{clausecoordination}), \isi{complement} clauses ({\S}\ref{complementclauses}), adverbial clauses ({\S}\ref{advclauses}), relative clauses ({\S}\ref{relclauses}), \isi{conditional} and \isi{temporal} clauses ({\S}\ref{condiclauses}) and \isi{direct speech} and thought ({\S}\ref{directspeechthought}).

\section{Coordinated clauses}\label{clausecoordination}

Coordination refers to syntactic constructions where two or more elements of equal status are connected \citep{Haspelmath:2007coordination}. Komnzo employs the same mechanisms for coordinating \isi{noun phrase}s as it does for coordinating clauses. The word \emph{a} `and' can be used for conjunctive \isi{coordination} (\ref{ex667}) and the word \emph{o} `or' can be used for disjunctive \isi{coordination} (\ref{ex714}).

\begin{exe}
	\ex \emph{mni wthomonwath \textbf{a} zräföfth.}\\
	\gll mni w\stem{thomon}wath a zrä\stem{föf}th\\
	fire \Stpl:\Sbj>\Tsg.\F:\Pst:\Pst:\Ipfv/prepare.fire and \Stpl:\Sbj>\Tsg.\F:\Irr:\Pfv/burn\\
	\trans `They piled the fire and burn it.'\Corpus{tci20120901-01}{MAK \#155}
	\label{ex667}
\end{exe}
\begin{exe}
	\ex \emph{nafaŋamaf wnfathwr \textbf{o} ynfathwr.}\\
	\gll nafa-ŋame=f wn\stem{fath}wr o yn\stem{fath}wr\\
	\Third.{\Poss}-mother=\Erg.{\Sg} \Stsg:\Sbj>\Tsg.\F:\Obj:\Nonpast:\Venit/hold or \Stsg:\Sbj>\Tsg.\Masc:\Obj:\Nonpast:\Venit/hold\\
	\trans `(The child's) mother holds her or holds him.'\Corpus{tci20111004}{RMA \#327-328}
	\label{ex714}
\end{exe}

For conjunctive \isi{coordination} it is quite common to have no overt marker (\ref{ex669}). Especially in sequences of events, two or more inflected verbs can follow each other. Example (\ref{ex669}) describes the felling of a sago palm.

\begin{exe}
	\ex \emph{wati yfarwake ... sabthake ... safümnzake fof.}\\
	\gll wati y\stem{far}wake (.) sa\stem{bth}ake (.) sa\stem{fümnz}ake fof\\
	then \Fpl:\Sbj>\Tsg.\Masc:\Pst:\Ipfv/chop (.) \Fpl:\Sbj>\Tsg.\Masc:\Pst:\Pfv/finish (.) \Fpl:\Sbj>\Tsg.\Masc:\Pst:\Pfv/pull.over {\Emph}\\
	\trans `Then we chopped it (and) finished it (and) pulled it over.'\Corpus{tci20120929-02}{SIK \#19-21}
	\label{ex669}
\end{exe}

Other ways of coordinating two clauses involve the manner \isi{demonstrative} \emph{nima} `like this', which is commonly used to introduce \isi{direct speech} ({\S}\ref{directspeechthought}). In example (\ref{ex689}), \emph{nima} indicates the manner of movement (accompanied by an appropriate gesture), but it also connects the two following clauses.

\begin{exe}
	\ex \emph{nabi tutin fä fof zumirwanzrm füsfüsf ... nima zfzänzrm fobo ... nima zfzänzrm.}\\
	\gll nabi tuti=n fä fof zu\stem{mirwa}nzrm füsfüs=f (.) nima zf\stem{zä}nzrm fobo (.) nima zf\stem{zä}nzrm\\
	bamboo branch={\Loc} {\Dist} {\Emph} \Stsg:\Sbj>\Tsg.\F:\Obj:\Pst:\Dur/swing wind={\Erg} (.) {like.this} \Stsg:\Sbj>\Tsg.\F:\Obj:\Pst:\Dur/carry \Dist.{\All} (.) {like.this} \Stsg:\Sbj>\Tsg.\F:\Obj:\Pst:\Dur/carry\\
	\trans `The wind was swinging (the lamp) on the bamboo (and) it was moving it there (and) it was moving it here.'\Corpus{tci20111119-03}{ABB \#117-118}
	\label{ex689}
\end{exe}

\section{Complement clauses}\label{complementclauses}

\subsection{Phasal verbs}\label{phasalcomplements}

The most common \isi{complement} taking predicates in Komnzo are the two phasal verbs \emph{thkäfksi} `start' and \emph{bthaksi} `finish'. Other verbs show similar behaviour, for example \emph{gathiksi} `stop, leave', \emph{mäyogsi} `continue, repeat'.

With phasal verbs, the indexation structure from the nominalised verb is raised into the \isi{matrix clause}. The values of those categories expressed in the verb form are marked on the phasal verb. This may include \isi{number}, \isi{person} and \isi{gender} of the arguments, but also \isi{tense}, \isi{aspect}, \isi{mood} and \isi{direction}. Example (\ref{ex670}) shows the `non-phasal' \isi{clause} \emph{bad wtharinzake} `we were digging the ground'. The verb indexes the actor (\Fpl) and the \isi{undergoer} \emph{bad} `ground' (\Tsg.\F). In the first \isi{clause} of the example, the same state of affairs is expressed, but the verb `dig' occurs in its \isi{infinitive} \emph{tharisi}, and its argument structure is raised into the phasal verb \emph{thkäfksi} `start'. Now it is the phasal verb which indexes a first \isi{plural} actor and a third singular \isi{feminine} \isi{undergoer}.

\begin{exe}
	\ex \emph{watik bad fof tharisi zathkäfake ... bad wtharinzake zabthake.}\\
	\gll watik bad fof thari-si za\stem{thkäf}ake (.) bad w\stem{thari}nzake za\stem{bth}ake\\
	then ground {\Emph} dig-{\Nmlz} \Fpl:\Sbj>\Tsg.\F:\Obj:\Pst:\Pfv/start (.) ground \Fpl:\Sbj>\Tsg.\F:\Obj:\Pst:\Ipfv/dig \Fpl:\Sbj>\Tsg.\F:\Obj:\Pst:\Pfv/start\\
	\trans `Then we started to dig the ground. We were digging the ground and finished it.'\\\Corpus{tci20120929-02}{SIK \#72-73}
	\label{ex670}
\end{exe}

This is also found with \isi{ditransitive} events, as in (\ref{ex671}). The verb \emph{thkäfksi} `start' indexes the indirect object. Note that the \isi{dative} noun phrase is omitted.

\begin{exe}
	\ex \emph{wri no n säthkäfath yarisi.}\\
	\gll wri no n sä\stem{thkäf}ath yari-si\\
	drunk water {\Imn} \Stpl:\Sbj>\Tsg.\Masc:\Io:\Pst:\Pfv/start give-{\Nmlz}\\
	\trans `They were about to give him alcohol.'\Corpus{tci20120925}{MAE \#158}
	\label{ex671}
\end{exe}

Verbs in the middle template also raise their respective indexation into the phasal verb. The middle template can be used with several functions ({\S}\ref{middletemplatesubsection}). Example (\ref{ex674}) shows the verb \emph{yonasi} `drink', which always occurs in a middle template. In the example, \emph{yonasi} occurs in the \isi{infinitive}. Consequently, the phasal verb takes over this indexing pattern and only encodes the \isi{subject}, but not the \isi{object}.

\begin{exe}
	\ex \emph{nä kayé ... watik yonasi zethkäfa.}\\
	\gll nä kayé (.) watik yona-si ze\stem{thkä}fa\\
	some day (.) then drink-{\Nmlz} \Sg:\Sbj:\Pst:\Pfv/start\\
	\trans `One day, he started to drink.'\Corpus{tci20120925}{MAE \#83}
	\label{ex674}
\end{exe}

In example (\ref{ex672}), the prefixing verb \emph{msaksi} `sit, dwell' is used in its \isi{infinitive}. Since, the phasal verb \emph{thkäfksi} `start' cannot enter the prefixing template, the middle template is used instead. As I describe in {\S}\ref{prefixingverbsec}, the prefixing template is a minor pattern in Komnzo and most intransitive verbs are encoded using the middle template. Furthermore, the prefixing template usually has stative semantics.

\begin{exe}
	\ex \emph{wati foba msaksi fefe zathkäfake.}\\
	\gll wati foba msak-si fefe za\stem{thkäf}ake\\
	then \Dist.{\Abl} dwell-{\Nmlz} really \Fdu:\Sbj:\Pst:\Pfv/start\\
	\trans `From there, we began our married life.' (lit. `We began dwelling.')\\\Corpus{tci20130823-08}{WAM \#47}
	\label{ex672}
\end{exe}

Example (\ref{ex673}) shows that for the middle verb \emph{yak} `run', the phasal verb takes over the indexation.\footnote{This verb is irregular: instead of a nominalised infinitive with \emph{-si}, the third singular masculine form \emph{yak} is used. However, \emph{yak} is the third singular of `walk' and not of `run'. This would be \emph{ŋakwir}. Thus, `walk' employs the noun \emph{moth} `path, way' as its nominalisation and `run' employs \emph{yak}.} Note that the \isi{directional} value (\Venit) is also raised into the phasal verb.

\begin{exe}
	\ex \emph{kabe ane zenthkäfath yak.}\\
	\gll kabe ane zen\stem{thkäf}ath yak\\
	man {\Dem} \Stpl:\Sbj:\Pst:\Pfv:\Venit/start run\\
	\trans `The people started to run here.'\Corpus{tci20131013-01}{ABB \#91}
	\label{ex673}
\end{exe}

\subsection{Complements of knowledge}\label{complknow}

Complements of knowledge are structured differently from phasal complements. They involve a \isi{property noun} with predicative function plus the copula to form a predicate of knowledge (\emph{miyatha} `knowledge(-able)') or ignorance (\emph{miyamr} `ignorant/ignorance'). Note that the latter has probably developed from a more overt marking that involved the \isi{privative} \isi{case} marker \emph{=mär}. The acquisition of knowledge is expressed by the \isi{property noun} \emph{miyatha} plus the \isi{light verb} \emph{ko-} `become'. The epistemic content of these predicates of knowledge and ignorance \textendash{} what is known or not known \textendash{} can be expressed by a number of different strategies. Examples (\ref{ex675}) and (\ref{ex676}) show complements in which a nominalised verb form in the absolutive is added. In (\ref{ex675}), the nominalised verb constitutes the \isi{head} of a compound `coconut climbing'.

\begin{exe}
	\ex \emph{nzä miyatha worä ŋazi sogsi.}\\
	\gll nzä miyatha wo\stem{rä} ŋazi sog-si\\
	\Fsg.{\Abs} knowledge \Fsg:\Sbj:\Nonpast:\Ipfv/be coconut climb-{\Nmlz}\\
	\trans `I know how to climbing coconut.' (lit. `I am knowledgeable (about) coconut climbing'){\hspace*{1pt}\hfill{\footnotesize{[overheard]}}}
	\label{ex675}
\end{exe}

The predicate of knowledge construction (\emph{miyatha}/\emph{miyamr} plus copula) is a frequent collocation. Therefore, it is possible to drop the copula altogether, as in (\ref{ex675}).

\begin{exe}
	\ex \emph{bäne ruga yfränzre ... afa fi miyamr ykwasi ... nzefénzo.}\\
	\gll bäne ruga y\stem{frä}nzre (.) afa fi miyamr ykwa-si (.) nzefé=nzo\\
	\Dem:\Med{} pig \Fpl:\Sbj>\Tsg.\Masc:\Nonpast:\Ipfv/singe.off (.) father \Third.{\Abs} ignorance cut.meat-{\Nmlz} (.) \Fsg.\Erg.\Emph={\Only}\\
	\trans `We burn the hair off that pig ... father doesn't know how to cut it ... only I (know).'\Corpus{tci20120821-02}{LNA \#61-62}
	\label{ex676}
\end{exe}

As described in {\S}\ref{charcase}, the characteristic \isi{case} can express a topic of conversation. Example (\ref{ex677}) shows that the epistemic content can also be marked with the characteristic \isi{case}.

\begin{exe}
	\ex \emph{zf wthkärwé zokwasi nzä monme miyatha worä no kzima.}\\
	\gll zf w\stem{thkär}wé zokwasi nzä mon=me miyatha wo\stem{rä} no kzi=ma\\
	{\Imm} \Fsg:\Sbj>\Tsg.\F:\Obj:\Nonpast:\Ipfv/start speech \Fsg.{\Abs} how={\Ins} knowledge \Fsg:\Sbj:\Nonpast:\Ipfv/be rain barktray={\Char}\\
	\trans `I will start the story how I know about the rain making (magic).'\\\Corpus{tci20110810-01}{MAB \#8}
	\label{ex677}
\end{exe}

The epistemic content can be expressed as a \isi{relative clause}, which takes the predicate of knowledge as its \isi{head}, as in example (\ref{ex678}).

\begin{exe}
	\ex \emph{bä z miyatha erä maf n zwämg?}\\
	\gll bä z miyatha e\stem{rä} maf n zwä\stem{mg}\\
	\Second.{\Abs} {\Iam} knowledge \Stpl:\Sbj:\Nonpast:\Ipfv/be who.{\Erg} {\Imn} \Stsg:\Sbj>\Fsg:\Obj:\Rpst:\Pfv/shoot\\
	\trans `Do you know who almost shot me?'\Corpus{tci20130927-06}{MAB \#37}
	\label{ex678}
\end{exe}

The epistemic content can also be expressed in an independent \isi{clause}, connected, for example, with \emph{nima} `like this' (\ref{ex679}). The use of \emph{nima} in this example can also be analysed as quoting inner thought ({\S}\ref{directspeechthought}).

\begin{exe}
	\ex \emph{fi miyamr sfrärm nima fi zbo ern.}\\
	\gll fi miyamr sf\stem{rä}rm nima fi zbo e\stem{rn}\\
	\Third.{\Abs} ignorance \Tsg.\Masc:\Sbj\Pst:\Dur/be {like.this} \Third.{\Abs} \Prox.{\All} \Stdu:\Sbj:\Nonpast:\Ipfv/be\\
	\trans `He did not know those two here.'\Corpus{tci20130927-06}{MAB \#123}
	\label{ex679}
\end{exe}

The acquisition of knowledge is expressed by replacing the copula with the \isi{light verb} \emph{ko-} `become'. Example (\ref{ex680}) is taken from a text about a punitive custom, whereby the perpetrator is humiliated by being given a large amount of yams, which he is expected to pay back the following year. The epistemic content is expressed by a \isi{relative clause}.

\begin{exe}
	\ex \emph{``miyatha käkor bä monwä zbrigwé bä ra nrä? daw kabe?'' nima kwakonzrmth.}\\
	\gll miyatha kä\stem{kor} bä mon=wä z\stem{brig}wé bä ra n\stem{rä} daw kabe nima kwa\stem{ko}nzrmth\\
	knowledge \Ssg:\Sbj:\Imp:\Pfv/become \Second.{\Abs} how={\Emph} \Ssg:\Sbj>\Tsg.\F:\Sbj:\Imp:\Ipfv/return \Second.{\Abs} what \Ssg:\Sbj:\Nonpast:\Ipfv/be garden man {\Quot} \Stpl:\Sbj:\Pst:\Dur/say\\
	\trans ```You see how you pay this back! What are you? A gardener?'' that is what they were saying.'\Corpus{tci20120805-01}{ABB \#241}
	\label{ex680}
\end{exe}

Note that the phrase \emph{miyatha käkor!} can be purely epistemic ``(Now) you know it!'' or it can express an experiential sensation ``(Now) you feel it!''.

\subsection{Complements of desire}\label{compldesire}

Much of what has been said about complements of knowledge, can be said about complements of desire. The \isi{property noun} \emph{miyo} `desire' is used for this.\footnote{Note that \emph{miyo} can also be a noun meaning `wish' and `taste'.} It can be negated with the \isi{privative} \isi{case} \emph{=mär}: \emph{miyomär}. Again, a \isi{property noun} plus copula construction expresses the concept of `want, wish or hope': \emph{ra miyo erä?} `What do you want' (lit. `What desire you are?'). The \isi{clause} encoding the desired (or undesired) can be expressed in a variety of ways. Example (\ref{ex681}) shows a nominalised verb \emph{mgthksi} `feed' in the absolutive. The verb is heading a compound `pig feeding'.

\begin{exe}
	\ex \emph{zena keke miyo worä ruga mgthksi ... znsä ttüfr.}\\
	\gll zena keke miyo wo\stem{rä} ruga mgthk-si (.) znsä t-tüfr\\
	today {\Neg} desire \Fsg:\Sbj:\Nonpast:\Ipfv/be pig feed-{\Nmlz} (.) work \Redup-plenty\\
	\trans `Today, I do not want to feed pigs ... too much work.' (lit. `I am not desirous for pig feeding')\Corpus{tci20120805-01}{ABB \#819-820}
	\label{ex681}
\end{exe}

In example (\ref{ex682}), the word \emph{zokwasi} is used as a nominalisation `speaking'.

\begin{exe}
	\ex \emph{keke zokwasi miyo nzä worärm yoganai worärm.}\\
	\gll keke zokwasi miyo nzä wo\stem{rä}rm yoganai wo\stem{rä}rm\\
	{\Neg} speech desire \Fsg.{\Abs} \Fsg:\Sbj:\Rpst:\Dur/be tired \Fsg:\Sbj:\Rpst:\Dur/be\\
	\trans `I did not want to talk. I was tired.'\Corpus{tci20120922-24}{MAA \#78}
	\label{ex682}
\end{exe}

The \isi{property noun} \emph{miyo} can also be used without the copula, as in (\ref{ex683}).

\begin{exe}
	\ex \emph{frzsi miyomäre fthé kafara znfonzo kerafith thämther. sayäfianme rifthzsi fath zn rä.}\\
	\gll frz-si miyo=märe fthé kafara zn=fo=nzo ke\stem{rafith} thä\stem{mther} sayäfi=anme rifthz-si fath zn \stem{rä}\\
	net-{\Nmlz} desire={\Priv} when {river pandanus} place=\All={\Only} \Ssg:\Sbj:\Imp:\Pfv/paddle \Ssg:\Sbj>\Stpl:\Obj:\Imp:\Pfv/lift.up {river.crayfish}=\Poss.{\Nsg} hide-{\Nmlz} place place \Tsg.\F:\Sbj:\Nonpast:\Ipfv/be\\
	\trans `If you don't want to catch by net, you can paddle to the river pandanus and lift up (the leaves). It is crayfish's hiding place.'\Corpus{tci20130907-02}{RNA \#450-451}
	\label{ex683}
\end{exe}

The desired proposition can also be expressed in an independent \isi{clause} which is only semantically connected to the desiderative proposition. In example (\ref{ex684}), a man threatens a young boy who shot an arrow at him.

\begin{exe}
	\ex \emph{zbo z fefe saththma ``nzä fthé miyo kwrarä zena zf mr kwa nwänzé.''}\\
	\gll zbo z fefe sa\stem{ththm}a nzä fthé miyo kwra\stem{rä} zena zf mr kwa n\stem{wä}nzé\\
	\Prox.{\All} {\Imm} really \Stsg:\Sbj>\Tsg,\Masc:\Io:\Pst:\Pfv/stick.on \Fsg.{\Abs} when desire \Fsg:\Sbj:\Irr:\Ipfv/be today {\Imm} neck {\Fut} \Fsg:\Sbj>\Ssg:\Obj:\Nonpast:\Ipfv/crack\\
	\trans `He stuck (the gun) right at him (saying): ``If I wanted I could breack your neck right here and now.'''\Corpus{tci20130927-06}{MAB \#45}
	\label{ex684}
\end{exe}

\section{Adverbial clauses}\label{advclauses}

Adverbial clauses show a wide range of possible constructions. These range from infinitival adjuncts to independent clauses. In the following section, \isi{purposive}, \isi{temporal} and manner adverbial clauses are described. Note that the domain of cause was used to introduce the reader to the various levels of syntactic intregration of two clauses. Therefore, I will not discuss this domain here, but refer to {\S}\ref{interclausintro}.

\subsection{Purposive adverbials}\label{purposeadverbials}

Purposive adverbials are found in different construction. Example (\ref{ex698})	is from a procedural about making a drum. The speaker explains how a bamboo ring will hold the membrane in place after it is glued to the drum.

\begin{exe}
	\ex \emph{nabi riwariwa kwa wäfiyokwre ... \textbf{narsir} fof.}\\
	\gll nabi riwariwa kwa wä\stem{fiyok}wre (.) nar-si=r fof\\
	bamboo ring {\Fut} \Fpl:\Sbj>\Tsg.\F:\Nonpast:\Ipfv/make (.) press.down-\Nmlz={\Purp} \Emph\\
	\trans `We make a bamboo ring ... for pressing down (the membrane).'\\\Corpus{tci20120824}{KAA \#87-88}
	\label{ex698}
\end{exe}

In example (\ref{ex699}), the speaker shows me a particular tree used for poison-root fishing. The example shows that the \isi{purposive} \isi{clause} can take an \isi{object} by forming a compound `for swamp poisoning' > `to poison the swamp'. Note that the \isi{recognitional} \isi{pronoun} is used just before the nominalised verb.

\begin{exe}
	\ex \emph{nä kayé zane zf yirwre \textbf{bänemr} ... \textbf{zra rsrsir}.}\\
	\gll nä kayé zane zf y\stem{r}wre bänemr (.) zra rsr-si=r\\
	{\Indf} day \Dem:{\Prox} {\Imm} \Fpl:\Sbj>\Tsg.\Masc:\Obj:\Nonpast:\Ipfv/scrape \Recog.{\Purp} (.) swamp poison.fishing-\Nmlz=\Purp\\
	\trans `Sometimes, we scrape (the root of) this one here for poisoning the waterholes.'\\\Corpus{tci20130907-02}{RNA \#340}
	\label{ex699}
\end{exe}

Purposive clauses can also be less syntactically integrated and form an independent \isi{clause}. In this case, they are usually introduced by the \isi{recognitional} flagged with \isi{purposive} \isi{case} \emph{bänemr}, which I translate with `in order to'. Example (\ref{ex700}) describes a tall structure used to show off the amount of a group's yams harvest. This structure involved a long post around which many layers of yam tubers were tied with thick rope.

\begin{exe}
	\ex \emph{wati far ane thden sfräzrmth \textbf{bänemr kwim ŋadme sfmthzgwrmth}.}\\
	\gll wati far ane thd=en sf\stem{räz}rmth bänemr kwim ŋad=me sf\stem{mthzg}wrmth\\
	then post {\Dem} middle={\Loc} \Stpl:\Sbj>\Tsg.\Masc:\Obj:\Pst:\Dur/erect \Recog.{\Purp} kwim rope={\Ins} \Stpl:\Sbj>\Tsg.\Masc:\Obj:\Pst:\Dur/encircle\\
	\trans `Then, they were erecting a post in the middle in order to wrap around the kwim (Acacia mangium) rope.'\Corpus{tci20120805-01}{ABB \#463}
	\label{ex700}
\end{exe}

\subsection{Temporal adverbials}\label{tempadverbials}

Temporal adverbials are found in a number of constructions. Example (\ref{ex690}) shows the \isi{locative} \isi{case} attached to a nominalised verb. The \isi{clause} \emph{ane yam fiyoksin} `doing that' is therefore subordinated to the \isi{matrix clause}. The relation between the two clauses is one of \isi{simultaneity}.

\begin{exe}
	\ex \emph{bäne zrazänzr ... fenz kzikaf ... mä ke kwa kabef sremar \textbf{ane yam fiyoksin}.}\\
	\gll bäne zra\stem{zä}nzr (.) fenz kzi=kaf (.) mä keke kwa kabe=f sre\stem{mar} ane yam fiyok-si=n\\
	\Recog.{\Abs} \Stsg:\Sbj>\Tsg.\F:\Obj:\Irr:\Ipfv/carry (.) {body liquid} barktray={\Prop} (.) where {\Neg} {\Fut} man=\Erg.{\Sg} \Stsg:\Sbj>\Tsg.\Masc:\Obj:\Irr:\Pfv/see {\Dem} event make-\Nmlz={\Loc}\\
	\trans `He will carry that one ... the body liquid with the barktray ... where no man will see him while doing that.'\Corpus{tci20130903-04}{RNA \#49-52}
	\label{ex690}
\end{exe}

In order to connecting more independent clauses, the word \emph{fthé} `if, when' is used. This is further described in {\S}\ref{condiclauses} together with \isi{conditional} \isi{clause}. A close \isi{temporal} connection between the two clauses can be established by the word \emph{fthémäsü} `meanwhile, during'. The words \emph{fthé} and \emph{fthémäsü} are historically related, but the etymology of the \emph{mäsü} part is unclear. In example (\ref{ex691}), the speaker talks about a particular tree which flowers during the planting season. Note that the first and last clause contain \emph{fthé} and the middle clause contains \emph{fthémäsü}: `when X, while Y, that is when Z'.

\begin{exe}
	\ex \emph{efthar fthé kräkor minzü ... \textbf{fthémäsü wawa worsi threthkäfth} ... nzram fthé fof kwa ŋarär.}\\
	\gll efthar fthé krä\stem{kor} minzü (.) fthémäsü wawa wor-si thre\stem{thkäf}th (.) nzram fthé fof kwa ŋa\stem{rä}r\\
	{dry season} when \Stsg:\Sbj:\Irr:\Pfv/become very (.) meanwhile yam plant-{\Nmlz} \Stpl:\Sbj>\Stpl:\Obj:\Irr:\Pfv/start (.) flower when {\Emph} {\Fut} \Stsg:\Sbj:\Nonpast:\Ipfv/do\\
	\trans `When it reaches the height of the dry season ... while they are starting to plant the yams ... that is when this one will flower.'\Corpus{tci20130907-02}{JAA \#220-221}
	\label{ex691}
\end{exe}

\emph{Fthémäsü} is not a subordinator because it can be used on independent clauses with the translation `in the meantime'. In example (\ref{ex692}), the speaker explains that after his father's death, the stones for rain-making were lost.

\begin{exe}
	\ex \emph{\textbf{nzenme ŋafe fthémäsü kwosi yara} ... watik foba ni miyamr nrä mafadben zena ethn.}\\
	\gll nzenme ŋafe fthémäsü kwosi ya\stem{r}a (.) watik foba ni miyamr n\stem{rä} mafa=dben zena e\stem{thn}\\
	\Fnsg.{\Poss} father meanwhile dead \Tsg.\Masc:\Sbj:\Pst:\Ipfv/be (.) then \Dist.{\Abl} {\Fnsg} ignorance \Fpl:\Sbj:\Nonpast:\Ipfv/be who=\Loc.\Anim.{\Sg} today \Stpl:\Sbj:\Nonpast:\Ipfv/lie.down\\
	\trans `In the meantime our father died ... and from then one we don't know with whom (the rain stones) are today.'\Corpus{tci20131013-01}{ABB \#399}
	\label{ex692}
\end{exe}

\largerpage
A third strategy to connect a \isi{clause} temporally is by using the \isi{recognitional} inflected with the \isi{locative} case \emph{bafen}. But this is an infrequent strategy because (i) the \isi{temporal} function is an extension of the \isi{locative} case and (ii) connecting clauses is only one function of the \isi{recognitional}. Example (\ref{ex693}) is about two men from Rouku who used to work on the Fly River. They run into another man from Rouku, who has been away for a long time. The \isi{recognitional} occurs twice. First, is it coreferential with holiday: `in that time ... during the holidays'. The second use is difficult to analyse because this is also a \isi{temporal}/\isi{conditional} construction, but one can assume that \emph{bafen} introduces the second clause.

\begin{exe}
	\ex \emph{fthé nima bafen kabrigrnoth holidayen \textbf{bafen fefe katrife ``fi bobo yé!''}}\\
	\gll fthé nima baf=en ka\stem{brig}rnoth holiday=en baf=en fefe ka\stem{trif}e fi bobo \stem{yé}\\
	when {like.this} \Recog={\Loc} \Du:\Sbj:\Imp:\Ipfv:\Andat/return holiday={\Loc} \Recog={\Loc} really \Sdu:\Sbj:\Imp:\Pfv/tell \Third.{\Abs} \Med.{\All} \Tsg.\Masc:\Sbj:\Nonpast:\Ipfv/be\\
	\trans `When you return in the holidays, then you have say: ``He is there!'''\\\Corpus{tci20130927-06}{MAB \#206}
	\label{ex693}
\end{exe}

\subsection{Manner adverbials}\label{manneradv}

The \isi{proprietive} and \isi{instrumental} case on a nominalised verb can be used to express a manner adverbial \isi{clause}. In the functional extension, the two case markers can also express a relation of association and \isi{temporal} overlaps respectively. Hence, the nominalised verb flagged with the \isi{proprietive} case in example (\ref{ex694}) can be translated as `He held hips while rejoicing' or `He held hips rejoicingly.'

\begin{exe}
	\ex \emph{\textbf{thweksikarä} gon z zefaf.}\\
	\gll thwek-si=karä gon z ze\stem{faf}\\
	rejoice-\Nmlz={\Prop} hip {\Iam} \Stsg:\Sbj:\Rpst:\Pfv/hold\\
	\trans `He held hips while rejoicing' or `He held hips rejoicingly.'\Corpus{tci20111004}{RMA \#174}
	\label{ex694}
\end{exe}

The \isi{recognitional} \isi{case} also serves to introduce a clause which expresses a manner (or \isi{temporal} association). In example (\ref{ex695}), the speaker explains how he and his friends were loading a heavy sago stem on a canoe. Some people from Morehead Station were sceptical about this plan. Thus, \emph{bäneme thfkogrm} `They were standing with/like this' connects to the following clause which expresses `They stood thinking ...'.

\begin{exe}
	\ex \emph{nä station kabe fä zämosirath \textbf{bäneme} thfkogrm ... \textbf{fam kwarärmth ``kwa ywokrakwr o kwa ŋabrüzr?''}}\\
	\gll nä station kabe fä zä\stem{mosir}ath bäne=me thf\stem{kogr}m (.) fam kwa\stem{rä}rmth kwa y\stem{wokrak}wr o kwa ŋa\stem{brüz}r\\
	{\Indf} station man {\Dist} \Stpl:\Sbj:\Pst:\Pfv/gather \Recog={\Ins} \Stpl:\Sbj:\Pst:\Dur/stand (.) thought \Stpl:\Sbj:\Pst:\Dur/do {\Fut} \Tsg.\Masc:\Sbj:\Nonpast:\Ipfv/float or {\Fut} \Stsg:\Sbj:\Nonpast:\Ipfv/submerge\\
	\trans `Some station people gathered there. They were standing thinking: ``Will it float or will it sink?'''\Corpus{tci20120929-02}{SIK \#30-31}
	\label{ex695}
\end{exe}

The most common way to encode a manner adverbial is by a \isi{relative clause} with \emph{mon} or \emph{monme} `how' ({\S}\ref{relclauses}). Example (\ref{ex696}) is taken from a picture task, where the participants were asked to arrange picture cards into a story. In the example, the speaker explains the task to a bystander. Note that the \isi{recognitional} \emph{bäneme} `with this, in this way' also appears in the first clause. The second \isi{recognitional} \emph{bäne} refers to \emph{trikasi} `story' as we can see in the last clause.

\largerpage
\begin{exe}
	\ex \emph{zena ane bäneme nzezinakwre \textbf{monme bäne wyak brä} ... \textbf{trikasi monme kma zrarä}.}\\
	\gll zena ane bäne=me nz=e\stem{zinak}wre mon=me bäne w\stem{yak} b=\stem{rä} (.) trik-si mon=me kma zra\stem{rä}\\
	now {\Dem} \Recog={\Ins} \Immpst=\Fpl:\Sbj>\Stpl:\Obj:\Nonpast:\Ipfv/put.down how={\Ins} \Recog.{\Abs} \Tsg.\F:\Sbj:\Nonpast:\Ipfv/walk \Med=\Tsg.\F:\Sbj:\Nonpast:\Ipfv/be (.) tell-{\Nmlz} how={\Ins} {\Pot} \Tsg.\F:\Sbj:\Irr:\Ipfv/be\\
	\trans `Now we are putting (the pictures) down how it goes there ... how the story should be.'\Corpus{tci20111004}{RMA \#313-314}
	\label{ex696}
\end{exe}

\section{Relative clauses}\label{relclauses}

I follow Andrews (\citeyear[206]{Andrews:2007relclauses}) in defining relative clauses as a ``subordinate clause which delimits the reference of an NP by specifying the role of the referent of that NP in the situation described by the \textsc{rc} [\isi{relative clause}]''. I adopt Andrews' label NP\textsubscript{\textsc{mat}} or matrix NP for the NP in the \isi{matrix clause}, and NP\textsubscript{\textsc{rel}} for the NP in the \isi{relative clause}. The latter is always expressed by \isi{interrogative} pronouns, which function as relative pronouns. Hence, Komnzo and other Yam languages employ the ``relative pronoun strategy'' for relativisation, which from a cross-linguistic perspective is found mostly in Europe \citep{Haspelmath:2001}.

Relative clauses in Komnzo are adjoined clauses in the sense of Hale (\citeyear{Hale:1976adjoined}), who notes that adjoined relative clauses are ``subordinate in some way, but [their] surface position with respect to the main clause is marginal rather than embedded'' (\citeyear[78]{Hale:1976adjoined}). Andrews defines them as having the \isi{relative clause} appear outside the matrix NP. Relative clauses in Komnzo are almost always right-adjoined, i.e. they follow the matrix NP. Alternatively, they may refer to the whole preceding (matrix) \isi{clause}. The matrix NP can be fronted together with the \isi{relative clause}, which is a common strategy used for topicalisation ({\S}\ref{info-cleft}).

We can represent the structure of relative clauses schematically, as in Figure \ref{relclause}. The matrix element, [...]\textsubscript{\textsc{mat}} in the figure, is usually a \isi{noun phrase}, which can be omitted if it is understood from context. Alternatively, the matrix element can be a \isi{matrix clause}. The \isi{relative clause}, [...]\textsubscript{\textsc{rc}} in the figure, consists of the relative \isi{pronoun} and the verb. There may be one \isi{noun phrase} preceding the relative \isi{pronoun}, but there cannot be more than one \isi{noun phrase} in this position.

\begin{figure}  
	\begin{tabular}{ll}
			[\textsc{np}\textsubscript{i}]\textsubscript{\textsc{mat}}& [(\textsc{np}) \textsc{rel.pron}\textsubscript{i} \textsc{v}]\textsubscript{\textsc{rc}}\\
	\end{tabular}
	\caption{Schematic representation of a relative clause (\textsc{rc})}
	\label{relclause} 
\end{figure}%Schematic representation of a \isi{relative clause}

I begin by describing the formal structure of relative clauses. Formally, they are similar to content questions because the relative pronouns are identical to the interrogative pronouns.\footnote{I refer the reader to {\S}\ref{interrogatives-sec} for a description of interrogative pronouns. See especially \tabref{interrogatives-table}, but also the interrogatives in \tabref{demonstratives-table}.} We could say that interrogatives function as relative pronouns, which is why I do not \isi{gloss} them as \textsc{rel} in the following examples. Instead, I \isi{gloss} them in the same manner, in which pronouns in interrogative function are glossed. However, relative clauses are semantically distinct from content questions because the answer to the question is already given in the form of the NP\textsubscript{\textsc{mat}}. Relative clauses are also syntactically different from content questions in that the relative \isi{pronoun} has to occur as the second element (\figref{relclause}). Such a restriction does not apply to content questions. This is illustrated in (\ref{ex541}-\ref{ex543}), where the \isi{relative clause} in each example is printed in bold face.

Example (\ref{ex541}) comes from a hunting story where the narrator had encountered a spirit which began chasing him. In the example, the relative \isi{pronoun} \emph{maf} follows the \isi{pronoun} \emph{nzä}. The \isi{relative clause} follows the NP\textsubscript{\textsc{mat}} \emph{ane kabe} `that man'.

\begin{exe}
	\ex \emph{nze nima ``byannor ane kabe fof \textbf{nzä maf wonrsoknwr}.''}\\
	\gll nze nima b=yan\stem{nor} ane kabe fof nzä maf won\stem{rsokn}wr\\
	\Fsg.{\Erg} {\Quot} \Med=\Tsg.\Masc:\Sbj:\Nonpast:\Ipfv:\Venit/shout {\Dem} man {\Emph} \Fsg.{\Abs} who.\Sg.{\Erg} \Stsg:\Sbj>\Fsg:\Obj:\Nonpast:\Ipfv:\Venit/bother\\
	\trans `I said: ``He is shouting out there. This man who bothers me.'''\\\Corpus{tci20111119-03}{ABB \#164-166}
	\label{ex541}
\end{exe}

In example (\ref{ex542}), the speaker describes why he did not pay attention to a fire that almost burned his garden. In the example, the relative \isi{pronoun} \emph{mane} is preceded by the \isi{noun phrase} \emph{mnz tharthar} `side of the house'. This is an adjoined \isi{relative clause} because it is outside the NP\textsubscript{\textsc{mat}}, which in this case is \emph{mni} `fire', whose antecedent is understood from the context.

\begin{exe}
	\ex \emph{ni fi ane zumarwrme \textbf{mnz thartharen mane zfrärm}.}\\
	\gll ni fi ane zu\stem{mar}wrme mnz tharthar=en mane zf\stem{rä}rm\\
	{\Fnsg} but {\Dem} \Fpl:\Sbj>\Tsg.\F:\Obj:\Pst:\Dur/look house side={\Loc} which(\Abs) \Tsg.\F:\Sbj:\Pst:\Dur/be\\
	\trans `But we were looking at that (fire), which was on the side of the house.'\\\Corpus{tci20120922-24}{STK \#5}
	\label{ex542}
\end{exe}

Finally, in (\ref{ex543}) the speaker describes how he was trying to remove a burning tree from his garden fence. The relative \isi{pronoun} \emph{mane} follows the \isi{ergative} marked \emph{wämne} `tree'. This is an adjoined \isi{relative clause} because it is outside the matrix NP, which in this case is \emph{ŋarake} `garden fence', whose antecedent is understood from the context.

\begin{exe}
	\ex \emph{kma wämne ane fof kwakarkwé ane fof \textbf{wämnef mane thänarfa} ... keke watikthémäre.}\\
	\gll kma wämne ane fof kwa\stem{kark}wé ane fof wämne=f mane thä\stem{narf}a (.) keke watik-thé=märe\\
	{\Pot} tree {\Dem} {\Emph} \Fsg:\Sbj:\Rpst:\Ipfv/pull {\Dem} {\Emph} tree={\Erg} which(\Abs) \Sg:\Sbj>\Stpl:\Obj:\Pst:\Ipfv/press.down (.) {\Neg} enough-\Adlzr={\Priv}\\
	\trans `I should have pulled that tree (from the fence), which the tree was pushing down. No, (I was) not (strong) enough!'\Corpus{tci20120922-24}{MAA \#42-43}
	\label{ex543}
\end{exe}

A second rule is needed for examples where the relative \isi{pronoun} occurs in initial position of the \isi{relative clause}. Although this is possible, such examples are much less frequent than the second position. The relative \isi{pronoun} can occur in first position only (i) if it is preceded by the NP\textsubscript{\textsc{mat}} (\ref{ex544} and \ref{ex545}), or (ii) if the only other element in the \isi{relative clause} is the verb (\ref{ex546}).

\begin{exe}
	\ex \emph{bundbonzo rä \textbf{mane zawokth}.}\\
	\gll bundbo=nzo \stem{rä} mane za\stem{wokth}\\
	\Ssg.\All={\Only} \Tsg.\F:\Nonpast:\Ipfv/be which(\Abs) \Ssg:\Sbj:\Imp:\Pfv/choose\\
	\trans `It is up to you which one you choose!'\Corpus{tci20111004}{RMA \#528}
	\label{ex546}
\end{exe}

Example (\ref{ex544}) is taken from a picture stimulus task. One of the participants is correcting the other. Note that the \ili{English} translation is misleading. The noun phrase \emph{mafanemäwä waniwani} is a complex noun phrase and the relative \isi{pronoun} \emph{mafanema} is marked flagged with the \isi{characteristic} \isi{case} in adnominal function. Thus, \emph{mafanemäwä waniwani} should be translated not as \isi{genitive} `whose picture', but as origin `picture of/about who'.

\begin{exe}
	\ex \emph{sukawi, nima keke rä. zane fthéthamane yé ... ane kabe fof \textbf{mafanemäwä waniwani zöbthé nzünmarwre}.}\\
	\gll sukawi nima keke \stem{rä} zane fthé=thamane \stem{yé} (.) ane kabe fof mafane=ma=wä waniwani zöbthé nz=wn\stem{mar}wre\\
	sukawi {like.this} {\Neg} \Tsg.\F:\Sbj:\Nonpast:\Ipfv/be \Dem:{\Prox} when=\Temp.{\Poss} \Tsg.\Masc:\Sbj:\Nonpast:\Ipfv/be (.) {\Dem} man {\Emph} who.\Sg.\Poss=\Char={\Emph} picture first \Immpst=\Fpl:\Sbj>\Tsg.\F:\Obj:\Nonpast:\Ipfv:\Venit/see\\
	\trans `Sukawi, it is not like that. This is from that time ... really this man whose picture we just saw before.'\Corpus{tci20111004}{RMA \#194}
	\label{ex544}
\end{exe}

In example (\ref{ex545}), the relative \isi{pronoun} occurs initially following the NP\textsubscript{\textsc{mat}} \emph{dödö}.

\begin{exe}
	\ex \emph{ane fathnzo zfrärm wämne keke ... dödönzo ... dödö \textbf{maneme ŋarenwre fath}.}\\
	\gll ane fath=nzo zf\stem{rä}rm wämne keke (.) dödö=nzo (.) dödö mane=me ŋa\stem{ren}wre fath\\
	{\Dem} clearing={\Only} \Tsg.\F:\Sbj:\Pst:\Dur/be tree {\Neg} (.) dödö={\Only} (.) dödö which={\Ins} \Fpl:\Sbj:\Nonpast:\Ipfv/sweep clearing\\
	\trans `This was a clearing, no trees ... only dödö (Sida acuta) ... dödö, with which we sweep the clear places.' \Corpus{tci20120821-02}{LNA \#25-27}
	\label{ex545}
\end{exe}

Next, I describe which kinds of argument roles can be relativised in the \isi{matrix clause} (NP\textsubscript{\textsc{mat}}), and which can occur in the \isi{relative clause} (NP\textsubscript{\textsc{rel}}). As the examples in this section show, there is virtually no restriction on the possible argument roles. NP\textsubscript{\textsc{rel}} is expressed by the relative \isi{pronoun}, which can inflect for all cases ({\S}\ref{interrogatives-sec}). The examples given in this section include the following cases: \isi{ergative} (\ref{ex541}), \isi{absolutive} (\ref{ex542}, \ref{ex543}, \ref{ex546}), \isi{characteristic} (\ref{ex544}), \isi{dative} (\ref{ex551}), \isi{locative} (\ref{ex547}), and \isi{instrumental} (\ref{ex545}).

It is harder to determine the argument role of NP\textsubscript{\textsc{mat}} because its presence is optional. We saw in (\ref{ex543}) that the relative \isi{pronoun} \emph{mane} referred to the fence, which the burning tree had pushed down. But this is not expressed by a noun phrase, nor is the fence indexed in the verb of the \isi{matrix clause}. We only know about it from the preceding context of the story, and the \isi{plural} prefix in the verb of the \isi{relative clause}.\footnote{The word \emph{ŋarake} `fence' is frequently used in the plural.} In (\ref{ex542}), the fire is not expressed as a noun phrase, but the prefix of the verbs \emph{zumarwrme} `we were seeing it' and \emph{zfrärm} `it was', both indexing \emph{mni} `fire' (3\Sg.\F). However, we can always determine the argument role of NP\textsubscript{\textsc{mat}} from the context. The following argument roles are found in the examples given in this section: the single argument of an intransitive verb (\ref{ex541}), \isi{patient} (\ref{ex542}), location (\ref{ex547}), discourse \isi{topic} (\ref{ex548}), actor (\ref{ex549}), and \isi{recipient} (\ref{ex550}).

\begin{exe}
	\ex \emph{mni wthomonwrth \textbf{yfö mä zfrärm}.}\\
	\gll mni w\stem{thomon}wrth yfö mä zf\stem{rä}rm\\
	fire \Stpl:\Sbj>\Tsg.\F:\Obj:\Nonpast:\Ipfv/pile.up.fire hole where \Tsg.\F:\Sbj:\Pst:\Dur\\
	\trans `They prepare the fire where the hole was.'\Corpus{tci20120901-01}{MAK \#153-154}
	\label{ex547}
\end{exe}
\begin{exe}
	\ex \emph{anema nä katan zokwasi nimamenzo fof zfrä ... nzone katan masisma ... ane mnima \textbf{zöbthé mane zukonzrmth kidn o zfth mni.}}\\
	\gll ane=ma nä katan zokwasi nima=me=nzo fof zf\stem{rä} (.) nzone katan masis=ma (.) ane mni=ma zöbthé mane zu\stem{ko}nzrmth kidn o zfth mni\\
	\Dem={\Char} {\Indf} small words {like.this}=\Ins={\Only} {\Emph} \Tsg.\F:\Rpst:\Ipfv/be (.) \Fsg.{\Poss} small matches={\Char} (.) {\Dem} fire={\Char} before which(\Abs) \Stpl:\Sbj>\Tsg.\F:\Obj:\Pst:\Dur/speak kidn or base fire\\
	\trans `This was another small story like this ... about my small matches ... about the fire, which they were calling Kidn or base fire before.'\Corpus{tci20120909-06}{KAB \#126-127}
	\label{ex548}
\end{exe}
\begin{exe}
	\ex \emph{kabef tauri samg \textbf{ŋatha tüfrkarä mane yé}.}\\
	\gll kabe=f tauri sa\stem{mg} ŋatha tüfr=karä mane \stem{yé}\\
	man=\Erg.{\Sg} wallaby \Stsg:\Sbj>\Tsg.\Masc:\Obj:\Rpst:\Pfv/shoot dog plenty={\Prop} who(\Abs) \Tsg.\Masc:\Sbj:\Nonpast:\Ipfv/be\\
	\trans `The man who has many dogs shot the wallaby.'{\hspace*{1pt}\hfill{\footnotesize{[overheard]}}}
	\label{ex549}
\end{exe}
\begin{exe}
	\ex \emph{be kmam nabi thar nafanm \textbf{mane wtri ŋarärth}.}\\
	\gll be kma=m nabi tha\stem{r} nafanm mane wtri ŋa\stem{rä}rth\\
	\Ssg.{\Erg} \Pot={\Appr} bow \Ssg:\Sbj>\Stpl:\Io:\Imp:\Pfv/give \Tnsg:{\Dat} who(\Abs) fear \Stpl:\Sbj:\Nonpast:\Ipfv/do\\
	\trans `You must not give a bow to those who are fearful.'{\hspace*{1pt}\hfill{\footnotesize{[overheard]}}}
	\label{ex550}
\end{exe}

It is also possible that the \isi{relative clause} is free in the sense that it refers to the whole \isi{matrix clause} and not to a particular \isi{nominal} (\citealt[213]{Andrews:2007relclauses}). Examples are given in (\ref{ex546}) and (\ref{ex551}).

\begin{exe}
	\ex \emph{be fam kwot karäré \textbf{tosin mafan kwa yarithr}.}\\
	\gll be fam kwot ka\stem{rä}ré tosin mafan kwa ya\stem{ri}thr\\
	\Ssg.{\Erg} thought properly \Ssg:\Sbj:\Imp:\Ipfv/do torch who.\Sg.{\Dat} {\Fut} \Stsg:\Sbj>\Tsg.\Masc:\Io:\Nonpast:\Ipfv/give\\
	\trans `You have to think properly to whom you will give the torch.'{\hspace*{1pt}\hfill{\footnotesize{[overheard]}}}
	\label{ex551}
\end{exe}

The NP\textsubscript{\textsc{mat}} can be fronted together with the \isi{relative clause}, as in (\ref{ex556}). This is commonly used for topicalisation. After showing me a traditional fishing basket, the speaker shifts the topic to more modern methods of fishing. The NP\textsubscript{\textsc{mat}} is \emph{net} in (\ref{ex556}).

\begin{exe}
	\ex \emph{wati, net \textbf{ane mane erä} markaianeme erä ane.}\\
	\gll wati net ane mane e\stem{rä} markai=aneme e\stem{rä} ane\\
	then {fishing.net} {\Dem} which \Stpl:\Sbj:\Nonpast:\Ipfv/be {outsider}=\Poss.{\Nsg} \Stpl:\Sbj:\Nonpast:\Ipfv/be \Dem\\
	\trans `Okay, as for the fishing nets, they are the white man's (things).'\\\Corpus{tci20120906}{SKK \#53-54}
	\label{ex556}
\end{exe}

In example (\ref{ex555}), the speaker talks about food taboos. He makes the point that a particular woman in the village has grown very old because she has always respected those food taboos. The \isi{relative clause} (in bold) marks a shift in topic to all those people who did not respect the food taboos. The antecedent of the \isi{relative clause} is omitted, since it is understood from context. It is different from the old woman, as can been seen by that fact that all following verbs index a plural argument, e.g. \emph{kwarirkwrmth} `they respected' and \emph{thufathwrm} `it grabbed them'.

\begin{exe}
	\ex \emph{watik, fi komnzo zathfär ... bänema fof nima rirksima brä ... nima kwarirkwrm ... fi \textbf{mafa keke kwarirkwrmth} ... watik, tekmär esufakwa kwikkwikf thufathwrm.}\\
	\gll watik fi komnzo za\stem{thfär} (.) bäne=ma fof nima rirk-si=ma b=\stem{rä} (.) nima kwa\stem{rirk}wrm (.) fi mafa keke kwa\stem{rirk}wrmth (.) watik tek=mär e\stem{sufak}wa kwik-kwik=f thu\stem{fath}wrm\\
	then \Third.{\Abs} only \Tsg.\F:\Sbj:\Rpst:\Pfv/jump (.) \Dem:\Med={\Char} {\Emph} {like.this} respect-\Nmlz={\Char} \Med=\Tsg.\F:\Sbj:\Nonpast:\Ipfv/be (.) {like.this} \Stsg:\Sbj:\Pst:\Dur/respect (.) but who.\Nsg.{\Erg} {\Neg} \Stpl:\Sbj:\Pst:\Dur/respect (.) then duration={\Priv} \Stpl:\Sbj:\Pst:\Ipfv/grow.old \Redup-sickness={\Erg} \Stsg:\Sbj>\Stpl:\Obj:\Pst:\Dur/hold\\
	\trans `She just lives on ... because of her respect ... she was respecting (the law) ... but those who did not respect (the law) ... well, they grew old quickly and they got sick.'\Corpus{tci20120922-26}{DAK \#22-27}
	\label{ex555}
\end{exe}

The fronted \isi{relative clause} as a topicalisation strategy is described in detail in {\S}\ref{info-cleft}.

\section{Conditional and time clauses}\label{condiclauses}

Conditional and time clauses are expressed in the same way, only the context resolves which of the two is meant. I will use the term \isi{conditional} in the subsequent description to cover both. Conditionals are formed by using the word \emph{fthé} `when, if'. Note that \emph{fthé} is not a subordinator per se because it can also occur in independent sentences with the meaning `that is when'. Thus, \emph{fthé} is required for a \isi{conditional}, but it is not sufficient. The word \emph{fthé} is used in the \isi{clause} which sets up the \isi{conditional}, often called the if-\isi{clause} (\citealt[255]{Thompson:2007relclauses}). The second clause, often called the then-\isi{clause}, receives no special marking.\footnote{Note that for time clauses, this would be the when-clause and then-clause respectively.} The clearest \isi{conditional} reading is found with the second \isi{person}. While an \isi{irrealis} verb inflection is also possible, in most cases, the \isi{imperative} is used in one of the two clauses, as in example (\ref{ex646}) and (\ref{ex647}).

\begin{exe}
	\ex \emph{ŋanzmäre fthé gnräré frasi kwa nrä.}\\
	\gll ŋanz=märe fthé gn\stem{rä}ré frasi kwa n\stem{rä}\\
	row={\Priv} when \Ssg:\Sbj:\Imp:\Ipfv/be hunger {\Fut} \Ssg:\Sbj:\Nonpast:\Ipfv/be\\
	\trans `If you are without a row (of yams in the garden), you will be hungry.'\\\Corpus{tci20130822-08}{LNA \#17}
	\label{ex646}
\end{exe}
\begin{exe}
	\ex \emph{wati, zena fthé zanmar ... yusi fr mane rä ... ane fof nzone farsima rä.}\\
	\gll wati zena fthé zan\stem{mar} (.) yusi fr mane \stem{rä} (.) ane fof nzone far-si=ma \stem{rä}\\
	then today when \Ssg:\Sbj>\Tsg.\F:\Obj:\Imp:\Pfv/see (.) grass stem which \Tsg.\F:\Sbj:\Nonpast:\Ipfv/be (.) {\Dem} {\Emph} \Fsg.{\Poss} fell-\Nmlz={\Char} \Tsg.\F:\Sbj:\Nonpast:\Ipfv/be\\
	\trans `If/When you look at it ... the grassland there ... that is from my cutting down (the trees).'\Corpus{tci20120805-01}{ABB \#614}
	\label{ex647}
\end{exe}

Both clauses can be marked for various TAM categories, for example \isi{imperative} in the if-clause and irrealis in the when-clause in (\ref{ex648}), where the speaker shows me the proper use of a toy bullroarer.

\begin{exe}
	\ex \emph{zbo fthé sakwr fefen o wämnen ... keke kwa srannor.}\\
	\gll zbo fthé sa\stem{kwr} fefe=n o wämn=en (.) keke kwa sran\stem{nor}\\
	\Prox.{\All} when \Ssg:\Sbj>\Tsg.\Masc:\Obj:\Imp:\Pfv/hit body={\Loc} or tree={\Loc} (.) {\Neg} {\Fut} \Tsg.\Masc:\Sbj:\Irr:\Ipfv/shout\\
	\trans `If you hit it here against the body or against a tree, it will not make a sound.'\\\Corpus{tci20120914}{RMA \#31-33}
	\label{ex648}
\end{exe}

In (\ref{ex649}), all clauses are in \isi{past} \isi{durative}, yet the \isi{conditional} construction can be interpreted as both actual (as in the translation) or counterfactual (`... the story man would have shot them with magic').

\begin{exe}
	\ex \emph{zizi zä keke kwarafinzrmth ŋoŋoyamkarä ... bänema fthé ŋoŋoyamkarä kwarafinzrmth menzf thfruthrm ... bthanme.}\\
	\gll zizi zä keke kwa\stem{rafi}nzrmth ŋoŋoyam=karä (.) bäne=ma fthé ŋoŋoyam=karä kwa\stem{rafi}nzrmth menz=f thf\stem{ru}thrm (.) bthan=me\\
	afternoon {\Prox} {\Neg} \Stpl:\Sbj:\Pst:\Dur/paddle noise={\Prop} ... \Recog={\Char} when noise={\Prop} \Stpl:\Sbj:\Pst:\Dur/paddle {story.man}=\Erg.{\Sg} \Stsg:\Sbj>\Stpl:\Obj:\Pst:\Dur/shoot (.) magic=\Ins\\
	\trans `They did not paddle her with a lot of noise in the afternoon. If they were paddling making a lot of noise, the story man was shooting them with.'\\\Corpus{tci20120922-19}{DAK \#14-15}
	\label{ex649}
\end{exe}

\section{Direct speech and thought}\label{directspeechthought}

Direct speech is a common construction in Komnzo. In most cases, \isi{direct speech} is introduced by a speech verb, for example \emph{ko-} `speak' or \emph{na-} `say', and the manner \isi{demonstrative} \emph{nima} `like this' ({\S}\ref{manner-demonstrative-subsec}). Direct speech receives a separate intonation contour and the whole clause is often produced at a slightly higher pitch to indicate that the speaker is taking on another person's role. An example is given in (\ref{ex650}).

\begin{exe}
	\ex 
	\label{ex650}
	\begin{xlist}
		\ex \emph{watik, srank kma \textbf{sakora nima} ``srank, ni krafare!''}\\
		\gll watik srank kma sa\stem{kor}a nima srank ni kra\stem{far}e\\
		then srank {\Pot} \Sg:\Sbj>\Tsg.\Masc:\Obj:\Pst:\Pfv/speak {\Quot} srank {\Fnsg} \Stdu:\Sbj:\Irr:\Pfv/set.off\\
		\trans `Well, he tried to tell Srank: ``Srank, we go!''
		\ex \emph{srankf \textbf{zenaftha} ``keke efoth zizi fefe rä nzä kayé woräro.''}\\
		\gll srank=f ze\stem{nafth}a keke efoth zizi fefe \stem{rä} nzä kayé wo\stem{rä}ro\\
		srank=\Erg.{\Sg} \Sg:\Sbj:\Pst:\Ipfv/say {\Neg} sun afternoon really \Tsg.\F:\Sbj:\Nonpast:\Ipfv/be \Fsg.{\Abs} tomorrow \Fsg:\Sbj:\Nonpast:\Ipfv:\Andat/be\\
		\trans Srank said: ``No, it is late afternoon. I will go tomorrow.'''\\\Corpus{tci20111107-01}{MAK \#44-45}
	\end{xlist}
\end{exe}

The manner \isi{demonstrative} functions as a \isi{quotative} marker. It can introduce \isi{direct speech} without a speech verb, as in example (\ref{ex651}).

\begin{exe}
	\ex \emph{naf \textbf{nima} ``nakre! wimäsen mni bŋasog.''}\\
	\gll naf nima nakre wimäs=en mni b=ŋa\stem{sog}\\
	\Tsg.{\Erg} {\Quot} nakre {mango.tree}={\Loc} fire \Med=\Stsg:\Sbj:\Nonpast:\Ipfv/climb\\
	\trans `He said: ``Nakre! The fire is climbing up the mango tree.'''\\\Corpus{tci20130901-04}{RNA \#152-153}
	\label{ex651}
\end{exe}

There is no dedicated construction for \isi{indirect speech}. Indirect speech equivalents can be expressed by a speech verb with an adverbial adjunct (\ref{ex652}) or a clause connected with \emph{mon} `how' (\ref{ex653}).

\begin{exe}
	\ex \emph{naf ŋanafr \textbf{drdr mäyogsir}.}\\
	\gll naf ŋa\stem{na}fr drdr mäyog-si=r\\
	\Tsg.{\Erg} \Stsg:\Sbj:\Nonpast:\Ipfv/speak old.garden repeat-\Nmlz=\Purp\\
	\trans `She said to continue the old garden.'\Corpus{tci20130823-06}{STK 161}
	\label{ex652}
\end{exe}
\begin{exe}
	\ex \emph{emothf ŋatrikwr \textbf{monme zfnzr}.}\\
	\gll emoth=f ŋa\stem{trik}wr mon=me z\stem{fn}nzr\\
	girl=\Erg.{\Sg} \Stsg:\Sbj:\Nonpast:\Ipfv/tell how={\Ins} \Stsg:\Sbj>\Tsg.\F:\Obj:\Pst:\Ipfv/hit\\
	\trans `The girl reports how he hit her.'\Corpus{tci20120925}{MAE \#102}
	\label{ex653}
\end{exe}

An individual's inner thoughts are treated like direct and \isi{indirect speech}. Hence, we find examples like (\ref{ex654}) and (\ref{ex655}), which mirror what has been described above for speech. The only difference lies in the framing expression, which is often the \isi{light verb} construction \emph{fam} `thought' + \emph{rä-} `do'.

\begin{exe}
	\ex \emph{fam zära ``kar bä rä a kar töna fobo fof wyak fof.''}\\
	\gll fam zä\stem{r}a kar bä \stem{rä} a kar töna fobo fof w\stem{yak} fof\\
	thought \Sg:\Sbj:\Pst:\Pfv/do village \Med{} \Tsg.\F:\Sbj:\Nonpast:\Ipfv/be and place {high.ground} \Dist.{\All} {\Emph} \Fsg:\Sbj:\Nonpast:\Ipfv/walk {\Emph}\\
	\trans `He thought ``There is a village. I will walk there to the high ground.'''\\\Corpus{tci20131013-01}{ABB \#259}
	\label{ex654}
\end{exe}
\begin{exe}
	\ex \emph{fam ane fof ŋarär monme sufnzrmth monme santhbath.}\\
	\gll fam ane fof ŋa\stem{rä}r mon=me su\stem{fn}nzrmth mon=me san\stem{thb}ath\\
	thought {\Dem} {\Emph} \Stsg:\Sbj:\Nonpast:\Ipfv/do how={\Ins} \Stpl:\Sbj>\Tsg.\Masc:\Obj:\Pst:\Dur/hit how={\Ins} \Stpl:\Sbj>\Tsg.\Masc:\Obj:\Pst:\Dur:\Venit/put.inside\\
	\trans `He is thinking how they were hitting him and how they locked him up.'\\\Corpus{tci20111004}{RMA \#457}
	\label{ex655}
\end{exe}