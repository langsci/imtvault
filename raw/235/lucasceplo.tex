\documentclass[output=paper]{langsci/langscibook}
\ChapterDOI{10.5281/zenodo.3744525}
\title{Maltese}

\author{Christopher Lucas\affiliation{SOAS University of London}\lastand Slavomír Čéplö\affiliation{Institute of Oriental Studies, Slovak Academy of Sciences/IMAFO Abteilung Byzanzforschung, Österreichische Akademie der Wissenschaften}}

\abstract{This chapter presents an overview of the most prominent contact-induced developments in the history of Maltese, a language which is genetically a variety of Arabic, but which has undergone significant changes, largely as a result of lengthy contact with Sicilian, Italian, and English. We first address the precise affiliation of Maltese and the nature of the historical and ongoing contact situations, before detailing relevant developments in the realms of phonology, inflectional and derivational morphology, syntax, and lexicon.}

% \IfFileExists{../localcommands.tex}{
%   \input{../localpackages}
%   %Copy this to localcommands.tex

\usepackage[english]{babel}
\usepackage{amsmath}
\usepackage{amssymb,amsfonts,textcomp}
\usepackage{array}
\usepackage{hhline}
\usepackage{hyperref}

\newenvironment{styleStandard}{}{}
\newenvironment{stylelsAbstract}{}{}
\newenvironment{stylelsSectioni}{}{}
\newenvironment{stylelsSectionii}{}{}
\newenvironment{stylelsBulletList}{}{}
\newenvironment{styleBibliographyi}{}{}
\newenvironment{listWWNumxxvleveli}{}{}
\newenvironment{listWWNumxxvlevelii}{}{}
\newenvironment{listWWNumxxvleveliii}{}{}
\newenvironment{listWWNumxxvleveliv}{}{}
\newenvironment{listWWNumixleveli}{}{}
\newenvironment{listWWNumixlevelii}{}{}
\newenvironment{listWWNumixleveliii}{}{}
\newenvironment{listWWNumixleveliv}{}{}

\newcommand\textstyleListLabelxvi[1]{#1}
\newcommand\labellistWWNumxxvleveli{\thelistWWNumxxvleveli.}
\newcommand\labellistWWNumxxvlevelii{\thelistWWNumxxvlevelii.}
\newcommand\labellistWWNumxxvleveliii{\thelistWWNumxxvleveliii.}
\newcommand\labellistWWNumxxvleveliv{\thelistWWNumxxvleveliv.}
\newcommand\labellistWWNumixleveli{[F0B7?]}
\newcommand\labellistWWNumixlevelii{\textstyleListLabelxvi{o}}
\newcommand\labellistWWNumixleveliii{[F0A7?]}
\newcommand\labellistWWNumixleveliv{[F0B7?]}

\newcounter{listWWNumxxvleveli}
\newcounter{listWWNumxxvlevelii}[listWWNumxxvleveli]
\newcounter{listWWNumxxvleveliii}[listWWNumxxvlevelii]
\newcounter{listWWNumxxvleveliv}[listWWNumxxvleveliii]
\newcounter{itemize} 
%   \input{../localhyphenation}
%   \togglepaper[1]%%chapternumber
% }{}

\begin{document}
\maketitle
\section{Maltese and Arabic}\largerpage %Chris
From a historical point of view, Maltese is a variety of spoken \ili{Arabic}, albeit one that has undergone far-reaching changes as a result of sustained and \isi{intensive} contact with Italo-\ili{Romance} varieties, and more recently also with \ili{English}. This is a fact about which there is no controversy among contemporary linguists. It should be noted, however, that a mix of social, cultural, historical, political, and indeed linguistic factors has led to a situation in which many Maltese people today view their language as \ili{Semitic}, but not a type of \ili{Arabic}. Since we are concerned here only with the historical perspective, we will not dwell on the vexed question of whether or not contemporary Maltese should be classified as an ``\ili{Arabic} dialect''.\footnote{Note that Maltese itself has a number of different dialects, one of which -- that of the major towns, and the variety used in media, literature and administration -- is referred to as  Standard Maltese. Except where specified, this chapter deals exclusively with the standard variety of Maltese.} Suffice it to say that the idea, first popularized by \citeauthor{desoldanis1750} (\citeyear{desoldanis1750}) and \cite{vassalli1791}, that Maltese is a variety of \ili{Phoenician} or \ili{Punic}, has been shown since at least since \cite{gesenius1810} and \cite{desacy1829} to be entirely without merit.

Since the \isi{Phoenicians} and then the \isi{Carthaginians} occupied Malta for much of the first millennium BCE, followed by Roman and Byzantine occupation for much of the first millennium CE, it would seem \textit{prima facie} likely that elements of the languages of these occupiers would survive into contemporary Maltese. \cite{brincat1995} shows, however, based on the account of \ia{al-Ḥimyarī@al-Ḥimyarī}al-Ḥimyarī, that Malta was to all intents and purposes uninhabited in the period between its conquest by the Arabs in 870 CE and the first concerted efforts at colonization by \ili{Arabic}-speaking \isi{Muslims} in 1048--1049 CE. It is for this reason that the \ili{Semitic} component of Maltese phonology, morphology, syntax and lexicon is \ili{Arabic} and \ili{Arabic} only (see also \citealt{grech1961}).


As for the provenance of the \ili{Arabic} component of contemporary Maltese, there is no doubt that the most important source is a variety of \ili{Maghrebi} (\ili{Western}) \ili{Arabic}. This is evident from grammatical features such as: the pan-\ili{Maghrebi} \isi{extension} to the singular of the first-person \textit{n-} prefix of the imperfect verbal paradigm
(see \tabref{niktib}); the loss of a \isi{gender} distinction in the second person singular, in pronouns and both perfect and imperfect verbs, as in urban \ili{Tunisian} \ili{Arabic} varieties \citep{Gibson2011}; variable rearticulation of the \isi{definite} \isi{article} on postnominal adjectives in \isi{definite} noun phrases, as in \REF{ex:optionalarticle} (cf. \citealt{Gatt2018}), found also in \ili{Casablanca} \ili{Arabic} (\citealt[205]{Harrell2004}); and the \textit{-il} suffix of the \isi{numerals} `eleven' to `nineteen' in determiner use, as in \REF{ex:apostles}, which also occurs in the \ili{Arabic} dialects of \ili{Casablanca} \citep{Caubet2011} and \ili{Tlemcen} \citep{Taine-Cheikh2011numerals}.\footnote{Unless otherwise specified, all numbered examples present data from Maltese. All Maltese examples in this chapter are rendered using  Standard Maltese \isi{orthography}.}\largerpage


\begin{table}
\caption{First-person imperfect `write' in Eastern and Western Arabic}
\label{niktib}
\fittable{ \begin{tabular}{lllll}
  \lsptoprule
         & \multicolumn{2}{c}{Eastern}       & \multicolumn{2}{c}{Western} \\\cmidrule(lr){2-3}\cmidrule(lr){4-5}
         & \ili{Classical Arabic} & \ili{Baghdad} \ili{Arabic} & \ili{Casablanca} \ili{Arabic} & Maltese \\
           \midrule
Singular & \textit{ʔaktub}            & \textit{aktib}          & \textit{nəktəb}            & \textit{nikteb}  \\
Plural   & \textit{naktub}           & \textit{niktib}         & \textit{nkətbu}            & \textit{niktbu}  \\
\lspbottomrule
\end{tabular}}
\end{table}

\begin{multicols}{2}
\ea\label{ex:optionalarticle}
\gll il-kelb (l-)abjad\\
     \textsc{def}-dog (\textsc{def})-white\\
\glt `the white dog'
\ex\label{ex:apostles}
\gll it-tnax-\textbf{il} appostlu\\
     \textsc{def}-twelve-\textsc{dep} apostle\\
\glt `the twelve apostles'
\z
\end{multicols}

Narrowing matters down further, Zammit's (\citeyear{zammit2014}) study of lexicon shared between Maltese and the \ili{Arabic} dialect of \ili{Sfax} offers yet more support (see also \citealt{Vanhove1998}) for the geographically unsurprising conclusion that Maltese is more closely related to the traditional (so-called pre-Hilalian; see Benkato, this volume)\ia{Benkato, Adam@Benkato, Adam} urban \ili{Tunisian} dialects than to any other extant \ili{Arabic} variety. This is not to suggest, however, that the \ili{Arabic} component of Maltese resembles these dialects in all respects. \cite{Borg1996} lists a number of areas in which Maltese accords more closely with \ili{Levantine} \ili{Arabic} dialects than with those of the Maghreb. But the social and political history of Malta after the end of direct Arab rule in 1127 CE is such that most or all of these similarities should be understood as the failure of Maltese to participate in innovations that later spread through the mainland \ili{Maghrebi} varieties, and not as evidence of influence of \ili{Eastern} \ili{Arabic} on the \isi{formation} of Maltese.

\section{Contact with Italo-Romance and English}
\subsection{Italo-Romance}

A comprehensive history of immigration to Malta in the medieval period is yet to be written (if indeed such a history is possible at all, given the apparently scarce documentary evidence). It is therefore impossible to give precise details of the sociolinguistic conditions under which the \ili{Arabic} variety spoken in Malta came into contact with varieties of Italo-\ili{Romance} in the course of the second millennium. We can, however, sketch the broad outlines of this process, and make some reasonable inferences.

The \ili{Arabic}-speaking settlers who colonized Malta in 1048--1049 CE can be assumed to have come from either \isi{Sicily} or southern Italy or both \citep[22]{brincat1995}, but in any case it seems likely that at least some of these came speaking a variety of \ili{Sicilian} in addition to \ili{Arabic}. Even after Malta was brought under Norman control in 1127 CE by \iai{Roger II of Sicily}, and went on to be part of the Kingdom of \isi{Sicily}, there does not seem to have been a large-scale immigration of non-\ili{Arabic} speakers to Malta at any point, a fact which is of course consistent with the survival of the Maltese language until today. Unsurprisingly from a geographical and political perspective, what immigration there was appears to have come overwhelmingly from \isi{Sicily} and southern Italy, with lesser numbers coming also from Spain (\citealt[134, 289]{ballou1893}; \citealt[43--46]{blouet1967}; \citealt{fiorini1986}; \citealt[26--32]{goodwin2002}).


Comprising mostly soldiers, craftsmen and churchmen of various types, it would appear that this immigration was disproportionately male. In addition to families in which the only language spoken was Maltese, there must, therefore, have been significant numbers of families in medieval Malta in which the father spoke only \ili{Sicilian} natively and the mother spoke only Maltese natively, with communication necessarily involving second-language speech by one or both parents. Children of such families would therefore have been exposed minimally to native and non-native Maltese speech and native \ili{Sicilian} speech.

From the perspective of Van Coetsem's (\citeyear{VanCoetsem1988,VanCoetsem2000}) framework for understanding contact-induced change, therefore, it seems highly likely that \isi{transfer} from \ili{Sicilian} to Maltese occurred both through \isi{imposition} under \isi{source-language} agentivity (by L1 \ili{Sicilian} speakers) and borrowing under \isi{recipient-language} agentivity (by L1 Maltese speakers).


There is no doubt that, alongside \ili{Sicilian}, (Tuscan) \ili{Italian} had an important place in Maltese life over many centuries, starting at the latest in 1530, when it became the official language of government under the regime of the Knights of Malta. But as Comrie \& Spagnol (\citeyear[316]{comriespagnol2016}) point out, \ili{Italian} did not gain a foothold at the expense of \ili{Sicilian} among bilingual Maltese until the later eighteenth century, and given its social function as a vehicle for government, education and high culture, rather than the native language of a significant proportion of ordinary Maltese, it is reasonable to say that \isi{transfer} from \ili{Italian} will have been mediated predominantly by borrowing under \isi{recipient-language} agentivity.

\subsection{English} \label{English}
Starting in 1800, when Malta became a protectorate of the British Empire, \ili{English} gradually began to supplant \ili{Italian} as the language of government, education and high culture, being joined in that role by the Maltese language itself only in the last few decades. \ili{English} is now widely spoken in Malta: according to 2011 \isi{census data} (\citealt[149]{census2011}), 94.6\% of the population of Malta reported speaking Maltese ``well" or ``average[ly]", while 82.1\% reported the same for \ili{English}. \ili{English} is a native language for only a very small percentage of Maltese residents, however: \cite{scirihavassallo2006} put the figure at 2\%. As with \ili{Italian}, then, \isi{transfer} from \ili{English} to Maltese will overwhelmingly have occurred through borrowing under \isi{recipient-language} agentivity. With the Maltese variety of \ili{English}, the reverse is true of course: here the \isi{transfer} from \ili{English} to Maltese will have been almost exclusively \isi{imposition} under \isi{source-language} agentivity by native speakers of Maltese, resulting in such hallmark features of \ili{Maltese English} as word-final obstruent \isi{devoicing} (cf. §\ref{mergers} below), and the use of \textit{but} in clause-final position (\citealt[527]{Lucas2015}).

Given that \isi{transfer} from \ili{English} was and is restricted to borrowing in Van Coetsem's sense,\ia{Van Coetsem, Frans@Van Coetsem, Frans} while the more extensive and long-lasting contact with \ili{Sicilian} will have involved both borrowing and \isi{imposition}, it is not surprising that a picture will emerge in the following sections whereby Italo-\ili{Romance} dominates as a source of contact-induced changes across all linguistic domains, with \ili{English} playing a much more modest role, largely restricted to lexicon and associated \isi{inflectional} morphology.

\section{Contact-induced changes
}
\subsection{Phonology}
\subsubsection{Consonants} %Chris
\subsubsubsection{Additions to the native phonemic inventory}
\label{sec:311}%Chris
One of the most salient -- and uncontroversially contact-induced -- innovations in Maltese phonology is the addition of at least five (arguably seven) consonant phonemes.\footnote{{For useful overviews of the phonology of Maltese, see \cite{Borg1997Maltese} and Cohen (\citeyear{cohen1966,Cohen1970}).}} This came about through the \isi{transfer} (presumably borrowing) of Italo-\ili{Romance} and \ili{English} lexical items without subsequent adaptation to the original native inventory (compare, e.g., Maltese \textit{pulizija} with unadapted initial [p] and \ili{Cairene} \ili{Arabic} \textit{bul\={i}s} `police'). The five uncontroversial additions are /p/, /v/, /ʦ/, /ʧ/ and /g/ (orthographically: 〈p〉, 〈v〉, 〈z〉, 〈ċ〉 and 〈g〉; see \tabref{tab:2:consonants}), as in \textit{evaporazzjoni} `evaporation' and \textit{granċ} `crab'. One can also make a case for an innovative borrowed \isi{phoneme} /ʣ/. There are no minimal pairs demonstrating a phonemic distinction between /ʣ/ and /ʦ/ (and both are represented by 〈z〉 in the \isi{orthography}), but Borg \& Azzopardi-Alexander (\citeyear[301]{BorgAzzopardi-Alexander1997}) point out that /ʣ/ occurs in environments not requiring a voiced obstruent, as in \textit{gazzetta} /gɐˈʣːɛtːɐ/ `newspaper'. More marginal is /ʒ/, which \citet{mifsud2011} and Borg \& Azzopardi-Alexander (\citeyear[303]{BorgAzzopardi-Alexander1997}) point out can be found in recent \isi{loanwords} from \ili{English}, such as \textit{televixin} `television' and \textit{bex} `beige', though whether all speakers voice the 〈x〉 in these items is uncertain.

\begin{table}[h]
\begin{tabularx}{\textwidth}{ l X X X X X X}
\lsptoprule
& \rotatebox{66}{Labial} & \rotatebox{66}{Alveolar} & \rotatebox{66}{Postalveolar} & \rotatebox{66}{Palatal} & \rotatebox{66}{Velar} & \rotatebox{66}{Laryngeal}\\\midrule
Plosive
& p & t & & & k & q \\
& b & d & & & g \\
Affricate
& & z & ċ & & \\
& & & ġ & & \\
Fricative
& f & s & x & & & ħ \\
& v & ż & & & \\
Nasal
& m & n & & & \\
Trill
& & r & & & & \\
Lateral
& & l & & & & \\
Approximant
& w & & & j & \\\lspbottomrule
\end{tabularx}
\caption{Inventory of consonants. Symbols are Maltese orthography.}
\label{tab:2:consonants}
\end{table}

Proto-\ili{Semitic} *g, represented as 〈\kern .75pt{\arabscript{ج}}〉 in \ili{Arabic} script, and usually rendered [ʤ] when \ili{Standard} \ili{Arabic} is spoken, is reflected as /ʤ/ (orthographic 〈ġ〉) in Maltese. This appears to be a retention of the original \ili{Maghrebi} realization of this \isi{phoneme}, other \ili{Maghrebi} varieties having in general deaffricated it to /ʒ/ (cf. \citealt[136]{Heath2002}). Unlike some other \ili{Maghrebi} varieties, however, the Maltese reflex of 〈\kern .75pt{\arabscript{ج}}〉 does not become /g/ before \isi{sibilants} (cp. Maltese \textit{ġewż} vs. \ili{Casablanca} \textit{gūz} `walnuts').\footnote{An exception is \textit{gżira} `island' < \ili{Arabic} \textit{ǧaz\={i}ra}, perhaps to be explained by direct contiguity with the \isi{sibilant}.} Similarly, Proto-\ili{Semitic} *q (on which more below), is never reflected as /g/ (orthographic 〈g〉) in Maltese (cf. \citealt[99]{Vanhove1998}), meaning that the presence of /g/ in the Maltese phonemic inventory is certainly due to its occurence in numerous lexical borrowings. The majority of these are from Italo-\ili{Romance} (e.g. \textit{gwerra} `war'), but some are from \ili{Berber} (e.g. \textit{gendus} `calf' < \ili{Berber} \textit{agenduz}; \citealt[827]{naitzerrad2002}), suggesting that /g/ as an independent \isi{phoneme} has been present in Maltese since the earliest days of \ili{Arabic} speech on the Maltese islands.\footnote{There are also some sporadic examples of /g/ < *k in \ili{Arabic} \isi{roots}, e.g. \textit{gideb} `to lie'. See Cohen (\citeyear[14--15]{cohen1966}) for further details.}


\subsubsubsection{Losses, mergers and shifts} \label{mergers}
Alongside these additions, the Maltese consonant \isi{phoneme} inventory has also witnessed a number of losses and mergers. Clearly it is not possible to establish with certainty whether or not these changes were due to contact, but various considerations make it reasonable to assume that contact at least accelerated these changes. For example, the inherited \isi{emphatic} (\isi{pharyngealized}/uvularized) consonants -- *\d{s}, *\d{t} and *\d{ð} -- have all \isi{merged} with their non-\isi{emphatic} counterparts, as in \textit{sħab} /sħɐːb/ `clouds' < \textit{sa\d{h}\={a}b}, and also `companions' < \textit{ʔa\d{s}\d{h}\={a}b}. Note in this connection that among other \ili{Arabic} varieties, it is only a handful of those most strongly affected by contact (such as pidgins and creoles, as well as \ili{Cypriot Maronite} \ili{Arabic}; see Avram, this volume; Walter, this volume) that have \isi{merged} the \isi{emphatic consonants} in this way. This suggests that non-native acquisition of Maltese by Italo-\ili{Romance} speakers precipitated this change (i.e. that it involves \isi{source-language} agentivity in Van Coetsem's \citeyear{VanCoetsem1988,VanCoetsem2000} terms).\ia{Avram, Andrei A.@Avram, Andrei A.}\ia{Walter, Mary Ann@Walter, Mary Ann}


In addition to the loss of the \isi{emphatic consonants}, Maltese has undergone significant losses and mergers among the velar and laryngeal phonemes.

Perhaps most saliently, an earlier version of what is today  Standard Maltese \isi{merged} and then lost the voiced uvular/velar fricative *ɣ and the voiced \isi{pharyngeal} fricative *ʕ. In Maltese's rather etymologizing \isi{orthography}, these historic phonemes are given the digraph symbol 〈għ〉. In general, this symbol either has no phonetic correlate, as in \textit{għajn} /ɐɪn/ `eye, spring' and \textit{għonq} /ɔnʔ/ `neck', or otherwise corresponds to the lengthening of a vowel in morphological patterns where the vowel would ordinarily be short, as in the \isi{stem} I {CaCeC} verb \textit{għamel} /ˈɐːmɛl/ `to do'. That the two original phonemes first \isi{merged} and were then lost in  Standard Maltese can be inferred from the behaviour of 〈għ〉 + 〈h〉 sequences. These are realised as /ħː/ in \isi{roots} where 〈għ〉 reflects *ʕ (e.g. \textit{semagħ-ha} /sɛˈmɐħːɐ/ hear.\textsc{prf.3sg.m-3sg.f}, `he heard it' < \textit{samaʕ} `to hear'), where other \ili{Arabic} varieties behave similarly (cf. \citealt[18] {Woidich2006}), but also, unlike other \ili{Arabic} varieties, in \isi{roots} where 〈għ〉 reflects *ɣ (e.g. \textit{ferragħ-ha} /fɛrˈrɐħːɐ/ pour.\textsc{prf.3sg.m-3sg.f}, `he poured it out' < \textit{farraɣ} `to empty'). This \isi{merger} and subsequent loss did not take place in all varieties of Maltese. To this day, there are apparently speakers of dialectal Maltese whose speech preserves both *ɣ as a velar fricative, and *ʕ as a \isi{pharyngeal} fricative \citep{klimiuk2017}. The fact that the \isi{merger} and loss of these two phonemes is more advanced in the standard language of the major conurbations and less so in the dialects of more isolated villages suggests that contact-induced change played an important role here, with non-native speakers of Maltese presumably being the principal agents of change.

Arguably the most interesting set of mergers and losses concerns the voiceless fricatives, which represent a case of considerable phonemic reorganization despite relatively little change at the phonetic level. The phonemic changes in this domain are as follows. First, *h, while maintained in the \isi{orthography} (as 〈h〉), has \isi{merged} with /ħ/ in \isi{codas} (e.g. \textit{ikrah} /ɪkˈrɐħ/ `ugly') and sporadically in onsets (e.g. \textit{naħaq} /ˈnɐħɐʔ/ < \textit{nahaq} `to bray (of donkeys)'), and is otherwise lost altogether (e.g. \textit{hemm} /ɛmː/ `there'). The Maltese \isi{phoneme} /ħ/ thus represents the continuation of the voiceless \isi{pharyngeal} fricative *\d{h}, as well as the partial \isi{merger} of *h. Moreover, original *ḫ, the voiceless uvular/velar fricative, has also \isi{merged} with /ħ/, as in \textit{ħajt} `thread' < \textit{ḫay\d{t}}, and also `wall' < \textit{\d{h}\={a}yi\d{t}}. Strikingly, however, the single Maltese \isi{phoneme} /ħ/ exhibits considerable inter- and intra-speaker variation in its precise realization, such that glottal, \isi{pharyngeal}, and velar/uvular voiceless fricative realizations may commonly be heard (\citealt[301]{BorgAzzopardi-Alexander1997}), and it is in this sense there has been little phonetic change despite the considerable phonological reorganization.

Like the loss of the \isi{emphatic consonants}, the loss or \isi{merger} of *h (as well as one or more of the \isi{pharyngeal} and velar/uvular fricatives) is restricted to a handful of \ili{Arabic} varieties that have been very strongly affected by contact (see, e.g., Walter, this volume).\ia{Walter, Mary Ann@Walter, Mary Ann} As such, these changes too are suggestive of \isi{imposition} by non-native speakers lacking these sounds in their native phonemic inventory (as was the case for speakers of the \ili{Romance} varieties with which Maltese has had the most intense contact, cf. \citealt[141--142] {Loporcaro2011}). On the other hand, the preservation of the glottal and \isi{pharyngeal} fricatives as allophones of /ħ/ complicates this picture, such that the role of contact in bringing about these particular changes must remain uncertain for now.

It is similarly hard to diagnose the causes of the shift of *q to glottal stop (nevertheless written as 〈q〉 in Maltese \isi{orthography}) and the stopping of the interdental fricatives *θ and *ð. In both cases, however, we can at least rule out with confidence any suggestion that these are ancient changes that predate the arrival of \ili{Arabic} in Malta, or are historically connected to similar realizations in the \ili{Arabic} dialects of urban centres in the Maghreb, Egypt, and the Levant. Written records of earlier Maltese clearly show that a dorsal realization of *q, as well as the interdental fricative realization of *θ and *ð, survived until at least the late eighteenth century \citep{Avram2012,Avram2014Maltese}. It is at least plausible, therefore, that contact with Italo-\ili{Romance} played a role in these changes too, but firm evidence on this point is so far lacking.

Finally, a well-known feature of contemporary Maltese (and Maltese \ili{English}) phonology is the \isi{devoicing} of word-final obstruents, as in \textit{ħadd} [ħɐtː] `nobody'. \cite{Avram2017devoicing} shows that \isi{devoicing} gradually diffused across the Maltese lexicon over the course  of about two centuries from the late sixteenth century onwards, and he makes a strong case that the initial trigger for this development was \isi{imposition} by native speakers of \ili{Sicilian} and \ili{Italian}, since  word-final obstruent \isi{devoicing} has been shown by various studies (e.g. \citealt{Flegemunromackay1995}) to be a frequent feature of the L2 speech of L1 speakers of \ili{Romance} languages.

\subsubsection{Vowels} %Chris

Maltese has a much richer vowel \isi{phoneme} inventory than typical \ili{Maghrebi} \ili{Arabic} dialects, with, among the \isi{monophthongs}, five short-vowel qualities /ɪ, ɛ, ɐ, ɔ, ʊ/ (orthographic 〈i, e, a, o, u〉), and six long-vowel qualities /iː, ɪː, ɛː, ɐː, ɔː, uː/ (orthographic 〈i, ie, e, a, o, u〉), as well as seven distinct \isi{diphthongs} (with a number of different orthographies -- see \citealt[299]{BorgAzzopardi-Alexander1997} for details): /ɪʊ, ɛɪ, ɛʊ, ɐɪ, ɐʊ, ɔɪ, ɔʊ/. Compare this with the three-vowel-quality system of \ili{Tunis} \ili{Arabic}, which also lacks \isi{diphthongs} \citep{Gibson2011}.


Since the Italo-\ili{Romance} languages have vowel systems of a similar richness to Maltese, one might assume that this proliferation of vowel phonemes is a straightforward case of \isi{transfer}. This is, in general, not the case, however. The majority of new phonemic distinctions are at least partially the result of the loss of \isi{emphatic consonants} and of *ʕ,\footnote{These latter changes are themselves, however, arguably contact-induced -- see §\ref{mergers}.} which led to the phonemicization of vowel qualities that were previously merely allophonic. Note also that the innovative lax close front \isi{long vowel} /ɪː/ is apparently an entirely internal development -- the outcome of an extreme raising of the front allophone of *\={a} (so-called \textit{im\={a}la}), as in \textit{ktieb} /ktɪːb/ `book' < \textit{kit\={a}b}.

Following Krier (\citeyear[21--22]{krier1976}), we can nevertheless point to three innovations in this domain which do seem to be the direct result of lexical borrowing from Italo-\ili{Romance}.

Krier (\citeyear[21]{krier1976}) points out first of all that, of the five short vowels, only four /ɪ, ɛ, ɐ, ɔ/ appear in all positions in \ili{Arabic}-derived lexicon. In contrast, /ʊ/ occurs only in final position in unstressed syllables in this portion of the lexicon, with the single exception of \textit{kull} `all'. Were it not for the (extensive) Italo-\ili{Romance} component of the Maltese lexicon, therefore, we can say that the distinction between [ɔ] and [ʊ] would remain allophonic, as it is in \ili{Tunis} \ili{Arabic}. As it is, the two sounds should probably be considered phonemically distinct in Maltese. Although minimal pairs are hard to find, possible examples include \textit{punt} `point' vs. \textit{pont} `bridge' and \textit{lotto} `lottery' vs. \textit{luttu} `mourning'.\footnote{Our thanks to Michael Spagnol for suggesting these examples.}


Among the long vowels, the presence of /ɛː/ and /ɔː/ phonemes in Maltese is also largely attributable Italo-\ili{Romance} loans containing these sounds. Although /\={e}/ and /\={o}/ do occur in certain \ili{Tunisian} \ili{Arabic} varieties \citep{Gibson2011,HerinZammit2017}, these are the result of historical monophthongization of the original *ay and *aw \isi{diphthongs}. The Maltese reflexes of these sounds remain \isi{diphthongs}, as in \textit{sejf} /sɛɪf/ `sword' and \textit{lewn} /lɛʊn/ `colour'. Other than in cases of compensatory lengthening in items where the consonants represented by 〈għ〉 and 〈h〉 have been lost (see §\ref{mergers}), /ɛː/ and /ɔː/ only occur in the non-\ili{Arabic} component of the Maltese lexicon, as in \textit{żero} /ˈzɛːrɔ/ `zero' and \textit{froġa} /ˈfrɔːʤɐ/ `omelette'.

To these three contact-induced monophthongal innovations we can add one new contact-induced \isi{diphthong}: /ɔɪ/. Mifsud (\citeyear{mifsud2011}) points out that this occurs only in non-\ili{Arabic} lexical items (e.g. \textit{vojt} /vɔɪt/ `empty space') in  Standard Maltese.

In summary, then, the majority of innovative vowel phonemes in Maltese are not the direct result of \isi{transfer}, but the three new monophthongal phonemes whose emergence is (at least partially) contact-induced, combine to create a near-symmetrical system in which all five short vowel phonemes have a long counterpart.

\subsubsection{Intonation}
Despite pioneering work by Alexandra Vella (e.g. \citealt{Vella1994,vella2003,Vella2009}; \citealt{GriceVellaBruggeman2019}), the study of \isi{intonation} in Maltese, as in most non-Indo-European languages, remains in its infancy (cf. Hellmuth, this volume).\ia{Hellmuth, Sam@Hellmuth, Sam} Impressionistically speaking, the tunes that can be heard in Maltese (and Maltese \ili{English}) speech are highly distinctive, and often quite unlike those of the Mediterranean \ili{Arabic} dialects. Several studies have demonstrated that \isi{intonation} patterns are highly susceptible to \isi{transfer} in language contact situations, especially through \isi{imposition} by source-language-dominant speakers (see the studies of \ili{Spanish} \isi{intonation} by \citealt{orourke2005,gabrielkireva2014}). Interestingly, however, this appears to be less true for the tunes associated with polar interrogatives, at least in the varieties of \ili{Spanish} described by the aforementioned authors, presumably because of the importance of \isi{intonation} in establishing interrogative force in the absence of syntactic cues in this language. What data we have on this issue for Maltese fits rather neatly into this larger picture. According to \cite{vella2003}, the intonational patterns of Maltese late-focus declaratives on the one hand, and wh-interrogatives on the other, pattern with Palermo \ili{Sicilian} and Tuscan \ili{Italian} respectively, while that of Maltese polar interrogatives more closely resembles counterparts in \ili{Arabic} dialects.

It seems safe to assume that \isi{imposition} by native speakers of Italo-\ili{Romance} varieties is the primary cause of the similarities in \isi{intonation} between Maltese and Italo-\ili{Romance}, but borrowing by Maltese-dominant bilinguals should not be ruled out as an additional factor.

\subsection{Morphology}
\subsubsection{Nouns and adjectives}
\subsubsubsection{Inflection} \label{inflection}
It has been shown (e.g. \citealt{Gardani2012,Seifart2017}) that plural affixes are, with case affixes, the most widely transferred \isi{inflectional} morphemes. Maltese conforms neatly to the general crosslinguistic picture: it has acquired plural morphemes from \ili{Sicilian} and \ili{English} and little in the way of other \isi{inflectional} morphology (but see §\ref{verbmorphology}).\footnote{One should note also, however, the appearance in a couple of items of a singulative suffix \textit{-u}, apparently borrowed from \ili{Sicilian}. Borg (\citeyear[57]{Borg1994}) cites \textit{wiżż-u} `geese-\textsc{sing}', \textit{dud-u} `worms-\textsc{sing}', and \textit{ful-u} `beans-\textsc{sing}'.}

In addition to a rich array of stem-altering (so-called ``broken'') plural patterns, most of which also serve as the plurals of at least some items of Italo-\ili{Romance} or, more rarely, \ili{English} origin (see \citealt{spagnol2011} for details), Maltese has six plural suffixes: \textit{-in, -a, -iet, -ijiet, -i, \textnormal{and} -s}.\footnote{There are also one or two examples of zero plurals, e.g. \textit{martri} `martyr(s)'.} Of these, \textit{-in, \textnormal{and} -iet} are straightforward retentions from \ili{Arabic} (nevertheless extended to numerous non-\ili{Arabic} items), \textit{-i} and \textit{-s} are straightforward cases of indirect affix borrowing (in the sense of \citealt{Seifart2015}), and \textit{-a, \textnormal{and} -ijiet} arguably involve a subtle interplay of internal and externally-caused developments.

The most recently borrowed plural suffix is the \ili{English}-derived \textit{-s}. This occurs exclusively with bases borrowed from \ili{English}, and may be considered only partially integrated into \isi{monolingual} Maltese (to the extent that such a thing exists; see §\ref{English}), in that it often alternates optionally with \textit{-ijiet} in items such as \textit{kejk} `cake' (pl. \textit{kejkijiet \textasciitilde \,kejks}). There are, however, a number of reasonably frequent items (e.g. \textit{friżer} `freezer') which appear never to take a plural suffix other than \textit{-s}.


The \ili{Sicilian}-derived suffix \textit{-i} can mark the plural of a far higher proportion of Maltese nouns than can \textit{-s}, and is demonstrably better integrated into the Maltese \isi{inflectional} system. In addition to marking the plural of \ili{Sicilian}-derived nouns which also take \textit{-i}, e.g. \textit{xkupa} `broom' < \ili{Sicilian} \textit{scupa} (pl. \textit{scupi}), \textit{fjakk} `weak' < \ili{Sicilian} \textit{fiaccu} (pl. \textit{fiacchi}), it has also been extended to: \ili{Italian}-derived nouns, including those with a plural in \textit{-e} in \ili{Italian}, e.g. \textit{statwa} `statue' < \ili{Italian} \textit{statua} (pl. \textit{statue}); nouns from other \ili{Romance} languages, e.g. \textit{pitrava} `beetroot' < \ili{French} \textit{betterave} with $\varnothing$-plural (orthographic \textit{-s}); \ili{English}-derived nouns, e.g. \textit{jard} `yard (unit of distance)'); and even a few \ili{Arabic}-derived nouns, e.g. \textit{saff} `layer' < \textit{\d{s}aff} `row', \textit{samm} `very hard' < \textit{ʔa\d{s}amm} `deaf, hard'.

\ili{Arabic} and \ili{Sicilian} coincidentally have an identical less frequently used plural (or \isi{collective}) suffix \textit{-a}, as in \ili{Arabic} \textit{m\={a}rra} `passers-by' (singular \textit{m\={a}rr}) and \ili{Sicilian} \textit{libbra} `books' (singular \textit{libbru}). A plural suffix of this form also occurs in Maltese, with nouns of both \ili{Arabic} and Italo-\ili{Romance} origin (e.g. \textit{kittieba} `writers' < \ili{Arabic} \textit{katt\={a}b}; \textit{nutara} `notaries' < \ili{Italian} \textit{notaro}). Evidence that this is perceived and treated as a single morpheme rather than two homophonous items comes from the fact that the restriction of this suffix to groups of people in \ili{Arabic} applies also to the Italo-\ili{Romance} part of the Maltese lexicon (\citealt{mifsud2011}).

A curious feature of Maltese plural morphology from a \isi{comparative} \ili{Arabic} perspective is the very frequent suffix \textit{-ijiet} (\textit{-jiet} after certain vowel-final stems), as in \textit{postijiet} `places' (singular \textit{post}) and \textit{ommijiet} `mothers' (singular \textit{omm}). While clearly based on the \ili{Arabic}-derived suffix \textit{-iet} (< \ili{Arabic} \textit{-\={a}t}, with characteristic Maltese \textit{im\={a}la}), the provenance of the initial \textit{-ij-} is not obvious. \cite{mifsud2011} plausibly suggests that \textit{-ijiet} as a whole is ``derived from the plural of verbal nouns with a weak final radical, like \textit{tiġrijiet} `races', \textit{tiswijiet} `repairs’'', but \cite{geary2017} makes a strong case that the large influx into Maltese of Italo-\ili{Romance} nouns whose singulars ended in \textit{-i} (e.g. \textit{affari} `affair, matter' < \ili{Sicilian} \textit{affari} or \ili{Italian} \textit{affare}) was instrumental in the emergence of this morpheme. On this account Maltese speakers originally pluralized such words with \textit{-iet}, with glide-insertion an automatic phonological consequence of the juncture of a vowel-final \isi{stem} and a vowel-initial suffix. Later, according to Geary, the whole string \textit{-ijiet} was reanalysed as constituting the marker of plurality, and this new plural suffix was extended to consonant-final stems, including \ili{Arabic}-derived items of basic vocabulary such as \textit{omm} `mother' and \textit{art} `land'.\footnote{Geary's contact-induced scenario for the emergence of this suffix may not be the whole story, however. Evidence on this point comes from \ili{Arabic} \isi{loanwords} in \ili{Siwa} \ili{Berber}. Souag (\citeyear[74]{Souag2013book}) lists a number of examples of \ili{Arabic}-origin nouns whose plural is formed by adding a suffix \textit{-iyyat} (e.g. \textit{sḥilfa} `turtle', pl. \textit{sḥilfiyyat}), despite the fact that both \ili{Classical} \ili{Arabic} and present-day \ili{Egyptian Arabic} lack plurals of this type. \ili{Siwa} \ili{Berber} must therefore have borrowed these items and their pluralization strategy from some early form of (eastern) \ili{Maghrebi} \ili{Arabic}, suggesting that the presence in Maltese of the \textit{-ijiet} suffix is, at least to some extent, an \ili{Arabic}-internal development that predates the large-scale borrowing of Italo-\ili{Romance} nouns into Maltese.} %you may want to add that the Cantilena a has \isi{broken plural} for art, it's a cool bit of info

\subsubsubsection{Derivation}\largerpage
Maltese displays a rich array of \isi{derivational} suffixes borrowed (presumably initially as part of polymorphemic lexical items) from Italo-\ili{Romance}. A definitive list of these has not been provided to date, but \cite{Saade2019} offers a detailed typology of such items, of which we present a simplified version here, drawing also on examples from \cite{brincatmifsud2015}, and \isi{focusing} just on the nominal, adjectival and adverbial domains (see §\ref{participles} for borrowed participial morphology).


First of all, there are at least twenty suffixes, such as the nominalizer \textit{-zzjoni}, which, though relatively frequent, only occur in items clearly borrowed wholesale from Italo-\ili{Romance} (e.g. \textit{dikjarazzjoni} `declaration' < \ili{Italian} \textit{dichiarazione}) or in coinages which, in a process that is relatively common in Maltese, represent borrowings from \ili{English} that are adapted to fit the phonology and morphology of \ili{Romance}-influenced Maltese, as in \textit{esplojtazzjoni} `exploitation' (cf. \citealt{gattfabri2018}). Given this restriction, there must be some doubt as to whether one can regard the suffixes themselves as borrowed, or only the polymorphemic items in which they occur.

Secondly, there are a number of borrowed suffixes which are sufficiently well integrated that they can attach to \ili{Arabic}-derived bases. Examples include:

\begin{description}[font=\normalfont,noitemsep,labelsep=\widthof{~}]
	\item[\textit{-ata},] e.g. \textit{xemxata} `sunstroke' (\textit{xemx} `sun')
    \item[\textit{-ezza},] e.g. \textit{mqarebezza} `naughtiness' (\textit{mqareb} `naughty')
    \item[\textit{-un}] (< \ili{Sicilian} \textit{-uni}, \ili{Italian} \textit{-one}), e.g. \textit{ħmarun} `great fool' (\textit{ħmar} `donkey')
\end{description}


Finally, there is at least one borrowed suffix: \textit{-tura}, which forms single-instance verbal nouns. The integration of this morpheme can be seen from the fact that it attaches to productively to \ili{English} bases, as in \textit{ċekkjatura} `an instance of checking' or \textit{weldjatura} `an instance of welding'.

\subsubsection{Verbs}\label{verbmorphology}\largerpage

\subsubsubsection{Loaned verbs}\label{loanedverbs}
Maltese has borrowed a large number of verbs from \ili{Sicilian} and \ili{Italian}, and more recently a smaller number from \ili{English}. The chief interest in these borrowings lies in the way in which they have been integrated into the Maltese \isi{inflectional} and \isi{derivational} verbal paradigms. An in-depth study of this phenomenon was provided by \cite{mifsudloanverbs}, who distinguished the following four types of loaned verbs:

\begin{description}[font=\normalfont,noitemsep,labelwidth={\widthof{Type D:}}]
	\item[Type A:] Full integration into \ili{Semitic} Maltese sound verbs
    \item[Type B:] Full integration into \ili{Semitic} Maltese weak-final verbs
    \item[Type C:] Undigested \ili{Romance} stems with a weak-final conjugation
    \item[Type D:] Undigested \ili{English} stems
\end{description}


Mifsud (\citeyear[58]{mifsudloanverbs}) points out that most (perhaps all) Type A verbs are so-called ``second generation'' loans, whereby a nominal or adjectival form has been borrowed, a \isi{root} extracted from it, and a verb formed on this \isi{root}, as in \textit{pitter} `to paint' -- a denominal \isi{derivation} from \textit{pittur} `painter', borrowed from \ili{Sicilian} \textit{pitturi} (and supported by \ili{Italian} \textit{pittore}). Such items do not, therefore, represent genuine cases of \isi{transfer} of verbs, and are reminiscent of similar coinages in other \ili{Arabic} varieties (e.g. \textit{fabrak} `to fabricate'). In \ili{Arabic} as in Maltese, such items are overwhelmingly restricted to the denominal verbal stems II and V of triliteral \isi{roots} and I and II of quadriteral \isi{roots} (CVCCVC and tCVCCVC).

In contrast to Type A, Mifsud's Types B and C are genuine cases of loaned verbs. Mifsud (\citeyear[110--116]{mifsudloanverbs}) shows that the imperative (rather than the homophonous 3\textsc{sg} present, or any other verb forms) was the most likely \isi{base form} of the \ili{Romance} models on which the Maltese loaned verbs were created.\footnote{This parallels the situation in \ili{Arabic}-based pidgins and creoles, for which \cite{Versteegh2014pidgin} shows that verbs generally appear to derive from imperatives in the \isi{lexifier} varieties.} In both \ili{Italian} and \ili{Sicilian} all verbs in the imperative end in either \textit{-i} or \textit{-a}. As it happens, Maltese weak-final verbs (in which the final radical element is a vowel rather than a consonant) also all end in either /ɪ/ or /ɐ/ in the imperfect and imperative singular, depending on which of the two weak-final conjugation classes they fall into. This coincidence resulted in borrowed \ili{Romance} verbs being integrated into one of the these two weak-final classes, as in \textit{kanta} `he sang', \textit{jkanta} `he sings' (< \ili{Sicilian}/\ili{Italian} imperative \textit{canta}); and \textit{serva} `he served', \textit{jservi} `he serves' (< \ili{Sicilian}/\ili{Italian} imperative \textit{servi}).

The difference between verbs of Types B and C is that the former are analysed as having \isi{root-and-pattern} morphology, with a triliteral or quadriliteral \isi{root}, whereas Type C are borrowed as a concatenative \isi{stem} without a \isi{root}. This can be seen from the fact that Type B verbs can give rise to new verbs with the same \isi{root} in other verbal stems, as in \textit{kompla} `to continue', \textit{tkompla} `to be continued' (< \ili{Sicilian} \textit{cumpliri} `to finish'), whereas Type C verbs cannot.\largerpage

Another difference between Types B and C is that no Type C verb begins with a single (ungeminated) consonant, whereas most Type B verbs do. In fact, apart from certain well-defined exceptions (see \citealt[152]{mifsudloanverbs}), all Type C verbs begin with a geminate consonant, as in \textit{ffolla} `to crowd' < \ili{Italian} \textit{affollare}. What exactly was the combination of historical factors that gave rise to this synchronic state of affairs is a complex matter (see \citealt[158--168]{mifsudloanverbs} for discussion), but the key point to note is that at least some of the instances of initial gemination in Type C verbs are apparently not attributable to phonological properties of the source item (e.g. \textit{pprova} `to try' < \ili{Italian} \textit{provare}). It seems that speakers of Maltese came to feel that all loan verbs must have an initial geminate consonant, whether or not this was actually true of the item being borrowed.

This state of affairs manifests itself rather spectacularly in more recent borrowings from \ili{English} (Type D verbs), in which initial consonants are duly geminated (despite this never being the case in the \ili{English} source items), but which also fall into the conjugation class of weak-final verbs, as in \textit{ddawnlowdja} `to download'. What underlies this treatment of loans from \ili{English} seems to be a type of reanalysis, which we can sketch as follows. In the initial stage, verbs without \isi{roots} (not necessarily identifiable to speakers as loans from Italo-\ili{Romance}) are analysed as falling into the weak-final conjugation class because they have a stem-final vowel. But since all verbs without \isi{roots} (at this pre-\ili{English} stage) have a stem-final vowel, it is possible to view the lack of a \isi{root}, not the presence of a stem-final vowel, as the reason that loan verbs obligatorily fall into the weak-final conjugation class; and it seems that speakers indeed made this reanalysis. In a parallel development, initial consonant gemination also came to be seen an obligatory feature of the class of verbs lacking a \isi{root}. As a result of these developments, when a verb is borrowed from \ili{English}, because it lacks a \isi{root} its initial consonant is geminated and it is conjugated as a weak-final verb, regardless of whether it has a stem-final vowel.\footnote{In addition, virtually all Type D verbs insert a palatal glide between the borrowed \isi{stem} and the added weak-final vowel, as in \textit{pparkja} `to park'. Similarly to the initial gemination and weak-final \isi{inflection} of Type D verbs, this glide insertion must be the result of \isi{analogical} \isi{extension} from numerous glide-final borrowed \ili{Romance} verbs, e.g. \textit{rdoppja} `to double' < \ili{Italian} \textit{raddoppiare}. See Mifsud (\citeyear[225--236]{mifsudloanverbs}) for a detailed discussion.}

\subsubsubsection{Participles}\label{participles}\largerpage %Slavo
%I assume we will draw our info from Manwel's book
Unsurprisingly, one of the additional ways in which Type A verbs differ from the remaining three classes of loaned verbs is the \isi{formation} of \isi{passive} participles: in Type A verbs, \isi{passive} participles are formed in accordance with the \ili{Semitic} pattern for the respective derived \isi{stem}, e.g. \textit{pejjep} `to smoke' (\isi{stem} II, from \ili{Italian} \textit{pipa} `pipe') produces \textit{mpejjep} `smoked' (\citealt[70]{mifsudloanverbs}). In contrast, some Type B verbs allow for the \isi{formation} of a \isi{passive} \isi{participle} using \ili{Romance} suffixes (\citealt[127--133]{mifsudloanverbs}), and this is the sole option for Type C and even Type D verbs: for Type C verbs, the choice of the actual suffix depends on the original form of the verb and, in some cases, the path of transfer (see below). For Type D verbs borrowed from \ili{English}, the suffix \textit{-at} is the only productive way to form a \isi{passive} \isi{participle} (e.g. \textit{inxurjat} `insured') with \textit{spellut} `spelled' as the only exception (\citealt[248]{mifsudloanverbs}).

And finally, there are two distinct classes of Type B and C verbs which can each derive two \isi{passive} participles. In the first class, one \isi{participle} is derived from the weak (regular) form \isi{root} and the other derived from the strong one, e.g. \textit{konfondut} `confused' vs. \textit{konfuż} (\citealt[134]{mifsudloanverbs}). In the second class, one \isi{participle} is derived using the \ili{Sicilian} suffix \textit{-ut}, the other using the \ili{Italian}-derived suffix \textit{-it}, e.g. \textit{preferut} `preferred' vs. \textit{preferit} (\citealt[230]{mifsudloanverbs}). The reason for these doublets is largely sociolinguistic: the variability of the first class echoes a similar situation in \ili{Italian} dialects (\citealt[134]{mifsudloanverbs}); that of the second class reflects a situation whereby the loaned verb effectively has two sources, spoken \ili{Sicilian} and  Standard (Tuscan) \ili{Italian}.
%%% rewrote the last sentence slightly
\subsection{Syntax}
\subsubsection{Phrase syntax}
\subsubsubsection{Word order} \label{wo} %Slavo
The expansion of Maltese lexicon with items borrowed from \ili{Sicilian} and \ili{Italian} had a profound effect on the syntax of Maltese. The primary example of this is \isi{word order} within the \isi{noun phrase}, involving the order of adjectives and their heads. In \ili{Arabic}, adjectives (with the exception of comparatives, superlatives and a number of specific cases) follow their heads. This is largely true of \ili{Italian} adjectives as well, with the exception of a small subclass some grammars term ``specificational adjectives" (e.g. \citealt[55--56]{italian2007}), such as \textit{stesso} `same' and \textit{certo} `certain', which precede their head. Such adjectives borrowed into Maltese retained their syntactic properties, as with the pre-nominal \textit{ċertu} (< \ili{Sicilian} \textit{certu}) in (\ref{wo1}).
%%% gloss
\ea\label{wo1}
{[BCv3: it-torca.8685]} \\
	\gll Kien bniedem ta' ċerta personalità. \\
	be.\textsc{prf.3sg.m} person \textsc{gen} certain.\textsc{f} personality\\
	\glt `He was a person with a certain personality.'\\
\z

In \ili{Italian}, specificational adjectives to a large extent overlap with a class of adjectives that perform double duty as \isi{quantifiers} (or perhaps determiners) and vary their position according to their respective roles: Adj--N for \isi{quantifiers}, N--Adj for adjectives. One could argue that it is in the former function that they were borrowed into Maltese and thus should be considered \isi{quantifiers} or determiners rather than adjectives, especially in light of the fact that they are (for the most part) in complementary distribution with the \isi{definite} \isi{article}, as determiners and \isi{quantifiers} are. Determiners and \isi{quantifiers} in Maltese precede their heads (as with the \isi{definite} \isi{article} \textit{il-}, \textit{kull} `all', \textit{xi} `some' etc.).

There are three arguments against such an account: first of all, borrowed pre-nominal specificational adjectives actually fall into two classes, where members of the first, such as \textit{ċertu} `certain', \textit{diversi} `diverse' (< \ili{Italian} \textit{diverso}) or \textit{varju} `various' (< \ili{Sicilian} \textit{varju}), do not (for the most part) allow the \isi{definite} \isi{article}. In contrast, words in the second class such as \textit{stess} `same' (< \ili{Italian} \textit{stesso}) or \textit{uniku} `unique' (< \ili{Sicilian} \textit{uniku}) predominantly co-occur with the \isi{definite} \isi{article} when pre-nominal. The same, incidentally, is true of the etymologically \ili{Arabic} pre-nominal \isi{quantifier} \textit{ebda} `no, none'.

Secondly, there are morphological considerations: pre-nominal specificational adjectives of both types mark \isi{gender} and/or number (\textit{varju} for the first, \textit{uniku} for the second) like Maltese adjectives do; Maltese determiners and \isi{quantifiers} do not inflect for either \isi{gender} or number.\footnote{With the exception of the very specific category of demonstrative pronouns where \isi{gender} and number are marked not by affixes, but rather a form of suppletion.}

The final argument against considering borrowed pre-nominal specificational adjectives as being borrowed into the slot for determiners involves ordinal \isi{numerals}. In \ili{Italian}, these also fall into the subclass of prenominal specificational adjectives (\citealt[55]{italian2007}) and thus precede their head. The same is invariably true of Maltese ordinal \isi{numerals}, as with \textit{ewwel} in (\ref{wo2}).

%%%
\ea\label{wo2}
	{[BCv3: l-orizzont.64586]}\\
	\gll wara l-ewwel sena\\
	after \textsc{def}-first year\\
	\glt `after the first year'\\
\z

In North African \ili{Arabic}, ordinal \isi{numerals} can either precede or follow their heads, but when they precede them, they never take the \isi{definite} \isi{article}, even when the \isi{noun phrase} is semantically \isi{definite} (see e.g. \citealt[284]{Ritt-Benmimoun2014} for \ili{Tunisian} \ili{Arabic}). In contrast, Maltese never allows its ordinal \isi{numerals} to follow their heads, and the \isi{definite} \isi{article} is obligatory.

All these arguments, including the comparison with related \ili{Arabic} varieties, suggest that the pre-nominal position of some adjectives and ordinal \isi{numerals} in Maltese is due to \isi{transfer} under \isi{recipient-language} agentivity from \ili{Italian}.

\subsubsubsection{The analytical passive}\largerpage %Slavo
As with adjectives (§\ref{wo}), lexical borrowings from Italo-\ili{Romance} have also had a significant impact on the syntax of Maltese verbs. One of the most conspicuous consequences of this development involves the \isi{passive} voice: as \ili{Romance}-origin verbs cannot generally form one of the \isi{passive} derived verbal stems (but see §\ref{loanedverbs}), they brought with them their \ili{Romance} syntax and thus a new type of \isi{passive} construction arose in Maltese -- the analytical \isi{passive}.

In Maltese, there are two types of analytical \isi{passive} construction containing a \isi{passive} \isi{participle}: the so-called ``dynamic \isi{passive}" (\citealt[321--324]{Vanhove1993}; \citealt[214]{BorgAzzopardi-Alexander1997}), which combines \isi{passive} participles with the \isi{passive} auxiliary \textit{ġie} `to come'; and the so-called ``\isi{stative} \isi{passive}" (\citealt[214]{BorgAzzopardi-Alexander1997}, \citealt[318--320]{Vanhove1993}), which has the same structure as \isi{copular} clauses (see §\ref{Copular}), the only difference being that \isi{stative} \isi{passive} constructions can feature an agentive NP introduced by the \isi{preposition} \textit{minn} `from' (see \citealt[104--107]{bulbul2018} for a detailed analysis).

The \isi{stative} \isi{passive} can be viewed as an \isi{extension} of the structurally identical construction which is sporadically attested already in \ili{Classical} \ili{Arabic} (\citealt[76--84]{Ullmann1989}), but becomes quite prominent in Christian \ili{Arabic} documents at least as early as tenth century, where, incidentally, it gained prominence under influence from \ili{Aramaic} and \ili{Greek} (\citealt[424]{Blau1967}).

The dynamic \isi{passive} (\ref{pass1}), on the other hand, is a straightforward \isi{calque} on either \ili{Italian} or \ili{Sicilian}, where a construction featuring a verb semantically equi\-valent to \textit{ġie} `to come' -- \textit{venire} in \ili{Italian} -- combines with a past \isi{participle} (see also Manfredi, this volume).\ia{Manfredi, Stefano@Manfredi, Stefano}

\ea\label{pass1}
{[MUDTv1: 30\_01P05]} \\
\gll Kif diġà għedt, ġie ppreżentat il-kuntratt.\\
	as already say.\textsc{prf.1sg} come.\textsc{prf.3sg.m} present.\textsc{ptcp.pass} \textsc{def}-contract\\
	\glt `As I already said, the contract was presented.'\\
\z

While the dynamic \isi{passive} must have originally functioned to fill a hole in the verbal system of Maltese by providing a way to passivize \ili{Romance} verbs, it has meanwhile spread to include native verbs as well, as with \textit{ta} `to give' (< \textit{√ʕṭy}) in (\ref{pass2}).
\ea\label{pass2}
{[BCv3: inewsmalta-ott.29.2013.1257-11045]} \\
\gll It-tagħrif ġie mogħti mill-Ministru Konrad Mizzi.\\
	\textsc{def}-information come.\textsc{prf.3sg.m} give.\textsc{ptcp.pass} from.\textsc{def}-minister Konrad Mizzi.\\
	\glt `The information was given by Minister Konrad Mizzi.'\\
\z
%Unlike most Maltese verbs of \ili{Romance} origin, \textit{ta} can occur in the \isi{passive} \isi{stem} VII as \textit{ngħata} `to be given' and does so quite frequently (all its forms add up to 1012 per million in BCv3). Corpus data indicates that the choice is stylistic: the VIIth \isi{stem} forms predominate, yet analytical \isi{passive} of \textit{ta} is more likely to be used in parliamentary records or scientific texts.

\subsubsubsection{Modality}
Another clear-cut example of grammatical \isi{calquing} comes from the domain of \isi{modality} and involves the pseudoverb \textit{għand-}. In Maltese, its primary function is that of a possessive (\ref{ghandposs}), as is the case with its cognates \textit{ʕind-/ʕand-} in many \ili{Arabic} varieties.
\ea\label{ghandposs}
{[MUDTv1: 22\_02J03]} \\
    \gll M' għandi xejn kontri-hom.\\
	\textsc{neg} have.\textsc{1sg} nothing against-\textsc{3pl}\\
	\glt `I have nothing against them.'\\
\z

In addition to this, however, the Maltese \textit{għand-} has also taken on a function as a deontic \isi{modal} of weak obligation `should, ought' taking verbal complement, as in (\ref{ghandmod}).\footnote{\textit{għand-} is the only Maltese pseudoverb (and verb) which exhibits a three-way distinction between present (\textit{għand-}), past (\textit{kell-}) and \isi{future}/habitual (\textit{ikoll-}) forms; all can occur in the \isi{modal} function.}

\ea\label{ghandmod}
{[MUDTv1: 22\_02J03]} \\
    \gll Naqbel	li għandhom jivvutaw aktar nies.\\
	agree.\textsc{impf.1sg} \textsc{comp} have.\textsc{3pl} vote.\textsc{impf.3pl} more people.\\
	\glt `I agree that more people should vote.'\\
\z

The use of \textit{għand-} in this kind of \isi{modal} function appears to be unique to Maltese; not even \ili{Cypriot Maronite} \ili{Arabic} with its many parallels to Maltese (on which see below) exhibits the same behavior for its \isi{cognate} \textit{ʕint-} (\citealt[346]{Borg2004}) and uses a different verb, \textit{salaḫ/pkyislaḫ} (\citealt[323]{Borg2004}), as the default deontic \isi{modal}. The Maltese development must therefore be another \isi{calque}, since the basic possessive verb of \ili{Sicilian}, \textit{aviri}, also doubles as a deontic \isi{modal}, as in (\ref{aviri}).
\ea\label{aviri}
    \langinfo{Sicilian}{}{\citealt[340]{piccitto1977}}\\
    \gll Cci l' àiu a-ddiri a-tto patri.\\
	\textsc{dat.3sg.m} \textsc{obj.3sg.m} have.\textsc{pres.1sg} to-say.\textsc{inf} \textsc{dat-2sg.m} father\\
	\glt `I have to say it to your father.'\\
\z

\subsubsection{Sentence syntax}\largerpage
\subsubsubsection{Differential object marking} %Slavo
Differential object marking (\isi{DOM}) is a phenomenon whereby direct objects are marked according to some combination of the semantic and pragmatic properties of the object in question. In \ili{Spanish}, for example, objects denoting humans (and equivalent entities) are marked by the particle \textit{a}, originally a directional \isi{preposition}. \isi{DOM} is a phenomenon attested cross-linguistically (see \citealt{Khan1984} for \ili{Semitic} languages), including in varieties of \ili{Arabic} such as \ili{Levantine}, \ili{Iraqi} (\citealt{Coghill2014} and references therein), and \ili{Andalusi} (\citealt[108]{Andalusi2013}).

\isi{DOM} is a well-documented feature of Maltese morphosyntax and largely conforms to the \ili{Spanish} prototype: in general, both pronominal and nominal direct objects denoting entities high in the ``\isi{animacy} hierarchy'' (\citealt[55]{BorgAzzopardi-Alexander1997}) take the object marker \textit{lil} (\ref{dom1}), which also does double duty as the indirect object marker for all objects. Inanimate direct objects do not take \textit{lil} (\ref{dom2}).
%%% gloss
\ea\label{dom1}
{[BCv3: ilgensillum.2011-Mejju-22.8230]}\\
\gll Min jara lili jara lil Missier-i.\\
	who see.\textsc{impf.3sg.m} \textsc{obj}.1sg see.\textsc{impf.3sg.m} \textsc{obj} father-\textsc{obl.1sg}\\
	\glt `Who looks at me, looks at my Father.'\\
\ex\label{dom2}
%%% gloss
{[BCv3: l-emigrant]}\\
\gll Min jara orrizzonti ġodda u min baħħ.\\
	who see.\textsc{impf.3sg.m} horizon.\textsc{pl} new.\textsc{pl} and who void.\\
	\glt `Some see new horizons, some see a void.'\\
\z
 \cite{dohla2016} examines \isi{DOM} in Maltese in some detail and arrives at the conclusion that while there is ``a certain predisposition for object marking in general within pan-\ili{Arabic} grammar" (\citeyear[169]{dohla2016}), Maltese \isi{DOM} cannot be ascribed to purely internal developments within \ili{Arabic}. A striking feature of the \ili{Arabic} varieties that exhibit \isi{DOM} is that they were all in prolonged contact with other languages: \ili{Aramaic} for \ili{Levantine} and \ili{Iraqi} \ili{Arabic} (and, by \isi{extension}, for \ili{Cypriot Maronite} \ili{Arabic}, cf. \citealt[412]{Borg2004}), \ili{Romance} for \ili{Andalusi} \ili{Arabic} and Maltese. In the case of Maltese, the \ili{Romance} variety in question is \ili{Sicilian}, where the object marker \textit{a} performs the same double duty as the Maltese \textit{lil}, and \isi{DOM} in both languages shows a number of remarkable similarities: in both \ili{Sicilian} and Maltese, \isi{DOM} is primarily triggered ``by humanness along with \isi{definiteness}/referentiality'' (\citealt[257]{iemmolo2010}, in reference to \ili{Sicilian}), it is obligatory with personal pronouns, but optional with plural ``kinship terms and human common nouns'' and disallowed with ``(in)\isi{animate} and indefinite non-specific nouns'' (\citealt[257]{iemmolo2010}, again in reference to \ili{Sicilian}), as exemplified by the non-specific Maltese \textit{nies} `people' in (\ref{dom3}).
 
%%% gloss
\ea\label{dom3}
{[BCv3: l-orizzont.41390]}\\
\gll Min irid jara nies jgħixu hekk?\\
	who want.\textsc{impf.3sg.m} see.\textsc{impf.3sg.m} people live.\textsc{impf.3pl.m} thus\\
	\glt `Who wants to see people live like that?'\\
\z

In Maltese \isi{DOM}, then, we have an instance of what Manfredi (this volume)\ia{Manfredi, Stefano@Manfredi, Stefano} labels ``\isi{calquing} of polyfunctionality of grammatical items inducing syntactic change'': Maltese acquired a rule of \isi{DOM} as a result of the indirect object marker \textit{lil} inheriting the dual function of its \ili{Sicilian} equivalent \textit{a}. It is clear that this is a contact-induced change. But since with this and the similar changes discussed below there is no \isi{transfer} of lexical matter, it seems impossible at present to judge whether they are the result of borrowing or \isi{imposition}, or whether they were \isi{actuated} by speakers for whom neither the \isi{source language} nor the \isi{recipient language} were dominant, in the process that Lucas (\citeyear{Lucas2015}) calls ``\isi{convergence}''.


\subsubsubsection{Clitic doubling (proper)} %Slavo
The existence of various reduplicative phenomena associated with direct and indirect \isi{clitic} pronouns in Maltese has been noted at least since Sutcliffe (\citeyear[179]{sutcliffe}), who identifies what classical tradition refers to as \textit{nominativus pendens}. This analysis has been elaborated on by \cite{fabri1993}, \cite{BorgAzzopardi-Alexander1997} and \cite{fabriborgtopicfocus}, primarily in the context of pragmatically determined \isi{constituent order} variation, especially topicalization. Building on these works and the analysis of Maltese clitics by \cite{camilleri2011}, \cite{bulbul2014} notes that in addition to these phenomena, which in one way or another entail dislocation, there exists in Maltese another related phenomenon, where lexical objects and \isi{clitic} pronouns co-occur, but without the dislocation of the lexical object. This phenomenon, termed Clitic Doubling Proper to distinguish it from similar constructions (see \citealt{krapovacinque2008} for a detailed analysis), involves the co-occurrence of a lexical object and the \isi{clitic} with the object in situ, which in Maltese is after the verb (see \citealt{bulbul2018}). Maltese Clitic Doubling Proper occurs with both direct (\ref{cdp01}) and indirect objects (\ref{cdp02}).

%%% gloss
\ea\label{cdp01}
{[BCv3: l-orizzont.36758]}\\
\gll Ftit nies jafu-\textbf{ha} l-istorja marbuta ma' dan il-proġett tant sabiħ.\\
	few people know.\textsc{impf.3pl.m}-\textsc{3sg.f} \textsc{def}-history connected.\textsc{sg.f} with \textsc{dem.sg.m} \textsc{def}-project such beautiful\\
	\glt `Few people know the history connected with such a beautiful project.'\\
\ex\label{cdp02}
%%% gloss
{[BCv3: 20020313\_714d\_par]}\\
\gll Hekk qed ngħidu-\textbf{lhom} lil dawn in-nies f' pajjiż-na.\\
	thus \textsc{prog} say.\textsc{impf.1pl}-\textsc{dat.3pl} \textsc{dat} \textsc{dem.pl} \textsc{def}-people in country-\textsc{1pl}\\
	\glt `This is what we say to these people in our country.'\\
\z
Unlike various types of dislocation with resumptive \isi{clitic} pronouns which are quite common in European languages (see e.g. \citealt{Decat2010}), Clitic Doubling Proper is a much rarer phenomenon; in Europe, it is largely confined to the Balkan \textit{Sprachbund} (\citealt{Friedman2008}) and some \ili{Romance} languages outside of the Balkans, like \ili{Spanish} (\citealt[7]{Zagona2002}) and varieties of \ili{Italian} (\citealt[231--233]{Russi2008}). The phenomenon is also attested in \ili{Semitic} languages (\citealt{Khan1984}), including \ili{Arabic}, where it was studied in detail by \cite{Souag2017clitic}. Comparing Clitic Doubling Proper in various varieties of \ili{Arabic} including Maltese, Souag (\citeyear[57]{Souag2017clitic}) notes parallels between Maltese and some varieties of Algerian \ili{Arabic}, especially in regard to the doubling of indirect objects. Ultimately, however, he arrives at the conclusion that Maltese Clitic Doubling Proper ``has little in common with any other \ili{Arabic} variety examined, but closely resembles that found in \ili{Sicilian}'' (\citealt[60]{Souag2017clitic}). This suggests that here too we have a contact-induced change, this time of the sort that Manfredi (this volume) labels ``narrow syntactic \isi{calquing}'', that is, without any accompanying \isi{calque} of lexical items.\ia{Manfredi, Stefano@Manfredi, Stefano}

\subsubsubsection{Copular constructions} \label{Copular} %Slavo
In Maltese, there are four types of \isi{copular} clauses \citep[53]{BorgAzzopardi-Alexander1997}:\footnote{In addition to these, \cite{Borg1987} and \cite{borgspagnol2015} also describe the \isi{copular} function of the verb \textit{jinsab} `to be found'. This being a finite verb, both Borg \& Azzopardi-Alexander (\citeyear[53]{BorgAzzopardi-Alexander1997}) and Čéplö (\citeyear[99--104]{bulbul2018}) exclude this type of clause, as well as similar ones, such as those featuring the verb \textit{sar} `to become', from the category of \isi{copular} clauses.}

\begin{description}[font=\normalfont,noitemsep,labelwidth={\widthof{Type 4:}}]
	\item[Type 1:] No \isi{copula}
    \item[Type 2:] The verb \textit{kien} as the \isi{copula}
	\item[Type 3:] Personal pronoun as the \isi{copula}
	\item[Type 4:] Present \isi{participle} \textit{qiegħed} as the \isi{copula}
\end{description}
Type 1 describes what traditional grammars of \ili{Semitic} languages refer to as nominal sentences; \isi{copular} clauses with an explicit verbal \isi{copula} (\textit{kien}) then fall into Type 2. Types 3 and 4, while not without parallel in other varieties \ili{Arabic},\footnote{See the analysis of Type 4 copulas in \cite{CamilleriSadler2019}.} feature much more prominently in Maltese. This is especially true of Type 3 \isi{copular} clauses, which involve the use of a personal pronoun as the \isi{copula} (\ref{cop1}).

\ea\label{cop1}
{[BCv3: 2010 Immanuel Mifsud - Fl-Isem tal-Missier (U tal-Iben)]}\\
\gll Din hi omm-ok.\\
	this.\textsc{f} \textsc{3sg.f} mother-\textsc{2sg}\\
	\glt `This is your mother.'\\
\z
Similar \isi{copular} constructions to that illustrated in \REF{cop1} have been described for several \ili{Maghrebi} varieties (cf. \citealt[355]{Vanhove1993}), but Maltese stands apart in terms of the \isi{frequency} with which Type 3 constructions occur: in MUDTv1, for example, 110 non-negative \isi{copular} clauses are of Type 1; 181 are Type 3. In this, Maltese Type 3 \isi{copular} clauses are comparable to equivalent \isi{copular} constructions in \ili{Anatolian} \ili{Arabic} (see \citealt[172--173]{Lahdo2009} for \ili{Tillo} \ili{Arabic} and the references therein, as well as Akkuş, this volume),\ia{Akkuş, Faruk@Akkuş, Faruk} \ili{Andalusi} \ili{Arabic} (\citealt[105]{Andalusi2013}), and especially \ili{Cypriot Maronite} \ili{Arabic} (\citealt[135]{Borg1985}; Walter, this volume), where they are but one piece of evidence linking \ili{Cypriot Maronite} \ili{Arabic} to \textit{qəltu} dialects (\citealt[31]{Borg2004}).\ia{Walter, Mary Ann@Walter, Mary Ann} The conclusion to be drawn here is the same as for \isi{DOM} and Clitic Doubling Proper above: it is no coincidence that these \isi{copular} constructions are in wide use and the \isi{copular} construction of choice especially in varieties of \ili{Arabic} which have been under contact influence from languages with a mandatory \isi{copula} -- \ili{Turkish} for \ili{Anatolian} \ili{Arabic}, \ili{Spanish} for \ili{Andalusi} \ili{Arabic}, \ili{Greek} for \ili{Cypriot Maronite} \ili{Arabic}, and \ili{Italian} for Maltese. Whether the origin of such constructions can be traced to a feature in (one of) these dialects' \ili{Old} \ili{Arabic} ancestors, or whether they came about through parallel development, contact undoubtedly triggered the widespread adoption of such constructions in these varieties of \ili{Arabic}.
%\subsection{Maltese and English: One system or two systems?} %Slavo
%As noted in §\ref{English}, the majority of the Maltese-speaking population is bilingual and consequently, \isi{code-switching} between \ili{English} and Maltese is a part of the everyday linguistic reality in the Maltese islands. This gives rise to sentences where Maltese and \ili{English} freely combine (\ref{onesys1}).
%\begin{exe}
%	\ex \label{onesys1} \gll Mr Chairman, se tibgħatli l- proposed list of witnesses.\\
%Mr chairman, \textsc{fut} \textsc{2.impf}-send-\textsc{dat.1sg} \textsc{def} proposed list of witnesses.\\
%	\glt `Mr. Chairman, you will send me the proposed list of witnesses.'\\
%    \glt [MUDTv1: 30_01P05]
%gloss
% 	\ex \label{onesys1} \gll Jien kont ġejt nominat to cochair dan l- isteering committee…\\
% 	I be.\textsc{pst-1sg} come.\textsc{pst-1sg}  nominate-\textsc{part.pass} to cochair this.\textsc{m} \textsc{det} steering committee...\\
% 	\glt `I had been nominated to co-chair this steering committee...'\\
%     \glt [MUDTv1: 30\_01P05]
% \end{exe}
% Here one must make a careful distinction between Maltese and \ili{Romance} contact (and early Maltese-\ili{English} contact) and the contact between Maltese and \ili{English}, as well as between borrowing of lexemes and \isi{code-switching}. While the conspicuous influence of \ili{Sicilian} and \ili{Italian} on Maltese has led many to describe Maltese as a mixed language (going as far back as \citealt{aquilinastructure1959}), even those studies that put such an assessment on a firm formal footing (notably \citealt{drewes1994} and \citealt{stolz2003}) arrive at the ultimate conclusion that the contact between Maltese and \ili{Romance} is simply a case of wide-scale borrowing of lexical items which in some cases (as shown above) also extended to the syntax of said (class of) lexical items. This is, inter alia, evident from the fact that the number of function words (§\ref{mtlexicon}) and \isi{inflectional} morphemes (§\ref{inflection}) is low and insignificant in terms of influence on the structure of Maltese: borrowed function words proper\footnote{See §\ref{mtlexicon}, especially \cite{comriespagnol2016}, who include adverbs and adjectives like \textit{stess} (\ref{wo}) in their list of function words.} are confined to a few \isi{prepositions} (\textit{skont}, \textit{apparti}) and a numeral (\cite{zero}) and as for \isi{inflectional} morphemes, only the most straightforward ones like plurals are borrowed with the \isi{passive} \isi{participle} suffix \textit{-at} applied to \ili{English} verbs as the only significant exception. In terms of Myers-Scotton's (\citeyear{myersscotton2002}) Matrix Language Frame model (MLF), Maltese remains the Matrix Language, as it is the sole source of the system morphemes\footnote{In MLF, this includes not only morphemes in the traditional sense, but also lexemes.} ``which have grammatical relations external to their head constituent" and the order of morphemes conforms to that of Maltese (\citealt[59]{myersscotton2002}). The borrowed verbs are perhaps the best example: no matter what the source and the type of borrowing, they were made to conform to Maltese conjugation (again, with \isi{passive} participles derived from \ili{English} verbs as the sole exception).

% The more recent contact between Maltese and \ili{English} is a completely different phenomenon, not in the least due to different sociolinguistic conditions: large-scale bilinguism is a relatively new phenomenon in Malta and only such large-scale bilinguism can give rise to sentences like (\ref{onesys1}). combining Maltese clauses and \ili{English} clauses within a single sentence and or utterance. (\ref{onesys1})

% "Embedded Language Islands" \cite[139-153]{myersscotton2002}

\subsection{Lexicon} \label{mtlexicon} %Slavo table with general composition. Plus frequency-based Spagnol and Comrie paper.
\subsubsection{Major sources}
That Maltese contains large numbers of \isi{loanwords} from \ili{Romance} and \ili{English} is a fact immediately obvious to even the most casual observer. Over the years, there have been a number of attempts to quantify the influence of other languages on Maltese by providing a classification of lexemes by their origin. The earliest, Fenech (\citeyear[216--217]{fenech1978}), compiled such statistics for journalistic Maltese, but also provided a comparison to literary and spoken Maltese (albeit using a very small data sample). Brincat analyzed the etymological composition of entries in Aquilina's dictionary, first examining the origin of 34,968 out of all 39,149 headwords (\citealt[115]{brincat1996}) and then applying the same analysis to the entire list (\citealt[407]{brincat2011});  \cite{mifsudborg1997} did the same with the vocabulary contained in an introductory textbook of Maltese as a foreign language. In 2006, \cite{bovingdondalli2006} analyzed the etymology of lexical items in a 1000-word sample obtained from a corpus of Maltese and, most recently, Comrie \& Spagnol (\citeyear[318]{comriespagnol2016}) did the same on a list of 1500 ``lexical meanings" within the framework of the \textit{Loanwords in the world’s languages} project (\citealt{HaspelmathTadmor2009}). Figure \ref{lexiconfig} summarizes all these findings.

\begin{figure}[H]
\centering
\includegraphics[width=12cm,trim=0 20 0 0]{figures/lexiconnewfont.eps}
\caption{A summary of previous studies of the composition of the Maltese lexicon}
\label{lexiconfig}
\end{figure}

The primary explanation for the sharp differences between these analyses is methodology: while \cite{fenech1978} analyzes entire texts and thus counts tokens, \cite{brincat1996} (including its updated version in \citealt{brincat2011}) and \cite{bovingdondalli2006} analyze lists of unique words, i.e. types. The later is also true of \cite{mifsudborg1997} and \cite{comriespagnol2016}, except where \cite{brincat1996} uses dictionary data and \cite{bovingdondalli2006} corpus data, \cite{mifsudborg1997} employ a list of lexical items with high \isi{frequency} of use in daily communication and \cite{comriespagnol2016} base their analysis on a list compiled for the purposes of cross-linguistic comparison. The high ratio of words of \ili{Semitic} origin in token-based analyses is thus due to the prevalence of function words, which are overwhelmingly \ili{Arabic}. The type-based analyses then provide a somewhat more accurate picture of the lexicon as a whole, even though they are not without their problems. Chief among these is the issue of what exactly counts as type, especially with regard to productive \isi{derivational} affixes, e.g. whether all the words with the prefix \textit{anti-} count as distinct types or not.


In addition to general analyses, both \cite{bovingdondalli2006} and \cite{comriespagnol2016} also provide breakdowns for individual parts of speech. Unfortunately, these analyses are not comparable, as each has a different focus: Bovingdon \& Dali (\citeyear[71]{bovingdondalli2006}) are interested in the composition of each etymological stock by word class (Table \ref{tab:posbovingdon}).

%table
\begin{table}[H]
\caption{Source language component of Maltese by word class (\citealt[71]{bovingdondalli2006}).}
\label{tab:posbovingdon}
\begin{tabular}{lrrrrrr}
\lsptoprule
Origin & Function words & V & Adj & N & Adv & Prn\\
\midrule
\ili{Semitic} & \multicolumn{1}{r}{3\%} & \multicolumn{1}{r}{70\%} & \multicolumn{1}{r}{2\%}
& \multicolumn{1}{r}{21\%} & \multicolumn{1}{r}{2\%} & \multicolumn{1}{r}{2\%} \\
\ili{Romance} & \multicolumn{1}{r}{0\%} & \multicolumn{1}{r}{38\%} & \multicolumn{1}{r}{11\%}
& \multicolumn{1}{r}{48\%} & \multicolumn{1}{r}{3\%} & \multicolumn{1}{r}{0\%} \\
\ili{English} & \multicolumn{1}{r}{0\%} & \multicolumn{1}{r}{29\%} & \multicolumn{1}{r}{8\%} & \multicolumn{1}{r}{63\%}
& \multicolumn{1}{r}{0\%} & \multicolumn{1}{r}{0\%} \\
\lspbottomrule
\end{tabular}
\end{table}
In contrast, Comrie \& Spagnol (\citeyear[328]{comriespagnol2016}) focus on the composition of individual word classes by their origin (Table \ref{tab:comriespagnol}).\footnote{The details of Comrie \& Spagnol's (\citeyear{comriespagnol2016}) methodology mean that loans in their dataset come from \ili{Romance} and \ili{English} but not from any other languages. The category we label ``Misc.'' in Tables \ref{tab:comriespagnol} and \ref{tab:comriespagnolsemantic} encompasses those meanings in the \textit{Loanwords in the world's languages} 1500-item set which have no corresponding single-word Maltese lexical item, and those where the etymology is at present unknown, or where the item in question is an innovative Maltese-internal coinage.}

%table
\begin{table}[H]
\caption{Word class composition by source language (\citealt[328]{comriespagnol2016})}
\label{tab:comriespagnol}
\begin{tabular}{lllll}
\lsptoprule
Word class & \ili{Arabic} & \ili{Romance} & \ili{English} & Misc.\\
\midrule
Function words & \multicolumn{1}{r}{84.7\%} & \multicolumn{1}{r}{6.2\%} & \multicolumn{1}{r}{0\%} & \multicolumn{1}{r}{9.1\%} \\
Verbs & \multicolumn{1}{r}{75.3\%} & \multicolumn{1}{r}{14.1\%} & \multicolumn{1}{r}{1.3\%} & \multicolumn{1}{r}{9.2\%} \\
Adjectives & \multicolumn{1}{r}{65.2\%} & \multicolumn{1}{r}{28.5\%} & \multicolumn{1}{r}{0.3\%} & \multicolumn{1}{r}{6.0\%} \\
Nouns & \multicolumn{1}{r}{44.7\%} & \multicolumn{1}{r}{39.6\%} & \multicolumn{1}{r}{7.2\%} & \multicolumn{1}{r}{8.6\%} \\
\lspbottomrule
\end{tabular}
\end{table}

\citet{comriespagnol2016} also provide a breakdown of their data by semantic field, permitting a comparison of the domains in which \ili{Romance} versus \ili{English} loans are more or less prominent. A number of generalizations can be made here (see Table \ref{tab:comriespagnolsemantic} for a summary), though ultimately they all follow naturally from the fact that contact with \ili{English} was more recent, and less \isi{intensive}, than contact with \ili{Sicilian} and \ili{Italian}.


\begin{table}[ht]
\caption{Composition of semantic fields by source language (\citealt[327]{comriespagnol2016})}
\label{tab:comriespagnolsemantic}
\begin{tabular}{lllll}
\lsptoprule
Semantic field & \ili{Arabic} & \ili{Romance} & \ili{English} & Misc.\\
\midrule
Modern world & \multicolumn{1}{r}{3.0\%} & \multicolumn{1}{r}{65.3\%} & \multicolumn{1}{r}{22.8\%} & \multicolumn{1}{r}{9.0\%} \\
Animals & \multicolumn{1}{r}{47.8\%} & \multicolumn{1}{r}{29.1\%} & \multicolumn{1}{r}{13.9\%} & \multicolumn{1}{r}{9.1\%} \\
Clothing and grooming & \multicolumn{1}{r}{38.7\%} & \multicolumn{1}{r}{47.2\%} & \multicolumn{1}{r}{10.4\%} & \multicolumn{1}{r}{3.8\%} \\
Warfare and hunting & \multicolumn{1}{r}{28.8\%} & \multicolumn{1}{r}{65.0\%} & \multicolumn{1}{r}{2.5\%} & \multicolumn{1}{r}{3.8\%} \\
Law & \multicolumn{1}{r}{36.0\%} & \multicolumn{1}{r}{50.0\%} & \multicolumn{1}{r}{0.0\%} & \multicolumn{1}{r}{14.0\%} \\
Social and political relations & \multicolumn{1}{r}{48.4\%} & \multicolumn{1}{r}{48.4\%} & \multicolumn{1}{r}{0.0\%} & \multicolumn{1}{r}{3.2\%} \\
\lspbottomrule
\end{tabular}
\end{table}

Unsurprisingly, \ili{English} is best represented in the category of items relating to the modern world, but even here \ili{Romance} dominates. Examples include \ili{English}-derived \textit{televixin} `television' and \ili{Italian}-derived \textit{kafè} `coffee'.


The domain of animals divides rather neatly as follows. Common animals (especially land animals) of the Mediterranean area are largely \ili{Arabic}-derived (e.g. \textit{fenek} `rabbit' < \ili{Maghrebi} \ili{Arabic} \textit{fanak} `fennec fox'), while well-known non-indigenous animals are largely \ili{Romance}-derived (e.g. \textit{ljunfant} `elephant' < \ili{Sicilian} \textit{liufanti}, the additional /n/ perhaps the result of influence from \textit{ljun} `lion'). More exotic animals, if there is a corresponding Maltese item at all, derive from \ili{English} (e.g. \textit{tapir} `tapir'). Clothing and grooming presents a similar picture, with \ili{Arabic}-derived \textit{suf} `wool', \ili{Sicilian}-derived \textit{ngwanta} `glove', and \ili{English}-derived \textit{fer} `fur', as does warfare and hunting, with \ili{Arabic}-derived \textit{sejf} `sword', \ili{Sicilian}-derived \textit{xkubetta} `gun', and \ili{English}-derived \textit{senter} `shotgun' (< \textit{centre-breech-loading shotgun}).

The total lack of \ili{English} loans in the domains of law and social and political relations, at least in Comrie and Spagnol's sample, is remarkable, given the extent to which the \ili{English} language dominated public life in Malta in the twentieth century. A generalization that underlies this finding is that while \ili{English} influence is strongest in the spheres of commerce, consumerism and, especially in the twenty-first century, popular culture (e.g. \textit{vawċer} `voucher', \textit{ċċettja} `to chat'),\footnote{Until at least 1991, when the Maltese government opened up television broadcasting rights to more than just the single state broadcaster TVM, Italian television stations, whose broadcasts from \isi{Sicily} could be received in Malta, were very widely watched, and there was consequently considerable Italian influence on Maltese popular culture \citep{sammut2007}. This influence has waned considerably at the expense of \ili{English} and American culture since the advent of broadcast pluralism in Malta, and especially with the rise of cable television and online video streaming.} at least as far as Maltese lexicon is concerned, it has not supplanted \ili{Italian} in the domains of high culture and the affairs of state (e.g. \textit{gvern} `government' < \ili{Italian} \textit{governo}, \textit{poeżija} `poem' < \ili{Italian} \textit{poesia}).

\subsubsection{Minor sources}

Considering its location and the nature of population movements in the Mediterranean, it is hardly surprising that the Maltese lexicon also contains borrowings from languages other than \ili{Sicilian}, \ili{Italian} and \ili{English}. %other romance
The most obvious of these are borrowings from other \ili{Romance} languages. First among them, as in other European languages, stands \ili{Latin}, which provided a large chunk of Maltese scientific and technical vocabulary, whether as terminology (e.g. \textit{ego},  \textit{rektum} or  \textit{sukkursu} `underground water'), biological nomenclature (\textit{fagu} `European beech, \textit{Fagus sylvatica}', \textit{mirla} `brown wrasse, \textit{Labrus merula}') or set phrases and expressions (\textit{ex cathedra}, \textit{ibidem}). Curiously for a Catholic country, \ili{Latin} is the source of very little religious vocabulary in Maltese; in this area, Maltese continues to rely almost exclusively on words of \ili{Arabic} origin. Those \ili{Latin} words related to religious matters employed in modern Maltese therefore typically refer to minutiae of Catholic Church rituals and procedures, such as \textit{ekseat} `a bishop's permission for a priest to leave the diocese' (< \textit{exeat}) or \textit{indult} `a Pope's authorization to perform an act otherwise not allowed by canon law'. Of the few \ili{Latin} terms related to religion still in common use, \textit{nobis} stands out as a rather curious lexical item: in Maltese, it is used as a (post-nominal) modifier indicating intensity or size, as in \textit{tkaxkira nobis} `a sound thrashing' or \textit{tindifa nobis} `a thorough cleaning'.


%\ili{Spanish}, \ili{Catalan} and \ili{Occitan}
Before the Order of Saint John gained control of Malta, the islands were for more than two centuries a part (whether officially or not) of the Crown of Aragon. As such, one would expect that speakers of Maltese during that era found themselves exposed the languages of the Crown like \ili{Catalan}, \ili{Spanish} and \ili{Occitan}, and that this was then reflected in the Maltese lexicon. In truth, however, there are only a few Maltese words that can clearly be traced to Ibero-\ili{Romance}. \cite{catalan2017} identify a number of lexical items with \ili{Catalan} or \ili{Occitan} origins, but note that many of them can also be found in \ili{Sicilian}, which in most cases can be clearly determined as the origin of the loan. On the other hand, there are Maltese words of obviously \ili{Romance} origin whose current shape cannot be easily explained by any of the processes by which \ili{Sicilian} or \ili{Italian} words were made to conform to Maltese phonology, and where the \ili{Catalan} or \ili{Occitan} origin postulated by \cite{catalan2017} may offer a better explanation than that of ``local \isi{formation}" resorted to by previous works. These may include: \textit{boxxla} `compass' < \ili{Catalan} \textit{búixola} vs. \ili{Italian} \textit{bussola}; \textit{frixa} `pancreas' < \ili{Catalan} \textit{freixura} `entrails' and even the very frequent \textit{żgur} `certain', which, due to its phonology, especially the /g/ (see §\ref{sec:311}), points to an origin in \ili{Catalan} \textit{segur} or \ili{Spanish} \textit{seguro}, rather than to its (Tuscan) \ili{Italian} or \ili{Sicilian} cognates, which both feature a /k/ in its place. These and other lexical items, onomastics (see \citealt[46]{catalan2017}), and even usage (such as the ubiquitous Maltese swear word \textit{l-ostja}, literally `the host, sacramental bread', which is very atypical for \ili{Italian} or \ili{Sicilian}, but has a counterpart in the \ili{Spanish} \textit{la hostia}) suggest some influence of Ibero-\ili{Romance} on Maltese which is yet to be thoroughly researched.

% \ili{French}
The much shorter French occupation of the Maltese islands left very little linguistic trace, and so it is internationalisms in the semantic field of culture (\textit{bonton} `high society', \textit{etikett} `etiquette'), fashion (\textit{manikin} `manequin') and the culinary arts (\textit{fundan} `fondant', \textit{ragu} `ragout') where \ili{French} borrowings in Maltese can be found. The few notable exceptions include \textit{berġa} (< \textit{auberge}), the term used for the residences of langues (chapters) of the Order of Saint John. The most prominent of these palaces, \textit{Berġa ta' Kastilja}, now houses the office of the Prime Minister of Malta, for which the term \textit{Berġa} is often used metonymically. The other two Maltese words of \ili{French} origin still in frequent daily use both happen to be connected to transportation: \textit{xufier} (< \textit{chauffeur}) `driver' and \textit{xarabank} (< \textit{char à bancs}) `bus'. The latter is particularly interesting due to its pronunciation /ʃɐrɐˈbɐnk/, which indicates that it was borrowed directly from \ili{French} and not from \ili{English} (which would give /ʃɛrɛˈbɛnk/, as well as for its connection to the \ili{French}-speaking Maghreb, where the same word was in use; this indicates the possibility that it was brought from there by Maltese expatriates.

%\ili{Greek}
In addition to \ili{Romance} languages, post-classical \ili{Greek}, with its ubiquitous presence all across the Mediterranean (including the neighboring \isi{Sicily}), could not help but leave a trace on Maltese vocabulary, small though it is. Aquilina (\citeyear[23]{aquilina1976}) gives \textit{Lapsi} `Feast of Ascension' (< \textit{análipsi}) as the solitary example of a Maltese religious term not inherited from Christian \ili{Arabic} or borrowed from \ili{Romance} languages. The other two examples of \ili{Greek} \isi{loanwords} involve a completely different sphere. The first is \textit{ħamallu} `lewd, vulgar person', from \ili{Greek} \textit{xamális} (\citealt[7781]{megaleksikon1958}). This word may ultimately be traceable to \ili{Arabic} (through \ili{Turkish}), as is evident from its other meaning in \ili{Greek}, namely `porter' (< \textit{ḥammāl}). However, the meaning in which it appears in Maltese is unique to the \ili{Greek} word, indicating that it was borrowed into Maltese from \ili{Greek}. The other such term is \textit{vroma} `complete failure, fiasco' which is quite straightforwardly traceable to the \ili{Greek} \textit{vróma} `dirt, filth' (\citealt[1506, 1516]{megaleksikon1958}).

%\ili{Berber}

With regard to the debates on the origin and history of Maltese, borrowings from other \ili{Afro-Asiatic} languages have long been at the centre of attention of Maltese etymological research. \ili{Berber} is perhaps the most notorious example here, with a number of items cited as having \ili{Berber} origins by \cite{colin1957} and Aquilina (\citeyear[25--39]{aquilina1976}). Aquilina's list is an expansion of Colin's and thus both feature the same conspicuous items, which for the most part involve zoology, such as \textit{fekruna} `tortoise' (< \textit{fekrun}; \citealt[553]{naitzerrad2002}) and \textit{gendus} `bull' (< \textit{agenduz}; \citealt[827]{naitzerrad2002}).
%or botany, e.g. \textit{lellux} "chrysanthemum" (< , \citealt[XXX]{naitzerrad2002}). - Don't know how easily we'll find an authoritative etymology for this: I don't think Nait-Zerrad got that far!
Additionally, Aquilina postulates a \ili{Berber} origin for a number of lexical items where this seems questionable. In some cases the items in question are obviously \ili{Arabic} \isi{loanwords} in \ili{Berber} (as with \textit{bilħaqq} `by the way', quite transparently from \ili{Arabic} \textit{b-il-\d{h}aqq} `in truth'). In other cases subsequent research has argued against a \ili{Berber} origin. For example, while Aquilina identifies \textit{żenbil} `a large carrying basket' as having a \ili{Berber} origin, Borg (\citeyear[261]{Borg2004}) notes that it can also be found in the \ili{Arabic} dialect of Aleppo and Arbil, and traces its ultimate origin to \ili{Akkadian} through \ili{Aramaic}. A large group of similarities between Maltese and \ili{Berber} identified by Aquilina involve ``\ili{Berber} nursery language'', containing items like \ili{Berber} \textit{papa} `bread' and Maltese \textit{pappa}, \ili{Berber} \textit{ppspps} or \textit{ppssi} `urine' and Maltese \textit{pixxa}, and \ili{Berber} \textit{kakka/qaqah} and Maltese \textit{kakka} (both having to do with defecation). These forms are actually attested cross-linguistically (\citealt{ferguson1964}) at least as far north as \ili{Slovak} (\citealt{ondrackova2010}) and cannot thus be considered loans from \ili{Berber}. Nevertheless, the fact that there is a \ili{Berber} lexical component in Maltese is well established, and \cite{Souag2018berber} has shown that it may be larger than previously thought (e.g. his case for the \ili{Berber} etymology of the frequent adjective \textit{ċkejken} `small').

%\ili{Aramaic}
Finally, in addition to \ili{Berber}, Maltese also contains a small number of words that can be reasonably traced back to \ili{Aramaic}. Along with obsolescent lexical items such as \textit{żenbil} given above or \textit{andar} `threshing floor' (\citealt[116--117]{Behnstedt2005}), this small list includes the frequent verb \textit{xandara} `to broadcast, to spread (news)', otherwise unattested in any other variety of \ili{Arabic} (\citealt[46]{Borg1996}). This verb is presumably derived from the common \ili{Aramaic} \isi{root} \textit{√šdr} `to dispatch, send' with cognates in \ili{Mandaic} (\citealt[450]{manddic1963}), \ili{Jewish Babylonian Aramaic} (\ili{JBA}; \citealt[1112-1113]{sokoloff2002}) and Christian \ili{Neo-Aramaic} (\citealt[1179]{Khan2008}). The insertion of [n] reflects the dissimilation of the geminated [dd] into [nd] (\citealt[175--176]{Lipinski1997}); the same phenomenon involving the original geminated [bb] can also account for \textit{żenbil} (cf. \ili{JBA} \textit{zabbīlā}; \citealt[397]{sokoloff2002}). These borrowings could on the one hand strengthen the case for a \ili{Levantine} \isi{substrate} in (if not origin of) Maltese, as \cite{Borg1996} insists; on the other hand, some of them can also be found in other North African varieties (\citealt{Behnstedt2005}).

\section{Conclusion}
This chapter has reviewed the extensive changes that have taken place in Maltese as a result of contact with \ili{Sicilian}, Tuscan \ili{Italian}, \ili{English}, and other languages. The changes due to contact with Italo-\ili{Romance} languages are so striking, especially but by no means only with respect to lexicon, that it is almost misleading to speak of these contacts having changed ``Maltese". Rather it might be argued that it was a \ili{Maghrebi} \ili{Arabic} dialect like any other that was subjected to these changes, and that Maltese, the distinct language that its speakers now feel it to be, was what emerged only once these changes were complete. The result is a language in which typically \ili{Semitic} and typically Indo-European elements exist side-by-side at all linguistic levels.

The elements of contemporary  Standard Maltese that are the result of contact, summarized in this chapter, are now relatively well understood. But the language has naturally also evolved in numerous ways that owe little or nothing to the effects of contact with other languages. With a few notable exceptions (e.g. \citealt{Borg1978,Vanhove1993}), these changes have received far less attention. A desideratum for \isi{future} historical linguistic work on Maltese is therefore to redress this imbalance.

Concerning contact-induced change specifically, \isi{future} research could fruitfully include \isi{comparative} work on the differential effects of contact on standard versus dialectal Maltese. And to the extent that it is possible, the field would benefit greatly from a detailed history of the sociolinguistic effects of language contact in Malta in the early modern period.

\section*{Further reading}
\begin{furtherreading}
\item \cite{krier1976} is a short monograph on the influence of Italo-\ili{Romance} on Maltese phonology, morphology, syntax, and lexicon.
\item \cite{mifsudloanverbs} gives an in-depth description of Maltese loaned verbs.
\item \cite{comriespagnol2016} examine lexical borrowing in Maltese in the context of \isi{loanword} typology crosslinguistically.
\item \cite{Drewes1994} and \cite{stolz2003} explore the question of whether Maltese is properly labeled a ``mixed language''.
\end{furtherreading}

\section*{Acknowledgements}
The research presented in this chapter was partly funded by Leadership Fellows grant AH/P014089/1 from the UK Arts and Humanities Research Council, grant number APVV-15-0030 from the Slovak Research and Development Agency\linebreak\relax(APVV), and ERC Starting Grant number 679083 from the European Research Council, whose support is hereby gratefully acknowledged.

\section*{Abbreviations}

\begin{tabularx}{.5\textwidth}[t]{@{}lQ@{}}
\textsc{1, 2, 3} & 1st, 2nd, 3rd person \\
Adj & adjective \\
Adv & adverb \\
BCE & before Common Era \\
\textsc BCv3 & Bulbulistan corpus malti v3 \\
CE & Common Era \\
\textsc{comp} & {complementizer} \\
\textsc{dat} & dative \\
\textsc{def} & {definite} {article} \\
\textsc{dem} & demonstrative \\
\textsc{dep} & dependent form \\
{DOM} & {differential object marking} \\
\textsc{f} & feminine \\
JBA & Jewish Babylonian Aramaic \\
\textsc{gen} & genitive \\
\textsc{impf} & imperfect (prefix conjugation) \\
\textsc{inf} & {infinitive} \\
\textsc{prg} & pragmatic marker \\
\end{tabularx}%
\begin{tabularx}{.5\textwidth}[t]{@{}lQ@{}}
\textsc{L1, L2} & 1st, 2nd language \\
\textsc{m} & masculine \\
MUDTv1 & Maltese Universal Dependencies Treebank v1 \\
N & noun \\
\textsc{neg} & negative (particle) \\
NP & {noun phrase}\\
\textsc{obj} & object \\
\textsc{obl} & oblique \\
\textsc{pass} & passive \\
\textsc{pl} & plural \\
\textsc{prf} & perfect (suffix conjugation) \\
Prn & pronoun \\
\textsc{prog} & progressive \\
\textsc{ptcp} & {participle} \\
\textsc{sg} & singular \\
\textsc{sing} & singulative \\
V & verb \\
\end{tabularx}%


%%% comment by S
%%% added my affiliation stuff
%%% also fixed a typo

\section*{Primary sources}
Maltese examples above are primarily cited from the general corpus of Maltese \textit{bulbulistan corpus malti v3} (accessible at www.bulbul.sk/bonito2, login: guest, password: Ghilm3), as well as from the \textit{Maltese Universal Dependencies Treebank v1} (accessible at www.bulbul.sk/annis-gui-3.4.4/), both described as to their composition and annotation in \cite{bulbul2018}. Each citation is accompanied by an abbreviation identifying the source (BCv3 and MUDTv1, respectively), as well as the specific document where it can be found.

\sloppy
\printbibliography[heading=subbibliography,notkeyword=this]

\end{document}
