\documentclass[output=paper]{langsci/langscibook} 
\ChapterDOI{10.5281/zenodo.3744543}
\author{Bruno Herin\affiliation{Inalco, IUF}}
\title{Northern Domari}
\abstract{This chapter provides an overview of the linguistic outcomes of contact between Arabic and Northern Domari. Northern Domari is a group of dialects spoken in Syria, Lebanon, Jordan and Turkey. It remained until very recently largely unexplored. This article presents unpublished first-hand linguistic data collected in Lebanon, Syria, Jordan and Turkey. It focuses on the Beirut/Damascus variety, with references to the dialects spoken in northern Syria and southern Turkey.}
\IfFileExists{../localcommands.tex}{
  \input{../localpackages}
  %Copy this to localcommands.tex

\usepackage[english]{babel}
\usepackage{amsmath}
\usepackage{amssymb,amsfonts,textcomp}
\usepackage{array}
\usepackage{hhline}
\usepackage{hyperref}

\newenvironment{styleStandard}{}{}
\newenvironment{stylelsAbstract}{}{}
\newenvironment{stylelsSectioni}{}{}
\newenvironment{stylelsSectionii}{}{}
\newenvironment{stylelsBulletList}{}{}
\newenvironment{styleBibliographyi}{}{}
\newenvironment{listWWNumxxvleveli}{}{}
\newenvironment{listWWNumxxvlevelii}{}{}
\newenvironment{listWWNumxxvleveliii}{}{}
\newenvironment{listWWNumxxvleveliv}{}{}
\newenvironment{listWWNumixleveli}{}{}
\newenvironment{listWWNumixlevelii}{}{}
\newenvironment{listWWNumixleveliii}{}{}
\newenvironment{listWWNumixleveliv}{}{}

\newcommand\textstyleListLabelxvi[1]{#1}
\newcommand\labellistWWNumxxvleveli{\thelistWWNumxxvleveli.}
\newcommand\labellistWWNumxxvlevelii{\thelistWWNumxxvlevelii.}
\newcommand\labellistWWNumxxvleveliii{\thelistWWNumxxvleveliii.}
\newcommand\labellistWWNumxxvleveliv{\thelistWWNumxxvleveliv.}
\newcommand\labellistWWNumixleveli{[F0B7?]}
\newcommand\labellistWWNumixlevelii{\textstyleListLabelxvi{o}}
\newcommand\labellistWWNumixleveliii{[F0A7?]}
\newcommand\labellistWWNumixleveliv{[F0B7?]}

\newcounter{listWWNumxxvleveli}
\newcounter{listWWNumxxvlevelii}[listWWNumxxvleveli]
\newcounter{listWWNumxxvleveliii}[listWWNumxxvlevelii]
\newcounter{listWWNumxxvleveliv}[listWWNumxxvleveliii]
\newcounter{itemize}  
  \input{../localhyphenation} 
  \togglepaper[1]%%chapternumber
}{}

\begin{document}
\maketitle
 
\section{Current state and historical development}

Domari is an Indic language spoken by the Doms in various countries of the Middle East. The Doms are historically itinerant communities who specialize in service economies. This occupational profile led the lay public to call them the Middle Eastern Gypsies. Common occupations are informal dentistry, metalwork, instrument crafting, entertainment and begging. Most claim Sunni Islam as their religion, with various degrees of syncretic practices. Although most have given up their semi-nomadic lifestyle and settled in the periphery of urban centres, mobility is still a salient element in the daily lives of many Doms.

The ethnonym Dom is mostly unknown to non-Doms, who refer to them with various appellations such as \textit{nawar,} \textit{qurbāṭ} or \textit{qarač}. The \ili{Standard} \ili{Arabic} word \textit{ɣa\v{g}ar} for `Gypsy' is variably accepted by the Doms, who mostly understand with this term European Gypsies. All these appellations are exonyms and the only endonym found across all communities is \textit{dōm}. Only the Gypsies of Egypt, it seems, use a reflex of \textit{ɣa\v{g}ar} to refer to themselves. 

From the nineteenth century onwards, European travellers reported the existence of Domari in the shape of word lists collected in the Caucasus, Iran, Iraq and the Levant (see \citealt{Herin2012} for a discussion of these sources). The first full-length grammatical description of a dialect of Domari is by \citet{Macalister1914}, who described the dialect spoken in Palestine in the first years of the twentieth century. At present, the language is known to be spoken in Palestine, Jordan, Lebanon, Syria and Turkey. No recent account can confirm that it is still spoken in Iraq and Iran. There are roughly two dialectal areas: \ili{Southern} Domari, spoken in Palestine and Jordan, and Northern Domari, spoken in Lebanon, Syria and southern Turkey. This geographical division is not clear cut, as I have recorded speakers of \ili{Southern} varieties in Lebanon and speakers of Northern dialects in Jordan. The main isogloss separating these two groups is the maintenance of a two-way {gender} system. \ili{Southern} dialects have maintained the {gender} distinction, whereas it has mostly disappeared in the north. Compare Northern \textit{gara} ‘(s)he went’ vs. \ili{Southern} \textit{gara} ‘he went’ and \textit{garī} ‘she went’. These are sufficiently different to allow us to posit an early split. Mutual intelligibility appears to be very limited. A case in point is {kinship terminology}, which is largely divergent in both groups. Within Northern Domari, the \ili{Beirut/Damascus} dialect stands out because of the glottal realization [ʔ] of etymological /q/ and the loss of the differential subject marker \textit{-ən}. 

No general statement can be made about language {endangerment}. \ili{Jerusalem Domari} is reported to have only one fluent speaker left (Matras, this volume),\ia{Matras, Yaron@Matras, Yaron} but the presence of speakers of Palestinian Domari in other places may not be excluded. Young fluent speakers of \ili{Southern} dialects are easy to find in Jordan. As far as Northern Domari is concerned, the language is no longer transmitted to the young generation in Beirut but it is in Damascus. In northern Syria, intergenerational {transmission} is quite solid. The situation in southern Turkey is, according to some consultants, more precarious, but I have personally witnessed quite a few children fully conversant with the language. In any case, bilingual Doms acquire both Domari and \ili{Arabic} in early childhood, making both languages equally “dominant” in Van Coestem’s (\citeyear{VanCoetsem1988,VanCoetsem2000}) terms.

Many Dom groups are also found in Eastern {Anatolia}. These groups have shifted to \ili{Kurdish} but maintained an in-group lexicon based on Domari, locally called Domani. According to what I could personally observe on the ground and what well-informed local actors reported to me, full-fledged Domari is not spoken beyond Urfa. East of Urfa, the shift to \ili{Kurdish} is complete and even the in-group lexicon is only remembered by elderly individuals. 

There are no reliable figures on the number of speakers of Domari. The language has often been mistaken for a variety of \ili{Romani} but this claim has no linguistic grounds, except that they are both classified as {Central} \ili{Indo-Aryan} Languages with a possible \ili{Dardic} {adstrate}.

\section{Contact languages}

Besides a {Central} Indic core and a \ili{Dardic} {adstrate}, the language exhibits various layers of influence. Easily identifiable sources of contact are \ili{Persian}, \ili{Kurdish}, \ili{Turkish} and finally \ili{Arabic}. This suggests, quite logically, that the ancestors of the Doms left the Indian subcontinent, and then travelled into \ili{Persian}-speaking lands, before reaching \ili{Kurdish}- and \ili{Turkish}-speaking areas (most probably in eastern {Anatolia}), before venturing into Arab lands. It is striking to see that the \ili{Iranian} and \ili{Turkic} elements in Domari are not uniform across Northern and \ili{Southern} varieties, which suggests an early split in eastern {Anatolia} between speakers of both groups. The impact of \ili{Arabic} is also not uniform both across \ili{Southern} and Northern Domari, nor even within Northern Domari. What this means is that the validity of any discussion of the \ili{Arabic} component in Domari is limited to the varieties considered.

The \ili{Beirut/Damascus} dialect is undoubtedly the most Arabized one within the Northern group, pointing to an earlier settlement of the community in an \ili{Arabic}-speaking environment. Bilingualism (Domari--\ili{Arabic}) is general in Lebanon and Syria. Except perhaps for very young children who have not yet acquired any other language, monolinguals in Domari are not to be found.

As far as Turkey is concerned, trilingualism in Domari, \ili{Turkish} and \ili{Kurdish} is not uncommon, especially in southern Turkey around Gaziantep. In Hatay province, many speakers above the age of forty are trilingual Domari--\ili{Arabic}--\ili{Turkish}. The generations born here in the eighties onwards did not acquire \ili{Arabic}. 

According to personal recollection from various consultants, the community of Beirut/\ili{Damascus} used to spend the winter in Lebanon, and would go back to Damascus in the summer. This semi-nomadic way of life seems to have stopped when the civil war in Lebanon began. Although movements between Beirut and Damascus remained frequent, this phenomenon ceased to be seasonal. In Damascus, they settled in the area of Sayyida Zaynab, in the suburbs of the city, and in Beirut many of them settled in Sabra. Since the civil war started in Syria, virtually all the Damascus community have moved to Lebanon and settled in refugee camps in the Bekaa Valley close to the Syrian border.\pagebreak

\section{Contact-induced changes in Northern Domari}

As noted above, Domari speakers in Lebanon and Syria are also fully proficient in \ili{Arabic}, to the point that I have never encountered or heard of any {monolingual} adult. The Dom community, although largely endogamous and socially isolated, cannot afford monolingualism, primarily because of their peripatetic profile. As far as one can judge, their proficiency in \ili{Arabic} is that of any {monolingual} native speaker of \ili{Arabic}. Their pronunciation, however, is often not fully congruent with the local dialect spoken in the immediate vicinity of their settlements. This is, as usual, due to the variety of inputs and migration after acquisition. The Doms of Beirut for instance, do not speak \ili{Beirut} \ili{Arabic} and their speech is immediately perceived as \ili{Syrian} by {Lebanese} because they do not raise /ā/. Raising of /ā/ towards [eː] is the hallmark of \ili{Lebanese} \ili{Arabic} in perceptual dialectology. Proficient speakers of Domari all exhibit \ili{Arabic}--Domari {bilingualism}. On the whole, there is a general license to integrate any \ili{Arabic} lexeme in Domari speech, even when a non-\ili{Arabic} morpheme exists. Code-switching is also very common and there seems to be no conservative ideology about linguistic practices, leading to a very permissive environment for language mixing.


 
 \subsection{Phonology}


All the segmental phonology of \ili{Arabic} has made its way into Domari. \ili{Arabic} stands out cross-linguistically because of its series of back consonants such as the {pharyngeals} /ḥ/ and /ʕ/, the post-velars /q/, /ḫ/ and /ɣ/, and a set of velarized consonants whose number varies from dialect to dialect. Typically, sedentary varieties in the Levant minimally exhibit contrast between /ḍ/, /ẓ/, /ṭ/ and /ṣ/. In Domari, the {pharyngeals} /ḥ/ and /ʕ/ are commonly found in loans from \ili{Arabic}: \textit{ḥḍər} \textit{h-} ‘watch’ (from \ili{Levantine} \ili{Arabic} \textit{ḥiḍir} ‘he watched’). The same goes for /ʕ/: \textit{ʕammər} \textit{kar-} `build' (from \ili{Arabic} \textit{ʕammar} `he built'). An oddity surfaces in the word for coffee, realized \textit{ʔaḥwa} from \ili{Arabic} \textit{ʔahwe}. These {pharyngeals} are also common in \ili{Kurdish}-derived items such as \textit{ḥazār} ‘thousand’, \textit{moʕōri} ‘ant’ and also in the inherited (Indic) stock in \textit{ʕaqqōr} ‘nut’. Post-velar /q/, /ḫ/ and /ɣ/ are found in all the layers of the language: \textit{qāla} ‘black’ (inherited), \textit{qāpī} ‘door’ (\ili{Turkish}), \textit{sāɣ} ‘alive’ (\ili{Kurdish}), \textit{ɣarīb} ‘strange’ (\ili{Arabic}). The most striking innovation of the \ili{Beirut/Damascus} dialect is the glottal realization [ʔ] of /q/: \textit{ʔər} ‘son’ (< \textit{qər}), \textit{ʔāyīš} ‘food’ (< \textit{qāyīš}). This innovation is very likely contact-induced because it is commonly found in the \ili{Arabic} dialects of both \ili{Damascus} and \ili{Beirut} and beyond. 

Velarized consonants mostly surface in the \ili{Arabic}-derived stock as in \textit{naḍḍəf} \textit{kar-} ‘clean’ (< \ili{Arabic} \textit{naḍḍaf} ‘he cleaned’), but also in pre-\ili{Arabic} items: \textit{ḍāwaṭ} ‘wedding’ (borrowed from \ili{Kurdish} but ultimately from \ili{Arabic} \textit{daʕwa} ‘invitation’), \textit{pạ̄ṣ} ‘at him’ (< Old \ili{Indo-Aryan} \textit{pārśvá} ‘side’). It is still unclear to what extent velarization in Domari continues \ili{Indo-Aryan} retroflexion \citep[64]{Matras2012}. Domari also kept a contrast between /p/ and /b/, not found in \ili{Arabic}: \textit{bīrōm} ‘I feared’ vs. \textit{pīrōm} ‘I drank’. 

As far as vowels are concerned, \ili{Levantine} \ili{Arabic} exhibits either a two-way distinction in the short vowel system (/a/ and /ə/) or a three-way distinction (/a/, /i/ and /u/). In Northern Domari, only the two short vowels /a/ and /ə/ are contrastive: \textit{kərī} ‘house’ vs. \textit{karī} ‘pot’. Such a paucity of contrastive short vowels is probably due to contact with \ili{Arabic} varieties which exhibit a two-way system (/a/ vs. /ə/), such as many \ili{Lebanese} and \ili{Syrian} dialects. Most \ili{Arabic} dialects in the area have a five-way system of long vowels because of the monophthongization of /ay/ and /aw/: /ā/, /ī/, /ū/, /ē/ and /ō/. In addition to these long vowels, Domari displays another contrast between /ā/ and a back /ạ̄/ (IPA [ɑː]): \textit{māsī} [maːsiː] ‘meat’ (< Old \ili{Indo-Aryan} \textit{māṁsá})  vs. \textit{mạ̄s-ī} [mɑːsiː] ‘month-\textsc{pl}’ (< Old \ili{Indo-Aryan} \textit{mā́sa}).

Domari has also preserved distinct suprasegmental features, such as final syllable {stress} assignment. \ili{Arabic}-derived items are fully integrated into this pattern and bear final primary {stress}, whether common nouns or proper nouns: Domari [faːˈdja] vs. \ili{Arabic} [ˈfaːdja] (personal name \textit{Fādya}). An interesting phenomenon is that \ili{Arabic} epenthetic vowels in final-syllable position are reinterpreted as plain vowels and bear primary {stress}.  Compare Domari [sˤaˈʕab] and \ili{Arabic} [ˈsˤaʕəb] ‘difficult’; Domari [waˈdˤaʕ] and \ili{Arabic} [ˈwadˤəʕ] ‘situation’.


 
\subsection{Morphology}\largerpage
Northern Domari has not borrowed any {derivational} or {inflectional} morphemes from \ili{Arabic}. This is of course due to the fact that \ili{Arabic} morphology is mostly {non-concatenative}. Borrowed morphology mostly comes from \ili{Kurdish} and \ili{Turkish}, whose morpheme segmentation is much more transparent. These borrowed morphemes must have entered Domari when \ili{Kurdish} and \ili{Turkish} were contact languages of Domari. A case in point is the \ili{Kurdish} {diminutive} \textit{-ək}, which has made its way into all layers of the lexicon: \textit{panč-ək} ‘tail’, \textit{ḫar-ək} ‘bone’ (both Indic), \textit{taḫt-ək} ‘wood’, \textit{qannīn-ək} ‘bottle’ (both derived from \ili{Arabic}: \textit{taḫt} ‘bed’ and \textit{qannīne} ‘bottle’). The dialects of northern Syria and southern Turkey have also borrowed from \ili{Kurdish} the {comparative} suffix \textit{{}-tar}, the \ili{Turkish} {conditional} marker \textit{{}-sa} and the \ili{Turkish} {superlative} marker \textit{ān}. These constructions are not available in the \ili{Beirut/Damascus} dialect, which relies entirely on \ili{Arabic}-derived material. Compare the translation of the \ili{Arabic} sentence \textit{inte} \textit{aḥsan} \textit{minni} ‘you are better than me’ into Sarāqib Domari (\ref{better}) and \ili{Beirut/Damascus} Domari (\ref{better1}):

\ea
{Sarāqib Domari}\\ \label{better}
\gll tō dēšōm bḫēz-tar ištōre\\
     \textsc{2sg} \textsc{1sg.abl} good\textsc{-cmpr} \textsc{cop.2sg}\\
\glt ‘You are better than me.’
\ex 
{\ili{Beirut/Damascus} Domari}\\ \label{better1}
\gll tō aḥsan wēšōm ištōr\\
 \textsc{2sg} better \textsc{1sg.abl} \textsc{cop.2sg} \\
\glt ‘You are better than me.’\\
\z

Sarāqib is located in northern Syria and the dialect spoken by the Doms of Sarāqib is a good representative of the Domari of northern Syria and southern Turkey. Three differences are immediately apparent. The first is morphological, whereby there are different forms for the ablative of the first-person pronoun. The second difference is syntactic: in (\ref{better}) the standard of comparison precedes the {comparative} adjective (\textit{dēšōm} \textit{bḫēz-tar}) and in (\ref{better1}) it follows it (\textit{aḥsan} \textit{wēšōm}).\footnote{Comparative constructions typically involve two noun phrases (NPs). \citet{Stassen2013} labels the object of comparison the “comparee NP” and the other the “standard NP”.} The \ili{Beirut/Damascus} Domari syntax exhibits full congruence with the \ili{Arabic} syntax. The third difference is lexical. Because \ili{Beirut/Damascus} Domari does not have at its disposal the morpheme \textit{{}-tar}, speakers are obliged to draw on \ili{Arabic} for the {comparative}. This phenomenon, labelled ``{bilingual suppletion}'' by Matras, is described at length for \ili{Jerusalem Domari} (\citealt[379--382]{Matras2012}; see also Matras, this volume: §3.5).

\ili{Beirut/Damascus} Domari also relies entirely on \ili{Arabic} material for the expression of time and date, as shown in (\ref{manane}). In northern Syria, speakers favour the use of inherited {numerals}, as exemplified in (\ref{hatta}).

\ea
{\ili{Beirut/Damascus} Domari}\\ \label{manane}
\gll mānane mi-s-sāʕa ʕašra la s-sāʕa sabʕa tmāne ōtanta sa čāɣ-an-sa\\
     stay.\textsc{ipfv.1pl} from-\textsc{det}{}-hour ten to \textsc{def}{}-hour seven eight there all children-\textsc{obl.pl-com}\\
   \glt ‘We stay there with the all the kids from ten o'clock to seven or eight o'clock.’\\
\ex \label{hatta}
{Sarāqib Domari}\\
\gll ḥatta saʕat štār ēwar mānde ē čōrt-ə{}-ma\\
     until hour four evening stay.\textsc{pfv.3sg} \textsc{dem.obl} wasteland-\textsc{obl-in}\\
\glt ‘He stayed until 4pm in this wasteland.’
\z
Some speakers of \ili{Beirut/Damascus} Domari also extend the use of \ili{Arabic} to higher {numerals} because, according to their own judgment, they have difficulties retrieving the pre-\ili{Arabic} options. A look at their distribution reveals that the main parameter that triggers the use of \ili{Arabic} items is not so much high {numerals}, but rather the complexity of the numeral. Compare in this regard (\ref{parda}) and (\ref{nacis}). In (\ref{parda}), the speaker uses \ili{Arabic} for the more complex numeral ‘95000’ but uses Domari items for simpler `2000', `3000' and `4000'.

\ea
{\ili{Beirut/Damascus} Domari}\\ \label{parda}
\gll pārda abōs šaʔʔ-āka ši ḫamse u tisʕīn alf dolar\\
     buy.\textsc{pfv.3sg} \textsc{3sg.ben} flat\textsc{{}-indf} about five and ninety thousand dollar\\
\glt ‘He bought a flat for her, about ninety-five thousand dollars.’
\ex\label{nacis}
{\ili{Beirut/Damascus} Domari}\\
\gll načīš-a-ki dī ḥazār trən ḥazār štār ḥazār dfaʕ kaštand dādōs kē\\
     dancing-\textsc{obl-abl} two thousand three thousand four thousand pay do.\textsc{prog.3pl} her.mother \textsc{ben}\\
\glt ‘They give two, three, four thousand (dollars) to her mother from dancing.’
\z

As noted above, it appears that the use of \ili{Arabic} {numerals} is closely linked to language dominance. Speakers themselves are aware of it and when asked why they do not use Domari {numerals}, they justify it claiming a lack of proficiency. Looking at the distribution of inherited and \ili{Arabic} {numerals} is therefore a good way to assess whether language attrition is incipient or not.

The impact of \ili{Arabic} is also apparent in some morphological differences between the \ili{Beirut/Damascus} variety and the dialects of northern Syria. For instance, the verb \textit{sək-} means ‘to learn’. The \ili{Beirut/Damascus} dialect adds the {passive} suffix \textit{-yā}/\textit{-ī}. The corresponding verb in \ili{Arabic} \textit{tʕallam} is marked with the valency-decreasing prefix \textit{t-}. What the speakers of the \ili{Beirut/Damascus} dialect have done is to replicate the valency-decreasing prefix \textit{t-} of \textit{tʕallam} by means of the Domari {passive} suffix  \textit{yā}/\textit{{}-ī}: \textit{skə-rd-ōm} (learn-\textsc{pfv-1sg}; northern Syria) vs. \textit{sk-ī-r-ōm} (learn-\textsc{pass-pfv-1sg}; \ili{Beirut/Damascus}) ‘I learnt’.

Unlike \ili{Southern} Domari, Northern Domari does not normally {transfer} \ili{Arabic} plurals. Speakers simply use the singular form and add the Domari plural marker \textit{-ī(n)}:  \textit{azʕar-īn} ‘thugs’ instead of the \ili{Arabic} plural \textit{zuʕrān}. \ili{Arabic} plurals do surface at times, but only when they exhibit a high degree of independence within the lexicon. Examples are \textit{ʔarāyb-ē-mā} (relatives-\textsc{pl-1pl}) ‘our relatives’, \textit{\v{g}īrān-ē-mā} (neighbors-\textsc{pl-1pl}) ‘our neighbors’, from \ili{Arabic} \textit{qarāyib} and \textit{\v{g}īrān}. Although these items have singular forms (respectively \textit{qarīb} and \textit{\v{g}ār}), they are arguably lexicalized plurals and independent entries in the \ili{Arabic} lexicon.


 
 \subsection{Syntax}
 \subsubsection{Constituent order}

The impact of \ili{Arabic} in the realm of syntax is not uniform across Domari dialects. Dialects of northern Syria and southern Turkey show a strong tendency towards a head-final constituent-order typology, both within the NP and the clause. This feature is areal, so its presence in Domari may well be contact-induced. The canonical syntax of the NP is (demonstrative) (numeral) (adjective) (noun) noun. Complex NPs could only be retrieved through elicitation (examples \REF{star} to \REF{dom}) and hardly occur in spontaneous speech. Example (\ref{star}) illustrates the canonical syntax, where all the modifiers appear to the left of the head. Speakers of \ili{Beirut/Damascus} Domari, however, tend to dislocate some modifiers to the right of the head, converging towards the \ili{Arabic} syntax, as in (\ref{dad}), (\ref{nam}) and (\ref{dom}).

\ea
{Sarāqib Domari}\\ \label{star}
\gll ē štār lāfty-ən-ki dād-ō-sā\\
     \textsc{dem} four girl-\textsc{obl.pl-abl} mother-\textsc{sg-3pl}\\
\glt ‘the mother of these four girls’
\ex \label{dad}
{\ili{Beirut/Damascus} Domari}\\
\gll dād-ō-sā štār lāfty-an-ki\\
     mother-\textsc{sg-3pl} four girl-\textsc{obl.pl-abl}\\
\glt ‘the mother of the four girls’
\ex  \label{nam}
{\ili{Beirut/Damascus} Domari}\\
\gll nām-ē-sā \v{g}əwr-an-ki tərn-an-ki\\
     name-\textsc{pl-3pl} woman-\textsc{obl.pl-abl} three-\textsc{obl.pl-obl}\\
\glt ‘the names of the three girls’
\ex \label{dom}
{\ili{Beirut/Damascus} Domari}\\
\gll dōm-an-sa ēr-an-sa štār-an-sa\\
     dom-\textsc{obl.pl-com} \textsc{dem-obl.pl-com} four-\textsc{obl.pl-com}\\
\glt ‘with these four Doms’
\z

In (\ref{nam}), the speaker also dislocates to the right the numeral \textit{trən} ‘three’ which normally appears to the left giving the expected order \textit{trən} \textit{\v{g}əwr-an-ki nām-ē-sā} (three woman-\textsc{obl}.\textsc{pl}{}-\textsc{abl} name-\textsc{pl}{}-\textsc{3pl}). The numeral remains unmarked for case when it appears to the left of the head. When it is placed to the right, it agrees in case with the head. This is also the case with the demonstrative in (\ref{dom}). Here the normal order would be \textit{ē} \textit{štār} \textit{dōm-an-sa} (\textsc{dem} four Dom-\textsc{obl.pl}{}-\textsc{com}). The fact that speakers replicate case marking on right-dislocated modifiers suggests that they feel the need to strengthen constituency in case of non-canonical syntax.

The influence of \ili{Arabic} also surfaces in the \ili{Beirut/Damascus} dialect in the syntax of the {quantifier} \textit{sa} ‘all’. This is normally located to the right of the head: \textit{ammat} \textit{sa} ‘all the people’ (`people all'). In \ili{Beirut/Damascus}, \textit{sa} consistently surfaces to the left, like the \ili{Arabic} {quantifier} \textit{kull}: \textit{sa} \textit{ammat} (\ili{Arabic} \textit{kull} \textit{in-nās}).\footnote{\ili{Arabic} \textit{kull} can also appear to the right as in \textit{in-nās} \textit{kull-ha} \textit{{\textasciitilde} kull-hum} ‘all the people’ but this is a marked syntax.}


 \subsubsection{Internal object}

Domari speakers regularly replicate \ili{Arabic} constructions and {idioms}, but tend to do so by recruiting inherited or pre-\ili{Arabic} material -- they do not borrow \ili{Arabic} material. For instance, all dialects have replicated the so-called {internal object} construction, commonly used in \ili{Arabic} as a predicate-modifying construction. Consider for instance (\ref{jordar}) in \ili{Jordanian} \ili{Arabic}, where the speaker narrows the scope of the predication using the verbal noun \textit{ʕirəf} ‘knowledge’, derived from the verb \textit{ʕirif} ‘he knew’, and modifies it with the adjective \textit{ṭayyib} ‘good’. In (\ref{dados}), the speaker has used the deverbal {derivation} \textit{kūš} from the {root} \textit{kū-} ‘throw’ and coded it as an object, as evident from the accusative marker \textit{-əs}. This replicates the \ili{Arabic} {internal object} construction as illustrated in (\ref{jordar}).

\ea
{\ili{Jordanian} Arabic}\\ \label{jordar}
\gll baʕrif-hum ʕirəf ṭayyib\\
     know.\textsc{impf.1sg-3pl} knowledge good\\
\glt ‘I know them well.’
\ex \label{dados}
{Sarāqib Domari}\\
\gll dād-ōs ibnḥarām e ē kūš-əs ktōs-s-e\\
     mother-\textsc{3sg} son.of.illicit \textsc{cop} \textsc{dem} throwing-\textsc{acc} throw.\textsc{pfv.3sg-obj.3sg-prs}\\
\glt ‘His mother is heartless for having thrown (her baby) in such a way.’
\z

 \subsubsection{Impersonal construction}

Speakers also replicate the \ili{Arabic} {impersonal construction} with the indefinite pronoun \textit{il-wāḥad} by way of the inherited noun \textit{mānəs} ‘individual, people’. Example (\ref{jordar1}) illustrates the use of \textit{il-wāḥad} in (\ili{Jordanian}) \ili{Arabic}. In (\ref{asti}), the sequence \textit{gzare} \textit{māns-as} corresponds to \ili{Arabic} \textit{biʕiḍḍ} \textit{il-wāḥad,} literally ‘it bites one’. The fact that \textit{māns-as} replicates \textit{il-wāḥad} is also apparent from the accusative marking in Domari, which normally surfaces only with {definite} objects. The referent here is by nature indefinite and non-referential, so accusative marking in Domari can only be explained by the presence of the {definite} {article} \textit{il-} in \ili{Arabic} \textit{il-wāḥad}.

\ea
{\ili{Jordanian} Arabic}\\ \label{jordar1}
\gll kān ʕēb il-wāḥad yrūḥ ʕala ʔutēl\\
     be.\textsc{prf.3sg.m} shameful \textsc{def-}one go.\textsc{impf.sbjv.2sg.m} to hotel\\
\glt ‘One was ashamed to spend the night in a hotel.’
\ex \label{asti}
{\ili{Beirut/Damascus} Domari}\\
\gll ašti ši hana lli baḥr-a-ma e gzare māns-as\\
     \textsc{exs} too \textsc{dem} \textsc{rel} sea-\textsc{obl-in} \textsc{cop} bite.\textsc{ipfv.3sg} man-\textsc{acc}\\
\glt ‘There is this thing in the sea, it bites you.’
\z


 \subsubsection{Auxiliaries}

Probably the most striking difference between \ili{Southern} and Northern Domari as far as the \ili{Arabic} component is concerned is the absence of \ili{Arabic} inflected material in the latter. Only the dialect of \ili{Beirut/Damascus} has borrowed the auxiliaries \textit{kān} (with its imperfect form \textit{bikūn}), \textit{ṣār} and \textit{ḫalli}.

\ea
{\ili{Beirut/Damascus} Domari}\\ \label{sar}
\gll ṣār \v{g}ahhəz lakand lāfty-a kē bḫēr\\
     become.\textsc{prf.3sg} prepare do\textsc{.sbjv.3pl} girl\textsc{{}-obl} \textsc{ben} well\\
\glt ‘They prepare the girl well now (for the wedding).’
\ex \label{xadra}
{\ili{Beirut/Damascus} Domari}\\
\gll ḫaḍra kān mə\v{g}nār-a\\
     Khadra be.\textsc{prf.3sg.m} breastfeed.\textsc{ipfv.3sg-pst}\\
\glt ‘Khadra was breastfeeding.’
\ex \label{awande}
{\ili{Beirut/Damascus} Domari}\\
\gll āwande bikūn krēnde mā kē kyāmōr\\
     come.\textsc{ipfv.3pl} be.\textsc{impf.3sg} do.\textsc{prf.3pl} \textsc{1sg} \textsc{ben} something\\
\glt ‘(My kids) would come and they would have done something (naughty).’ 
\z

In (\ref{sar}), the subject is in the 3\textsc{pl} but \textit{ṣār} remains invariable, as the \textsc{3pl} is \textit{ṣāru}. In (\ref{xadra}), the subject is feminine so if there was {agreement} one would expect \textit{kānat}, not masculine \textit{kān}. A further intriguing feature in (\ref{xadra}) is the redundancy in past marking, first with \textit{kān} and second with the past suffix \textit{{}-a}, which in northern Syria and southern Turkey Domari suffices to mark past {tense}. The same invariability is apparent in (\ref{awande}) where the \textsc{3pl} of \textit{bikūn} should be \textit{bikūnu}. These auxiliaries have the same semantic load as in \ili{Arabic}. The morpheme \textit{ṣār} puts {emphasis} on the inception of the event, \textit{kān} followed by the imperfect places the event in the past and gives it an iterative/habitual aspect and \textit{bikūn} describes a possible state of affairs not attested at the time of utterance. \ili{Arabic} \textit{ṣār}, \textit{kān} and \textit{bikūn} are absent in the dialects of northern Syria and southern Turkey. The only auxiliary that has been replicated here is \textit{ṣār}. These dialects, however, have only replicated the structure, not the substance, that is they rely on inherited morphemes, as exemplified in (\ref{har}). The speaker simply translates \ili{Arabic} \textit{ṣār} with the Domari equivalent \textit{hra}, replicating the \ili{Arabic} structure \textit{ṣār} + subjunctive (see Manfredi, this volume).\ia{Manfredi, Stefano@Manfredi, Stefano} A further difference is {word order}, with the verb placed clause-finally in the subordinate clause. 

\ea \label{har}
{Sarāqib Domari}\\
\gll hər wārsīndạ lwār\\
     become.\textsc{pfv.3sg} rain hit.\textsc{sbjv.3sg}\\
\glt ‘It started raining.’
\z

As noted above, in these dialects the functions of \ili{Arabic} \textit{kān} are expressed by the inherited past suffix \textit{-a}. The functions covered by \ili{Arabic}, \textit{bikūn}, however do not seem to be encoded in the grammar of these dialects.

In \ili{Levantine} \ili{Arabic}, the imperative form \textit{ḫalli} ‘let’ of \textit{ḫalla} ‘he let’ is often used to soften an order and allows the speaker to avoid using an imperative, flagging a suggestion or an invitation, as shown in (\ref{xalli}):

\ea \label{xalli}
{\ili{Jordanian} Arabic}\\
\gll ḫalli ibn-ak yrūḥ la \v{g}-\v{g}ēš\\
     let son-\textsc{2sg.m} go.\textsc{impf.3sg.m} to \textsc{def}{}-army\\
\glt ‘Let your son serve in the army.’
\z

This auxiliary has been borrowed into \ili{Beirut/Damascus} Domari with the exact same function, as illustrated in (\ref{xalli1}). In this case too, \textit{ḫalli} remains invariable and does not surface as \textit{ḫallī-(h)un} (let.\textsc{imp.2sg-3pl}) as it would in \ili{Beirut/Damascus} \ili{Arabic}. Here again, the dialects of northern Syria and southern Turkey have borrowed the structure, but not the substance, and use the inherited {root} \textit{mək} ‘let’, as exemplified in (\ref{mek}).

\ea \label{xalli1}
{\ili{Beirut/Damascus} Domari}\\
\gll ḫalli \v{g}ānd dfən lakrand-əs\\
     let go.\textsc{sbjv.3pl} bury do.\textsc{sbjv.3pl-3sg}\\
\glt ‘Let them go and bury him.’
\ex \label{mek}
{Aleppo Domari}\\
\gll mək pāʋər pạ̄sōr\\
     let come.\textsc{sbjv.3sg} \textsc{2sg.ad}\\
\glt ‘Let him come to your place.’
\z


 \subsubsection{Negation}

Only two \ili{Arabic} negators have made their way into the grammar of Northern Domari: \ili{Damascus} \ili{Arabic} \textit{mū} and the contrastive negative {coordination} markers \textit{lā...walā} ‘neither…nor’. \ili{Arabic} \textit{mū} is only available in the dialect of \ili{Beirut/Damascus}. Its distribution and functions, however, only partially match those of \ili{Damascus} \ili{Arabic}. The primary function of \textit{mū} in \ili{Damascus} \ili{Arabic} is to negate non-verbal predicates. This is not attested in Domari, which relies for this purpose only on inherited \textit{nye}. \textit{mū} surfaces first when {negation} has scope over non-clausal constituents, as shown in (\ref{seff}), and second when the predicate is in a non-indicative mood (subjunctive, jussive and imperative) as in (\ref{guz}):

\ea
{\ili{Beirut/Damascus} Domari}\\ \label{seff}
\gll səff (h)ra wāšya mū wāšōm\\
     side become\textsc{.pfv.3sg} \textsc{3pl.com} \textsc{neg} \textsc{1sg.com}\\
\glt ‘He took sides with them, not with me.’
\ex \label{guz}
{\ili{Beirut/Damascus} Domari}\\
\gll bi\v{g}ūz mū māntyar wāš məṣrī\\
     possible \textsc{neg} stay\textsc{.sbjv.3sg} \textsc{3sg.com} money\\
\glt ‘He might not have any money left.’
\z

The \ili{Arabic} structure \textit{lā…walā} is readily available in all varieties, but whereas it is the only option in \ili{Beirut/Damascus}, it competes with the inherited structure \textit{nə…nə} in northern Syria and southern Turkey. Interestingly, this clash has led to a mixed form \textit{nə…walā}, as shown in (\ref{mix}). The Domari syntax is also reminiscent of the \ili{Turkish} possessive predication syntax with possessive marking on the noun and an {existential} morpheme.\pagebreak

\ea \label{mix}
{\ili{Antioch} Domari (southern Turkey)}\\
\gll nə lawr-ōs ašti wala šarš-ōs ašti\\
     \textsc{neg} tree\textsc{-3sg} \textsc{exs} \textsc{neg} {root}\textsc{-3sg} \textsc{exs}\\
\glt ‘It doesn’t grow on a tree nor has it {roots}.’
\z


 \subsubsection{Complex sentences}

Complex sentences minimally include coordinated and subordinate clauses. The \ili{Arabic} coordinators \textit{w} ‘and’, \textit{aw} ‘or’, \textit{walla} ‘or’, \textit{bass} `but' and others have all made their way into Domari. Originally, Domari seems to have distinguished clausal {coordination} from phrasal {coordination}, a not so frequent feature from a typological point of view. Nominal categories are coordinated with the \ili{Turkish}-derived morpheme \textit{la} and clauses are coordinated with the \ili{Kurdish}-derived enclitic -\textit{ši}. The intrusion of \ili{Arabic} \textit{w}, which in \ili{Arabic} is used indiscriminately for both kinds of {coordination}, has led to the marginalization of the original system in \ili{Beirut/Damascus} Domari, which now tends to favour the use of \ili{Arabic} \textit{w}.

\ea
{\ili{Beirut/Damascus} Domari}\\
\gll illi mangar tōre māṣṭ-a-ma w illi mangar ʔār-s-e nāšif\\
     \textsc{rel} want.\textsc{ipfv.3sg} put.\textsc{ipfv.3sg} yoghurt-\textsc{obl-in} and \textsc{rel} want.\textsc{ipfv.3sg} eat.\textsc{ipfv.3sg-obj.3sg-prs} dry\\
\glt ‘Some eat it in yoghurt, some eat it dry.’
\z

As far as phrasal {coordination} is concerned, some alternation between \ili{Arabic} \textit{w} and \ili{Turkish}-derived \textit{la} is still observed: \textit{dōmwārī} \textit{w} \textit{ṭāṭwārī} ‘Domari and \ili{Arabic}’ {\textasciitilde} \textit{dōm} \textit{la} \textit{ʕarabi} ‘Domari and \ili{Arabic}’.

Virtually all the conjunctions of subordination found in Domari are borrowed from \ili{Arabic}. This includes the {relativizer} \textit{illi}, the {complementizer} \textit{inno} and potentially all the adverbial conjunctions found in \ili{Levantine} \ili{Arabic}: \textit{lamma} ‘when’, \textit{qabəl-mā} ‘before’, \textit{baʕəd-mā} ‘after’, \textit{ʕa-bēn-mā} ‘by the time’, and many more. Pre-\ili{Arabic} constructions are attested for relativization and {conditional} clauses, but these only survive in the dialects of northern Syria and southern Turkey, and tend to be replaced by \ili{Arabic} material (except in the varieties spoken in Turkey). A case in point is {conditional} clauses. \ili{Arabic} \textit{iza} and \textit{law} are available everywhere, even in Turkey, as shown in (\ref{aza}), recorded in \ili{Antioch}. In this example, the speaker uses the \ili{Arabic} morpheme \textit{aza} (< \textit{iza}) in the first sentence of the utterance, and no overt marking in the protasis, making parataxis a possible means to express condition. As far as counterfactual conditions are concerned, it appears that the dialect of \ili{Beirut/Damascus} is fully congruent with \ili{Arabic} in having borrowed also the morpheme \textit{kān} in both the protasis and the apodosis, as shown in (\ref{law}). The dialects of northern Syria and southern Turkey exhibit a native strategy using subjunctive mood and past marking in the protasis and perfective and past marking in the apodosis. The two clauses are coordinated with the \ili{Kurdish} derived enclitic \textit{ši} (\ref{all}).

\ea \label{aza}
{\ili{Antioch} Domari}\\
\gll aza kām karne qāne kām nə-karne nə-qāne\\
     if work do.\textsc{ipfv.1pl} eat.\textsc{ipfv.1pl} work \textsc{neg}{}-do.\textsc{ipfv.1pl} \textsc{neg}{}-eat.\textsc{ipfv.1pl}\\
\glt ‘If we work, we eat, (if) we don’t work, we don’t eat.’
\ex \label{law}
{\ili{Beirut/Damascus} Domari}\\
\gll law kān nəčnār-sā bāb-ōm kān abṣar kaki (h)re\\
     if be.\textsc{prf.3sg} make.dance\textsc{.ipfv.3sg-obj.3pl} father-\textsc{1sg} be.\textsc{prf.3sg} not.know what become.\textsc{pfv.3sg}\\
\glt ‘If my father had put them to dance, I don’t know what would have happened.’
\ex \label{all}
{Sarāqib Domari}\\
\gll aḷḷ-əs byātyənd-a nə-ktēnd-s-a ši\\
     God-\textsc{acc} fear.\textsc{sbjv.3pl}{}-\textsc{pst} \textsc{neg}{}-throw.\textsc{pfv.3pl-obj.3sg-pst} and\\
\glt ‘Had they feared God, they would not have thrown him.’
\z

 \subsection{Lexicon}
 \subsubsection{Function words}

\ili{Arabic} {prepositions} do occur in Domari, but these are mostly non-core {prepositions} such as \textit{qabəl} ‘before’, \textit{baʕad} ‘after’, \textit{minšān} ‘for’, \textit{ɣēr} ‘other’. Some have made their way into Domari only recently, and still alternate with pre-\ili{Arabic} options, such as the \ili{Iranian} equative morpheme \textit{war,} which tends to be replaced by \ili{Arabic} \textit{mitəl} ‘as, like’ especially in the dialect of \ili{Beirut/Damascus}. Currently, \textit{war} and \textit{mitəl} are in a quasi-complementary distribution, with \textit{war} being used with full NPs and \textit{mitəl} with pronouns, as shown below in (\ref{rom}) and (\ref{tani}):

\ea
{\ili{Beirut/Damascus} Domari}\\ \label{rom}
\gll tō ʔr-ōm war ištōr\\
     you son-\textsc{1sg} like \textsc{cop.2sg}\\
\glt ‘You are like my son.’
\ex \label{tani}
{\ili{Beirut/Damascus} Domari}\\
\gll tāni ʔər gēna mitl-ōs kām karre\\
     second son also like-\textsc{3sg} work do.\textsc{ipfv.3sg}\\
\glt ‘My second son has the same job.’
\z

The \ili{Arabic} core {preposition} \textit{b-} ‘in, with’ occurs in Domari, but it appears to be restricted to certain constructions and {idioms} such as \textit{gāl} \textit{b-gāl} ‘discussion’ (word in-word), \textit{ārāt} \textit{əb-dīs} ‘night and day’ (night in-day), \textit{b-rəbʕ-āk} ‘for a quarter of a pound’ (with-quarter-\textsc{indf}). The {preposition} \textit{min} ‘from’ also sporadically occurs in \ili{Beirut/Damascus} Domari:

\ea
{\ili{Beirut/Damascus} Domari}\\
\gll min ši šēš mạ̄s ǧərsa krōm ḍāwaṭ-ōs\\
     from about six month wedding do.\textsc{pfv.1sg} wedding-\textsc{3sg}\\
\glt ‘Some six months ago I married him off.’
\z
 
Domari also borrows high-{frequency} adverbs, fillers, connectors and all kinds of discourse-structuring devices, such as \textit{masalan} ‘for instance’, \textit{abadan} ‘at all, never’, \textit{yaʕni}  ‘I mean’, \textit{aywa} ‘yes, so’, \textit{waḷḷa} ‘I swear’, \textit{inno} ({complementizer} and discourse marker) and many more. One finds also common adverbial phrases such as \textit{ṭūl} \textit{in-nhār} ‘all day long’, \textit{ṭūl} \textit{il-waʔət} ‘all the time’, and \textit{ʕala} \textit{ṭūl} ‘immediately’.  The very common Domari phrase \textit{tīka} \textit{tīka} ‘slowly’ replicates \ili{Arabic} \textit{šwayy} \textit{əšwayy}.


 \subsubsection{Content words}

In Syria and Lebanon, \ili{Arabic} is the \textit{de facto} lexical reservoir of Domari, so there is a general licence to integrate any element from \ili{Arabic} if no pre-\ili{Arabic} option exists. The issue is the replacement of pre-\ili{Arabic} options with \ili{Arabic} material. There is of course a certain amount of variation in lexical knowledge across speakers, but it seems possible to differentiate several levels of {replaceability}. Some items have long been replaced by \ili{Arabic} words, and only a handful of speakers are able to retrieve them, such as \textit{lōrga} ‘tomato’ or \textit{pīsənga} ‘bulgur’, replaced respectively by \ili{Arabic} \textit{bandōra} and \textit{bərɣəl}. Other items tend to be replaced by \ili{Arabic} equivalents but may still surface in the speech of some speakers, such as \textit{čatīn} ‘hard’, \textit{čirkī} ‘bird’, \textit{alčāḫ} ‘low’ replaced by \ili{Arabic} \textit{ṣaʕab}, \textit{ṭēr}/\textit{ʕaṣfūr} and \textit{wāṭi}. Some items seem stable but are sporadically replaced with \ili{Arabic}-derived items such as \textit{drəs} \textit{kar-} ‘study’ instead of inherited \textit{sək}{}-. Finally, other items  such as \textit{\v{g}awwəz h-} ‘get married’ and \textit{\v{g}irsāwī h-} freely alternate. It appears therefore that every pre-\ili{Arabic} item is somewhere on a continuum of {replaceability} from ``very unlikely'' to ``completely disappeared''. To illustrate the variablility in {replaceability} judgment, I remember an elicitation session in Aleppo with a father and his son. One of the sentences contained the \ili{Arabic} word \textit{baṣal} ‘onion’. The son simply translated the sentence with the \ili{Arabic} word \textit{baṣal} but the father strongly objected to this answer, stating that the proper Domari word is \textit{pīwāz}.

As noted above, \ili{Arabic} nouns are integrated in their singular form, except in the case of lexicalized plurals. Adjectives are borrowed in their masculine form and never agree in {gender}, as shown in (\ref{hala}). Other than the past {copula} \textit{a}, all the words in this example are \ili{Arabic}. Two features, however, allow its identification as Domari. First, \textit{ḥāla} is realized without raising (also stressed on the last syllable [ħaːˈla]), unlike \ili{Levantine} \ili{Arabic} \textit{ḥāle,} and second \textit{taʕbān} does not agree in {gender} with \textit{ḥāla} and surfaces in its masculine form, instead of feminine \textit{taʕbāne}, as it would normally occur in \ili{Arabic}.

\protectedex{
\ea \label{hala}
{\ili{Beirut/Damascus} Domari}\\
\gll ʔabəl ḥāla taʕbān a\\
     Before situation tired \textsc{cop.pst}\\
\glt ‘Before, the situation was bad’
\z
}\largerpage[2]

\ili{Arabic} verbs are easily integrated into Domari, because Domari has a {light verb} strategy. Roughly, transitive verbs tend to be integrated with the {light verb} \textit{kar-} ‘do’: \textit{rabbī} \textit{kar-} ‘raise’ from \ili{Arabic} \textit{rabba}, \textit{yrabbi} ‘raise’. Intransitive verbs are integrated with \textit{h-} ‘become’: \textit{ʕīš} \textit{h-} ‘live’ from \ili{Arabic} \textit{ʕāš}, \textit{yʕīš} ‘live’. While all the verbs that are integrated with \textit{kar-} are transitive, some verbs integrated with \textit{h-} are not intransitive: \textit{lməs} \textit{(h)rōs-s-e} ‘he has touched it’ (touch become.\textsc{pfv.3sg-3sg-prs}) from \ili{Arabic} \textit{lamas}, \textit{yilmis} ‘touch’. This seems to happen with transitive verbs that are lower on the transitivity scale, or at least perceived to be so. In the case of \textit{lamas}, \textit{yilmis}, its integration into Domari by way of the {light verb} \textit{h-} suggests that speakers perceive it as less transitive. Formally, speakers isolate the imperfect {stem} of the verb, and apply a vocalism in /i/: \textit{nsī} \textit{kar-} ‘forget’ and \textit{stannī} \textit{kar-} ‘wait’, from the \ili{Arabic} imperfect stems of \textit{nsa} ‘forget’ and \textit{stanna} ‘wait’.\footnote{These verbs are only available in \ili{Beirut/Damascus}, other dialects use respectively \textit{ziwra} \textit{kar-} and \textit{akī} \textit{kar-}.} An exception to this tendency occurs with the so-called hollow {roots} in \ili{Arabic} whose imperfect {stem} is CūC. In this case, speakers simply extract the imperfect {stem} and leave it unchanged: \textit{zūr} \textit{h-} ‘visit’, \textit{dūr} \textit{h-} ‘turn’, \textit{ʕūz} \textit{h-} ‘need’, from the \ili{Arabic} imperfect stems \textit{zūr}, \textit{dūr} and \textit{ʕūz}.

Some \ili{English}-derived items were also recorded in the \ili{Beirut/Damascus} dialect, such as \textit{mōmari} ‘memory card’, \textit{hambarga} ‘hamburger’ and, more surprisingly, \textit{tōman\v{g}īre} ‘Tom and Jerry’ [toːmanʤiːˈre], expectedly stressed on the last syllable.


 \subsubsection{Speech sample}

Probably the best way to capture how \ili{Arabic} integrates into Domari is to consider a piece of spontaneous speech, reproduced below in (\ref{speech}). It is part of a recorded discussion I had with a consultant in her mid-thirties in Beirut. It illustrates the level of {endangerment} of \ili{Beirut/Damascus} Domari. The consultant belongs to the last generation of fluent speakers. Her children did not acquire the language. According to what she reports, she was unable to speak to her children in their early childhood because her husband, who is a semi-speaker of Domari, prevented her from transmitting the language. Her daughter-in-law, aged twenty-one at that time, is also a fluent speaker of Domari because she grew up in Damascus, where language {transmission} was more solid than in Lebanon. Both of them use Domari in the household. Her son reacts negatively when he hears it, and even labels it \textit{a\v{g}nabi} ‘foreign, non-\ili{Arabic}’. Linguistically, the text illustrates some of the features discussed above. \ili{Arabic}-derived items are marked in boldface. 


\ea
{\ili{Beirut/Damascus} Domari}\\ \label{speech}
\gll nā n-ǧib karre pānǧī gāl karre gāl karre dōm wāšōm mā gāl kame wāšī \textbf{ʕādi} \textbf{bass} əʔr-ōm ʔzīn karre wat ftyare ma-gāl ka \textbf{aǧnabí} nə-\textbf{fəmm} (h)ōme watōr, gāl karse \textbf{ʕarabiy}{}-a-ma \textbf{yaʕni} ma-gāl k(a) ēhānī \textbf{laʔanno} n-\textbf{fəmm} (h)ōre watī \textbf{bass} mā l pānǧī ǧib kane \textbf{ṭūl} \textbf{il-waʔət} kəry-a-ma \textbf{yaʕni} \textbf{iza} mā l pānǧi štēn kəry-a-ma \textbf{ṭūl} \textbf{in-nhār} gāl kane dōm-a-ma \textbf{yaʕni} ʔr-ōm wāri \textbf{ʕəmr}-ōs \textbf{wāḥad} \textbf{u} \textbf{ʕišrīn} \textbf{sane} \textbf{akbar} ʔr-ōm-ki \textbf{b}{}-trən wars  \textbf{mū} \textbf{ʕādi} \textbf{ʕādi} nye amīn \textbf{lāzim} lpāran \textbf{azɣar} wēšōma \textbf{bass} bxēz e \textbf{u} \textbf{ādami} e \textbf{u} \textbf{maḥšūm} e mā ēhāny-a xr-a kē pārdōm-əs ʔr-ōm kē \textbf{u} \textbf{ǧamāʕt}-ēm kē skīr(a) ēta \textbf{baʕdēn} skīra \textbf{mahná} \textbf{baʕdēn} kām əkra wars-ā wars-ā nīm \textbf{makanīk} \textbf{baʕdēn} wəndrārda \textbf{u} īsa nə-kām kištar \textbf{wala} kkyā wēsre kəry-a-ma\\
no \textsc{neg}{}-tongue do\textsc{.ipfv.3sg} \textsc{3sg} word do.\textsc{ipfv.3sg} word do\textsc{.ipfv.3sg} Dom \textsc{1sg.com} \textsc{1sg} word do.\textsc{ipfv.1sg} \textsc{3sg.com} normal but son-\textsc{1sg} shout do\textsc{.ipfv.3sg} \textsc{3sg.supr} say.\textsc{ipfv.3sg} \textsc{neg}{}-word do.\textsc{sbjv.2sg} foreign \textsc{neg}{}-understand become.\textsc{ipfv.3sg} \textsc{2sg.supr} word do.\textsc{ipfv.2pl} \ili{Arabic}-\textsc{obl-in} I.mean \textsc{neg}{}-word do.\textsc{sbjv.2sg} so because \textsc{neg}{}-understand become.\textsc{ipfv.3sg} \textsc{3sg.supr} but \textsc{1sg} and \textsc{3sg} tongue do.\textsc{ipfv.1pl} length \textsc{def}{}-time house-\textsc{obl-in} I.mean if \textsc{1sg} and \textsc{3sg} \textsc{cop.1pl} house-\textsc{obl-in} length \textsc{def}{}-day word do\textsc{.ipfv.1pl} Dom-\textsc{obl-in} I.mean son-\textsc{1sg} bride age-\textsc{3sg.f} one and twenty year bigger son-\textsc{1sg-abl} with-three year \textsc{neg} normal normal \textsc{cop.neg} \textsc{1pl} must take\textsc{.sbjv.1pl} smaller \textsc{1pl.abl} but good \textsc{cop} and humane \textsc{cop} and respectful \textsc{cop} so so-\textsc{obl} heart-\textsc{obl} \textsc{ben} take.\textsc{pfv}.\textsc{1sg-obj.3sg} son-\textsc{1sg} \textsc{ben} and folks-\textsc{1sg} \textsc{ben} learn.\textsc{pfv.3sg} here then learn.\textsc{pfv.3sg} profession then work do.\textsc{pfv.3sg} year-\textsc{indf} year-\textsc{indf} half mechanic then fire.\textsc{pfv.3sg} and now \textsc{neg}-work do.\textsc{prog.3sg} nor thing sit.\textsc{pfv.3sg} house-\textsc{obl-in}\\
\glt ‘No, [my son] doesn’t speak [Domari], [my daughter-in-law] does, she speaks with me, I speak with her normally but my son shouts at her and tells her: “Don’t speak foreign, I don’t understand you, you all speak in \ili{Arabic}, don’t speak like this”, because he doesn’t understand her. But me and her we speak all the time in Domari, that is, if both of us are in the house, all day long we speak in Domari. The bride of my son, she is twenty-one years old, three years older than my son, it’s not usual, we [women] have to take someone older, but she is a good person, humane and respectful. That’s why I took her for my son and my family. [My son] studied here [in the school]. After that he went for vocational training and worked for a year a year and a half as a mechanic -- then he quit. And now he doesn’t do anything, he stays at home.’ 
\z

\section{Conclusion}

Multilingualism seems to have been a normal state of affairs amongst the Doms for a very long time, probably since the genesis of the community. The reason for this is mostly because the sociolinguistics of Domari has in likelihood remained unchanged throughout the centuries: Domari is a community language whose use is restricted to in-group communication. Out-group interactions imply the use of the majority language. Due to the very nature of their occupational profile, peripatetic groups are forced to have frequent interactions with outsiders. This involves \textit{de} \textit{facto} high levels of {bilingualism}. Although it is hard to assess whether the dominant language is the insider code or the outsider code, it makes sense to suspect that balanced {bilingualism} was the norm, as much in the past as in the present.

Van Coetsem (\citeyear{VanCoetsem1988,VanCoetsem2000}) uses the term “{transfer}” generically for any kind of contact-induced phenomenon. If the {transfer} is triggered by speakers who are dominant in the {source language}, he uses the term “{imposition}”. If it originates from {recipient-language} dominance, it is called “borrowing”. Lucas (\citeyear[525]{Lucas2015}) further introduces two categories, the first of which he calls ``restructuring'', defined as a “type of change […] brought about by speakers for whom the changing language is an L2, but it does not involve {transfer}”.  He notes that for individuals who acquired two languages simultaneously (in early childhood), “the distinction between borrowing and {imposition} breaks down”. In this case, both languages typically undergo ``{convergence}'', that is the fourth category of contact-induced change. Because I posit balanced \ili{Arabic}--Domari {bilingualism} as the norm, the question that needs to be answered is whether all the contact-induced changes happening in Domari are the product of {convergence}, or whether there are changes that can be attributed to \ili{Arabic} dominance ({source-language} agentivity or {imposition}). Another problem concerns the sociolinguistic limits of the model. Speakers with two first languages are expected to initiate changes that target both languages. When languages exhibit unbalanced sociolinguistic statuses (minority versus majority), one wonders how changes originating from minority language agentivity can diffuse to the majority. Although it cannot be ruled out, it remains very unlikely. Consequently, {convergence} will always happen in the direction of the minority language. And this is indeed what is happening between \ili{Arabic} and Domari: they become more and more similar at all levels, but only Domari is moving towards \ili{Arabic}.

In the realm of phonology, it was shown that Domari has kept a distinct inventory from \ili{Arabic}, although {convergence} with \ili{Arabic} is almost complete for short vowels. A possible consonantal {imposition} is found in \ili{Beirut/Damascus} Domari where etymological /q/ is realized as /ʔ/, as in neighbouring \ili{Arabic} dialects. As far as morphology is concerned, eligible candidates for {imposition} are the \ili{Kurdish} {diminutive} \textit{{}-ək}, the \ili{Turkish} {conditional} {clitic} \textit{sa} and {superlative} \textit{ān}. An evident case of {imposition} is the phenomenon that seems the most sensitive to dominance: so-called ``{bilingual suppletion}" \citep{Matras2012}. Bilingual suppletion in Northern Domari can be observed only in the dialect of \ili{Beirut/Damascus} in the case of comparatives, and incipiently in the case of {numerals}. As far as syntax is concerned, cases of {imposition} are probably the {transfer} of \ili{Arabic} auxiliaries and the negator \textit{mū}. The {transfer} of utterance modifiers such as fillers, adverbs, conjunctions and virtually all discourse structuring devices is so prone to {replication} in contact situations \citep{Matras1998} that it is difficult to assess the source of agentivity. Other features discussed in this paper, such as {constituent order}, the {internal object} and the {impersonal construction} are clear instances of {convergence}.

As noted above, the main direction of change in Domari is towards {convergence} with \ili{Arabic}, as expected in cases of absence of dominance. The dialect of \ili{Beirut/Damascus} is the most convergent of all the Northern dialects, which in itself suggests that \ili{Arabic}--Domari {bilingualism} is older in that variety. The \ili{Arabic} component in Domari is largely uneven cross-dialectally and no overall statement about its nature can be made. The general picture that arises is that the impact of \ili{Arabic} gradually increases from north to south, with the dialects of northern Syria and southern Turkey being the least Arabized, the \ili{Southern} dialects spoken in Palestine and Jordan being the most influenced by \ili{Arabic}, and the dialect of \ili{Beirut/Damascus} exhibiting an intermediary stage. It was also shown that the main difference between Northern and \ili{Southern} Domari as far as \ili{Arabic} is concerned is the reluctance in Northern Domari to {transfer} \ili{Arabic} inflections and the general tendency to favour the {transfer} of structures without substance.

\section*{Further reading}\largerpage
\begin{furtherreading}
\item For a general account of the \ili{Arabic} component in all the varieties of Domari documented so far, see \citet{Herin2018}. The paper discusses the \ili{Arabic} component in \ili{Southern} and Northern dialects. This is the only paper that tackles extensively the issue of contact-induced change in Domari from a global perspective.
\item For a description of the Domari dialect of Aleppo, readers can refer to \citet{Herin2012}.
\item \citet{Herin2014Domari} identifies the grammatical features that make Northern Domari a coherent dialectal group.
\item \citet{Herin2016} investigates the full extent of variation in Domari as a whole, drawing on data from both Northern and \ili{Southern} Domari.
\item Readers can refer to Matras (this volume) for a number of references relating to {Jerusalem Domari}.\ia{Matras, Yaron@Matras, Yaron}
\end{furtherreading}

\section*{Abbreviations}

\begin{multicols}{2}
\begin{tabbing}
1, 2, 3 \= imperfect (prefix conjugation)\kill
1, 2, 3 \> 1st, 2nd, 3rd person \\
\textsc{abl} \> ablative \\
\textsc{acc} \> accusative \\
\textsc{ad} \> adessive \\
\textsc{ben} \> benefactive \\
\textsc{cmpr} \> {comparative} \\
\textsc{com} \> comitative \\
\textsc{cop} \> {copula} \\
\textsc{def} \> {definite} {article} \\
\textsc{dem} \> demonstrative \\
\textsc{exs} \> {existential} \\
\textsc{f} \> feminine \\
\textsc{impf} \> imperfect (prefix conjugation) \\
IPA \> International Phonetic \\ \> Alphabet\\
\textsc{in} \> inessive \\
\textsc{ind} \> indicative \\
\textsc{indf} \> indefinite \\
\textsc{ipfv} \> imperfective \\
\textsc{m} \> masculine \\
\textsc{neg} \> {negation} \\
NP \> {noun phrase} \\
\textsc{obj} \> object \\
\textsc{obl} \> oblique \\
\textsc{pfv} \> perfective \\
\textsc{pl} \> plural \\
\textsc{prs} \> present \\
\textsc{prf} \> perfect (suffix conjugation) \\
\textsc{prog} \> progressive \\
\textsc{pst} \> past \\
\textsc{rel} \> {relative} \\
\textsc{sbjv} \> subjunctive \\
\textsc{sg} \> singular \\
\textsc{supr} \> superessive \\
\end{tabbing}
\end{multicols}


{\sloppy\printbibliography[heading=subbibliography,notkeyword=this]}
\end{document}
