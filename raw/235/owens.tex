\documentclass[output=paper]{langsci/langscibook}
\ChapterDOI{10.5281/zenodo.3744515}
\author{Jonathan Owens\affiliation{University of Bayreuth}}
\title{Nigerian Arabic}
\abstract{Nigerian Arabic displays an interesting interplay of maintenance of inherited structures along with striking contact-induced innovations in a number of domains. This chapter summarizes the various domains where contact-based change has occurred, concentrating on those less studied not only in Arabic linguistics, but in linguistics in general, namely idiomatic structure and an expanded functionalization of demonstratives. Methodologically, comparative corpora are employed to demonstrate the degree of contact-based influence.}
\maketitle

\begin{document}


\section{
Historical and linguistic background
}

Nigerian \ili{Arabic} (NA) is spoken by perhaps – there are no reliable demographic figures from the last 50 years – 500,000 speakers. These are found mainly in northeast Nigeria in the state of {Borno} where their homeland is concentrated along the Cameroon--Chad border as far south as Banki, spreading westwards towards Gubio, and south of Maiduguri towards Damboa. Mirroring a larger trend in Nigerian demographics, the past 40 years have seen a considerable degree of rural–urban migration. This has seen, above all, the development of large Arab communities in cities in {Borno} – the capital Maiduguri has at least 50,000 alone\footnote{A report in the 1970s by an urban planning company, the Max Lock Group (\citeyear{MaxLockGroup1976}), estimated that 10\% of the then estimated population of 200,000 Maidugurians were Arabs. Today the population of Maiduguri is not less than one million and may be considerably larger, which proportionally would estimate an Arab population in Maiduguri alone of at least 100,000. Of course, if one included the refugee camps today, the number would be much higher.} – though they are now found throughout cities in Nigeria.

Arabs in Nigeria are traditionally cattle nomads, part of what the anthropologist Ulrich Braukämper (\citeyear{Braukämper1994}) has called the “Baggara belt”, named after the Arab tribe in the western Sudan (Kordofan, Darfur; see \citealt{Manfredi2010}) whose culture and dialect are very similar to those of the Nigerian Arabs. Until the very recent Bokko Haram tragedy, besides nomadism, Arabs practiced subsistence farming. As of the writing of this chapter, nearly all rural Nigerian Arabs have been forced to flee their home villages and cattle camps, and are living mainly in refugee camps in northeast Nigeria and neighboring countries.

Arabs first came to the {Lake Chad} area – whether territorial Nigeria is at this point undetermined – in the late fourteenth century. They were part of what initially was a slow migration out of Upper Egypt towards the northern Sudan beginning in the early thirteenth century, which gained momentum after the fall of the northern \ili{Nubian} kingdom of Nobadia (or Maris) in the fourteenth century. All in all, NA exhibits a series of significant isoglosses which link it to Upper Egypt, via \ili{Sudanese} \ili{Arabic}, even if it displays interesting “archaisms” linking it to regions far removed from Africa \citep{Owens2013}. Its immediate congeners are found in what I have termed {Western} Sudanic \ili{Arabic} (\ili{WSA}; \citealt{Owens1994editor,Owens1994author}), stretching between northeast Nigeria in the west and Kordofan in the east \citep{Manfredi2010}. When properties of NA are contrasted with other varieties of \ili{Arabic}, it is implicitly understood that these do not necessarily include other \ili{WSA} varieties. Much more empirical work is necessary in this regard, but, to give one example, many of the extended functions of the NA demonstrative described in §\ref{demons} below are also found in Kordofanian \ili{Arabic} \citep{Manfredi2014demonstratives}. Moreover, where thoughout the Sudanic region as a whole any given isogloss lies is also an open question, as is the issue of the degree to which the contact-induced changes suggested here represent broad areal phenomena. As my own in many cases detailed data derives from NA, I limit most observations to this area. NA itself divides into two dialect areas, a western and an eastern one that I have also termed \ili{Bagirmi} \ili{Arabic}, since it is spoken by Arabs in the \ili{Bagirmi}-speaking region.

In {Borno}, Arabs are probably the largest minority ethnic group, though still a minority. The entire area bordering {Lake Chad}, both to the east and to the west, is dominated by \ili{Kanuri}-speaking peoples (\ili{Kanembu} in Chad). This was a domination which the Arabs already met in their first migrations into the region, both a political and a linguistic domination. As will be seen, this has left dramatic influences in some domains of NA, while leaving others untouched.

While until about 1970 \ili{Kanuri} was the dominant co-territorial language, Arabs in the {Lake Chad} area have been in close contact with other languages and ethnic groups as well, for instance \ili{Fulfulde}, \ili{Kotoko} (just south of {Lake Chad}) and \ili{Bagirmi} (south of Ndjammena in Chad). Furthermore, \ili{Kanuri} established itself in {Borno} in an area already populated by speakers of \ili{Chadic} languages, so it as well was probably influenced by some of the co-territorial languages Arabs met. Since 1970, \ili{Hausa} has become the dominant lingua franca in all urban areas in northeastern Nigeria (indeed throughout the north of the country). In a sample of 58 Maiduguri speakers for instance (\citealt{Owens1998}; \citealt{Owens2000article}: 324), 50 professed knowing \ili{Hausa}, and 46 \ili{Kanuri}. In the only study of its type, \citet{Broß2007} shows that urban Maiduguri Nigerian Arabs have a high degree of accuracy for a number of complex variables in \ili{Hausa}, while, using a similar sample, in one of the few interactional studies available, \citet{Owens2002} also documents a high multi-lingual proficiency between \ili{Arabic} and \ili{Hausa}, and for some speakers, \ili{English}. How such micro-studies can be interpreted against the over 400 years of NA contact with area languages remains a question for the {future}. Rural areas have not yet experienced such a high penetration of \ili{Hausa}. In a second, rural sample consisting of 48 individuals, only sixteen self-reported knowing \ili{Hausa} versus forty \ili{Kanuri}. Note that as of the 1990s, there were still a considerable number of {monolingual} \ili{Arabic} speakers, particularly in the area along the Cameroonian border which among Nigerian Arabs is known as the Kala--Balge region.

While \ili{Standard} \ili{Arabic} (\ili{Classical} \ili{Arabic}) has always been a variety known among a small educated elite in {Borno} (of all ethnic backgrounds), along with \ili{Hausa} it has gained considerable momentum in recent years. Whereas traditionally \ili{Classical} \ili{Arabic}, as a part of Koranic memorization, has always been a part of Arabs’ linguistic repertoire, it is only since about 1990 that the teaching of \ili{Standard} \ili{Arabic} as a school subject has spread oral fluency in this variety.


To this point, conditions have been described which, on paper at least, would favor influence via borrowing under RL-agentivity (in the terminology of \citealt{VanCoetsem1988,VanCoetsem2000}). Nigerian Arabs as a linguistic minority tend to be bilingual, and, it may be assumed, have had a history of {bilingualism} in \ili{Kanuri} and locally other languages going back to their first migrations into the region. Equally, however, Nigerian \ili{Arabic} society has itself integrated other ethnic groups creating conditions of shift to \ili{Arabic}. According to Braukämper’s (\citeyear{Braukämper1994}) thesis, the very basis of Nigerian Arab nomadism is cattle nomadism based on a Fulani model. This is said to have arisen around the mid-seventeenth century as Arabs coming from the east met Fulani moving west. Today there is very little \ili{Fulfulde} spoken in {Borno} or Chad, so it may be surmised that the result of the Fulani--Arab contact was {language shift} in favor of \ili{Arabic}. Furthermore, slavery was a well-established institution which incorporated speakers from other ethnic groups (see recording TV57b-Mule-Hawa in \citealt{OwensHassan2011}, as an instance of a slave descendant). Intermarriage is another mechanism by which L1 speakers would switch to \ili{Arabic}. In contemporary Nigeria, intermarriage in fact tends to favor Arab women marrying outside their group, rather than marriage into Arab society, though there is no cultural proscription of the practice, and such practices tend, inter alia, to be influenced by the {relative} {prestige} and power of the groups involved. Today Arabs are dominated politically by the \ili{Kanuri}, though there are eras, for example the period of Kanemi in the mid-nineteenth century, or the rule of \iai{Rabeh} at the beginning of the twentieth, when Arabs were more dominant and perhaps had greater access to marriage from outside groups. I will return to these summaries in §\ref{conc}.

The data for this chapter comes from long years of working on \ili{Arabic} in the {Lake Chad} region. More concretely, a large oral corpus of about 400,000 words (\citealt{OwensHassan2011}) forms the basis of much of the research, and this corpus will be referred to in a number of places in the chapter. When a form is said to be rare, frequent, etc., these evaluations are made {relative} to what can be found in the corpus. All examples come from this corpus. The source of the recording in the data bank is indicated by the number in brackets at the end of the example.

\section{Contact and historical linguistics
}


Language contact is an integral part of historical linguistics. In the case of \ili{Arabic}, the history of \ili{Arabic} has different interpretations, so it is relevant here to very briefly reiterate my own views \citep{Owens2006}. All varieties of contemporary \ili{Arabic} derive from a reconstructed ancestor or ancestors. Whether singular or plural is a crucial matter, but one answered legitimately only within historical linguistic methodology (see e.g. \citealt{Retsö2013}, who appears to favour the plural). As is usually accepted (perhaps not by some working within {grammaticalization} theory, e.g. \citealt{HeineKuteva2011}), historical linguistics operates at the juncture of inheritance and contact, and examines change due to internal developments and change due to contact. In the case of \ili{Arabic}, contact extends well into the pre-Islamic era (\citealt{Owens2013,Owens2016Aramaic,Owensforthcoming}).

Furthermore, it operates at the level of the {speech community}, and \ili{Arabic} has and had many speech communities, each with its own linguistic history. The history of speech communities is not co-terminous with political history, usually not with the history of individual countries, or even with cultural entities such as a nomadic lifestyle. It follows that \ili{Arabic} linguistic history is quite complicated, its large population being the product of and reflecting many individual social entities.


Any individual contemporary \ili{Arabic} {speech community} therefore lies at the end of many influences. Interpreting whether and when a particular change occurred due to contact is anything but straightforward, as I will discuss very briefly in the following phonological issue.

Ostensibly NA shows the loss of *θ:

\ea\label{t}
*\textup{θ} > t, \textup{*}θawr > \textit{tōr}  \textup{‘bull’}\\
\z
  or in the eastern area:

\ea\label{s}
  *θ > s, \textup{*}θawr > \textit{sōr} \textup{‘bull’}\\
\z


There is no space to go into the detailed historical linguistic arguments here, but it would be incorrect to assert that these changes, quite plausibly originally due to contact, took place in the territorial NA or \ili{WSA} region. This can be seen inter alia in the fact that all of \ili{Egyptian Arabic} (\ili{EA}) and all of the Sudanic region including the \ili{WSA} area has (\ref{t}). Whenever the shift occurred, it was well before Arabs came to the Sudanic region, let alone Nigeria. The changes in (\ref{t}) and, I would argue, (\ref{s}) as well, are part of the historical linguistics of ancestral Sudanic \ili{Arabic}, but the changes themselves are antecedent to \ili{Arabic} in the Sudanic region and therefore are not treated here.

\section{Contact-induced changes
}

\subsection{Phonology}
\label{phon}

Excluding cases like \xxref{t}{s} on methodological grounds, other than marginal effects due to borrowing, discussed briefly in §\ref{loans}, there are no significant instances of contact-induced phonological change limited only to NA. Two changes confined to all or part of the \ili{WSA} region can be suspected, however.

Throughout Nigeria, Cameroon, and most of \ili{Chadian} \ili{Arabic}, *ḥ/ʕ have de-{pharyngealized}.

\ea
*\textup{ḥ/ʕ} > h/ʔ\\
         \textit{ḥilim} ‘dream’ > \textit{hilim}\\
        \textit{gaʕad} ‘stay, sit’ > \textit{gaʔad}
\z

As a set, the change is attested only in this region. Moreover, the area it is attested in begins by and large in the region where \ili{Arabic} fades into minority status.


A second candidate for a local \ili{WSA} innovation is the reflex of *ṭ, which is a voiced, {emphatic} implosive /ɗ/. The implosive /ɗ/ is also found in \ili{Fulfulde}, as well as other possible contact languages such as \ili{Bagirmi}, which, as noted above, are one source of shifters to \ili{Arabic}. Manfredi (\citeyear{Manfredi2010}: 44; and personal communication) notes that /ɗ/ is an allophonic variant in Kordofanian \ili{Baggara} \ili{Arabic}.

The status of one {phoneme}, /č/, is still open. It is fairly frequent (about 100 entries out of about 8,500 (excluding proper names) in a dictionary currently in preparation begin with /č/). In a minority of cases an \ili{Arabic} origin is certain or likely, e.g. \textit{čāl} ‘come’ (eastern variant) < *tāl and perhaps \textit{čatt} ‘all’, < *šattā ‘various’, with [š + t] > /č/ recalling some \ili{Gulf} dialects \textit{ičūf} ‘you see’. /č/ is never a reflex of *k. However, most instances of /č/ are still unaccounted for (e.g \textit{ču} ‘very red’, \textit{čāqab} ‘wade through’).

All in all then, there has not been a great deal of fundamental phonological change due to contact. Note that NA maintains all inherited emphatics, and probably inherited its phonemically contrastive {emphatic} /ṃ/, /{\R}/ and perhaps its /ḷ/ as well.

\subsection{Loanwords}\label{loans}

Despite its long period as a minority language in the {Lake Chad} region, NA has only a modest number of {loanwords} (see \citealt{Owens2000article} for a much more detailed treatment of all aspects of {loanwords} in the classical sense). In a token count based on about 500,000 words, only about 3\% of all words were loans. On a type basis the percentage rises considerably, though still is far from overwhelming. Table \ref{1} presents {loanword} provenance data from the dictionary currently in progress.

\begin{table}
\begin{tabularx}{.5\textwidth}{Xr}
\lsptoprule
Language & Types\\
\midrule
English \il{English} & 509 \\
Hausa  \il{Hausa} & 255 \\
Kanuri \il{Kanuri} & 252\\
Standard Arabic\il{Standard} & 212\\
French \il{French} & 21 \\
Fulfulde\il{Fulfulde} & 12 \\
Kotoko \il{Kotoko} & 2 \\
\lspbottomrule
\end{tabularx}
\caption{Loanwords in NA, types, $N = 1263$}
\label{1}
\end{table}

The figures in Table \ref{1} are probably a slight underestimation, as there are about sixty words, like \textit{bazingir} ‘soldier of \iai{Rabeh}' which clearly are not of \ili{Arabic} origin but whose precise origin has not been found.\largerpage[-2]

There are many interesting issues in understanding the {loanwords}, a few of which I mention very cursorily here. The semantic domains differ from source to source. \ili{Standard} \ili{Arabic}, for instance, has mainly learned words. \ili{Kanuri} covers a fairly wide spectrum, and strikingly includes a large number of discourse markers and conjunctions, on a token basis. \textit{dugó} ‘then, so’ (< \textit{dugó}) for instance has something in the range of 630 occurrences and \textit{yo,} \textit{yō,} \textit{iyō} ‘so, okay, aha’ has 938. In Owens (\citeyear[303]{Owens2000article}), discourse particles and conjunctions are shown to make up no less than 23.3\% of all {loanword} tokens in the sample. It is noticeable that although a few \ili{Hausa} discourse marker tokens (\textit{to} ‘right, okay, so’) do occur, there are hardly twenty in all, this being indicative of the much shorter time span \ili{Hausa} has been in large-scale contact with NA as compared to \ili{Kanuri}.

The question of origin has two aspects, one the ultimate origin, the other how it got into NA. \textit{bel} ‘belt’ is ultimately of \ili{English} origin, but the same word is also found in \ili{Hausa} (\textit{bel}) and in \ili{Kanuri} (\textit{bêl}). Given that both of these languages are dominant ones, it is likely that \textit{bel} entered NA from one of these, not directly from \ili{English}. The statistics above are the ultimate origin. The medial origin (travel words) is much harder to trace. Using the corpus, it is possible to discern likely paths. For instance, NA \textit{sanāʔa} {\textasciitilde}~\textit{saɲa} ‘trade, occupation, profession’ is {cognate} with both \ili{Standard} \ili{Arabic} \textit{ṣināʕa} ‘art, occupation, craft’ and \ili{Hausa} \textit{sanāʔā} `trade, craft, profession'. Considering the distribution of \textit{saɲa} among speakers who have no knowledge of \ili{Standard} \ili{Arabic}, it is likely that the word reached NA via \ili{Hausa}.

Non-\ili{Arabic} phonology will often be maintained in the {loanword}. However, as can be discerned from {loanwords} of higher {frequency}, usually there is variation between retention of the source {phoneme} and adaptation. For instance ‘police’ comes in two forms, \textit{polīs} and \textit{folīs} \citep[278]{Owens2000article}. The [p] variant occurs in 19 tokens distributed among eight speakers, the [f] in 18 tokens among six speakers. Inspection of the statistics shows only a tendential bias towards [f] among women and villagers. Both variants appear therefore to be widespread. Note in this case that variation between [p] and [f] is also endemic to \ili{Kanuri}, so it is likely here that the variation itself was borrowed.

\subsection{Syntax}

There are three strong candidates for contact-induced change in the syntactic domain: {word order}, {ideophones} and an expansion and realignment of {demonstratives}.\largerpage[-2]

\subsubsection{Word order and ideophones}

NA has only two pre-noun modifiers, \textit{gōlit} ‘each’, \textit{kunni} ‘each’.

\ea \gll gōlit ʔīd nulummu\\
     each holiday gather\textsc{.impf.1pl}\\
\glt ‘We would gather at each festival.’
\z

Otherwise NA is head-\textsc{n}-initial, which means that \textit{čatt} ‘all’ and \textit{kam} ‘how many’ are post-\textsc{n}, while {demonstratives} only have a post-\textsc{n} position (as in \ili{EA}).

\ea \gll numšu be ʔaḫuwāt-na čatt-ina\\
     go.\textsc{impf.1pl} with sisters-1\textsc{pl} all-\textsc{1pl}\\
\glt ‘We go with our sisters, all of us.’
\ex \gll taǧīb ḍahaḅ kam\\
     bring.\textsc{impf.2sg.m} gold how.much\\
\glt ‘How much gold do you bring?’
\z

The post-nominal-only demonstrative would have been inherited from \ili{EA}. \textit{čatt} ‘all’ mirrors the post-nominal alternative for \textit{kull}, both taking a pronoun cross-referencing the head noun. Therefore, strictly speaking, the only innovation is the post-nominal position of \textit{kam} ‘how many’, and an argument could be made that internal analogies lead NA towards a more consistent head-first noun-phrase order. By the same token, \ili{Kanuri} is also consistently head-first order in the \textsc{np}, so it could be that contact with \ili{Kanuri} accelerated an inherited trend.

The numeral phrase has undergone considerable re-structuring. From `twenty' upwards, the order is decade--ones.

\ea \gll talātīn haw wāhid\\
     thirty and one\\
\glt ‘thirty one’
\z

Though inherited teens do occur, the usual structure is ten--ones.

\ea \gll ʔasara haw wāhid\\
     ten and one\\
\glt ‘eleven’
\z

This order mirrors that of \ili{Kanuri} \citep[203]{Hutchison1981}, and indeed that of most languages in the immediate {Lake Chad} area. \ili{Uzbekistan} \ili{Arabic} has the same numeral order and structure as NA, and in these cases independent contact events are likely the reason for the shift from an inherited structure.


A new syntactic category (for \ili{Arabic}), that of {ideophones}, is described in detail in \citet{OwensHassan2004} (see \textit{tul} in (\ref{tul}) below). To date in the dictionary of NA in progress there are 342 {ideophones}, about 4\% of the lemma total.

\subsubsection{Demonstratives} \label{demons}

Formally, NA {demonstratives} reproduce their inherited forms, and therefore are virtually identical to paradigms found in various Egyptian dialects, except that, in consonance with NA morphology, feminine plural has a distinct form, which most Egyptian dialects have neutralized (see Table \ref{demtab}).

\begin{table}

\begin{tabular}{lllll}
\lsptoprule
     & \multicolumn{2}{c}{Near}       & \multicolumn{2}{c}{Far} \\\cmidrule(lr){2-3}\cmidrule(lr){4-5}
& Singular &  Plural & Singular & Plural\\
\midrule
Masculine & da & dōl & ɗāk & ɗōlak\\
Feminine & di & dēl & ɗīk & ɗēlak\\
\lspbottomrule
\end{tabular}
\caption{\label{bkm:Ref520909612}NA demonstratives}
\label{demtab}
\end{table}

As with all \ili{Arabic} {demonstratives}, NA {demonstratives} are used both as modifiers and pronominally. The traditional, inherited functions are entity referential (\textit{al-bēt} \textit{da} ‘this house’), and propositional anaphoric (ʔ\textit{ašān} \textit{da} ‘because of this’, where ‘this’ references an introduced proposition).

Additionally, however, the {demonstratives} occur in several contexts which either are not attested at all, or are attested only on an extremely infrequent basis in other \ili{Arabic} dialects. I summarize these here.

\begin{enumerate}[leftmargin=*]

\item Marking the end of dependent clauses, whether {relative}, {conditional} or adverbial.

Usually \textit{da} is the default form in this function, though in the case of {relative} clauses the {demonstratives} often agrees with the head noun.


\ea\label{gul}
 {\label{bkm:Ref520916112}Conditional clause}\\
\gll [kan gul balkallam kalām-hum da] ma bukūn\\
     [if say.\textsc{prf.1sg} speak.\textsc{impf.1sg} language-\textsc{3pl.m} \textsc{dem.sg.m]} \textsc{neg} possible\\
\glt ‘If I said I speak their language, it is not possible.’
\ex \label{luqqa}
{}{}{\label{bkm:Ref520916704}Relative clause}\\
\gll balkallam le-əm be l-luqqa l-biyarifū-ha di\\
     speak.\textsc{impf.1sg} to-\textsc{3pl.m} with \textsc{def}-language \textsc{rel-}know.\textsc{impf.3pl.m-3sg.f} \textsc{dem.sg.f}\\
\glt ‘I speak to them in the language which they know.’
\z

\item Text referential, cataphoric.

\textit{da} is used cataphorically to foreshadow a propositional expansion. In (\ref{farm}) the speaker is asked how he farms. Instead of answering directly, he introduces his answer with the cataphoric use of \textit{da}, which is then expanded upon in the following independent proposition.

\ea\label{farm}
\ea
\gll kēf tihērit\\
     how farm.\textsc{impf.2sg.m}\\
\glt ‘How do you farm?’

\ex
\gll baharit da, al-hirāta l-wād-e tul di d-duḫun\\
     farm.\textsc{impf.1sg} \textsc{dem.sg.m} \textsc{def-}farming \textsc{def-}one-\textsc{f} only \textsc{dem.sg.f} \textsc{def-}millet\\
\glt ‘How I farm? The one type of farming is only millet.’\label{tul}
\z
\z

\item Deictics.

A number of deictic words, mainly adverbs, are marked by {demonstratives}, in this case nearly always \textit{da}. The deictics include \textit{hassa} ‘now’, \textit{dugut} ‘now’, \textit{wakit} {\textasciitilde}~\textit{waqit} ‘now’, \textit{tawwa} ‘previously, formerly’, \textit{hine/hinēn} ‘here’, \textit{awwal} ‘first, before’, \textit{gabul} ‘previously, before, \textit{baʔad} ‘afterwards’, \textit{alōm}/\textit{alyōm} ‘today’, \textit{bukura} ‘day after tomorrow’, \textit{amis} ‘yesterday’, \textit{albāre} ‘yesterday evening’, \textit{ambākir} ‘tomorrow’, \textit{m\kern 0.75ptǝ\kern -0.5ptṇṇaṣabá} ‘in the morning’, \textit{qādi} ‘there’, \textit{hināk} ‘there’, \textit{haǧira} ‘(a place) away from here’, \textit{bilhēn} ‘much’.

\ea\label{hajira}
\gll haǧira da ma mašēt\\
     away \textsc{dem.sg.m} \textsc{neg} go.\textsc{prf.1sg}\\
\glt ‘I didn’t go away anywhere.’
\ex\label{albare}
\gll albāre da as-sarārīk daḫalo\\
     yesterday \textsc{dem.sg.m} \textsc{def-}thieves enter.\textsc{prf.3pl.m}\\
\glt ‘Yesterday evening thieves broke in.’
\z

\item Demonstratives mark pronouns, in this case often agreeing with the pronoun in terms of number and {gender}, and other {demonstratives}, where usually \textit{da} occurs.

\ea\label{suqul}
\ea
\gll inti di ǧībi le-i š-suqúl da\\
     \textsc{2sg.f} \textsc{dem.sg.f} bring.\textsc{imp.sg.f} to-\textsc{obl.1sg} \textsc{def-}thing \textsc{dem.sg.m}\\
\glt ‘You there bring me the watchamacallit.’
\ex \label{ard}
\gll ʔard gaydam dōla da kula ʔarab\\
     land Geidam \textsc{dem.pl.m} \textsc{dem.sg.m} also Arab\\
\glt ‘In the land around Geidam and the like are also Arabs.’
\z
\z

\end{enumerate}

\noindent Basic attributes of these expanded functions can be given in cursory manner.

Concerning {frequency}, the occurrence of {demonstratives} in these functions on a token basis is high. For instance, there are 887 tokens of \textit{qādi} ‘there’ in the corpus, of which 108 or 12\% are marked by \textit{da}. The highest percentages of {demonstratives} in these functions occur with dependent clauses and the 3\textsc{sg} pronouns \textit{hu} ‘he’ and \textit{hi} ‘she’. For \textit{hu}, nearly 25\% of all tokens occur with \textit{da} (586/2407 24.3\%). As far as the four innovative functions summarized above are concerned, a sample of 1318 tokens of \textit{da} gathered from an arbitrary selection of 45 texts in the corpus reveals the data presented in Table \ref{tab:funct}. While the inherited referential functions constitute the largest single class, they make up only 53\% of the total. The remaining 47\% are functionally innovative.


\begin{table}\begin{tabularx}{.67\textwidth}{lr}
\lsptoprule
Function & Percentage of total\\
\midrule
Inherited functions &  53.4\%\\
        \hspace{5mm}Entity referential  & 42.3\%\\
        \hspace{5mm}Proposition-anaphoric  & 11.1\%\\
\addlinespace
Innovative functions & 46.7\%\\
        \hspace{5mm}Cataphoric-propositional & 7.2\%\\
        \hspace{5mm}Dependent clause & 18.7\%\\
        \hspace{5mm}Adverbs/deictic & 12\%\\
        \hspace{5mm}Pronouns, {demonstratives} & 8.8\%\\
\lspbottomrule
\end{tabularx}
\caption{
\label{bkm:Ref520915528}Functions of \textit{da} in NA
}
\label{tab:funct}
\end{table}


The syntactic, pragmatic and semantic nuances of using or not using the dem\-onstratives in these innovative contexts have yet to be worked out. The two examples in (\ref{qayyar}) and (\ref{nas}) illustrate different ways the innovative functions are integrated with other elements of the grammar.

Syntactically, for instance, based on the sample described above, \textit{da} marks the end of about 30\% of all {conditional} clauses. When it does not occur, its final clause boundary marking position commutes with an alternative pragmatically-marked element, such as the discourse marker \textit{kula} ‘even’. (No tokens of *\textit{kula da} closing a {conditional} clause occur in the corpus).

\ea\label{qayyar}
\gll kan qayyart-a kula\\
     if change.\textsc{prf.2sg.m-3sg.m} \textsc{dm}\\
\glt ‘Even if you changed it.’
\z

Pragmatically there are many instances where \textit{da} has a {focusing} function, as in the following, where a mixed linguistic region ‘here’ is contrasted with another ‘there’, which is linguistically homogeneous.

\ea\label{nas}
\gll nās gadé gadé kula hinēn katīrīn fi [qādi \textbf{da}] nafar-na nafara wāhid\\
     people different different \textsc{dm} here many \textsc{exs} there \textsc{dem.sg.m} type-\textsc{1pl} type one\\
\glt ‘Here there are a lot of different (types) but [over there] there is just our one ethnic group.’
\z

The functions outlined in Table \ref{tab:funct} are therefore both of high {frequency} and are systematically embedded in the syntax and pragmatics.

It should be intuitively clear that the functions in examples \xxref{gul}{nas} are innovative in their systematicity {relative} to other varieties of \ili{Arabic}. To show this in detail it would, however, be necessary to look at large-scale corpora of other \ili{Arabic} dialects. This can very briefly be done with \ili{EA}, which, as noted above, is an ancestral homeland of NA. The \ili{EA} corpus is from \textit{LDC Callhome} \citep{Canavanetal1997}, \citet{Nakano1982}, \citet{BehnstedtWoidich1987}, and \citet{WoidichDrop2007}, comprising about 417,000 words. It is thus of comparable size to the NA corpus. In this corpus there do occasionally occur collocations of pronoun + demonstrative in the same contexts as illustrated in (\ref{suqul}), in particular as in (\ref{hiwwa}).


\ea\label{hiwwa}
\gll hiwwa da lli mawgūd ʕandi-na\\
     \textsc{3sg.m} \textsc{dem.sg.m} \textsc{rel} present at-1\textsc{pl}\\
\glt ‘That is what we have.’
\z

It clearly, however, has a different functionality from NA pronoun + demonstrative. In \ili{EA} the construction consistently is anaphoric to a previous proposition or situation, as in (\ref{hiwwa}), where it introduces a previously-established topic to a following descriptive qualification. In 11 of the 58 tokens in the \ili{EA} corpus it is followed by a {relative} clause, as in (\ref{hiwwa}). Most tellingly, there are 2,677 \textit{huwwa} ({\textasciitilde}~\textit{hu,} \textit{hū,} \textit{hiwwa,} \textit{hūwa}) tokens, of which only 58, or 2\% are followed by \textit{da} ({\textasciitilde}~\textit{dah,} \textit{dih,} \textit{deh,} \textit{dī}). This compares to the nearly 25\% \textit{hu} \textit{+} \textit{da} tokens in NA noted above. Moreover, in the NA sample, no tokens of \textit{hu} \textit{da} are followed by a {relative} clause. In this same statistical vein, the total number of singular proximal {demonstratives} in NA amounts to 16,774 tokens (14,591 \textit{da}, 2,183 \textit{di}). In the \ili{EA} corpus there are only 8,239 (4,996 \textit{da}, 3,243 \textit{di}). Given that the corpora sizes are comparable (\ili{EA} in fact a little larger), the {demonstratives} in NA are vastly over-proportional. This preponderance is due to \textit{da}. Clearly there is a case to be answered: what accounts for the vastly higher {frequency} of the 3\textsc{sg.m} demonstrative in NA? Recall, in answering this question, that behind the simple statistical comparison is a fundamental historical one as well. Ancestral NA came from ancestral \ili{EA}. The initial populations, it needs to be assumed, had a demonstrative system like that of \ili{EA}, and the majority of NA demonstrative tokens (see Table \ref{tab:funct}) still reflect this system. A blunt historical linguistic question is what caused the vast shift in frequencies.

From these initial, basic observations, it does not appear that the greatly expanded functionality of the demonstrative in NA can be explained by an increasing {grammaticalization} of the demonstrative.\footnote{I do not at all agree with \citet{HeineKuteva2011} and Leddy-Cecere (this volume)\ia{Leddy-Cecere, Thomas@Leddy-Cecere, Thomas} that changes due to contact can be assimilated to a type of {grammaticalization} process, so the following contact-based account is independent of {grammaticalization}. Grammaticalization, in Meillet's original sense, pertained only to internally-motivated changes.} This follows from two observations. First, the expanded functions of the demonstrative in Table \ref{tab:funct} are, with the exception of the boundary-marking of dependent clauses \REF{luqqa}, not those associated with the {grammaticalization} of {demonstratives} (e.g. the 17 trajectories of {demonstratives} in \citealt{Diessel1999}). Secondly, NA and \ili{EA} split over 400 years ago. One of the branches, represented by NA, underwent the considerable changes outlined here, whereas the other branch, \ili{EA}, probably did not change at all (i.e. sentences such as (\ref{hiwwa}) were probably present in \ili{EA} in 1200, and before).\footnote{Cf. \ili{Damascus}, which has an identical construction to that of \ili{EA}. There are parallels also in \ili{Classical} \ili{Arabic}, so this type of construction is probably proto-\ili{Arabic}. If so, it only heightens the degree to which NA has innovated away from an original, stable structure.} There is thus no natural or inherent tendency for {demonstratives} to expand as in Table \ref{tab:funct}. It can thus be safely assumed that the expanded functionality of the NA demonstrative was due to contact.

In fact, there is a good deal of prima facie evidence supporting this supposition. However, as is so frequently the case when one suspects pattern (metatypical)-type contact influence which is probably centuries old, support for the position will be indexical. Moreover, in the current case one is most probably dealing with a large-scale areal phenomenon in the {Lake Chad} area (and perhaps beyond) which encompasses well over a hundred languages. In this summary chapter it will therefore have to suffice to rather peremptorily indicate that throughout the region there is a referential marker, sometimes a demonstrative, sometimes an article-like element, sometimes an element with both demonstrative and article-like properties, which consistently has the distribution of \xxref{gul}{nas}. Some languages have a better fit than others, and, of course, they will differ in detail in their language-internal functionality. A basic pattern is illustrated in (\ref{kanuri}) with \ili{Kanuri} (\citealt{Hutchison1981}: 47, 207, 218, 234, 241, 270), and summary references are made for \ili{Bagirmi}, \ili{Wandala} and \ili{Fali}. So far as is known, \ili{Fali} and \ili{Wandala} had no significant contact with NA or its \ili{WSA} relatives.

The \ili{Kanuri} determinative -\textit{də} has the following functions.


\ea\label{kanuri}
{}{}{\ili{Kanuri} (\ili{Nilo-Saharan}/Saharan)}\\

{}{}{Anaphoric entity reference}\\
  \ea
{}{}{obligatorily ends RC and optionally many adverbial clauses}; = (\ref{gul}), (\ref{luqqa})\\
\ex
{}{}{pronoun focus}; = (\ref{suqul})\\
\ex
{}{}{marks adverbs}; = (\ref{hajira}), (\ref{albare})\\
\z
\z

The only \ili{Kanuri} structure missing from the list appears to be the propositional cataphoricity illustrated in (\ref{farm}).

\ili{Wandala} (\citealt{Frajzyngier2012}: 507–34, 603) has two morphemes: -\textit{na} which is broadly glossed as a determiner and -\textit{w} ‘that’. -\textit{na}, besides marking entity reference, obligatorily marks the ends of a {relative} clause, and optionally a {conditional} (=\ref{gul}, \ref{luqqa}); it occurs as an obligatory element in certain time/place adverbs (=\ref{hajira}, \ref{albare}); it is part of the previous mention marker \textit{ŋán-na;} \textit{ŋán} itself is said to originally be a third person singular pronoun, so there is a structural parallel to \textit{hu} \textit{+} \textit{da}. -\textit{w} functions as a topic marker that marks pronouns (=\ref{suqul}).

In \ili{Fali} (Adamawa; Niger-Congo) the {demonstratives} \textit{gi/go} also obligatorily mark the end of {relative} and {conditional} clauses (=\ref{gul}, \ref{luqqa}), subject focus (=\ref{suqul}), and occur with some adverbs (=\ref{hajira}, \ref{albare}).

In \ili{Bagirmi} a “determiner particle” -\textit{na} is a constitutive part of the demonstrative \textit{enna} < \textit{et-na} ‘this’, and -\textit{na} alone obligatorily marks the end of {relative} clauses, and can emphasize pronouns, adverbs and entire sentences (\citealt{Stevenson1969}: 40, 51, 54).


Areal features typically are not sensitive to language family, and this appears to be the case in this brief exemplification. \ili{Kanuri} and \ili{Bagirmi} are \ili{Nilo-Saharan}, \ili{Wandala} is \ili{Chadic}, \ili{Fali} is Niger-Congo, and \ili{Arabic} is \ili{Semitic}. Only \ili{Wandala} and \ili{Arabic} are very distantly related genetically. Nonetheless, in all of the languages there is a deictic–referential marker (demonstrative, determiner, demonstrative–determiner) which, besides a classic deictic or anaphoric function, surfaces in an extended range of identical (cf. marking boundary of dependent clause) or similar (pronouns, adverbs)\footnote{The comparativist is limited to the extant reference grammars. These are in many instances excellent. Still, I suspect that they understate the flexibility of distribution of elements such as the deictic marker discussed here. \textit{Mea culpa}, in Owens (\citeyear{Owens1993}: 88, 221, 235) the extended functions of the demonstrative described in this chapter for NA were treated in disparate sections, with no overall focus.} functions. These extended functions are precisely those which distinguish NA from other varieties of \ili{Arabic}. The case for contact follows from two directions: in certain (not all) respects, NA deviates markedly from a putative ancestral source shared with \ili{EA}, and where it does, its deviation corresponds broadly to analogous categories in co-territorial languages.

\subsection{Semantics}

The innovative distribution of the NA {demonstratives} is striking for the degree to which it appears to have raised the overall demonstrative token count, {relative} to \ili{EA}. Discerning its presence in a text, however, is a straightforward matter. A much subtler, but no less pervasive instance of contact-based change pertains to {idiomaticity}. Like the demonstrative, this has a semantic and a formal aspect. Semantically, meanings emerge which are, for \ili{Arabic}, unique, as in the following.

\ea\label{bet}
\ea
\gll rās al-bēt\\
     head \textsc{def-}house\\
\glt ‘roof’\label{roof}

\ex
\gll nādim rās-a\\
     person head-\textsc{3sg.m}\\
\glt ‘an independent person, person of his own means’\label{nadim}
\z
\ex\label{galb}
\ea
\gll tallafo gaḷb-i\\
     spoil.\textsc{prf.3pl.m} heart-1\textsc{sg}\\
\glt ‘They angered me.’

\ex
\gll gaḷb-a helu\\
     heart-\textsc{3sg.m} sweet\\
\glt ‘He is happy.’\label{helu}
\z
\z

Formally the {idioms} are distinctive (as \ili{Arabic} collocations) in bringing together lexemes which in other dialects would hardly co-occur, like [tallaf + gaḷb] or [gaḷb + helu]. The {idiomatic} meanings of the keywords (e.g. \textit{tallaf,} \textit{gaḷb}) are, in usage terms, often the typical usage for a given lexeme. In the NA corpus, for instance, of 101 tokens of \textit{gaḷb} ‘heart’ all of them, 100\%, are {idiomatic}. There is no reference to a physical heart. Similarly, \textit{rās} is 80\% {idiomatic} (247/308 tokens; \citealt{Ritt-Benmimounetc2017}: 53). Thus, while {idiomaticity} has been consistently ignored as a theoretical issue in historical linguistics in general and in \ili{Arabic} in particular, on a usage basis it is an integral aspect of understanding the lexical texture of the language.


Here as well NA is strikingly different from \ili{EA}, as again can be determined from corpora-based comparison. In general, though both NA and \ili{EA} share {idiomatic} keywords (\textit{gaḷb/ʔalb} and \textit{rās} are frequent in both, for instance), their meanings and their collocational environments hardly overlap. For instance, in the \ili{EA} corpus there are 110 tokens of \textit{gaḷb/ʔalb} ‘heart’, of which 102 or 93\% are {idiomatic}. This percentage closely parallels that of NA {idiomatic} \textit{gaḷb}. The typical \ili{EA} {collocate} of {idiomatic} \textit{ʔalb}, however, is very different. The most frequent meaning is ‘center of X’, \textit{ʔalb} \textit{il-baḥr} ‘middle of the sea’. This meaning is entirely lacking in NA, and consequently {collocates} like !\textit{gaḷb} \textit{al-bahar} (! = collocationally\slash semantically odd) are also lacking.

How different NA {idiomaticity} (meaning and collocational environment) is from \ili{EA} was shown recently in \citet{Ritt-Benmimounetc2017}. There a three-way comparison was conducted between \ili{EA}, southern \ili{Tunisian} \ili{Arabic} and NA, looking at three {idiomatic} keywords frequent in all three dialects: \textit{{\R}ās,} \textit{gaḷb} ‘heart’, and \textit{ʕēn} ‘eye’. \ili{EA} and southern \ili{Tunisian}, though separated by a longer period of time (ca. 1035–present) than \ili{EA}–NA (ca. 1300–present), showed a much higher {identity} of {idiomatic} structure than \ili{EA}–NA (or NA–southern \ili{Tunisian}). Both \ili{EA} (\ref{owens:egy}) and \ili{Tunisian} \ili{Arabic} (\ref{tun}), for instance, maintain the same lexemes, same structure, same {idiomaticity} in a highly specific meaning.

\ea
\ea \ili{Egyptian Arabic} \\
\gll ḥaṭṭ {\R}ās-u fi t-turāb\\
     put.\textsc{prf.3sg.m} head-\textsc{3sg.m} in \textsc{def}{}-ground\\
\glt ‘He humiliated him.’ \label{owens:egy}

\ex \ili{Tunisian} \ili{Arabic} \\
\gll ḥaṭṭ-l-a {\R}ās-a fi t-t{\R}āb\\
     put.\textsc{prf.3sg.m}-to-\textsc{3sg.m} head-\textsc{3sg.m} in \textsc{def}{}-ground\\
\glt ‘He humiliated him.’ \label{tun}
\z
\z

These are nonsensical, or literal collocations in NA.

The comparison between \ili{EA} and southern \ili{Tunisian} \ili{Arabic} serves as a similar baseline to comparing the overall demonstrative frequencies between \ili{EA} and NA. The same question occurs. Why is NA different?


In this case the answer is even clearer than with the demonstrative. Essentially, NA has calqued its {idiomatic} structure (meaning and collocation) from \ili{Kanuri}. The \ili{Kanuri} of (\ref{roof}) and (\ref{helu}), for instance, are as in (\ref{kan2}).

\ea\label{kan2}
\ea
\gll kəla fato-be\\
     head house-\textsc{gen}\\
\glt ‘roof’\label{fato}

\ex
\gll kam kəla-nzə-ye\\
     person head-\textsc{3sg.m}{}-\textsc{gen}\\
\glt ‘an independent person, person of his own means’\label{kam}
\z
\z

A ‘roof’ in both languages is the ‘head of a house’, an independent person is a ‘person of his head’, and so on, for something in the range of 70–80\% of all the approximately 340 {idioms} studied (see \citealt{Owens1996,Owens2014,Owens2015,Owens2016idioms} for details).

In summary, a large part of NA lexical structure is, as it were, not \ili{Arabic}, but rather, as termed in \citet{Owens1998}, part of the {Lake Chad} {idiomatic} area. This {identity}, however, exists only at a semantic and collocational level. In their basic meaning, and their phonology, morphology and syntax, even in the context of {idioms} (\citealt{OwensDodsworth2017}), the constituent lexemes \textit{rās}, \textit{bēt}, \textit{tallaf}, \textit{gaḷb} etc. in NA are indistinguishable from any variety of \ili{Arabic} at all.


There doubtless remains a good deal more systematic, contact-based correspondence between NA and languages of the {Lake Chad} area to be explored. The influence on NA is significant.

\section{Conclusion
} \label{conc}

According to the historico-demographic background to NA, this variety did and does live with co-territorial languages, particularly \ili{Kanuri}, today increasingly with \ili{Hausa}, and in the past, \ili{Fulfulde} and other smaller languages. NA {bilingualism} should, presumably, manifest itself in borrowing. Equally, NA speech communities have incorporated speakers of other languages into its fabric. The expectation here is that NA would be influenced via shift ({imposition}) from other languages.

In the domains summarized here, it is hard to discern a clear correlation between linguistic outcome and type of contact. There has been some phonological change, which in Van Coetsem’s (\citeyear{VanCoetsem1988,VanCoetsem2000}) model is suggestive of change via shift ({imposition}), but the influence is limited to the features discussed in §\ref{phon}. What I believe is more striking than the contact-induced phonological change is the maintenance of inherited structures. NA still maintains a robust series of emphatics, has a non-reductive syllable structure reminiscent of, inter alia, Tihāma varieties, has classic distinguishing syllable structure attributes such as the \textit{gahawa} syndrome (\textit{ah}\textbf{\textit{a}}\textit{mar} ‘red’) and the \textit{bukura} syndrome (\textit{bi-ǧ}\textbf{\textit{i}}\textit{ri} ‘he runs’), to mention but a few. If the changes in \xxref{gul}{nas} are due to {imposition}, it is equally clear that the “imposers” otherwise learned/learn a very normal \ili{Arabic}.


Classic borrowing is moderate. The fact that discourse markers and conjunctions are token-wise frequent suggests that speakers were/are conversant in both \ili{Kanuri} and \ili{Arabic}. This does not, however, indicate whether these loans arose through imposers or borrowers. Moreover, to complicate matters even more, assuming \ili{Kanuri} to have been the widespread lingua franca in the past, it would not need to have been native \ili{Kanuri} speakers who imposed the \ili{Kanuri} into \ili{Arabic}. Speakers of \ili{Fulfulde}, \ili{Kotoko}, \ili{Malgwa} or other languages would have been involved as well. As shown in \citet{OwensHassan2010}, discourse markers are prevalent in {code-switching}, which here would be conducted by Arabs {code-switching} between \ili{Arabic} and \ili{Kanuri}. From this scenario the discourse markers entered as borrowed elements.

The interpretation of {demonstratives} and {idiomatic} structure is equally ambiguous. The easiest development to envisage is L2 \ili{Arabic} speakers imposing their L1 \ili{Kanuri}, \ili{Fulfulde} etc. usage onto their L2 \ili{Arabic}. What makes this interpretation attractive is that it explains why in both cases such a massive importation of non-\ili{Arabic} structure came into \ili{Arabic}. As the name implies, these speakers could simply have imposed their own semantics and collocational alignment onto \ili{Arabic}. Equally, however, it is not impossible that L1 \ili{Arabic} speakers, fully bilingual in \ili{Kanuri} and/or other languages simply shifted their \ili{Arabic} usage to accommodate to their L2. Full fluency implies knowing {idiomatic} structure and the use of {demonstratives}, which the Arab borrowers could eventually incorporate into their own \ili{Arabic}.

The only obvious common denominator to these musings is that the speakers would have been highly fluent in their respective L2s, whether L2 \ili{Arabic} speakers shifting to \ili{Arabic} or L1 \ili{Arabic} speakers fluent in \ili{Kanuri} or other languages borrowing from their L2. The issue is only partly who the L1 and L2 speakers are. It is equally how well the populations knew/know \ili{Arabic}/other languages, and how the high level of fluency produces the results shown.\pagebreak

Adding to the interpretive problem is that neither of the domains, {idiomaticity} or the expansion of {demonstratives} as it occurred in NA, have a {comparative} basis. Idiomaticity in the recent western linguistic tradition has been all but entirely subordinated to metaphor theory (\citealt{LakoffJohnson1999}; see \citealt{Haser2005} for one critical perspective). It has received very little principled historical interpretation, and what work has been done (e.g. \citealt{Sweetser1990}) tends to follow a Lakoffian paradigm and to be confined to European languages and to societies quite different from that of Nigerian Arabs. As far as {demonstratives} go, the little work that has been done on the languages co-territorial with NA (e.g. \citealt{Kramer2014}: 141 on \ili{Fali}), assume a {grammaticalization} of demonstrative usage \textit{ab} \textit{novo} via {grammaticalization} processes. Assuming such a perspective for the development of NA gives the lie to this simple assumption for the following reason. It would need to explain why the {grammaticalization} process did not take place in \ili{EA} or other \ili{Arabic} varieties, but did in NA, which is spoken in an area where the co-territorial languages, historically antecedent to \ili{Arabic}, have the structures which NA acquired. If change via contact is the only plausible explanation for NA, it equally needs to be entertained for any language in the {Lake Chad} region.


Given so many open variables, it might be interesting to approach the issue from the opposite perspective, namely, what parts of language were not influenced by contact. Most of phonology was not, morphology hardly at all, syntax to a degree, basic vocabulary little.\footnote{A Swadesh 100-word list gives something in the range of 79–83\% cognacy with   other varieties of \ili{Arabic}.} This minimally implies that if the contact changes were due to shift, the shifters in other domains (those where they did not impose {idiomaticity} or demonstrative usage) acquired a native-like competence in \ili{Arabic}. In this respect it might be easier to envisage L1 \ili{Arabic} borrowers maintaining these structures, and borrowing {idiomaticity}/demonstrative usage via their L2.

At the end of the day I think the range of {questions} evoked far surpasses the ability of currently-formulated linguistic theories of contact or {language change}, whether based on sociolinguistic or on cognitive perspectives \citep[523]{Lucas2015} to provide profound insight into how the obvious, and in some cases pervasive influence on NA via contact came about. It would be more fruitful to turn the question around and ask how rich databases such as exist for NA, \ili{EA} and some other \ili{Arabic} dialects inform the overall issue of change via contact.


\pagebreak\section*{Abbreviations}\largerpage[2]
{\multicolsep=0pt\begin{multicols}{2}
\begin{tabbing}
\textsc{ipfv} \hspace{1em} \= before common era\kill
\textsc{1, 2, 3} \> 1st, 2nd, 3rd person \\
\textsc{def} \> {definite} {article} \\
\textsc{dem} \> demonstrative \\
\textsc{dm} \> discourse marker \\
{EA} \> {Egyptian} {Arabic} \\
\textsc{exs} \> {existential}  \\
\textsc{f} \> feminine \\
\textsc{gen} \> genitive \\
\textsc{impf} \> imperfect (prefix conjugation) \\
\textsc{m} \> masculine \\
N \> number \\
\textsc{n} \> noun \\
{NA} \> {{Nigerian} Arabic} \\
\textsc{neg} \> negative \\
\textsc{np} \> {noun phrase} \\
\textsc{obl} \> oblique \\
\textsc{pl} \> plural \\
\textsc{prf} \> perfect (suffix conjugation) \\
\textsc{rel} \> {relative} \\
\textsc{sg} \> singular \\
{WSA} \> Western {Sudanic} {Arabic}
\end{tabbing}
\end{multicols}}


{\sloppy\printbibliography[heading=subbibliography,notkeyword=this]}
\end{document}
