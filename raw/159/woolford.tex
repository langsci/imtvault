\documentclass[output=paper]{LSP/langsci} 
\author{Ellen Woolford 	\affiliation{University of Massachussetts}
}
\title{Mainland Scandinavian object shift and the puzzling ergative pattern in Aleut} 
% \epigram{Change epigram}
\abstract{Eskimo-Aleut languages turn out to have the same two types of object shift that \citet{holmberg1986word} describes for Scandinavian. Specific objects move out of the VP in Inuit \citep{bittner1996ergativity} and I argue that object shift also occurs in Aleut, but it is limited to pronouns as in Mainland Scandinavian. Aleut differs from Mainland Scandinavian in that, for independent reasons, only third pronouns successfully undergo object shift. Shifting first and second person pronouns is blocked by PCC-like constraints on the portmanteau agreement that occurs in object shift constructions. Shifting reflexives is also blocked, because it would incur a violation of the Anaphor Agreement Effect. 
The surface pattern in Aleut has been described as one where ergative case marks the subject only when another argument in the clause is null. I argue that there is no direct cause and effect relationship between these. The key is the fact that pronouns that agree are not spelled out. Agreement correlates with ergative case because, as in Inuit, ergative case marks the subject in Aleut only when the object moves out of the VP, and in this situation, again as in Inuit, there is portmanteau agreement with the ergative subject and nominative object in object shift constructions. Like Inuit, Aleut has possessor raising/stranding so that the possessor of an object can undergo object shift, trigger agreement, and thus pro drop. From the English translations of Aleut sentences, it initially appears that null objects of prepositions also correlate with ergative subjects, but Aleut, like Inuit, has possessed relational nouns which function like prepositions if they take locative case. These also allow possessor raising and object shift, with the same consequences described above. 
}

\ChapterDOI{10.5281/zenodo.1117744}
\maketitle

\begin{document}
\section{Introduction}

\citet{holmberg1986word} establishes that \ili{Scandinavian} languages divide into two types with respect to the kind of objects that undergo object shift (movement of an object out of the VP to a position below the subject): In \ili{Mainland Scandinavian} languages such as \ili{Swedish}, only pronouns undergo object shift, as in \REF{ex:woolford:1}, while \ili{Icelandic} allows all specific objects to undergo object shift, as in \REF{ex:woolford:2}\footnote{The verb moves out of the VP to ‘second position’. Objects that precede the negative have undergone object shift.
\ea
\ea Structure without object shift:  [subject V\textsubscript{i} neg [\textsubscript{VP} t\textsubscript{i} object]]
\ex Structure with object shift:  [subject V\textsubscript{i} object\textsubscript{j} neg [\textsubscript{VP} t\textsubscript{i} tj]]
\z
\z}:

\ea\label{ex:woolford:1}
\langinfo{Swedish}{}{\citealt[242]{holmberg1986word}}\\
\gll Johan  köpte    den  inte. \\
     John  bought   it  not     \\
\glt ‘John did not buy it.’
\z


\ea\label{ex:woolford:2}
\langinfo{Swedish}{}{\citealt[242]{holmberg1986word}}\\
\gll *Johan  köpte    boken    inte.\\
       John  bought  book.the  not  \\
\glt  ‘John did not buy the book.’
\z

\ea\label{ex:woolford:3}
\langinfo{Icelandic}{}{\citealt[2]{holmberg1986word}}\\
\gll Jón  keypti    bókina   ekki.\\
     John  bought  book.the  not  \\
\glt ‘John did not buy the book.’
\z

The goal of this paper is to show that the Eskimo-Aleut languages divide into the same two types. Inuit parallels \ili{Icelandic} in that specific objects undergo object shift \citep{bittner1996ergativity}.\footnote{Although object shift is typically associated with \ili{Scandinavian} languages, it is known to occur in other languages, such as Turkish \citep{diesing1996semantic} and Hindi \citep{bhatt1996object}. See \citet{thrainsson2001object} for an overview of object shift. \citet{bittner1996ergativity} show that specific objects move out of the VP in Inuit/West Greenlandic; although they do not label this movement as object shift, it nevertheless fits the definition of object shift as movement of an object out of the VP to a position below the subject.} In this paper I argue that, despite its obfuscating surface complexity, Aleut is like the \ili{Mainland Scandinavian} languages in that only pronouns undergo object shift.\is{agreement|(}

There are some additional consequences of object shift in the Eskimo-Aleut languages which are not present in the \ili{Scandinavian} languages. In clauses with object shift in the Eskimo-Aleut languages, the case pattern is \isi{ergative}-nominative, and the agreement is portmanteau, reflecting features of both the \isi{ergative} subject and the nominative object.\footnote{Exactly why object shift alters the case pattern in Eskimo-Aleut languages is controversial. \citet{bittner1996ergativity}, working within dependency case theory, claim that shifting the object brings it close enough to the subject to allow it to serve as a case competitor. \citet{woolford2015ergativity} suggests that object shift creates a defective intervention effect that can be avoided by a ‘last resort’ use of \isi{ergative} case on the subject. } The two types of object shift produce two slightly different surface case and agreement patterns in Inuit versus Aleut. In Inuit, the subject is \isi{ergative} and the agreement is portmanteau when the object is specific, because specific objects undergo object shift in Inuit. In contrast, in Aleut, the subject is \isi{ergative} and the agreement is portmanteau only if the object is a pronoun, because only pronouns undergo object shift in Aleut. 

Additional complications in Aleut (to be discussed below) produce a surface \isi{ergative} pattern which initially seems entirely unlike that of other \isi{ergative} languages, even that of the related language Inuit. The Aleut pattern has been described as marking the subject with \isi{ergative} case if and only if there is a null argument elsewhere in the clause \citep{bergsland1997aleut,boyle2000aleut,sadock2000aleut,Sadock2009}. This surprising correlation between \isi{ergative} case and null arguments has been referred to as the Aleut Effect:


\ea\label{ex:woolford:4}
\textit{Description of The Aleut \isi{Ergative}\is{ergative} Pattern (the Aleut Effect)}:\\
In Aleut, the subject gets \isi{ergative} case if and only if there is a null NP elsewhere in the clause.
\z

\noindent This description of the Aleut pattern contrasts sharply with the \isi{ergative} pattern in Inuit, following \citet{bittner1996ergativity}:

\ea\label{ex:woolford:5}
\textit{The Inuit \isi{Ergative}\is{ergative} Pattern}:\\

In Inuit Greenlandic, the subject is marked with \isi{ergative} case if and only if the object moves out of the VP, and only specific objects move out of the VP.
\z

Despite this apparent dissimilarity, \citet{hale1997misumalpan} nevertheless suggests that the analysis of Inuit in \citet{bittner1996ergativity} can be extended to Aleut.\footnote{\citet[6]{fortescue1985anaphoric} also recognizes the parallel between Aleut and other Eskimo languages with respect to constructions with and without an \isi{ergative} subject and agreement with both the subject and the object.} Hale sketches an analysis wherein only null pronouns move out of the VP in Aleut.\footnote{\citet{hale1997misumalpan} proposes that only null objects move out of the VP in Aleut, and assuming that case drives movement, following \citet{chomsky1995minimalist}, he postulates that null pronouns in Aleut lack case and must therefore move to get case. \citet{boyle2000aleut} shares this assumption that only null pronouns that move out of the VP, although his motivation for the movement is different: a need for null pronouns to be licensed by moving to Spec \isi{TP}. \citet{merchant2011aleut} suggests the movement of null pronouns in Aleut is like \isi{clitic} movement in other languages.} \citet{sadock2000aleut} and \citet{boyle2000aleut} point out that Hale’s account of Aleut suffers from various technical problems, and does not account for all of the Aleut data. While Hale’s proposal to extend Bittner and Hale’s account of Inuit to Aleut is incomplete, and handicapped by some now outmoded assumptions of the framework of \citet{chomsky1995minimalist}, I argue that his basic idea is correct: object shift is the key to understanding the case patterns of both Inuit and Aleut. I show that the differences between these two languages stem from the difference in what type of objects undergo object shift; specific objects object shift in Inuit but only pronouns undergo object shift in Aleut, a difference that parallels the two types of object shift that \citet{holmberg1986word} shows occurs in \ili{Scandinavian} languages. 

With the benefit of improvements in syntactic theory since Hale’s \citeyear{hale1997misumalpan} paper, particularly in the area of case and agreement, and with the benefit of excellent subsequent work on other Native American languages, I am able to present an account of the complex Aleut pattern which preserves Hale’s basic idea: subjects get \isi{ergative} case only when the object moves out of the VP in both Inuit and Aleut, however the two languages differ in what kind of objects move out of the VP. Nevertheless, I argue that Hale’s proposal has one important factor backwards: Hale postulates that pronouns undergo object shift only if they are null in Aleut.\footnote{To limit object shift to null pronouns, \citet{hale1997misumalpan} claims that overt pronouns get case in situ, but null pronouns have to move to get case. Aside from the problem of why this would be so, case no longer drives movement in more recent versions of theory (e.g. \citealt{chomsky1995minimalist,Chomsky2000}). Moreover, under the assumption that the decision as to which pronouns will and will not be pronounced (spelled out) is not made until PF, there is no distinction between overt and null pronouns in syntax.} In contrast, I argue that cause and effect goes in the other direction: pronouns are null only if they undergo object shift, because only shifted pronouns are in a position to trigger agreement, and only agreeing pronouns can be pro dropped (not spelled out/pronounced at PF), in both Aleut and Inuit.

A complication in Aleut is that object shifting some kinds of pronouns is blocked because it would cause the derivation to crash. Object shifting first and second \isi{person} pronouns out of the VP is blocked by PCC-like effects on the resulting portmanteau agreement (with the \isi{ergative} subject and nominative object), similar to what we see in dative-nominative constructions in \ili{Icelandic} with first and second \isi{person} objects \citep{Sigurðsson1996}.\footnote{PCC\is{Person Case Constraint} stands for \isi{Person Case Constraint}. \citet{bonet1994person} formulated this constraint to account for \isi{person} restrictions on sequences of a dative and accusative \isi{clitic} in \ili{Romance}. This term has been extended to apply to \isi{person} restrictions on agreement in clauses with a dative subject and a nominative object in \ili{Icelandic} and other languages. See \citet{anagnostopoulou2005strong} for discussion and additional references.} Thus first and second \isi{person} object do not undergo object shift in Aleut, and thus they do not agree. Because they do not agree, they cannot be pro dropped (null). Thus first and second \isi{person} object pronouns are overt and in situ in Aleut. \isi{Object shift} of reflexives is also blocked, due to the Anaphor Agreement Effect \citep{rizzi1990anaphor,woolford1999more}.\footnote{Reflexives are prohibited in positions construed with agreement \citep{rizzi1990anaphor}.} 

The proposed analysis of Aleut does not posit anything new. The complex Aleut pattern results from the interaction of nine known syntactic and \isi{morphological} properties, each of which is independently motivated in other languages. Almost all of these properties have been documented in Inuit and/or other Native American languages such as Nez Perce and Navajo:\footnote{This contrasts with the view expressed in \citealt{boyle2000aleut} that Aleut syntax is unlike that of any other language.}

\ea\label{ex:woolford:6}
\textit{Nine Independently Motivated Properties of Aleut}:

\begin{enumerate}
\item
\ili{Mainland Scandinavian} type object shift (pronouns only)
\item
Possessor raising\is{possessor}, as in Inuit  \citep[71]{bittner1994case} and Nez Perce \citep{deal2010ergative}.
\item
Possessed positional nouns/nominal stems in Aleut which have the function of postpositions \citep{fortescue1985anaphoric,bergsland1997aleut}
\item
\isi{Ergative}\is{ergative} subject only in object shift constructions (as in Inuit and Nez Perce)\footnote{Since \citet{bittner1996ergativity} established that there is a type of \isi{ergative} language where \isi{ergative} case is used on the subject only when the object moves out of the VP, additional languages of this type have emerged, e.g. Nez Perce and Niuean. See \citet{woolford2015ergativity} for a survey and discussion of this type of \isi{ergative} language. }
\item
Portmanteau agreement with subject and object in object shift constructions (as in Inuit)
\item
\isi{Object shift} of reflexives is blocked in Aleut, as in Inuit \citep[82]{bittner1994case} and Nez Perce \citep{deal2010ergative}, which is an instances of the Anaphor Agreement Effect \citep{rizzi1990anaphor,woolford1999more}
\item
\isi{Object shift} of first and second \isi{person} objects is blocked because the resulting portmanteau agreement from probing both the \isi{ergative} subject and the nominative object would incur a PCC-type violation (cf. \ili{Icelandic} dative subject constructions \citep{Sigurðsson1996})
\item
Pro-drop of pronouns that agree (as in Inuit \citep{bok1983on,bok1991case})
\item
Fronted topical objects, with a resumptive pronoun in situ (as in Navajo \citealt{Speas1990,Willie1991}) and as argued for Aleut by \citet{boyle2000aleut})
\end{enumerate}
\z

The conclusion of this paper is that the Aleut Effect, that is, the observed surface correlation between \isi{ergative} case on the subject and a null argument elsewhere in the clause in Aleut is a true correlation, but there is no direct causal relation. Both \isi{ergative} subjects and null objects are independent, indirect consequences of Mainland Type Object Shift of pronouns: 

\begin{itemize}
\item
Pronouns that undergo object shift agree, and agreeing pronouns are dropped (null) in \ili{Eskimo Aleut} languages, as in many languages.
\item
\isi{Object shift} places the object in a local relation with the subject, which requires \isi{ergative} case on the subject in the Eskimo-Aleut languages, as in Niuean and Nez Perce.\footnote{See \citet{woolford2015ergativity} for a survey and discussion of this type of \isi{ergative} language.}
\end{itemize}

This paper is organized as follows. The data and the proposed analysis of basic transitive clauses in Aleut, with a comparison to Inuit, is presented in \sectref{sec:woolford:2}. This section includes a discussion of clauses with a fronted topical object, clauses with a first or second \isi{person} object, and clauses with a reflexive object. \sectref{sec:woolford:3} focuses on \isi{possessor} raising constructions in Aleut, where the raised \isi{possessor} behaves like any object, just as in Inuit and Nez Perce. \sectref{sec:woolford:4} turns to possessed relational/positional nouns in Aleut, which also allow \isi{possessor} raising. \sectref{sec:woolford:5} discusses some remaining questions and \sectref{sec:woolford:6} is the conclusion.

\section{Transitive clauses in Inuit and Aleut}\label{sec:woolford:2} 
\subsection{Object shift}

\isi{Object shift} occurs in both Inuit and Aleut, but the two languages differ as to what kind of object undergoes object shift, paralleling the \ili{Scandinavian} languages. Inuit is like \ili{Icelandic} wherein specific objects undergo object shift, while non-specific objects do not. \citet{bittner1994case} gives the following minimal pair. In the example in \REF{ex:woolford:7}, the object is specific: there is one specific book which Juuna has not got:

\ea\label{ex:woolford:7}
\langinfo{Inuit}{}{\citealt[2]{bittner1994case}}\\
\gll Junna-p    atuagaq  ataasiq  tigu-sima-nngi-laa. \\
     Junaa-\textsc{erg}  book    one    get-\textsc{perf-neg-ind.\oldstylenums{3}sg.sg}\\
\glt ‘There is one book which Juuna hasn’t got (yet).’
\z

Bittner argues that the specific object has moved out of the VP, based on evidence involving the scope of \isi{negation}. 

\ea\label{ex:woolford:8}
[ Junna-\textsc{erg}  one book\textsubscript{i}  [\textsubscript{VP}  t\textsubscript{i}     got.neg ]]
\z

The case pattern in object shift constructions in Inuit is \isi{ergative}-nominative, with portmanteau agreement resulting from the multiple agree relation created when T probes to and through the \isi{ergative} subject down to the nominative object. 

In contrast, in the example in \REF{ex:woolford:9}, the object is non-specific. This object “can only take narrow scope, indicating that it remains below \isi{negation}, inside the VP” \citep[35]{bittner1994case}.

\ea\label{ex:woolford:9}
\langinfo{Inuit}{}{\citealt[35]{bittner1994case}}\\
\gll Juuna  ataukka-mik  ataasi-mik  tigu-si-sima-nngi-la-q.                           \\
     Junna  book-\textsc{instr} one-\textsc{instr}  get-\textsc{ap-perf-neg-ind-\oldstylenums{3}sg}  \\
\glt ‘Juuna has not got one book.’ 
\z

\ea\label{ex:woolford:10}
[ Junna-\textsc{erg}  [\textsubscript{VP}  one book     got ]]
\z

Here the subject has nominative case and the agreement is only with the subject.

Although we don’t see a word order change in the above pair of examples, more \isi{ergative} languages of this type have emerged in the linguistics literature, and for some there is clear word order evidence. One of these is Niuean. \citet{massam2010} gives the following pair of examples from Niuean where word order changes indicate whether or not object shift has occurred before the \isi{VP fronting} that characterizes this verb initial language. In the example in \REF{ex:woolford:11}, there is a specific object and object shift occurred before \isi{VP fronting}. The resulting word order is VSO and the case pattern is \isi{ergative} nominative:

\ea\label{ex:woolford:11}
\langinfo{Niuean}{}{\citealt[98]{massam2010}}\\{}
    [VSO   specific object]\footnote{In Niuean the \isi{ergative} case morpheme for proper nouns, \textit{e}, happens to look just like the nominative case morpheme for common nouns, \textit{e}.}\\
\gll Ne  inu  e  Sione    e  kofe.\\
     past  drink  \textsc{erg} Sione    \textsc{nom} coffee\\
\glt ‘Sione drank the coffee.’
\z

\ea\label{ex:woolford:12}
\ea Base order:        S  [\textsc{\textsubscript{vp}} V O]\\
\ex Order after object shift:    S  O\textsubscript{i}  [\textsc{\textsubscript{vp}} V  t\textsubscript{i} ]\\
\ex Order after \isi{VP fronting}    [\textsc{\textsubscript{vp}} V  t ]  S  O
\z
\z

In contrast, the example in \REF{ex:woolford:13} has a non-specific object and no object shift has occurred. The object thus fronts with the verb inside the fronted VP, so that the resulting word order is VOS:



\ea\label{ex:woolford:13}
\langinfo{Niuean}{}{\citealt[98]{massam2010}}\\{}
     [VOS   non-specific object]\\
\gll Ne  inu    kofe    a   Sione.                \\
     \textsc{past}  drink    coffee    \textsc{nom} Sione\\
\glt Sione drank coffee.
\z

\ea\label{ex:woolford:14}
\ea Base order:      S  [\textsc{\textsubscript{vp}} V O]\\
\ex Order after \isi{VP fronting}:  [\textsc{\textsubscript{vp}} V O]  S
\z
\z

I argue that Aleut manifests a pattern similar to the related Inuit language, the difference being that only pronouns undergo object shift in Aleut, paralleling \ili{Mainland Scandinavian}. Transitive clauses with non-pronominal arguments have an ordinary nominative-accusative case pattern in Aleut (although neither case is marked) and agreement is only with the subject:

  


\ea\label{ex:woolford:15}
\langinfo{Aleut}{}{\citealt[2 (1a)]{boyle2000aleut} from \citealt[32]{bergsland1981atkan}}\\
\gll Piitra-\^{x}    Ivaana-\^{x}  kidu-ku-\^{x}.\\
     Peter-\oldstylenums{3}\textsc{sg}  John-\oldstylenums{3}\textsc{sg}  help-\textsc{pres}-\oldstylenums{3}\textsc{sg}\\
\glt ‘Peter is helping John.’ 
\z

Non-pronominal objects remain in situ in the VP in Aleut:

\ea\label{ex:woolford:16}
[ Peter(\textsc{nom})    [\textsubscript{VP} John(\textsc{acc})    V-\oldstylenums{3}\textsc{sg}]]
\z

In contrast, I argue that (third \isi{person}) pronoun objects undergo object shift in Aleut, as in \ili{Mainland Scandinavian}.\footnote{We will see below that the object shift construction is blocked with first and second \isi{person} objects, as well as with reflexive objects in Aleut.} The case pattern in object shift constructions in Aleut is \isi{ergative}-nominative, as in Inuit \citep{bittner1994case,bittner1996ergativity}. The agreement is portmanteau, expressing features from both the \isi{ergative} subject and the nominative object:\footnote{The \isi{ergative} case is labeled ‘\isi{relative}’ in descriptive work on Aleut. The portmanteau agreement series is referred to as the anaphoric series, reflecting the fact that this series identifies the feature of a null object. Work on Aleut generally uses the traditional typological label ‘absolutive’ for the object of a clause with an \isi{ergative} subject, however the term ‘absolutive’ was meant to be the neutral label of a typological pattern \citep{dixon1994ergativity} rather than the label of an actual case. As \citet{bittner1994case} argues for the related language Inuit/West Greenlandic, the identity of this case is nominative in Aleut. There are \isi{ergative} languages that have accusative objects, e.g. Warlpiri \citep{legate2006split}, but accusative objects do not participate in portmanteau agreement with the subject. }

 


\ea\label{ex:woolford:17}
\langinfo{Aleut}{}{\citealt[3 (1b)]{boyle2000aleut} from \citealt[32]{bergsland1981atkan}}\\
\gll Piitra-m   kidu-ku-u.        \\
     Peter-\textsc{erg}  help-\textsc{pres}-\oldstylenums{3}\textsc{sg}/\oldstylenums{3}\textsc{sg} \\
\glt ‘Peter is helping him.' 
\z

\ea\label{ex:woolford:18}
[ Peter-\textsc{erg}  pro.\textsc{nom}\textsubscript{i}  [\textsubscript{VP} t\textsubscript{i}  V-\oldstylenums{3}\textsc{sg}/\oldstylenums{3}\textsc{sg}]]
\z

The object pronoun is not pronounced (not spelled out at PF) because it agrees and agreeing pronouns drop in Aleut, as in Inuit. We see both subject and object pronoun drop in the following Aleut example:

 


\ea\label{ex:woolford:19}
\langinfo{Aleut}{}{\citealt[3 (5)]{boyle2000aleut} from \citealt[10]{bergsland1981atkan}}\\
\gll Kidu-ku-ngis.       \\
     help-\textsc{pres}-\oldstylenums{3}/\oldstylenums{3}\textsc{pl}    \\
\glt ‘They are helping him/her/them.’
\z

\citet{bok1983on} gives the following examples with subject and object pronoun drop from Yupi’k:



\ea\label{ex:woolford:20}
\langinfo{Yup’ik}{}{\citealt[1]{bok1983on}}\\
\gll Yurar-tug.   \\
     dance-\textsc{ind}.\oldstylenums{3}\textsc{sg}\\
\glt ‘She/he is dancing.’
\z

 


\ea\label{ex:woolford:21}
\langinfo{Yup’ik}{}{\citealt[2]{bok1983on}}\\
\gll Tangrr-aa.   \\
     see-\textsc{ind}.\oldstylenums{3}/\oldstylenums{3} \\
\glt ‘He/she sees him/her/it.
\z

\subsection{Topical objects}

Aleut allows a topical object to precede the clause, with a resumptive pronoun in situ, as in example \REF{ex:woolford:23}: 

\ea\label{ex:woolford:22}
\langinfo{Aleut}{}{\citealt[3 (6a)]{boyle2000aleut} from \citealt[27]{Bergsland1969}}\\
\gll Taya\^{g}u-\^{x}   qa-\^{x}     qa-ku-\^{x}.\\
     man-\textsc{sg}    fish-\textsc{sg}  eat-\textsc{pres}-\oldstylenums{3}\textsc{sg}.           \\
\glt 'The man is eating the fish.'
\z


\ea\label{ex:woolford:23}
\langinfo{Aleut}{}{\citealt[4 (6b)]{boyle2000aleut} from \citealt[27]{Bergsland1969}}\\
\gll qa-\^{x}     taya\^{g}u-m  qa-ku-u.                \\
     fish-\textsc{sg}\textsubscript{i}    man-\textsc{erg}  eat-\textsc{pres}-\oldstylenums{3}/\oldstylenums{3}\textsc{sg}\\
\glt 'The fish, the man is eating it.'
\z

This is not movement leaving a \isi{trace}, but rather a base generated topic linked to a resumptive pronoun in situ \citep{boyle2000aleut}, paralleling what \citet{Willie1991} and \citet{Speas1990} argue for in Navajo. As with any pronoun object, this resumptive pronoun undergoes object shift, triggers agreement on the verb, and is thus not spelled out at PF (pro-dropped):

\ea\label{ex:woolford:24}
fish\textsubscript{i}  [ man-\textsc{erg}  pro.\textsc{nom}\textsubscript{i  }[\textsubscript{VP} t\textsubscript{i  }V-\oldstylenums{3}/\oldstylenums{3}]]
\z

\subsection{First and second person objects }

Something different happens with first and second \isi{person} objects; these do not undergo object shift, do not agree, and must thus be spelled out at PF. The case pattern is nominative-accusative, as in any transitive clause without object shift in Aleut: 

 
\ea\label{ex:woolford:25}
\langinfo{Aleut}{}{\citealt[344]{bergsland1997aleut}}\\
\gll Tayag\^{ }u-\^{x} ting   kidu-ku-x\^{}\^{}.\\
     man-\oldstylenums{3}\textsc{sg}(\textsc{nom})    me(\textsc{acc})  help-\textsc{pres}-\oldstylenums{3}\textsc{sg}     \\
\glt ‘The man is helping me.’ 
\z

I argue that the object shift construction is blocked/crashes with first and second \isi{person} objects because the portmanteau agreement that would result violates PCC-type constraints, similar to what has been observed in \ili{Icelandic} dative subject constructions with first and second \isi{person} nominative objects, as in \citealt{Sigurðsson1996,sigurdhsson2004icelandic}:\footnote{See \citet{anagnostopoulou2005strong} for discussion and additional references on PCC\is{Person Case Constraint} effects.} 

\ea\label{ex:woolford:26}
\langinfo{Icelandic}{}{\citealt[148]{sigurdhsson2004icelandic}}\\
\gll Honum    mundu  alltaf    líka  þeir.         \\
     him.\textsc{dat}  would.\oldstylenums{3}\textsc{pl}  always  like  they.\textsc{nom}\\
\glt ‘He would always like them.’
\z

\ea\label{ex:woolford:27}
\langinfo{Icelandic}{}{\citealt[148]{sigurdhsson2004icelandic}}\\
\gll *Honum    munduð  alltaf    líka  þið.                  \\
     \textit{ }him.\textsc{dat}  would.2\textsc{pl}  always  like   you.\textsc{nom} \\
\glt ‘He would always like you.’
\z

Thus the only grammatical version of this construction in Aleut is the one without object shift, as in \REF{ex:woolford:25}, because it has a nominative-accusative case pattern and agreement only with the subject.\footnote{In contrast to Aleut, the related language Inuit allows first and second \isi{person} objects to undergo object shift. We know that PCC-type effects are not \isi{universal}, and that languages that do manifest PCC\is{Person Case Constraint} effects vary widely in exactly which feature combinations are prohibited; however, we may be able to identify a more specific reason why first and second \isi{person} objects are allowed to undergo object shift in Inuit, in contrast to Aleut. In addition to portmanteau agreement, Inuit has pronominal clitics which suffix to the portmanteau agreement. When a first or second \isi{person} object moves out of the VP, Inuit may avoid some PCC-type violations by encoding only the \isi{number} of the object in the portmanteau agreement, and encoding the first or second \isi{person} feature of the object in a separate pronominal \isi{clitic}. See \citet{fortescue1985anaphoric} and \citet{woolford2016two}.}

\subsection{The pattern with reflexive objects}

In both Inuit and Aleut, object shift is blocked (or the object shift construction crashes) when the object is a reflexive. We see this for Inuit in the following pair of examples from \citealt{bittner1994case}. The Inuit example in \REF{ex:woolford:28} has a specific object, which undergoes object shift, with the resulting \isi{ergative}-nominative case pattern. In contrast, in the example in \REF{ex:woolford:29} with a reflexive object, object shift has not occurred and the case pattern is nominative-dative:

\ea\label{ex:woolford:28}
\langinfo{Inuit}{}{\citealt[82]{bittner1994case} }\\
\gll Suulu(t)-p  Kaali    aallaa-vaa.                \\
     Suulu-\textsc{erg}  Kaali    shoot-\textsc{ind}.\oldstylenums{3}\textsc{sg}/\oldstylenums{3}\textsc{sg}   \\
\glt ‘Suulut shot Kaali.’
\z

\ea\label{ex:woolford:29}
Inuit \citep[82]{bittner1994case}\\
\gll Sullut    immi-nut  aallaa-vuq.          \\
     Sullut    self-\textsc{dat}  shoot-\textsc{ind}.\oldstylenums{3}\textsc{sg}\\
\glt ‘Suulut shot himself.’
\z

The object shift version of the reflexive object example is ungrammatical:

\ea\label{ex:woolford:30}
\langinfo{Inuit}{}{\citealt[82]{bittner1994case}}\\
\gll *Suulu(t)-p  immi  aallaa-vaa.                   \\
      Sullut-\textsc{erg}  self  shoot-\textsc{ind}.\oldstylenums{3}\textsc{sg}/\oldstylenums{3}\textsc{sg}     \\
\glt  ‘Suulut shot himself.’ 
\z

This is an example of the Anaphor Agreement effect \citep{rizzi1990anaphor,woolford1999more}: anaphors cannot occur in positions normally associated with agreement.\footnote{The Anaphor Agreement Effect appears to be \isi{universal} \citep{woolford1999more}, but why it holds is still a mystery.} 

We see the same pattern with reflexive objects in Aleut. The following example in \REF{ex:woolford:31} has a pronoun subject (which is not spelled out because it agrees), and an overt reflexive object. The agreement, which is only with the subject, tells us that object shift has not occurred. Although the subject pronoun is not spelled out, because it agrees, the reflexive pronoun has to be spelled out because it does not agree:

\ea\label{ex:woolford:31}
\langinfo{Aleut}{}{\citealt[139]{Bergsland1969}}\\
\gll Txin     achixa{}-ku-\^{x}.     \\
     \oldstylenums{3}.\textsc{refl.sg}    teach-\textsc{pres}-\oldstylenums{3}\textsc{sg}    \\
\glt ‘He taught himself.'
\z

This contrasts with the example in \REF{ex:woolford:32} with a pronoun object that is disjoint in \isi{reference} with the subject. Here we see by the portmanteau agreement on the verb that object shift has occurred. The shifted object is dropped (not spelled out at PF) because it agrees.

\ea\label{ex:woolford:32}
\langinfo{Aleut}{}{\citealt[139]{Bergsland1969}}\\
\gll Kidu-ku-u.        \\
     help-\textsc{pres}-\oldstylenums{3}/\oldstylenums{3}\textsc{sg}  \\
\glt ‘He is helping him.’
\z

\section{Possessed objects}\label{sec:woolford:3} 
\subsection{Possessor raising}

Pronominal possessors of objects behave like pronominal objects in Aleut. I argue that this is due to \isi{possessor} raising, paralleling Nez Perce \citep{Deal2013}.\footnote{\citet[71--72]{bittner1994case} shows that Inuit allows \isi{possessor} raising, but only with some verbs.} Nez Perce is like Aleut in that it has object shift resulting in an \isi{ergative} subject and portmanteau agreement, and it also has clearly transitive clauses without object shift where the subject is nominative and agreement is only with the subject.

 
\ea\label{ex:woolford:33}
\langinfo{Nez Perce}{}{\citealt[552]{rude1988ergative}}\\
\gll Háama    hi-'wí-ye  wewúkiye.  \\
     man(\textsc{nom})  \oldstylenums{3}-shoot-\textsc{asp}  elk(\textsc{acc})\\
\glt ‘The man shot an elk.’
\z

\ea\label{ex:woolford:34}
\langinfo{Nez Perce}{}{\citealt[552]{rude1988ergative}}\\
\gll Háama-nm  pée-'wi-ye    wewúkiye-ne.\\
       man-\textsc{erg}  \oldstylenums{3}/\oldstylenums{3}-shoot-\textsc{asp}  elk-\textsc{obj}   \\
\glt   ‘The man shot an elk.’  
\z

In terms of what kind of object undergoes object shift, Nez Perce is more like Inuit and \ili{Icelandic} in that non-pronominal arguments can undergo object shift. According to \citet{Rude1982,Rude1986}, it is the more topical objects that undergo object shift in Nez Perce, and less topical objects do not.

Nez Perce allows \isi{possessor} raising, as \citet{Deal2013} shows, and raised possessors undergo object shift; as in other object shift constructions in Nez Perce, the subject is \isi{ergative} and the agreement is portmanteau and the object takes the case glossed as objective:\footnote{See \citet{Deal2013}  for a discussion of the morpheme she glosses as μ.}

\ea\label{ex:woolford:35}
\langinfo{Nez Perce}{}{\citealt[398 (14)]{Deal2013} from \citealt[119]{Rude1986}}\\
\gll Hi-nees-hex-ne’ny-e   ma-may’as-na   pist.                               \\
     3\textsuperscript{rd}{}-\textsc{pl.obj}-see-μ{}-\textsc{past}  \textsc{pl}-child-\textsc{obj}    father \\
\glt ‘He saw the children’s father.’
\z

In contrast, we see an unraised \isi{possessor} in the following example in \REF{ex:woolford:36}. The first object has undergone object shift, and we see the genitive \isi{possessor} in the second object:

\ea\label{ex:woolford:36}
\langinfo{Nez Perce}{}{\citealt[400 (20a)]{Deal2013}}\\
\gll ’Ew-’nii-se    Tatlo-na   Angel-nim   taaqmaaɬ.\\
     \oldstylenums{3}\textsc{obj}-give-\textsc{imperf}  Tatlo-\textsc{obj}  Angel-\textsc{gen}  hat    \\
\glt ‘I’m giving Tatlo Angel’s hat.’
\z

In contrast to Nez Perce, Aleut only allows pronominal possessors to undergo \isi{possessor} raising and object shift.\footnote{\citet{Deal2013} argues that object shift is directly from the \isi{possessor} position, without a separate step of \isi{possessor} raising in Nez Perce. This could be true of Aleut as well.} The pair of examples below shows the contrast between an NP \isi{possessor} in \REF{ex:woolford:37}, and a pronominal \isi{possessor} in \REF{ex:woolford:38}. In \REF{ex:woolford:37}, there is no object shift, which is what we expect with non-pronominal objects in Aleut; thus the subject case is nominative and the agreement on the verb is only with the nominative subject:

\ea\label{ex:woolford:37}
\langinfo{Aleut}{}{\citealt[3 (4a)]{boyle2000aleut} from \citealt[144]{bergsland1997aleut}}\\
\gll Piitra-\^{x}     [hal-s    ada-a]   kidu-ku-\^{x}.\\
     Peter-\oldstylenums{3}\textsc{sg}  [boy-\textsc{pl}  father-\oldstylenums{3}/\oldstylenums{3}]  help-\textsc{pres}-\oldstylenums{3}\textsc{sg}    \\
\glt ‘Peter is helping the boys' father.'  
\z

In contrast in \REF{ex:woolford:38} below the \isi{possessor} of the object is a pronoun. This pronoun undergoes object shift, and we see all the usual consequences of object shift in Aleut. The subject is \isi{ergative} and the shifted plural pronoun is cross-referenced in the portmanteau agreement on the verb. Because it agrees, the object shifted pronoun is not pronounced (i.e. it is pro-dropped):

\ea\label{ex:woolford:38}
\langinfo{Aleut}{}{\citealt[144]{bergsland1997aleut}}\\
\gll Piitra-m    ada-ngis  kidu-ku-ngis.      \\
     Peter-\textsc{erg}  father-\oldstylenums{3}/\oldstylenums{3}\textsc{pl}  help-\textsc{tns}-\oldstylenums{3}/\oldstylenums{3}\textsc{pl} \\
\glt ‘Peter is helping their father.’
\z

\ea\label{ex:woolford:39}{}
[ Peter-\textsc{erg}  their.\textsc{nom}\textsubscript{i} [ [t\textsubscript{ i}  father]  help-\oldstylenums{3}/\oldstylenums{3}\textsc{pl}]]
\z

\subsection{No object shift of coreferent possessors}
Aleut parallels Nez Perce as well in constructions where the pronominal \isi{possessor} of the object is coreferent with the subject. Here object shift is blocked in both languages, and Deal’s account of why extends to Aleut. \citet[413]{Deal2013} points out that Binding Condition B rules out a pronominal object being coreferent with the subject, as in the Nez Perce example in \REF{ex:woolford:40} where the pronominal \isi{possessor} has undergone object shift (as evidenced by the \isi{ergative} subject and portmanteau agreement on the verb):

\ea\label{ex:woolford:40}
\langinfo{Nez Perce}{}{\citealt[413 (54a)]{Deal2013}}\\
\gll Pit’iin’-im   paa-’ya\^{x}-na’ny-a  ’ip-ne picpic. \\
     girl-\textsc{erg}    \oldstylenums{3}/\oldstylenums{3}-find-μ-\textsc{rem.past}    his/her-\textsc{obj}  cat               \\
\glt ‘The girl\textsubscript{i} found his/her\textsubscript{j} cat.’    (no coreference)  
\z

If there is coreference, only the version without object shift (and without \isi{ergative} case and portmanteau agreement) in \REF{ex:woolford:41} is grammatical:

\ea\label{ex:woolford:41}
\langinfo{Nez Perce}{}{\citealt[413 (54b)]{Deal2013}}\\
\gll Pit’iin’    hi-’yaa\^{x}-n-a     [’ip-nimi  picpic].\\
     girl(\textsc{nom})  \oldstylenums{3}-find-\textsc{asp}-\textsc{rem.past}  [her-\textsc{gen}  cat]      \\
\glt ‘The girl\textsubscript{i} found her\textsubscript{i} cat.’    (coreference)  
\z

If the object shifted pronoun were to be interpreted as a reflexive rather than an ordinary pronoun, coreference in \REF{ex:woolford:41} would also be ruled out, but for a different reason: the reflexive pronount would agree and that would be an instance of the Anaphor Agreement Effect: anaphors are barred from positions that agree. Thus the only solution is to block object shift when there is co-\isi{reference} between the subject and the \isi{possessor} of an object.

The same pattern is found in Aleut. When a pronoun \isi{possessor} undergoes object shift, it cannot be interpreted as coreferent with the subject:

\ea\label{ex:woolford:42}
\langinfo{Aleut}{}{\citealt[54]{bergsland1997aleut}}\\
\gll Hla-m                               ~                     [ ~ ada-a                                                       ]  kidu-ku-u.            \\
     boy-\textsc{erg}\textsubscript{i}    pro\textsubscript{j}  [  t\textsubscript{j}  father-\oldstylenums{3}/\oldstylenums{3} ]  help-\textsc{pres}-\oldstylenums{3}/\oldstylenums{3}   \\
\glt ‘The boy\textsubscript{i} is helping his\textsubscript{ j}  father.’  (no coreference)
\z

As \citet{Deal2013} concludes for Nez Perce, this is due to Binding Condition B which prohibits coreference between a pronoun object and a c-commanding subject in the same clause.

Co\isi{reference} is only possible if the pronoun \isi{possessor} does \textit{not} undergo object shift, as in \REF{ex:woolford:43}. We can tell that object shift has not occurred in \REF{ex:woolford:43} because the subject is not \isi{ergative} and the verbal agreement is only with the subject.


\ea\label{ex:woolford:43}
\langinfo{Aleut}{}{\citealt[54]{bergsland1997aleut}}\\
\gll Hla-\^{x}     [  ada-an ]    kidu-ku-\^{x}.                                           \\
     boy-\textsc{sg}(\textsc{nom})\textsubscript{i}  [pro\textsubscript{i}  father-\oldstylenums{3}.\textsc{refl} ]    help-\textsc{tns}-\textsc{sg}   \\
\glt ‘Peter is helping his (own) father.’
\z

Note that the pronoun \isi{possessor} is null (not spelled out) even in its base position inside the object NP in \REF{ex:woolford:43} because it agrees with the head noun, and agreeing pronouns are dropped (not spelled out).

\section{Possessed relational nouns in Aleut}\label{sec:woolford:4}

From the English translations of the following examples, one could easily get the impression that the subject is ergative in Aleut when the object of a PP is null, and as far as we know, object shift does not occur out of PPs. However, these constructions do not actually involve PPs, but rather possessed \isi{directional} nouns with an oblique case suffix (\citealt{fortescue1985anaphoric}, \citealt[47]{bergsland1997aleut}). These constructions work like the possessed object constructions discussed above in \sectref{sec:woolford:3}. The version of the sentence in \REF{ex:woolford:44} has a non-pronominal NP \isi{possessor}, which remains in situ, since only pronouns undergo object shift in Aleut. In contrast, the version of the sentence in \REF{ex:woolford:45} has a pronominal \isi{possessor}, which does undergo object shift. 

\ea\label{ex:woolford:44}
\langinfo{Aleut}{}{\citealt[126]{bergsland1997aleut}}\\
\gll Piitra-\^{x}    tayagu-m  had-a-     huya-ku-\^{x}. \\
     Peter-\oldstylenums{3}\textsc{sg}  man-\textsc{gen}   direction-\oldstylenums{3}/\oldstylenums{3}-\textsc{loc}  go-\textsc{pres}-\oldstylenums{3}\textsc{sg} \\
\glt ‘Peter is going toward the man (in the man’s direction).’
\z

\ea\label{ex:woolford:45}
\langinfo{Aleut}{}{\citealt[127]{bergsland1997aleut}}\\
\gll Piitra-m    had-a-n    huya-ku-u.              \\
     Peter-\textsc{erg}   direction-\oldstylenums{3}/\oldstylenums{3}-\textsc{loc}   go-\textsc{pres}-\oldstylenums{3}/\oldstylenums{3}\textsc{sg}\\
\glt ‘Peter is going toward him/her (in his/her direction).’
\z

We observe the same differences in the case and agreement patterns in these examples as we do in examples \REF{ex:woolford:37} and \REF{ex:woolford:38} in \sectref{sec:woolford:3}.

\section{Remaining questions}\label{sec:woolford:5}

There are remaining issues concerning Aleut grammar that have not been discussed in this paper. One is the question that \citet{sadock2000aleut} asks: what determines which features are expressed by the agreement \isi{morphology} in different constructions in Aleut? This paper addresses only part of this question, predicting when verbal agreement in clauses can and cannot be portmanteau; verbal agreement cannot be portmanteau unless object shift has occurred. However, the question of which agreement features will be spelled out at PF is a separate issue. Sadock suggests that in some instances, the choice of what agreement features to realize at PF, especially in agreement in DPs (which have not been discussed in this paper), can depend on functional/communicative factors.

Another remaining {question} concerns Aleut examples where (under the analysis proposed in this paper) object shift appears to occur out of some kind of embedded/adjunct clause. \citet{berge2010unexpected} cites the following pair of examples from \citet[248]{bergsland1997aleut}. In the first example in \REF{ex:woolford:46}, the verb ‘go.to.sleep’ agrees only with its subject. In contrast, in the second example in \REF{ex:woolford:47}, the agreement is portmanteau, also encoding the \oldstylenums{3}\textsc{pl} features of what looks like the \isi{null subject} of the adjunct/embedded clause, “When (they) stopped talking”: 

\ea\label{ex:woolford:46}
\langinfo{Aleut, Atkan dialect}{}{\citealt[10]{berge2010unexpected}}\\
\gll Hla-s  tunum-kada-ku-z-iin   ting   sa\^{g}ani-na-q.         \\
     boy-\textsc{pl}  talk-\textsc{cess}-\textsc{ind}-\textsc{pl}-\textsc{encl}   \oldstylenums{1}\textsc{sg}   go.to.sleep-part-\oldstylenums{1}\textsc{sg}  \\
\glt ‘When the boys stopped talking, I went to sleep.’
\z

\ea\label{ex:woolford:47}
\langinfo{Aleut, Atkan dialect}{}{\citealt[10]{berge2010unexpected}}\\
\gll  Tunum-kada-ku-z-iin   ting  sa\^{g}ani-qa-ning.           \\
     talk-\textsc{cess}-\textsc{ind}-\textsc{pl}-\textsc{encl}  \oldstylenums{1}\textsc{sg}  go.to.sleep-part.an-\oldstylenums{1}\textsc{sg}/\oldstylenums{3}\textsc{pl}.an \\
\glt ‘When they stopped talking, I went to sleep.’
\z

One possible clue to understanding these particular examples is that the morpheme -\textit{iin} which Berge glosses simply as ‘enclitic’ is actually the 2\textsuperscript{nd}/3\textsuperscript{rd}plural \isi{possessive} ending on nouns –\textit{iin}, as in \textit{adam-aziin} ‘to our fathers’ \citep[149]{bergsland1997aleut}. This suggests the possibility that what is translated as ‘when they stopped talking’ might be more accurately translated as ‘(at) their stopping-talking’. If so, this could be another instance of \isi{possessor} raising.

Merchant (personal communication) suggests the possibility of a similar analysis of Aleut examples which, from their English translations, would appear to involve movement out of a \isi{relative} clause. As Merchant notes in his \citeyear{merchant2011aleut} paper ``Some Aleut \isi{relative} clauses have something like the form of a possessed clause.'' \citep[397]{merchant2011aleut}

\section{Summary and conclusions}\label{sec:woolford:6}


The goal of this paper has been to show that the Eskimo-Aleut languages parallel the \ili{Scandinavian} languages in manifesting two types of object shift, one where only pronouns undergo object shift, and one where all specific objects undergo object shift. Aleut is like the \ili{Mainland Scandinavian} languages in allowing only pronouns to undergo object shift, while Inuit is like \ili{Icelandic} in allowing all specific objects to shift.

We have seen that the consequences of object shift are much more complex in the \ili{Eskimo Aleut} languages than they are in \ili{Scandinavian} languages. In the Eskimo-Aleut languages, clauses with object shift have an \isi{ergative} subject and portmanteau agreement, in contrast to clauses without object shift which have a nominative subject and agreement only with the subject. These consequences of object shift have also been observed in other \isi{ergative} languages such as Nez Perce and Niuean. Additional factors interacting with object shift in Aleut include possessed relational nouns instead of PPs and \isi{possessor} raising.

A second goal of this paper has been to show that, although the surface pattern of Aleut shows a perfect correlation between \isi{ergative} case on the subject and a null object (The Aleut Effect), this correlation does not reflect causation. Instead, object shift (interacting with other factors) causes both \isi{ergative} subjects and null objects in Aleut. While the surface complexity of the Aleut pattern might initially seem to warrant adding significant generative machinery to the grammar, I have shown in this paper that the Aleut Effect follows automatically from a combination of nine grammatical constructions/factors, each of which is independently motivated in other languages.\is{agreement|)}

\section*{Acknowledgements}
It is a privilege to have been invited to contribute to this volume dedicated to Anders Holmberg in honor of his important contributions to linguistic theory. In my case, without his contributions in the area of object shift typology, I would never have been able to solve the mystery of the puzzle known as the Aleut Effect.
I would like to thank Jason Merchant and Margaret Speas and two anonymous reviewers for reading this manuscript and providing many helpful comments.

\printbibliography[heading=subbibliography,notkeyword=this]
\end{document}