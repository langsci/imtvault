\documentclass[output=paper]{LSP/langsci} 
\author{Richard S. Kayne\affiliation{New York University}}
\title{A note on some even more unusual relative clauses}  
\abstract{Relative clauses can be found that contain a relative pronoun whose antecedent is not the head of the relative. The familiar relation between the head of a relative and the relative pronoun can thus be seen as a special case of a more general relation between a relative pronoun (a stranded determiner) and its antecedent (whose movement has stranded that determiner). The piece of relative clause syntax that is the antecedent-relative pronoun relation is less specific to relative clauses that it might have seemed.}

\ChapterDOI{10.5281/zenodo.1117708}
\maketitle

\begin{document}
 
\section{Some general points on relative clauses}

     In the spirit of \citet{Chomsky1970} on ‘\isi{passive}’, the notion ‘\isi{relative} clause’ is unlikely to be a primitive of the language faculty.  This was explicitly recognized in \citet{Chomsky1977}, to the extent that the \isi{wh-movement} operation that plays a role in the derivation of \isi{relative} clauses also plays a role elsewhere (e.g in interrogatives).  \citet{Rizzi1997} might be interpreted as backtracking from this position insofar as the landing site for \isi{wh-movement} in relatives is different (Spec,ForceP) from the landing site in interrogatives (Spec,FocP/IntP\is{Int}).

     The difference in landing site, though, could be factored out from the common movement operation, and taken instead as something to be explained.  The following proposal is based on the fact that the wh-phrase in headed relatives is in a relation to the ‘head’ of the \isi{relative} in a way that has no exact counterpart in interrogatives, which lack a comparable ‘head’:

\ea\label{ex:kayne:1}
   Wh-movement in relatives cannot (normally) land below ForceP (or TopP\footnote{Cf. \citet{Cinque1982} on links with topicalization.}) because of locality requirements holding between the ‘head’ of the \isi{relative} and the wh-phrase.
\z

     The informal formulation in \REF{ex:kayne:1} abstracts away from the {question} of the correctness of the raising analysis of relatives.\footnote{See \citealt[125]{Brame1976}, \citealt{Schachter1973,Vergnaud1974,Vergnaud1985}; \citealt[chap. 9]{Kayne1994}; \citealt{Bianchi1999};  and \citealt{KatoNunes2009}.
     Headless relatives may be hidden instances of (adjunct) interrogatives, thinking of the similarity between:
\ea     We’ll buy whatever you suggest.\z
and
\ea  No matter what you suggest, we’ll buy it.
\z
For a suggestion along such lines, see \citet{Lin1996}.} 
 In what follows, I will assume the raising approach (perhaps not crucially).

     In addition to \isi{wh-movement}, a second, related aspect of \isi{relative} clauses that is not specific to them is the very presence of overt wh-words.  A proposal expressing this non-specificity would be (cf. \citealt{Postma1994}):

  \ea\label{ex:kayne:2}
  \ea  The \textit{which} of English (headed) relatives is identical to the \textit{which} of English interrogatives (and to the \textit{which} of \textit{every which way}).
\ex  The \textit{where} of English relatives is identical to the \textit{where} of English interrogatives, as well as to the \textit{where} of \textit{somewhere, nowhere, anywhere, everywhere, elsewhere}.
\ex  and similarly for other wh-words in whatever language.
\z
\z

Needless to say, the surrounding syntactic environment must be at least partially different in relatives, interrogatives and indefinites.\footnote{It is not essential to this discussion whether \textit{everywhere} is a true indefinite -- \citet{BeghelliStowell1997}.}

     Note that \REF{ex:kayne:2} does not state that the sets of wh-words occurring in relatives and interrogatives and indefinites have to match perfectly.  In English \textit{where} occurs in all three, but \textit{who} occurs only in relatives and interrogatives\textit{.} In \ili{Italian} \textit{quale} (‘which’) occurs in both relatives and interrogatives, but \textit{cui} (‘who/what’)\footnote{\ili{Italian} \textit{cui} is arguably an oblique form of \textit{che}, i.e. \textit{ch-+-ui}, with oblique (possibly bimorphemic - cf. \citet{Martín2012}) \textit{-ui} lacking in Spanish (and similarly for \ili{Italian} \textit{lui}, \textit{altrui}).  (Note that non-oblique \textit{che} does occur in interrogatives in \ili{Italian}.)} occurs only in relatives and \textit{chi} (‘who’) occurs only in interrogatives (and free relatives).

     This point about wh-words not being specific to \isi{relative} clauses carries over to those \isi{relative} pronouns that are clearly related to demonstratives (such as \ili{German} \isi{relative} \textit{d}{}-words).  If \citet{Kayne2010a} is correct, this point also holds for English \textit{that}, which occurs both as a \isi{relative} pronoun and as an ordinary \isi{demonstrative}.

     The proposal in \REF{ex:kayne:2} can be understood as a particular case of a more general approach that is also illustrated by English numerals.\footnote{Cf. \citet{Kayne2016} on English \textit{there} and more generally on anti-homophony.}  Consider:

\ea\label{ex:kayne:3}
   They have seven children.
\z

\ea\label{ex:kayne:4}
   Their youngest child has just turned seven.
\z

\ea\label{ex:kayne:5}
   It’ll be exactly seven in a couple of minutes.
\z

Example \REF{ex:kayne:3} shows an ordinary instance of the numeral \textit{seven}.  In \REF{ex:kayne:4} and \REF{ex:kayne:5}, a bare \textit{seven} appears to be interpreted as an age and as a time of day, respectively.  \citet{Kayne2003} argued that cases like \REF{ex:kayne:4} and \REF{ex:kayne:5} are best analyzed in terms of the presence of silent nouns, with \REF{ex:kayne:4} containing (at least) the noun YEAR (capitalization indicates silence) and \REF{ex:kayne:5} containing (at least) HOUR.\footnote{This approach, in which interpretations are constrained by the availability of silent elements, looks likely to be more restrictive that the allosemy-based approach of \citet{Marantz2010} and \citet{WoodMarantztoappear}.  This will be especially clear if the language faculty disallows elements that would be consistently silent in all languages.}

\section{Unusual relative clauses (with more than one relative pronoun)}

     Like interrogatives, relatives can sometimes to some extent contain more than one wh-word:

\ea\label{ex:kayne:6}
   (?)Mary Smith, whose husband’s love for whom knows no bounds, is a famous linguist.
\z

\ea\label{ex:kayne:7}
   ?The only woman whose husband’s love for whom knows no bounds is Mary Smith.
\z

In \REF{ex:kayne:6} and \REF{ex:kayne:7}, both of the wh-words/\isi{relative} pronouns are related to the head of the \isi{relative}. It may be that \textit{whose husband’s love for whom} in \REF{ex:kayne:6} and \REF{ex:kayne:7} has been pied-piped by the initial \textit{who(se)}, rather than by \textit{whom}.  This \textit{whom} appears in any case to be ‘in situ’ within the larger wh-phrase.  Yet there is evidence that this \textit{whom} is involved in a movement relation, perhaps of the parasitic gap sort.\footnote{For some discussion, see \citet[239ff]{Kayne1983}.}  This is suggested by the existence of ECP-like effects, as in:\footnote{On the Empty Category Principle, see \citet{Chomsky1981}.}

\ea\label{ex:kayne:8}
   ?Mary Smith, whose husband’s desire for me to paint a picture of whom is perfectly understandable, is a very famous linguist.
\z

\ea\label{ex:kayne:9}
   *Mary Smith, whose husband’s desire for whom to paint a picture of me is perfectly understandable, is a very famous linguist.
\z
\largerpage[-3]
\pagebreak\section{Even more unusual relative clauses}
     There also exist \isi{relative} clauses containing two \isi{relative} pronouns such that only one of them is related to the head of the \isi{relative}.\footnote{\widowpenalty=10000\clubpenalty=10000\relax There is a point of similarity here with \citegen{Stowell1985} discussion of parasitic gap examples such as:
   \ea Who did your stories about amuse?\z
which for some speakers (but not me, in this case) allow an interpretation in which two distinct individuals are at issue.

     It remains to be understood what underlies the variation in speaker judgments, both in the case of (i) and in the case of the unusual relatives discussed in the text.} 
  These are for me somewhat more marginal than the preceding, but are still surprisingly close to acceptability (in the English of some speakers).  An example is:\footnote{Another is:
   \ea  ?That car over there just ran into my old friend John Smith, whose inability to get a good view of which was a determining factor in the accident.\z
     This kind of \isi{relative} is more difficult as a restrictive:
   \ea  ???That car over there belongs to the very {person} whose attachment to which is so well-known.\z}

\ea\label{ex:kayne:10}
  ?That car over there belongs to my old friend John Smith, whose long-standing attachment to which is well-known to all his friends.
\z

Here, \textit{who(se)} is related to the head of the \isi{relative} \textit{my old friend John Smith}, but \textit{which} is not; rather, \textit{which} is related to the subject of the matrix sentence, \textit{that car over there}.

     As in \REF{ex:kayne:8}-\REF{ex:kayne:9}, sentences like \REF{ex:kayne:10} show ECP-like effects.  These can be detected by comparing the following two examples. The first is:

\ea\label{ex:kayne:11}
   ??That car over there belongs to my old friend John Smith, whose long-standing desire for me to buy which is well-known to all his friends.
\z

Although more marginal than \REF{ex:kayne:10}, \REF{ex:kayne:11} nonetheless contrasts sharply with:

\ea\label{ex:kayne:12}
   *That car over there belongs to my old friend John Smith, whose long-standing desire for which to be sold quickly is well-known to all his friends.
\z

Replacing the embedded \isi{infinitive} following \textit{desire} with a finite\is{finiteness} sentence results in an appreciable drop in acceptability, but the contrast remains clear:

\ea\label{ex:kayne:13}
   ???That car over there belongs to my old friend John Smith, whose long-standing desire that I buy which is well-known to all his friends.
\z

\ea\label{ex:kayne:14}
   **That car over there belongs to my old friend John Smith, whose long-standing desire that which be sold quickly is well-known to all his friends.
\z

It seems clear that the extra deviance of \REF{ex:kayne:12} and \REF{ex:kayne:14}, as compared with \REF{ex:kayne:11} and \REF{ex:kayne:13}, is akin to the greater difficulty that holds in a general way for \isi{extraction} of or from within subjects as compared with \isi{extraction} of or from within objects.

\section{Steps toward an analysis}

     The raising approach to ordinary \isi{relative} clauses, when extended to cover \isi{relative} pronouns, leads one to take what we call \isi{relative} pronouns to come about as the result of stranding a particular kind of \isi{determiner}.\footnote{Various details are discussed in \citet{Kayne2008a,Kayne2010a}.}  For example, a head + \isi{relative} clause structure such as:

\ea\label{ex:kayne:15}
  books which I’ve read
\z

will have a derivation that looks like:\footnote{I abstract away from questions concerning the ``outside'' \isi{determiner}, for example \textit{the} in:
   \ea  the books which I’ve read\z
For relevant discussion, see \citet{Leu2014}.}

\ea\label{ex:kayne:16}
   I’ve read which books  →  \isi{wh-movement} 

        which books I’ve read <which books>  →  raising of NP to ‘head’ position, stranding the \isi{relative} \isi{determiner} \textit{which}

        books which <books> I’ve read <which books>
\z 

     The convenient informal term ‘\isi{relative} pronoun’, then, is usually to be understood as short for ‘\isi{determiner} occuring within a \isi{relative} clause and stranded by movement of its associated NP to the position of the ‘head’ of the \isi{relative}’.\footnote{Alongside \isi{relative} \textit{who} there is no *\textit{who person}.  Possibly, \textit{who = wh- + -o}, with the latter a noun, thinking of \citet{Bernstein1993} on Spanish \textit{uno}.  Alternatively, \textit{who} is a \isi{determiner} and there is a link to *\textit{mine book} (cf. \citet{BernsteinTortora2005}) and/or to \ili{French} \textit{Lequel (*livre) veux-tu}? (‘the-which (book) want-you) (cf. \citealt{Kayne2008b}) and other cases of the same sort.}   Let me call the movement operation that strands \textit{which} in the last pair of lines in \REF{ex:kayne:16} ‘\isi{relative} pronoun stranding’, henceforth abbreviated as RPS.

     It seems natural, however, to also take the \textit{which} of (\ref{ex:kayne:10}--\ref{ex:kayne:14}) to be a \isi{relative} pronoun (in almost exactly the same sense), despite the unusual position of its antecedent.  This is supported by the fact that it is also possible to find examples of such unusual relatives in which the unusual \isi{relative} pronoun is \textit{who(m)}:

\ea\label{ex:kayne:17}
   ?My old friend Mary Jones is still unaware of yesterday’s discovery, the capacity of which to surprise whom cannot be exaggerated.
\z

In \REF{ex:kayne:17}, \textit{which} is related to the nearby ‘head’ \textit{yesterday’s discovery} in a familiar way, whereas \textit{whom} is related not to that head, but rather to the matrix subject \textit{my old friend Mary Jones}.

      To say that the \textit{which} of (\ref{ex:kayne:10}--\ref{ex:kayne:14}) and the \textit{whom} of \REF{ex:kayne:17} are \isi{relative} pronouns is to say, then, that they have been stranded by RPS, despite the fact that the antecedent in {question} is not the head of the \isi{relative}.  Put another way, in (\ref{ex:kayne:10}--\ref{ex:kayne:14}) and in \REF{ex:kayne:17} RPS has moved the NP associated with \textit{which} and \textit{whom} to the position of matrix subject, hence out of the \isi{relative} clause entirely.

     That RPS can apply out of a \isi{relative} clause might seem surprising, but the difficulty of \isi{extraction} out of a \isi{relative} clause is often exaggerated.  For a detailed survey, see \citet{Cinque2010complexnp}.  To his examples of extractions leaving a gap might well be added, thinking back to \citet{Ross1967},\footnote{Cf. \citet{Boeckx2001,Boeckx2003}{}.  \citet{Kayne2002} extends this tradition to all pronouns, even those with antecedents in A-positions.

     That (resumptive) pronouns may reflect movement is not taken into account by \citet{Bošković2015}.} examples in which the \isi{extraction} leaves behind a resumptive pronoun.

     For all of (\ref{ex:kayne:10}--\ref{ex:kayne:14}) and \REF{ex:kayne:17} the {question} arises as to what precisely has been moved.  RPS may perhaps be moving a full DP in such examples, rather than a NP.  Alternatively, RPS may be moving just NP, in a more familiar way, if sideways movement is allowed.\footnote{On sideways movement, see \citet{BobaljikBrown1997} and \citet{Nunes2001}.}

     That the \textit{which} of (\ref{ex:kayne:10}--\ref{ex:kayne:14}) and the \textit{whom} of \REF{ex:kayne:17} are \isi{relative} pronouns (and not just pronouns) is also suggested by the following considerations.  Sentences like (\ref{ex:kayne:10}--\ref{ex:kayne:14}) and \REF{ex:kayne:17} require that \textit{which} or \textit{whom} be pied-piped as part of a phrase containing the other (ordinary) \isi{relative} pronoun.  This is shown by the contrast between \REF{ex:kayne:17}, for example, and the unacceptable:

\ea\label{ex:kayne:18}
   *My old friend Mary Jones is still unaware of yesterday’s discovery, which will definitely surprise who(m).
\z

The pied-piping in \REF{ex:kayne:17} now recalls the pied-piping of ordinary \isi{relative} pronouns seen in:

\ea\label{ex:kayne:19}
   the book the first chapter of which is being widely discussed.
\z

That the \textit{which} of (\ref{ex:kayne:10}--\ref{ex:kayne:14}) and the \textit{whom} of \REF{ex:kayne:17} are not just ordinary pronouns is shown by:

\ea\label{ex:kayne:20}
   *My old friend Mary Jones is still unaware of yesterday’s discovery, even though it’s very likely to surprise who(m).
\z

     As a final point to this squib, we can note that the ‘head’ of the \isi{relative} cannot be ‘skipped’ entirely (even if the \isi{relative} contains a resumptive pronoun linked to it):

\ea\label{ex:kayne:21}
   **That car over there belongs to my old friend John Smith, a picture of which shows how tall he is.
\z

This may be due to a requirement that the head of a \isi{relative} clause must in all cases originate together with some \isi{relative} pronoun (and that in \REF{ex:kayne:21} there is no option for a silent \isi{relative} pronoun).

\section{Conclusion}

     Relative clauses can be found that contain a \isi{relative} pronoun whose antecedent is not the head of the \isi{relative}.  The familiar relation between the head of a \isi{relative} and the \isi{relative} pronoun can thus be seen as a special (even if overwhelmingly frequent\footnote{In languages that have \isi{relative} pronouns.  For a proposal on why prenominal relatives lack \isi{relative} pronouns, see \citet[chap. 9]{Kayne1994}.}) case of a more general relation between a \isi{relative} pronoun (a stranded \isi{determiner}) and its antecedent (whose movement has stranded that \isi{determiner}).  The piece of \isi{relative} clause syntax that is the antecedent--\isi{relative} pronoun relation is less specific to \isi{relative} clauses that it might have seemed.

\section*{Acknowledgments}
An earlier version of this squib was presented (as part of a longer talk on {relative} pronouns) in June, 2010 at the Comparative Germanic Syntax Workshop, University of Tromsø and at the Workshop: `Adjectives and Relative Clauses: Syntax and Semantics', University of Venice; in October 2010 at Rencontres d'Automne en Linguistique Formelle: Langage, Langues et Cognition, University of Paris 8; in May, 2011 at the University of Poitiers, at the Linguistics Institute, Academy of Sciences, Budapest, and at the University of Bucharest; in June, 2011 at the University of Vienna; and in October, 2011 at Leiden University.  I am grateful to all those audiences, as well as to two anonymous reviewers of an earlier version of this paper, for useful questions and comments.

{\sloppy
\printbibliography[heading=subbibliography,notkeyword=this]
}
\end{document}