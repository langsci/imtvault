\documentclass[output=paper]{LSP/langsci}

\title{The common syntax of deixis and affirmation}
\author{George Tsoulas \affiliation{University of York}}
\abstract{This paper pursues a formal analysis of the idea that affirmative answers to Yes/No questions correspond to a sort of propositional deixis whereby the relevant proposition is pointed at.  The empirical case involves an analysis of the deictic particle \na in \ili{Greek} and a comparison of its syntax with that of the affirmative particle \nai.  It is shown that both involve an extra head which in the case of the deictic particle is uniformly externalised as the pointing \isi{gesture}.  It is argued that gestural externalisation of syntactic structure should be considered on a par with phonetic externalisation (not only in sign languages). The grammar of the affirmative particle gives us also an account of the observed facts about \ili{Greek} whereby both the truth and the polarity answering system appear to coexist.}

\ChapterDOI{10.5281/zenodo.1117722}
\maketitle  
\begin{document}

\section{Introduction} 


\citet{holmberg:15} begins thus: `It is certainly not obvious that expressions like \textit{Yes} and \textit{No} have syntactic structure.'
It is even less obvious that elements like  \textit{Yes} and 
\textit{No} have \textit{complex} internal syntactic and semantic\is{semantics} structure.  In the literature on the \isi{semantics} of Yes/No questions an explicit \isi{semantics} for \textit{Yes} is rarely given. \citet{groenedijk-stokhof:84} is one of these exceptions and their \isi{semantics} is given in \Next:

\Lsciex.
\dbr{yes}=λp p(a)


\citeposs{groenedijk-stokhof:84} syntactic assumption is that \textit{Yes} and \textit{No} are sentential adverbs of type S/S.  It would then seem that there is not much of interest that either the \isi{semantics} or the implied syntax would give us.  In this paper I will take Holmberg's stance and try to show that interesting insights and conclusions can follow from pursuing the non-obvious. Yes/No questions in \ili{Greek} can receive either a verb-echo answer \ref{vecho} or a particle answer \ref{particle-answer}:

\Lsciex.
\ag.  Petai o gaidaros?\\
 flies the.\textsc{nom}  donkey\\
\glt `Do  donkeys fly?'
\bg. Petai\\
     flies\\ \label{vecho}
\cg. Ne\\
Yes\\ \label{particle-answer}
     

Our focus in this paper will be on particle answers only and verb echo answers will not be considered.   \citeposs{holmberg:15} proposal for the particles \textit{Yes} and \textit{No} is that they are the spell-out values of a focused \isi{polarity} variable.  If this is so, and given that languages usually have small unanalysable particles for this function, it makes little sense to ask why these particles take the form that they do.  They just do.  But now imagine that there is a language where the affirmative particle is, if not immediately transparently complex, at least arguably so.  Then it does make sense to ask why it is this, rather than a different complex form that has this meaning and  function.  Furthermore, if the particle is indeed complex, the \isi{question} of its internal syntax and compositional \isi{semantics} justifiably arises over and above that of its external distribution.  This seems correct, but is there such a language?  In this paper I will argue that \ili{Greek}, at least concerning the affirmative particle \nai (Yes), corresponds quite precisely to the above description and, therefore, gives us a very good opportunity to formulate and explore questions that may lead to a better understanding of affirmation.

My  ultimate goal here is to understand the affirmation particle \nai in \ili{Greek}.  Anecdotally, speakers of Indo-European languages are often surprised not only because the way to say \textit{Yes} in \ili{Greek} resembles more the way to say \textit{No} in other languages but also that the language does not use the \textit{-n-} element in negatives.  \ili{Greek} \textit{n}-words have no /\textit{n}/ in them.

To understand this particle, however, we will have to take a somewhat circuitous route starting from the properties and analysis of the deictic\is{deixis/deictic} (or presentational) particle \textit{N\`{a}}.  \textit{N\`{a}} and \nai share the initial element \textit{N-} and the hypothesis that I will explore is that this is not an accident.  In other words, deixis\is{deixis/deictic} and affirmation have a common core.  I contend here that understanding what I call \textit{N-} deixis\is{deixis/deictic} leads us to a particular understanding of affirmation as essentially a sub-case of deixis\is{deixis/deictic}, namely propositional deixis\is{deixis/deictic}.
The paper is structured as follows.  \sectref{sec2} begins with two apparently unrelated observations regarding, on the one hand, an intuitive understanding of what it means to answer a Yes/No \isi{question} and, on the other, an observation regarding the (possible) origin of the word \nai in \ili{Greek}.  \sectref{sec3} consists of a short primer on \ili{Greek} particles focusing mainly on an observation from \citet{tsoulas:15b} on the meaning of the endings of two classes of particles (speaker and addressee oriented particles respectively).
\sectref{sec4} is an analysis of the deictic  particle \textit{N\`{a}} which relates its syntax and \isi{semantics} directly to the required presence of a pointing \isi{gesture}.  An extension of the proposal to other gestural elements is also discussed.
Having established the syntax and \isi{semantics} of \textit{N\`{a}}, \sectref{sec5} applies the same principles  to the affirmation particle \nai using the analysis of \citet{holmberg:15} as point of departure.


\section{Two apparently unrelated observations}
\label{sec2}
To make the argument that I want to make here, I will start with two seemingly unrelated observations.  The first is a generally offered intuitive and informal description of what answering a Yes/No \isi{question} amounts to.  I will again {borrow} this intuitive description from \citet{holmberg:15} who writes:

\Lsciex.
\begin{quote}
  [\ldots] The answer provides a value for the variable in the \isi{question}, and thereby indicates which of the two disjunctive propositions posed by the \isi{question} the respondent \textit{\textbf{presents as being true}} [Emphasis mine, GT]. 
\end{quote}


The first part of the above quote will be important later on in this paper in \sectref{sec5}.  It is the emphasised part that I would like to draw attention to for now.  Although it is an informal way of describing what answering a Yes/No \isi{question} amounts to, it can be taken to express an important intuition regarding the formal relationship between \textit{the respondent} and the relevant proposition.

The second observation is an etymological one.  The \ili{Greek} word for \textit{Yes}, i.e. \nai, has a somewhat mysterious etymology.  It is already found in Homer and is a very common marker of affirmation and agreement in Plato's dialogues as well as the major tragic poets of the 5th century BCE, and it is also found frequently in the New Testament and in Medieval texts.  So it seems that this particle was part of \ili{Greek} from the start though it is trickier to establish its origin with certainty.  There is, however, general agreement between Indo-Europeanists that it incorporates the Proto-Indo-European element *\textit{n}  whose function is deictic\is{deixis/deictic}.\footnote{There is a vast literature on deictic\is{deixis/deictic} and \isi{demonstrative} pronouns in (proto)-Indo-European and their uses as elements of inflection.  \citet{brugman:04,brugman:11} are classic references.  \citet{shields:92} and references therein provide further context.}  There is an obvious intuitive connection here: if answering a Yes/No \isi{question} amounts to presenting or pointing at a proposition (or its truth-value) then it is not unexpected that elements with a deictic\is{deixis/deictic} function appear in the formation of the Yes/No particles.
In other words, the speech act that the speaker performs by answering a Yes/No \isi{question} amounts to, or is at least related to, a kind of propositional deixis\is{deixis/deictic}, as if in order to say \textit{Yes} one had to point at the relevant proposition and state that it is true by providing the relevant \isi{polarity} value (which, of course, is positive in the case of \textit{Yes}).  And this is what the \ili{Greek} case shows more clearly.  If the above connection remains merely an (informed) intuition it is not of great value.   I contend here that it can be cashed out in formal structural and semantic\is{semantics} terms.  To see this we need to start with a short primer on \ili{Greek} particles.

\section{Greek particles: A primer}
\label{sec3}
\ili{Greek} has a large {number} of particles of different types and functions.  \citeposs{tsoulas:15b} study of the higher field particles in \ili{Greek} shows that so-called discourse particles in \ili{Greek}, though small and monosyllabic, are consistently complex elements that are made up of (at least) two heads, one that encodes anchoring and perspective and another that encodes attitude/evidentiality.  The most relevant observation for the purposes of the present work is that the elements \textit{E} and \textit{A}, while particles in their own right, as in \ref{a1}--\ref{e1}, also combine with other elements to create complex particles \ref{ea1}, \ref{ea2}\footnote{To avoid unnecessarily complicating the glosses, and when there is no possibility of confusion, I will simply gloss the particles \textsc{prt}.}:

\Lsciex.  \label{a1}
\ag. A, irthe o Kostas\\
\textsc{prt} came.3rd.\textsc{sg} the Kostas\\ 
\glt `I see, Kostas has arrived.'
\bg. A, o giannis agorase aftokinito\\
\textsc{prt} the Giannis bought car\\ \label{esi}
\glt `I see, Giannis bought a car.'

\Lsciex. \label{e1}
\ag. E, irthe o Kostas\\
\textsc{prt} came.3rd.\textsc{sg} the Kostas\\ 
\glt `As you know/would expect, Kostas has arrived.'
\bg. E, agorase o Giannis aftokinito\\
\textsc{prt} bought the Giannis car\\ \label{esf}
\glt `As you know/would expect, Giannis bought a car.'


\Lsciex. \label{ea1}
E-series\\
Re, De, Vre etc...

\Lsciex. \label{ea2}
A-series\\
Na/Nà, $\delta$a, Ba etc...



The meanings of the particles in \ref{ea1}--\ref{ea2} are complex and difficult to describe.\footnote{\citeposs{blakemore:02} term for the difficulty in formulating descriptively the meanings of discourse particles is \textit{descriptive ineffability}.  Speakers mostly provide contexts where the particles are felicitous in order to explain their meanings.}  Except for \na (to which we return) they are not directly relevant to the present paper, and an illustration will be given shortly.
The point to retain from this is the following:\footnote{See \citet{tsoulas:15b} for more details.}  the particles in the \textit{E} series are what I will call addressee-anchored particles while those in the \textit{A}-series are speaker-anchored ones.  We will elaborate further on the notion of anchoring later on. Comparing the distribution of the different particles, \citet{tsoulas:15b} observes that the complex particles of the \textit{E}-series have a wider distribution than the bare particle \textit{E}, which is restricted to the sentence initial and sentence final position.\footnote{\label{footslift}In fact, \citet{tsoulas:15b} suggests that only the sentence initial position is available to these particles and the sentence final one results from slifting of the clausal complement of the particle.}  In other words, complex \textit{E}-particles can appear at various positions inside the sentence.\footnote{With some restrictions for some particles which can be derived from their meanings.}  Interestingly, this is not true of the \textit{A}-series particles, which remain restricted in their distribution to sentence initial position.  From these facts, the conclusion is that the two series of particles have different syntactic structures.  The \textit{E}-series is headed by the evidential/attitudinal morpheme, which allows and accounts for their wider distribution, while the \textit{A}-series is headed by what we called the \textit{anchor}.\largerpage  The two structures are as follows:


\Lsciex. E-series\label{structure1}

% \begin{center}
\begin{forest}
[Evid/AttitudeP [X [X,]
[-E, name=E]] [AnchorP [anchor [-e, draw]
{\draw[->] () to[out=south west,in=south] (E);
}
]]]
\end{forest}
% \end{center}

\Lsciex. A-series\footnote{We will return shortly to the nature of the element noted as Y in the structure.} \label{structure2}

% \begin{center}
\begin{forest}
  [AnchorP [Anchor$^0$[Y, name=E] [A]][YP [Y, draw ]{\draw[->] () to[out=south west,in=south] (E);
} [XP[\hspace*{1cm}, roof]]]]
\end{forest}
% \end{center}


With this in mind, let us illustrate with one example from the list of particles in \ref{ea2} the general approach. Consider the particle \textit{De}.  This particle only appears in sentence final position.  Deriving this restriction would take us too far afield but the final structure will look similar to the following:\footnote{I also set aside here the {question} whether the CP\is{complementizer} has moved from a lower position.}

\Lsciex.
% \begin{center}
\begin{forest}
[AttitudeP [CP\is{complementizer}] [\isi{Attitude} [D [D,]
[-E, name=E]] [AnchorP [anchor [-e, draw]
{\draw[->] () to[out=south west,in=south] (E);
}
]]]]
\end{forest}
% \end{center}

The attitude head relates a proposition to an anchor.   The semantic\is{semantics} \isi{question}, given that \textit{e} is the anchor is the content of the attitude expressed by \textit{D}.  To formulate this the following examples will help:

\Lsciex.
\ag. Graps-to De.\\
write-it \textsc{prt}\\ \label{de1}
\glt `Write it (for goodness sake).'
\bg.
To vlepo De.\\
it see.1.\textsc{sg} \textsc{prt}\\ \label{de2}
\glt `I see it.'


As can be seen from the above example the particle can follow imperatives as well as declaratives -- therefore the notion of \textit{proposition} should be understood broadly. The attitude is one of exasperation.  \ref{de1} is felicitous in a context where the addressee has perhaps been talking repeatedly about writing something but never does.  \ref{de1} is then an expression of the speaker's exasperation with the addressee's failure to do the relevant writing.  A similar description applies to \ref{de2} where the addressee has repeatedly drawn the attention of the speaker to a particular object.  Although formalising precisely these notions remains to be done, this example can serve as a general illustration of the relevant types of meaning.

Now if the above is on the right track, it seems reasonable, or at least possible, to identify the \textit{-e} ending of the \textit{Yes} particle \nai with the addressee oriented element seen above.  But what about the \textit{N} part?  It may be that there was a deictic\is{deixis/deictic} element \textit{N} in Proto-Indo-European but what is the evidence, if any, in the Modern version of \ili{Greek}?  In the next section I show that the evidence is rather strong and that the analysis leads us to unexpected considerations.


\section{\textit{N-} deixis in Greek}
\label{sec4}
The particle that is of immediate interest here is the particle \textit{N\`{a}}.  There are two versions of this particle, one that is a \is{Modal}modal particle and marks the subjunctive \ref{subjunctive}, and one that is a deictic\is{deixis/deictic} particle \ref{na1}.

\exg. 
I Maria theli na fai.\\
The.\textsc{nom} Maria wants \textsc{sbjv} eat\\ \label{subjunctive}
\glt `Maria wants to eat.'

 Although the two may be related and perhaps, ultimately the same particle\footnote{This is the claim made by \citet{christidis:90}.  We will return to his account, though not the issue of whether there is one or two different elements \textit{N\`{a}}, which is not directly relevant to our concerns here.} we will set aside for the purposes of this discussion the \is{Modal}modal particle and concentrate on the deictic\is{deixis/deictic} version.  Deictic\is{deixis/deictic} uses of \textit{N\`{a}} are accompanied by a pointing \isi{gesture}:

\exg.
\na $\ $  \xspace\Pointinghand $\ $ o Giannis\\
\na $\ $ \xspace\Pointinghand $\ $ the Giannis\\ \label{na1}
\glt `There is Giannis'


The orientation of the deixis\is{deixis/deictic} is always with respect to the speaker, since it is the speaker who actually gestures towards the thing that is pointed at.  This is confirmed by the fact that this particle contains the speaker anchoring morpheme \textit{-a} that we saw in the previous section.  Let me set aside the {question} of the pointing \isi{gesture} and return to it in \sectref{sec:gesture}.  A closer look at the properties of the deictic\is{deixis/deictic} particle \na reveals an interesting set of properties and is necessary in order to substantiate the claim that particles have complex structure.  I will focus here on three aspects of the grammar of this particle, namely the \isi{Case} patterns of the DPs following the particle, ethical datives and \isi{Person Case Constraint} effects, and the plural \isi{agreement} that is manifested on the particle in some dialects.

\subsection{Case patterns}
To begin with, as \citet{Tzartzanos:46} has observed, the DP following the particle can surface in either nominative or accusative.  Note, however, that in the case of the accusative, a \isi{clitic} pronoun must accompany the particle:\footnote{\label{jofoot}\citet{joseph:81} claims that \ref{na-acc} is grammatical without the \isi{clitic} pronoun though he acknowledges that some speakers reject it.  In my dialect \na+accusative is completely ungrammatical.   These judgements are shared by all those speakers I asked too.  There may be dialects where \na+Acc is grammatical.  It is, however, unclear whether any significant conclusion can be drawn from that fact.}


\exg.
\na *(ton) ton Kosta.\\
\na him.\textsc{acc} the.\textsc{acc} Kosta.\textsc{acc}\\ \label{na-acc}
\glt `Here is Kostas.'

\exg.
\na (tos) o Kostas.\\
\na he.\textsc{nom} the.\textsc{nom} Kostas.\textsc{nom}\\\label{na-nom}
\glt `Here is Kostas.'

It is also possible that the DP following the accusative \isi{clitic} is in the nominative:

\exg.
\na ton o Kostas.\\
\na him.\textsc{acc} the.\textsc{nom} Kostas.\textsc{nom}\\ \label{na-acc-nom}
\glt `Here is Kostas.'


The origin of the case marking here is unclear.  An ellipsis-based account whereby the DP is the object or subject of some verb that has been deleted immediately suggests itself. Unfortunately, there appears to be little justification for postulating an elided sentence here.  More importantly, sentences like  \ref{na-acc-nom} above seem to militate openly against such an account given that not only would there be no source for the nominative on the overt DP but even if we accepted that it surfaces in some sort of default case\footnote{And it seems appropriate to think that if there is a default case in \ili{Greek} it would be the nominative.} as we eventually might have to do, the problem is that the overt counterpart of \ref{na-acc-nom} is generally ungrammatical.  By generally, I mean that with some verbs, a nominative DP co-referential with the \isi{clitic} can appear in the post-verbal position as an apposition after a markedly long pause, which suggests that these cases are indeed examples of elliptical constructions where a T level constituent has been omitted.  It is significant that there should be a pause in these cases especially as the pause is not required in the cases with \textit{N\`{a}}:

\exg.
??/*Kita ton, o Kostas.\\
Look him the.\textsc{nom} Kostas.\textsc{nom}\\
\glt `Look at him, Kostas.'

But with other verbs this is impossible:

\exg.
*Pare ton [\ldots]  o ipologistis.\\
Take him [\ldots] the.\textsc{nom} computer.\textsc{nom}\\
\glt `Take the computer.'

Note that \Last with  the deictic\is{deixis/deictic} particle and an understood\footnote{By \textit{Understood} I mean roughly accommodated. There is no suggestion here that there is a verb that has been deleted/left unpronounced.} \textit{take} is perfectly fine in the \textsc{acc}-\textsc{nom} pattern:

\exg.
\na  ton o ipologistis.\\
here him the.\textsc{nom} ipologistis.\textsc{nom}\\
\glt `Here is the computer.'

This suggests that if an ellipsis account were the right approach then there ought to be some way, grammatical or contextual, to ensure that the right verb is chosen.  But there is no such way, at least none that I can think of. The content of the putative elided predicate cannot readily be recovered (though guesses can be made).  Furthermore, there can be follow ups with further specification of the intended predicate which could not have been the origin of the elided material as the overt counterpart is ungrammatical, as in \Next-\NNext respectively:

\Lsciex.
\na ton o ipologistis \ldots  hrisimopiise/katharise/spase/kan' ton (oti thelis.)\\
here him the.\textsc{nom} computer.\textsc{nom} \ldots use/clean/break/do him (what want-you)
\glt `Here is the computer, use it, clean it, break it do whatever you want to it.'

\exg.
*Hrisimopiise/etc ton o ipologistis.\\
Use him the.\textsc{nom} computer.\textsc{nom}\\
\glt `Use the computer.'


It therefore appears that the ellipsis account is not \textit{prima facie} at least a viable one.\footnote{A significantly different variant of the ellipsis account is \citet{joseph:81}.  We return to his account in \sectref{sec:lit}.}  

\subsection{Ethical datives and Person-Case Constraint effects}
Another interesting property of these constructions is that in certain contexts, mostly narrative, an ethical dative \isi{clitic} (\textit{su}) can appear before the DP or accusative \isi{clitic}:

\exg.
Kai opos strivo stin Kalidromiu na su mia kluva me MAT\\
And as turn.1\textsc{sg} in Kalidromiu here 2\textsc{sg}.\textsc{dat} one bus with MAT\\
\glt `And as I turn into Kalidromiu street there is a bus full of riot police.'


Interestingly in the presence of the ethical dative the \isi{clitic} cluster is subject to the \isi{Person Case Constraint} (PCC\is{Person Case Constraint}).  At least in my \ili{Greek} these \isi{clitic} clusters are subject to the strong version of the PCC\is{Person Case Constraint} (the one that bans all first/second \isi{person} direct object clitics if any dative \isi{clitic} is present):\footnote{On ethical datives in \ili{Greek} and more generally, see \citet{michelioudakis-kapogianni:13} and references therein.  It would be interesting to juxtapose the ideas in this paper with the analysis in \citet{michelioudakis-kapogianni:13}. Unfortunately, this will have to be left for another occasion.}

\Lsciex.
\ag.
*\na su me\\
\na 2.\textsc{dat} 1.\textsc{acc}\\
\bg.
*\na me su\\
\na 1.\textsc{acc} 2.\textsc{dat}\\
\cg.
*\na se mu\\
\na 2.\textsc{acc} 1.\textsc{dat}\\
\dg.
*\na mu se\\
\na 1.\textsc{dat} 2.\textsc{acc}\\


I take this as another indication of more complex covert structure.  

\subsection{Plural agreement}
The final property of this particle that we will mention here is that in certain dialects of \ili{Greek} the deictic\is{deixis/deictic} particle \textit{N\`{a}} shows \isi{person} and \isi{number} \isi{agreement}.\footnote{Although I have not been able to check in many dialects, the plural versions of the particle are certainly found in Cretan \ili{Greek} and in North-Western dialects (Epirus).  It can be found in texts and transcripts of folktales from Epirus and it is very common in Cretan \ili{Greek} as well as -- seemingly at least -- other Island varieties. This form of the particle first appeared in Medieval Greek}
In the dialects that have it, the plural version of the particle is \textit{\nate}:

\exg.
\nate ta pedia sas.\\
Here.2\textsc{pl} the.\textsc{pl} children yours\\
\glt `Here are your children.'

\exg.
\nate enan para ke min ton skotosete.\\
Here.2\textsc{pl} one coin and not him kill\\
\glt `Here you all, take some money and don't kill him.'


In general, no other particles of this type show this sort of \isi{agreement}.  There is, however, another particle, \textit{Ade}, meaning roughly \textit{Go} when followed by a second \isi{person} verb in the imperative:

\Lsciex.
\ag. 
Ade na vrite to Gianni.\\
Go \textsc{sbjv} find the Gianni\\
\glt `Go find Gianni.'
\bg. Ade gamisou.\\
Go fuck.yourself\\
\glt `Go fuck yourself.'


When followed by a first \isi{person} verb the meaning is more complex.  In certain cases it can mean roughly \textit{Let's}:

\exg.
Ade na pigenoume.\\
Let's \textsc{sbjv} go\\
\glt `Let's go.'

It also conveys the meaning that it is difficult (for whatever reason) to do what is described in the proposition:

\exg.
Ade na vroume trapezi tetia ora.\\
\textsc{prt} \textsc{sbjv} find table such hour\\
\glt  `Go find a table at this hour.' (meaning: `It is virtually impossible.')

\exg.
Ade parkare sto kentro.\\
\textsc{prt} park in.the centre\\
\glt `Go find a parking place in the centre...' (meaning: `It is virtually impossible.')


Now, in the same dialects as those that show a plural form of \textit{N\`{a}}, \textit{Ade} also has a plural form \textit{adeste}:\footnote{The meaning of the plural here is somewhat unclear in the sense that judgements vary on whether the plurality in {question} includes the speaker or not.  I want to thank Anna Roussou for giving me this example and also Evi Sifaki and Stella Gryllia for discussing their intuitions on the interpretation of these plurals.}


\exg.
Adeste tora pame.\\
Adeste now go.1\textsc{pl}\\
\glt `Come on now, let's go.'

Again, the origin of this \isi{agreement} remains unclear.  It would be difficult to incorporate it into an ellipsis account as it would require us to accept that while the whole verb has been elided the \isi{agreement} ending would somehow stay and stick to the particle.

Having said that, accounting for the presence of \isi{agreement} on these particles is not straightforward in the model presented here either.  Assuming that the anchor head can optionally carry a [+PL] feature is descriptively adequate but no more.  This idea is also generally in line with a suggestion made by an anonymous reviewer for this volume.  The reviewer suggests that in these cases we may be dealing with something akin to allocutive \isi{agreement}.  If this is so, it makes the argument against an ellipsis account a little weaker as the \isi{agreement} is not the one that is found on the verb.  In other words while there is ellipsis, the \isi{agreement} is independent from what we find on the verb.  While this is an interesting possibility it does not rescue the ellipsis account from the earlier objections.  This suggestion, of course, faces the same difficulties.  There is no reason why this \isi{agreement} would appear only with these particles.  I will leave this issue open for further research at this point.  More needs to be discovered about the \isi{agreement} patterns in the relevant dialects before a more convincing account can be developed.


To summarise, we have seen that the deictic\is{deixis/deictic} particle shows properties that would push us to associate it with a larger structure and yet as far as we can tell an account that is based on mere ellipsis of a larger, fully clausal, structure seems unwarranted and unsupported by the evidence. 

\subsection{Two earlier proposals} \label{sec:lit}
The issues surrounding the  particle \textit{N\`{a}} have been the focus of some attention in the literature. \citet{joseph:81} and \citet{christidis:90} are the most complete accounts. The two accounts differ sharply but from the perspective pursued here, they both contain valid insights and intuitions. Both \citet{joseph:81} and \citet{christidis:90} are concerned with the proper categorisation of \textit{N\`{a}}.   I will briefly present their accounts below.

\citeposs{joseph:81} careful study  considers a {number} of issues regarding the status of deictic\is{deixis/deictic} \textit{N\`{a}}. His central claim is that \textit{N\`{a}} is a verb and more specifically a non-finite\is{finiteness} imperative form of a verb meaning roughly \textit{look} or \textit{take}.  This analysis allows us to understand the presence of an accusative (in  \isi{clitic} form or bare, see footnote \ref{jofoot}) after the particle as well as the fact that it does not appear before a verb. It also affords an understanding of the plural \isi{agreement} that appears dialectally on the particle.  At the same time, the analysis runs into problems (as Joseph himself observes) in the cases where the particle is followed by a nominative which would have no source.  He offers a view according to which this is the result of reanalysis that is mainly due to the case ambiguity found in \ili{Greek} with neuter nouns (where nominative and accusative are not differentiated).  The result is that there is a finite\is{finiteness} version of the verb \textit{N\`{a}}  which takes the nominative DP as its subject, deriving from an abstract underlying \textit{Here comes DP$_{NOM}$}.

Although ultimately I disagree that \na itself is a verb in the sense that it carries a category determining  V feature, I think that Joseph's intuition that there are two types of deictic\is{deixis/deictic} \na and that the way to capture the difference is by appealing to something predicative is correct.  My general implementation will differ greatly though.

Another aspect of \citet{joseph:81} is the discussion of the etymology of the particle.  He discusses what he calls the generally accepted etymology \citep[due to][]{hadzidakis:05} and which he describes as follows:

\Lsciex.
\begin{quote}
 [\ldots]  The Classical \ili{Greek} form \textit{\={e}n\`{i}de} `see there!', composed of the interjection \textit{\={e}n} `see there' plus the imperatival form \textit{\'{i}de} `see!' was reanalyzed as \textit{\={e}n\'{i}} plus \textit{de}, with the result that a new form \textit{\={e}n\'{i}} was abstracted from  \textit{\={e}n\'ide}.  Then by the substitution of the final \textit{-a} of adverbs [\ldots] and other particles (as in \textit{δά} [\ldots]) for the \textit{ί}, along with the regular aphaeresis of the unstressed initial vowel \textit{\=e-}, the form \textit{nά} arose. \citep[p.141]{joseph:81}
\end{quote}

Quite rightly perhaps, Joseph notes that this is a rather involved etymology for a very simple word and that, most importantly `It is not at all clear why a particle like \textit{\={e}ni} (or even \textit{d\'e} for that matter) would be influenced by the form of adverbial elements like  \textit{k\'alista} `very well',  \textit{katak\'efala} `on the head',  \textit{akóma} [`more'], and so forth'.

Instead, Joseph proposes that \textit{N\`{a}} is a \isi{borrowing} from South \ili{Slavic} where similar elements are found.  What the ultimate truth about the etymology of deictic\is{deixis/deictic} \textit{N\`{a}} is I don't know.  However, if we assume that the \textit{a} that was substituted for \textit{ί} is that of the speaker anchoring particle \textit{a}, which is both independently found in the language and appropriate for the final nature of the particle, as in the structure in \ref{structure2} then the accepted etymology becomes less problematic and more attractive than a \isi{borrowing} from South \ili{Slavic}.
Having said that, not much really turns on the etymology anyway.  I take it that, should it be the case that \na is indeed a \isi{borrowing} from South \ili{Slavic}, \ili{Greek} must have projected on it the morphosyntactic structures existing in the language. 

The etymological issue notwithstanding, \citeposs{joseph:81} account expresses some important insights as already mentioned.

Let me now turn briefly to the account offered by \citet{christidis:90}, who, unlike Joseph, suggests that searching for a category to assign \textit{N\`{a}} to is futile and in the end misleading as the particle resists all categorisation attempts\footnote{He rightly also rejects the view held, albeit rather halfheartedly, by \citet{Householder-Kazazis-Koutsoudas:64} that \textit{N\`{a}} is a \isi{preposition}.} simply because it is just not the sort of element that falls within any of the traditionally recognised categories.  He suggests that this is natural if we assume \citep[following][]{ross:72} that categorial distinctions are elastic and are better understood as a continuum rather than a set of discrete points. He also rejects \citeposs{zwicky:85a} ban on acategorial words.  His analysis of \textit{N\`{a}} makes it a \textit{holophrase}.  Holophrases are syntactically undifferentiated units that often express fully sentential meanings.  To fully understand the idea it  is best to quote at some length:

\Lsciex.
\begin{quote}
  Holophrasis is a term meant to describe linguistic formations where, to use \citegen[26]{halliday-m-hasan:1976a} terminology, the differentiation between the `ideational' and the `interpersonal' components of language is `still' undifferentiated.  The `ideational' component [\ldots] concerned with the expression of content [\ldots] the interpersonal component [\ldots] is concerned with expressing the speaker's angle. [\ldots] The holophrastic nature of \textit{N\`{a}} is a manifestation of an archaic fusion of the interpersonal and ideational component \citep[67]{christidis:90}. 
\end{quote}

It must be said too that while Christidis does offer an account of \na, his main objective is to argue against the views on categorisation championed by \citet{zwicky:85a} and instantiated in a sense in Joseph's work. 
Nonetheless, and despite the fact that his account is couched in very different analytical and theoretical terms, it is clear  that it contains important insights.  Overall it seems that both Joseph and Christidis, despite their differences, see the elliptical or incomplete character of the particle as essential to understanding its nature.  

The {question} of categorisation is rendered rather moot, however, if we adopt a view where traditional categories are mostly epiphenomenal and where  labels on syntactic objects often include none of the traditional \textit{categorial} features.  This  allows us to build a theory that preserves, and eventually analyses away, the important insights of apparent incompleteness, while circumventing the problems of the ellipsis view, whether one holds that the particle is a verb or something altogether different.  I attempt this in the next section. The novelty of the account I develop is that I don't take, like \citet{joseph:81}, \citet{christidis:90}, and others who mention this particle in passing, the pointing \isi{gesture} as a mere optional accompaniments to the particle, nor do I subscribe to Christidis' view that the particle is `the linguistic substitute of the pointing \isi{gesture}'. I  think both these statements are wrong.  The idea is that the \isi{gesture} is a fully integrated part of the syntactic makeup of the particle. 

\subsection{The role of the gesture \xspace\Pointinghand}
\label{sec:gesture}


The proposal I will put forward here is that, as the evidence suggests, there is indeed invisible (or rather inaudible) structure involved in this particle but it is not structure that has been elided.  Rather, the idea is that the central element of the structure that appears as the complement of the particle is the pointing \isi{gesture} itself, notated for convenience simply as \xspace\Pointinghand.   

Thinking about example \ref{na1}, repeated here:

\exg.
\na \xspace\Pointinghand o Giannis\\
\na \xspace\Pointinghand the Giannis\\
\glt `There is Giannis.'

It needs to be emphasised that the pointing \isi{gesture} does not merely
accompany the particle or vice versa in fact. Without the \isi{gesture}
itself the sentence is ungrammatical or at the very least completely
uninterpretable.  Note here that I use the word ``\isi{gesture}" in a general
sense, not confined to hand pointings: it could be a head nod or an
eye movement or something else altogether (we will see another example
shortly).  The crucial point is that it is not omissible. The relation
of the \isi{gesture} to the particle must, however, be elucidated.  First of
all, merely pointing at an object does achieve some effect, albeit a
rather limited one: the object has been pointed at and that's about
it.  Clearly, the {person} perceiving the \isi{gesture} might, and often will,
extract some meaning from it but whatever that meaning is will be
reached via the application of standard Gricean principles (i.e. if
the {person} sitting across from me is winking at me he is either
deranged or \ldots, where the [\dots] part can be filled with
reasoning guided by the Cooperative principle).  The point here is
that while a \isi{gesture} itself can be related to the overall meaning\pagebreak[4] of
the exchange this can only happen through global \isi{pragmatic} principles
rather than local compositional processes.  As a result the object
pointed at is not fully integrated in the
discourse as a discourse referent. \citet{lascarides-stone:09} claim that a \isi{gesture} on its own
is limited in what it can contribute to linguistic meaning through
inference.  They show that while gesturally introduced referents remain
available for the interpretation of subsequent \textit{gestures}, it
is not the same for subsequent discourse. For example, merely pointing
at an object does not suffice to create a discourse referent which
would license subsequent pronominal anaphora.  This is not surprising,
as they point out, since pronominal anaphora require a linguistic
antecedent. \footnote{A reviewer raises an interesting objection at
  this point, namely that in some \isi{pro-drop} languages the \isi{gesture} by
  itself does suffice to create a discourse referent that would be
  available for subsequent pronominal anaphora.  It is clear that more
  research is needed in order to establish the extent to which this is
  true and the specific contexts where it applies, including the
  specific grammatical positions where anaphora may be licensed; it is
  for example conceivable that there is a difference between pronouns
  in subject and object position. But if we assume that the
  observation is correct, within the present analysis we may speculate
  that a pointing to a cup, which may license an \isi{utterance} of
  \textit{wash it}, where \textit{it} refers to the cup, has the
  structure in \ref{cof2} rather than the one in \ref{cof1}:

  
  \Lsciex. \label{cof1} \xspace\Pointinghand   \Coffeecup$_{K}$ 
  

   \Lsciex. \label{cof2} \xspace\Pointinghand   pro$_{K}$


 Given that the relevant languages certainly license \textit{pro} it is not unreasonable to assume that they could take it as an argument. }
  
While the above is true for the \isi{gesture} on its own, interestingly, the particle on its own does not have the required \isi{demonstrative} effect either.  The particle without a \isi{gesture} is ungrammatical/uninterpretable.  The composition of the two has the effect of making the thing that is being pointed at relevant to the current discourse, relating it to the epistemic state of the speaker, and adding it to the common ground.  This may be in contrast to other deictic\is{deixis/deictic} elements which are, apparently, interpretable without a pointing.  Concerning the particle at hand, unsupported (gestural) uses lead to more than just infelicity or too much underspecification.  These uses are as ungrammatical as a transitive verb missing its object. \ref{p1} corresponds to \ref{ad1} and the status of \ref{p2} is analogous to that of \ref{ad2}:


\exg. 
\na \xspace\Pointinghand o Giannis.\\
\na \xspace\Pointinghand the Giannis\\ \label{p1}
\glt `There is Giannis.'


\exg. 
\na o Giannis.\\
\na the Giannis\\  \label{p2}
\glt `There is Giannis.'

\Lsciex. \label{ad1}
John admires Mary.

\Lsciex. \label{ad2}
John admires.


Consider now a \isi{gesture} together with a naming act\footnote{As a reviewer correctly points out, we should ensure that we make the distinction between an Austinian \textit{naming act} \citep{austin-jl:1962b} which corresponds to the use of an example like \Next which provides a new name for something that was nameless prior to the relevant act:
  \Lsciex.
  I name this ship the \textit{Heart of Gold}


  and the \textit{naming act} as used in the text which amounts to supplying the pre-existing name to an entity that one points at and does amount to a presentational statement:

  \Lsciex.
  This is the Nostromo

} 
in a neutral context.\footnote{In this case by ``neutral context" I mean specifically that this is not an answer to a \isi{question}.} This amounts, I think, to a presentational statement, a thetic statement which has little effect on the discourse.\footnote{Indeed these are difficult to integrate in a coherent discourse save for special cases.} Again, it seems that it is the complex [Particle + \isi{Gesture}\is{gesture} + Naming Act] that foregrounds the relevance of the object referred to  to the concerns of the participants. 
To clarify the position I am defending here:  many proposals exist according to which speakers use both language and \isi{gesture} in tandem to construct meaning and ultimately, a single semantic\is{semantics} representation.  
Many, such as \citet{mcneil:05}, \citet{lascarides-stone:09}, \citet{kopp-et-al:04} among others, have suggested that \isi{gesture} is fully integrated with speech.  The position that I take here for the gestures following the particle under discussion is that they are more than just vaguely integrated or just semantically integrated.  
They are in fact the ``pronunciation'' of specific syntactic heads.  In this I differ from some of the authors cited above in that although they assume that coverbal \isi{gesture}  is timed to align with prosodic units and that sometimes it fills a vacant grammatical slot, they do not assume, at least not overtly, that gestures have syntactic reality and syntactic effects.  The strong position that I take in this paper concerning the syntax-\isi{semantics} interface is that elements that contribute to meaning  and are present in the semantic\is{semantics} representation must have some kind of syntactic substance.  This is true of intonation, which contributes to meaning but not, say, palatalisation or other phonetic processes that do not contribute to the construction of a semantic\is{semantics} representation.  This, it goes without saying, leaves completely open the possibility that such processes offer \isi{pragmatic} clues that lead to additional ``meaning'' distinctions, sociolinguistic or other. I am not interested in those here.  
I think the deictic\is{deixis/deictic} gestures accompanying the particles show that they do.  So I would like to take \citeposs{lascarides-stone:09} idea that since \isi{gesture} and language contribute to the construction of meaning, they should be represented in the same logical language one step further and suggest that at least for some gestures they should be represented in the same syntactic representation too. This is in exactly the same spirit as \citeposs{jouitteau:04} proposal that gestures can be expletives filling the \textit{\isi{EPP} position} in Atlantic \ili{French}.  The basic claim here is that at least some gestures are fully grammatical elements.  It would not be appropriate here to talk of grammaticalised gestures.  Rather, certain syntactic heads have a gestural rather than an oral externalisation.  It follows that, under this view, we should not be talking of PF as the relevant interface level but of EF, for \textit{\isi{Externalization}\is{externalization} Form}, which will contain in most cases more\largerpage[2] than just phonetic information.  I will briefly return to this discussion after showing more precisely the relevant structures.

\section{The syntax of \na}

 In this section my aim is to put some syntactic meat on these semantic\is{semantics} bones. The proposal is that in the case of the deictic\is{deixis/deictic} \textit{N\`{a}} the pointing \isi{gesture} heads its own \isi{projection}, \xspace\Pointinghand P.  The \isi{gesture} takes a DP as its argument and is in turn merged with the \textit{N} element which is then merged with the speaker-anchoring particle \textit{A}.  Movement of \textit{N} into \textit{A} completes the derivation.\footnote{The reason for the N-to-A movement is somewhat unclear.}  Schematically: \footnote{Clearly, the pointing can take many forms as we noted before depending on the context (a rather famous one involved a kiss) but the relation remains constant.}

\Lsciex. \label{structure2a}
\begin{forest}
  [AnchorP [Anchor$^0$[N, name=E] [A]][DeicticP [N, draw ]{\draw[->] () to[out=south west,in=south] (E);
} [\xspace\Pointinghand P [\xspace\Pointinghand] [DP [O Giannis, roof]]]]]
\end{forest}


The labels in the tree in \Last are chosen for ease of {reference} purposes.  The \textit{N} morpheme then can be understood as a relation between a pointing and an anchor.  One might {question} the need for a special element to indicate this relation between the anchor and the pointing; usually the pointing instrument tends to be attached to the speaker's body, after all.  This is true in most cases but in contexts of reported speech/\isi{gesture} (at least in \ili{Greek}) it is not the case that the anchor is the individual who actually makes the \isi{gesture} while reporting:

\exg.
Otan bike o Giannis sto grafio girnai o Kostas ke xoris na me proidopiisi mou lei na \xspace\Pointinghand o Giannis\\
When entered the Giannis into office turns the Kostas and without \textsc{sbjv} me warn to.me say there \xspace\Pointinghand the Giannis\\
\glt `When Giannis came into the office Kostas turned to me and without warning he said there \xspace\Pointinghand is Giannis.'

In this case the \isi{gesture} is made by the reporting speaker but the anchor is the original one.  There is no requirement that the pointing be at Giannis or at anything identifiable really.  There has got to be a pointing though, this is the important requirement.
One way to explain this is that gestures cannot be  very easily embedded partly due their nature as an externalisation device, so \Next, with an overt complementiser, is actually ungrammatical:

\exg.
*O Giannis ipe oti N\`{a} \Pointinghand \xspace o Alexis pou tha figi.\\
The giannis said that here \Pointinghand \xspace the Alexis who will leave\\
\glt `Giannis said that this is the Alexis who will leave.'

This suggests an incompatibility between the anchor \textit{-a} and the complementiser. This is also true of other particles in the \textit{-A} series.  Consider the negative particle \textit{Ba}:

\exg.
*O Kostas ipe oti ba dhen tha erthi.\\
The Kostas said that \textsc{prt} \textsc{neg} will come\\
\glt `Kostas said that he will not come.'

There is no incompatibility between the negative particle and sentential \isi{negation}:

\exg.
Ba, dhen tha ertho.\\
\textsc{prt} \textsc{neg} will come\\
\glt `I will not come.'


Also, the notion of \textit{Anchor} should not be understood in too limited a fashion.  Although for the limited purposes of this paper I just link it to the speaker\slash addressee, it should be underlined that this linkage will interact with the rest of the discourse in complex ways which we have to set aside for now.  
In sum, the idea is that there is no reason to postulate elided clausal structure in order to understand the behaviour of the complex category that surfaces as the deictic\slash presentational particle \na in \ili{Greek}.  The \isi{gesture} that must accompany the particle is not a ``parallel'' yet independent act.  It is part of the syntactic structure like any other morpheme might have been, say the \isi{agreement} morpheme on a verb. 

Before we turn to a more explicit formulation of the \isi{semantics} we need to consider the case and \isi{agreement} properties  mentioned earlier in the light of the proposed structure.

\subsubsection{Case and agreement}
 The pieces of the case puzzle are the following:  The DP complement to the particle can be a single DP in the nominative or a complex [\isi{clitic}+DP] where the following combinations are possible:


\Lsciex.
\a. CL$_{\text{-\textsc{acc}}}$ DP$_{\text{-\textsc{acc}}}$
\b. CL$_{\text{-\textsc{acc}}}$ DP$_{\text{NOM}}$
\c. *CL$_{\text{NOM}}$ DP$_{\text{-\textsc{acc}}}$
\d. CL$_{\text{NOM}}$ DP$_{\text{NOM}}$


Recall also that, \textit{pace} \citet{joseph:81}, an accusative DP without the \isi{clitic} is ungrammatical.
Note further that nominative clitics are very rare in \ili{Greek}.  In fact they only appear following deictic\is{deixis/deictic} \textit{N\`{a}} and in the \isi{interrogative} \textit{Puntos/i/to}, `where is he/she/it'.  \textit{Puntos} is a contracted form of \textit{Pu} `where', \textit{n}, which is a reduced form of the copula (\textit{Ine}) and the nominative \isi{clitic}:

\Lsciex.
\ag. Puntos\\
 where.is.he\\
\bg. Natos\\
here.he.is\\


Anticipating somewhat the evidence in \sectref{sec:moutza} on \textit{intransitive gestures} I would like to suggest that the solution to the case puzzle lies in the recognition that the gesture-head (\xspace\Pointinghand) has two versions, one which is transitive and one which is unaccusative.  This is not particularly strange since we take the \isi{gesture} to be the externalisation of a linguistic morpheme.  The cases where a \isi{clitic} is present are cases of \isi{clitic} doubling (which is independently found in \ili{Greek}). 
When the gesturally expressed element is transitive it assigns accusative case to its object and nothing more needs to be said. When the DP associated with the \isi{clitic} surfaces in the nominative it is attached in a higher position and surfaces in the default case, which is nominative.  In the unaccusative case, there is no appropriate case assigner and the DP and \isi{clitic} appear in nominative case.\footnote{The details of the analysis might slightly differ with respect to the view one takes of \isi{clitic} doubling and nominative case but the central points will remain unchanged.}  Given how restricted nominative clitics are it seems correct to suggest that this is a realisation of default \isi{Case}.  What is impossible is for the \isi{clitic} to appear in the nominative, which signals that the unaccusative version of the pointing is selected, and the associated DP to appear in the accusative.  Given that nominative is the default, it is predicted that an accusative marked DP in the absence of a case assigner will lead to ungrammaticality.  Why the \isi{clitic} is obligatory with the accusative in many dialects is a {question} I have no answer to at present and will leave it for future work.

Turning now to the \isi{agreement} issue involving the appearance of second \isi{person} plural \isi{agreement} on the particle, we should recognise that the gestural expression of the pointing, while, as I argued, fully integrated into the structure, differs from other lexical items in that it cannot act as host to other bound morphemes by its very nature.  As a result, the \isi{agreement} morpheme will attach to the next (only) available host which is the particle complex itself.  Presumably this is not generally allowed since the agreeing\is{agreement} version of the particle is only found in some dialects.
 

 Let's now turn to the \isi{semantics} of these structures. 

\subsection{The semantics of deictic \na}
In this section we turn to the compositional \isi{semantics} of the structure in
the \ref{structure2a}.  A pointing \isi{gesture} can be understood broadly as an event.  For maximal regularity in the \isi{semantics} of different heads I will take the \isi{gesture} head to have the \isi{semantics} in \Next


\Lsciex.
$\llbracket$ \xspace\Pointinghand $\rrbracket$ = $\lambda x. \lambda e \  $\xspace\Pointinghand$(x), e$

In other words the \isi{gesture} functions, at the relevant level of abstraction, unsurprisingly, as a \isi{demonstrative} (event).  Now clearly, a pointing entails a pointer. However, I want to propose that the pointing in itself is not syntactically a two place relation.  Rather, the anchor or subject of the pointing is introduced by the \textit{N} morpheme, much like v introduces the external argument (which is the Anchor rather than the actual pointer).  In this way N introduces a relation between an individual that was pointed at and the anchor of that pointing, simplifying somewhat:

\Lsciex.
$\llbracket N \rrbracket$ = $\lambda w \lambda y \lambda\ \xspace\Pointinghand. \ Anchor($\xspace\Pointinghand$, y)\ in\ w$

To keep things simple, I will assume that the \textit{A} particle introduces the relevant anchor as an individual.\footnote{A different approach is possible which would keep more in line with the fact that \textit{A} is not directly referring to an individual (the speaker).  The ensuing complications are, however, not relevant to the points of this paper.}

Once the Anchor argument has been introduced the result is:

\Lsciex.
$ \lambda w.\  Anchor(${\xspace\Pointinghand},\ x $)\  in \ w$



It all really works in a manner parallel to the way vPs and nPs are built, a welcome result.  

Before we turn to the case of \textit{Yes (Ne)} I would like to show briefly how this approach generalises to other gestures with one example.


\subsection{Generalising to other gestures} \label{sec:moutza}
The approach sketched here also allows us to understand cases where the particle \textit{N\`{a}} is accompanied by a \isi{gesture} but no naming act,  in other words the use of intransitive gestures.  One case in point is the rather notorious \ili{Greek} \textit{moutza}.  This is a very commonplace insulting \isi{gesture} in \ili{Greek} which consists in the palm and fingers open wide pointed towards the addressee (in the same way that an English speaker might indicate the \isi{number} five). Roughly like this:

\Lsciex.

\vspace*{-6mm}
% \begin{center}
\hspace*{1cm}
 \includegraphics[height=2cm]{figures/mou4.jpg}
% \end{center}

This \isi{gesture} is made \textit{towards} the addressee but crucially it does not point at the addressee.  Using it to point at someone in a neutral situation would be roughly as felicitous as pointing to the next questioner during the {question} period after a presentation and identifying him as \textit{the bastard in the second row}.   A felicitous use of this \isi{gesture} would be to another driver who has just moved at high speed across the path of your car almost causing an accident. 

 Given the framework adopted here which shares much of the underlying objectives and guiding principles of \citeposs{lascarides-stone:09}\footnote{Though my implementation of these ideas differs significantly from theirs.}, especially the idea that linguistic discourse and at least co-speech \isi{gesture} must be represented in the same logical and syntactic language, I want to propose that the \textit{moutza} functions much as a particle in \ili{Greek}.  More precisely, it can combine with \na, by taking the place of {\xspace\Pointinghand}.  There are two empirical arguments for this position.  First, semantically, its meaning is qualitatively similar to that of particles in that it is \textit{descriptively ineffable}, to use \citeposs{blakemore:02} term. 
 A speaker of \ili{Greek} will have great trouble explaining what a \textit{moutza} actually means, beyond the fact that it is an insult. As for the actual content of the insult, he or she will most likely resort to a series of contexts where the use is felicitous.  The same is true of discourse particles.  
The second argument is that there are both coocurrence restrictions with other particles and ordering effects when used with more than one particle. The \isi{gesture} can be used on its own, which suggests that in some cases the anchoring particle can remain silent, but when used in conjunction with other particles it must be simultaneous/immediately adjacent to \na.  Interestingly, it can never occur alone with an addressee oriented particle such as \textit{Re}.\footnote{On \textit{Re}, see \citet{tsoulas-alexiadou:06}}  
If used, an addressee oriented particle must follow the \isi{gesture}.  Consider the following examples:\footnote{This \isi{gesture} has further characteristics, found in linguistic elements, that we cannot go into here in detail.  For example, it can be reduplicated and this reduplication leads to two different meanings.  If the hands overlap then a focus or emphatic reading is obtained.  When the hands are further apart and do not overlap a distributive reading is favoured.}

\Lsciex.
\a. \na \includegraphics[scale=0.7]{figures/mou4.jpg}
\b. * Re  \includegraphics[scale=0.7]{figures/mou4.jpg}
\c. \na \includegraphics[scale=0.7]{figures/mou4.jpg} Re
\d. * \na Re  \includegraphics[scale=0.7]{figures/mou4.jpg}

The proposal above that \includegraphics[scale=0.7]{figures/mou4.jpg} is merged as the complement of \textit{N} predicts precisely these patterns. For the cases at hand (no pun intended) the structure will be as follows:\largerpage[2]

\Lsciex. \label{structure4a}
\hspace*{3em}\begin{forest}
 [EvidP [AnchorP$_i$, name=u [Anchor$^0$[N, name=R] [A]]
           [DeicticP [N, draw ]{\draw[->] () to[out=south west,in=south](R);} [\includegraphics{figures/mou4.jpg}\,P [\includegraphics{figures/mou4.jpg}]]]]
      [Evid'
        [Evid$^0$ [R] [E, name=T]]
               [AnchorP
                      [E, draw]{\draw[->] () to[out=south west,in=south](T);} [t$_i$,draw,name=TsoulasTI1]]]]
                      \draw[->,overlay] (TsoulasTI1) to[in=west,out=-90,looseness=2.8](u);
\end{forest}\newpage              

Semantically, given the contextual dependency of the \isi{gesture}'s meaning we can give the following general \isi{semantics}:

\Lsciex.
$\llbracket$ \raisebox{-1mm}{\includegraphics[scale=0.7]{figures/mou4.jpg}} $\rrbracket$= $\lambda s \lambda k $(\raisebox{-1mm}{\includegraphics[scale=0.7]{figures/mou4.jpg}} ${}^k$ in $s$)

where the variable $k$ indexes a contextual parameter which independently determines the nature and felicity of the insult in situation $s$ and relates it causally to $s$.  

My purpose in this section was to show that the idea of integrating \isi{gesture} within both the syntactic and the semantic\is{semantics} representation of the sentences with which they are co-temporaneous is a viable and perhaps illuminating option.  We will turn now to the issue of the affirmative particle \nai.



\section{The affirmative particle \nai}
\label{sec5}
With this analysis of \textit{N\`{a}} in mind let us now turn to the affirmative particle \nai.  This particle seems to combine deixis\is{deixis/deictic} with what we called \textit{addressee anchoring}.  Intuitively, this does not seem quite right since in a question-answer situation the questioner asks (in a yes/no \isi{question}) for the respondent to tell which of the two values of the proposition is the true one.  Admittedly, this ought to be done from the point of view of the participant who provides the answer, \textit{the respondent}.  \isi{Anchoring} to the addressee seems like an odd thing to do.  I want to suggest here that we should take a closer look at the notion of anchoring in order to understand what is going on.  The importance of the notion of anchoring in deixis\is{deixis/deictic} is obvious especially given the analysis of \textit{N\`{a}} in the previous section.  In the case of answers to questions, however, perhaps less so.  However, anchoring is not the only notion that these speaker/addressee elements can express. Consider questions in general.  There are many types of questions and many speech acts that may be performed using the \isi{interrogative} form.  However, when we restrict attention to open, non-confirmation, non-rhetorical questions, we can say that in a relatively standard setting uttering in good faith a \isi{question} addressed to a particular addressee entails the following:\footnote{We also have to exclude the sort of \isi{question} that may have no answer.  Scientific questions are sometimes like that.  The qualification \textit{in good faith} in the text is meant to also exclude sarcasm, irony etc.  For very relevant commentary and analysis on these issues see \citet{fiengo:07}.  I am indebted to an anonymous reviewer for comments on this point and for bringing the relevance of Fiengo's book to my attention.} 

\Lsciex.
\a. That there is a proposition $p$ such that $p$ is the answer to the \isi{question}. \label{qua}
\b. That knowledge of that proposition lies with the addressee. \label{qub} 
\c. That $p$  is relevant to the questioner.  \label{quc}


Setting \ref{qua} aside, we can identify \ref{qub} with \citeposs{holmberg:15} \textit{Q-force} operator (see \ref{hq1}) equivalent to \textit{Tell me which...} (an imperative which is clearly addressee-anchor\-ed) which is externalised in various way in different languages.  Marking \ref{quc} in the answer is what we find in \ili{Greek}.  This idea can be formalised using \citeposs{truckenbrodt:06} notion of \textit{Context Index}.  The following \ref{trind} are\largerpage context indices for declaratives and interrogatives based on  \citet{truckenbrodt:06}:

\Lsciex. \label{trind}
\a. Declarative: $<\text{Deont}_{S}, \text{A}, <\text{Epist}> >$ 
\glt `S wants from A that it is common ground that p.'
\b. Interrogative: $<\text{Deont}_{S}, \text{A}, <\text{Epist}> >$ 
\glt `S wants from A that it is common ground whether p.'


The effects of these context indices are given in the paraphrases.  In these cases, in Truckenbrodt's words, \textit{S wants to change the world by changing the epistemic state of S or A}.  So the intuition here is that the particle \nai in \ili{Greek} encapsulates a deictic\is{deixis/deictic} element that relates a proposition to that participant to whose epistemic state it is relevant.  Clearly this participant is the addressee.  It is therefore expected that the \textit{E} particle will be part of this particle complex (\nai). To formalise this I introduce the notion of Anchor$_{\text{\textbf{REL(EVANCE)}}}$ which encodes a relevance-related perspectival dependence as opposed to an \textit{origo}-related one.  In informal, intuitive terms this particle says: \textit{This is the proposition that is of relevance to your belief/knowledge}.\footnote{This notion has implications for the analysis of indexicality and perspective (especially in the context of \cite{giorgi:10,hinzen-sheehan:13}) as it implies that in particle answers the speaker is, if not absent, at least somewhat removed from the representation. Space and time constraints prevent me from delving deeper into  these questions on this occasion.}   Now let's turn to a more specific syntactic implementation. \citeposs{holmberg:15} structure for Yes/No questions is \ref{hq1}, and for a \textit{Yes} answer, \ref{ha2}:

\Lsciex. \label{hq1}  
\begin{forest}
     [[Q-\isi{force}] [CP\is{complementizer} [ $\pm$Pol\is{polarity} [does][$\pm$Pol\is{polarity}]] [CP\is{complementizer} [C] [PolP [DP [John]] [Pol\is{polarity}' [$\pm$ Pol\is{polarity} [does] [$\pm$ Pol\is{polarity}]] [\isi{TP} [like this book, roof]]]]]]]
  \end{forest}

\Lsciex. \label{ha2}  
\begin{forest}
     [FocP [ $+$Pol\is{polarity} [Yes]]  [Foc\is{Focus}' [Foc\is{Focus}] [PolP [DP [John]] [Pol\is{polarity}' [$+$Pol\is{polarity}] [\isi{TP} [like this book, roof]]]]]]
  \end{forest}


One thing that we should observe is that in face-to-face dialogue, saying \textit{Yes} is also accompanied by a \isi{gesture} (usually a head nod).  It is, however, not the case that this \isi{gesture} is as necessary as the deictic\is{deixis/deictic} pointing discussed in the previous section.  I assume that the reason for this is that the \isi{question} provides enough context for the interpretation.  At the same time, I also understand it to be evidence, beyond what was discussed up to now, that an account in parallel with the deictic\is{deixis/deictic} particle is called for and may prove fruitful.  I will therefore assume that the structure of the particle \nai is roughly equivalent to that of \textit{N\`{a}} except that the Anchor  \textit{A} will now be replaced with what I called Anchor$_{\text{\textbf{REL(EVANCE)}}}$.
For maximum consistency with \textit{N\`{a}} (but hopefully avoiding potential confusion) I use the label \textbf{\textit{{G}}} for the head which is the complement of \textit{N-}.  So the structure that we start with is \ref{ne1}:


\Lsciex. \label{ne1}
\begin{forest}
  [Anchor$_{REL}$P [Anchor$_{REL}^0$[N, name=E] [E]][DeicticP [N, draw ]{\draw[->] () to[out=south west,in=south] (E);
} [GP [G] [XP [\ldots, roof]]]]]
\end{forest}

Turning to the nature of the XP, I will assume that \ili{Greek} Yes/No questions have, for the relevant part, a structure that is similar to the one proposed by Holmberg.  Most importantly, I assume that there is a PolP, which is the highest functional head in the IP domain.  I also assume, again following Holmberg, that the PolP is copied from the \isi{question} with its Pol\is{polarity} feature unvalued.  Unvalued Pol\is{polarity} can be valued in two ways:  it can be valued by \isi{negation} which is lower in the structure.  It can also be valued by the G head.    
I propose that the G head is specified with two features.  First, a selectional uPol feature. Technically, if selectional features are by definition uninterpretable then this feature should be a \textbf{uuPol}.  In other words  G selects for a PolP whose head Pol\is{polarity} is specified for a uPol feature.  Presumably, if the uPol feature is carried by a different element then that element can in principle be selected by G.  The second feature for which G is specified is +Pol\is{polarity}, which represents the effect that it has on the PolP.  
This is not contradictory, it is simply a way to express modification.  Modifiers are of type $<$x,x$>$. Syntactically we need to express that they are also selective (``*a passionately car'' is, after all, not a well formed expression), and this is what uPol expresses.  +Pol\is{polarity} tells us what the result of the composition is, i.e. a syntactic object of category Pol\is{polarity} specified for +Pol\is{polarity}.  What happens next is relatively unremarkable:  G values the uPol feature of PolP and the derivation proceeds in exactly the same way as with the deictic\is{deixis/deictic} particle \textit{N\`{a}}.  The deictic\is{deixis/deictic} head \textit{N-} merges\is{Merge} with GP and licenses the introduction of the Anchor$_{REL}$ argument which this time is \textit{E}.  The final result is produced following head movement of \textit{N} in  Anchor$_{REL}^0$, which is pronounced \textit{NE}.  Ideally, nothing more would need to be said.  However, a complication arises with negative questions to which we turn in the next section.  To summarise, the account of the affirmative particle \nai stands as follows:

\Lsciex.
\a. There is a head G specified for uuPol and + Pol\is{polarity}.
\b. G merges\is{Merge} with the PolP (with uPol) inherited from the \isi{question}.
\c. G values uPol as +.
\d.  Deictic N- merges\is{Merge} with G and introduces the Anchor argument.
\e.  Anchor$_{REL}$ E is merged.
\f.  N- moves to E.


Now for negative questions.

\subsection{Negative polar questions}
The approach that we have pursued up to this point brings the issue of negative polar questions into sharp focus.  This is so in two related ways.  First, assuming that uPol acts just like any other uninterpretable feature, then one expects it to act as a probe and find a  matching goal in Neg\is{negation}$^0$ and will be valued -Pol\is{polarity}.  We could argue here that Neg\is{negation} is an inactive goal for the Pol\is{polarity} probe.  But this, together with the idea that saying \textit{Yes} involves propositional deixis\is{deixis/deictic}, makes the general prediction that \ili{Greek} should pattern more like Chinese in that it would always be possible to answer \textit{Yes} to a negative \isi{question} and confirm the negative alternative.  In other words \ili{Greek} should have a truth-based answering system rather than a \isi{polarity} based one.  As it turns out, \ili{Greek} seems to allow both. There seems to be two major dialects in \ili{Greek} in this respect. \largerpage  
It has been reported \citep{Holton-Mackridge-Philippaki:97}\footnote{See also the data in SSWL (http://sswl.railsplayground.net/)} that \ili{Greek} allows \textit{Yes} answers to negative questions whereby the negative alternative is confirmed:\footnote{Example \ref{nq1} is adapted slightly from \citet[414]{Holton-Mackridge-Philippaki:97}}

\exg. 
Dhen to pire to danio o Giannis telika?\\
\textsc{neg} it took the loan the Giannis finally\\  \label{nq1}
\glt `In the end, didn't Giannis get the loan?'

\exg. 
Ne dhen to pire\\
Yes \textsc{neg} it took\\ \label{nq2}
\glt `Yes he didn't get it.'

Call this Dialect 1.  Dialect 2\footnote{As far as I know Dialect 1 and 2 are not geographically circumscribed.} does not allow this type of confirmatory answer except in the presence of special elements to which we return.  
Observe, however, that the pattern above is \textit{only} possible, in Dialect 1, if the answer contains an overtly realised copy of the PolP of the \isi{question}.  If the PolP is not overt the result is ungrammatical:


\exg.
Dhen to pire to danio o Giannis telika?\\
\textsc{neg} it took the loan the Giannis finally\\   \label{nq3}
\glt `In the end, didn't Giannis get the loan?'

\exg.
*\nai\\
Yes\\ \label{nq4}

How can we account for these patterns with the theory that we have so far?  To begin with Dialect 1, the data suggest that there is a formal recoverability condition that must be met in order for ellipsis of the PolP to be licensed.  We can formulate this condition as follows:

\Lsciex. \label{recov}
\textit{Recoverability Condition for PolP}\\
PolP can be elided if the valuer of uPol is overt.

This principle derives Dialect 1 directly.  If Neg\is{negation} values Pol\is{polarity}, then ellipsis of PolP is not licensed.  Note that \ref{recov} is a formal condition and therefore immune to being circumvented pragmatically given that recovering the meaning of PolP is rather easy.  It may be but it is not allowed.  Note that in the case of a \textit{No} answer ellipsis will be licensed since there is no way to distinguish which -Pol\is{polarity} feature valued uPol of Pol\is{polarity}$^0$.   So despite appearances, Dialect 1 seems to be a polarity-based system.

Let me turn now to Dialect 2. This dialect does not allow \textit{Yes} confirmation  answers to negative questions quite generally except when there is overt material between the affirmative particle and the PolP.  The material in \isi{question} is not just any material.  As far as I can tell what is required is either certain particle clusters like \textit{Re gamoto}, meaning roughly \textit{bloody hell} or, more interestingly, elements like \textit{siga} and \textit{kala}.  The literal meaning of \textit{siga} is `slowly' or `quietly' and of \textit{kala} is `fine/OK'.   Their meaning \textit{qua} particles in this context is scalar and indicates that the proposition with the opposite \isi{polarity} value of that of the PolP is, in fact, the most unlikely.\footnote{The proposition whose position at the scale is made salient through the particle can be made overt with the particle \textit{siga} but not with \textit{kala}.  Offering a detailed analysis of the resulting patterns, including the appearance of \isi{expletive} \isi{negation} with \textit{siga}, would take us too far afield and are set aside for future work.}
Note, also, that these elements are prosodically grouped with the affirmative particle rather than with what follows. Consider the following examples:
 

\exg. 
Dhen to pire to danio o Giannis telika?\\
\textsc{neg} it took the loan the Giannis finally\\  \label{nq3a}
\glt `In the end, didn't Giannis get the loan?'

\Lsciex.
\ag. 
Ne [re gamoto]  dhen to pire.\\
Yes \textsc{prt} \textsc{prt} \textsc{neg} it took\\ \label{nq4a}
\glt `Yes, bloody hell, he didn't get it.'
\bg. 
Ne kala,  dhen to pire.\\
Yes \textsc{prt} \textsc{neg} it took\\ \label{nq5}
\glt `Yes, how could he, he didn't get it.'
\cg. 
Ne siga,  dhen to pire.\\
Yes \textsc{prt} \textsc{neg} it took\\ \label{nq6}
\glt `Yes, how could he, he didn't get it.'


I believe that we can understand these patterns if we take seriously the idea that the syntax of the G head is really very closely aligned with the syntax of the \xspace\Pointinghand head in the case of the deictic\is{deixis/deictic} particle \na.  Specifically, the pattern above is reminiscent of the cases of the deictic\is{deixis/deictic} particle when it is followed by a \isi{clitic} and a full DP.  We can think of the complement of G as the Pol\is{polarity} equivalent of the [\isi{clitic} DP] doublet where the \isi{clitic} element can be realised overtly by elements like \textit{siga} or \textit{kala}.  As the intonational pattern suggests, just as in the case of the DPs, the PolP ends up higher in an extraposed position.  Whether it moves there or it is base-generated in the higher position is an important {question} but one that can be safely set aside for now.  The result, in either case,  is that the PolP ends up outside the scope domain of the G-head.  To capture this I want to propose that there is a second constraint alongside recoverability which can be formulated as follows:

\Lsciex. 
\textit{G-Scope Constraint}\\ \label{gscope}
If G has a determinate feature specification (+ or -) it does not tolerate contradictory Pol\is{polarity} values in its scope.

Perhaps \ref{gscope} is ultimately formulable in terms of AGREE but there are complications as we saw earlier. 

We could also draw a parallel with the  \isi{Case} patterns observed with \na too.  Recall that an accusative \isi{clitic} could associate with a higher DP in the nominative while the opposite is impossible.   Depending on the speaker's attitude to the fact that Giannis \textit{did} get the loan or to \textit{Giannis} himself, the following patterns are observed. Assuming \textsc{nom}=+Pol\is{polarity} and \textsc{acc}=-Pol\is{polarity} and that the particle complex in \ref{nq7} corresponds to a negative value:

\Lsciex.
\ag. 
*Ne [re gamoto]  to pire.\\
Yes \textsc{prt} \textsc{prt}  it took\\ \label{nq7}
\glt `Yes, bloody hell, he got it.'
\bg. 
Ne kala,   to pire.\\
Yes \textsc{prt}   it took\\ \label{nq8}
\glt `Yes, how couldn't he, he got it.'
\cg. 
Ne siga,  to pire.\\
Yes \textsc{prt}  it took\\ \label{nq9}
\glt `Yes, how couldn't he, he got it.'


At the same time, if the particle  position in the [\textsc{prt} PolP] doublet is filled by the element \textit{pos} which is a polarity-reversing positive particle\footnote{Akin to \ili{French} \textit{Si}, although \textit{Si} cannot coocur with \textit{Oui}, unlike pos. There is another version of the particle \textit{pos} which is scalar along the same lines of \textit{siga/kala}.  The claim here is for the \isi{polarity} reversing one.} a negative PolP is impossible:

\exg. 
Ne pos  to pire.\\
Yes \textsc{prt} it took\\ \label{nq10}
\glt `Yes, on the contrary, he got it.'
 
\exg. 
*Ne pos,  dhen to pire.\\
Yes \textsc{prt} \textsc{neg} it took\\ \label{nq11}
\glt `Yes, on the contrary, he didn't get it.'

Intuitively the effect of this is to produce a two part answer, roughly:  `Correct, $(\neg) p$'.
We can then think of the difference between the two dialects as involving the interplay  of the two constraints: 

\Lsciex.
\a. Recoverability \label{rec}
\b. G-Scope Constraint. \label{scope}

While Dialect 1 tolerates violations of \ref{scope}, Dialect 2 does not.  Neither tolerates violations of \ref{rec}.

 
\section{Conclusion}
\label{sec6}
This paper explored the hypothesis that affirmation and deixis\is{deixis/deictic} share a common core.  \ili{Greek} shows this common core overtly in the \isi{morphological} composition of its deictic\is{deixis/deictic} and affirmative particles.  But looking at the details of the syntactic representations we observed very striking similarities.  Most importantly the fact that there is a head that corresponds to a pointing, real or metaphorical, which is clearly part of the syntactic representation.  Accepting this leads us to take at least a subset of \isi{gesture} as an integral part of linguistic representations.  The syntactic complexity of seemingly very simple words was also a surprising conclusion.  Of course much work remains to be done.  The role of focus and intonation deserves closer study than I have been able to offer here, the negative particle \textit{όχι} `No' was set aside (for good reason, as the \isi{morphology} is far less transparent), and the cross-linguistic applicability of the account was not considered.  I would expect that this account extends crosslinguistically quite widely although one cannot expect that the \isi{morphology} will be as transparent as what we saw in \ili{Greek} (although there is suggestive evidence).  This I leave for future work. 

C. P. Cavafy wrote in a famous poem:\footnote{\textit{Che fece \ldots Il gran rifiuto}, which I quote in Edmund Keely's translation.} 

% \vspace*{-1cm}

% \begin{poem}[center] {}{}%[\versewidth]
\begin{quote}
[\ldots] It’s clear at once who has the Yes \\
ready within him; and saying it,\\
 he goes forward in honor and self-assurance.  \\
He who refuses does not repent. Asked again,\\
he would still say no. Yet that no—the right no—\\
undermines him all his life. 
\end{quote}
% \end{poem}

I am sure the poet is right, but if I am right too then saying \textit{Yes} in \ili{Greek} is no mean feat either.

\section*{Acknowledgments}
It is with great pleasure that I dedicate this work to Anders on his birthday.  Much of what is in here was prompted by his work on \textit{Yes} and \textit{No}. The influence of his work on the present paper is pervasive.  

Parts of this paper have been presented at the 12th International Conference on {Greek} Linguistics in Berlin, and at seminars in York and Newcastle.  I want to thank these audiences for many useful comments and suggestions.  More specifically I would like to thank: Kook-Hee Gil,  Stella Gryllia, Anders Holmberg, Shin-Sook Kim, Margarita Makri,  Dimitris Michelioudakis, Moreno Mitrovic, Bernadette Plunkett, Anna Roussou,  Peter Sells, Evi Sifaki, and Rebecca Woods for suggestions and comments.  I would also like to thank the two anonymous reviewers for this volume for comments that led to a {number} of improvements.  Finally, I would like to thank the editors, Laura Bailey and Michelle Sheehan, for giving me this opportunity and for being so patient.
{\sloppy
\printbibliography[heading=subbibliography,notkeyword=this]
}



\end{document}
