\documentclass[output=paper]{langscibook} 
\title{Repairing Final-Over-Final Condition violations: Evidence from Basque verb clusters} 
\shorttitlerunninghead{Repairing Final-Over-Final Condition violations}
\author{%
 Ricardo Etxepare\affiliation{IKER UMR 5478-CNRS}\lastand 
 Bill Haddican\affiliation{CUNY, Queens College/Graduate Center}
}

\abstract{%
This article discusses implications of Basque modal constructions for representational models of Final-Over-Final Condition (FOFC) effects.  We argue that FOFC-violating structures at an intermediate derivational level can be repaired by subsequent movement steps.  The analysis entails that FOFC-violating structures are buildable by the syntax, contra narrow syntactic approaches to FOFC, and that FOFC evaluation instead applies in the phonology after copy deletion.  Such a view of FOFC helps explain several recalcitrant word order restrictions on Basque modal constructions as well as variation in the effect of focus and negation on modal placement across Basque dialects.
}
%%% REPLACE ABSTRACT TEXT WITH YOUR OWN ABSTRACT
%\noindent \textit{Abstract}. %{This paper discusses implications of Basque modal constructions for representational models of Final-Over-Final Constraint (FOFC) effects.  We argue that FOFC-violating structures at an intermediate derivational level can be repaired by subsequent movement steps.  The analysis entails that FOFC-violating structures are buildable by the syntax, contra narrow syntactic approaches to FOFC, and that FOFC evaluation instead applies in the phonology after copy deletion.  Such a view of FOFC helps explain several recalcitrant word order restrictions on Basque modal constructions as well as variation in the effect of focus and negation on modal placement across Basque dialects.}

%
% \bibliography{anders}
\ChapterDOI{10.5281/zenodo.1117702}
\maketitle 
\begin{document}
%%% ENTER A BRIEF TITLE: this appears in the running head
% \brieftitle{Reparing Final-Over-Final Constraint Violations}
% \runninghead

%%% ENTER AN ACKNOWLEDGMENT, which will be the first, unnumbered footnote:
%\ackfootnote{We would like to thank ...}

\section{Introduction} \protect\label{introduction}
This article discusses some implications of \ili{Basque} \is{Modal}modal constructions for recent approaches to  Final-Over-Final Condition (FOFC\is{Final-over-Final Condition}) effects.  FOFC\is{Final-over-Final Condition} is a  generalization originally by \cite{Holmberg2000deriving} about the interaction between dominance relations and \{head, complement\} ordering cross-linguistically.   In particular, following much previous typological literature, Holmberg  noted that  ``harmonic'' sequences of  head-initial and head-final phrases, as in (\ref{fig:haddican:sub1},\ref{fig:haddican:sub2}) are common cross-linguistically, as are ``disharmonic'' se-\linebreak quences where a head-initial phrase dominates a head-final phrase, as in \REF{fig:haddican:sub3} \citep{hawkins1983, hawkins1995}.  Holmberg noted that what is much rarer -- possibly unattested in relevant domains -- are instances of a head-final phrase dominating a head-initial phrase, as in \REF{fig:haddican:sub4}.

\newpage
\ea\label{fofcstructures}
\begin{multicols}{4}
\ea \label{fig:haddican:sub1} Harmonic, right-branching\\\vspace{\baselineskip}\begin{tikzpicture}[baseline]\tikzset{level distance=25pt, sibling distance=5pt}
\Tree
[.ɣP ɣ [.βP β α ] ]
\end{tikzpicture}\\
\ex \label{fig:haddican:sub2} Harmonic, left-branching\\\vspace{2\baselineskip}\begin{tikzpicture}[baseline]\tikzset{level distance=25pt, sibling distance=5pt}
\Tree
[.ɣP [.βP α β ] ɣ ]
\end{tikzpicture}\\\vspace{\baselineskip} 
\ex \label{fig:haddican:sub3} Disharmonic, attested\\\vspace{2\baselineskip}\begin{tikzpicture}[baseline]\tikzset{level distance=25pt, sibling distance=5pt}
\Tree
[.ɣP ɣ [.βP α β ] ]
\end{tikzpicture}\\\vspace{\baselineskip}
\ex \label{fig:haddican:sub4} Disharmonic, unattested (in relevant domains)\\\vspace{\baselineskip}\begin{tikzpicture}[baseline]\tikzset{level distance=25pt, sibling distance=5pt}
\Tree
[.ɣP [.βP β α ] ɣ ]
\end{tikzpicture}
\z \end{multicols} \z

For the moment, let us summarize Holmberg's observation about the above interaction as in \REF{ex:haddican:2} (taken from \cite{biberaueretal2014}).
 
\ea\label{ex:haddican:2}\begin{justify} \textit{The Final-over-Final condition (preliminary version)}\\\noindent\protect\label{fofc1}If β is a head-initial phrase and  ɣ is a phrase immediately dominating β, then ɣ must
be head-initial. If β is a head-final phrase, and  ɣ is a phrase immediately dominating
β, then  ɣ can be head-initial or head-final.\\
(adapted from \citealt{biberaueretal2014})\end{justify}
\z

Following \cite{Holmberg2000deriving}, a now-considerable body of literature has described\linebreak FOFC-effects cross-linguistically \citep{Holmberg2000deriving,BiberauerEtAl2008,biberaueretal2014,Sheehan2013fofc,sheehan2013some} and diachronically \citep{biberaueretal2009, biberaueretal2010}. Formal approaches to FOFC\is{Final-over-Final Condition} effects have generally been of two types.\footnotemark\protect\footnotetext{We do not consider consider here Hawkins' (to appear) processing based approach to FOFC.  See \cite{sheehan2012fofchff} for discussion.}  One approach, by \cite{biberaueretal2014}, takes FOFC\is{Final-over-Final Condition} effects to be a narrow syntactic phenomenon.  Assuming the \isi{Linear Correspondence Axiom} (LCA) \citep{Kayne1994}, \cite{biberaueretal2014} take effects such as \REF{fofcstructures} to reflect restrictions on roll-up movement, which follow, in turn, from minimality effects on the spreading of features which drive such movement.  A second approach by Sheehan (\citeyear{Sheehan2013fofc, sheehan2013some, sheehan2012fofchff}) takes FOFC\is{Final-over-Final Condition} effects to be phonological in nature.  On this approach, structures such as \REF{fig:haddican:sub4} are bad because they cannot be linearized by the LCA (in Sheehan's modified form) at PF.  

The two approaches crucially make different predictions about the possibility of deri-\linebreak vational repair. The PF approach, but not the narrow syntax approach, predicts the possibility of a derivation where a FOFC-violating structure is built by the syntax, but repaired in some way before linearization -- for instance by copy deletion of FOFC-violating structure.  In contrast, the narrow syntax approach holds that FOFC-violating structures are never buildable by the syntax, and therefore predicts that the syntax should never have occasion to repair a FOFC-violating structure.  The goal of this chapter is to describe a set of \is{Modal}modal constructions in \ili{Basque} where copy deletion appears to bleed FOFC\is{Final-over-Final Condition}.  Assuming that copy deletion applies in the phonology, our evidence that FOFC\is{Final-over-Final Condition} evaluation follows copy deletion is therefore in keeping with a PF approach to FOFC\is{Final-over-Final Condition} effects in these dialects and not with a narrow syntactic approach. We do not take a position on how FOFC\is{Final-over-Final Condition} effects might be derived at PF.

The discussion is organized as follows. In \sectref{sec:haddican:2}, we introduce FOFC\is{Final-over-Final Condition} and describe PF vs. narrow syntactic approaches to this phenomenon.  \sectref{sec:haddican:3} reviews a set of facts described by \cite{etxepare-uribeetxebarria2009} about the interaction between word order and structural complexity of \is{Modal}modal complements in \ili{Basque} \is{Modal}modal constructions.  In \sectref{sec:haddican:4}, we spell out the nature of the FOFC\is{Final-over-Final Condition} violation and FOFC\is{Final-over-Final Condition} repair involved in such constructions.  


\section{The Final-Over-Final Condition}  \protect\label{fofc}\label{sec:haddican:2}
\subsection{Word order (dis)-harmony in mixed-head languages}  \protect\label{disharmony}

We begin by illustrating FOFC\is{Final-over-Final Condition} effects with some examples from the literature. \citegen{Holmberg2000deriving} original characterization of FOFC\is{Final-over-Final Condition} was in the context of \{Aux\is{Auxiliary}, O and VP\} order patterns in \ili{Finnish} as in \REF{ex:haddican:3}. \ili{Finnish} is typically VO, but in certain contexts allows both the object to precede the V and the V to precede the auxiliary\is{Auxiliary}, as illustrated in (\ref{ex:haddican:3a}--\ref{ex:haddican:3c}).  What is not permitted, however, is a V-O-Aux\is{Auxiliary} order as shown in \REF{ex:haddican:3a}, that is, where a head-final auxiliary\is{Auxiliary} selects a head-initial V, in violation of \REF{ex:haddican:2}.

\ea\protect\label{finnish}\label{ex:haddican:3}
\ea[]{\label{ex:haddican:3a} 
\gll 
\textit{Milloin} \textit{Jussi} \textit{olisi}	\textit{kirjoittanut} \textit{romaanin?} 		\\ 
        when Jussi would-have written	\textsc{indef}-novel \\
	    \glt `When would Jussi have written a novel?' \hfill{[Aux-V-O]}}
	\ex[]{\gll	\textit{Milloin} \textit{Jussi} \textit{olisi}	\textit{romaanin}	\textit{kirjoittanut?} 		 \\
		when Jussi would-have \textsc{indef}-novel written \\
	    \glt `When would Jussi have written a novel?'  \hfill{[Aux-O-V]}}
	\ex[]{\label{ex:haddican:3c} \gll \textit{Milloin} \textit{Jussi} \textit{romaanin}	 \textit{kirjoittanut} \textit{olisi?}  \\
		When Jussi \textsc{indef}-novel written	would-have \\
	     \glt `When would Jussi have written a novel?' \hfill{[O-V-Aux\is{Auxiliary}]}}
	\ex[*]{\label{ex:haddican:3d} \gll \textit{Milloin} \textit{Jussi} \textit{kirjoittanut} \textit{romaanin}  \textit{olisi?} \\
		    when Jussi written \textsc{indef}-novel would-have\\
	     \glt `When would Jussi have written a novel?'\hfill{\textbf{[*V-O-Aux\is{Auxiliary}]}}\\}
		\citep{Holmberg2000deriving}
\z		\z

Similar facts come from the relative order of modals, infinitival verbs and their objects in \ili{Basque}. \ili{Basque} is canonically OV, but many speakers allow objects -- especially heavy objects -- to occur postverbally \citep{derijk1969, ortizdeurbina1989, elordieta2001}.  In addition, infinitival complements of modals may appear either to the right or the left of the selecting \is{Modal}modal + auxiliary\is{Auxiliary}.  When the infinitival complement appears to the right of its selecting \is{Modal}modal as in (\ref{ex:haddican:4a}, \ref{ex:haddican:4b}) both OV and VO orders are possible.  When the infinitival appears to the left of its selecting \is{Modal}modal, only the OV order is possible, a pattern again in keeping with \protect\REF{fofc1}.


\ea \label{ex:haddican:4}
\ea \label{ex:haddican:4a} \gll  \textit{Nahi} \textit{zuen} [\textit{hobetu} \textit{bere} \textit{ingelesa.}]   \\
want \textsc{aux} improve \textsc{refl} English  \\
\glt `He/She wanted to improve his/her English.'\hfill{[Modal-Infin-Obj]}
\ex \label{ex:haddican:4b} \gll  \textit{Nahi} \textit{zuen} [\textit{bere} \textit{ingelesa} \textit{hobetu.}]    \\
want \textsc{aux} \textsc{refl} English improve \\
\glt `He/She wanted to improve his/her English.'\hfill{[Modal-Obj-Infin]}
\ex \label{ex:haddican:4c} \gll  [\textit{Bere} \textit{ingelesa} \textit{hobetu}] \textit{nahi} \textit{zuen.} \\
 \textsc{refl} English improve want \textsc{aux} \\
\glt `He/She wanted to improve his/her English.' \hfill{[Obj-Infin-\isi{Modal}]}
\ex \label{ex:haddican:4d} \gll  [\textit{*Hobetu} \textit{bere} \textit{ingelesa}] \textit{nahi} \textit{zuen.}\\
improve \textsc{refl} English  want \textsc{aux} \\
\glt `He/She wanted to improve his/her English.'  \hfill{\textbf{[*Infin-Obj-\isi{Modal}]}}
\z\z

\cite{biberaueretal2014} note that without further qualification, \protect\REF{fofc1} incorrectly rules out commonplace, well-formed structures in \ili{German} of the kind shown in \REF{ex:haddican:5}.

\ea \label{ex:haddican:5}
\ea \label{ex:haddican:5a} \gll \textit{Johann} \textit{hat} [\textsubscript{\textit{VP}} [\textsubscript{\textit{DP}} \textit{einen} \textit{Mann}] \textit{gesehen}.] \\
	John	has	 {} {}      a        man	  seen \\
	\glt `John has seen a man.'
\ex \label{ex:haddican:5b} \gll \textit{Johann} \textit{ist} [\textsubscript{\textit{VP}} [\textsubscript{\textit{PP}} \textit{nach} \textit{Berlin}] \textit{gefahren}.]\\ 
	John is	{} {} to Berlin gone \\
	\glt `John has gone to Berlin.' \\
	\citep{biberaueretal2014}
\z\z 

\REF{ex:haddican:5a} involves a head-final VP containing a head-initial DP, and in \REF{ex:haddican:5b}, the head-final VP contains a head initial PP, both in violation of \protect\REF{fofc1}. Biberauer et al. note that such exceptions to \protect\REF{fofc1} can be explained in terms of the categorial status of α and β.  That is, Biberauer et al. note that the cases in \REF{ex:haddican:5} differ from the \ili{Basque} and \ili{Finnish} cases just discussed in that the relevant α and β heads in \REF{ex:haddican:5}, are categorially distinct -- V is clearly of a different categorial status from both D \REF{ex:haddican:5a} and P \REF{ex:haddican:5b}.  By contrast, the \ili{Finnish} examples in (\protect\ref{finnish}) all crucially involve a sequence of heads in the extended \isi{projection} of the verb.   Biberauer et al. capture this class of exceptions to \protect\REF{fofc1} by restricting FOFC\is{Final-over-Final Condition} evaluation to an extended \isi{projection}, and by defining the extended \isi{projection} as in \REF{ex:haddican:6}, where \textsc{spine} is defined as in \REF{ex:haddican:7}.

\noindent\parbox{\textwidth}{\ea \label{ex:haddican:6} The Extended \isi{Projection}\is{extended projection} of a lexical head L (EP(L)) is the sequence of categories EP = \{α\textsubscript{1} \ldots\ α\textsubscript{i} \ldots\ α\textsubscript{n}\} such that:  \protect\label{fofc4}
\begin{xlisti}
\ex  α\textsubscript{i} is in the spine defined by L; for each pair of heads <H\textsubscript{i}, H\textsubscript{i+1}> in EP;
\ex H\textsubscript{i} c-selects H\textsubscript{i+1}; 
\ex H\textsubscript{i} is categorially non-distinct from H\textsubscript{i+1}.
\end{xlisti}
\z}

\noindent\parbox{\textwidth}{\ea \label{ex:haddican:7} \textsc{spine:} A sequence of nodes Σ = \{α\textsubscript{1} \ldots\ α\textsubscript{i} \ldots\ α\textsubscript{n}\} 
% \todo{I would add commas here between the elements of this set} 
forms a spine iff:  \protect\label{fofc5}
\begin{xlisti}
\ex \label{ex:haddican:7a} α\textsubscript{n} is a lexical head H\textsuperscript{min};
\ex \label{ex:haddican:7b} α\textsubscript{i} is H\textsuperscript{-min}, a \isi{projection} of α\textsubscript{n};
\ex  for all α\textsubscript{m > i}  α is a head H´ which c-selects either H or some α\textsubscript{j} ∊ Σ, or α is a \isi{projection} of some α\textsubscript{j} ∊ Σ.
\end{xlisti}\z}

Biberauer et al.'s \REF{ex:haddican:6} and \REF{ex:haddican:7} are intended to formalize the intuition that the extended \isi{projection} of a lexical head consists of all of the functional material in the c-selecting sequence above that lexical head up to the first categorially distinct element.  Biberauer et al. assume that a sequence C-T-v-V is all part of the extended \isi{projection} of V and will count as categorially non-distinct. In this way, Biberauer et al. intend FOFC\is{Final-over-Final Condition} to encompass disharmonies of the above type involving heads in a canonical clausal spine as well as those in a canonical nominal spine, but will not extend to sequences of heads with distinct categorial features, such as in cases where a CP\is{complementizer} is selected by n.

A second class of exceptions that Biberauer et al. focus on concerns Ā-movement.   Biberauer et al. note that, across languages, topic- and focus-movements appear able to violate FOFC\is{Final-over-Final Condition} as described so far. In \REF{ex:haddican:8}, for example, the head-initial, non-satellite VP raises to the left-periphery and spells-out to the left of its dominating head in violation of \protect\REF{fofc1}.
 
\ea \label{ex:haddican:8} We expected John to eat the pies, and [eat the pies] he did \st{eat the pies}.\\
\citep{biberaueretal2014} \z
 
Biberauer et al. therefore also exclude Ā-movement from the scope of FOFC\is{Final-over-Final Condition}.   Let us therefore adopt as our working characterization of FOFC\is{Final-over-Final Condition} the following from \cite{biberaueretal2014}.

\ea \textit{The Final-over-Final condition (amended version)}\\\protect\label{fofc2}If β is a head-initial phrase and ɣ is a phrase immediately dominating β, then ɣ must be head-initial. If β is a head-final phrase, and ɣ is a phrase immediately dominating β, then  ɣ can be head-initial or head-final, where: 
\begin{xlisti}
\ex  β and ɣ are in the same Extended \isi{Projection}\is{extended projection}; 
\ex βP has not been Ā-moved to Spec, ɣP.
\end{xlisti}
(adapted from \cite{biberaueretal2014}) 
\z 

We consider two main formal approaches to this generalization in the following sections.
 
\subsection{Biberauer et al.'s narrow syntactic approach to FOFC}  
 
Biberauer et al. propose that FOFC\is{Final-over-Final Condition} effects, as described above, are  a property of the syntactic component, reflecting a condition on movement.  In particular, Biberauer et al. follow \citet{Kayne1994} in assuming a \isi{universal} spec-head-complement merged order, and that complement-head orders are a consequence of ``roll up'' -- iterative complement-to-specifier movement in a given sequence.  FOFC\is{Final-over-Final Condition} effects, from this perspective, are explained if the following two conditions apply to roll up: (i) it must start at the base of a given extended \isi{projection}; and (ii) it proceeds monotonically, that is, it cannot start and stop and start again.

Biberauer et al. model these conditions in terms of constraints on spreading of a general movement-driving feature which they represent with the caret symbol, ``\textsuperscript{∧}''.  This feature drives different kinds of movement depending on the formal features that it associates with:  when ``\textsuperscript{∧}'' associates with edge features of a \isi{phase} head, it will trigger Ā-movement; when associated with phi-features it will drive \isi{A-movement}; and crucially for FOFC\is{Final-over-Final Condition} effects, when it associates with c-selectional features, it triggers movement of a complement to the spec of its selecting head.
 
Biberauer et al. assume further that this feature can ``spread'' up the tree.  This spreading is crucially constrained in a way typically assumed for head movement, namely that it can skip no intervening heads.  Biberauer et al. state this condition as in \REF{ex:haddican:10}.
   
\ea\label{ex:haddican:10} 	If a head α\textsubscript{i} in the Extended \isi{Projection}\is{extended projection} E of a lexical head L has \textsuperscript{∧} associated with its selection feature for a lower head α\textsubscript{i+1}, then so does α\textsubscript{i+1}.
\z

The assumption of monotonic spreading therefore excludes the unattested start-stop-start pattern that will produce FOFC\is{Final-over-Final Condition} violations:
 
\ea \textit{Non-monotonic spreading of \textsuperscript{∧}}\\ \protect\label{haddican:wedge} *[X\textsuperscript{∧}  [ Y [Z\textsuperscript{∧} ]]] \z

Importantly, on Biberauer et al.'s approach, FOFC\is{Final-over-Final Condition} effects are a narrow syntactic phenomenon. FOFC-violating structures are not filtered out by interface conditions; rather they are simply not derivable on the approaches to merge and locality proposed by Biberauer et al.  In the next section, we briefly contrast this approach with Sheehan's PF approach.
	
\subsection{Sheehan's PF approach} \protect\label{sheehanpf}
Sheehan (\citeyear{Sheehan2013fofc, sheehan2013some, sheehan2012fofchff}) proposes that FOFC\is{Final-over-Final Condition} effects are a consequence of the way the phonology linearizes syntactic structures on a modified version of the LCA \citep{Kayne1994}.   Following \citet{Chomsky1995} and \citet{Nunes2004}, Sheehan (\citeyear{Sheehan2013fofc,sheehan2013some}) takes the LCA to be a linearization algorithm that orders syntactic objects in the phonological component. Sheehan's version of the LCA, however, differs from Kayne's in that it assumes that  linearization maps not just according to c-command relations, but also \textit{c-selection} relations.  Indeed, in Sheehan's algorithm, precedence relations are first mapped by c-selection; c-command is an elsewhere condition.  In addition, it adopts from head-parameter approaches the assumption that linearization of two categories in a c-selection relation is parametrized to the selecting head.  We summarize this proposal in \REF{ex:haddican:11} from \cite{Sheehan2013fofc, sheehan2013some}.

\ea\label{ex:haddican:11} \textit{Sheehan's (\citeyear{sheehan2013some})
revised LCA}  \protect\label{lca}
 \begin{xlisti}
 \ex \label{ex:haddican:11i} If a category A c-selects a category B, then A precedes/follows B at PF.
 \ex \label{ex:haddican:11ii} If no order is specified between A and B even transitively by (i), then A precedes B at PF if A asymmetrically c-commands B.	
 \end{xlisti}
\z 	

Let us consider now how these assumptions help derive the FOFC\is{Final-over-Final Condition} effects described in \sectref{disharmony}, returning to the structures in \sectref{fofcstructures}, repeated here.

\begin{exe}
\exr{fofcstructures}
\begin{multicols}{4}
\ea Harmonic, right-branching\\\vspace{\baselineskip}\begin{tikzpicture}[baseline]\tikzset{level distance=25pt, sibling distance=5pt}
\Tree
[.ɣP ɣ [.βP β α ] ]
\end{tikzpicture}\\
\ex Harmonic, left-branching\\\vspace{2\baselineskip}\begin{tikzpicture}[baseline]\tikzset{level distance=25pt, sibling distance=5pt}
\Tree
[.ɣP [.βP α β ] ɣ ]
\end{tikzpicture}\\\vspace{\baselineskip} 
\ex Disharmonic, attested\\\vspace{2\baselineskip}\begin{tikzpicture}[baseline]\tikzset{level distance=25pt, sibling distance=5pt}
\Tree
[.ɣP ɣ [.βP α β ] ]
\end{tikzpicture}\\\vspace{\baselineskip}
\ex Disharmonic, unattested (in relevant domains)\\\vspace{\baselineskip}\begin{tikzpicture}[baseline]\tikzset{level distance=25pt, sibling distance=5pt}
\Tree
[.ɣP [.βP β α ] ɣ ]
\end{tikzpicture}
\z \end{multicols} \end{exe}

In the harmonic (a) and (b) structures in \REF{fofcstructures}, precedence relations are established unproblematically by parameter setting attaching to the c-selection relations between β and α and α and ɣ, pursuant to \REF{ex:haddican:11i}. In (a), the precedence relations {β > α and  ɣ > β} are established and by transitivity ɣ > α. In (b), {α > β, β > ɣ} are established by c-selection, and by transitivity α > ɣ. In the case of the disharmonic orders in (c) and (d), the condition in \REF{ex:haddican:11ii} becomes relevant.  In the attested disharmonic order, (c), c-selectional relations will determine the orders ɣ > β and α > β.  C-selectional relations, however, leave underdetermined the relative order of ɣ and α, that is, the choice between outputs ɣ > α > β and α > ɣ > β.  The fall back c-command criterion in \REF{ex:haddican:11ii}, however, determines ɣ > α. In the disharmonic structure in (d), c-selectional relations will likewise determine β > ɣ and β > α, leaving underdetermined the relative order of ɣ and α.  Crucially, the c-command condition in \REF{ex:haddican:11ii} will then determine ɣ > α, yielding the output β > ɣ > α, and \textit{not} the FOFC-violating order, β > α > ɣ.  On this approach, the unavailability of FOFC-violating structures in the general case falls out of Sheehan's modified LCA, since the (d) structure in \REF{fofcstructures} is not linearizable on this approach.\footnote{\citet{sheehan2013some} does not take up the issue of how to express the exceptionality of Ā-movement and c-selectional relations between different extended projections as raised by \cite{biberaueretal2014}.}

Sheehan's PF approach and Biberauer et al.'s narrow syntactic approach therefore make different predictions about the reparability of FOFC-violating structures in the syntax.  Again, on the PF approach, but not the narrow syntactic approach a FOFC-violating structure should in principle be derivable in the syntax; it just will not be linearizable.  
 
One possible case of FOFC\is{Final-over-Final Condition} repair noted in previous literature involves \textit{Head-Final Filter} violations \citep{greenberg1963, williams1982, sheehan2012fofchff}.  A well-known restriction on adjectival modification cross-linguistically is a ban on complements of prenominal adjectives where the adjectival complement appears between the adjective and noun. (\citealt{williams1982} called this the \textit{Head-Final Filter}.)  \citet{sheehan2012fofchff}, in particular, argues that strings like \REF{ex:haddican:14c} should be analyzed as a FOFC\is{Final-over-Final Condition} effect and proposes a PF approach akin to the one described in \sectref{sheehanpf}.

\ea\label{ex:haddican:14} 
 \begin{xlista}
 \ex[]{the proud man}
 \ex[]{John is proud of his children.}
 \ex[*]{the [ɣP [βP proud [αP of his children]] man]\label{ex:haddican:14c}}
 \end{xlista}
 (adapted from \citealt{williams1982})
\z

As Sheehan notes, different languages employ different ``compliance strategies'' for contexts where a Head-Final Filter would otherwise arise.  One such case involves  extraposition of CP/PP complements of the prenominal adjective as in \REF{extraposition1} and \REF{extraposition2} in English and \ili{Slovenian} respectively.\is{complementizer}

\ea\protect\label{extraposition1} 
\begin{xlista}
\ex[]{a difficult book [for anyone to read]}
\ex[*]{a difficult [for anyone to read] book}
\end{xlista}
(adapted from \citealt{sheehan2012fofchff})\\
\z
 
\ea \protect\label{extraposition2}  
\gll {zavesten} {otrok,} {da} {je} {vojna} \\
      aware.\textsc{m} child.\textsc{m} that is.\oldstylenums{3}\textsc{sg} war.\textsc{f}\\
\glt `a child aware that there is a war'\\
(adapted from \citealt{sheehan2012fofchff})\\
\z

\citet{sheehan2012fofchff} follows \citet{Kayne1994} in taking prenominal adjectives to be reduced \isi{relative} clauses where the adjective raises from a postnominal position. 
 
\ea\relax [\textsubscript{DP} [\textsubscript{CP} [\textsubscript{AP} Adj [\textsubscript{NP} Noun \st{AP} ]]]] \z

\cite{sheehan2012fofchff} proposes that these repair effects might be reconciled with the PF approach to FOFC\is{Final-over-Final Condition} effects introduced above where FOFC-violating structures are not linearizable by the LCA.  In particular, Sheehan  suggests that the FOFC-violating structures might be repaired at copy deletion by ``scattered deletion'', whereby  ``extraposition'' of the FOFC-offending CP\is{complementizer}\slash PP in cases like \REF{extraposition1} and \REF{extraposition2} are achieved by deleting the higher rather than the lower copy of these constituents in order for them to be linearizable by the modified LCA.

In the following discussion, we describe a similar set of facts from \ili{Basque} verb clusters which suggest that chain reduction may bleed FOFC-violations in a similar way in the absence of scattered deletion.

\section{Word order and the functional richness of modal complements in Basque}\label{sec:haddican:3}\is{agreement|(}

The core set of facts that we focus on come from observations by Etxepare, Uribe-Etxe\-bar\-ria and colleagues concerning word order and the functional richness of infinitival complements of the modals  \textit{behar} `need' and \textit{nahi} `want' \citep{etxepare-uribeetxebarria2009, etxepare-uribeetxebarria2012, balza2010}.  As illustrated in \REF{ex:haddican:18} the constituent headed by \textit{ikusi}, `see', can appear either to the left or the right of the selecting \is{Modal}modal, \textit{nahi}, `want'.  

\ea\protect\label{E-UE-2}\label{ex:haddican:18}
\ea[]{ \label{ex:haddican:18a} \gll  {[Horrelakoak} {maiz-ago}   {ikusi]} {nahi}  {nituzke.} \\
    like.that\textsc{.pl}    often-more see   want  \textsc{aux}     \\ \protect\label{E-UE-2a}}
\ex[]{\label{ex:haddican:18b} \gll {Nahi} {nituzke} {[horrelakoak} {maiz-ago} {ikusi.]} \\   
         want  \textsc{aux}      like.that\textsc{.pl}   often-more  see  \\ \protect\label{E-UE-2b}
  \glt    `I'd like to see things like that more often.'\\}
\z\z
 
\cite{etxepare-uribeetxebarria2009, etxepare-uribeetxebarria2012} and \cite{balza2010} note that the word order difference illustrated in \REF{ex:haddican:18} correlates with three other properties suggesting that the modal-infinitival order in \REF{ex:haddican:18b} can involve a functionally richer infinitival complement than \REF{ex:haddican:18a}.  We describe these in  turn below.

\subsection{Temporal modification}

A first way in which in\-fin\-i\-ti\-val>mo\-dal and mo\-dal>in\-fin\-i\-ti\-val orders differ is in terms of the temporal independence of the non-finite\is{finiteness} constituent.  In in\-fin\-i\-ti\-val>mo\-dal orders, the infinitival phrase cannot contain a temporal modifier forcing a temporal interpretation of the event in the infinitival phrase that is different from that of the \is{Modal}modal+auxiliary\is{Auxiliary}.  In \REF{ex:haddican:19a}, the infinitival phrase contains \textit{gaur} `today' with a temporal interpretation different from the past interpretation of the \is{Modal}modal+auxiliary\is{Auxiliary}, and the result is poor.  On the other hand, Etxepare and Uribe-Etxebarria report that this temporal difference is fine in mo\-dal>in\-fin\-i\-ti\-val contexts such as \REF{ex:haddican:19b}.

\ea 
\ea[*]{ \label{ex:haddican:19a} 
\gll Jon-ek {atzo} {[gaur}    {etxe-a-n} {ego-n]}        {behar} {zuen.} \\
     Jon-\textsc{erg} yesterday today house-\textsc{def}-in be-\textsc{inf} need   \textsc{aux}\\}
\ex[ ]{ \label{ex:haddican:19b} 
\gll 	{Jon-ek}    {atzo}         {behar}  {zuen}  {[gaur}   {etxe-a-n }        {ego-n.]}\\
	Jon-\textsc{erg} yesterday need   \textsc{aux}   today house-\textsc{def}-in be-\textsc{inf}\\
\glt	`Yesterday Jon needed to be home today.'}
\z\z

\citet{etxepare-uribeetxebarria2009, etxepare-uribeetxebarria2012} take these facts to indicate that, in mo\-dal>in\-fin\-i\-ti\-val orders, the non-finite\is{finiteness} constituent may contain a T head with a \isi{tense} value different from that of the matrix clause.  In infinitive-\is{Modal}modal orders, on the other hand, the non-finite\is{finiteness} constituent cannot contain a separate T head.

\subsection{Agreement}
A second difference between the word orders concerns agreement.  Open class finite\is{finiteness} verbs in \ili{Basque} are formed periphrastically, with a verb root (bearing any aspectual \isi{morphology}) separate from the auxiliary\is{Auxiliary} that agrees in \isi{person} and \isi{number} with \isi{ergative}, absolutive and dative arguments of the main verb.  We illustrate this agreement in \REF{ex:haddican:20}.\footnote{On a closed class of synthetic verbs, \isi{tense} and agreement marking appear affixed to the verb root in some aspectual contexts.  \isi{Addressee} agreement works similarly for these forms but we set these forms aside for expositional convenience.} In the examples in \REF{ex:haddican:20}, \isi{ergative}, absolutive and dative arguments are all overt; however, we note that \ili{Basque} allows \isi{pro-drop} with all three of these argument types. 
 
\ea \label{ex:haddican:20}\label{agreement}
    \ea \gll  {Ni} {joa-n} {na-iz.}\\	
              \oldstylenums{1}\textsc{sg.abs} go.\textsc{prf} \oldstylenums{1}\textsc{sg.abs-root}\\
 	    \glt  `I have gone.'  \hfill{[unaccusative]}
    \ex \gll  {Katu-ek} {ni} {ikus-i} {na-u-te.}\\	
 	      cat-\oldstylenums{3}\textsc{pl.erg} \oldstylenums{1}\textsc{sg.abs} see-\textsc{prf} \oldstylenums{1}\textsc{sg.abs-root}-\oldstylenums{3}\textsc{pl.erg}\\
        \glt    `The cats have seen me.' \hfill{[monotransitive]}
    \ex \gll  {Ni-k} {liburu-ak} {Jon-i} {ema-n} {d-i-zki-o-t.}\\	
          \oldstylenums{1}\textsc{sg-erg} books-\textsc{pl.abs} Jon-\textsc{dat} give-\textsc{prf} \oldstylenums{3}\textsc{sg.abs-root-abs.pl}-\oldstylenums{3}\textsc{sg.dat}-\oldstylenums{1}\textsc{sg.erg} \\
        \glt `I have given Jon the book.'  \hfill{[ditransitive]}
    \ex \gll  {Ni} {Jon-i} {mintza-tzen} {na-tzai-o.}\\	
 		\oldstylenums{1}\textsc{sg.abs} Jon-\textsc{dat} speak-\textsc{ipfv} \oldstylenums{1}\textsc{sg.abs-root}-\oldstylenums{3}\textsc{sg.dat}   \\
 		\glt `I speak to Jon.'   \hfill{[\isi{applicative} unaccusative]}
\z\z

In addition, \is{Modal}modal verbs that take infinitival complements are transparent to plural absolutive and dative agreement marking in transitive constructions.  In sentences with the \is{Modal}modal \textit{behar} `must',  agreement marking on the auxiliary\is{Auxiliary} is exhaustively determined by the argument structure of the lower verb as shown in \REF{ex:haddican:21}, below.  

\ea \label{ex:haddican:21}
    \ea[\%]{\gll {Joan} {behar}   {na-iz.}	\\						
     go    must   \oldstylenums{1}\textsc{sg.abs-root}\\
    \glt `I must go.'\footnotemark \hfill{[unaccusative]}}
    \ex[]{\gll  {Katu-ek} {ni} {ikusi} {behar} {na-u-te.}\\	
	      cat-\oldstylenums{3}\textsc{pl.erg} \oldstylenums{1}\textsc{sg.abs} see need \oldstylenums{1}\textsc{sg.abs-root}-\oldstylenums{3}\textsc{pl.erg}\\
   \glt     `The cats must see me.' \hfill{[monotransitive]}}
   \ex[]{ \gll  {Jon-i}       {liburu-ak}       {eman} {behar}   {d-i-zki-o-t.} \\
        Jon-\textsc{dat} books-\textsc{pl.abs}  give    need  \oldstylenums{3}\textsc{.abs-root-pl.abs-\oldstylenums{3}sg.dat-\oldstylenums{1}sg.erg}  \\
   \glt  `I must give Jon the books.'   \hfill{[ditransitive]}}
\z\z
\footnotetext{In some dialects, the \is{Modal}modal \textit{behar} determines the transitive auxiliary \textit{*edun} in unaccusative contexts.}  
 
As \cite{etxepare-uribeetxebarria2009} note, both absolutive plural agreement and dative agreement patterns are constrained by the position of the infinitival.  As shown in \REF{ex:haddican:22}, in the mo\-dal>in\-fin\-i\-ti\-val order, absolutive plural agreement is optional.
 
\ea \label{ex:haddican:22} 
	\ea[]{ \gll  	{Nahi}    {n-\textbf{it}-u-z-ke} {[horr-ela-ko-a-k }           {maiz-ago }         {ikus-i].}    \\
 	want   \oldstylenums{1}\textsc{sg.erg-\textbf{pl.abs}-root-pl.abs-irr}    that-like-\textsc{gen-def-pl}  frequent-more see-\textsc{inf} \\}
\ex[]{ \gll  	{Nahi}    {n-u-ke}  {[horr-ela-ko-a-k}   {maiz-ago}  {ikus-i].} \\
         	want   \oldstylenums{1}\textsc{sg.erg-root-irr}    that-like-\textsc{gen-def-pl}  frequent-more see-\textsc{inf} \\
    \glt   `I'd like to see things like that more often.'\\}
    \citep{etxepare-uribeetxebarria2009}
\z\z

In the in\-fin\-i\-ti\-val>mo\-dal order, on the other hand, plural absolutive agreement on the auxiliary\is{Auxiliary} is obligatory.

\ea	
\ea[]{ \gll   {[Horr-ela-ko-a-k }      {maiz-ago}     {ikus-i]}     {nahi}   {n-\textbf{it}-u-z-ke.}\\
               that-like-\textsc{gen-def-pl}  frequent-more see-\textsc{inf}  want \oldstylenums{1}\textsc{sg.erg-\textbf{pl.abs}-root-pl.abs-irr}  \\}
\ex[*]{ \gll  [\textit{Horr-ela-ko-a-k} \textit{maiz-ago} \textit{ikus-i}] \textit{nahi} \textit{nuke}.   \\  
         that-like-\textsc{gen-def-pl}  frequent-more see-\textsc{inf}  want \oldstylenums{1}\textsc{sg.erg-root-irr}  \\   
        \glt `I'd like to see things like that more often.'\\}
        \citep{etxepare-uribeetxebarria2009}
\z\z
 
Agreement with dative arguments is similarly constrained.  \REF{ex:haddican:24} shows that that dative agreement is optional in the mo\-dal>in\-fin\-i\-ti\-val order. 
 
\ea \label{ex:haddican:24} 
	\ea \gll  	\textit{Behar}  \textit{zen-i-e-ke}   [\textit{zure} \textit{guraso-ei}   \textit{obeditu}]. \\
             	must   \oldstylenums{2}\textsc{abs-root-dat.pl-irr}    your parent-\textsc{dat.pl}   obey  \\
 	\ex \gll 	\textit{Behar} \textit{zen-u-ke}    [\textit{zure}  \textit{guraso-ei} \textit{obeditu}]. \\
 	    must   \oldstylenums{2}\textsc{abs-root-irr}   your parent-\textsc{dat.pl}   obey    \\    
 	\glt  	`You should obey your parents.'\\
    \citep{etxepare-uribeetxebarria2009}
\z\z	

As shown in \REF{ex:haddican:25}, this agreement is obligatory when the order is in\-fin\-i\-ti\-val>mo\-dal:
 
\ea \label{ex:haddican:25} 
    \ea[]{ \gll  {[Zure} {guraso-ei} {obeditu]} {behar}  {zen-i-e-ke.}  \\
              	your parent-\textsc{dat.pl}   obey     must   \oldstylenums{2}\textsc{sg.abs-root-dat.pl-irr}  \\}
 	\ex[*]{ \gll 	{[Zure} {guraso-ei} {obeditu]} {behar} {zen-u-ke.} \\
 	    your parent-\textsc{dat.pl}   obey     must   \oldstylenums{2}\textsc{sg.abs-root-irr}  \\
 		\glt `You should obey your parents.' \\}
        \citep{etxepare-uribeetxebarria2009}
\z\z 			
 
 
These agreement restrictions stand to reason on the assumption that the loci for dative and absolutive case in transitive contexts is not T but rather some set of vP-internal heads -- v and \isi{Appl} for instance -- and that the agreement morphemes on the auxiliary\is{Auxiliary} reflect head movement from v/\isi{Appl} to T \citep{arregi-molinaazaola2004, rezac2008}.  On this approach, the unavailability of dative and plural absolutive agreement on the finite\is{finiteness} auxiliary\is{Auxiliary} plausibly reflects the presence of a lower T blocking movement to the higher T.\is{agreement|)}

\subsection{Negation}
 
The above sets of facts plausibly indicate that, in mo\-dal>in\-fin\-i\-ti\-val but not infinitival>\linebreak \is{Modal}modal orders, the non-finite\is{finiteness} constituent may contain a T head.  A final set of facts, however, suggests that in modal-\isi{infinitive} orders the non-finite\is{finiteness} constituent can be somewhat larger than \isi{TP} -- containing minimally a TP-external \isi{negation} \isi{projection}. \cite{balza2010} and \cite{etxepare-uribeetxebarria2009} observe that non-finite\is{finiteness} constituents to the left of the \is{Modal}modal can never contain the sentential \isi{negation} morpheme \textit{ez}, which appears to the left of the auxiliary\is{Auxiliary} in \ili{Basque} \citep{Laka1990}.  In contrast, when the \isi{infinitive} appears to the right of the \is{Modal}modal, \textit{ez} can indeed appear.  This contrast is illustrated in \REF{ex:haddican:26}.  
 
\ea	\label{ex:haddican:26} 
 \ea[*]{\label{ex:haddican:26a} \gll 	{[Ez} {eros-i]}  {nahi/behar} {n-u-ke.}\\
    	\textsc{neg} buy-\textsc{inf} want/need   \oldstylenums{1}\textsc{sg.erg-root-irr}   \\}
 \ex[]{\label{ex:haddican:26b} \gll 	{Nahi/behar} {n-u-ke} {[ez} {eros-i].}\\
      want/need  \oldstylenums{1}\textsc{sg.erg-root-irr}    \textsc{neg} buy-\textsc{inf} \\
     \glt `I want\slash need not to buy it.' \\}
     \citep{etxepare-uribeetxebarria2009}\\
\z\z

As \cite{etxepare-uribeetxebarria2009} note, the \isi{negation} in \REF{ex:haddican:26b} is not plausibly an instance of constituent \isi{negation} since constituent \isi{negation} does not license a higher, clausemate negative \isi{polarity} item (NPI).  Example \REF{ex:haddican:27a}, illustrating constituent \isi{negation} in a non-\is{Modal}modal context, shows that the higher NPI \textit{inork} `anybody' is not licensed, unlike a true sentential \isi{negation} context such as \REF{ex:haddican:27b}.  

\ea	
\ea[*]{\label{ex:haddican:27a} \gll 	{Inork} {(ere)}     {du}     {ez}  {eros-i.} \\
    	Anybody (at-all) \textsc{aux} \textsc{neg} buy-\textsc{inf} \\
    \glt    `Nobody at all bought it.'}
 	\ex[]{\label{ex:haddican:27b} \gll  	{Inork} {(ere)}       {ez}    {du}   {eros-i.}   \\
 		Anybody (at-all) \textsc{neg} \textsc{aux} buy-\textsc{inf}  \\
     \glt `Nobody at all bought it.'}
\z\z
 
\REF{ex:haddican:28} shows that \textit{ez} in mo\-dal>in\-fin\-i\-ti\-val contexts behaves like sentential \isi{negation} in licensing the higher NPI, \textit{deus}, `anything'.  \cite{balza2010} and \cite{etxepare-uribeetxebarria2009} take these facts to indicate that the non-finite\is{finiteness} constituents in these environments can contain a negative head.
 
\ea\label{ex:haddican:28}
\gll {Nahi}  {nuke} {deus} {(ere)} {ez} {eros-i.}\\
     want \textsc{aux}   anything at.all \textsc{neg} buy-\textsc{inf} \\
\glt	`I'd like to not buy anything (at all).'
\z
 
To summarize, we have described four sets of facts drawn mainly from \citet{etxepare-uribeetxebarria2009, etxepare-uribeetxebarria2012} and \citet{balza2010} suggesting that the two word orders discussed above correspond to different internal structures of the non-finite\is{finiteness} constituent. The infinitival phrase in in\-fin\-i\-ti\-val>mo\-dal orders is smaller than \isi{TP} -- a vP, we'll assume -- while the infinitival in \is{Modal}modal>aux>infinitival orders can be a \isi{TP} and may contain a TP-external \isi{negation} position as well.  We illustrate this proposal with the sequences of functional heads in \protect\REF{structuresensitivity}, (repeated here) which abstract away from surface linear order.
 
\ea	\protect\label{structuresensitivity} \ea \isi{Infinitival}>\is{Modal}modal orders: [T [\isi{Modal} [v [V \ldots  
 	                                     \ex \isi{Modal}>infinitival orders: [T [\isi{Modal} ([Neg\is{negation}) ([T)  [v [V \ldots
\z\z										 

As noted earlier, \citet{balza2010} and \citet{etxepare-uribeetxebarria2009} do not provide an account of the contrast in \REF{structuresensitivity}.  In the following discussion, we argue that this contrast is explainable as a garden variety FOFC\is{Final-over-Final Condition} effect from the perspective of antisymmetric approaches to \ili{Basque}.

\section{FOFC and word order in Basque verb clusters}\label{sec:haddican:4}
\subsection{Antisymmetry and polarity-sensitive word order alternations}

\ili{Basque} is a \textit{mixed-head} language: heads in the clausal spine below T appear to the right of their complements, while heads above T, including preverbal speech act and evidential particles appear to the left of their complements \citep{derijk1969, ortizdeurbina1989, ortizdeurbina1994, Laka1990,elordieta2001, irurtzun2007, elordieta2008}. Most generative approaches to \ili{Basque} have modeled these facts in terms of a head-directionality\is{head directionality} parameter: T and clausal heads below it take their complements to the left, while those heads above T take their complements to the right. The head-final nature of TP-internal projections, on this approach, usefully accounts the fact that in neutral declarative sentences like \protect\REF{affirmativeclauses}, the finite\is{finiteness} verb -- presumably in T -- appears sentence finally.

\ea \label{ex:haddican:30} \textit{{Affirmative main clauses}} \protect\label{affirmativeclauses}\\
\gll {Miren-ek}   {Jon} {ikus-i}      {du.}   \\				
      Miren-\textsc{erg} Jon-\textsc{abs} see-\textsc{prf}   \textsc{aux}.\oldstylenums{3}\textsc{sg.erg} \\
\glt `Miren has seen Jon.'
\z
 
In negative sentences, the negative morpheme \textit{ez} appears left-adjacent to the auxiliary\is{Auxiliary} and the VP appears to the right of the auxiliary\is{Auxiliary} as in \REF{ex:haddican:31}.\largerpage[2]

\ea \label{ex:haddican:31} \textit{{Negative main clauses}}\protect\label{negativeclauses}\\
\gll {Miren-ek}   {ez}     {du}   {Jon} {ikus-i.} 	\\			
      Miren-\textsc{erg} \textsc{neg} \textsc{aux}.\oldstylenums{3}\textsc{sg.erg} Jon-\textsc{abs} see-\textsc{prf} \\
\glt      `Miren hasn't seen Jon.'
\z 

\citet{Laka1990} and \citet{elordieta2001, elordieta2008} propose that these \isi{polarity} effects reflect the fact that \isi{negation} -- which is head-initial on this approach -- is first-merged outside \isi{TP} in \ili{Basque}, and that the inflected verb must head adjoin to \isi{negation} as a way of providing lexical support for the clitic-like auxiliary\is{Auxiliary}. The Neg\is{negation}>Aux\is{Auxiliary} word order requires that this be right head adjunction as shown in \REF{ex:haddican:32}.  In affirmative sentences, Neg\is{negation} is not merged, and the auxiliary\is{Auxiliary} stays in its first-merged position in \isi{TP}.  
 
\ea \label{ex:haddican:32} \textit{The head movement approach \citep{Laka1990}} \label{tree-2} \is{AspP}\\ 
 
 \begin{tikzpicture}[baseline]\tikzset{level distance=30pt, sibling distance=2.5pt}
 \Tree
 [.{NegP/ΣP} {[Neg-[Aux\is{Auxiliary}]]}  [.\isi{TP} [.AspP [.vP [.VP V ] {\st{[V]-v}}  ] {[[[V]-v]-Asp\is{Aspect}]}  ] \st{Aux} ] ]
 \end{tikzpicture}
\z
  

On an approach that eschews head-directionality\is{head directionality} parametrization, a different account is required for the polarity-sensitive word order alternations illustrated in \REF{affirmativeclauses} and \protect\REF{negativeclauses}.  In particular, following \citet{haddican2004a,haddican2008}, we propose that  (i) the left-branch\-ing structure of the extended VP is derived via roll up \citep{Kayne1994}, and (ii) the relative order of the auxiliary\is{Auxiliary} and extended verbal \isi{projection} reflects the presence or absence of \isi{fronting} of the extended verbal \isi{projection}, a constituent that we will label PolP, for reasons to be made clear shortly.  We adopt Laka's (\citeyear{Laka1990}) seminal proposal that \ili{Basque} has a left peripheral \isi{polarity} head, Σ.  We propose that this head probes for polarity-specified elements.  Two such elements will be the the negative and the emphatic affirmative morphemes \textit{ez} and \textit{bai}, which we take to be  \isi{polarity} adverbs merged in the Spec of  PolP. These forms, where present, will raise to Spec,ΣP, as illustrated in the negative example in \REF{ex:haddican:34}.\largerpage[1]
	
\ea \label{ex:haddican:34} Ez-\textit{raising, negative contexts} \protect\label{tree-5} \\  \samepage
\begin{tikzpicture}[baseline]\tikzset{level distance=25pt, sibling distance=5pt}
\Tree
[.ΣP \textit{ez} [.Σ′ Σ [.\isi{TP} Aux\is{Auxiliary} [.PolP \st{\textit{ez}} [.Pol\is{polarity}′ Pol\is{polarity} {\ldots} ] ] ] ] ]
\end{tikzpicture}
\z\newpage

In affirmative root contexts, the position to the left of the auxiliary\is{Auxiliary} is not occupied by the negative morpheme \textit{ez}, but rather by the extended \isi{projection} of the verb.  We propose that, in these contexts, in the absence of \textit{ez}, the extended verbal phrase raises to Σ to satisfy the latter's \isi{polarity} feature.  Specifically, in the spirit of predicate \isi{fronting} approaches to VSO and VOS word orders  \citep{massam2000, massam2001, massam2010, coon2010, coon2012}, let us assume that what raises is a PolP whose head contains an affirmative \isi{polarity} [Aff] feature.  In the verb-initial orders they analyze, Massam and Coon take the landing site of this movement to be \isi{TP}\slash IP, and relate this movement to the featural needs of T/C.  In \ili{Basque}, we take this movement to be related to featural needs of a polarity-related morpheme, namely  Σ. We illustrate this in \REF{ex:haddican:35}.\protect\footnotemark \protect\footnotetext{See \cite{haddican2004a, haddican2008} and \cite{etxepare-uribeetxebarria2009} for similar approaches.}
 	 
\ea \label{ex:haddican:35} \textit{Affirmative orders} \protect\label{tree-6} \\ \samepage
 \begin{tikzpicture}[baseline]\tikzset{level distance=25pt, sibling distance=5pt}
 \Tree
 [.ΣP [.PolP Pol\is{polarity} {\ldots}  ] [.Σ′ Σ [.\isi{TP} Aux\is{Auxiliary} \st{PolP} ] ] ]
 \end{tikzpicture}
\z

Evidence in favor of predicate \isi{fronting} in affirmative clauses comes from \isi{TP} ellipsis sentences like \REF{ex:haddican:36} and \REF{ex:haddican:37}. In both cases, the auxiliary\is{Auxiliary} in the second sentence is left unpronounced, plausibly as a banal case of \isi{TP} ellipsis \citep{Laka1990}.  Interestingly, the elements which escape TP-ellipsis are different in affirmative and negative sentences: whereas in negative sentences (and in those involving contrastive affirmation) the verbal predicate is elided together with the finite\is{finiteness} auxiliary\is{Auxiliary} \REF{ex:haddican:37}, in simple affirmative sentences the verbal predicate escapes TP-ellipsis, by virtue of obligatory predicate raising \REF{ex:haddican:36}. On the head directionality\is{head directionality} approach, additional movement operations are required to derive such sentences.
 
\ea \label{ex:haddican:36} 
\gll  Jon-ek kafe-a erosi du, eta Ane-k, [\textsubscript{ΣP} [\textsubscript{PolP} [liburu-a leitu] Σ [\textsubscript{\sout{TP}} \sout{du}]. \\
	Jon-\textsc{erg} coffee bought \textsc{aux} and Ane-\textsc{erg} \textcolor{white}{x} \textcolor{white}{x} book-the read\\
\glt `Jon has bought coffee, and Ane has read the book.'
\z

\ea \label{ex:haddican:37}
\gll  Jon-ek kafe-a erosi du, baina Ane-k, [\textsubscript{ΣP}  ez Σ \textsubscript{\sout{TP}} \sout{du kafe-a erosi}].\\
	Jon-\textsc{erg} coffee bought \textsc{aux} but Ane-\textsc{erg}  \textcolor{white}{x} \textsc{neg} \textcolor{white}{x} \\ 
\glt `Jon has bought coffee, but Ane hasn't.'
\z
 
Evidence that the extended VP indeed contains a \isi{polarity} feature in affirmative contexts comes from \isi{polarity} focus sentences like \REF{ex:haddican:38}.  Here, the extended VP raises to a left peripheral focus position and co-occurs with an affirmative denial interpretation, suggesting the raised verbal constituent is the locus of the affirmative feature.

\ea \label{ex:haddican:38}
    \gll [\textsubscript{\textit{FocP}} [\textsubscript{\textit{PolP}} \textit{Etorri}] [\textsubscript{\textit{TP}} \textit{da} \textit{Iker.}]] \\
	{} {} come {} \textsc{aux} Iker\\
    \glt `Iker HAS (indeed) come.' \protect\label{iker}
\z

What is important about this approach for the FOFC-effects focused on here is that \isi{TP} is a left-headed \isi{projection} that does not participate in roll-up movement; that is, the complement of T does not move to its spec.  From this perspective, and assuming that non-fintite T is like finite\is{finiteness} T in not participating in roll-up movement, Etxepare and Uribe-Etxebarria's observed correlation between word order and size of the \is{Modal}modal complement is explicable as a vanilla FOFC\is{Final-over-Final Condition} effect.  That is, what makes the functionally richer constituents impossible in the \isi{infinitive}>\is{Modal}modal order is the presence of a head-complement structure in the spec of the \is{Modal}modal phrase, in violation of \protect\REF{fofc2}.  Specifically, the complement of the infinitival T is not spelled out in the spec of the non-finite\is{finiteness} T, but rather as the sister of T.  The infinitival T itself then moves to the spec of the \is{Modal}modal \isi{projection} and runs afoul of \protect\REF{fofc1}. In contrast, vP-sized infinitives will not run afoul of FOFC\is{Final-over-Final Condition}, as stated in \protect\REF{fofc1} and \protect\REF{fofc2}, since v \textit{does} participate in roll-up; that is, it attracts its complement to its spec. We illustrate this proposal in \REF{ex:haddican:39} and \REF{ex:haddican:40}.


\ea \label{ex:haddican:39} \textit{FOFC-violating TP-raising} \protect\label{tree-8} \\  \samepage
\begin{tikzpicture}[baseline]\tikzset{level distance=25pt, sibling distance=5pt}
\Tree
[.ModalP [.\isi{TP} T vP ] [.\isi{Modal}′ \isi{Modal} \st{TP} ] ]
\end{tikzpicture}
\z 

\ea \label{ex:haddican:40}  \textit{FOFC-compliant vP-raising} \protect\label{tree-9} \\  \samepage
\begin{tikzpicture}[baseline]\tikzset{level distance=25pt, sibling distance=5pt}
\Tree
[.ModalP [.vP VP [.v′ v \st{VP} ] ] [.\isi{Modal}′ \isi{Modal} \st{vP} ]  ]\end{tikzpicture}
\z

From the perspective of the antisymmetric approach to polarity-sensitive word order alternations described above, the structure-sensitivity of the word order alternations described by \cite{etxepare-uribeetxebarria2009} is therefore predicted as a FOFC\is{Final-over-Final Condition} effect. From a mixed head perspective, where T takes its complement on the left, some other account of Etxepare and Uribe-Etxebarria's observation is required.\footnote{For the same reason, Sheehan's (\citeyear{Sheehan2013fofc,sheehan2013some,sheehan2012fofchff}) approach will fail to express the structure-sensitivity of these word order alternations as a FOFC\is{Final-over-Final Condition} phenomenon if T is parameterized to take its complement to its left.  Sheehan's theory, though, entails no commitment to such a derivation versus an XP movement approach of the kind just proposed.}


\subsection{Repairing the violation}

The account so far explains why vP-, but not TP-sized \is{Modal}modal complements can raise to the specifier of the \is{Modal}modal.  Unaddressed so far is why TP-sized \is{Modal}modal complements are licit when they appear to the right of the \is{Modal}modal as in \protect\REF{E-UE-2b}.  A further fact about the alternation in \protect\REF{E-UE-2} that we take to be central to this issue is the fact that the modal-aux-\isi{infinitive} order is most readily available in contexts in which the non-finite\is{finiteness} constituent to the right is focalized or contains a focus-bearing constituent. The \is{Modal}modal+auxiliary\is{Auxiliary} sequence to the left is preferably defocused.  From this perspective, sentences like \protect\REF{E-UE-2b} are reminiscent of cases of right peripheral focus constructions as in \REF{ex:haddican:41} and \REF{ex:haddican:42}.


\ea \label{ex:haddican:41} 
\gll \textit{Ardoa} \textit{ekarri} \textit{diot} \textit{(\#)} \textit{ANDONI-RI.} \\
wine brought \textsc{aux} {} Andoni-\textsc{dat} \\
\glt `I brought the wine to ANDONI.'\\
\citep{elordieta2001} \\
\z

\ea  \label{ex:haddican:42} 
\gll \textit{Monjak} \textit{egin} \textit{zigun} [\textit{barruan} \textit{utz-i}.] \\
nuns do \textsc{aux} inside leave-\textsc{inf} \\
\glt `The nuns LEFT US INSIDE.' \\
\citep{haddican2007} \\
\z

\citet{ortizdeurbina2002} and \citet{uribeetxebarria2003} propose that sentences such as \REF{ex:haddican:41} and \REF{ex:haddican:42} are derived by movement of the focused constituents to a left-peripheral focus position, followed by remnant movement of the non-focused portion of the sentence to a higher topic phrase.  We illustrate this proposal in \REF{ex:haddican:43}. As \cite{ortizdeurbina2002} notes, this approach is supported by the fact that the remnant-moved material shares intonational properties with other pre-focus topic constituents.  

\ea \label{ex:haddican:43} 
\gll [TopP [ \textit{Ardoa} \textit{\st{Andoni-ri}} \textit{ekarri} \textit{diot}] \textit{Top} [FocP [ \textit{Andoni-ri} ]\ldots  ]  \\
{} {} wine {} brought \textsc{aux} {} {} {} Andoni-\textsc{dat} \\\protect\label{ardoa}
\glt `I brought the wine to ANDONI.'  
\z

Some support for remnant movement comes from the relative scope of focus and \isi{negation}. When following the lexical verb, the favored scope of the focal constituent is maximal with regard to \isi{negation}, as diagnosed by the continuation \textit{and not DP}. In this regard, it behaves like left-peripheral foci as in \REF{ex:haddican:44b} \citep{ortizdeurbina2002}.
	
\ea	
\ea \label{ex:haddican:44a} \gll  \textit{Ez} \textit{diot} \textit{liburua} \textit{oparitu} \textit{ANDONI-RI,} \textit{eta} \textit{ez} \textit{Miren-i.}\\
\textsc{neg} \textsc{aux} book-the offered Andoni-\textsc{dat}, and \textsc{neg} Miren-\textsc{dat}\\
\glt `The one I did not offer the book to is Andoni, and not Miren.'
\ex \label{ex:haddican:44b} \gll  \textit{ANDONI-RI} \textit{ez} \textit{diot} \textit{liburua} \textit{oparitu,} \textit{eta} \textit{ez} \textit{Miren-i.}\\
Andoni-\textsc{dat} \textsc{neg} \textsc{aux} book-the offered and \textsc{neg} Miren-\textsc{dat}\\
\glt `It is Andoni that I didn't offer the book to, and not Miren.'
\z\z

Also, note that wide-scope foci in non-initial position must occupy the right edge of the clause as suggested by the fact that they cannot be followed linearly by clausal material:

\ea
\ea \gll  \textit{Jon-ek} \textit{ez} \textit{du} \textit{liburu-rik} \textit{irakurri} \textit{BULEGOAN,} \textit{eta} \textit{ez} \textit{trenean.}\\
Jon-\textsc{erg} \textsc{neg} \textsc{aux} book-\textsc{inf} read office-in and \textsc{neg} train-in\\
\glt `The place Jon did not read any book is the office, not the train.'
\ex \gll  \textit{Jon-ek} \textit{ez} \textit{du} \textit{irakurri} \textit{(liburu-rik)} \textit{BULEGOAN} \textit{(*liburu-rik),} \textit{eta} \textit{ez} \textit{trenean.}\\
Jon-\textsc{erg} \textsc{neg} has read book-\textsc{part} office-in book-\textsc{part} and \textsc{neg} train-in\\
\glt `The place Jon did not read any book is the office, not the train.'	
\z\z 	
	
The crux of our proposal about FOFC\is{Final-over-Final Condition} repair is as follows.  In mo\-dal>in\-fin\-i\-ti\-val orders such as \protect\REF{E-UE-2b}, affirmative PolP moves to ΣP, as usual. The FOFC-offending infinitival \isi{TP} then subextracts to a \isi{Focus} phrase, as in \REF{ex:haddican:46}. The position of the \is{Modal}modal to the left of the infinitival is derived via remnant topicalization, not shown here. Crucially, because the \isi{TP} targets an A-bar position, this movement step is FOFC-exempt. (See \cite{biberaueretal2014} for discussion.)

\ea \label{ex:haddican:46} \textit{PolP movement to ΣP and sub-\isi{extraction} of infinitival TP} \\
\begin{tikzpicture}
\Tree
[.FocusP \node(infin-3){\isi{TP}\scriptsize{infinitival}}; [.Foc\is{Focus}´ Foc\is{Focus} [.ΣP [.PolP Pol\is{polarity} [.ModalP \node(infin-2){\st{TP}\scriptsize{\st{infinitival}}}; [.\isi{Modal}´ \isi{Modal} \node(infin-1){\st{TP}\scriptsize{\st{infinitival}}}; ] ] ]  [.Σ´ Σ [.\isi{TP} Aux\is{Auxiliary} \st{PolP} ] ] ] ] ]
\draw[semithick,->] (infin-1)..controls +(south:1.5) and +(south:1.5)..(infin-2);
\draw[semithick,->] (infin-2)..controls +(west:2.2) and +(south:2.2)..(infin-3);
\end{tikzpicture}
\z

The derivation in \REF{ex:haddican:46} requires that freezing effects do not apply in this context \citep{collins2005a, collins2005b}.  We do not consider in detail what conditions the availability of subextraction here, but note that independent evidence of the ability of focused constituents to extract from moved XPs come from examples like \REF{ex:haddican:47} and discussed by \cite{elordieta2008}.  Here, the \textit{wh-}phrase, \textit{norekin} `with who', cyclically moves from within a moved CP\is{complementizer}.

\ea \label{ex:haddican:47} 
\gll \textit{Nor-ekin} \textit{pentsa-tu} \textit{duzu} \textit{[CP} \textit{\st{nor-ekin}} \textit{ezkondu} \textit{behar} \textit{naiz-ela]} \textit{agindu} \textit{didate-la} \textit{\st{CP}?}\\
who-with think-\textsc{prf} \textsc{aux} \textcolor{white}{text}  who-with marry must \textsc{aux-C} order \textsc{aux-C}\\
\glt `Who did you think they told me I had to get married with?'
\z

Sub\isi{extraction} of \isi{TP} to the \isi{Focus} Phrase is followed by remnant topicalization of ΣP, as in (\protect\ref{ardoa}). Evidence that the \is{Modal}modal in the relevant cases sits in a derived position comes from complex functional sequences preceding the non-finite\is{finiteness} constituent that cannot be generated in-situ \citep{etxepare-uribeetxebarria2009}. Consider \REF{ex:haddican:48}:

\ea \label{ex:haddican:48}
\gll \textit{Nahi} \textit{izan}   \textit{du}   \textit{beranduago} \textit{etorri.}	\\
        want \textsc{prf} \textsc{aux} later            come\\
\glt `She/he has wanted to come later.' 
\z


In \REF{ex:haddican:48}, the perfect head follows the \is{Modal}modal, which it selects, and precedes the auxiliary\is{Auxiliary}, which in turn precedes the non-finite\is{finiteness} verb. The hierarchical relations among the different components of the matrix-clause functional sequence can be represented in terms of either a head-final structure or roll-up movement, but the relative ordering of that sequence and the non-finite\is{finiteness} verb cannot: the \is{Modal}modal verb selects the non-finite\is{finiteness} \isi{TP}, but the two elements appear on opposite sides of the sequence, and separated by other clausal heads. Remnant movement provides a simple rationale for this ordering, and is well attested in other \ili{Basque} focal constructions. \REF{ex:haddican:49} lays out the derivational steps necessary to arrive to a configuration such as \REF{ex:haddican:48}, starting from the merger\is{Merge} of the \isi{Modal} head with the TP-infinitival \REF{ex:haddican:49a}:


\ea \label{ex:haddican:49} \is{AspP}
\ea \label{ex:haddican:49a} \isi{Merge} \is{Modal}modal \textit{nahi} with infinitival \isi{TP}:\\\relax
[ModalP \textit{nahi} [\isi{TP} \textit{etorri} ]]  
\ex \isi{Infinitival} \isi{TP} rolls up with ModalP:\\\relax
 [ModalP [\isi{TP} \textit{etorri}] \textit{nahi} \st{TP} ]
\ex \isi{Merge} Aspect (\isi{participle} \textit{izan}):\\\relax
 [AspP \textit{izan} [ModalP [\isi{TP} \textit{etorri} ] \textit{nahi}  ]] 
\ex ModalP rolls up with AspP:\\\relax
 [AspP [ModalP [\isi{TP} \textit{etorri} ] \textit{nahi} ]  \textit{izan} \st{ModalP}  ]
\ex \isi{Merge} Pol\is{polarity} and finite\is{finiteness} auxiliary\is{Auxiliary} in T:\\\relax
 [\isi{TP} \textit{du} [PolP Pol\is{polarity} [AspP [ModalP [\isi{TP} \textit{etorri} ] \textit{nahi} ]  \textit{izan}   ]]]
\ex \isi{Merge} Σ:\\\relax
 [ΣP Σ [\isi{TP} \textit{du} [PolP Pol\is{polarity} [AspP [ModalP [\isi{TP} \textit{etorri} ] \textit{nahi} ]  \textit{izan} ]]]]
\ex Predicate \isi{fronting} -- PolP raising to ΣP:\\\relax
 [ΣP  [PolP Pol\is{polarity} [AspP [ModalP [\isi{TP} \textit{etorri} ] \textit{nahi} ]  \textit{izan}   ]] Σ [\isi{TP} \textit{du} \st{PolP} ]]
\ex \isi{Merge} focus head and move infinitival \isi{TP} to spec, Foc\is{Focus}:\\\relax
 [FocP [\isi{TP} \textit{etorri}] Foc\is{Focus} [ΣP [PolP Pol\is{polarity} [AspP [ModalP \st{\textit{etorri}} \textit{nahi} ]  \textit{izan}]] Σ [\isi{TP} \textit{du} ]]]
\ex \isi{Merge} \isi{Topic} head and remnant move ΣP to spec, \isi{Topic}:\\\relax
 [TopP   [ΣP [PolP Pol\is{polarity} [AspP [ModalP  \textit{nahi} ]  \textit{izan}]] Σ [\isi{TP} \textit{du} ]] Top\is{Topic} [FocP [\isi{TP} \textit{etorri}] Foc\is{Focus} \st{ΣP}]
\z\z

To summarize, the importance of the foregoing facts for the debate between narrow syntactic and PF-based approaches to FOFC\is{Final-over-Final Condition} is that they suggest a derivation whereby a FOFC-violating structure is assembled, but then repaired by a subsequent movement step. The analysis, if correct, entails that Biberauer et al.'s narrow-syntax approach to FOFC\is{Final-over-Final Condition}, where FOFC-violating structures are simply not buildable in the syntax, cannot be correct. Rather, they suggest that copy-deletion can bleed FOFC\is{Final-over-Final Condition}. This, in turn, means that FOFC-evaluation is derivationally subsequent to copy-deletion, in the phonological component of the grammar on standard approaches \citep{Nunes2004}.
	
\section{Conclusion}\label{sec:haddican:5}
This paper has presented an analysis of word-order restrictions in \ili{Basque} \is{Modal}modal constructions described in recent work by \cite{etxepare-uribeetxebarria2009, etxepare-uribeetxebarria2012}.  We have shown that the relevant restrictions are explained as an utterly banal FOFC\is{Final-over-Final Condition} effect on antisymmetric approaches to \ili{Basque}, but not on a traditional mixed head approach.  The analysis of \ili{Basque} verb clusters presented, if correct, entails that Biberauer et al.'s narrow syntactic approach to FOFC\is{Final-over-Final Condition} effects is not correct and instead recommends a PF-based approach. How this might be achieved, whether by Sheehan's promising analysis (\citeyear{Sheehan2013fofc, sheehan2013some}) or another approach, might usefully be investigated in future work.
	
\section*{Acknowledgements}
This chapter is dedicated to Anders Holmberg for his inspirational leadership in theoretical linguistics for a generation.  We are grateful to him for his ideas, his mentorship and friendship.  Thanks to Laura Bailey, Michelle Sheehan and an anonymous reviewer for helpful comments on this paper. Many thanks also to Beñat Oyhar\c cabal, members of the {Basque} Dialect Grammar team and an audience at GLOW. This research is supported by a grant from the Spanish Ministerio de Ciencia e Innovaci\'on FFI2008-00240/FILO, FFI2008-05135/FILO and from the {Basque} Government GIC07/144-IT-210-07. All errors are our own.



	
{\sloppy\printbibliography[heading=subbibliography,notkeyword=this]}
\end{document}  