\documentclass[output=paper]{langsci/langscibook} 
\title{Grammaticalization of participles and gerunds in Indo-Aryan: Preterite, future, infinitive} 
\shorttitlerunninghead{Grammaticalization of participles and gerunds in Indo-Aryan}
\author{%
 Annie Montaut\affiliation{Inalco (Paris)}
}
% (UMR 8202 INALCO/CNRS/IRD, Labex 083)
\ChapterDOI{10.5281/zenodo.823240} %will be filled in at production

\abstract{The development of the Sanskrit passive past participle and gerund or passive verbal adjective of obligation in \ili{Indo-Aryan} are up to a certain point parallel and resulted respectively in an ergative alignment in past sentences in Western languages and a nominative realignment in both future and past in Eastern languages. Only Eastern languages grammaticalized the old endings into the specific temporal markers -\textit{l}- for past and -\textit{b}{}- for future, while throughout the IA area the obligative passive verbal adjective also evolved into an infinitive. The aim of the paper is to account for the various grammaticalization paths of these forms in a unified manner, taking into account the whole range of other competing constructions in the various IA languages considered, as well as comparable instances of grammaticalization in Latin and Romance languages. Grammaticalization is understood here in the meaning of less grammatical to more grammatical as in \citet{Hopper2003} with a special attention to the shift in syntactic construction as in \citet{Benveniste1966[1952]} and \citet{Kuryłowicz1965}.}

\maketitle
\begin{document}

 
 

% \textbf{Keywords}: \isi{participle}, \isi{gerund}, \isi{infinitive}, Indo-Aryan, \isi{ergative} alignment
\section{Introduction}\label{sec:montaut:1}

The \isi{ergative} realignment of the \ili{Sanskrit} passive past participle\is{past passive participle} construction has been extensively studied in many Western \ili{Indo-Aryan} languages, and is often taken to be a unique development in the area and the wider family of Indo-European. However, its \isi{nominative} realignment in Eastern languages has raised far less interest. Moreover, the parallel development of the passive future \isi{verbal adjective} of obligation into a \isi{future marker}, and the \isi{infinitive}, is still unstudied. The aim of the paper, in continuation of \citet{MontautInPress}, is to account for the various grammaticalization paths of these forms in a unified manner. The words which developed in specific TAM markers with specific constructions are not lexical items but verbal adjectives, and grammaticalization is understood here in the meaning of less grammatical to more grammatical as in \citet{Hopper2003} with a special attention to the shift in syntactic construction as in \citet{Benveniste1966[1952]} and \citet{Kuryłowicz1965}. To understand why different paths can lead to different outcomes in different languages, the whole range of other competing constructions is considered. The paper also includes a comparison with similar instances of grammaticalization in \ili{Latin} and \ili{Romance} languages.
\ili{Indo-Aryan} (IA) languages (apart from \ili{Dardic} languages) are now conventionally classified into two main groups \citep{Cardona2003}, genetically associated respectively to the Western \ili{Sauraseni Prakrits} in Middle \ili{Indo-Aryan} (from which derive \ili{Gujarati}, \ili{Hindi}/\ili{Urdu}, \ili{Punjabi}, \ili{Rajasthani}, \ili{Braj}, among others), and to the Eastern \ili{Magadhean} Prakrits (from which \ili{Bengali}, \ili{Oriya}, \ili{Assamese}, and \ili{Maithili} derive, among others). I first analyse the different developments of the past \isi{verbal adjective} and \isi{past passive participle}, since the \isi{ergative} development is limited to the first group, at the same time questioning theories of both passive and \isi{possessive} origin of the perfect (\sectref{sec:montaut:2}). In \sectref{sec:montaut:3}, I present the parallel grammaticalization of the passive obligation \isi{participle} (or \isi{gerund}), and its different outcomes in Eastern and in Western \ili{Indo-Aryan}, since the anomalous situation of \ili{Marathi} invites to question the nature of this \isi{gerund}. This question, along with the discussion of the \isi{gerund}’s grammaticalization as a \isi{verbal noun}, is the subject of \sectref{sec:montaut:4}. 
This paper also attempts to clarify areal relations and the conditions for a given path of grammaticalization/reanalysis to actualize here and not there, while at the same time inquiring into the reasons for re-alignments.

\section{The grammaticalization of the passive participle into finite past and “ergative alignments” in Indo-Aryan}\label{sec:montaut:2}

As is well-known since \citet{Kellogg1875} and \citet{Grierson1903-1928}, what is today called the \isi{ergative} construction or alignment in \ili{Indo-Aryan}, and what was in the 19th and early 20th centuries described as ‘a kind of passive construction’, stems from a particular type of the \ili{Sanskrit} \isi{nominal sentence}, generalized in the classical language\footnotemark{}. The \isi{past passive participle} (henceforth glossed PPP) or \isi{verbal adjective} ending in -\textit{(i)ta} was used as a \isi{predicate}, replacing the finite \ili{Vedic} synthetic \isi{past tense} forms. In this \isi{predicative} use, the PPP agreed in gender and number with the patient, the agent appearing in the instrumental. As the \isi{predicative} uses of PPP increased in frequency, it grammaticalized into the standard expression of past (cf. \citealt{Bybee2003} for the role of use frequency). The original instrumental case marking of the agent was later replaced by a postpositional marker, the so-called \isi{ergative} \isi{case marker} in Western IA languages, while the -\textit{ita} (>\textit{ia}) form acquired modern gender/number endings. 
\footnotetext{In Late Classical \ili{Sanskrit}, for instance in the tales of the Vampire (\textit{Vetāla}), participial forms represent about 95\% of past sentences 
\citep[60]{Bloch1906}.}
The following example from Classical \ili{Sanskrit} \REF{ex:montaut:1} is a historical antecedent of the modern \isi{ergative} construction in \ili{Hindi} (\ref{ex:montaut:2}a), in contrast to the present or future \isi{nominative} sentences (\ref{ex:montaut:2}b), a construction extensively studied since the 1980s for its syntactic as well as pragmatic properties (for \ili{Hindi} cf. e.g. \citealt{Kachru1987,Montaut2004,Davison2002}). In the \isi{ergative} construction, the agent has most of the subject properties (particularly in control constructions) and the patient has only few discourse-related ‘subject properties’.

\ea
\label{ex:montaut:1}
\ili{Sanskrit}\\
\gll mayā     /  mama       tat              kr̥tam\\
\textsc{1sg.ins} /   \textsc{1sg.gen}     \textsc{dem nom.n.sg}   do.\textsc{ppp.nom.n.sg} \\ 
\glt ‘I did/have done that. (lit. ‘by me/ of me this done’)’\footnotemark{} 
\footnotetext{The genitive marking of the agent is restricted to pronouns, instrumental being by far the most usual marker elsewhere.}
\z 


\ea
\label{ex:montaut:2}
\ea 
\ili{Hindi}\\
\gll mai͂.ne      yah    /    apnā   kām kiyā\\
 \textsc{1sg.erg} \textsc{dem.m.sg}  {}   \textsc{refl}     work.\textsc{m.sg}     do.\textsc{m.sg} \\
\glt ‘I did this\slash my work.’
\ex 
\ili{Hindi}\\
\gll maĩ.  yah      /    apnā   kām        kar.ū͂.g.ī\\
\textsc{1sg}  \textsc{dem.m.sg} {} \textsc{refl}    work.\textsc{m.sg}   do.\textsc{1sg.fut.f} \\
\glt ‘I will do this\slash my work.’ 
\z 
\z


\subsection{Early New Indo-Aryan (NIA) data: alignment shift and acquisition of temporal meaning}\label{sec:montaut:2.1}

Equally well-known is the fact that this TAM-based pattern of split ergativity (found in \isi{definite past} and derived tenses such as present perfect, pluperfect and all compound forms involving the \isi{past participle}) is now restricted to the Western part of IA. What has been commented on less is the fact that it was prevailing throughout \ili{Indo-Aryan} up to the middle stages of NIA (14th--16th century, depending on the regional variety). Here are early examples of the extension of the construction, both from Western and Eastern IA, right from the last stage of Old \ili{Indo-Aryan} or Prakrit (PRK); in \REF{ex:montaut:3} from Ashoka’s first edict which displays both Western (\ref{ex:montaut:3}a: Girnar) and Eastern (\ref{ex:montaut:3}b: Jaugada) dialectal varieties, and \REF{ex:montaut:4}, by the playwright Kalidasa in Saurasenic Prakrit, which shows the contrast between \isi{instrumental agent} and nominal \isi{predicate} in the past, and \isi{nominative agent} and finite \isi{predicate} in the present: 

\ea
\label{ex:montaut:3}
\ea 
\textit{iyam dhammalipī devānampriyena     priyadassina       ranna   lekhapita} \\
\ex 
\glll iyam   dhammalipi   devānampiyena    piyadassina         [lajina]  lekhita\\
this  law-scripture    of-gods-friend    friendly-looking  king     inscribed \\
\textsc{nom.f.sg}  \textsc{nom.f.sg}     \textsc{inst.m.sg}      \textsc{inst.m.sg}     \textsc{inst.m.sg}  \textsc{nom.f.sg} \\
\glt ‘The friendly looking king beloved of gods has (made) engraved this law-edict.’  (PRK)
\z 
\z 

\ea
\label{ex:montaut:4}
\ea 
\gll hau  pai  pucchimi …   diṭṭhī       pia   pai    sāmuha   jāntī\\ 
\textsc{1sg.nom} \textsc{2.obl}    ask.\textsc{prs.1sg} ~   seen.\textsc{f.sg}  loved.\textsc{f.sg}  \textsc{2.obl}  in.front   passing.\textsc{nom.f.sg} \\
\glt ‘I ask you… Did you see (my) beloved passing in front (of you)?’\footnote{The form \textit{pai} for 2nd person is already used as a syncretic marker for several oblique cases.}     (PRK)
\z 
\z 

All Western and Central IA languages, now \isi{ergative}, displayed at an older stage contrast similar to \REF{ex:montaut:4}, with personal endings on the \isi{finite verb} in the present whereas in the past the verb retains nominal morphology. In the latter pattern, the \isi{predicate} shows gender and number agreement with the patient, while the agent is marked by the oblique case (a polyfunctional case, as a result of the usual syncretism \textsc{abl/dat/loc} in the area). The pattern is attested throughout the Western and Central languages (\ref{ex:montaut:5}a--d); example (\ref{ex:montaut:5}e) from Kabir (13th c.) illustrates the so-called ‘saint language’, a transregional literary koine which belongs to the vast category of old \ili{Hindi} \citep{Kellogg1875}:\newpage

\ea
\label{ex:montaut:5}
\ea 
\ili{Old Punjabi}\\
\gll guri  dānu ditta\\
guru.\textsc{obl/loc}    gift.\textsc{m.sg}    given.\textsc{m.sg}    \\    
\glt ‘The guru gave the gift.’ (\textit{Guru Granth Sahib})
\ex 
\ili{Old Rajasthani}\\
\gll sundari.nai  Bharath.ai           rākhī\\
beautiful.lady.\textsc{f.sg.acc}   Bharath.\textsc{obl.m.sg} kept.\textsc{f.sg} \\ 
\glt ‘Bharath kept the beautiful lady.’ \citep[167]{Tessitori19141916}
\ex 
\gll kr̥pā  kel.i  tumhĩ  (\ili{Old Marathi}) \\
pity.\textsc{f.sg}   do.l.\textsc{f.sg}     \textsc{2.obl} \\
\glt ‘You have had pity.’ (Jñaneśvari 11.255, in \citealt[261]{Bloch1970[1920]})
\ex 
\ili{Old Braj}\\
\gll maĩ   nahĩ    mākhan          khāyau\\
      \textsc{1.sg.ins}   \textsc{neg}  butter.\textsc{m.sg}   eat.\textsc{m.sg}  \\    
\glt  ‘I did not eat the butter.’ (Surdas 25.1, mid 16\textsuperscript{th} c.)
\ex 
\ili{Sant Bhasha}\\
\gll   gur.i   diyā   palītā\\
guru.\textsc{loc/obl} give.\textsc{m.sg}   stick.\textsc{m.sg} \\   
\glt ‘The guru gave the stick.’ (Kabir 8.3)
\ex 
\ili{Old Pahari}\\
\gll Virrsigh Joysĩ.yã  bhās̩               pāi\\
Virsingh Joshi.\textsc{obl}    proclamation/bond\textsc{.f.sg}   get\textsc{.f.sg} \\
\glt ‘Virsingh Joshi (the king) received the bond.’ (\citealt[283]{Stroński2014}) 
\z 
\z 


Some variation appears in this pattern already; note agreement with a marked object in Old \ili{Rajasthani} (\ref{ex:montaut:5}b), a marking itself rather recent, as well as a -\textit{l}{}- \isi{suffixation} on the \isi{predicative} \isi{participle} in Old \ili{Marathi} (\ref{ex:montaut:5}c), with variations in the oblique case marking (\textit{ai, -hi /-i, -yã} ending). Still, the basic pattern is the same. 
  This pattern was also maintained unchanged in the Eastern languages, with a similar contrast between \isi{nominative agent} and finite form agreeing with agent in the present, as opposed to non-finite participial forms in the past with \isi{oblique agent}. In the 16th century, the \ili{Bhojpuri} first person pronoun still had a \isi{nominative} form inherited from the \ili{Sanskrit} \isi{nominative} \textit{aham} (\textit{hau manus} ‘I [am] a man’), whereas it displayed an oblique form stemming from the \ili{Sanskrit} instrumental in the past (\textit{maĩ pāi} ‘I obtained’) (cf. \citealt[158]{Tiwari1966}).
  The examples in \REF{ex:montaut:6} illustrate a few of these Eastern varieties in early NIA, starting with a sample from the oldest Buddhist poems or \textit{caryas}, in Old \ili{Bengali} \citep{Chatterji1926}, the predecessor of Middle \ili{Maithili}, \ili{Oriya}, \ili{Bengali} and \ili{Assamese}, then other Eastern languages in their old or middle stages:

\ea
\label{ex:montaut:6}
\ea 
\ili{Old Bengali}\\
\gll mo.e  ghalil.i  hāḍerī-māli\\
 \textsc{1sg.ins}   cast.off.\textit{il}.\textsc{f}   bone-garland.\textsc{f}  \\
\glt ‘I have cast off the chaplets of bones.’  (carya 10, in \citealt[964]{Chatterji1926})
\ex 
\ili{Old Bengali}\\
\gll ebe    maï   bujh.ila\\
now  \textsc{1sg.ins} understand.\textit{ila} \\
\glt ‘Now I have understood.’  (carya 35, in \citealt[964]{Chatterji1926})
\ex 
\ili{Old Maithili}\\
\gll bhala  na  ka.la  mañe  de.la      bisavāsa\\
 good.\textsc{m.sg}    \textsc{neg}    do.\textit{la}   \textsc{1.sg.obl}   give.\textit{la}  trust.\textsc{m.sg} \\
\glt ‘I did not [the] good, [that] I gave trust.’ 
\ex 
\gll Tirahuti    le.li    jānhi\\
Tirahuti.\textsc{f.sg}   take.\textit{li}.\textsc{f.sg}   \textsc{rel.obl} \\ 
\glt ‘By whom (the city) Tirahuti was taken = who took Tirahuti.’ (from \citealt[491]{Jha1958}) 
 \ex 
\ili{Old Awadhi}\\
\gll eka  rāta sapnā mai  dekhā\\
one  night    dream.\textsc{m.sg}    \textsc{1sg.ins}    see.\textsc{m.sg}  \\      
\glt ‘One night I saw a dream.’ (Nur Mohammed 4)
\z 
\z 

Slight differences start emerging also here, such as the variety of oblique forms for the first-person pronoun (alternate forms of the instrumental in (\ref{ex:montaut:6}a--b), syncretic oblique in (\ref{ex:montaut:6}c--d), and the extension of a \textit{-l/il} suffix to the \isi{predicative} \isi{participle}). This suffix, with no particular meaning, was used with nominal and adjectival bases and is now used for deriving adjectives in many IA languages.\footnote{ For instance, \textit{rangīlā} ‘colourful’ from \textit{rang} ‘colour’ or \textit{kānthīlā} ‘thorny’ from \textit{kānthā} ‘thorn’ in \ili{Hindi}.} All Eastern varieties, stemming from the \ili{Magadhean} Prakrits (as opposed to those stemming from the Western \ili{Sauraseni Prakrits}) display a progressive erosion of gender marking between the 14th and 16th century, so that agreement grows less distinctive. But again, the basic pattern is essentially maintained.
As indicated by the translations of the examples in \REF{ex:montaut:5} and \REF{ex:montaut:6}, by and large taken from the publications providing the respective examples,\footnote{Except when they were translated by passive sentences as (\ref{ex:montaut:6}a). As for the gloss, I adopted the gloss \textsc{ins} wherever the pronoun has a distinctively instrumental form, not a syncretic oblique form.} the temporal meaning of the form extends from present perfect and pluperfect to the mere representation of anterior events as a preterit. As soon as the old \isi{passive participle} started developing as the only expression for past, it assumed both the original \isi{stative} / resultative meaning (perfect) and a new anterior meaning. This change in meaning has been well documented in \citet[190]{Peterson1998} for \ili{Pali} and in \citet{Breunis1990} for Classical \ili{Sanskrit}. I have accounted for it (\citealt{Montaut1996}, \citeyear{Montaut2007}) along the lines of \citet{BybeeEtAl1994} as a gradual process of grammaticalization of the new \isi{periphrastic} form: as long as the nominal form, initially a marked innovation aiming at stylistic expressivity, competed with the old tensed forms, it retained its original restricted meaning (resultative-\isi{stative}). When the old forms disappeared, the new form, no longer stylistically expressive, occupied the whole space of past reference and acquired what \citet{BybeeEtAl1994} calls an open meaning. This meaning conveyed the values of preterit or anterior, resultative, stative-resultative and \isi{stative}, as already observed by \citet[60]{Bloch1906}.\footnote{ “Du contexte et du sens de la racine dépendent la valeur active ou passive et la nuance temporelle et modale du participe. Il est donc le substitut de toutes les formes verbales du passé à tous les modes et à toutes les voix”. As for the \isi{copula}, it was originally used only for disambiguating an omitted agent in the first or second person.}
Later on, through foregrounding, then conventionalizing, the implicature (action leading to the resulting state), the adjectival form acquired an anterior meaning, while a new \isi{periphrastic} form with a \isi{copula} emerged for the perfect (around the 17th--18th ct.). This foregrounding of dynamic aspect resulted in the agent becoming foregrounded. Apparently, the topicalizing fronting of the agent becomes the rule during Middle \ili{Indo-Aryan} (MIA) (\citealt{Breunis1990}: Chapter 6 on \isi{word order}), although \citet[116--117]{Bubenik1996} date the linguistic perception of the oblique noun as a semantic subject to late MIA. It should however be emphasized that the agent was endowed with control and binding properties from the very beginning of the use of the \isi{predicative} \isi{past participle} (cf. \citealt{Hock1992}), which means that in Late \ili{Sanskrit} too dynamicity and prominence of the agent where already present. Such a parallel acquisition of first dynamic perfect, then \isi{aorist} and preterit meaning, and of syntactic subjecthood of the agent echoes the evolution of the North \ili{Russian} and Circum-Baltic perfect (cf. \citealt{Seržant2012}).

\subsection{Further grammaticalization in Modern Eastern New Indo-Aryan: a shift from passive to active or ‘possessive perfect’?}\label{sec:montaut:2.2}

Whereas Western IA languages developed the non-\isi{nominative} alignment with oblique agents further into a fully-fledged \isi{ergative} pattern such as in (\ref{ex:montaut:2}a), Modern Eastern IA languages shifted back to a \isi{nominative} alignment between the 14th and 16th century. Furthermore, most of them added personal endings to the \isi{participle}, making it a finite tensed form, as in \ili{Bengali}:


\ea
\label{ex:montaut:7}
\ea 
\ili{Bengali}\\   
\gll āmi     boi.t̩a       por.l.ām\\
\textsc{1sg}     book.\textsc{def}   read.\textsc{pst.1sg} \\ 
\glt ‘I read the book.’
\ex 
\ili{Bengali}\\
\gll tumi   boi.t̩a  por.l.e\\
\textsc{2}  book.\textsc{def}    read.\textsc{pst.2} \\
\glt ‘You read the book.’
\z 
\z 

As noted by \citet{Chatterji1926}, the new \isi{past tense} personal endings (1 -\textit{ām}, 2-\textit{i}, -\textit{e}), derived from personal pronouns, are distinct from the inherited present \isi{tense} personal endings (1 -\textit{i,} 2\textit{ -ish, -o}). As for the \textit{{}-l-,} which is now analysed as a past tense marker\is{past tense}, and which also occurs in \ili{Marathi}, it originates from an adjectival suffix (cf. \ili{Hindi} -\textit{il}{}- in \textit{rang.il.ā} ‘coloured’). This \isi{suffixation} is further evidence for the adjectival nature of the \isi{predicate} in the pre-\isi{ergative} alignments (cf. \citealt[928]{Chatterji1926} and \citealt{Tessitori19141916}). Its reanalysis as a past tense marker\is{past tense} corresponds to the renewal of the “pre-\isi{ergative}” alignment into a \isi{nominative} alignment. \citet{Chatterji1926}, following the then-usual interpretation, considers this evolution a shift from passive to active,\footnote{An ‘archaic’ remnant of the old system survived in classical \ili{Bengali} with the -\textit{e} ending for agents in \isi{transitive} past, and is still present in \ili{Assamese}.} but he clearly recognizes the resultative reading, as is also clear from his literal translation of this pattern, when the object is marked: ‘there has been V by X as concerns Y’.
Similarly, the evolution of this pattern in Eastern \ili{Hindi}, which also displays the same suffix -\textit{l}{}- in the \isi{definite past}, is systematically interpreted as a shift from passive to active in \citet[247 ff]{Saxena1937} for \ili{Bhojpuri}, in \citet[492 ff]{Jha1985[1958]} for \ili{Maithili}, and in \citet[171]{Tiwari1966} for \ili{Awadhi}. The active reading, with concomitant acquisition of subject behavioural properties by the agent, was already present centuries before, so the modern shift in coding properties can be considered a mere final step in the grammaticalization of the construction. This evolution is in line with \citegen{Haspelmath2010} principle of Behavioural before Coding Properties.
  This is also the case in other Indo-European languages such as \ili{Persian} (cf. \citealt{Cardona1970}) and \ili{Latin}/\ili{Romance} languages, which went through a similar cycle: finite \isi{past tense} > \isi{nominal sentence} with \isi{predicative} \isi{participle} agreeing with the patient and \isi{oblique agent} > verbal sentence with \isi{nominative agent} and verb agreeing with agent, as stated in \citet{Kuryłowicz1931,Kuryłowicz1965}. Here is the Old (\ref{ex:montaut:8}a) and modern (\ref{ex:montaut:8}b) \ili{Persian} data considered by the author:
   
\begin{exe}\label{ex:montaut:8}
\ex 
a.
\parbox[t]{.52\textwidth}{
\gll mana  tyâ  karta.m\\
\textsc{1sg.gen}   \textsc{dem.nom.n.sg}  do.\textsc{ppp.nom.n.sg}\\
\glt ‘I have done that.’  
}
{\Large >~~}
b.
\parbox[t]{.35\textwidth}{
\gll       man in kar.d.am\\
      \textsc{1sg.nom}     this    
do.\textsc{pst.1sg}\\
\glt     ‘I did that.’
} 
\end{exe}

The corresponding \ili{Latin} data involves a \isi{periphrastic} past with \isi{participle} and \isi{dative} of the agent, renewed by means of the ‘have’ \isi{auxiliary} and a \isi{nominative subject}. In present-day French, a vestige of the old ‘pre-\isi{ergative}’ pattern is still available: if the \isi{direct object} is preposed, the \isi{participle} agrees with it, and not with the subject (\textit{les lettres que j’ai écrit.es, je les ai écrit.es).} Examples (\ref{ex:montaut:9}a) and (\ref{ex:montaut:9}b) illustrate this shift to \isi{nominative} alignment in \ili{Latin} as analyzed in \citet{Kuryłowicz1931}, example \REF{ex:montaut:10} shows the output in modern \ili{Romance}, with the French, \ili{Italian} and \ili{Spanish} translations of \REF{ex:montaut:9}:

\ea
\label{ex:montaut:9}
\ea 
\ili{Latin}\\ 
\gll mihi  id   factum (est)\\
\textsc{1sg.dat}   \textsc{dem.nom.n.sg}     do.\textsc{ppp.nom.n.sg} (be.\textsc{prs.3s})\\
\glt ‘I did/have done that.’ (\textit{lit}. ‘to.me this done (is)’)
\ex 
\ili{Latin}\\
\gll ego  id   factum   habeo\\
\textsc{1sg.nom}   \textsc{dem.acc.n.sg}   do.\textsc{ppp.acc.n.sg} have.\textsc{prs.1sg}\\
\glt ‘I have done that/it.’ (\textit{lit}. I have this done)
\z 
\z 

\ea\label{ex:montaut:10}
\ili{French}\\
\textit{j’    ai       fait     ceci}       \\
\ili{Italian}\\
\textit{io   ho      fatto   questo}     \\
\ili{Spanish}\\
\textit{yo   he      hecho   esto}       \\
\glt `I     have  done   that'
\z 

As stated by \citet{Kuryłowicz1931} in his paper on the formation of tenses in \ili{Romance} languages, further developed in his study on the evolution of grammatical categories \citep{Kuryłowicz1965}, “the decisive step is the replacement of the \isi{dative} + \textit{esse} [be] + \isi{nominative} by \isi{nominative} + \textit{habere} [have] + accusative. The passive construction has been transformed into an active one” \citep[107]{Kuryłowicz1931}. 
  Against this classic analysis of a \isi{nominative} shift paralleling the `active transformation’, also proposed by the \ili{Indian} scholarly literature mentioned above, \citet{Benveniste1952}, in a pioneering paper on the meaning of the perfect, proposed his own view of the perfect as basically \isi{possessive}: the \isi{dative} \isi{case marker} in such \isi{periphrastic} expressions with a \isi{participle} was never an agent marker but a \isi{possessor} marker, since Classical \ili{Latin} uses the \isi{dative} and verb ‘be’ to represent possession.\footnote{By contrast, the standard \isi{case marker} for agents in passive clause is \textit{a} + ablative in \ili{Latin}, and similarly in Old \ili{Persian} it is not a genitive but \textit{hacama} + ablative.}  The \isi{possessive} construction \textit{mihi filius est} (‘I have a son’) later shifted to the ‘have’ construction (\textit{ego}) \textit{filium habeo}, in the same way as the perfect. 
Consequently, the evolution of the \ili{Latin} (and \ili{Persian}) perfect has, according to \citet{Benveniste1952}, nothing to do with a shift from passive to active, but is a mere reversal (“renversement, retournement”) of the old \isi{possessive} pattern. Similar conclusions were presented later by \citet{Pirejko1979} and by \citet[397]{Trask1979} who associates the \isi{possessive} origin with “the incorporation into the \isi{inflectional} paradigm of a nominal form” with a genitive \isi{agentive} complement. 

\subsection{Re-interpretation of the IA grammaticalization path: the new agent case markers in ergative Western languages as localizers}\label{sec:montaut:2.3}

This famous analysis of the Indo-European perfect by Benveniste raises however a problem regarding the \ili{Indo-Aryan} data, since it is exclusively based on case marking. As already mentioned, the agent in classical \ili{Sanskrit}, particularly if it is a full NP, is standardly in the instrumental, the standard agent marker in passive sentences (see \citealt{MontautInPress} for details), and not in a \isi{possessive} case. Yet, further developments of the `pre-\isi{ergative}’ pattern into a fully \isi{ergative} alignment in Western IA languages provide arguments for considering the agent in a way similar to \citeauthor{Benveniste1966[1952]}’s analysis of \ili{Latin}. The new markers used to reinforce the old syncretic oblique case are indeed massively derived from location nouns. Moreover, they also served as a \isi{locative} marker (such as the -\textit{i} ending in \ili{Sant Bhasha} in example (\ref{ex:montaut:5}e) above). Indeed, the most widespread form of the now \isi{ergative} case markers is \textit{ne} (\textit{ni, nai, nẽ, ne}), found in \ili{Hindi}/\ili{Urdu}, \ili{Panjabi}, \ili{Marathi}, \ili{Gujarati}, and it is derived from the \isi{locative} (*\textit{karn̩asmin}, a renewal of the classical form \textit{karne} by analogy with the major paradigm) of the noun \textit{karn̩a} `ear” (\citealt[65ff]{Tessitori19141916}; for more details and examples see \citealt{MontautInPress}). Then this form underwent reduction along the following lines: \textit{karnasmin > kannahĩ > kan̩n̩haĩ > kan̩n̩aĩ > naĩ > nai > nẽ > ne}. Tessitori was the first scholar who identified the correct origin of \textit{ne}, but already \citet[401]{Trumpp1872} had traced \textit{kane} (`near, at the edge’, then `to’) to the \ili{Sanskrit} noun \textit{karna} `ear’. The origin of the \isi{ergative} marker has later on been accepted by all traditional grammarians (\citealt{Tiwari1961}, \citealt{Tiwari1966,Saxena1937,Chatterji1926,Chatak1966}). \citet[68--70]{Tessitori19141916} gives examples such as (\ref{ex:montaut:11}a) with a clearly \isi{locative} meaning and (\ref{ex:montaut:11}b) with an \isi{agentive} meaning:


\ea\label{ex:montaut:11}
\ea  
\gll mithyādr̥sthi  loka  kanhai    sravai  vasirau   nahĩ\\
false.look  people  \textsc{loc}   hermit.\textsc{m.sg}   dwell.\textsc{prs.3m.sg}  \textsc{neg} \\
\glt ‘A shravaka (hermit) should [does] not live near heretics.’
\ex 
\ili{Old Rajasthani}\\
\gll adiśvara  naï        diksā                  lidhi\\
Adishwara  \textsc{loc/erg} consecration.\textsc{f.sg}    take.\textsc{f.sg} \\
\glt ‘The Adishvara took the consecration.’
\z 
\z 

The second most widespread \isi{ergative} marker is \textit{le} (\textit{lai, al}), found in Kumauni, \ili{Garhwali}, early \ili{Nepali}, and it is derived from the verbal root \textit{lag} ‘to be in contact, touch’ (\textit{lagi/lāgi} > \textit{laï, lai, le}) originally meaning ‘having come in touch with’, ‘up to’, ‘for the sake of’. Reflexes of both markers are far more often used for \isi{dative} (\textit{nai/ne/n\~{ū}} in \ili{Rajasthani}, \ili{Panjabi}, \ili{Gujarati}; \textit{lā} in \ili{Marathi}) than for instrument arguments (\textit{ne} in \ili{Marathi}, -\textit{an}/\textit{le} in \ili{Garhwali}/Kumaoni). These connections should however not be overestimated, given the extremely weak semantic content of the initial etymon, and also given that the same ending \textit{–i} was used in the \ili{Sant Bhasha} of Kabir both for agents, as in (\ref{ex:montaut:5}e) above, and for \isi{locative} complements, as in \REF{ex:montaut:12}:

\ea
\label{ex:montaut:12}
\ili{Sant Bhasha}\\
\gll ābari dīsai                ketā   tārā\\
sky.\textsc{loc}   be.seen.\textsc{3m.pl}.     how.many  star.\textsc{m.pl}  \\
\glt ‘How many stars we have seen in the sky!’ (146.1)
\z 

Finally, there is the \ili{Gujarati} \isi{ergative} marker with its alternation \textit{ne/e}, the first form being related to the above-mentioned \textit{ne} whereas the -\textit{e} ending is also used as a \isi{locative} \isi{case marker} (cf. \citealt[678]{Cardona2003}) and seems to have been the initial marking, since we find it in both functions in the \textit{Jain Gurjar Kavyo}, dating to the 14--15th century \REF{ex:montaut:14}:

\ea
\label{ex:montaut:13}
\ili{Modern Gujarati}\\
\gll Sītā.e      kāgal         vācyo\\
    Sita.\textsc{erg}    letter.\textsc{m.sg}   read.\textsc{m.sg} \\
\glt ‘Sita read the letter.’  \citep[426]{Mistry1997}
\z    
    
\ea
\label{ex:montaut:14}
\ea 
\ili{Old Gujarati}\\
\gll jamunājī.n.e                tat.e\\
Yamuna.\textsc{hon.gen.loc} bank.\textsc{loc}\\
\glt ‘On the bank of the Yamuna river.’\footnote{Like in \ili{Hindi} and all IA languages that maintained grammatical gender, the genitive postposition is an adjectiving suffix: the noun in the genitive agrees with the head noun in gender, number and case.} \citep[630]{Desai1926}
\ex 
\gll ame   jamunā  gay\~{ā} (…) roky\~{ā}       nandanā     nānhad̩ī.e\\
 \textsc{1pl}   Yamuna  go.\textsc{m.pl}  ~   stop.\textsc{m.pl}  \ili{Nanda}.\textsc{gen}    small.boy.\textsc{loc/erg}\\
\glt ‘We have been to the Yamuna (…), [we] were stopped by \ili{Nanda}’s boy /\ili{Nanda}’s boy stopped [us].’ \citep[630]{Desai1926}
\z 
\z 

The specialization of a given \isi{case marker} for a given function is clearly recent (and in some languages still very weak: \citeauthor{Montaut2015} (\citeyear{Montaut2015,MontautInPress}) and \citet{Stroński2010} quote various examples where the same marker is used for \isi{ergative}, \isi{dative}/accusative, and instrumental).\footnote{\ili{Bangaru} (Panjab-Haryana) is well-known for displaying the same case-marker, \textit{nai} in certain dialects, \textit{s\~{i}} in others, for \textsc{erg}, \textsc{dat}/\textsc{acc} and \textsc{ins} (for examples see \citealt{Montaut2007}, \citealt{Montaut2015}). Kului, a Western \ili{Pahari} (Himachal, north of Panjab) also exhibits this peculiarity (example in \citealt{Stroński2014}). Similarly, \ili{Maithili} too displays “contradictory”{uses of the postposition} {\textit{so}{\textasciitilde}} , \textsc{ins}/\textsc{abl}, including for agents, and \textsc{dat}/\textsc{acc} \citep[30]{Jha1958}.} The original semantics of the words that grammaticalized to \isi{ergative} case markers may explain such anomalous associations as agent and patient as distinct from subject (cf. \citealt{Malchukov2009}). Yet, a straightforward evolution of a construction with the \isi{passive participle} into an \isi{ergative} pattern with agents marked by originally localizing words is hardly plausible. One might be tempted to view the development from the \isi{passive participle} in nominal sentences such as in \REF{ex:montaut:1} to the modern \isi{ergative} construction given in (\ref{ex:montaut:2}a) and \REF{ex:montaut:13} as involving an intermediate stage which was neither a passive nor an active construction but a type of impersonal construction. This is a construction which \citet[374]{Seržant2012} calls a ‘free-dative-adverbial’ complement as the agent (subsuming under ‘\isi{dative}’ the adhesive, \isi{dative}, \isi{locative} and genitive case). This construction becomes ambiguous with respect to subject properties (control and discourse-related) during the early NIA period, as shown in (\ref{ex:montaut:14}b). Later, marked objects blocked agreement, which then facilitated reanalysis: This construction has been already identified as ‘impersonal’ in constructions with a marked object in traditional grammars such as \citet{Kellogg1875} and \citet{Chatterji1926}, as their literal translations such as ‘there has been V by X regarding Y’ clearly demonstrate. It is in this abstract sense that the construction can be considered as a \isi{predication} of localization: the process or the result is merely located in relation to the agent, not represented as the direct source as in the \isi{transitive} model. 
Coming back to \citeauthor{Benveniste1952} and the \ili{Latin} scenario involving the ‘have’ \isi{auxiliary}, \citet{Benveniste1952} clearly states that ‘have’ is but ‘an inverted be’, that is, a \isi{stative} verb, \isi{transitive} only apparently. He later developed this notion in an article on French auxiliaries (\citeyear{Benveniste1960}) to explain the use of \isi{possessive} \isi{predication} in French and \ili{Romance} languages for expressing transient states (\textit{avoir faim, peur, mal} ‘be hungry, cold, in pain’). On similar grounds (inverted \isi{stative} \isi{predication} rather than the direct \isi{transitive} source-goal model), I suggested an analogy between \isi{ergative} as well as \isi{dative} subject sentences widespread in modern IA to experiential \isi{predicates} (physiological or psychological processes) and predications of localisation \citep{Montaut2004}. 
  Currently, \isi{dative} experiential subjects have case markers that are distinct from \isi{ergative} agents in modern languages (except in \ili{Bangaru}). Both fully-fledged \isi{ergative} patterns and \isi{dative} subjects arose, depending on the language concerned, between the 14th and 16th century. They most probably derived from such localizing predications. This development is thus comparable to the reanalysis of impersonal \isi{intransitive} constructions attested elsewhere; and it is in particular similar to the \ili{Slavic} \textit{{}-n/t} \isi{participle} developing into perfect, as described by \citet{Seržant2012}. Significantly, constructions of the type exemplified in \REF{ex:montaut:1}, \REF{ex:montaut:5}, and \REF{ex:montaut:6} were also found with \isi{intransitive} \isi{predicates}, in \ili{Sanskrit} as well as in early Western and Eastern NIA:

\ea
\label{ex:montaut:15}
\ea 
\ili{Sanskrit}\\
\gll ināhina {\ob}= ina+    ahina{\cb}   st̥ptam\\
 ~ ~ \textsc{dem.ins} serpent.\textsc{ins}   crawl.\textsc{ppp.nom.n.sg} \\
\glt ‘The serpent crawled.’ (there has been crawling by the serpent) \citep[198]{Renou1952}
\ex 
\gll asmakam  abhipretam  bhavantam kimcid  artham  abhiprastum\\
\textsc{1pl.dual.gen}  come.\textsc{ppp.nom.n.sg}  \textsc{2.acc}  \textsc{indef}    for       ask.\textsc{inf} \\ 
\glt ‘We have come here to ask you something.’ (\textit{Mahabharata} IIIg9) \newpage
\ex   
\ili{Middle Bengali}\\
\gll duhe͂ thak.ila  eka     paśe \\
both.\textsc{obl} remain.\textit{ila}   one    side.\textsc{loc}  \\
\glt ‘They both remained on one side.’ \citep[947]{Chatterji1926} 
\z 
\z 

Even the now highly anomalous clause \textit{mai͂.ne gayā} ‘I went’, with the \textit{ne} mark\-er, reported to be frequently heard in the speech of migrants from Punjab to Delhi in the mid-20th century \citep[71]{Chatterji1986}, can be explained by this scenario. This, as well as the diverging Eastern and Western evolutions, raises questions about the now standard labelling of such patterns as ‘pre-er\-ga\-tive’, also reflected in glossing of case markers, whether \isi{inflectional} or postpositional (\ili{Assamese} -\textit{e} is never glossed \isi{ergative}, and the language is not considered \isi{ergative}, \ili{Gujarati} \textit{ne/e} is glossed Agent in \citet{Cardona2003} – although the authors consider the language \isi{ergative} – and \isi{ergative} by \citet{Mistry1997}. Syncretic obliques are usually glossed \isi{ergative} in modern studies on Old \ili{Hindi}). 

\section{The grammaticalization of the modal verbal adjective: an areal complementary distribution?}\label{sec:montaut:3}

Whereas the construction featuring the PPP developed into \isi{ergative} alignments in Western languages only via the \isi{nominal sentence} with an \isi{instrumental agent}, this construction was realigned into \isi{nominative} alignments in the East. In parallel, another \isi{passive participle} also developed, also via nominal sentences, into patterns with non-canonical alignment, which were further realigned into \isi{nominative} patterns in Eastern languages only. This \isi{passive participle} is the form ending in -\textit{tavya}, ‘to be V.ed’, also called \isi{gerund} or \isi{verbal adjective} of obligation. I will henceforth gloss this form as \isi{gerund}, although it is morphologically a \isi{participle}, inflecting like an adjective. This form underwent an evolution strikingly parallel to the evolution of \isi{past participle} in Eastern languages, a fact which has long been noticed by the historians of those Eastern languages but not considered as a highly relevant fact before \citet{Montaut1996,Montaut2007,MontautInPress}. Let us first see this type of evolution before turning to its developments in Western languages.

\subsection{The parallel grammaticalization paths of future and past in Bengali and other Eastern IA languages}\label{sec:montaut:3.1}

In Classical \ili{Sanskrit}, the \isi{periphrastic} construction involving a \isi{verbal adjective} (sometimes called \isi{gerund} or gerundive) was the standard way to express obligation. This pattern, illustrated in (\ref{ex:montaut:16}a), is parallel to the \isi{past tense} pattern in \REF{ex:montaut:1}, but it is found both with \isi{transitive} (\ref{ex:montaut:16}a) and \isi{intransitive} verbs as shown in (\ref{ex:montaut:16}b) from \citet[31]{Bloch1906}. Example \REF{ex:montaut:17} illustrates both past and \isi{modal} nominal sentences pervasive in the narrative register (\textit{Tales of the Vampire}, in \citealt[59]{Bloch1906}):

\ea\label{ex:montaut:16}
\ea 
\ili{Sanskrit}\\
\gll mayā   tat  kartavyam\\
 \textsc{1sg.ins}    \textsc{dem nom.n.sg}   do.\textsc{ger.nom.n.sg} \\
\glt ‘I have to/should do that. (lit. ‘by me this to-be-done’)’
{\rmfamily} 
\ex 
\gll yamayor  apramattayā         tvayā bhavitavyam\\
regulations.\textsc{loc.dual}    attentive.\textsc{ins.f.sg}  \textsc{2.ins}      be.\textsc{ger.n.sg} \\
\glt ‘You should be attentive (non-distracted) regarding the regulations.’
\z 
\z 

\ea\label{ex:montaut:17} 
\gll mantriputren̩oktam~(mantriputren̩a-uktam)    “adya   tvayā   gantavyam”; tayoktam~(=tayā.uktam) “gantavyam”\\
{minister.son.\textsc{ins.m.sg} say.\textsc{ppp.n.sg}} now \textsc{2.ins}    go.\textsc{ger.n.sg} {\textsc{3.ins.f.sg} say.\textsc{ppp.n.sg}}     go.\textsc{ger.n.sg} \\
\glt ‘The son of the minister told her: ‘Now you should go’; she said: ‘should go’.’
\z

This pattern remained unchanged in the various Prakrits of Middle \ili{Indo-Aryan} such as the \ili{Magadhean} variety in Ashoka \REF{ex:montaut:18}, the direct ancestor of \ili{Bengali}: 

\ea
\label{ex:montaut:18}
\glll {hida no kimci jive}       alabhitu    pajohitavye    {no pi   ca  samāje}      kattavye\\
  {here no some   living}        kill           sacrifice.  {no even  and assembly}  do \\
\textsc{nom.n.sg}   \textsc{cv}   \textsc{ger.nom.n.sg}          \textsc{nom.m.sg}   \textsc{ger.nom.m.sg} \\
\glt ‘Here one should not sacrifice by killing a living creature nor hold a meeting here.’ \\
  (lit. here living being killing should not be sacrificed nor meeting should be held)
  \z 
  
The forms involving the \isi{gerund} seem to have included a temporal meaning from as early as the 3rd century in the eastern region: according to \citet[966]{Chatterji1926}, the corresponding form in \REF{ex:montaut:19} from Ashoka’s rock edict in Sarnath has “a vague mandatory sense, with an express future implication”:

\ea
\label{ex:montaut:19}
\ili{Prakrit}\\
\gll iyam    sāsane               vĩnapayitavye\\
\textsc{dem nom.m.sg}   principle.\textsc{m.sg} make.known.\textsc{ger.nom.m.sg} \\
\glt ‘This principle should/will be made known’. (made to be known)
\z 

Old \ili{Bengali} (illustrated in (\ref{ex:montaut:20}a--c) from \citep[967 ff]{Chatterji1926} displays the same construction, both for \isi{intransitive} and \isi{transitive} verbs with the characteristic -\textit{b}{}- suffix derived from the old -\textit{tavya}, and agreement with the patient of \isi{transitive} verbs, before the loss of gender blurred agreement and the ending -\textit{ba} became invariable:

\ea\label{ex:montaut:20}
\ea 
\ili{Old Bengali}\\
\gll toe sāma   kariba   maï       sānga\\
 \textsc{2.obl}   with     do.\textit{ba}(\textsc{m.sg?})   \textsc{1sg.obl}   company\textsc{.m.sg} \\
\glt ‘I shall have union (do company) with you.’ 
\ex 
\ili{Old Bengali}\\
\gll maï dibi   piricha\\
\textsc{1sg.ins}  give.\textit{b}.\textsc{f}   question.\textsc{f.pl}  \\
\glt ‘I shall ask questions.’ 
\ex 
\ili{Old Bengali}\\
\gll tab\~{e} to.ka   rakhiba  kona  jāne\\
then   \textsc{2.acc}  protect.\textit{ba}    which   person.\textsc{obl} \\
\glt ‘Then who will protect you?’  
\ex 
\gll mai͂  jaivo (=jaiba?)  govinda   saha    khelaṇa\\
 \textsc{1sg.ins}  go.\textit{b}    ~            Govinda   with    play.\textsc{inf} \\
\glt ‘I shall go to sport with Govinda.’ \citep[30]{Chatterji1986}
\z 
\z 
 
Around the 15th century, this construction was transformed into a \isi{nominative} one, with \isi{nominative} (unmarked) agents and tensed verbs agreeing with the agent. The verbal form consists of the old -\textit{ba}- form now suffixed with finite person endings. These suffixed person endings are the same as the ones used for past, both distinct from the present endings (-\textit{ish} and -\textit{o} for second familiar and second respectful person respectively):

\ea
\label{ex:montaut:21}
\ea 
\ili{Modern Bengali}\\
\gll tu  boi.ta       por.bi\\
\textsc{2sg}   book.\textsc{def}   read.\textsc{fut.f.2sg} \\
\glt ‘You (familiar) will read the book.’
\ex  
\ili{Modern Bengali}\\
\gll tumi boi.ta         por.b.e\\
\textsc{2hon} book.\textsc{def}    read.\textsc{fut.f.2hon} \\
\glt ‘You (respectful) will read the book.’
\z 
\z 

The verb morphology is now analysed as base + \isi{future marker} -\textit{b}- + person ending, in the same way as the past is now analysed as base + past marker -\textit{l}- + person ending, both processes of reanalysis occurring at the same time. The above-mentioned grammarians also acknowledged this parallel in the evolution of Eastern IA languages as a common ‘active transformation’. \citet[987]{Chatterji1926} notes for \ili{Bengali} that “the affixes are exactly on the lines of the past”, in contrast with those for the present, and also that the shift in alignment occurred at the same time for future and past.
  Other Eastern IA languages also went through this ‘active transformation’ but the respective outcomes are not as clear as in \ili{Bengali}. In \ili{Awadhi}, for instance, the sigmatic inherited future persisted and is still prevailing in certain persons (and similarly in \ili{Bhojpuri}), and \ili{Maithili}, which now consistently displays a -\textit{b}- future, has acquired a complex agreement system indexing several participants. But in older times the construction was maintained with the now-lost gender agreement still visible, for instance in \ili{Vidyapati} (14th century):

\ea\label{ex:montaut:22}
\ea 
\ili{Old Maithili}\\
\gll sumarabi  mori name  …     prema sumaraba\\
remember.\textit{b}\textsc{.f}    my.\textsc{f}  name\textsc{.f}  ~ love\textsc{.m} remember.\textit{b}.\textsc{m}  \\
\glt ‘You will remember my name, you will remember my love.’ (V 9, in \citealt[494]{Jha1958})
\ex  
\gll hamahũ   nāgari   sabe        sikhaūbi\\   
\textsc{1.obl}   lady.\textsc{f}     all(=pl)   teach.\textit{b}.\textsc{f} \\
\glt ‘I shall instruct the ladies.’ (V 52, \citealt[495]{Jha1958} 495)
\ex 
\gll man͂e ki bolaba sakhi apana gen͂āna\\ 
\textsc{1sg.obl} how   speak.\textit{b}   friend   \textsc{refl}     experience.\textsc{m.pl}\\
\glt ‘O friend, how shall I speak out (my) feelings.’\footnote{In example (\ref{ex:montaut:22}b) \textit{hamahũ} is the oblique form of the first person (singular and plural, today \textit{hama}), whereas (\ref{ex:montaut:22}c) displays an oblique case of the base for first singular person, now gone out of use, with a case marking left unglossed by the author, but reminiscent of the palatal oblique forms for first person (\ili{Hindi} \textit{mujh, mujhe}).} (V 24, \citealt[495]{Jha1958})
\z 
\z 

\ili{Awadhi} and \ili{Bhojpuri}, which have now a complex paradigm in the future, also displayed the same morpho-syntactic pattern in their early stages before the shift to the \isi{nominative subject} occurred. Similarly, \citet[171]{Tiwari1966} observed for \ili{Bhojpuri} that “the future affixes for the first, second and third persons masculine and feminine singular and plural are in a line with those of \isi{simple past}”. Remarkably, in \ili{Awadhi}, the -\textit{ba} \isi{predicates} combining with an \isi{oblique agent} of the older pattern that prevailed till the 16th century could still convey necessity, which was no longer the case for the \ili{Bengali} or \ili{Maithili} equivalents. The resilience of the old \isi{sigmatic future} in certain persons in \ili{Awadhi} may explain why the -\textit{b}- form took longer to grammaticalize into a \isi{future marker}:\footnote{Depending on the dialects: only the first person forms have the -\textit{b}{}- ending in western dialects of \ili{Awadhi}, and first and second person in eastern dialects. The -\textit{h} form (inherited \isi{sigmatic future} \textit{s̩} > \textit{s} > \textit{h}) prevails in the third person, but in early \ili{Awadhi} -\textit{b}{}- was found in all persons, although not systematically \citep[264--266]{Saxena1937}.} 

\ea\label{ex:23}
\ea 
\ili{Awadhi}\\
\gll kathā           bhāśā-baddha       karabi                     mai͂               so-saba hetu     kahaba mai͂        gāī\\
story.\textsc{f.sg}  language-ridden    do.\textit{b}.\textsc{f}  \textsc{1sg.ins}   this-all because   say.\textit{b}. \textsc{1sg.ins}  sing.\textsc{cv} \\
\glt ‘That story is to be composed by me in the vernacular, for this reason it is to narrated by me by singing.’ (or: I shall compose, I shall tell by singing)\footnote{Same example is translated in \citet[260]{Saxena1937} by a future form: \textit{bhasa-badha karabi maĩ} `I shall render it in popular language'.}   
\ex 
\gll ghara  kaisai  pait̩haba   maĩ       ch\~{ū}che, kaunu utara  debau?\\
house  how  enter.\textit{ba}   \textsc{1.sg.ins}  deprived   which answer.\textsc{m.sg}    give.\textit{b}.\textsc{m.sg} \\
\glt ‘Empty how shall I enter the house, what answer shall I give?’ \citep[261]{Saxena1937}
\ex 
\gll prāna-priya  siya  jānibi,          nija  kinkarī-kari  mānibi\\
life-dear  Sita.\textsc{f.sg}   know.\textit{b}.\textsc{f.sg} \textsc{refl} slave      consider.\textit{b}.\textsc{f.sg} \\
\glt ‘Sita is to/will be regarded as beloved like life, she is to/will be accepted as thy slave.’ (\citet[96]{Chatterji1986} translates this form as \isi{obligative})
\z 
\z 

\subsection{Kuryłowicz’s explanation of the parallel nature and development of perfect and future}\label{sec:montaut:3.2}

This parallel evolution of past and future systems was also noticed by \citet{Kuryłowicz1931,Kuryłowicz1965} for \ili{Romance} languages, based on different, but equally convincing morpho-syntactic evidence. The future in modern \ili{Romance} languages is indeed a very peculiar innovation, involving the verb `have' constructed with the \isi{infinitive}. The same \isi{auxiliary} is used as in the perfect, the difference being only that in the future, the \isi{infinitive} precedes the \isi{auxiliary}, resulting in a fused form, whereas the \isi{auxiliary} after the verb in the perfect is still free (cf. \ref{ex:montaut:10} above). Here are the forms in French and \ili{Spanish}, with similar formation in \ili{Italian}:
%%please move \begin{table} just above \begin{tabular
\begin{table}
\begin{tabularx}{\textwidth}{lQQQ}
\lsptoprule
 & 1st person\newline `I will sing' & 2nd person\newline `You will sing' & 3rd person\newline `He will sing'\\
\midrule
French & Je chanter.ai & Tu chanter.as & Il chanter.a\\
\ili{Spanish} & Yo cantar.é & Tu cantar.ás & El cantar.á\\
& \textsc{1sg} sing.\textsc{inf-1sg} & \textsc{2sg} sing.\textsc{inf-2sg} & \textsc{3sg} sing.\textsc{inf-3sg}\\
\lspbottomrule
\end{tabularx}
\caption{First, second, and third person singular future forms of ‘sing’ in French and Spanish.}
\label{tabmontaut:1}
\end{table}

The endings are either identical to (French) or derived from (\ili{Spanish}) the present paradigm of ‘have’ (\textit{j’ai, tu as, il a}). The French pattern is clearer, since past and future \isi{auxiliary} have the same form (\textit{j’ai chanté, tu as chanté, il a chanté}), whereas the past \isi{auxiliary} displays an initial \textit{h} in \ili{Spanish} as a separate \isi{auxiliary} (\textit{he cantado, has cantado, ha cantado}).
  Historically, the “have” \isi{periphrastic} future acquired its temporal meaning from the original meaning of necessity in Vulgar \ili{Latin}, with the \isi{infinitive} shifting from passive to active morphology (\textit{cantari} > \textit{cantare}). According to \citeauthor{Kuryłowicz1965} (\citeyear[107]{Kuryłowicz1931}),
  the striking parallelism between perfect and future (in contrast to the present) crucially relies on the origin of the \textit{habere} \isi{periphrastic} future: the ‘have’ periphrasis is for him the continuation of the \ili{Latin} \isi{verbal adjective} of obligation (sometimes called \isi{gerund} or gerundive in -\textit{nd}{}-):\footnote{The initially passive \isi{infinitive} was replaced by the active \isi{infinitive} as soon as Late Classical and Vulgar \ili{Latin}. The probable original pattern for (\ref{ex:montaut:24}b), as duly noticed by Benveniste, was then:\textit{ego id cant.ari   habeo} \textsc{1sg} this sing.\textsc{inf.pass} have.\textsc{1sg}  ‘I have this to be sung, to sing.’ As for the \isi{dative} pattern with obligation \isi{gerund}, on the popular patterns in \textit{delenda est Carthago} (‘Carthago is to be destroyed’) or \textit{mihi colenda est virtus} (‘Virtue is to-be-culvated to me’), all the numerous examples quoted in \citet[164 ff]{Touratier1994} as well as other  examples in the most classical writers such as Cicero, it always has a \isi{modal}, not temporal, meaning : \textit{senibus labores corporis minuendi sint} (‘[So that] old people should minimize physical works’ Cic \textit{De Officiis} 1.123) ; \textit{cum haec nobis quaerenda sit in causis} (‘As in our causes we have to research these [goals]’ Cic \textit{De Oratore} II. 120).}

\ea\label{ex:montaut:24}
\ili{Latin}\\
\gll {\upshape a.} mihi  cantandum  est    {> \upshape b.} (ego)     cantare  habeo \\
{} \textsc{1sg.dat}   sing.\textsc{ger} {} {}  be.\textsc{prs.3sg}      \textsc{1sg.nom} sing.\textsc{inf.act} have.\textsc{1sg} \\
\glt ‘I have this to be sung, to sing  > I will sing’    (\citealt[107]{Kuryłowicz1931})
\z 


The -\textit{nd}{}- \isi{verbal adjective} was used with a \isi{dative agent} and agreed in gender and number with the \isi{nominative} patient, like the -\textit{tavya} sentences in (\ref{ex:montaut:16}--\ref{ex:montaut:17}). But unlike in the latter, the \isi{copula} was required (agreeing with the \isi{nominative} patient), as shown in \REF{ex:montaut:25}: 

\ea\label{ex:montaut:25}
\ea 
\ili{Latin}\\
\gll mihi  sit  referenda  omnis illa oratio\\
\textsc{1sg.dat}   be.\textsc{prs.3sg}   relate.\textit{nd}.\textsc{f.sg}   all this discourse\textsc{.f.sg.nom}   \\
\glt ‘[to which]  I should relate this whole discourse.’ (Cicero: \textit{De Oratore} II.114)
\ex  
\gll nunc est  bibendum\\
now   be.\textsc{prs.3sg}    drink.\textit{nd}.\textsc{n.sg}   \\
\glt ‘Now one should drink.’ (Horatius)
\z 
\z 

Thus, we can summarize the \ili{Latin} evolution of perfect and future in \tabref{tab:montaut:2} below, which highlights similarities to the IA data. In a very illuminating paper on the evolution of grammatical categories, \citet{Kuryłowicz1965} further developed the hypothesis of a deep similarity between these two tenses: on his account, perfect and future are both basically non-active because they do not aim at depicting an action, but at representing viewpoints, from the present, on this action. This contrasts with the present that aims at directly representing a process.
%%please move \begin{table} just above \begin{tabular
\begin{table}
\begin{tabularx}{\textwidth}{QQQ}
\lsptoprule
alignment & perfect & future\\
\midrule
Nominative: synthetic verb form, \textsc{nom} 
subject, person agreement (Early/Classical \ili{Latin}) & \textit{(ego) feci/cantavi} & \textit{(ego) faciam /cantabo}\\
\tablevspace
Non-\isi{nominative}: participial verb, \textsc{dat} 
agent, no person agreement (Late \ili{Latin}) & \textit{mihi factum/cantatum est} & \textit{mihi faciendum /cantandum est}\\
\tablevspace
Nominative: V + have, \textsc{nom} subject, 
Person agreement (Vulgar \ili{Latin}) & \textit{(ego) factum/cantatum habeo} & \textit{(ego) fieri/cantari habeo}\\
\tablevspace
Nominative: V+ have, \textsc{nom} subject, 
person agreement (French) & \textit{j’ai fait/ chanté} & \textit{je fer.ai/chanter.ai}\\
\lspbottomrule
\end{tabularx}
\caption{Alignment development from Early/Classical Latin to French.}
\label{tab:montaut:2}
\end{table}

\citet{Benveniste1966[1965]} reached similar conclusions regarding the symmetry of past and future (both ‘orthogonal’ relatively to the present, both representing perspectives on an action rather than processes proper). However, \citet[131]{Benveniste1966[1965]} radically rejected \citeauthor{Kuryłowicz1965}’ interpretation of a parallel evolution from \ili{Latin}, arguing that the \ili{Latin} construction involving \textit{habere} never had an \isi{obligative} meaning and was used as a ‘future of predestination’ in Christian preachers, in the meaning ‘fated to happen’. This latter argument has later on been proven wrong, as well as the accusation against \citeauthor{Kuryłowicz1965} of mistaking a passive \isi{infinitive} for an active \isi{infinitive}.  Yet, \citeauthor{Benveniste1966[1965]}’s major reason for rejecting the symmetry of the two developments might have had something to do with his own thesis of the ‘\isi{possessive} perfect’, since it seems more difficult to derive the \isi{future meaning} from the \isi{possessive} (more details in \citealt{Montaut1997}).
  The\largerpage ambiguity between the original meaning of obligation (or potential) and the new meaning of future was attested between the 2nd and 3rd century by grammarians such as Tertullian and Pompeius. On the other hand, the \isi{future meaning} was found only later (in conformity with \citet{BybeeEtAl1994}’s generalizations), mainly in texts by Christian writers and grammarians, with no particular connotation of predestination. Moreover, it is almost exclusively used in learned texts, chiefly from Africa. Its earliest – and still ambiguous – instance in a colloquial register is found in \ili{Wâdi} Fawâki’s letters (cf. \citealt[659]{Adams2011,Adams2013}).\footnote{Compare the ambiguous (future/alethic) example from Tertulian quoted in \citet[148]{Adams2011}: \textit{si enim sustuleris istam tertiam, remanere habent duae} ‘for if you take away the third (syllable) two will (have to) remain’ (GL 129.6). \ili{Wâdi} Fawâki’s letter to Rustius Barbarus is also ambiguous between future and deontic modality: \textit{adferre habes} ‘you have to bring’ or ‘you will bring’ \citep[659]{Adams2013}.} 

To my knowledge, no study points to a direct transformation of the \ili{Latin} \isi{dative} alignment with the -\textit{nd}{}- \isi{gerund} into the \isi{nominative} alignment with \textit{habere}, but this reanalysis is basically equivalent to the transformation regarding the perfect. The striking event in the history of the \ili{Romance} languages is the total extinction of the \ili{Latin} synthetic future and the subsequent formation of new futures, often on the basis of an \isi{obligative} periphrasis (\textit{habere}). Thus, grammaticalization of obligation into future is undoubted (\citealt{Adams2011,Bourova2007}), not least because it conforms to a typologically common source for futures (\citeauthor{Heine2002}, this volume). In this respect, it parallels the Eastern IA data, which display an observable, continuous history from non-\isi{nominative} alignment to \isi{nominative} alignment.

\subsection{Divergent evolution of gerund in Western IA languages}\label{sec:montaut:3.3}

However, this evolution from modality to future is not pan-\ili{Indian}, and this is a strong difference with the past in the general evolution of \ili{Indo-Aryan}. Whereas the -\textit{ta} form, being the normal expression of past, rapidly became the substitute for all verbal forms of the past, the -\textit{tavya} form never became the normal expression of future because the old synthetic future was maintained in many regions and prevented the new periphrasis from extending to the field of future.
In Western IA languages, the old \isi{verbal adjective} (V.ADJ) of obligation was maintained in its original meaning up to the middle stage of NIA. For instance, the ergative-like pattern of Ashoka (early MIA), presented as the origin of the Eastern -\textit{b}{}- future in \REF{ex:montaut:18} above, has a Western equivalent in Girnar (now Pakistan):


\ea\label{ex:montaut:26}
\glll hida na kimci   jivam     arābhitpā  prajuhitavyam    na ca    samājo      kattavyo\\
 here no some   living     kill         sacrifice.          no and  assembly    do \\
\textsc{nom.n.sg}  ~ ~  ~  \textsc{cv}    \textsc{v.adj.nom.n.sg} ~ ~   \textsc{nom.m.sg} \textsc{v.adj.nom.m.sg}  \\
\glt ‘Here one should not sacrifice by killing a living creature nor hold a meeting.’ \\
 (lit. here should not be sacrificed killing a living being nor should meeting be held)
\z 

The same pattern, with a -\textit{v}- form, continued till late MIA (Apabhramsha stage, turn of the millennium) for instance in the well-known Western Jain text \textit{Paumacariu}, with or without \isi{oblique agent}, clearly patterning like the past\largerpage sentences exemplified in (\ref{ex:montaut:27}c):

\ea \label{ex:montaut:27}
\ea 
\ili{Apabhramsha}\\
\gll annu na nam.ev.au\\
 other.\textsc{m.sg}   \textsc{neg}   respect.v.\textsc{m.sg}  \\
\glt ‘No other is to be respected.’ (Paumacariu 26.3.2)
\ex 
\gll navara  ekku vau  mai͂  pālevau\\
 only one  vow.\textsc{m.sg}   \textsc{1sg.ins}   keep.ev.\textsc{m.sg}  \\
\glt ‘I shall/should observe only one vow.’ \citep[194]{Bubenik1998}
\ex    
\gll tā  keumaie͂  hau͂   gharaho  nīya\\
     then   Ketumati.\textsc{ins/obl} \textsc{1sg}   home.\textsc{loc} conduct.\textsc{ppp.m.sg}\\
\glt ‘Then I was taken home by Ketumati’/‘Then K took me home.’ \citep[148]{Bubenik1998}
\z 
\z 

Those NIA languages that inherited this state of affairs and also retained the \isi{sigmatic future} (e.g. Old \ili{Gujarati}, \ili{Marathi}, \ili{Rajasthani}) maintained the -\textit{tavya} pattern with an \isi{obligative} meaning till their middle NIA stage, contrary to the Eastern languages which started shifting to a \isi{future meaning} during their early NIA stage. Old \ili{Gujarati}, for instance, shows agreement with patient and \isi{instrumental agent} (-\textit{im}), and the \isi{obligative} meaning in the 16\textsuperscript{th}century text \textit{Upadeśamālā}:

\ea
\label{ex:montaut:28}
\ea 
\ili{Old Gujarati}\\
 \gll isī  upamā                jāṇivi   \\
 such.\textsc{f.sg}   comparison.\textsc{f.sg} know.\textit{v}.\textsc{f.sg} \\
\glt ‘Such a comparison should be known.’ \citep[64--65]{Dave1935}
\ex  
\gll ṣisyii͂   te      kārya         tatkāla ācarivau͂\\
 pupil.\textsc{ins} \textsc{dem.m.sg}     work.\textsc{m.sg}   immediately   do-\textit{v}{}-\textsc{m.sg}  \\
\glt ‘The pupil should immediately do that work.’ \citep[94]{Dave1935}
\ex
\gll te~pāpiu   jāṇi.v.au  \\
 \textsc{3m.sg.ins} know.\textit{v}.\textsc{m.sg}\\
\glt ‘He should be considered as a sinner.’ \citep[54]{Dave1935}
\z 
\z 

This is a direct continuation of the 14th\textsuperscript{} century language found in the \textit{ṣadāvaśyaka}, the oldest written testimony of Old \ili{Gujarati}. According to \citet[23]{Pandit1976}, the “\isi{gerund}” used as a \isi{predicate} had an “\isi{imperative} [= \isi{obligative}] sense” in this variety: \textit{rakhivau͂} in the masculine ‘is to be saved’, \textit{karivau͂} ‘is to be done’, \textit{vyavasthāpivi} in the feminine ‘is to be founded’.
  In Old \ili{Rajasthani}, similar \isi{obligative} constructions are still found in the 16th century, the time when the language is supposed to have diverged from Old \ili{Gujarati}. The \isi{obligative} construction is attested both with \isi{intransitive} as shown in (\ref{ex:montaut:29}c) and \isi{transitive} \isi{predicates}, the latter both with and without an agent, given in (\ref{ex:montaut:29}b) and (\ref{ex:montaut:29}a) respectively:


\ea\label{ex:montaut:29}
\ea 
\ili{Old Rajasthani}\\
\gll hisā   na   kar.av.ī\\
violence.\textsc{f.sg}  \textsc{neg}   do.\textit{av}\textsc{.f.sg} \\
\glt ‘Injury is not to be done.’ \citep[120]{Tessitori19141916}
\ex 
\gll anere vidya  lete  vinay             kar.iv.um\\
other.\textsc{pl.ins}  knowledge taking     humility.\textsc{m.sg}     do.\textit{iv}.\textsc{m.sg}  \\
\glt ‘Humility should be observed by others acquiring knowledge.’ \citep[120]{Tessitori19141916}
\ex 
\gll tai͂    na   jāi.vu\\
 \textsc{2.ins} \textsc{neg} go.v.\textsc{m.sg} \\
\glt ‘It should not be gone by you.’ \citep[101]{Khokhlova2013}
\z 
\z 

Old \ili{Marathi} also displays similar constructions, which \citet[264]{Bloch1970[1920]} presents as parallel to the past constructions, with example \REF{ex:montaut:30}: the “syntax, with the logical subject in the instrumental, [is] very similar to that of the form for past”.  

\ea\label{ex:montaut:30}
\ili{Old Marathi}\\
 \gll majhyān\~{e}    cālavl\~{e}               / dhadā         sikhavlā  \\
 \textsc{1sg.ins} go.\textsc{pot.pst.n.sg}   / lesson.\textsc{m.sg}    learn.\textsc{pot.pst.m.sg}  \\
\glt ‘I could/was able to go / to learn the lesson.’ \citep[265]{Bloch1970[1920]}
\z 

Bloch also mentions a \isi{dative} alternation for the agent (\textit{majhyāne͂} or \textit{mālā}).

\subsection{The Marathi case: an exception to the areal complementary distribution}\label{sec:montaut:3.4}

The -\textit{tavya} \isi{obligative} \isi{gerund} was, as a rule, lost in modern Western IA languages, whereas it developed into a future with \isi{nominative} realignment in the Eastern IA languages. On the other hand, the construction with the \ili{Sanskrit} \isi{participle} -\textit{ita} in the past was maintained with the original alignment and developed into an \isi{ergative} construction in the West, whereas it was realigned to a \isi{nominative} pattern in the East. One could conceive of this as a complementary distribution between languages maintaining the non-\isi{nominative} construction of the PPP on the one hand and languages maintaining the -\textit{tavya} form as a \isi{predicate} on the other. \ili{Marathi} is however an exception, since the modern language still displays the same pattern as in \REF{ex:montaut:30}, albeit with the new case markers. Note that in modern grammars, the -\textit{av}{}- suffix is usually glossed as ‘\isi{subjunctive}’ (sometimes as ‘potential’), and the marker \textit{ne} is usually glossed as agent in such constructions. However, it is glossed \isi{ergative} in past constructions \citep[44]{DhongdeWali2009},  and now occurs only in the 3rd person, although in the unmarked 1st and 2nd person, the verb still agrees with the object.\footnote{The verb also agrees with the agent in the 2nd person: \textit{tyāne samayā ghas.l.yā} (\textsc{3sg.erg} lamp.\textsc{n.pl} wash.\textsc{pst.n.pl}) ‘he washed the lights’, vs \textit{tu samayā ghas.l.yā.s} (\textsc{2sg.erg} lamp.\textsc{n.pl} wash.\textsc{pst.n.pl2sg) ‘}you washed the lights’.} \ili{Marathi} is hence exceptional in two respects: one, unlike other Western languages, it maintained the -\textit{tavya} \isi{predicative} form; two, unlike Eastern languages, it maintained this form in its original \isi{obligative} meaning and alignment.

\ea\label{ex:montaut:31}
\ili{Marathi}\\
\ea   
\gll tyāne           ghari       yā.v.e\\
\textsc{3m.sg.ins/ag}   home.\textsc{loc}    come.\textit{av}.\textsc{n.sg} \\
\glt ‘He should come home.’
\ili{Marathi}\\
\ex  
\gll tyāne ci͂c                   khā.v.ī\\
\textsc{3m.sg.ins/ag}   tamarind.\textsc{f.sg}  eat.\textit{av}\textsc{.f.sg}  \\
\glt ‘He should eat tamarind.’
\z 
\z 

This construction has a \isi{nominative} counterpart \REF{ex:montaut:32}, with the verb agreeing with the subject. It is analysed as potential and not \isi{obligative} in \citet{Wali2004} and \citet{DhongdeWali2009}, but for \citet[290]{Pandharipande1997} both meanings are conveyed by the \REF{ex:montaut:31} construction. 

\ea
\label{ex:montaut:32}
\gll to             ghari          yā.vā\\
\textsc{3m.sg.nom}  home.\textsc{loc}   come.\textsc{sbjv.3m.sg}   \\ 
\glt ‘He may come home.’
\z 

What is clear from these various evolutions is that wherever a different form for the future was available – be it the inherited \isi{sigmatic future} in \ili{Gujarati}, \ili{Rajasthani}, and \ili{Braj}, or a new form -\textit{l}{}- as in \ili{Marathi} – the -\textit{tavya} \isi{verbal adjective} retained its \isi{modal} value (\ili{Marathi}) or disappeared from the TAM paradigm altogether (other Western languages). All Western languages except \ili{Marathi} indeed developed \isi{periphrastic} constructions with verb ‘be’ or ‘need’ and \isi{dative} ‘subjects in non-canonical alignments (see ex. \REF{ex:montaut:36} below). The reason why only \ili{Marathi} maintained the original construction and verbal form is most probably because it is also the only language which retained the old \ili{Sanskrit} \isi{infinitive} in \textit{-tum}. In contrast we observe the presence of \textit{-v- /-b-} infinitives inherited from the -\textit{tavya} form in all languages which lost the \isi{obligative} meaning of the \isi{predicative} -\textit{tavya}.

\section{From the -\textit{tavya} gerund to infinitive and verbal noun}\label{sec:montaut:4}

The original, invariable \ili{Sanskrit} \isi{infinitive} (-\textit{tum}) rapidly fell out of use in \ili{Indo-Aryan} and the category itself almost disappeared, replaced in most modern languages by verbal nouns in~-\textit{an/ana}  or, in the East, by former \isi{participle}s. \ili{Marathi} is the only language to maintain the old form and category. \ili{Gujarati} is another exception regarding the \isi{infinitive} and \isi{verbal noun}, since it never developed an -\textit{an} \isi{verbal noun} and instead marked both the \isi{infinitive} and the \isi{verbal noun} by a single form derived from the -\textit{tavya} \isi{gerund}. This development also occurred in other languages but was ultimately inhibited by the growth of the -\textit{an} forms.

\subsection{The Gujarati infinitive: infinitive and inflecting verbal noun}
The single form of the \isi{infinitive} in \ili{Gujarati} is -\textit{vu͂}, which scholars relate to the old \isi{verbal adjective} in -\textit{tavya}: \citet[64]{Dave1935} and \citet[966]{Chatterji1926} for instance derive the \ili{Gujarati} \isi{verbal noun} \textit{karv\~{ū}} from the \isi{verbal adjective} or passive \isi{obligative} \isi{participle} \textit{kartavyam}. This \isi{verbal noun} agrees in case, similar to the -{\textit{an/ -ana}} {form}{} in other IA languages: \textit{jovu͂} ‘to see’, \textit{jovā lāyak} ‘worth seeing’.

\ea\label{ex:montaut:33} 
\gll a khā.v.ā  lāyak   vastu  che\\
\textsc{dem}    eat.\textit{v}.\textsc{obl}    worth   thing   be.\textsc{3sg}  \\
\glt ‘This is a thing worth eating.’
\z 

In \isi{periphrastic} permissive constructions with the verb \textit{de} ‘give’ and inceptive constructions with the verb \textit{lag} ‘touch/start’, the \isi{verbal noun} similarly displays the oblique form -\textit{ā}: \textit{karvā de} ‘permit to do’, \textit{khāvā lag} ‘start to eat’ (cf. \citealt[688]{Cardona2003}; \citealt[52 ff.]{Dave1935}).\footnote{In these two constructions, languages with -\textit{an/na} infinitives also display the oblique form of the \isi{verbal noun} (\ili{Hindi} \textit{karne do} [do.\textsc{inf.obl} give.\textsc{imper}] ‘allow [x] to do/let X do’, \textit{khāne lagā} [eat.\textsc{inf.obl} start.\textsc{m.sg}] ‘(he) started to eat’.} Used in \isi{obligative} constructions with ‘be’, it agrees in gender and number with the object, like the -\textit{an} \isi{infinitive} forms of other NIA languages in \isi{obligative} constructions. But unlike other NIA languages, the \ili{Gujarati} -\textit{v}{}- infinitival form, in combination with \isi{auxiliary} ‘be’ or ‘need’, can be used to convey not only obligation, but wish too, as shown in \REF{ex:montaut:34b} and \REF{ex:montaut:35b}, respectively. The meaning of obligation is conveyed by an extended form of the \isi{infinitive} (-\textit{v-an}{}-), with the relational suffix -\textit{an} added (cf. \citealt[677]{Cardona2003}), whereas the desiderative meaning requires the short form -\textit{v}{}-. Both forms are constructed with the \isi{copula}:

\ea\label{ex:montaut:34}
\ea\label{ex:montaut:34a} 
\gll mar.e caupḍī vanc.van.ī       che    /  lekh   lakh.van.o   che\\
\textsc{1sg.ag} book.\textsc{f.sg}  read.\textit{van}.\textsc{f.sg} be.\textsc{prs.3}  / article.\textsc{m.sg}  write.\textit{van}.\textsc{m.sg} be.\textsc{prs.3s}  \\
\glt ‘I have to read a book                        /     {to write an article.’}
\ex \label{ex:montaut:34b}
\gll tam.ne kyā  jā.vān.u  che\\
\textsc{2.dat}  where   go.\textsc{inf.n.sg}   be.\textsc{prs.3sg}  \\
\glt ‘Where do you have to go?’  \citep{Cardona2003}
\z 
\z 

\ea
\label{ex:montaut:35}
\ea \label{ex:montaut:35a}
\gll mar.e  caupḍī vanc.v.i che       lekh lakh.vo  che\\
\textsc{1sg.ag}  book.\textsc{f.sg}  read.v.\textsc{f.sg} be.\textsc{prs.3}  article.\textsc{m.sg}  write.v.\textsc{m.sg} be\textsc{.prs.3sg} \\
\glt ‘I want to read a book, to write an article.’
\ex \label{ex:montaut:35b}
\gll tam ne  kyā  ja.v.u   che\\
\textsc{2}   \textsc{dat}  where  go.\textsc{inf.n}   be\textsc{.prs}.\textsc{3}  \\
\glt ‘Where do you want to go?’ \citep{Cardona2003}
\z 
\z 

It should be noted that, although the meanings are very close to the \ili{Marathi} \isi{obligative} or potential sentences with finite forms in -\textit{v}{}-, the morphology here is that of a \isi{verbal noun}. This is exactly parallel to the -\textit{an} formations of \ili{Hindi} or other languages with gender/number agreement of the \isi{verbal noun}. Interestingly in \ili{Gujarati}, case marking is not limited to the \isi{dative} as in other \ili{Indo-Aryan} languages like \ili{Hindi}, but exhibits an alternation between \isi{dative} and \isi{agentive}. Both are possible with the same pronoun as shown in example (\ref{ex:montaut:36}a), which expresses necessity with the ‘need’ \isi{auxiliary} \textit{joie} + \isi{verbal noun}. The \ili{Hindi}/\ili{Urdu} equivalent, with \isi{verbal noun} in -\textit{n}{}- and \textit{cāhie} (etymologically from a ‘look’ verb, like the \ili{Gujarati} \textit{joi}), has the same agreement pattern (default agreement with intransitives, object with transitives), but no alternation with the \isi{ergative}/\isi{agentive}. Punjab \ili{Hindi} does display case alternation, depending on dialects (cf. \citealt{Khokhlova2013}):

\noindent\parbox{\textwidth}{\ea\label{ex:montaut:36}
\ea 
\ili{Gujarati}\\
\gll mār.e  /  ma.ne ghar  jā.v.u͂        (gujarati bol.v.i)  joie\\
{\textsc{1sg.}}\textsc{ag}  / \textsc{1sg}.\textsc{dat} home go.\textsc{inf.n.sg}   \ili{Gujarati}.\textsc{f} speak.v.\textsc{f} need \textsc{prs.3sg} \\
\glt ‘I should go home. (speak \ili{Gujarati})’ \citep{Cardona2003}
\ex  
\ili{Hindi}\slash\ili{Urdu}\\
\gll mujhe ghar  jā.nā            (gujarati bol.nī) cāhie\\
\textsc{1sg}.\textsc{dat} home   go.\textsc{inf.m.sg} \ili{Gujarati}.f speak.v.f  need \textsc{prs.3sg} \\
\glt ‘I should go home (speak \ili{Gujarati}).’ 
\ex    
\ili{Punjab Hindi/Urdu}\\
\gll mai͂.ne  /  mujhe  ghar jānā  hai  \\
{\textsc{1sg.}}\textsc{erg} /  \textsc{1sg}.\textsc{dat} home go.\textsc{inf.m.sg} be.\textsc{prs.3sg} \\
\glt ‘I should go home.’ 
\z 
\z}

The agent in (\ref{ex:montaut:36}c) is usually glossed \isi{ergative} because of its identical form (\textit{ne}) with the specific \isi{ergative} marker in \isi{transitive} past clauses in \ili{Hindi}/\ili{Urdu} and \ili{Punjabi}. In sentences like (\ref{ex:montaut:36}a) and (\ref{ex:montaut:33}--\ref{ex:montaut:34}) above, the \ili{Gujarati} -\textit{e} is usually glossed agent whereas the same marker is glossed \isi{ergative} in \isi{transitive} past clauses with similar agreement pattern (like in example \ref{ex:montaut:13} above). Despite the variable glosses, it seems that \ili{Gujarati} is shifting to \isi{dative agent}, the standard expression of obligation in all languages with the new -\textit{an} verbal nouns, originating from a construction with an \isi{instrumental agent}. The \isi{predicative} -\textit{tavya} \isi{verbal adjective} was used to express obligation in the old language (like throughout middle IA), but a \isi{copula} construction also emerged in the 16th century, where the \textit{–tavya} form is an \isi{infinitive} and no longer a \isi{predicate} by itself, similarly constructed with an \isi{instrumental agent}:

\ea\label{ex:montaut:37}
\ili{Old Gujarati}\\
\gll jīṇaī͂    jīvii͂          jīhā͂  jāi.v.au͂  chai\\
\textsc{rel.ins}   person.\textsc{ins}    there go.v.\textsc{m.sg} be.\textsc{3m.sg} \\
\glt ‘The person who is designated to go there.’ (= who should go) \citep[67]{Dave1935}
\z 

Outside this construction, verbal nouns in -\textit{v}{}- were also as in the modern language commonly used with the meaning ‘worth of V’, and as an action nominal (\textit{āsana nau͂ le.v.au͂} ‘the taking of the seat’, \textit{viṇasi.v.ā nai-kāji} ‘in order to be destroyed’, cf. \citealt[54]{Dave1935}).
 
\ea \label{ex:montaut:38}    
\ea 
\gll pāsachā siu͂ aṇamila.v.au͂ bhalau͂\\
depraved.\textsc{obl}   with   not.mix.\textit{v}.\textsc{m.sg}   good.\textsc{m.sg} \\
\glt ‘It is good (the good is) not to mix with the depraved.’ \citep[64]{Dave1935}
\ex  
\gll  pāsachā  siu͂ boli.v.u͂, ekai͂        upāśrayi rahi.v.u͂\\
depraved.\textsc{obl} with   speak.\textit{v}.\textsc{n.sg} one.\textsc{loc}   hostel.\textsc{loc}   stay.\textit{v}.\textsc{n.sg} \\
\glt ‘To speak with the depraved, to stay at the same hostel.’ \citep[64]{Dave1935}
\z 
\z 

\subsection{Development and specialization of the -v/-\textit{b}{}- infinitive}\label{sec:montaut:4.2}

This -\textit{v}{}- \isi{infinitive} is present from late {MIA onwards} \citep[388]{Pischel1900}. In the 11th century Jain Digambara texts (cf. \citealt[322]{Tagare1948}) it is attested as a \isi{verbal noun}, in parallel to its \isi{predicative} use as in examples \REF{ex:montaut:27} and \REF{ex:montaut:28}. Both uses are maintained throughout early NIA and continue to be used in many modern Western languages such as \ili{Braj} (\textit{māribau͂, māribau} ‘to strike’), \ili{Rajasthani} (\textit{mārabo}), and \ili{Kanauji} further north (\textit{māribo}). This long tradition is documented for the older stages of \ili{Rajasthani} \ili{Gujarati} in \citet[121]{Tessitori19141916}:

\ea \label{ex:montaut:39}
\ili{Old Rajasthani}\\
\gll jīp.ava vaṃchai\\
win.\textit{ava}   want.\textsc{pst.3sg} \\
\glt ‘He wishes to conquer.’ (\citealt[121]{Tessitori19141916}; Chaya to Yogasāstra III-134)
\z 

In modern languages with two forms of the \isi{infinitive}, the distribution is not quite clear. \ili{Braj} for instance seems to select the -\textit{n}{}- form for goal functions and as a complement of inceptive verbs (\ref{ex:montaut:40}c--d) and the -\textit{b-/-v}{}- form for more nominal uses (\ref{ex:montaut:40}a--b), as illustrated by the 18--19th c. examples given by \citet[289]{Kellogg1875}:

\ea \label{ex:montaut:40}
\ea 
\ili{Braj}\\
\gll sadā kahu  sau͂  rah.iv.au        nāhi \\
always   somebody   with     stay.\textit{iv}.\textsc{m.sg}  \textsc{neg} \\
\glt ‘One does not remain always in the same company.’ \\
(Lit. ‘There is no staying always with anybody’)
\ex   
\gll mere putrani kau͂   pandit       kar.iv.e  jog  hai\\
 my   son.\textsc{pl} \textsc{acc}   wise.men   do.\textit{iv}.\textsc{obl}  worth  be.\textsc{sg} \\
\glt ‘You are competent for making my sons wise men.’
\ex 
\gll tum sau͂  kah.an kau͂  āyau  hau͂\\
\textsc{2}  \textsc{soc}   say.\textit{an}  \textsc{dat}   come  \textsc{pft.1sg} \\
\glt ‘I have come to tell you.’
\ex  
\gll rājā   kah.ani   lāgyau\\     
king  say.\textit{an}  start.\textsc{m.sg}\\
\glt ‘The king began to say.’
\z 
\z 

\citet[16]{Snell1991} suggests that in Classical \ili{Braj} (16--18th ct.), the -\textit{b}{}- form was chiefly used with inflected verbal nouns, while the -\textit{na}{}- form occurred with semi auxiliaries such as \textit{de}{}- (permissive), \textit{pā} (ability) and \textit{lag} (inception). Both forms were also employed to signal goal functions.\footnote{For instance, as a complement of movement verbs: \textit{moko͂ pakaran ko͂ āyo} ‘he came to touch me’, \textit{moko͂ spars karive kau͂ doryau} ‘he ran to touch me’.}
\ili{Bundeli}, a South-western language sometimes considered a \ili{Hindi} dialect, also has two types of infinitives. The -\textit{na} form is preferred in permissive and inceptive constructions (\textit{mor̩ā kho͂ khānā do} ‘let the boy eat’ (cf. \citealt[132]{Jaiswal1962}) while the -\textit{b}{}- form typically occurs in nominal uses.\footnote{ The -\textit{na}{}- form alternates with the -\textit{b}- form in presence of the agent suffix -\textit{baro} (\textit{karna baro} ‘doer’, \textit{khabe baro} ‘eater’) whereas the -\textit{b}{}- form is only allowed with the -\textit{aiya} suffix (\textit{dekhabaiya} ‘seer’).}

\ea\label{ex:montaut:41}
\ea 
Modern \ili{Bundeli}\\
\gll daur.b.o  ūke  cala.b.e       barābar hai\\
run.\textit{b}.\textsc{m.sg}     \textsc{3.sg.gen}    walk.\textit{b.}\textsc{obl}   same        be.\textsc{3.sg} \\
\glt ‘Your running is equal to his walking.’ \citep[133]{Jaiswal1962}
\ex 
\gll tumāe   kara.b.e  kho͂  ite bohota   hai\\
\textsc{2.obl} do.\textit{b}.\textsc{obl}   \textsc{dat} here much be.\textsc{3sg} \\
\glt ‘There is much for you to do here.’ \citep[133]{Jaiswal1962}
\z 
\z 

Whereas all Western languages at some point displayed, and many of them still display a -\textit{v-/-b-} \isi{infinitive}, the \ili{Marathi} exception can be accounted for by the resilience of the inherited \ili{Sanskrit} \isi{infinitive}. It is also the only language which maintained the \isi{predicative} \isi{gerund} in its original \isi{obligative} meaning. Since in Eastern languages the \isi{gerund} developed into a \isi{future marker}, one might also think that languages which lost the \isi{predicative} construction turned the \isi{gerund} into a \isi{verbal noun}. On this account, three grammaticalization paths of the \isi{gerund} are in complementary distribution: one, obligation and potential with new case markers as in \ili{Marathi}; two, future with realignment in Eastern languages; and three, \isi{infinitive} and \isi{verbal noun} as in \ili{Gujarati} and other Western languages. But it is obvious that the \textit{-b-/-v-} \isi{verbal noun} is also present in Eastern languages, although to a lesser extent than in the West.

\subsection{The infinitive in the Eastern languages}\label{sec:montaut:4.3}

Modern \ili{Maithili} has three forms of \isi{infinitive} (-\textit{ana/āna, -al/ala, -aba/ab}), the two latter forms alternate with postpositions: \textit{dekh.lā me sunnar} or \textit{dekh.bā me sunnar} ‘beautiful to look at’ \citep[519]{Jha1958}. Similar examples were already attested in Middle \ili{Maithili} with the same -\textit{ba/-va} form, and with a wider range of functions, such as \textit{kopahū kara.vā͂ joga} ‘fit for showing anger’ (\ili{Vidyapati} 50, in \citealt[519]{Jha1958}), or as a \isi{verbal noun} in inceptive and volitive constructions:

\ea
\label{ex:montaut:42}
\ili{Middle Maithili}\\
\gll kamal.āsana kichu  kahavā͂        lāgu\\
lotus.seat  something   say.\textit{v}.\textsc{inf;obl}     start.\textsc{m.sg} \\
\glt ‘The lotus-seated [god] began to say something.’ \citep[611]{Jha1958}
\z 

\ili{Assamese} (North East) also has an infinitival form, ending in -\textit{iba}, which is included into the suffix -\textit{(i)ba.logya} ‘worth to’ (\textit{sa.ba.logyia} ‘worth to see’), and is required as a nominal formative and in concatenation with the \isi{modal} ‘be able’:

\ea\label{ex:montaut:43}
\ea 
\ili{Assamese}\\
\gll za.b.âr   xâm.ât\\
leave.\textit{b}.\textsc{gen}   time.\textsc{loc} \\
\glt ‘Time of leaving.’
\ex 
\gll xi  saikel sâla.bo  par.e\\
\textsc{3m.sg}   bicycle    ride.\textit{bo}   be.able.\textsc{3m.sg} \\
\glt ‘He can ride a bicycle.’ \citep[425]{Goswami2003}
\z 
\z 

Similarly, the standard \ili{Oriya} \isi{infinitive} ends in -\textit{ibā} and is used as a noun before postpositions (\textit{ās.ibā ku} ‘(in order) to come’). Although modern standard \ili{Bengali} does not display -\textit{b-/-v}{}- infinitives, the variety spoken in Assam \citep{Tunga1995} has similar forms used for complements of \isi{modal} verbs (\textit{karibar paro} ‘I am able to do’). \citet{Grierson1903-1928} mentions \ili{Bengali} constructions such as \textit{ja.b.ār somoy.i} [go.\textit{b}.\textsc{gen} time.\textsc{loc]} ‘at the time of leaving’, \textit{korbā lāgil} ‘he began to do’, continuing from Middle \ili{Bengali} \textit{lage balibar} ‘he began to say’ (cf. \citealt[1008]{Chatterji1926}). Besides, standard \ili{Bengali} still uses it in nominal functions, particularly in the genitive (-\textit{ar} ending): \textit{ja.b.ar somoi} ‘at the time of coming’; \textit{as.(i)b.ar janye} ‘for coming’. As for mid-Eastern languages such as \ili{Bhojpuri} and \ili{Awadhi}, they have -\textit{be/-bu} verbal nouns alternating with the -\textit{ana} forms in both their modern and early stages (cf. \citealt[282--285]{Saxena1937}). So, it seems that, apart from \ili{Marathi}, which maintained the old \ili{Sanskrit} \isi{infinitive}, only standard \ili{Hindi}/\ili{Urdu} and \ili{Panjabi}, the so-called central IA languages, do not have the -\textit{b/v} \isi{infinitive}. However, it should be remembered that \ili{Hindi} and \ili{Urdu} , sharing as common ancestors \ili{Braj}, Avadhi, and, most importantly, the popular literary \textit{koine} of the mystic preachers, the \ili{Sant Bhasha}, and \ili{Panjabi} also shares the Sant Basha ancestry.\footnote{\ili{Panjabi}’s founding texts (\textit{Guru Granth Sahib}) displays a mixture of \ili{Sant Bhasha} texts, many from Kabir and early \ili{Panjabi} still not much differentiated from other dialects of what will become the \ili{Hindi} language later. Although associated to Fariduddin Ganjshakar (13\textsuperscript{th} ct.), the rise of \ili{Panjabi} as distinct from various neighboring ‘Old \ili{Hindi}’ dialects occurs only later. As for \ili{Hindi}/\ili{Urdu}, recent under these names, their earlier forms are regional languages from \ili{Braj} to \ili{Awadhi}.} They retained only the -\textit{an/-ana} \isi{infinitive}, which is used in \isi{obligative} constructions as well as a \isi{verbal noun}, and they developed a new \isi{periphrastic} -\textit{gā} future, so that the old \isi{verbal adjective} has left no trace in the standard modern stage of these languages. But on the whole, more languages still display the \isi{infinitive} more than any other \isi{tense} or \isi{modal} form derived from the old -\textit{tvaya} \isi{verbal adjective}. This can be seen as an indication that the basic meaning of the form was more nominal than verbal (or \isi{modal}).

\subsection{Original meaning of the so-called obligation gerund}\label{sec:montaut:4.4}

A last piece of evidence for the remarkable persistence of the nominal meaning associated with the -\textit{tavya} form in New \ili{Indo-Aryan} comes from \ili{Romani}, a language separated from West-central IA in the beginning of the 2\textsuperscript{nd} millennium. In all \ili{Romani} dialects, the \isi{infinitive} is notoriously absent as in most languages of the Balkans (cf. \citealt{Boretzky1996}). While an \isi{infinitive} form has been created out of contact with various local languages, the old -\textit{tavya} form was however maintained as a noun of action in -\textit{iben}, which is the only suffix for deverbal and deadjectival abstract nouns in the South Balkan dialects. For instance, the noun referring to a beverage, that is, something worth drinking, is \textit{zspiben}, from the verbal IA root \textit{pi} ‘drink’ (cf. \citealt{Beniśek2010}). Similarly, in most \ili{Romani} dialects, deverbal abstracts end in -\textit{iben},\footnote{Deadjectival abstracts usually end in -\textit{ipen}, a suffix inherited from the IA form -\textit{pan}, and many dialects present a merger of the two forms.} a formation parallel to the other gerundival suffix -\textit{niya/n͂ya} ‘able to’, ‘worthy of’ which also produces deverbal abstracts such as the noun \textit{pāj} ‘drink’ (common \ili{Romani}).
These various evolutions cast doubt on the supposedly original meaning of the \isi{gerund}, considered by \citet[966]{Chatterji1926} to have developed “side by side” from obligation into \isi{infinitive} and future.\footnote{“The simple future notion evolved gradually; side by side with it, the old notion of an action to be done continued, and was modified into simply the notion of an act”.} Far more plausible from a typological perspective is the evolution from \isi{verbal noun} to \isi{modal} and future.
Going back to the origin of the -\textit{tavya} form in Old \ili{Indo-Aryan}, this form is itself derived from the \isi{verbal noun} in -\textit{tu}, the same which in the accusative was used throughout Classical \ili{Sanskrit} as an \isi{infinitive} (-\textit{tum}) and maintained in \ili{Marathi} with the -\textit{un} \isi{infinitive} ending (-\textit{idum} > Maharashtri -\textit{iuṃ} > Mod. \ili{Marathi} -\textit{un}). In the \isi{dative} case, -\textit{tave}, the form was also used as an \isi{infinitive} (-\textit{tave} > MIA -\textit{tae}, > Ardhamagadhi -\textit{ttae}). This -\textit{tave} \isi{verbal noun} could also accept the gerundival suffix -\textit{ya} (cf. \citealt[612--615]{Debrunner1954}), which only later specialized in necessity constructions while the other gerundive occurring on the -\textit{ana} nouns of action (-\textit{anlya}) tended to be lexicalized (cf. \citealt[190]{Bubenik1998}). It is also worth mentioning that the nominalising (deadjectival) suffix -\textit{pan}, still used in NIA languages, is also supposed to derive from a gerundive ending -\textit{tva} \citep{Tagare1948}, suffixed with -\textit{an} (-\textit{tvana} > -\textit{pan}). This suffix is very productive and is either directly suffixed to the base (such as in \textit{bac.pan} ‘childhood’ {\textless} \textit{baccā} ‘child’) or to the oblique or direct form of the adjective (such as in \textit{akele.pan} or \textit{akelā.pan} ‘solitude’{ {\textless}} {\textit{akel}ā} ‘alone, lonely’). Not surprisingly, it occurs in Classical \ili{Sanskrit} as a postpositional NP, patterning exactly like (\ref{ex:montaut:43}a): 


\ea
\label{ex:montaut:44}
\gll na  ayam vaktavya.sya  kālaḥ\\
\textsc{neg}   this   speak-\textit{tavya}.\textsc{gen}   time \\
\glt ‘It is not the time of speaking.’ (from \textit{Panchatantra}, in \citep[278]{Bloch1970[1920]}
\z 

The \ili{Latin} data leads to similar questions regarding the original meaning of the \isi{verbal adjective} or \isi{gerund} of necessity in -\textit{nd}{}-. Whereas the well-known construction with \isi{copula} and possible agent in the \isi{dative} mentioned in \sectref{sec:montaut:3.2} has always a necessity reading, conveying both passive voice and obligation, the form itself seems to be more general in meaning. This conclusion was first formulated by \citet[285]{ErnoultThomas1951}: “il exprime simplement l’idée verbale”, then developed by \citet[164ff]{Touratier1994} who insists that the construction, not the form, conveys the meaning of necessity in predications with the verb ‘be’, with an optional \isi{dative agent} (\ref{ex:montaut:45}a). Used as an attribute the -\textit{nd}{}- \isi{verbal adjective} obtains a vaguely abilitative meaning like the -\textit{able} suffix (\textit{orator legendus} ‘an orator worth reading’, ‘a readable orator’), but it is mostly used, in Touratier’s words, “merely as an \isi{infinitive}” since the beginning of the common era, devoid of any \isi{modal} meaning \REF{ex:montaut:45b}:

\ea\label{ex:montaut:45}
\ea \label{ex:montaut:45a}   
\ili{Latin}\\
\gll consola-nd-us hic  mih-ist   (= \textit{mihi est})  \\
console-\textit{nd}{}-\textsc{nom.m.sg}   this.\textsc{nom.m.sg}   \textsc{1sg.dat}{}-be.\textsc{3m.sg} \\
\glt ‘I have to console him.’ (= he is to be consoled to me) 
(Plautus, in \citealt[165]{Touratier1994})
\ex  \label{ex:montaut:45b}
\gll de consilio relique-nd-i  Italiam\\
 about   decision.\textsc{abl}   leave-\textit{nd}{}-\textsc{gen}   Italy.\textsc{acc} \\
\glt ‘Regarding my project of leaving Italy.’ (Cicero from \citealt[165]{Touratier1994})
\z 
\z 

In non-\isi{predicative} constructions, the -\textit{nd}{}- \isi{gerund} is clearly devoid of obligation, a meaning which grammaticalized only in \isi{predicative} constructions with the verb ‘be’.{} Countless formations attest to this nominal behaviour, particularly its use as genitive complements, from the well-known \textit{Ars ama.ndi} (‘art of loving’) to the three \textit{libidines} (‘desires’) stigmatized by Augustinus (\textit{libido domina.ndi, scie.ndi, frue.ndi} ‘desire {of domination, knowledge, enjoyment’}). The difference with the \ili{Indo-Aryan} evolution is that a further development into a standard \isi{infinitive} never took place, because of the preservation of the inherited \isi{infinitive} in -\textit{re} throughout Classical and Vulgar \ili{Latin}, still present in modern \ili{Romance} languages. As a result, the \isi{gerund} form disappeared altogether in \ili{Romance} languages. But its initial meaning of expressing a mere verbal notion, common to both Old IA and \ili{Latin}, nicely accounts for the various grammaticalization paths of this ‘\isi{gerund}’, richly illustrated in NIA and also in various stages of \ili{Latin} (though not in modern \ili{Romance} languages). 

\section{Conclusions}\label{sec:montaut:5}

Regarding the grammaticalization paths of TAM-markers, the various developments in \ili{Indo-Aryan} do not neatly mirror areal sub-classification, neither in meaning nor in form. Two pure \isi{tense} markers, -\textit{l}{}- for past and -\textit{b}{}- for future, were the Eastern outcome of the grammaticalization of the \isi{past passive participle} and the \isi{gerund}, respectively. Except for the \ili{Marathi} -\textit{l}{}- past, none of these occur in Western languages in these functions. As for the meaning, the development of \isi{definite past} out of the old resultative \isi{participle} is omnipresent, whereas the development of the \isi{gerund} into future is limited to the Eastern group, and its development into a \isi{modal} \isi{predicate} (-\textit{av}{}-) is restricted to \ili{Marathi}. Both developments contrast strikingly with the \ili{Latin}-\ili{Romance} data. In the latter languages, the perfect of comparable origin remained tense-wise a perfect (only recently encroaching on the \isi{definite past} domain in spoken French). This is because of the \isi{copula} being present right from the initial step, but even more because of the resilience of the old \isi{aorist}. The omnipresence of the ‘have’ future based on a \isi{modal} \isi{predication} also differs from the restricted development of future in IA, even though it is of comparable origin: the resilience of the old future in the West IA accounts for its limited development in IA as a whole, whereas the \ili{Latin} future disappeared in all \ili{Romance} languages. Similarly, the development of the \isi{gerund} form into an \isi{infinitive} was apparently inhibited by the persistence of the old \isi{infinitive} throughout \ili{Romance}, a form which was lost. By contrast, the loss of the \isi{infinitive} in all IA languages, except \ili{Marathi}, allowed for an extensive development of the \isi{gerund} into \isi{verbal noun} and \isi{infinitive}. The blocking effect of existing forms in the same function has obviously been decisive, and the co-existence of old and new forms, for instance for IA infinitives, usually corresponds to a specialization (i.e. one of them behaving more like a noun, the other more like a verb).
  As for the various alignment shifts, which show an areal distribution in IA, in contrast to the \ili{Romance} uniform realignment in \isi{nominative} patterns, they can be better understood by taking into account all the correlated patterns. Labelling the original \isi{nominal sentence} with \isi{instrumental agent} ‘pre-\isi{ergative}’, because of its further developments in the West, has been shown to be misleading. It is also misleading as it masks relevant correlations with both \isi{intransitive} \isi{predicates} and modal-future meanings which prevailed throughout IA up to the 15\textsuperscript{th} century. As to why the Eastern non-\isi{nominative} perfect realigned with \isi{nominative agent} (and not the Western one), a possible answer lies in the parallel realignment of future clauses. I consider this explanation more plausible than contact with \ili{Dravidian} as suggested by \citet{Chatterji1926}. Once the \isi{future meaning} wiped out the \isi{modal} one, the upgrading of the agent to a \isi{nominative subject} is expected. Note in this respect that \isi{ergative} alignment in the future is not common in languages with \isi{nominative} alignment in the present. Since both past and future constructions pattern alike, both were simultaneously realigned, unlike what we find in West IA. The formerly instrumental subject of obligation sentences shifting to \isi{dative} may be correlated with the \isi{dative} subject sentences. The latter construction gained ground in the 15th-17th century with experiential \isi{predicates}, and was extended to obligation constructions, once the formal link with the old form in -\textit{av}{}- was lost. The old case pattern prevailed only in \ili{Marathi} and \ili{Gujarati}, which kept the -\textit{av}{}- \isi{predicate} and an -\textit{av}{}- \isi{infinitive}, respectively.
  Finally, one may wonder why the two forms studied changed in meaning and category so dissimilarly: whereas the PPP became a \isi{definite past} everywhere, the \isi{gerund} grammaticalized into future, deontic modality, \isi{verbal noun} and \isi{infinitive}. In the case of the former, as a resultative \isi{participle} it initially foregrounded the result of an action, while the implicature of the underlying process was later conventionalized. The latter form originally simply conveyed verbal meaning, there is no such unique implicature associated. Thus, the construction shows different developments depending on its function: it is interpreted as an intended action if used predicatively (Eastern IA languages), or as a \isi{verbal noun} if used non-predicatively. Here again, \ili{Marathi} is an exception: it is the only regional language not having (had) the non-\isi{predicative} use, and the only Western IA language to have maintained the \isi{predicative} \isi{modal} meaning, which calls for an explanation. Emphasizing its conservative character by pointing to the preservation of the old \isi{infinitive} and of the three-gender system does not make much sense, in view of the neighbouring \ili{Gujarati} also maintaining three genders. Social factors may provide an explanation, given the historical, self-claimed cultural \ili{Marathi} specificity. The texts forming the active basis of this cultural heritage should be analysed with this in mind, in the same way as the bulk of literature decisive for the formation of \ili{Hindi}/\ili{Urdu}, another exception since they just lost the -\textit{v}{}- form.
 
 

{\sloppy
\printbibliography[heading=subbibliography,notkeyword=this]
}
\end{document}