\documentclass[output=paper,colorlinks,citecolor=brown,
% hidelinks,
% showindex
]{langscibook}

\author{Adolfo Ausín\affiliation{Michigan State University} and Francisco J. Fernández-Rubiera\affiliation{University of Central Florida}}
\title{Towards a uniform account of accusative and dative clitic doubling in Spanish}
\abstract{


}

\IfFileExists{../localcommands.tex}{%hack to check whether this is being compiled as part of a collection or standalone
   \input{../localpackages}
   %Copy this to localcommands.tex

\usepackage[english]{babel}
\usepackage{amsmath}
\usepackage{amssymb,amsfonts,textcomp}
\usepackage{array}
\usepackage{hhline}
\usepackage{hyperref}

\newenvironment{styleStandard}{}{}
\newenvironment{stylelsAbstract}{}{}
\newenvironment{stylelsSectioni}{}{}
\newenvironment{stylelsSectionii}{}{}
\newenvironment{stylelsBulletList}{}{}
\newenvironment{styleBibliographyi}{}{}
\newenvironment{listWWNumxxvleveli}{}{}
\newenvironment{listWWNumxxvlevelii}{}{}
\newenvironment{listWWNumxxvleveliii}{}{}
\newenvironment{listWWNumxxvleveliv}{}{}
\newenvironment{listWWNumixleveli}{}{}
\newenvironment{listWWNumixlevelii}{}{}
\newenvironment{listWWNumixleveliii}{}{}
\newenvironment{listWWNumixleveliv}{}{}

\newcommand\textstyleListLabelxvi[1]{#1}
\newcommand\labellistWWNumxxvleveli{\thelistWWNumxxvleveli.}
\newcommand\labellistWWNumxxvlevelii{\thelistWWNumxxvlevelii.}
\newcommand\labellistWWNumxxvleveliii{\thelistWWNumxxvleveliii.}
\newcommand\labellistWWNumxxvleveliv{\thelistWWNumxxvleveliv.}
\newcommand\labellistWWNumixleveli{[F0B7?]}
\newcommand\labellistWWNumixlevelii{\textstyleListLabelxvi{o}}
\newcommand\labellistWWNumixleveliii{[F0A7?]}
\newcommand\labellistWWNumixleveliv{[F0B7?]}

\newcounter{listWWNumxxvleveli}
\newcounter{listWWNumxxvlevelii}[listWWNumxxvleveli]
\newcounter{listWWNumxxvleveliii}[listWWNumxxvlevelii]
\newcounter{listWWNumxxvleveliv}[listWWNumxxvleveliii]
\newcounter{itemize} 
   \input{../localhyphenation}
    \bibliography{localbibliography}
    \togglepaper[23]
}{}

\begin{document}
\maketitle

\section{Introduction}

As illustrated in (1), Spanish has both accusative and dative pronominal clitics. 


% Examples


Numerous issues regarding Spanish pronominal clitics have been discussed in the literature. One of the questions that has been at the core of these discussions is whether accusative and dative clitics should be given a uniform analysis. Putting aside the obvious difference that accusative and dative clitics are used to refer to accusative and dative arguments respectively, we can see that accusative and dative pronouns have a similar syntactic behavior (e.g., with respect to clitic placement and clitic climbing). However, there is one aspect that clearly distinguish accusative and dative clitics: clitic doubling (ClD henceforth). The differences with respect to ClD have been used as evidence that accusative and dative clitics should be analyzed in a completely different manner. Our goal in this paper is to argue for a more uniform account of these differences. In short, while we recognize that accusative and dative clitic doubling are different, we will argue that the best way to capture those differences is to assume that dative clitics should be decomposed into an invariant applicative (or prepositional) morpheme plus an optional agreement (or pronominal) morpheme. We will show that pronominal clitic doubling works essentially the same with both accusative and dative clitics. The alleged different behavior of dative clitic doubling will be attributed to the fact that some instances of dative clitics are not the full pronominal clitic, but just the manifestation of the invariant component of the dative clitic, thus making it different from accusative clitic doubling.

The paper is organized as follows. In section 1, we will review some general properties of clitics in Spanish, paying special attention to the differences with respect to clitic doubling, as well as to the proposals that have been made to account for these differences. In section 2, we will review what is frequently described as \textit{le-for-les}: an unexpected lack of agreement between the singular dative clitic and its plural double. These facts are the empirical bases for our proposal to be presented in section 3, namely that dative clitics should be decomposed into an invariant morpheme and a pronominal morpheme. Finally, in section 4, we will show how the lexical decomposition of dative clitics opens the possibility to a new analysis of the alleged differences between accusative and dative ClD. Under our analysis, a more uniform account of dative and clitic doubling will become evident. 


\section{Accusative and dative clitic doubling in Spanish}

The defining characteristic of Spanish pronominal clitics is that they need to attach to a verb, as illustrated in the examples in (3). 


% Examples


In (2) \textit{Marta} appears in the canonical object position: direct and indirect object in (2a) and (2b) respectively. On the other hand, the clitics in (3) necessarily appear next to the sole inflected verb, attached to its left.\footnote{\textrm{ Clitic placement is different for imperatives and non-finite verbal forms, in which case clitics appear attached to the verb’s right.}} If the clitic were to appear in any other position, or something were to intervene between the clitic and the verb, the resulting sentence would be ungrammatical as illustrated in (4) for (3a).


% Examples


Many aspects of clitics have been studied under different grammatical frameworks. Scholars have studied their placement in the sentence and their grammatical status, as well as the variation both within a given language and across related languages – \citet{kayne1975}, \citet{strozer1976}, \citet{rivas1977}, \citet{suñer1988}, Uriagereka (\citeyear{uriagereka1988} and subs.), \citet{raposo2005}, among many others. The aspect we are going to focus on in this paper is the issue of clitic doubling, a well-known characteristic of Spanish clitics. 

Clitic doubling refers to the situation where the clitic and the object in canonical position coappear. The conditions under which clitic doubling is possible in Spanish are complex and subject to many different factors, including dialectal variation. We will be discussing some of those factors as needed, keeping in mind that the main goal of this paper is the proper characterization of the differences and similarities between accusative and dative clitic doubling, ultimately arguing for an analysis that captures both doubling phenomena under a single proposal. 

As a way of introduction, let’s consider the simplest cases. If we somehow combine the sentences in (2) and (3), we obtain the clitic doubling examples in (5). In these examples, the accusative and dative clitic (in (5a) and (5b)) are doubling \textit{Marta}, the object in canonical position. 


% Examples


These examples already showcase interesting contrasts. First, (5a) is subject to dialectal variation, which we signal with the “\%”: while it is grammatical in some dialects (specially in Argentinian Spanish), it is not accepted in most dialects. More generally, accusative clitic doubling (hereafter, AccClD) with non-pronominal objects in canonical position is subject to dialect variation. On the other hand, dative clitic doubling (hereafter, DatClD) as in (5b) is accepted in every dialect, in fact frequently preferred and sometimes required depending on the predicate. 

The situation is more uniform when the argument in canonical object position is a pronoun as in (6). In these cases, clitic doubling is required in every dialect. In (6a) the accusative clitic \textit{los} needs to double the full pronouns \textit{a ellos}, and the same is true for the dative clitic \textit{les} in (6b).


% Examples


While the examples in (6) show no difference between accusative and dative ClD, other contexts do. We will next review these contexts since they are the empirical bases for proposals that claim that accusative and dative ClD (and by extension accusative and dative clitics) are essentially different. 

Let’s start considering what happens when the double in canonical position is a negative quantifier such as \textit{nada} as in (7). As can be easily verified in (7), there is a clear contrast with respect to clitic doubling when the element being doubled by the clitic is a negative quantifier.


% Examples


In (7a) \textit{nada} appears in direct object position and AccClD is impossible in any dialect, including those dialects with more liberal accusative clitic doubling. On the other hand, in (7b), \textit{nada} appears in indirect object position and DatClD is fine in all Spanish dialects.

A similar situation is found when the element in canonical position is a bare plural as in (8). With bare plurals, AccClD is impossible as illustrated in (8a) even in those dialects that allow AccClD with full DPs, but DatClD is possible at least for some authors as illustrated in (8b).


% Examples


As for the status of b), the literature is divided with respect to acceptability of DatClD with bare plurals. Some of them \citep{suñer1988} accept them, while others (\citealt{jaeggli1982} and \citealt{fernandez1989reccion}) reject them. In fact, \citet{suñer1988} presents nuanced assessment of this type of sentences by contrasting the grammaticality of (8b) to the ungrammaticality of (9). According to Suñer, the main difference between (8b) and (9) is that in (9) the bare noun is “unqualified” \citep[395, fn6]{suñer1988}. \citet[43-44]{cuervo2003datives} provides further support for Suñer’s views by claiming that (9) becomes grammatical if the unqualified bare plural \textit{museos} gets qualified as in (10a) or heavier as in (10b).\footnote{\textrm{ We thank one of the reviewers for bringing these facts to our attention.}}


% Examples


Although we believe that there is a contrast between (9) and (10), it is not clear to us that the contrast relies just on the presence of the clitic as the reported judgements seem to indicate. We believe that (9) is better than (10) even without the presence of the clitic. As discussed by \citet{bosque1996que}, among many others, the distribution of Spanish bare plurals in positions other than direct object is subject to intricate restrictions. The following examples from \citet{bosque1996que}, who is building on examples and observations by \citet{laca1997sustantivos}, illustrate some of these restrictions for indirect objects:


% Examples


Bosque uses examples like these to argue that bare plurals in positions other than direct object need to be licensed by focal markings, which can be done with derivative morphemes as in (11e). We believe that these restrictions are the reasoning behind the contrasts between (9) and (10) independently of the presence of the clitic.\footnote{\textrm{ We believe that both (9) and (10) improve if the invariant }\textrm{\textit{le}}\textrm{ is used instead of the plural }\textrm{\textit{les}}\textrm{, which will play a crucial role in our argumentation. However, even with the invariant }\textrm{\textit{le}}\textrm{, (9) is worse than (10) for the reasons discussed in the text. }}

We have seen that some scholars accept DatClD with bare plurals while others reject it. A different take on the acceptability of DatClD with bare plurals is presented by Roca (\citeyear{roca1992}, \citeyear{roca1996}), and Laca (\citeyear{laca1997sustantivos}, \citeyear{laca1999}), who claim that the acceptability of this type of sentences increases when the clitic shows up as an invariant \textit{le}. This is the view that we believe is correct and that we will assume. This will play a crucial role in our argumentations later in the paper. 

Comparative NPs as in (12) seem to behave like bare plurals. AccClD is impossible in (12a): the accusative clitic \textit{los} cannot double the NP \textit{menos estudiantes}. However, (12b) shows that DatClD is possible with the dative clitic \textit{le(s)} doubling \textit{menos estudiantes}.


% Examples


Finally, AccClD and DatClD are also different when the element in canonical position is a full DP as illustrated earlier in (5) and now in (13). \ With full DPs, Dative ClD is possible (and sometimes required) in any dialect, but AccClD with full DPs is only possible in some dialects and with some DPs, such as definite DPs.


% Examples


\tabref{tab:ausin:1} summarizes the differences between AccClD and DatClD.

\begin{figure}
\begin{tabular}{lll}
\lsptoprule
                        & Accusative clitic doubling    & Dative clitic doubling    \\\midrule
With negative QPs       & No                            & Yes                       \\\midrule
With bare plural NPs    & No                            & Yes                       \\\midrule
With comparative NPs    & No                            & Yes                       \\\midrule
With a full DP          & No (but yes in some dialects) & Yes                       \\
\lspbottomrule
\end{tabular}
\caption{Differences between accusative clitic doubling and dative clitic doubling}
\label{tab:ausin:1}
\end{figure}

These differences have been used by some authors to argue that accusative clitics and dative clitics are essentially different.

Under some approaches, the feature composition of dative and accusative clitics is different. For instance, \citet{suñer1988} proposes that accusative clitics have a [+specific] feature, whereas dative clitics do not have such a feature. Thus, in an example like (7a) there would be a clash between the [+specific] feature of the accusative clitic and the non-specific nature of the negative quantifier. No such a clash takes place in (7b), since the dative clitic is assumed not to carry the [+specific] feature, thus accounting for the grammaticality observed.


% Examples


Under some other approaches, such as Roca (\citeyear{roca1992}, \citeyear{roca1996}) or \citet{ormazabal2013} and others, dative and accusative clitics are categorically different. For these scholars, accusative clitics are determiners, whereas dative clitics are agreement morphemes. Under such proposals, accusative clitics (as determiners) show restrictions on what they can coappear with, but dative clitics (as agreement morphemes) show no such restrictions. 

These proposals assume (and try to account for) that accusative and dative ClD are essentially different. The proposal that we would like to develop in this paper is the opposite. Contrary to appearances, we will argue that clitic doubling is the same for accusative and dative complements. If Acc and Dat ClD look different, it is because some of the instances that are described as Dat ClD are not so, an idea already present in \citet[113--115]{roca1996}. 

In order to start our defense of a uniform account of accusative and dative ClD, let us point out some clear similarities between accusative and dative clitics. For instance, earlier when we were discussing the examples in (6), we saw that Acc and Dat ClD is required with strong pronouns. Furthermore, accusative and dative clitics behave similarly when there is no double (or there is a null argument) as in (14), or there is a topicalized full DP (pronominal or not) as illustrated in (15):


% Examples


A proposal that claims that Acc and Dat ClD are essentially different would have a hard time explaining the similarities that we just reviewed. On the other hand, a proposal like the one to be defended in this paper that AccClD and DatClD are essentially the same needs to provide an account of the seeming differences that are found between AccClD and DatClD. In this line, we would like to argue that those differences are more apparent than real. The departure point of our argumentation is our previous proposal that dative clitics should be decomposed into a pronominal or agreement morpheme plus an applicative morpheme \citep{ausin2017}). We will make a detailed presentation of that proposal in Section 3, but before, in Section 2, we will discuss the empirical basis of that proposal: \textit{le-for-les} phenomena.


\section{le-for-les}

In all the clitic doubling examples that we have discussed so far, there is agreement between the clitic and the DP in canonical position. This is a feature that is taken for granted in most of the literature on clitic doubling. However, it is normal to find cases in all dialects of Spanish where there is lack of agreement between the clitic and the plural full DP. The following example, from Rufino José Cuervo, illustrates this phenomenon.


% Examples


In (16), we do not find the expected \textit{les}, which would agree with the plural \textit{las garzas}, but we find the singular \textit{le} instead, hence the name that is normally used to refer to this phenomenon: \textit{le-for-les}. The use of singular \textit{le} instead of the expected plural \textit{les} has been acknowledged, documented, and discussed in a number of places, such as Cuervo (\citeyear{cuervo1955}), \citet[107--120]{casares1918}, \citet{sturgis1927}, \citet{rinni1988}, \citet{demello1992}, Roca (\citeyear{roca1992}, \citeyear{roca1996}), \citet{soriano1999pronombre}, \citet{boeckx2004}, \citet{huerta2005}, \citet{rae2009}, \citet{ausin2017}. However, it seems to us that \textit{le-for-les} has not played a significant role in linguistic theorizing. What we would like to show in this paper is that a proper understanding of \textit{le-for-les} can help us develop a better description of clitic doubling in Spanish.

\footnote{\textrm{ Similar facts are attested in Catalan, as discussed in Pineda (\citeyear{pineda2018}, \citeyear{pineda2019}) who offers examples like the following: }\par \textrm{(i)\ \ Li}\textrm{\textsubscript{[SG]}}\textrm{ \ \ vaig \ \ dir \ als \ \ \ \ clients}\textrm{\textsubscript{[PL]}}\textrm{ \ \ que \ tancaríem. }\par \textrm{\ \ CLDAT go}\textrm{\textsubscript{1SG}}\textrm{ tell to-the customers that would.close}\textrm{\textsubscript{1PL}}\par \textrm{\ \ ‘I told the customers that we would close.’}
\todo[inline]{When adding examples, please rework this to use the example environment and use \textbackslash\textsc{acc} instead of ACC etc for glosses}
}

Since the original works by \citeauthor{cuervo1955} (1907/1955) and \citet{casares1918}, it has been noted that there are clear syntactic restrictions on the distribution of \textit{le-for-les}. In \citet{ausin2017} we noted that there is a clear parallelism between the contexts that do not allow \textit{le-for-les} and the contexts that require the presence of accusative clitic. These contexts are preverbal objects (namely, \textsc{clld} constructions), null objects (namely, \textit{pro}), and stressed pronouns in canonical object positions. We discuss these contexts next.

Let’s start with preverbal objects. As already noted by R. J. Cuervo, \textit{le-for-les} is not allowed when the double DP appears in preverbal position, as illustrated in the examples in (17)-(18). In (17), \textit{le-for-les} is allowed because the doubling full DPs (i.e., \textit{a las garzas}, \textit{a las balas}, \textit{a todos}) appear in postverbal position. However, \textit{le-for-les} is not allowed in (18) because the double DP appears in preverbal position (a \textsc{clld}ed position), which requires the agreeing \textit{les}.\footnote{\textrm{ This pattern was confirmed in Huerta-Flores’ (\citeyear{huerta2005}) corpus study of the phenomenon in Mexican Spanish.}} 


% Examples


Interestingly, the distribution of accusative clitics is subject to the same restriction, since it is required when the DO appears in preverbal (\textsc{clld}ed) position as illustrated in the example in (19). The preverbal DO \textit{las garzas} cooccurs with the accusative clitic, and the sentence is grammatical. In the absence of the accusative clitic, the sentence is rendered ungrammatical. 


% Examples


Similarly, and already discussed by \citet{casares1918}, if there is no doubling, or put differently, if there is a null \textit{pro} in canonical position, both the agreeing dative \textit{les} and the accusative clitic are required as illustrated in (20). With an overt IO in the question, \textit{le-for-les} is not possible and the agreeing \textit{les} needs to be used. The same thing happens in (21): in the answer with the overt DO, the accusative clitic needs to be used.


% Examples


There is some parallelism, admittedly weaker, when the double is a full, strong pronoun as in (22)-(23). In both (22) and (23) the element in canonical position is a full stressed/tonic pronoun (\textit{a ellos}). With the dative, \textit{le }makes the sentence to be worse than with \textit{les}.\footnote{\textrm{ \citet[67]{arechalde1992} confirms this in a corpus study when she claims that “Si la [frase complemento indirecto] es pronominal suele mantenerse }\textrm{\textit{les}}\textrm{” (‘If the double IO is pronominal, }\textrm{\textit{les}}\textrm{ is normally kept.’)}} Similarly, the absence of the accusative clitic makes (23) ungrammatical.\footnote{\textrm{ The parallelism is not complete because absence of the accusative clitic triggers a higher degree of unacceptability than the absence of the agreeing dative clitic. We will put this issue aside and hopefully we will come back to it in future research.}}


% Examples


Finally, it could also be argued that the variability that is found in the choice between \textit{les} and the non-agreeing \textit{le} with definite DPs mimics the variability that is found across dialects of clitic doubling with postverbal direct objects. In DatClD with full DPs in canonical position as in (24), we find a considerable amount of variation between the agreeing \textit{les }and the non-agreeing \textit{le}.\footnote{\textrm{ See footnote 6.}}\textit{ }We would like to argue that this is similar to the variation that we find with AccClD as illustrated in (25). Remember that the possibility of having AccClD in examples like (25) is restricted to some dialects.


% Examples


\tabref{tab:ausin:2} summarizes the parallelism between \textit{les} and the presence of accusative clitic.


\begin{figure}
\begin{tabular}{lll}
\lsptoprule
                        & Agreeing \textit{les} required   & Presence of accusative clitic \\
                        & (i.e.  \textit{le-for-les} not possible) & required               \\\midrule
\textit{pro}           & Yes                            & Yes                       \\\midrule
Left dislocated     & Yes                           & Yes                       \\\midrule
Full pronoun        & Yes?                            & Yes                       \\\midrule
Full DP             & Optional & Dialectal                       \\
\lspbottomrule
\end{tabular}
\caption{Parallelism between impossibility of le-for-les and the presence of accusative clitic}
\label{tab:ausin:2}
\end{figure}


In the next section, we will introduce the formal analysis of \textit{le-for-les} that we presented in Ausín and Fernández-Rubiera (\citeyear{ausin2017}, \citeyear{ausin2021}).


\section[A formal account of le{}-for{}-les: a decompositional account of dative clitics]{A formal account of \textup{le-for-les}: a decompositional account of dative clitics}

\citet{cuervo1955} described \textit{le-for-les} as a brilliant mistake but he did not go further. \citet{casares1918} agreed with Cuervo in the benevolent assessment of this “mistake,” and claimed that many of the supposedly ungrammatical examples felt better than the supposedly grammatical one (i.e., better with \textit{le }than with \textit{les}). However, Casares went a bit further and proposed an analysis. He claimed that in the \textit{le-for-les} examples, \textit{le} “becomes a little arrow (‘\textit{flechita}’) pointing forward to indicate the presence of an indirect object”. At some other point, he described this non-agreeing \textit{le} as an expletive or adverbial particle. What we have done in our previous work (\citeyear{ausin2017}, \citeyear{ausin2021}) is to present a more formal, elaborate version of Casares’ “little arrow”, “expletive particle”.

Although we considered Casares’ account to be essentially right, we identified two problems. First, we felt that the notion of “little arrow” needed a more solid grammatical background. Second, it was not clear to us how the non-agreeing \textit{le} (the “little arrow”) and the agreeing \textit{le(s)} were related to each other\textit{.} The way we decided to tackle the first problem was to bring into the picture the applicative analysis of dative pronouns.\footnote{\textrm{ According to this analysis, the applicative morpheme is a functional head that introduces the indirect object as in (i). See C. Cuervo (\citeyear{cuervo2003datives}, \citeyear{cuervo2010against}) and Pineda (\citeyear{pineda2019}, \citeyear{pineda2020}) for more details and references. }\par \textrm{(i)\ \ [V’ V [ApplP DP-IO [Appl’ [Appl le] DP-DO]]]}} As for the second problem, we proposed that non-agreeing \textit{le }should be a regular dative clitic but stripped down of the agreement features. To be able to do that, we proposed that dative clitics should be lexically decomposed into an agreement morpheme and an applicative morpheme.\footnote{\textrm{ The proposal is similar to the one made for Catalan by Martín (2012), Martín and Boeckx (2013) and \citealt{ordóñez2019}, according to which the dative clitic has a locative component. For instance, the proposal in Martín (2012) is that the dative clitic is composed of an accusative clitic and a deictic morpheme.\todo[inline]{Please add bibliography entries for Martín (2012) and Martín \& Boeckx (2013)}}} When both the agreement morpheme and the applicative morpheme are spelled out, the agreeing \textit{le(s) }occurs. The non-agreeing \textit{le} would be the result of spelling out the applicative morpheme without the agreement morpheme. 

In what follows, we will review our proposal and the evidence that supports it. We want to do that because we believe it provides new descriptive generalizations regarding clitic doubling in Spanish, and we believe that this new way of viewing clitic doubling will provide the relevant framework to reevaluate the alleged differences between Acc and Dat ClD, in turn providing a uniform account of clitic doubling in Spanish.

One of the goals in our previous work (\citeyear{ausin2017}, \citeyear{ausin2021}) was to capture the parallelism between the presence of the agreeing dative clitic and the presence of the accusative clitic that was discussed in the previous section and summarized in \tabref{tab:ausin:2}. We accomplished that by assimilating the agreement element of the dative to the accusative clitic. Assuming Uriagereka’s (\citeyear{uriagereka1995}) “big DP” approach to clitic doubling,\footnote{\textrm{ In our previous work, we assumed an agreement approach to clitics (doubling) constructions à-la-\citet{suñer1988}. We are adopting now Uriagereka’s big DP because we believe it captures better the contrasts discussed in this paper.}} according to which the structure of an example like (26) is (27), we propose that the structure of a ditransitive example as that in (28) should be analyzed as in (29).\footnote{\textrm{ An anonymous reviewer suggested potential connections between our proposal and that in \citet{zdrojewski2014}, which investigates Andean Spanish varieties that exhibit lack of agreement (i.e., not only number, as in our contribution, but also gender) between the accusative clitic and its double. Providing an account of both }\textrm{\textit{le-for-les}}\textrm{ and the data that \citet{zdrojewski2014} analyze falls beyond our scope. However, consider the following examples that at first sight, may show some parallelism in terms of “lack of agreement”:}\par 


% Examples


\textrm{rfl}\textrm{\textsubscript{CL }}\textrm{CL}\textrm{\textsubscript{ACC[MASC] }}\textrm{took}\textrm{\textsubscript{3SG}}\textrm{ a \ \ \ \ box \ \ \ sweets}\par \textrm{‘S/he took a box/candy (with him/her)’ \ }\par \textrm{\ \ \ \ [\citet{lujan1987} }\textrm{\textit{apud}}\textrm{ \citet{zdrojewski2014}]}\par \textrm{We could say that }\textrm{\textit{lo}}\textrm{ in these examples is just the invariant accusative counterpart of the invariant applicative morpheme }\textrm{\textit{le}}\textrm{. Alternatively, }\textrm{\textit{lo }}\textrm{(i) is not a licensor for the presence of the direct object, but rather a morphological “accusative” marker of sorts. In our analysis, }\textrm{\textit{le }}\textrm{instantiates an Applicative head, which licenses the presence of the indirect object. We leave this open for future research.}}


Examples

% Diagram 1
%% Original:
%\input{figures/ausin_diagram1.tex}
\begin{forest}  
delay={where n children=0{if={instr("P",content("!u"))}{roof}{}}{}},
[VP
  [V$^0$ [la vimos, name=end]]
  [DP
    [Double [(a ella)]]
    [D'
      [D$^0$ [la, name=startD]]
      [DP [pro]]]]]
\draw[-Latex] (startD) to[out=south west,in=south] (end);
\end{forest}

% Examples

% Diagram 2
%%% Original:
% \input{figures/ausin_diagram2.tex}
\begin{forest}  
delay={where n children=0{if={instr("P",content("!u"))}{roof}{}}{}},
[VP
  [V$^0$ [les entregamos, name=end]]
  [ApplP
    [DP-Goal
      [Double [(a ellas)]]
      [D'
        [D$^0$ [(la)s, name=startD]]
        [DP [pro]]]]
    [Appl'
      [Appl$^0$ [le, name=startApp]]
      [DP-Theme
        [Det [el]]
        [NP [libro]]]]]]]
\draw[-Latex] (startD) to[out=south west,in=south] (end);
\draw[-Latex] (startApp) to[out=south,in=south,looseness=2] (end);
\end{forest}

Under this account, the agreeing dative \textit{les} arises as a result of combining the determiner \textit{las} and the applicative \textit{le}, surfacing as \textit{les}, as shown in the derivation above. In turn, when there is no agreement between the dative clitic and its double, as in the cases of \textit{le-for-les} (cf. (30)), the underlying structure we assume is that in (31). As it is shown, under our proposal, it is the absence of the D° what triggers the lack of agreement, in which case the clitic surfacing is exclusively an applicative morpheme.\footnote{\textrm{ An anonymous reviewer is concerned that this account may overgenerate and fail to prevent “}\textrm{\textit{les-for-le}}\textrm{”; that is, an instance in which plural agreement surfaces in the dative clitic and not in the doubled dative DP. However, since we are arguing for an analysis in which the presence/absence of agreement in the dative clitic is dependent on the presence/absence of Dº in the doubled dative DP, a singular DP (e.g., }\textrm{\textit{al niño – }}\textrm{“to the kid”) would fail to generate a Dº [+plural] that may combine with Appº and produce “}\textrm{\textit{les}}\textrm{” instead of “}\textrm{\textit{le}}\textrm{”. }\par \textrm{In turn, the same reviewer wonders how the narrow syntactic computation knows that Dº+Appº computes as }\textrm{\textit{les}}\textrm{, and not as two independently realized heads. We can consider two potential underlying reasons. First, both the Appº and the Dº mediate the relation between the clitic(s) and the dative argument, thus forming a unique “cluster” of sorts. Second, if Dº instantiates an affix, the combination of Appº + Dº ensures that the affix is not stranded. Cases in which the two clitics remain separate, as in “}\textrm{\textit{se lo dije}}\textrm{” – ‘I said it to him/her’, illustrate a different clitic corresponding to a different internal argument, ensuing the two visible heads we have in these instances, and thus different from our Appº + Dº, which corresponds to one (and only one) internal argument, namely the dative. }}


% Examples

% Diagram 3
%% Original:
%\input{figures/ausin_diagram3.tex}
\begin{forest}  
delay={where n children=0{if={instr("P",content("!u"))}{roof}{}}{}},
[VP
  [V$^0$ [le entregamos, name=end]]
  [ApplP
    [DP-Goal
      [D [a las]]
      [NP [estudiantes]]]
    [Appl'
      [Appl$^0$ [le, name=startApp]]
      [DP-Theme
        [D$^0$ [el]]
        [NP [libro]]]]]]]
\draw[-Latex] (startApp) to[out=south,in=south,looseness=1.5] (end);
\end{forest}

By assuming that the same element underlies accusative clitics and agreeing dative clitics (i.e., a Dº related to each of the verbal arguments), we automatically capture the parallel distribution of these two elements that was summarized in \tabref{tab:ausin:2}. Whatever may explain the distribution of accusative clitics (see \citealt{leonetti2008} for a sound proposal and for alternatives), will also account for the distribution of dative agreeing clitics. 

One could argue that this proposal incorrectly predicts a closer parallelism between accusative and dative clitics. That is, if we are claiming that dative clitics have a determiner like the one that is found in accusative clitics, then we incorrectly predict that DatClD and AccClD have the same distribution. That this is not the case is clearly shown by the examples discussed in Section 1. However, note that under our account, it is not true that all instances of dative clitics contain a determiner. Agreeing dative clitics do have a determiner. Non-agreeing dative clitics do not. What we are going to do in the next section is to show that the cases that are frequently used to show that DatClD and AccClD are different, are precisely those with invariant \textit{le}, which under our proposal means that the structure for the dative does not have a determiner, and therefore, should not be part of the data used to evaluate the similarities between accusative and dative clitics. 


\section{Re-examining the differences between accusative and dative clitic doubling}

With our proposal regarding dative clitics in mind, let’s reexamine the evidence that is typically used to claim that accusative and dative clitic doubling are different. Starting with examples involving negative quantifiers, as in (32), AccClD is ungrammatical with a negative quantifier like \textit{nada}, but DatClD is fine in the same environment.


% Examples


Remember that the contrast between these two examples was supposed to show that clitic doubling is essentially different for datives and for accusatives. However, if the \textit{le} in (32b) is just an applicative morpheme, not a pronominal clitic (in our terms, the structure does not contain a D°), then a new descriptive generalization is possible. We can say that accusative and dative pronominal clitics are incompatible with negative quantifiers like \textit{nada}, probably due to the incompatibility between the head Dº and \textit{nada} \citep{leonetti2008}. (32b) is not a counterexample as there is no real pronominal clitic (i.e., D°), but just an applicative morpheme as shown in (31). \ In other words, (32a) is ungrammatical because the only possible source for the clitic \textit{lo} is the big DP with a Dº head, which is incompatible with the negative quantifier \textit{nada}. On the other hand, it is possible to generate a syntactic structure for (32b) which does not have the big DP, but just the applicative morpheme.\footnote{\textrm{ An anonymous reviewer indicates that the doubling in (32b) is optional, as the dative clitic }\textrm{\textit{le }}\textrm{may surface or not. Whether a dative clitic doubles a postverbal IO is subject to several syntactic (as well as dialectal) restrictions. For instance, verbs of transfer are more likely to tolerate lack of doubling whereas stressed full personal pronouns need to be doubled. See RAE \& AALE (2009) for a general description of the facts, and \citet{pineda2020} for a recent proposal that attempts to account for the optionality.}} 

\ \ Our account of the contrast in (32) can be easily extended to other contexts. The examples will be different, but the general strategy will be the same: the cases where DatClD and AccClD seem to be different are those cases where the applicative morpheme is the only option. The contrast in (32) is somehow obscured by the fact that negative quantifiers like \textit{nada} show default singular agreement, and lack of agreement is not obvious. Next, we will discuss examples with bare plurals where lack of agreement will be more transparent. Consider the examples in (8), repeated here as (33), which exhibit bare plural objects. 


% Examples


The view presented by \citet{suñer1988} and others after her is that ClD is possible with bare plural datives but not possible with bare plural accusatives. However, there is an interesting controversy regarding the acceptability of examples like (33b): Whereas \citet{suñer1988} judges examples like this one grammatical, others judge them ungrammatical. Roca (\citeyear{roca1992}, \citeyear{roca1996}), building on work by \citet{jaeggli1982} and \citet{fernandez1989reccion}, as well as \citet{laca1999}, note that bare plurals are frequently deemed ungrammatical in DatClD constructions. In these contexts, the presence of the doubling clitic is ruled out:


% Examples


Interestingly, what \citet{roca1992} reports is that for some speakers, dative bare plurals are preferred either without any doubling, or crucially for the purposes of our discussion, with the ‘defective’ \textit{le}. 


% Examples


\citet[909]{laca1999} also provides similar data and reports that ‘defective’ \textit{le} is the preferred option with bare plural datives for those who have it.
\footnote{\textrm{ \citet{kany1945} reports \ref{ex:ausin:defectivele-i}, a similar example, and examples like \ref{ex:ausin:defectivele-ii} are easily found in Spanish. See also \citet{laca1997sustantivos} for illuminating discussion and examples. }
% \par \textrm{(i)\ \ Cuéntenle a \ carneros esta historia.}\par \textrm{\ \ tell.LE \ \ \ \ \ to rams \ \ \ \ \ \ that story}\par \textrm{\ \ ‘Go tell that story to the rams.’}\par \textrm{(ii)\ \ No \ le \ \ abras \ \ \ \ la \ \ puerta a desconocidos.}\par \textrm{\ \ Not LE open}\textrm{\textsubscript{2SG}}\textrm{ the door \ \ to strangers}\par \textrm{\ \ ‘Don’t open the door to strangers.’}}
\ea \label{ex:ausin:defectivele}
\ea \label{ex:ausin:defectivele-i}
\gll Cuéntenle a carneros esta historia.\\
 tell.\textsc{le} to rams that story.\\
\glt `Go tell that story to the rams.'
\ex \label{ex:ausin:defectivele-ii}
\gll No le abras la puerta a desconocidos. \\
not \textsc{le} open\textsubscript{2SG} the door to strangers \\
\glt `Don’t open the door to strangers.' 
\z
\z}


% Examples


What we may conclude from the literature is that, at least for some speakers, bare plurals are possible with the non-agreeing \textit{le}, but clearly impossible with the agreeing \textit{les }across the board. We believe that this state of affairs can be easily accounted for under our current proposal. According to \citet{leonetti2008}, all the semantic restrictions on AccClD follow from the definiteness of the accusative clitic: the D° in the big DP structure. Since we are proposing that this D° is also part of the agreeing dative clitics, the incompatibility of agreeing (i.e., plural) dative clitics and bare plurals follow. No incompatibility arises with the defective \textit{le} since it is just an applicative morpheme. \ 



\ \ Thus, the ungrammaticality of (34a) is due to the presence of the [+definite] D° that make up the agreeing dative clitic \textit{les} as illustrated in (37). The [+definite] D° of the big DP is incompatible with the [-definite] nature of the bare DP \textit{a personas necesitadas}. 

% Diagram 4
%%% Original:
% \input{figures/ausin_diagram4.tex}
\begin{forest}  
where level=0{s sep=7em}{},
delay={where n children=0{if={instr("P",content("!u"))}{roof}{}}{}},
[VP
  [V$^0$ [les daré, name=end]]
  [ApplP
    [DP-Goal
      [Double-DP [a personas\\necesitadas\\{[-definite]}, align=center, base=bottom]]
      [D'
        [D$^0$ [(la)s\\{[+definite]}, align=center, base=bottom, name=startD]]
        [DP [pro]]]]
    [Appl'
      [Appl$^0$ [le, name=startApp]]
      [DP-Theme [todo mi dinero]]]]]
\draw[-Latex] (startD) to[out=south west,in=south] (end);
\draw[-Latex] (startApp) to[out=south,in=south,looseness=2] (end);
\end{forest}

Our proposal captures and predicts that no such incompatibility arises if the non-agreeing \textit{le} appears, which for us is just an applicative morpheme. Consider the example in (38), and its structure in (39), which contrast with (34a)/(37). Since no Dº appears in the structure, no incompatibility with the bare plural (and thus [-definite]) arises.

% Diagram 5
%%% Original:
% \input{figures/ausin_diagram5.tex}
\begin{forest}  
where level=0{s sep=7em}{},
delay={where n children=0{if={instr("P",content("!u"))}{roof}{}}{}},
[VP
  [V$^0$ [les daré, name=end]]
  [ApplP
    [DP-Goal [a personas\\necesitadas, align=center, base=bottom]]
    [Appl'
      [Appl$^0$ [le, name=startApp]]
      [DP-Theme [todo mi dinero]]]]]
\draw[-Latex] (startApp) to[out=south west,in=south] (end);
\end{forest}

The same analysis can be extended to account for other type of indefinite expressions such as comparative NPs. For instance, in the same way that AccClD is not possible in (40a), we believe that the option with the agreeing \textit{les }is worse than the one with the non-agreeing \textit{le} in (40b). 


% Examples


In (40a), the comparative NP \textit{menos estudiantes} is incompatible with the [+definite] D° head of the big DP. The same incompatibility shows up with the agreeing \textit{les}, which under our analysis contains the definite D° head of the big DP. No such incompatibility arises in the case of the non-agreeing \textit{le}: as there is no big DP, it is just an applicative morpheme.

\ \ Admittedly, one of the weak aspects of our analysis is that the parallelism between the presence of the clitic in (40a) and the presence of the agreeing \textit{les} in (40b) is far from perfect. That is, the presence of the accusative clitic in (40a) feels a lot worse than the presence of the agreeing \textit{les} in (40b). This is also related to our analysis inability to provide an account for the judgements reported in (8b) and (10) (which showcased agreeing \textit{les }with bare plurals). Although we do not have a clear answer right now, we see two possible ways to account for these facts. One possibility would be that indirect object bare plurals are more likely to be assigned a definite interpretation than direct object bare plurals, and therefore more likely to be associated with a definite D. Another possibility would be that in all the cases where there is optional agreement, there is some type of low level, post syntactic agreement between the double and the applicative morpheme. In fact, next we will discuss some evidence that seems to indicate that this low level syntactic agreement might be needed anyway. 

% \citet{suñer1988} notes that there is a clear contrast between AccClD and DatClD with interrogative pronouns. DatClD is fine (in (41a)) but AccClD is not (in (41b)). If agreeing \textit{les }were to have the same distribution of accusative clitics we would expect no contrast in (41).\footnote{\textrm{ It is not the case that agreeing }\textrm{\textit{les}}\textrm{ is required with wh-datives as illustrated in (i) from \citet{kany1945}. Examples like (i) do behave as expected in our proposal, as opposed to (41a).}\par \textrm{(i)\ \ A \ cuántas \ \ \ \ \ muchachas le \ habrá dicho Usted lo \ \ \ mismo?}\par \textrm{\ \ to how-many girls \ \ \ \ \ \ \ \ \ \ \ \ LE have said \ \ \ \ \ you \ \ \ \ the same}\par \textrm{\ \ ‘How many girls would you have said the same to?’}}

\citet{suñer1988} notes that there is a clear contrast between AccClD and DatClD with interrogative pronouns. DatClD is fine (in (41a)) but AccClD is not (in (41b)). If agreeing \textit{les} were to have the same distribution of accusative clitics we would expect no contrast in (41).
\footnote{\textrm{ It is not the case that agreeing }\textrm{\textit{les}}\textrm{ is required with wh-datives as illustrated in \ref{ex:ausin:agreeingles} from \citet{kany1945}. Examples like \ref{ex:ausin:agreeingles} do behave as expected in our proposal, as opposed to (41a).}
% \par \textrm{(i)\ \ A \ cuántas \ \ \ \ \ muchachas le \ habrá dicho Usted lo \ \ \ mismo?}\par \textrm{\ \ to how-many girls \ \ \ \ \ \ \ \ \ \ \ \ LE have said \ \ \ \ \ you \ \ \ \ the same}\par \textrm{\ \ ‘How many girls would you have said the same to?’}}
\ea \label{ex:ausin:agreeingles}
\gll A cuántas muchachas le habrá dicho Usted lo mismo?\\
 to how-many girls \textsc{le} have said you the same\\
\glt `How many girls would you have said the same to?'
\z}


% Examples


In the examples above, our analysis predicts that both sentences should be similar, but they are not. Plural agreement surfaces with DatClD, but AccClD is ruled out. In the previous paragraph we hinted at the possibility that some cases of agreeing \textit{les} were due to some low-level syntactic agreement. We believe that this solution is particularly well suited for (41a) since wh-movement has placed the indirect object next to the clitic. That there is a connection between wh-movement and the agreeing \textit{les} is shown by the fact that, when wh-movement does not take place, as with \textit{wh-in-situ}, the option without agreement seems the most natural, as expected under our analysis:


% Examples


\section{Conclusion}

We started our paper reviewing the evidence that is frequently used to support the view that dative and accusative clitics are essentially different. According to this view, the Spanish clitic field should be divided between accusative clitics and dative clitics. Under the current proposal, there is also a dividing line but in a different place. We have proposed that the main division should be between non-agreeing dative \textit{le} (an applicative morpheme) on one side and on the other agreeing dative \textit{le(s)} and accusative clitics. The latter are characterized by the big DP component that makes them incompatible with indefinite objects (negative quantifiers, bare plurals, comparative NPs). With the analysis we are advocating for, not only the parallelisms between accusative and dative clitic doubling can receive a principled and elegant explanation, but also accounts for data that had failed to receive a principled account in the literature and are key to understand the intricate processes that regulate clitic doubling in Spanish: \textit{le-for-les}.

\section*{Acknowledgements}

\printbibliography[heading=subbibliography,notkeyword=this]

\end{document}
