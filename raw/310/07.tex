\documentclass[output=paper,colorlinks,citecolor=brown,
% hidelinks,
% showindex
]{langscibook}
\author{Gabriela Matos\affiliation{ULisboa -- Universidade de Lisboa} and Patrícia Rodrigues\affiliation{UFPR -- Universidade Federal do Paraná}}
\title{Que-clauses in Portuguese}
\abstract{In this paper we discuss the properties of Portuguese que-clauses that modify a speech act, as in \textit{Vamos comer, que eu estou com fome} ‘Let’s eat, because I’m hungry’. Generally, they are characterized as explicative causal clauses, because of their frequent justification value. However, other readings can be associated to these clauses. We propose that these different readings are derived from the context instead of from the meaning of the connector que, which would have a semantic bleached value. Extending previous work on explicative causal clauses in European Portuguese to these different instances of que-clauses in the speech act domain, we will show that, despite their different readings, they are involved in paratactic constructions and may be characterized as parenthetical clauses.}

\IfFileExists{../localcommands.tex}{%hack to check whether this is being compiled as part of a collection or standalone
   \input{../localpackages}
   %Copy this to localcommands.tex

\usepackage[english]{babel}
\usepackage{amsmath}
\usepackage{amssymb,amsfonts,textcomp}
\usepackage{array}
\usepackage{hhline}
\usepackage{hyperref}

\newenvironment{styleStandard}{}{}
\newenvironment{stylelsAbstract}{}{}
\newenvironment{stylelsSectioni}{}{}
\newenvironment{stylelsSectionii}{}{}
\newenvironment{stylelsBulletList}{}{}
\newenvironment{styleBibliographyi}{}{}
\newenvironment{listWWNumxxvleveli}{}{}
\newenvironment{listWWNumxxvlevelii}{}{}
\newenvironment{listWWNumxxvleveliii}{}{}
\newenvironment{listWWNumxxvleveliv}{}{}
\newenvironment{listWWNumixleveli}{}{}
\newenvironment{listWWNumixlevelii}{}{}
\newenvironment{listWWNumixleveliii}{}{}
\newenvironment{listWWNumixleveliv}{}{}

\newcommand\textstyleListLabelxvi[1]{#1}
\newcommand\labellistWWNumxxvleveli{\thelistWWNumxxvleveli.}
\newcommand\labellistWWNumxxvlevelii{\thelistWWNumxxvlevelii.}
\newcommand\labellistWWNumxxvleveliii{\thelistWWNumxxvleveliii.}
\newcommand\labellistWWNumxxvleveliv{\thelistWWNumxxvleveliv.}
\newcommand\labellistWWNumixleveli{[F0B7?]}
\newcommand\labellistWWNumixlevelii{\textstyleListLabelxvi{o}}
\newcommand\labellistWWNumixleveliii{[F0A7?]}
\newcommand\labellistWWNumixleveliv{[F0B7?]}

\newcounter{listWWNumxxvleveli}
\newcounter{listWWNumxxvlevelii}[listWWNumxxvleveli]
\newcounter{listWWNumxxvleveliii}[listWWNumxxvlevelii]
\newcounter{listWWNumxxvleveliv}[listWWNumxxvleveliii]
\newcounter{itemize} 
   \input{../localhyphenation}
    \bibliography{localbibliography}
    \togglepaper[23]
}{}

\begin{document}
\maketitle

\section{Introduction}
Clauses initiated by the conjunctional form \textit{que} in Romance have been classified, according to their degree of grammatical integration, as cases of \textit{hypotaxis}, commonly identified with subordination \pef{ex:matos:fome}, \textit{parataxis}, often associated with coordination, but also with parentheses \pef{ex:matos:ovos} (e.g. \citealt{quirk_comprehensive_1985}; \citealt{de_vries_invisible_2007}; \citealt{colaco_explicative_2016}) and \textit{insubordination} \pef{ex:matos:vien}, in the sense of \citet{evans_insubordination_2007}, i.e. “the conventionalized main clause use of what, on prima facie grounds, appear to be formally subordinate clauses” (p. 367).\footnote{For a more detailed analysis of these notions and their relation with non-argumental que-clauses, see \sectref{sec:matos:sect4} of this paper.}

\ea \label{ex:matos:fome}
\gll Ele disse que estava com fome.\\
     he said that was with hunger\\
\glt `He said that he was hungry.'
\z

\ea \label{ex:matos:ovos}
 \ea \label{ex:matos:ovos-fazer}
 \gll Compre os ovos, que eu quero fazer o bolo de banana.\\
      buy.\textsc{imp} the eggs, that I want make.\textsc{inf} the cake of banana\\
 \glt ‘Buy the eggs, because I want to make the banana cake.’ 
 \ex \label{ex:matos:ovos-faco}
 \gll Compre os ovos, que eu faço o bolo de banana.\\
      buy.\textsc{imp} the eggs, that I make.\textsc{pres} the cake of banana\\
 \glt `Buy the eggs, then I will make the banana cake.'
 \ex \label{ex:matos:ovos-manteiga}
 \gll Compre os ovos, que eu compro a manteiga.\\
      buy.\textsc{imp} the eggs, that I buy.\textsc{pres} the butter\\
 \glt ‘Buy the eggs, (and/while) I buy the butter.’ 
 \z
\z
\il{Portuguese}

\ea \label{ex:matos:vien}
\gll Que vien-e. \\
     \textsc{comp} come-\textsc{pres}.\textsc{ind}.\textsc{3sg}\\\jambox{(Spanish)}
\glt ‘(S)he is coming.’  \citep[507]{gras_interactional_2015}
\z
\il{Spanish}

In this paper we will focus on the sentences in \pef{ex:matos:ovos}, to which we will refer, as a whole, as (non-argumental) \textit{que}-clauses.
The \textit{que}-clause in \pef{ex:matos:ovos-fazer} is generally characterized as an explicative causal clause. Explicative causal clauses are traditionally said to express a \textit{de dicto} cause, be it (i) a justification for the utterance of the main clause, as in \pef{ex:matos:ovos-fazer}, or (ii) the reason why the speaker believes the main clause to be true, as in (4). Thus, the \textit{que}-clause in \pef{ex:matos:ovos-fazer} modifies a speech act, while the \textit{que}-clause in \pef{ex:matos:maria-saiu} modifies a proposition.\footnote{In Sweetser's \citeyearpar{sweetser_etymology_1990} terms, these are causal relations pertaining, respectively, to the speech act domain and to the epistemic domain. Other authors also acknowledge this kind of distinction, as \citet{peres_notes_2005} and \citet{frey_correlations_2016}.}

\ea \label{ex:matos:maria-saiu}
\gll A Maria saiu, que/porque a luz ‘tá apagada.\\
     the Maria left, that/because the light is off\\
\glt ‘Maria left, because the lights are off.’ 
\z

\textit{Que}-clauses as in \pef{ex:matos:ovos-faco} and \pef{ex:matos:ovos-manteiga}, despite the formal similarities with \pef{ex:matos:ovos-fazer}, – in all of these sentences the \textit{que}-clause is related to an injunctive matrix clause (i.e. they modify a speech act) – convey different meanings, as it is shown by the English translations – a resulting interpretation in \pef{ex:matos:ovos-faco} and a reading of temporal simultaneity in \pef{ex:matos:ovos-manteiga}.

The literature on explicative causal clauses in Portuguese is vast. However, the study of non-argumental \textit{que}-clauses like the ones in \pef{ex:matos:ovos-faco} and \pef{ex:matos:ovos-manteiga} in Portuguese has been most neglected. Usually the analyses of explicative causal clauses in Portuguese do not distinguish between explicative clauses that modify a speech act and explicative clauses that modify a proposition. Thus, as \textit{que}-clauses in the epistemic domain have consistently an explicative meaning, \textit{que}-clauses in the speech act domain have also been considered as explicative, and the different meanings we identified in Portuguese \textit{que}-clauses in \pef{ex:matos:ovos-faco}-\pef{ex:matos:ovos-manteiga} are not normally acknowledged. Hence the main aim of this paper is to discuss the properties of Portuguese \textit{que}-clauses in (2)\footnote{Although we analyzed data from European and Brazilian Portuguese, we did not find relevant differences regarding the syntax, meaning and behavior of connective \textit{que}-clauses.}, comparing the cases in \pef{ex:matos:ovos-fazer} with those in \pef{ex:matos:ovos-faco} and \pef{ex:matos:ovos-manteiga}, and to show that, despite their different readings, they have a similar syntactic behavior. Therefore, we will restrain our analysis to \textit{que}-clauses in the speech act domain, since it is in this domain that the non-explicative \textit{que}-clauses are found.

Studies on explicative causal clauses typically analyze sentences beginning with \textit{que} ‘that’, \textit{porque} ‘because’, and \textit{pois} ‘for/because’ as a whole (\citealt{cunha_nova_2001,bechara_moderna_2019,neves_gramatica_2011,matos_estruturas_2013,matos_coordenacao_2004,matos_alternation_2013,peres_notes_2005,lopes_contributos_2012,colaco_explicative_2016}). An exception to this practice is Corr (\citeyear{corr_ibero-romance_2016,corr_matrix_2018})\footnote{See also \citet{kocher_unselected_2019}, in the sequence of Corr’s \citeyearpar{corr_ibero-romance_2016} work.}, who restricts her analysis to \textit{que}–clauses in Ibero-Romance languages, including European Portuguese, and ranges together \textit{que}-clauses with an explicative import and insubordinate \textit{que}-clauses occurring in the colloquial register of some of these languages.\footnote{We will detail further this analysis in \sectref{sec:matos:sect3}.}

Previous approaches to explicative causal clauses in Portuguese often disagreed on their status. They have characterized them as subordinate sentences (e.g. \citealt{bechara_moderna_2019,neves_gramatica_2011,matos_estruturas_2003,matos_coordenacao_2004})\todo{matos\_estruturas\_2003 missing from bibliography. Did you mean 2013?}, as coordinate sentences \citep{lobo_aspectos_2003}, or both, depending on the linguistic context where they occur \citep{cunha_nova_2001}, as cases of parentheses \citep{colaco_explicative_2016} or supplementation (\citealt{peres_notes_2005,lopes_contributos_2012}), or even, in the case of \textit{que}-explicative, as an instance of insubordination, which contextualize utterance information for benefit of the addressee and involves the sentence in the speech act domain, in the sense of \citet{speas_configurational_2003}, \citet{haegeman_syntacticization_2013} and \citet{haegeman_west_2014} (cf. \citealt{corr_ibero-romance_2016,corr_matrix_2018}). 

In the current paper, in contrast with \citet{colaco_explicative_2016}, we will mainly deal with non-explicative occurrences of connective \textit{que}-clauses in Portuguese, as in \pef{ex:matos:ovos-faco}-\pef{ex:matos:ovos-manteiga}, and we will focus on the cases where the \textit{que}-clause modifies the speech act expressed by the related sentence. However, the analysis developed shows that the examples under study are adequately analyzed as parenthetic clauses; in this sense, the present work may be conceived as an extension of \citet{colaco_explicative_2016}. We will also claim that, in this language, there is no evidence for the insubordinate status of connective \textit{que}-clauses in Portuguese, a fact also noticed in \citet{corr_ibero-romance_2016}, but whose consequences have been apparently disregarded.

The paper has the following structure: \sectref{sec:matos:sect2} presents a comparison between the \textit{que}-clauses in \pef{ex:matos:ovos} and discusses their different meanings; \sectref{sec:matos:sect3} presents some previous analyses proposed in the literature; \sectref{sec:matos:sect4} discusses the paratactic status of \textit{que}-clauses and \sectref{sec:matos:sect5} develops an analysis of \textit{que}-clauses in Portuguese as parentheses. \sectref{sec:matos:sect6} presents our final remarks.

\section{\textit{Que}-clauses and meaning variation} \label{sec:matos:sect2}

The \textit{que}-clause in \pef{ex:matos:ovos-fazer} can alternate with a \textit{porque}-clause without a change in the explicative meaning \pef{ex:matos:ovos2-porque}. For this reason, explicative causal clauses are usually compared to proper causal clauses \pef{ex:matos:maria-pedro}, which are often introduced by \textit{porque} ‘because’ and express a \textit{de re} cause, the direct reason for the situation described in the main clause. However, in proper causal clauses, the alternation \textit{porque} ‘because’/ \textit{que} ‘that’ is unacceptable.

\ea \label{ex:matos:ovos2}
 \ea \label{ex:matos:ovos2-porque}
 \gll Compre os ovos, que/porque eu quero fazer o bolo de banana.\\
      buy.\textsc{imp} the eggs, that/because I want make.\textsc{inf} the cake of banana\\
 \glt ‘Buy the eggs, because I want to make the banana cake.’
 \ex \label{ex:matos:maria-pedro}
 \gll A Maria foi embora porque/*que o Pedro chegou.\\
	  the Maria went away because/that the Pedro arrived\\
 \glt ‘Maria left because Pedro has arrived.’
\z
\z

The alternation between \textit{que} and \textit{porque} in \pef{ex:matos:ovos2-porque} is available because it is possible to establish a causal/explicative link between the connected sentences, but also because the \textit{que} and \textit{porque} clauses have the same distributional status, as signaled by the comma punctuation.
\footnote{Notice that in Portuguese \textit{porque} may head adverbial integrated clauses, as in \pef{ex:matos:maria-pedro}, and non-integrated clauses as in \pef{ex:matos:ovos2-porque}. In opposition to integrated clauses, non-integrated clauses are outside the scope of the related clause (\citealt{haegeman_parenthetical_1991}; \citeyear{haegeman_adverbial_2012}; \citealt{lobo_aspectos_2003}; \citealt{matos_coordenacao_2004}; \citealt{colaco_explicative_2016}). This is apparent, for instance, in cases of sentence negation of the associate clause: while integrated \textit{porque} clauses can be negated, as in \ref{ex:matos:fn1-i}, this does not happen to non-integrated ones, as shown in \ref{ex:matos:fn1-ii}:

\ea 
 \ea\label{ex:matos:fn1-i}
 \gll Não compre os ovos porque eu quero fazer o bolo de banana. (Compre-os por outro motivo.)\\
      	 not buy.\textsc{imp} the eggs because I want make.\textsc{inf} the cake of banana (buy.\textsc{imp}-them for other reason)\\
 \glt ‘Don’t buy the eggs because I want to make the banana cake. (Buy them for another reason.)’
 \ex\label{ex:matos:fn1-ii}
 \gll Não compre os ovos, que/porque eu quero fazer o bolo da banana.\\
      	  not buy.\textsc{imp} the eggs that/because I want make.\textsc{inf} the cake of banana\\
      	\glt ‘Don’t buy the eggs, because I want to make the banana cake.’\\
     	 (Implied interpretation: eggs are not an ingredient of the banana cake.) 
\z
\z} 
In contrast, the impossibility of replacing \textit{porque} by \textit{que} in \pef{ex:matos:maria-pedro} indicates that these two connectives do not have the same meaning, nor the same distribution. While \textit{porque} ‘because’ express a causal/explicative meaning, \textit{que} ‘that’ would be neutral in this respect, the explicative reading being obtained from the context; in addition, the acceptability judgement of the \textit{porque}-clause in \pef{ex:matos:maria-pedro} obtains when the sentence forms a single prosodic unit and is syntactically integrated  in the matrix clause. In this case, it could be the answer, for instance, to the question \textit{Why did Maria leave?}; in opposition, non-argumental \textit{que}-clauses are always prosodically and syntactically non-integrated in the associated sentence.

The neutral semantic status of \textit{que} is confirmed when we observe that, in \pef{ex:matos:ovos-faco}, repeated in \pef{ex:matos:ovos3-que}, which does not convey a causal/explicative relation, \textit{the} \textit{que}-clause’s replacement by a \textit{porque}-clause \pef{ex:matos:ovos3-porque} causes the loss of the original’s meaning in \pef{ex:matos:ovos3-que}:
\footnote{The pragmatic contrast observed in \pef{ex:matos:ovos3} does not seem to result from a TMA effect; in the sentence in \ref{ex:matos:fn2-i}, for instance, the explicative meaning obtains, and the sentence in \ref{ex:matos:fn2-ii} is ambiguous between an explicative and a resulting reading. Further investigation is needed though to confirm this assumption:

\ea 
 \ea\label{ex:matos:fn2-i}
 \gll Compre o bolo, que eu faço aniversário hoje.\\
      Buy.\textsc{}{imp} the cake, that I do.\textsc{pres} birthday today\\
 \glt ‘Buy the cake, because it’s my birthday today.’
 \ex\label{ex:matos:fn2-ii}
 \gll Compre os ovos, que eu vou fazer o bolo de                        banana.\\
      buy.\textsc{imp} the eggs, that I will make.\textsc{inf} the cake of banana\\
 \glt ‘Buy the eggs, because I will make the banana         cake.’\\ 
      ‘Buy the eggs, then I will make the banana cake.’
\z\z}

\ea \label{ex:matos:ovos3}
 \ea \label{ex:matos:ovos3-que}
 \gll Compre os ovos, que eu faço o bolo de banana.\\
	    buy.\textsc{imp} the eggs, that I make.\textsc{pres} the cake of banana\\
 \glt ‘Buy the eggs, then I will make the banana cake.’ 
 \ex[\#]{
 \gll Compre os ovos, porque eu faço o bolo.\\
	  buy.imp the eggs, because I make.\textsc{pres} the cake\\
 \glt ‘Buy the eggs, as I will make the cake.’ 
 }\label{ex:matos:ovos3-porque}
 \z
 \z
 
Thus, we defend that the different readings of sentences in \pef{ex:matos:ovos} come from the context, not from the meaning of the connective \textit{que}, which has a highly bleached semantic value (cf. also \citealt{corr_ibero-romance_2016}). This connective’s main import is to relate the sentence it introduces with the preceding associate clause (\citealt{corr_ibero-romance_2016}; \citealt{etxepare_coordination_2014}; a.o.), allowing them to establish a loose syntactic link that we analyze as paratactic.

In fact, the sentence in \pef{ex:matos:ovos-faco} has more than one reading. It can be, for instance, paraphrased by a construction with a conditional meaning, as illustrated in \pef{ex:matos:ovos4}. In this case, the \textit{que}-clause in \pef{ex:matos:ovos-faco}/\pef{ex:matos:ovos3-que} expresses the result or consequence to be obtained if the action expressed in the connected clause takes place.
 
\ea \label{ex:matos:ovos4}
\gll Se você comprar os ovos, eu faço o bolo.\\ 
	 if you buy.\textsc{inf} the eggs, I make.\textsc{pres} the cake\\
\glt ‘If you buy the eggs, I will make a cake.’
\z 

It may also be the case that, in uttering \pef{ex:matos:ovos-faco}/\pef{ex:matos:ovos3-que}, the speaker is just assigning the tasks s/he and her/his interlocutor must perform in order to have a cake, without establishing the purchase of eggs as a condition to have the cake done. 
 
Therefore, we assume that the \textit{que}-clause in the speech act domain essentially introduces a comment by the speaker on the command expressed in the main clause. The \textit{que}-clause can then assume different readings, depending on the context, including the situation context as well as, for example, the clause’s tense and mood. Hence, while in \pef{ex:matos:ovos-fazer} (a control structure with a volitive verb), the justification value is more salient, in \pef{ex:matos:ovos-faco} (in which the \textit{que}-clause is in the present tense and the accomplishment of the task referred to in it depends on the accomplishment of the task referred to on the related clause), the most salient reading is the result reading. Differently, in \pef{ex:matos:ovos-manteiga}/\pef{ex:matos:ovos5} (also with the present tense in the \textit{que}-clause, but with two independent tasks), the assign-tasks reading is more salient: the \textit{que}-clause may be being used to refer to a task the speaker intends to accomplish at the same time as the task referred to in the injunctive main clause. 

\ea \label{ex:matos:ovos5}
\gll Compre os ovos, que eu compro a manteiga.\\
	buy.\textsc{imp} the eggs, that I buy.\textsc{pres} the butter\\
\glt ‘Buy the eggs, (and/while) I buy the butter.’
\z 

Moreover, \textit{que}-clauses can convey what may also be seen as a result meaning in sentences like \pef{ex:matos:nao-sei}, which involves the speech act domain as well, but whose related clause is not injunctive; \pef{ex:matos:nao-sei} could be used to express, for example, an apology. The \textit{que}-clause expresses the consequence of a potential situation that made someone cry.

\ea \label{ex:matos:nao-sei}
\gll Não sei {o que} aconteceu, que ele está a chorar.\\
	not know.\textsc{pres}.\textsc{1sg} what happened, that he is to cry.\textsc{inf}\\
\glt ‘I don’t know what happened that made him cry.’
\z 
 
Despite these kinds of result readings, \textit{que}-clauses in Portuguese \pef{ex:matos:sereia} differ from Spanish predicative subordinate clauses with a result import \pef{ex:matos:salio}, which have been classified as consecutive clauses with a covert degree/quantifying expression \citep{demonte_predicacion_1999} and related to regular consecutive or comparative-consecutive clauses (\citealt{giusti_frasi_1991}; \citealt{corver_internal_1997}), with overt degree/quantifying elements, as the one in \pef{ex:matos:comeu-tanto}. Unlike these clauses, \textit{que}-clauses in \pef{ex:matos:ovos-faco} and \pef{ex:matos:salio} modify a speech act and are not related with an overt or covert degree/quantifying expression \pef{ex:matos:ovos6}.

\ea \label{ex:matos:salio}
\gll Salió de la bañera que semejaba una sirena. \\
     left from the bathtub that seemed a mermaid\\
\glt ‘She came out of the bathtub looking like a mermaid.’

\citep[2473]{demonte_predicacion_1999}
\z 

\ea \label{ex:matos:sereia}
\gll Ela saiu da banheira que parecia uma sereia. \ding{51}(BP)/??(EP){\footnotemark}\\
     she left from.the bathtub that seemed a mermaid\\
\footnotetext{BP=Brazilian Portuguese and EP=European Portuguese}
\glt ‘She came out of the bathtub looking like a mermaid.’
\z 

\ea \label{ex:matos:comeu-tanto}
\gll Ela comeu tanto/de tal modo que ficou doente.\\
	 she ate {so much}/in such way that became.\textsc{pst}.\textsc{prf}.\textsc{3sg} sick\\ 
\glt ‘She ate so much that she felt sick.’
\z 

\ea \label{ex:matos:ovos6}
 \ea \label{ex:matos:ovos6-talmodo}
 \gll Compre os ovos (*de tal modo / *tanto) que eu faço um bolo.\\
      buy-\textsc{imp} the eggs (in such way / {so much})  that I  make.\textsc{pres} a cake\\
 \ex \label{ex:matos:nao-sei-talponto}
 \gll Não sei {o que} aconteceu (*a tal ponto / *tanto) que ele está a        chorar.\\  
      not know.\textsc{pres}.\textsc{1sg} what happened (to such extent / {so much}) that he is to cry\\ 
 \z
\z 

Summing up, we saw that \textit{que}-clauses in the speech act domain convey different readings, and we assumed that these readings are obtained from the context, because the conjunctional \textit{que} has a neutral semantic import. In the remainder of the paper, we will refer to \textit{que}-clauses semantic values as explicative meaning, temporal simultaneity meaning and result/sequential meaning. 

\section{Previous proposals} \label{sec:matos:sect3}

Previous studies on \textit{que}-clauses in Portuguese, with the exception of \citet{corr_ibero-romance_2016} and \citet{kocher_unselected_2019}, conceive them just as an instance of a subset of explicative causal clauses, which put together clauses starting with \textit{que} ‘that’, \textit{pois} ‘for’ and \textit{porque} ‘because’ (\citealt{cunha_nova_2001,neves_gramatica_2011,bechara_moderna_2019,lobo_aspectos_2003,matos_estruturas_2003,matos_coordenacao_2004,peres_notes_2005,lopes_contributos_2012,colaco_explicative_2016}). These clauses are usually assumed to present a partially different behavior from other clauses with a causal explicative import headed by connectors such as \textit{já que} ‘since’ or \textit{uma vez que} ‘once’, etc (see \citealt{lobo_aspectos_2003,lobo_subordinacao_2013}). The common concerns of these studies were to characterize the specificity of the semantic and pragmatic value of these explicative causal clauses and the type of relation they established with the associate sentence. 

Thus, most work developed in the last two decades on European Portuguese explicative causal clauses based their claims on the syntactic properties internal to the explicative clauses and their formal behavior with respect to the related clause. It has been claimed that explicative clauses should be taken as coordinate sentences in European Portuguese \citep{lobo_subordinacao_2013}, because (i) they allow enclisis of clitic pronouns, proclisis being considered as inherent triggered by complementizers in this language variety \pef{ex:matos:telefonei-novamente}; (ii) they preclude fronting, a fact interpreted as a consequence of the coordinate structure constraint \citep{ross_constraints_1967} \pef{ex:matos:porque-eu-vi}; and (iii) they may present marginal results when coordinated, a property assumed as indicating that the explicative and the coordinate conjunction both occupy the same distributional position \pef{ex:matos:vamos-ao-trabalho}:

\ea \label{ex:matos:telefonei}
 \ea[] {\label{ex:matos:telefonei-novamente}
 \gll Telefonei novamente, pois lembrei-me/*me lembrei de outra       coisa.\\
      called.\textsc{1sg} again, for remembered.\textsc{1sg}-myself/myself remembered.\textsc{1sg} of other thing\\
 \glt ‘I called again because I remembered something else.’\\ \citep[50]{lobo_aspectos_2003}}
 \ex[*]{
 \gll Porque eu vi a luz acesa, o director está cá. \\
      because I saw the light on, the director is here\\ \label{ex:matos:porque-eu-vi}
 \glt \citep[48]{lobo_aspectos_2003}}
 \ex[??/*]{\label{ex:matos:vamos-ao-trabalho}
 \gll Vamos ao trabalho, que ainda há muito a fazer e que o sol já vai alto.\\
      go.\textsc{pres}.\textsc{1pl} to.the work, that still {there is} {a lot} to do and that the sun already goes high\\
 \glt ‘Let’s go work, because there is a lot to do and because the sun is already high.’\\
 \citep[49]{lobo_aspectos_2003}} 
 \z
\z 

As argued in \citet{matos_coordenacao_2004}, these criteria are not entirely undisputable: regarding clitic placement, there are some inconsistency among the speakers on the acceptability of enclisis or proclisis, specially with \textit{que} and \textit{porque}, as shown in \pef{ex:matos:telefonei-novamente} vs. \pef{ex:matos:crianca-magoada}, where proclisis occurs; as for coordination, several native speakers of European Portuguese find \pef{ex:matos:vamos-ao-trabalho} fully acceptable; finally, in what concerns the ban of fronting, it may be imputed to causes other than the coordinate structure constraint, as shown in \pef{ex:matos:trabalho-elaborado}, where the fronting of an appositive specifier (classically not taken as a case of coordination) clause produces an unacceptable result \pef{ex:matos:trabalho-elaborado-bad}\footnote{\citet{fieis_para_2009} claim that explicative causal clauses with \textit{pois} are the most consistently representative of these criteria in contemporary European Portuguese, although not in Medieval Portuguese, where they could be fronted and coordinated. This shows that these explicative causal connectors do not have a uniform behavior and suggests that they may constitute an area of change.}

\ea \label{ex:matos:crianca-magoada}
\gll A criança está magoada, que/porque a ouvimos chorar!\\
     the child is hurt, that/because her hear.\textsc{pst}.\textsc{prf}.\textsc{1pl} cry\\
\glt ‘The child is hurt, because we heard her cry.’ 
\z 

\ea \label{ex:matos:trabalho-elaborado}
 \ea \label{ex:matos:trabalho-elaborado-ok}
 \gll O trabalho estava bem elaborado, coisa que muito nos agradou.\\
      the work was well elaborated, thing that {a lot} us pleased\\
 \glt ‘The work was well elaborated, what pleased us a lot.’
 \ex \label{ex:matos:trabalho-elaborado-bad}
 \gll {*} Coisa que muito nos agradou, o trabalho estava bem elaborado.\\
      {} thing that {a lot} us pleased, the work was well elaborated\\
 \z
\z 

Still, explicative causal exhibits properties characteristic of CPs, which casts doubts on the coordinator status of the connectors that introduce them. Explicative connectors cannot connect (other) CPs \pef{ex:matos:eles-chegaram}, require sentential complements \pef{ex:matos:joao-luz-acesa}, are sensitive to finiteness, being introduced by conjunctional connectors that only select finite TPs \pef{ex:matos:vamos-dormir}, do not allow gapping \pef{ex:matos:estao-em-casa}, and prevent across-the-board movement \pef{ex:matos:que-joao-comprou} (\citealt{matos_coordenacao_2004}; \citealt{colaco_explicative_2016}):

\ea[*]{
\gll Eles chegaram [quando tu entrastes] que/porque/pois [quando eu saí].\\
	 they arrived [when you entered] that/because/for [when I left]\\
} \label{ex:matos:eles-chegaram}
\z 

\ea[*]{
\gll O João está em casa, porque [com a luz acesa]. \\
	 the João is at home, because [with the light on]\\
}\citep[49]{lobo_aspectos_2003} \label{ex:matos:joao-luz-acesa}
\z 

\ea \label{ex:matos:vamos-dormir}
\gll Vamos dormir, pois/porque/que [estamos/*estar] cansados.\\
	 go.\textsc{pres}.\textsc{1pl} sleep.\textsc{inf}, for/because/that [be.\textsc{pres}.\textsc{1pl}/*be.\textsc{inf}] tired\\
\glt ‘Let’s sleep, because we are tired.’
\z 

\ea[*]{
\gll Eles estão em casa, pois/porque/que a luz \_\_ acesa.\\
	 they are at home, for/because/that the light \_\_ on.\\
} \label{ex:matos:estao-em-casa}
\z 

\ea[*]{
\gll Que é que o João comprou \_\_ pois/porque/que está \_\_ na         garagem?\\
	 what is that the João bought \_\_ for/because/that is \_\_ in.the garage\\
} \label{ex:matos:que-joao-comprou}
\z 

Considering this behavior, \citet{matos_coordenacao_2004}, \citet{colaco_explicative_2016} and \citet{matos_alternation_2013} assumed that, to a certain extent, explicative causal clauses in European Portuguese should be characterized as subordinate clauses. 

However, taking into account the loose relation established by these explicative clause with the associate sentence in Portuguese, it has been proposed that they should be interpreted as syntactically independent discourse fragments that do not participate in the predication of the related clause, i.e., according to \citet{huddleston_coordination_2002}, as supplements (cf. \citealt{peres_notes_2005}; \citealt{lopes_contributos_2012}), or as clausal parenthetic constituents, in some way, syntactically connected with the host sentence (\citealt{matos_coordenacao_2004}\footnote{\citet{matos_coordenacao_2004} takes this relation as apposition.}; \citealt{colaco_explicative_2016}).

Grounded on the studies on parentheticals (\citealt{dehe_parentheticals_2007}; \citealt{de_vries_unconventional_2012}; \citealt{kluck_representing_2015}, a.o), \citet{colaco_explicative_2016} show that explicative causal clauses headed by \textit{pois}, \textit{porque} and \textit{que}, despite apparently presenting a subordinate behavior, are related to the host sentence as parenthetical clauses, appended or interpolated in their host sentences, as in \pef{ex:matos:joao-incapaz}. Like parentheses, they have prosodic autonomy (commonly marked in the writing by specific punctuation, often a comma), they have relative semantic autonomy, since they express secondary information with respect to the host sentence, and they behave syntactically as non-integrated sentences, preventing c-command from the host clauses, as illustrated for Principle C effects in \pef{ex:matos:ela-saiu-pressa}.

\ea \label{ex:matos:joao-incapaz}
 \ea \label{ex:matos:joao-incapaz-a}
 \gll O João era incapaz de tal atitude, pois/que/porque eu conheço-o bem.\\
	  the João was incapable of such attitude, for/that/because I know-him well\\
 \glt ‘João was incapable of such an attitude, because I know him well.’
 \ex \label{ex:matos:joao-incapaz-b}
 \gll O João, pois/que/porque eu conheço-o bem, era incapaz de tal atitude.\\
	 the João, for/that/because I know-him well, was incapable of such attitude\\
 \z
 \citep[49]{colaco_explicative_2016}
\z 

\ea \label{ex:matos:ela-saiu-pressa}
\gll Ela$_i$ saiu à pressa, pois a Maria$_i$ nunca deixa as luzes acesas.\\
     she left to.the hurry, for the Maria never leaves the lights on\\
\glt ‘She left in a hurry, because Maria never leaves the lights on.’\\
\citep[49]{colaco_explicative_2016}
\z 

Colaço \& Matos adopt \citet{de_vries_unconventional_2012} proposal to account for the parenthetical status of these explicative clauses. Thus, they consider that the paratactic status of these clauses has to do with their parenthetical nature, not with coordination.

All the preceding studies group together non-argumental \textit{que}-clauses with explicative clauses beginning with \textit{pois} ‘for’and \textit{porque} ‘because’. However, recent work on root \textit{que}-clauses, by Corr (\citeyear{corr_ibero-romance_2016}; \citeyear{corr_matrix_2018}) and Kocher (\citeyear{kocher_unselected_2019}), relate explicative causal clauses headed by \textit{que} to instances of insubordinate \textit{que}-clauses.\footnote{Previous work on insubordinate \textit{que}-clauses in Spanish was developed by Etxepare (\citeyear{etxepare_hearsay_2010}; \citeyear{etxepare_coordination_2014}) and Demonte \& Soriano (\citeyear{demonte_force_2009}; \citeyear{demonte_evidentiality_2014}), among others.}

Corr (\citeyear{corr_ibero-romance_2016}; \citeyear{corr_matrix_2018}), distinguishes three types of root complementizers instantiated by \textit{que} in Ibero Romance-Languages: exclamative-\textit{que} \pef{ex:matos:ai-cansat-a}, quotative-\textit{que} \pef{ex:matos:estais-liados} and conjunctive-\textit{que} \pef{ex:matos:estropeada}.  The examples are from Corr (\citeyear[79--80]{corr_matrix_2018}):

\ea \label{ex:matos:ai-cansat}
\settowidth\jamwidth{(Portuguese)}
 \ea \label{ex:matos:ai-cansat-a}
 \gll Ai que estic cansat! \\	
	  \textsc{prt} \textsc{excl} be.\textsc{1sg} tired\\\jambox{(Catalan)}
 \glt ‘Oh, I’m tired!!!’
 \ex \label{ex:matos:estais-liados}
 \begin{itemize}
     \item[A:] ¿Estáis liados? \\
 '              Are you hooking up?’\jambox{(Spanish)}
     \item[B:] ¿Perdona?\\
                ‘Pardon?’
     \item[A:] \gll Que si te enrollas con mi padre.\\
	                \textsc{quot} if you {hook up}.\textsc{2sg} with my father\\
                \glt ‘I asked if you’re hooking up with my dad.’
 \end{itemize}
  \ex \label{ex:matos:estropeada}
 Context: the addressee is trying to switch on the light.\\
 \gll Que está estropeada.\\
	  \textsc{conj} be.\textsc{3sg} broken\\
 \glt ‘It’s broken.’
 \z
\z 
\il{Catalan}

According to Corr (\citeyear{corr_ibero-romance_2016}; \citeyear{corr_matrix_2018}), it is conjuntive-\textit{que} that occurs in explicative clauses, in cases of sentence connection, as illustrated in \pef{ex:matos:no-llores} (examples from Corr \citeyear[201]{corr_ibero-romance_2016}):

\ea \label{ex:matos:no-llores}
\settowidth\jamwidth{(Portuguese)}
 \ea \label{ex:matos:no-llores-a}
 \gll No llores que yo te perdono. \\
	  not cry.\textsc{sbjv}.\textsc{2sg} \textsc{conj} I you forgive.\textsc{1sg}\\\jambox{(Spanish)}
 \glt ‘Don’t cry, I forgive you.’
 \ex \label{ex:matos:embarcai}
 \gll Embarcai, que haveis de ir à derradeira.{\footnotemark} \\
      embark.\textsc{imp} \textsc{conj} have.\textsc{2pl} of go.\textsc{inf} to.the final\\\jambox{(Portuguese)}
 \glt ‘Embark, for you have to go to the last one.’
 \footnotetext{Notice that this is a 16th century’s example of Portuguese, from Auto da Barca do Inferno, by Gil Vicente, the pioneer of theatre in Portugal. Still, this use of \textit{que} is similar to its use in contemporary Portuguese.}
 \z
\z 

She claims that the form \textit{que} has no intrinsic causative meaning. It is an item with a bleached content that, according to the context, assumes different values, including no specific meaning besides the one of being a connective discourse marker, as it happens when the conjunctive \textit{que}-clause occurs in isolation (cf. \ref{ex:matos:estropeada}).

Despite noticing that conjunctive-\textit{que} clauses in Portuguese cannot be licensed when they occur in isolation, in contrast to what happens in other Ibero-Romance languages, Corr (\citeyear{corr_ibero-romance_2016}; \citeyear{corr_matrix_2018}) assumes that they should be analyzed as root sentences, in terms of the cartographic approach originally proposed by \citet{rizzi_fine_1997}, developed according \citet{ledgeway_latin_2012}, and extended to incorporate the speech act domain, as proposed in \citet{speas_configurational_2003}, re-elaborated by \citet{haegeman_west_2014}, \citet{haegeman_syntacticization_2013}. In this cartographic structure, conjunctive-\textit{que} would occupy the highest position in the root sentence, Speech Act high (SAhigh), as shown in \pef{ex:matos:saphigh}:

\ea \label{ex:matos:saphigh}
[\capsub{SAPhigh} QUE\capsub{CONJ} [\capsub{SAPlow} [QU\capsub{EXCL} [\capsub{Force} QUE\capsub{QUOT/DE\ DICTO} [\capsub{Topic} [\capsub{Pol-int} [\capsub{Excl} [\capsub{Wh-Int} [\capsub{Focus} [\capsub{Fin} [\capsub{IP} \ldots ]]]]]]]]]]\\
\citep[90]{corr_matrix_2018}
\z 

In favor of her proposal, Corr argues that conjunctive-\textit{que} clauses are not embeddable in colloquial registers of some languages, hence are true cases of insubordination \citep[91]{corr_matrix_2018}. Thus, in Corr’s work, the correlation of the \textit{que}-clause with the related sentence is not structurally marked and only relies on a discourse connection. Notice, however, that non-argumental \textit{que}-clauses in Portuguese may occur embedded (\sectref{sec:matos:sect4-3}).

\citet[148]{kocher_unselected_2019} also assumes that \textit{que}-initial declarative clauses can appear juxtaposed to directive sentences, often taking a causal or explicative meaning. She remarks that the presence of \textit{que} does not contribute to the causal meaning of the sentence \pef{ex:matos:keep-calm-a}, and sometimes this meaning may not available, as in the Spanish example in \pef{ex:matos:salid}:

\ea \label{ex:matos:keep-calm}
\settowidth\jamwidth{(Portuguese)}
 \ea \label{ex:matos:keep-calm-a}
 \gll Keep calm. (Que) hoje é sexta-feira. \\
	  keep calm \textsc{que} today is Friday\\\jambox{(Portuguese)}
 \glt ‘Keep calm. Today is Friday.’
 \ex \label{ex:matos:salid}
 \gll Salid. Que no vos mataré \\
	  leave-\textsc{imp}.\textsc{2sg} \textsc{que} not \textsc{cl}.\textsc{2p} kill.\textsc{fut}.\textsc{1sg}\\\jambox{(Spanish)}
 \glt ‘Come out! I won’t kill you.’
 \z
 \citep[148]{kocher_unselected_2019}
\z 

However, \citet{kocher_unselected_2019} considers these juxtaposed clauses as adverbial clause modifiers and she takes them as instances of what she calls attributive-\textit{que} clauses. She claims that in all attributive-\textit{que} clauses, the complementizer is merged in FinP, where it is valued with an attributive feature, but, whenever required by the different constructions, it moves to higher projections \citep[161]{kocher_unselected_2019}. In opposition to Corr (\citeyear{corr_ibero-romance_2016}; \citeyear{corr_matrix_2018}), the author suggests that in the case of non-argumental \textit{que}-clauses, the \textit{que} complementizer moves to SubP (ForceP in \citealt{rizzi_fine_1997}), since there is no evidence that can occur with discourse markers or vocatives instancing the speech act functional projections,  highest positions of the left periphery (\citealt[235]{corr_ibero-romance_2016}: 235; \citealt[74]{kocher_unselected_2019}): 

\ea \label{ex:matos:escuchame}
\gll ¡Escúchame, (*oye/María) que (*oye/María) vamos a llegar tarde!\\
	 listen.\textsc{imp}.\textsc{cl}.\textsc{1s} \textsc{dm}/Maria \textsc{conj} \textsc{dm}/Maria go.\textsc{1p} to arrive late\\
\glt (‘Listen, *hey we’re going to arrive late!’)
\z 

Finally, notice that not all the examples proposed for Spanish find an acceptable correlate in Portuguese. Thus, for instance, \citet[223]{demonte_evidentiality_2014} illustrate that non-argumental \textit{que}-clauses in Spanish can present a purposive meaning \pef{ex:matos:acercate}, which we did not find in Portuguese \pef{ex:matos:aproximate}:

\ea \label{ex:matos:acercate}
\gll Acércate, que te vea mejor.\\
     come.closer that to-you see.\textsc{1sg}.\textsc{subj} better\\ 
\glt ‘Come closer, so that I can see you better.’ (Spontaneous, Madrid)
\z 

\ea[*]{
\gll Aproxima-te, que (eu) te veja melhor.\\
     come.closer that (I) you.\textsc{cl} see.\textsc{1sg}.\textsc{sbjv} better\\
} \label{ex:matos:aproximate}
\z 

In sum, while Corr (\citeyear{corr_ibero-romance_2016}; \citeyear{corr_matrix_2018}) and Kocher (\citeyear{kocher_unselected_2019}) assign a juxtaposed status to connective \textit{que}-clauses, their analyses differ as to the autonomy of the \textit{que}-clause regarding the related linguistic context. In addition, there are relevant differences between other Iberian Romance languages and Portuguese that suggest that a different analysis must be proposed for connective \textit{que}-clauses in this language. 

\section{The syntax of connective \textit{que}-clauses in Portuguese} \label{sec:matos:sect4}

\subsection{General remarks} \label{sec:matos:sect4-1}

Recent work on European Portuguese has claimed that in this language variety \textit{que}-clauses with causal explicative meaning have properties that relate them to hypotaxis and parataxis \citep{colaco_explicative_2016} or are instances of insubordination (\citealt{corr_ibero-romance_2016}; \citeyear{corr_matrix_2018}). In this section, before analyzing of \textit{que}-clauses with result/sequential or simultaneous meaning in Portuguese, we will review the concepts of parataxis, hypotaxis and insubordination.

\subsection{On the notions of parataxis, hypotaxis and insubordination} \label{sec:matos:sect4-2}

Taking as a starting point the general meaning of hypotaxis (dependent arrangement) and parataxis (side by side/equal arrangement), we will assume that hypotaxis occurs when a linguistic constituent is syntactically (and sometimes semantically) dependent of another, while parataxis relates constituents with equivalent syntactic status of (in)dependency. Though the current usage of hypotaxis mainly identifies it with subordination\footnote{The characterization of these notions is not consensual. \citet[919]{quirk_comprehensive_1985} assume as cases of hypotaxis both subordination and embedding of a sentence into another. \citet[3513]{lopez_garcia_relaciones_1999} assumes hypotaxis and parataxis mainly apply to clause connections, while subordination and coordination are used for clauses and non-sentential phrasal constituents.}, the same does not happen with the notion of parataxis which is presently taken to range over regular coordination, juxtaposition, but also parentheses (\citealt[911--919]{quirk_comprehensive_1985}; \citealt[178]{heycock_embedded_2006}\footnote{Heycock defines \textit{parataxis}, as “a combination of clauses neither of which is subordinate to the other” and claims that, on a par of symmetric coordination, other cases have been considered as paratactic, among which adverbial clauses expressing causation, ‘so’ clauses in English, and free \textit{dass} clauses in German.}). We will adopt this point of view of hypotaxis and parataxis. However, we assume that, despite being instances of parataxis, integrated structured coordinate structures can be distinguished from parentheticals. In the former, a coordinate conjunction (overt or null) selects two terms of any categorial nature but with identical structural status, and none of the terms acts as an argument or a modifier of the other. As for parentheses, they are non-integrated constituents, with prosodic, syntactic and semantic autonomy, that typically act as comments of the host sentence or of one of its constituents (see § 5.2). Several empirical properties typically distinguish the range of phenomena included under parataxis from subordination. However within the generative grammar framework, they are all built by Merge, in its different modalities, in particular, by External Merge, which combines two distinct objects, and by Internal Merge (displacement), when one of the merged objects already occurs as a term of the other merge operation. Merge is a recursive binary operation that builds up a hierarchical structure. It combines two syntactic objects, X and Y and produces a new one, call it K (\citealt{chomsky_problems_2013}; \citeyear{chomsky_problems_2015}; \citealt{chomsky_generative_2019}): 

\ea \label{ex:matos:merge-xy}
Merge X and Y: $\xrightarrow{}$ K = {X, Y}
\z

The nature of the new object K is established by a \textit{label algorithm}. In the trivial case, one syntactic object is a Lexical Item (LI), a head, and the other one a phrase, a XP. In this case, usually mentioned as Set Merge, the label algorithm assigns to the new object K the label of the closest lexical item (LI) (\citealt[43]{chomsky_problems_2013}; \citeyear[7]{chomsky_problems_2015}):

\ea \label{ex:matos:given-k}
Given K = {H, XP}, if H is a LI and XP a complex object, H will be chosen as the label.
\z 

More problematic cases for labelling are those that apply to structurally identical objects, produced by Pair Merge \{X(P), Y(P)\}, where both objects may be understood as heads. Chomsky suggests that in this case “there is no label, unless they agree” \citep[7]{chomsky_problems_2015} or the syntactic object is modified by Internal Merge, as he proposes for regular structured coordination in \citet[46]{chomsky_problems_2013}, which comprises integrated coordination.\footnote{Notice that Set Merge, which was conceived to build Specifier-Head-Complement structures, is assumed to be present in argumental subordination (and, for some authors, also in adverbial subordination) as well as in integrated Coordination (e.g. \citealt{johannessen_coordination_1998}). Similarly, Pair Merge has been classically assigned to adverbial subordinate clauses, but it is also present in analyses of coordinate structures \citep{munn_topics_1993}.}

We accept that most of the distinct properties of subordinate, coordinate and parenthetical structures, result from the functional heads that instantiate them, C (complementizer), Conj (conjunction) or Par (parenthesis) (see § 5). However, regarding C, its presence does not in itself guarantee that a subordinate sentence occurs. A CP may only be considered a subordinate sentence when it is embedded in another one, as an argument or a modifier of this clause or one of its constituents. In fact, root sentences are CPs, and the cases of insubordination show that an overt complementizer, such as que (standardly used to introduce finite subordinate clauses), may head matrix CPs, in colloquial registers of (some) Ibero-Romance languages (cf. \ref{ex:matos:vien} and \ref{ex:matos:estropeada} above).\footnote{As previously mentioned, in Spanish and Catalan, the complementizer \textit{que} can be used at the beginning of a clause that is not dependent from any previous constituent (cf. \citealt[113]{gras_revisiting_2016}). However not all occurrences of the \textit{que} form in insubordinate clauses have been identified with the finite complementizer and, in some cases have been characterized as homonymous markers instantiating different functional heads of the C domain (e.g. \citealt{demonte_evidentiality_2014}; \citealt{corr_ibero-romance_2016}; \citeyear{corr_matrix_2018})}

We will analyze connective-\textit{que} clauses in Portuguese taking into account the above considerations.


\subsection{Connective-\textit{que} clauses in Portuguese} \label{sec:matos:sect4-3}

Independently of their meaning, connective-\textit{que} clauses in Portuguese exhibit properties of finite CPs, also present in subordination. First of all, they are introduced by the finite complementizer \textit{que}, which selects finite sentences in this language
\footnote{In Portuguese, \textit{que} is the complementizer that introduces finite subordinate sentences in subordinate clauses, as in \ref{ex:matos:fn3-i}. Infinitival sentences are introduced by prepositional complementizer (e.g. \textit{para} ‘for’), as in \ref{ex:matos:fn3-ii}, or by the null complementizer, as in \ref{ex:matos:fn3-iii}:

\ea
  \ea\label{ex:matos:fn3-i}
    \gll Ana disse que ela faz / *fazer o bolo.\\
                    Ana said that she makes / *make.\textsc{inf} the cake\\
    \glt ‘Ana said that she will make the cake.’ 
  \ex\label{ex:matos:fn3-ii}
    \gll Ela pediu-me para fazer o bolo.\\ 
                    she asked-me for make.\textsc{inf} the cake\\
    \glt ‘She asked me to make the cake.’
  \ex\label{ex:matos:fn3-iii}
    \gll Ela admitiu $\varnothing$ estar cansada.\\
                    she admitted $\varnothing$  be tired\\
    \glt ‘She admitted to be tired.’ 
  \z 
\z 

As illustrated in \ref{ex:matos:fn3-vi}--\ref{ex:matos:fn3-vii}, Portuguese does not allow sentences similar to the Spanish examples in \ref{ex:matos:fn3-iv}--\ref{ex:matos:fn3-v}, where \textit{que} cooccurs with an infinitival sentence:

\ea
\settowidth\jamwidth{(Portuguese)}
  \ea\label{ex:matos:fn3-iv}
    \gll Dijo que a no molestarle. \\
                    said that for not bother.him.\textsc{cl}\\
    \glt ‘He says not to bother him.’\\ \citep[551]{rivero_indirect_1994}
  \ex\label{ex:matos:fn3-v}
    \gll Dijo que por qué no molestarle. \\
                said that why not bother him-\textsc{cl}\\
    \glt ‘He said: Why not bother him.’\\\citep[551]{rivero_indirect_1994}
  \ex[*]{\label{ex:matos:fn3-vi}
    \gll Disse que para não o incomodar.    \\
        say.\textsc{prf}.\textsc{3sg} that for not him.\textsc{cl} bother\\\jambox{(Portuguese)}}
  \ex[*]{\label{ex:matos:fn3-vii}
    \gll Disse que por que não o incomodar.                           \\ 
        say.\textsc{prf}.\textsc{3sg} that why not him.\textsc{cl} bother\\\jambox{(Portuguese)}}
  \z 
\z 

According to Rivero, the examples in \ref{ex:matos:fn3-iv}--\ref{ex:matos:fn3-v} are cases of recursive CP, in which a quotative marker, \textit{que}, co-occurs with an embedded CP: \textit{que} co-occurs with the prepositional complementizer that encodes (command) illocutionary force in \ref{ex:matos:fn3-iv}; and with an embedded CP with interrogative illocutionary force in \ref{ex:matos:fn3-v}. 

The unacceptability of \ref{ex:matos:fn3-vi}--\ref{ex:matos:fn3-vii} indicates indicate that contemporary Portuguese does not have a \textit{que}-quotative complementizer, a fact also patent in finite embedded interrogative sentences in European Portuguese, as shown in \citet[90]{matos_alternation_2013}.} % END FOOTNOTE
As shown in \pef{ex:matos:inf-ovos}, infinitival verb forms are excluded from \textit{que}-clauses, despite the existence of personal inflected infinitive in Portuguese: 

\ea \label{ex:matos:inf-ovos}
 \ea \label{ex:matos:inf-ovos-a}
 \gll Compre os ovos, que eu quero / *querer fazer o bolo.\\
 	  buy.\textsc{imp} the eggs, that I want / *want.\textsc{inf} make.\textsc{inf} the cake\\
 \glt ‘Buy the eggs, that I want (/*I to want) to make the cake.’
 \ex \label{ex:matos:inf-ovos-b}
 \gll Compre os ovos, que eu faço / *fazer o bolo.\\
	buy.\textsc{imp} the eggs, that I make / *make.\textsc{inf} the cake\\ 
 \glt ‘Buy the eggs, that I will make (/*I to make) the cake.’
 \ex \label{ex:matos:nao-sei-aconteceu}
 \gll Não sei {o que} aconteceu, que ele está / *estar a chorar.\\
	  not know.\textsc{pres}.\textsc{1sg} what happened, that he is / *be.\textsc{inf} to cry.\textsc{inf}\\
 \glt ‘I don’t know what happened, that he is crying (/*that he to be       to cry).’ 
 \z
 \z 

As a complementizer, \textsc{que} only selects a sentential complement. The examples in \pef{ex:matos:vamos-embora}-\pef{ex:matos:doente} indicate that, in contrast with TPs, small clauses or DPs are rejected:

\ea \label{ex:matos:vamos-embora}
 \ea \label{ex:matos:vamos-embora-a}
 \gll Vamos embora, que [a criança está cansada.]\\
      go.\textsc{imp}.\textsc{1pl} away, that the child is tired\\
 \glt ‘Let’s go, because the child is tired.’
 \ex \label{ex:matos:vamos-embora-b}
 \gll {*} Vamos embora, que [a criança cansada.]\\
      {} go.\textsc{imp}.\textsc{1pl} away, that the child tired\\
 \ex \label{ex:matos:vamos-embora-c}
 \gll {*} Vamos embora, que [o cansaço.]\\
      {} go.\textsc{imp}.\textsc{1pl} away, that the tiredness\\
 \z
\z 

\ea \label{ex:matos:bata-claras}
 \ea \label{ex:matos:bata-claras-a}
 \gll Bata as claras, que eu faço o bolo.\\
      whisk.\textsc{imp} the {egg whites}, that I make.\textsc{pres} the cake\\
 \glt ‘Whisk the egg whites, then I will make the cake.’
 \ex \label{ex:matos:bata-claras-b}
 \gll {*} Bata as claras, que [o bolo delicioso].\\
	  {} whisk.\textsc{imp} the {egg whites}, that [the cake delicious]\\
 \z
\z 

\ea \label{ex:matos:doente}
 \ea \label{ex:matos:doente-a}
 \gll Não sei {o que} aconteceu, que ele ficou doente.\\
      not know.\textsc{pres}.\textsc{1sg} what happened, that he became ill\\
 \glt ‘I don’t know what happened, that he became ill.’
 \ex \label{ex:matos:doente-b}
 \gll {*} Não sei {o que} aconteceu, que ele doente.\\
      {} not know.\textsc{pres}.\textsc{1sg} what happened, that he ill\\
 \ex \label{ex:matos:doente-c}
 \gll {*} Não sei {o que} aconteceu, que doença.\\
      {} not know.\textsc{pres}.\textsc{1sg} what happened, that illness\\
 \z
\z 

In addition, the \textit{que}-clause does not allow gapping \pef{ex:matos:compre-elipse}, an ellipsis phenomenon typically excluded from CPs, in particular from proper causal clauses \pef{ex:matos:gelado-elipse}:

\ea[*]{
\gll Compre os ovos, que eu \sout{compro} a manteiga.\\
	 buy.\textsc{imp} the eggs, that I \sout{buy} the butter\\
\glt (‘Buy the eggs, and/while I buy the butter.’)
} \label{ex:matos:compre-elipse}
\z 

\ea[*]{
\gll Ela só não comprou gelado porque nós \sout{compramos} um bolo\\ 
     she only not bought {ice cream} because we \sout{bought} a cake\\
\glt (‘She didn’t buy the ice cream, only because we bought a cake.’)
} \label{ex:matos:gelado-elipse}
\z 

This behavior shows that connective \textit{que}-clauses are CPs and seem to correlate them with adverbial subordinate sentences. However, as it happened with causal explicatives considered in \sectref{sec:matos:sect3}, whenever the link with the associate sentence is at stake, connective \textit{que}-clauses exhibit properties characteristic of paratactic structures, especially juxtaposition and parentheses.

The main arguments for the correlation of connective \textit{que}-clauses with insubordinate juxtaposed sentences or with parenthetical clauses are the prosodic autonomy of the \textit{que}-clause, their autonomous illocutionary force, and the weak semantic content of \textit{que}, which mainly express a connection with the associate clause. These properties are present in the \textit{que} clauses considered, independently of their explicative, \pef{ex:matos:deme-dicionario-a}, result \pef{ex:matos:assine-papeis}-\pef{ex:matos:nao-sei-chorar} or simultaneous temporal \pef{ex:matos:jornal-revista} preferential reading: 

\ea \label{ex:matos:deme-dicionario}
 \ea \label{ex:matos:deme-dicionario-a}
 \gll Dê-me o dicionário, que eu quero pôr na estante.\\
	  give.\textsc{imp}-me the dictionary, that I want put.\textsc{inf} on.the shelf\\
 \glt ‘Give me that dictionary, because I want to put it on the shelf.’
 \ex \label{ex:matos:assine-papeis}
 \gll Assine os papéis, que eu levo para o advogado.\\
	  sign.\textsc{imp} the papers, that I take to the lawyer\\
 \glt ‘Sign the papers, then I take them to the lawyer.’
 \ex \label{ex:matos:nao-sei-chorar}
 \gll Não sei {o que} aconteceu, que ele está a chorar.\\
	  not know.\textsc{pres}.\textsc{1sg} what happened, that he is to cry.\textsc{inf}\\
 \glt ‘I don’t know what happened, that he is crying.’
 \ex \label{ex:matos:jornal-revista}
 \gll Leia o jornal, que eu leio a revista.\\
	read.\textsc{imp} the newspaper, that I read.\textsc{pres} the magazine\\
 \glt ‘Read the newspaper, while I read the magazine.’
 \z
\z 
 
Also, the impossibility of fronting of the \textit{que}-clause, illustrated in \pef{ex:matos:estante}, may constitute an argument for the juxtaposed or parenthetical status of the \textit{que}-clause, because juxtaposition typically requires a fixed order of the sentences involved \pef{ex:matos:dicionario}; and parenthetical clauses mostly require their host to precede them, see, for instance, \pef{ex:matos:trabalho-elaborado}, repeated in \pef{ex:matos:trabalho}, where a complex DP modifies the host clause:

\ea \label{ex:matos:estante}
 \ea[*]{
 \gll Que eu quero pôr na estante, dê-me o dicionário.\\
	  that I want put.\textsc{inf} on.the shelf, give.\textsc{imp}-me the dictionary\\
 } \label{ex:matos:estante-a}
 \ex[*]{
 \gll Que eu levo para o advogado, assine os papéis.\\
	  that I take to the lawyer, sign.\textsc{imp} the papers\\
 } \label{ex:matos:advogado-papeis}
 \ex[*]{ 
 \gll Que ele está a chorar, não sei {o que} aconteceu.\\
	  that he is to cry.\textsc{inf}, not know.\textsc{pres}.\textsc{1sg} what happened\\
 } \label{ex:matos:chorar-aconteceu}
 \ex[*]{
 \gll Que eu leio a revista, leia o jornal.\\
	  that I read the magazine, read.\textsc{imp} the newspaper\\
 } \label{ex:matos:revista-jornal}
 \z
\z 

\ea \label{ex:matos:dicionario}
 \ea \label{ex:matos:dicionario-a}
 \gll Dê-me o dicionário. Eu ponho na estante.\\
	  give.\textsc{imp}-me the dictionary. I put in.the shelf\\	
 \glt ‘Give-me the dictionary. I will put it on the shelf.’
 \ex[??]{
 \gll Eu ponho na estante. Dê-me o dicionário.\\
	  I {put (it)} in.the shelf. give.\textsc{imp}-me the dictionary.\\
 } \label{ex:matos:dicionario-b}
 \z
\z 

\ea \label{ex:matos:trabalho}
 \ea \label{ex:matos:trabalho-a}
 \gll O trabalho estava bem elaborado, coisa que muito nos agradou.\\
      the work was well elaborated, thing that {a lot} us.\textsc{cl} pleased\\
 \glt ‘The work was well elaborated, what pleased us a lot.’
 \ex[*]{
 \gll Coisa que muito nos agradou, o trabalho estava bem elaborado.\\
      thing that {a lot} us.\textsc{cl} pleased, the work was well elaborated\\
 } \label{ex:matos:trabalho-b}
 \z
\z 

Notice, however, that the assumption that connective-\textit{que} clauses are juxtaposed to the previous sentence implies that they are independent insubordinate clauses; in contrast, their inclusion in the parentheticals does not necessarily lead to such an assumption, since it depends on the approach to parenthesis we adopt.

Yet, the hypothesis that connective \textit{que}-clauses are Insubordinate CPs is empirically challenged by the unacceptability of examples like \pef{ex:matos:contexts-bad}, which show that these sentences cannot occur in isolation, merely grounded in a situational context, but require the spell-out of the associate sentence, in opposition to conjunctive-\textit{que} in Spanish and Catalan, as shown by Corr (\citeyear{corr_ibero-romance_2016}; \citeyear{corr_matrix_2018}) for conjunctive-\textit{que} in Spanish and Catalan, languages where examples like \pef{ex:matos:vien}, and \pef{ex:matos:vacances} are possible:

\ea \label{ex:matos:contexts-bad}
 \ea \label{ex:matos:context-bad-a}
 Context: B is handling a dictionary. A says:\\
 \gll {*} Que eu quero pôr na estante.\\
	  {} that I want put.\textsc{inf} on.the shelf\\
 \ex \label{ex:matos:context-bad-b}
 Context: To convince B that (s)he has to sign the documents, A says:\\
 \gll {*} Que eu levo para o advogado.\\
	  {} that I take to the lawyer\\
 \ex \label{ex:matos:context-bad-c}
 Context: A is aware that something caused the child to cry and says:\\
 \gll {*} Que ele está a chorar.\\
	  {} that he is to cry.{inf}\\
 \ex \label{ex:matos:context-bad-d}
 Context: On the table there are a newspaper and a magazine. B hesitates on which to take. A says:\\ 
 \gll {*} Que eu leio a revista.\\
 	  {} that I read the magazine\\
 \z
\z 

\ea \label{ex:matos:vacances}
Context: the addressee glances at some boarding passes on the speaker’s desk and the speaker notices what the addressee is looking at.\\
\gll Que me’n vaig de vacances. \\
	 \textsc{conj} me.\textsc{part} go.\textsc{1} of holidays\\
\glt ‘I’m going on holiday.’\\\jambox{(Catalan)}
	\citep[226]{corr_ibero-romance_2016}
\z 

In addition, the fact that connective \textit{que}-clauses in Portuguese, together with their associated clause, may be embedded in a subordinate clause, corroborates that they are not insubordinate CPs:

\ea \label{ex:matos:vestisses}
 \ea \label{ex:matos:vestisses-a}
 \gll Eu já disse que vestisses o impermeável, que vai chover à tarde. \\
      I already told that put.on.\textsc{pst}.\textsc{sbjv}.\textsc{2sg} the raincoat, that will rain at.the afternoon\\\jambox{(EP)}
 \glt ‘I already told you to put on the raincoat, because it will be raining in the afternoon.'
 \ex \label{ex:matos:peco-que-assines}
 \gll Peço que assine logo os papéis, que eu levo para o advogado.\\
	  ask.\textsc{pres}.\textsc{1sg} that sign.\textsc{sbjv}.\textsc{2sg} soon the papers, that I take to the lawyer\\
 \glt ‘I ask you to sign the papers as soon as possible, and I’ll take      them to the lawyer.’
 \z
\z 

These data argue for the non-insubordinate status of connective \textit{que}-clauses in Portuguese. Still, examples like the one in \pef{ex:matos:friday-a}, from \citet{kocher_unselected_2019}, which present a case of juxtaposition, seem to challenge this claim. Nevertheless, as shown in \pef{ex:matos:friday-b}, the \textit{que}-clause in this context may not occur in isolation. Moreover, the \textit{que}-clause in \pef{ex:matos:friday-a} is interpreted as an afterthought, which comprises an elided version of the correlate sentence. This is clear in \pef{ex:matos:friday-c}, where the sentential positive polarity item \textit{Sim} ‘yes’, clearly recover the omitted sentence. Thus, these examples present some similarity with the preceding ones, the \textit{que}-clause apparently modifying an elided main sentence.


\ea \label{ex:matos:friday}
 \ea \label{ex:matos:friday-a}
 \gll Keep calm. (Que) hoje é sexta-feira! \\
      keep calm that today is Friday\\
 \glt ‘Keep calm. Today is Friday.’ \\\citep[148]{kocher_unselected_2019}
 \ex \label{ex:matos:friday-b}
 Context: B is nervous, and A says:\\
 \gll {*} Que hoje é sexta-feira!\\
      {} that today is Friday\\
 \ex \label{ex:matos:friday-c}
 \gll Keep calm. Sim \sout{(keep calm)}, que hoje é sexta-feira!\\
      keep calm. yes \sout{(keep calm)} that today is Friday\\
 \z
\z 

Considering these data, we conclude that connective-\textit{que} clauses in Portuguese are not insubordinate sentences. Thus, their weak integration in the associated sentence they depend on should be dealt with otherwise. We claim that they are parenthetical CPs, whose C domain, in a cartographic approach, involves the functional heads, Force and Fin.  

\section{\textit{Que}-clauses as parenthesis} \label{sec:matos:sect5}

\subsection{The parenthetical nature of connective \textit{que}-clauses} \label{sec:matos:sect5-1}

One piece of evidence that connective \textit{que}-clauses are parenthetical clauses is the fact that they may occur appended, \pef{ex:matos:joao-ferro-a}-\pef{ex:matos:assine-logo-a}, or interpolated, \pef{ex:matos:joao-ferro-b}-\pef{ex:matos:joao-ferro-c}-\pef{ex:matos:assine-logo-b} in the host sentence, often behaving as floating parentheticals:

\ea \label{ex:matos:joao-ferro}
 \ea \label{ex:matos:joao-ferro-a}
 \gll O João tirou dois meses de férias, que ninguém é de ferro.\\
	  the João took two months of vacation, that nobody is of iron.\\ 
 \glt ‘John took a two-month vacation, because nobody is made of iron.’
 \ex \label{ex:matos:joao-ferro-b}
 \gll O João, que ninguém é de ferro, tirou dois meses de férias.\\
	  the João, that nobody is of iron, took two months of vacation\\
 \ex \label{ex:matos:joao-ferro-c}
 \gll O João tirou, que ninguém é de ferro, dois meses de férias.\\
	  the João took that nobody is of iron two months of vacation\\
 \z
\z

\ea \label{ex:matos:assine-logo}
 \ea \label{ex:matos:assine-logo-a}
 \gll Assine logo os papéis do divórcio, que eu levo para o advogado.\\
      sign.\textsc{imp} soon the papers of.the divorce, that I take to the lawyer\\
 \glt ‘Sign the divorce papers as soon as possible, and I’ll take them to the lawyer.’  
 \ex \label{ex:matos:assine-logo-b}
 \gll Assine logo, que eu levo para o advogado, os papéis do divórcio.\\
	  sign.\textsc{imp} soon, that I take to the lawyer, the papers of.the divorce\\
 \glt ‘Sign them as soon as possible, and I’ll take to the lawyer the divorce papers.’
 \z
\z 

As parentheses, connective \textit{que}-clauses have prosodic autonomy, often marked in writing by punctuation, do not intervene in the propositional content of the host clause, corresponding to an additional information, and present syntactic autonomy with respect to the host clause.
The syntactic autonomy of the \textit{que}-clauses is clearly evidenced in the blocking of c-command effects, illustrated in the examples \pef{ex:matos:des} and \pef{ex:matos:ele-tirou-ferias}, respectively presenting the blocking of the scope of sentence negation and of principle-C effects:

\ea \label{ex:matos:des}
 \ea[\#]{
 \gll Não me dês esse dicionário, que eu quero pôr na estante.\\
      not me.\textsc{cl} give.\textsc{imp}.\textsc{2sg} that dictionary, that I want put.\textsc{inf} on.the shelf\\
 \glt ‘Do not give me that dictionary, because I want to put it on the shelf.’
 } \label{ex:matos:nao-des}
 \ex[??]{
 \gll Não assine os papéis, que eu levo para o advogado.\\
      not sign.\textsc{imp} the papers, that I take.\textsc{pres} to the lawyer\\
 \glt ‘Do not sign the papers, then I will take them to the lawyer.’
 } \label{ex:matos:nao-assine}
 \z
\z 

\ea \label{ex:matos:ele-tirou-ferias}
\gll Ele$_i$ tirou dois meses de férias, que o João$_i$ não é de ferro.\\
	 he took two months of vacation, that the João not is of iron\\
\glt ‘He took a two-month vacation, because John is not made of iron.’
\z 

In \pef{ex:matos:des} the negation in the host sentence does not have scope over the \textit{que}-clause, thus producing pragmatically odd sentences. In \pef{ex:matos:ele-tirou-ferias}, c-command from the subject position of the host clause is blocked and the coreference between the pronoun subject, \textit{ele} ‘he’, and the R-expression produces an acceptable result.

The assumption that connective \textit{que}-clauses are parentheses allow us to accommodate their CP dependent behavior with the loose paratactic link they establish with their host sentences. As it is well known, parentheses include a large diversity of constituents. Just to mention sentential constituents, both subordinate \pef{ex:matos:lotaria-a} and coordinate sentences \pef{ex:matos:carro-novo} may occur as parentheticals:

\ea \label{ex:matos:lotaria}
 \ea \label{ex:matos:lotaria-a}
 \gll O  João  comprava, se  ganhasse a lotaria, um avião a jacto pessoal.\\
      the João buy.\textsc{pst}.\textsc{ipfv}.\textsc{}ind if win.\textsc{pst}.\textsc{ipfv}.\textsc{sbjv} the lottery a plane to jet personal\\
 \glt ‘John would buy, if he won the lottery, a personal jet plane.’
 \ex \label{ex:matos:carro-novo}
 \gll Eu gostaria, e não estou a pedir muito, que tu me oferecesses um      carro novo.\\
      I like.\textsc{cond}, and not be.\textsc{pres}.\textsc{1sg} to ask much, that you me offer.\textsc{pst}.\textsc{ipfv}.\textsc{sbjv} a car new\\
 \glt ‘I would like, and I am not asking too much, that you give me a new car.’
 \z
\z 

In sum, we assume that connective \textit{que}-clauses may be categorized as parenthetical CPs, closely related to non-integrated adverbial clauses.	

\subsection{The approach to parentheses adopted} \label{sec:matos:sect5-3}

Two main approaches to parentheses had received special attention in recent literature: (i) the radical structural autonomy approach, proposed by \citet{haegeman_parenthetical_1991}, which argues that parentheses are constructed syntactically as radically independent of their hosts, and the insertion and linearization in the hosts occurs at the discourse level, an approach also adopted in \citet[1351--1353]{huddleston_coordination_2002}; and (ii) the syntactic inclusion approach to parentheses, that merges them as adjunct constituents marked from the lexicon with a specific parenthetical feature (for different implementations of this idea, see \citealt{potts_lexical_2002}, \citealt{matos_appositive_2009}, \citealt{de_vries_unconventional_2012}).

The first approach faces several problems, namely to account for the dependency of the parentheticals anchored to specific constituents, as \textit{e ganhou bem} in (52), as shown by the contrast in acceptability between \pef{ex:matos:partido-a} and \pef{ex:matos:partido-b}:

\ea \label{ex:matos:partido}
 \ea \label{ex:matos:partido-a}
 \gll O partido que ganhou, e ganhou bem, não atinge 50\% dos votos.\\ 
      the party that won, and won well, not get 50\% of.the votes\\
 \glt ‘The party that won, and it won well, did not get 50\% of the         vote.’\\ 
     (Público 21/10/2008)
 \ex[??]{
 \gll O partido que ganhou não atinge, e ganhou bem, 50\% dos          votos.\\
      the party that won not get, and won well, 50\% of.the votes\\
 } \label{ex:matos:partido-b}
 \z
\z 

The impossibility of parentheticals to be pragmatically licensed by situational antecedent, in isolation from a linguistic host, also argues for the claim that there is some syntactic dependency between the parenthetical and its host. This is shown in \pef{ex:matos:situation} for connective \textit{que} clauses:

\ea \label{ex:matos:situation}
Situation: the addressee is trying to switch on the light.
 \ea[*]{
 \gll Que está estragado(EP)/quebrado(BP).\\
	  that   is   broken\\
 } \label{ex:matos:estragado-a}
 \ex \label{ex:matos:estragado-b}
 \gll Não toque, que está estragado/quebrado.\\
	  not touch.\textsc{imp}, that is broken\\
 \glt ‘Do not touch, because it is broken.’ 
 \z
\z 

This impossibility distinguishes connective-\textit{que} clauses in Portuguese from insubordinate conjunctive \textit{que}-clauses, illustrated for Spanish and Catalan by Gras (\citeyear{gras_revisiting_2016}), Corr (\citeyear{corr_ibero-romance_2016}; \citeyear{corr_matrix_2018}) (see examples \pef{ex:matos:vien}, \pef{ex:matos:estropeada} and \pef{ex:matos:situation}, above). Although in both cases the \textit{que} connector has no intrinsic content, they differ with respect to the relation they establish with the associate constituent: the Portuguese connective \textit{que}-clauses analyzed depend on the overt presence of the related clause; in contrast, in the Spanish and Catalan insubordinate conjunctive \textit{que}-clauses, \textit{que} is a discourse connector that allows to retrieve information from a previous non-linguistic context. Thus, we will adopt \citet{de_vries_unconventional_2012} syntactic approach to parentheticals, extending \citet{colaco_explicative_2016} proposal for explicative causal clauses to all connective-\textit{que} clauses in Portuguese. 

In \citet{de_vries_invisible_2007} proposal, parentheses were conceived a specific case of coordination, taken in a broad sense, which recovered “conjunction, disjunction, specification (bivalent and monovalent), other” \citep[231]{de_vries_invisible_2007}. However, as we mentioned in \sectref{sec:matos:sect4-2}, we do not identify integrated coordination with parenthesis. In fact, integrated conjunctive and disjunctive coordination, differently from parentheses, allows c-command from the outermost constituent of the first coordinated term over the second one. Thus, the following examples of Portuguese respectively display bound anaphora, \pef{ex:matos:anafora-a}, setting of the negative sense of the underspecified polarity expression \textit{quem quer que seja} ‘anyone’, \pef{ex:matos:anafora-b}, and principle C effects, \pef{ex:matos:anafora-c}: 

\ea \label{ex:matos:anafora}
 \ea \label{ex:matos:anafora-a}
 \gll [Todo o aluno]$_i$ fez um projeto e [seu$_i$ professor] analisou-o cuidadosamente.\\
	  every the student made a project and his teacher analyzed-it carefully\\
 \glt ‘Every student made a project and his teacher analyzed it carefully.’
 \ex \label{ex:matos:anafora-b}
 \gll Ninguém o viu ou {quem quer que seja} o ouviu chorar.\\
      nobody him.\textsc{cl} saw or anyone him.\textsc{cl} heard cry.\textsc{inf}\\ 
 \glt ‘Nobody saw him nor did anyone heard him crying.’
 \ex[*]{
 \gll Ela/pro$_i$ entrou e a professora$_i$ cumprimentou os alunos.\\
      she/pro entered and the teacher greeted the students\\
 } \label{ex:matos:anafora-c}
 \z
\z 

However, we consider that \citet{de_vries_invisible_2007,de_vries_unconventional_2012} de Vries (2007; 2012) approach to parentheticals adequately accounts for the properties of connective \textit{que}-clauses.\footnote{We will not discuss the cases of Specifying Coordination, mainly conceived to account for Appositive Relative Clauses (see \citealt{de_vries_invisible_2007}; \citeyear{de_vries_relative_2018}), which are analyzed as parentheses in \citet{griffiths_parenthesis_2019}.}

According to this proposal, parenthetical clauses are the projection of \textit{Par}, a monovalent a functional head, which triggers the application of \textit{Par-Merge}. \textit{Par} selects a complement, XP\textsubscript{par}, corresponding to the parenthetical constituent, and projects ParP*, a functional projection that right or left adjoins to its host, YP, depending on the position it occupies regarding the host expression.  ParP* starts a new prosodic and c-command domain, and blocks c-command from the host expression. The resulting structure is schematized in \pef{ex:matos:yp}:

\ea \label{ex:matos:yp}
 \ea {[\capsub{YP}  YP  [ParP*   Par  XPpar ]]}
 \ex {[\capsub{YP}  [ParP*   Par  XPpar ] YP]}
 \z
\z 

The following representations illustrate this analysis for connective \textit{que}-clauses in \pef{ex:matos:parenteticos-ovos-a} and \pef{ex:matos:parenteticos-ferias-a}:

\ea \label{ex:matos:parenteticos-ovos}
 \ea \label{ex:matos:parenteticos-ovos-a} 
 \gll Compre os ovos, que eu faço o bolo de banana.\\
	  buy.\textsc{imp} the eggs, that I make the cake of banana\\
 \glt ‘Buy the eggs, then I will make the banana cake.’ 
 \ex \label{ex:matos:parenteticos-ovos-b}
 [\capsub{CP} [\capsub{CP} Compre os ovos] [\capsub{ParP*} Par [\capsub{CP} que eu faço o bolo de banana]]]
 \z
\z 

\ea \label{ex:matos:parenteticos-ferias}
 \ea \label{ex:matos:parenteticos-ferias-a}
 \gll O João tirou, que ninguém é de ferro, dois meses de férias.\\ 
	  the João took, that nobody is of iron, two months of vacation\\
 \glt ‘John took a two-month vacation, because nobody is made of iron.’
 \ex \label{ex:matos:parenteticos-ferias-b}
 [\capsub{TP} [\capsub{DP} O João$_i$] [\capsub{T} took$_j$ ] [\capsub{vP} [\capsub{ParP*} Par [\capsub{CP} que ninguém é de ferro]] [\capsub{vP} \sout{o João$_i$ tirou$_j$} dois meses de férias]]]
 \z
\z 

In sum, this analysis seems to adequately capture the CP status of the \textit{que}-clause and its syntactic dependence on its linguistic host, while accounting for its prosodic, semantic, syntactic and discursive autonomy regarding the host sentence. Thus, we claim that the paratactic status of Connective-\textit{que} clauses relies on their parenthetical nature.

\section{Final remarks} \label{sec:matos:sect6}

This paper discussed what we called the connective \textit{que}-clauses in Portuguese, which have been previously characterized as explicative causal clauses, because of their frequent justification value. We showed that the causal reading is not the only one that arises in these non-argumental clauses headed by \textit{que}, and some of them may convey a result or sequential or a simultaneous temporal event interpretation, although these cases have been mostly neglected in Portuguese. We argued that these different readings are derived from the context, the connector \textit{que} being rather neutral. However, despite their different readings, we claimed that connective-\textit{que} clauses can be analyzed alike. We restrained our analysis to connective-\textit{que} clauses related to a speech act. 

We argued connective \textit{que}-clauses are CPs that depend on a linguistic constituent overtly spelled-out and may occur embedded in subordinate sentences; in this sense, they are closely related to subordinated adverbial sentences. Yet, they are markedly non-integrated in the connected clause, and exhibit a paratactic behavior when it comes to the connection of the \textit{que}-clause with the related sentence. Considering that there is no evidence to treat these clauses in Portuguese as instances of Insubordination (as proposed in Corr (\citeyear{corr_ibero-romance_2016}; \citeyear{corr_matrix_2018}), for Spanish and Catalan), we analyzed connective \textit{que}-clauses as parenthetical CPs. 

Extending \citet{colaco_explicative_2016} proposal for explicative causal clauses in European Portuguese to all connective \textit{que}-clauses in Brazilian and European Portuguese, we adopted de Vries' (\citeyear{de_vries_unconventional_2012}) approach to parentheticals. According to his proposal, the paratactic relation obtains from the adjunction to the host expression of a parenthetical phrase, ParP*, resulting from the projection of a functional head \textit{Par} that selects the parenthetical constituent as its complement. ParP* starts a new prosodic domain and blocks c-command, accounting for the parenthesis’ opacity in respect with its host.

The data presented in this paper reveal a particularity of Portuguese in comparison with other Romance languages, like Spanish and Catalan, in the sense that, in Portuguese, \textit{que}-clauses are never insubordinate.


%\section*{Acknowledgements}
%\citet{Nordhoff2018} is useful for compiling bibliographies

\printbibliography[heading=subbibliography,notkeyword=this]

\end{document}