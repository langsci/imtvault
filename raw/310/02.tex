%%%%%%%%%%
%% Editor Notes
%% 3/28/2021
%% 1. Several commands were not recognized (e.g., \textopeno. Not sure which package these belong to.


\documentclass[output=paper]{langscibook} 
\title{French constituent unconditionals relativize a free choice item} 
\author{Aurore Gonzalez\affiliation{Harvard University} and Karoliina Lohiniva\affiliation{New York University}} 

\abstract{The realization of unconditionals is subject to considerable cross-linguistic variation \citep{haspelmathkonig1998}. In this paper, we zoom in on constituent unconditionals in European French, and argue that they contain free choice items. Specifically, we propose that the adjunct clauses of these structures are derived through the relativization of a free choice item which may itself be partly elided. In doing so, we provide the first unified syntactic analysis of two types of French constituent unconditionals.  We also provide novel empirical arguments against a potential analysis of French constituent unconditionals as involving \emph{wh}-questions (like English constituent unconditionals, \citealt{rawlins2013}).}

%\def\checkmark{\tikz\fill[scale=0.4](0,.35) -- (.25,0) -- (1,.7) -- (.25,.15) -- cycle;} 

\IfFileExists{../localcommands.tex}{%hack to check whether this is being compiled as part of a collection or standalone
  \input{../localpackages}
  %Copy this to localcommands.tex

\usepackage[english]{babel}
\usepackage{amsmath}
\usepackage{amssymb,amsfonts,textcomp}
\usepackage{array}
\usepackage{hhline}
\usepackage{hyperref}

\newenvironment{styleStandard}{}{}
\newenvironment{stylelsAbstract}{}{}
\newenvironment{stylelsSectioni}{}{}
\newenvironment{stylelsSectionii}{}{}
\newenvironment{stylelsBulletList}{}{}
\newenvironment{styleBibliographyi}{}{}
\newenvironment{listWWNumxxvleveli}{}{}
\newenvironment{listWWNumxxvlevelii}{}{}
\newenvironment{listWWNumxxvleveliii}{}{}
\newenvironment{listWWNumxxvleveliv}{}{}
\newenvironment{listWWNumixleveli}{}{}
\newenvironment{listWWNumixlevelii}{}{}
\newenvironment{listWWNumixleveliii}{}{}
\newenvironment{listWWNumixleveliv}{}{}

\newcommand\textstyleListLabelxvi[1]{#1}
\newcommand\labellistWWNumxxvleveli{\thelistWWNumxxvleveli.}
\newcommand\labellistWWNumxxvlevelii{\thelistWWNumxxvlevelii.}
\newcommand\labellistWWNumxxvleveliii{\thelistWWNumxxvleveliii.}
\newcommand\labellistWWNumxxvleveliv{\thelistWWNumxxvleveliv.}
\newcommand\labellistWWNumixleveli{[F0B7?]}
\newcommand\labellistWWNumixlevelii{\textstyleListLabelxvi{o}}
\newcommand\labellistWWNumixleveliii{[F0A7?]}
\newcommand\labellistWWNumixleveliv{[F0B7?]}

\newcounter{listWWNumxxvleveli}
\newcounter{listWWNumxxvlevelii}[listWWNumxxvleveli]
\newcounter{listWWNumxxvleveliii}[listWWNumxxvlevelii]
\newcounter{listWWNumxxvleveliv}[listWWNumxxvleveliii]
\newcounter{itemize} 
  \input{../localhyphenation}
    \bibliography{localbibliography}
    \togglepaper[23]
}{}

\begin{document}
\maketitle

\section{Introduction}\label{sec:gonzalez:1}

Across languages, we find two main types of unconditionals: alternative (\ref{ex:gonzalez:englishalt}) and constituent (\ref{ex:gonzalez:englishconst}).\footnote{In previous work, unconditionals have been called concessive conditionals \citep{haspelmathkonig1998}.} The names make reference to the adjunct clause, which -- in English -- bears a resemblance to an embedded alternative question (\emph{I wonder whether Mia will bring wine or not}) in the former case, and to a constituent question or \emph{wh}-question (\emph{What will Zoé cook for dinner?}) in the latter. 

\ea\label{ex:gonzalez:english}
    \ea Whether Mia brings wine or not, Lou will be happy. \label{ex:gonzalez:englishalt} 
    \ex Whatever Zoé cooks for dinner, Lou will be happy.
    \label{ex:gonzalez:englishconst}
    \z 
\z 

While English unconditionals are indeed often connected to questions in the literature \citep{haspelmathkonig1998, rawlins2013}, there is quite a lot of variation in how unconditionals are realized syntactically both across languages \citep{haspelmathkonig1998, quer2009, rawlins2013, lohiniva2019, simik2018, szabolcsi2018, falausnicolae2020, balusu2019} and within languages  \citep{quer2009,simik2018}. 


In this paper, we focus on two types of constituent unconditionals (CUs) in European French (henceforth, `French'). In the first type, the adjunct clause of the unconditional is headed by a bare-looking \emph{wh}-word followed by a  relative clause with a subjunctive lexical verb (\ref{ex:gonzalez:1a}). In the second type, the adjunct clause additionally involves a subjunctive copular structure (\ref{ex:gonzalez:1b}). These two types of CU-adjuncts will be referred to as 
`short' and `long', respectively.\footnote{In this paper, we do not discuss constructions that  only involve a copular relative clause in the adjunct clause (i) or constructions that contain a pronoun in the main clause (ii). Although similar on the surface and intuitively related in meaning, these constructions have different syntactic and semantic properties from constructions in (\ref{ex:gonzalez:1a}) and (\ref{ex:gonzalez:1b}), suggesting that a distinct analysis may be needed. We leave constructions like (i) and (ii) for future investigation.

\ea
\gll [\textsubscript{adjunct} Quelle que soit sa d\'ecision ],  [\textsubscript{main} Lou sera contente. $]$\\
{} which \textsc{rel} is.\textsc{sbj} her decision {} {} Lou is.\textsc{fut} happy {}\\
\glt `Whichever decision they make, Lou will be happy.'
\z

\ea
\gll [\textsubscript{adjunct} Quoi que Zo\'e cuisine $]$,  [\textsubscript{main} c'est excellent. $]$\\
{} what \textsc{rel} Zo\'e cooks.\textsc{sbj} {} {} it-is excellent {}\\
\glt `Whatever Zo\'e cooks, it is excellent.' 
\z

}

\ea\label{ex:gonzalez:1a} Short CU-adjunct \\
 \gll [\textsubscript{adjunct} Quoi que Zo\'e cuisine ], Lou sera contente.\\
 {} what \textsc{rel} Zo\'e cooks.\textsc{sbj} {}  Lou is.\textsc{fut} happy  \\ %\jambox{[Short]}
 \glt `Whatever Zo\'e cooks, Lou will be happy.'
 \z

\ea\label{ex:gonzalez:1b} Long CU-adjunct \\
 \gll [\textsubscript{adjunct} Quoi que ce soit que Zo\'e cuisine ], Lou sera contente. \\
{} what \textsc{rel} it is.\textsc{sbj} \textsc{rel} Zo\'e cooks.\textsc{sbj} {} Lou is.\textsc{fut} happy \\ %\jambox{[Long]}
 \glt `Whatever Zo\'e cooks, Lou will be happy.'
\z



The main claim of our paper is that both short and long CU-adjuncts contain a \emph{wh que ce soit} free choice item (\textsc{fci}) that has been relativized. The difference between short and long CUs is that in short CUs, a part of this \textsc{fci} is elided. As a result, the short CU looks as if it contained a bare \emph{wh}-word (i.e., a \emph{wh}-word that could appear by itself in a question). The gist of our proposal is shown in schematic form in (\ref{ex:gonzalez:tree}). $(\Delta)$ marks the relative clause that is optionally elided.



\ea\label{ex:gonzalez:tree}
\ea
\gll [\textsubscript{adjunct}\ [\capsub{\textsc{fci}} Quoi$_i$ [$_{(\Delta)}$ que ce soit $t_i$ ]]$_j$ que  Zoé cuisine $t_j$ ], ... \\
 {} what {} \textsc{rel} it is.\textsc{sbj} {} {} \textsc{rel} Zoé cooks.\textsc{sbj}  \\
\glt `Whatever Zoé cooks, ...' 
\ex \Tree [.{CP} 
 [.{DP$_j$} quoi$_i$ 
 [.{$(\Delta)$} \edge[roof]; {que ce soit $t_i$} 
 ]
 ]
 {que Zoé cuisine $t_j$} 
 ]
\z

\z


The \textsc{fci}-based syntactic analysis that we put forth in this paper allows for a unified treatment of short and long CUs in French, reducing apparent intralinguistic variation to a matter of ellipsis. While the connection between \textsc{fci}s and (long) CUs has been pointed out previously in the literature \citep{muller2006,vlachou2007,corblin2010}, no other unified syntactic analysis of short and long CUs exists, as far as we know. Moreover, the only explicit syntactic account that has been previously put forth for long CUs \citep{corblin2010} faces some issues that our analysis resolves.

In this paper, we present novel empirical evidence for the involvement of \textsc{fci}s in short and long CUs. The same data also constitute evidence against a potential question-based analysis of these structures. While we are not aware of a question-based analysis of French CUs (following e.g. the work of \citealt{rawlins2013} on English), one of our main crosslinguistic contributions with respect to the literature on CUs is that such an analysis would indeed not be tenable for French. 



This paper is structured as follows. We begin in \sectref{sec:gonzalez:2} by presenting previous work on French CUs. In \sectref{sec:gonzalez:3}, we put forth a number of arguments in favor of our claim that French CU-adjuncts involve \textsc{fci}s, and should not be analyzed as questions. In \sectref{sec:gonzalez:4}, we provide evidence in favor of relativization within the CU-adjunct. \sectref{sec:gonzalez:5} spells out the details of our syntactic analysis, and \sectref{sec:gonzalez:6} goes over some predictions it leads to. \sectref{sec:gonzalez:7} concludes.

\section{Previous work}\label{sec:gonzalez:2}



The guiding insight of our paper is that both short and long CU-adjuncts contain a \textsc{fci}, which itself contains a \emph{wh}-word. In particular, the type of \textsc{fci} we focus on in this paper has the shape \emph{wh que ce soit}, and appears -- when properly licensed \citep[a.o.]{vlachou2007} -- in argumental positions, as in (\ref{ex:gonzalez:fci-arg}).\footnote{Like English \textit{any}, French \emph{wh que ce soit} items are both \textsc{fci}s and negative polarity items.} 


\ea\label{ex:gonzalez:fci-arg} \emph{Wh que ce soit} \textsc{fci} \\
 \gll Le dimanche, Max peut cuisiner [\capsub{\textsc{fci}} quoi que ce soit ]. \\
 the Sunday Max can cook.\textsc{inf} {} what \textsc{rel} it is.\textsc{sbj}  \\ %\jambox{[Long]}
 \glt `On Sundays, Max can cook whatever.' 
\z



The connection between \emph{wh que ce soit} \textsc{fci}s and CUs is also made in previous work on French, although with varying levels of specificity \citep{muller2006,vlachou2007, corblin2010}. \citet[][13]{muller2006} simply notes that \emph{wh que ce soit} \textsc{fci}s such as the one in (\ref{ex:gonzalez:fci-arg}) also form CU-adjunct clauses. \citet[][]{vlachou2007} mainly focuses on the licensing conditions of \emph{wh que ce soit} \textsc{fci}s, and only includes one example where such an item appears in a (long) CU-adjunct (\ref{ex:gonzalez:cu-vlachou}) \citep[][103]{vlachou2007}. In neither paper is the structure of the CU-adjunct analyzed or discussed in detail. 




\ea\label{ex:gonzalez:cu-vlachou} 
 \gll [\subscript{adjunct} Quoi que ce soit qu'il m'advienne ],  je le mets aussitôt au passé. \\
 {} what that it is.\textsc{sbj}  that-it me-happens.\textsc{sbj} {} I it put immediately {to the} past \\ %\jambox{[Long]}
 \glt `Whatever happens to me, I immediately consider it as belonging to the past.' 
\z

 

 \citet[][9]{corblin2010} proposes an explicit syntactic analysis for long CU-adjuncts after discussing the connection between \emph{wh que ce soit} \textsc{fci}s and CUs. The intention appears to be an analysis of long CU-adjuncts where the \emph{wh que ce soit} \textsc{fci} is modified by a relative clause (RC). Unfortunately, however, the syntactic implementation given in (\ref{ex:gonzalez:corblinsyntax}) does not seem to deliver this. Based on the indexing on \emph{quoi} and the empty category $e$, the analysis begins with the relativization of the (bare) \emph{wh}-word within the lower TP \emph{je dise}. We may therefore  assume that the RC \emph{que je dise} in  (\ref{ex:gonzalez:corblinsyntax}) should be attached to the NP \emph{quoi} directly. This whole phrase -- \emph{quoi que je dise} -- appears in the complement position of a copular structure.  Given that \emph{quoi} can never be pronounced in this position, the \emph{wh}-word is obligated to move upwards to a position above \emph{que}, which is assumed to be a normal complementizer (movement not shown in (\ref{ex:gonzalez:corblinsyntax})). 


\ea\citet[][9]{corblin2010}
\label{ex:gonzalez:corblinsyntax}
    \ea \gll [\subscript{adjunct} Quoi que ce soit que je dise ], ... \\
    {} what \textsc{comp} it is.\textsc{sbj} \textsc{rel} I say.\textsc{sbj} {} \\
    \glt `Whatever I say, ...' 
    \ex 
    \Tree 
    [.CP \emph{que}
    [.TP [.NP \emph{ce} ]
     [.V \emph{soit} ]
     [.NP \emph{quoi}$_i$ ]
    [.{RC} \emph{que}
    [.TP [.NP \emph{je} ]
     [.V \emph{dise} ] [.NP $e_i$ ]
    ]]]]
    \z 
\z 


Putting aside the small technical problems of this analysis (attachment site of RC, stipulation concerning the movement of \emph{quoi} to CP), we are still left with an analysis under which long CUs do not contain a \textsc{fci}. Indeed, in (\ref{ex:gonzalez:corblinsyntax}), there is no constituent that corresponds to \emph{quoi que ce soit}. Thus, this analysis does not  capture the key fact -- for which we will provide evidence in the next section -- that long CUs involve \textsc{fci}s in French. 


Thus, \citet{muller2006}, \citet{vlachou2007} and \citet{corblin2010} all make the connection between \emph{wh que ce soit} \textsc{fci}s and CUs in French. As mentioned in \sectref{sec:gonzalez:1}, however, English CU-adjuncts have been previously analyzed as questions \citep{rawlins2013}. Rawlins defends the question-based analysis of English CU-adjuncts against previous analyses that treat them as free relatives\footnote{An analysis based on free relatives is not tenable for French. While there is indeed some overlap, and both short CUs and free relatives can be headed by \emph{où} `where' in French (\ref{ex:gonzalez:french-fr-a}), free relatives cannot be headed by \emph{quoi} `what' (\ref{ex:gonzalez:french-fr-b}), requiring a free relative built with \emph{ce que} `that which' instead. Nevertheless, \emph{quoi} does appear in well-formed CU-adjuncts, as we have seen earlier. 

\ea 
    \ea \label{ex:gonzalez:french-fr-a}
    \gll Je veux aller $[$ où tu vas $]$. \\
    I want go.\textsc{inf} {} where you go \\
    \glt `I want to go where you go'
    \ex \label{ex:gonzalez:french-fr-b}
    \gll Je veux faire $[$ *quoi tu fais / ce que tu fais $]$. \\
    I want do.\textsc{inf} {} what you do {} it that you do \\
    \glt `I want to do what you do.'
    \z 
\z} \citep{dayal1997, izvorski2000} or as an ``intermediate" structure between interrogatives and free relatives \citep{gawron2001}. In the analysis, English CU-adjuncts are directly related to English \emph{wh-ever} questions (\ref{ex:gonzalez:whatever}) \citep[][148]{rawlins2013}.

\ea \label{ex:gonzalez:whatever}
    \ea Whatever happened to Alfonso? 
    \ex [\subscript{adjunct} Whatever happened to Alfonso ], ...
    \z
\z
 
 Rawlins presents three arguments in favor of a question-based syntactic analysis of English CU-adjuncts over previous analyses: compatibility with the \textit{what was X doing Y} idiom, replacement of an interrogative clause with a \emph{wh}-item, and acceptability of multiple \textit{wh}-items (see \citealt[][148--150]{rawlins2013} for  details). Following this line of work, one could also entertain an analysis where at least the short type of French CU-adjuncts is built from a question. The closest that we have found to such an analysis is \citet{kayne2010}, who briefly suggests that in French, short CUs are made up of a bare \emph{wh}-word followed by a silent EVER, i.e., a covert counterpart of the English \emph{wh}-attaching morpheme \emph{-ever} (\ref{ex:gonzalez:kayne}).
 

  

\ea \label{ex:gonzalez:kayne}
    \gll [\subscript{adjunct} Où EVER qu'il aille ], ... \\
    {} where {} that-he goes.\textsc{sbj}  \\
    \glt `Wherever he goes, ...' 
\z

As Kayne's focus is not on CUs, the structural details of (\ref{ex:gonzalez:kayne}) are not discussed further. It therefore remains unclear whether the aim is to analyze the structure in (\ref{ex:gonzalez:kayne}) on par with its English cousin -- i.e., as a question -- or not. If it is, then the evidence we provide in the next section can be taken as evidence against Kayne's idea. If it is not, a proper discussion of (\ref{ex:gonzalez:kayne}) would require a much better understanding of the analysis. 


In sum, this section has shown that the syntax of short and long CUs in French has not received much attention in the literature. In our analysis, we retain the main insight of previous work -- namely, that long CUs in French involve \textsc{fci}s. We further argue that short CUs also involve \textsc{fci}s. By relying on the process of ellipsis, our account allows for a unified analysis of short and long CUs in French. In addition, it avoids the problems faced by previous work. Before discussing our analysis in detail, we provide novel evidence for the involvement of \textsc{fci}s in French CUs in the next section.

  
%%%%%%%%%%%%%%%%%%%%%%%%%%%%%%%%%%%%%%%%%%%%%
%%%%%%%%%%%%%%%%%%%%%%%%%%%%%%%%%%%%%%%%%%%%%

\section{Arguments for \textsc{fci}s and against \emph{wh}-questions}\label{sec:gonzalez:3}

In this section, we present three novel arguments in favor of a unified, \textsc{fci}-based analysis of short and long CUs in French, and against a potential question-based analysis of the same structures (cf. \citealt{rawlins2013}).

\subsection{Matching paradigm gaps}

The first argument comes from an interesting match in paradigm gaps. Specifically, only those \emph{wh}-words that appear in \emph{wh que ce soit} \textsc{fci}s appear in short and long CUs. For example, (\ref{gap-ou}) shows that \emph{où} can occur in both a \textsc{fci} and  CU, and (\ref{gap-quand}) shows that \emph{quand} can appear in neither.

\ea \label{gap-ou}
\ea
\gll Zoé peut cuisiner $[$ \textbf{o\`u} que ce soit $]$.\\
 Zoé can cook.\textsc{inf} {} where \textsc{rel} it is.\textsc{sbj} \\ \jambox{[\textsc{fci}]}
\glt `Zoé can cook anywhere.'
\ex
\gll \textbf{O\`u} (que ce soit) que Zoé cuisine, ... \\
where \textsc{rel} it is.\textsc{sbj} \textsc{rel} Zoé cooks.\textsc{sbj} \\ \jambox{[CU]}
\glt `Wherever Zoé cooks, ...' 
\z
\z

\ea \label{gap-quand}
\ea [*] { 
\gll Zoé peut cuisiner $[$ \textbf{quand} que ce soit $]$.\\
 Zoé can cook.\textsc{inf} {} when \textsc{rel} it is.\textsc{sbj} \\ \jambox{[\textsc{fci}]}
\glt Int. `Zoé can cook anytime/whenever.' 
}
\ex [*] {
\gll \textbf{Quand} (que ce soit) que Zoé cuisine, ... \\
 when \textsc{rel} it is.\textsc{sbj} \textsc{rel} Zoé cooks.\textsc{sbj} \\ \jambox{[CU]}
\glt Int. `Whenever Zoé cooks, ...' 
}
\z

\z

\begin{table} 
\begin{tabular}{lll}
    \lsptoprule
    \textit{Wh}-word & \textsc{fci} & Short or long CU \\
    & $\_\_\_$ \textit{ que ce soit} & $\_\_\_$ \textit{(que ce soit) que...} \\
    \midrule
    \textit{quoi} ~`what'  & \langscicheckmark & \langscicheckmark \\
    \textit{qui} ~`who'  & \langscicheckmark & \langscicheckmark \\
    \textit{où} ~`where'  & \langscicheckmark & \langscicheckmark \\
    \textit{quand} ~`when'  & * & * \\
    \textit{comment} ~`how'  & * & * \\
    \textit{pourquoi} ~`why'  & * & * \\
    \textit{combien} ~`how much' & * & * \\
    \lspbottomrule
    \end{tabular}
    \caption{Matching paradigm gaps in \emph{wh que ce soit} \textsc{fci}s and CUs}
    \label{tab:gonzalez:table1}
\end{table}

\tabref{tab:gonzalez:table1} summarizes the  situation for all French \emph{wh}-words, and shows that \emph{wh que ce soit} \textsc{fci}s and CUs pattern exactly alike regarding the \emph{wh}-words they use. If short and long CUs in French involve \textsc{fci}s, as we argue in this paper, the ungrammaticality of certain \emph{wh que ce soit} \textsc{fci}s -- whatever its reason -- explains the ungrammaticality of the corresponding CUs (i.e. CUs that contain the same \emph{wh}-word). In contrast, an analysis of CUs based on \textit{wh}-questions would need additional assumptions to account for the data. Given that \emph{quand}, \emph{comment}, \emph{pourquoi} and \emph{combien} form \emph{wh}-questions in French, they would be expected to be able to form at least short CU-adjuncts under such an analysis.



\subsection{\emph{D'autre}}



The second argument for the involvement of \textsc{fci}s and against the involvement of \emph{wh}-questions comes from the position of the modifier \emph{d'autre} `else'. First, in French \emph{wh}-questions, a \emph{wh}-modifying \emph{d'autre} `else' is allowed to either move with the \emph{wh}-word (\ref{ex:gonzalez:autre-wh-a}) or be stranded in its base position (\ref{ex:gonzalez:autre-wh-b}).

\ea 
    \ea \label{ex:gonzalez:autre-wh-a}
    \gll \textbf{Quoi} \textbf{d'autre} {est-ce que} je devrais faire? \\
 what of-other \textsc{q} I should do.\textsc{inf} \\ \jambox{[Q]}
    \glt `What else should I do?' 
    \ex \label{ex:gonzalez:autre-wh-b}
    \gll \textbf{Qu'} {est-ce que} je devrais faire \textbf{d'autre}? \\
 what  \textsc{q} I should do.\textsc{inf} of-other \\ \jambox{[Q]}
    \glt `What else should I do?' 
    \z
\z

In contrast, inside \textsc{fci}s, \emph{d'autre} cannot appear adjacent to the \emph{wh}-part of the \textsc{fci} (\ref{ex:gonzalez:autre-fci-a}). It must instead appear at the very end of the \textsc{fci} (\ref{ex:gonzalez:autre-fci-b}).


\ea \label{ex:gonzalez:autre-fci}
\ea [*] {
\gll Tu peux manger $[$ \textbf{quoi} \textbf{d'autre} que ce soit $]$, mais pas ce gâteau. \label{ex:gonzalez:autre-fci-a}\\
you can eat.\textsc{inf} {} what of-other \textsc{rel} it is.\textsc{sbj} {} but not this cake \\ \jambox{[\textsc{fci}]}
}
\ex \label{ex:gonzalez:autre-fci-b}
\gll Tu peux manger $[$ \textbf{quoi} que ce soit \textbf{d'autre} $]$, mais pas ce gâteau. \\
you can eat.\textsc{inf} {} what \textsc{rel} it is.\textsc{sbj} of-other {} but not this cake \\ \jambox{[\textsc{fci}]}
\glt `You can eat anything else, but not this cake.'
\z 

\z

Crucially, both short and long CU-adjuncts pattern like \textsc{fci}s and not like \emph{wh}-questions with respect to \emph{d'autre}. They disallow a \emph{wh}-adjacent \emph{d'autre} (\ref{ex:gonzalez:autre-cu-a}), but allow \emph{d'autre} to appear stranded at the  end of the \textsc{fci} (\ref{ex:gonzalez:autre-cu-b}), or even at the very end of the CU-adjunct (\ref{ex:gonzalez:autre-cu-c}).\footnote{While we do not have the space to discuss the syntax of these modification structures any further, under our analysis, it is possible that \emph{d'autre} in (\ref{ex:gonzalez:autre-cu-b}) is stranded during the relativization of the \emph{wh}-word inside the \textsc{fci} (which means that it must be elided when \emph{que ce soit} is elided), while in (\ref{ex:gonzalez:autre-cu-c}), \emph{d'autre} is stranded during the relativization of the \textsc{fci}. We leave this issue for future work.} 

\ea 
    \ea[*] { 
    \gll \textbf{Quoi} \textbf{d'autre} {(que ce soit)} que tu cuisines, ... \label{ex:gonzalez:autre-cu-a}\\
    what of-other {\textsc{rel} it is.\textsc{sbj}} \textsc{rel} you cook.\textsc{sbj} \\ \jambox{[CU]}
    }
    \ex\label{ex:gonzalez:autre-cu-b}
    \gll  \textbf{Quoi} {que ce soit} \textbf{d'autre} que tu cuisines, ...\\
    what {\textsc{rel} it is.\textsc{sbj}} of-other \textsc{rel} you cook.\textsc{sbj} \\ \jambox{[CU]}
    \glt `Whatever else you cook, ...'
    \ex\label{ex:gonzalez:autre-cu-c}
    \gll  \textbf{Quoi} {que ce soit}  que tu cuisines \textbf{d'autre}, ... \\
    what {\textsc{rel} it is.\textsc{sbj}}  \textsc{rel} you cook.\textsc{sbj} of-other \\ \jambox{[CU]}
    \glt `Whatever else you cook, ...'
    \z
\z

Again, if CU-adjuncts were built from \emph{wh}-questions, the unacceptability of (\ref{ex:gonzalez:autre-cu-a}) would have to be explained. In contrast, on our analysis,  the ungrammaticality of (\ref{ex:gonzalez:autre-fci-a}) -- whatever its reason -- explains the ungrammaticality of the corresponding CU in (\ref{ex:gonzalez:autre-cu-a}).




\subsection{\emph{Diable}}

The final argument against \emph{wh}-syntax and for our \textsc{fci}-based analysis comes from another modifier, \emph{diable} lit. `devil', which plays the role of \emph{the hell} in French \emph{wh}-questions. First, (\ref{ex:gonzalez:diable-wh}) shows that French \emph{wh}-questions allow the \emph{wh}-word to be modified by \emph{diable} lit. `devil':

\ea \label{ex:gonzalez:diable-wh}
\ea 
\gll \textbf{Qui} \textbf{diable} êtes-vous? \\
 who devil are-you \\ \jambox{[Q]}
\glt `Who the hell are you?' 
\ex
\gll \textbf{Où} \textbf{diable} êtes-vous? \\
where devil are-you  \\ \jambox{[Q]}
\glt `Where the hell are you?' 
\z 

\z


As with \emph{d'autre} `else',  \textit{diable} cannot directly modify the \textit{wh}-part of a \textsc{fci} (\ref{ex:gonzalez:diable-fci}). Crucially, both short and long CU-adjuncts pattern like \textsc{fci}s, and not \emph{wh}-questions: neither allow the \textit{wh}-part to combine with \emph{diable}.


\ea
\ea [*] {
\gll \textbf{qui} \textbf{diable} que ce soit \label{ex:gonzalez:diable-fci}\\
 who devil \textsc{rel} it is.\textsc{sbj} \\ \jambox{[\textsc{fci}]}
 \glt Int. `whoever the hell', `anyone the hell'
}
\ex [*] {
\gll \textbf{Qui} \textbf{diable} {(que ce soit)} qui cuisine, ... \label{ex:gonzalez:diable-cu}\\
who devil {\textsc{rel} it is.\textsc{sbj}} \textsc{rel} cook.\textsc{sbj} \\ \jambox{[CU]}
\glt Int. `Whoever the hell cooks, ...' 
}
\z 

\z


Once again, both short and long CUs pattern together with \textsc{fci}s in that they disallow the modification of the \emph{wh}-word by \emph{diable} `devil'. 
Thus, just like the other arguments discussed in this section, the distribution of \emph{diable} `devil' suggests that CU-adjuncts should be analyzed in parallel with \textsc{fci}-phenomena, and not be given a question-based analysis.


To sum up, in this section, we have provided three types of arguments in favor of a unified, \textsc{fci}-based analysis of short and long CUs in French, and against a potential question-based analysis of said structures. We therefore align with previous literature on the topic in our reliance on free choice.



\section{Arguments for relativization}\label{sec:gonzalez:4}


Having argued that both short and long CU-adjuncts in French contain a \emph{wh que ce soit} \textsc{fci}, we now provide two novel arguments for the claim that in French CUs, this \textsc{fci} is relativized.



\subsection{Form of relative operator}

The first argument for the involvement of relativization in the syntax of French CU-adjuncts comes from the classic \emph{qui} vs. \emph{que} alternation of the relativizer (\textsc{rel}) in subject (a) vs. non-subject (b) headed relative clauses:


\ea \label{ex:gonzalez:RCop}%Standard headed RC
\ea
\gll la table $\{$ \textbf{qui}/*qu' $\}$ $\_$ est dans la cuisine \label{ex:gonzalez:RCop-a}\\
 the table  {} \textsc{rel} {} {} is in the kitchen \\ \jambox{[subject]}
\glt `the table that is in the kitchen' 
\ex  
\gll la table $\{$ \textbf{que}/*qui $\}$ Zoé a construite $\_$ \label{ex:gonzalez:RCop-b}\\
 the table {} \textsc{rel}  {} Zoé has built \\ \jambox{[object]}
\glt `the table that Zoé built'
\z 

\z

As (\ref{ex:gonzalez:RCop}) shows, relativization from subject position requires \textsc{rel} to take the form \emph{qui}, while relativization from non-subject positions -- such as the object position in (\ref{ex:gonzalez:RCop-b}) -- requires the form \emph{que}. Crucially, this same form alternation is present in CUs (\ref{ex:gonzalez:RCop-cu}), which strongly suggests that we are dealing with a relativization process within the CU-adjunct.


\ea \label{ex:gonzalez:RCop-cu}
\ea \gll Quoi (que ce soit) $\{$\textbf{qui}/*qu'$\}$ $\_$ fasse ce bruit, \\
 what \textsc{rel} it is.\textsc{sbj}  \textsc{rel}  {} does.\textsc{sbj} this noise \\ \jambox{[CU]}
\glt `Whatever is making this noise, ...' 
\ex
\gll Quoi (que ce soit) $\{$\textbf{que}/*qui$\}$  Zoé fasse $\_$ , ...\\
what \textsc{rel} it is.\textsc{sbj} \textsc{rel}  Zoé does.\textsc{sbj} \\ \jambox{[CU]}
\glt `Whatever Zoé does, ...' 
\z 

\z

\subsection{Free choice licensing}\label{sec:gonzalez:fcilicensing}

The second argument for our claim that French CUs involve the relativization of a \textsc{fci} comes from \textsc{fci} licensing. Specifically, \textsc{fci}s oftentimes appear in the presence of a modal and are usually not licensed in episodic statements (\ref{ex:gonzalez:fci-licensing}). However, \textsc{fci}s may be licensed in episodic statements if they are modified by a relative clause (\citealt{legrand1975}, \citealt{dayal1998}, a.o.), as shown in (\ref{ex:gonzalez:fci-subtrigging}).\footnote{Licensing by a relative clause has been called \textit{subtrigging} by \cite{legrand1975}.}

\ea \label{ex:gonzalez:fci-licensing}
\ea \gll Pour r\'eparer l'\'evier, je peux utiliser quoi que ce soit.\\
 to fix the-sink I can use what \textsc{rel} it is.\textsc{sbj}\\ \jambox{[\textsc{fci}]}
\glt ` I could use anything to fix the sink.'
\ex [*] {
\gll L'ann\'ee derni\`ere, elle a lu quoi que ce soit.\\
the-year last she has read what \textsc{rel} it is.\textsc{sbj}\\
}
\z 

\z

\ea \label{ex:gonzalez:fci-subtrigging}
\gll L'ann\'ee derni\`ere, elle a lu quoi que ce soit qui puisse \^etre pertinent pour sa th\`ese.\\
the-year last she has read what \textsc{rel} it is.\textsc{sbj} \textsc{rel} can.\textsc{sbj} be relevant for her dissertation\\ \jambox{[\textsc{fci}]}
\glt `Last year, she has read anything that could have been relevant for her dissertation.'
\z 

 Under our analysis, CU-adjuncts involve a \emph{wh que ce soit} \textsc{fci} that is relativized, and thus modified by a relative clause. This configuration is crucial for the licensing of the \textsc{fci} in the CU-adjunct. 

It is also noteworthy that relative clauses that license \textsc{fci}s have to occur in the subjunctive mood in many Romance languages.\footnote{See \citet{quer1998,quer2000}  on Catalan, and \citet{chierchia2013} on Italian.} Similarly, under our analysis, French CUs involve \textsc{fci}s modified by a subjunctive-mood relative clause.

\section{The proposal}\label{sec:gonzalez:5}

 In this section, we provide the details of our unified syntactic analysis of short and long CU-adjuncts in French. 

\subsection{A raising analysis of \textsc{fci}s and CUs}

We begin by repeating the general form of our analysis (repeated from (\ref{ex:gonzalez:tree})).

\ea 
\ea \gll [\textsubscript{adjunct} [\capsub{\textsc{fci}} Quoi$_i$ [$_{(\Delta)}$ que ce soit $t_i$ ]]$_j$ que  Zoé cuisine $t_j$ ], ... \\
 {} {} what {} \textsc{rel} it is.\textsc{sbj} {} {} \textsc{rel} Zoé cooks.\textsc{sbj}  \\
\glt `Whatever Zoé cooks, ...' 
\ex \Tree [.{CP} 
 [.{DP$_j$} quoi$_i$ 
 [.{$(\Delta)$} \edge[roof]; {que ce soit $t_i$} 
 ]
 ]
 {que Zoé cuisine $t_j$} 
 ]
\z

\z

More formally, we adopt the raising analysis of relativization for French \citep{kayne1994,bianchi1999, bianchi2000,bhatt2002}. Thus, both \emph{qui} and \emph{que} are glossed as \textsc{rel}: syntactically, they correspond to relative determiners (D).

The internal syntax of a \emph{wh que ce soit} \textsc{fci} is shown in (\ref{tree:gonzalez:tree-fci}).\footnote{At this point, we do not provide any strong independent arguments for our analysis of \emph{wh que ce soit} \textsc{fci}s. However, note that our analysis of CU-adjuncts has much the same form regardless of whether the inner syntax of the \textsc{fci} involves relativization. We leave it for further work to determine whether our preliminary analysis of \emph{wh que ce soit} \textsc{fci}s is correct.} First, the relative determiner D$_1$ (\emph{que}) selects a \emph{wh}-NP. This DP is merged in the complement position of a copular clause, from where it moves to  Spec,XP \citep{bianchi2000}. The \emph{wh}-NP selected by D$_1$ then moves further to Spec,CP. Finally, a second determiner D$_2$ is merged, and the whole structure is labeled as DP.\footnote{Within the \textsc{fci}, the form of \textsc{rel} remains constant -- it appears as \emph{que} -- which is not a surprise, as the \emph{wh}-containing DP is always relativized from the copular complement position.}





\ea\label{tree:gonzalez:tree-fci} 

\begin{tikzpicture}
  \tikzset{level 4/.style={sibling distance=2.5cm}, level 2/.style={sibling distance=3cm}}
  \Tree [.DP D$_2$ [.CP [.\node(np){NP$_j$}; \node(quoi){quoi}; ] [.C' C [.XP 
[.\node(DP){DP$_i$}; [.D$_1$ que ] \node(tracequoi){$t_j$}; ] 
[.X' X [.TP ce [ soit \node(tracewh){$t_i$}; ]]]]]]]

     %\draw[->] (tracewh.south) to [out=220, in=0] (DP.east);
     %\draw[->] (tracequoi.south) to [out=220, in=0] (np.east);
    \draw[->] (tracequoi.south)..controls +(south:6) and +(east:2)..(np.east);
    \draw[->] (tracewh.south)..controls +(south:3) and +(east:2)..(DP.east);
    
    \end{tikzpicture} 
\z

In CU-adjuncts (\ref{tree:gonzalez:tree-cu}), the \textsc{fci}-DP is itself selected by a relative D. This DP moves to Spec,XP, and the \textsc{fci} contained in the DP then moves to Spec,CP.


\newpage
\ea \label{tree:gonzalez:tree-cu}

\begin{tikzpicture} 
  \tikzset{level 3/.style={sibling distance=2cm}, level 1/.style={sibling distance=2cm}}
 \Tree [.CP [.\node(highDP){DP$_j$}; \edge[roof]; {quoi que ce soit} 
 ]
  [.C' C [.XP [.\node(lowDP){DP$_i$}; [.D que ] \node(hightrace){$t_j$}; ] 
  [.X' X [.TP \edge[roof]; \node(lowtrace){Zoé cuisine $t_i$};
 ]]]]]
  
    \draw[->] (hightrace.south)..controls +([xshift=-.6cm]south:6) and +([xshift=1cm]east:2)..(highDP.east);

    \draw[->] ([xshift=.95cm]lowtrace.south)..controls +([xshift=-.7cm]south:3) and +(east:1)..(lowDP.east);
  
 \end{tikzpicture} 
\z

The derivation of CU-adjuncts mirrors that of \emph{wh que ce soit} \textsc{fci}s almost entirely, as the trees in (\ref{tree:gonzalez:tree-fci}) and (\ref{tree:gonzalez:tree-cu}) show. The main structural difference between (\ref{tree:gonzalez:tree-fci}) and (\ref{tree:gonzalez:tree-cu}) is that the latter structure is not topped off with a second D, which means that the CU-adjunct is labeled as a CP, and not a DP. At this point, we do not provide independent evidence for the possibility of leaving out the highest D, and leave the task of justifying it for further work.\footnote{In this section, we presented our syntactic analysis of French CUs. It differs from that of \citet{corblin2010} mainly in that it contains a copular constituent corresponding to a \textsc{fci}. However, there is a third possible analysis of French CUs that we have not discussed, one based on clefted questions. On such an analysis, the CU-adjunct would start out as a \textit{wh}-question (\emph{Qu'est-ce que Zoé cuisine?}). The \emph{wh}-word would  first be clefted (\emph{C'est quoi que Zoé cuisine?}) and then fronted further, accompanied by the insertion of a high \emph{que}, producing \emph{Quoi que ce soit que Zoé cuisine} once mood is adjusted (\ref{ex:gonzalez:cleft-c}). Clefting being optional, short CUs would have the same underlying structure as long CUs, but only long CUs would involve clefting.

\ea
\gll [\capsub{CU-adjunct} \textbf{Quoi$_i$} que ce soit $t_i$ que Zoé cuisine $t_i$ $]$, ... \label{ex:gonzalez:cleft-c}\\
 {} what that it is.\textsc{sbj} {} that Zo\'e cooks.\textsc{sbj} {} {}\\
\glt `Whatever Zo\'e cooks, ...'
\z 

 Unfortunately, being question-based, this analysis would still not capture the data we presented in \sectref{sec:gonzalez:3}. For instance, while the \emph{wh}-word \emph{quand} can be used in clefted questions, it cannot occur in short or long CU-adjuncts. Moreover, this analysis would not capture the key fact that French CUs involve free choice items.}  








\subsection{Deriving short CUs through ellipsis}


Now that the syntactic structure of \emph{wh que ce soit} \textsc{fci}s and CU-adjuncts is in place, we are ready for the formal analysis of the difference between short and long CUs. We propose that the only syntactic difference between short and long CUs lies in whether the relative clause \emph{que ce soit} inside the \textsc{fci} has been elided (short CU) or not (long CU). 

Formally, we follow \cite{merchant2001} and assume that ellipsis requires some head X within the \textsc{fci} to carry the feature $[E]$. The complement of X is then elided. We propose that in French CUs, this X is C, and thus, XP is elided (\ref{tree:gonzalez:tree-ellipsis}).



\ea \label{tree:gonzalez:tree-ellipsis}
  \Tree [.CP
  [.DP$_l$ 
  [. D
  ]
  [.CP [.{NP$_j$} quoi ] 
  [.C' [.C\capsub{[E]} ]
  [.{XP$_{\Delta}$} \edge[roof]; {[que $t_j$]$_i$ ce soit $t_i$}
  ]]]]
  [.C' C
  [.{XP} \edge[roof]; {[que $t_l$]$_k$ Zo\'e cuisine $t_k$} ]
  ]
  ]
  
\z 


Thus, in short CU-adjuncts, the sole remnant of the ellipsis process is what appears to be a bare \emph{wh}-word (NP in Spec,CP). In some ways, then, the type of ellipsis we evoke for the derivation of short CU-adjuncts resembles sluicing. Sluicing is usually discussed in the context of \emph{wh}-questions (\ref{ex:gonzalez:sluicing}) \citep[a.o.]{ross1969, merchant2001}, where it deletes a TP and leaves a \emph{wh}-remnant:



\ea \label{ex:gonzalez:sluicing}
Zoé built something, but I don't know \textbf{what} {\tiny \sout{TP}.}
\z

 It is interesting to note that in some languages, sluicing can also target relative clauses \citep{liptakaboh2013}:

\ea
\gll Kòfí n\'a {yr\textopeno' m\`\textepsilon} \textrtaild\'e {\`am\'\textopeno} m\'a  ny\'\textopeno n {m\`\textepsilon}  \textrtaild\u{e} {w\`\textepsilon}  \sout{\tiny TP}.  \\
  K\`of\'i \textsc{fut} {call person}  \textsc{ind} but \textsc{1sg.neg} know  person \textsc{rel} \textsc{foc} \\ \jambox{[Gungbe]}
\glt lit. `Kofi  will call someone, but I don’t know the person who.' 
\z

 Moreover, in other languages (e.g., Spanish), sluicing can target copular structures \citep{vicente2008,vancranenbroeck2009,rodrigues2009}:

\ea
\gll Juan ha hablado com una chica pero no sé cuál \sout{es pro}.\\
Juan has talked with a girl but not know which {is it}\\
\glt lit. `Juan talked to a girl, but I don’t know which girl it was.'
\z


Thus, while we do not as of yet provide a detailed answer as to what type of ellipsis French  CU-adjuncts may undergo, there are indications that this process could be related to sluicing. Regardless of the exact subtype of ellipsis we are dealing with, the final question that we must answer concerns licensing. In the literature, it has been proposed that ellipsis can be licensed by both syntactic and semantic antecedents. Due to restrictions of space, we do not pursue this question further, and simply follow previous work on copular ellipsis and assume that it is licensed semantically \citep{rodrigues2009}. 


\section{Predictions}\label{sec:gonzalez:6}

Up to now, we have argued that French CU-adjuncts involve the relativization of a \emph{wh que ce soit} \textsc{fci}, which itself involves the relativization of a \emph{wh}-word. The difference between short and long versions is due to relative clause ellipsis. In this section, we explore two predictions that our analysis leads to.




\subsection{More \textsc{fci}s in CUs}

If French CU-adjuncts are derived through the relativization of \textsc{fci}s, we expect \textsc{fci}s other than \emph{wh que ce soit} to occur in CUs as well. The following examples show that this prediction is borne out: \emph{n'importe wh} \textsc{fci}s  \citep{muller2006} (\ref{ex:gonzalez:more-fcis-1}) and \emph{quelque N que ce soit} \textsc{fci}s (\ref{ex:gonzalez:more-fcis-2}) also appear in CU-adjuncts. 


\ea \label{ex:gonzalez:more-fcis-1}
\ea \gll Zoé peut cuisiner $[$ \textbf{n'importe} \textbf{quoi} $]$. \\
 Zoé can cook.\textsc{inf} {} \textsc{ne}-matters what \\ \jambox{[\textsc{fci}]}
\glt `Zoé can cook anything.' 
\ex
\gll  \textbf{N'importe} \textbf{quoi}  que Zoé cuisine, ...  \\
  \textsc{ne}-matters what \textsc{rel} Zoé cooks.\textsc{sbj} \\ \jambox{[CU]}
\glt `Whatever Zoé cooks, ...' \\
\z 

\z


\ea \label{ex:gonzalez:more-fcis-2}
\ea \gll  Zoé peut cuisiner $[$ \textbf{quelque} \textbf{plat}  que ce soit $]$. \\ 
Zoé can cook.\textsc{sbj}  {} some dish \textsc{rel} it is.\textsc{sbj} \\ \jambox{[\textsc{fci}]}
\glt `Zoé can cook any dish.'
\ex \gll \textbf{Quelque} \textbf{plat}  {(que ce soit)}  que Zoé cuisine, ... \label{ex:gonzalez:more-fcis-2b}\\
 some dish {~\textsc{rel} it is.\textsc{sbj}} \textsc{rel} Zoé cooks.\textsc{sbj}  \\ \jambox{[CU]}
\glt `Whichever dish Zoé cooks, ...'
\z 

\z
 
(\ref{ex:gonzalez:more-fcis-2b}) moreover shows that CUs headed by a \emph{quelque N que ce soit} \textsc{fci} can be short or long. This is accounted for by our analysis: the same ellipsis process that can delete the \emph{que ce soit} relative clause in CU-adjuncts headed by a \emph{wh que ce soit} \textsc{fci} can also delete the \textit{que ce soit} relative clause in (\ref{ex:gonzalez:more-fcis-2b}).


\subsection{More relative clause ellipsis}

Our analysis  predicts that relative clause ellipsis should be available elsewhere in French. (\ref{ex:gonzalez:more-ellipsis}) shows that this prediction is  borne out. If \emph{aucun linguiste} is underlyingly modified by the relative clause \emph{qui aime le hockey} which is then elided (as schematized in (\ref{ex:gonzalez:more-ellipsis-b})), the sentence in (\ref{ex:gonzalez:more-ellipsis-a}) should have the following interpretation: `I know three philosophers that love hockey, but no linguist that loves hockey.' In contrast, if (\ref{ex:gonzalez:more-ellipsis-a}) does not involve an elided relative clause, we expect it to be interpreted as `I know three philosophers that love hockey, but no linguist (at all).' For the speakers we consulted,  (\ref{ex:gonzalez:more-ellipsis-a}) can only have the former reading. This shows that this sentence has the structure in (\ref{ex:gonzalez:more-ellipsis-b}), thus providing independent evidence that relative clause ellipsis is available in French.

\ea \label{ex:gonzalez:more-ellipsis}
\ea \gll Je connais trois philosophes qui aiment le hockey, mais aucun linguiste. \label{ex:gonzalez:more-ellipsis-a}\\
 I know three philosophers \textsc{rel} love the {hockey} but no linguist \\
\glt `I know three philosophers that love hockey, but no linguist that loves hockey.'
\ex ..., mais [ [ aucun linguiste ]$_i$ [$_{\Delta}$ qui aime le hockey $t_i$ ]] \label{ex:gonzalez:more-ellipsis-b}
\z 

\z

Moreover, it even appears  that the ellipsis process that produces our short CUs may also take place within the second relative clause of the structure. Examples like (\ref{ex:gonzalez:cu-muller}) (from \citealt[][13]{muller2006}) illustrate this option: in the structure, the \emph{wh que ce soit} \textsc{fci} appears to form a CU-adjunct clause on its own, because the second relative clause \emph{qui veuille me voir} `who wants to see me' has been elided (with the antecedent being retrievable from the context).

\ea\label{ex:gonzalez:cu-muller} 
 \gll -- {Il y a} quelqu'un qui veut vous voir. -- [$_{}$ Qui que ce soit $]$, il attendra que j'aie terminé. \\
 {} {there is} someone \textsc{rel} wants you see.\textsc{inf} {} {} who \textsc{rel} it is.\textsc{sbj} {} he waits.\textsc{fut} that I-have.\textsc{sbj} finished \\ %\jambox{[Long]}
 \glt `-- There is someone to see you. -- Whoever it is, they will wait until I've  finished.' 
\z


In sum, the data that we have presented in this section clearly support our argument that short and long CUs in French involve a \textsc{fci}, and that the short type can be derived by eliding a part of the \textsc{fci}, allowing for a unified treatment of the two structures.






\section{Conclusion}\label{sec:gonzalez:7}
 In this paper, we provided novel empirical arguments in favor of a syntactic analysis of French CUs that is \textsc{fci}-based. In doing so, we argued against a potential question-based analysis of the structures. Moreover, our analysis is the first unified syntactic analysis of short and long CU-adjuncts in French. It reduces the apparent intralinguistic variation between short and long CUs to a matter of relative clause ellipsis. 

By showing that French CU-adjuncts cannot be analyzed as \emph{wh}-questions (as has been proposed for English by \citealt{rawlins2013}), our paper contributes to the ongoing debate concerning the nature of unconditionals crosslinguistically.  In particular, it provides more evidence for the claim that unconditionals can be formed using
various morphosyntactic and semantic ingredients across languages \citep{lohiniva2019, simik2018, szabolcsi2018, falausnicolae2020, balusu2019}.


%\section*{Abbreviations}
%\begin{tabularx}{.45\textwidth}{lQ}
%... & \\
%... & \\
%\end{tabularx}
%\begin{tabularx}{.45\textwidth}{lQ}
%... & \\
%... & \\
%\end{tabularx}

\section*{Acknowledgements}
Our names appear in alphabetical order. We would like to thank Anamaria F\u{a}l\u{a}u\c{s} and the audiences of the New York Philosophy of Language Workshop, Linguistic Symposium on Romance Languages 49, Sinn und Bedeutung 24, the Synsem seminar in Nantes, and CSSP 2019 in Paris for discussion and insightful comments. We would also like to thank Juliette Angot, Léna Baunaz, Adélie Crépin, Paloma Jereti\v{c}, Chloé Odet, Mélanie Claire Peak, Louise Raynaud, and Johan Rooryck  for sharing their  judgments with us at various points of this project. The second author would like to thank the Swiss National Science Foundation for their generous support of this work. 
%All mistakes are our own.

\printbibliography[heading=subbibliography,notkeyword=this]

%{\sloppy\printbibliography[heading=subbibliography,notkeyword=this]}
\end{document}
