\documentclass[output=paper,colorlinks,citecolor=brown,
% hidelinks,
% showindex
]{langscibook}
\author{Eugenia Mangialavori Rasia\affiliation{National Scientific and Technical Research Council (CONICET)} and Josep Ausensi\affiliation{Universitat Pompeu Fabra}}
\title{Deconstructing the conative alternation in Romance}
\abstract{In conative alternations, an alternate argument structure realization (object of transitive realized as a PP; e.g., \textit{John hit at the fence}) coincides with a relevant semantic shift (e.g.,, in telicity/resultativity drop). The conative alternation is generally seen with verbs whose meaning includes both motion and contact (\citealt{Guersseletal1985}; \citealt{Levin1993}), and is widely attested in various languages (e.g.,, English, Danish, German). Yet, it is generally believed to be unavailable in Romance. Here we draw on historical corpus data from Old Spanish, consonant with findings in Galician (\citealt{Miguez2016}), to argue that the conative alternation is possible and productive in Romance. According to data, conative alternation in Old Spanish allows verb classes similar to those commonly attested (across languages): i.e., cutting (e.g., \textit{cortar} ‘cut’), contact (e.g., \textit{dar} ‘hit’, \textit{disparar} ‘shoot’, \textit{morder} ‘bite’) and consumption verbs (e.g., \textit{beber} ‘drink’, \textit{comer} ‘eat’, \textit{picar} ‘nibble’). Further, the alternation in Old Spanish seems productive with verb classes believed to be excluded from/predicted not to allow conative alternation, like change-of-state (e.g., \textit{matar} ‘kill’, \textit{destrozar} ‘shatter’, \textit{destruir} ‘destroy’, \textit{romper} ‘break’) and pure contact verbs (e.g., \textit{tocar} ‘touch'). Thus, data from Old Spanish suggest that (i) the required motion plus contact denotation for verbs to alternate is not a necessary condition; (ii) further verb classes could be conative alternation-productive; (iii) widely-claimed restrictions on the conative alternation, linking meaning and syntactic behavior, could be actually language-specific.}

\IfFileExists{../localcommands.tex}{%hack to check whether this is being compiled as part of a collection or standalone
  \input{../localpackages}
  %Copy this to localcommands.tex

\usepackage[english]{babel}
\usepackage{amsmath}
\usepackage{amssymb,amsfonts,textcomp}
\usepackage{array}
\usepackage{hhline}
\usepackage{hyperref}

\newenvironment{styleStandard}{}{}
\newenvironment{stylelsAbstract}{}{}
\newenvironment{stylelsSectioni}{}{}
\newenvironment{stylelsSectionii}{}{}
\newenvironment{stylelsBulletList}{}{}
\newenvironment{styleBibliographyi}{}{}
\newenvironment{listWWNumxxvleveli}{}{}
\newenvironment{listWWNumxxvlevelii}{}{}
\newenvironment{listWWNumxxvleveliii}{}{}
\newenvironment{listWWNumxxvleveliv}{}{}
\newenvironment{listWWNumixleveli}{}{}
\newenvironment{listWWNumixlevelii}{}{}
\newenvironment{listWWNumixleveliii}{}{}
\newenvironment{listWWNumixleveliv}{}{}

\newcommand\textstyleListLabelxvi[1]{#1}
\newcommand\labellistWWNumxxvleveli{\thelistWWNumxxvleveli.}
\newcommand\labellistWWNumxxvlevelii{\thelistWWNumxxvlevelii.}
\newcommand\labellistWWNumxxvleveliii{\thelistWWNumxxvleveliii.}
\newcommand\labellistWWNumxxvleveliv{\thelistWWNumxxvleveliv.}
\newcommand\labellistWWNumixleveli{[F0B7?]}
\newcommand\labellistWWNumixlevelii{\textstyleListLabelxvi{o}}
\newcommand\labellistWWNumixleveliii{[F0A7?]}
\newcommand\labellistWWNumixleveliv{[F0B7?]}

\newcounter{listWWNumxxvleveli}
\newcounter{listWWNumxxvlevelii}[listWWNumxxvleveli]
\newcounter{listWWNumxxvleveliii}[listWWNumxxvlevelii]
\newcounter{listWWNumxxvleveliv}[listWWNumxxvleveliii]
\newcounter{itemize} 
  \input{../localhyphenation}
  \bibliography{localbibliography}
  \togglepaper[23]
}{}

\begin{document}
\maketitle

\section{Introduction}\label{sec:mangialavori:1}
Argument structure alternations are a major and fruitful research topic in linguistics. In the Conative Alternation (CA) (\citealt{Guersseletal1985}; \citealt{Laughren1988}; \citealt{Pinker1989}; \citealt{Levin1993}; \citealt{Goldberg1995}; \citealt{vanderLeek1996}; \citealt{Croft1998}; \citealt{Broccias2003}; \citealt{Beavers2006}, \textit{i.a.}) the object (in the transitive variant) is realized as a PP -- in English, usually headed by \textit{at} (but also \textit{on}, see \citealt{Levin1993}; \citealt{Beavers2006}) -- as in  \ref{ex:mangialavori:1}. Alternation is not trivial, as it carries a slight -- but grammatically-relevant -- change in meaning. In contrast to the transitive variant, the entailment that the action denoted by the verb was completed  \ref{ex:mangialavori:1} or successful  \ref{ex:mangialavori:2} (\citealt{Levin1993}) is not necessarily present. CA is also distinguished by an aspectual shift (most notably, in default (a)telicity), shown in  \ref{ex:mangialavori:3}--\ref{ex:mangialavori:4}.

\ea\label{ex:mangialavori:1}
  \ea The boy ate the fries, \#but didn’t eat them all.
  \ex The boy ate at the fries, but didn’t eat them all.
  \z 
\z 
\ea\label{ex:mangialavori:2}
  \ea The gangster shot the sheriff, \#but didn’t hit him.
  \ex The gangster shot at the sheriff, but didn’t hit him.
  \z 
\z 
\ea\label{ex:mangialavori:3}
  \ea The boy ate the fries in/?\#for an hour.
  \ex The boy ate at the fries for/\#in an hour.
  \z 
\z 
\ea\label{ex:mangialavori:4}
  \ea John shot the sheriff in/?\#for an hour.
  \ex John shot at the sheriff for/\#in an hour.
  \z 
\z 

Three specific observations follow. First, \citet{Levin1993} claims that the CA denotes an attempted action, therefore leaving the P complement unaffected (cf.  \ref{ex:mangialavori:2}). Nonetheless, as \citet{Broccias2003} (also \citealt{Beavers2006}) points out, at least for some verbs (e.g., \textit{eat} as in  \ref{ex:mangialavori:5}), the patient in the intransitive (P-marked) variant is nonetheless affected, at least, to some degree.\footnote{By \textit{patient} we merely refer to the entity denoted by the P-complement in intransitive variants or the entity denoted by the object in transitive variants.} The conative may hence diverge by entailing a lower degree of affectedness (vs. the transitive frame, cf. \citealt{Beavers2011,Beavers2013}).

\ea\label{ex:mangialavori:5}
  \ea[\#] {John ate (at) the fries but didn’t eat any.}
  \ex[\#] {John ate (at) the fries but left all of them on the plate.}
  \z 
\z 

Second, the CA is considered available for verbs whose meaning includes both motion and contact (\citealt{Guersseletal1985}; \citealt{Levin1993}; \citealt{Goldberg1995}; \citealt{vanderLeek1996}, \textit{i.a.}). Namely, the CA is productive with verbs of contact (e.g., \textit{hit}, \textit{bang}, \textit{bash}), verbs of cutting (e.g., \textit{chip}, \textit{cut}, \textit{hack}) and spray/load verbs (e.g., \textit{dab}, \textit{rub}, \textit{splash}). Not all verb classes are productive (\citealt{Levin1993}), at least in English. Change-of-state verbs  \ref{ex:mangialavori:6} (e.g., \textit{break}, \textit{kill}, \textit{destroy}) or touch verbs  \ref{ex:mangialavori:7} (e.g., \textit{touch}, \textit{kiss}, \textit{caress}) are not expected to allow the CA.

\ea\label{ex:mangialavori:6}
  \ea[] {John killed the enemies.}
  \ex[*] {John killed at the enemies.}
  \z 
\z 
\ea\label{ex:mangialavori:7}
  \ea[] {John touched the wall.}
  \ex[*] {John touched at the wall.}
  \z 
\z 

Third, CA appears in various languages apart from English, e.g., German (\citealt{FrenseandBennet1996}), Warlpiri (\citealt{Laughren1988}), Swedish (\citealt{Andersson1983}) or Danish (\citealt{Durst-AndersenandHerslund1996}). It is generally claimed to be unavailable across Romance.

\ea
\settowidth\jamwidth{(Spanish)}
  \ea[*]{
    \gll Juan golpeó en los hombres.\\
    Juan hit.\textsc{pfv}.\textsc{3sg} at the men\\ \jambox{(Spanish)}
    \glt Intended: `Juan hit at the men.'
    }
  \ex[*]{
    \gll Jean a mange à une pomme.\\
    Jean have.\textsc{prs}.\textsc{3sg} eat.\textsc{ptcp} at an apple\\\jambox{(French)}
    \glt Intended: `Juan has eaten at an apple.'
    }
  \ex[*]{
    \gll En Joan ha mossegat a la poma.\\
    the Joan have.\textsc{prs}.\textsc{3sg} bite.\textsc{ptcp} at the apple\\ \jambox{(Catalan)}
    \glt Intended: `Joan has bitten at the apple.'
    }
  \z 
\z 

% Old linguex example:
% \ex. \ag. *Juan golpeó en los hombres.\\
% Juan hit.\textsc{pfv}.\textsc{3sg} atthemen\\
% Intended: `Juan hit at the men.' \hfill \raisebox{1.9\baselineskip}[0pt][0pt]{(Spanish)}
% \bg. *Jean a mange à une pomme.\\ 
% Jean have.\textsc{prs}.\textsc{3sg} eat.\textsc{ptcp} at an apple.\\
% Intended: `Juan has eaten at an apple.' \hfill \raisebox{1.9\baselineskip}[0pt][0pt]{(French)}
% \bg. *En Joan ha mossegat a la poma.\\ 
% the Joan have.\textsc{prs}.\textsc{3sg} bite.\textsc{ptcp} at the apple\\
% Intended: `Joan has bitten at the apple.' \hfill \raisebox{1.9\baselineskip}[0pt][0pt]{(Catalan)}

Nonetheless, corpus data from Old Spanish (12th to 16th centuries) (hereafter, OSp) indicates CA is possible and productive in Romance grammar.\footnote{All data were extracted from three different historical corpora, namely the Old Spanish corpus developed in \citet{Sanchez-Marcoetal2010} (hereafter, SM), \textit{Corpus del Español} (\citealt{Davies2002}) (hereafter, CES), and \textit{Corpus Diacrónico del español} (\citealt{Corde}) (hereafter, CORDE).} Moreover, it involves further verb classes. We show that, in contrast to English, the CA in OSp is possible with change-of-state (e.g., \textit{matar} ‘kill’, \textit{destrozar} ‘destroy’, \textit{romper} ‘break’, \textit{derribar} ‘knock down’) and touch verbs (e.g., \textit{tocar} ‘touch’).\footnote{Compatible examples figure sporadically in the literature (\citealt{Cano-Aguilar1977}; \citealt{Folgar1988}, \textit{apud} \citealt{Garcia-Miguel2006}). Yet, a formal/systematic account is missing.} If correct, (i) data from OSp indicate that motion plus contact denotation may not be a necessary condition for the CA; (ii) this widely-claimed restriction on CA may be language-specific.

We proceed as follows. In \sectref{sec:mangialavori:2}, we describe verb classes that productively alternate in OSp. In \sectref{sec:mangialavori:3}, we analyze the properties of each verb class in detail. In \sectref{sec:mangialavori:4}, we develop the present analysis of conative variants in OSp. In \sectref{sec:mangialavori:5}, we argue against recent claims that certain cases in Modern Spanish involve true conatives. \sectref{sec:mangialavori:6} concludes the paper.

\section{The conative alternation in Old Spanish}\label{sec:mangialavori:2}
\subsection{Productive verb classes}\label{sec:mangialavori:2-1}

In OSp, the above appears with verb classes crosslinguistically associated with CA. Like English, OSp allows verbs of cutting (i.e., \textit{cortar} ‘cut’)  \ref{ex:mangialavori:10}, contact (i.e., \textit{disparar} ‘shoot’, \textit{dar} ‘hit’, \textit{ferir} ‘attack’, \textit{morder} ‘bite’)  \ref{ex:mangialavori:11}, and consumption  \ref{ex:mangialavori:12} (i.e., \textit{comer} ‘eat’, \textit{beber} ‘drink’, \textit{picar} ‘nibble’).

\ea
  \label{ex:mangialavori:10}
  \gll  Ni su espada cortava en las armas del caballero.\\
      Nor his sword cut.\textsc{ipfv}.\textsc{3sg} in the weapons of-the knight\\
  \glt  `Not even his sword cut at the arms of the knight.' (Melchor de Ortega, \textit{Felixmarte de Hircania}, 1556; CES)
\z

\ea\label{ex:mangialavori:11}
  \ea
    \gll Y comienço a cenar y morder en mis tripas.\\
    and start.\textsc{prs}.\textsc{1sg} to dine.\textsc{inf} and bite.\textsc{inf} in my guts\\
    \glt `And [I] start to dine and bite at my intestines.' (Anonymous, \textit{Lazarillo de Tormes}, 1517; SM)
  \ex
    \gll Y llegando a la Puerta el artillería disparó en los Moros.\\
    and arriving at the door the artillery shot.\textsc{pfv}.\textsc{3sg} in the moors\\
    \glt ‘Arriving at the door the artillery shot at the Moors.’ (Diego de Torres, \textit{Relación del origen y suceso de los Xarifes y del estado de los reinos de Marruecos, Fez y Tarudante}, 1575; CORDE)
  \ex
    \gll Los lobos seguramente dieron en las ouejas.\\
    the wolves probably hit.\textsc{pfv}.\textsc{3sg} in the sheep\\
    \glt ‘The wolves probably hit at the sheep.’ (Sánchez de Vercial, Clemente, \textit{Libro de los exemplos por A. B. C.}, 1400--1421; CORDE)
  \ex
    \gll La hueste d el conde hauia ferido en los turcos.\\
    the army of the count  have.\textsc{ipfv}.\textsc{3sg} hit.\textsc{ptcp} in the turks\\
    \glt ‘The army of the count had attacked/hit at the Turks.’ (Anonymous, \textit{Gran conquista de Ultramar}, 1295; SM)
  \z 
\z 

\ea\label{ex:mangialavori:12}
  \ea
    \gll El buitre cesso de comer en la molleja de tiero.\\
    the vulture stop.\textsc{pfv}.\textsc{3sg} of eat.\textsc{inf} in the gizzard of tero\\
    \glt ‘The vulture stopped eating at the gizzard of tero.’ (Anonymous, \textit{Coronación de Juan Mena}, 1499; SM)
  \ex
    \gll Despues que han comido van a beuer en algun agua clara.\\
    after that have.\textsc{prs}.\textsc{3pl} eat.\textsc{ptcp} go to drink in some water clear\\
    \glt ‘After they have eaten, they drink at some clear water.’ (Diego de Valera, \textit{Crónica de España}, 1482; CES)
  \z 
\z 


But also, as anticipated, CA in OSp seems more flexible. In principle, the pattern above is productive with causative verbs of change of state (i.e., \textit{matar} ‘kill’, \textit{destrozar} ‘destroy’, \textit{romper} ‘break’, \textit{derribar} ‘knock down’)  \ref{ex:mangialavori:COS} and (pure) contact (i.e., \textit{tocar} ‘touch’)  \ref{ex:mangialavori:tocar}.

\ea\label{ex:mangialavori:COS}
  \ea
    \gll Mato en ellos fasta que llego a la espannas.\\
    kill.\textsc{pfv}.\textsc{3sg} in them until that arrive.\textsc{pfv}.\textsc{3sg} at the Spain\\
    \glt Lit. `S/he killed at them until s/he reached the Spanish realms.' (Alfonso X, \textit{Estoria de España}, 1270; SM)
  \ex
    \gll Arrebatadamente destroçan en los griegos por duros.\\
    violently shatter.\textsc{prs}.\textsc{3pl} in the Greeks for hard\\
    \glt Lit. ‘They violently shatter at the Greeks as they are strong.’ (Anonymous, \textit{Crónica Troyana}, 1490; SM)
  \ex
    \gll Aquel emperador yendo destruyendo en aquella yent.\\
    that emperor going destroying in that people\\
    \glt Lit. ‘That emperor was destroying at that people.’ (Alfonso X, \textit{General Estoria. Cuarta parte}, 1280; CORDE)
  \ex
    \gll No se atrevio a romper en ellos porque los vio fuertes.\\
not \textsc{refl} dare.\textsc{pfv}.\textsc{3sg} to break in them because \textsc{acc}.\textsc{m}.\textsc{3pl} see.\textsc{pfv}.\textsc{3sg} strong\\
    \glt Lit. ‘He didn’t dare to break at them apart as he saw they were strong.’ (Francisco López de Gómara, \textit{Historia de la conquista de México}, 1538; CES)
  \ex
    \gll Fue feriendo y derribando en ellos fasta en la meatad de la Puente.\\
be.\textsc{pfv}.\textsc{3sg} hitting and knocking-down in them  until in the middle of the bridge\\
    \glt Lit. ‘S/he was hitting and knocking at them down up to the middle of the bridge.’ (Alfonso X, \textit{Estoria de España}, 1270; CES) 
  \z 
\z 

\ea\label{ex:mangialavori:tocar}
    \gll El rey de Portugal prohibio que tocassen en las rayzes.\\
the king of Portugal forbid.\textsc{pfv}.\textsc{3sg} that touch.\textsc{sbjv}.\textsc{prs}.\textsc{3pl} in the roots\\
    \glt Lit. ‘The king of Portugal forbade them to touch at the roots.’ (Juan Fragoso, \textit{Discvrsos de las cosas Aromaticas}, 1572; SM)
\z

Such verbs classes (i.e., pure contact, change-of-state verbs) are not expected to be productive in CA, basically because they lack motion plus contact components (\citealt{Guersseletal1985}; \citealt{Laughren1988}; \citealt{Levin1993}; \citealt{Goldberg1995}). The claim that verbs must denote motion and contact at the same time to enter CA is designed to predict that verbs such as \textit{move}, which do not denote contact, or \textit{touch}, which do not denote motion, do not allow CA.

\ea
  \ea[*] {John moved at the dog. (cf. John moved the dog)}
  \ex[*] {Tina touched at the table. (cf. Tina touched the table)}
  \z 
\z 

Moreover, OSp data suggest that causative verbs may allow causative/inchoative alternation and CA concurrently. This is assumed not to occur in natural languages (\citealt{Levin1993, Levin2017}). Nevertheless, in OSp, a prototypical change-of-state verb such as \textit{romper} ‘break’ occurs both in causative and conative alternations. The possibility of causative verbs allowing both causative/inchoative  \ref{ex:mangialavori:causative} and CA  \ref{ex:mangialavori:conative} thus becomes empirically relevant for various reasons.

\ea
  \ea\label{ex:mangialavori:causative}
    \gll Sus vestiduras non se \textbf{rompieron}, nin se enbeiecieron.\\
her clothes no \textsc{se} break.\textsc{pfv}.\textsc{3pl} nor \textsc{se} age.\textsc{pfv}.\textsc{3pl}\\
    \glt ‘Her clothes did not break, nor they aged.’ (Anonymous, \textit{Leyes nuevas}, 1255 - 1280; CORDE)
  \ex\label{ex:mangialavori:conative}
    \gll Anduvieron hasta ellos y \textbf{rompieron} \textbf{en} los moros y[...] los moros fueron vencidos.\\
walk.\textsc{pfv}.\textsc{3pl} until them and break.\textsc{pfv}.\textsc{3pl} in the moors and [...] the moors be.\textsc{pfv}.\textsc{3pl} beat.\textsc{ptcp}.\textsc{m}.\textsc{pl}\\
    \glt Lit. ‘They walked up to them and broke at them apart and the moors were defeated.’ (Anonymous, \textit{Relación circunstanciada de lo acaecido en la prisión del Rey Chico de Granada}, 1483 - 1500; CORDE)
  \z 
\z 


\subsection{Characteristic properties}\label{sec:mangialavori:2-2}

Five specific facts indicate that the present patterns can be arguably taken as CA instantiations, even if the PP is headed by a different spatial preposition (i.e., \textit{en} ‘in’, instead of \textit{a}, which is, potentially, a closer equivalent to English \textit{at}). First, OSp alternates crucially show the shift in aspectual type generally seen in CAs. Conative alternants of verbs of contact such as \textit{disparar} ‘shoot’ necessarily give semelfactive readings (cf. \textit{The gangster shot at the sheriff for hours}), yielding the denotation of an event repeated several times.  \ref{ex:mangialavori:semelfactive} hence expresses that the entity denoted by the subject shot iteratively at someone, which is not the default reading in transitive frames (cf. \textit{John shot Tom ??for hours/in 2 minutes}).

\ea\label{ex:mangialavori:semelfactive}
    \gll Empezáronse á allegar á ellos y á disparar en ellos.\\
 start.\textsc{pfv}.\textsc{3pl}=\textsc{refl} to arrive at them and to shoot.\textsc{inf} in them\\
    \glt ‘They started to get closer to them and to shoot at them.’ (Anonymous, \textit{Relación del Licenciado Pedro de la Gasca al Consejo de Indias sobre la campaña de pacificación del Perú}, 1548; CORDE) 
\z

Crucially, OSp P-marked variants also lack the expected result entailment. Consequently, the ‘attempted’ action and missed contact entailments typically ascribed to conatives are also possible in OSp. For instance,  \ref{ex:mangialavori:failedattempt} is understood as a description of a subject shooting with the intention to make the enemies move away (note the \textit{para que se apartasen} `so that they would move away'). Unlike the transitive variant,  \ref{ex:mangialavori:failedattempt} does not entail that the enemies were actually shot, much less killed, but merely targeted. This semantic flavor draws a thus far unnoticed contrast with the P distribution usually seen with \textit{disparar} in Old and Modern Spanish (cf. \textit{disparar a los enemigos} ‘shoot the enemies’) (see \sectref{sec:mangialavori:5}).

\ea\label{ex:mangialavori:failedattempt}
    \gll Ordeno el Emperador [...] disparasen en los enemigos para que  se apartasen.\\
order.\textsc{pfv}.\textsc{3sg} the Emperor [...] shoot.\textsc{sbjv}.\textsc{prs}.\textsc{3pl} in the enemies for that\textsc{refl}move-away.\textsc{sbjv}.\textsc{prs}.\textsc{3pl}\\
    \glt ‘The Emperor ordered them to shoot \#(at) the enemies so that they would move away.’ (Alonso de Santa Cruz, \textit{Crónica del Emperador Carlos V}, 1550; CORDE)
\z

In addition, OSp constructions at stake also yield the ‘incompletive’ interpretations generally seen in conatives (\citealt{Krifka1999}; \citealt{Beavers2006} on English). As \citet{Beavers2006, Beavers2011, Beavers2013} points out, in verb classes like verbs of consumption, conatives render a lower degree of affectedness of the patient. Hence, only a part of the entity denoted by the P complement must be consumed for the construction to hold (cf. \textit{John ate at his fries, but didn’t finish them all/\#but left all of them on the counter for Sam}). In this vein, OSp examples like  \ref{ex:mangialavori:incompleted} are interpreted as describing that an unspecified quantity of the entity denoted by the P complement undergoes the change of state encoded in the verb \textit{matar} ‘kill’, accordingly. This entailment would not be possible in the transitive frame. According to data, the entity denoted by the subject in the transitive would be interpreted as killing something -- i.e., the iterative reading expected from punctual verbs -- until reaching a motion boundary (consider \textit{\#John killed them until he arrived in Spain}\footnote{Avoiding, of course, interpretation involving a rotating set of enemies.}).

\ea\label{ex:mangialavori:incompleted}
    \gll Mato en ellos fasta que llego a la espannas.\\
kill.\textsc{pfv}.\textsc{3sg} in them until that arrive.\textsc{pfv}.\textsc{3sg} at the Spain\\
    \glt Lit. `S/he killed at them until s/he reached the Spanish realms.' (Alfonso X, \textit{Estoria de España}, 1270; SM)
\z

Similarly, the OSp CA also involves the expected change in telicity. The oblique variant behaves as atelic, as in English (recall  \ref{ex:mangialavori:3}--\ref{ex:mangialavori:4}). Consistently, verbs in the oblique frame frequently appear with modifiers only natural in atelic predicates. The cases of \textit{hasta} ‘until’ in  \ref{ex:mangialavori:hasta} (cf. an English equivalent as \textit{\#John killed them until the next day}), or \textit{quanto el dia les duró} ‘for as long as the day lasted’ in  \ref{ex:mangialavori:quanto} (cf. \textit{\#John killed them for as long as the day lasted}) are good examples of this. Similar observations are suggested by incompletive readings of the DP ( \ref{ex:mangialavori:300} below). 

\ea
  \ea \label{ex:mangialavori:hasta}
    \gll Començo a matar en ellos en guijsa que fasta la luz d el dia nunca descansaron.\\
start.\textsc{pfv}.\textsc{3sg} to kill.\textsc{inf} in them in way that until the light of the day never rest.\textsc{pfv}.\textsc{3pl}\\
    \glt Lit. ‘S/he started to kill at them in a way they did not rest until the next day.’ (Anonymous, \textit{Caída de princípes de Boccacio}, 1495; SM)
  \ex \label{ex:mangialavori:quanto}
    \gll Enrrique con los suyos hiriendo y matandoen ellos quanto el dia les duro.\\
Enrique with the his hurting and killing in them how-much theday\textsc{dat}.\textsc{3pl}last.\textsc{pfv}.\textsc{3sg}\\
    \glt Lit. ‘Enrique, with his team, hurting and killing at them for as long the day lasted.’ (Anonymous, \textit{Enrique fi de Oliva}, 1450; SM) 
  \z 
\z 

Finally, a certain Romance regularity further supports the argument for ‘true’ conatives in OSp. A parallel alternation is well attested in Galician. Crucially, the same P (i.e. \textit{in}) is involved.\footnote{\citet[117]{Miguez2016} argues alternation in Galician is less restrictive (compared to Germanic languages), as any durative verb is allowed. This is crucial in two ways: in both Galician and OSp, contrast between (in)transitive frames reflects (a)telicity shifts. More crucially, they show productivity patterns wider than those considered in mainstream (crosslanguage) literature.}

\ea
  \ea
    \gll O rato roeu o queixo.\\
the mouse nibble.\textsc{pfv}.\textsc{3sg} the cheese\\
    \glt ‘The mouse nibbled the cheese.’
  \ex
    \gll O rato roeu no queixo.\\
the mouse nibble.\textsc{pfv}.\textsc{3sg} in-the cheese\\\jambox{(Galician)}
    \glt ‘The mouse nibbled at the cheese.’ \\ 
    (examples from \citealt[169]{Miguez2016}) 
  \z 
\z 

\ea\label{ex:mangialavori:lechugas}
    \gll Si llega una gallina, pica en las lechugas; un tordo come los higos.\\
if arrive.\textsc{prs}.\textsc{3sg} a chicken nibble.\textsc{prs}.\textsc{3sg} in the lettuces a thrush eat.\textsc{prs}.\textsc{3sg} the figs\\
    \glt ‘If a chicken arrives, it nibbles at the lettuce; a thrush eats the figs.’ (Francisco Terrones del Caño, \textit{Instrucción de predicadores}, 1605; CORDE)
\z

Alternate (oblique/direct) realizations with verbs of consumption draw a semantic contrast that strictly reflects the aspectual contrast noted in English. The data above illustrate two verbs of consumption, \textit{picar} and \textit{comer}, in intransitive and transitive variants. Distinct entailments are involved; namely while in  \ref{ex:mangialavori:lechugas} \textit{las lechugas} in \textit{picar en} need not be completely affected (in the sense that they are not completely consumed, but simply \textit{nibbled at}, cf.  \ref{ex:mangialavori:5}), \textit{comer los higos} necessarily entails that the whole object is consumed (full/complete affectation, see \citealt{Beavers2011}).

\section{Verb classes with productive alternation}\label{sec:mangialavori:3}
\subsection{Verbs of contact}\label{sec:mangialavori:3-1}

Contact verbs in oblique variants show aspectual shifts similar to those seen in CA crosslinguistically (\citealt{Vendler1957}; \citealt{Dowty1979}). Namely, activities render semelfactive predicates. Prepositional variants describe an activity that is being repeated, also in OSp.

\ea
  \ea
    \gll Y comienço a cenar y morder en mis tripas.\\
and start.\textsc{prs}.\textsc{1sg} to dine and bite in my guts\\
    \glt ‘And I start to dine and bite at my intestines.’ (Anonymous, \textit{Lazarillo de Tormes}, 1517; SM)
  \ex
    \gll Yo [...] cortaba en aquellos cueros y nervios \textbf{tantos} que ya casi nada había quedado.\\
I [...] cut.\textsc{ipfv}.\textsc{1sg} in those hides and nerves many that already almost nothing have.\textsc{ipfv}.\textsc{3sg} remain.\textsc{ptcp}\\
    \glt ‘I cut at those hides and nerves so much that there remained almost nothing of that nervous shield.’ (Pedro Hernández de Villaumbrales, \textit{Peregrinación de la vida del hombre}, 1552; CES)
  \z 
\z 

Basic generalizations proposed for CA in English for verbs of contact seem to hold for OSp also, insofar as these OSp occurrences do not necessarily involve the result expected from the (lexically-coded) denotation of the verb. Instead, a repetitive reading arises. Significantly, this reading correctly combines with the ‘attempted’ reading widely reported in CA (e.g., \textit{John shot at the door but missed}; see \citealt{Levin1993}). To give one example, \textit{morder en} ‘bite at’ in  \ref{ex:mangialavori:morder} need not entail a properly accomplished event with ‘bitten bread’ as a result—i.e., a denotation necessarily entailing that contact was made with the bread (cf. \textit{John bit at the apple, but failed to bite it}). Thus, an entailment obtains that the action denoted by the verb has been initiated to no specific result and, most likely, has been repeated an indefinite number of times. In this sense,  \ref{ex:mangialavori:morder} reflects the pattern given by of verbs of contact (e.g., \textit{disparar en} in  \ref{ex:mangialavori:failedattempt} repeated below as  \ref{ex:mangialavori:failedattempt2}) seen above, which were also noted not to necessarily entail an accomplished event either. In both cases, the interpretation can be derived that contact need not obtain as a result of the event.

\ea
  \ea\label{ex:mangialavori:morder}
    \gll Con grande rabia buelve a morder en el pan.\\
with great rage return.\textsc{prs}.\textsc{3sg} to bite.\textsc{inf} in the bread\\
    \glt ‘With great rage s/he bites at the bread again.' (Jerónimo de Huerta, \textit{Traducción de los libros de Historia natural de los animales de Plinio}, 1599; CORDE) 
  \ex\label{ex:mangialavori:failedattempt2}
    \gll Ordeno el Emperador [...] disparasen en los enemigos para que  se apartasen.\\
order.\textsc{pfv}.\textsc{3sg} the Emperor [...] shoot.\textsc{sbjv}.\textsc{prs}.\textsc{3pl} in the enemies for that\textsc{refl}move-away.\textsc{sbjv}.\textsc{prs}.\textsc{3pl}\\
    \glt ‘The Emperor ordered them to shoot \#(at) the enemies so that they would move away.’ (Alonso de Santa Cruz, \textit{Crónica del Emperador Carlos V}, 1550; CORDE) 
  \z 
\z 

\subsection{Verbs of consumption}\label{sec:mangialavori:3-2}
As anticipated, a key characteristic of consumption verbs in the CA yields a distinct effect. In both (transitive and oblique) variants, the patient is (necessarily) affected. This sets a contrast with the effect seen in the verbs seen above, which differ by allowing a ‘missed contact’ reading. OSp constructions with consumption verbs -- according to data attested (thus far) with the verbs \textit{comer} ‘eat’, \textit{beber} ‘drink’, but also \textit{picar} ‘nibble’ -- , however, retain the telic-to-atelic shift seen in other cases as a hallmark of the CA. Notably, they render, by default, atelic predicates. In this respect, OSp data are in keeping the observation that conative alternates are set apart by their lack of result entailment and noncompletive interpretations (\citealt{Krifka1999}). The contrast widely reported in English (from \citealt[64]{Beavers2006}) in  \ref{ex:mangialavori:consumption} can be thus argued to hold for OSp, as in  \ref{ex:mangialavori:consumption2}.

\ea\label{ex:mangialavori:consumption}
  \ea The forlorn diner ate his sandwich. \jambox{(Sandwich totally eaten)}
  \ex The forlorn diner ate at his sandwich. \jambox{(Sandwich maybe not totally eaten)}
  \z 
\z 

\ea\label{ex:mangialavori:consumption2}
  \ea
    \gll El solo comiesse el pan.\\
He alone eat.\textsc{sbjv}.\textsc{prs}.\textsc{3sg} the bread\\\jambox{(Bread totally eaten)}
    \glt ‘He ate the bread by himself.’ (Clemente Sánchez de Vercial, \textit{Libro de los exemplos por A. B. C.}, 1400 - 1421; CORDE) 
  \ex
    \gll Desçiende a comer en las carnes puestas en la tierra.\\ 
descend.\textsc{prs}.\textsc{3sg} to eat.\textsc{inf} in the meats put.\textsc{ptcp}.\textsc{f}.\textsc{pl} on the ground\\\jambox{(Meat maybe not totally eaten)}
    \glt ‘It descends to eat at the meats on the ground.’ (Antonio de Villalpando, \textit{Razonamiento de las Reales Armas de los Católicos Reyes don Fernando y doña Isabel}, 1474 - 1500; CORDE)
  \z 
\z 

Also in OSp, oblique realizations of the DP would yield the interpretive shift from telic to atelic predications, thus reflecting the expected semantic contrast. This is in keeping with the (predicted) shift in argument structure realization (cf. \textit{John ate (\#at) the apple in a minute}). Significantly, all attested telic occurrences of consumption verbs in OSp do not involve the P associated with CA; instead, they instantiate standard transitives. In contrast, according to CES, CORDE and SM data, oblique frames yield patterns generally associated with atelic predicates. Namely, they consistently appear with \textit{cessar de} ‘stop’ but not with \textit{acabar de} ‘finish’, as illustrated above in  \ref{ex:mangialavori:12}, repeated below as  \ref{ex:mangialavori:tero}.\footnote{Bare plurals are also productive in OSp, e.g., \textit{Puede al principio començar a comer en passas sin granillos}. ‘At the beginning, you can start eating [at] seedless raisins.’ (Nicolás Monardes, \textit{Primera y Segvnda y Tercera Partes de la Historia Medicinal}, 1580; SM).} 

\ea\label{ex:mangialavori:tero}
    \gll El buitre cesso de comer en la molleja de tiero.\\
the vulture stop.\textsc{pfv}.\textsc{3sg} of eat.\textsc{inf} in the gizzard of tero\\
    \glt ‘The vulture stopped eating at the gizzard of tero.’ (Anonymous, \textit{Coronación de Juan Mena}, 1499; SM)
\z 


\subsection{Verbs of change of state}\label{sec:mangialavori:3-3}
In verbs of change of state, OSp data offer particularly promising findings. This class of verbs stands out in the analysis because it has been extensively argued to disallow the CA (specifically in English, cf. \textit{John broke (*at) the vase}). Nevertheless, a representative amount of change-of-state verbs such as \textit{romper} ‘break’ and \textit{matar} ‘kill’ productively alternate between transitive and oblique variants. Shifts in meaning (recall  \ref{ex:mangialavori:COS}) noted here are consistent. To begin with, P-marked variants are invariably atelic. This explains that \textit{matar en} ‘kill at’ can productively combine with modifiers like \textit{hasta} ‘until’, imposing an event boundary that is otherwise defective in the transitive frame (cf. \textit{??Los matamos hasta que nos cansamos} ‘??We killed them until we became tired.’). 

\ea
  \ea \label{ex:mangialavori:300}
    \gll Matamos en ellos fasta que queremos que somos  uengados.\\
kill.\textsc{prs}.\textsc{1pl} in them until that want.\textsc{prs}.\textsc{1pl} that be.\textsc{prs}.\textsc{1pl} avenge.\textsc{ptcp}.\textsc{m}.\textsc{pl}\\
    \glt Lit. ‘We kill at them until we fee like we are avenged.’ (Alfonso X, \textit{General Estoria. Segunda parte}, 1275; CORDE)
  \ex \label{ex:mangialavori:alcançava}
    \gll Mataua en ellos quantos alcançava.\\
kill.\textsc{ipfv}.\textsc{3sg} in them how-many reach.\textsc{ipfv}.\textsc{3sg}\\
    \glt Lit. ‘S/he killed at them as many as s/he could reach.’ (Alfonso X, \textit{General Estoria. Segunda parte}, 1275; CORDE) 
  \z 
\z 

Further, transitive realizations of the same verbs entail total affectedness, yielding telicity accordingly. P-marked variants, however, allow the entailment that even if all entities denoted by the P complement must be affected by the change of state lexically specified by the verb (cf. \textit{\#John killed them, but some of them survived}) also here, not all tokens in the set need to instantiate the expected result (i.e., a successfully accomplished event), but only a subset of them may. As a consequence, compatibility with modifiers like \textit{quantos alcançava} ‘as many as he could reach’ in  \ref{ex:mangialavori:alcançava} also becomes possible. Interestingly, however, other causative-alternating verbs allow an atelic reading paralleling the case of consumption verbs, as they consent an atelic reading in which all entities may have been affected, though not in their whole extension. Thus,  \ref{ex:mangialavori:almas} is interpreted as atelic as destruction affects all souls, but not necessarily integrally, i.e., destruction may only hold for a part of the souls (cf. \textit{John ate at the sandwich, but he only ate a tiny portion}).

\ea\label{ex:mangialavori:almas}
    \gll Este santísimo fuego destruye en las almas y consume lo que es mortal.\\
this most-holy fire destroy.\textsc{prs}.\textsc{3sg} in the souls and consume.\textsc{prs}.\textsc{3sg} itthat is mortal\\
    \glt Lit. ‘This holy fire destroys at the souls and consumes what is deadly.’ (Juan Boscán, \textit{Traducción de El cortesano de Baltasar de Castiglione}, 1534; CORDE) 
\z

Moreover, incompletive (vs. totally affected theme) readings also arise in the constructions at hand. While in  \ref{ex:mangialavori:32a} not all of the enemies are (assumed to be) killed (only those in the killers’ reach), in  \ref{ex:mangialavori:32b} nonresultativity follows as a consequence, as enemies were eventually corralled into the city (see \citealt{Kratzer2000} \textit{i.a.}, for similar crosslanguage data). Similar entailments arise in numerous examples. Notably, in constructions like \textit{y matando en ellos, los echaron fuera del pueblo} ‘and (by) killing (at) them, they kicked them out of town' (Hernan Cortés, \textit{Cartas de relación}, 1519--1526; CORDE), were noncompletive or irresultative interpretations are necessary. 

\ea
  \ea \label{ex:mangialavori:32a}
    \gll Cogieron los romanos en fuya matando en ellos quantos podien en so alcanço.\\
catch.\textsc{pfv}.\textsc{3pl} the Romans in run killing in them much be-able.\textsc{ipfv}.\textsc{3pl} in their reach\\
    \glt Lit. ‘They caught the Romans that were fleeing, killing at them as many as they could.’ (Alfonso X, \textit{General Estoria. Cuarta Parte}, 1280; CORDE) 
  \ex \label{ex:mangialavori:32b}
    \gll Herodes \& los ssuyos ffueron matando en ellos ffata quelos çeraron en la çibdat onde ssalieran.\\  
Herodes and the his go.\textsc{pfv}.\textsc{3pl} killing in them until that-\textsc{acc}.\textsc{m}.\textsc{3pl} close.\textsc{pfv}.\textsc{3pl} in the city where exit.\textsc{sbjv}.\textsc{ipfv}.\textsc{3pl}\\
    \glt Lit. ‘Herodes and his soldiers went killing at them until enclosing them in the city they came from.’ (Alfonso X, \textit{General Estoria. Sexta Parte}, 1284; SM) 
  \z 
\z 

Further specification in  \ref{ex:mangialavori:35} below -- marking both a part of the affected set and final state -- is quite telling in this respect, just as the occurrence of degree modifier \textit{tanto} `so much' in  \ref{ex:mangialavori:33}.

\ea\label{ex:mangialavori:33}
    \gll Los españoles cruelmente mataban en ellos tanto, que por muchas partes no se veía sino sangre.\\
the Spaniards cruelly kill.\textsc{ipfv}.\textsc{3pl} in them so-much that for many parts not \textsc{imp} see.\textsc{ipfv}.\textsc{3sg} but blood\\
    \glt Lit. ‘The Spaniards cruelly killed at them so much, that you could only see blood around.’ (Pedro Cieza de León, \textit{Crónica del Perú}, 1551; CES)
\z

Atelicity is also brought out by the distribution of aspectually-sensitive verbs like \textit{dejar} ‘stop’, \textit{cessar} ‘cease’, and incompatibility with \textit{terminar} ‘finish’, illustrated below. The data below are consistent with the (atelic) pattern noted in consumption verbs in conative variants (exemplified in  \ref{ex:mangialavori:tero} above). 

\ea
  \ea 
    \gll Alli dexaron los iudios de matar en ellos.\\
 there stop.\textsc{pfv}.\textsc{3pl} the jews of kill.\textsc{inf} in them\\
    \glt Lit. ‘In that location, the Jews stopped killing at them.’ (Alfonso X, \textit{General Estoria. Cuarta parte}, 1280; CORDE)
  \ex 
    \gll Sus enemigos [...] no dexaron de matar en ellos fasta la tarde.\\
 their enemies [...] not stop.\textsc{pfv}.\textsc{3pl} of kill.\textsc{inf} in them until the afternoon\\
    \glt Lit. ‘Their enemies would not stop killing at them until the afternoon.’ (Anonymous, \textit{Primaleón}, 1512; CORDE)
  \z 
\z 

In sum, such (e.g., incompletive) readings crucially depend on P-headed realization and are absent in the transitive form -- in fact, it suffices to drop the P for the sentence to become odd. As alternative (direct/oblique) expressions of the DP correlate with a contrast between telic and atelic uses of verbs (crosslinguistically, \citealt{vanHout1996}), argument structure realization arguably determines, also here, the (a)telic nature of the predicate. This is a distinctive property of CA. In addition, along with (non)resultativity and consistent (a)telicity, and also according to the examples above, the constructions at hand do not show the same change-of-state and resultative entailments typically seen in direct object variants (i.e., in causative alternation). Thus, in  \ref{ex:mangialavori:35}, the adjunct (i.e. \textit{morieron y muchos}) is natural, as the necessary implication that a large amount of people were affected is available, crucially, though, the result lexically specified by the verb does not apply to the complete set.\footnote{In the reading whereby, in  \ref{ex:mangialavori:300} above, the subject kept killing (at) a group of people until something/someone made him stop, thus preventing the completion of the task (partial set affected).} The action thus naturally leads to a nonspecific result.

\ea\label{ex:mangialavori:35}
    \gll E los moros que estauan en las torres començaron de apedrear \& matar en ellos \& morieron y muchos.\\
and the Moors that be.\textsc{ipfv}.\textsc{3pl} in the towers begin.\textsc{pfv}.\textsc{3pl} of stone.\textsc{inf} and kill.\textsc{inf} in them and die.\textsc{pfv}.\textsc{3pl} there many\\
    \glt Lit. ‘And the Moors that where in the towers started throwing stones and killing at them, and many (of them) died there.’ (Anonymous, \textit{Crónica de once reyes}, 1280; SM)
\z

A caveat, however, is in order. In OSp, change-of-state verbs participating in the CA still retain the change-of-state denotation insofar as the change of state specified by the verb (lexical root) must affect the entity denoted by the P complement (i.e. the ‘affected object’ here realized as target), even if not to completion. Thus, the P-marked variant of \textit{matar} gives an atelic/unbounded event interpretation as it leaves unspecified (i.e. it does not entail) whether all the entities denoted by the plural DP headed by P are killed (i.e., become dead) -- or, instead, just a set of them are. This is similar to what occurs with verbs of consumption in CA: while \textit{eat at} something does not entail that the whole argument is consumed, it nevertheless entails that at least a part of it must undergo the consumption event (cf. \textit{\#John ate at the fries but left them all untouched}). This is consistent with the well-known observation (\citealt{BeaversandFrancez2006}; \citealt{Beavers2011}) that there are subtle differences depending on verb class. Thus, with contact verbs, the object is necessarily impacted in the transitive variant (cf. \textit{\#John hit the table but didn’t hit it}) whereas in the oblique variant the entailment is absent (cf. \textit{John hit at the table but didn’t hit it}). With consumption verbs, the difference between transitive and P variants reflects a contrast between complete consumption (cf. \textit{\#John ate the fries but left some of them}) and noncompletive (hence atelic) readings (cf. \textit{John ate at the fries but left some of them}). In verbs of change of state, P-marked alternates also yield irresultativity in the sense of an event, initiated, lacking successful completion (see \citealt{Vincent2013}) (i.e., not all the entities (only a subset) denoted by the complement of P need be killed). The aspectual contrast between transitive and oblique variants is evidenced by the distribution of aspectually-sensitive modifiers and adjuncts, as shown above. In this way, important arguments for change-of-state-verbs-as-conative-allowing verbs analysis obtain, insofar as the relation between semantic and morphosynactic (oblique/prepositional) realization is consistent in this respect.

Based on evidence of this sort, we contend that verbs in OSp do alternate between causative and conative variants, contrary to what is assumed to be possible in natural languages. Further, we consider this pattern to be available for verbs believed to be excluded from the CA. Nonetheless, we will not discuss here whether verbs like \textit{matar} and \textit{destruir} are correctly classed as change of state -- or whether they should be rather (re)analized as target verbs in these contexts. The important point here is that a representative set of verbs appear in causative-inchoative and conative alternations at the same time. This poses a challenge (if not a constraint) on the widespread generalization that verbs denoting change of state do not allow CA. From the perspective of the Manner/Result Complementarity (\citealt{RappaportHovavAndLevin2010}), in turn, and the lexicalization constraint derived from it (\citealt{RappaportHovavAndLevin1998}; \citealt{RappaportHovav2008,RappaportHovav2014}; \citealt{Levin1999, Levin2017}, \textit{i.a.}), some verbs productive in PP frames in OSp class as ‘result’ verbs. Thus, another other widely-assumed generalization -- that ‘result’ verbs do not consent CA--could also lose predictive force.

\section{Essential analysis (recap)}\label{sec:mangialavori:4}
To account for the constructions at the table here, we would assume a light verb that merges with the verb to form a complex head. This complex would be responsible for licensing the PP complement deriving conative readings. On the one hand, a similar configuration has been proposed before for Spanish (\citealt{Basilico2010}).\footnote{ Crucially, however, the data proposed by \citet{Basilico2010} as CA select for a source DP (\textit{comer de} ‘eat from’), which is not the type of PP seen in conatives (target), crosslinguistically. The proposal thus involves empirical pitfalls, but different to those seen in Modern Spanish alleged data below, which features a trivial alternation instead (CA being nontrivial).} On the other hand, the proposal is compatible with the constructional analysis generally adopted. Yet, in our analysis, we take conatives to introduce a Target, rather than a Goal argument. Among other reasons, because conatives do not specify an endstate but merely an entity towards which the action is oriented. To do this, we follow previous analyses (see \citealt{Adams2001} for review) claiming that a Target argument can capture the fact that (i) contact is not (necessarily) made with in the conative; (ii) the trajectory/path does not (necessarily) end at the entity identified by the PP. This distinct meaning would be therefore brought about by the particular components that combine in the present frame.

An important challenge however comes with the novel data attested in OSp involving additional/yet unattested verb types (e.g., \textit{matar}, \textit{destruir}). To explain these facts, we would like to follow non-radical constructional proposals, like \citet{FolliAndHarley2005, FolliAndHarley2007}. We also consider that verbs entering the causative/inchoative alternation involve an external-argument-introducing head with the predicted properties (causative semantics, interpretive restrictions (probable cause)). Yet, we speculate that some causative verbs in OSp could also recruit/be compatible with an external-argument-introducing head of a different sort. Specifically, with the \textit{v} head associated with unergatives and verbs typically productive in conative frames. 

As we will argue below, such a constructional alternative would correctly capture distribution asymmetries in: (i) interpretation of the external argument (probable cause vs. volitional agent); (ii) aspectual effects ((ir)resultativity); along with (iii) structural (morphosyntactic) realization of the complement. Significantly, the prospect offered by a double (causative/conative) alternation seems to be in keeping with the behavior generally assumed in the literature. Namely, that the causative head—the one producing causative alternation—characteristically selects for a Small Clause complement corresponding to the endstate/result of the event. A different \textit{v} head associated with CA could explain that, in the examples at the table, the \textit{v} head takes a target (PP) complement instead. For some reason, oblique variants of causative verbs do not instantiate the selectional restrictions imposed by the \textit{v} head associated with the causative alternation (i.e., \textit{v}\textsubscript{\textsc{cause}}) according to the three criteria described above ((i)-(iii)), but rather those associated with the alternative head (i.e., \textit{v}\textsubscript{\textsc{do}}).\footnote{Yet, in \citet{FolliAndHarley2005, FolliAndHarley2007} there is no indication as to which verbs should fall within their analysis.} Nonetheless, as more (empirical) work is needed, it is yet unclear whether the analysis could be fruitfully extended to other (CA-allowing) verbs.

\subsection{Causative verbs and CA (finer specifications)}\label{sec:mangialavori:4-1}
Additional observations allowed by causative verbs point to facts relevant to the analysis. Significantly, OSp conatives differ in the realization of the external argument, instantiating a Cause-Agent opposition. Such an asymmetry is reminiscent of a distribution linked to distinct \textit{v} heads in the literature (\citealt{Chomsky1995}) consistent with asymmetric syntactic and semantic selectional and interpretive behavior. Several authors like \citet{FolliAndHarley2005,FolliAndHarley2007} propose that the little \textit{v} responsible for introducing the external argument (see \citealt{Kratzer1996} \textit{et seq.}) comes in two ‘flavors.’ Notably, \textit{v}\textsubscript{\textsc{do}} requires an animate agent subject, while the subject licensed by \textit{v}\textsubscript{\textsc{cause}} need not be, as it only requires to be interpretable as probable cause. 

Apparently, this constructional difference in realization of the external argument can also be appreciated in OSp with verbs like \textit{matar}. From this perspective, a natural explanation of why argument structure alternations generally believed to be incompatible (causative-inchoative and conative) are productive becomes possible. Significantly, and as we argue, selectional and interpretive contrasts standing between the two alternations are consistent with the properties ascribed to the configurations proposed: while the external argument in the causative only needs to be easily interpreted as (possible) Cause (cf.  \ref{ex:mangialavori:transitive} transitive and  \ref{ex:mangialavori:unaccusative} unaccusative variants in  \ref{ex:mangialavori:ponçoña}), the external argument of \textit{matar} in the CA  \ref{ex:mangialavori:37} requires an intentional agent. 

An important observation in this respect is that the same behavior appears across all data retrieved thus far.\footnote{In exhaustive searches, none of the attested oblique occurrence of \textit{matar} -- retrieved across (all of) the corpora cited here -- allows for nonagentive (nonvolitional) readings (i.e., cause-like as in  \ref{ex:mangialavori:ponçoña}), even in constructions with tacit subject.} Distribution of adverbials requiring intentionality/agentivity (vs. probable cause) (e.g., \textit{Hiriendo y matando en ellos a su placer} lit. ‘Hurting and killing at them with pleasure’ (Francisco Cervantes de Salazar, \textit{Crónica de la Nueva España}, 1560; CORDE), \textit{cruelmente} ‘cruelly’ in  \ref{ex:mangialavori:33}, etc.) is quite telling in this sense. 

\ea\label{ex:mangialavori:ponçoña} [\textit{v}P [DP\textsubscript{1} \textit{v}' [\textit{v}\textsubscript{\textsc{cause}} SC [DP\textsubscript{2} \textsc{res} $\surd$ ] ] ] ]\\
  \ea\label{ex:mangialavori:transitive}
    \gll La ponçoña de su maliçia lo mato.\\
the poison of his malice \textsc{acc}.\textsc{m}.\textsc{3sg} kill.\textsc{pfv}.\textsc{3sg}\\
    \glt ‘The poison of his malice killed him.’ (Anonymous, \textit{Exemplario por ABC}, 1493; SM) 
  \ex\label{ex:mangialavori:unaccusative}
    \gll Cuando estando con él se mató la luz y quedó a escuras.\\ 
when being with him \textsc{unacc} kill.\textsc{pfv}.\textsc{3sg} the light and became at dark\\
    \glt ‘Being with him, the light(s) died and he was left in the dark.’ (Luis de la Puente, \textit{Sentimientos espirituales}, 1589; CES) 
  \ex\label{ex:mangialavori:36a}
    \gll El fuego destruye las cosas calientes \& húmidas.\\
the fire destroy.\textsc{prs}.\textsc{3sg} the things hot and wet\\
    \glt ‘The fire destroys hot and wet things.’ (Anonymous, \textit{Libro de los caballos}, 1275; CORDE) 
  \z 
\z 

\ea\label{ex:mangialavori:37} [\textit{v}P [DP\textsubscript{1} \textit{v}' [\textit{v}\textsubscript{\textsc{do}} PP [\textsc{target} DP\textsubscript{2} ] ] ] ]\\
  \ea
    \gll Mato en ellos Julio cesar fasta que los destruxo.\\
kill.\textsc{pfv}.\textsc{3sg} in them Julius Cesar until that \textsc{acc}.\textsc{m}.\textsc{pl} destroy.\textsc{pfv}.\textsc{3sg}\\
    \glt Lit. ‘Julius Cesar killed at them until he destroyed them.’ (Alfonso X, \textit{Estoria de Espanna}, 1270; CORDE)
  \ex
    \gll Aquel emperador yendo destruyendo en aquella yent.\\
 that emperor going destroying in that people\\
    \glt Lit. ‘That emperor was going destroying at that people.’ (Alfonso X, \textit{General Estoria. Cuarta parte}, 1280; CORDE)
  \z 
\z 

Nonetheless, such an analysis picks up on a notion not unanimously shared by (constructionist) approaches. Specifically, it involves a minimal inventory of lexical heads (i.e. light verbs, in the sense of, e.g., \citealt{HaleAndKeyser2002}; \citealt{FolliAndHarley2005,FolliAndHarley2007}) with distinct semantic and syntactic properties. Still, an important advantage is that, under this sort of approach, semantic structure would not be \textit{ad liv}, but transparently reflect a grammatical (derivational) option; insofar as the semantic asymmetries just observed could be related to composition with distinct \textit{v} heads. In particular, it could capture the distribution drawn by restricted subject (agent) interpretations in the CA in OSp, in contrast to causative-alternation instances of the same verbs. Availability of two major argument structure alternations could be thus related to a choice of \textit{v}-head. 

In any event, co-occurrence of the noted argument structure alternations in causative verbs still raises major theoretical questions for any approach. Notably, for the widely-embraced notion that Manner/Result Complementarity (\citealt{RappaportHovavAndLevin2010}) correctly predicts CA (un)availability.\footnote{\citet{LevinAndRappaportHovav2013,LevinAndRappaportHovav2014} argue that result verbs (e.g., \textit{break}) consistently fail CA, while manner verbs (e.g., \textit{shoot}) allow it. CA has hence been taken as a diagnostic to tell manner and result verbs apart. In our view, OSp data challenge the generalization, as canonical result verbs (e.g., \textit{kill}, \textit{break}) -- which also happen to allow causative/inchoative alternation, like \textit{matar} -- have a productive CA.} In this connection, OSp could offer data suggesting currently untapped patterns bearing important theoretical ramifications.

\section{Conative variants in Modern Spanish}\label{sec:mangialavori:5}
Finally, we would like to consider some differences with Modern Spanish. Even if apparently contradictory, such considerations would rather support the proposal that the OSp constructions at the table can be arguably described as conatives.

In recent works on (Modern) Spanish, some authors (e.g., \citealt{Acevedo2011}) argue that Modern Spanish does show instances of true conatives by analyzing examples of the type in  \ref{ex:mangialavori:acevedo}. If correct, this would go against our current claims, namely that only in OSp but not in Modern Spanish, some verb classes can alternate between transitive and oblique (conative) variants.

\ea\label{ex:mangialavori:acevedo}
  \ea
    \gll El presidente golpeó a la mesa.\\
the president hit.\textsc{pfv}.\textsc{3sg} at the table\\
    \glt ‘The president hit (at) the table.’
  \ex
    \gll El balón golpeó al larguero.\\
 the ball hit.\textsc{pfv}.\textsc{3sg} at-the crossbar\\\jambox{(Modern Spanish)}
    \glt ‘The ball hit (at) the crossbar.’
  \z 
\z 

Yet, it is interesting to notice that the Modeern Spanish constructions here crucially lack the typical patterns expected from the CA -- which are correctly displayed above in OSp. First, these Modern Spanish alleged conative variants do not show missed contact entailments which are distinctive of the CA. 

\ea
    \gll El balón golpeó al larguero, \#pero no lo tocó.\\
 the ball hit.\textsc{pfv}.\textsc{3sg} at-the crossbar but no \textsc{acc}.\textsc{m}.\textsc{3sg} touch.\textsc{pfv}.\textsc{3sg}\\
    \glt ‘The ball hit at the crossbar, but didn’t hit it.’
\z 

Second, Modern Spanish intransitive variants do not show the expected aspectual shift between telic and atelic uses: in Modern Spanish, the PP variant is nevertheless telic. Similarly, it does not license either the entailment of repeated event (iterativity) seen in conatives crosslinguistically (e.g., hit at the ball for five minutes)—and which is seen, conversely, in the OSp construction (see  \ref{ex:mangialavori:semelfactive} above). 

\ea
    \gll El balón golpeó al larguero, \#por horas/en un segundo.\\
 the ball hit.\textsc{pfv}.\textsc{3sg} at-the crossbar for hours/in one second \\
    \glt ‘The ball hit at the crossbar for hours/in a second.’
\z 

The alternation in Modern Spanish is therefore trivial in relevant respects. 

\ea
    \gll Golpeó (a) la mesa, \#pero no le dio/\#no la tocó.\\
 hit.\textsc{pfv}.\textsc{3sg} (at) the table, but no \textsc{dat}.\textsc{sg} hit.\textsc{pfv}.\textsc{3sg}/ no \textsc{acc}.\textsc{f}.\textsc{3sg} touch.\textsc{pfv}.\textsc{3sg}\\
    \glt Intended: ‘S/he hit at the table, but missed.’ (vs. He hit the table, \#but missed)
\z 

Finally, in Modern Spanish, alternation is possible in significantly few verbs. Most importantly, verb classes typically allowing CA across languages (e.g., verbs of consumption, verbs of cutting) fail to alternate in Modern Spanish. By contrast, quite crucially, these classes are fully productive in OSp.\footnote{A caveat on the P itself is necessary. As known, in Spanish, the preposition \textit{a} ‘at’ is also used for Differential Object Marking (thus raising potential questions on the analysis). Yet, two completely different matters, semantically, and syntactically, are at stake. In Differential Object Marking, the morphosyntactic expression of the DP alternates between null and prepositional marking of case (accusative/dative/genitive), depending on variables such as animacy or definiteness of the NP. By contrast, the conative alternation depends on alternative syntactic (configurational) realizations of the (direct/oblique) patient. More crucially, in the CA, the syntactic alternation correlates with alternative structural semantic structures (theme/target), consequent with a nontrivial aspectual shift (telicity drop). Differential Object Marking is, conversely, not accompanied by systematic changes in meaning (as seen in CA); P does not centrally contribute to event aspect.}

\ea
  \ea
    \gll Juan comió / bebió / cortó(*a / *en)la manzana / sopa.\\
Juan eat.\textsc{pfv}.\textsc{3sg} / drink.\textsc{pfv}.\textsc{3sg} / cut.\textsc{pfv}.\textsc{3sg} at / in the apple / soup\\
    \glt Intended: `Juan ate/drank/cut at the apple/soup.'
    
  \ex
    \gll Juan mató / destruyó (*a / *en) los perros / los enemigos.\\
Juan kill.\textsc{pfv}.\textsc{3sg} / destroy.\textsc{pfv}.\textsc{3sg} at / in the dog / the enemies\\
    \glt Intended: `Juan killed/destroyed at the dogs/enemies.'
  \z 
\z 

Facts lead us to conclude that the Modern Spanish PP constructions in question do not represent standard conative instances, at least according to these observations and the empirical data behind them. In opposition, expected contrasts can be noted between OSp and varieties like Galician, thus establishing a certain productive regularity deserving to be considered. In short, even if the P used in OSp is different from the one generally recruited in the CA (e.g., English \textit{at}), we can nonetheless draw on data from two Romance varieties indicating the event-type shifts typical of CAs (\citealt{vanHout1996}, \textit{i.a}.). In fact, there is no reason to expect identical morpho-lexical-syntactic realization across languages, especially as argument against CA analysis of such constructions. Actually, it would be contrary to fact (\citealt{Kiparsky1988}, \textit{i.a}.).

\section{Conclusion}\label{sec:mangialavori:6}
Drawing on data from OSp, we argue that CA is productive in Romance grammars. Although OSp is similar concerning verb classes crosslinguistically allowed in CA (i.e., verbs of contact, cutting, consumption), CA in OSp allows causative change-of-state verbs, posing a challenge for the long-argued restriction that change-of-state verbs of the \textit{break} or \textit{kill} sort never participate in the CA (concurrent Causative/Conative alternation). 

Data from OSp shows that the required motion and contact denotation for verbs to be able to participate in the CA is not a necessary condition, or rather, that the long-claimed restriction of motion and contact is language specific, as it may only hold for English but not for Romance. Further, such data challenge the so-called Manner/Result complementarity, since such a complementarity is argued to predict CA (un)availability, assuming result-type verbs should fail to allow CA (e.g., \textit{break}, \textit{kill}), and only manner-type verbs alternate in this respect (e.g. \textit{hit}, \textit{shoot}).

Finally, we argue that Modern Spanish does not show instances of true conatives, despite recent claims in the literature. Among other things, Modern Spanish intransitive variants do not allow the prototypical shifts in meaning and aspectual structure typically associated with the CA, effectively seen in OSp. 


\section*{Acknowledgements}
We are grateful to two anonymous reviewers for the valuable comments they provided. We are also grateful to the discussions held at the following events where this paper was presented: \textit{Anglia Ruskin – Cambridge Romance Linguistics Seminars}, \textit{The 49th Linguistic Symposium on Romance Languages}, \textit{The European Dialect Syntax Workshop IX}, and \textit{Seminars of the Centre de Lingüística Teòrica} at Universitat Autònoma de Barcelona. Any remaining errors are our own.

The second author acknowledges support from the project Connecting Conceptual and Referential Models of Meaning 2 (CONNECT 2) (PI: Louise McNally) from the Spanish Ministerio de Economía y Competitividad (FFI2016-76045-P) (AEI/MINEICO/FEDER, UE).

\printbibliography[heading=subbibliography,notkeyword=this]

\end{document}
