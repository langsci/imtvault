\addchap{Preface}
\begin{refsection}

This book is a revised version of my PhD Thesis,
submitted August 2016 awarded July 2017.
The analyses in this work remain broadly the same as those in my PhD thesis,
but several areas have undergone substantial revision.

The most significant analytical revision
is in the analysis of phonologically conditioned
metathesis (Chapter \ref{ch:PhoMet}).
This revision is informed by a proper analysis of
the prosodic structures of Amarasi including
the Prosodic Word (\srf{sec:ProsStr}).
A proper understanding and description
of Amarasi prosodic structures was mostly lacking in the earlier work.

Structurally, chapter \ref{ch:SynchMet}
has been significantly reorganised on the basis
of comments from the reviewers in order to clarify
the distinction between the three types of metathesis
I identify, as well as to avoid shoe-horning languages
with multiple kinds of metathesis into a single category.

I have also included a more proper discussion
of alternate analyses of the structure of Amarasi
metathesis (\srf{sec:AltApp}), as well
as syntactic metathesis (\srf{sec:ProsMet}).

Since the submission of my PhD thesis much more data
on other varieties of Meto has also become available.
In addition to including such data where it is relevant,
it has also informed my analysis of Amarasi in several important respects.
Thus, for instance, vowel initial enclitics which were mostly represented
as monosyllables in my earlier work are now properly
represented as containing two vowels (\srf{sec:HisDev}).

Of particular prominence is the additional data
on Ro{\Q}is Amarasi (the other major dialect of Amarasi)
for which I only had limited preliminary data for my PhD thesis.
I have included several sections discussing
the Ro{\Q}is data as it has several important
differences to that of Kotos Amarasi
(e.g. \srf{sec:RoqAnaCCIniMod}, \srf{sec:DisDriMetRoqAma}).

Finally, numerous errors have been corrected.
These include typographical errors
and minor errors in the actual data.

%\printbibliography[heading=subbibliography]
\end{refsection}

