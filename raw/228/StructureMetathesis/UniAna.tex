\section{Unified analysis}\label{sec:UniAna}
A number of surface phonological operations
occur to derive the M\=/form in Amarasi.
Such phonological processes include metathesis,
consonant deletion, and assimilation of /a/.
Furthermore, metathesis itself can trigger further
processes of consonant deletion, vowel deletion,
and vowel height assimilation.

\begin{table}[h]
	\caption{Amarasi surface basic M-forms}\label{tab:AmaSurMfor}
	\centering
		\stl{0.38em}\begin{tabular}{llcl|lclll}\lsptoprule
				&		U\=/form																	&		&M\=/form											&\mc{2}{l}{U\=/form}&\mc{2}{l}{M\=/form}		&processes\\ \midrule
			1.&	V\sub{1}C\sub{1}V\sub{2}									&\ra&V\sub{1}V\sub{2}C\sub{1}		&\ve{fatu}	&\ra&\ve{faut}	&`pig'		&metathesis\\
			2.& V\sub{1}C\sub{1}V\sub{2}C\sub{2}					&\ra&V\sub{1}V\sub{2}C\sub{1}		&\ve{muʔit}	&\ra&\ve{muiʔ}	&`animal'	&{\&} C deletion\\
			3.& V\sub{1}\sub{\tsc{hi}}C\sub{1}V\sub{2}\sub{\tsc{mid}}
																										&\ra&V\sub{1}V\sub{2}\sub{\tsc{hi}}C\sub{1}	
																																										&\ve{ume}		&\ra&\ve{uim}		&`house'	&{\&} height ass.\\
			4.& V\sub{1}{\sub{α}}C\sub{1}a(C\sub{2})					&\ra&V\sub{1}{\sub{α}}V{\sub{α}}C\sub{1}&\ve{nima}	&\ra&\ve{niim}	&`five'		&{\&} /a/ ass.\\
			5.& V\sub{1}V\sub{2}C\sub{1}V\sub{3}					&\ra&V\sub{1}V\sub{2}C\sub{1}		&\ve{aunu}	&\ra&\ve{aun}		&`spear'	&V deletion\\
			6.& V\sub{1}V\sub{2}C\sub{1}V\sub{3}C\sub{2}	&\ra&V\sub{1}V\sub{2}C\sub{1}		&\ve{nautus}&\ra&\ve{naut}	&`beetle'	&V deletion\\
			7.& V\sub{1}V\sub{2}C\sub{1}									&\ra&V\sub{1}V\sub{2}						&\ve{kaut}	&\ra&\ve{kau}		&`papaya'	&C deletion\\
			8.& V\sub{1}V\sub{2}													&\ra&V\sub{1}V\sub{2}						&\ve{ai}		&\ra&\ve{ai}		&`fire'		&\\
		\lspbottomrule
		\end{tabular}
\end{table}

\largerpage
Which operations apply to a word 
is determined by the phonotactic structure of that word,
as well as the quality of the vowels it contains.
The different structures of the M\=/form
are summarised in Table \ref{tab:AmaSurMfor}.
With the exception of M\=/forms with a double vowel (\srf{sec:DouVow}),
all M\=/forms are phonetically disyllabic.

The M\=/form must be derived from the U\=/form
as there is a large amount of ambiguity among M\=/forms.
For instance, given an M\=/form with the shape VVC{\#},
we cannot predict whether the U\=/form will have a shape
corresponding to any of shapes 1--6 in Table \ref{tab:AmaSurMfor}.
A concrete example is the form \ve{n-neen},
which is the M\=/form of both
\ve{n-nene} `pushes' and \ve{n-nena} `hears'.

In this section I propose an analysis of all the different M\=/forms.
I use an autosegmental model of phonology \citep{go76}
and a rule-based model of process morphology \citep{ma74,an92}.
Adopting these models allows me to formulate a single, unified
analysis of the diverse processes which occur in the formation of Amarasi M\=/forms.
In \srf{sec:AltApp} I discuss alternate analyses
which I propose cannot account for all the Amarasi data.

My analysis consists of a single process of metathesis at the CV tier
and an associated morphemically conditioned process (/a/ assimilation).
These processes, combined with an obligatory CVCVC foot structure
and the general phonotactic constraints of Amarasi,
generate all the different M\=/forms.

In my autosegmental diagrams in the following sections
empty C-slots are occasionally `filled' with {\0}
in order to make it explicit that they behave identically to filled C-slots.
This is a notational convenience.
Similarly, the x-tier (or timing tier) is used as a notational device to
illustrate clearly the effect of metathesis.
Use of the x-tier should not be taken as a claim about its theoretical status.

\subsection{Obligatory CVCVC Foot}\label{sec:ThePhoRul}
I posit that the Amarasi foot obligatorily has the structure CVCVC
and that C-slots may be empty.
This stipulation is given in \qf{ex2:Ft->CVCVC} below,
and has already been discussed in \srf{sec:TheFoo}.
Extensive evidence for the existence
of empty C-slots in Amarasi has been given in \srf{sec:EmpCSlo}.
This foot structure applies to all words

\begin{exe}
	\ex{Ft {\ra} CVCVC}\label{ex2:Ft->CVCVC}
\end{exe}

The structures of the words \ve{fafi} `pig', \ve{muʔit} `animal', \ve{kaut} `papaya',
\ve{ai} `fire', \ve{nautus} `beetle' and \ve{aunu} `spear' under this analysis
are given in \qf{as:fafi}--\qf{as:nautus} below.
The initial C-slots of the words \ve{ai} `fire' and \ve{aunu} `spear'
have been filled with an automatic glottal stop,
as is the case for all vowel-initial words (\srf{sec:GloStoIns}).

\begin{multicols}{3}
	\begin{exe}
		\exa{\label{as:fafi}\xy
			<0pt,1cm>*\as{C}="C1",<1em,1cm>*\as{V}="V1",<2em,1cm>*\as{C}="C2",<3em,1cm>*\as{V}="V2",<4em,1cm>*\as{C}="C3",
			<0pt,0pt>*\as{f}="c1",<1em,0pt>*\as{a}="v1",<2em,0pt>*\as{f}="c2",<3em,0pt>*\as{i}="v2",
			"c1"+U;"C1"+D**\dir{-};"c2"+U;"C2"+D**\dir{-};"v1"+U;"V1"+D**\dir{-};"v2"+U;"V2"+D**\dir{-};
		\endxy}
		\exa{\xy
			<0pt,1cm>*\as{C}="C1",<1em,1cm>*\as{V}="V1",<2em,1cm>*\as{C}="C2",<3em,1cm>*\as{V}="V2",<4em,1cm>*\as{C}="C3",
			<0pt,0pt>*\as{m}="c1",<1em,0pt>*\as{u}="v1",<2em,0pt>*\as{ʔ}="c2",<3em,0pt>*\as{i}="v2",<4em,0pt>*\as{t}="c3",
			"c1"+U;"C1"+D**\dir{-};"c2"+U;"C2"+D**\dir{-};"c3"+U;"C3"+D**\dir{-};"v1"+U;"V1"+D**\dir{-};"v2"+U;"V2"+D**\dir{-};
		\endxy}
		\exa{\xy
			<0pt,1cm>*\as{C}="C1",<1em,1cm>*\as{V}="V1",<2em,1cm>*\as{C}="C2",<3em,1cm>*\as{V}="V2",<4em,1cm>*\as{C}="C3",
			<0pt,0pt>*\as{k}="c1",<1em,0pt>*\as{a}="v1",<3em,0pt>*\as{u}="v2",<4em,0pt>*\as{t}="c3",
			"c1"+U;"C1"+D**\dir{-};"c3"+U;"C3"+D**\dir{-};"v1"+U;"V1"+D**\dir{-};"v2"+U;"V2"+D**\dir{-};
		\endxy}
	\end{exe}
\end{multicols}
\begin{multicols}{3}
	\begin{exe}
		\exa{\xy
			<0pt,1cm>*\as{C}="C1",<1em,1cm>*\as{V}="V1",<2em,1cm>*\as{C}="C2",<3em,1cm>*\as{V}="V2",<4em,1cm>*\as{C}="C3",
			<0pt,0pt>*\as{[ʔ]}="c1",<1em,0pt>*\as{a}="v1",<3em,0pt>*\as{i}="v2",
			"c1"+U;"C1"+D**\dir{-};"v1"+U;"V1"+D**\dir{-};"v2"+U;"V2"+D**\dir{-};
		\endxy}
		\exa{\xy
			<0pt,1cm>*\as{C}="C1",<1em,1cm>*\as{V}="V1",<2em,1cm>*\as{C}="C2",<3em,1cm>*\as{V}="V2",<4em,1cm>*\as{C}="C3",
			<0pt,0pt>*\as{[ʔ]}="c1",<0.75em,0pt>*\as{a}="v1",<1.25em,0pt>*\as{u}="v1.5",<2em,0pt>*\as{n}="c2",<3em,0pt>*\as{u}="v2",
			"c1"+U;"C1"+D**\dir{-};"c2"+U;"C2"+D**\dir{-};"v1"+U;"V1"+D**\dir{-};"v1.5"+U;"V1"+D**\dir{-};"v2"+U;"V2"+D**\dir{-};
		\endxy}
		\exa{\label{as:nautus}\xy
			<0pt,1cm>*\as{C}="C1",<1em,1cm>*\as{V}="V1",<2em,1cm>*\as{C}="C2",<3em,1cm>*\as{V}="V2",<4em,1cm>*\as{C}="C3",
			<0pt,0pt>*\as{n}="c1",<0.75em,0pt>*\as{a}="v1",<1.25em,0pt>*\as{u}="v1.5",<2em,0pt>*\as{t}="c2",<3em,0pt>*\as{u}="v2",<4em,0pt>*\as{s}="c3",
			"c1"+U;"C1"+D**\dir{-};"c2"+U;"C2"+D**\dir{-};"c3"+U;"C3"+D**\dir{-};"v1"+U;"V1"+D**\dir{-};"v1.5"+U;"V1"+D**\dir{-};"v2"+U;"V2"+D**\dir{-};
		\endxy}
	\end{exe}
\end{multicols}

Metathesis at the CV tier yields a derived
foot structure with the form CVVC.
%The structures of the M\=/form of the words
%in \qf{as:fafi}--\qf{as:nautus} above
%are given in \qf{as:faif}--\qf{as:naut} below.
This structure is shown in \qf{as:faif}--\qf{as:naut} below.

\begin{multicols}{3}
	\begin{exe}
		\exa{\label{as:faif}\xy
			<0pt,1cm>*\as{C}="C1",<1em,1cm>*\as{V}="V1",<2em,1cm>*\as{V}="C2",<3em,1cm>*\as{C}="V2",
			<0pt,0pt>*\as{f}="c1",<1em,0pt>*\as{a}="v1",<2em,0pt>*\as{i}="c2",<3em,0pt>*\as{f}="v2",
			"c1"+U;"C1"+D**\dir{-};"c2"+U;"C2"+D**\dir{-};"v1"+U;"V1"+D**\dir{-};"v2"+U;"V2"+D**\dir{-};
		\endxy}
		\exa{\xy
			<0pt,1cm>*\as{C}="C1",<1em,1cm>*\as{V}="V1",<2em,1cm>*\as{V}="C2",<3em,1cm>*\as{C}="V2",
			<0pt,0pt>*\as{m}="c1",<1em,0pt>*\as{u}="v1",<2em,0pt>*\as{i}="c2",<3em,0pt>*\as{ʔ}="v2",
			"c1"+U;"C1"+D**\dir{-};"c2"+U;"C2"+D**\dir{-};"v1"+U;"V1"+D**\dir{-};"v2"+U;"V2"+D**\dir{-};
		\endxy}
		\exa{\xy
			<0pt,1cm>*\as{C}="C1",<1em,1cm>*\as{V}="V1",<2em,1cm>*\as{V}="C2",<3em,1cm>*\as{C}="V2",
			<0pt,0pt>*\as{k}="c1",<1em,0pt>*\as{a}="v1",<2em,0pt>*\as{u}="c2",<3em,0pt>*\as{}="v2",
			"c1"+U;"C1"+D**\dir{-};"c2"+U;"C2"+D**\dir{-};"v1"+U;"V1"+D**\dir{-};
		\endxy}
	\end{exe}
\end{multicols}
\begin{multicols}{3}
	\begin{exe}
		\exa{\xy
			<0pt,1cm>*\as{C}="C1",<1em,1cm>*\as{V}="V1",<2em,1cm>*\as{V}="C2",<3em,1cm>*\as{C}="V2",
			<0pt,0pt>*\as{[ʔ]}="c1",<1em,0pt>*\as{a}="v1",<2em,0pt>*\as{i}="c2",<3em,0pt>*\as{ }="v2",
			"c1"+U;"C1"+D**\dir{-};"c2"+U;"C2"+D**\dir{-};"v1"+U;"V1"+D**\dir{-};
		\endxy}
		\exa{\xy
			<0pt,1cm>*\as{C}="C1",<1em,1cm>*\as{V}="V1",<2em,1cm>*\as{V}="C2",<3em,1cm>*\as{C}="V2",
			<0pt,0pt>*\as{[ʔ]}="c1",<1em,0pt>*\as{a}="v1",<2em,0pt>*\as{u}="c2",<3em,0pt>*\as{n}="v2",
			"c1"+U;"C1"+D**\dir{-};"c2"+U;"C2"+D**\dir{-};"v1"+U;"V1"+D**\dir{-};"v2"+U;"V2"+D**\dir{-};
		\endxy}
		\exa{\label{as:naut}\xy
			<0pt,1cm>*\as{C}="C1",<1em,1cm>*\as{V}="V1",<2em,1cm>*\as{V}="C2",<3em,1cm>*\as{C}="V2",
			<0pt,0pt>*\as{n}="c1",<1em,0pt>*\as{a}="v1",<2em,0pt>*\as{u}="c2",<3em,0pt>*\as{t}="v2",
			"c1"+U;"C1"+D**\dir{-};"c2"+U;"C2"+D**\dir{-};"v1"+U;"V1"+D**\dir{-};"v2"+U;"V2"+D**\dir{-};
		\endxy}
	\end{exe}
\end{multicols}

\subsection{The morphological rule: metathesis}\label{sec:TheMorRul}
The process required to generate M\=/forms is metathesis,
given in \qf{ex:CV->VC/'V-} below,
which states that a C-slot and a V-slot metathesise after a stressed V-slot.
This rule is a morphological process, in the style of \cite{an92}.

In \qf{ex:CV->VC/'V-} I have included the phonological environment
in which metathesis takes place; after a stressed V-slot.
This is \emph{not} the environment which triggers metathesis
but rather the environment by which metathesis is constrained.

\begin{exe}
	\ex{CV {\ra} VC /\'V{\gap}}\label{ex:CV->VC/'V-}
\end{exe}

The operation of metathesis for the words \ve{muʔit} `animal',
\ve{kaut} `papaya', and \ve{fafi} `pig' %\ve{ai} `fire',
is given in \qf{as:muiq} below.
\qf{as:muiq1} shows the underlying U\=/form of each of these words.
In \qf{as:muiq2} metathesis of the penultimate C-slot and final V-slot takes place.
This results in a disallowed word-final cluster of two C-slots in \qf{as:muiq3}.
To resolve this, the final C-slot is deleted in (\ref{as:muiq}d),
producing the M\=/forms in (\ref{as:muiq}e).
%
%\begin{multicols}{3}
%\begin{exe}
	%\ex{\begin{xlist}
		%\exa{\xy
			%<0pt,3.5cm>*\as{\x}="x1",<1em,3.5cm>*\as{\x}="x2",<2em,3.5cm>*\as{\x}="x3",<3em,3.5cm>*\as{\x}="x4",<4em,3.5cm>*\as{\x}="x5",
			%<0pt,2.5cm>*\as{C}="c1",<1em,2.5cm>*\as{V}="v1",<2em,2.5cm>*\as{C}="c2",<3em,2.5cm>*\as{V}="v2",<4em,2.5cm>*\as{C}="c3",<4.75em,2.5cm>*\as{\#}="ed",	
			%<0pt,1.5cm>*\as{m}="pc1",<1em,1.5cm>*\as{u}="pv1",<2em,1.5cm>*\as{ʔ}="pc2",<3em,1.5cm>*\as{i}="pv2",<4em,1.5cm>*\as{t}="pc3",
			%<0pt,1cm>*\as{k}="pc1.2",<1em,1cm>*\as{a}="pv1.2",<2em,1cm>*\as{ }="pc2.2",<3em,1cm>*\as{u}="pv2.2",<4em,1cm>*\as{t}="pc3.2",
			%<0pt,0.5cm>*\as{f}="pc1.1",<1em,0.5cm>*\as{a}="pv1.1",<2em,0.5cm>*\as{f}="pc2.1",<3em,0.5cm>*\as{i}="pv2.1",<4em,0.5cm>*\as{ }="pc3.1",
			%<0pt,0cm>*\as{[ʔ]}="pc1.3",<1em,0cm>*\as{a}="pv1.3",<2em,0cm>*\as{ }="pc2.3",<3em,0cm>*\as{i}="pv2.3",<4em,0cm>*\as{ }="pc3.3",
			%"c1"+U;"x1"+D**\dir{-};"c2"+U;"x3"+D**\dir{-};"v1"+U;"x2"+D**\dir{-};"v2"+U;"x4"+D**\dir{-};"c3"+U;"x5"+D**\dir{-};
			%"pc1"+U;"c1"+D**\dir{-};"pc2"+U;"c2"+D**\dir{-};"pv1"+U;"v1"+D**\dir{-};"pv2"+U;"v2"+D**\dir{-};"pc3"+U;"c3"+D**\dir{-};
		%\endxy}\label{as:muiq1}
		%\exa{\xy
			%<0pt,3.5cm>*\as{\x}="x1",<1em,3.5cm>*\as{\x}="x2",<2em,3.5cm>*\as{\x}="x3",<3em,3.5cm>*\as{\x}="x4",<4em,3.5cm>*\as{\x}="x5",
			%<0pt,2.5cm>*\as{C}="c1",<1em,2.5cm>*\as{V}="v1",<2em,2.5cm>*\as{C}="c2",<3em,2.5cm>*\as{V}="v2",<4em,2.5cm>*\as{C}="c3",<4.75em,2.5cm>*\as{\#}="ed",	
			%<0pt,1.5cm>*\as{m}="pc1",<1em,1.5cm>*\as{u}="pv1",<2em,1.5cm>*\as{ʔ}="pc2",<3em,1.5cm>*\as{i}="pv2",<4em,1.5cm>*\as{t}="pc3",
			%<0pt,1cm>*\as{k}="pc1.2",<1em,1cm>*\as{a}="pv1.2",<2em,1cm>*\as{ }="pc2.2",<3em,1cm>*\as{u}="pv2.2",<4em,1cm>*\as{t}="pc3.2",
			%<0pt,0.5cm>*\as{f}="pc1.1",<1em,0.5cm>*\as{a}="pv1.1",<2em,0.5cm>*\as{f}="pc2.1",<3em,0.5cm>*\as{i}="pv2.1",<4em,0.5cm>*\as{ }="pc3.1",
			%<0pt,0cm>*\as{[ʔ]}="pc1.3",<1em,0cm>*\as{a}="pv1.3",<2em,0cm>*\as{ }="pc2.3",<3em,0cm>*\as{i}="pv2.3",<4em,0cm>*\as{ }="pc3.3",
			%"c1"+U;"x1"+D**\dir{-};"c2"+U;"x4"+D**\dir{.};"v1"+U;"x2"+D**\dir{-};"v2"+U;"x3"+D**\dir{.};"c3"+U;"x5"+D**\dir{-};
			%"pc1"+U;"c1"+D**\dir{-};"pc2"+U;"c2"+D**\dir{-};"pv1"+U;"v1"+D**\dir{-};"pv2"+U;"v2"+D**\dir{-};"pc3"+U;"c3"+D**\dir{-};
			%<2.5em,3cm>*\as{\tikz[red,thick,dashed,baseline=0.9ex]\draw (0,0) rectangle (0.8cm,1.5cm);}="box",
		%\endxy}\label{as:muiq2}
		%\exa{\xy
			%<0pt,3.5cm>*\as{\x}="x1",<1em,3.5cm>*\as{\x}="x2",<3em,3.5cm>*\as{\x}="x3",<2em,3.5cm>*\as{\x}="x4",<4em,3.5cm>*\as{\x}="x5",
			%<0pt,2.5cm>*\as{C}="c1",<1em,2.5cm>*\as{V}="v1",<3em,2.5cm>*\as{C}="c2",<2em,2.5cm>*\as{V}="v2",<4em,2.5cm>*\as{C}="c3",<4.75em,2.5cm>*\as{\#}="ed",	
			%<0pt,1.5cm>*\as{m}="pc1",<1em,1.5cm>*\as{u}="pv1",<3em,1.5cm>*\as{ʔ}="pc2",<2em,1.5cm>*\as{i}="pv2",<4em,1.5cm>*\as{t}="pc3",
			%<0pt,1cm>*\as{k}="pc1.2",<1em,1cm>*\as{a}="pv1.2",<3em,1cm>*\as{ }="pc2.2",<2em,1cm>*\as{u}="pv2.2",<4em,1cm>*\as{t}="pc3.2",
			%<0pt,0.5cm>*\as{f}="pc1.1",<1em,0.5cm>*\as{a}="pv1.1",<3em,0.5cm>*\as{f}="pc2.1",<2em,0.5cm>*\as{i}="pv2.1",<4em,0.5cm>*\as{ }="pc3.1",
			%<0pt,0cm>*\as{[ʔ]}="pc1.3",<1em,0cm>*\as{a}="pv1.3",<3em,0cm>*\as{ }="pc2.3",<2em,0cm>*\as{i}="pv2.3",<4em,0cm>*\as{ }="pc3.3",
			%"c1"+U;"x1"+D**\dir{-};"c2"+U;"x3"+D**\dir{-};"v1"+U;"x2"+D**\dir{-};"v2"+U;"x4"+D**\dir{-};"c3"+U;"x5"+D**\dir{-};
			%"pc1"+U;"c1"+D**\dir{-};"pc2"+U;"c2"+D**\dir{-};"pv1"+U;"v1"+D**\dir{-};"pv2"+U;"v2"+D**\dir{-};"pc3"+U;"c3"+D**\dir{-};
			%<3.5em,2.55cm>*\as{\tikz[red,thick,dashed,baseline=0.9ex]\draw (0,0) rectangle (0.75cm,0.4cm);}="box",
		%\endxy}\label{as:muiq3}
	%\end{xlist}}\label{as:muiq}
%\end{exe}
%\end{multicols}
%
%\begin{multicols}{3}
%\begin{exe}
	%\sn{\begin{xlist}
		%\exi{d.}\exia{\xy
			%<0pt,3.5cm>*\as{\x}="x1",<1em,3.5cm>*\as{\x}="x2",<3em,3.5cm>*\as{\x}="x3",<2em,3.5cm>*\as{\x}="x4",<4em,3.5cm>*\as{\x}="x5",
			%<0pt,2.5cm>*\as{C}="c1",<1em,2.5cm>*\as{V}="v1",<3em,2.5cm>*\as{C}="c2",<2em,2.5cm>*\as{V}="v2",<4em,2.5cm>*\as{\xc{C}}="c3",<4.75em,2.5cm>*\as{\#}="ed",	
			%<0pt,1.5cm>*\as{m}="pc1",<1em,1.5cm>*\as{u}="pv1",<3em,1.5cm>*\as{ʔ}="pc2",<2em,1.5cm>*\as{i}="pv2",<4em,1.5cm>*\as{t}="pc3",
			%<0pt,1cm>*\as{k}="pc1.2",<1em,1cm>*\as{a}="pv1.2",<3em,1cm>*\as{ }="pc2.2",<2em,1cm>*\as{u}="pv2.2",<4em,1cm>*\as{t}="pc3.2",
			%<0pt,0.5cm>*\as{f}="pc1.1",<1em,0.5cm>*\as{a}="pv1.1",<3em,0.5cm>*\as{f}="pc2.1",<2em,0.5cm>*\as{i}="pv2.1",<4em,0.5cm>*\as{ }="pc3.1",
			%<0pt,0cm>*\as{[ʔ]}="pc1.3",<1em,0cm>*\as{a}="pv1.3",<3em,0cm>*\as{ }="pc2.3",<2em,0cm>*\as{i}="pv2.3",<4em,0cm>*\as{ }="pc3.3",
			%"c1"+U;"x1"+D**\dir{-};"c2"+U;"x3"+D**\dir{-};"v1"+U;"x2"+D**\dir{-};"v2"+U;"x4"+D**\dir{-};{\ar@{-}|-(.425)*@{|} |-{\hole} |-(.575)*@{|} "c3"+U;"x5"+D};
			%"pc1"+U;"c1"+D**\dir{-};"pc2"+U;"c2"+D**\dir{-};"pv1"+U;"v1"+D**\dir{-};"pv2"+U;"v2"+D**\dir{-};{\ar@{-}|-(.425)*@{|} |-{\hole} |-(.575)*@{|} "pc3"+U;"c3"+D};
			%<4em,1.75cm>*\as{\tikz[red,thick,dashed,baseline=0.9ex]\draw (0,0) rectangle (0.35cm,4cm);}="box",
		%\endxy}
		%\exi{e.}\exia{\xy
			%<0pt,3.5cm>*\as{\x}="x1",<1em,3.5cm>*\as{\x}="x2",<3em,3.5cm>*\as{\x}="x3",<2em,3.5cm>*\as{\x}="x4",
			%<0pt,2.5cm>*\as{C}="c1",<1em,2.5cm>*\as{V}="v1",<3em,2.5cm>*\as{C}="c2",<2em,2.5cm>*\as{V}="v2",<3.75em,2.5cm>*\as{\#}="ed",	
			%<0pt,1.5cm>*\as{m}="pc1",<1em,1.5cm>*\as{u}="pv1",<3em,1.5cm>*\as{ʔ}="pc2",<2em,1.5cm>*\as{i}="pv2",
			%<0pt,1cm>*\as{k}="pc1.2",<1em,1cm>*\as{a}="pv1.2",<3em,1cm>*\as{ }="pc2.2",<2em,1cm>*\as{u}="pv2.2",
			%<0pt,0.5cm>*\as{f}="pc1.1",<1em,0.5cm>*\as{a}="pv1.1",<3em,0.5cm>*\as{f}="pc2.1",<2em,0.5cm>*\as{i}="pv2.1",
			%<0pt,0cm>*\as{[ʔ]}="pc1.3",<1em,0cm>*\as{a}="pv1.3",<3em,0cm>*\as{ }="pc2.3",<2em,0cm>*\as{i}="pv2.3",
			%"c1"+U;"x1"+D**\dir{-};"c2"+U;"x3"+D**\dir{-};"v1"+U;"x2"+D**\dir{-};"v2"+U;"x4"+D**\dir{-};
			%"pc1"+U;"c1"+D**\dir{-};"pc2"+U;"c2"+D**\dir{-};"pv1"+U;"v1"+D**\dir{-};"pv2"+U;"v2"+D**\dir{-};
		%\endxy}
		%\sna{\xy
			%<0pt,3.5cm>*\as{}="x1",<0pt,2.5cm>*\as{}="c1",<0pt,1.5cm>*\as{}="pc1",
			%<0pt,1cm>*\as{}="pc1.2",<0pt,0.5cm>*\as{}="pc1.1",<0pt,0cm>*\as{}="pc1.3",
		%\endxy}
	%\end{xlist}}
%\end{exe}
%\end{multicols}

\begin{multicols}{3}
\begin{exe}
	\ex{\begin{xlist}
		\exa{\xy
			<0pt,3cm>*\as{\x}="x1",<1em,3cm>*\as{\x}="x2",<2em,3cm>*\as{\x}="x3",<3em,3cm>*\as{\x}="x4",<4em,3cm>*\as{\x}="x5",
			<0pt,2cm>*\as{C}="c1",<1em,2cm>*\as{V}="v1",<2em,2cm>*\as{C}="c2",<3em,2cm>*\as{V}="v2",<4em,2cm>*\as{C}="c3",<4.75em,2cm>*\as{\#}="ed",	
			<0pt,1cm>*\as{m}="pc1",<1em,1cm>*\as{u}="pv1",<2em,1cm>*\as{ʔ}="pc2",<3em,1cm>*\as{i}="pv2",<4em,1cm>*\as{t}="pc3",
			<0pt,0.5cm>*\as{k}="pc1.2",<1em,0.5cm>*\as{a}="pv1.2",<2em,0.5cm>*\as{ }="pc2.2",<3em,0.5cm>*\as{u}="pv2.2",<4em,0.5cm>*\as{t}="pc3.2",
			<0pt,0cm>*\as{f}="pc1.1",<1em,0cm>*\as{a}="pv1.1",<2em,0cm>*\as{f}="pc2.1",<3em,0cm>*\as{i}="pv2.1",<4em,0cm>*\as{ }="pc3.1",
			"c1"+U;"x1"+D**\dir{-};"c2"+U;"x3"+D**\dir{-};"v1"+U;"x2"+D**\dir{-};"v2"+U;"x4"+D**\dir{-};"c3"+U;"x5"+D**\dir{-};
			"pc1"+U;"c1"+D**\dir{-};"pc2"+U;"c2"+D**\dir{-};"pv1"+U;"v1"+D**\dir{-};"pv2"+U;"v2"+D**\dir{-};"pc3"+U;"c3"+D**\dir{-};
		\endxy}\label{as:muiq1}
		\exa{\xy
			<0pt,3cm>*\as{\x}="x1",<1em,3cm>*\as{\x}="x2",<2em,3cm>*\as{\x}="x3",<3em,3cm>*\as{\x}="x4",<4em,3cm>*\as{\x}="x5",
			<0pt,2cm>*\as{C}="c1",<1em,2cm>*\as{V}="v1",<2em,2cm>*\as{C}="c2",<3em,2cm>*\as{V}="v2",<4em,2cm>*\as{C}="c3",<4.75em,2cm>*\as{\#}="ed",	
			<0pt,1cm>*\as{m}="pc1",<1em,1cm>*\as{u}="pv1",<2em,1cm>*\as{ʔ}="pc2",<3em,1cm>*\as{i}="pv2",<4em,1cm>*\as{t}="pc3",
			<0pt,0.5cm>*\as{k}="pc1.2",<1em,0.5cm>*\as{a}="pv1.2",<2em,0.5cm>*\as{ }="pc2.2",<3em,0.5cm>*\as{u}="pv2.2",<4em,0.5cm>*\as{t}="pc3.2",
			<0pt,0cm>*\as{f}="pc1.1",<1em,0cm>*\as{a}="pv1.1",<2em,0cm>*\as{f}="pc2.1",<3em,0cm>*\as{i}="pv2.1",<4em,0cm>*\as{ }="pc3.1",
			"c1"+U;"x1"+D**\dir{-};"c2"+U;"x4"+D**\dir{.};"v1"+U;"x2"+D**\dir{-};"v2"+U;"x3"+D**\dir{.};"c3"+U;"x5"+D**\dir{-};
			"pc1"+U;"c1"+D**\dir{-};"pc2"+U;"c2"+D**\dir{-};"pv1"+U;"v1"+D**\dir{-};"pv2"+U;"v2"+D**\dir{-};"pc3"+U;"c3"+D**\dir{-};
			<2.5em,2.5cm>*\as{\tikz[red,thick,dashed,baseline=0.9ex]\draw (0,0) rectangle (0.8cm,1.5cm);}="box",
		\endxy}\label{as:muiq2}
		\exa{\xy
			<0pt,3cm>*\as{\x}="x1",<1em,3cm>*\as{\x}="x2",<3em,3cm>*\as{\x}="x3",<2em,3cm>*\as{\x}="x4",<4em,3cm>*\as{\x}="x5",
			<0pt,2cm>*\as{C}="c1",<1em,2cm>*\as{V}="v1",<3em,2cm>*\as{C}="c2",<2em,2cm>*\as{V}="v2",<4em,2cm>*\as{C}="c3",<4.75em,2cm>*\as{\#}="ed",	
			<0pt,1cm>*\as{m}="pc1",<1em,1cm>*\as{u}="pv1",<3em,1cm>*\as{ʔ}="pc2",<2em,1cm>*\as{i}="pv2",<4em,1cm>*\as{t}="pc3",
			<0pt,0.5cm>*\as{k}="pc1.2",<1em,0.5cm>*\as{a}="pv1.2",<3em,0.5cm>*\as{ }="pc2.2",<2em,0.5cm>*\as{u}="pv2.2",<4em,0.5cm>*\as{t}="pc3.2",
			<0pt,0cm>*\as{f}="pc1.1",<1em,0cm>*\as{a}="pv1.1",<3em,0cm>*\as{f}="pc2.1",<2em,0cm>*\as{i}="pv2.1",<4em,0cm>*\as{ }="pc3.1",
			"c1"+U;"x1"+D**\dir{-};"c2"+U;"x3"+D**\dir{-};"v1"+U;"x2"+D**\dir{-};"v2"+U;"x4"+D**\dir{-};"c3"+U;"x5"+D**\dir{-};
			"pc1"+U;"c1"+D**\dir{-};"pc2"+U;"c2"+D**\dir{-};"pv1"+U;"v1"+D**\dir{-};"pv2"+U;"v2"+D**\dir{-};"pc3"+U;"c3"+D**\dir{-};
			<3.5em,2.55cm>*\as{\tikz[red,thick,dashed,baseline=0.9ex]\draw (0,0) rectangle (0.75cm,0.4cm);}="box",
		\endxy}\label{as:muiq3}
	\end{xlist}}\label{as:muiq}
\end{exe}
\end{multicols}
\begin{multicols}{3}
\begin{exe}
	\sn{\begin{xlist}
		\exi{d.}\exia{\xy
			<0pt,3cm>*\as{\x}="x1",<1em,3cm>*\as{\x}="x2",<3em,3cm>*\as{\x}="x3",<2em,3cm>*\as{\x}="x4",<4em,3cm>*\as{\x}="x5",
			<0pt,2cm>*\as{C}="c1",<1em,2cm>*\as{V}="v1",<3em,2cm>*\as{C}="c2",<2em,2cm>*\as{V}="v2",<4em,2cm>*\as{\xc{C}}="c3",<4.75em,2cm>*\as{\#}="ed",	
			<0pt,1cm>*\as{m}="pc1",<1em,1cm>*\as{u}="pv1",<3em,1cm>*\as{ʔ}="pc2",<2em,1cm>*\as{i}="pv2",<4em,1cm>*\as{t}="pc3",
			<0pt,0.5cm>*\as{k}="pc1.2",<1em,0.5cm>*\as{a}="pv1.2",<3em,0.5cm>*\as{ }="pc2.2",<2em,0.5cm>*\as{u}="pv2.2",<4em,0.5cm>*\as{t}="pc3.2",
			<0pt,0cm>*\as{f}="pc1.1",<1em,0cm>*\as{a}="pv1.1",<3em,0cm>*\as{f}="pc2.1",<2em,0cm>*\as{i}="pv2.1",<4em,0cm>*\as{ }="pc3.1",
			"c1"+U;"x1"+D**\dir{-};"c2"+U;"x3"+D**\dir{-};"v1"+U;"x2"+D**\dir{-};"v2"+U;"x4"+D**\dir{-};{\ar@{-}|-(.425)*@{|} |-{\hole} |-(.575)*@{|} "c3"+U;"x5"+D};
			"pc1"+U;"c1"+D**\dir{-};"pc2"+U;"c2"+D**\dir{-};"pv1"+U;"v1"+D**\dir{-};"pv2"+U;"v2"+D**\dir{-};{\ar@{-}|-(.425)*@{|} |-{\hole} |-(.575)*@{|} "pc3"+U;"c3"+D};
			<4em,1.75cm>*\as{\tikz[red,thick,dashed,baseline=0.9ex]\draw (0,0) rectangle (0.35cm,3cm);}="box",
		\endxy}
		\exi{e.}\exia{\xy
			<0pt,3cm>*\as{\x}="x1",<1em,3cm>*\as{\x}="x2",<3em,3cm>*\as{\x}="x3",<2em,3cm>*\as{\x}="x4",
			<0pt,2cm>*\as{C}="c1",<1em,2cm>*\as{V}="v1",<3em,2cm>*\as{C}="c2",<2em,2cm>*\as{V}="v2",<3.75em,2cm>*\as{\#}="ed",	
			<0pt,1cm>*\as{m}="pc1",<1em,1cm>*\as{u}="pv1",<3em,1cm>*\as{ʔ}="pc2",<2em,1cm>*\as{i}="pv2",
			<0pt,0.5cm>*\as{k}="pc1.2",<1em,0.5cm>*\as{a}="pv1.2",<3em,0.5cm>*\as{ }="pc2.2",<2em,0.5cm>*\as{u}="pv2.2",
			<0pt,0cm>*\as{f}="pc1.1",<1em,0cm>*\as{a}="pv1.1",<3em,0cm>*\as{f}="pc2.1",<2em,0cm>*\as{i}="pv2.1",
			"c1"+U;"x1"+D**\dir{-};"c2"+U;"x3"+D**\dir{-};"v1"+U;"x2"+D**\dir{-};"v2"+U;"x4"+D**\dir{-};
			"pc1"+U;"c1"+D**\dir{-};"pc2"+U;"c2"+D**\dir{-};"pv1"+U;"v1"+D**\dir{-};"pv2"+U;"v2"+D**\dir{-};
		\endxy}
		\sna{\xy
			<0pt,3cm>*\as{}="x1",<0pt,2cm>*\as{}="c1",<0pt,1cm>*\as{}="pc1",
			<0pt,0.5cm>*\as{}="pc1.2",<0pt,0cm>*\as{}="pc1.1",<0pt,0cm>*\as{}="pc1.3",
		\endxy}
	\end{xlist}}
\end{exe}
\end{multicols}

\subsubsection{Metathesis and mid vowel assimilation}
As discussed in \srf{sec:MidVowAss}, any final mid vowel
assimilates to the height of a previous high vowel after metathesis.
This vowel height assimilation is an instance of vowel harmony,
arising from the fact that sequences of a high vowel and mid vowel
are disallowed in Amarasi (\srf{sec:VowSeq}).

This process is illustrated for \ve{ume} {\ra} \ve{uim} `house' in \qf{as:uim} below.
After metathesis in \qf{as:uim2}, the feature \tsc{[+high]}
of the stressed vowel spreads in (\ref{as:uim}d)
resulting in a sequence of two high vowels in (\ref{as:uim}e).
Unless the \tsc{[+high]} feature of the penultimate vowel is analysed
as privative we would also have to propose that the height features
\tsc{[-high, +mid]} of the final vowel /e/ de-link in (\ref{as:uim}d).

\begin{multicols}{3}
\begin{exe}
	\ex{\begin{xlist}
		\exa{\xy
			<0em,3cm>*\as{\x}="x1",<1em,3cm>*\as{\x}="x2",<2em,3cm>*\as{\x}="x3",<3em,3cm>*\as{\x}="x4",<4em,3cm>*\as{\x}="x5",
			<0em,2cm>*\as{C}="C1",<1em,2cm>*\as{V}="V1",<2em,2cm>*\as{C}="C2",<3em,2cm>*\as{V}="V2",<4em,2cm>*\as{C}="C3",
			<0em,1cm>*\as{[ʔ]}="c1",<1em,1cm>*\as{u}="v1",<2em,1cm>*\as{m}="c2",<3em,1cm>*\as{e}="v2",<4em,1cm>*\as{}="c3",
			<1em,0cm>*\as{\tsc{[+high]}}="f","f"+U;"v1"+D**\dir{-};
			"C1"+U;"x1"+D**\dir{-};"C2"+U;"x3"+D**\dir{-};"V1"+U;"x2"+D**\dir{-};"V2"+U;"x4"+D**\dir{-};"C3"+U;"x5"+D**\dir{-};
			"c1"+U;"C1"+D**\dir{-};"c2"+U;"C2"+D**\dir{-};"v1"+U;"V1"+D**\dir{-};"v2"+U;"V2"+D**\dir{-};
		\endxy}\label{as:uim1}
		\exa{\xy
			<0em,3cm>*\as{\x}="x1",<1em,3cm>*\as{\x}="x2",<2em,3cm>*\as{\x}="x3",<3em,3cm>*\as{\x}="x4",<4em,3cm>*\as{\x}="x5",
			<0em,2cm>*\as{C}="C1",<1em,2cm>*\as{V}="V1",<2em,2cm>*\as{C}="C2",<3em,2cm>*\as{V}="V2",<4em,2cm>*\as{C}="C3",
			<0em,1cm>*\as{[ʔ]}="c1",<1em,1cm>*\as{u}="v1",<2em,1cm>*\as{m}="c2",<3em,1cm>*\as{e}="v2",<4em,1cm>*\as{}="c3",
			<1em,0cm>*\as{\tsc{[+high]}}="f","f"+U;"v1"+D**\dir{-};
			"C1"+U;"x1"+D**\dir{-};"C2"+U;"x4"+D**\dir{.};"V1"+U;"x2"+D**\dir{-};"V2"+U;"x3"+D**\dir{.};"C3"+U;"x5"+D**\dir{-};
			"c1"+U;"C1"+D**\dir{-};"c2"+U;"C2"+D**\dir{-};"v1"+U;"V1"+D**\dir{-};"v2"+U;"V2"+D**\dir{-};
			<2.5em,2.5cm>*\as{\tikz[red,thick,dashed,baseline=0.9ex]\draw (0,0) rectangle (0.8cm,1.5cm);}="box",
		\endxy}\label{as:uim2}
		\exa{\xy
			<0em,3cm>*\as{\x}="x1",<1em,3cm>*\as{\x}="x2",<2em,3cm>*\as{\x}="x3",<3em,3cm>*\as{\x}="x4",<4em,3cm>*\as{\x}="x5",
			<0em,2cm>*\as{C}="C1",<1em,2cm>*\as{V}="V1",<3em,2cm>*\as{C}="C2",<2em,2cm>*\as{V}="V2",<4em,2cm>*\as{C}="C3",
			<0em,1cm>*\as{[ʔ]}="c1",<1em,1cm>*\as{u}="v1",<3em,1cm>*\as{m}="c2",<2em,1cm>*\as{e}="v2",<4em,1cm>*\as{}="c3",
			<1em,0cm>*\as{\tsc{[+high]}}="f","f"+U;"v1"+D**\dir{-};
			"C1"+U;"x1"+D**\dir{-};"C2"+U;"x4"+D**\dir{-};"V1"+U;"x2"+D**\dir{-};"V2"+U;"x3"+D**\dir{-};"C3"+U;"x5"+D**\dir{-};
			"c1"+U;"C1"+D**\dir{-};"c2"+U;"C2"+D**\dir{-};"v1"+U;"V1"+D**\dir{-};"v2"+U;"V2"+D**\dir{-};
			<1.5em,1.05cm>*\as{\tikz[red,thick,dashed,baseline=0.9ex]\draw (0,0) rectangle (0.75cm,0.4cm);}="box",
		\endxy}\label{as:uim3}
	\end{xlist}}\label{as:uim}
\end{exe}
\end{multicols}
\begin{multicols}{3}
\begin{exe}
	\sn{\begin{xlist}
		\exi{d.}\exia{\xy
			<0em,3cm>*\as{\x}="x1",<1em,3cm>*\as{\x}="x2",<2em,3cm>*\as{\x}="x3",<3em,3cm>*\as{\x}="x4",<4em,3cm>*\as{\x}="x5",
			<0em,2cm>*\as{C}="C1",<1em,2cm>*\as{V}="V1",<3em,2cm>*\as{C}="C2",<2em,2cm>*\as{V}="V2",<4em,2cm>*\as{C}="C3",
			<0em,1cm>*\as{[ʔ]}="c1",<1em,1cm>*\as{u}="v1",<3em,1cm>*\as{m}="c2",<2em,1cm>*\as{e}="v2",<4em,1cm>*\as{}="c3",
			<1em,0cm>*\as{\tsc{[+high]}}="f","f"+U;"v1"+D**\dir{-};"f"+U;"v2"+D**\dir{.};
			"C1"+U;"x1"+D**\dir{-};"C2"+U;"x4"+D**\dir{-};"V1"+U;"x2"+D**\dir{-};"V2"+U;"x3"+D**\dir{-};"C3"+U;"x5"+D**\dir{-};
			"c1"+U;"C1"+D**\dir{-};"c2"+U;"C2"+D**\dir{-};"v1"+U;"V1"+D**\dir{-};"v2"+U;"V2"+D**\dir{-};
			<1.5em,1.05cm>*\as{\tikz[red,thick,dashed,baseline=0.9ex]\draw (0,0) rectangle (0.75cm,0.4cm);}="box",
		\endxy}\label{as:uim4}
		\exi{e.}\exia{\xy
			<0em,3cm>*\as{\x}="x1",<1em,3cm>*\as{\x}="x2",<2em,3cm>*\as{\x}="x3",<3em,3cm>*\as{\x}="x4",<4em,3cm>*\as{\x}="x5",
			<0em,2cm>*\as{C}="C1",<1em,2cm>*\as{V}="V1",<3em,2cm>*\as{C}="C2",<2em,2cm>*\as{V}="V2",<4em,2cm>*\as{\xc{C}}="C3",
			<0em,1cm>*\as{[ʔ]}="c1",<1em,1cm>*\as{u}="v1",<3em,1cm>*\as{m}="c2",<2em,1cm>*\as{i}="v2",<4em,1cm>*\as{}="c3",
			<1.5em,0cm>*\as{\tsc{[+high]}}="f","f"+U;"v1"+D**\dir{-};"f"+U;"v2"+D**\dir{-};
			"C1"+U;"x1"+D**\dir{-};"C2"+U;"x4"+D**\dir{-};"V1"+U;"x2"+D**\dir{-};"V2"+U;"x3"+D**\dir{-};"C3"+U;"x5"+D**\dir{-};
			"c1"+U;"C1"+D**\dir{-};"c2"+U;"C2"+D**\dir{-};"v1"+U;"V1"+D**\dir{-};"v2"+U;"V2"+D**\dir{-};
		\endxy}\label{as:uim5}
		\exi{f.}\exia{\xy
			<0em,3cm>*\as{\x}="x1",<1em,3cm>*\as{\x}="x2",<2em,3cm>*\as{\x}="x3",<3em,3cm>*\as{\x}="x4",
			<0em,2cm>*\as{C}="C1",<1em,2cm>*\as{V}="V1",<3em,2cm>*\as{C}="C2",<2em,2cm>*\as{V}="V2",
			<0em,1cm>*\as{[ʔ]}="c1",<1em,1cm>*\as{u}="v1",<3em,1cm>*\as{m}="c2",<2em,1cm>*\as{i}="v2",
			<1.5em,0cm>*\as{\tsc{[+high]}}="f","f"+U;"v1"+D**\dir{-};"f"+U;"v2"+D**\dir{-};
			"C1"+U;"x1"+D**\dir{-};"C2"+U;"x4"+D**\dir{-};"V1"+U;"x2"+D**\dir{-};"V2"+U;"x3"+D**\dir{-};
			"c1"+U;"C1"+D**\dir{-};"c2"+U;"C2"+D**\dir{-};"v1"+U;"V1"+D**\dir{-};"v2"+U;"V2"+D**\dir{-};
		\endxy}\label{as:uim6}
	\end{xlist}}
\end{exe}
\end{multicols}

\subsubsection{Metathesis and vowel deletion}
The vowel deletion in words with a phonetic diphthong,
such as \ve{nautus} [ˈnəwtʊs] {\ra} \ve{nautus} [ˈnə.ʊt] `beetle'
results from metathesis and the fact that Amarasi does
not allow sequences of three surface vowels.
Recall from \srf{sec:Str} that the first two vowels
of words with a phonetic diphthong are associated to a single V-slot,
as shown by the fact that stress falls on the antepenultimate
vowel rather than the penultimate vowel.

The formation of the M\=/form for
\ve{nautus} [ˈnəwtʊs] {\ra} \ve{nautus} [ˈnə.ʊt] `beetle' 
is illustrated in \qf{as:aun} below.
Metathesis in \qf{as:aun2} results in a
surface sequence of three vowels in \qf{as:aun3};
the first V-slot is associated to two vowels
which are adjacent to another vowel associated to a single V-slot.
As a result, the final vowel is deleted in (\ref{as:aun}d),
with subsequent re-association of the adjacent vowel into the now empty V-slot in (\ref{as:aun}e).
The final C-slot is also deleted yielding the output shown in (\ref{as:aun}f).

\begin{multicols}{3}
\begin{exe}
	\ex{\begin{xlist}
		\exa{\xy
			<0em,2cm>*\as{\x}="x1",<1em,2cm>*\as{\x}="x2",<2em,2cm>*\as{\x}="x3",<3em,2cm>*\as{\x}="x4",<4em,2cm>*\as{\x}="x5",
			<0em,1cm>*\as{C}="C1",<4em,1cm>*\as{C}="C3",<2em,1cm>*\as{C}="C2",
			"C1"+U;"x1"+D**\dir{-};"C3"+U;"x5"+D**\dir{-};"C2"+U;"x3"+D**\dir{-};
			<0em,0cm>*\as{n}="c1",<2em,0cm>*\as{t}="c2","c2"+U;"C2"+D**\dir{-};<4em,0cm>*\as{s}="c3","c3"+U;"C3"+D**\dir{-};
			<1em,1cm>*\as{V}="V1",<3em,1cm>*\as{V}="V2","V1"+U;"x2"+D**\dir{-};"V2"+U;"x4"+D**\dir{-};
			<0.7em,0cm>*\as{a}="v1",<1.3em,0cm>*\as{u}="v3",<3em,0cm>*\as{u}="v2",
			"c1"+U;"C1"+D**\dir{-};"v1"+U;"V1"+D**\dir{-};"v2"+U;"V2"+D**\dir{-};"v3"+U;"V1"+D**\dir{-};
		\endxy}\label{as:aun1}
		\exa{\xy
			<0em,2cm>*\as{\x}="x1",<1em,2cm>*\as{\x}="x2",<2em,2cm>*\as{\x}="x3",<3em,2cm>*\as{\x}="x4",<4em,2cm>*\as{\x}="x5",
			<0em,1cm>*\as{C}="C1",<4em,1cm>*\as{C}="C3",<2em,1cm>*\as{C}="C2",
			"C1"+U;"x1"+D**\dir{-};"C3"+U;"x5"+D**\dir{-};"C2"+U;"x4"+D**\dir{.};
			<0em,0cm>*\as{n}="c1",<2em,0cm>*\as{t}="c2","c2"+U;"C2"+D**\dir{-};<4em,0cm>*\as{s}="c3","c3"+U;"C3"+D**\dir{-};
			<1em,1cm>*\as{V}="V1",<3em,1cm>*\as{V}="V2","V1"+U;"x2"+D**\dir{-};"V2"+U;"x3"+D**\dir{.};
			<0.7em,0cm>*\as{a}="v1",<1.3em,0cm>*\as{u}="v3",<3em,0cm>*\as{u}="v2",
			"c1"+U;"C1"+D**\dir{-};"v1"+U;"V1"+D**\dir{-};"v2"+U;"V2"+D**\dir{-};"v3"+U;"V1"+D**\dir{-};
			<2.5em,1.5cm>*\as{\tikz[red,thick,dashed,baseline=0.9ex]\draw (0,0) rectangle (0.8cm,1.5cm);}="box",
		\endxy}\label{as:aun2}
		\exa{\xy
			<0em,2cm>*\as{\x}="x1",<1em,2cm>*\as{\x}="x2",<2em,2cm>*\as{\x}="x3",<3em,2cm>*\as{\x}="x4",<4em,2cm>*\as{\x}="x5",
			<0em,1cm>*\as{C}="C1",<4em,1cm>*\as{C}="C3",<3em,1cm>*\as{C}="C2",
			"C1"+U;"x1"+D**\dir{-};"C3"+U;"x5"+D**\dir{-};"C2"+U;"x4"+D**\dir{-};
			<0em,0cm>*\as{n}="c1",<3em,0cm>*\as{t}="c2","c2"+U;"C2"+D**\dir{-};<4em,0cm>*\as{s}="c3","c3"+U;"C3"+D**\dir{-};
			<1em,1cm>*\as{V}="V1",<2em,1cm>*\as{V}="V2","V1"+U;"x2"+D**\dir{-};"V2"+U;"x3"+D**\dir{-};
			<0.7em,0cm>*\as{a}="v1",<1.3em,0cm>*\as{u}="v3",<2em,0cm>*\as{u}="v2",
			"c1"+U;"C1"+D**\dir{-};"v1"+U;"V1"+D**\dir{-};"v2"+U;"V2"+D**\dir{-};"v3"+U;"V1"+D**\dir{-};
			<1.35em,0.025cm>*\as{\tikz[red,thick,dashed,baseline=0.9ex]\draw (0,0) rectangle (0.8cm,0.4cm);}="box",
		\endxy}\label{as:aun3}
	\end{xlist}}\label{as:aun}
\end{exe}
\end{multicols}
\begin{multicols}{3}
\begin{exe}
	\sn{\begin{xlist}
		\exi{d.}\exia{\xy
			<0em,2cm>*\as{\x}="x1",<1em,2cm>*\as{\x}="x2",<2em,2cm>*\as{\x}="x3",<3em,2cm>*\as{\x}="x4",<4em,2cm>*\as{\x}="x5",
			<0em,1cm>*\as{C}="C1",<4em,1cm>*\as{C}="C3",<3em,1cm>*\as{C}="C2",
			"C1"+U;"x1"+D**\dir{-};"C3"+U;"x5"+D**\dir{-};"C2"+U;"x4"+D**\dir{-};
			<0em,0cm>*\as{n}="c1",<3em,0pt>*\as{t}="c2","c2"+U;"C2"+D**\dir{-};<4em,0cm>*\as{s}="c3","c3"+U;"C3"+D**\dir{-};
			<1em,1cm>*\as{V}="V1",<2em,1cm>*\as{V}="V2","V1"+U;"x2"+D**\dir{-};"V2"+U;"x3"+D**\dir{-};
			<0.7em,0cm>*\as{a}="v1",<1.3em,0pt>*\as{u}="v3",<2em,0pt>*\as{\xc{u}}="v2","c1"+U;"C1"+D**\dir{-};"v1"+U;"V1"+D**\dir{-};"v3"+U;"V1"+D**\dir{-};
			{\ar@{-}|-(.425)*@{|} |-{\hole} |-(.575)*@{|} "v2"+U;"V2"+D};
			<2em,0.5cm>*\as{\tikz[red,thick,dashed,baseline=0.9ex]\draw (0,0) rectangle (0.4cm,1.5cm);}="box",
		\endxy}
		\exi{e.}\exia{\xy
			<0em,2cm>*\as{\x}="x1",<1em,2cm>*\as{\x}="x2",<2em,2cm>*\as{\x}="x3",<3em,2cm>*\as{\x}="x4",<4em,2cm>*\as{\x}="x5",
			<0em,1cm>*\as{C}="C1",<4em,1cm>*\as{\xc{C}}="C3",<3em,1cm>*\as{C}="C2",
			"C1"+U;"x1"+D**\dir{-};"C3"+U;"x5"+D**\dir{-};"C2"+U;"x4"+D**\dir{-};
			<0em,0cm>*\as{n}="c1",<3em,0pt>*\as{t}="c2","c2"+U;"C2"+D**\dir{-};<4em,0cm>*\as{s}="c3","c3"+U;"C3"+D**\dir{-};
			<1em,1cm>*\as{V}="V1",<2em,1cm>*\as{V}="V2","V1"+U;"x2"+D**\dir{-};"V2"+U;"x3"+D**\dir{-};
			<0.7em,0pt>*\as{a}="v1",<1.3em,0pt>*\as{u}="v3",
			"c1"+U;"C1"+D**\dir{-};"v1"+U;"V1"+D**\dir{-};{\ar@{-}|-(.35)*@{|} |-{\hole} |-(.5)*@{|} "v3"+U;"V1"+D};"v3"+U;"V2"+D**\dir{.};
			<1.5em,0.5cm>*\as{\tikz[red,thick,dashed,baseline=0.9ex]\draw (0,0) rectangle (0.9cm,1.5cm);}="box",
		\endxy}
		\exi{f.}\exia{\xy
			<0em,2cm>*\as{\x}="x1",<1em,2cm>*\as{\x}="x2",<2em,2cm>*\as{\x}="x3",<3em,2cm>*\as{\x}="x4",
			<0em,1cm>*\as{C}="C1",<3em,1cm>*\as{C}="C2",
			"C1"+U;"x1"+D**\dir{-};"C2"+U;"x4"+D**\dir{-};
			<0em,0cm>*\as{n}="c1",<3em,0pt>*\as{t}="c2","c2"+U;"C2"+D**\dir{-};
			<1em,1cm>*\as{V}="V1",<2em,1cm>*\as{V}="V2","V1"+U;"x2"+D**\dir{-};"V2"+U;"x3"+D**\dir{-};
			<1em,0cm>*\as{a}="v1",<2em,0pt>*\as{u}="v2","v1"+U;"V1"+D**\dir{-};"v2"+U;"V2"+D**\dir{-};
			"c1"+U;"C1"+D**\dir{-};
		\endxy}
	\end{xlist}}
\end{exe}
\end{multicols}

Evidence that it is the final vowel and not the penultimate vowel
which is deleted comes from the word
\ve{n-aena} `runs, flees' with the M\=/form \ve{n-aen}.
If the second vowel were deleted after metathesis
in words with an initial phonetic diphthong, \ve{n-aena} `runs, flees'
would have the M\=/form \it{*n-aan}.

There are no other processes in Amarasi which create a sequence
of three vowels within a morpheme.
All other potential VVV sequences would occur across a morpheme
boundary in which case consonants are inserted;
a voiced obstruent morpheme finally (\srf{sec:EmpCSloConIns}),
and a glottal stop morpheme initially (\srf{sec:GloStoIns}).

\subsection{The morphemically conditioned rule: assimilation of /a/}\label{sec:MorRulAssOfA}
The morphological process of metathesis
triggers assimilation of final /a/,
such as in \ve{ni\tbr{ma}} {\ra} \ve{ni\tbr{im}} `five'.
This rule is given as rule \qf{ex:'VaC->VC} below.
This rule states that the features (represented by \tsc{[+F.]}) of the stressed vowel spread
when immediately followed by /a/ and a filled C-slot.

\begin{exe}
	\exa{\xy
		<0em,2cm>*\as{\'V}="V1",<1em,2cm>*\as{V}="V2",<2em,2cm>*\as{C}="C1",
		<0em,1cm>*\as{}="v1",<1em,1cm>*\as{a}="v2",<2em,1cm>*\as{}="c1",
		<0em,0cm>*\as{\tsc{[+F.]}}="f","f"+U;"v1"+D**\dir{-};"f"+U;"v2"+D**\dir{.};
		"v1"+U;"V1"+D**\dir{-};"c1"+U;"C1"+D**\dir{-};"v2"+U;"V2"+D**\dir{-};
	\endxy}\label{ex:'VaC->VC}
\end{exe}

Assimilation of /a/ is a derived environment effect.
It is not dissimilar to umlaut in German plurals,
in that both occur only in morphologically derived environments.
In German, a floating autosegment triggers fronting of the root vowel
only in morphologically derived environments,
such as in plurals \citep[181ff]{wi96}.
In Amarasi /a/ assimilation only occurs
in a morphologically derived environment: the M\=/form.
This, I take it, is what is meant by a
\emph{morphemically conditioned rule}:
a rule the operation of which is dependent on
and occurs only after an independent morphological rule.

The rule of /a/ assimilation is formulated 
in \qf{ex:'VaC->VC} as a general phonological rule.
This is possible because under the analysis involving
the obligatory CVCVC foot the environment which triggers
assimilation of /a/ -- two immediately adjacent V-slots --
only arises after metathesis.

That the only vowel which assimilate in Amarasi is /a/
can be partially explained by the fact that it is almost featureless.\footnote{
		There is also evidence that /a/ is the default vowel,
		as it is the vowel used in epenthesis (\srf{sec:Epe}).}
Perhaps apart from the feature \tsc{[+low]},
/a/ is not specified for \tsc{front} or \tsc{back}.
This lack of features allows the features of the stressed vowel to spread
when the V-slot to which /a/ is associated occurs immediately after it.

The formation \ve{nima} {\ra} \ve{niim} `five' is given in \qf{as:niim} below.
Metathesis occurs in \qf{as:niim2},
resulting in the V-slot to which /a/ is associated
occurring immediately after a stressed V-slot and before a filled C-slot in \qf{as:niim3}.
Thus, the features of the stressed vowel spread in (\ref{as:niim}d),
creating a sequence of two identical vowels in (\ref{as:niim}e).
The final C-slot is then deleted yielding the final output shown in (\ref{as:niim}f).

\newpage
\begin{multicols}{3}
\begin{exe}
	\ex{\begin{xlist}
		\ex\raisebox{\dimexpr-\totalheight+5ex\relax}{\xy
			<0em,3.3cm>*\as{\x}="x1",<1em,3.3cm>*\as{\x}="x2",<2em,3.3cm>*\as{\x}="x3",<3em,3.3cm>*\as{\x}="x4",<4em,3.3cm>*\as{\x}="x5",
			<0em,2.3cm>*\as{C}="C1",<1em,2.3cm>*\as{\'V}="V1",<2em,2.3cm>*\as{C}="C2",<3em,2.3cm>*\as{V}="V2",<4em,2.3cm>*\as{C}="C3",
			<0em,1.3cm>*\as{n}="c1",<1em,1.3cm>*\as{i}="v1",<2em,1.3cm>*\as{m}="c2",<3em,1.3cm>*\as{a}="v2",<4em,1.3cm>*\as{}="c3",
			<1em,0cm>*\as{{$\left[\hspace{-2mm}\begin{array}{l}\textrm{\tsc{+high}}\\\textrm{\tsc{+front}}\end{array}\hspace{-2mm}\right]$}}="f",
			"f"+U;"v1"+D**\dir{-};
			"C1"+U;"x1"+D**\dir{-};"C2"+U;"x3"+D**\dir{-};"V1"+U;"x2"+D**\dir{-};"V2"+U;"x4"+D**\dir{-};"C3"+U;"x5"+D**\dir{-};
			"c1"+U;"C1"+D**\dir{-};"c2"+U;"C2"+D**\dir{-};"v1"+U;"V1"+D**\dir{-};"v2"+U;"V2"+D**\dir{-};
		\endxy}\label{as:niim1}
		\ex\raisebox{\dimexpr-\totalheight+5ex\relax}{\xy
			<0em,3.3cm>*\as{\x}="x1",<1em,3.3cm>*\as{\x}="x2",<2em,3.3cm>*\as{\x}="x3",<3em,3.3cm>*\as{\x}="x4",<4em,3.3cm>*\as{\x}="x5",
			<0em,2.3cm>*\as{C}="C1",<1em,2.3cm>*\as{\'V}="V1",<2em,2.3cm>*\as{C}="C2",<3em,2.3cm>*\as{V}="V2",<4em,2.3cm>*\as{C}="C3",
			<0em,1.3cm>*\as{n}="c1",<1em,1.3cm>*\as{i}="v1",<2em,1.3cm>*\as{m}="c2",<3em,1.3cm>*\as{a}="v2",<4em,1.3cm>*\as{}="c3",
			<1em,0cm>*\as{{$\left[\hspace{-2mm}\begin{array}{l}\textrm{\tsc{+high}}\\\textrm{\tsc{+front}}\end{array}\hspace{-2mm}\right]$}}="f",
			"f"+U;"v1"+D**\dir{-};
			"C1"+U;"x1"+D**\dir{-};"C2"+U;"x4"+D**\dir{.};"V1"+U;"x2"+D**\dir{-};"V2"+U;"x3"+D**\dir{.};"C3"+U;"x5"+D**\dir{-};
			"c1"+U;"C1"+D**\dir{-};"c2"+U;"C2"+D**\dir{-};"v1"+U;"V1"+D**\dir{-};"v2"+U;"V2"+D**\dir{-};
			<2.5em,2.7cm>*\as{\tikz[red,thick,dashed,baseline=0.9ex]\draw (0,0) rectangle (0.8cm,1.5cm);}="box",
		\endxy}\label{as:niim2}
		\ex\raisebox{\dimexpr-\totalheight+5ex\relax}{\xy
			<0em,3.3cm>*\as{\x}="x1",<1em,3.3cm>*\as{\x}="x2",<2em,3.3cm>*\as{\x}="x3",<3em,3.3cm>*\as{\x}="x4",<4em,3.3cm>*\as{\x}="x5",
			<0em,2.3cm>*\as{C}="C1",<1em,2.3cm>*\as{\'V}="V1",<3em,2.3cm>*\as{C}="C2",<2em,2.3cm>*\as{V}="V2",<4em,2.3cm>*\as{C}="C3",
			<0em,1.3cm>*\as{n}="c1",<1em,1.3cm>*\as{i}="v1",<3em,1.3cm>*\as{m}="c2",<2em,1.3cm>*\as{a}="v2",<4em,1.3cm>*\as{}="c3",
			<1em,0cm>*\as{{$\left[\hspace{-2mm}\begin{array}{l}\textrm{\tsc{+high}}\\\textrm{\tsc{+front}}\end{array}\hspace{-2mm}\right]$}}="f",
			"f"+U;"v1"+D**\dir{-};
			"C1"+U;"x1"+D**\dir{-};"C2"+U;"x4"+D**\dir{-};"V1"+U;"x2"+D**\dir{-};"V2"+U;"x3"+D**\dir{-};"C3"+U;"x5"+D**\dir{-};
			"c1"+U;"C1"+D**\dir{-};"c2"+U;"C2"+D**\dir{-};"v1"+U;"V1"+D**\dir{-};"v2"+U;"V2"+D**\dir{-};
			<1.5em,1.35cm>*\as{\tikz[red,thick,dashed,baseline=0.9ex]\draw (0,0) rectangle (0.75cm,0.4cm);}="box",
		\endxy}\label{as:niim3}
	\end{xlist}}\label{as:niim}
\end{exe}
\end{multicols}
\begin{multicols}{3}
\begin{exe}
	\sn{\begin{xlist}
		\exi{d.}\raisebox{\dimexpr-\totalheight+5ex\relax}{\xy
			<0em,3.3cm>*\as{\x}="x1",<1em,3.3cm>*\as{\x}="x2",<2em,3.3cm>*\as{\x}="x3",<3em,3.3cm>*\as{\x}="x4",<4em,3.3cm>*\as{\x}="x5",
			<0em,2.3cm>*\as{C}="C1",<1em,2.3cm>*\as{\'V}="V1",<3em,2.3cm>*\as{C}="C2",<2em,2.3cm>*\as{V}="V2",<4em,2.3cm>*\as{C}="C3",
			<0em,1.3cm>*\as{n}="c1",<1em,1.3cm>*\as{i}="v1",<3em,1.3cm>*\as{m}="c2",<2em,1.3cm>*\as{a}="v2",<4em,1.3cm>*\as{}="c3",
			<1em,0cm>*\as{{$\left[\hspace{-2mm}\begin{array}{l}\textrm{\tsc{+high}}\\\textrm{\tsc{+front}}\end{array}\hspace{-2mm}\right]$}}="f",
			"f"+U;"v1"+D**\dir{-};"f"+U;"v2"+D**\dir{.};
			"C1"+U;"x1"+D**\dir{-};"C2"+U;"x4"+D**\dir{-};"V1"+U;"x2"+D**\dir{-};"V2"+U;"x3"+D**\dir{-};"C3"+U;"x5"+D**\dir{-};
			"c1"+U;"C1"+D**\dir{-};"c2"+U;"C2"+D**\dir{-};"v1"+U;"V1"+D**\dir{-};"v2"+U;"V2"+D**\dir{-};
			<1.5em,1.35cm>*\as{\tikz[red,thick,dashed,baseline=0.9ex]\draw (0,0) rectangle (0.75cm,0.4cm);}="box",
		\endxy}\label{as:niim4}
		\exi{e.}\raisebox{\dimexpr-\totalheight+5ex\relax}{\xy
			<0em,3.3cm>*\as{\x}="x1",<1em,3.3cm>*\as{\x}="x2",<2em,3.3cm>*\as{\x}="x3",<3em,3.3cm>*\as{\x}="x4",<4em,3.3cm>*\as{\x}="x5",
			<0em,2.3cm>*\as{C}="C1",<1em,2.3cm>*\as{\'V}="V1",<3em,2.3cm>*\as{C}="C2",<2em,2.3cm>*\as{V}="V2",<4em,2.3cm>*\as{\xc{C}}="C3",
			<0em,1.3cm>*\as{n}="c1",<1em,1.3cm>*\as{i}="v1",<3em,1.3cm>*\as{m}="c2",<2em,1.3cm>*\as{i}="v2",<4em,1.3cm>*\as{}="c3",
			<1.5em,0cm>*\as{{$\left[\hspace{-2mm}\begin{array}{l}\textrm{\tsc{+high}}\\\textrm{\tsc{+front}}\end{array}\hspace{-2mm}\right]$}}="f",
			"f"+U;"v1"+D**\dir{-};"f"+U;"v2"+D**\dir{-};
			"C1"+U;"x1"+D**\dir{-};"C2"+U;"x4"+D**\dir{-};"V1"+U;"x2"+D**\dir{-};"V2"+U;"x3"+D**\dir{-};"C3"+U;"x5"+D**\dir{-};
			"c1"+U;"C1"+D**\dir{-};"c2"+U;"C2"+D**\dir{-};"v1"+U;"V1"+D**\dir{-};"v2"+U;"V2"+D**\dir{-};
		\endxy}\label{as:niim5}
		\exi{f.}\raisebox{\dimexpr-\totalheight+5ex\relax}{\xy
			<0em,3.3cm>*\as{\x}="x1",<1em,3.3cm>*\as{\x}="x2",<2em,3.3cm>*\as{\x}="x3",<3em,3.3cm>*\as{\x}="x4",
			<0em,2.3cm>*\as{C}="C1",<1em,2.3cm>*\as{\'V}="V1",<3em,2.3cm>*\as{C}="C2",<2em,2.3cm>*\as{V}="V2",
			<0em,1.3cm>*\as{n}="c1",<1em,1.3cm>*\as{i}="v1",<3em,1.3cm>*\as{m}="c2",<2em,1.3cm>*\as{i}="v2",
			<1.5em,0cm>*\as{{$\left[\hspace{-2mm}\begin{array}{l}\textrm{\tsc{+high}}\\\textrm{\tsc{+front}}\end{array}\hspace{-2mm}\right]$}}="f",
			"f"+U;"v1"+D**\dir{-};"f"+U;"v2"+D**\dir{-};
			"C1"+U;"x1"+D**\dir{-};"C2"+U;"x4"+D**\dir{-};"V1"+U;"x2"+D**\dir{-};"V2"+U;"x3"+D**\dir{-};
			"c1"+U;"C1"+D**\dir{-};"c2"+U;"C2"+D**\dir{-};"v1"+U;"V1"+D**\dir{-};"v2"+U;"V2"+D**\dir{-};
		\endxy}\label{as:niim6}
	\end{xlist}}
\end{exe}
\end{multicols}

Under this analysis /a/ assimilation is triggered by the presence
of two immediately adjacent V-slots;
an environment which only occur in M\=/forms.
The lack of assimilation in U\=/forms such as \ve{kuan} `village' is
explained by the fact that there is an intervening C-slot between
the two vowels; \ve{ku{\gap}an}.
The environment necessary for the operation
of rule \qf{ex:'VaC->VC} is not present.

Assimilation of /a/ only occurs before filled C-slots.
That is, it does not occur in the M\=/form of words
such as \ve{kuan} {\ra} \ve{kua} `village'.
I analyse the lack of assimilation in such forms as being due
to the lack of a following filled C-slot.

The formation of \ve{kuan} {\ra} \ve{kua} `village'
is given in \qf{as:kua} below.
Metathesis at the CV tier occurs in \qf{as:kua2},
resulting in the V-slot to which /a/ is associated occurring directly after the stressed V-slot.
However, the following C-slot is empty.
This means the environment under which /a/ assimilation occurs is not present.
Thus, no assimilation takes place.
The final C-slot is then deleted in (\ref{as:kua}d){\textendash}(\ref{as:kua}e).

\begin{multicols}{3}
\begin{exe}
	\ex{\begin{xlist}
		\exa{\xy
			<0em,2cm>*\as{\x}="x1",<1em,2cm>*\as{\x}="x2",<2em,2cm>*\as{\x}="x3",<3em,2cm>*\as{\x}="x4",<4em,2cm>*\as{\x}="x5",
			<0em,1cm>*\as{C}="C1",<4em,1cm>*\as{C}="C3",<2em,1cm>*\as{C}="C2",
			"C1"+U;"x1"+D**\dir{-};"C2"+U;"x3"+D**\dir{-};"C3"+U;"x5"+D**\dir{-};
			<0em,0cm>*\as{k}="c1",<4em,0cm>*\as{n}="c3","c1"+U;"C1"+D**\dir{-};"c3"+U;"C3"+D**\dir{-};
			<1em,1cm>*\as{\'V}="V1",<3em,1cm>*\as{V}="V2","V1"+U;"x2"+D**\dir{-};"V2"+U;"x4"+D**\dir{-};
			<1em,0cm>*\as{u}="v1",<3em,0cm>*\as{a}="v2","v1"+U;"V1"+D**\dir{-};"v2"+U;"V2"+D**\dir{-};
		\endxy}\label{as:kua1}
		\exa{\xy
			<0em,2cm>*\as{\x}="x1",<1em,2cm>*\as{\x}="x2",<2em,2cm>*\as{\x}="x3",<3em,2cm>*\as{\x}="x4",<4em,2cm>*\as{\x}="x5",
			<0em,1cm>*\as{C}="C1",<4em,1cm>*\as{C}="C3",<2em,1cm>*\as{C}="C2",
			"C1"+U;"x1"+D**\dir{-};"C3"+U;"x5"+D**\dir{-};"C2"+U;"x4"+D**\dir{.};
			<0em,0cm>*\as{k}="c1",<4em,0cm>*\as{n}="c3","c1"+U;"C1"+D**\dir{-};"c3"+U;"C3"+D**\dir{-};
			<1em,1cm>*\as{\'V}="V1",<3em,1cm>*\as{V}="V2","V1"+U;"x2"+D**\dir{-};"V2"+U;"x3"+D**\dir{.};
			<1em,0cm>*\as{u}="v1",<3em,0cm>*\as{a}="v2","v1"+U;"V1"+D**\dir{-};"v2"+U;"V2"+D**\dir{-};
			<2.5em,1.5cm>*\as{\tikz[red,thick,dashed,baseline=0.9ex]\draw (0,0) rectangle (0.8cm,1.5cm);}="box",
		\endxy}\label{as:kua2}
		\exa{\xy
			<0em,2cm>*\as{\x}="x1",<1em,2cm>*\as{\x}="x2",<2em,2cm>*\as{\x}="x3",<3em,2cm>*\as{\x}="x4",<4em,2cm>*\as{\x}="x5",
			<0em,1cm>*\as{C}="C1",<4em,1cm>*\as{C}="C3",<3em,1cm>*\as{C}="C2",
			"C1"+U;"x1"+D**\dir{-};"C3"+U;"x5"+D**\dir{-};"C2"+U;"x4"+D**\dir{-};
			<0em,0cm>*\as{k}="c1",<4em,0cm>*\as{n}="c3","c1"+U;"C1"+D**\dir{-};"c3"+U;"C3"+D**\dir{-};
			<1em,1cm>*\as{\'V}="V1",<2em,1cm>*\as{V}="V2","V1"+U;"x2"+D**\dir{-};"V2"+U;"x3"+D**\dir{-};
			<1em,0cm>*\as{u}="v1",<2em,0cm>*\as{a}="v2","v1"+U;"V1"+D**\dir{-};"v2"+U;"V2"+D**\dir{-};
			<1.5em,0cm>*\as{\tikz[red,thick,dashed,baseline=0.9ex]\draw (0,0) rectangle (0.8cm,0.6cm);}="box",
			<3em,1cm>*\as{\tikz[red,thick,dashed,baseline=0.9ex]\draw (0,0) rectangle (0.4cm,0.6cm);}="box",
		\endxy}\label{as:kua3}
	\end{xlist}}\label{as:kua}
\end{exe}
\end{multicols}
\begin{multicols}{3}
\begin{exe}
	\sn{\begin{xlist}
		\exi{d.}\exia{\xy
			<0em,2cm>*\as{\x}="x1",<1em,2cm>*\as{\x}="x2",<2em,2cm>*\as{\x}="x3",<3em,2cm>*\as{\x}="x4",<4em,2cm>*\as{\x}="x5",
			<0em,1cm>*\as{C}="C1",<4em,1cm>*\as{\xc{C}}="C3",<3em,1cm>*\as{C}="C2",
			"C1"+U;"x1"+D**\dir{-};"C2"+U;"x4"+D**\dir{-};{\ar@{-}|-(.425)*@{|} |-{\hole} |-(.575)*@{|} "C3"+U;"x5"+D};
			<0em,0cm>*\as{k}="c1",<4em,0cm>*\as{n}="c3","c1"+U;"C1"+D**\dir{-};{\ar@{-}|-(.425)*@{|} |-{\hole} |-(.575)*@{|} "c3"+U;"C3"+D};
			<1em,1cm>*\as{\'V}="V1",<2em,1cm>*\as{V}="V2","V1"+U;"x2"+D**\dir{-};"V2"+U;"x3"+D**\dir{-};
			<1em,0cm>*\as{u}="v1",<2em,0cm>*\as{a}="v2","v1"+U;"V1"+D**\dir{-};"v2"+U;"V2"+D**\dir{-};
			<4em,1cm>*\as{\tikz[red,thick,dashed,baseline=0.9ex]\draw (0,0) rectangle (0.4cm,2.5cm);}="box",
		\endxy}
		\exi{e.}\exia{\xy
			<0em,2cm>*\as{\x}="x1",<1em,2cm>*\as{\x}="x2",<2em,2cm>*\as{\x}="x3",<3em,2cm>*\as{\x}="x4",
			<0em,1cm>*\as{C}="C1",<3em,1cm>*\as{C}="C2",
			"C1"+U;"x1"+D**\dir{-};"C2"+U;"x4"+D**\dir{-};
			<0em,0cm>*\as{k}="c1","c1"+U;"C1"+D**\dir{-};
			<1em,1cm>*\as{\'V}="V1",<2em,1cm>*\as{V}="V2","V1"+U;"x2"+D**\dir{-};"V2"+U;"x3"+D**\dir{-};
			<1em,0cm>*\as{u}="v1",<2em,0cm>*\as{a}="v2","v1"+U;"V1"+D**\dir{-};"v2"+U;"V2"+D**\dir{-};
		\endxy}
		\sna{\xy
			<0em,2cm>*\as{}="x1",<0em,1cm>*\as{}="C1",<0em,0cm>*\as{}="c1",
		\endxy}
	\end{xlist}}
\end{exe}
\end{multicols}

That /a/ is protected from assimilation by a following empty C-slot
finds comparative support from Ro{\Q}is Amarasi.
Ro{\Q}is Amarasi has a process whereby any final unstressed /a/
in a closed syllable (optionally) assimilates to the quality of the stressed vowel.
Examples are given in Table \ref{tab:RoqPosStrAAss}.\footnote{
		Assimilation of /a/ in Ro{\Q}is Amarasi does not usually occur
		before or after the glottal stop: i.e. \ve{keʔ\tbr{a}n}
		`room' and \ve{in\tbr{a}ʔ} `mother'.
		Assimilation of /a/ in closed final syllables also occurs in Timaus
		and the variety of Meto spoken in the village of Oepaha ---
		both adjacent to Ro{\Q}is Amarasi speaking areas.}

\begin{table}[h]
	\centering\caption{Ro{\Q}is Amarasi post-stress /a/ assimilation}\label{tab:RoqPosStrAAss}
		\begin{threeparttable}
			\begin{tabular}{lll}\lsptoprule
				Kotos Amarasi					& Ro{\Q}is Amarasi			& \\ \midrule
				\ve{ʔnim\tbr{a}-f}		&\ve{nim\tbr{i}-f}			& `arm, hand'\\
				\ve{sbet\tbr{a}-f}		&\ve{sbet\tbr{e}-f}			& `upper arm' \\
				\ve{ek\tbr{a}m}				&\ve{er\tbr{e}m, eram}	& `pandanus'  \\
				\ve{na-ten\tbr{a}b}		&\ve{na-ten\tbr{e}b}		& `thinks' \\
				\ve{ok\tbr{a}m}				&\ve{ok\tbr{o}m, okam}	& `melon, gourd'\\
				\ve{or\tbr{a}s}\su{†}	&\ve{or\tbr{o}s}				& `time' \\
				\ve{rum\tbr{a}n}			&\ve{rum\tbr{u}n}				& `empty' \\
				\ve{ut\tbr{a}n}				&\ve{ut\tbr{u}k, utak}	& `vegetables; pumpkin, squash'\su{‡}  \\
				\ve{sur\tbr{a}t}\su{\#}&\ve{sur\tbr{u}t}			& `paper, book' \\
			\lspbottomrule
		\end{tabular}%}
			\begin{tablenotes}
				\item [†]	From Portuguese \it{horas} [ɔras] `hours'.
				\item [‡]	Kotos Amarasi \ve{utan} means only `vegetables'.
				\item [\#]	From Malay \it{surat} `letter'.
			\end{tablenotes}
		\end{threeparttable}
\end{table}

Assimilation of /a/ in closed syllables in Ro{\Q}is is productive, as seen by forms such
as Ro{\Q}is \ve{in\tbr{a}ʔ} `mother' and \ve{maun in\tbr{i}-f} `hen',
as well as borrowings such as Portuguese \it{horas} [ɔr\tbr{a}s]
{\ra} Ro{\Q}is \ve{or\tbr{o}s}.
However, such assimilation does not occur in final
open syllables of a stem, even those which are followed by a consonant.
Two examples are Ro{\Q}is \ve{boʔ=es\tbr{a} =m hiut} `seventeen'
and \ve{n-ok\tbr{a} skoor} `with a school'.
This lack of assimilation can be explained by
the same mechanism which blocks assimilation of /a/ in M\=/forms:
a following empty C-slot.

The reason assimilation of /a/ does not
occur in Kotos Amarasi in the derivation of
an M\=/form from a VVC{\#} final root can also be accounted
for due to the difficulty of recovering
the U\=/form based on the M\=/form if assimilation took place.
In a stem such as \ve{nima} {\ra} \ve{niim} `five'
the majority of its segments are preserved in the M\=/form
with three out of four segments retained.
Similarly, a stem such \ve{rinah} {\ra} \ve{riin} `pomegranate, wood apple'
also preserves most of its segments with the M\=/form
attesting three out of five segments of the U\=/form.
However, in a hypothetical case such as \ve{kuan} {\ra} \it{*kuu} `village'
only two out of four segments would be preserved after metathesis.\footnote{
		M\=/forms such as \ve{mainuan} {\ra} \ve{mainua} `openness' which
		also do not undergo assimilation of final /a/
		show that the final foot, not the stem, is the relevant domain
		which determines whether enough segments are preserved after metathesis.}

Finally, as discussed in \srf{sec:AssOfA}, in Ro{\Q}is Amarasi
/a/ does not assimilate when the metathesising consonant is a glottal stop,
as seen in Ro{\Q}is \ve{n-ro\tbr{ʔa}} {\ra} \ve{n-ro\tbr{aʔ}} `vomit'
which contrasts with Kotos \ve{n-ro\tbr{ʔa}} {\ra} \ve{n-ro\tbr{oʔ}} `vomit'.
No matter how this is explained, it does provide evidence that
the kind of consonant in the C-slot which moves after
metathesis plays a role in /a/ assimilation.
In Kotos Amarasi only an unfilled final C-slot protects
/a/ from undergoing assimilation after metathesis,
while in Ro{\Q}is Amarasi this is extended to final
C-slots which are filled by glottal stop --- the minimal consonant.

\subsection{Summary}
In this section I have proposed a single unified
analysis of the formation of the M\=/form
from the U\=/form in Amarasi.
This analysis is framed under an autosegmental model of phonology \citep{go76}
and a rule-based model of process morphology \citep{ma74,an92}.
My analysis consists of three parts:
one phonological stipulation, one morphological rule,
and one morphemically conditioned rule.
The phonological stipulation and the rules
are repeated in \qf{ex3:Ft->CVCVC}--\qf{ex2:'VaC->VC} below.

\begin{exe}
	\ex{Ft {\ra} CVCVC}\label{ex3:Ft->CVCVC}
	\ex{CV {\ra} VC /\'V{\gap}}\label{ex2:CV->VC/'V-}
	\exa{\xy
		<0em,2cm>*\as{\'V}="V1",<1em,2cm>*\as{V}="V2",<2em,2cm>*\as{C}="C1",
		<0em,1cm>*\as{}="v1",<1em,1cm>*\as{a}="v2",<2em,1cm>*\as{}="c1",
		<0em,0cm>*\as{\tsc{[+F.]}}="f","f"+U;"v1"+D**\dir{-};"f"+U;"v2"+D**\dir{.};
		"v1"+U;"V1"+D**\dir{-};"c1"+U;"C1"+D**\dir{-};"v2"+U;"V2"+D**\dir{-};
	\endxy}\label{ex2:'VaC->VC}
\end{exe}

These two rules and single stipulation,
combined with the general phonotactic constraints of Amarasi,
are sufficient to account for formation of the M\=/forms.
The general phonotactic constraints of Amarasi with which these rules
interact are given below:

\begin{itemize}
	\item clusters of two C-slots are prohibited word finally
	\item sequences of three surface vowels are prohibited
	\item vowel sequences consisting of a high-vowel and mid-vowel are prohibited
\end{itemize}

This rule-based analysis of Amarasi accounts for
all of the data in a single consistent way.
In the next section I consider some alternate analyses.
While these analyses can account for some of the data Amarasi data
they cannot account for all of the data.
