\subsection{Vowel assimilation}\label{sec:VowAss}
Two kinds of vowel assimilation occur in the derivation of Amarasi M\=/forms:
mid vowel height assimilation and assimilation of final /a/.

\subsubsection{Mid vowel assimilation}\label{sec:MidVowAss}
When the final vowel is mid and the penultimate vowel is high,
the penultimate vowel is raised to high after metathesis.
The surface relationship between the U\=/form and M\=/form of
\ve{tune} [ˈt̪ʊnɛ] {\ra} \ve{tuin} [ˈt̪ʊ.in] `gewang palm' is shown in \qf{as:tune/tuin} below,
with more examples given in \qf{ex:V+hiCV+mid->V+hiV+hiC}.

\begin{exe}
	\exa{\xy
		<0em,2.5cm>*\as{`gewang'}="gloss",
		<2.5em,2cm>*\as{t}="u1",<3.5em,2cm>*\as{u}="u2",<4.5em,2cm>*\as{n}="u3",<5.5em,2cm>*\as{e}="u4",<0em,2cm>*\as{U\=/form:}="u",
		<2.5em,1.5cm>*\as{C}="uC1",<3.5em,1.5cm>*\as{V}="uC2",<4.5em,1.5cm>*\as{C}="uC3",<5.5em,1.5cm>*\as{V}="uC4",
		<2.5em,0.5cm>*\as{C}="mC1",<3.5em,0.5cm>*\as{V}="mC2",<4.5em,0.5cm>*\as{V}="mC4",<5.5em,0.5cm>*\as{C}="mC3",
		<2.5em,0cm>*\as{t}="m1",<3.5em,0cm>*\as{u}="m2",<4.5em,0cm>*\as{i}="m4",<5.5em,0cm>*\as{n}="m3",<0em,0cm>*\as{M\=/form:}="m",<4.05em,0cm>*\as{>}="ar",
		{\ar@{->} "uC1"+D;"mC1"+U};{\ar@{->} "uC2"+D;"mC2"+U};{\ar@{->} "uC3"+D;"mC3"+U};{\ar@{->} "uC4"+D;"mC4"+U};
	\endxy}\label{as:tune/tuin} 
	\ex{{\ldots}V\sub{1}\tsc{[+high]}C\sub{1}V\sub{2}\tsc{[-high,-low]}{\#} {\ra} 
			{\ldots}V\sub{1}\tsc{[+high]}V\sub{2}\tsc{[+high]}C\sub{1}{\#}}\label{ex:V+hiCV+mid->V+hiV+hiC}
	\gw\sn{\begin{tabular}{rcll}
		 U\=/form					&			&\mc{2}{l}{M\=/form}								\\
		\ve{u\tbr{me}}	&{\ra}&\ve{u\tbr{im}}		&`house'				\\
		\ve{pu\tbr{ne}ʔ}&{\ra}&\ve{pu\tbr{in}}	&`grain-head'		\\
		\ve{tu\tbr{ne}}	&{\ra}&\ve{tu\tbr{in}}	&`gewang palm'	\\
		\ve{ni\tbr{ne}ʔ}	&{\ra}&\ve{ni\tbr{in}}	&`edge; wing'	\\
		\ve{na-hi\tbr{ne}}	&{\ra}&\ve{n-hi\tbr{in}}&`knows'	\\
		\ve{n-si\tbr{mo}}	&{\ra}&\ve{n-si\tbr{um}}&`receives	(poetic)'\\
		\end{tabular}}
\end{exe}

Words with this shape are uncommon in my corpus
with only 26 attestations out of a total of 1,913 unique lexical roots (1.4\%).
Additionally, the majority of such words have
variant U\=/forms in which the final vowel is raised to high.
Examples include \ve{um\tbr{e}} {\tl} \ve{um\tbr{i}} `house',
\ve{tun\tbr{e}} {\tl} \ve{tun\tbr{i}} `gewang palm',
\ve{na-hin\tbr{e}} {\tl} \ve{na-hin\tbr{i}} `knows',
and \ve{nin\tbr{e}ʔ} {\tl} \ve{nin\tbr{i}ʔ} `edge; wing'.

Vowel sequences of a high vowel followed by a mid vowel are not found in Amarasi;
there are no attestations of \ve{*ie}, \ve{*io}, \ve{*ue} or \ve{*uo}.
For this reason, the mid vowel assimilation observed when the final vowel
is high and the penultimate vowel is mid
can be explained by the phonotactic constraints of the language.

Different kinds of height assimilation occur to different
extents in other varieties of Meto.
For instance, in some varieties of Baikeno
/e/ is not raised after /u/, as seen in \ve{u\tbr{me}} {\ra} \ve{u\tbr{em}} `house',
though it is raised after /i/, as seen in \ve{na-hi\tbr{ne}} {\ra} \ve{na-hi\tbr{in}} `know'.
On the other hand, in some varieties of Baikeno, Miomafo, Amanuban, and
Amanatun high vowels lower to mid after /a/.
Baikeno examples include \ve{a\tbr{su}} {\ra} \ve{a\tbr{os}} `dog',
\ve{ma\tbr{nu}} {\ra} \ve{ma\tbr{on}} `chicken', \ve{n-a\tbr{mi}} {\ra} \ve{n-a\tbr{em}} `look for',
and \ve{la\tbr{si}} {\ra} \ve{la\tbr{es}} `matter'.

\subsubsection{Assimilation of /a/}\label{sec:AssOfA}
The second kind of vowel assimilation in the formation of M\=/forms is assimilation of /a/.
The M\=/form of words which end in CVa{\#} is formed via consonant-vowel metathesis with
complete assimilation of /a/ to the quality of the first vowel.
The surface relationship between the forms
\ve{nima} [ˈnimɐ] {\ra} \ve{niim} [ˈniːm] `five' is shown in \qf{as:nuka/nuuk}.
More examples are given in \qf{ex:VCa->VVC}.

\begin{exe}
	\exa{\xy
		<0em,2.5cm>*\as{`five'}="gloss",
		<2.5em,2cm>*\as{n}="u1",<3.5em,2cm>*\as{i}="u2",<4.5em,2cm>*\as{m}="u3",<5.5em,2cm>*\as{a}="u4",<0em,2cm>*\as{U\=/form:}="u",
		<2.5em,1.5cm>*\as{C}="uC1",<3.5em,1.5cm>*\as{V}="uC2",<4.5em,1.5cm>*\as{C}="uC3",<5.5em,1.5cm>*\as{V}="uC4",
		<2.5em,0.5cm>*\as{C}="mC1",<3.5em,0.5cm>*\as{V}="mC2",<4.5em,0.5cm>*\as{V}="mC4",<5.5em,0.5cm>*\as{C}="mC3",
		<2.5em,0cm>*\as{n}="m1",<3.5em,0cm>*\as{i}="m2",<4.5em,0cm>*\as{i}="m4",<5.5em,0cm>*\as{m}="m3",<0em,0cm>*\as{M\=/form:}="m",
		{\ar@{->} "uC1"+D;"mC1"+U};{\ar@{->} "uC2"+D;"mC2"+U};{\ar@{->} "uC3"+D;"mC3"+U};{\ar@{->} "uC2"+D;"mC4"+U};
	\endxy}\label{as:nuka/nuuk}
	\ex{{\ldots}V{\sub{α}}Ca{\#} {\ra} {\ldots}V{\sub{α}}V{\sub{α}}C{\#}}\label{ex:VCa->VVC}
	\gw\sn{\begin{tabular}{rcll|rcll}
		 U\=/form						&		&\mc{2}{l|}{M\=/form}					&U\=/form						&		&\mc{2}{l}{M\=/form}					\\
		\ve{ni\tbr{ma}}		&\ra&\ve{ni\tbr{im}}	&`five'		&\ve{n-bi\tbr{ba}}&\ra&\ve{n-bi\tbr{ib}}&`massage'\\
		\ve{n-ne\tbr{na}}	&\ra&\ve{n-ne\tbr{en}}&`hears'	&\ve{fe\tbr{fa}-f}&\ra&\ve{fe\tbr{ef}}	&`mouth'	\\
		\ve{n-so\tbr{sa}}	&\ra&\ve{n-so\tbr{os}}&`buys'		&\ve{so\tbr{na}-f}&\ra&\ve{so\tbr{on}}	&`palace'	\\
		\ve{n-tu\tbr{pa}}	&\ra&\ve{n-tu\tbr{up}}&`sleeps'	&\ve{su\tbr{na}-f}&\ra&\ve{su\tbr{un}}	&`horn'		\\
		\end{tabular}}
\end{exe}

Vowel sequences in which the second vowel is /a/ freely
occur before consonants in U\=/forms,
with 55 examples in my database.
Examples include \ve{siah} `part of the loom',
\ve{mneas} `hulled rice', \ve{koaʔ} `friarbird',
and \ve{kuan} `village'.

The assimilation of /a/ in M\=/forms
is an example of a derived environment effect \citep{ki73,keki77},
a phonological rule which only operates after the application of another rule.
In this case, metathesis triggers assimilation of /a/.

Assimilation of /a/ after metathesis is regionally common.
It is attested in most varieties of Meto for which data is available,
as well as Mambae (\srf{sec:Mam}) and Funai Helong (\srf{sec:Hel}).
However, there are at least two cases from Meto
in which final /a/ does not assimilate after metathesis.

Firstly, in Kusa-Manea final /a/ is preserved after metathesis.
Examples of Kusa-Manea forms with final /a/ retained after metathesis
are given in \trf{tab:KusManMforFinA}
alongside their unmetathesised and metathesised
Kotos Amarasi cognates for comparison.
\trf{tab:KusManMforFinA} also shows several nouns
metathesised before a vowel-initial enclitic (Chapter \ref{ch:PhoMet}).
%(Unlike Amarasi, Kusa-Manea verbs are often cited without a prefix.)

\begin{table}[h]
	\centering\caption{Kusa-Manea M-forms with final /a/}\label{tab:KusManMforFinA}
	\begin{tabular}{llll}\lsptoprule
			\mc{2}{l}{Kotos Amarasi}				& \mc{2}{l}{Kusa-Manea} \\
			U\=/form				& M\=/form 					& M\=/form 							& gloss\\ \midrule
			\ve{nima}			& \ve{niim}				& \ve{ni\tbr{am}}			& `five'\\
			\ve{n-nena}		& \ve{n-neen}			& \ve{ne\tbr{an}}			& `hear'\\
			\ve{nema}			& \ve{neem}				& \ve{ne\tbr{am}}			& `(s/he) comes'\\
			\ve{n-sena}		& \ve{n-seen}			& \ve{se\tbr{an}}			& `plant (v.)'\\
			\ve{na-tefa}	& \ve{na-teef}		& \ve{na-te\tbr{af}}	& `meet'\\
			\ve{n-roʔa}		& \ve{n-rooʔ}			& \ve{ro\tbr{aʔ}}			& `vomit'\\
			\ve{n-sosa}		& \ve{n-soos}			& \ve{so\tbr{as}}			& `buy'\\
			\ve{na-ʔura}	& \ve{na-ʔuur}		& \ve{na-ʔu\tbr{ar}}	& `rain'\\
			\ve{n-runa}		& \ve{n-ruun}			& \ve{ru\tbr{an}}			& `tattoo'\\
			\ve{suna-n}		& \ve{suun-n=aa}	& \ve{su\tbr{an}-n=aa}	& `horn'\\
			\ve{bonak}		& \ve{boonk=aa}		& \ve{bo\tbr{an}k=aa}		& `pandanus'\\
			\ve{funan}		& \ve{fuunn=aa}		& \ve{fu\tbr{an}n=aa}		& `moon'\\
%			\ve{}	& \ve{}	& \ve{\tbr{}}	& `'\\
%			\ve{}	& \ve{}	& \ve{\tbr{}}	& `'\\
		\lspbottomrule
	\end{tabular}
\end{table}

That this is not a process of dissimilation of the second
vowel of a sequence is shown by forms such as Kusa-Manea \ve{ba{\tl}booh} `cough'
and \ve{skiik} `brush teeth' each of which can be compared with
Amarasi \ve{n-boho} {\ra} \ve{n-booh} `cough'
and \ve{na-skiki} {\ra} \ve{na-skiik} `brush teeth'.

Secondly, assimilation of /a/ does not occur in Ro{\Q}is Amarasi
after metathesis when the consonant before
the final vowel is a glottal stop.
Examples are given in \trf{tab:RoqMforFinA}
which shows a number of Ro{\Q}is metathesised ʔa(C){\#} final words
alongside their U\=/forms and Kotos cognates.
There is some evidence that final /a/ does undergo
assimilation after the glottal stop in Ro{\Q}is
for M\=/forms marking nominal attributive modification,
though there is only one example in my database:
\ve{ke\tbr{ʔa}n} `room' + \ve{susu-f} `corner' {\ra}
\ve{ke\tbr{eʔ} susu-f} `corner of a room'.
If this is a regular pattern, it would be consistent
with the data from consonant deletion whereby
the M\=/forms marking nominal attribution have an
additional phonological process compared with other M\=/forms.

\begin{table}[h]
	\centering\caption[Ro{\Q}is Amarasi M\=/forms with final /aʔ/]
	{Ro{\Q}is Amarasi M\=/forms with final /aʔ/\su{†}}\label{tab:RoqMforFinA}
		\begin{threeparttable}
	\begin{tabular}{lllll}\lsptoprule
			\mc{2}{l}{Ro{\Q}is Amarasi}			& \mc{2}{l}{Kotos Amarasi} 					&\\
				U\=/form	&	M\=/form 	&	U\=/form	&	M\=/form	&	gloss	\\\midrule
				\ve{poʔan}	&	\ve{po\tbr{aʔ}n=ini}	&	\ve{poʔon}	&	\ve{pooʔn=eni}	&	`orchard(s)'	\\
				\ve{n-peʔa}	&	\ve{n-pe\tbr{aʔ}}	&	\ve{n-peʔe}	&	\ve{n-peeʔ}	&	`break, crack'	\\
				\ve{na-koʔan}	&	\ve{na-ko\tbr{aʔ}n}	&	\ve{na-koʔon}	&	\ve{}	&	`bark (dog)'	\\
				\ve{amfoʔan}	&	\ve{amfo\tbr{aʔ}n}	&	\ve{amfoʔan}	&	\ve{}	&	`Amfo{\Q}an'	\\
				\ve{na-peʔan}	&	\ve{na-pe\tbr{aʔ}n}	&	\ve{}	&	\ve{}	&	`create'	\\
				\ve{}	&	\ve{n-ro\tbr{aʔ}}	&	\ve{n-roʔa}	&	\ve{n-rooʔ}	&	`vomit'	\\
				\ve{}	&	\ve{n-tu\tbr{aʔ}}	&	\ve{n-tuʔu}	&	\ve{n-tuuʔ}	&	`end'	\\
				\ve{}	&	\ve{n-po\tbr{aʔ}}	&	\ve{n-poʔo}	&	\ve{n-pooʔ}	&	`herd'	\\
				\ve{}	&	\ve{n-ri\tbr{aʔ}}	&	\ve{n-riʔi}	&	\ve{n-riiʔ}	&	`fold'	\\
				\ve{}	&	\ve{tri\tbr{aʔ}-n=aa}	&	\ve{triʔi-f}	&	\ve{}	&	`cubital fossa'	\\
%			\ve{ʔ}		&\ve{ʔ}	& \ve{ʔ}		& \ve{\tbr{aʔ}}		&`'\\
%			\ve{ʔ}		&\ve{ʔ}	& \ve{ʔ}		& \ve{\tbr{aʔ}}		&`'\\
		\lspbottomrule
			\end{tabular}%}
			\begin{tablenotes}
				\item [†]	Empty cells are currently unattested.
									Some Kotos U\=/forms have assimilation of final /a/
									after a glottal stop to the quality of the penultimate vowel.
			\end{tablenotes}
		\end{threeparttable}
\end{table}

While only /a/ undergoes complete assimilation in Kotos Amarasi,
in some other varieties of Meto \emph{all} vowels
undergo assimilation after metathesis.
Varieties in which this has been attested to some extent
include Baikeno, Amfo{\Q}an, Timaus, Kopas, and Miomafo.
Examples of vowel assimilation after metathesis
from Naitbelak Amfo{\Q}an are given in \trf{tab:AssOfV2AmfNaiMfo}.

\begin{table}[ht]
	\caption{Amfo{\Q}an (Naitbelak) vowel assimilation after metathesis}\label{tab:AssOfV2AmfNaiMfo}
	\centering
		\begin{tabular}{llll} \lsptoprule
										&Amarasi						&	Amfo{\Q}an 				&\\
			U\=/form				&M\=/form  						&M\=/form 						& gloss \\ \midrule
			\ve{hitu}			&\ve{h\tbr{iu}t}		&\ve{h\tbr{ii}t}		& `seven' \\
			\ve{n-inu}		&\ve{n-\tbr{iu}n}		&\ve{n-\tbr{ii}n}		& `drink' \\
			\ve{na-kinu}	&\ve{na-k\tbr{iu}n}	&\ve{na-\tbr{kii}n}	& `spit' \\
			\ve{na-r/leko}&\ve{na-r\tbr{eo}k}	&\ve{na-l\tbr{ee}k}	& `good' \\
			\ve{na-henu}	&\ve{na-h\tbr{eu}n}	&\ve{na-h\tbr{ee}n}	& `fill' \\
			\ve{tenu}			&\ve{t\tbr{eu}n}		&\ve{t\tbr{ee}n}		& `three' \\
			\ve{n-mani}		&\ve{n-m\tbr{ai}n}	&\ve{a|n-m\tbr{aa}n}& `laugh' \\
			\ve{n-mate}		&\ve{n-m\tbr{ae}t}	&\ve{a|n-m\tbr{aa}t}& `die' \\
			\ve{n-hake}		&\ve{n-h\tbr{ae}k}	&\ve{a|n-h\tbr{aa}k}& `stand' \\
			\ve{fanu}			&\ve{f\tbr{au}n}		&\ve{f\tbr{aa}n}		& `eight' \\ \lspbottomrule
		\end{tabular}
\end{table}

Complete vowel assimilation in Naitbelak Amfo{\Q}an
means that the allophones of the mid vowels /e/ and /o/
are marginally contrastive in this variety.
As discussed in \srf{sec:Vow}, the mid vowels are realised as mid-high [e] and [o]
before high vowels and as mid-low [ɛ] and [ɔ] elsewhere.
In Naitbelak Amfo{\Q}an these vowels often retain
this quality after assimilation of the conditioning vowel.
This results in contrasts such as \ve{na-leko} `is good'
{\ra} \ve{na-leok} {\ra} \ve{na-leek} {\ra} [naˈl\tbr{ɛː}k]
and \ve{na-henu} {\ra} \ve{na-heun} {\ra} \ve{na-heen} {\ra} [naˈh\tbr{eː}n].
\citet{st93,st96,st96b} reports a similar phenomenon in his Miomafo data. %\footnote{
		%In my Amfo'an data assimilation of vowels other than /a/ appears to be optional.
		%Forms such as \ve{na-leko} {\ra} \ve{na-leok} `good' occur alongside assimilated
		%forms such as \ve{na-leek}.
		%Vowel assimilation is overwhelmingly the most common pattern in my data.}
		
Kirsten Culhane (p.c. October 2018) reports that in Nai{\Q}bais Amfo{\Q}an
all vowels undergo complete assimilation after metathesis except for \ve{io},
for which only height assimilation is unattested
(e.g. \ve{n-s\tbr{i}m\tbr{o}} {\ra} \ve{n-s\tbr{iu}m} `receive'),
as well as \ve{oi} and \ve{oe}, for which full assimilation is optional.

\subsubsection{Quantification of M\=/forms ending in V{\A}V{\A}C}\label{sec:QuaMfoEndVVC}
In my description, words ending in VCa{\#} and V{\sA}CV{\sA}{\#}
yield M\=/forms ending in a sequence of two identical vowels followed
by a consonant; V{\sA}V{\sA}C{\#}.
This differs from previous descriptions of Meto,
which report deletion of the final vowel, with no further changes
\citep{st93,st96,st96b,mccko96,blga98,st08}.
%Thus, for instance, \citet[285--286]{st08} gives the M\=/form of \it{bibi} `goat' as \it{bib}
%and the M\=/form of \it{penaʔ} `maize' as \it{pɛn}.

I have not found vowel deletion of this kind
for any variety of Meto for which I have data.
Instead, with the exception of Kusa-Manea, for which
VCa{\#} words simply metathesise,
the M\=/form of such words has a sequence of two identical vowels.
This was demonstrated for Amarasi
by refining the instrumental acoustic study of vowel length
summarised in \srf{sec:QuaLenVowSeq}
(see \trf{tab:VowLenAma} on \prf{tab:VowLenAma}),
in which I showed a sequence of two identical vowels
is on average 31{\%} longer than a single vowel in Amarasi.

I further refined this study by separating
vowel sequences created through metathesis
from vowel sequences which are underlying in U\=/forms.
Of the 314 sequences of identical vowels measured,
242 represent a sequence of identical vowels in an M\=/form,
such as \ve{nima} {\ra} \ve{niim} `five' or \ve{fini} {\ra} \ve{fiin} `seed'
and 72 represent words which contain a sequence of two identical vowels in the U\=/form,
such as \ve{too} `citizens' or \ve{toon} `year'.
The average lengths of each of these kinds of words is given in \trf{tab:SeqIdeVowMfoUfor}.

\begin{table}[h]%For significance http://xkcd.com/1478/
	\centering
	\caption{Sequences of identical Vowels in M-forms and U-forms}\label{tab:SeqIdeVowMfoUfor}
			\begin{tabular}{rrrrr}\lsptoprule
			V{\sA}V{\sA}(C){\#}		&\mc{1}{c}{all}&U\=/form	& M\=/form		&\mc{1}{c}{V}\\ \midrule
%			sum all tokens (sec.)	&40.373	& 9.115					& 31.257		& 46.414 \\
			average length (sec.)	&0.129	&0.127					&\tbr{0.129}&\tbr{0.098}	\\
			number of tokens			&314		& 72						& 242				& 472					\\
			standard deviation		&0.05		& 0.059					& 0.047			& 0.034			\\
			t-test (vs. M\=/form)		&				&\it{p} = 0.759	&						&\it{p} <0.001	\\
			\lspbottomrule
		\end{tabular}
\end{table}

\trf{tab:SeqIdeVowMfoUfor} shows that vowel sequences created through metathesis
are on average 31{\%} longer than a single vowel in Amarasi.
This difference is statistically significant, as shown by a two tailed t-test.
Sequences of two identical vowels created through metathesis
are also a fraction longer than those which are underlying in U\=/forms.
This difference is not statistically significant.\footnote{
		Separating M\=/forms created through metathesis and assimilation of final /a/,
		(e.g. \ve{nima} {\ra} \ve{niim} `five') from those created through
		metathesis with identical penultimate and final vowels,
		(e.g. \ve{fini} {\ra} \ve{fiin} `seed'),
		revealed no statistically significant differences.}

Given this acoustic analysis, the simplest
analysis of the data is to propose that words
whose U\=/forms ends in VCa{\#} or  V{\sA}CV{\sA}{\#}
have M\=/forms with a double vowel in Amarasi.
This is the pattern for all lexical words in Amarasi.
There are a small number of functors in Amarasi for which the M\=/form
\emph{is} formed by deletion of the final vowel.
Such examples are discussed in \srf{sec:IrrMfor}.

Data collected from other varieties of Meto, including
Amanatun, Amanuban, Amfo{\Q}an, Baikeno, Fatule{\Q}u,
Kopas, Ketun, Kusa-Manea, Molo, and Timaus provides no
evidence that U\=/forms ending in VCa{\#} or V{\sA}CV{\sA}{\#}
derive their M\=/form by final vowel deletion.
Thus, earlier reports of vowel deletion in such forms
appear to have arisen due to the realisation of
two identical vowels as a single long vowel (\srf{sec:DouVow}),
with analysts interpreting phonetically long vowels
as a realisation of a single vowel.
