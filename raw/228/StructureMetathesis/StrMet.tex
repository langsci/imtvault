\chapter{Structure of metathesis}\label{ch:StrMetAma}

\SelectTips{lu}{12}
%\SelectTips{cm}{12}
%\SelectTips{eu}{12}

\section{Introduction}\label{sec:Int}
In this chapter I describe and analyse the form of metathesis in Amarasi.
At its most simple, metathesis involves the
reversal of the final consonant-vowel sequence of a word.
One example is the word `stone' which has the unmetathesised form
\ve{fa\tbr{tu}} [ˈfat̪ʊ] and the metathesised form \ve{fa\tbr{ut}} [ˈfa.ʊt̪].
This example shows the pattern C\sub{1}V\sub{2}C\sub{3}V\sub{4}
{\ra} C\sub{1}V\sub{2}V\sub{4}C\sub{3},
illustrated in \qf{as:fatu->faut} below.

\begin{multicols}{3}
	\begin{exe}
		\ex{\begin{xlist}
			\exa{\xy
				<1.1em,2cm>*\as{\x}="x1",<2.2em,2cm>*\as{\x}="x2",<3.3em,2cm>*\as{\x}="x3",<4.4em,2cm>*\as{\x}="x4",
				<1.1em,1cm>*\as{\hp{\sub{1}}C\sub{1}}="CV1",<2.2em,1cm>*\as{\hp{\sub{2}}V\sub{2}}="CV2",
				<3.3em,1cm>*\as{\hp{\sub{3}}C\sub{3}}="CV3",<4.4em,1cm>*\as{\hp{\sub{4}}V\sub{4}}="CV4",
				<1.1em,0cm>*\as{f}="cv1",<2.2em,0cm>*\as{a}="cv2",<3.3em,0cm>*\as{t}="cv3",<4.4em,0cm>*\as{u}="cv4",
				"cv1"+U;"CV1"+D**\dir{-};"cv2"+U;"CV2"+D**\dir{-};"cv3"+U;"CV3"+D**\dir{-};"cv4"+U;"CV4"+D**\dir{-};
				"CV1"+U;"x1"+D**\dir{-};"CV2"+U;"x2"+D**\dir{-};"CV3"+U;"x3"+D**\dir{-};"CV4"+U;"x4"+D**\dir{-};
			\endxy}
			\exa{\xy
				<1.1em,2cm>*\as{\x}="x1",<2.2em,2cm>*\as{\x}="x2",<3.3em,2cm>*\as{\x}="x3",<4.4em,2cm>*\as{\x}="x4",
				<1.1em,1cm>*\as{\hp{\sub{1}}C\sub{1}}="CV1",<2.2em,1cm>*\as{\hp{\sub{2}}V\sub{2}}="CV2",
				<3.3em,1cm>*\as{\hp{\sub{3}}C\sub{3}}="CV3",<4.4em,1cm>*\as{\hp{\sub{4}}V\sub{4}}="CV4",
				<1.1em,0cm>*\as{f}="cv1",<2.2em,0cm>*\as{a}="cv2",<3.3em,0cm>*\as{t}="cv3",<4.4em,0cm>*\as{u}="cv4",
				"cv1"+U;"CV1"+D**\dir{-};"cv2"+U;"CV2"+D**\dir{-};"cv3"+U;"CV3"+D**\dir{-};"cv4"+U;"CV4"+D**\dir{-};
				"CV1"+U;"x1"+D**\dir{-};"CV2"+U;"x2"+D**\dir{-};"CV4"+U;"x3"+D**\dir{-};"CV3"+U;"x4"+D**\dir{-};
			\endxy}
			\exa{\xy
				<1.1em,2cm>*\as{\x}="x1",<2.2em,2cm>*\as{\x}="x2",<3.3em,2cm>*\as{\x}="x3",<4.4em,2cm>*\as{\x}="x4",
				<1.1em,1cm>*\as{\hp{\sub{1}}C\sub{1}}="CV1",<2.2em,1cm>*\as{\hp{\sub{2}}V\sub{2}}="CV2",
				<3.3em,1cm>*\as{\hp{\sub{4}}V\sub{4}}="CV3",<4.4em,1cm>*\as{\hp{\sub{3}}C\sub{3}}="CV4",
				<1.1em,0cm>*\as{f}="cv1",<2.2em,0cm>*\as{a}="cv2",<3.3em,0cm>*\as{u}="cv3",<4.4em,0cm>*\as{t}="cv4",
				"cv1"+U;"CV1"+D**\dir{-};"cv2"+U;"CV2"+D**\dir{-};"cv3"+U;"CV3"+D**\dir{-};"cv4"+U;"CV4"+D**\dir{-};
				"CV1"+U;"x1"+D**\dir{-};"CV2"+U;"x2"+D**\dir{-};"CV3"+U;"x3"+D**\dir{-};"CV4"+U;"x4"+D**\dir{-};
			\endxy}
		\end{xlist}}\label{as:fatu->faut}
	\end{exe}
\end{multicols}

Metathesis is mostly straightforward with words
that instantiate all and only CVCV.
However, words with other shapes also occur in Amarasi (\srf{sec:RooStr}).
Depending on the phonotactic structure of the word,
metathesis is associated with other phonological processes, including
vowel deletion, consonant deletion, and two kinds of vowel assimilation.
These different processes are illustrated in Table \ref{tab:PhoProAssMetAma}.

\begin{table}[h]
	\centering\caption{Phonological processes associated with metathesis}\label{tab:PhoProAssMetAma}
		\stl{0.2em}\begin{tabular}{rrclcccccl}\lsptoprule
			shape			&U						&\ra&M	&met.&C{\#}{\ra}{\0}	&/a/ {\ra} V\sub{1} &V{\ra}\tsc{high}	&V{\#}{\ra}{\0}	\\ \midrule
			VCV{\#}		& \ve{fatu}		&\ra&\ve{faut}	&✔&&&&&`stone'\\
			VCVC{\#}	& \ve{muʔit}	&\ra&\ve{muiʔ}	&✔&✔&&&&`animal' \\
			VCa{\#}		& \ve{nima}		&\ra&\ve{niim}	&✔&&✔&&&`five'\\
			VCV{\#}		& \ve{ume}		&\ra&\ve{uim}		&✔&&&✔&&`house' \\
			VVC{\#}		& \ve{kaut}		&\ra&\ve{kau}		&&✔&&&&`papaya' \\
			VVCV{\#}	& \ve{aunu}		&\ra&\ve{aun}		&&&&&✔&`spear' \\
			VVCVC{\#}	& \ve{nautus}	&\ra&\ve{naut}	&&✔&&&✔&`beetle' \\
		\lspbottomrule
	\end{tabular}
\end{table}

From the examples in Table \ref{tab:PhoProAssMetAma}, 
it is clear that many of the forms before
and after the arrow do not differ only
in the order of the final CV sequence.
This is the reason for referring to forms
paradigmatically equivalent to \ve{fatu} `stone'
as the `U\=/form' and forms paradigmatically
equivalent to \ve{faut} `stone' as the `M\=/form'.

By positing an obligatory CVCVC foot
in which C-slots can be empty,
under a process-based model of morphology
and an autosegmental model of phonology,
all the phonological processes
in the formation of the M\=/form arise
from a single rule of metathesis at the CV tier,
an associated morphemically conditioned rule,
and the general phonotactic constraints of Amarasi.
This analysis is superior to alternate analyses
under different frameworks, such as prosodic morphology,
or purely concatenative morphology,
which cannot account for all the data
in a consistent, typologically plausible manner.

For some word shapes it is possible to identify
formally distinct M\=/forms which occur in different environments.
The distribution of each of these three M\=/forms
is summarised in \trf{tab:MfoMuqAni}.

\begin{table}[h]
	\centering
	\caption{M\=/forms of \ve{muʔit} `animal' and \ve{fatu} `stone'}\label{tab:MfoMuqAni}
		\begin{tabular}{lcllll} \lsptoprule
			U\=/form		&		&	M\=/forms			& gloss	& environments 										&\\ \midrule
			\ve{muʔit}&\ra&	\ve{muiʔ}		&	{\M}	& (C)V-initial nominal modifiers,	&\srf{sec:BasMfo}\\
			\ve{fatu}	&		&	\ve{faut}		& 			&default verbal form							&\\
								&\ra&	\ve{muʔi}		&	{\Mc}	& CC-initial modifiers 						&\srf{sec:CCIniMod}\\
								&		&	\ve{fatu}		&				&																	&\\
								&\ra&	\ve{muiʔt}	& {\Mv}	& vowel-initial enclitics 				&Ch. \ref{ch:PhoMet}\\
								&		&	\ve{faatgw}	&				&																	&\\
			\lspbottomrule
		\end{tabular}
\end{table}

Firstly, there is the basic M\=/form which is the form
taken by nouns when modified by another nominal
and which is also the default form of vowel-final verbs.
Examples include \ve{fa\tbr{tu}} {\ra} \ve{fa\tbr{ut}} `stone'
and \ve{mu\tbr{ʔit}} {\ra} \ve{mu\tbr{iʔ}} `animal'.
Basic M\=/forms are indicated with the gloss `{\M}'.
Their structure is discussed in \srf{sec:BasMfo}.

Secondly, there is an M\=/form taken by nominals
when modified by a word with an initial consonant cluster.
One example is \ve{muʔi\tbr{t}} {\ra} \ve{muʔi} `animal'.
These \mbox{M\=/forms} are indicated with the gloss {\Mc}.
Their structure is discussed in \srf{sec:CCIniMod}.

Thirdly, there is an M\=/form taken by all words before vowel-initial enclitics.
Examples include \ve{fa\tbr{tu}} {\ra} \ve{fa\tbr{atgw}} `stone'
and \ve{mu\tbr{ʔi}t} {\ra} \ve{mu\tbr{iʔ}t} `animal'.
M\=/forms before enclitics are indicated with the gloss {\Mv}.
Unlike the other M\=/forms, M\=/forms used before enclitics
can be analysed as purely phonologically conditioned
and their structure is described and analysed in Chapter \ref{ch:PhoMet}.

The M\=/forms discussed in this chapter
are the M\=/forms with a morphological use:
the basic M\=/form ({\M}) and the {\MC}-form
used before attributive modifiers with an initial consonant cluster.

While these two forms occur in
different phonological environments,
this does not mean that the use of these forms
is triggered by these environments.
Instead, which form is used is triggered
by syntactic or discourse factors.
An analogy is the English verbal third person singular agreement suffix \it{-s}.
This suffix has three forms /əz/, /z/ and /s/,
each of which is used in different environments.
When the English \tsc{3sg} agreement suffix occurs,
it has different forms as determined by its phonological environment.
However, it is not these phonological environments
which determine whether this suffix occurs or not;
instead, morphosyntactic factors determine this.

While the focus in this chapter is on the structure
of metathesis in Kotos Amarasi as spoken in Nekmese{\Q} by
inhabitants of the old hamlet of Koro{\Q}oto, I provide
comparative data from other varieties of Meto when they are
known to differ from Kotos Amarasi.

\section{Basic M\=/form}\label{sec:BasMfo}
In this section I describe the structure of the basic M\=/form.
This is the form taken by nouns when modified by a nominal
which does not begin with a consonant cluster
and is the default form for vowel-final verbs.
While the functions of these M\=/forms are described
in full detail in chapters \ref{ch:SynMet} and \ref{ch:DisMet},
I provide here a brief overview as context for the following discussion.

Within the noun phrase M\=/forms are a construct form (\srf{sec:ConFor})
used when an attributive modifier occurs within the noun phrase.
Compare examples \qf{ex:StoAreBig} and \qf{ex:BigSto}.
Each phrase consists of the noun \ve{fatu} {\tl} \ve{faut} `stone'
followed by the modifier \ve{koʔu} `big, great'.
In \qf{ex:StoAreBig} \ve{fatu} `stone' is in the U\=/form
and the modifier has a predicative reading.
In \qf{ex:BigSto} \ve{faut} `stone' is in the M\=/form,
and the modifier has an attributive meaning.
The syntactic structures of each of these phrases
are represented with trees in \qf{tr:StoAreBig} and \qf{tr:BigSto} respectively.

\begin{multicols}{2}
	\begin{exe}
		\ex{\gll \brac{NP} fa\tbr{tu} \bracr{} \brac{NP} koʔu \bracr{}\\
						%	\hp{\brac{NP}} fatu {} {} koʔu {}\\
							{} stone {} {} big {}\\
				\glt \lh{\brac{NP} }`Stones are big.'}\label{ex:StoAreBig}
		\ex{\gll \brac{NP} fa\tbr{ut} koʔu \bracr{}\\
						%	\hp{\brac{NP}} fatu koʔu {}\\
							{} stone big {}\\
				\glt \lh{\brac{NP} }`(a) big stone'}\label{ex:BigSto}
	\end{exe}
\end{multicols}

\begin{multicols}{2}
	\begin{exe}
		\ex{\begin{forest} %where n children=0{tier=word}{}
			[S,[NP,[N,[\ve{fa\tbr{tu}}\\stlone{\tbrU}]]][NP,[N,[\ve{koʔu}\\big{\U}]]]]
		\end{forest}}\label{tr:StoAreBig}
		\ex{\begin{forest} %where n children=0{tier=word}{}
			[S,[\vp{NP}{\ldots},[,phantom]][NP,[N,[\ve{fa\tbr{ut}}\\stlone{\tbrM}]][N,[\ve{koʔu}\\big{\U}]]]]
		\end{forest}}\label{tr:BigSto}
	\end{exe}
\end{multicols}

A number of different phonological processes occur in
the formation of the basic M\=/form according to the shape of the U\=/form stem.
These processes include metathesis (\srf{sec:BasPro}),
consonant deletion (\srf{sec:MetConDel}, \srf{sec:ConDel}),
two kinds of vowel assimilation (\srf{sec:VowAss}), and vowel deletion (\srf{sec:VowDel}).

\subsection{Metathesis}\label{sec:BasPro}
When a root ends in VCV{\#},
the M\=/form is formed by metathesis of the final consonant-vowel sequence.
The surface relationship between the segments of
\ve{fatu} [ˈfat̪ʊ] {\ra} \ve{faut} [ˈfa.ʊt̪] `stone' is shown in \qf{as:fatu/faut1},
with more examples in \qf{ex:V1CV2->V1V2C}.

\begin{exe}
	\exa{\xy
		<0em,2.5cm>*\as{`stone'}="gloss",
		<2.5em,2cm>*\as{f}="u1",<3.5em,2cm>*\as{a}="u2",<4.5em,2cm>*\as{t}="u3",<5.5em,2cm>*\as{u}="u4",<0em,2cm>*\as{U\=/form:}="u",
		<2.5em,1.5cm>*\as{C}="uC1",<3.5em,1.5cm>*\as{V}="uC2",<4.5em,1.5cm>*\as{C}="uC3",<5.5em,1.5cm>*\as{V}="uC4",
		<2.5em,0.5cm>*\as{C}="mC1",<3.5em,0.5cm>*\as{V}="mC2",<4.5em,0.5cm>*\as{V}="mC4",<5.5em,0.5cm>*\as{C}="mC3",
		<2.5em,0cm>*\as{f}="m1",<3.5em,0cm>*\as{a}="m2",<4.5em,0cm>*\as{u}="m4",<5.5em,0cm>*\as{t}="m3",<0em,0cm>*\as{M\=/form:}="m",
		{\ar@{->} "uC1"+D;"mC1"+U};{\ar@{->} "uC2"+D;"mC2"+U};{\ar@{->} "uC3"+D;"mC3"+U};{\ar@{->} "uC4"+D;"mC4"+U};
	\endxy}\label{as:fatu/faut1}
\end{exe}

\begin{exe}
\ex{{\ldots}V\sub{1}CV\sub{2}{\#} {\ra} {\ldots}V\sub{1}V\sub{2}C{\#}}\label{ex:V1CV2->V1V2C}
	\sn{\gw\begin{tabular}{rcll|rcll}
		 U\=/form					&			&\mc{2}{l}{M\=/form}					&U\=/form						&			&\mc{2}{l}{M\=/form}	\\
		\ve{fi\tbr{ni}}	&{\ra}&\ve{fi\tbr{in}}	&`seed'		&\ve{ne\tbr{no}}	&{\ra}&\ve{ne\tbr{on}}	&`day; sky'\\
		\ve{be\tbr{si}}	&{\ra}&\ve{be\tbr{is}}	&`knife'	&\ve{kna\tbr{fo}}	&{\ra}&\ve{kna\tbr{of}}	&`mouse'\\
		\ve{fa\tbr{fi}}	&{\ra}&\ve{fa\tbr{if}}	&`pig'		&\ve{ko\tbr{ro}}	&{\ra}&\ve{ko\tbr{or}}	&`bird'\\
		\ve{o\tbr{ni}}	&{\ra}&\ve{o\tbr{in}}		&`bee'		&\ve{hi\tbr{tu}}	&{\ra}&\ve{hi\tbr{ut}}	&`seven'\\
		\ve{u\tbr{ki}}	&{\ra}&\ve{u\tbr{ik}}		&`banana'	&\ve{te\tbr{nu}}	&{\ra}&\ve{te\tbr{un}}	&`three'	\\
		\ve{re\tbr{ne}}	&{\ra}&\ve{re\tbr{en}}	&`field'	&\ve{fa\tbr{tu}}	&{\ra}&\ve{fa\tbr{ut}}	&`stone'\\
		\ve{ba\tbr{re}}	&{\ra}&\ve{ba\tbr{er}}	&`place'	&\ve{no\tbr{pu}}	&{\ra}&\ve{no\tbr{up}}	&`hole'\\
		\ve{no\tbr{pe}}	&{\ra}&\ve{no\tbr{ep}}	&`cloud'	&\ve{hu\tbr{tu}}	&{\ra}&\ve{hu\tbr{ut}}	&`louse'\\
		\end{tabular}}
\end{exe}

It is worth emphasising that in most cases the order
of the final consonant and vowel of the word is the only phonetic difference
between the U\=/form and the M\=/form of VCV{\#} final roots.
Metathesis is not accompanied by any reduction in the number of syllables
nor by any change in the placement of stress.\footnote{
		The only exceptions are words with identical penultimate
		and final vowels such as \ve{fini} [ˈfini] {\ra} \ve{fiin} [ˈfiːn] `seed',
		in which case there is a reduction in the number of phonetic syllables
		and thus arguably also in the placement of stress.
		As discussed in \srf{sec:DouVow}
		there is no basis for analysing sequence of two identical vowels
		differently from sequences of different vowels.}

Such metathesis applies to all VCV{\#} final roots,
with the exception of roots in which the final vowel is /a/ (\srf{sec:AssOfA})
or when the penultimate vowel is high and the final vowel is mid (\srf{sec:MidVowAss}).
Such roots undergo metathesis followed by vowel assimilation.

	\subsection{Vowel assimilation}\label{sec:VowAss}
Two kinds of vowel assimilation occur in the derivation of Amarasi M\=/forms:
mid vowel height assimilation and assimilation of final /a/.

\subsubsection{Mid vowel assimilation}\label{sec:MidVowAss}
When the final vowel is mid and the penultimate vowel is high,
the penultimate vowel is raised to high after metathesis.
The surface relationship between the U\=/form and M\=/form of
\ve{tune} [ˈt̪ʊnɛ] {\ra} \ve{tuin} [ˈt̪ʊ.in] `gewang palm' is shown in \qf{as:tune/tuin} below,
with more examples given in \qf{ex:V+hiCV+mid->V+hiV+hiC}.

\begin{exe}
	\exa{\xy
		<0em,2.5cm>*\as{`gewang'}="gloss",
		<2.5em,2cm>*\as{t}="u1",<3.5em,2cm>*\as{u}="u2",<4.5em,2cm>*\as{n}="u3",<5.5em,2cm>*\as{e}="u4",<0em,2cm>*\as{U\=/form:}="u",
		<2.5em,1.5cm>*\as{C}="uC1",<3.5em,1.5cm>*\as{V}="uC2",<4.5em,1.5cm>*\as{C}="uC3",<5.5em,1.5cm>*\as{V}="uC4",
		<2.5em,0.5cm>*\as{C}="mC1",<3.5em,0.5cm>*\as{V}="mC2",<4.5em,0.5cm>*\as{V}="mC4",<5.5em,0.5cm>*\as{C}="mC3",
		<2.5em,0cm>*\as{t}="m1",<3.5em,0cm>*\as{u}="m2",<4.5em,0cm>*\as{i}="m4",<5.5em,0cm>*\as{n}="m3",<0em,0cm>*\as{M\=/form:}="m",<4.05em,0cm>*\as{>}="ar",
		{\ar@{->} "uC1"+D;"mC1"+U};{\ar@{->} "uC2"+D;"mC2"+U};{\ar@{->} "uC3"+D;"mC3"+U};{\ar@{->} "uC4"+D;"mC4"+U};
	\endxy}\label{as:tune/tuin} 
	\ex{{\ldots}V\sub{1}\tsc{[+high]}C\sub{1}V\sub{2}\tsc{[-high,-low]}{\#} {\ra} 
			{\ldots}V\sub{1}\tsc{[+high]}V\sub{2}\tsc{[+high]}C\sub{1}{\#}}\label{ex:V+hiCV+mid->V+hiV+hiC}
	\gw\sn{\begin{tabular}{rcll}
		 U\=/form					&			&\mc{2}{l}{M\=/form}								\\
		\ve{u\tbr{me}}	&{\ra}&\ve{u\tbr{im}}		&`house'				\\
		\ve{pu\tbr{ne}ʔ}&{\ra}&\ve{pu\tbr{in}}	&`grain-head'		\\
		\ve{tu\tbr{ne}}	&{\ra}&\ve{tu\tbr{in}}	&`gewang palm'	\\
		\ve{ni\tbr{ne}ʔ}	&{\ra}&\ve{ni\tbr{in}}	&`edge; wing'	\\
		\ve{na-hi\tbr{ne}}	&{\ra}&\ve{n-hi\tbr{in}}&`knows'	\\
		\ve{n-si\tbr{mo}}	&{\ra}&\ve{n-si\tbr{um}}&`receives	(poetic)'\\
		\end{tabular}}
\end{exe}

Words with this shape are uncommon in my corpus
with only 26 attestations out of a total of 1,913 unique lexical roots (1.4\%).
Additionally, the majority of such words have
variant U\=/forms in which the final vowel is raised to high.
Examples include \ve{um\tbr{e}} {\tl} \ve{um\tbr{i}} `house',
\ve{tun\tbr{e}} {\tl} \ve{tun\tbr{i}} `gewang palm',
\ve{na-hin\tbr{e}} {\tl} \ve{na-hin\tbr{i}} `knows',
and \ve{nin\tbr{e}ʔ} {\tl} \ve{nin\tbr{i}ʔ} `edge; wing'.

Vowel sequences of a high vowel followed by a mid vowel are not found in Amarasi;
there are no attestations of \ve{*ie}, \ve{*io}, \ve{*ue} or \ve{*uo}.
For this reason, the mid vowel assimilation observed when the final vowel
is high and the penultimate vowel is mid
can be explained by the phonotactic constraints of the language.

Different kinds of height assimilation occur to different
extents in other varieties of Meto.
For instance, in some varieties of Baikeno
/e/ is not raised after /u/, as seen in \ve{u\tbr{me}} {\ra} \ve{u\tbr{em}} `house',
though it is raised after /i/, as seen in \ve{na-hi\tbr{ne}} {\ra} \ve{na-hi\tbr{in}} `know'.
On the other hand, in some varieties of Baikeno, Miomafo, Amanuban, and
Amanatun high vowels lower to mid after /a/.
Baikeno examples include \ve{a\tbr{su}} {\ra} \ve{a\tbr{os}} `dog',
\ve{ma\tbr{nu}} {\ra} \ve{ma\tbr{on}} `chicken', \ve{n-a\tbr{mi}} {\ra} \ve{n-a\tbr{em}} `look for',
and \ve{la\tbr{si}} {\ra} \ve{la\tbr{es}} `matter'.

\subsubsection{Assimilation of /a/}\label{sec:AssOfA}
The second kind of vowel assimilation in the formation of M\=/forms is assimilation of /a/.
The M\=/form of words which end in CVa{\#} is formed via consonant-vowel metathesis with
complete assimilation of /a/ to the quality of the first vowel.
The surface relationship between the forms
\ve{nima} [ˈnimɐ] {\ra} \ve{niim} [ˈniːm] `five' is shown in \qf{as:nuka/nuuk}.
More examples are given in \qf{ex:VCa->VVC}.

\begin{exe}
	\exa{\xy
		<0em,2.5cm>*\as{`five'}="gloss",
		<2.5em,2cm>*\as{n}="u1",<3.5em,2cm>*\as{i}="u2",<4.5em,2cm>*\as{m}="u3",<5.5em,2cm>*\as{a}="u4",<0em,2cm>*\as{U\=/form:}="u",
		<2.5em,1.5cm>*\as{C}="uC1",<3.5em,1.5cm>*\as{V}="uC2",<4.5em,1.5cm>*\as{C}="uC3",<5.5em,1.5cm>*\as{V}="uC4",
		<2.5em,0.5cm>*\as{C}="mC1",<3.5em,0.5cm>*\as{V}="mC2",<4.5em,0.5cm>*\as{V}="mC4",<5.5em,0.5cm>*\as{C}="mC3",
		<2.5em,0cm>*\as{n}="m1",<3.5em,0cm>*\as{i}="m2",<4.5em,0cm>*\as{i}="m4",<5.5em,0cm>*\as{m}="m3",<0em,0cm>*\as{M\=/form:}="m",
		{\ar@{->} "uC1"+D;"mC1"+U};{\ar@{->} "uC2"+D;"mC2"+U};{\ar@{->} "uC3"+D;"mC3"+U};{\ar@{->} "uC2"+D;"mC4"+U};
	\endxy}\label{as:nuka/nuuk}
	\ex{{\ldots}V{\sub{α}}Ca{\#} {\ra} {\ldots}V{\sub{α}}V{\sub{α}}C{\#}}\label{ex:VCa->VVC}
	\gw\sn{\begin{tabular}{rcll|rcll}
		 U\=/form						&		&\mc{2}{l|}{M\=/form}					&U\=/form						&		&\mc{2}{l}{M\=/form}					\\
		\ve{ni\tbr{ma}}		&\ra&\ve{ni\tbr{im}}	&`five'		&\ve{n-bi\tbr{ba}}&\ra&\ve{n-bi\tbr{ib}}&`massage'\\
		\ve{n-ne\tbr{na}}	&\ra&\ve{n-ne\tbr{en}}&`hears'	&\ve{fe\tbr{fa}-f}&\ra&\ve{fe\tbr{ef}}	&`mouth'	\\
		\ve{n-so\tbr{sa}}	&\ra&\ve{n-so\tbr{os}}&`buys'		&\ve{so\tbr{na}-f}&\ra&\ve{so\tbr{on}}	&`palace'	\\
		\ve{n-tu\tbr{pa}}	&\ra&\ve{n-tu\tbr{up}}&`sleeps'	&\ve{su\tbr{na}-f}&\ra&\ve{su\tbr{un}}	&`horn'		\\
		\end{tabular}}
\end{exe}

Vowel sequences in which the second vowel is /a/ freely
occur before consonants in U\=/forms,
with 55 examples in my database.
Examples include \ve{siah} `part of the loom',
\ve{mneas} `hulled rice', \ve{koaʔ} `friarbird',
and \ve{kuan} `village'.

The assimilation of /a/ in M\=/forms
is an example of a derived environment effect \citep{ki73,keki77},
a phonological rule which only operates after the application of another rule.
In this case, metathesis triggers assimilation of /a/.

Assimilation of /a/ after metathesis is regionally common.
It is attested in most varieties of Meto for which data is available,
as well as Mambae (\srf{sec:Mam}) and Funai Helong (\srf{sec:Hel}).
However, there are at least two cases from Meto
in which final /a/ does not assimilate after metathesis.

Firstly, in Kusa-Manea final /a/ is preserved after metathesis.
Examples of Kusa-Manea forms with final /a/ retained after metathesis
are given in \trf{tab:KusManMforFinA}
alongside their unmetathesised and metathesised
Kotos Amarasi cognates for comparison.
\trf{tab:KusManMforFinA} also shows several nouns
metathesised before a vowel-initial enclitic (Chapter \ref{ch:PhoMet}).
%(Unlike Amarasi, Kusa-Manea verbs are often cited without a prefix.)

\begin{table}[h]
	\centering\caption{Kusa-Manea M-forms with final /a/}\label{tab:KusManMforFinA}
	\begin{tabular}{llll}\lsptoprule
			\mc{2}{l}{Kotos Amarasi}				& \mc{2}{l}{Kusa-Manea} \\
			U\=/form				& M\=/form 					& M\=/form 							& gloss\\ \midrule
			\ve{nima}			& \ve{niim}				& \ve{ni\tbr{am}}			& `five'\\
			\ve{n-nena}		& \ve{n-neen}			& \ve{ne\tbr{an}}			& `hear'\\
			\ve{nema}			& \ve{neem}				& \ve{ne\tbr{am}}			& `(s/he) comes'\\
			\ve{n-sena}		& \ve{n-seen}			& \ve{se\tbr{an}}			& `plant (v.)'\\
			\ve{na-tefa}	& \ve{na-teef}		& \ve{na-te\tbr{af}}	& `meet'\\
			\ve{n-roʔa}		& \ve{n-rooʔ}			& \ve{ro\tbr{aʔ}}			& `vomit'\\
			\ve{n-sosa}		& \ve{n-soos}			& \ve{so\tbr{as}}			& `buy'\\
			\ve{na-ʔura}	& \ve{na-ʔuur}		& \ve{na-ʔu\tbr{ar}}	& `rain'\\
			\ve{n-runa}		& \ve{n-ruun}			& \ve{ru\tbr{an}}			& `tattoo'\\
			\ve{suna-n}		& \ve{suun-n=aa}	& \ve{su\tbr{an}-n=aa}	& `horn'\\
			\ve{bonak}		& \ve{boonk=aa}		& \ve{bo\tbr{an}k=aa}		& `pandanus'\\
			\ve{funan}		& \ve{fuunn=aa}		& \ve{fu\tbr{an}n=aa}		& `moon'\\
%			\ve{}	& \ve{}	& \ve{\tbr{}}	& `'\\
%			\ve{}	& \ve{}	& \ve{\tbr{}}	& `'\\
		\lspbottomrule
	\end{tabular}
\end{table}

That this is not a process of dissimilation of the second
vowel of a sequence is shown by forms such as Kusa-Manea \ve{ba{\tl}booh} `cough'
and \ve{skiik} `brush teeth' each of which can be compared with
Amarasi \ve{n-boho} {\ra} \ve{n-booh} `cough'
and \ve{na-skiki} {\ra} \ve{na-skiik} `brush teeth'.

Secondly, assimilation of /a/ does not occur in Ro{\Q}is Amarasi
after metathesis when the consonant before
the final vowel is a glottal stop.
Examples are given in \trf{tab:RoqMforFinA}
which shows a number of Ro{\Q}is metathesised ʔa(C){\#} final words
alongside their U\=/forms and Kotos cognates.
There is some evidence that final /a/ does undergo
assimilation after the glottal stop in Ro{\Q}is
for M\=/forms marking nominal attributive modification,
though there is only one example in my database:
\ve{ke\tbr{ʔa}n} `room' + \ve{susu-f} `corner' {\ra}
\ve{ke\tbr{eʔ} susu-f} `corner of a room'.
If this is a regular pattern, it would be consistent
with the data from consonant deletion whereby
the M\=/forms marking nominal attribution have an
additional phonological process compared with other M\=/forms.

\begin{table}[h]
	\centering\caption[Ro{\Q}is Amarasi M\=/forms with final /aʔ/]
	{Ro{\Q}is Amarasi M\=/forms with final /aʔ/\su{†}}\label{tab:RoqMforFinA}
		\begin{threeparttable}
	\begin{tabular}{lllll}\lsptoprule
			\mc{2}{l}{Ro{\Q}is Amarasi}			& \mc{2}{l}{Kotos Amarasi} 					&\\
				U\=/form	&	M\=/form 	&	U\=/form	&	M\=/form	&	gloss	\\\midrule
				\ve{poʔan}	&	\ve{po\tbr{aʔ}n=ini}	&	\ve{poʔon}	&	\ve{pooʔn=eni}	&	`orchard(s)'	\\
				\ve{n-peʔa}	&	\ve{n-pe\tbr{aʔ}}	&	\ve{n-peʔe}	&	\ve{n-peeʔ}	&	`break, crack'	\\
				\ve{na-koʔan}	&	\ve{na-ko\tbr{aʔ}n}	&	\ve{na-koʔon}	&	\ve{}	&	`bark (dog)'	\\
				\ve{amfoʔan}	&	\ve{amfo\tbr{aʔ}n}	&	\ve{amfoʔan}	&	\ve{}	&	`Amfo{\Q}an'	\\
				\ve{na-peʔan}	&	\ve{na-pe\tbr{aʔ}n}	&	\ve{}	&	\ve{}	&	`create'	\\
				\ve{}	&	\ve{n-ro\tbr{aʔ}}	&	\ve{n-roʔa}	&	\ve{n-rooʔ}	&	`vomit'	\\
				\ve{}	&	\ve{n-tu\tbr{aʔ}}	&	\ve{n-tuʔu}	&	\ve{n-tuuʔ}	&	`end'	\\
				\ve{}	&	\ve{n-po\tbr{aʔ}}	&	\ve{n-poʔo}	&	\ve{n-pooʔ}	&	`herd'	\\
				\ve{}	&	\ve{n-ri\tbr{aʔ}}	&	\ve{n-riʔi}	&	\ve{n-riiʔ}	&	`fold'	\\
				\ve{}	&	\ve{tri\tbr{aʔ}-n=aa}	&	\ve{triʔi-f}	&	\ve{}	&	`cubital fossa'	\\
%			\ve{ʔ}		&\ve{ʔ}	& \ve{ʔ}		& \ve{\tbr{aʔ}}		&`'\\
%			\ve{ʔ}		&\ve{ʔ}	& \ve{ʔ}		& \ve{\tbr{aʔ}}		&`'\\
		\lspbottomrule
			\end{tabular}%}
			\begin{tablenotes}
				\item [†]	Empty cells are currently unattested.
									Some Kotos U\=/forms have assimilation of final /a/
									after a glottal stop to the quality of the penultimate vowel.
			\end{tablenotes}
		\end{threeparttable}
\end{table}

While only /a/ undergoes complete assimilation in Kotos Amarasi,
in some other varieties of Meto \emph{all} vowels
undergo assimilation after metathesis.
Varieties in which this has been attested to some extent
include Baikeno, Amfo{\Q}an, Timaus, Kopas, and Miomafo.
Examples of vowel assimilation after metathesis
from Naitbelak Amfo{\Q}an are given in \trf{tab:AssOfV2AmfNaiMfo}.

\begin{table}[ht]
	\caption{Amfo{\Q}an (Naitbelak) vowel assimilation after metathesis}\label{tab:AssOfV2AmfNaiMfo}
	\centering
		\begin{tabular}{llll} \lsptoprule
										&Amarasi						&	Amfo{\Q}an 				&\\
			U\=/form				&M\=/form  						&M\=/form 						& gloss \\ \midrule
			\ve{hitu}			&\ve{h\tbr{iu}t}		&\ve{h\tbr{ii}t}		& `seven' \\
			\ve{n-inu}		&\ve{n-\tbr{iu}n}		&\ve{n-\tbr{ii}n}		& `drink' \\
			\ve{na-kinu}	&\ve{na-k\tbr{iu}n}	&\ve{na-\tbr{kii}n}	& `spit' \\
			\ve{na-r/leko}&\ve{na-r\tbr{eo}k}	&\ve{na-l\tbr{ee}k}	& `good' \\
			\ve{na-henu}	&\ve{na-h\tbr{eu}n}	&\ve{na-h\tbr{ee}n}	& `fill' \\
			\ve{tenu}			&\ve{t\tbr{eu}n}		&\ve{t\tbr{ee}n}		& `three' \\
			\ve{n-mani}		&\ve{n-m\tbr{ai}n}	&\ve{a|n-m\tbr{aa}n}& `laugh' \\
			\ve{n-mate}		&\ve{n-m\tbr{ae}t}	&\ve{a|n-m\tbr{aa}t}& `die' \\
			\ve{n-hake}		&\ve{n-h\tbr{ae}k}	&\ve{a|n-h\tbr{aa}k}& `stand' \\
			\ve{fanu}			&\ve{f\tbr{au}n}		&\ve{f\tbr{aa}n}		& `eight' \\ \lspbottomrule
		\end{tabular}
\end{table}

Complete vowel assimilation in Naitbelak Amfo{\Q}an
means that the allophones of the mid vowels /e/ and /o/
are marginally contrastive in this variety.
As discussed in \srf{sec:Vow}, the mid vowels are realised as mid-high [e] and [o]
before high vowels and as mid-low [ɛ] and [ɔ] elsewhere.
In Naitbelak Amfo{\Q}an these vowels often retain
this quality after assimilation of the conditioning vowel.
This results in contrasts such as \ve{na-leko} `is good'
{\ra} \ve{na-leok} {\ra} \ve{na-leek} {\ra} [naˈl\tbr{ɛː}k]
and \ve{na-henu} {\ra} \ve{na-heun} {\ra} \ve{na-heen} {\ra} [naˈh\tbr{eː}n].
\citet{st93,st96,st96b} reports a similar phenomenon in his Miomafo data. %\footnote{
		%In my Amfo'an data assimilation of vowels other than /a/ appears to be optional.
		%Forms such as \ve{na-leko} {\ra} \ve{na-leok} `good' occur alongside assimilated
		%forms such as \ve{na-leek}.
		%Vowel assimilation is overwhelmingly the most common pattern in my data.}
		
Kirsten Culhane (p.c. October 2018) reports that in Nai{\Q}bais Amfo{\Q}an
all vowels undergo complete assimilation after metathesis except for \ve{io},
for which only height assimilation is unattested
(e.g. \ve{n-s\tbr{i}m\tbr{o}} {\ra} \ve{n-s\tbr{iu}m} `receive'),
as well as \ve{oi} and \ve{oe}, for which full assimilation is optional.

\subsubsection{Quantification of M\=/forms ending in V{\A}V{\A}C}\label{sec:QuaMfoEndVVC}
In my description, words ending in VCa{\#} and V{\sA}CV{\sA}{\#}
yield M\=/forms ending in a sequence of two identical vowels followed
by a consonant; V{\sA}V{\sA}C{\#}.
This differs from previous descriptions of Meto,
which report deletion of the final vowel, with no further changes
\citep{st93,st96,st96b,mccko96,blga98,st08}.
%Thus, for instance, \citet[285--286]{st08} gives the M\=/form of \it{bibi} `goat' as \it{bib}
%and the M\=/form of \it{penaʔ} `maize' as \it{pɛn}.

I have not found vowel deletion of this kind
for any variety of Meto for which I have data.
Instead, with the exception of Kusa-Manea, for which
VCa{\#} words simply metathesise,
the M\=/form of such words has a sequence of two identical vowels.
This was demonstrated for Amarasi
by refining the instrumental acoustic study of vowel length
summarised in \srf{sec:QuaLenVowSeq}
(see \trf{tab:VowLenAma} on \prf{tab:VowLenAma}),
in which I showed a sequence of two identical vowels
is on average 31{\%} longer than a single vowel in Amarasi.

I further refined this study by separating
vowel sequences created through metathesis
from vowel sequences which are underlying in U\=/forms.
Of the 314 sequences of identical vowels measured,
242 represent a sequence of identical vowels in an M\=/form,
such as \ve{nima} {\ra} \ve{niim} `five' or \ve{fini} {\ra} \ve{fiin} `seed'
and 72 represent words which contain a sequence of two identical vowels in the U\=/form,
such as \ve{too} `citizens' or \ve{toon} `year'.
The average lengths of each of these kinds of words is given in \trf{tab:SeqIdeVowMfoUfor}.

\begin{table}[h]%For significance http://xkcd.com/1478/
	\centering
	\caption{Sequences of identical Vowels in M-forms and U-forms}\label{tab:SeqIdeVowMfoUfor}
			\begin{tabular}{rrrrr}\lsptoprule
			V{\sA}V{\sA}(C){\#}		&\mc{1}{c}{all}&U\=/form	& M\=/form		&\mc{1}{c}{V}\\ \midrule
%			sum all tokens (sec.)	&40.373	& 9.115					& 31.257		& 46.414 \\
			average length (sec.)	&0.129	&0.127					&\tbr{0.129}&\tbr{0.098}	\\
			number of tokens			&314		& 72						& 242				& 472					\\
			standard deviation		&0.05		& 0.059					& 0.047			& 0.034			\\
			t-test (vs. M\=/form)		&				&\it{p} = 0.759	&						&\it{p} <0.001	\\
			\lspbottomrule
		\end{tabular}
\end{table}

\trf{tab:SeqIdeVowMfoUfor} shows that vowel sequences created through metathesis
are on average 31{\%} longer than a single vowel in Amarasi.
This difference is statistically significant, as shown by a two tailed t-test.
Sequences of two identical vowels created through metathesis
are also a fraction longer than those which are underlying in U\=/forms.
This difference is not statistically significant.\footnote{
		Separating M\=/forms created through metathesis and assimilation of final /a/,
		(e.g. \ve{nima} {\ra} \ve{niim} `five') from those created through
		metathesis with identical penultimate and final vowels,
		(e.g. \ve{fini} {\ra} \ve{fiin} `seed'),
		revealed no statistically significant differences.}

Given this acoustic analysis, the simplest
analysis of the data is to propose that words
whose U\=/forms ends in VCa{\#} or  V{\sA}CV{\sA}{\#}
have M\=/forms with a double vowel in Amarasi.
This is the pattern for all lexical words in Amarasi.
There are a small number of functors in Amarasi for which the M\=/form
\emph{is} formed by deletion of the final vowel.
Such examples are discussed in \srf{sec:IrrMfor}.

Data collected from other varieties of Meto, including
Amanatun, Amanuban, Amfo{\Q}an, Baikeno, Fatule{\Q}u,
Kopas, Ketun, Kusa-Manea, Molo, and Timaus provides no
evidence that U\=/forms ending in VCa{\#} or V{\sA}CV{\sA}{\#}
derive their M\=/form by final vowel deletion.
Thus, earlier reports of vowel deletion in such forms
appear to have arisen due to the realisation of
two identical vowels as a single long vowel (\srf{sec:DouVow}),
with analysts interpreting phonetically long vowels
as a realisation of a single vowel.

	\subsection{Consonant deletion}\label{sec:ConDel (Higher Level)}
Word-final consonants of nominals are deleted in the formation of the M\=/form.
This process is unique to the derivation of nominal M\=/forms
and does not affect other word classes, of which the largest is verbs.
This means that consonant-final verbs, such as \ve{na-tuin} `follow'
or \ve{n-boʔis} `praise' do not have basic M\=/forms.

\subsubsection{Metathesis and consonant deletion}\label{sec:MetConDel}
Words with a final consonant (CVC{\#}) derive their M\=/form through metathesis
of the penultimate consonant with the final vowel
and deletion of the final consonant.
The surface relationship between
\ve{muʔit} [ˈmʊʔit̪] {\ra} \ve{muiʔ} [ˈmʊ.iʔ] `animal'
is shown in \qf{as:muqit/muiq} below,
with more examples given in \qf{ex:VCVC->VVC}.

\begin{exe}
	\exa{\xy
		<0em,2.5cm>*\as{`animal'}="gloss",
		<2.5em,2cm>*\as{m}="u1",<3.5em,2cm>*\as{u}="u2",<4.5em,2cm>*\as{ʔ}="u3",<5.5em,2cm>*\as{i}="u4",<6.5em,2cm>*\as{t}="u5",<0em,2cm>*\as{U\=/form:}="u",
		<2.5em,1.5cm>*\as{C}="uC1",<3.5em,1.5cm>*\as{V}="uC2",<4.5em,1.5cm>*\as{C}="uC3",<5.5em,1.5cm>*\as{V}="uC4",<6.5em,1.5cm>*\as{C}="uC5",
		<2.5em,0.5cm>*\as{C}="mC1",<3.5em,0.5cm>*\as{V}="mC2",<4.5em,0.5cm>*\as{V}="mC4",<5.5em,0.5cm>*\as{C}="mC3",
		<2.5em,0cm>*\as{m}="m1",<3.5em,0cm>*\as{u}="m2",<4.5em,0cm>*\as{i}="m4",<5.5em,0cm>*\as{ʔ}="m3",<0em,0cm>*\as{M\=/form:}="m",
		{\ar@{->} "uC1"+D;"mC1"+U};{\ar@{->} "uC2"+D;"mC2"+U};{\ar@{->} "uC3"+D;"mC3"+U};{\ar@{->} "uC4"+D;"mC4"+U};
	\endxy}\label{as:muqit/muiq}
\end{exe}
%\newpage
\begin{exe}
	\ex{{\ldots}V\sub{1}C\sub{1}V\sub{2}C\sub{2}{\#} {\ra} {\ldots}V\sub{1}V\sub{2}C\sub{1}{\#}}\label{ex:VCVC->VVC}
	\sn{\stl{0.45em}\gw\begin{tabular}{rcll|rcll}
		 U\=/form					&			&\mc{2}{l}{M\=/form}					&U\=/form						&			&\mc{2}{l}{M\=/form}\\
		\ve{mu\tbr{ʔit}}&{\ra}&\ve{mu\tbr{iʔ}}&`animal'		&\ve{po\tbr{ʔon}}	&{\ra}&\ve{po\tbr{oʔ}}&`orchard'	\\
		\ve{te\tbr{nuk}}&{\ra}&\ve{te\tbr{un}}&`umbrella'	&\ve{o\tbr{ʔof}}	&{\ra}&\ve{o\tbr{oʔ}}&`pen, corral'	\\
		\ve{te\tbr{noʔ}}&{\ra}&\ve{te\tbr{on}}&`egg'			&\ve{ma\tbr{nus}}	&{\ra}&\ve{ma\tbr{un}}&`betel vine'	\\
		\ve{uk\tbr{um}}	&{\ra}&\ve{u\tbr{uk}}	&`cuscus'		&\ve{a\tbr{nah}}	&{\ra}&\ve{a\tbr{an}}	&`child'	\\
		\end{tabular}}
\end{exe}

Word-final consonant clusters are not permitted in Amarasi.
The consonant deletion in the M\=/form of VCVC{\#} final words
can be accounted for by language specific phonotactic constraints.
Metathesis occurs, resulting in a disallowed final consonant cluster
which is resolved by deletion of the final consonant.

\begin{table}[p]
	\centering\caption[Ro{\Q}is Final Consonant Clusters]
	{Ro{\Q}is Final Consonant Clusters\su{†}}\label{tab:RoqFinConClu}
		\begin{threeparttable}
			\begin{tabular}{lllll}\lsptoprule
				Kotos	&	Ro{\Q}is	&	Ro{\Q}is CC{\#}	&	Kotos/Ro{\Q}is	&		\\
				U\=/form	&	U\=/form	&	M\=/form	&	M\=/form	&	gloss	\\ \midrule
					\ve{batan}	&	\ve{batan}	&	\ve{baa\tbr{tn}}	&	\ve{baat}	&	`generation'	\\
					\ve{funan}	&	\ve{funun}	&	\ve{fuu\tbr{nn}}	&	\ve{fuun}	&	`month'	\\
					\ve{knapan}	&	\ve{knapan}	&	\ve{knaa\tbr{pn}}	&	\ve{knaap}	&	`butterfly'	\\
					\ve{manas}	&	\ve{manas}	&	\ve{maa\tbr{ns}}	&	\ve{maan}	&	`sun'	\\
					\ve{metan}	&	\ve{meten}	&	\ve{mee\tbr{tn}}	&	\ve{meet}	&	`black'	\\
					\ve{prenat}	&	\ve{prenet}	&	\ve{pree\tbr{nt}}	&	\ve{preen}	&	`government'	\\
					\ve{ranan}	&	\ve{ranan}	&	\ve{raa\tbr{nn}}	&	\ve{raan}	&	`road'	\\
					\ve{surat}	&	\ve{surut}	&	\ve{suu\tbr{rt}}	&	\ve{suur}	&	`paper'	\\
					\ve{uran}	&	\ve{urun}	&	\ve{uu\tbr{rn}}	&	\ve{uur}	&	`rain'	\\
					\ve{amfoʔan}	&	\ve{amfoʔan}	&	\ve{amfoa\tbr{ʔn}}	&	\ve{}	&	`Amfo{\Q}an'	\\
					\ve{benas}	&	\ve{fenes/fenas}	&	\ve{fee\tbr{ns}}	&	\ve{}	&	`machete'	\\
					\ve{bonak}	&	\ve{bonak}	&	\ve{boo\tbr{nk}}	&	\mc{2}{r}{`fragrant pandanus'} 	\\
					\ve{ekam}	&	\ve{erem/eram}	&	\ve{ee\tbr{rm}}	&	\ve{}	&	`wild pandanus'	\\
					\ve{koor{\gap}kapiten}	&	\ve{koor{\gap}kapitin}	&	\ve{koor{\gap}kapii\tbr{tn}}	&	\ve{}	&	`swiftlet'	\\
					\ve{kopan}	&	\ve{kopon/kopan}	&	\ve{koo\tbr{pn}}	&	\ve{}	&	`Kupang'	\\
					\ve{ksamun}	&	\ve{ksa͡unum}\su{‡}	&	\ve{ksau\tbr{nm}}	&	\ve{}	&	`starling'	\\
					\ve{oras}	&	\ve{oros}	&	\ve{oo\tbr{rs}}	&	\ve{}	&	`time'	\\
					\ve{ruman}	&	\ve{rumun}	&	\ve{ruu\tbr{mn}}	&	\ve{}	&	`empty'	\\
					\ve{ukum}	&	\ve{urum}	&	\ve{uu\tbr{rm}}	&	\ve{}	&	`cuscus'	\\
					\ve{ʔhenes}	&	\ve{henes}	&	\ve{hee\tbr{ns}}	&	\ve{}	&	`winter melon'	\\
					\ve{anin}	&	\ve{}	&	\ve{ai\tbr{nn}}	&	\ve{ain}	&	`wind'	\\
					\ve{menas}	&	\ve{}	&	\ve{mee\tbr{ns}}	&	\ve{meen}	&	`sickness'	\\
					\ve{krisan}	&	\ve{}	&	\ve{krii\tbr{sn}}	&	\mc{2}{r}{`red-cheeked parrot'} 	\\		
					\ve{meisʔokan}	&	\ve{}	&	\ve{meisiʔnoo\tbr{rn}}	&	\ve{}	&	`dark(ness)'	\\
					\ve{nini{\gap}tboran}	&	\ve{}	&	\ve{niin{\gap}tboo\tbr{rn}}	&	\ve{}	&	`dollarbird'	\\
					\ve{ninik}	&	\ve{}	&	\ve{nii\tbr{nk}}	&	\ve{}	&	`wax'	\\
					\ve{onen}	&	\ve{}	&	\ve{oe\tbr{nn}}	&	\ve{}	&	`prayer'	\\
					\ve{paah{\gap}pinan}	&	\ve{}	&	\ve{paah{\gap}pii\tbr{nn}}	&	\ve{}	&	`earth, world'	\\
					\ve{pinis}	&	\ve{}	&	\ve{pii\tbr{ns}}	&	\ve{}	&	`dew'	\\
				\lspbottomrule
			\end{tabular}%}
			\begin{tablenotes}
				\item [†]	Empty cells indicate forms which are
									currently unattested in my data.
				\item [‡]	One of Kotos \it{ksamun} or Ro{\Q}is \it{ksa͡unum}
									`startling' has undergone historical metathesis of the
									penultimate and final consonants.
									Ro{\Q}is \it{ksa͡unum} further has diphthongisation (\srf{sec:RoqAmaDip}).
			\end{tablenotes}
		\end{threeparttable}
\end{table}

In Ro{\Q}is Amarasi some
consonant-final words with certain phonological properties
(see \srf{sec:MforFinConClu}) have two M\=/forms:
an M\=/form derived in the same way as Kotos Amarasi
by metathesis and deletion of the final consonant,
and an M\=/form derived by metathesis but with preservation
of the final consonant cluster.
Examples of Ro{\Q}is nouns which have been attested with
this second CC-final M\=/form are given in \trf{tab:RoqFinConClu}
alongside Kotos U\=/forms and Kotos/Ro{\Q}is basic M\=/forms.

The Ro{\Q}is M\=/forms with final consonant deletion
are used to mark attributive modification
in the same way as their Kotos equivalents.
Ro{\Q}is M\=/forms with a final cluster are used
phrase finally with a discourse function like
Kotos verbs, where U\=/forms mark lack of resolution and
M\=/forms resolution (Chapter \ref{ch:DisMet}).

The data from Ro{\Q}is Amarasi in which certain
word-final consonant clusters are permitted phrase finally
indicates that the deletion of final consonants in the basic M\=/form
is not simply due to a general prohibition
against word-final clusters but rather against
word-final consonant clusters in medial members of the noun phrase.
Additional evidence for this comes from Amfo{\Q}an,
in which certain CVC{\#} final words delete their
final consonant without metathesis when modified.
This Amfo{\Q}an data is discussed in \srf{sec:LosFinCon}.

\subsubsection{Consonant deletion}\label{sec:ConDel}
Nominals which end in VVC{\#} in the U\=/form derive their
M\=/form by deletion of the final consonant.
The surface relationship between the segments of
\ve{kaut} [ˈkə.ʊt̪] {\ra} \ve{kau} [ˈkə.ʊ] `papaya' is shown in \qf{as:kaut/kau},
with more examples in \qf{ex:VVC->VV}.
Assimilation of /a/ does not occur in such M\=/forms.
In \srf{sec:AssOfA} I analyse this as being due to
a final empty C-slot in the M\=/form of these words.

\begin{exe}
	\exa{\xy
		<0em,2.5cm>*\as{`papaya'}="gloss",
		<2.5em,2cm>*\as{k}="u1",<3.5em,2cm>*\as{a}="u2",<4.5em,2cm>*\as{u}="u3",<5.5em,2cm>*\as{t}="u4",<0em,2cm>*\as{U\=/form:}="u",
		<2.5em,1.5cm>*\as{C}="uC1",<3.5em,1.5cm>*\as{V}="uC2",<4.5em,1.5cm>*\as{V}="uC3",<5.5em,1.5cm>*\as{C}="uC4",
		<2.5em,0.5cm>*\as{C}="mC1",<3.5em,0.5cm>*\as{V}="mC2",<4.5em,0.5cm>*\as{V}="mC4",
		<2.5em,0cm>*\as{k}="m1",<3.5em,0cm>*\as{a}="m2",<4.5em,0cm>*\as{u}="m4",<0em,0cm>*\as{M\=/form:}="m",
		{\ar@{->} "uC1"+D;"mC1"+U};{\ar@{->} "uC2"+D;"mC2"+U};{\ar@{->} "uC3"+D;"mC4"+U};
	\endxy}\label{as:kaut/kau} 
	\ex{{\ldots}VVC{\#} {\ra} {\ldots}VV{\#}}\label{ex:VVC->VV}
	\stl{0.4em}\gw\sn{\begin{tabular}{rcll|rcll}
		 U\=/form					&		&\mc{2}{l|}{M\=/form}		&U\=/form						&		&\mc{2}{l}{M\=/form}\\
		\ve{kau\tbr{t}}	&\ra&\ve{kau}		&`papaya'	&\ve{kua\tbr{n}}	&\ra&\ve{kua}		&`village'	\\
		\ve{bruu\tbr{k}}&\ra&\ve{bruu}	&`pants'	&\ve{ʔnae\tbr{f}}	&\ra&\ve{ʔnae}	&`old man'	\\
		\ve{knaa\tbr{ʔ}}&\ra&\ve{knaa}	&`beans'	&\ve{poe\tbr{s}}	&\ra&\ve{poe}		&`prawn/shrimp'	\\
		\ve{heu\tbr{m}}	&\ra&\ve{heu}		&`mango'	&\ve{noa\tbr{h}}	&\ra&\ve{noa}		&`coconut'	\\
		\end{tabular}}
\end{exe}

Unlike the consonant deletion seen for VCVC{\#} words (\srf{sec:MetConDel}),
this consonant deletion cannot be accounted for by surface phonotactic constraints of the language.
By positing medial empty C-slots this consonant deletion
can be analysed as an automatic result of metathesis
and a prohibition against word-final consonant clusters,
including clusters involving empty C-slots.

	\subsection{Vowel deletion}\label{sec:VowDel}
The final complication in the formation of the basic M\=/form
involves words which end in VVCV(C){\#}
in the U\=/form; words with a phonetic diphthong.
Such words derive their M\=/form by deletion of the final vowel as well as any final consonant.
The surface relationship between the segments of the U\=/form and M\=/form of
\ve{nautus} [ˈnəwt̪ʊs] {\ra} \ve{naut} [ˈnə.ʊt̪] `beetle' is given in \qf{as:nautus/naut},
with more examples given in \qf{ex:VVCV->VVC} below.

\begin{exe}
	\exa{\xy
		<0em,2.5cm>*\as{`beetle'}="gloss",
		<2.5em,2cm>*\as{n}="u1",<3.5em,2cm>*\as{a}="u2",<4.5em,2cm>*\as{u}="u3",<5.5em,2cm>*\as{t}="u4",<6.5em,2cm>*\as{u}="u5",<7.5em,2cm>*\as{s}="u6",<0em,2cm>*\as{U\=/form:}="u",
		<2.5em,1.5cm>*\as{C}="uC1",<3.5em,1.5cm>*\as{V}="uC2",<4.5em,1.5cm>*\as{V}="uC3",<5.5em,1.5cm>*\as{C}="uC4",<6.5em,1.5cm>*\as{V}="uC5",<7.5em,1.5cm>*\as{C}="uC6",
		<2.5em,0.5cm>*\as{C}="mC1",<3.5em,0.5cm>*\as{V}="mC2",<4.5em,0.5cm>*\as{V}="mC3",<5.5em,0.5cm>*\as{C}="mC4",
		<2.5em,0cm>*\as{n}="m1",<3.5em,0cm>*\as{a}="m2",<4.5em,0cm>*\as{u}="m4",<5.5em,0cm>*\as{t}="m5",<0em,0cm>*\as{M\=/form:}="m",
		{\ar@{->} "uC1"+D;"mC1"+U};{\ar@{->} "uC2"+D;"mC2"+U};{\ar@{->} "uC3"+D;"mC3"+U};{\ar@{->} "uC4"+D;"mC4"+U};
	\endxy}\label{as:nautus/naut} 
	\ex{{\ldots}V\sub{1}V\sub{2}C\sub{1}V\sub{3}(C\sub{2}){\#} {\ra} {\ldots}V\sub{1}V\sub{2}C\sub{1}{\#}}\label{ex:VVCV->VVC}
	\gw\sn{\begin{tabular}{llll|llll}
		 U\=/form						&		&\mc{2}{l|}{M\=/form}		& U\=/form						&		&\mc{2}{l}{M\=/form}		\\
		\ve{aun\tbr{u}}		&\ra&\ve{aun}		&`spear'	&\ve{naut\tbr{us}}	&\ra&\ve{naut}	&`beetle'\\
		\ve{n-ait\tbr{i}}	&\ra&\ve{n-ait}	&`pick up'&\ve{kaun\tbr{aʔ}}	&\ra&\ve{kaun}	&`snake'\\
		\ve{n-aen\tbr{a}}	&\ra&\ve{n-aen}	&`run, flee'		&\ve{aik\tbr{aʔ}}		&\ra&\ve{aik}		&`thorn'\\
		\ve{uab\tbr{aʔ}}	&\ra&\ve{uab}		&`speech'	&\ve{ain\tbr{a}-f}	&\ra&\ve{ain}		&`mother'\\
		\end{tabular}}
\end{exe}

Sequences of three surface vowels do not occur in Amarasi.
Thus, this vowel deletion can be analysed as resulting from phonological constraints of the language.
If consonant-vowel metathesis were to occur, it would result in a disallowed sequence
of three vowels which is resolved by vowel deletion.

\subsection{Irregular M\=/forms and U\=/forms}\label{sec:IrrMfor}
There are a handful of morphemes in my database which have irregular M\=/forms.
Firstly, the plural enclitic \ve{=enu}
has the M\=/form \ve{=uun} with irregular assimilation of the initial vowel.
This plural enclitic is uncommon in my data.
Instead, the form \ve{=eni/=ein} is most frequent (\srf{sec:PluEnc}).

Secondly, \ve{ai{\j}oʔo} `casuarina tree'
and \ve{naisoʔo} `garlic, shallot' have M\=/forms
derived by deleting the final /ʔo/ sequence.
Examples include \ve{ai{\j}o\tbr{ʔo}} + \ve{teas} `heartwood'
{\ra} \ve{ai{\j}o teas} `heartwood of a casuarina tree'
and \ve{naiso\tbr{ʔo}} + \ve{meʔe} `red' {\ra} \ve{naiso meʔe} `shallot'.

There are also a number of functors with final
/a/ in the U\=/form which form their M\=/form by deletion of this vowel.
These functors are given in \qf{ex:LexDel} below.
These functors usually occur in the M\=/form
and only take the U\=/form before consonant clusters
or when the plural enclitic \ve{=n} (\srf{sec:PluEnc}) is attached.

\begin{exe}
	\ex{{\ldots}V\sub{1}Ca{\#} /{\gap}CC {\ra} {\ldots}V\sub{1}C{\#}}\label{ex:CvaCC}
	\gw\sn{\begin{tabular}{lcll}
		U\=/form 		&		&\mc{2}{l}{M\=/form}					\\
		\ve{eta}	&\ra&\ve{et}	&`{\et}; at, in, on'\\
		\ve{ofa}	&\ra&\ve{of}	&`later, surely'	\\
		\ve{fina}	&\ra&\ve{fin}	&`because, so'		\\
		\ve{tara}	&\ra&\ve{tar}	&`until'					\\
		\ve{n-aka}	&\ra&\ve{n-ak}	&`say'						\\
%		\ve{n-oka}	&\ra&\ve{n-ok}	&`with accompany'	\\
	%	\ve{a}		&\ra&\ve{}		&`'	&\tsc{}		\\
		\end{tabular}}
\end{exe}

Finally, while most pronouns are VV{\#} final and thus
do not have distinct U\=/forms and M\=/forms,
those pronouns which do have both forms have multiple U\=/forms;
one with final /a/ and one with final /i/.
These pronouns are given in \qf{ex:LexDel} below.

\begin{exe}
	\ex{{\ldots}V\sub{1}CV\sub{2}{\#} {\ra} {\ldots}V\sub{1}C{\#}}\label{ex:LexDel}
	\gw\sn{\begin{tabular}{lclclll}
		U\=/form\sub{1} 	& 	&U\=/form\sub{2}		&		&M\=/form			&						&										\\
		\ve{\hp{=}ina}	&\ra&\ve{\hp{=}ini}		&\ra&\ve{iin}		&`s/he, it'	&\tsc{3sg.nom}			\\
		\ve{\hp{=}sina}	&\ra&\ve{\hp{=}sini}	&\ra&\ve{siin}	&`they'			&\tsc{3pl}					\\
		\ve{=sina}			&\ra&\ve{=sini}				&\ra&\ve{=siin}	&`them'			&\tsc{3pl}					\\
		\ve{\hp{=}hita}	&\ra&\ve{\hp{=}hiti}	&\ra&\ve{hiit}	&`we'				&\tsc{1pl.incl.nom}	\\
		\ve{=kita}			&\ra&\ve{=kiti}				&\ra&\ve{=kiit}	&`us'				&\tsc{1pl.incl.acc}	\\
		\end{tabular}}
\end{exe}

U\=/forms ending in /a/ are historically conservative.
Thus, PMP *sida > \ve{sina} > \ve{sini} `they',
and PMP *kita > \ve{hita} > \ve{hiti} `we',
as well as *kita > \ve{=kita} > \ve{=kiti} `us'.
The U\=/forms ending in /a/ tend only to
be used before consonant clusters, while the other
U\=/forms tend to be used with a morphological function,
though there are counterexamples in both cases.

While at an abstract level of phonological
organisation the M\=/form of these pronouns
must be analysed as containing a vowel sequence,
these pronouns are usually unstressed and, as a result, the vowel
sequence is usually realised as a single short vowel.
The vowel sequence in the M\=/form of these pronouns
is usually only realised as phonetically long in certain environments,
such as before vowel-initial enclitics (\srf{sec:PosDet}).

Comparative evidence that these pronouns
have an underlying sequence of two vowels comes from Kusa-Manea
in which M\=/forms of these pronouns have a /Va/ sequence:
\ve{ian} `{\iin}, s/he, it', \ve{sian} `{\siin}, they',
and \ve{hiat} `\tsc{1pl.incl}, we'.

%\footnote{
%		That the M\=/form of these functors has only a single vowel
%		was confirmed by an instrumental phonetic study.
%		Although excluded from the data used to measure vowel length in \srf{sec:QuaLenVowSeq},
%		there are 207 instances of these words
%		in the four texts used for this phonetic study.
%		They have an average length of 0.071 seconds,
%		below the average of 0.098 seconds for a single vowel,
%		and well below the average of 0.129 seconds
%		for a sequence of identical vowels.}

\subsection{No change}\label{sec:NoCha}
Words which end in a vowel sequence do not have distinct U\=/forms and M\=/forms.
Some examples are given in \qf{ex1:VV->VV} below.

\begin{exe}
	\ex{VV{\#} {\ra} VV{\#}}\label{ex1:VV->VV}
	\gw\sn{\begin{tabular}{lcll}
		 U\=/form			&		&\mc{2}{l}{M\=/form}				\\
		\ve{hau}		&\ra&\ve{hau}		&`tree, wood'	\\
		\ve{pui}		&\ra&\ve{pui}		&`quail'			\\
		\ve{biʤae}	&\ra&\ve{biʤae}	&`cow'				\\
		\ve{meo}		&\ra&\ve{meo}		&`cat'				\\
		\ve{ai}			&\ra&\ve{ai}		&`fire'				\\
		\ve{kee}		&\ra&\ve{kee}		&`turtle, tortoise'	\\
		\ve{pansoe}	&\ra&\ve{pansoe}&`earthworm'	\\
		\ve{ʔsao}		&\ra&\ve{ʔsao}	&`viper'			\\
		\end{tabular}}
\end{exe}
\section{Unified analysis}\label{sec:UniAna}
A number of surface phonological operations
occur to derive the M\=/form in Amarasi.
Such phonological processes include metathesis,
consonant deletion, and assimilation of /a/.
Furthermore, metathesis itself can trigger further
processes of consonant deletion, vowel deletion,
and vowel height assimilation.

\begin{table}[h]
	\caption{Amarasi surface basic M-forms}\label{tab:AmaSurMfor}
	\centering
		\stl{0.38em}\begin{tabular}{llcl|lclll}\lsptoprule
				&		U\=/form																	&		&M\=/form											&\mc{2}{l}{U\=/form}&\mc{2}{l}{M\=/form}		&processes\\ \midrule
			1.&	V\sub{1}C\sub{1}V\sub{2}									&\ra&V\sub{1}V\sub{2}C\sub{1}		&\ve{fatu}	&\ra&\ve{faut}	&`pig'		&metathesis\\
			2.& V\sub{1}C\sub{1}V\sub{2}C\sub{2}					&\ra&V\sub{1}V\sub{2}C\sub{1}		&\ve{muʔit}	&\ra&\ve{muiʔ}	&`animal'	&{\&} C deletion\\
			3.& V\sub{1}\sub{\tsc{hi}}C\sub{1}V\sub{2}\sub{\tsc{mid}}
																										&\ra&V\sub{1}V\sub{2}\sub{\tsc{hi}}C\sub{1}	
																																										&\ve{ume}		&\ra&\ve{uim}		&`house'	&{\&} height ass.\\
			4.& V\sub{1}{\sub{α}}C\sub{1}a(C\sub{2})					&\ra&V\sub{1}{\sub{α}}V{\sub{α}}C\sub{1}&\ve{nima}	&\ra&\ve{niim}	&`five'		&{\&} /a/ ass.\\
			5.& V\sub{1}V\sub{2}C\sub{1}V\sub{3}					&\ra&V\sub{1}V\sub{2}C\sub{1}		&\ve{aunu}	&\ra&\ve{aun}		&`spear'	&V deletion\\
			6.& V\sub{1}V\sub{2}C\sub{1}V\sub{3}C\sub{2}	&\ra&V\sub{1}V\sub{2}C\sub{1}		&\ve{nautus}&\ra&\ve{naut}	&`beetle'	&V deletion\\
			7.& V\sub{1}V\sub{2}C\sub{1}									&\ra&V\sub{1}V\sub{2}						&\ve{kaut}	&\ra&\ve{kau}		&`papaya'	&C deletion\\
			8.& V\sub{1}V\sub{2}													&\ra&V\sub{1}V\sub{2}						&\ve{ai}		&\ra&\ve{ai}		&`fire'		&\\
		\lspbottomrule
		\end{tabular}
\end{table}

\largerpage
Which operations apply to a word 
is determined by the phonotactic structure of that word,
as well as the quality of the vowels it contains.
The different structures of the M\=/form
are summarised in Table \ref{tab:AmaSurMfor}.
With the exception of M\=/forms with a double vowel (\srf{sec:DouVow}),
all M\=/forms are phonetically disyllabic.

The M\=/form must be derived from the U\=/form
as there is a large amount of ambiguity among M\=/forms.
For instance, given an M\=/form with the shape VVC{\#},
we cannot predict whether the U\=/form will have a shape
corresponding to any of shapes 1--6 in Table \ref{tab:AmaSurMfor}.
A concrete example is the form \ve{n-neen},
which is the M\=/form of both
\ve{n-nene} `pushes' and \ve{n-nena} `hears'.

In this section I propose an analysis of all the different M\=/forms.
I use an autosegmental model of phonology \citep{go76}
and a rule-based model of process morphology \citep{ma74,an92}.
Adopting these models allows me to formulate a single, unified
analysis of the diverse processes which occur in the formation of Amarasi M\=/forms.
In \srf{sec:AltApp} I discuss alternate analyses
which I propose cannot account for all the Amarasi data.

My analysis consists of a single process of metathesis at the CV tier
and an associated morphemically conditioned process (/a/ assimilation).
These processes, combined with an obligatory CVCVC foot structure
and the general phonotactic constraints of Amarasi,
generate all the different M\=/forms.

In my autosegmental diagrams in the following sections
empty C-slots are occasionally `filled' with {\0}
in order to make it explicit that they behave identically to filled C-slots.
This is a notational convenience.
Similarly, the x-tier (or timing tier) is used as a notational device to
illustrate clearly the effect of metathesis.
Use of the x-tier should not be taken as a claim about its theoretical status.

\subsection{Obligatory CVCVC Foot}\label{sec:ThePhoRul}
I posit that the Amarasi foot obligatorily has the structure CVCVC
and that C-slots may be empty.
This stipulation is given in \qf{ex2:Ft->CVCVC} below,
and has already been discussed in \srf{sec:TheFoo}.
Extensive evidence for the existence
of empty C-slots in Amarasi has been given in \srf{sec:EmpCSlo}.
This foot structure applies to all words

\begin{exe}
	\ex{Ft {\ra} CVCVC}\label{ex2:Ft->CVCVC}
\end{exe}

The structures of the words \ve{fafi} `pig', \ve{muʔit} `animal', \ve{kaut} `papaya',
\ve{ai} `fire', \ve{nautus} `beetle' and \ve{aunu} `spear' under this analysis
are given in \qf{as:fafi}--\qf{as:nautus} below.
The initial C-slots of the words \ve{ai} `fire' and \ve{aunu} `spear'
have been filled with an automatic glottal stop,
as is the case for all vowel-initial words (\srf{sec:GloStoIns}).

\begin{multicols}{3}
	\begin{exe}
		\exa{\label{as:fafi}\xy
			<0pt,1cm>*\as{C}="C1",<1em,1cm>*\as{V}="V1",<2em,1cm>*\as{C}="C2",<3em,1cm>*\as{V}="V2",<4em,1cm>*\as{C}="C3",
			<0pt,0pt>*\as{f}="c1",<1em,0pt>*\as{a}="v1",<2em,0pt>*\as{f}="c2",<3em,0pt>*\as{i}="v2",
			"c1"+U;"C1"+D**\dir{-};"c2"+U;"C2"+D**\dir{-};"v1"+U;"V1"+D**\dir{-};"v2"+U;"V2"+D**\dir{-};
		\endxy}
		\exa{\xy
			<0pt,1cm>*\as{C}="C1",<1em,1cm>*\as{V}="V1",<2em,1cm>*\as{C}="C2",<3em,1cm>*\as{V}="V2",<4em,1cm>*\as{C}="C3",
			<0pt,0pt>*\as{m}="c1",<1em,0pt>*\as{u}="v1",<2em,0pt>*\as{ʔ}="c2",<3em,0pt>*\as{i}="v2",<4em,0pt>*\as{t}="c3",
			"c1"+U;"C1"+D**\dir{-};"c2"+U;"C2"+D**\dir{-};"c3"+U;"C3"+D**\dir{-};"v1"+U;"V1"+D**\dir{-};"v2"+U;"V2"+D**\dir{-};
		\endxy}
		\exa{\xy
			<0pt,1cm>*\as{C}="C1",<1em,1cm>*\as{V}="V1",<2em,1cm>*\as{C}="C2",<3em,1cm>*\as{V}="V2",<4em,1cm>*\as{C}="C3",
			<0pt,0pt>*\as{k}="c1",<1em,0pt>*\as{a}="v1",<3em,0pt>*\as{u}="v2",<4em,0pt>*\as{t}="c3",
			"c1"+U;"C1"+D**\dir{-};"c3"+U;"C3"+D**\dir{-};"v1"+U;"V1"+D**\dir{-};"v2"+U;"V2"+D**\dir{-};
		\endxy}
	\end{exe}
\end{multicols}
\begin{multicols}{3}
	\begin{exe}
		\exa{\xy
			<0pt,1cm>*\as{C}="C1",<1em,1cm>*\as{V}="V1",<2em,1cm>*\as{C}="C2",<3em,1cm>*\as{V}="V2",<4em,1cm>*\as{C}="C3",
			<0pt,0pt>*\as{[ʔ]}="c1",<1em,0pt>*\as{a}="v1",<3em,0pt>*\as{i}="v2",
			"c1"+U;"C1"+D**\dir{-};"v1"+U;"V1"+D**\dir{-};"v2"+U;"V2"+D**\dir{-};
		\endxy}
		\exa{\xy
			<0pt,1cm>*\as{C}="C1",<1em,1cm>*\as{V}="V1",<2em,1cm>*\as{C}="C2",<3em,1cm>*\as{V}="V2",<4em,1cm>*\as{C}="C3",
			<0pt,0pt>*\as{[ʔ]}="c1",<0.75em,0pt>*\as{a}="v1",<1.25em,0pt>*\as{u}="v1.5",<2em,0pt>*\as{n}="c2",<3em,0pt>*\as{u}="v2",
			"c1"+U;"C1"+D**\dir{-};"c2"+U;"C2"+D**\dir{-};"v1"+U;"V1"+D**\dir{-};"v1.5"+U;"V1"+D**\dir{-};"v2"+U;"V2"+D**\dir{-};
		\endxy}
		\exa{\label{as:nautus}\xy
			<0pt,1cm>*\as{C}="C1",<1em,1cm>*\as{V}="V1",<2em,1cm>*\as{C}="C2",<3em,1cm>*\as{V}="V2",<4em,1cm>*\as{C}="C3",
			<0pt,0pt>*\as{n}="c1",<0.75em,0pt>*\as{a}="v1",<1.25em,0pt>*\as{u}="v1.5",<2em,0pt>*\as{t}="c2",<3em,0pt>*\as{u}="v2",<4em,0pt>*\as{s}="c3",
			"c1"+U;"C1"+D**\dir{-};"c2"+U;"C2"+D**\dir{-};"c3"+U;"C3"+D**\dir{-};"v1"+U;"V1"+D**\dir{-};"v1.5"+U;"V1"+D**\dir{-};"v2"+U;"V2"+D**\dir{-};
		\endxy}
	\end{exe}
\end{multicols}

Metathesis at the CV tier yields a derived
foot structure with the form CVVC.
%The structures of the M\=/form of the words
%in \qf{as:fafi}--\qf{as:nautus} above
%are given in \qf{as:faif}--\qf{as:naut} below.
This structure is shown in \qf{as:faif}--\qf{as:naut} below.

\begin{multicols}{3}
	\begin{exe}
		\exa{\label{as:faif}\xy
			<0pt,1cm>*\as{C}="C1",<1em,1cm>*\as{V}="V1",<2em,1cm>*\as{V}="C2",<3em,1cm>*\as{C}="V2",
			<0pt,0pt>*\as{f}="c1",<1em,0pt>*\as{a}="v1",<2em,0pt>*\as{i}="c2",<3em,0pt>*\as{f}="v2",
			"c1"+U;"C1"+D**\dir{-};"c2"+U;"C2"+D**\dir{-};"v1"+U;"V1"+D**\dir{-};"v2"+U;"V2"+D**\dir{-};
		\endxy}
		\exa{\xy
			<0pt,1cm>*\as{C}="C1",<1em,1cm>*\as{V}="V1",<2em,1cm>*\as{V}="C2",<3em,1cm>*\as{C}="V2",
			<0pt,0pt>*\as{m}="c1",<1em,0pt>*\as{u}="v1",<2em,0pt>*\as{i}="c2",<3em,0pt>*\as{ʔ}="v2",
			"c1"+U;"C1"+D**\dir{-};"c2"+U;"C2"+D**\dir{-};"v1"+U;"V1"+D**\dir{-};"v2"+U;"V2"+D**\dir{-};
		\endxy}
		\exa{\xy
			<0pt,1cm>*\as{C}="C1",<1em,1cm>*\as{V}="V1",<2em,1cm>*\as{V}="C2",<3em,1cm>*\as{C}="V2",
			<0pt,0pt>*\as{k}="c1",<1em,0pt>*\as{a}="v1",<2em,0pt>*\as{u}="c2",<3em,0pt>*\as{}="v2",
			"c1"+U;"C1"+D**\dir{-};"c2"+U;"C2"+D**\dir{-};"v1"+U;"V1"+D**\dir{-};
		\endxy}
	\end{exe}
\end{multicols}
\begin{multicols}{3}
	\begin{exe}
		\exa{\xy
			<0pt,1cm>*\as{C}="C1",<1em,1cm>*\as{V}="V1",<2em,1cm>*\as{V}="C2",<3em,1cm>*\as{C}="V2",
			<0pt,0pt>*\as{[ʔ]}="c1",<1em,0pt>*\as{a}="v1",<2em,0pt>*\as{i}="c2",<3em,0pt>*\as{ }="v2",
			"c1"+U;"C1"+D**\dir{-};"c2"+U;"C2"+D**\dir{-};"v1"+U;"V1"+D**\dir{-};
		\endxy}
		\exa{\xy
			<0pt,1cm>*\as{C}="C1",<1em,1cm>*\as{V}="V1",<2em,1cm>*\as{V}="C2",<3em,1cm>*\as{C}="V2",
			<0pt,0pt>*\as{[ʔ]}="c1",<1em,0pt>*\as{a}="v1",<2em,0pt>*\as{u}="c2",<3em,0pt>*\as{n}="v2",
			"c1"+U;"C1"+D**\dir{-};"c2"+U;"C2"+D**\dir{-};"v1"+U;"V1"+D**\dir{-};"v2"+U;"V2"+D**\dir{-};
		\endxy}
		\exa{\label{as:naut}\xy
			<0pt,1cm>*\as{C}="C1",<1em,1cm>*\as{V}="V1",<2em,1cm>*\as{V}="C2",<3em,1cm>*\as{C}="V2",
			<0pt,0pt>*\as{n}="c1",<1em,0pt>*\as{a}="v1",<2em,0pt>*\as{u}="c2",<3em,0pt>*\as{t}="v2",
			"c1"+U;"C1"+D**\dir{-};"c2"+U;"C2"+D**\dir{-};"v1"+U;"V1"+D**\dir{-};"v2"+U;"V2"+D**\dir{-};
		\endxy}
	\end{exe}
\end{multicols}

\subsection{The morphological rule: metathesis}\label{sec:TheMorRul}
The process required to generate M\=/forms is metathesis,
given in \qf{ex:CV->VC/'V-} below,
which states that a C-slot and a V-slot metathesise after a stressed V-slot.
This rule is a morphological process, in the style of \cite{an92}.

In \qf{ex:CV->VC/'V-} I have included the phonological environment
in which metathesis takes place; after a stressed V-slot.
This is \emph{not} the environment which triggers metathesis
but rather the environment by which metathesis is constrained.

\begin{exe}
	\ex{CV {\ra} VC /\'V{\gap}}\label{ex:CV->VC/'V-}
\end{exe}

The operation of metathesis for the words \ve{muʔit} `animal',
\ve{kaut} `papaya', and \ve{fafi} `pig' %\ve{ai} `fire',
is given in \qf{as:muiq} below.
\qf{as:muiq1} shows the underlying U\=/form of each of these words.
In \qf{as:muiq2} metathesis of the penultimate C-slot and final V-slot takes place.
This results in a disallowed word-final cluster of two C-slots in \qf{as:muiq3}.
To resolve this, the final C-slot is deleted in (\ref{as:muiq}d),
producing the M\=/forms in (\ref{as:muiq}e).
%
%\begin{multicols}{3}
%\begin{exe}
	%\ex{\begin{xlist}
		%\exa{\xy
			%<0pt,3.5cm>*\as{\x}="x1",<1em,3.5cm>*\as{\x}="x2",<2em,3.5cm>*\as{\x}="x3",<3em,3.5cm>*\as{\x}="x4",<4em,3.5cm>*\as{\x}="x5",
			%<0pt,2.5cm>*\as{C}="c1",<1em,2.5cm>*\as{V}="v1",<2em,2.5cm>*\as{C}="c2",<3em,2.5cm>*\as{V}="v2",<4em,2.5cm>*\as{C}="c3",<4.75em,2.5cm>*\as{\#}="ed",	
			%<0pt,1.5cm>*\as{m}="pc1",<1em,1.5cm>*\as{u}="pv1",<2em,1.5cm>*\as{ʔ}="pc2",<3em,1.5cm>*\as{i}="pv2",<4em,1.5cm>*\as{t}="pc3",
			%<0pt,1cm>*\as{k}="pc1.2",<1em,1cm>*\as{a}="pv1.2",<2em,1cm>*\as{ }="pc2.2",<3em,1cm>*\as{u}="pv2.2",<4em,1cm>*\as{t}="pc3.2",
			%<0pt,0.5cm>*\as{f}="pc1.1",<1em,0.5cm>*\as{a}="pv1.1",<2em,0.5cm>*\as{f}="pc2.1",<3em,0.5cm>*\as{i}="pv2.1",<4em,0.5cm>*\as{ }="pc3.1",
			%<0pt,0cm>*\as{[ʔ]}="pc1.3",<1em,0cm>*\as{a}="pv1.3",<2em,0cm>*\as{ }="pc2.3",<3em,0cm>*\as{i}="pv2.3",<4em,0cm>*\as{ }="pc3.3",
			%"c1"+U;"x1"+D**\dir{-};"c2"+U;"x3"+D**\dir{-};"v1"+U;"x2"+D**\dir{-};"v2"+U;"x4"+D**\dir{-};"c3"+U;"x5"+D**\dir{-};
			%"pc1"+U;"c1"+D**\dir{-};"pc2"+U;"c2"+D**\dir{-};"pv1"+U;"v1"+D**\dir{-};"pv2"+U;"v2"+D**\dir{-};"pc3"+U;"c3"+D**\dir{-};
		%\endxy}\label{as:muiq1}
		%\exa{\xy
			%<0pt,3.5cm>*\as{\x}="x1",<1em,3.5cm>*\as{\x}="x2",<2em,3.5cm>*\as{\x}="x3",<3em,3.5cm>*\as{\x}="x4",<4em,3.5cm>*\as{\x}="x5",
			%<0pt,2.5cm>*\as{C}="c1",<1em,2.5cm>*\as{V}="v1",<2em,2.5cm>*\as{C}="c2",<3em,2.5cm>*\as{V}="v2",<4em,2.5cm>*\as{C}="c3",<4.75em,2.5cm>*\as{\#}="ed",	
			%<0pt,1.5cm>*\as{m}="pc1",<1em,1.5cm>*\as{u}="pv1",<2em,1.5cm>*\as{ʔ}="pc2",<3em,1.5cm>*\as{i}="pv2",<4em,1.5cm>*\as{t}="pc3",
			%<0pt,1cm>*\as{k}="pc1.2",<1em,1cm>*\as{a}="pv1.2",<2em,1cm>*\as{ }="pc2.2",<3em,1cm>*\as{u}="pv2.2",<4em,1cm>*\as{t}="pc3.2",
			%<0pt,0.5cm>*\as{f}="pc1.1",<1em,0.5cm>*\as{a}="pv1.1",<2em,0.5cm>*\as{f}="pc2.1",<3em,0.5cm>*\as{i}="pv2.1",<4em,0.5cm>*\as{ }="pc3.1",
			%<0pt,0cm>*\as{[ʔ]}="pc1.3",<1em,0cm>*\as{a}="pv1.3",<2em,0cm>*\as{ }="pc2.3",<3em,0cm>*\as{i}="pv2.3",<4em,0cm>*\as{ }="pc3.3",
			%"c1"+U;"x1"+D**\dir{-};"c2"+U;"x4"+D**\dir{.};"v1"+U;"x2"+D**\dir{-};"v2"+U;"x3"+D**\dir{.};"c3"+U;"x5"+D**\dir{-};
			%"pc1"+U;"c1"+D**\dir{-};"pc2"+U;"c2"+D**\dir{-};"pv1"+U;"v1"+D**\dir{-};"pv2"+U;"v2"+D**\dir{-};"pc3"+U;"c3"+D**\dir{-};
			%<2.5em,3cm>*\as{\tikz[red,thick,dashed,baseline=0.9ex]\draw (0,0) rectangle (0.8cm,1.5cm);}="box",
		%\endxy}\label{as:muiq2}
		%\exa{\xy
			%<0pt,3.5cm>*\as{\x}="x1",<1em,3.5cm>*\as{\x}="x2",<3em,3.5cm>*\as{\x}="x3",<2em,3.5cm>*\as{\x}="x4",<4em,3.5cm>*\as{\x}="x5",
			%<0pt,2.5cm>*\as{C}="c1",<1em,2.5cm>*\as{V}="v1",<3em,2.5cm>*\as{C}="c2",<2em,2.5cm>*\as{V}="v2",<4em,2.5cm>*\as{C}="c3",<4.75em,2.5cm>*\as{\#}="ed",	
			%<0pt,1.5cm>*\as{m}="pc1",<1em,1.5cm>*\as{u}="pv1",<3em,1.5cm>*\as{ʔ}="pc2",<2em,1.5cm>*\as{i}="pv2",<4em,1.5cm>*\as{t}="pc3",
			%<0pt,1cm>*\as{k}="pc1.2",<1em,1cm>*\as{a}="pv1.2",<3em,1cm>*\as{ }="pc2.2",<2em,1cm>*\as{u}="pv2.2",<4em,1cm>*\as{t}="pc3.2",
			%<0pt,0.5cm>*\as{f}="pc1.1",<1em,0.5cm>*\as{a}="pv1.1",<3em,0.5cm>*\as{f}="pc2.1",<2em,0.5cm>*\as{i}="pv2.1",<4em,0.5cm>*\as{ }="pc3.1",
			%<0pt,0cm>*\as{[ʔ]}="pc1.3",<1em,0cm>*\as{a}="pv1.3",<3em,0cm>*\as{ }="pc2.3",<2em,0cm>*\as{i}="pv2.3",<4em,0cm>*\as{ }="pc3.3",
			%"c1"+U;"x1"+D**\dir{-};"c2"+U;"x3"+D**\dir{-};"v1"+U;"x2"+D**\dir{-};"v2"+U;"x4"+D**\dir{-};"c3"+U;"x5"+D**\dir{-};
			%"pc1"+U;"c1"+D**\dir{-};"pc2"+U;"c2"+D**\dir{-};"pv1"+U;"v1"+D**\dir{-};"pv2"+U;"v2"+D**\dir{-};"pc3"+U;"c3"+D**\dir{-};
			%<3.5em,2.55cm>*\as{\tikz[red,thick,dashed,baseline=0.9ex]\draw (0,0) rectangle (0.75cm,0.4cm);}="box",
		%\endxy}\label{as:muiq3}
	%\end{xlist}}\label{as:muiq}
%\end{exe}
%\end{multicols}
%
%\begin{multicols}{3}
%\begin{exe}
	%\sn{\begin{xlist}
		%\exi{d.}\exia{\xy
			%<0pt,3.5cm>*\as{\x}="x1",<1em,3.5cm>*\as{\x}="x2",<3em,3.5cm>*\as{\x}="x3",<2em,3.5cm>*\as{\x}="x4",<4em,3.5cm>*\as{\x}="x5",
			%<0pt,2.5cm>*\as{C}="c1",<1em,2.5cm>*\as{V}="v1",<3em,2.5cm>*\as{C}="c2",<2em,2.5cm>*\as{V}="v2",<4em,2.5cm>*\as{\xc{C}}="c3",<4.75em,2.5cm>*\as{\#}="ed",	
			%<0pt,1.5cm>*\as{m}="pc1",<1em,1.5cm>*\as{u}="pv1",<3em,1.5cm>*\as{ʔ}="pc2",<2em,1.5cm>*\as{i}="pv2",<4em,1.5cm>*\as{t}="pc3",
			%<0pt,1cm>*\as{k}="pc1.2",<1em,1cm>*\as{a}="pv1.2",<3em,1cm>*\as{ }="pc2.2",<2em,1cm>*\as{u}="pv2.2",<4em,1cm>*\as{t}="pc3.2",
			%<0pt,0.5cm>*\as{f}="pc1.1",<1em,0.5cm>*\as{a}="pv1.1",<3em,0.5cm>*\as{f}="pc2.1",<2em,0.5cm>*\as{i}="pv2.1",<4em,0.5cm>*\as{ }="pc3.1",
			%<0pt,0cm>*\as{[ʔ]}="pc1.3",<1em,0cm>*\as{a}="pv1.3",<3em,0cm>*\as{ }="pc2.3",<2em,0cm>*\as{i}="pv2.3",<4em,0cm>*\as{ }="pc3.3",
			%"c1"+U;"x1"+D**\dir{-};"c2"+U;"x3"+D**\dir{-};"v1"+U;"x2"+D**\dir{-};"v2"+U;"x4"+D**\dir{-};{\ar@{-}|-(.425)*@{|} |-{\hole} |-(.575)*@{|} "c3"+U;"x5"+D};
			%"pc1"+U;"c1"+D**\dir{-};"pc2"+U;"c2"+D**\dir{-};"pv1"+U;"v1"+D**\dir{-};"pv2"+U;"v2"+D**\dir{-};{\ar@{-}|-(.425)*@{|} |-{\hole} |-(.575)*@{|} "pc3"+U;"c3"+D};
			%<4em,1.75cm>*\as{\tikz[red,thick,dashed,baseline=0.9ex]\draw (0,0) rectangle (0.35cm,4cm);}="box",
		%\endxy}
		%\exi{e.}\exia{\xy
			%<0pt,3.5cm>*\as{\x}="x1",<1em,3.5cm>*\as{\x}="x2",<3em,3.5cm>*\as{\x}="x3",<2em,3.5cm>*\as{\x}="x4",
			%<0pt,2.5cm>*\as{C}="c1",<1em,2.5cm>*\as{V}="v1",<3em,2.5cm>*\as{C}="c2",<2em,2.5cm>*\as{V}="v2",<3.75em,2.5cm>*\as{\#}="ed",	
			%<0pt,1.5cm>*\as{m}="pc1",<1em,1.5cm>*\as{u}="pv1",<3em,1.5cm>*\as{ʔ}="pc2",<2em,1.5cm>*\as{i}="pv2",
			%<0pt,1cm>*\as{k}="pc1.2",<1em,1cm>*\as{a}="pv1.2",<3em,1cm>*\as{ }="pc2.2",<2em,1cm>*\as{u}="pv2.2",
			%<0pt,0.5cm>*\as{f}="pc1.1",<1em,0.5cm>*\as{a}="pv1.1",<3em,0.5cm>*\as{f}="pc2.1",<2em,0.5cm>*\as{i}="pv2.1",
			%<0pt,0cm>*\as{[ʔ]}="pc1.3",<1em,0cm>*\as{a}="pv1.3",<3em,0cm>*\as{ }="pc2.3",<2em,0cm>*\as{i}="pv2.3",
			%"c1"+U;"x1"+D**\dir{-};"c2"+U;"x3"+D**\dir{-};"v1"+U;"x2"+D**\dir{-};"v2"+U;"x4"+D**\dir{-};
			%"pc1"+U;"c1"+D**\dir{-};"pc2"+U;"c2"+D**\dir{-};"pv1"+U;"v1"+D**\dir{-};"pv2"+U;"v2"+D**\dir{-};
		%\endxy}
		%\sna{\xy
			%<0pt,3.5cm>*\as{}="x1",<0pt,2.5cm>*\as{}="c1",<0pt,1.5cm>*\as{}="pc1",
			%<0pt,1cm>*\as{}="pc1.2",<0pt,0.5cm>*\as{}="pc1.1",<0pt,0cm>*\as{}="pc1.3",
		%\endxy}
	%\end{xlist}}
%\end{exe}
%\end{multicols}

\begin{multicols}{3}
\begin{exe}
	\ex{\begin{xlist}
		\exa{\xy
			<0pt,3cm>*\as{\x}="x1",<1em,3cm>*\as{\x}="x2",<2em,3cm>*\as{\x}="x3",<3em,3cm>*\as{\x}="x4",<4em,3cm>*\as{\x}="x5",
			<0pt,2cm>*\as{C}="c1",<1em,2cm>*\as{V}="v1",<2em,2cm>*\as{C}="c2",<3em,2cm>*\as{V}="v2",<4em,2cm>*\as{C}="c3",<4.75em,2cm>*\as{\#}="ed",	
			<0pt,1cm>*\as{m}="pc1",<1em,1cm>*\as{u}="pv1",<2em,1cm>*\as{ʔ}="pc2",<3em,1cm>*\as{i}="pv2",<4em,1cm>*\as{t}="pc3",
			<0pt,0.5cm>*\as{k}="pc1.2",<1em,0.5cm>*\as{a}="pv1.2",<2em,0.5cm>*\as{ }="pc2.2",<3em,0.5cm>*\as{u}="pv2.2",<4em,0.5cm>*\as{t}="pc3.2",
			<0pt,0cm>*\as{f}="pc1.1",<1em,0cm>*\as{a}="pv1.1",<2em,0cm>*\as{f}="pc2.1",<3em,0cm>*\as{i}="pv2.1",<4em,0cm>*\as{ }="pc3.1",
			"c1"+U;"x1"+D**\dir{-};"c2"+U;"x3"+D**\dir{-};"v1"+U;"x2"+D**\dir{-};"v2"+U;"x4"+D**\dir{-};"c3"+U;"x5"+D**\dir{-};
			"pc1"+U;"c1"+D**\dir{-};"pc2"+U;"c2"+D**\dir{-};"pv1"+U;"v1"+D**\dir{-};"pv2"+U;"v2"+D**\dir{-};"pc3"+U;"c3"+D**\dir{-};
		\endxy}\label{as:muiq1}
		\exa{\xy
			<0pt,3cm>*\as{\x}="x1",<1em,3cm>*\as{\x}="x2",<2em,3cm>*\as{\x}="x3",<3em,3cm>*\as{\x}="x4",<4em,3cm>*\as{\x}="x5",
			<0pt,2cm>*\as{C}="c1",<1em,2cm>*\as{V}="v1",<2em,2cm>*\as{C}="c2",<3em,2cm>*\as{V}="v2",<4em,2cm>*\as{C}="c3",<4.75em,2cm>*\as{\#}="ed",	
			<0pt,1cm>*\as{m}="pc1",<1em,1cm>*\as{u}="pv1",<2em,1cm>*\as{ʔ}="pc2",<3em,1cm>*\as{i}="pv2",<4em,1cm>*\as{t}="pc3",
			<0pt,0.5cm>*\as{k}="pc1.2",<1em,0.5cm>*\as{a}="pv1.2",<2em,0.5cm>*\as{ }="pc2.2",<3em,0.5cm>*\as{u}="pv2.2",<4em,0.5cm>*\as{t}="pc3.2",
			<0pt,0cm>*\as{f}="pc1.1",<1em,0cm>*\as{a}="pv1.1",<2em,0cm>*\as{f}="pc2.1",<3em,0cm>*\as{i}="pv2.1",<4em,0cm>*\as{ }="pc3.1",
			"c1"+U;"x1"+D**\dir{-};"c2"+U;"x4"+D**\dir{.};"v1"+U;"x2"+D**\dir{-};"v2"+U;"x3"+D**\dir{.};"c3"+U;"x5"+D**\dir{-};
			"pc1"+U;"c1"+D**\dir{-};"pc2"+U;"c2"+D**\dir{-};"pv1"+U;"v1"+D**\dir{-};"pv2"+U;"v2"+D**\dir{-};"pc3"+U;"c3"+D**\dir{-};
			<2.5em,2.5cm>*\as{\tikz[red,thick,dashed,baseline=0.9ex]\draw (0,0) rectangle (0.8cm,1.5cm);}="box",
		\endxy}\label{as:muiq2}
		\exa{\xy
			<0pt,3cm>*\as{\x}="x1",<1em,3cm>*\as{\x}="x2",<3em,3cm>*\as{\x}="x3",<2em,3cm>*\as{\x}="x4",<4em,3cm>*\as{\x}="x5",
			<0pt,2cm>*\as{C}="c1",<1em,2cm>*\as{V}="v1",<3em,2cm>*\as{C}="c2",<2em,2cm>*\as{V}="v2",<4em,2cm>*\as{C}="c3",<4.75em,2cm>*\as{\#}="ed",	
			<0pt,1cm>*\as{m}="pc1",<1em,1cm>*\as{u}="pv1",<3em,1cm>*\as{ʔ}="pc2",<2em,1cm>*\as{i}="pv2",<4em,1cm>*\as{t}="pc3",
			<0pt,0.5cm>*\as{k}="pc1.2",<1em,0.5cm>*\as{a}="pv1.2",<3em,0.5cm>*\as{ }="pc2.2",<2em,0.5cm>*\as{u}="pv2.2",<4em,0.5cm>*\as{t}="pc3.2",
			<0pt,0cm>*\as{f}="pc1.1",<1em,0cm>*\as{a}="pv1.1",<3em,0cm>*\as{f}="pc2.1",<2em,0cm>*\as{i}="pv2.1",<4em,0cm>*\as{ }="pc3.1",
			"c1"+U;"x1"+D**\dir{-};"c2"+U;"x3"+D**\dir{-};"v1"+U;"x2"+D**\dir{-};"v2"+U;"x4"+D**\dir{-};"c3"+U;"x5"+D**\dir{-};
			"pc1"+U;"c1"+D**\dir{-};"pc2"+U;"c2"+D**\dir{-};"pv1"+U;"v1"+D**\dir{-};"pv2"+U;"v2"+D**\dir{-};"pc3"+U;"c3"+D**\dir{-};
			<3.5em,2.55cm>*\as{\tikz[red,thick,dashed,baseline=0.9ex]\draw (0,0) rectangle (0.75cm,0.4cm);}="box",
		\endxy}\label{as:muiq3}
	\end{xlist}}\label{as:muiq}
\end{exe}
\end{multicols}
\begin{multicols}{3}
\begin{exe}
	\sn{\begin{xlist}
		\exi{d.}\exia{\xy
			<0pt,3cm>*\as{\x}="x1",<1em,3cm>*\as{\x}="x2",<3em,3cm>*\as{\x}="x3",<2em,3cm>*\as{\x}="x4",<4em,3cm>*\as{\x}="x5",
			<0pt,2cm>*\as{C}="c1",<1em,2cm>*\as{V}="v1",<3em,2cm>*\as{C}="c2",<2em,2cm>*\as{V}="v2",<4em,2cm>*\as{\xc{C}}="c3",<4.75em,2cm>*\as{\#}="ed",	
			<0pt,1cm>*\as{m}="pc1",<1em,1cm>*\as{u}="pv1",<3em,1cm>*\as{ʔ}="pc2",<2em,1cm>*\as{i}="pv2",<4em,1cm>*\as{t}="pc3",
			<0pt,0.5cm>*\as{k}="pc1.2",<1em,0.5cm>*\as{a}="pv1.2",<3em,0.5cm>*\as{ }="pc2.2",<2em,0.5cm>*\as{u}="pv2.2",<4em,0.5cm>*\as{t}="pc3.2",
			<0pt,0cm>*\as{f}="pc1.1",<1em,0cm>*\as{a}="pv1.1",<3em,0cm>*\as{f}="pc2.1",<2em,0cm>*\as{i}="pv2.1",<4em,0cm>*\as{ }="pc3.1",
			"c1"+U;"x1"+D**\dir{-};"c2"+U;"x3"+D**\dir{-};"v1"+U;"x2"+D**\dir{-};"v2"+U;"x4"+D**\dir{-};{\ar@{-}|-(.425)*@{|} |-{\hole} |-(.575)*@{|} "c3"+U;"x5"+D};
			"pc1"+U;"c1"+D**\dir{-};"pc2"+U;"c2"+D**\dir{-};"pv1"+U;"v1"+D**\dir{-};"pv2"+U;"v2"+D**\dir{-};{\ar@{-}|-(.425)*@{|} |-{\hole} |-(.575)*@{|} "pc3"+U;"c3"+D};
			<4em,1.75cm>*\as{\tikz[red,thick,dashed,baseline=0.9ex]\draw (0,0) rectangle (0.35cm,3cm);}="box",
		\endxy}
		\exi{e.}\exia{\xy
			<0pt,3cm>*\as{\x}="x1",<1em,3cm>*\as{\x}="x2",<3em,3cm>*\as{\x}="x3",<2em,3cm>*\as{\x}="x4",
			<0pt,2cm>*\as{C}="c1",<1em,2cm>*\as{V}="v1",<3em,2cm>*\as{C}="c2",<2em,2cm>*\as{V}="v2",<3.75em,2cm>*\as{\#}="ed",	
			<0pt,1cm>*\as{m}="pc1",<1em,1cm>*\as{u}="pv1",<3em,1cm>*\as{ʔ}="pc2",<2em,1cm>*\as{i}="pv2",
			<0pt,0.5cm>*\as{k}="pc1.2",<1em,0.5cm>*\as{a}="pv1.2",<3em,0.5cm>*\as{ }="pc2.2",<2em,0.5cm>*\as{u}="pv2.2",
			<0pt,0cm>*\as{f}="pc1.1",<1em,0cm>*\as{a}="pv1.1",<3em,0cm>*\as{f}="pc2.1",<2em,0cm>*\as{i}="pv2.1",
			"c1"+U;"x1"+D**\dir{-};"c2"+U;"x3"+D**\dir{-};"v1"+U;"x2"+D**\dir{-};"v2"+U;"x4"+D**\dir{-};
			"pc1"+U;"c1"+D**\dir{-};"pc2"+U;"c2"+D**\dir{-};"pv1"+U;"v1"+D**\dir{-};"pv2"+U;"v2"+D**\dir{-};
		\endxy}
		\sna{\xy
			<0pt,3cm>*\as{}="x1",<0pt,2cm>*\as{}="c1",<0pt,1cm>*\as{}="pc1",
			<0pt,0.5cm>*\as{}="pc1.2",<0pt,0cm>*\as{}="pc1.1",<0pt,0cm>*\as{}="pc1.3",
		\endxy}
	\end{xlist}}
\end{exe}
\end{multicols}

\subsubsection{Metathesis and mid vowel assimilation}
As discussed in \srf{sec:MidVowAss}, any final mid vowel
assimilates to the height of a previous high vowel after metathesis.
This vowel height assimilation is an instance of vowel harmony,
arising from the fact that sequences of a high vowel and mid vowel
are disallowed in Amarasi (\srf{sec:VowSeq}).

This process is illustrated for \ve{ume} {\ra} \ve{uim} `house' in \qf{as:uim} below.
After metathesis in \qf{as:uim2}, the feature \tsc{[+high]}
of the stressed vowel spreads in (\ref{as:uim}d)
resulting in a sequence of two high vowels in (\ref{as:uim}e).
Unless the \tsc{[+high]} feature of the penultimate vowel is analysed
as privative we would also have to propose that the height features
\tsc{[-high, +mid]} of the final vowel /e/ de-link in (\ref{as:uim}d).

\begin{multicols}{3}
\begin{exe}
	\ex{\begin{xlist}
		\exa{\xy
			<0em,3cm>*\as{\x}="x1",<1em,3cm>*\as{\x}="x2",<2em,3cm>*\as{\x}="x3",<3em,3cm>*\as{\x}="x4",<4em,3cm>*\as{\x}="x5",
			<0em,2cm>*\as{C}="C1",<1em,2cm>*\as{V}="V1",<2em,2cm>*\as{C}="C2",<3em,2cm>*\as{V}="V2",<4em,2cm>*\as{C}="C3",
			<0em,1cm>*\as{[ʔ]}="c1",<1em,1cm>*\as{u}="v1",<2em,1cm>*\as{m}="c2",<3em,1cm>*\as{e}="v2",<4em,1cm>*\as{}="c3",
			<1em,0cm>*\as{\tsc{[+high]}}="f","f"+U;"v1"+D**\dir{-};
			"C1"+U;"x1"+D**\dir{-};"C2"+U;"x3"+D**\dir{-};"V1"+U;"x2"+D**\dir{-};"V2"+U;"x4"+D**\dir{-};"C3"+U;"x5"+D**\dir{-};
			"c1"+U;"C1"+D**\dir{-};"c2"+U;"C2"+D**\dir{-};"v1"+U;"V1"+D**\dir{-};"v2"+U;"V2"+D**\dir{-};
		\endxy}\label{as:uim1}
		\exa{\xy
			<0em,3cm>*\as{\x}="x1",<1em,3cm>*\as{\x}="x2",<2em,3cm>*\as{\x}="x3",<3em,3cm>*\as{\x}="x4",<4em,3cm>*\as{\x}="x5",
			<0em,2cm>*\as{C}="C1",<1em,2cm>*\as{V}="V1",<2em,2cm>*\as{C}="C2",<3em,2cm>*\as{V}="V2",<4em,2cm>*\as{C}="C3",
			<0em,1cm>*\as{[ʔ]}="c1",<1em,1cm>*\as{u}="v1",<2em,1cm>*\as{m}="c2",<3em,1cm>*\as{e}="v2",<4em,1cm>*\as{}="c3",
			<1em,0cm>*\as{\tsc{[+high]}}="f","f"+U;"v1"+D**\dir{-};
			"C1"+U;"x1"+D**\dir{-};"C2"+U;"x4"+D**\dir{.};"V1"+U;"x2"+D**\dir{-};"V2"+U;"x3"+D**\dir{.};"C3"+U;"x5"+D**\dir{-};
			"c1"+U;"C1"+D**\dir{-};"c2"+U;"C2"+D**\dir{-};"v1"+U;"V1"+D**\dir{-};"v2"+U;"V2"+D**\dir{-};
			<2.5em,2.5cm>*\as{\tikz[red,thick,dashed,baseline=0.9ex]\draw (0,0) rectangle (0.8cm,1.5cm);}="box",
		\endxy}\label{as:uim2}
		\exa{\xy
			<0em,3cm>*\as{\x}="x1",<1em,3cm>*\as{\x}="x2",<2em,3cm>*\as{\x}="x3",<3em,3cm>*\as{\x}="x4",<4em,3cm>*\as{\x}="x5",
			<0em,2cm>*\as{C}="C1",<1em,2cm>*\as{V}="V1",<3em,2cm>*\as{C}="C2",<2em,2cm>*\as{V}="V2",<4em,2cm>*\as{C}="C3",
			<0em,1cm>*\as{[ʔ]}="c1",<1em,1cm>*\as{u}="v1",<3em,1cm>*\as{m}="c2",<2em,1cm>*\as{e}="v2",<4em,1cm>*\as{}="c3",
			<1em,0cm>*\as{\tsc{[+high]}}="f","f"+U;"v1"+D**\dir{-};
			"C1"+U;"x1"+D**\dir{-};"C2"+U;"x4"+D**\dir{-};"V1"+U;"x2"+D**\dir{-};"V2"+U;"x3"+D**\dir{-};"C3"+U;"x5"+D**\dir{-};
			"c1"+U;"C1"+D**\dir{-};"c2"+U;"C2"+D**\dir{-};"v1"+U;"V1"+D**\dir{-};"v2"+U;"V2"+D**\dir{-};
			<1.5em,1.05cm>*\as{\tikz[red,thick,dashed,baseline=0.9ex]\draw (0,0) rectangle (0.75cm,0.4cm);}="box",
		\endxy}\label{as:uim3}
	\end{xlist}}\label{as:uim}
\end{exe}
\end{multicols}
\begin{multicols}{3}
\begin{exe}
	\sn{\begin{xlist}
		\exi{d.}\exia{\xy
			<0em,3cm>*\as{\x}="x1",<1em,3cm>*\as{\x}="x2",<2em,3cm>*\as{\x}="x3",<3em,3cm>*\as{\x}="x4",<4em,3cm>*\as{\x}="x5",
			<0em,2cm>*\as{C}="C1",<1em,2cm>*\as{V}="V1",<3em,2cm>*\as{C}="C2",<2em,2cm>*\as{V}="V2",<4em,2cm>*\as{C}="C3",
			<0em,1cm>*\as{[ʔ]}="c1",<1em,1cm>*\as{u}="v1",<3em,1cm>*\as{m}="c2",<2em,1cm>*\as{e}="v2",<4em,1cm>*\as{}="c3",
			<1em,0cm>*\as{\tsc{[+high]}}="f","f"+U;"v1"+D**\dir{-};"f"+U;"v2"+D**\dir{.};
			"C1"+U;"x1"+D**\dir{-};"C2"+U;"x4"+D**\dir{-};"V1"+U;"x2"+D**\dir{-};"V2"+U;"x3"+D**\dir{-};"C3"+U;"x5"+D**\dir{-};
			"c1"+U;"C1"+D**\dir{-};"c2"+U;"C2"+D**\dir{-};"v1"+U;"V1"+D**\dir{-};"v2"+U;"V2"+D**\dir{-};
			<1.5em,1.05cm>*\as{\tikz[red,thick,dashed,baseline=0.9ex]\draw (0,0) rectangle (0.75cm,0.4cm);}="box",
		\endxy}\label{as:uim4}
		\exi{e.}\exia{\xy
			<0em,3cm>*\as{\x}="x1",<1em,3cm>*\as{\x}="x2",<2em,3cm>*\as{\x}="x3",<3em,3cm>*\as{\x}="x4",<4em,3cm>*\as{\x}="x5",
			<0em,2cm>*\as{C}="C1",<1em,2cm>*\as{V}="V1",<3em,2cm>*\as{C}="C2",<2em,2cm>*\as{V}="V2",<4em,2cm>*\as{\xc{C}}="C3",
			<0em,1cm>*\as{[ʔ]}="c1",<1em,1cm>*\as{u}="v1",<3em,1cm>*\as{m}="c2",<2em,1cm>*\as{i}="v2",<4em,1cm>*\as{}="c3",
			<1.5em,0cm>*\as{\tsc{[+high]}}="f","f"+U;"v1"+D**\dir{-};"f"+U;"v2"+D**\dir{-};
			"C1"+U;"x1"+D**\dir{-};"C2"+U;"x4"+D**\dir{-};"V1"+U;"x2"+D**\dir{-};"V2"+U;"x3"+D**\dir{-};"C3"+U;"x5"+D**\dir{-};
			"c1"+U;"C1"+D**\dir{-};"c2"+U;"C2"+D**\dir{-};"v1"+U;"V1"+D**\dir{-};"v2"+U;"V2"+D**\dir{-};
		\endxy}\label{as:uim5}
		\exi{f.}\exia{\xy
			<0em,3cm>*\as{\x}="x1",<1em,3cm>*\as{\x}="x2",<2em,3cm>*\as{\x}="x3",<3em,3cm>*\as{\x}="x4",
			<0em,2cm>*\as{C}="C1",<1em,2cm>*\as{V}="V1",<3em,2cm>*\as{C}="C2",<2em,2cm>*\as{V}="V2",
			<0em,1cm>*\as{[ʔ]}="c1",<1em,1cm>*\as{u}="v1",<3em,1cm>*\as{m}="c2",<2em,1cm>*\as{i}="v2",
			<1.5em,0cm>*\as{\tsc{[+high]}}="f","f"+U;"v1"+D**\dir{-};"f"+U;"v2"+D**\dir{-};
			"C1"+U;"x1"+D**\dir{-};"C2"+U;"x4"+D**\dir{-};"V1"+U;"x2"+D**\dir{-};"V2"+U;"x3"+D**\dir{-};
			"c1"+U;"C1"+D**\dir{-};"c2"+U;"C2"+D**\dir{-};"v1"+U;"V1"+D**\dir{-};"v2"+U;"V2"+D**\dir{-};
		\endxy}\label{as:uim6}
	\end{xlist}}
\end{exe}
\end{multicols}

\subsubsection{Metathesis and vowel deletion}
The vowel deletion in words with a phonetic diphthong,
such as \ve{nautus} [ˈnəwtʊs] {\ra} \ve{nautus} [ˈnə.ʊt] `beetle'
results from metathesis and the fact that Amarasi does
not allow sequences of three surface vowels.
Recall from \srf{sec:Str} that the first two vowels
of words with a phonetic diphthong are associated to a single V-slot,
as shown by the fact that stress falls on the antepenultimate
vowel rather than the penultimate vowel.

The formation of the M\=/form for
\ve{nautus} [ˈnəwtʊs] {\ra} \ve{nautus} [ˈnə.ʊt] `beetle' 
is illustrated in \qf{as:aun} below.
Metathesis in \qf{as:aun2} results in a
surface sequence of three vowels in \qf{as:aun3};
the first V-slot is associated to two vowels
which are adjacent to another vowel associated to a single V-slot.
As a result, the final vowel is deleted in (\ref{as:aun}d),
with subsequent re-association of the adjacent vowel into the now empty V-slot in (\ref{as:aun}e).
The final C-slot is also deleted yielding the output shown in (\ref{as:aun}f).

\begin{multicols}{3}
\begin{exe}
	\ex{\begin{xlist}
		\exa{\xy
			<0em,2cm>*\as{\x}="x1",<1em,2cm>*\as{\x}="x2",<2em,2cm>*\as{\x}="x3",<3em,2cm>*\as{\x}="x4",<4em,2cm>*\as{\x}="x5",
			<0em,1cm>*\as{C}="C1",<4em,1cm>*\as{C}="C3",<2em,1cm>*\as{C}="C2",
			"C1"+U;"x1"+D**\dir{-};"C3"+U;"x5"+D**\dir{-};"C2"+U;"x3"+D**\dir{-};
			<0em,0cm>*\as{n}="c1",<2em,0cm>*\as{t}="c2","c2"+U;"C2"+D**\dir{-};<4em,0cm>*\as{s}="c3","c3"+U;"C3"+D**\dir{-};
			<1em,1cm>*\as{V}="V1",<3em,1cm>*\as{V}="V2","V1"+U;"x2"+D**\dir{-};"V2"+U;"x4"+D**\dir{-};
			<0.7em,0cm>*\as{a}="v1",<1.3em,0cm>*\as{u}="v3",<3em,0cm>*\as{u}="v2",
			"c1"+U;"C1"+D**\dir{-};"v1"+U;"V1"+D**\dir{-};"v2"+U;"V2"+D**\dir{-};"v3"+U;"V1"+D**\dir{-};
		\endxy}\label{as:aun1}
		\exa{\xy
			<0em,2cm>*\as{\x}="x1",<1em,2cm>*\as{\x}="x2",<2em,2cm>*\as{\x}="x3",<3em,2cm>*\as{\x}="x4",<4em,2cm>*\as{\x}="x5",
			<0em,1cm>*\as{C}="C1",<4em,1cm>*\as{C}="C3",<2em,1cm>*\as{C}="C2",
			"C1"+U;"x1"+D**\dir{-};"C3"+U;"x5"+D**\dir{-};"C2"+U;"x4"+D**\dir{.};
			<0em,0cm>*\as{n}="c1",<2em,0cm>*\as{t}="c2","c2"+U;"C2"+D**\dir{-};<4em,0cm>*\as{s}="c3","c3"+U;"C3"+D**\dir{-};
			<1em,1cm>*\as{V}="V1",<3em,1cm>*\as{V}="V2","V1"+U;"x2"+D**\dir{-};"V2"+U;"x3"+D**\dir{.};
			<0.7em,0cm>*\as{a}="v1",<1.3em,0cm>*\as{u}="v3",<3em,0cm>*\as{u}="v2",
			"c1"+U;"C1"+D**\dir{-};"v1"+U;"V1"+D**\dir{-};"v2"+U;"V2"+D**\dir{-};"v3"+U;"V1"+D**\dir{-};
			<2.5em,1.5cm>*\as{\tikz[red,thick,dashed,baseline=0.9ex]\draw (0,0) rectangle (0.8cm,1.5cm);}="box",
		\endxy}\label{as:aun2}
		\exa{\xy
			<0em,2cm>*\as{\x}="x1",<1em,2cm>*\as{\x}="x2",<2em,2cm>*\as{\x}="x3",<3em,2cm>*\as{\x}="x4",<4em,2cm>*\as{\x}="x5",
			<0em,1cm>*\as{C}="C1",<4em,1cm>*\as{C}="C3",<3em,1cm>*\as{C}="C2",
			"C1"+U;"x1"+D**\dir{-};"C3"+U;"x5"+D**\dir{-};"C2"+U;"x4"+D**\dir{-};
			<0em,0cm>*\as{n}="c1",<3em,0cm>*\as{t}="c2","c2"+U;"C2"+D**\dir{-};<4em,0cm>*\as{s}="c3","c3"+U;"C3"+D**\dir{-};
			<1em,1cm>*\as{V}="V1",<2em,1cm>*\as{V}="V2","V1"+U;"x2"+D**\dir{-};"V2"+U;"x3"+D**\dir{-};
			<0.7em,0cm>*\as{a}="v1",<1.3em,0cm>*\as{u}="v3",<2em,0cm>*\as{u}="v2",
			"c1"+U;"C1"+D**\dir{-};"v1"+U;"V1"+D**\dir{-};"v2"+U;"V2"+D**\dir{-};"v3"+U;"V1"+D**\dir{-};
			<1.35em,0.025cm>*\as{\tikz[red,thick,dashed,baseline=0.9ex]\draw (0,0) rectangle (0.8cm,0.4cm);}="box",
		\endxy}\label{as:aun3}
	\end{xlist}}\label{as:aun}
\end{exe}
\end{multicols}
\begin{multicols}{3}
\begin{exe}
	\sn{\begin{xlist}
		\exi{d.}\exia{\xy
			<0em,2cm>*\as{\x}="x1",<1em,2cm>*\as{\x}="x2",<2em,2cm>*\as{\x}="x3",<3em,2cm>*\as{\x}="x4",<4em,2cm>*\as{\x}="x5",
			<0em,1cm>*\as{C}="C1",<4em,1cm>*\as{C}="C3",<3em,1cm>*\as{C}="C2",
			"C1"+U;"x1"+D**\dir{-};"C3"+U;"x5"+D**\dir{-};"C2"+U;"x4"+D**\dir{-};
			<0em,0cm>*\as{n}="c1",<3em,0pt>*\as{t}="c2","c2"+U;"C2"+D**\dir{-};<4em,0cm>*\as{s}="c3","c3"+U;"C3"+D**\dir{-};
			<1em,1cm>*\as{V}="V1",<2em,1cm>*\as{V}="V2","V1"+U;"x2"+D**\dir{-};"V2"+U;"x3"+D**\dir{-};
			<0.7em,0cm>*\as{a}="v1",<1.3em,0pt>*\as{u}="v3",<2em,0pt>*\as{\xc{u}}="v2","c1"+U;"C1"+D**\dir{-};"v1"+U;"V1"+D**\dir{-};"v3"+U;"V1"+D**\dir{-};
			{\ar@{-}|-(.425)*@{|} |-{\hole} |-(.575)*@{|} "v2"+U;"V2"+D};
			<2em,0.5cm>*\as{\tikz[red,thick,dashed,baseline=0.9ex]\draw (0,0) rectangle (0.4cm,1.5cm);}="box",
		\endxy}
		\exi{e.}\exia{\xy
			<0em,2cm>*\as{\x}="x1",<1em,2cm>*\as{\x}="x2",<2em,2cm>*\as{\x}="x3",<3em,2cm>*\as{\x}="x4",<4em,2cm>*\as{\x}="x5",
			<0em,1cm>*\as{C}="C1",<4em,1cm>*\as{\xc{C}}="C3",<3em,1cm>*\as{C}="C2",
			"C1"+U;"x1"+D**\dir{-};"C3"+U;"x5"+D**\dir{-};"C2"+U;"x4"+D**\dir{-};
			<0em,0cm>*\as{n}="c1",<3em,0pt>*\as{t}="c2","c2"+U;"C2"+D**\dir{-};<4em,0cm>*\as{s}="c3","c3"+U;"C3"+D**\dir{-};
			<1em,1cm>*\as{V}="V1",<2em,1cm>*\as{V}="V2","V1"+U;"x2"+D**\dir{-};"V2"+U;"x3"+D**\dir{-};
			<0.7em,0pt>*\as{a}="v1",<1.3em,0pt>*\as{u}="v3",
			"c1"+U;"C1"+D**\dir{-};"v1"+U;"V1"+D**\dir{-};{\ar@{-}|-(.35)*@{|} |-{\hole} |-(.5)*@{|} "v3"+U;"V1"+D};"v3"+U;"V2"+D**\dir{.};
			<1.5em,0.5cm>*\as{\tikz[red,thick,dashed,baseline=0.9ex]\draw (0,0) rectangle (0.9cm,1.5cm);}="box",
		\endxy}
		\exi{f.}\exia{\xy
			<0em,2cm>*\as{\x}="x1",<1em,2cm>*\as{\x}="x2",<2em,2cm>*\as{\x}="x3",<3em,2cm>*\as{\x}="x4",
			<0em,1cm>*\as{C}="C1",<3em,1cm>*\as{C}="C2",
			"C1"+U;"x1"+D**\dir{-};"C2"+U;"x4"+D**\dir{-};
			<0em,0cm>*\as{n}="c1",<3em,0pt>*\as{t}="c2","c2"+U;"C2"+D**\dir{-};
			<1em,1cm>*\as{V}="V1",<2em,1cm>*\as{V}="V2","V1"+U;"x2"+D**\dir{-};"V2"+U;"x3"+D**\dir{-};
			<1em,0cm>*\as{a}="v1",<2em,0pt>*\as{u}="v2","v1"+U;"V1"+D**\dir{-};"v2"+U;"V2"+D**\dir{-};
			"c1"+U;"C1"+D**\dir{-};
		\endxy}
	\end{xlist}}
\end{exe}
\end{multicols}

Evidence that it is the final vowel and not the penultimate vowel
which is deleted comes from the word
\ve{n-aena} `runs, flees' with the M\=/form \ve{n-aen}.
If the second vowel were deleted after metathesis
in words with an initial phonetic diphthong, \ve{n-aena} `runs, flees'
would have the M\=/form \it{*n-aan}.

There are no other processes in Amarasi which create a sequence
of three vowels within a morpheme.
All other potential VVV sequences would occur across a morpheme
boundary in which case consonants are inserted;
a voiced obstruent morpheme finally (\srf{sec:EmpCSloConIns}),
and a glottal stop morpheme initially (\srf{sec:GloStoIns}).

\subsection{The morphemically conditioned rule: assimilation of /a/}\label{sec:MorRulAssOfA}
The morphological process of metathesis
triggers assimilation of final /a/,
such as in \ve{ni\tbr{ma}} {\ra} \ve{ni\tbr{im}} `five'.
This rule is given as rule \qf{ex:'VaC->VC} below.
This rule states that the features (represented by \tsc{[+F.]}) of the stressed vowel spread
when immediately followed by /a/ and a filled C-slot.

\begin{exe}
	\exa{\xy
		<0em,2cm>*\as{\'V}="V1",<1em,2cm>*\as{V}="V2",<2em,2cm>*\as{C}="C1",
		<0em,1cm>*\as{}="v1",<1em,1cm>*\as{a}="v2",<2em,1cm>*\as{}="c1",
		<0em,0cm>*\as{\tsc{[+F.]}}="f","f"+U;"v1"+D**\dir{-};"f"+U;"v2"+D**\dir{.};
		"v1"+U;"V1"+D**\dir{-};"c1"+U;"C1"+D**\dir{-};"v2"+U;"V2"+D**\dir{-};
	\endxy}\label{ex:'VaC->VC}
\end{exe}

Assimilation of /a/ is a derived environment effect.
It is not dissimilar to umlaut in German plurals,
in that both occur only in morphologically derived environments.
In German, a floating autosegment triggers fronting of the root vowel
only in morphologically derived environments,
such as in plurals \citep[181ff]{wi96}.
In Amarasi /a/ assimilation only occurs
in a morphologically derived environment: the M\=/form.
This, I take it, is what is meant by a
\emph{morphemically conditioned rule}:
a rule the operation of which is dependent on
and occurs only after an independent morphological rule.

The rule of /a/ assimilation is formulated 
in \qf{ex:'VaC->VC} as a general phonological rule.
This is possible because under the analysis involving
the obligatory CVCVC foot the environment which triggers
assimilation of /a/ -- two immediately adjacent V-slots --
only arises after metathesis.

That the only vowel which assimilate in Amarasi is /a/
can be partially explained by the fact that it is almost featureless.\footnote{
		There is also evidence that /a/ is the default vowel,
		as it is the vowel used in epenthesis (\srf{sec:Epe}).}
Perhaps apart from the feature \tsc{[+low]},
/a/ is not specified for \tsc{front} or \tsc{back}.
This lack of features allows the features of the stressed vowel to spread
when the V-slot to which /a/ is associated occurs immediately after it.

The formation \ve{nima} {\ra} \ve{niim} `five' is given in \qf{as:niim} below.
Metathesis occurs in \qf{as:niim2},
resulting in the V-slot to which /a/ is associated
occurring immediately after a stressed V-slot and before a filled C-slot in \qf{as:niim3}.
Thus, the features of the stressed vowel spread in (\ref{as:niim}d),
creating a sequence of two identical vowels in (\ref{as:niim}e).
The final C-slot is then deleted yielding the final output shown in (\ref{as:niim}f).

\newpage
\begin{multicols}{3}
\begin{exe}
	\ex{\begin{xlist}
		\ex\raisebox{\dimexpr-\totalheight+5ex\relax}{\xy
			<0em,3.3cm>*\as{\x}="x1",<1em,3.3cm>*\as{\x}="x2",<2em,3.3cm>*\as{\x}="x3",<3em,3.3cm>*\as{\x}="x4",<4em,3.3cm>*\as{\x}="x5",
			<0em,2.3cm>*\as{C}="C1",<1em,2.3cm>*\as{\'V}="V1",<2em,2.3cm>*\as{C}="C2",<3em,2.3cm>*\as{V}="V2",<4em,2.3cm>*\as{C}="C3",
			<0em,1.3cm>*\as{n}="c1",<1em,1.3cm>*\as{i}="v1",<2em,1.3cm>*\as{m}="c2",<3em,1.3cm>*\as{a}="v2",<4em,1.3cm>*\as{}="c3",
			<1em,0cm>*\as{{$\left[\hspace{-2mm}\begin{array}{l}\textrm{\tsc{+high}}\\\textrm{\tsc{+front}}\end{array}\hspace{-2mm}\right]$}}="f",
			"f"+U;"v1"+D**\dir{-};
			"C1"+U;"x1"+D**\dir{-};"C2"+U;"x3"+D**\dir{-};"V1"+U;"x2"+D**\dir{-};"V2"+U;"x4"+D**\dir{-};"C3"+U;"x5"+D**\dir{-};
			"c1"+U;"C1"+D**\dir{-};"c2"+U;"C2"+D**\dir{-};"v1"+U;"V1"+D**\dir{-};"v2"+U;"V2"+D**\dir{-};
		\endxy}\label{as:niim1}
		\ex\raisebox{\dimexpr-\totalheight+5ex\relax}{\xy
			<0em,3.3cm>*\as{\x}="x1",<1em,3.3cm>*\as{\x}="x2",<2em,3.3cm>*\as{\x}="x3",<3em,3.3cm>*\as{\x}="x4",<4em,3.3cm>*\as{\x}="x5",
			<0em,2.3cm>*\as{C}="C1",<1em,2.3cm>*\as{\'V}="V1",<2em,2.3cm>*\as{C}="C2",<3em,2.3cm>*\as{V}="V2",<4em,2.3cm>*\as{C}="C3",
			<0em,1.3cm>*\as{n}="c1",<1em,1.3cm>*\as{i}="v1",<2em,1.3cm>*\as{m}="c2",<3em,1.3cm>*\as{a}="v2",<4em,1.3cm>*\as{}="c3",
			<1em,0cm>*\as{{$\left[\hspace{-2mm}\begin{array}{l}\textrm{\tsc{+high}}\\\textrm{\tsc{+front}}\end{array}\hspace{-2mm}\right]$}}="f",
			"f"+U;"v1"+D**\dir{-};
			"C1"+U;"x1"+D**\dir{-};"C2"+U;"x4"+D**\dir{.};"V1"+U;"x2"+D**\dir{-};"V2"+U;"x3"+D**\dir{.};"C3"+U;"x5"+D**\dir{-};
			"c1"+U;"C1"+D**\dir{-};"c2"+U;"C2"+D**\dir{-};"v1"+U;"V1"+D**\dir{-};"v2"+U;"V2"+D**\dir{-};
			<2.5em,2.7cm>*\as{\tikz[red,thick,dashed,baseline=0.9ex]\draw (0,0) rectangle (0.8cm,1.5cm);}="box",
		\endxy}\label{as:niim2}
		\ex\raisebox{\dimexpr-\totalheight+5ex\relax}{\xy
			<0em,3.3cm>*\as{\x}="x1",<1em,3.3cm>*\as{\x}="x2",<2em,3.3cm>*\as{\x}="x3",<3em,3.3cm>*\as{\x}="x4",<4em,3.3cm>*\as{\x}="x5",
			<0em,2.3cm>*\as{C}="C1",<1em,2.3cm>*\as{\'V}="V1",<3em,2.3cm>*\as{C}="C2",<2em,2.3cm>*\as{V}="V2",<4em,2.3cm>*\as{C}="C3",
			<0em,1.3cm>*\as{n}="c1",<1em,1.3cm>*\as{i}="v1",<3em,1.3cm>*\as{m}="c2",<2em,1.3cm>*\as{a}="v2",<4em,1.3cm>*\as{}="c3",
			<1em,0cm>*\as{{$\left[\hspace{-2mm}\begin{array}{l}\textrm{\tsc{+high}}\\\textrm{\tsc{+front}}\end{array}\hspace{-2mm}\right]$}}="f",
			"f"+U;"v1"+D**\dir{-};
			"C1"+U;"x1"+D**\dir{-};"C2"+U;"x4"+D**\dir{-};"V1"+U;"x2"+D**\dir{-};"V2"+U;"x3"+D**\dir{-};"C3"+U;"x5"+D**\dir{-};
			"c1"+U;"C1"+D**\dir{-};"c2"+U;"C2"+D**\dir{-};"v1"+U;"V1"+D**\dir{-};"v2"+U;"V2"+D**\dir{-};
			<1.5em,1.35cm>*\as{\tikz[red,thick,dashed,baseline=0.9ex]\draw (0,0) rectangle (0.75cm,0.4cm);}="box",
		\endxy}\label{as:niim3}
	\end{xlist}}\label{as:niim}
\end{exe}
\end{multicols}
\begin{multicols}{3}
\begin{exe}
	\sn{\begin{xlist}
		\exi{d.}\raisebox{\dimexpr-\totalheight+5ex\relax}{\xy
			<0em,3.3cm>*\as{\x}="x1",<1em,3.3cm>*\as{\x}="x2",<2em,3.3cm>*\as{\x}="x3",<3em,3.3cm>*\as{\x}="x4",<4em,3.3cm>*\as{\x}="x5",
			<0em,2.3cm>*\as{C}="C1",<1em,2.3cm>*\as{\'V}="V1",<3em,2.3cm>*\as{C}="C2",<2em,2.3cm>*\as{V}="V2",<4em,2.3cm>*\as{C}="C3",
			<0em,1.3cm>*\as{n}="c1",<1em,1.3cm>*\as{i}="v1",<3em,1.3cm>*\as{m}="c2",<2em,1.3cm>*\as{a}="v2",<4em,1.3cm>*\as{}="c3",
			<1em,0cm>*\as{{$\left[\hspace{-2mm}\begin{array}{l}\textrm{\tsc{+high}}\\\textrm{\tsc{+front}}\end{array}\hspace{-2mm}\right]$}}="f",
			"f"+U;"v1"+D**\dir{-};"f"+U;"v2"+D**\dir{.};
			"C1"+U;"x1"+D**\dir{-};"C2"+U;"x4"+D**\dir{-};"V1"+U;"x2"+D**\dir{-};"V2"+U;"x3"+D**\dir{-};"C3"+U;"x5"+D**\dir{-};
			"c1"+U;"C1"+D**\dir{-};"c2"+U;"C2"+D**\dir{-};"v1"+U;"V1"+D**\dir{-};"v2"+U;"V2"+D**\dir{-};
			<1.5em,1.35cm>*\as{\tikz[red,thick,dashed,baseline=0.9ex]\draw (0,0) rectangle (0.75cm,0.4cm);}="box",
		\endxy}\label{as:niim4}
		\exi{e.}\raisebox{\dimexpr-\totalheight+5ex\relax}{\xy
			<0em,3.3cm>*\as{\x}="x1",<1em,3.3cm>*\as{\x}="x2",<2em,3.3cm>*\as{\x}="x3",<3em,3.3cm>*\as{\x}="x4",<4em,3.3cm>*\as{\x}="x5",
			<0em,2.3cm>*\as{C}="C1",<1em,2.3cm>*\as{\'V}="V1",<3em,2.3cm>*\as{C}="C2",<2em,2.3cm>*\as{V}="V2",<4em,2.3cm>*\as{\xc{C}}="C3",
			<0em,1.3cm>*\as{n}="c1",<1em,1.3cm>*\as{i}="v1",<3em,1.3cm>*\as{m}="c2",<2em,1.3cm>*\as{i}="v2",<4em,1.3cm>*\as{}="c3",
			<1.5em,0cm>*\as{{$\left[\hspace{-2mm}\begin{array}{l}\textrm{\tsc{+high}}\\\textrm{\tsc{+front}}\end{array}\hspace{-2mm}\right]$}}="f",
			"f"+U;"v1"+D**\dir{-};"f"+U;"v2"+D**\dir{-};
			"C1"+U;"x1"+D**\dir{-};"C2"+U;"x4"+D**\dir{-};"V1"+U;"x2"+D**\dir{-};"V2"+U;"x3"+D**\dir{-};"C3"+U;"x5"+D**\dir{-};
			"c1"+U;"C1"+D**\dir{-};"c2"+U;"C2"+D**\dir{-};"v1"+U;"V1"+D**\dir{-};"v2"+U;"V2"+D**\dir{-};
		\endxy}\label{as:niim5}
		\exi{f.}\raisebox{\dimexpr-\totalheight+5ex\relax}{\xy
			<0em,3.3cm>*\as{\x}="x1",<1em,3.3cm>*\as{\x}="x2",<2em,3.3cm>*\as{\x}="x3",<3em,3.3cm>*\as{\x}="x4",
			<0em,2.3cm>*\as{C}="C1",<1em,2.3cm>*\as{\'V}="V1",<3em,2.3cm>*\as{C}="C2",<2em,2.3cm>*\as{V}="V2",
			<0em,1.3cm>*\as{n}="c1",<1em,1.3cm>*\as{i}="v1",<3em,1.3cm>*\as{m}="c2",<2em,1.3cm>*\as{i}="v2",
			<1.5em,0cm>*\as{{$\left[\hspace{-2mm}\begin{array}{l}\textrm{\tsc{+high}}\\\textrm{\tsc{+front}}\end{array}\hspace{-2mm}\right]$}}="f",
			"f"+U;"v1"+D**\dir{-};"f"+U;"v2"+D**\dir{-};
			"C1"+U;"x1"+D**\dir{-};"C2"+U;"x4"+D**\dir{-};"V1"+U;"x2"+D**\dir{-};"V2"+U;"x3"+D**\dir{-};
			"c1"+U;"C1"+D**\dir{-};"c2"+U;"C2"+D**\dir{-};"v1"+U;"V1"+D**\dir{-};"v2"+U;"V2"+D**\dir{-};
		\endxy}\label{as:niim6}
	\end{xlist}}
\end{exe}
\end{multicols}

Under this analysis /a/ assimilation is triggered by the presence
of two immediately adjacent V-slots;
an environment which only occur in M\=/forms.
The lack of assimilation in U\=/forms such as \ve{kuan} `village' is
explained by the fact that there is an intervening C-slot between
the two vowels; \ve{ku{\gap}an}.
The environment necessary for the operation
of rule \qf{ex:'VaC->VC} is not present.

Assimilation of /a/ only occurs before filled C-slots.
That is, it does not occur in the M\=/form of words
such as \ve{kuan} {\ra} \ve{kua} `village'.
I analyse the lack of assimilation in such forms as being due
to the lack of a following filled C-slot.

The formation of \ve{kuan} {\ra} \ve{kua} `village'
is given in \qf{as:kua} below.
Metathesis at the CV tier occurs in \qf{as:kua2},
resulting in the V-slot to which /a/ is associated occurring directly after the stressed V-slot.
However, the following C-slot is empty.
This means the environment under which /a/ assimilation occurs is not present.
Thus, no assimilation takes place.
The final C-slot is then deleted in (\ref{as:kua}d){\textendash}(\ref{as:kua}e).

\begin{multicols}{3}
\begin{exe}
	\ex{\begin{xlist}
		\exa{\xy
			<0em,2cm>*\as{\x}="x1",<1em,2cm>*\as{\x}="x2",<2em,2cm>*\as{\x}="x3",<3em,2cm>*\as{\x}="x4",<4em,2cm>*\as{\x}="x5",
			<0em,1cm>*\as{C}="C1",<4em,1cm>*\as{C}="C3",<2em,1cm>*\as{C}="C2",
			"C1"+U;"x1"+D**\dir{-};"C2"+U;"x3"+D**\dir{-};"C3"+U;"x5"+D**\dir{-};
			<0em,0cm>*\as{k}="c1",<4em,0cm>*\as{n}="c3","c1"+U;"C1"+D**\dir{-};"c3"+U;"C3"+D**\dir{-};
			<1em,1cm>*\as{\'V}="V1",<3em,1cm>*\as{V}="V2","V1"+U;"x2"+D**\dir{-};"V2"+U;"x4"+D**\dir{-};
			<1em,0cm>*\as{u}="v1",<3em,0cm>*\as{a}="v2","v1"+U;"V1"+D**\dir{-};"v2"+U;"V2"+D**\dir{-};
		\endxy}\label{as:kua1}
		\exa{\xy
			<0em,2cm>*\as{\x}="x1",<1em,2cm>*\as{\x}="x2",<2em,2cm>*\as{\x}="x3",<3em,2cm>*\as{\x}="x4",<4em,2cm>*\as{\x}="x5",
			<0em,1cm>*\as{C}="C1",<4em,1cm>*\as{C}="C3",<2em,1cm>*\as{C}="C2",
			"C1"+U;"x1"+D**\dir{-};"C3"+U;"x5"+D**\dir{-};"C2"+U;"x4"+D**\dir{.};
			<0em,0cm>*\as{k}="c1",<4em,0cm>*\as{n}="c3","c1"+U;"C1"+D**\dir{-};"c3"+U;"C3"+D**\dir{-};
			<1em,1cm>*\as{\'V}="V1",<3em,1cm>*\as{V}="V2","V1"+U;"x2"+D**\dir{-};"V2"+U;"x3"+D**\dir{.};
			<1em,0cm>*\as{u}="v1",<3em,0cm>*\as{a}="v2","v1"+U;"V1"+D**\dir{-};"v2"+U;"V2"+D**\dir{-};
			<2.5em,1.5cm>*\as{\tikz[red,thick,dashed,baseline=0.9ex]\draw (0,0) rectangle (0.8cm,1.5cm);}="box",
		\endxy}\label{as:kua2}
		\exa{\xy
			<0em,2cm>*\as{\x}="x1",<1em,2cm>*\as{\x}="x2",<2em,2cm>*\as{\x}="x3",<3em,2cm>*\as{\x}="x4",<4em,2cm>*\as{\x}="x5",
			<0em,1cm>*\as{C}="C1",<4em,1cm>*\as{C}="C3",<3em,1cm>*\as{C}="C2",
			"C1"+U;"x1"+D**\dir{-};"C3"+U;"x5"+D**\dir{-};"C2"+U;"x4"+D**\dir{-};
			<0em,0cm>*\as{k}="c1",<4em,0cm>*\as{n}="c3","c1"+U;"C1"+D**\dir{-};"c3"+U;"C3"+D**\dir{-};
			<1em,1cm>*\as{\'V}="V1",<2em,1cm>*\as{V}="V2","V1"+U;"x2"+D**\dir{-};"V2"+U;"x3"+D**\dir{-};
			<1em,0cm>*\as{u}="v1",<2em,0cm>*\as{a}="v2","v1"+U;"V1"+D**\dir{-};"v2"+U;"V2"+D**\dir{-};
			<1.5em,0cm>*\as{\tikz[red,thick,dashed,baseline=0.9ex]\draw (0,0) rectangle (0.8cm,0.6cm);}="box",
			<3em,1cm>*\as{\tikz[red,thick,dashed,baseline=0.9ex]\draw (0,0) rectangle (0.4cm,0.6cm);}="box",
		\endxy}\label{as:kua3}
	\end{xlist}}\label{as:kua}
\end{exe}
\end{multicols}
\begin{multicols}{3}
\begin{exe}
	\sn{\begin{xlist}
		\exi{d.}\exia{\xy
			<0em,2cm>*\as{\x}="x1",<1em,2cm>*\as{\x}="x2",<2em,2cm>*\as{\x}="x3",<3em,2cm>*\as{\x}="x4",<4em,2cm>*\as{\x}="x5",
			<0em,1cm>*\as{C}="C1",<4em,1cm>*\as{\xc{C}}="C3",<3em,1cm>*\as{C}="C2",
			"C1"+U;"x1"+D**\dir{-};"C2"+U;"x4"+D**\dir{-};{\ar@{-}|-(.425)*@{|} |-{\hole} |-(.575)*@{|} "C3"+U;"x5"+D};
			<0em,0cm>*\as{k}="c1",<4em,0cm>*\as{n}="c3","c1"+U;"C1"+D**\dir{-};{\ar@{-}|-(.425)*@{|} |-{\hole} |-(.575)*@{|} "c3"+U;"C3"+D};
			<1em,1cm>*\as{\'V}="V1",<2em,1cm>*\as{V}="V2","V1"+U;"x2"+D**\dir{-};"V2"+U;"x3"+D**\dir{-};
			<1em,0cm>*\as{u}="v1",<2em,0cm>*\as{a}="v2","v1"+U;"V1"+D**\dir{-};"v2"+U;"V2"+D**\dir{-};
			<4em,1cm>*\as{\tikz[red,thick,dashed,baseline=0.9ex]\draw (0,0) rectangle (0.4cm,2.5cm);}="box",
		\endxy}
		\exi{e.}\exia{\xy
			<0em,2cm>*\as{\x}="x1",<1em,2cm>*\as{\x}="x2",<2em,2cm>*\as{\x}="x3",<3em,2cm>*\as{\x}="x4",
			<0em,1cm>*\as{C}="C1",<3em,1cm>*\as{C}="C2",
			"C1"+U;"x1"+D**\dir{-};"C2"+U;"x4"+D**\dir{-};
			<0em,0cm>*\as{k}="c1","c1"+U;"C1"+D**\dir{-};
			<1em,1cm>*\as{\'V}="V1",<2em,1cm>*\as{V}="V2","V1"+U;"x2"+D**\dir{-};"V2"+U;"x3"+D**\dir{-};
			<1em,0cm>*\as{u}="v1",<2em,0cm>*\as{a}="v2","v1"+U;"V1"+D**\dir{-};"v2"+U;"V2"+D**\dir{-};
		\endxy}
		\sna{\xy
			<0em,2cm>*\as{}="x1",<0em,1cm>*\as{}="C1",<0em,0cm>*\as{}="c1",
		\endxy}
	\end{xlist}}
\end{exe}
\end{multicols}

That /a/ is protected from assimilation by a following empty C-slot
finds comparative support from Ro{\Q}is Amarasi.
Ro{\Q}is Amarasi has a process whereby any final unstressed /a/
in a closed syllable (optionally) assimilates to the quality of the stressed vowel.
Examples are given in Table \ref{tab:RoqPosStrAAss}.\footnote{
		Assimilation of /a/ in Ro{\Q}is Amarasi does not usually occur
		before or after the glottal stop: i.e. \ve{keʔ\tbr{a}n}
		`room' and \ve{in\tbr{a}ʔ} `mother'.
		Assimilation of /a/ in closed final syllables also occurs in Timaus
		and the variety of Meto spoken in the village of Oepaha ---
		both adjacent to Ro{\Q}is Amarasi speaking areas.}

\begin{table}[h]
	\centering\caption{Ro{\Q}is Amarasi post-stress /a/ assimilation}\label{tab:RoqPosStrAAss}
		\begin{threeparttable}
			\begin{tabular}{lll}\lsptoprule
				Kotos Amarasi					& Ro{\Q}is Amarasi			& \\ \midrule
				\ve{ʔnim\tbr{a}-f}		&\ve{nim\tbr{i}-f}			& `arm, hand'\\
				\ve{sbet\tbr{a}-f}		&\ve{sbet\tbr{e}-f}			& `upper arm' \\
				\ve{ek\tbr{a}m}				&\ve{er\tbr{e}m, eram}	& `pandanus'  \\
				\ve{na-ten\tbr{a}b}		&\ve{na-ten\tbr{e}b}		& `thinks' \\
				\ve{ok\tbr{a}m}				&\ve{ok\tbr{o}m, okam}	& `melon, gourd'\\
				\ve{or\tbr{a}s}\su{†}	&\ve{or\tbr{o}s}				& `time' \\
				\ve{rum\tbr{a}n}			&\ve{rum\tbr{u}n}				& `empty' \\
				\ve{ut\tbr{a}n}				&\ve{ut\tbr{u}k, utak}	& `vegetables; pumpkin, squash'\su{‡}  \\
				\ve{sur\tbr{a}t}\su{\#}&\ve{sur\tbr{u}t}			& `paper, book' \\
			\lspbottomrule
		\end{tabular}%}
			\begin{tablenotes}
				\item [†]	From Portuguese \it{horas} [ɔras] `hours'.
				\item [‡]	Kotos Amarasi \ve{utan} means only `vegetables'.
				\item [\#]	From Malay \it{surat} `letter'.
			\end{tablenotes}
		\end{threeparttable}
\end{table}

Assimilation of /a/ in closed syllables in Ro{\Q}is is productive, as seen by forms such
as Ro{\Q}is \ve{in\tbr{a}ʔ} `mother' and \ve{maun in\tbr{i}-f} `hen',
as well as borrowings such as Portuguese \it{horas} [ɔr\tbr{a}s]
{\ra} Ro{\Q}is \ve{or\tbr{o}s}.
However, such assimilation does not occur in final
open syllables of a stem, even those which are followed by a consonant.
Two examples are Ro{\Q}is \ve{boʔ=es\tbr{a} =m hiut} `seventeen'
and \ve{n-ok\tbr{a} skoor} `with a school'.
This lack of assimilation can be explained by
the same mechanism which blocks assimilation of /a/ in M\=/forms:
a following empty C-slot.

The reason assimilation of /a/ does not
occur in Kotos Amarasi in the derivation of
an M\=/form from a VVC{\#} final root can also be accounted
for due to the difficulty of recovering
the U\=/form based on the M\=/form if assimilation took place.
In a stem such as \ve{nima} {\ra} \ve{niim} `five'
the majority of its segments are preserved in the M\=/form
with three out of four segments retained.
Similarly, a stem such \ve{rinah} {\ra} \ve{riin} `pomegranate, wood apple'
also preserves most of its segments with the M\=/form
attesting three out of five segments of the U\=/form.
However, in a hypothetical case such as \ve{kuan} {\ra} \it{*kuu} `village'
only two out of four segments would be preserved after metathesis.\footnote{
		M\=/forms such as \ve{mainuan} {\ra} \ve{mainua} `openness' which
		also do not undergo assimilation of final /a/
		show that the final foot, not the stem, is the relevant domain
		which determines whether enough segments are preserved after metathesis.}

Finally, as discussed in \srf{sec:AssOfA}, in Ro{\Q}is Amarasi
/a/ does not assimilate when the metathesising consonant is a glottal stop,
as seen in Ro{\Q}is \ve{n-ro\tbr{ʔa}} {\ra} \ve{n-ro\tbr{aʔ}} `vomit'
which contrasts with Kotos \ve{n-ro\tbr{ʔa}} {\ra} \ve{n-ro\tbr{oʔ}} `vomit'.
No matter how this is explained, it does provide evidence that
the kind of consonant in the C-slot which moves after
metathesis plays a role in /a/ assimilation.
In Kotos Amarasi only an unfilled final C-slot protects
/a/ from undergoing assimilation after metathesis,
while in Ro{\Q}is Amarasi this is extended to final
C-slots which are filled by glottal stop --- the minimal consonant.

\subsection{Summary}
In this section I have proposed a single unified
analysis of the formation of the M\=/form
from the U\=/form in Amarasi.
This analysis is framed under an autosegmental model of phonology \citep{go76}
and a rule-based model of process morphology \citep{ma74,an92}.
My analysis consists of three parts:
one phonological stipulation, one morphological rule,
and one morphemically conditioned rule.
The phonological stipulation and the rules
are repeated in \qf{ex3:Ft->CVCVC}--\qf{ex2:'VaC->VC} below.

\begin{exe}
	\ex{Ft {\ra} CVCVC}\label{ex3:Ft->CVCVC}
	\ex{CV {\ra} VC /\'V{\gap}}\label{ex2:CV->VC/'V-}
	\exa{\xy
		<0em,2cm>*\as{\'V}="V1",<1em,2cm>*\as{V}="V2",<2em,2cm>*\as{C}="C1",
		<0em,1cm>*\as{}="v1",<1em,1cm>*\as{a}="v2",<2em,1cm>*\as{}="c1",
		<0em,0cm>*\as{\tsc{[+F.]}}="f","f"+U;"v1"+D**\dir{-};"f"+U;"v2"+D**\dir{.};
		"v1"+U;"V1"+D**\dir{-};"c1"+U;"C1"+D**\dir{-};"v2"+U;"V2"+D**\dir{-};
	\endxy}\label{ex2:'VaC->VC}
\end{exe}

These two rules and single stipulation,
combined with the general phonotactic constraints of Amarasi,
are sufficient to account for formation of the M\=/forms.
The general phonotactic constraints of Amarasi with which these rules
interact are given below:

\begin{itemize}
	\item clusters of two C-slots are prohibited word finally
	\item sequences of three surface vowels are prohibited
	\item vowel sequences consisting of a high-vowel and mid-vowel are prohibited
\end{itemize}

This rule-based analysis of Amarasi accounts for
all of the data in a single consistent way.
In the next section I consider some alternate analyses.
While these analyses can account for some of the data Amarasi data
they cannot account for all of the data.

\section{Alternate approaches}\label{sec:AltApp}
In the final part of this chapter I consider
the ways alternate approaches would handle the Amarasi data.
There are no existing proposals in the literature which can
adequately handle all of the Amarasi data.

In \srf{sec:ProMor} I consider whether an approach framed within
prosodic morphology \citep{mccpr90,mccpr93} can explain the Amarasi data.
In particular, I consider whether the analysis of \cite{mcc00} for Rotuman metathesis
or that of \cite{he04} for Kwara'ae can be extended to Amarasi.
While such analyses can explain a small amount of the Amarasi data,
they cannot explain it all.
Furthermore, because there is no consistent single prosodic structure present
in Amarasi surface M\=/forms, an analysis framed within prosodic morphology
does not seem appropriate,

In \srf{sec:AltAppPhoMet} I show that Amarasi
metathesis cannot be analysed as phonologically conditioned
as has been proposed for Rotuman \citep{haki98,mcc00},
Kwara'ae \citep{he04}, and Luang \citep{tata15},
as well as Meto \citep{mccko96}.

Finally, in \srf{sec:AffCVMel} I show that it
is typologically implausible to analyse the
Amarasi data as a fundamentally concatenative process
involving affixation of a CV melody to the segmental information of a word.

\subsection{Prosodic morphology}\label{sec:ProMor}
One alternate approach to the data would be to analyse
Amarasi metathesis within the framework of prosodic morphology.
Such an analysis has been proposed by \citet{mcc00} for Rotuman metathesis.
Similarly, \cite{he04} proposes an analysis of synchronic metathesis in
Kwara'ae which is compatible with a prosodic morphological approach.
After a discussion of each of these analyses,
I show that neither can be adapted and/or extended to the Amarasi data.

\subsubsection{Rotuman}\label{sec:ProMorRot}
The forms and functions of metathesis in Rotuman
has been comprehensively summarised in \srf{sec:Rot}.
As discussed in \srf{sec:Rot},
a variety of different processes are used to form the
M\=/form from the U\=/form. These processes include
vowel deletion, metathesis, umlaut, and diphthongisation.
Which process applies depends on the quality of the penultimate
and final vowels of the U\=/form, as well as
whether the U\=/form is VCV{\#} final or VV{\#} final.
There are also a certain number of word shapes with no distinction between the two forms.
Examples of each of these processes extracted from \cite{ch40}
are given in Table \ref{tab:RotPhaAlt} in standard IPA transcription.

\begin{table}[h]
	\caption{Rotuman U-form and M-form alternations}\label{tab:RotPhaAlt}
	\centering
		\begin{tabular}{llll}\lsptoprule
				Process 			& U\=/form	 		& M\=/form			& gloss \\ \midrule
				deletion			&	\it{haŋa}		& \it{haŋ}		& `feed'	\\
											&	\it{tokiri}	& \it{tokir}	& `roll'	\\
				metathesis		&	\it{pure}		& \it{puer} 	& `rule, decide'	\\
											&	\it{hoʔa}		& \it{hoaʔ} 	& `take'	\\
				umlaut				&	\it{mori}		& \it{mør} 		& `orange (fruit)'	\\
											&	\it{ʔuli}		& \it{ʔyl} 		& `skin'	\\
		diphthongisation	&	\it{pupui}	& \it{pupŭi}	& `floor'	\\
											&	\it{foʔou}	& \it{foʔŏu}	& `new'	\\
					no change		&	\it{rii}		& \it{rii} 		& `house'	\\
											&	\it{ree}		& \it{ree}		& `do'	\\
											\lspbottomrule
		\end{tabular}
\end{table}

An analysis of the Rotuman data within the framework of
prosodic morphology and Optimality Theory \citep{prsm93}
is presented in \cite{mcc00}.
\citet[159]{mcc00} bases his analysis on the observation that
{``The incomplete phase [M\=/form] is identical to the complete phase [U\=/form],
except that the \emph{final foot} of the complete phase is realized as a
\emph{monosyllabic foot} in the incomplete phase.''}
Regarding words which form the M\=/form
via metathesis, such as \it{hoʔa} {\ra} \it{hoaʔ} `take',
\cite{mcc00} argues that the M\=/form consists of a single syllable,
as is consistent with more recent descriptions of Rotuman including \cite{be87} and \cite{va02}.\footnote{\label{fn:Problem1}
		It not certain that the M\=/form was
		monosyllabic when \citeauthor{ch40} conducted his fieldwork
		\citet[86]{ch40} states that
		``the stress seems to be levelled out, so to speak,
		in the inc[omplete] phase [M\=/form]. Thus: \it{fo}ra becomes \it{foar},
		which is pronounced almost, though perhaps not quite, as one syllable,
		the stress being evenly distributed {\ldots}''
		While \cite{mcc00} takes this statement to suggest the M\=/form
		is a monosyllabic word, it can also be taken
		to mean that the M\=/form is shorter in phonetic length
		even though it remains two syllables.}

Under \citeauthor{mcc00}'s analysis Rotuman is a weight sensitive language.
Because a (metathesised) monosyllable such as \it{hoaʔ} `take' is
consonant final, it has two morae and bears stress
as expected for a heavy syllable in a weight sensitive language.

\cite{mcc00} also draws upon the observation
by \cite{haki98} that in most cases the use of the two stems in Rotuman
is conditioned by the number of syllables of a following suffix or enclitic.
Suffixes and enclitics consisting of two or more syllables trigger the U\=/form,
while monosyllabic or non-syllabic suffixes and enclitics occur
with a stem in the M\=/form.\footnote{\label{fn:Problem2}
		As noted by \cite[162]{mcc00}, \citet{haki98} identify
		two zero suffixes which occur with the U\=/form
		and one monosyllabic suffix which occurs with the M\=/form.
		\cite{mcc00} does not offer any explanation for their
		aberrant behaviour beyond mentioning that \citet{haki98} analysed such
		forms as taking zero suffixes which bear moraic weight.}

\citet[156]{mcc00} then draws upon the principle of Foot Binarity,
whereby feet are required to consist of a minimum of either two syllables or two morae.
\citeauthor{mcc00} proposes that polysyllabic suffixes and enclitics
are prosodically external to the stem, as they are eligible to form independent feet.
Non-syllabic and monosyllabic suffixes, on the other hand cannot form feet and are thus bound to the stem.

\citet[163]{mcc00} represents forms like \emph{hoaʔ-kia}
`take-\tsc{transitive}' with the structure in \qf{hoaq-kia}.
The stem of this form is in the M\=/form
(a heavy syllable consisting of two morae)
with a polysyllabic suffix/enclitic attached.
\citeauthor{mcc00} represents forms like \emph{hoʔa-ŋe} `take-away'
with the structure in \qf{hoqa-nge}.
The stem of this form is in the U\=/form (a disyllable) with a monosyllabic suffix.
In both these diagrams `PrWd' stands for `prosodic word'.\footnote{
		\cite{mcc00} uses different examples to illustrate the prosodic
		structures of the U\=/form and M\=/form of words.
		I have selected these examples in order to illustrate
		clearly the difference between metathesised and unmetathesised forms.}

\begin{multicols}{2}
	\begin{exe}
			\ex{M\=/form and long suffix:}\label{hoaq-kia}
			\sn{\it{hoʔa} `take' + \it{-kia} `transitive'}
			\sna{\xy
				<0em,3cm>*\as{PrWd}="s1",<2.5em,3cm>*\as{PrWd}="s2",
				<0em,2cm>*\as{Ft}="c1",<2.5em,2cm>*\as{Ft}="c2",
				<-0.35em,1cm>*\as{}="g1",<0.35em,1cm>*\as{}="g2",<2.15em,1cm>*\as{}="g3",<2.85em,1cm>*\as{}="g4",
				<0em,1cm>*\as{hoaʔ}="p1",<2.5em,1cm>*\as{-kia}="p2",
				<0em,0cm>*\as{stem}="f1",<2.5em,0cm>*\as{sfx}="f2",
				"f1"+U;"p1"+D**\dir{-};"f2"+U;"p2"+D**\dir{-};
				"g1"+U;"c1"+D**\dir{-};"g2"+U;"c1"+D**\dir{-};"g3"+U;"c2"+D**\dir{-};"g4"+U;"c2"+D**\dir{-};
				"g1"+U;"g2"+U**\dir{-};"g3"+U;"g4"+U**\dir{-};
				"c1"+U;"s1"+D**\dir{-};"c2"+U;"s2"+D**\dir{-};
			\endxy}
			
			\ex{U\=/form and short suffix:}\label{hoqa-nge}
			\sn{\it{hoʔa} `take' + \it{-ŋe} `away'}
			\sna{\xy
				<1em,3cm>*\as{PrWd}="s1",
				<2em,2cm>*\as{Ft}="c2",
				<1.65em,1cm>*\as{}="g3",<2.35em,1cm>*\as{}="g4",
				<0em,1cm>*\as{hoʔa}="p1",<2em,1cm>*\as{-ŋe}="p2",
				<0em,0cm>*\as{stem}="f1",<2em,0cm>*\as{sfx}="f2",
				"f1"+U;"p1"+D**\dir{-};"f2"+U;"p2"+D**\dir{-};
				"g3"+U;"c2"+D**\dir{-};"g4"+U;"c2"+D**\dir{-};"g3"+U;"g4"+U**\dir{-};
				"p1"+U;"s1"+D**\dir{-};
				"c2"+U;"s1"+D**\dir{-};
			\endxy}
	\end{exe}
\end{multicols}

The constraint \tsc{Align-Head}-σ,
which requires stressed syllables to be word final,
is  crucial in \citeauthor{mcc00}'s analysis.
This constraint regulates prosodic words.
When a stem occurs with a long suffix, each prosodic word occurs in the M\=/form.
This means that both the stem and the
suffix/enclitic in \qf{hoaq-kia} would occur in the M\=/form.
When a stem occurs with a short suffix the whole prosodic word,
--- the stem combined with the suffix/enclitic ---
would occur in the M\=/form.

For Rotuman, the constraint \tsc{Align-Head}-σ
and the constraint \tsc{max} (prohibiting deletion)
are ranked more highly than the constraint \tsc{Linearity} (prohibiting metathesis).
Each of these three constraints is given in \qf{ex:ConRan} below.

\begin{exe}
	\ex{Constraint Ranking in \citet{mcc00}:}\label{ex:ConRan}
		\begin{xlist}
			\ex{\tsc{Align-Head}-σ: Align(H′(PrWd), R, PrWd, R)
			(The main-stressed syllable is final in every prosodic word)}\label{ex:Ali}
			\ex{\tsc{Max}: Every element of S\sub{1} has a correspondent in S\sub{2}
			(No deletion)}\label{ex:Max}
			\ex{\tsc{Linearity}: S\sub{1} is consistent with the precedence structure of S\sub{2} and vice versa
			(No metathesis)}\label{ex:Lin}
		\end{xlist}
\end{exe}

Metathesis results as it is ``the most faithful constraint mapping of a
/{\ldots}VCV/ input that still satisfies \tsc{Align-Head}-σ.''
\citet[174]{mcc00} gives an equivalent of the optimality tableau in \qf{ex:RotOT} below.

\begin{exe}
\ex{\rule{0pt}{0pt}{} \\[-3ex]
	\begin{tabular}{|rrl||c|c|c|} \hline
		\mc{3}{|c||}{Input:~/hoʔ\sub{1}a\sub{2}/} & \tsc{Align-Head}-σ & \tsc{max} & \tsc{Linearity} \\[0.5ex]
		\hline \hline a. & \hand & hoa\sub{2}ʔ\sub{1} & & & {\cgr}* \\
		\hline b. & & hoʔ\sub{1} & & *! & {\cgr} \\
		\hline c. & & hoʔ\sub{1}a\sub{2} & *! & {\cgr} & {\cgr} \\
		\hline \end{tabular}}\label{ex:RotOT}
\end{exe}

Under \citeauthor{mcc00}'s analysis Rotuman metathesis
(along with other processes which form the M\=/form)
occurs in order to create final stressed syllables
within the domain of the prosodic word.
I refer the reader to \cite{mcc00} for a full analysis
of the different phonological processes in Rotuman
and the ways in which these are handled.

\subsubsection{Kwara'ae}\label{sec:ProMorKwa}
The forms and functions of metathesis in Kwara'ae are summarised in \srf{sec:Kwa}.
In Kwara'ae the metathesised form of words is the form of words used in everyday normal speech
while the unmetathesised form is used in traditional songs,
for clarification, and when calling/yelling out.
Some examples of metathesis in Kwara'ae
are given in \qf{ex2:KwVCV->VVC} below.

\begin{exe}
	\ex{Metathesis in Kwara'ae \hfill\citep{he04}}\label{ex2:KwVCV->VVC}
	\sn{\begin{tabular}{rcll}
\rcl U\=/form				 &		& M\=/form				&  \\
		\it{ˈlo.\tbr{ʔi}} &\ra& \it{ˈlo\tbr{i̯ʔ}}	& `snake' \\
\rcl\it{ˈbu.\tbr{ri}} &\ra& \it{ˈbu̯\tbr{ir}}	& `behind' \\
		\it{ˈbo.\tbr{re}} &\ra& \it{ˈbo̯\tbr{er}}	& `although' \\
\rcl\it{ˈki.\tbr{ni}} &\ra& \it{ˈk\tbr{iˑn}} 	& `woman' \\
		\it{ˈde.\tbr{ŋe}} &\ra& \it{ˈd\tbr{ɛˑŋ}} 	& `shrimp' \\
\rcl\it{ˈke.\tbr{ta}.ˌla.\tbr{ku}} 	&\ra& \it{ˈke̯\tbr{at}.ˌla\tbr{u̯k}} & `my height' \\
		\it{da.ˈ\tbr{ro}.ʔa.ˌni.\tbr{da}}&\ra& \it{ˈda\tbr{o̯r}.ʔa.ˌni̯\tbr{ɛd}} & `to share them' \\
	\end{tabular}}
\end{exe}

According to the description in \citet{he04},
each Kwara'ae unmetathesised form
contains one or more stressed syllables with a single vowel
which are followed by an unstressed vowel.
Each metathesised form contains stressed syllables,
each of which is a heavy syllable containing either
a vowel and glide or a long vowel.

Based on this observation \cite{he04} analyses metathesis in Kwara'ae
as being conditioned by the placement of stress.
His analysis is framed under Optimality Theory
and is based on ranking the Stress to Weight Principle constraint (requiring
stressed syllables to be heavy) more highly than the constraint of Linearity,
which requires segments to occur in their underlying order.
Given an underlying form such as /salo/ `sky',
the metathesised form with a heavy syllable consisting of a vowel and glide
is selected rather than the unmetathesised form with two light syllables,
as shown in the optimality tableau in \qf{KwMetOT} below

\begin{exe}
		\ex{Metathesis in Kwara'ae \hfill \cite[53]{he04}}\label{KwMetOT}
	\sn{\begin{tabular}{|rrl||c|c|} \hline
		\mc{3}{|c||}{Input:~/salo/} & SWP & \tsc{Linearity} \\[0.5ex]
		\hline \hline a. & & ˈsao̯l & & {\cgr}* \\
		\hline b. & & ˈsa.lo & *! & {\cgr} \\
		\hline \end{tabular}
	}
\end{exe}

\subsubsection{Amarasi}
Although the analysis of \cite{mcc00} for Rotuman
and the analysis of \cite{he04} for Kwara'ae
are different in several details,
in each case metathesis is analysed as occurring in
order to create a monosyllabic form which bears stress.
Whatever the merits these analyses may have for Rotuman or Kwara'ae,
they cannot be extended to describe the Amarasi data.
This is because in most cases the stress
and number of syllables of a U\=/form and M\=/form are identical.

This is true of forms with different penultimate and final vowels
which undergo metathesis; such as \ve{hitu} [ˈhi.t̪ʊ] {\ra} \ve{hiut} [ˈhi.ʊt̪] `seven'.
It is true of words which undergo metathesis and deletion of a final consonant;
\ve{muʔit} [ˈmʊ.ʔit̪] {\ra} \ve{muiʔ} [ˈmʊ.iʔ] `animal'.
It is also true of words which undergo deletion of a final consonant
such as \ve{kuan} [ˈkʊ.ɐn] {\ra} \ve{kua} [ˈkʊ.ɐ] `village'.\footnote{
		It is also worth noting that while words which end in VV{\#},
		such as \ve{ai} [ʔa.i] `fire',
		do not have distinct M\=/forms and U\=/forms in Amarasi,
		the prosodic structure of such a word is identical
		when it occurs in either a U\=/form or M\=/form environment.}

The only cases in which such an analysis could be successfully
applied are those in which the M\=/form is, arguably, a reduced form.
This includes words in which the penultimate and final vowel
are identical, such as \ve{fini} [ˈfɪ.ni] {\ra} \ve{fiin} [ˈfiːn] `seed',
or those in which assimilation of /a/ occurs, such as \ve{nima} [ˈni.mɐ] {\ra} \ve{niim} [ˈniːm] `five'.
It also perhaps includes words with an initial phonetic diphthong,
such as \ve{nautus} [ˈnəw.t̪ʊs] {\ra} \ve{naut} [ˈnə.ʊt̪] `beetle'.

In cases in which the M\=/form
has a final sequence of two identical vowels,
such as \ve{fini} [ˈfɪ.ni] {\ra} \ve{fiin} [ˈfiːn] `seed',
there is a reduction in the number of phonetic syllables,
with a resultant change in stress from a light syllable to a heavy syllable.
In such instances alone an analysis along the lines of \cite{mcc00} or \cite{he04}
in which metathesis occurs to create a heavy stressed syllable could be proposed.
However, given the evidence for treating phonetically long vowels
as a sequence of two identical vowels (\srf{sec:DouVow}),
and the fact that Amarasi is otherwise not a weight sensitive language (\srf{sec:Str}),
this analysis is not consistent with other facts of the language.

Not only does such an analysis contradict other
facts of Amarasi, but it also cannot account for all the data,
as the M\=/form of words such as \ve{hitu} [ˈhit̪ʊ] {\ra} \ve{hiut} [ˈhi.ʊt̪] `seven'
simply do not contain a stressed heavy syllable.
Therefore, neither of the analyses proposed by \citet{mcc00} or \citet{he04}
can explain all the Amarasi data.
Furthermore, it is not at all clear that \emph{any} analysis
framed within prosodic morphology could account for the Amarasi data.

One approach, broadly in line with the 
notion that consonant-vowel metathesis is a reduction strategy,
would be to propose that the M\=/forms occur in order
to create a phonetically shorter form.
An M\=/form with a vowel sequence such as \ve{hiut} [ˈhi.ʊt̪] `seven'
is usually, though not always, phonetically shorter than the U\=/form \ve{hitu} [ˈhit̪ʊ]
in which the two vowels are separated by a consonant.\footnote{
		That the total length of a vowel sequence is usually shorter
		than the combined length of two vowels separated by a consonant
		is confirmed by an instrumental phonetic study.
		I marked vowels in Praat in three recorded texts from a single speaker
		and extracted their lengths with a script.
		(Words with a distinctive pause intonation were discarded.)
		This resulted in 255 measurements of both the vowels in forms such as \ve{hitu} `seven'
		(that is, the measurements of 510 vowels with the length
		of the penultimate and final vowels summed)
		and 243 lengths of vowel sequences in M\=/forms such as \ve{hiut} `seven'.
		The average length of two vowels separated by a consonant was 0.17 seconds
		and the average length of a vowel sequence was 0.13 seconds.
		A two tailed t-test showed that this difference was statistically significant (p > 0.001).
		The lengths of 385 vowel sequences in U\=/forms such as \ve{kuan} `village'
		were also extracted and had an average duration of 0.12 seconds.
		A two tailed t-test showed that the difference between vowel sequences
		in M\=/forms and U\=/forms is probably not statistically significant (p = 0.11).}
Apart from the fact that the phonetic length of a word in Amarasi
is primarily determined by speech speed, sentence stress, and pragmatics
rather than the phonotactic shape of a word,
this approach would leave unexplained forms in which the U\=/form
and M\=/form both have a vowel sequence, such as \ve{kuan} [ˈkʊ.ɐn] {\ra} \ve{kua} [ˈkʊ.ɐ] `village'.

Another possibility would be to propose M\=/forms
occur in order to create consonant-final forms.
Such a proposal does not explain why VVC{\#} U\=/forms
such as  \ve{kuan} [ˈkʊ.ɐn] {\ra} \ve{kua} [ˈkʊ.ɐ] `village' delete their final consonant
to form an M\=/form, nor why already consonant-final U\=/forms
such as  [ˈmʊ.ʔit̪] {\ra} \ve{muiʔ} [ˈmʊ.iʔ] `animal'
undergo metathesis in the M\=/form.

The diversity in the surface prosodic structures of M\=/forms in Amarasi
confounds an analysis framed within prosodic morphology.
While in Rotuman there is a comparable diversity in forms,
nearly all instances of the M\=/form
have been analysed as bi-moraic monosyllables,
thus an analysis framed within prosodic morphology is successful.

A prosodic morphology approach to the Amarasi data encounters
serious challenges which cannot obviously be overcome.
Instead, by adopting a process based model of morphology
and an obligatory CVCVC foot structure for U\=/forms,
the different processes in the formation of the Amarasi M\=/form
can be explained with a single process of metathesis
and a single morphemically conditioned rule of /a/ assimilation.
This analysis accounts for Amarasi M\=/forms
in a simple, consistent, unified way.

Finally, regarding Optimality Theory, which is
utilised in the analyses of \cite{mcc00} and \cite{he04},
the high level of opacity/ambiguity in the derivation of M\=/forms
--- including at least one derived environment effect (\srf{sec:AssOfA}) ---
indicates that standard Optimality Theory would not fare particularly well in Amarasi.
This is one of the reasons I have not employed it in this book.

\subsection{Phonologically conditioned metathesis}\label{sec:AltAppPhoMet}
Another possible approach to the Amarasi data would
be to analyse metathesis as phonologically conditioned.
This is, in fact, part of the analysis of Rotuman
given by \citet[168]{mcc00} who explicitly rejects
the idea that there is a metathesis morpheme and states:
``[\ldots] earlier accounts of the phase [U\=/form/M\=/form] difference are inadequate empirically,
since it does not make sense to talk of a phase morpheme or template.''\footnote{
		That \cite{mcc00} explicitly rejects the notion of a ``phase morpheme'' is surprising
		given that \cite{ch40} gives many examples in which the use of the
		U\=/form or M\=/form is the only marker of a semantic difference,
		as summarised in \srf{sec:RotFun}.
		\cite{mcc00} does not offer any analysis of
		such uses of the U\=/form and M\=/form.}

Another language with synchronic consonant-vowel metathesis
which has been analysed as phonologically conditioned
is Luang in the Timor region \citep{tata15}.
For Luang, \citet[24]{tata15} propose that metathesis
is one of several phonological processes which operates to join
adjacent words into a single rhythm unit with only one stressed syllable.

Whatever the case may be for Rotuman or Luang,
metathesis in Amarasi cannot be reduced
to a phonologically conditioned process.
As discussed in Chapters \ref{ch:SynMet} and \ref{ch:DisMet},
metathesis in Amarasi is the only phonological marking
of certain syntactic and/or discourse structures.

Amarasi nouns followed by cardinal and ordinal numerals
provide the clearest demonstration that Amarasi metathesis
is not phonologically conditioned.
When followed by a cardinal number, nouns occur in the U\=/form.
However, when followed by an ordinal number, nouns occur in the M\=/form.
Examples are given in Table \ref{tab:AmaNouNum1},
which shows the noun \ve{neno} `day' followed by cardinal and ordinal numbers.\footnote{
		The ordinal numbers in \ref{tab:AmaNouNum1}
		are those used for counting days and months,
		and are derived from the cardinal numbers
		with the addition of a glottal stop as a suffix or infix.}

\begin{table}[ht]
	\caption{Amarasi nouns and numerals}\label{tab:AmaNouNum1}
	\centering
		\begin{threeparttable}[b]
		\begin{tabular}{llll} \lsptoprule
	Underlying	&Phonetic	&	& gloss\\\midrule
	\ve{ne\tbr{no} meseʔ}	&[ˌnɛnɔˈmɛsɛʔ]	&\emb{neno-meseq.mp3}{\spk{}}{\apl}	&`a single day'\su{†}\\
	\ve{ne\tbr{on} meseʔ}	&[ˌnɛ.ɔnˈmɛsɛʔ]	&\emb{neon-meseq.mp3}{\spk{}}{\apl}	&`first day (Monday)'\su{‡}	\\
	\ve{ne\tbr{no} nua}	&[ˌnɛnɔˈnʊ.ɐ]	&\emb{neno-nua.mp3}{\spk{}}{\apl}	&`two days'	\\
	\ve{ne\tbr{on} nua-ʔ}	&[ˌnɛ.ɔˈnːʊ.ɐʔ]	&\emb{neon-nuaq.mp3}{\spk{}}{\apl}	&`second day (Tuesday)'	\\
	\ve{ne\tbr{no} teun}\su{\#}&[ˌnɛnɔˈt̪ɛ.ʊn]	&\emb{neno-teun.mp3}{\spk{}}{\apl}	&`three days' \\
	\ve{ne\tbr{on} tenu-ʔ}	&[ˌnɛ.ɔn̪ˈt̪ɛnʊʔ]	&\emb{neon-tenuq.mp3}{\spk{}}{\apl}	&`third day (Wednesday)'	\\
	\ve{ne\tbr{no} haa}	&[ˌnɛnɔˈhaː]	&\emb{neno-haa.mp3}{\spk{}}{\apl}	&`four days'	\\
	\ve{ne\tbr{on} haa-ʔ}	&[ˌnɛ.ɔnˈhaˑʔ]	&\emb{neon-haaq.mp3}{\spk{}}{\apl}	&`fourth day (Thursday)'	\\
	\ve{ne\tbr{no} niim}	&[ˌnɛnɔˈniˑm]	&\emb{neno-niim.mp3}{\spk{}}{\apl}	&`five days'	\\
	\ve{ne\tbr{on} nima-ʔ}	&[ˌnɛ.ɔˈnːimɐʔ]	&\emb{neon-nimaq.mp3}{\spk{}}{\apl}	&`fifth day (Friday)'	\\
	\ve{ne\tbr{no} nee}	&[ˌnɛnɔˈnɛː]	&\emb{neno-nee.mp3}{\spk{}}{\apl}	&`six days'	\\
	\ve{ne\tbr{on} ne\<ʔ\>e}	&[ˌnɛ.ɔˈnːɛʔɛ]	&\emb{neon-neqe.mp3}{\spk{}}{\apl}	&`sixth day (Saturday)'	\\
		\lspbottomrule
		\end{tabular}
			\begin{tablenotes}
				\item [†]
					The phrase \it{neno meseʔ} has the sense of `a single day'.
					The normal phrase for `one day' would be \ve{neeŋgw=ees},
					from \ve{neno} + \ve{=ees}.
				\item [‡]
					The normal reference for the phrases with ordinal numbers
					is to the days of the week, with \ve{neon meseʔ} `first day'
					being Monday and \ve{neon ne\<ʔ\>e} `sixth day' being Saturday.
					The normal phrase for Sunday is \ve{neno krei} `day + church'.
					The phrase \ve{neon hitu-ʔ} `seventh day' is attested twice
					in the Amarasi Bible translation in Genesis 2.
					This shows that the phrases with ordinal numbers
					are not just completely lexicalised phrases.
				\item [\#] The default form for cardinal numerals is the M\=/form.
			\end{tablenotes}
		\end{threeparttable}
\end{table}

There is no phonetic difference between each kind of phrase,
with the exception of the metathesis of the noun
and, where applicable, the addition of the glottal stop forming ordinal numbers.
In every phrase the noun has two syllables
and stress falls on the penultimate vowel of the numeral.
Compare especially the phrase \ve{neno meseʔ} `a single day'
with that of \ve{neon meseʔ} `first day (Monday)',
in which the only difference is in the metathesis of the noun.

Instead, the M\=/form marks that these phrases have different syntactic structures.
Ordinal numbers occur within the noun phrase
while cardinal numbers occur outside of the noun phrase
as the head of a number phrase.
The different syntactic structures of the phrases
\ve{neno meseʔ} and \ve{neon meseʔ}
are shown in \qf{ex2:NenoMeseq} and \qf{ex2:NeonMeseq} below.
See Chapter \ref{ch:SynMet}, especially \srf{sec:OrdNum}, for full details.

%\newpage
\begin{multicols}{2}
	\begin{exe} \let\eachwordone=\textnormal \let\eachwordtwo=\ve
		\ex{\glll {}				ˌnɛnɔ {} {} ˈmɛsɛʔ	{} \\
							\brac{NP} ne\tbr{no} \bracr{} \brac{Num} meseʔ \bracr{}\\
							{} day{{\textbackslash}\tbr{\tsc{u}}} {} {} one{{\textbackslash}\tsc{u}} {}\\
				\glt \lh{\brac{NP}} `a single day' }\label{ex2:NenoMeseq}
		\ex{\glll	{} ˌnɛ.ɔn ˈmɛsɛʔ {} \\
							\brac{NP} ne\tbr{on} meseʔ \bracr{}\\
							{} day{{\textbackslash}\tbr{\tsc{m}}} one{{\textbackslash}\tsc{u}} {}\\
				\glt \lh{\brac{NP}}`first day (i.e. Monday)' }\label{ex2:NeonMeseq}
	\end{exe}
\end{multicols}

Such data rule out an analysis of Amarasi metathesis
as phonologically conditioned,
unless we posit that different
syntactic structures are associated with different abstract
phonological structures with no phonological realisation.

\subsubsection{Metathesis conditioned by intonation}\label{sec:MetConInt}
Based on preliminary data on a north-eastern variety of Meto, \cite{mccko96}
raise the possibility that metathesis in Meto could be prosodically conditioned.\footnote{
		Thanks go to Patrick McConvell for providing me with his unpublished notes.
		Based on this material, the variety described appears to be Miomafo, Insana, or Beboki.}
They noted that they had two examples of a U\=/form verb with falling or low pitch,
and two examples of an M\=/form verb with rising or high pitch.

To test the hypothesis that metathesis in Amarasi could be conditioned by intonation,
I took a random selection of 80 U\=/form verbs and 80 M\=/form verbs
in different sentence positions from a number of natural texts.
The pitch of each verb was recorded as either rise, fall, high, mid, or low.
The results are summarised in \trf{tab:VerMetInt}.

\begin{table}[h]
	\caption{Verbal metathesis and intonation}\label{tab:VerMetInt}
	\centering
		\begin{tabular}{rll}\lsptoprule
						& U\=/form	& M\=/form	\\ \midrule
			rise	& 10			& 9				\\
			high	& 14			& 15			\\
			mid		& 14			& 5				\\
			low		& 2				& 10			\\
			fall	&	40			& 41			\\
			\lspbottomrule
		\end{tabular}
\end{table}

\trf{tab:VerMetInt} shows that the pitch of both U\=/form
and M\=/form verbs is very similar.
About half of both U\=/forms and M\=/forms have a falling pitch
and about a quarter have a rising or high pitch.
The only difference is in the frequency  of mid pitch and low pitch,
with M\=/forms occurring with a low pitch more frequently than U\=/forms
--- the opposite to what would be predicted by \citeauthor{mccko96}'s preliminary hypothesis.
%This difference could be a result of the difficulty in consistently
%distinguishing between these two pitches in running text.

\subsection{Affixation of consonant-vowel melody}\label{sec:AffCVMel}
A final possible analysis of Amarasi metathesis
would be to analyse it under an item and arrangement model of morphology
in which the consonant-vowel template itself is a kind of affix
which combines with to the segments of a word.
This is the analysis proposed by \citet[160f]{st94} for Rotuman metathesis
and would be similar to the analysis of Arabic morphology in \cite{mcc81}.

Under such an analysis,
each consonant of an Amarasi word would be ordered with respect to each other consonant
and each vowel would be ordered with respect to each other vowel,
but consonants and vowels would not be ordered with respect to one another.
An Amarasi word such as \ve{fatu {\tl} faut} `stone'
could then be represented as  either /ft,au/ or /au,ft/.
This segmental information then combines with the appropriate consonant-vowel melody.
This is shown in \qf{ex:fatu/faut} below
which makes explicit the concatenative nature of this analysis,
and in \qf{as:fatu/faut} with autosegmental notation.
Examples \qf{ex:ft,au+CVCV} and \qf{as:fatu} show U\=/forms
and examples \qf{ex:ft,au+CVVC} and \qf{as:faut} show M\=/forms.

\begin{multicols}{2}
	\begin{exe}
		\ex{\begin{xlist}
			\ex{/ft,au/ + CVCV {\ra} \ve{fatu}}\label{ex:ft,au+CVCV}
			\ex{/ft,au/ + CVVC {\ra} \ve{faut}}\label{ex:ft,au+CVVC}
		\end{xlist}}\label{ex:fatu/faut}
	\end{exe}
\end{multicols}
\begin{multicols}{2}
	\begin{exe}
		\ex{\begin{xlist}
			\exa{\xy
				<0em,2cm>*\as{f}="c1",<2em,2cm>*\as{t}="c2",
				<0em,1cm>*\as{C}="C1",<2em,1cm>*\as{C}="C2",<1em,1cm>*\as{V}="V1",<3em,1cm>*\as{V}="V2",
				<1em,0cm>*\as{a}="v1",<3em,0cm>*\as{u}="v2",
				"C1"+U;"c1"+D**\dir{-};"C2"+U;"c2"+D**\dir{-};"v1"+U;"V1"+D**\dir{-};"v2"+U;"V2"+D**\dir{-};
			\endxy}\label{as:fatu}
			\exa{\xy
				<0em,2cm>*\as{f}="c1",<3em,2cm>*\as{t}="c2",
				<0em,1cm>*\as{C}="C1",<3em,1cm>*\as{C}="C2",<1em,1cm>*\as{V}="V1",<2em,1cm>*\as{V}="V2",
				<1em,0cm>*\as{a}="v1",<2em,0cm>*\as{u}="v2",
				"C1"+U;"c1"+D**\dir{-};"C2"+U;"c2"+D**\dir{-};"v1"+U;"V1"+D**\dir{-};"v2"+U;"V2"+D**\dir{-};
			\endxy}\label{as:faut}
		\end{xlist}}\label{as:fatu/faut}
	\end{exe}
\end{multicols}

Analysing Amarasi metathesis as affixation
with different consonant-vowel melodies is possible.
Under such an analysis,
the selection of the appropriate melody
would be determined by morphosyntactic criteria.
While a concatenative analysis accurately describes the data,
there are two ways in which the process-based
analysis adopted in this chapter better fits the Amarasi data.

Firstly, the affixal analysis misses the generalisation
that the M\=/form is always derivable from the U\=/form
by reversal of the final CV sequence.
While morphological consonant-vowel metathesis is rare,
it \emph{is} cross-linguitically attested
(see Chapter \ref{ch:SynchMet}).
Under the concatenative analysis it is not
immediately clear why the derived M\=/form
does not involve other kinds of metathesis,
or other arbitrary substitutions.
Under the concatenative analysis
there is no principled reason that the M\=/form
should not be, for instance, VCVV yielding \ve{fatu} `stone' {\ra} \ve{*afuu}.

Secondly -- and closely linked to the first reason --
the affixal approach to metathesis
misses the cross-linguistic generalisation discussed
in \srf{sec:OriMorMet} and \srf{sec:For ch:SynchMet}
that processes of compensatory metathesis (see further \srf{sec:OriMetAma} below)
are located adjacent to a stressed syllable.
The placement of stress plays no role
in the derivation of M\=/forms under the affixal analysis.
The rule based approach, on the other hand,
achieves this straightforwardly by including the stressed
syllable as the constraining environment after which metathesis occurs.
\section{Origins of Amarasi metathesis}\label{sec:OriMetAma}
As discussed in \srf{sec:OriMorMet}, \cite{blga98,blga04} propose
a number of ways in which a language can acquire a synchronic process
of metathesis through a number of phonetically natural steps.
Under their account the kind of metathesis seen
in Amarasi is compensatory metathesis,
which arose originally in certain prosodically conditioned environments:

\begin{quote}
``Compensatory metatheses originate when VCV sequences
are pronounced with extreme coarticulation of one vowel,
resulting in a seepage or shift of that vowel to the other side of the medial consonant.
This extreme form of coarticulation occurs in syllables which are already long due to stress.
The peripheral unstressed vowel, whose cues are now primarily on the opposite side of the consonant,
withers into a reduced form, and is ultimately lost.
The migration of the peripheral vowel across the intervening
consonant into tonic position is complete.'' \hfill\citep[529]{blga98}
\end{quote}

Under this account a noun such as \ve{fatu} `stone' goes 
through a process like that illustrated in \qf{ex:fautu} below.
This process would only occur in certain prosodic environments,
with the end result that the forms \ve{fatu} and \ve{faut}
are found in different phonological environments.

\begin{exe}
	\ex{\ve{fatu} > *fautu > *faut\u{u} > \ve{faut}}\label{ex:fautu}
\end{exe}

In Kotos Amarasi only the first stage (\ve{fatu})
and the final stage (\ve{faut}) are attested.
If this is indeed the process that gave rise to metathesis in Amarasi,
we would expect to find data attesting the hypothesised medial stages.
Indeed, data from other varieties of Meto attests these stages.

\subsection{First intermediate stage}
Intermediate stages with an initial diphthong and final full vowel
(i.e. *fautu) are found in Ro{\Q}is Amarasi.
As discussed in \srf{sec:RoqAmaDip},
in Ro{\Q}is the U\=/form of certain consonant final roots
show spread of the final vowel to the first syllable.
The examples from \prf{tab:RoqStrVSloDip}
are repeated in \trf{tab:RoqStrVSloDip2},
which shows Kotos Amarasi U\=/forms, Ro{\Q}is Amarasi U\=/forms
and M\=/forms before enclitics (Chapter \ref{ch:PhoMet}).
%See \srf{sec:SurVVCVWor} for more discussion of such forms.

\begin{table}[ht]
	\centering\caption{Ro{\Q}is Amarasi diphthongisation}\label{tab:RoqStrVSloDip2}
	\begin{threeparttable}[b]
		\begin{tabular}{llll}\lsptoprule
				Kotos 								& Ro{\Q}is									&	Kotos/Ro{\Q}is				&		\\
				U\=/form								&	U\=/form										& M̿-form								&	gloss	\\ \midrule
				\ve{t\tbr{e}fis}			&	\ve{t\tbrtb{e}{\i}fik}		&	\ve{t\tbr{ei}fs/k=}		&	`roof'	\\
				\ve{m\tbr{a}sik}			&	\ve{m\tbrtb{a}{\i}sik}		&	\ve{m\tbr{ai}sk=}			&	`salt'	\\
				\ve{t\tbr{o}ʔis}			&	\ve{t\tbrtb{o}{\i}ʔis}		&	\ve{t\tbr{oi}ʔs=}			&	`trumpet'	\\
				\ve{h\tbr{u}nik}			&	\ve{h\tbrtb{u}{\i}nik}		&	\ve{h\tbr{ui}nk=}			&	`turmeric'	\\
				\ve{\tbr{a}net}				&	\ve{\tbrtb{a}{e}net}			&	\ve{\tbr{ae}nt=}			&	`needle'	\\
				\ve{r\tbr{o}ne-f}			&	\ve{r\tbrtb{o}{e}ne-f}		&	\ve{r\tbr{oe}n-f=}		&	`brain'	\\
				\ve{niis \tbr{e}no-f}	&	\ve{niis \tbrtb{e}{o}no-f}&	\ve{niis \tbr{eo}n-f=}&	`incisors'	\\
				\ve{n-ʔ\tbr{a}tor}		&	\ve{n-ʔ\tbrtb{a}{o}tor}		&	\ve{n-ʔ\tbr{ao}tr=}		&	`arrange'	\\
				\ve{s\tbr{i}ʔu-f}			&	\ve{s\tbrtb{\i}{u}ʔu-f}		&	\ve{s\tbr{iu}ʔ-f=}		&	`elbow'	\\
				\ve{\tbr{e}suk}				&	\ve{\tbrtb{e}{u}suk}			&	\ve{\tbr{eu}sk=}			&	`mortar'	\\
				\ve{m\tbr{a}nus}			&	\ve{m\tbrtb{a}{u}nus}			&	\ve{m\tbr{au}ns=}			&	`beetle vine'	\\
				\ve{p\tbr{o}nu-f}			&	\ve{p\tbrtb{o}{u}nu-f}		&	\ve{p\tbr{ou}n-f=}		&	`moustache'\su{†}	\\
			\lspbottomrule
		\end{tabular}
			\begin{tablenotes}
				\item [†] Kotos \ve{ponu-f} is `moustache'
									and Ro{\Q}is \ve{po͡unu-f} is `body hair'.
			\end{tablenotes}
		\end{threeparttable}
\end{table}

This diphthongisation is productive and, as discussed in \srf{sec:RoqAnaCCIniMod},
either this diphthongisation or complete metathesis
occurs in Ro{\Q}is Amarasi before modifiers which
begin with a consonant cluster.
Thus, the complete process of compensatory metathesis as
hypothesised by \citet{blga98} is attested
in a phrase such as \ve{umi} `house' + \ve{kbubuʔ} `round'.
Kotos Amarasi \ve{umi kbubuʔ} attests the first
stage, while Ro{\Q}is Amarasi \ve{u͡{\i}mi kbubuʔ} and \ve{uim kbubuʔ}
attest the intermediate and final stages.

Additionally, there is one instance in which diphthongisation
is attested before a modifier with a single consonant in my
corpus of Ro{\Q}is Amarasi texts.
This is the phrase \ve{rasi} `matter' + \ve{matsao-s} `marriage'
which occurs as \ve{ra͡{\i}si matsaos} `marriage arrangements' four
times in one of my Ro{\Q}is texts.
One of these examples is given in \qf{ex:RO-170830-1-06-54} below.

\begin{exe}
	\ex{\glll	\sf{{\j}adi} na-ʔuabaʔ r\tbrtb{a}{\i}\tbr{si} matsao-s=ii\j=ii\\
						\sf{{\j}adi} na-ʔuabaʔ r\tbr{asi} matsao-s=ii=ii\\
						so \na-speak matter marry-{\at}={\ii}={\reqt}\\
			\glt	`So they talk about the marriage arrangements'\txrf{RO-170830-1, 6.54}
						{\emb{RO-170830-1-06-54.mp3}{\spk{}}{\apl}}}\label{ex:RO-170830-1-06-54}
\end{exe}

Ro{\Q}is Amarasi attests spread of the final
vowel to the penultimate position creating a medial
diphthong. This is the first intermediate
stage which can give rise to synchronic metathesis.

\subsection{Second intermediate stage}
The second intermediate stage showing forms with
a reduced final vowel (i.e. *faut\u{u}),
is found in some varieties of South Amanuban,
Timaus, Fatule{\Q}u, Amfo{\Q}an and Kopas.
My discussion here focusses on the variety of
South Amanuban spoken in Se{\Q}i village.\footnote{
		In known varieties of Fatule{\Q}u and Amfo{\Q}an
		with such vowel reduction, it only affects words with a final
		back vowel /o/ or /u/. This is also the case
		for Timaus spoken in Sanenu.}

In Se{\Q}i Amanuban verbs and numerals are usually
cited in a form with a medial double vowel
corresponding to the penultimate vowel of the root
and a final non-syllabic or voiceless vowel
corresponding to the root final vowel.
A simple example is \ve{tenu} {\ra} \ve{teenu̯} `three'
More examples are given in \trf{tab:SouAmaMetFor} on the next page.
Such words also have metathesised forms,
such as \ve{tenu} {\ra} \ve{teun} `three'.
The only words in my Se{\Q}i Amanuban data which do not have
forms with a final non-syllabic vowel are those with final /a/
and a penultimate vowel other than /a/ such as \ve{nima} {\ra} \ve{niim} `five'.

\begin{table}[ht]
	\caption[Se{\Q}i Amanuban Citation Forms]
					{Se{\Q}i Amanuban Citation Forms\su{†}}\label{tab:SouAmaMetFor}
	\centering
		\begin{threeparttable}
			\begin{tabular}{llllll}	\lsptoprule
					&Se{\Q}i 	&							&							&Amarasi	&\\
		Root	&citation	&Phonetic			&							&M\=/form		&gloss\\ \midrule
	\ve{{\rt}mani}	&\ve{n-maani̯}		&[ˈn͡maˑnj]		&\emb{NB-Sei-nmaani.mp3}{\spk{}}{\apl}		&\ve{n-main}	&`laugh'\\
%	&&[naˑˈmnaˑsj̊]&\emb{NB-Sei-namnaasi.mp3}{\spk{}}{\apl}&&\\
%	\multirow{-2}{*}{\ve{{\rt}mnasi}}	&\multirow{-2}{*}{\ve{na-mnaasi̯}}	&[naˈmna̟ˑçː]
%	&\emb{NB-Sei-namnaasi2.mp3}{\spk{}}{\apl}	&\multirow{-2}{*}{\ve{na-mnais}}&\multirow{-2}{*}{`old'}\\
	\ve{{\rt}mnasi}	&\ve{na-mnaasi̯}	&[naˑˈmnaˑsj̊]	&\emb{NB-Sei-namnaasi.mp3}{\spk{}}{\apl}	&\ve{na-mnais}&`old'\\
									&								&[naˈmna̟ˑçː]	&\emb{NB-Sei-namnaasi2.mp3}{\spk{}}{\apl}	&&\\
	\ve{{\rt}honi}	&\ve{na-hooni̯}	&[naˈhoˑnj]		&\emb{NB-Sei-nahooni.mp3}{\spk{}}{\apl}		&\ve{na-hoin}	&`be born'\\
	\ve{{\rt}luli}	&\ve{n-luuli̯}		&[ˈnlʊːlʝ̥]		&\emb{NB-Sei-nluuli.mp3}{\spk{}}{\apl}		&\ve{}				&`burn'\\
	\ve{{\rt}hake}	&\ve{n-haake̯}		&[ˈnhaˑkɜ̥̆] 		&\emb{NB-Sei-nhaake.mp3}{\spk{}}{\apl}		&\ve{n-haek}	&`stand'\\
	\ve{{\rt}mate}	&\ve{n-maate̯}		&[ˈn͡maːt̪ə̥]		&\emb{NB-Sei-nmaate.mp3}{\spk{}}{\apl}		&\ve{n-maet}	&`die'\\
	\ve{{\rt}lole}	&\ve{na-loole̯}	&[naˈlɔˑlɜ̯̆]		&\emb{NB-Sei-naloole.mp3}{\spk{}}{\apl}		&\ve{}				&`far'\\
	\ve{{\rt}loʔe}	&\ve{t-looʔe̯}		&[ˈt̪lɔˑʔɛ̥̆]		&\emb{NB-Sei-tlooqe.mp3}{\spk{}}{\apl}		&\ve{}				&`swim'\\
	\ve{{\rt}paumaka}&\ve{paumaaka̯}	&[ˌpɐwˈmaˑkɐ̥̆]	&\emb{NB-Sei-paumaaka.mp3}{\spk{}}{\apl}	&\ve{n-paumaak}	&`near'\\
	\ve{{\rt}mahata}&\ve{n-mahaata̯}	&[ˈn͡maˑhaːt̪ə̥]	&\emb{NB-Sei-nmahaata.mp3}{\spk{}}{\apl}	&\ve{n-mahaat}&`itchy'\\
	\ve{{\rt}kiso}	&\ve{n-kiiso̯}		&[ˈnkiˑsw̥]		&\emb{NB-Sei-nkiiso.mp3}{\spk{}}{\apl}		&\ve{n-kius}	&`see'\\
	\ve{{\rt}tselo}	&\ve{na-tseelo̯}	&[naːˈt̪sɛlɔ̆]	&\emb{NB-Sei-natseelo.mp3}{\spk{}}{\apl}	&\ve{}				&`fall'\\
	\ve{{\rt}meno}	&\ve{n-meeno̯}		&[ˈn͡mɛːnɔ̯]		&\emb{NB-Sei-nmeeno.mp3}{\spk{}}{\apl}		&\ve{n-meon}	&`thirsty'\\
	\ve{{\rt}meto}	&\ve{n-meeto̯}		&[ˈnmɛˑt̪ɔ̯̊]		&\emb{NB-Sei-nmeeto.mp3}{\spk{}}{\apl}		&\ve{n-meot}	&`be dry'\\
	\ve{{\rt}nano}	&\ve{na-naano̯}	&[nɐˈnaˑnɔ̯]		&\emb{NB-Sei-nanaano.mp3}{\spk{}}{\apl}		&\ve{na-kaon}	&`braid'\\
	\ve{{\rt}toko}	&\ve{t-tooko̯}		&[ˈt̪ɔˑkw̥]			&\emb{NB-Sei-ttooko.mp3}{\spk{}}{\apl}		&\ve{t-took}	&`sit'\\
	\ve{{\rt}hitu}	&\ve{hiitu̯}			&[ˈhɪːt̪w̥]			&\emb{NB-Sei-hiitu.mp3}{\spk{}}{\apl}			&\ve{hiut}		&`seven'\\
	\ve{{\rt}inu}		&\ve{t-iinu̯}		&[ˈt̪ɪˑnw]			&\emb{NB-Sei-tiinu.mp3}{\spk{}}{\apl}			&\ve{t-iun}		&`drink'\\
	\ve{{\rt}matleʔu}	&\ve{n-matleeʔu̯}	&[ˈn͡mat̪l̥eːʔw]	&\emb{NB-Sei-nmatleequ.mp3}{\spk{}}{\apl}	&\ve{}		&`dream'\\
	\ve{{\rt}tenu}	&\ve{teenu̯}			&[ˈt̪eːnw]			&\emb{NB-Sei-teenu.mp3}{\spk{}}{\apl}			&\ve{teun}		&`three'\\
	\ve{{\rt}fanu}	&\ve{faanu̯}			&[ˈfaːnw]			&\emb{NB-Sei-faanu.mp3}{\spk{}}{\apl}			&\ve{faun}		&`eight'\\
	\ve{{\rt}ʔapu}	&\ve{na-ʔaapu̯}	&[naˈʔaːpw̥]		&\emb{NB-Sei-naqaapu.mp3}{\spk{}}{\apl}		&							&`pregnant'\\
	\ve{{\rt}mofu}	&\ve{n-moofu̯}		&[ˈn͡moˑfw]		&\emb{NB-Sei-nmoofu.mp3}{\spk{}}{\apl}		&\ve{n-mouf}	&`fall'\\
%	\ve{{\rt}}	&\ve{}	&[ˈ]	&\emb{NB-Sei-.mp3}{\spk{}}{\apl}		&\ve{}	&`'\\
					\lspbottomrule
				\end{tabular}
			\begin{tablenotes}
				\item [†] Words were elicited from a group of three speakers
									and several sound files have multiple speakers
									giving the word at the same time.
			\end{tablenotes}
		\end{threeparttable}
\end{table}

Phonetically, such final non-syllabic vowels are usually realised by the organs
of the mouth taking the position for the articulation of the root final vowel,
but without any subsequent vibration of the vocal cords.
When the final consonant is a voiceless plosive there is also a subsequent puff of air.
After other consonants there is not usually any additional sound or air expelled.
%In the case of final /o/ or /u/, it is visually quite clear that speakers round their lips
%after the root final consonant.

Final non-syllabic vowels were judged by my Se{\Q}i consultants
to be different from normal vowels, and forms
with a syllabic vowel were interpreted as U\=/forms.
In the case of \ve{na-naano̯} `braid' [nɐˈnaˑnɔ̯]
\emb{NB-Sei-nanaano.mp3}{\spk{}}{\apl} one of my consultants stated,
``There's clearly an \emph{o} but it doesn't leave [the mouth].''
\it{(kentara \emph{o}, tapi tidak keluar)}
and regarding \ve{na-maani̯} `laugh' [ˈn͡maˑnj] \emb{NB-Sei-nmaani.mp3}{\spk{}}{\apl}
they stated ``It's like there is an \emph{i}
at the end, but the \emph{i} is lost.'' \it{(ke ada \emph{i}
di belakang, tapi \emph{i}-nya hilang})

Based on current textual data it appears that the
Se{\Q}i Amanuban forms are an additional M\=/form
which might only be used phrase finally while normal M\=/forms with
metathesis are used phrase medially.\footnote{
		This is not to say that the use of each
		form in Se{\Q}i Amanuban is purely conditioned
		by phrase position. Instead, when an M\=/form is
		grammatically appropriate the selection of M\=/form
		might be determined by phrase position.}
However, a more comprehensive investigation of Se{\Q}i Amanuban
is needed to properly determine how U\=/forms and
different M\=/forms are used in this variety of Meto.
An example of each kind of M\=/form is given in
\qf{ex:NB-171026-4, 0.56} which shows a
normal M\=/form of \ve{{\rt}honi} `give birth'
medially and an M\=/form with non-syllabic vowel
phrase finally.

\begin{exe}
	%\ex{\glll	amaʔ =ma au enaʔ na-ho\tbr{in} =kau\\
	%					amaʔ =ma au enaʔ na-ho\tbr{ni} =kau\\
	%					father =and {\au} mother \na-born =kau\\
	%		\glt	`My father and mother gave birth to me' \txrf{NB-171026-4, 0.28}
	%					{\emb{NB-171026-4-00-28.mp3}{\spk{}}{\apl}}}\label{ex:NB-171026-4, 0.28}
	\ex{\glll	n-ak: ``ena hoo mu-ho\tbr{in}'' n-ak: ``au u-ho\tbr{oni̯}''\\
						n-ak \hp{``}ena hoo mu-ho\tbr{ni} n-ak \hp{``}au u-ho\tbr{ni}\\
						\n-ay \hp{``}mother {\hoo} \muu-born{\tbrM} \n-say \hp{``}{\au} \qu-born{\tbrM}\\
			\glt	`He said ``Mother, have you given birth?'', she said ``I've given birth''.'
						\\ \txrf{NB-171026-4, 0.56}
						{\emb{NB-171026-4-00-56.mp3}{\spk{}}{\apl}}}\label{ex:NB-171026-4, 0.56}
\end{exe}

In Timaus from Sanenu verbs and numerals with final /o/
and /u/ also have M\=/forms with a final non-syllabic vowel.
In Timaus these are the M\=/forms used in all phrase positions
and words which take such M\=/forms have
not been attested with a normal M\=/form derived by simple metathesis.
Three examples of such Timaus M\=/forms are
given in \qf{ex:FGT-171013-1, 0.31} and
\qf{ex:FGT-171016-2, 2.05} below.
The second instance in \qf{ex:FGT-171013-1, 0.31} is phrase
final while the other two instances in \qf{ex:FGT-171013-1, 0.31}
and \qf{ex:FGT-171016-2, 2.05} are phrase medial.

\begin{exe}
	\ex{\glll	atoniʔ te\tbr{enu̯}, bifee-l te\tbr{enu̯}\\
						atoniʔ te\tbr{nu} bifee-l te\tbr{nu}\\
						man three{\tbrM} woman-\tsc{u} three{\tbrM}\\
			\glt	`Three men and three women.' \txrf{FGT-171013-1, 0.31}
						{\emb{FGT-171013-1-00-31.mp3}{\spk{}}{\apl}}}\label{ex:FGT-171013-1, 0.31}
	\ex{\glll	hai m-e\tbr{eku̯} kotugw leʔ iin\\
						hai m-e\tbr{ku} koto-gw leʔ iin\\
						{\hai} \m-eat{\tbrM} hyacinth.bean-\tsc{u} {\req} {\ia}\\
			\glt	`We ate these hyacinth beans.' \txrf{FGT-171016-2, 2.05}
						{\emb{FGT-171016-2-02-05.mp3}{\spk{}}{\apl}}}\label{ex:FGT-171016-2, 2.05}
\end{exe}

Final non-syllabic vowels in varieties of Meto such as Se{\Q}i Amanuban
are intermediate between fully unmetathesised and fully metathesised forms.
However, while I have attested forms with
an intermediate sequence of two identical vowels and
final non-syllabic vowel such as \ve{tenu} {\ra}
\ve{teenu̯} `three', I do not yet have any clear examples
of forms with a final non-syllabic vowel and intermediate
sequence of two different vowels such as *teunu̯.
% In wordlist Se'i speakers were quite insistent and took great delight in the fact that their logat was like that,
% laughed at the fact that Om Nus's was different

\subsection{Loss of final consonants}\label{sec:LosFinCon}
The final process which needs to be accounted
for in the derivation of M\=/forms in Amarasi
is deletion of final consonants of nominals.
This is seen in the formation of M\=/forms of CVC{\#} final words
such as \ve{muʔit} {\ra} \ve{muiʔ} `animal' (\srf{sec:MetConDel}),
as well as VVC{\#} final words such as
\ve{kaut} {\ra} \ve{kau} `papaya' (\srf{sec:ConDel}).
There are several pieces of evidence
indicating that, diachronically, final consonant
deletion preceded metathesis.

\largerpage[-2]
Firstly, in some varieties of Meto
certain nominals derive their M\=/form
only by consonant deletion.
This is the case in Naitbelak and Nai{\Q}bais Amfo{\Q}an
in which all VVC{\#} final nominals, as well as CVC{\#}
final nominals whose final vowel is not /a/ mark attributive
modification simply through consonant deletion.
\mbox{Examples}{\pagebreak} are given in \trf{tab:AmfConDel}.
This system may attest an older system, with Amarasi then
applying metathesis to (newly) CV{\#} final words.\footnote{
		Naitbeak/Nai{\Q}bais Amfo{\Q}an attributive modification
		for vowel-final words is marked by a lack of consonant insertion,
		an example is \ve{fafi-\j} `pig' + \ve{anaʔ} `small, baby' {\ra} \ve{fafi anaʔ} `piglet'.
		CVC{\#} final words with final /a/ mark modification through
		consonant deletion and metathesis. An example is
		\ve{ekam} `pandanus' + \ve{neno-g} `day/sky' {\ra} \ve{eek neno-g} `wild pandanus'.
		See \cite{cu18} for more details.}


\begin{table}[ht]
	\caption{Naitbelak/Nai{\Q}bais Amfo{\Q}an consonant deletion}\label{tab:AmfConDel}
	\centering
		%\begin{tabular}{r@{\hspace{0.4em}}c@{\hspace{0.4em}}llr@{\hspace{0.4em}}c@{\hspace{0.4em}}l} \lsptoprule
		\begin{tabular}{rclllcl} \lsptoprule
			N\sub{1}							&+&N\sub{2}			&Phrase									&N\sub{1}			&+&N\sub{2}		\\\midrule
			\ve{muke\tbr{ʔ}}			&+&\ve{kase-l}	&\ve{muke kase-l}				&`citrus'			&+&`foreign'	\\
			\ve{muʔi\tbr{t}}			&+&\ve{fui-\j}	&\ve{muʔi fui-\j}				&`animal'			&+&`wild'			\\
			\ve{manu\tbr{s}}			&+&\ve{noo-f}		&\ve{manu noo-f}				&`betel vine'	&+&`leaves'		\\
			\ve{fee mnasi\tbr{ʔ}}	&+&\ve{amenat}	&\ve{fee mnasi amenat}	&`old woman'	&+&`sick'			\\
			\ve{kua\tbr{n}}				&+&\ve{tuaf}		&\ve{kua tuaf}					&`village'		&+&`person'		\\
			\ve{kau\tbr{t}}				&+&\ve{noo-f}		&\ve{kau noo-f}					&`papaya'			&+&`leaves'		\\
			%\ve{\tbr{}}			&+&\ve{}		&\ve{}			&`'	&+&`'			&`'\\
		\lspbottomrule
		\end{tabular}
\end{table}

Secondly, only nouns used attributively have final
consonant deletion. As discussed in
\srf{sec:ConDel (Higher Level)} and further exemplified in \srf{sec:ConFinVer},
verbs with a final consonant do not usually have distinct
M\=/forms in Amarasi; the presence of a final consonant blocks verbal metathesis.
This indicates that consonant deletion
is a necessary precondition for metathesis
to apply to consonant-final roots.

Thirdly, before CC-initial modifiers final consonant
deletion is the only marker of the M\=/form,
with a word like \ve{muʔit} `animal' taking
the {\MC}-form \ve{muʔi}. This consonant deletion is
discussed further in \srf{sec:CCIniMod} below.

Fourthly, despite the fact that Ro{\Q}is Amarasi
permits clusters of three consonants (\srf{sec:RoqAnaCCIniMod}),
the final consonant of VVC{\#} words is still deleted
before CC-initial modifiers, i.e. \ve{kniit} `crab' + \ve{snaen} `sand'
{\ra} \ve{knii{\gap}snaen} `horned ghost crab'.

These facts indicate that at an earlier
stage of Amarasi final consonants of nouns were
deleted before attributive modifiers;
in the same way as they still are before CC-initial modifiers,
or as is still found with the surface M\=/form of VVC{\#} final words.

While deletion of final consonants
can be analysed synchronically as a result of a prohibition against
final consonant clusters created after metathesis,
from a diachronic perspective deletion
of final consonants probably occurred first, with
this consonant deletion then opening the way for
metathesis to apply to consonant-final roots.

\subsection{Morphologisation of metathesis}
The Ro{\Q}is Amarasi data with stressed vowel diphthongisation
and varieties of Meto with M\=/forms with final non-syllabic vowels
attest intermediate stages between fully unmetathesised
forms (e.g. \ve{fatu}) and fully metathesised forms
(e.g. \ve{faut}) which are consistent
with the development of compensatory metathesis
as predicted by \citet{blga98}.

The final stage in the development of Amarasi metathesis was
for the prosodic environments in which each form occurred
to be reinterpreted as different morphological environments
(Chapters \ref{ch:SynMet} and \ref{ch:DisMet}).
This creation of a paradigm of morphological metathesis
probably partly led to the imposition of the CVCVC template to all words of the language
in order to provide the necessary machinery for consonant-vowel metathesis to operate
and thereby allow each word to fill both cells of the morphological paradigm.

Finally, recall from Chapter \ref{ch:SynchMet} that final CV {\ra} VC
metathesis occurs in several languages of Timor including Helong and Mambae.
The presence of metathesis in these three languages
is almost certainly due to historic contact.
In the case of Helong and Meto, such contact is still ongoing,
while Mambae and Meto are no longer in contact with one another.
Thus, while final CV {\ra} VC metathesis may have first arisen
according to the process of compensatory metathesis as described above,
it is unlikely that these processes occurred independently
in each of Meto, Helong, and Mambae. Instead they likely occurred
in one of these languages from which they then diffused into the others.
Which of these languages first acquired metathesis remains to be investigated.

\section{M\=/forms before consonant clusters}\label{sec:CCIniMod}
In the previous sections I described and analysed the basic M\=/form.
This is the M\=/form taken by vowel-final verbs,
as well as nouns before an attributive modifier
with only a single initial consonant.
In addition to the basic M\=/form,
Amarasi has an additional M\=/form which is used
by nouns before attributive modifiers 
which begin with a consonant cluster.
This M\=/form is derived by deletion of any final consonant with no further changes.

M\=/forms before consonant clusters are glossed with `{\Mc}'
(\tsc{m} with a `c' for consonant above it).
{\MC}-forms are the paradigmatic and morphological equivalents of basic M\=/forms
in a different phonological environment.
Basic M\=/forms occur before attributive modifiers
which begin with a single consonant,
while {\MC}-forms occur before attributive modifiers with an initial cluster.
This process is not predictable consonant deletion to
avoid a cluster of three consonants. 
(Such clusters are avoided in other situations by
epenthesis of /a/ as discussed in \srf{sec:Epe}).

Examples of {\MC}-forms are shown in \qf{ex:C->0/CC} below for each word shape.
The modifiers used to illustrate are \ve{mnasiʔ} `old', \ve{kbubuʔ} `round'
\ve{mnanuʔ} `long' and \ve{mnatuʔ} `ripe, cooked' as semantically appropriate.
Words which end in a vowel in the U\=/form do not have distinct
M\=/forms before modifiers which begin with a consonant cluster.
It is possible at an abstract level to analyse the M\=/form
of such words as being formed by deletion of the final empty C-slot.

\begin{exe}
	\ex{C{\#} {\ra} {\0}/{\gap}CC }\label{ex:C->0/CC}
	\gw\sn{\begin{tabular}{rlll}
			 U\=/form						&			&{\MC}-form						&\\
			\ve{muʔi\tbr{t}}	&{\ra}&\ve{muʔi mnasiʔ}	&`old animal'	\\
			\ve{kau\tbr{t}}		&{\ra}&\ve{kau mnatuʔ}	&`cooked/ripe papaya'	\\
			\ve{nautu\tbr{s}}	&{\ra}&\ve{nautu kbubuʔ}&`round beetle'	\\
			\ve{fafi}					&{\ra}&\ve{fafi mnasiʔ}	&`old pig'	\\
			\ve{ume}					&{\ra}&\ve{ume kbubuʔ}	&`round house'	\\
			\ve{aunu}					&{\ra}&\ve{aunu mnanuʔ}	&`long spear'	\\
			\ve{oo}						&{\ra}&\ve{oo kbubuʔ}		&`round (piece of) bamboo'	\\
		\end{tabular}}
\end{exe}

The relationship between the surface forms of the U\=/form and {\MC}-form
\ve{muʔit} {\ra} \ve{muʔi} `animal' and \ve{fafi} {\ra} \ve{fafi} `pig'
are shown in \qf{as:muqit/muqi} and \qf{as:fatu/fatu} below.

\begin{multicols}{2}
	\begin{exe}
		\exa{\xy
			<0em,2.5cm>*\as{`animal'}="gloss",
			<2.5em,2cm>*\as{m}="u1",<3.5em,2cm>*\as{u}="u2",<4.5em,2cm>*\as{ʔ}="u3",<5.5em,2cm>*\as{i}="u4",<6.5em,2cm>*\as{t}="u5",<0em,2cm>*\as{U\=/form:}="u",
			<2.5em,1.5cm>*\as{C}="uC1",<3.5em,1.5cm>*\as{V}="uC2",<4.5em,1.5cm>*\as{C}="uC3",<5.5em,1.5cm>*\as{V}="uC4",<6.5em,1.5cm>*\as{C}="uC5",
			<2.5em,0.5cm>*\as{C}="mC1",<3.5em,0.5cm>*\as{V}="mC2",<4.5em,0.5cm>*\as{C}="mC3",<5.5em,0.5cm>*\as{V}="mC4",
			<2.5em,0cm>*\as{m}="m1",<3.5em,0cm>*\as{u}="m2",<4.5em,0cm>*\as{ʔ}="m4",<5.5em,0cm>*\as{i}="m3",<0em,0cm>*\as{{\MC}-form:}="m",
			{\ar@{->} "uC1"+D;"mC1"+U};{\ar@{->} "uC2"+D;"mC2"+U};{\ar@{->} "uC3"+D;"mC3"+U};{\ar@{->} "uC4"+D;"mC4"+U};
		\endxy}\label{as:muqit/muqi}
	\end{exe}
	\begin{exe}
		\exa{\xy
			<0em,2.5cm>*\as{`pig'}="gloss",
			<2.5em,2cm>*\as{f}="u1",<3.5em,2cm>*\as{a}="u2",<4.5em,2cm>*\as{f}="u3",<5.5em,2cm>*\as{i}="u4",<0em,2cm>*\as{U\=/form:}="u",%<6.5em,2cm>*\as{t}="u5",
			<2.5em,1.5cm>*\as{C}="uC1",<3.5em,1.5cm>*\as{V}="uC2",<4.5em,1.5cm>*\as{C}="uC3",<5.5em,1.5cm>*\as{V}="uC4",%<6.5em,1.5cm>*\as{C}="uC5",
			<2.5em,0.5cm>*\as{C}="mC1",<3.5em,0.5cm>*\as{V}="mC2",<4.5em,0.5cm>*\as{C}="mC3",<5.5em,0.5cm>*\as{V}="mC4",
			<2.5em,0cm>*\as{f}="m1",<3.5em,0cm>*\as{a}="m2",<4.5em,0cm>*\as{f}="m4",<5.5em,0cm>*\as{i}="m3",<0em,0cm>*\as{{\MC}-form:}="m",
			{\ar@{->} "uC1"+D;"mC1"+U};{\ar@{->} "uC2"+D;"mC2"+U};{\ar@{->} "uC3"+D;"mC3"+U};{\ar@{->} "uC4"+D;"mC4"+U};
		\endxy}\label{as:fatu/fatu}
	\end{exe}
\end{multicols}

VVC{\#} words with a final /n/ form a partial exception to this rule
when they occur before a modifier which begins with two nasals.
In such instances either the final consonant is deleted,
or it is retained and epenthesis occurs.
One example is \ve{kuan} `village' modified by \ve{mnaa{ʔ}} `old, former'
in which case both \ve{kua mnaa{ʔ}} or \ve{kuan a|mnaa{ʔ}} 
occur with an attributive meaning.\footnote{
		Nekmese{\Q} village was founded in the 1970s
		and many people still maintain fields and gardens near the old village
		(see \srf{sec:LanBac} for more details).
		Thus, the phrase \ve{kua(n a)mnaa{ʔ}} is frequently heard.
		The form \ve{kua mnaaʔ} is much more common in my experience.}

Likewise, when asked to translate `old tap' into Amarasi (\ve{kraan} `tap' + \ve{mnaaʔ} `old'),
Roni (my main consultant) produced the string \ve{kraan a|mnaa{ʔ}}.
I immediately then presented him with the string \ve{kraa mnaa{ʔ}}
which he interpreted as being `old glass', from \ve{kraas} + \ve{mnaa{ʔ}}.

In \srf{sec:ConDel/CC} and \srf{sec:CVFinWor} below I sketch
a partial analysis of {\MC}-forms within Optimality Theory.
I do this because the tableaux that this theory employs
illustrate well the large number of potential outputs
the combination of a noun followed by an attributive
modifier could potentially generate.
The purpose of this book is not to give a
complete Optimality Theory account of metathesis in Amarasi.
Indeed, the high level of opacity in the formation of M\=/forms
-- including at least one derived environment effect (\srf{sec:AssOfA}) --
indicates that standard Optimality Theory
would not fare particularly well in Amarasi.
Nonetheless Optimality Theory is still a useful tool
to illuminate certain aspects of the structure of the language.

\subsection{Consonant deletion}\label{sec:ConDel/CC}
When a consonant-final word, such as \ve{muʔit} `animal',
occurs before an attributive modifier
with an initial cluster, such as \ve{mnasiʔ} `old',
the final consonant of the first noun is deleted.
This yields \brac{NP} \ve{muʔi} \ve{mnasiʔ}] `an old animal'.

In such instances, there are a large number of potential
outputs involving combinations of: metathesis, consonant deletion, and/or epenthesis.
Each of these potential outputs is given in the Optimality Theory
tableau in \qf{ex:muqi mnasiq} below, along with the constraints they violate.
The definitions constraints are given in \qf{ex:Constraints}.
Their ranking is according to the order given.

\begin{exe}
	\ex{\begin{xlist}
		\ex{\tsc{*CC{\#}}: No final consonant clusters}
		\ex{\tsc{*-CC-}: No foot medial consonant clusters}
		\ex{\tsc{*CCC}: No clusters of three consonants}
		\ex{\tsc{Dep}: No epenthesis}
		\ex{\tsc{Max}: No deletion}
		\ex{\tsc{\M}: Mark the M\=/form}\label{ex:MarMfo}
		\ex{\tsc{Linearity}: No metathesis}
	\end{xlist}}\label{ex:Constraints}
\end{exe}

Constraint \qf{ex:MarMfo} is equivalent to
\tsc{RealizeMorpheme} in the sense of \citet{ku01}.
This constraint is included as the forms under discussion
are those which are paradigmatically and morphologically
equivalent to the basic M\=/form which occurs before attributive
modifiers with no initial cluster.

\begin{exe}
	\ex{\rule{0pt}{0pt}{} \\[-5ex]\stl{0.41em}
	\begin{tabular}[t]{|rrl||c|c|c|c|c|c|c|} \hline
		\multicolumn{3}{|c||}{\brac{NP} \ve{muʔit} + \ve{mnasiʔ} ]} 
																						&*CC{\#}&*-CC-	&*CCC 		&\tsc{Dep}&\tsc{Max}&{\M}			& \tsc{Lin} \\[0.5ex]\hline
		\hline a. &				& \ve{muiʔt mnasiʔ} 	&*! 		&{\cgr}	&{\cgr}**	&{\cgr}		&{\cgr} 	&{\cgr}		& {\cgr}* \\
		\hline b. &				& \ve{muʔti mnasiʔ}		&				&*!			&{\cgr}		&{\cgr}		&{\cgr}		&{\cgr}		&	{\cgr}*	\\
		\hline c. &				& \ve{muiʔ mnasiʔ}		&				&				&*! 			&{\cgr} 	&{\cgr}* 	&{\cgr}		& {\cgr}* \\
		\hline d. &				& \ve{mui mnasiʔ} 		&				&				&					&					&**! 			&{\cgr}		& {\cgr} 	\\
		\hline e. &				& \ve{muiʔt a|mnasiʔ} &*! 		&{\cgr}	&{\cgr} 	&{\cgr}*	&{\cgr} 	&{\cgr}		& {\cgr}* \\
		\hline f. &				& \ve{muʔit mnasiʔ} 	&				&				&*! 			&{\cgr} 	&{\cgr} 	&{\cgr}*	& {\cgr}	\\
		\hline g. &				& \ve{muʔit a|mnasiʔ} &				&				&					&*! 			&{\cgr} 	&{\cgr}*	& {\cgr}	\\
		\hline h. &{\hand}& \ve{muʔi mnasiʔ} 		&				&				&					&					&* 				&{\cgr}		& {\cgr}	\\
	\hline \end{tabular}}\label{ex:muqi mnasiq}
\end{exe}

Table \qf{ex:muqi mnasiq} shows that the output with deletion of the
final consonant, \ve{muʔi mnasiʔ}, is the best output.
This candidate marks the M\=/form and also
avoids final consonant clusters, foot medial clusters,
clusters of three consonants, and epenthesis.
While it does have consonant deletion,
it only deletes one consonant while the next best candidate
\ve{\tcb{*}mui mnasiʔ} has two consonants deleted.

When a consonant-final word occurs before a predicative modifier
with an initial consonant cluster,
epenthesis usually occurs between the two words.
This is shown in \qf{ex:muqit-amnasiq} and \qf{tr:muqit-amnasiq}
below, which can be contrasted with the attributive phrases
in \qf{ex:muqi-mnasiq} and \qf{tr:muqi-mnasiq}.

\begin{multicols}{2}
	\begin{exe}
		\ex{\gll \brac{NP} {muʔit \bracr{}} \brac{NP} {a|mnasiʔ \bracr{}}\\
							{} animal{\U} {} {\a}old\\
				\glt \lh{\brac{NP} }`Animals are old.'}\label{ex:muqit-amnasiq}
		\ex{\gll \brac{NP} muʔi mnasiʔ \bracr{}\\
							{} animal{\Mc} old {}\\
				\glt \lh{\brac{NP} }`(an) old animal'}\label{ex:muqi-mnasiq}
	\end{exe}
\end{multicols}
%\newpage
\begin{multicols}{2}
	\begin{exe}
		\ex{\begin{forest} %where n children=0{tier=word}{}
			[S,[NP,[N,[\ve{muʔit}\\animal{\U}]]][NP,[N,[\ve{a|mnasiʔ}\\old]]]]
		\end{forest}}\label{tr:muqit-amnasiq}
		\ex{\begin{forest} %where n children=0{tier=word}{}
			[S,[\vp{NP}{\ldots},[,phantom]][NP,[N,[\ve{muʔi}\\animal{\Mc}]][N,[\ve{mnasiʔ}\\old]]]]
		\end{forest}}\label{tr:muqi-mnasiq}
	\end{exe}
\end{multicols}

This can be explained by positing that while epenthesis
is not allowed within a single phrase,
it \emph{is} allowed between two separate phrases.
In the terminology of Optimality Theory,
the constraint \tsc{Dep} is more highly ranked than \tsc{Max} within a single phrase,
while between two phrases \tsc{Max} is more highly ranked than \tsc{Dep}.
A modified version of table \qf{ex:muqi mnasiq}
is given in \qf{ex:muqit amnasiq} below for predicative phrase
with the constraints re-ordered as appropriate.
The constraint \tsc{\M} `mark the M\=/form' has been removed as
this is not a requirement of predicative phrases.

\begin{exe}
	\ex{\rule{0pt}{0pt}{} \\[-5ex]\stl{0.45em}
	\begin{tabular}[t]{|rrl||c|c|c|c|c|c|} \hline
		\multicolumn{3}{|c||}{\brac{NP} \ve{muʔit} ] + \brac{NP} \ve{mnasiʔ} ]}
																						&*CC{\#}&*-CC-	&*CCC 		&\tsc{Max}&\tsc{Dep}& \tsc{Lin} \\[0.5ex]\hline
		\hline a. &				& \ve{muiʔt mnasiʔ} 	&*! 		&{\cgr}	&{\cgr}**	&{\cgr}		&{\cgr} 	& {\cgr}* \\
		\hline b. &				& \ve{muʔti mnasiʔ}		&				&*!			&{\cgr}		&{\cgr}		&{\cgr}		&	{\cgr}*	\\
		\hline c. &				& \ve{muiʔ mnasiʔ}		&				&				&*! 			&{\cgr}* 	&{\cgr} 	& {\cgr}* \\
		\hline d. &				& \ve{mui mnasiʔ} 		&				&				&					&**! 			&{\cgr}		& {\cgr} 	\\
		\hline e. &				& \ve{muiʔt a|mnasiʔ} &*! 		&{\cgr}	&{\cgr} 	&{\cgr} 	&{\cgr}*	& {\cgr}* \\
		\hline f. &				& \ve{muʔit mnasiʔ} 	&				&				&*! 			&{\cgr} 	&{\cgr} 	& {\cgr}	\\
		\hline g. &{\hand}& \ve{muʔit a|mnasiʔ} &				&				&					&					&*	 			& {\cgr}	\\
		\hline h. &				& \ve{muʔi mnasiʔ} 		&				&				&					&*!				& 				& {\cgr}	\\
	\hline \end{tabular}}\label{ex:muqit amnasiq}
\end{exe}

Table \qf{ex:muqit amnasiq} shows that when two separate
noun phrases occur next to one another
a cluster of three consonants is resolved by epenthesis.
It is better to epenthesise between noun phrases
than to have a cluster of three consonants.

\begin{multicols}{2}
	\begin{exe}
		\exa{\label{as:muqit-amnasiq}\xy
			<3em,6cm>*\as{PrWd}="PrWd1",<9.5em,6cm>*\as{PrWd}="PrWd2",
			<3em,5cm>*\as{Ft}="ft1",<11em,5cm>*\as{Ft}="ft2",
			<2em,4cm>*\as{σ}="s1",<4em,4cm>*\as{σ}="s2",<7em,4cm>*\as{σ}="s3",<10em,4cm>*\as{σ}="s4",<12em,4cm>*\as{σ}="s5",
			<1em,3cm>*\as{C}="CV1",<2em,3cm>*\as{V}="CV2",<3em,3cm>*\as{C}="CV3",<4em,3cm>*\as{V}="CV4",<5em,3cm>*\as{C}="CV5",
			<6em,3cm>*\as{C}="CV6",<7em,3cm>*\as{V}="CV7",<8em,3cm>*\as{C}="CV8",
			<9em,3cm>*\as{C}="CV9",<10em,3cm>*\as{V}="CV10",<11em,3cm>*\as{C}="CV11",<12em,3cm>*\as{V}="CV12",<13em,3cm>*\as{C}="CV13",
			<1em,2cm>*\as{m}="cv1",<2em,2cm>*\as{u}="cv2",<3em,2cm>*\as{ʔ}="cv3",<4em,2cm>*\as{i}="cv4",<5em,2cm>*\as{t}="cv5",
			<6em,2cm>*\as{ʔ}="cv6",<7em,2cm>*\as{a}="cv7",<8em,2cm>*\as{m}="cv8",<9em,2cm>*\as{n}="cv9",
			<10em,2cm>*\as{a}="cv10",<11em,2cm>*\as{s}="cv11",<12em,2cm>*\as{i}="cv12",<13em,2cm>*\as{ʔ}="cv13",
			<3em,1cm>*\as{M}="m1",<10.5em,1cm>*\as{M}="m2",
			<3em,0cm>*\as{SynWd}="SynWd1",<10.5em,0cm>*\as{SynWd}="SynWd2",
			"SynWd1"+U;"m1"+D**\dir{-};"SynWd2"+U;"m2"+D**\dir{-};
			"m1"+U;"cv1"+D**\dir{-};"m1"+U;"cv2"+D**\dir{-};"m1"+U;"cv3"+D**\dir{-};"m1"+U;"cv4"+D**\dir{-};"m1"+U;"cv5"+D**\dir{-};
			"m2"+U;"cv8"+D**\dir{-};"m2"+U;"cv9"+D**\dir{-};"m2"+U;"cv10"+D**\dir{-};
			"m2"+U;"cv11"+D**\dir{-};"m2"+U;"cv12"+D**\dir{-};"m2"+U;"cv13"+D**\dir{-};
			"cv1"+U;"CV1"+D**\dir{-};"cv2"+U;"CV2"+D**\dir{-};"cv3"+U;"CV3"+D**\dir{-};"cv4"+U;"CV4"+D**\dir{-};
			"cv5"+U;"CV5"+D**\dir{-};"cv6"+U;"CV6"+D**\dir{-};"cv7"+U;"CV7"+D**\dir{-};
			"cv8"+U;"CV8"+D**\dir{-};"cv9"+U;"CV9"+D**\dir{-};"cv10"+U;"CV10"+D**\dir{-};
			"cv11"+U;"CV11"+D**\dir{-};"cv12"+U;"CV12"+D**\dir{-};"cv13"+U;"CV13"+D**\dir{-};
			"CV1"+U;"s1"+D**\dir{-};"CV2"+U;"s1"+D**\dir{-};"CV3"+U;"s1"+D**\dir{-};
			"CV3"+U;"s2"+D**\dir{-};"CV4"+U;"s2"+D**\dir{-};"CV5"+U;"s2"+D**\dir{-};
			"CV6"+U;"s3"+D**\dir{-};"CV7"+U;"s3"+D**\dir{-};"CV8"+U;"s3"+D**\dir{-};
			"CV9"+U;"s4"+D**\dir{-};"CV10"+U;"s4"+D**\dir{-};"CV11"+U;"s4"+D**\dir{-};
			"CV11"+U;"s5"+D**\dir{-};"CV12"+U;"s5"+D**\dir{-};"CV13"+U;"s5"+D**\dir{-};
			"s1"+U;"ft1"+D**\dir{-};"s2"+U;"ft1"+D**\dir{-};"s4"+U;"ft2"+D**\dir{-};"s5"+U;"ft2"+D**\dir{-};
			"ft1"+U;"PrWd1"+D**\dir{-};"s3"+U;"PrWd2"+D**\dir{-};"ft2"+U;"PrWd2"+D**\dir{-};
		\endxy}
		\exa{\label{as:muqi-mnasiq}\xy
			<3em,6cm>*\as{PrWd}="PrWd1",<8em,6cm>*\as{PrWd}="PrWd2",
			<3em,5cm>*\as{Ft}="ft1",<8em,5cm>*\as{Ft}="ft2",
			<2em,4cm>*\as{σ}="s1",<4em,4cm>*\as{σ}="s2",<7em,4cm>*\as{σ}="s3",<9em,4cm>*\as{σ}="s4",
			<1em,3cm>*\as{C}="CV1",<2em,3cm>*\as{V}="CV2",<3em,3cm>*\as{C}="CV3",
			<4em,3cm>*\as{V}="CV4",<5em,3cm>*\as{C}="CV5",
			<6em,3cm>*\as{C}="CV6",<7em,3cm>*\as{V}="CV7",<8em,3cm>*\as{C}="CV8",
			<9em,3cm>*\as{V}="CV9",<10em,3cm>*\as{C}="CV10",
			<1em,2cm>*\as{m}="cv1",<2em,2cm>*\as{u}="cv2",<3em,2cm>*\as{ʔ}="cv3",<4em,2cm>*\as{i}="cv4",
			<5em,2cm>*\as{m}="cv5",<6em,2cm>*\as{n}="cv6",<7em,2cm>*\as{a}="cv7",<8em,2cm>*\as{s}="cv8",
			<9em,2cm>*\as{i}="cv9",<10em,2cm>*\as{ʔ}="cv10",
			<2.5em,1cm>*\as{M}="m1",<7.5em,1cm>*\as{M}="m2",
			<5em,0cm>*\as{SynWd}="SynWd1","SynWd1"+U;"m1"+D**\dir{-};"SynWd1"+U;"m2"+D**\dir{-};
			"m1"+U;"cv1"+D**\dir{-};"m1"+U;"cv2"+D**\dir{-};"m1"+U;"cv3"+D**\dir{-};"m1"+U;"cv4"+D**\dir{-};
			"m2"+U;"cv5"+D**\dir{-};"m2"+U;"cv6"+D**\dir{-};"m2"+U;"cv7"+D**\dir{-};
			"m2"+U;"cv8"+D**\dir{-};"m2"+U;"cv9"+D**\dir{-};"m2"+U;"cv10"+D**\dir{-};
			"cv1"+U;"CV1"+D**\dir{-};"cv2"+U;"CV2"+D**\dir{-};"cv3"+U;"CV3"+D**\dir{-};"cv4"+U;"CV4"+D**\dir{-};"cv5"+U;"CV5"+D**\dir{-};
			"cv6"+U;"CV6"+D**\dir{-};"cv7"+U;"CV7"+D**\dir{-};"cv8"+U;"CV8"+D**\dir{-};"cv9"+U;"CV9"+D**\dir{-};
			"cv10"+U;"CV10"+D**\dir{-};
			"CV1"+U;"s1"+D**\dir{-};"CV2"+U;"s1"+D**\dir{-};"CV3"+U;"s1"+D**\dir{-};
			"CV3"+U;"s2"+D**\dir{-};"CV4"+U;"s2"+D**\dir{-};"CV5"+U;"s2"+D**\dir{-};
			"CV6"+U;"s3"+D**\dir{-};"CV7"+U;"s3"+D**\dir{-};"CV8"+U;"s3"+D**\dir{-};
			"CV8"+U;"s4"+D**\dir{-};"CV9"+U;"s4"+D**\dir{-};"CV10"+U;"s4"+D**\dir{-};
			"s1"+U;"ft1"+D**\dir{-};"s2"+U;"ft1"+D**\dir{-};"s3"+U;"ft2"+D**\dir{-};"s4"+U;"ft2"+D**\dir{-};
			"ft1"+U;"PrWd1"+D**\dir{-};"ft2"+U;"PrWd2"+D**\dir{-};
		\endxy}
	\end{exe}
\end{multicols}

In an analysis which is considered in \srf{sec:ProsMet},
I propose that members of an attributive phrase
are members of a single category -- the syntactic word --
while each member of a predicative phrase
is a member of a different syntactic word.\footnote{
	This is essentially the same as proposing that
	attributive phrases are a (syntactic) compound,
	even though their members may belong to different
	prosodic categories.}
The relationship between the prosodic structure, morphological structure,
and the Syntactic Word(s) of the attributive phrase
\ve{muʔi mnasiʔ} `(an) old animal',
and the predicative phrase \ve{muʔit a|mnasiʔ} `animals are old'
are shown in \qf{as:muqit-amnasiq} and \qf{as:muqi-mnasiq} above respectively.

When a consonant-final nominal occurs before a 
modifier with an initial consonant cluster,
the cluster of three consonants is usually resolved in Amarasi.
In an attributive phrase, such as that
represented in \qf{as:muqi-mnasiq},
the M\=/form must be realised to mark the presence of this attributive modifier.
Metathesis is blocked as it would result in a cluster of three consonants,
as exemplified in \qf{ex:muqi mnasiq}.
As a result, the final consonant of the first
noun is deleted to express the M\=/form.
This also has the effect of resolving the cluster of three consonants.

However, when the phrase consists of two syntactic words,
such as that represented in \qf{as:muqit-amnasiq},
there is no need to mark the M\=/form.
As a result, the cluster of three consonants remains.
Epenthesis of /a/ (preceded by an automatic glottal stop
--- see \srf{sec:GloStoIns}) then occurs between these two syntactic words,
thus resolving the cluster of three consonants.

\subsection{No change}\label{sec:CVFinWor}
Vowel-final words do not have a distinct M\=/form
before attributive modifiers with an initial cluster.
Metathesis in this environment is blocked as it would
create a cluster of three consonants.
It is more important in Kotos Amarasi to avoid a cluster of three
consonants than it is to mark the M\=/form.

However, there are at least two logical ways in which Amarasi
could avoid a cluster of three consonants and still mark the
M\=/form for CV{\#} final words.
Firstly, metathesis could occur with subsequent epenthesis: i.e.
\ve{fafi} `pig' + \ve{mnasiʔ} `old' {\ra}
\ve{\tcb{*}fai\tbr{f} \tbr{mn}asiʔ} {\ra} \ve{\tcb{*}faif a|mnasiʔ}.
Epenthesis is attested elsewhere in Amarasi
to break up sequences of three consonants (\srf{sec:Epe}).
Secondly, metathesis could take place with subsequent deletion of the final consonant,
\ve{\tcb{*}fai\tbr{f} \tbr{mn}asiʔ} {\ra} \ve{\tcb{*}fai mnasiʔ}.
Consonant deletion is attested elsewhere in
the derivation of M\=/forms (\srf{sec:MetConDel}, \srf{sec:ConDel}, \srf{sec:ConDel/CC}).

We thus have at least four possible outputs when a CV{\#}
final word is modified by a nominal with an initial consonant cluster.
Each of these potential outputs is given in the Optimality Theory
tableau in \qf{ex:fafi mnasiq} below, along with the constraint(s) they violate.
These constraints and their ranking
were given in \qf{ex:Constraints} above.

\newpage
\begin{exe}
	\ex{\rule{0pt}{0pt}{} \\[-5ex]
	\begin{tabular}[t]{|rrl||c|c|c|c|c|} \hline
		\multicolumn{3}{|c||}{\brac{NP} \ve{fafi} + \ve{mnasiʔ} ]} 
																					& *CCC&\tsc{Dep}&\tsc{Max}&\tsc{\M}	&\tsc{Lin} \\[0.5ex]\hline
		\hline a. & 			& \ve{faif mnasiʔ}	& *!	&{\cgr} 	&{\cgr} 	&{\cgr}		&{\cgr}*\\
		\hline b. & 			& \ve{faif a|mnasiʔ}& 		&*!				&{\cgr} 	&{\cgr}		&{\cgr}*\\
		\hline c. & 			& \ve{fai mnasiʔ}		& 		&  				&*! 			&{\cgr}		&{\cgr} \\
		\hline d. &{\hand}& \ve{fafi mnasiʔ} 	& 		& 				& 				&*				&{\cgr} \\
	\hline \end{tabular}}\label{ex:fafi mnasiq}
\end{exe}

Potential output (\ref{ex:fafi mnasiq}a.) \ve{\tcb{*}faif mnasiʔ}
does not occur because it is worse to have a cluster of three consonants
than it is to mark the M\=/form.
Potential output (\ref{ex:fafi mnasiq}b.) \ve{\tcb{*}faif a|mnasiʔ}
does not occur because it is worse to epenthesise
(within a single phrase) than it is to mark the M\=/form.
Potential output (\ref{ex:fafi mnasiq}c.) \ve{\tcb{*}fai mnasiʔ}
does not occur because it is worse to delete a medial consonant
than it is to mark the M\=/form.
This leaves the occurring output \ve{fafi mnasiʔ},
which fails to mark the M\=/form but does not violate
any of the more highly ranked constraints.

\subsection{Ro{\Q}is Amarasi modifiers with an initial cluster}\label{sec:RoqAnaCCIniMod}
In Ro{\Q}is Amarasi metathesis occurs before words
which begin with a consonant cluster.
Examples are given in \trf{tab:RoqMetConClu} on the next page
alongside Kotos Amarasi equivalents (where known) for comparison.

Metathesis of CV(C){\#} final words is the most common
pattern before CC-initial modifiers in my Ro{\Q}is data.
However, two other patterns are also found.\footnote{
		These alternate patterns are most frequent for
		one of my consultants from Tunbaun,
		though do sporadically occur in the speech of others.}
Firstly, there are two examples in my corpus in
which metathesis does not occur:
\ve{smana-f} `spirit' + \ve{kninuʔ} `clean, holy'
{\ra} \ve{smana kninuʔ} `Holy Spirit'
and \ve{hana-f} `voice' + \ve{tbaat} `lies across, in-between'
{\ra} \ve{hana tbaat} `intermediate dialect'.\footnote{
		The same speaker also uses \ve{haan tbaat} `intermediate dialect'
		at another point in the same text.}

Secondly, diphthongisation of the stressed vowel (\srf{sec:RoqAmaDip})
without metathesis can also occur. There are six examples in my data
of which three are: \ve{u\tbr{mi}} `house' + \ve{kbubuʔ} `round' {\ra}
\ve{u͡\tbr{{\i}mi} kbubuʔ} `round house'
(alongside Ro{\Q}is \ve{uim kbubuʔ}, Kotos \ve{umi kbubuʔ}),
\ve{ne\tbr{no}} `day' + \ve{krei} `church' {\ra} \ve{ne͡\tbr{ono} krei} `Sunday'
(alongside Ro{\Q}is \ve{neon krei}, Kotos \ve{neno krei}), and
\ve{ra\tbr{si}} `matter' + \ve{skoor} `school' {\ra} \ve{ra͡\tbr{{\i}si} skoor}
`school matters' (Kotos \ve{rasi skoor}).
Given that diphthongisation of stressed vowels
followed by a closed syllable is an automatic process
in Ro{\Q}is (see \srf{sec:RoqAmaDip}),
it is probably best to analyse such instances as underlyingly
unmetathesised, with an automatic phonological rule applying.
Nonetheless, diachronically, such forms are intermediate
between metathesised and unmetathesised forms,
as was discussed in \srf{sec:OriMetAma}.

\begin{table}[ht]
	\centering\caption{Ro{\Q}is metathesis before consonant clusters}\label{tab:RoqMetConClu}
	\begin{tabular}{l@{\hspace{0.4em}}c@{\hspace{0.4em}}l@{ }llll}\lsptoprule
			 Noun				&	& mod.					&		& Ro{\Q}is 								& Kotos & \\ \midrule
		\ve{kruru-f} 	&+& \ve{tnana-f}	&\ra& \ve{kru\tbr{ur tn}anaf} & \ve{kruru tnanaf} & `middle finger' \\
		`finger'			&+&	`middle'			&&&&	\\
		\ve{umi}			&+& \ve{kbubuʔ}		&\ra& \ve{u\tbr{im kb}ubuʔ} 	& \ve{umi kbubuʔ}		& `round house' \\
		`house'				&+&	`round'				&&&&	\\
		\ve{tenoʔ}		&+& \ve{kmoro-f}	&\ra& \ve{te\tbr{on km}orof}	& \ve{teno kmorof}	& `egg yolk' \\
		`egg'					&+&	`yellow'			&&&&	\\
		\ve{fatu} 		&+& \ve{kruru-f}	&\ra& \ve{fa\tbr{ut}{\gap}\tbr{kr}uruf} 	& \ve{fatu kruruf}	& `soft coral' \\
		`stone'				&+&	`finger'			&&&&	\\
		\ve{ikaʔ} 		&+& \ve{tnopos}		&\ra& \ve{i\tbr{ik}{\gap}\tbr{tn}opos} 	& \ve{} & `silver moony \\
		`fish'				&+&	`silver'			&& \mc{3}{r}{(\it{Monodactylus argentus})'}	\\
		\ve{ikaʔ} 		&+& \ve{kbiti}		&\ra& \ve{i\tbr{ik}{\gap}\tbr{kb}iti} 		& \ve{} & `spinefoot \\
		`fish'				&+&	`scorpion'		&&&&\hp{`}(\it{Siganus spp.})'	\\
%		\ve{} 		&+& \ve{}	&\ra& \ve{} 	& \ve{} & `' \\
%		`'				&+&	`'			&&&&	\\
		\lspbottomrule
	\end{tabular}
\end{table}

Finally, VVC{\#} final words in Ro{\Q}is delete their
final consonant before all modifiers,
including modifiers with an initial cluster,
to derive an M\=/form in the same way as Kotos (\srf{sec:ConDel}).
Two examples from my Ro{\Q}is data are: \ve{knaaʔ}
+ \ve{mnanuʔ} `long' {\ra} \ve{knaa mnanuʔ} `long beans'
and \ve{kniit} `crab' + \ve{snaen} `sand' {\ra} \ve{knii snaen}
`horned ghost crab (\emph{Ocypode ceratophthalma})'.
Given that Ro{\Q}is permits clusters of three consonants,
deletion of the final consonant in these instances
provides additional evidence that the consonant deletion
Kotos Amarasi is indeed a morphological process.

%Buraen
%uim krei
%uim skoor
%uim preent
%niim mnonof
%teon kmorof
%teon kmuitif
%kruur tnanaf
%faut kruruf
%knaam fno'ot (k.o. katydid) knamat
%akbeun bjae/bjakase' (big fly)
%kuum treukus
%teem brahu'
%niin tboorn
%koor bjakase'
%seor bjae te'i
%kroom bjakase' (kroma')
%kbaut bjakase' (kbautus)
%kroom snaen (p. 16)
%iik knaapn (p. 18)
%iik knaes
%iik snaen
%iik knaap fafi
%iik tnopos
%iik kbiti
%iik kmuru
%iik mneas
%
%Tunbaun
%uim krei, uimi krei
%uim skoor, uimi skoor
%kaun smata'
%
%Texts:
%atoin mnasi'
%haan tbaat
%
%neono tnana'
%neono krei
%raisi skoor
%raisi krei
%raisi kninu'
%uiri mnatu'
%
%smana kninu'
%hana tbaat ~ haan tbaat