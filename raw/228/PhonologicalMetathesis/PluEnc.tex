\section{The plural enclitic}\label{sec:PluEnc}
The plural enclitic has a number of allomorphs and variant forms,
partly depending on the shape of the host to which it attaches.
This enclitic marks plurals for nouns and for verbs it marks that one
or more of the core verbal arguments (subject or object) is plural.
The allomorphy of the plural enclitic for verbs and nouns is similar, though not identical.
This allomorphy is summarised in \trf{tab:PluEncAll}.
The main difference is the allomorphs taken by
nouns and verbs ending in a vowel sequence.

\begin{table}[h]
	\caption{Plural enclitic allomorphy}\label{tab:PluEncAll}
	\centering
		\begin{tabular}{rll}\lsptoprule
							Stem	& Nominals				&Verbs	 \\ \midrule
			{\ldots}C{\#}	& \ve{=ein/=eni},	&\ve{=ein/=eni}, \\
										& \ve{=enu/=uun}	&\ve{=enu/=uun}\\
			{\ldots}CV{\#}& \ve{=n}					&\ve{=n}\\
			{\ldots}VV{\#}& \ve{=n=gwein},	&\ve{=n}\\
										& \ve{=nu}				&\\
			\lspbottomrule
		\end{tabular}
\end{table}

After consonant-final stems, the plural enclitic
usually has the form \ve{=ein/=eni}.
The M\=/form \ve{=ein} is usually realised as [ɪn], and U\=/form \ve{=eni} as [ɛni].
The choice between the U\=/form and M\=/form of this enclitic
is discourse driven (Chapter \ref{ch:DisMet}) and
the M\=/form is the default form (\srf{sec:DefFor1}).
Before this enclitic CVC{\#} stems undergo metathesis,
as is expected before vowel-initial enclitics.
Examples of pluralised consonant-final verbs and nouns
are given in \qf{ex:pl->-ein/C} below.

\begin{exe}
	\ex{\{\tsc{pl}\} {\ra} \ve{=ein} /C{\#}{\gap}}\label{ex:pl->-ein/C}
		\sn{\gw\begin{tabular}{rcll}
			\ve{anah}				&\ra&\ve{aanh=\tbr{ein}}				&`children'\\
			\ve{kaes mutiʔ}	&\ra&\ve{kaes muitʔ=\tbr{ein}}	&`Europeans'\\
			\ve{enoʔ}				&\ra&\ve{eonʔ=\tbr{ein}}				&`doors'\\
			\ve{tuaf}			&\ra&\ve{tuaf=\tbr{ein}}				&`people'\\
			\ve{kuan}				&\ra&\ve{kuan=\tbr{ein}}				&`villages'\\
			\ve{n-fesat}		&\ra&\ve{n-feest=\tbr{ein}}			&`(they) throw a party'\\
			\ve{na-barab}		&\ra&\ve{na-baarb=\tbr{ein}}		&`(they) prepare'\\
			\ve{n-ʔonen}		&\ra&\ve{n-ʔoenn=\tbr{ein}}			&`(they) pray'\\
			\ve{na-tuin}		&\ra&\ve{na-tuin=\tbr{ein}}			&`(they) follow'\\
%			\ve{}	&\ra&\ve{=n}	&`'\\
%			\ve{}	&\ra&\ve{=n}	&`'\\
	\end{tabular}}
\end{exe}

This enclitic also has the variant forms \ve{=uun} and \ve{=enu},
of which the form \ve{=uun} is the M\=/form of \ve{=enu}.
The expected M\=/form \ve{\tcb{*}=eun} does not occur in my data,
thus \ve{=enu} {\ra} \ve{=uun} is an irregular M\=/form (\srf{sec:IrrMfor}).
The forms \ve{=enu} and \ve{=uun} are rare in my data.
There are twelve attestations of \ve{=uun} in my corpus and three attestations of \ve{=enu}.
This is compared with 157 attestations of \ve{=ein}
and 21 attestations of \ve{=eni}.
Examples of \ve{=uun} and \ve{=enu} are given in \qf{ex:pl->-uun,enu/C} below.
The clitic hosts shown in \qf{ex:pl->-uun,enu/C} also occur with \ve{=ein}.\footnote{
		In some varieties of Amanuban the plural enclitic
		usually has the form \ve{=enu/=eun}.}

\begin{exe}
	\ex{\{\tsc{pl}\} {\ra} \ve{=uun {\tl} =enu} /C{\#}{\gap}}\label{ex:pl->-uun,enu/C}
		\sn{\gw\begin{tabular}{rcll}
			\ve{abas}				&\ra&\ve{aabs=\tbr{uun}}			&`threads'\\
			\ve{na-ʔkoroʔ}	&\ra&\ve{na-ʔkoorʔ=\tbr{uun}}	&`(they) hide'\\
			\ve{Timor}			&\ra&\ve{Tiamr=\tbr{uun}}			&`Timorese people'\\
			\ve{faif anaʔ}	&\ra&\ve{faif aanʔ=\tbr{enu}}	&`piglets'\\
			\ve{kana-k}			&\ra&\ve{kaan-k=\tbr{enu}}		&`their names'\\
	\end{tabular}}
\end{exe}

After stems which end in CV,
the plural enclitic usually takes the form \ve{=n}.
Examples are given in \qf{ex:pl->-n/CV} below.

\begin{exe}
	\ex{\{\tsc{pl}\} {\ra} \ve{=n} /CV{\#}{\gap}}\label{ex:pl->-n/CV}
		\sn{\gw\begin{tabular}{rcll}
			\ve{kase}		&\ra&\ve{kase=\tbr{n}}	&`foreigners'\\
			\ve{hutu}		&\ra&\ve{hutu=\tbr{n}}	&`head-lice'\\
			\ve{kbiti}	&\ra&\ve{kbiti=\tbr{n}}	&`scorpions'\\
			\ve{koro}		&\ra&\ve{koro=\tbr{n}}	&`birds'\\
			\ve{tuni}		&\ra&\ve{tuni=\tbr{n}}	&`eels'\\
			\ve{n-moʔe}	&\ra&\ve{n-moʔe=\tbr{n}}	&`(they) do/make'\\
			\ve{na-tona}	&\ra&\ve{na-tona=\tbr{n}}	&`(they) tell'\\
			\ve{n-eki}		&\ra&\ve{n-eki=\tbr{n}}		&`(they) bring'\\
			\ve{na-hana}	&\ra&\ve{na-hana=\tbr{n}}	&`(they) cook'\\
%			\ve{}	&\ra&\ve{=n}	&`'\\
%			\ve{}	&\ra&\ve{=n}	&`'\\
	\end{tabular}}
\end{exe}

Similarly, after verbs which end in a vowel sequence,
the plural enclitic also has the form \ve{=n}.
A number of examples are given in \qf{ex:pl->=n/VVverb} below.

\begin{exe}
	\ex{\{\tsc{pl}\} {\ra} \ve{=n} Verb, /VV{\#}{\gap}}\label{ex:pl->=n/VVverb}
		\sn{\gw\begin{tabular}{rcll}
			\ve{n-sii}		&\ra&\ve{n-sii=\tbr{n}}	&`(they) sing'\\
			\ve{n-murai}	&\ra&\ve{n-murai=\tbr{n}}	&`(they) start'\\
			\ve{n-tui}		&\ra&\ve{n-tui=\tbr{n}}	&`(they) write'\\
			\ve{n-kae}		&\ra&\ve{n-kae=\tbr{n}}	&`(they) cry'\\
			\ve{n-nao}		&\ra&\ve{n-nao=\tbr{n}}	&`(they) go'\\
			\ve{na-niu}		&\ra&\ve{na-niu=\tbr{n}}	&`(they) bathe'\\
			\ve{na-mnau}	&\ra&\ve{na-mnau=\tbr{n}}	&`(they) remember'\\
			\ve{n-poi}		&\ra&\ve{n-poi=\tbr{n}}	&`(they) exit/go out'\\
%			\ve{}	&\ra&\ve{=n}	&`'\\
%			\ve{}	&\ra&\ve{=n}	&`'\\
	\end{tabular}}
\end{exe}

When nouns which end in a vowel sequence are pluralised,
a number of different forms occur.
Firstly, there is the form \ve{=nu}
which I have encountered once as a simple plural 
during my fieldwork. This example
is given in \qf{ex:Obs06/10/14} below.

\begin{exe}
	\ex{\gll	hiit t-hormaat hau=\tbr{nu}!\\
						{\hiit} {\t}-honour tree={\ein}\\
			\glt	`We're giving honour to the trees!'
						(Joke made when ducking branches of trees while riding in the back of a truck.)
						\txrf{Observation 06/10/14}}\label{ex:Obs06/10/14}
\end{exe}

The clitic \ve{=nu} also attaches to VV{\#} final pronouns
to mark an otherwise unexpressed plural possessum.
Thus, \ve{au=nu} `mine/my things', \ve{hoo=nu} `yours/your (sg.) things',
\ve{hai=nu} `ours (excl.)/our things', \ve{hii=nu} `yours (pl.)/your things'.
See \srf{sec:PosDet} for more discussion.

However, the normal way in which VV{\#} final nouns mark plural
and the normal way VV{\#} final pronouns mark plural possessums
is with a form [ŋɡwɪn].
This is analysable as a combination of \ve{=nu} + \ve{=ein}
with insertion of /ɡw/ before the second enclitic.
Examples with are given in \qf{ex:pl->=n/VVnoun} below.\footnote{
		In the Baikeno variety of Meto the plural enclitic has the form \ve{=mbini}
		after words which end in a vowel sequence, e.g. \ve{bi{\j}ae=mbini} `cows'.
		Insertion of Baikeno /b/ also corresponds 
		to insertion of Amarasi /ɡw/ in other environments.}

\begin{exe}
	\ex{\{\tsc{pl}\} {\ra} \ve{=ŋgwein} /VV{\#}{\gap}}\label{ex:pl->=n/VVnoun}
		\sn{\gw\begin{tabular}{rcll}
			\ve{bifee}	&{\ra}&\ve{bifee=\tbr{ŋgwein}}		&`women'	\\
			\ve{bi\j ae}&{\ra}&\ve{bi\j ae=\tbr{ŋgwein}}	&`cows'	\\
			\ve{oe}			&{\ra}&\ve{oe=\tbr{ŋgwein}}				&`kinds of water'	\\
			\ve{pentua}	&{\ra}&\ve{pentua=\tbr{ŋgwein}}		&`church elders'	\\
			\ve{too}		&{\ra}&\ve{too=\tbr{ŋgwein}}			&`citizens'	\\
			\ve{hau}		&{\ra}&\ve{hau=\tbr{ŋgwein}}			&`trees'	\\
%		\end{tabular}}
%\end{exe}
%\begin{exe}
%	\ex{\{\tsc{pl}\} {\ra} \ve{=ŋgwein} /VV{\#}{\gap}}\label{ex:pl->=n/VVpronoun}
%		\sn{\gw\begin{tabular}{rcll}
			\ve{au}		&{\ra}&\ve{au=\tbr{ŋgwein}}		&`mine/my things'	\\
			\ve{hoo}	&{\ra}&\ve{hoo=\tbr{ŋgwein}}	&`yours (sg.)/your things'	\\
			\ve{hai}	&{\ra}&\ve{hai=\tbr{ŋgwein}}	&`ours (excl.)/our things'	\\
			\ve{hii}	&{\ra}&\ve{hii=\tbr{ŋgwein}}	&`yours (pl.)/your things'	\\
		\end{tabular}}
\end{exe}

The noun \ve{kfuu} `star' is an exception.
This word has the plural form \ve{kfuu=n} `stars' for some speakers.
In this case singular \ve{kfuu} `star' is a back formation,
as the final /n/ of plural \ve{kfuu=n} is a reflex
of the final consonant of Proto-Malayo-Polynesian *bituqən.
Similarly, while the loan word \ve{partei} `friend'
(from Dutch \it{partij} [partɛi]) usually has the plural
\ve{partei=ŋgwein} `friends', it has been attested once
with \ve{=n}; thus \ve{partei=n} `friends' \citep[3]{or16}.\footnote{
		Another allomorph for VV{\#} final nouns
		is \ve{=ŋgonu/=ŋgoun} which is attested
		from a single speaker, and then only on the loan \ve{oraŋ tua} `parents'
		(from Malay \it{orang tua}).
		There is one example each in my data of \ve{oraŋ tua=ŋgonu}
		and \ve{oraŋ tua=ŋgoun} `parents'.}

There are also three examples in which
a CV{\#} or C{\#} final noun takes double plural marking
with both \ve{=n} and \ve{=ein},
given in \qf{ex:130914-2, 1.17}--\qf{ex:130902-1, 3.28} below.

\begin{exe}
	\ex{\glll	feʔe n-ʔoban naan rauk=\tbr{n}=\tbr{ein}, nopu nua mes \hspace{20mm} ka= n-eku =f, n-ʔoobn=aah.\\
						feʔe n-ʔoban naan raku=\tbr{n}=\tbr{ein} nopu nua mes {} ka= n-eku =f, n-ʔoban=aah.\\
						earlier \n-furrow {\naan} sweet.potato=\tbr{\ein}=\tbr{\ein} hole two but {} {\ka}= \n-eat ={\fa} \n-furrow=just\\
			\glt	`Earlier it had dug up the sweet potatoes, there were two holes
						but it hadn't eaten anything, it just dug around.'
						\txrf{130914-2, 1.17} {\emb{130914-2-01-17.mp3}{\spk{}}{\apl}}}\label{ex:130914-2, 1.17}
	\ex{\glll	hoo m-fee areʔ kana=n hau fua-f maut \hspace{30mm} he koor=\tbr{n}=\tbr{ein} bisa n-eku=n.\\
						hoo m-fee areʔ kana=n hau fua-f maut {} he koro=\tbr{n}=\tbr{ein} bisa n-eku=n.\\
						{\hoo} \m-give every name={\ein} tree fruit-{\f} let {}
						{\he} bird=\tbr{\ein}=\tbr{\ein} can \n-eat={\einV}\\
			\glt	`You gave all kinds of fruit trees in order that all the different birds could eat.'
						\hfill{\citep[11]{or16b}}}\label{ex:BigBook}
	\ex{\glll	rari =te, n-ma-taeb n-ok ahh baroit=\tbr{n}=\tbr{eni} =ma\\
						rari =te, n-ma-tabe n-oka {} baroit=\tbr{n}=\tbr{eni} =ma\\
						finish ={\te} \n-{\mak}-shake.hands \n-{\ok} {} bride/groom=\tbr{\ein}=\tbr{\ein} =and\\
			\glt	`After that he shook hands with both the bride and groom and'
						\txrf{130902-1, 3.28} {\emb{130902-1-03-28.mp3}{\spk{}}{\apl}}}\label{ex:130902-1, 3.28}
\end{exe}

Despite the complexities in the data,
the forms of the plural enclitic(s) can
be mostly described as allomorphy,
as summarised in \trf{tab:PluEncAll}.
For verbs the analysis is a straightforward case
of phonologically conditioned allomorphy.
Vowel-final stems take \ve{=n} while
consonant-final stems take \ve{=ein/=eni},
or its variant \ve{=uun/=enu}.
For nouns the data is more complex.
Consonant-final nouns take \ve{=ein/=eni},
CV{\#} final nouns take \ve{=n},
and VV{\#} final nouns normally take
\ve{=ŋgwein} (analysable as \ve{=nu} + \ve{=ein})
but also are attested with \ve{=n} or \ve{=nu}.
Double plural marking with \ve{=n=ein} also occasionally
occurs with nouns.

The examples with double plural marking may
indicate that the different allomorphs have come from different sources
and may have once been different morphemes with different functions.
While there may be traces of these different functions
in some of the synchronic data, in most cases, and for most speakers,
they appear to have semantically merged and both mark plural.\footnote{
		The allomorph \ve{=n} may have originally marked plurals with
		an emphasis on the group as a collection of individuals,
		thus paralleling the use of the quantifier \ve{areʔ} `every, all'
		while \ve{=ein} marked plurals as a whole mass,
		thus paralleling the use of the quantifier \ve{okeʔ} `all'.}

That the vowel-initial forms of the plural enclitic \ve{=ein}
do not occur with vowel-final stems means that consonant
insertion is not usually observed before this enclitic.
There is one exception in my database:
the verb \ve{na-ʔbaʔe} `play', which has been attested once
with the plural enclitic allomorph \ve{=ein} as \ve{na-ʔbaaʔ\j=ein}.
This verb is also exceptional in not otherwise taking M\=/forms.\footnote{
		There are also three other vowel-final stems occurring with
		the enclitic \ve{=ein} in the Amarasi Bible translation:
		\ve{na-ʔtaʔi} `trembles' + \ve{=ein} {\ra} \it{<na{\Q}tai{\Q} jein>} (one example),
		\ve{koʔu} `big' + \ve{=ein} {\ra} \it{<kou{\Q} guin>} (five examples)
		and \ve{na-ʔseʔ{\tl}seʔo} `whispers' {\ra} \it{<na{\Q}se{\Q}-seo{\Q} guin>} (two examples).}

\subsection{Consonant insertion after \it{=n}}\label{sec:ConInsPluEnc}
When the \ve{=n} allomorph of the plural enclitic
attaches to a stem which ends in a vowel sequence,
any subsequent clitic triggers insertion of /ɡw/.
This is analysable as resulting from historic/underlying \ve{=nu}.

Such consonant insertion does not occur 
when \ve{=n} attaches to a CV{\#} final stem.
Instead, when a vowel-initial enclitic follows,
the host is treated like a CVC{\#} stem with regular metathesis.
Examples are given in \qf{ex:CV=n+=V->VC=n=V} below.

\begin{exe}
	\ex{CV\ve{=n} + =V {\ra} VC\ve{=n}=V}\label{ex:CV=n+=V->VC=n=V}
	\sn{\stl{0.31em}\gw\begin{tabular}{rclclcll}
				\ve{sepa\tbr{tu}}	&+&\ve{=n}&+&\ve{=ii}	&\ra&\ve{sepa\tbr{ut}=n=ii}	&`the shoes'\\
				\ve{hu\tbr{tu}}		&+&\ve{=n}&+&\ve{=aan}	&\ra&\ve{hu\tbr{ut}=n=aan}	&`the head-lice'\\
				\ve{ka\tbr{se}}		&+&\ve{=n}&+&\ve{=ee}	&\ra&\ve{ka\tbr{es}=n=ee}	&`the foreigners'\\
				\ve{kbi\tbr{ti}}	&+&\ve{=n}&+&\ve{=ee}	&\ra&\ve{kbi\tbr{it}=n=ee}	&`the scorpions'\\
				\ve{ko\tbr{ro}}		&+&\ve{=n}&+&\ve{=ee}	&\ra&\ve{ko\tbr{or}=n=ee}	&`the birds'\\
				\ve{fa\tbr{fi}}		&+&\ve{=n}&+&\ve{=ee}	&\ra&\ve{fa\tbr{if}=n=ee}	&`the pigs'\\
				\ve{n-to\tbr{ti}}	&+&\ve{=n}&+&\ve{=aah}	&\ra&\ve{n-to\tbr{it}=n=aah}	&`(they) just ask'\\
				\ve{n-he\tbr{ra}}	&+&\ve{=n}&+&\ve{=ee}	&\ra&\ve{n-he\tbr{er}=n=ee}	&`(they) pull it'\\
				\ve{n-fa\tbr{ni}}	&+&\ve{=n}&+&\ve{=een}	&\ra&\ve{n-fa\tbr{in}=n=een}	&`(they've) now returned'\\
				\ve{na-hi\tbr{ni}}&+&\ve{=n}&+&\ve{=ii}	&\ra&\ve{na-hi\tbr{in}=n=een}	&`(they) now know'\\
		%		\ve{}	&+&\ve{=n}&+&\ve{=ii}	&\ra&\ve{}=n=}	&`'\\
		%		\ve{}	&+&\ve{=n}&+&\ve{=ii}	&\ra&\ve{}=n=}	&`'\\
	\end{tabular}}
\end{exe}

\largerpage
As discussed above, the regular allomorph of
the plural enclitic on VV{\#} verbs is \ve{=n}.
When either of the enclitics \ve{=een} or \ve{=aah}
follows, /ɡw/ usually occurs before the second enclitic.
Examples are given in \qf{ex:VV=ngw=een} below.

\newpage
\begin{exe}
	\ex{VV + \ve{=n} + =V {\ra} VV\ve{=ŋgw}V}\label{ex:VV=ngw=een}
		\sn{\stl{0.2em}\gw\begin{tabular}{rclclcll}
			\ve{n-sii}	&+&\ve{=n}&+&\ve{=een}&{\ra}&\ve{n-sii=ŋ\tbr{gw}=een}&`(they've) now sung'\\
			\ve{n-murai}&+&\ve{=n}&+&\ve{=een}&{\ra}&\ve{n-murai=ŋ\tbr{gw}=een}&`(they've) now started'\\
			\ve{n-tui}	&+&\ve{=n}&+&\ve{=een}&{\ra}&\ve{n-tui=ŋ\tbr{gw}=een}&`(they've) now written'\\
			\ve{n-kae}	&+&\ve{=n}&+&\ve{=een}&{\ra}&\ve{n-kae=ŋ\tbr{gw}=een}&`(they've) now cried'\\
			\ve{n-tea}	&+&\ve{=n}&+&\ve{=een}&{\ra}&\ve{n-tea=ŋ\tbr{gw}=een}&`(they've) now arrived'\\
			\ve{na-bua}	&+&\ve{=n}&+&\ve{=een}&{\ra}&\ve{na-bua=ŋ\tbr{gw}=een}&`(they've) now gathered'\\
			\ve{n-nao}	&+&\ve{=n}&+&\ve{=een}&{\ra}&\ve{n-nao=ŋ\tbr{gw}=een}&`(they've) now gone'\\
			\ve{na-niu}	&+&\ve{=n}&+&\ve{=een}&{\ra}&\ve{na-niu=ŋ\tbr{gw}=een}&`(they've) now bathed'\\
%			\ve{na-mnau}&+&\ve{=n}&+&\ve{=een}&{\ra}&\ve{na-mnau=ŋ\tbr{gw}=een}&`(they've) now remembered'\\
			\ve{n-poi}	&+&\ve{=n}&+&\ve{=aah}&{\ra}&\ve{n-poi=ŋ\tbr{gw}=aah}&`(they) just went out'\\
		\end{tabular}}
\end{exe}

Such insertion of /ɡw/ does not occur for verbs before other enclitics.
There are five examples in my corpus,
two with the discourse marker \ve{=ii}
and three with the \tsc{3sg.acc} pronoun \ve{=ee}.
These examples are given in \qf{ex:VV+-n+=ee/i/a->VV-n=ee/i/a} below.\footnote{
		There is also one example in my corpus of \ve{=n} + \ve{=ein} on a VV{\#}
		final host without consonant insertion: \ve{n-tea=n=ein} `they've arrived'.}

\begin{exe}
	\ex{VV + =n + =ee/=ii {\ra} VV=n=ee/=ii \label{ex:VV+-n+=ee/i/a->VV-n=ee/i/a}}
		\sn{\gw\begin{tabular}{lll}
			\ve{na-ʔ-rau=n}				&+ \ve{=ii} {\ra}			&\ve{na-ʔ-rau=n=ii}	\\ \hhline{~}
			{\na-\qV-bite=\einV}	&\hp{+ }{\ii}					&`made these ones bite'\\
			\ve{m-foe{\tl}foe=n}	&+ \ve{=ii} {\ra}			&\ve{m-foe{\tl}foe=n=ii}\\ \hhline{~}
			{\m-{\frd}move=\einV}	&\hp{+ }{\ii}					&`(we've) worked hard'\\
			\ve{n-nao=n}					&+ \ve{=ee} {\ra}			&\ve{n-nao=n=ee}\\ \hhline{~}
			{\n-go=\einV}					&\hp{+ }{\eeV}				&`went to him/her'\\
			\ve{n-sae=n}					&+ \ve{=ee} {\ra}			&\ve{n-sae=n=ee}\\ \hhline{~}
			{\n-go.up=\einV}			&\hp{+ }{\eeV}				&`increased for him/her'\\
			\ve{t-fee=n}					&+ \ve{=ee=siin} {\ra}&\ve{t-fee=n=ee=siin}\\ \hhline{~}
			{\t-give	=\einV}			&\hp{+ }{\eeV}={\siin}&`gave it to them'\\
		\end{tabular}}
\end{exe}

\largerpage
When a combination of a plural enclitic
and a vowel-initial enclitic occur on a VV{\#} noun,
the plural enclitic takes the form \ve{=n}
with insertion of /ɡw/ before the second enclitic.
For nouns, this includes enclitics other than
\ve{=een} and \ve{=aah}.
Examples are given in \qf{ex:VV=ngw=ee} below.

\begin{exe}
	\ex{VV + =n + =V {\ra} VV=ŋgw=V}\label{ex:VV=ngw=ee}
		\sn{\gw\begin{tabular}{rlllll}
			\ve{oe=n}		&+&\ve{=aan}	&{\ra}&\ve{oe=ŋ\tbr{gw}=aan}&`the kinds of water'\\
			\ve{mei=n}	&+&\ve{=ee}	&{\ra}&\ve{mei=ŋ\tbr{gw}=ee}&`the tables'\\
			\ve{too=n}	&+&\ve{=ii}	&{\ra}&\ve{too=ŋ\tbr{gw}=ii}&`the citizens'\\
%			\ve{=n}	&+&\ve{=ii}	&{\ra}&\ve{=n=gw}&`the citizens'\\
		\end{tabular}}
\end{exe}

\newpage
Examples of \ve{=n} and another enclitic
are judged as ungrammatical without insertion of /ɡw/.
Two examples are \ve{\tcb{*}n-sii=n=een} `sung' and \ve{\tcb{*}n-kae=n=een} `cried'.
This creates near-minimal pairs between forms in which a final /n/
is part of the root and ones in which it is the plural enclitic.
Thus, \ve{n-sii=n} + \ve{=een} {\ra} \ve{n-sii=ŋgw=een} `they've sung'
can be compared with \ve{n-pina} + \ve{=een} {\ra} \ve{n-piin=een} `blazed'.

Similarly, among nouns, insertion of /ɡw/ occurs after plural \ve{=n},
but not after the \tsc{3sg.gen} suffix \ve{-n}.
Thus \ve{too=ŋgw=ee} `citizen={\ein}={\ee}' (`the citizens')
with insertion of /ɡw/ can be compared
with \ve{ao-n=ee} `body-{\N}={\ee}' (`someone's body') without insertion.

When the \ve{=n} allomorph of the plural enclitic
attaches to a stem which ends in a vowel sequence,
any subsequent clitic usually triggers insertion of /ɡw/.
This is regular for nouns before all vowel-initial
enclitics and regular for verbs before \ve{=een} `{\een}'
and \ve{=aah} `just'.
