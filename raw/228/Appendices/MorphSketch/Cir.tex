\section{Circumfixes}\label{sec:Cir ch:Pho}

\subsection{Nominalising \it{a-{\ldots}-t}}\label{sec:NomA--t}
The nominalising circumfix \ve{a-{\ldots}-t} (Ro{\Q}is Amarasi has \ve{ka-{\ldots}-t})
has the allomorph \ve{a-{\ldots}-s} on stems which contain a /t/.
This circumfix typically derives nouns referring to people
who carry out or who are characterised by the event/state encoded by the root.\footnote{
		The root \ve{{\rt}reko} `good' has irregular semantics:
		with \ve{a-reko-t} `very good, best'.}
When the root to which it attaches ends in a consonant,
the suffixal part of this circumfix does not surface.
Examples are given in \qf{ex:NomCira-...t} below.

\begin{exe}
	\ex{Nominalising circumfix \ve{a-{\ldots}-t}}\label{ex:NomCira-...t}
	\sn{\gw\stl{0.4em}\begin{tabular}{lllll}
		`work'	&\ve{{\rt}mepu} &+ \ve{a-{\ldots}-t} \ra& \ve{a-mepu-t} & `worker'\\
		`read'	&\ve{{\rt}resa} &+ \ve{a-{\ldots}-t} \ra& \ve{a-resa-t} & `reader'\\
		`stand'	&\ve{{\rt}hake} &+ \ve{a-{\ldots}-t} \ra& \ve{a-hake-t} & `one who stands'\\
		`invite'&\ve{{\rt}skau} &+ \ve{a-{\ldots}-t} \ra& \ve{a-skau-t} & `inviter'\\
		`pray'	&\ve{{\rt}ʔonen}&+ \ve{a-{\ldots}-t} \ra& \ve{a-ʔonen} & `one who prays'\\
		`agape, 				&\ve{{\rt}tafiʔ} &+ \ve{a-{\ldots}-t} \ra& \ve{a-tafiʔ} & `one who is agape, \\ \hhline{~}
\hphantom{`}random'	&									&  									 		&								& \hphantom{`}does things randomly'\\
		`die'		&\ve{{\rt}mate} &+ \ve{a-{\ldots}-s} \ra& \ve{a-mate-s} & `dead one'\\
		`parallel'&\ve{{\rt}tnoe} &+ \ve{a-{\ldots}-s} \ra& \ve{a-tnoe-s} & `one(s) sitting opposite'\\
		`sleep'	&\ve{{\rt}tupa} &+ \ve{a-{\ldots}-s} \ra& \ve{a-tupa-s} & `sleeping one'\\
		\end{tabular}}
\end{exe}

One root in which the final consonant is apparently replaced
by the suffixal element of \ve{a-{\ldots}-t} is \ve{munif} `young'
{\ra} \ve{a-muni-t} `youngest one'.
This is due to the final /f/ of \ve{munif} `young'
being a fossilised suffix, in this case the \tsc{0gen} suffix \ve{-f}.

When a vowel-initial root or the monosyllabic root \ve{{\rt}ha} `eat (soft food)'
is nominalised with \ve{a-\ldots-t},
the stem consists of the stative prefix \ve{m-} (\srf{sec:StaPre})
attached to the third person form of the verb.
Examples are given in \qf{ex:a-...-t} below.\footnote{
		Evidence that it is the third person from of the
		verb to which the stative prefix attaches rather
		than having an allomorph \ve{amn-} comes from the
		partially suppletive verb \ve{{\rt}Vma} `come'
		with the third person form \ve{nema}
		and derivation \ve{a-m-nema-t} `one who comes, origin'.}

\begin{exe}
	\ex{Nominalising circumfix \ve{a-{\ldots}-t}}\label{ex:a-...-t}
	\sn{\stl{0.3em}\gw\begin{tabular}{lllll}
		`run'		& \ve{{\rt}aena}&+ \ve{a-{\ldots}-t} \ra& \ve{a-m-n-aena-t} &`runner'\\
		`search'& \ve{{\rt}ami}	&+ \ve{a-{\ldots}-t} \ra& \ve{a-m-n-ami-t}	&`one who searches'\\
		`eat'		& \ve{{\rt}eku}	&+ \ve{a-{\ldots}-t} \ra& \ve{a-m-n-ekut-t} &`eater (of hard food)'\\
		`eat' 	& \ve{{\rt}ha}	&+ \ve{a-{\ldots}-t} \ra& \ve{a-m-na-ha-t}	&`eater (of soft food)'\\
		`come'	& \ve{{\rt}Vma}	&+ \ve{a-{\ldots}-t} \ra& \ve{a-m-nema-t}		&`one who comes, origin’\\
		%`' & \ve{{\rt}} &+ \ve{a-{\ldots}-t} \ra& \ve{a-m-n--t} &`'\\
	\end{tabular}}
\end{exe}

In this case the third person prefix usually fills the first C-slot
of the foot while the prefix combination \ve{a-m-} form a syllable
separate to the foot. The prosodic and morphological structures
of \ve{amnamit} `one who searches', \ve{amnahat} `eater',
and \ve{amnemat} `one who comes, origin' are shown in
\qf{as:amnamit}--\qf{as:amnemat} below for comparison.

\begin{multicols}{3}
	\begin{exe}
		\exa{\label{as:amnamit}\xy
				<3.6em,5.7cm>*\as{PrWd}="PrWd",<5.4em,4.7cm>*\as{Ft}="ft1",
				<1.8em,3.7cm>*\as{σ}="s1",<4.5em,3.7cm>*\as{σ}="s2",<6.3em,3.7cm>*\as{σ}="s3",
				<0.9em,2.7cm>*\as{C}="CV1",<1.8em,2.7cm>*\as{V}="CV2",<2.7em,2.7cm>*\as{C}="CV3",<3.6em,2.7cm>*\as{C}="CV4",
				<4.5em,2.7cm>*\as{V}="CV5",<5.4em,2.7cm>*\as{C}="CV6",<6.3em,2.7cm>*\as{V}="CV7",<7.2em,2.7cm>*\as{C}="CV8",
				<0.9em,1.7cm>*\as{}="cv1",<1.8em,1.7cm>*\as{a}="cv2",<2.7em,1.7cm>*\as{m}="cv3",<3.6em,1.7cm>*\as{n}="cv4",
				<4.5em,1.7cm>*\as{a}="cv5",<5.4em,1.7cm>*\as{m}="cv6",<6.3em,1.7cm>*\as{i}="cv7",<7.2em,1.7cm>*\as{t}="cv8",
				<2.7em,0.8cm>*\as{M}="m1",<3.6em,0.8cm>*\as{M}="m2",<5.4em,0.8cm>*\as{M}="m3",<4.5em,0cm>*\as{M}="m4",
				<1.8em,0.4cm>*\as{}="m0",<7.2em,0.4cm>*\as{}="m00",
				"m1"+U;"cv3"+D**\dir{-};"m2"+U;"cv4"+D**\dir{-};"m3"+U;"cv5"+D**\dir{-};"m3"+U;"cv6"+D**\dir{-};"m3"+U;"cv7"+D**\dir{-};
				"m4"+U;"m0"+U**\dir{-};"m4"+U;"m00"+U**\dir{-};"m0"+U;"cv2"+D**\dir{-};"m00"+U;"cv8"+D**\dir{-};
				"cv2"+U;"CV2"+D**\dir{-};"cv3"+U;"CV3"+D**\dir{-};"cv4"+U;"CV4"+D**\dir{-};
				"cv5"+U;"CV5"+D**\dir{-};"cv6"+U;"CV6"+D**\dir{-};"cv7"+U;"CV7"+D**\dir{-};"cv8"+U;"CV8"+D**\dir{-};
				"CV1"+U;"s1"+D**\dir{-};"CV2"+U;"s1"+D**\dir{-};"CV3"+U;"s1"+D**\dir{-};
				"CV4"+U;"s2"+D**\dir{-};"CV5"+U;"s2"+D**\dir{-};"CV6"+U;"s2"+D**\dir{-};
				"CV6"+U;"s3"+D**\dir{-};"CV7"+U;"s3"+D**\dir{-};"CV8"+U;"s3"+D**\dir{-};
				"s1"+U;"PrWd"+D**\dir{-};"s2"+U;"ft1"+D**\dir{-};"s3"+U;"ft1"+D**\dir{-};
				"ft1"+U;"PrWd"+D**\dir{-};
		\endxy}
		\exa{\label{as:amnahat}\xy
				<3.6em,5.7cm>*\as{PrWd}="PrWd",<5.4em,4.7cm>*\as{Ft}="ft1",
				<1.8em,3.7cm>*\as{σ}="s1",<4.5em,3.7cm>*\as{σ}="s2",<6.3em,3.7cm>*\as{σ}="s3",
				<0.9em,2.7cm>*\as{C}="CV1",<1.8em,2.7cm>*\as{V}="CV2",<2.7em,2.7cm>*\as{C}="CV3",<3.6em,2.7cm>*\as{C}="CV4",
				<4.5em,2.7cm>*\as{V}="CV5",<5.4em,2.7cm>*\as{C}="CV6",<6.3em,2.7cm>*\as{V}="CV7",<7.2em,2.7cm>*\as{C}="CV8",
				<0.9em,1.7cm>*\as{}="cv1",<1.8em,1.7cm>*\as{a}="cv2",<2.7em,1.7cm>*\as{m}="cv3",<3.6em,1.7cm>*\as{n}="cv4",
				<4.5em,1.7cm>*\as{a}="cv5",<5.4em,1.7cm>*\as{h}="cv6",<6.3em,1.7cm>*\as{a}="cv7",<7.2em,1.7cm>*\as{t}="cv8",
				<2.7em,0.8cm>*\as{M}="m1",<4.05em,0.8cm>*\as{M}="m2",<5.85em,0.8cm>*\as{M}="m3",<4.5em,0cm>*\as{M}="m4",
				<1.8em,0.4cm>*\as{}="m0",<7.2em,0.4cm>*\as{}="m00",
				"m1"+U;"cv3"+D**\dir{-};"m2"+U;"cv4"+D**\dir{-};"m2"+U;"cv5"+D**\dir{-};"m3"+U;"cv6"+D**\dir{-};"m3"+U;"cv7"+D**\dir{-};
				"m4"+U;"m0"+U**\dir{-};"m4"+U;"m00"+U**\dir{-};"m0"+U;"cv2"+D**\dir{-};"m00"+U;"cv8"+D**\dir{-};
				"cv2"+U;"CV2"+D**\dir{-};"cv3"+U;"CV3"+D**\dir{-};"cv4"+U;"CV4"+D**\dir{-};
				"cv5"+U;"CV5"+D**\dir{-};"cv6"+U;"CV6"+D**\dir{-};"cv7"+U;"CV7"+D**\dir{-};"cv8"+U;"CV8"+D**\dir{-};
				"CV1"+U;"s1"+D**\dir{-};"CV2"+U;"s1"+D**\dir{-};"CV3"+U;"s1"+D**\dir{-};
				"CV4"+U;"s2"+D**\dir{-};"CV5"+U;"s2"+D**\dir{-};"CV6"+U;"s2"+D**\dir{-};
				"CV6"+U;"s3"+D**\dir{-};"CV7"+U;"s3"+D**\dir{-};"CV8"+U;"s3"+D**\dir{-};
				"s1"+U;"PrWd"+D**\dir{-};"s2"+U;"ft1"+D**\dir{-};"s3"+U;"ft1"+D**\dir{-};
				"ft1"+U;"PrWd"+D**\dir{-};
		\endxy}
		\exa{\label{as:amnemat}\xy
				<3.6em,5.7cm>*\as{PrWd}="PrWd",<5.4em,4.7cm>*\as{Ft}="ft1",
				<1.8em,3.7cm>*\as{σ}="s1",<4.5em,3.7cm>*\as{σ}="s2",<6.3em,3.7cm>*\as{σ}="s3",
				<0.9em,2.7cm>*\as{C}="CV1",<1.8em,2.7cm>*\as{V}="CV2",<2.7em,2.7cm>*\as{C}="CV3",<3.6em,2.7cm>*\as{C}="CV4",
				<4.5em,2.7cm>*\as{V}="CV5",<5.4em,2.7cm>*\as{C}="CV6",<6.3em,2.7cm>*\as{V}="CV7",<7.2em,2.7cm>*\as{C}="CV8",
				<0.9em,1.7cm>*\as{}="cv1",<1.8em,1.7cm>*\as{a}="cv2",<2.7em,1.7cm>*\as{m}="cv3",<3.6em,1.7cm>*\as{n}="cv4",
				<4.5em,1.7cm>*\as{e}="cv5",<5.4em,1.7cm>*\as{m}="cv6",<6.3em,1.7cm>*\as{a}="cv7",<7.2em,1.7cm>*\as{t}="cv8",
				<2.7em,0.8cm>*\as{M}="m1",<4.05em,0.8cm>*\as{M}="m2",<5.4em,0.8cm>*\as{M}="m3",<4.5em,0cm>*\as{M}="m4",
				<1.8em,0.4cm>*\as{}="m0",<7.2em,0.4cm>*\as{}="m00",
				"m1"+U;"cv3"+D**\dir{-};"m2"+U;"cv4"+D**\dir{-};"m2"+U;"cv5"+D**\dir{-};
				"m3"+U;"cv5"+D**\dir{-};"m3"+U;"cv6"+D**\dir{-};"m3"+U;"cv7"+D**\dir{-};
				"m4"+U;"m0"+U**\dir{-};"m4"+U;"m00"+U**\dir{-};"m0"+U;"cv2"+D**\dir{-};"m00"+U;"cv8"+D**\dir{-};
				"cv2"+U;"CV2"+D**\dir{-};"cv3"+U;"CV3"+D**\dir{-};"cv4"+U;"CV4"+D**\dir{-};
				"cv5"+U;"CV5"+D**\dir{-};"cv6"+U;"CV6"+D**\dir{-};"cv7"+U;"CV7"+D**\dir{-};"cv8"+U;"CV8"+D**\dir{-};
				"CV1"+U;"s1"+D**\dir{-};"CV2"+U;"s1"+D**\dir{-};"CV3"+U;"s1"+D**\dir{-};
				"CV4"+U;"s2"+D**\dir{-};"CV5"+U;"s2"+D**\dir{-};"CV6"+U;"s2"+D**\dir{-};
				"CV6"+U;"s3"+D**\dir{-};"CV7"+U;"s3"+D**\dir{-};"CV8"+U;"s3"+D**\dir{-};
				"s1"+U;"PrWd"+D**\dir{-};"s2"+U;"ft1"+D**\dir{-};"s3"+U;"ft1"+D**\dir{-};
				"ft1"+U;"PrWd"+D**\dir{-};
		\endxy}
	\end{exe}
\end{multicols}

\subsection{Property \it{ma-{\ldots}-ʔ}}\label{sec:PropCir}
Amarasi \ve{ma-{\ldots}-ʔ} attaches to verbal and nominal roots to form property nouns.
For nominal roots, the new word typically describes particular
characterisation by the presence of the root noun,
while for verbs it typically describes the resulting state of the verb.
When the stem to which this circumfix attaches ends in a vowel sequence,
the final glottal stop occurs as an infix between these two vowels.
Examples of \ve{ma-{\ldots}-ʔ} are given in \qf{ex:AdjPreMa} below.

\newpage
\begin{exe}
	\ex{Property circumfix \ve{ma-{\ldots}-ʔ}\label{ex:AdjPreMa}}
	\sn{\gw\begin{tabular}{lllll}
		`rock, stone'&\ve{\hp{\rt}fatu}	&+ \ve{ma-{\ldots}-ʔ} \ra& \ve{ma-fatu-ʔ}		& `rocky, stony'\\
		`hair'			&\ve{\hp{\rt}funu-f}	&+ \ve{ma-{\ldots}-ʔ} \ra& \ve{ma-funu-ʔ}	& `hairy'\\
%		`betel nut'	&\ve{\hp{\rt}puah}	&+ \ve{ma-{\ldots}-ʔ} \ra& \ve{ma-pua-ʔ} 		& `exchanging betel nut'\footnotemark\\
		`wing'			&\ve{\hp{\rt}niniʔ}&+ \ve{ma-{\ldots}-ʔ} \ra& \ve{ma-nini-ʔ}		& `winged'\\
		`key'				&\ve{\hp{\rt}retuʔ}&+ \ve{ma-{\ldots}-ʔ} \ra& \ve{ma-retu-ʔ}		& `locked'\\
		`thorn'			&\ve{\hp{\rt}aikaʔ}&+ \ve{ma-{\ldots}-ʔ} \ra& \ve{ma-ʔaika-ʔ}		& `thorny'\\
		`price'			&\ve{\hp{\rt}osa-f}	&+ \ve{ma-{\ldots}-ʔ} \ra& \ve{ma-ʔosa-ʔ}		& `expensive,\\ \hhline{~}
								&										&												 &									& \hp{`}valuable'\\
		`hear'			&\ve{		{\rt}nena}	&+ \ve{ma-{\ldots}-ʔ} \ra& \ve{ma-nena-ʔ}		& `heard'\\
		`call, name'&\ve{		{\rt}teka}	&+ \ve{ma-{\ldots}-ʔ} \ra& \ve{ma-teka-ʔ}		& `famous,\\ \hhline{~}
								&										&												 &									& \hp{`}well known'\\
		`receive'		&\ve{		{\rt}topu}	&+ \ve{ma-{\ldots}-ʔ} \ra& \ve{ma-topu-ʔ} 	& `received'\\
		`write'			&\ve{		{\rt}tui}		&+ \ve{ma-{\ldots}-ʔ} \ra& \ve{ma-tu\<ʔ\>i} & `written'\\
		`be aware'	&\ve{		{\rt}keo}		&+ \ve{ma-{\ldots}-ʔ} \ra& \ve{ma-ke\<ʔ\>o} & `aware'\\
	\end{tabular}}
\end{exe}

When \ve{ma-{\ldots}-ʔ} attaches to a consonant-final root,
the final glottal stop of this circumfix appears
to replace any root final consonants, though there
is only one clear putative example: \ve{puah} `betel-nut'
{\ra} \ve{ma-pua-ʔ} `exchanging betel-nut'.
This example may involve the reciprocal prefix \ve{ma-} (\srf{sec:RecPre}).\footnote{
		Culturally, betel-nut is chewed by all parties before
		any social gathering. Thus, \ve{ma-pua-ʔ} `exchanging betel-nut' 
		is used metaphorically to mean `preface, prelude, introduction'.}

When \ve{ma-{\ldots}-ʔ} attaches to a vowel-initial root,
the stem consists of the stative prefix \ve{m-} (\srf{sec:StaPre})
attached to the third person form of the verb.
However, there is only one clear example in my corpus:
\ve{{\rt}ita} `see' {\ra} \ve{ma-m-n-ita-ʔ} `visible'.

There are also 36 property nominals in my corpus which
begin with /ma/ but which have no corresponding root without initial /ma/.
Many, but not all, of these forms also end in /ʔ/.
Examples are given in \qf{ex:FroAdjPre}.

\begin{exe}
	\ex{Frozen property prefix}\label{ex:FroAdjPre}
	\sn{\gw\begin{tabular}{ll|ll}
		\ve{mahataʔ}		& `itchy'	&\ve{mainikin}		& `cold'	\\
		\ve{maʔsenoʔ}	& `spicy'	&\ve{maʔfenaʔ}	& `heavy'\\
		\ve{makoe}				& `diligent'&\ve{masʔekiʔ}	& `slippery'\\
%		\ve{maʔfenaʔ}	& `heavy'\\
%		\ve{makoe}			& `diligent'\\
%		\ve{masʔekiʔ}	& `slippery'\\
	\end{tabular}}
\end{exe}

There are also a number of property nominals with initial /ma/
which \it{do} have a corresponding root,
but either the property nominal or root has undergone semantic shift
such that the semantic link between the two is no longer regular.
One such example is the nominal \ve{maputuʔ} `hot'
which is connected with \ve{putuʔ} `charcoal'.

Of such /ma/ initial nominals there are eleven which have a corresponding 
semantically related causative verb in which the initial /ma/ is replaced by /ha/,
a reflex of the Proto-Malayo-Polynesian causative prefix *pa-.
These eleven words are given in \trf{tab:AdjCorHaIniCau} below.
Some of the resulting verbs also take the transitive suffix \ve{-b} (\srf{sec:TraSuf}).

\begin{table}[h]
	\caption[Property nominals with \it{ha} initial causatives]
					{Property nominals with \ve{ha} initial causatives\su{†}}\label{tab:AdjCorHaIniCau}
	\centering
		\begin{threeparttable}[b]
		\begin{tabular}{llll} \lsptoprule
			\mc{2}{l}{Nominal} & \mc{2}{l}{Causative Verb} \\ \midrule
			\ve{maiʔnisin}	& `repugnated by' & \ve{n-haiʔnisi}		& `repels s.o.' \\
			\ve{maiʔnura} 	& `feeble' 				& \ve{n-haiʔnura}		& `enfeeble' \\
			\ve{mainikin} 	& `cold' 					& \ve{n-hainiki} 		& `cool down (tr.)' \\
			\ve{mainuan} 		& `open' 					& \ve{n-hainua-b} 	& `make open' \\
			\ve{maʔekiʔ}		& `fine, smooth' 	& \ve{n-haʔeki} 		& `smoothen' \\
			\ve{maʔfenaʔ}		& `heavy' 				& \ve{n-haʔfena-b}	& `weigh down' \\
			\ve{maʔkafaʔ}		& `light' 				& \ve{n-haʔkafa}		& `lighten' \\
			\ve{maʔtaniʔ}		& `excessive, earnest'	& \ve{n-haʔtani} & `strengthen, motivate' \\
			\ve{maputuʔ}		& `hot' 					& \ve{n-haputu} 		& `heat up (tr.)' \\
			\ve{marine} 		& `happy' 				& \ve{n-harine-b} 	& `make happy' \\
			\ve{maʔmuʔi} 		& `poor' 					& \ve{n-haʔmuʔi}		& `cause difficulty'\su{‡} \\ \lspbottomrule
		\end{tabular}
		\begin{tablenotes}
			\item [†] When verbs are derived from nominals,
								any root final consonant is deleted.
								This is a regular process in Amarasi
								and is described in more detail in \srf{sec:BasVerDer}.
			\item[‡]	The putative (historic) root of both these forms is \ve{{\rt}muʔi} `have, own'.
								This root may be the source of Amarasi \ve{muʔit} `animal',
							  which would be a regular nominalisation of \ve{{\rt}muʔi} (\srf{sec:Nom-t}).
								The word \ve{muʔit} refers to animals which are domesticated,
								or have the potential to be domesticated.
								The possession of livestock is a sign of wealth in Timor
								and provides a highly plausible semantic pathway between \ve{{\rt}muʔi} `have'
								and \ve{muʔit} `(domestic) animal'.
								However, the existence of the Helong words \it{hmukit} `(domestic) animal'
								and Dhao \it{mukit} `animal' may caution against establishing
								Amarasi \ve{muʔit} `animal' as being historically derived from \ve{{\rt}muʔi} `have, own'.
								%though speakers \emph{do} report the connection as a folk etymology.
		\end{tablenotes}
	\end{threeparttable}
\end{table}

This alternation is no longer productive.
For most property nominals which begin with /ma/,
a corresponding causative verb can be derived through
addition of the transitive suffix \ve{-b} (\srf{sec:TraSuf}).
Two examples are \ve{makoe} `diligent' {\ra} \ve{n-makoe-b} `makes s.o. diligent'
and \ve{mainukiʔ} `young' {\ra} \ve{n-mainuki-b} `makes s.o. young'.

\subsection{Nominalising \it{ʔ-{\ldots}-ʔ}}\label{sec:NomQ--q}
The circumfix \ve{ʔ-{\ldots}-ʔ}
typically derives nouns which refer to physical objects, often tools, from verbs.
When this circumfix attaches to a surface CVCV root,
the initial element occurs as a prefix and the second element as a suffix.
Examples are given in \qf{ex:NomCir2} below.

\begin{exe}
	\ex{Nominalising circumfix \ve{ʔ-{\ldots}-ʔ}}\label{ex:NomCir2}
	\sn{\gw\begin{tabular}{llcccll}
			`grate'	&\ve{{\rt}fona} &+&\ve{ʔ{\ldots}ʔ}&\ra& \ve{ʔ-fona-ʔ} &`grater'\\
			`bind' 	&\ve{{\rt}futu} &+&\ve{ʔ{\ldots}ʔ}&\ra& \ve{ʔ-futu-ʔ} &`cloth band'\\
			`sit' 	&\ve{{\rt}toko} &+&\ve{ʔ{\ldots}ʔ}&\ra& \ve{ʔ-toko-ʔ} &`chair'\\
			`sweep' &\ve{{\rt}sapu} &+&\ve{ʔ{\ldots}ʔ}&\ra& \ve{ʔ-sapu-ʔ} &`broom'\\
		\end{tabular}}
\end{exe}

When this circumfix occurs on a root with a final vowel sequence,
the second glottal stop occurs between these two vowels as an infix.
Examples are given in \qf{ex:NomCirInf2} below.
The behaviour of this circumfix when it attaches to a surface
CVCVC or CVVC root is currently unknown due to my non-exhaustive data.\footnote{
		Comparison with the property circumfix \ve{ma-{\ldots}-ʔ} (\srf{sec:PropCir})
		indicates that the suffixal element of this nominaliser would probably
		replace the final consonant of such roots.}

\begin{exe}
	\ex{Nominalising circum-/infix \ve{ʔ-{\ldots}\<ʔ\>}}\label{ex:NomCirInf2}
	\sn{\gw\begin{tabular}{llcccll}
			`cover'	&\ve{{\rt}neo} &+&\ve{ʔ{\ldots}ʔ}&\ra& \ve{ʔ-ne\<ʔ\>o} & `umbrella'\\
			`pound'	&\ve{{\rt}pau} &+&\ve{ʔ{\ldots}ʔ}&\ra& \ve{ʔ-pa\<ʔ\>u} & `mortar and pestle'\\
			`exit'	&\ve{{\rt}poi} &+&\ve{ʔ{\ldots}ʔ}&\ra& \ve{ʔ-po\<ʔ\>i} & `exit (noun)'\\
			`sing'	&\ve{{\rt}sii} &+&\ve{ʔ{\ldots}ʔ}&\ra& \ve{ʔ-si\<ʔ\>i} & `song'\\
			`write'	&\ve{{\rt}tui} &+&\ve{ʔ{\ldots}ʔ}&\ra& \ve{ʔ-tu\<ʔ\>i} & `pen, pencil'\\
	\end{tabular}}
\end{exe}

In addition to productive uses, there are 50 nominal roots
in my corpus which appear to have a fossil of this circumfix attached.
Of these roots, 70{\%} (35/50) refer to physical entities which are of a size or shape 
such that they could be held in one hand, such as tools, containers, or fruit.
Examples of roots which appear to have a fossil of \ve{ʔ-{\ldots}-ʔ} are given in \qf{ex:RooFosNomCir} below.

\begin{exe}
	\ex{Roots with putative fossil of the nominalising circumfix \ve{ʔ-{\ldots}-ʔ}}\label{ex:RooFosNomCir}
	\sn{\gw\begin{tabular}{ll|ll}
				\ve{ʔfaneʔ} & `hammer', `bowl'&\ve{ʔmukiʔ}	& `lime (fruit)' \\
				\ve{ʔfiʔu} 	& `sling' 				&\ve{ʔnisaʔ}	& `gewang palm seed' \\
				\ve{ʔkaroʔ} & `sack' 					&\ve{ʔsoʔo}		& `rice planting tool' \\
%				\ve{ʔfaneʔ} & `hammer', `bowl'	\\
%				\ve{ʔfiʔu} 	& `sling' 					\\
%				\ve{ʔkaroʔ} & `sack' 						\\
%				\ve{ʔmukiʔ} & `lime (fruit)' \\
%				\ve{ʔnisaʔ} & `gewang palm seed' \\
%				\ve{ʔsoʔo} & `rice planting tool' \\
		\end{tabular}}
\end{exe}

\subsection{Stative \it{m-{\ldots}-ʔ}}\label{sec:StaPre}
The stative circumfix \ve{m-{\ldots}-ʔ} does not appear to be very productive in Amarasi,
though the prefixal element \ve{m-} co-occurs regularly with the nominalising circumfix
\ve{a-{\ldots}-t} (\srf{sec:NomA--t}) and property circumfix \ve{ma-{\ldots}-ʔ} (\srf{sec:PropCir})
when these circumfixes attach to vowel-initial stems.
Stative \ve{m-} attaches to verbal roots and derives forms for which
the subject has a more patient-like semantic role.
Examples of statives derived with this affix are given in \qf{ex:StaVerM-} below.

\begin{exe}
	\ex{Stative verbs with \ve{m-}}\label{ex:StaVerM-}
	\sn{\gw\stl{0.27em}\begin{tabular}{lllll}
		`finish'			&\ve{{\rt}sopu} &+ \ve{m-} \ra& \ve{na-m-sopu} & `is finished'\\
		`loose'				&\ve{{\rt}neku} &+ \ve{m-} \ra& \ve{na-m-neku} & `is lost'\\
		`stand upright'&\ve{{\rt}tetu} &+ \ve{m-} \ra& \ve{na-m-tetu} & `is standing upright'\\
		`set, place'	&\ve{{\rt}teke} &+ \ve{m-} \ra& \ve{na-m-teke} & `is set/placed'\\
		`straighten, correct'	&\ve{{\rt}nono} &+ \ve{m-} \ra& \ve{m-nono-ʔ} & `straight, correct'\\
		\end{tabular}}
\end{exe}

This circumfix is clearly related to the property circumfix \ve{ma-{\ldots}-ʔ} (\srf{sec:PropCir})
and the prefixal elements of both these circumfixes are reflexes of
Proto-Malayo-Polynesian *ma- `stative verb prefix' \cite[473]{bl03}.

When vowel-initial roots or the monosyllabic root
\ve{{\rt}ha} `eat (soft food)' occur with this prefix,
the stem takes the third person form.
Examples are given in \qf{ex:StaM-N-} below.
Not all these forms clearly have a stative meaning.
This is additional evidence that this prefix is no longer productive.

\begin{exe}
	\ex{Statives with \ve{m-n-}}\label{ex:StaM-N-}
	\sn{\gw\stl{0.4em}\begin{tabular}{lllll}
		`praise'	&\ve{{\rt}aikas}	&+ \ve{m-} \ra& \ve{m-n-aikas}	& `praise' (nominal)\\
	%	`eat'			&\ve{{\rt}ha} 		&+ \ve{m-} \ra& \ve{na-m-na-ha} & `hungry'\\
		`drink'		&\ve{{\rt}inu}		&+ \ve{m-} \ra& \ve{m-n-inu-ʔ}	& `drinking, drinkable'\\
		`see'			&\ve{{\rt}ita}		&+ \ve{m-} \ra& \ve{m-n-ita-ʔ}	& `seen, visible'\\
		\end{tabular}}
\end{exe}

The final glottal stops in \ve{na-m-n-inu-ʔ}
`drinking, drinkable', \ve{m-n-ita-ʔ} `seen, visible',
and \ve{m-nono-ʔ} `straight, correct'
provide evidence that the stative affix is indeed a circumfix \ve{m-{\ldots}-ʔ}
rather than simply a prefix \ve{m-}.
The failure of this glottal stop to appear on other forms in \qf{ex:StaVerM-} and \qf{ex:StaM-N-}
can be ascribed to them either being consonant final (e.g. \ve{aikas} `praise')
or because they are verbs.
Final consonants of nominals are regularly deleted
when they are the base for verbal derivation (\srf{sec:BasVerDer}).

The eight words given in \qf{ex:StaVerM-} and \qf{ex:StaM-N-}
are the only forms in my corpus for which a clearly related root
without the stative circumfix has been identified.
There are also eleven property nominals in my corpus which begin with /mC/ for which there
is no corresponding synchronic form without this initial /m/.
Of these, half (6/11) also end with /ʔ/.
These eleven forms are: \ve{mnaaʔ} `old, former, previous',
\ve{mnasiʔ} `old, aged', \ve{mnanuʔ} `long, deep, length, depth',
\ve{mnee} `calm', \ve{{\rt}mnees} {\ra} \ve{mnees{\tl}a|mnees} `quiet',
\ve{mfaun} `many', \ve{mneo} `straight, erect; really, truly',
\ve{mneraʔ} `flat, wide open valley',  \ve{mnutuʔ} `fine, tiny',
\ve{mnuʔir} `wrinkled', and \ve{mtasaʔ} `cooked, ripe'.
