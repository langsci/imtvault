\subsection{M\=/form tail and U\=/form head}\label{sec:MforTaiUforHea}
There are 26 instances of tail-head linkage
in my corpus in which the tail is in the M\=/form and the head in the U\=/form.
In most instances the head is followed by one of the connectors
\ve{=ma} `and' or \ve{=te} \tsc{set} `when, as'.

The structure of these tail-head linkage constructions
is given in \qf{ex:THL M/U} below.
The tail occurs in the M\=/form followed by the head in the U\=/form.
This introduces a second event which resolves the previous U\=/form.

\begin{exe}
	\ex{\xytext{\fbox{event\sub{1}{\M}}\xybarconnect[2][-](D,D){1}&\fbox{event\sub{1}{\U}}\xybarconnect[2][->]{2}&(\ve{=ma}/\ve{=te})&\fbox{event\sub{2}\vp{/}}}}\label{ex:THL M/U}
\end{exe}

In about one quarter (7/26) of these examples
the tail occurs at the point where the plot structure
shifts from background information to the storyline,
in or after the Setting part of a story,
with the U\=/form head occurring in the Inciting Incident,
which then leads to the Climax.
If we examine only the low-level structure
of the immediate sentences or clauses
such U\=/forms are usually resolved fairly quickly.
However, at the higher level of the plot structure of a narrative,
the problems introduced by such U\=/forms
are often not resolved until the Dénouement of the story.

One example is given in \qf{ex:130902-1, 1.43} below.
In this instance the M\=/form tail occurs in the first part
of the Inciting Incident of the narrative.
At a low level the U\=/form \ve{n-mofu} `fall' in \qf{ex:130902-1, 1.43-c}
is resolved by the following event which it causes, \ve{na-mneuk} `lost'.
However, at a higher level of the discourse this
entire incident is not resolved until
several clauses later in the Dénouement when the
problem introduced by \qf{ex:130902-1, 1.43} is resolved.

\begin{exe}
	\ex{Going to a party -- Inciting Incident: \xytext{\fbox{fall{\M}}\xybarconnect[2][-](D,D){1}&\fbox{fall{\U}}\xybarconnect[2][->]{2}&and&\fbox{lost{\M}}}}\label{ex:130902-1, 1.43}
	\begin{xlist}
		\ex{\gll	oras hai m-nao =te, \\
							time {\hai} {\m}-go ={\te}\\
				\glt	`While we were going,' \txrf{}}
		\ex{\glll	naiʔ Owen ina ʔpiurʔ=ee n-mo\tbr{uf},\\
							naiʔ Owen ina ʔpiruʔ=ee n-mofu\\
							{\naiq} Owen {\iin} cloth={\ee} {\n}-fall{\tbrM} \\
				\glt	`Owen's handkerchief fell,'}
		\ex{\glll	\hp{naiʔ} \hp{Owen} \hp{ina} \hp{ʔpiurʔ=ee} n-mo\tbr{fu} =m na-mneuk.\\
							\hp{naiʔ} \hp{Owen} \hp{ina} \hp{ʔpiruʔ=ee} n-mofu =ma na-mneku\\
							\hp{\naiq} \hp{Owen} \hp{\iin} \hp{cloth={\ee}} {\n}-fall{\tbrU} =and {\na}-lose{\M}\\
				\glt	\lh{`Owen's handkerchief fell,'} `it fell and got lost'
							\txrf{130902-1, 1.43}{\emb{130902-1-01-43.mp3}{\spk{}}{\apl}}}\label{ex:130902-1, 1.43-c}
	\end{xlist}
\end{exe}

Another example is given in \qf{ex:130928-1, 0.02-0.11} below,
which consists of the first three clauses of a story.
The first clause in \qf{ex2:130928-1, 0.02}
is the Setting of the story with the M\=/form verb \ve{n-maet} `dies'.
This verb is then repeated as a U\=/form in \qf{ex:130928-1, 0.06}
to introduce the Inciting Incident in \qf{ex:130928-1, 0.11}.
At a low level, the U\=/form verb \ve{n-mate} in \qf{ex:130928-1, 0.06}
is resolved by the event in \qf{ex:130928-1, 0.11}.
However, at a higher level, the chain of events introduced by this U\=/form
is not resolved until much later in this story.

\begin{exe}
	\ex{Nahor Bani's death: \xytext{\fbox{die{\M}}\xybarconnect[2][-](D,D){1}&\fbox{die{\U}}\xybarconnect[2][->]{2}&when&\fbox{dug before{\M}}}
		\txrf{130928-1} {\emb{130928-1-00-02-00-11.mp3}{\spk{}}{\apl}}}\label{ex:130928-1, 0.02-0.11}
	\begin{xlist}
		\ex{\glll	neno ia aam Nahor Bani iin n-ma\tbr{et}. \\
							neno ia ama Nahor Bani ini n-mate \\
							day {\ia} father Nahor Bani {\iin} {\n}-die{\tbrM} \\
				\glt `Today father Nahor Bani died.' \txrf{0.02}}\label{ex2:130928-1, 0.02}
		\ex{\gll	\hspace{38mm} oras iin n-ma\tbr{te} =te, \\
							{} time {\iin} {\n}-die{\tbrU} ={\te} \\
				\glt \hspace{38mm} `When he died,' \txrf{0.06}}\label{ex:130928-1, 0.06}
	\ex{\glll	iin aan moon\j=ees kaan-n=ee naiʔ, Fanu, \hspace{20mm} a|n-hain n-ain nopu.\\
						ini anah mone=esa kana-n=ee naiʔ Fanu {} {\a}n-hani n-ani nopu\\
						{\iin} child male={\es} name-{\N}={\ee} {\naiq} Fanu {} {\a\n}-dig{\M} {\n}-before{\M} hole\\
			\glt	\lh{\a}`One of his sons, called Fanu, had dug the grave beforehand.' \txrf{0.11}}\label{ex:130928-1, 0.11}
	\end{xlist}
\end{exe}

Another example is given in \qf{ex:130825-8, 1.06-1.13} below.
In this example \qf{ex:130825-8, 1.06} is the final part of the Setting:
the narrator is relaxing in his hotel room.
The Setting ends with the M\=/form \ve{ʔ-iiŋgw=een} `drank'.
As with the previous two examples, this verb occurs as a U\=/form in the following
clause \qf{ex:130825-8, 1.10} to introduce the Inciting Incident:
the narrator enters the bathroom.
%After which we have the Climax,
%the beginning of which is given in \qf{ex:130825-8, 1.13}.
%The other verbal U\=/forms in \qf{ex:130825-8, 1.10}
%are discussed in \srf{sec:SceSet} on \prf{ex2:130825-8, 1.10}.

\begin{exe}
	\ex{Exploring a hotel room: \txrf{130825-8} {\emb{130825-8-01-06-01-13.mp3}{\spk{}}{\apl}}}\label{ex:130825-8, 1.06-1.13}
	\sn{\xytext{\fbox{drink{\Mv}\vp{|}}\xybarconnect[2][-](D,D){1}&\fbox{drink{\U}\vp{|}}\xybarconnect[2][->]{2}&and&\fbox{finish{\U}\vp{|}}\xybarconnect[2][->](D,D){2}&when&\fbox{enter{\M} turn.on tap\vp{|}}}}
	\begin{xlist}
		\ex{\glll	ʔ-took ʔ-oka bruuk=ii =m ʔ-ait biir \sf{kaleŋ} \hspace{5mm} =siin =m ʔ-i\tbr{iŋgw}=een\\
							ʔ-toko ʔ-oka bruuk=ii =ma ʔ-aiti biir \sf{kaleŋ} {} =sini =ma ʔ-inu=ena\\
							{\q}-sit{\M} {\q}-{\ok\Uc} pants={\ii} =and {\q}-pick.up{\M} beer can {} ={\siinN} =and {\q}-drink{\tbrMv}={\een}\\
				\glt	`I sat down in the pants, picked up some beer cans and drank.' \txrf{1.06}}\label{ex:130825-8, 1.06}
		\ex{\glll	ʔ-i\tbr{nu} =m u-rari =t, a|ʔ-taam ʔ-ai kraan=ii,\\
							ʔ-inu =ma u-rari =te {\a}ʔ-taam ʔ-ai kraan=ii\\
							{\q}-drink{\tbrU} =and {\qu}-finish{\U} ={\te} \a\q-enter{\M} {\q}-push tap={\ii}\\
				\glt	`I drank and when I finished, I went in and turned on the tap,'
							\txrf{1.10}}\label{ex:130825-8, 1.10}
		\ex{\glll	mu-hiin he oe mapuutʔ=ee es mee =m \hspace{10mm} oe mainiikn=ee es mee? \\
							mu-hini he oe maputuʔ=ee es mee =ma {} oe mainikin=ee es mee \\
							{\muu}-know{\M} {\he} water hot={\ee} {\et} where =and {} water cold={\ee} {\et} where \\
				\glt	`Do you know where the hot water is and where the cold water is?'
							(implied: I didn't know)
							\txrf{1.13}}\label{ex:130825-8, 1.13}
	\end{xlist}
\end{exe}

While about a quarter of tail-head linkages with a U\=/form head
are used to introduce the Climax part of the plot,
others are simply used to introduce some extra information.
One such example is given in \qf{ex:130825-8, 0.17} below.
In this example the speaker is encouraging the main narrator
to keep telling his story. The M\=/form \ve{m-ait} `pick up'
in \qf{ex:130825-8, 0.17 a} is repeated as a U\=/form in \qf{ex:130825-8, 0.17 b}
which introduces the event which is presumed to have occurred next (\ve{m-bukae} `consume').

\begin{exe}
	\ex{Exploring a hotel:
			\xytext{\fbox{pick.up{\M}}\xybarconnect[2][-](D,D){1}&\fbox{pick.up{\U}}\xybarconnect[2][->]{1}&\fbox{drink\vp{p}}}
		\txrf{130825-8, 0.17} {\emb{130825-8-00-17-00-21.mp3}{\spk{}}{\apl}}}\label{ex:130825-8, 0.17}
	\begin{xlist}
		\ex{\glll	hoo meu-- m-a\tbr{it} \sf{biir} \sf{kaleŋ},\\
							hoo {} m-aiti \sf{biir} \sf{kaleŋ}\\
							{\hoo} {} \m-pick.up{\tbrM} beer can\\
				\glt	`You picked up a beer can,' \txrf{0.17}}\label{ex:130825-8, 0.17 a}
		\ex{[others interrupt]}
		\ex{\glll	hoo m-a\tbr{iti}, hoo m-bukae.\\
							hoo m-aiti hoo m-bukae\\
						%	hoo m-aiti hoo m-bukae\\
							{\hoo} \m-pick.up{\tbrU} {\hoo} \m-consume\\
				\glt	`you picked it up, you drank.' \txrf{0.21}}\label{ex:130825-8, 0.17 b}
	\end{xlist}
\end{exe}

No connectors occur in \qf{ex:130825-8, 0.17},
but nonetheless the head of the tail-head linkage construction occurs in the U\=/form.
My main consultant rejected the equivalent of \qf{ex:130825-8, 0.17}
above with two M\=/forms, as shown in (\ref{ex:130825-8, 0.17}′) below.
This is evidence that tail-head linkage with alternate U\=/forms and M\=/forms
is a grammaticalised pattern in Amarasi,
independent of the presence or absence of connectors.

\begin{exe}
	\exp{ex:130825-8, 0.17}[*]{\glll	
						hoo m-a\tbr{it} \sf{biir} \sf{kaleŋ}, hoo m-a\tbr{it} hoo m-bukae\\
						hoo m-aiti \sf{biir} \sf{kaleŋ} hoo m-aiti hoo m-bukae\\
						{\hoo} \m-pick.up{\tbrM} beer can {\hoo} \m-pick.up{\tbrM} {\hoo} \m-drink\\
			\glt	`(You picked up a beer can, you picked it up, you drank.)'
						\txrf{elicit. 25/02/16 p.29}}
\end{exe}

Another example of tail-head linkage with a U\=/form head is given in \qf{ex:130909-6, 2.20} below.
In this example the M\=/form verb \ve{ta-mnaas\j=een} `grow old'
is repeated as a U\=/form \ve{ta-mnasi} in the next clause,
which in turn introduces a new event \ve{ta-smeruʔ} `look at angrily'.
The equivalent of \qf{ex:130909-6, 2.20} with
a second M\=/form was judged unacceptable, as shown in (\ref{ex:130909-6, 2.20}′).

\newpage
\begin{exe}
	\ex{Growing old together:  \xytext{\fbox{old{\Mv}\vp{\Uc}}\xybarconnect[2][-](D,D){1}&\fbox{old{\U}\vp{\Uc}}\xybarconnect[2][->]{2}&when&\fbox{stare{\Uc}}}
		\xrf{130909-6, 2.20} {\emb{130909-6-02-20.mp3}{\spk{}}{\apl}}}\label{ex:130909-6, 2.20}
	\begin{xlist}
		\ex[]{\glll	haa \sf{ja}. on reʔ naan, ta-mna\tbr{as\j}=een =t,\\
							haa \sf{ja} on reʔ naan ta-mnasi=ena =te\\
							hey yes like \req {\naan} \ta-old{\tbrMv}={\een} ={\te}\\
				\glt	`What's that? Yes. That's how it is. When we grow old,' \txrf{}}
		\ex[]{\gll	au ʔ-ak aiʔ ehh ta-mna\tbr{si} aiʔ ia =t, of aiʔ ta-smeruʔ uis fee mnasiʔ aiʔ fee mnasiʔ\\
							{\au} \q-say or {} {\ta}-old{\tbrU} or {\ia} ={\te} later or {\ta}-glare{\Uc} lord wife old or wife old\\
				\glt	`I think, when we grow old now, we glare angrily at the lord of the old woman, or the old woman.' \txrf{}}
	\end{xlist}
	\exp{ex:130909-6, 2.20}{\begin{xlist}
		\exi{b.}[*]{\gll	au ʔ-ak aiʔ ta-mna\tbr{is} aiʔ ia =t, of aiʔ ta-smeruʔ uis fee mnasiʔ aiʔ fee mnasiʔ\\
							{\au} \q-say or {\ta}-old{\tbrM} or {\ia} ={\te} later or {\ta}-glare{\Uc} lord wife old or wife old\\
				\glt	`(I thought when we grow old or now, we glare angrily at the lord of the old woman, or the old woman.)' \txrf{elicit. 25/02/16 p.28}}
	\end{xlist}}
\end{exe}

The ungrammatical examples in (\ref{ex:130825-8, 0.17}′)
and (\ref{ex:130909-6, 2.20}′) above are ungrammatical
because the tail-head linkage construction contains two M\=/forms.
Tail-head linkage constructions with two U\=/forms are also unacceptable.
This is shown in (\ref{ex:130902-1, 1.43}′) below,
manipulated versions of example \qf{ex:130902-1, 1.43} above (repeated below),
showing every possible combination of two U\=/form verbs
with and without the connector \ve{=ma}.
None of these were judged acceptable.

\begin{exe}
	\exr{ex:130902-1, 1.43}{
		\begin{xlist}
			\sn[]{\gll naiʔ Owen ina ʔpiurʔ=ee n-mo\tbr{uf}, n-mo\tbr{fu} =m na-mneuk\\
								 {\naiq} Owen {\iin} {{\a}cloth={\ee}} {\n}-fall{\tbrM} {\n}-fall{\tbrU} =and {\na}-lose{\M} \\}
		\end{xlist}}
	\exp{ex:130902-1, 1.43}{
		\begin{xlist}
			\ex[*]{\gll	naiʔ Owen ina ʔpiurʔ=ee n-mo\tbr{fu}, n-mo\tbr{fu} =ma na-mneuk\\
								{\naiq} Owen {\iin} {cloth={\ee}} {\n}-fall{\tbrU} {\n}-fall{\tbrU} =and {\na}-lose{\M}\\}
			\ex[*]{\gll	naiʔ Owen ina ʔpiurʔ=ee n-mo\tbr{fu} =ma, n-mo\tbr{fu} na-mneuk\\
								{\naiq} Owen {\iin} {cloth={\ee}} {\n}-fall{\tbrU} =and {\n}-fall{\tbrU} {\na}-lose{\M}\\}
			\ex[*]{\gll	naiʔ Owen ina ʔpiurʔ=ee n-mo\tbr{fu}, n-mo\tbr{fu} na-mneuk\\
								{\naiq} Owen {\iin} {cloth={\ee}} {\n}-fall{\tbrU} {\n}-fall{\tbrU} {\na}-lose{\M}\\
					\glt	`(Owen's handkerchief fell, it fell and was lost.)' \xrf{elicit. 15/03/16 p.45}}
		\end{xlist}}
\end{exe}

One pattern of tail-head linkage in Amarasi is for the head to be in the M\=/form
and the tail to be in the U\=/form.
In this case the U\=/form introduces a new event into the storyline
which resolves the event described by the tail-head linkage construction.
U\=/forms must be used in combination with M\=/forms and it is not acceptable
for both parts of the tail-head linkage construction to be in the M\=/form
or for both to be in the U\=/form.\footnote{
		Two U\=/forms occur in a restricted set of circumstances.
		See \srf{sec:UforTaiUforHea} for more details.}

