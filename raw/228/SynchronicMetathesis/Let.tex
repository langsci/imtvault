\subsection{Leti}\label{sec:Let}
\il{Leti|(}Leti is an Austronesian language of Indonesia spoken on an island
with the same name off the eastern-most tip of the island of Timor
(see \frf{fig:CVMetTimReg}).
It is closely related to Luang (\srf{sec:PhoMet}),
which has phonologically conditioned metathesis.
Leti metathesis has been described by \citet{en94,en96,en04}
and formal analyses of it have been proposed by \cite{huen95},
as well as \cite{hu98}.

\subsubsection{Forms}\label{sec:LetFor}
In Leti each word has a least two forms;
a vowel-final U\=/form and an M\=/form which is often consonant final.
%Broadly speaking, the U\=/forms are used when the word occurs phrase finally,
%including in the citation form \citep[90]{en04},
%while M\=/forms are used in certain conditions when before another word or morpheme.
A single Leti U\=/form does not necessarily correspond to a single M\=/form.
Rather, the phonological shape of both the form in question
and the following morpheme must be taken into account when determining the shape of the M\=/form.
For instance, the Leti U\=/form \it{iina} `fish'
can have either of the M\=/forms \it{iin} or \it{ian},
depending on the phonological shape of the following morpheme.
In this respect, Leti is similar to Amarasi in which a single
U\=/form can have up to three different M\=/forms in different environments (see Chapter \ref{ch:StrMetAma}).

Four different phonological processes operate in Leti to derive each different form:
glide formation, internal metathesis, external metathesis, and apocope.
Each of these processes is described with reference to the phonological shape of the U\=/form of the first word.
There are four possible shapes for Leti U\=/forms:

\begin{itemize}
	\item[i.] VV{\#} final e.g. \it{nia} `snake'
	\item[ii.] VCV{\#} final e.g. \it{kusa} `cat'
	\item[iii.] V{\sub{α}}V{\sub{α}}CV{\#} final e.g. \it{iina} `fish'
	\item[iv.] VCCV{\#} final e.g. \it{ɛmna} `moray eel'
\end{itemize}

\paragraph{No change}\label{sec:LetNoc}
When the second word begins with a consonant cluster,
the first word does not undergo any phonological
processes and appears in the vowel-final U\=/form.

\begin{exe}
	\ex{No phonological process \hfill\citep[91]{en04}}\label{LetNoc-1}
		\sn{\gw\begin{tabular}{rlllll}
			U\=/form			&	& 						&		& M\=/form	&	\\
			\it{lau}		&+& \it{tniɛi}	&\ra& \it{lau tniɛi} 		&`civet + guts'	\\
			\it{ruuni}	&+& \it{tniɛi}	&\ra& \it{ruuni tniɛi} 	&`dugong + guts'	\\
		\end{tabular}}%
\end{exe}

\paragraph{Glide formation}\label{sec:LetGliFor}
When the first word ends with a high vowel
and the second word begins with a non-high vowel
/e/, /ɛ/, /a/, /ɔ/ or /o/,
the final vowel of the first word is realised as a glide.
Examples are given in \qf{LetGliFor-1}.

This is an automatic phonetic process,
as glides do not contrast phonemically with high vowels in Leti.
A high vowel is automatically realised as a glide when it occurs before a stressed non-high vowel \citep[59]{en04}.

\begin{exe}
	\ex{CV\tsc{[+high]} {\ra} CV̯\tsc{[+high]} /{\_}V\tsc{[-high]}\hfill\citep[91]{en04}}\label{LetGliFor-1}
		\sn{\gw\stl{0.5em}
		\begin{tabular}{rllllll}
			U\=/form						&	& 					&		& M\=/form	&&	\\
			\it{la\tbr{u}}		&+& \it{aana}	&\ra& \it{la\tbr{u} aana} 	& [la\tbr{w}ˈaːna]	 	&`civet + child'	\\
			\it{ruun\tbr{i}}	&+& \it{aana}	&\ra& \it{ruun\tbr{i} aana}	& [ruːn\tbr{j}ˈaːna] &`dugong + child'	\\
		\end{tabular}}
\end{exe}

\paragraph{Internal metathesis}\label{sec:LetIntMet}
%\paragraph{CVC {\ra} CCV}\label{sec:VC->CV}
If the second word begins with a CV sequence,
or a sequence of a high vowel followed by a vowel
(phonetically a glide followed by a vowel;
as discussed in \srf{sec:LetGliFor} above),
and the U\=/form of the first word ends in CCV{\#},
then the M\=/form of the first word corresponds to
the U\=/form via metathesis of the final CV sequence.

	\begin{exe}
		\ex{C\sub{1}VC\sub{2} {\ra} C\sub{1}C\sub{2}V /{\_}CV \hfill\citep[91]{en04}}\label{ex:IntMetCVC->CCV}
			\sn{\gw\begin{tabular}{rlllll}
				U\=/form						&	& 								&		&M\=/form			&	\\  
				\it{ɛm\tbr{na}}		&+& \it{nama}				&\la& \it{ɛm\tbr{an} nama} 	&`moray + tongue'  \\
				\it{plil\tbr{ki}}	&+& \it{ruri}				&\la& \it{plil\tbr{ik} ruri} &`k.o. lizard + bone' \\
				\it{trut\tbr{nu}}	&+& \it{u̯ata}&\la& \it{trut\tbr{un} u̯ata} &`Blurr-fish + head' \\
			\end{tabular}}
	\end{exe}

There is a process of consonant assimilation which operates in Leti
which provides evidence that the underlying form of CCV final U\=/forms is the M\=/form.
A penultimate /d/ or /l/ in the M\=/form assimilates to a final /n/ in the U\=/form.
Likewise, a penultimate /d/ in the M\=/form assimilates to a U\=/form final /l/.
Examples are given in \qf{ex:LetConAss} below.

\begin{exe}
	\ex{Consonant assimilation \hfill\citep[74]{en04}}\label{ex:LetConAss}
		\sn{\gw\begin{tabular}{rlll}
			M\=/form										&			& U\=/form 						&	\\
			\it{{\B}ɛ\tbr{n}a\tbr{n}}	&{\ra}&\it{{\B}ɛ\tbr{nn}a}&`kill'  \\
			\it{ɛ\tbr{d}a\tbr{n}}			&{\ra}&\it{ɛ\tbr{nn}a} 		&`pineapple' \\
			\it{{\B}u\tbr{l}a\tbr{n}} &{\ra}&\it{{\B}u\tbr{ll}a}&`moon' \\
			\it{su\tbr{d}a\tbr{l}}		&{\ra}&\it{su\tbr{ll}a} 	&`prop' \\
		\end{tabular}}
\end{exe}

Given a U\=/form such as \it{ɛnna} `pineapple',
it is impossible to predict whether the M\=/form will be \it{*ɛnan} or \it{ɛdan}.
Likewise, given the U\=/form \it{{\B}ulla} either the correct M\=/form \it{{\B}ulan}
or the incorrect form \it{*{\B}udal} can be derived.
This provides evidence that the M\=/form in such examples
is morphologically underlying with the U\=/form
being formed by final VC {\ra} CV metathesis.

Another kind of internal metathesis occurs when the
antepenultimate and penultimate vowels of the first word are identical;
a V{\sub{α}}V{\sub{α}}CV{\sub{β}} final word.
In the M\=/form the final consonant and vowel metathesise and the penultimate vowel is deleted.
Like the process of VC {\ra} CV metathesis shown in \qf{ex:IntMetCVC->CCV} above,
this only occurs when the second word begins with CV.
Examples are given in \qf{ex:IntMetVVCV->VVC} below.

\begin{exe}
	\ex{V{\sub{α}}V{\sub{α}}CV{\sub{β}} {\ra} V{\sub{α}}V{\sub{β}}C/{\_}CV
	\hfill\citep[91]{en04}}\label{ex:IntMetVVCV->VVC}
		\sn{\gw\begin{tabular}{rlllll}
			U\=/form						&	& 									&		& M\=/form	&	\\  
			\it{i\tbr{ina}}		&+& \it{nama}					&\ra& \it{i\tbr{an} nama} 	&`fish + tongue'  \\
			\it{ru\tbr{uni}}	&+& \it{ruri}					&\ra& \it{ru\tbr{in} ruri} 	&`dugong + bone' \\
			\it{ma\tbr{anu}}	&+& \it{u̯ata}	&\ra& \it{ma\tbr{un} u̯ata} &`bird + head' \\
		\end{tabular}} %
\end{exe}

\paragraph{External metathesis}\label{sec:LetExtMet}
When the first word ends in a vowel sequence or VCV{\#}
and the second word begins with {\#}CV with a non-high
initial vowel, metathesis occurs across the word boundary.
According to the regular phonetic rule of glide formation,
the final V of the first word becomes a glide.
This process is similar to phonological metathesis
of glides in Selaru (\srf{sec:PhoMet}).

\begin{exe}
	\ex{V\sub{1}\tsc{[+high]}\#CV\sub{2}\tsc{[-high]} {\ra} CV̯\sub{1}V\sub{2} \hfill\citep[91]{en04}}\label{LetExtMet}
		\sn{\gw\stl{0.4em}
		\begin{tabular}{rllllll}
			U\=/form					&	& 								&		&	M\=/form	&&	\\  
			\it{sru\tbr{i}}	&+& \it{\tbr{n}ama}	&\ra& \it{sru\tbr{ni}ama}	& [sru\tbr{nj}ˈama]	&`garfish + tongue'  \\
			\it{la\tbr{u}}	&+& \it{\tbr{n}ama}	&\ra& \it{la\tbr{nu}ama} 	& [la\tbr{nw}ˈama]	&`civet + tongue' \\
			\it{nik\tbr{i}}	&+& \it{\tbr{n}ama}	&\ra& \it{nik\tbr{ni}ama}	& [nik\tbr{nj}ˈama]	&`bat + tongue' \\
			\it{as\tbr{u}}	&+& \it{\tbr{n}ama}	&\ra& \it{as\tbr{nu}ama}	& [as\tbr{nw}ˈama]	&`dog + tongue' \\
		\end{tabular}}%
\end{exe}

\paragraph{Apocope}\label{sec:LetApo}
Apocope (a.k.a truncation or vowel deletion) occurs in three environments in Leti.
Firstly, apocope occurs when the first segment of the second word is a high vowel
(but not a glide),
no matter the shape of the first word.
Examples are shown in \qf{LetApo-1} below:

\newpage
\begin{exe}
	\ex{V {\ra} {\0} /{\_}V\tsc{[+hi]} \hfill\citep[91]{en04}}\label{LetApo-1}
		\sn{\gw\begin{tabular}{rlllll}
			U\=/form						&	& 					&		&M\=/form	&	\\  
			\it{sru\tbr{i}}		&+&\it{irnu}	&\ra&\it{sru irnu}	&`garfish + nose'	\\
			\it{la\tbr{u}}		&+&\it{irnu}	&\ra&\it{la irnu}		&`civet + nose'  \\
			\it{nik\tbr{i}}		&+&\it{irnu}	&\ra&\it{nik irnu} 	&`bat + nose' \\
			\it{as\tbr{u}}		&+&\it{irnu}	&\ra&\it{as irnu} 	&`dog + nose' \\
			\it{ruun\tbr{i}}	&+&\it{irnu}	&\ra&\it{ruun irnu}	&`dugong + nose'	\\
			\it{maan\tbr{u}}	&+&\it{irnu}	&\ra&\it{maan irnu}	&`bird + nose'	\\
			\it{plilk\tbr{i}}	&+&\it{irnu}	&\ra&\it{plilk irnu}&`k.o. lizard + nose'	\\
			\it{trutn\tbr{u}}	&+&\it{irnu}	&\ra&\it{trutn irnu}&`Blurr-fish + nose'	\\
	\end{tabular}}
\end{exe}

Secondly apocope takes place when the first word ends in VCV{\#} 
or VV{\#} (but not VVCV),
and the second word begins with a high vowel,
as seen in \qf{LetApo-1} above,
or a consonant (including glides) followed by a high vowel,
as shown in \qf{LetApo-2}.

\begin{exe}
	\ex{V {\ra} {\0} /{\_}(C)V\tsc{[+hi]}, /{\_} V̯V \hfill\citep[91]{en04}}\label{LetApo-2}
		\sn{\gw\begin{tabular}{rlllll}
			U\=/form						&	& 					&		&M\=/form	&	\\  
			\it{sru\tbr{i}}	&+& \it{ruri}	&{\ra}& \it{sru ruri} 	&`garfish + bone' \\
			\it{la\tbr{u}}	&+& \it{u̯ata}	&{\ra}& \it{la u̯ata} 	&`civet + head' \\
			\it{nik\tbr{i}}	&+& \it{u̯ata}	&{\ra}& \it{nik u̯ata}	&`bat + head' \\
			\it{as\tbr{u}}	&+& \it{ruri}	&{\ra}& \it{as ruri}	&`dog + bone' \\
		\end{tabular}}
\end{exe}

Thirdly, apocope takes place when the first word ends in VV or VCV with a non-high final vowel
and the first vowel of the second word is also a non-high vowel.
This is shown in \qf{LetApo-3} below.

\begin{exe}
	\ex{Apocope: V\tsc{[-high]} {\ra} {\0} /{\_}(C)V\tsc{[-high]} \hfill\citep[91]{en04}}\label{LetApo-3}
		\sn{\gw\begin{tabular}{rlllll}
			U\=/form					&	& 					&			&M\=/form	&	\\  
			\it{ni\tbr{a}}	&+& \it{aana}	&{\ra}& \it{ni aana} 	&`snake + baby'  \\
			\it{kus\tbr{a}}	&+& \it{aana}	&{\ra}& \it{kus aana} &`cat + baby' \\
			\it{ɛmn\tbr{a}}	&+& \it{aana}	&{\ra}& \it{ɛmn aana} &`moray + baby' \\
			\it{ni\tbr{a}}	&+& \it{nama}	&{\ra}& \it{ni nama}	&`snake + tongue' \\
			\it{kus\tbr{a}}	&+& \it{nama}	&{\ra}& \it{kus nama}	&`cat + tongue' \\
		\end{tabular}}%
\end{exe}

\paragraph{Summary}\label{sec:LetSum}
The different processes which operate in Leti to derive the M\=/form from the U\=/form
are summarised in \trf{tab:LetFreBouFor}.
This table is followed by \trf{tab:LetInsFreBouFor}
which shows instantiated examples of each of these processes.
Metathesis in Leti is only one of several
phonological processes which operate in the language to derive M\=/forms.

Which form is the underlying form is not consistent in Leti.
In some cases the U\=/form must be posited as underlying as
the quality of the final vowel cannot be recovered after apocope,
while in other cases the M\=/form must be posited as underlying as the quality
of the penultimate consonant cannot be recovered after metathesis.
This is different to the Amarasi data, in which the U\=/form
must be posited as underlying in all circumstances.

\newcommand{\CV}{{\{}$\frac{\textrm{C}}{\textrm{V}}${\}}}
\newcommand{\cbl}{\cellcolor{blue!40}}
\newcommand{\ccy}{\cellcolor{green!50}}
\newcommand{\cye}{\cellcolor{yellow!75}}

\begin{table}[h]
	\caption{Leti U-forms and M-forms}\label{tab:LetFreBouFor}
		\begin{tabular}{lr|lllll}
		\lsptoprule
U\=/form{\da} 					& before:		&CCV				&CV\sub{+h}			&C\sub{2}V\sub{-h}		&V\sub{+h}			&V\sub{-h}			\\\midrule
\mc{2}{l|}{V(C)V}									&V(C)V			&V(C)			{\cgr}&V(C)						{\cgr}&V(C)			{\cgr}&V(C)			{\cgr}\\	
\mc{2}{l|}{V(C)V\tsc{[+high]}}		&V(C)V			&V(C)			{\cgr}&V(C)C\sub{2}V̯	{\ccy}&V(C)			{\cgr}&V(C)V̯		{\cye}\\	
\mc{2}{l|}{V{\CV}CV}							&V{\CV}CV		&V{\CV}VC	{\cbl}&V{\CV}VC				{\cbl}&V{\CV}C	{\cgr}&V{\CV}C	{\cgr}\\	
\mc{2}{l|}{V{\CV}CV\tsc{[+high]}}	&V{\CV}CV		&V{\CV}VC	{\cbl}&V{\CV}VC				{\cbl}&V{\CV}C	{\cgr}&V{\CV}CV̯	{\cye}\\
		\lspbottomrule
\mc{7}{c}{
			\tikz{\path[draw=black,fill={rgb:black,0.07;white,0.93}] (0,0) rectangle (3.75mm,2mm);} apocope,
			\tikz{\path[draw=black,fill=blue!40] (0,0) rectangle (3.75mm,2mm);} metathesis,
			\tikz{\path[draw=black,fill=green!50] (0,0) rectangle (3.75mm,2mm);} ext. metathesis,
			\tikz{\path[draw=black,fill=yellow!75] (0,0) rectangle (3.75mm,2mm);} glide formation} \\
	\end{tabular}
\end{table}

\begin{table}[h]
	\caption{Leti instantiated U-forms and M-forms}\label{tab:LetInsFreBouFor}
	\begin{tabular}{ll|lllll|l}
		\lsptoprule
U\=/form{\da}	&before:		&CCV	&CV\sub{+h}	&C\sub{2}V\sub{-h}&V\sub{+h}							&V\sub{-h}				&\mc{1}{l}{}\\\midrule
VV								&\it{nia}		&\it{nia}		&\it{ni}		{\cgr}&\it{ni}					{\cgr}&\it{ni}		{\cgr}&\it{ni}		{\cgr}&`snake'\\
VV\sub{\tsc{+h}}	&\it{lau}		&\it{lau}		&\it{la}		{\cgr}&\it{laC\sub{2}u̯}	{\ccy}&\it{la}		{\cgr}&\it{lau̯}		{\cye}&`civet'\\ 
VCV								&\it{kusa}	&\it{kusa}	&\it{kus}		{\cgr}&\it{kus}					{\cgr}&\it{kus}		{\cgr}&\it{kus}		{\cgr}&`cat'\\
VCV\sub{\tsc{+h}}	&\it{asu}		&\it{asu}		&\it{as}		{\cgr}&\it{asC\sub{2}u̯}	{\ccy}&\it{as}		{\cgr}&\it{asu̯}		{\cye}&`dog'\\ 
VVCV							&\it{iina}	&\it{iina}	&\it{ian}		{\cbl}&\it{ian}					{\cbl}&\it{iin}		{\cgr}&\it{iin}		{\cgr}&`fish'\\
VVCV\sub{\tsc{+h}}&\it{maanu}	&\it{maanu}	&\it{maun}	{\cbl}&\it{maun}				{\cbl}&\it{maan}	{\cgr}&\it{maanu̯}	{\cye}&`bird'\\ 
VCCV							&\it{ɛmna}	&\it{ɛmna}	&\it{ɛman}	{\cbl}&\it{ɛman}				{\cbl}&\it{ɛmn}		{\cgr}&\it{ɛmn}		{\cgr}&`eel'\\
VCCV\sub{\tsc{+h}}&\it{plilki}&\it{plilki}&\it{plilik}{\cbl}&\it{plilik}			{\cbl}&\it{plilk}	{\cgr}&\it{plilki̯}{\cye}&`lizard'\\
		\lspbottomrule
			\mc{8}{c}{
			\tikz{\path[draw=black,fill={rgb:black,0.07;white,0.93}] (0,0) rectangle (3.75mm,2mm);} apocope,
			\tikz{\path[draw=black,fill=blue!40] (0,0) rectangle (3.75mm,2mm);} metathesis,
			\tikz{\path[draw=black,fill=green!50] (0,0) rectangle (3.75mm,2mm);} ext. metathesis,
			\tikz{\path[draw=black,fill=yellow!75] (0,0) rectangle (3.75mm,2mm);} glide formation} \\
	\end{tabular}
\end{table}

\subsubsection{Functions}\label{sec:LetFun}
The M\=/form of words occur in Leti in two main environments:
when the word is non-final in the noun phrase and when it is indefinite.
Note that the enclitic \it{=e} in many of the following examples replaces
a final /a/ of the word to which it attaches.
This enclitic is a kind of definiteness marker
(for a more detailed discussion of this enclitic see \citet[159--61]{en04}).
Words which do not end in /a/ do not occur with this enclitic.

\paragraph{Attributive modification}\label{sec:LetMod}
All non-final words in the noun phrase occur in the M\=/form.
This function also occurs in Rotuman (\srf{sec:RotPhoConMfo})
and Amarasi (Chapter \ref{ch:SynMet}).
In \qf{ex:LetAM2} below, the noun \it{asu} is in the U\=/form and the adjective is predicative,
as shown in the tree in \qf{tr:LetTre1}.
In \qf{ex:LetAM1} below, the noun is in the M\=/form,
and the adjective is attributive, as shown in the tree in \qf{tr:LetTre2}.

\begin{multicols}{2}
	\begin{exe}\let\eachwordone=\itshape
		\ex{\glll as\tbr{u}  lala{\B}ne \\
						as\tbr{u} la{\tl}la{\B}na=e \\
						dog{\tbrU} \tsc{red}{\tl}big=\tsc{def} \\
						\glt `The dog is big.'}\label{ex:LetAM2}
		\ex{\glll as\tbr{lu}ala{\B}ne {}\\
						{as\tbr{u}\hspace{4.85mm} \tbr{l}a{\tl}la{\B}na=e} {}\\
						{dog{\tbrM} \tsc{red}{\tl}big=\tsc{def}} {} \\
						\glt `The big dog.'}\label{ex:LetAM1}
	\end{exe}
\end{multicols}

\begin{multicols}{2}
	\begin{exe}
		\ex{\begin{forest} where n children=0{tier=word}{}
			[S,[NP,[N,[\it{as\tbr{u}}\\dog]]][PRED,[\it{lala{\B}ne}\\big]]]
		\end{forest}}\label{tr:LetTre1}
		\ex{\begin{forest} %where n children=0{tier=word}{}
			[S,[NP,[N,[\it{as}\\dog]][ADJ,[\it{l\tbr{u}ala{\B}ne}\\big]]][\ldots{\vp{b}},[,phantom]]]
		\end{forest}}\label{tr:LetTre2}
	\end{exe}
\end{multicols}

Any non-final word of the noun phrase,
including adjectives and adverbs, also occur in the M\=/form.
This is shown in example \qf{LetAM3} below
in which a noun, adjective, and adverb all occur in the M\=/form
when the final word of the noun phrase is a demonstrative.

\begin{exe}
\let\eachwordone=\itshape
	\ex{\glll kus mɛmɛt\tbr{am} da{\B}\tbr{ar} d\'{i}\\ 
			kus\tbr{a} mɛ{\tl}mɛt\tbr{ma} da{\B}\tbr{ra} d\'{i} \\
			cat{\tbrM} \tsc{red}{\tl}black{\tbrM} very{\tbrM} \tsc{dem1} \\
			\glt `This very black cat.' \hfill\citep[177]{en04}\label{LetAM3}}
\end{exe}

However, nouns followed by a numeral do not occur
in the M\=/form, as shown in \qf{LetAM4}.
%Nouns are analysed by \citet[176]{en04} analysed
%as constituting independent heads within the noun phrase.
This is similar to Amarasi in which (cardinal)
numerals do not induce M\=/forms on nouns (\srf{sec:NumPhr}).

%\begin{multicols}{2}
	\begin{exe}\let\eachwordone=\itshape
	%	\ex{\begin{xlist}
			\ex{\glll asluala{\B}\tbr{na} {\B}ɔrue \\
				{asu\hspace{4.85mm} la{\tl}la{\B}\tbr{na}} {\B}ɔrua=e \\
				{dog{\M} \tsc{red}{\tl}big{\tbrU}} two=\tsc{def} \\
				\glt `The two big dogs.' \hfill\citep[176]{en04}}\label{LetAM4}
	%		\ex[*]{\glll as\und{lu}ala{\B}an={\B}ɔrue \\
	%			asu=la{\tl}la\Bna={\Bɔ}rua={e} \\
	%			dog=\tsc{red}{\tl}big=two=\tsc{det} \\
	%			\glt `(The two big dogs.)'}\end{xlist}}
	\end{exe}
%\end{multicols}

\paragraph{Indefinite}\label{sec:LetInd}
The second function of the M\=/form in Leti
is to mark words as indefinite.
Like Rotuman, it is the metathesised form of words which is indefinite
and the unmetathesised form which is definite.

This is shown by the contrast between examples \qf{LetAM5} and \qf{LetAM6} below.
In \qf{LetAM5}, the noun \emph{iina} `fish' is definite,
and thus occurs in the U\=/form.
In \qf{LetAM6}, however, it is indefinite and occurs in the M\=/form.

\begin{exe}\let\eachwordone=\itshape
	\ex{\glll rɔɔne i\tbr{ine} {\B}alio. \\
		r-ɔɔna-e i\tbr{ina}-e {\B}ali-o \\
		\tsc{3pl}-eat{\U}-\tsc{def} fish{\tbrU}-\tsc{def} also-\tsc{indicative} \\
		\glt `They eat the fish also.' \label{LetAM5}}
	\ex{\glll rɔɔne i\tbr{an} {\B}alio. \\
		r-ɔɔna-e i\tbr{ina} {\B}ali-o \\			
		\tsc{3pl}-eat{\U}-\tsc{def} fish{\tbrM} also-\tsc{indicative} \\
		\glt `They eat a fish also.' \hfill\citep[210]{en96}\label{LetAM6}}
\end{exe}

Verbs also occur in the M\=/form when they are ``indefinite''.
Such ``indefinite'' verbs have a semantics indicating that
the action specified by the verb was not properly carried out,
translated by \citeauthor{en96} with the phrase `kind of'.
This use seems comparable to the imperfective aspect.

An example of such an ``indefinite'' verb can be seen with the verb \emph{rɔɔna} `eat'
in example \qf{LetAM7} below, which has undergone apocope.
This example contrasts with the ``definite'' verb in example \qf{LetAM5} above.
When both verb and noun are indefinite,
both occur in the M\=/form,
as shown in \qf{LetAM8}.

\begin{exe}\let\eachwordone=\itshape
	\ex{\glll rɔɔn iine {\B}alio. \\
						r-ɔɔn\tbr{a} iina-e {\B}ali-o \\
						\tsc{3pl}-eat{\tbrM} fish{\U}-\tsc{def} also-\tsc{indicative} \\
						\glt `They kind of eat the fish also.' \hfill\citep[210]{en96}\label{LetAM7}}
	\ex{\glll rɔɔn i\tbr{an} {\B}alio. \\
						r-ɔɔn\tbr{a} i\tbr{ina} {\B}ali-o \\			
						\tsc{3pl}-eat{\tbrM} fish{\tbrM} also-\tsc{indicative} \\
						\glt `They kind of eat a fish also.' \hfill\citep[209]{en96}\label{LetAM8}}
\end{exe}
	
Metathesis, as one of the processes by which the M\=/form is derived,
has two functions in Leti:
it marks following nominal modifiers as attributive 
and it is employed to mark words as indefinite.
Of these functions, the first also occurs in Amarasi.\il{Leti|)}