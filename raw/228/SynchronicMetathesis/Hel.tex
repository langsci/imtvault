\subsection{Helong}\label{sec:Hel}
\il{Helong|(}Helong is an Austronesian language spoken in the westernmost part of the island of Timor,
and the neighbouring island of Semau (see \frf{fig:CVMetTimReg}).
Helong metathesis is described by \citet{baca12} and \citet{ba15,ba17b}
who describe the Semau dialect,
as well as by \citeauthor{st96b} (\citeyear{st96b}; \citeyear{st08}),
who presents data from the Bolok dialect.
In addition, I have carried out fieldwork on the Funai dialect
as spoken in Oeletsala village, published as \cite{ed18c}.\footnote{
		There are three recognised dialects of Helong,
		Semau Helong is spoken on Semau island, while Funai
		Helong and Bolok Helong are spoken in Timor mainland}

\subsubsection{Forms}\label{sec:HelFor}
Words in Helong have two forms,
which I call the U\=/form and the M\=/form.
In most cases the M\=/form is derived from the U\=/form
by metathesis of the final CV {\ra} VC.
Examples of each relevant vowel combination are given in \qf{ex:VCV->VVC-Hel} below.
(Helong does not appear to have words in which the penultimate vowel
is high and the final vowel mid,
or in which the final vowel is high and the penultimate vowel mid.)

\begin{exe}
	\ex{Helong V\sub{1}C\sub{1}V\sub{2} {\ra} V\sub{1}V\sub{2}C\sub{1} \hfill\citep[11, 33--52]{baca12}}\label{ex:VCV->VVC-Hel}
	\sn{\gw\stl{0.5em}
	\begin{tabular}{rcll|rcll}
				U\=/form					&		&\mc{2}{l}{M\=/form}					&U\=/form						&		&\mc{2}{l}{M\=/form} \\
			\it{ni\tbr{ni}}		&\ra&\it{ni\tbr{in}}	&`use'		&\it{le\tbr{ko}}	&\ra&\it{le\tbr{ok}}	&`beautiful' \\
			\it{da\tbr{ni}}		&\ra&\it{da\tbr{in}}	&`song'		&\it{la\tbr{ko}}	&\ra&\it{la\tbr{ok}}	&`go' \\
			\it{atu\tbr{li}}	&\ra&\it{atu\tbr{il}}	&`person'	&\it{so\tbr{do}}	&\ra&\it{so\tbr{od}}	&`greet' \\
			\it{de\tbr{he}}		&\ra&\it{de\tbr{eh}}	&`some'		&\it{su\tbr{su}}	&\ra&\it{su\tbr{us}}	&`breast' \\
			\it{da\tbr{ke}}		&\ra&\it{da\tbr{ek}}	&`work'		&\it{di\tbr{ku}}	&\ra&\it{di\tbr{uk}}	&`beat' \\
			\it{o\tbr{ne}}		&\ra&\it{o\tbr{en}}		&`they'		&\it{la\tbr{lu}}	&\ra&\it{la\tbr{ul}}	&`palm-wine' \\
			\it{kla\tbr{pa}}	&\ra&\it{kla\tbr{ap}}	&`garden'
	\end{tabular}}
\end{exe}

\citet[47]{ba17b} also gives a handful of VCCV{\#} words
with metathesis of the final vowel across a consonant cluster:
\it{ba\tbr{ŋku}} {\ra} \it{ba\tbr{uŋk}} `bench', \it{sa\tbr{mpe}} {\ra} \it{sa\tbr{emp}} `until',
\it{ba\tbr{ndu}} {\ra} \it{ba\tbr{und}} `authority', and \it{sa\tbr{ksi}} {\ra} \it{sa\tbr{iks}} `witness'.
All these words are Malay loans with the exception of \it{bandu},
the etymology of which is unknown.

When the U\=/form ends in VCa,
the M\=/form in Semau Helong is described as being
derived by deleting the final /a/,
with the exception of words in which the penultimate vowel is also /a/
(such as \it{klapa} {\ra} \it{klaap} `garden' in \qf{ex:VCV->VVC-Hel} above).
Examples of such putative vowel deletion are given in \qf{ex:VCa->VC-Hel} below.

\begin{exe}
	\ex{Semau Helong V{\sub{α}}Ca{\sub{β}} {\ra} V{\sub{α}}C \hfill\citep[13f]{baca12}}\label{ex:VCa->VC-Hel}
	\sn{\gw\begin{tabular}{rcll}
			U\=/form					&			&M\=/form		& \\
			\it{hid\tbr{a}}	&{\ra}&\it{hid}	&`promise' \\
			\it{del\tbr{a}}	&{\ra}&\it{del}	&`chair' \\
			\it{doh\tbr{a}}	&{\ra}&\it{doh}	&`watch' \\
			\it{um\tbr{a}}	&{\ra}&\it{um}	&`house' \\
	\end{tabular}}
\end{exe}

Similarly, \citeauthor{st96b} (\citeyear{st96b}; \citeyear{st08}), describes M\=/forms VCa{\#}
final words as being derived by apocope,
though in his description this also includes
words whose penultimate vowel is /a/.

In Funai Helong the M\=/form of such words
is derived by metathesis with assimilation
of the final /a/ -- the same process found
in Mambae (\srf{sec:Mam}) and most varieties of Meto.
Examples are given in \qf{ex:VCa->VVC-FunHel} below.

\newpage
\begin{exe}
	\ex{Funai Helong V{\sub{α}}Ca {\ra} V{\sub{α}}V{\sub{α}}C}\label{ex:VCa->VVC-FunHel}
	\sn{\gw\begin{tabular}{rcll}
			U\=/form						&			&M\=/form						& \\
			\it{i\tbr{na}-ʔ}	&{\ra}&\it{i\tbr{in}}		&`mother' \\
			\it{ko\tbr{ma}}		&{\ra}&\it{ko\tbr{om}}	&`agree' \\
			\it{me\tbr{sa}}		&{\ra}&\it{me\tbr{es}}	&`one' \\
			\it{u\tbr{ma}}		&{\ra}&\it{u\tbr{um}}		&`house' \\
	\end{tabular}}
\end{exe}

When a stem ends in CVC{\#} the M\=/form
can be derived by CV {\ra} VC metathesis and deletion of the final consonant.
Two examples are \it{u\tbr{nit}} {\ra} \it{u\tbr{in}} `sugar'
and \it{pa\tbr{ni}ŋ} {\ra} \it{pa\tbr{in}} `give without concern' \citep[47]{ba17b}.
However, it appears that not all words which end
in a consonant have distinct U\=/forms and M\=/forms.

Similarly, words which end in a vowel sequence
do not seem to have distinct U\=/forms and M\=/forms.
A number of such words do appear to have contracted forms which are shorter than the full form,
however, the use of these contracted forms is not the same as the use
of M\=/forms formed by metathesis or apocope.

\subsubsection{Functions}\label{sec:HelFun}
All words which end in VCV are attested with both U\=/forms and M\=/forms in Helong.
However, the function of each form varies according to the word class of the word in question.
We can identify four different distributions of M\=/forms,
three of which are morphological and one of which is phonological.
For nouns M\=/forms mark specificity
and the presence of an attributive modifier (\srf{sec:MetNP}),
while for verbs M\=/forms mark imperfective aspect (\srf{sec:VerMet})
and with other word classes U\=/forms are used phrase finally
and M\=/forms phrase initially and phrase medially (\srf{sec:MetOthWorCla}).

\paragraph{Nominal metathesis}\label{sec:MetNP}
In the noun phrase metathesis marks specificity in Semau Helong.
Examples are given in \qf{ex:KatBukuEneng}--\qf{ex:*KatTotoangBuuk} below.
In \qf{ex:KatBukuEneng} the noun \it{buku} `book' is in the U\=/form, and is non-specific.
In sentence \qf{ex:KatTotoangBuuk} the noun \it{buuk}
`book' is in the M\=/form and is specific,
in \qf{ex:KatTotoangBuuk} specificity is further marked by the specific demonstrative \it{nas}.
Example \qf{ex:*KatTotoangBuuk} shows it is ungrammatical to use an M\=/form noun
with the non-specific demonstrative \it{las}.

\begin{exe}\let\eachwordone=\itshape
	\ex[]{\gll	kat bu\tbr{ku} ene\ng.\\
							take book{\tbrU} six\\
				\glt	`Take (any) six books.'}\label{ex:KatBukuEneng}
\end{exe}
\newpage
\begin{exe}\let\eachwordone=\itshape
	\ex[]{\gll	kat to{\tl}toa{\ng} bu\tbr{uk} ene{\ng} \tbr{n}-a-s.\\
							take \tsc{red}{\tl}all book{\tbrM} six \tsc{\tbr{spec}-dem-pl}\\
				\glt	`Take all (of those) six books.'}\label{ex:KatTotoangBuuk}
	\ex[*]{\gll	kat to{\tl}toa{\ng} buuk ene{\ng} l-a-s\\
							take \tsc{red}{\tl}all book{\M} six \tsc{nspec-dem-pl} \\ 
				\glt	`(Take all (of those) six books.)' \hfill\cite{ba15}}\label{ex:*KatTotoangBuuk}
\end{exe}

A similar example can be seen by comparing
sentences \qf{ex:NolKlapaLas} and \qf{ex:OenTamasLakos} below.
In \qf{ex:NolKlapaLas} the unmetathesised noun \it{klapa} `garden' agrees
in specificity with the non-specific demonstrative \it{las},
while in \qf{ex:OenTamasLakos} the same noun occurs metathesised
agreeing with the specific demonstrative \it{na}.

\begin{exe}\let\eachwordone=\itshape
	\ex{\gll	nol kla\tbr{pa} \tbr{l}-a-s puti{\ng} isin banan.\\
						and garden{\tbrU} \tsc{\tbr{nspec}-dem-pl} produce contents good\\
			\glt	`And gardens produce good contents.'}\label{ex:NolKlapaLas}
	\ex{\gll	oen tama-s lako-s se kla\tbr{ap} \tbr{n}-a.\\
						\tsc{3pl}{\M} enter\tsc{-pl} go\tsc{-pl} \tsc{all} garden{\tbrM} \tsc{\tbr{spec}-dem}\\
			\glt	`They go into that garden.' \hfill\citep[15]{baca12}}\label{ex:OenTamasLakos}
\end{exe}

When a noun occurs with a following modifier,
the modifier occurs in the U\=/form or M\=/forms
to signal the specificity of the noun.
In both \qf{ex:KatBukuDehe} and \qf{ex:KatBukuDeeh} below
the specificity of the noun phrase is marked by thi quantifier \it{dehe} `some'.
In \qf{ex:KatBukuDehe} this quantifier occurs in the U\=/form
and the referent is non-specific,
while in the equivalent example \qf{ex:KatBukuDeeh} it occurs in the M\=/form
and has a specific referent.

\begin{exe}\let\eachwordone=\itshape
	\ex{\gll	kat buku de\tbr{he}.\\
						take book{\U} some{\tbrU}\\
			\glt `Take some books.'}\label{ex:KatBukuDehe}
	\ex{\gll	kat buku de\tbr{eh} de{\ng} \tbr{n}-i-a.\\
						take book{\U} some{\tbrM} from \tsc{\tbr{spec}-prox-dem}\\
			\glt `Take some (of those) books from here.' \hfill\cite{ba15}}\label{ex:KatBukuDeeh}
\end{exe}

In the description of metathesis given by \citeauthor{st96b} (\citeyear{st96b}; \citeyear{st08}),
metathesis is reported as affecting every non-final element in the noun phrase.
Examples are given in \qf{ex:N+adj->Nm adj} below.
The change of medial /h/ {\ra} /s/ after metathesis
is a phenomenon specific to the Bolok dialect.

\newpage
\begin{exe}
	\ex{Bolok Helong noun + adj. {\ra} noun{\M} adj. \hfill\citep[477]{st96b}}\label{ex:N+adj->Nm adj}
	\sn{\gw\begin{tabular}{rcllll}
		\it{atu\tbr{li}}&+&\it{{\ng}e{\ng}oʔ}&{\ra}&\it{atu\tbr{il} {\ng}e{\ng}oʔ}&`person + stupid'\\
		\it{a\tbr{le}}	&+&\it{kunis}					&{\ra}&\it{a\tbr{el} kunis}				&`paddy + yellow'\\
		\it{le\tbr{lo}}	&+&\it{lima-ʔ}				&{\ra}&\it{le\tbr{ol} limaʔ}			&`day + fifth' = `Friday'\\
		\it{bla\tbr{ho}}&+&\it{mutiʔ}					&{\ra}&\it{bla\tbr{os} mutiʔ}			&`mouse + white'\\
		\it{ba\tbr{hi}}	&+&\it{mea}						&{\ra}&\it{ba\tbr{is} mea}				&`pig + red'\\
	\end{tabular}}
\end{exe}

\citeauthor{st96b} also identifies metathesis as marking the
difference between attributive modifiers and predicative modifiers,
as shown in \qf{ex:LeloDeneq} and \qf{ex:LeolDeneq} below.
This description matches closely one of the functions of metathesis
found in both Rotuman (\srf{sec:RotFun}) and Leti (\srf{sec:LetFun}).

\begin{exe}\let\eachwordone=\itshape
	\ex{\gll \brac{NP} le\tbr{lo} \bracr{} \brac{PRED} deneʔ \bracr{} \\
						{} sun{\tbrU} go.down \\
			\glt \lh{\brac{NP}}`The sun sets.' }\label{ex:LeloDeneq}
	\ex{\gll \brac{NP} le\tbr{ol} deneʔ \bracr{} \\
						{} sun{\tbrM} go.down\\
			\glt \lh{\brac{NP}}`west' \hfill\citep[477]{st96b}}\label{ex:LeolDeneq}
\end{exe}

The functions of metathesis within the noun phrase as described by
\cite{st96b} is different to the function described by \cite{baca12}.
It is possible that these two descriptions
of Helong metathesis differ due to each describing
a different dialect of Helong.
\cite{baca12} describe Semau Helong
and \cite{st96b} describes Bolok Helong.

In my Funai Helong data
attributive modifiers normally trigger metathesis on CV{\#} head nouns.
Examples include \it{a\tbr{le}} `rice' + \it{kpohot} `husk'
{\ra} \it{a\tbr{el} kpohot} `unhusked rice' and
\it{ba\tbr{tu}} `rock' + \it{tlokon} `coral' {\ra}
\it{ba\tbr{ut} tlokon} `coral rock'.
My Funai Helong data is currently insufficient to
determine whether metathesis also marks specificity in this dialect.

To summarise, metathesis marks specificity in the Helong noun phrase
and/or marks the presence of an attributive modifier.
Marking of an attributive modifier is similar to the functions
of metathesis in Rotuman, Leti, Roma, and Amarasi.
Marking of specificity is similar to one of the functions of metathesis in Rotuman and Leti.
In both Rotuman and Leti it is the (historically) metathesised form which is indefinite,
while in Helong it is the \it{un}metathesised form which is non-specific.
Although a similar morphological process is used in each instance,
the function of that morphological process in Helong is the opposite
to that found in Rotuman and Leti.

\paragraph{Verbal metathesis}\label{sec:VerMet}
Metathesis has two main distributions for verbs in Helong.
Firstly, verbs are in the U\=/form when they do not take an object
and in the M\=/form when they do take an object.
This is similar to the use of M\=/forms in the noun
phrase with an attributive modifier as described by \cite{st96b}.

Compare examples \qf{ex:HelObj1} and \qf{ex:HelObj2} below.
In \qf{ex:HelObj1} the verb \it{dake} `work, do' occurs in the U\=/form
as there is no overt object.
However, in example \qf{ex:HelObj2} there is an object
and the verb takes the M\=/form.

\begin{exe}\let\eachwordone=\itshape
	\ex{\gll	un lako da\tbr{ke}.\\
						\tsc{3sg} go{\U} work{\tbrU}\\
			\glt `S/he's gone to (do some) work.'}\label{ex:HelObj1}
	\ex{\gll	un lako da\tbr{ek} kukis.\\
						\tsc{3sg} go{\U} work{\tbrM} cake\\
			\glt `S/he's gone to make cake.' \hfill\citep[11]{baca12}}\label{ex:HelObj2}
\end{exe}

The other use of metathesis on verbs is to mark aspect.
U\=/forms mark perfective or completed aspect,
while M\=/forms mark imperfective or progressive aspect.
Compare the examples in \qf{ex:HelAsp1a}--\qf{ex:HelAsp2b} below.
In examples \qf{ex:HelAsp1a} and \qf{ex:HelAsp2a} the verb \it{lako} `go'
is in the U\=/form and the sentence thus has perfective aspect.
In examples \qf{ex:HelAsp1b} and \qf{ex:HelAsp2b}, however,
the verb \it{laok} is in the M\=/form,
and each example is imperfective.

\begin{exe}\let\eachwordone=\itshape
	\ex{\gll	un la\tbr{ko} daek kukis.\\
						\tsc{3sg} go{\tbrU} work{\M} cake\\
			\glt `S/he's (already) left to make cake.' }\label{ex:HelAsp1a}
	\ex{\gll	un la\tbr{ko} daek kukis nol asii l-a?\\
						\tsc{3sg} go{\tbrU} work{\M} cake with who \tsc{nspec-dem}\\
			\glt `With whom has s/he (already) left to make cake?'}\label{ex:HelAsp2a}
	\ex{\gll	un la\tbr{ok} daek kukis.\\
						\tsc{3sg} go{\tbrM} work{\M} cake\\
			\glt `S/he's (in the process of) leaving to make cake.' }\label{ex:HelAsp1b} %\hfill\citep[11]{baca12}
	\ex{\gll	un la\tbr{ok} daek kukis nol asii l-a?\\
						\tsc{3sg} go{\tbrM} work{\M} cake with who \tsc{nspec-dem}\\
			\glt `With whom is s/he (in the process of) leaving to make cake?' }\label{ex:HelAsp2b} %\hfill\citep[11]{baca12}
\end{exe}

\paragraph{Metathesis with other word classes}\label{sec:MetOthWorCla}
In addition to nouns and verbs, other word classes including
adjectives, adverbs, ``particles'' and most pronouns also have a U\=/form and an M\=/form.
For these other word classes U\=/forms are (mostly) used phrase finally
and M\=/forms are used phrase initially and phrase medially.
This may be a case of phonologically conditioned metathesis.
Examples of non-final adjectival M\=/forms are
given in \qf{ex:HelAdj3}--\qf{ex:HelAdj4} below,
which can be contrasted with the final U\=/form in \qf{ex:HelAdj1}.

\begin{exe}\let\eachwordone=\itshape
	\ex{\gll	auk le\tbr{ko}.\\
						\tsc{1sg} beautiful{\tbrU}\\
			\glt	`I'm beautiful.'}\label{ex:HelAdj1}
	\ex{\gll	auk le\tbr{ok} dui.\\
						\tsc{1sg} beautiful{\tbrM} more\\
			\glt	`I'm more beautiful.'}\label{ex:HelAdj3}
	\ex{\gll	auk le\tbr{ok} isi.\\
						\tsc{1sg} beautiful{\tbrM} very{\U}\\
			\glt	`I'm very beautiful.'}\label{ex:HelAdj2}
	\ex{\gll	auk le\tbr{ok} baktetebes.\\
						\tsc{1sg} beautiful{\tbrM} truly\\
			\glt	`I'm truly beautiful.' \hfill\citep[12f]{baca12}}\label{ex:HelAdj4}
\end{exe}

\cite{baca12} give one exception to this pattern.
This is the adverbial \it{ana lo} `a lot, enough, exceptional(ly)'.
This phrase is a conventionalised understatement, literally meaning `not a little'
and is formally a separate clause.
Before this adverbial, an adjective occurs in the U\=/form.
This is shown in \qf{ex:HelAdj5} below.

\begin{exe}\let\eachwordone=\itshape
	\ex{\gll	auk le\tbr{ko} {ana lo}.\\
							\tsc{1sg} beautiful{\tbrU} exceptionally\\
				\glt	`I'm exceptionally beautiful.' \hfill\citep[12]{baca12}}\label{ex:HelAdj5}
\end{exe}

An example of a metathesised adverbial, \it{nabael} `still'
is given in \qf{ex:HelAdv2} below.
This can be contrasted with the unmetathesised
and phrase-final \it{nabale} in \qf{ex:HelAdv1}.

\begin{exe}\let\eachwordone=\itshape
	\ex{\gll	mo nahin n-a la{\ng}a isi lo naba\tbr{le}.\\
						but meaning \tsc{spec-dem} clear very \tsc{neg} still{\tbrU}\\
			\glt	`But the meaning is still not very clear.'}\label{ex:HelAdv1}
	\ex{\gll	mo nahin n-a la{\ng}a isi lo naba\tbr{el} tuun.\\
						but meaning \tsc{spec-dem} clear very \tsc{neg} still{\tbrM} just\\
			\glt	`But the meaning is just still not very clear.'
						\citep[13]{baca12}}\label{ex:HelAdv2} %\hfill\citep[13]{baca12}
\end{exe}

Finally, most of the pronouns have both U\=/forms and M\=/forms.
The Semau Helong pronouns are given in \trf{tab:HelPro} below.
Note that despite the fact that the \tsc{2pl} pronoun \it{mia} ends in a vowel sequence,
it has an M\=/form, formed by deleting the final /a/.
The \tsc{1sg} pronoun \it{auk} does not have distinct U\=/forms and M\=/forms,
however, it does have a shorter form \it{au},
which is used when the pronoun is an object.
The form \it{auk} is used for \tsc{1sg} subjects.\footnote{
		The Semau Helong \tsc{1sg} subject pronoun \it{auk}
		is historically a result of reanalysis of the pronoun \it{au} followed
		by a \tsc{1sg} verbal agreement prefix \it{k-}.}

\begin{table}[h]
	\caption[Semau Helong pronouns]{Semau Helong pronouns \citep[16]{baca12}}\label{tab:HelPro}
	\begin{tabular}{lll}	\lsptoprule
							&U\=/form					&M\=/form		\\	\midrule
		\tsc{1sg}	&\it{auk}				&\it{auk}	\\
		\tsc{2sg}	&\it{ku}				&\it{ku}	\\
		\tsc{3sg}	&\it{un\tbr{a}}	&\it{un}	\\
		\tsc{1pi}	&\it{kit\tbr{a}}&\it{kit}	\\
		\tsc{1px}	&\it{ka\tbr{mi}}&\it{ka\tbr{im}}\\
		\tsc{2pl}	&\it{mi\tbr{a}}	&\it{mi}	\\
		\tsc{3pl}	&\it{o\tbr{ne}}	&\it{o\tbr{en}}\\	\lspbottomrule
	\end{tabular}
\end{table}

Examples of the U\=/form and M\=/form of the \tsc{3pl} pronoun \it{one}
are given in \qf{ex:HelPro1}--\qf{ex:HelPro3} below.
In \qf{ex:HelPro1} the \tsc{3pl} pronoun \it{one}
is sentence final and thus occurs unmetathesised.
In sentences \qf{ex:HelPro2} and \qf{ex:HelPro3} the same pronoun occurs non-finally
and is thus in the M\=/form.

\begin{exe}\let\eachwordone=\itshape
	\ex{\gll	kaim lako-{\ng} meo o\tbr{ne}.\\
						\tsc{1pl.excl\M} go-1 visit \tsc{3pl\tbrU}\\
			\glt `We're going to visit them.' }\label{ex:HelPro1}
	\ex{\gll	o\tbr{en} lako-s meo in-ama-n n-u-a-s.\\
						\tsc{3pl\tbrM} go-\tsc{pl} visit parents-\tsc{pl.gen} \tsc{spec-remote-dem-pl}\\
			\glt `They're going to visit their parents.'}\label{ex:HelPro2}
	\ex{\gll	tiata o\tbr{en} lako-s meo o\tbr{en} in-ama-n n-a-s.\\
						so \tsc{3pl\tbrM} go-\tsc{pl} visit \tsc{3pl\tbrM} parents-\tsc{pl.gen} \tsc{spec-dem-pl}\\
			\glt `So, they're going to visit their parents.' \hfill\citep[12]{baca12}}\label{ex:HelPro3}
\end{exe}

\subsubsection{Summary}
Formally, metathesis in Helong is associated with
two other phonological process: deletion of final /a/
in Semau and Bolok and assimilation of final /a/ in Funai.
In terms of function, there are two kinds of metathesis in Helong.
There is a process of phonological metathesis whereby
non-final words which are neither verbs nor members of the
noun phrase occur metathesised phrase medially.
Metathesis is morphological for nouns and verbs.
It marks specificity and/or the presence of an attributive modifier
in the noun phrase and imperfective aspect in verbs.

There are two main similarities between metathesis in Helong and in Amarasi.
Firstly, in each language words with a final /a/ do not follow the normal
pattern of CV {\ra} VC metathesis.
In particular, assimilation of final /a/ after metathesis
occurs in both Funai Helong and Amarasi.
Secondly, in both languages metathesis has
different functions for nouns and verbs.

Even though Amarasi and Helong are immediate neighbours,
the similarities between metathesis in each
are fewer and less striking than the similarities between metathesis
in Amarasi and Mambae or between Amarasi and Leti.

While much more work on the history of these languages
is needed to properly understand why this is the case,
one possible source could be to do with metathesis being
manipulated (consciously or otherwise) as a marker of ethnic identity;
when groups are in contact with one another there may be a greater
imperative to differentiate between one another.
Metathesis as marker of identity is explored in more detail in \srf{sec:MetIde}.\il{Helong|)}