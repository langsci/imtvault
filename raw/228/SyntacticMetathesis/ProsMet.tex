\section{A prosodic analysis}\label{sec:ProsMet}
It is possible to analyse the metathesis discussed in this chapter
as a prosodic process whereby each member
of an attributive phrase is a part of a compound
with metathesis being triggered by an internal prosodic word boundary.
While such an analysis is possible, I conclude by maintaining
the morphological analysis originally proposed in \srf{sec:UniAna}.

In this chapter I have treated attributive modification as occurring within the syntax.
However, such modification can also occur in the morphology.
Under such an approach, attributive phrases would be analysed as compounds.

With this analysis, attributive phrases
such as \ve{neon meseʔ} [ˌnɛ.ɔnˈmɛsɛʔ] `first day'
would be analysed as compounds, and hence a single word,
while phrases which are non-attributive,
such as \ve{neno meseʔ} [ˌnɛnɔˈmɛsɛʔ] `one day',
would constitute two separate words.

The different structures of these two phrases under
this analysis are shown in \qf{as:neon-meseq}
and \qf{as:neno-meseq} respectively,
in which the prosodic structures are further
linked to their syntactic structures.
I leave aside for the moment the analysis
involving empty C-slots (\srf{sec:EmpCSlo}).

\begin{multicols}{2}
	\begin{exe}
		\exa{\xy
			<4.75em,6cm>*\as{NP}="np1",
			<4.75em,5cm>*\as{N}="n1",
			<2.5em,4cm>*\as{\hp{\sub{1}}PrWd\sub{1}}="PrWd1",<7em,4cm>*\as{\hp{\sub{2}}PrWd\sub{2}}="PrWd2",
			<2.5em,3cm>*\as{\hp{\sub{\tsc{m}}}Ft\sub{\tsc{m}}}="Ft1",<7em,3cm>*\as{\hp{\sub{2}}Ft\sub{2}}="Ft2",
			<1.5em,2cm>*\as{\hp{\sub{1}}σ\sub{1}}="s1",<3.5em,2cm>*\as{\hp{\sub{2}}σ\sub{2}}="s2",<6em,2cm>*\as{\hp{\sub{3}}σ\sub{3}}="s3",<8em,2cm>*\as{\hp{\sub{4}}σ\sub{4}}="s4",
			<1em,1cm>*\as{n}="cv1",<2em,1cm>*\as{e}="cv2",<3em,1cm>*\as{o}="cv3",<4em,1cm>*\as{n}="cv4",<5em,1cm>*\as{m}="cv5",<6em,1cm>*\as{e}="cv6",<7em,1cm>*\as{s}="cv7",<8em,1cm>*\as{e}="cv8",<9em,1cm>*\as{ʔ}="cv9",
			<2.5em,0cm>*\as{\hp{\sub{1}}M\sub{1}}="m1",<7em,0cm>*\as{\hp{\sub{2}}M\sub{2}}="m2",
			"m1"+U;"cv1"+D**\dir{-};"m1"+U;"cv2"+D**\dir{-};"m1"+U;"cv3"+D**\dir{-};"m1"+U;"cv4"+D**\dir{-};"m2"+U;"cv5"+D**\dir{-};"m2"+U;"cv6"+D**\dir{-};"m2"+U;"cv7"+D**\dir{-};"m2"+U;"cv8"+D**\dir{-};"m2"+U;"cv9"+D**\dir{-};
			"cv1"+U;"s1"+D**\dir{-};"cv2"+U;"s1"+D**\dir{-};"cv3"+U;"s2"+D**\dir{-};"cv4"+U;"s2"+D**\dir{-};
			"cv5"+U;"s3"+D**\dir{-};"cv6"+U;"s3"+D**\dir{-};"cv7"+U;"s3"+D**\dir{-};"cv7"+U;"s4"+D**\dir{-};"cv8"+U;"s4"+D**\dir{-};"cv9"+U;"s4"+D**\dir{-};
			"s1"+U;"Ft1"+D**\dir{-};"s2"+U;"Ft1"+D**\dir{-};"s3"+U;"Ft2"+D**\dir{-};"s4"+U;"Ft2"+D**\dir{-};
			"Ft1"+U;"PrWd1"+D**\dir{-};"Ft2"+U;"PrWd2"+D**\dir{-};
			"PrWd1"+U;"n1"+D**\dir{-};"PrWd2"+U;"n1"+D**\dir{-};"n1"+D**\dir{-};"n1"+U;"np1"+D**\dir{-};
			<4.5em,2.5cm>*\as{\tikz[red,thick,dashed,baseline=0.9ex]\draw (0,0) -- (0,3.5cm);}="line",
		\endxy}\label{as:neon-meseq}
		\exa{\xy
			<5em,7cm>*\as{NumP}="NumP1",
			<3em,6cm>*\as{NP}="np1",
			<3em,5cm>*\as{N}="n1",<7em,5cm>*\as{Num}="num1",
			<3em,4cm>*\as{\hp{\sub{1}}PrWd\sub{1}}="PrWd1",<7em,4cm>*\as{\hp{\sub{2}}PrWd\sub{2}}="PrWd2",
			<3em,3cm>*\as{\hp{\sub{1}}Ft\sub{1}}="Ft1",<7em,3cm>*\as{\hp{\sub{2}}Ft\sub{2}}="Ft2",
			<2em,2cm>*\as{\hp{\sub{1}}σ\sub{1}}="s1",<4em,2cm>*\as{\hp{\sub{2}}σ\sub{2}}="s2",<6em,2cm>*\as{\hp{\sub{3}}σ\sub{3}}="s3",<8em,2cm>*\as{\hp{\sub{4}}σ\sub{4}}="s4",
			<1em,1cm>*\as{n}="cv1",<2em,1cm>*\as{e}="cv2",<3em,1cm>*\as{n}="cv3",<4em,1cm>*\as{o}="cv4",<5em,1cm>*\as{m}="cv5",<6em,1cm>*\as{e}="cv6",<7em,1cm>*\as{s}="cv7",<8em,1cm>*\as{e}="cv8",<9em,1cm>*\as{ʔ}="cv9",
			<2.5em,0cm>*\as{\hp{\sub{1}}M\sub{1}}="m1",<7em,0cm>*\as{\hp{\sub{2}}M\sub{2}}="m2",
			"m1"+U;"cv1"+D**\dir{-};"m1"+U;"cv2"+D**\dir{-};"m1"+U;"cv3"+D**\dir{-};"m1"+U;"cv4"+D**\dir{-};"m2"+U;"cv5"+D**\dir{-};"m2"+U;"cv6"+D**\dir{-};"m2"+U;"cv7"+D**\dir{-};"m2"+U;"cv8"+D**\dir{-};"m2"+U;"cv9"+D**\dir{-};
			"cv1"+U;"s1"+D**\dir{-};"cv2"+U;"s1"+D**\dir{-};"cv3"+U;"s1"+D**\dir{-};"cv3"+U;"s2"+D**\dir{-};"cv4"+U;"s2"+D**\dir{-};"cv5"+U;"s2"+D**\dir{-};
			"cv5"+U;"s3"+D**\dir{-};"cv6"+U;"s3"+D**\dir{-};"cv7"+U;"s3"+D**\dir{-};"cv7"+U;"s4"+D**\dir{-};"cv8"+U;"s4"+D**\dir{-};"cv9"+U;"s4"+D**\dir{-};
			"s1"+U;"Ft1"+D**\dir{-};"s2"+U;"Ft1"+D**\dir{-};"s3"+U;"Ft2"+D**\dir{-};"s4"+U;"Ft2"+D**\dir{-};
			"Ft1"+U;"PrWd1"+D**\dir{-};"Ft2"+U;"PrWd2"+D**\dir{-};
			"PrWd1"+U;"n1"+D**\dir{-};"PrWd2"+U;"num1"+D**\dir{-};
			"n1"+U;"np1"+D**\dir{-};"np1"+U;"NumP1"+D**\dir{-};"num1"+U;"NumP1"+D**\dir{-};
			<5em,1cm>*\as{\tikz[red,thick,dashed,baseline=0.9ex]\draw (0,0) rectangle (0.45cm,0.5cm);}="box",
		\endxy}\label{as:neno-meseq}
	\end{exe}
\end{multicols}

The attributive phrase \ve{neon meseʔ} `first day' in \qf{as:neon-meseq}
consists of two prosodic words which are linked to a single syntactic word,
while for the non-attributive phrase \ve{neno meseʔ} `one day' in \qf{as:neno-meseq}
each prosodic word is linked to separate syntactic words.\footnote{
		In each phrase the individual morphemes must be assigned individual
		prosodic words as each bears penultimate primary or secondary stress.}
As illustrated by the red line in \qf{as:neon-meseq},
the attributive phrase has a crisp edge between
each prosodic word while the non-attributive phrase does not.
Metathesis could thus be analysed as an
automatic response to a crisp edge constraint.

In Chapter \ref{ch:PhoMet} I proposed that metathesis
before vowel-initial enclitics occurs in response to
a constraint \tsc{Crisp-Edge}[PrWd] which prohibits
elements from being linked to more than one prosodic word.
This constraint could be reformulated to only disallow
fuzzy borders after \emph{internal} prosodic words,
while allowing them between prosodic words
which are do not share a higher prosodic structure.

Individual syntactic words, such as N, Num, and so on,
would be prosodic elements under this analysis.
This means that the prosodic words of attributive
phrases are internal prosodic words,
while those of a non-attributive phrase are not,
as represented in \qf{as:neon-meseq} and \qf{as:neno-meseq} above.

As it currently stands this analysis does
account for metathesis of CVC{\#} final roots
which are the first member of an attributive phrase
in examples such as \ve{mu\tbr{ʔi}t} `animal' + \ve{fui} `wild'
{\ra} \brac{NP} \ve{mu\tbr{iʔ} fui} \bracr{} `wild animal',
as such phrases would already have an internal prosodic
word boundary as shown in unattested \qf{as:muqit-fui}.

\begin{multicols}{2}
	\begin{exe}
		\exa{\xy
			<1em,5cm>*\as{*}="*",<5.125em,5cm>*\as{N}="n1",
			<3em,4cm>*\as{\hp{\sub{1}}PrWd\sub{1}}="PrWd1",<7.25em,4cm>*\as{\hp{\sub{2}}PrWd\sub{2}}="PrWd2",
			<3em,3cm>*\as{\hp{\sub{1}}Ft\sub{1}}="Ft1",<7.25em,3cm>*\as{\hp{\sub{2}}Ft\sub{2}}="Ft2",
			<2em,2cm>*\as{\hp{\sub{1}}σ\sub{1}}="s1",<4em,2cm>*\as{\hp{\sub{2}}σ\sub{2}}="s2",<6.5em,2cm>*\as{\hp{\sub{3}}σ\sub{3}}="s3",<8em,2cm>*\as{\hp{\sub{4}}σ\sub{4}}="s4",
			<1em,1cm>*\as{m}="cv1",<2em,1cm>*\as{u}="cv2",<3em,1cm>*\as{ʔ}="cv3",<4em,1cm>*\as{i}="cv4",<5em,1cm>*\as{t}="cv5",<6em,1cm>*\as{f}="cv6",<7em,1cm>*\as{u}="cv7",<8em,1cm>*\as{i}="cv8",
			<3em,0cm>*\as{\hp{\sub{1}}M\sub{1}}="m1",<7em,0cm>*\as{\hp{\sub{2}}M\sub{2}}="m2",
			"m1"+U;"cv1"+D**\dir{-};"m1"+U;"cv2"+D**\dir{-};"m1"+U;"cv3"+D**\dir{-};"m1"+U;"cv4"+D**\dir{-};"m1"+U;"cv5"+D**\dir{-};"m2"+U;"cv6"+D**\dir{-};"m2"+U;"cv7"+D**\dir{-};"m2"+U;"cv8"+D**\dir{-};
			"cv1"+U;"s1"+D**\dir{-};"cv2"+U;"s1"+D**\dir{-};"cv3"+U;"s1"+D**\dir{-};"cv3"+U;"s2"+D**\dir{-};"cv4"+U;"s2"+D**\dir{-};"cv5"+U;"s2"+D**\dir{-};
			"cv6"+U;"s3"+D**\dir{-};"cv7"+U;"s3"+D**\dir{-};"cv8"+U;"s4"+D**\dir{-};
			"s1"+U;"Ft1"+D**\dir{-};"s2"+U;"Ft1"+D**\dir{-};"s3"+U;"Ft2"+D**\dir{-};"s4"+U;"Ft2"+D**\dir{-};
			"Ft1"+U;"PrWd1"+D**\dir{-};"Ft2"+U;"PrWd2"+D**\dir{-};
			"PrWd1"+U;"n1"+D**\dir{-};"PrWd2"+U;"n1"+D**\dir{-};
			<5.5em,2.5cm>*\as{\tikz[red,thick,dashed,baseline=0.9ex]\draw (0,0) -- (0,3.5cm);}="line",
		\endxy}\label{as:muqit-fui}
	\end{exe}
\end{multicols}

To account for this, we could propose that nominal
attribution is marked by consonant deletion.
That is, the construct form is fundamentally
marked by a morphological rule of subtraction.
Subtraction creates an illicit fuzzy border in which
the first consonant of the third syllable is also the
final consonant of the second syllable
and thus shared between two prosodic words,
as represented in \qf{as:muqi-fui}.
This is resolved by metathesis which creates a crisp edge at
the internal prosodic word boundary,
as illustrated in \qf{as:muiq-fui}.

\begin{multicols}{2}
	\begin{exe}
		\ex{\begin{xlist}
			\exa{\xy
				<4.625em,5cm>*\as{N}="n1",
				<3em,4cm>*\as{\hp{\sub{1}}PrWd\sub{1}}="PrWd1",<6.25em,4cm>*\as{\hp{\sub{2}}PrWd\sub{2}}="PrWd2",
				<3em,3cm>*\as{\hp{\sub{1}}Ft\sub{1}}="Ft1",<6.25em,3cm>*\as{\hp{\sub{2}}Ft\sub{2}}="Ft2",
				<2em,2cm>*\as{\hp{\sub{1}}σ\sub{1}}="s1",<4em,2cm>*\as{\hp{\sub{2}}σ\sub{2}}="s2",<5.5em,2cm>*\as{\hp{\sub{3}}σ\sub{3}}="s3",<7em,2cm>*\as{\hp{\sub{4}}σ\sub{4}}="s4",
				<1em,1cm>*\as{m}="cv1",<2em,1cm>*\as{u}="cv2",<3em,1cm>*\as{ʔ}="cv3",<4em,1cm>*\as{i}="cv4",<5em,1cm>*\as{f}="cv5",<6em,1cm>*\as{u}="cv6",<7em,1cm>*\as{i}="cv7",
				<2.5em,0cm>*\as{\hp{\sub{1}}M\sub{1}}="m1",<6em,0cm>*\as{\hp{\sub{2}}M\sub{2}}="m2",
				"m1"+U;"cv1"+D**\dir{-};"m1"+U;"cv2"+D**\dir{-};"m1"+U;"cv3"+D**\dir{-};"m1"+U;"cv4"+D**\dir{-};"m2"+U;"cv5"+D**\dir{-};"m2"+U;"cv6"+D**\dir{-};"m2"+U;"cv7"+D**\dir{-};
				"cv1"+U;"s1"+D**\dir{-};"cv2"+U;"s1"+D**\dir{-};"cv3"+U;"s1"+D**\dir{-};"cv3"+U;"s2"+D**\dir{-};"cv4"+U;"s2"+D**\dir{-};"cv5"+U;"s2"+D**\dir{-};
				"cv5"+U;"s3"+D**\dir{-};"cv6"+U;"s3"+D**\dir{-};"cv7"+U;"s4"+D**\dir{-};
				"s1"+U;"Ft1"+D**\dir{-};"s2"+U;"Ft1"+D**\dir{-};"s3"+U;"Ft2"+D**\dir{-};"s4"+U;"Ft2"+D**\dir{-};
				"Ft1"+U;"PrWd1"+D**\dir{-};"Ft2"+U;"PrWd2"+D**\dir{-};"PrWd1"+U;"n1"+D**\dir{-};"PrWd2"+U;"n1"+D**\dir{-};
				<5em,1cm>*\as{\tikz[red,thick,dashed,baseline=0.9ex]\draw (0,0) rectangle (0.4cm,0.5cm);}="box",
			\endxy}\label{as:muqi-fui}
			\exa{\xy
				<4.375em,5cm>*\as{N}="n1",
				<2.5em,4cm>*\as{\hp{\sub{1}}PrWd\sub{1}}="PrWd1",<6.25em,4cm>*\as{\hp{\sub{2}}PrWd\sub{2}}="PrWd2",
				<2.5em,3cm>*\as{\hp{\sub{\tsc{m}}}Ft\sub{\tsc{m}}}="Ft1",<6.25em,3cm>*\as{\hp{\sub{2}}Ft\sub{2}}="Ft2",
				<1.5em,2cm>*\as{\hp{\sub{1}}σ\sub{1}}="s1",<3.5em,2cm>*\as{\hp{\sub{2}}σ\sub{2}}="s2",<5.5em,2cm>*\as{\hp{\sub{3}}σ\sub{3}}="s3",<7em,2cm>*\as{\hp{\sub{4}}σ\sub{4}}="s4",
				<1em,1cm>*\as{m}="cv1",<2em,1cm>*\as{u}="cv2",<3em,1cm>*\as{i}="cv3",<4em,1cm>*\as{ʔ}="cv4",<5em,1cm>*\as{f}="cv5",<6em,1cm>*\as{u}="cv6",<7em,1cm>*\as{i}="cv7",
				<2.5em,0cm>*\as{\hp{\sub{1}}M\sub{1}}="m1",<6em,0cm>*\as{\hp{\sub{2}}M\sub{2}}="m2",
				"m1"+U;"cv1"+D**\dir{-};"m1"+U;"cv2"+D**\dir{-};"m1"+U;"cv3"+D**\dir{-};"m1"+U;"cv4"+D**\dir{-};"m2"+U;"cv5"+D**\dir{-};"m2"+U;"cv6"+D**\dir{-};"m2"+U;"cv7"+D**\dir{-};
				"cv1"+U;"s1"+D**\dir{-};"cv2"+U;"s1"+D**\dir{-};"cv3"+U;"s2"+D**\dir{-};"cv4"+U;"s2"+D**\dir{-};
				"cv5"+U;"s3"+D**\dir{-};"cv6"+U;"s3"+D**\dir{-};"cv7"+U;"s4"+D**\dir{-};
				"s1"+U;"Ft1"+D**\dir{-};"s2"+U;"Ft1"+D**\dir{-};"s3"+U;"Ft2"+D**\dir{-};"s4"+U;"Ft2"+D**\dir{-};
				"Ft1"+U;"PrWd1"+D**\dir{-};"Ft2"+U;"PrWd2"+D**\dir{-};"PrWd1"+U;"n1"+D**\dir{-};"PrWd2"+U;"n1"+D**\dir{-};
				<4.5em,2.5cm>*\as{\tikz[red,thick,dashed,baseline=0.9ex]\draw (0,0) -- (0,3.5cm);}="line",
			\endxy}\label{as:muiq-fui}
		\end{xlist}}
	\end{exe}
\end{multicols}

While this analysis may initially seem somewhat counter-intuitive,
it is in fact consistent with the history
of nominal M\=/forms in which consonant deletion
historically preceded metathesis, as discussed in \srf{sec:LosFinCon}.

Under this analysis serial verb constructions would similarly be viewed as compounds.
Metathesis of CV{\#} final verbs when they are the first
member of an SVC occurs for the same reason as it does for nouns:
to create a crisp edge between the internal prosodic words of the compound.
This is represented in \qf{as:misoup-main} below for
\ve{mi-sopu} + \ve{m-ain} {\ra} \ve{mi-soup m-ain} `(we) first completed'.

The lack of metathesis for CVC{\#} final verbs (\srf{sec:PhoResMfrSVC}),
such as \ve{na-honiʔ n-ain} `(she) first gave birth' can be explained
by the fact that there is already a crisp edge between the prosodic
words of this SVC, as represented in \qf{as:nahoniq-nain}.
We would simply posit that, unlike nouns, deletion of final
consonants is not a marker of attribution for verbs.
This could be explained by the fact that deletion of final consonants
is already used to derive nouns from verbs (\srf{sec:BasVerDer}).

\begin{multicols}{2}
	\begin{exe}
		\exa{\xy
			<5.75em,5cm>*\as{V}="n1",
			<3em,4cm>*\as{\hp{\sub{1}}PrWd\sub{1}}="PrWd1",<8.5em,4cm>*\as{\hp{\sub{2}}PrWd\sub{2}}="PrWd2",
			<4.5em,3cm>*\as{\hp{\sub{\tsc{m}}}Ft\sub{\tsc{m}}}="Ft1",<8.5em,3cm>*\as{\hp{\sub{\tsc{m}}}Ft\sub{\tsc{m}}}="Ft2",
			<2em,2cm>*\as{\hp{\sub{1}}σ\sub{1}}="s1",<3.5em,2cm>*\as{\hp{\sub{2}}σ\sub{2}}="s2",<5.5em,2cm>*\as{\hp{\sub{3}}σ\sub{3}}="s3",<7.5em,2cm>*\as{\hp{\sub{4}}σ\sub{4}}="s4",<9.5em,2cm>*\as{\hp{\sub{5}}σ\sub{5}}="s5",
			<1em,1cm>*\as{m}="cv1",<2em,1cm>*\as{i}="cv2",<3em,1cm>*\as{s}="cv3",<4em,1cm>*\as{o}="cv4",<5em,1cm>*\as{u}="cv5",<6em,1cm>*\as{p}="cv6",<7em,1cm>*\as{m}="cv7",<8em,1cm>*\as{a}="cv8",<9em,1cm>*\as{i}="cv9",<10em,1cm>*\as{n}="cv10",
			<1.5em,0cm>*\as{\hp{\sub{1}}M\sub{1}}="m1",<4.5em,0cm>*\as{\hp{\sub{2}}M\sub{2}}="m2",<7em,0cm>*\as{\hp{\sub{3}}M\sub{3}}="m3",<9em,0cm>*\as{\hp{\sub{4}}M\sub{4}}="m4",<2.5em,0cm>*\as{-}="-1",<8em,0cm>*\as{-}="-2",
			"m1"+U;"cv1"+D**\dir{-};"m1"+U;"cv2"+D**\dir{-};"m2"+U;"cv3"+D**\dir{-};"m2"+U;"cv4"+D**\dir{-};"m2"+U;"cv5"+D**\dir{-};"m2"+U;"cv6"+D**\dir{-};"m3"+U;"cv7"+D**\dir{-};"m4"+U;"cv8"+D**\dir{-};"m4"+U;"cv9"+D**\dir{-};"m4"+U;"cv10"+D**\dir{-};
			"cv1"+U;"s1"+D**\dir{-};"cv2"+U;"s1"+D**\dir{-};"cv3"+U;"s1"+D**\dir{-};"cv3"+U;"s2"+D**\dir{-};"cv4"+U;"s2"+D**\dir{-};"cv5"+U;"s3"+D**\dir{-};"cv6"+U;"s3"+D**\dir{-};"cv7"+U;"s4"+D**\dir{-};"cv8"+U;"s4"+D**\dir{-};"cv9"+U;"s5"+D**\dir{-};"cv10"+U;"s5"+D**\dir{-};
			"s2"+U;"Ft1"+D**\dir{-};"s3"+U;"Ft1"+D**\dir{-};"s4"+U;"Ft2"+D**\dir{-};"s5"+U;"Ft2"+D**\dir{-};
			"s1"+U;"PrWd1"+D**\dir{-};"Ft1"+U;"PrWd1"+D**\dir{-};"Ft2"+U;"PrWd2"+D**\dir{-};"PrWd1"+U;"n1"+D**\dir{-};"PrWd2"+U;"n1"+D**\dir{-};
			<6.5em,2.5cm>*\as{\tikz[red,thick,dashed,baseline=0.9ex]\draw (0,0) -- (0,3.5cm);}="line",
		\endxy}\label{as:misoup-main}
	\end{exe}
	\begin{exe}
		\exa{\xy
			<6.5em,5cm>*\as{V}="n1",
			<3.5em,4cm>*\as{\hp{\sub{1}}PrWd\sub{1}}="PrWd1",<9.5em,4cm>*\as{\hp{\sub{2}}PrWd\sub{2}}="PrWd2",
			<5em,3cm>*\as{\hp{\sub{1}}Ft\sub{1}}="Ft1",<9.5em,3cm>*\as{\hp{\sub{\tsc{m}}}Ft\sub{\tsc{m}}}="Ft2",
			<2em,2cm>*\as{\hp{\sub{1}}σ\sub{1}}="s1",<4em,2cm>*\as{\hp{\sub{2}}σ\sub{2}}="s2",<6em,2cm>*\as{\hp{\sub{3}}σ\sub{3}}="s3",<8.5em,2cm>*\as{\hp{\sub{4}}σ\sub{4}}="s4",<10.5em,2cm>*\as{\hp{\sub{5}}σ\sub{5}}="s5",
			<1em,1cm>*\as{n}="cv1",<2em,1cm>*\as{a}="cv2",<3em,1cm>*\as{h}="cv3",<4em,1cm>*\as{o}="cv4",<5em,1cm>*\as{n}="cv5",<6em,1cm>*\as{i}="cv6",<7em,1cm>*\as{ʔ}="cv7",<8em,1cm>*\as{n}="cv8",<9em,1cm>*\as{a}="cv9",<10em,1cm>*\as{i}="cv10",<11em,1cm>*\as{n}="cv11",
			<1.5em,0cm>*\as{\hp{\sub{1}}M\sub{1}}="m1",<5em,0cm>*\as{\hp{\sub{2}}M\sub{2}}="m2",<8em,0cm>*\as{\hp{\sub{3}}M\sub{3}}="m3",<10em,0cm>*\as{\hp{\sub{4}}M\sub{4}}="m4",<2.5em,0cm>*\as{-}="-1",<9em,0cm>*\as{-}="-2",
			"m1"+U;"cv1"+D**\dir{-};"m1"+U;"cv2"+D**\dir{-};"m2"+U;"cv3"+D**\dir{-};"m2"+U;"cv4"+D**\dir{-};"m2"+U;"cv5"+D**\dir{-};"m2"+U;"cv6"+D**\dir{-};"m2"+U;"cv7"+D**\dir{-};"m3"+U;"cv8"+D**\dir{-};"m4"+U;"cv9"+D**\dir{-};"m4"+U;"cv10"+D**\dir{-};"m4"+U;"cv11"+D**\dir{-};
			"cv1"+U;"s1"+D**\dir{-};"cv2"+U;"s1"+D**\dir{-};"cv3"+U;"s1"+D**\dir{-};"cv3"+U;"s2"+D**\dir{-};"cv4"+U;"s2"+D**\dir{-};"cv5"+U;"s2"+D**\dir{-};"cv5"+U;"s3"+D**\dir{-};"cv6"+U;"s3"+D**\dir{-};"cv7"+U;"s3"+D**\dir{-};"cv8"+U;"s4"+D**\dir{-};"cv9"+U;"s4"+D**\dir{-};"cv10"+U;"s5"+D**\dir{-};"cv11"+U;"s5"+D**\dir{-};
			"s2"+U;"Ft1"+D**\dir{-};"s3"+U;"Ft1"+D**\dir{-};"s4"+U;"Ft2"+D**\dir{-};"s5"+U;"Ft2"+D**\dir{-};
			"s1"+U;"PrWd1"+D**\dir{-};"Ft1"+U;"PrWd1"+D**\dir{-};"Ft2"+U;"PrWd2"+D**\dir{-};"PrWd1"+U;"n1"+D**\dir{-};"PrWd2"+U;"n1"+D**\dir{-};
			<7.5em,2.5cm>*\as{\tikz[red,thick,dashed,baseline=0.9ex]\draw (0,0) -- (0,3.5cm);}="line",
		\endxy}\label{as:nahoniq-nain}
	\end{exe}
\end{multicols}

This prosodic analysis encounters no problems in
accounting for consonant deletion before
modifiers with an initial consonant cluster (\srf{sec:ConDel/CC}),
such as in \ve{muʔi\tbr{t}} `animal' + \ve{mnasiʔ} `old' {\ra} \ve{muʔi mnasiʔ}.
After consonant deletion occurs to mark nominal attribution,
the resulting structure has a crisp edge between each prosodic word,
as shown in \qf{as:muqi-mnasiq2}.

However, the prosodic analysis does not deal straightforwardly
with consonant deletion of VVC{\#} final nouns,
as seen in \ve{kau\tbr{t}} `papaya' + \ve{sufaʔ} `blossom'
{\ra} \ve{kau sufaʔ} `papaya blossom'.
Consonant deletion in this case results in an illicit
fuzzy border between the prosodic words, as shown in \qf{as:kau-sufaq}.
Metathesis cannot occur due to the structure of the first noun,
and thus this fuzzy border is left unresolved.\footnote{
		In the case of \qf{as:kau-sufaq} we might expect
		vowel deletion to occur to resolve the fuzzy border
		giving unattested \ve{kaut} `papaya' + \ve{sufaʔ} `blossom'
		{\ra} \ve{*ka sufaʔ}, or we might expect
		consonant deletion to be blocked to preserve the prosodic word boundary.}

%However, this would straightforwardly explain why /a/ assimilation does
%not affect the M\=/form VVC{\#} final words (\srf{sec:MorRulAssOfA}),
%in forms such as \ve{kuan} `village' + \ve{feʔu} `new'
%{\ra} \ve{kua feʔu} `new village' which contrast
%with forms such as \ve{nima} {\ra} \ve{niim} `five' (\srf{sec:AssOfA}).
%Metathesis simply does not occur in \ve{kua feʔu} `new village'
%meaning that the derived environment for /a/ assimilation is not present.

\begin{multicols}{2}
	\begin{exe}
		\exa{\xy
			<5.5em,5cm>*\as{N}="n1",
			<3em,4cm>*\as{\hp{\sub{1}}PrWd\sub{1}}="PrWd1",<8em,4cm>*\as{\hp{\sub{2}}PrWd\sub{2}}="PrWd2",
			<3em,3cm>*\as{\hp{\sub{1}}Ft\sub{1}}="Ft1",<8em,3cm>*\as{\hp{\sub{2}}Ft\sub{2}}="Ft2",
			<2em,2cm>*\as{\hp{\sub{1}}σ\sub{1}}="s1",<4em,2cm>*\as{\hp{\sub{2}}σ\sub{2}}="s2",<7em,2cm>*\as{\hp{\sub{3}}σ\sub{3}}="s3",<9em,2cm>*\as{\hp{\sub{4}}σ\sub{4}}="s4",
			<1em,1cm>*\as{m}="cv1",<2em,1cm>*\as{u}="cv2",<3em,1cm>*\as{ʔ}="cv3",<4em,1cm>*\as{i}="cv4",<5em,1cm>*\as{m}="cv5",<6em,1cm>*\as{n}="cv6",<7em,1cm>*\as{a}="cv7",<8em,1cm>*\as{s}="cv8",<9em,1cm>*\as{i}="cv9",<10em,1cm>*\as{ʔ}="cv10",
			<2.5em,0cm>*\as{\hp{\sub{1}}M\sub{1}}="m1",<7.5em,0cm>*\as{\hp{\sub{2}}M\sub{2}}="m2",
			"m1"+U;"cv1"+D**\dir{-};"m1"+U;"cv2"+D**\dir{-};"m1"+U;"cv3"+D**\dir{-};"m1"+U;"cv4"+D**\dir{-};"m2"+U;"cv5"+D**\dir{-};
			"m2"+U;"cv6"+D**\dir{-};"m2"+U;"cv7"+D**\dir{-};"m2"+U;"cv8"+D**\dir{-};"m2"+U;"cv9"+D**\dir{-};"m2"+U;"cv10"+D**\dir{-};
			"cv1"+U;"s1"+D**\dir{-};"cv2"+U;"s1"+D**\dir{-};"cv3"+U;"s1"+D**\dir{-};"cv3"+U;"s2"+D**\dir{-};"cv4"+U;"s2"+D**\dir{-};"cv5"+U;"s2"+D**\dir{-};
			"cv6"+U;"s3"+D**\dir{-};"cv7"+U;"s3"+D**\dir{-};"cv8"+U;"s3"+D**\dir{-};"cv8"+U;"s4"+D**\dir{-};"cv9"+U;"s4"+D**\dir{-};"cv10"+U;"s4"+D**\dir{-};
			"s1"+U;"Ft1"+D**\dir{-};"s2"+U;"Ft1"+D**\dir{-};"s3"+U;"Ft2"+D**\dir{-};"s4"+U;"Ft2"+D**\dir{-};
			"Ft1"+U;"PrWd1"+D**\dir{-};"Ft2"+U;"PrWd2"+D**\dir{-};
			"PrWd1"+U;"n1"+D**\dir{-};"PrWd2"+U;"n1"+D**\dir{-};
			<5.5em,2.25cm>*\as{\tikz[red,thick,dashed,baseline=0.9ex]\draw (0,0) -- (0,4cm);}="line",
		\endxy}\label{as:muqi-mnasiq2}
		\exa{\xy
			<4.25em,5cm>*\as{N}="n1",
			<2.5em,4cm>*\as{\hp{\sub{1}}PrWd\sub{1}}="PrWd1",<6em,4cm>*\as{\hp{\sub{2}}PrWd\sub{2}}="PrWd2",
			<2.5em,3cm>*\as{\hp{\sub{1}}Ft\sub{1}}="Ft1",<6em,3cm>*\as{\hp{\sub{2}}Ft\sub{2}}="Ft2",
			<1.5em,2cm>*\as{\hp{\sub{1}}σ\sub{1}}="s1",<3.5em,2cm>*\as{\hp{\sub{2}}σ\sub{2}}="s2",<5em,2cm>*\as{\hp{\sub{3}}σ\sub{3}}="s3",<7em,2cm>*\as{\hp{\sub{4}}σ\sub{4}}="s4",
			<1em,1cm>*\as{k}="cv1",<2em,1cm>*\as{a}="cv2",<3em,1cm>*\as{u}="cv3",<4em,1cm>*\as{s}="cv4",<5em,1cm>*\as{u}="cv5",<6em,1cm>*\as{f}="cv6",<7em,1cm>*\as{a}="cv7",<8em,1cm>*\as{ʔ}="cv8",
			<2em,0cm>*\as{\hp{\sub{1}}M\sub{1}}="m1",<6em,0cm>*\as{\hp{\sub{2}}M\sub{2}}="m2",
			"m1"+U;"cv1"+D**\dir{-};"m1"+U;"cv2"+D**\dir{-};"m1"+U;"cv3"+D**\dir{-};"m2"+U;"cv4"+D**\dir{-};"m2"+U;"cv5"+D**\dir{-};"m2"+U;"cv6"+D**\dir{-};"m2"+U;"cv7"+D**\dir{-};"m2"+U;"cv8"+D**\dir{-};
			"cv1"+U;"s1"+D**\dir{-};"cv2"+U;"s1"+D**\dir{-};"cv3"+U;"s2"+D**\dir{-};"cv4"+U;"s2"+D**\dir{-};
			"cv4"+U;"s3"+D**\dir{-};"cv5"+U;"s3"+D**\dir{-};"cv6"+U;"s3"+D**\dir{-};"cv6"+U;"s4"+D**\dir{-};"cv7"+U;"s4"+D**\dir{-};"cv8"+U;"s4"+D**\dir{-};
			"s1"+U;"Ft1"+D**\dir{-};"s2"+U;"Ft1"+D**\dir{-};"s3"+U;"Ft2"+D**\dir{-};"s4"+U;"Ft2"+D**\dir{-};
			"Ft1"+U;"PrWd1"+D**\dir{-};"Ft2"+U;"PrWd2"+D**\dir{-};
			"PrWd1"+U;"n1"+D**\dir{-};"PrWd2"+U;"n1"+D**\dir{-};
			<4em,1cm>*\as{\tikz[red,thick,dashed,baseline=0.9ex]\draw (0,0) rectangle (0.4cm,0.5cm);}="box",
		\endxy}\label{as:kau-sufaq}
	\end{exe}
\end{multicols}

To account for the problems involved in the analysis
of VVC{\#} final words under the prosodic analysis,
we can reintroduce empty C-slots,
and posit that metathesis has indeed occurred in \ve{kau sufaʔ} `papaya blossom',
as shown in \qf{as:kauC-sufaq} in which
there is a crisp edge between each prosodic word.

\begin{exe}
	\exa{\xy
		<4.75em,6cm>*\as{N}="n1",
		<2.5em,5cm>*\as{\hp{\sub{1}}PrWd\sub{1}}="PrWd1",<7em,5cm>*\as{\hp{\sub{2}}PrWd\sub{2}}="PrWd2",
		<2.5em,4cm>*\as{\hp{\sub{\tsc{m}}}Ft\sub{\tsc{m}}}="Ft1",<7em,4cm>*\as{\hp{\sub{2}}Ft\sub{2}}="Ft2",
		<1.5em,3cm>*\as{\hp{\sub{1}}σ\sub{1}}="s1",<3.5em,3cm>*\as{\hp{\sub{2}}σ\sub{2}}="s2",<6em,3cm>*\as{\hp{\sub{3}}σ\sub{3}}="s3",<8em,3cm>*\as{\hp{\sub{4}}σ\sub{4}}="s4",
		<1em,2cm>*\as{C}="CV1",<2em,2cm>*\as{V}="CV2",<3em,2cm>*\as{V}="CV3",<4em,2cm>*\as{C}="CV4",<5em,2cm>*\as{C}="CV5",<6em,2cm>*\as{V}="CV6",<7em,2cm>*\as{C}="CV7",<8em,2cm>*\as{V}="CV8",<9em,2cm>*\as{C}="CV9",
		<1em,1cm>*\as{k}="cv1",<2em,1cm>*\as{a}="cv2",<3em,1cm>*\as{u}="cv3",<4em,1cm>*\as{ }="cv4",<5em,1cm>*\as{s}="cv5",<6em,1cm>*\as{u}="cv6",<7em,1cm>*\as{f}="cv7",<8em,1cm>*\as{a}="cv8",<9em,1cm>*\as{ʔ}="cv9",
		<2em,0cm>*\as{\hp{\sub{1}}M\sub{1}}="m1",<7em,0cm>*\as{\hp{\sub{2}}M\sub{2}}="m2",
		"m1"+U;"cv1"+D**\dir{-};"m1"+U;"cv2"+D**\dir{-};"m1"+U;"cv3"+D**\dir{-};"m1"+U;"cv4"+D**\dir{};"m2"+U;"cv5"+D**\dir{-};"m2"+U;"cv6"+D**\dir{-};"m2"+U;"cv7"+D**\dir{-};"m2"+U;"cv8"+D**\dir{-};"m2"+U;"cv9"+D**\dir{-};
		"cv1"+U;"CV1"+D**\dir{-};"cv2"+U;"CV2"+D**\dir{-};"cv3"+U;"CV3"+D**\dir{-};"cv4"+U;"CV4"+D**\dir{};"cv5"+U;"CV5"+D**\dir{-};"cv6"+U;"CV6"+D**\dir{-};"cv7"+U;"CV7"+D**\dir{-};"cv8"+U;"CV8"+D**\dir{-};"cv9"+U;"CV9"+D**\dir{-};
		"CV1"+U;"s1"+D**\dir{-};"CV2"+U;"s1"+D**\dir{-};"CV3"+U;"s2"+D**\dir{-};"CV4"+U;"s2"+D**\dir{-};
		"CV5"+U;"s3"+D**\dir{-};"CV6"+U;"s3"+D**\dir{-};"CV7"+U;"s3"+D**\dir{-};"CV7"+U;"s4"+D**\dir{-};"CV8"+U;"s4"+D**\dir{-};"CV9"+U;"s4"+D**\dir{-};
		"s1"+U;"Ft1"+D**\dir{-};"s2"+U;"Ft1"+D**\dir{-};"s3"+U;"Ft2"+D**\dir{-};"s4"+U;"Ft2"+D**\dir{-};
		"Ft1"+U;"PrWd1"+D**\dir{-};"Ft2"+U;"PrWd2"+D**\dir{-};
		"PrWd1"+U;"n1"+D**\dir{-};"PrWd2"+U;"n1"+D**\dir{-};"n1"+D**\dir{-};
		<4.5em,3.25cm>*\as{\tikz[red,thick,dashed,baseline=0.9ex]\draw (0,0) -- (0,4.5cm);}="line",
	\endxy}\label{as:kauC-sufaq}
\end{exe}

However, the reintroduction of empty C-slots
is problematic for the analysis of verbal metathesis in SVCs.
Because consonant deletion is not a marker of attribution for verbs,
a vowel-final verb such as \ve{mi-sopu} `(we) complete'
would end in an empty C-slot which would be retained
when it occurs in an SVC, thus resulting in a crisp edge
after the first internal prosodic word edge, as represented in
unattested \ve{*mi-sopu m-ain} `(we) first completed' given in \qf{as:misopu-main}.
The attested structure is \ve{mi-soup m-ain} given in \qf{as:misoup-main}
above and repeated with the CV-tier shown in \qf{as:misoup-main2} below.

\begin{multicols}{2}
	\begin{exe}
		\exa{\xy
			<1em,6cm>*\as{*}="*",<6.5em,6cm>*\as{V}="n1",
			<3.5em,5cm>*\as{\hp{\sub{1}}PrWd\sub{1}}="PrWd1",<9.5em,5cm>*\as{\hp{\sub{2}}PrWd\sub{2}}="PrWd2",
			<5em,4cm>*\as{\hp{\sub{1}}Ft\sub{1}}="Ft1",<9.5em,4cm>*\as{\hp{\sub{\tsc{m}}}Ft\sub{\tsc{m}}}="Ft2",
			<2em,3cm>*\as{\hp{\sub{1}}σ\sub{1}}="s1",<4em,3cm>*\as{\hp{\sub{2}}σ\sub{2}}="s2",<6em,3cm>*\as{\hp{\sub{3}}σ\sub{3}}="s3",<8.5em,3cm>*\as{\hp{\sub{4}}σ\sub{4}}="s4",<10.5em,3cm>*\as{\hp{\sub{5}}σ\sub{5}}="s5",
			<1em,2cm>*\as{C}="CV1",<2em,2cm>*\as{V}="CV2",<3em,2cm>*\as{C}="CV3",<4em,2cm>*\as{V}="CV4",<5em,2cm>*\as{C}="CV5",<6em,2cm>*\as{V}="CV6",<7em,2cm>*\as{C}="CV7",<8em,2cm>*\as{V}="CV8",<9em,2cm>*\as{C}="CV9",<10em,2cm>*\as{V}="CV10",<11em,2cm>*\as{C}="CV11",
			<1em,1cm>*\as{m}="cv1",<2em,1cm>*\as{i}="cv2",<3em,1cm>*\as{s}="cv3",<4em,1cm>*\as{o}="cv4",<5em,1cm>*\as{p}="cv5",<6em,1cm>*\as{u}="cv6",<7em,1cm>*\as{ }="cv7",<8em,1cm>*\as{m}="cv8",<9em,1cm>*\as{a}="cv9",<10em,1cm>*\as{i}="cv10",<11em,1cm>*\as{n}="cv11",
			<1.5em,0cm>*\as{\hp{\sub{1}}M\sub{1}}="m1",<4.5em,0cm>*\as{\hp{\sub{2}}M\sub{2}}="m2",<8em,0cm>*\as{\hp{\sub{3}}M\sub{3}}="m3",<10em,0cm>*\as{\hp{\sub{4}}M\sub{4}}="m4",<2.5em,0cm>*\as{-}="-1",<9em,0cm>*\as{-}="-2",
			"m1"+U;"cv1"+D**\dir{-};"m1"+U;"cv2"+D**\dir{-};"m2"+U;"cv3"+D**\dir{-};"m2"+U;"cv4"+D**\dir{-};"m2"+U;"cv5"+D**\dir{-};"m2"+U;"cv6"+D**\dir{-};"m3"+U;"cv8"+D**\dir{-};"m4"+U;"cv9"+D**\dir{-};"m4"+U;"cv10"+D**\dir{-};"m4"+U;"cv11"+D**\dir{-};
			"cv1"+U;"CV1"+D**\dir{-};"cv2"+U;"CV2"+D**\dir{-};"cv3"+U;"CV3"+D**\dir{-};"cv4"+U;"CV4"+D**\dir{-};"cv5"+U;"CV5"+D**\dir{-};"cv6"+U;"CV6"+D**\dir{-};"cv7"+U;"CV7"+D**\dir{};"cv8"+U;"CV8"+D**\dir{-};"cv9"+U;"CV9"+D**\dir{-};"cv10"+U;"CV10"+D**\dir{-};"cv11"+U;"CV11"+D**\dir{-};
			"CV1"+U;"s1"+D**\dir{-};"CV2"+U;"s1"+D**\dir{-};"CV3"+U;"s1"+D**\dir{-};"CV3"+U;"s2"+D**\dir{-};"CV4"+U;"s2"+D**\dir{-};"CV5"+U;"s2"+D**\dir{-};
			"CV5"+U;"s3"+D**\dir{-};"CV6"+U;"s3"+D**\dir{-};"CV7"+U;"s3"+D**\dir{-};"CV8"+U;"s4"+D**\dir{-};"CV9"+U;"s4"+D**\dir{-};"CV10"+U;"s5"+D**\dir{-};"CV11"+U;"s5"+D**\dir{-};
			"s2"+U;"Ft1"+D**\dir{-};"s3"+U;"Ft1"+D**\dir{-};"s4"+U;"Ft2"+D**\dir{-};"s5"+U;"Ft2"+D**\dir{-};
			"s1"+U;"PrWd1"+D**\dir{-};"Ft1"+U;"PrWd1"+D**\dir{-};"Ft2"+U;"PrWd2"+D**\dir{-};"PrWd1"+U;"n1"+D**\dir{-};"PrWd2"+U;"n1"+D**\dir{-};
			<7.5em,3.25cm>*\as{\tikz[red,thick,dashed,baseline=0.9ex]\draw (0,0) -- (0,4.5cm);}="line",
		\endxy}\label{as:misopu-main}
		\exa{\xy
			<5.75em,6cm>*\as{V}="n1",
			<3em,5cm>*\as{\hp{\sub{1}}PrWd\sub{1}}="PrWd1",<8.5em,5cm>*\as{\hp{\sub{2}}PrWd\sub{2}}="PrWd2",
			<4.5em,4cm>*\as{\hp{\sub{\tsc{m}}}Ft\sub{\tsc{m}}}="Ft1",<8.5em,4cm>*\as{\hp{\sub{\tsc{m}}}Ft\sub{\tsc{m}}}="Ft2",
			<2em,3cm>*\as{\hp{\sub{1}}σ\sub{1}}="s1",<3.5em,3cm>*\as{\hp{\sub{2}}σ\sub{2}}="s2",<5.5em,3cm>*\as{\hp{\sub{3}}σ\sub{3}}="s3",<7.5em,3cm>*\as{\hp{\sub{4}}σ\sub{4}}="s4",<9.5em,3cm>*\as{\hp{\sub{5}}σ\sub{5}}="s5",
			<1em,2cm>*\as{C}="CV1",<2em,2cm>*\as{V}="CV2",<3em,2cm>*\as{C}="CV3",<4em,2cm>*\as{V}="CV4",<5em,2cm>*\as{V}="CV5",<6em,2cm>*\as{C}="CV6",<7em,2cm>*\as{C}="CV7",<8em,2cm>*\as{V}="CV8",<9em,2cm>*\as{V}="CV9",<10em,2cm>*\as{C}="CV10",
			<1em,1cm>*\as{m}="cv1",<2em,1cm>*\as{i}="cv2",<3em,1cm>*\as{s}="cv3",<4em,1cm>*\as{o}="cv4",<5em,1cm>*\as{u}="cv5",<6em,1cm>*\as{p}="cv6",<7em,1cm>*\as{m}="cv7",<8em,1cm>*\as{a}="cv8",<9em,1cm>*\as{i}="cv9",<10em,1cm>*\as{n}="cv10",
			<1.5em,0cm>*\as{\hp{\sub{1}}M\sub{1}}="m1",<4.5em,0cm>*\as{\hp{\sub{2}}M\sub{2}}="m2",<7em,0cm>*\as{\hp{\sub{3}}M\sub{3}}="m3",<9em,0cm>*\as{\hp{\sub{4}}M\sub{4}}="m4",<2.5em,0cm>*\as{-}="-1",<8em,0cm>*\as{-}="-2",
			"m1"+U;"cv1"+D**\dir{-};"m1"+U;"cv2"+D**\dir{-};"m2"+U;"cv3"+D**\dir{-};"m2"+U;"cv4"+D**\dir{-};"m2"+U;"cv5"+D**\dir{-};"m2"+U;"cv6"+D**\dir{-};"m3"+U;"cv7"+D**\dir{-};"m4"+U;"cv8"+D**\dir{-};"m4"+U;"cv9"+D**\dir{-};"m4"+U;"cv10"+D**\dir{-};
			"cv1"+U;"CV1"+D**\dir{-};"cv2"+U;"CV2"+D**\dir{-};"cv3"+U;"CV3"+D**\dir{-};"cv4"+U;"CV4"+D**\dir{-};"cv5"+U;"CV5"+D**\dir{-};"cv6"+U;"CV6"+D**\dir{-};"cv7"+U;"CV7"+D**\dir{-};"cv8"+U;"CV8"+D**\dir{-};"cv9"+U;"CV9"+D**\dir{-};"cv10"+U;"CV10"+D**\dir{-};
			"CV1"+U;"s1"+D**\dir{-};"CV2"+U;"s1"+D**\dir{-};"CV3"+U;"s1"+D**\dir{-};"CV3"+U;"s2"+D**\dir{-};"CV4"+U;"s2"+D**\dir{-};"CV5"+U;"s3"+D**\dir{-};"CV6"+U;"s3"+D**\dir{-};"CV7"+U;"s4"+D**\dir{-};"CV8"+U;"s4"+D**\dir{-};"CV9"+U;"s5"+D**\dir{-};"CV10"+U;"s5"+D**\dir{-};
			"s2"+U;"Ft1"+D**\dir{-};"s3"+U;"Ft1"+D**\dir{-};"s4"+U;"Ft2"+D**\dir{-};"s5"+U;"Ft2"+D**\dir{-};
			"s1"+U;"PrWd1"+D**\dir{-};"Ft1"+U;"PrWd1"+D**\dir{-};"Ft2"+U;"PrWd2"+D**\dir{-};"PrWd1"+U;"n1"+D**\dir{-};"PrWd2"+U;"n1"+D**\dir{-};
			<6.5em,3.25cm>*\as{\tikz[red,thick,dashed,baseline=0.9ex]\draw (0,0) -- (0,4.5cm);}="line",
		\endxy}\label{as:misoup-main2}
	\end{exe}
\end{multicols}

\largerpage
Under the prosodic approach it seems that
the analyst must decide whether to deal with
verbal metathesis in the same way as nominal metathesis,
or whether to introduce empty C-slots and thus
deal with the M\=/form of VVC{\#} nominals in the same way as other nominals.\footnote{
		This is an oversimplification and ignores other possibilities
		such as only positing medial empty slots and/or only positing
		the obligatory CVCVC foot for nouns.
		However, in my view, the strongest evidence for empty C-slots
		in Amarasi comes from the process of consonant insertion
		at clitic boundaries (\srf{sec:ConIns}) and word-final
		consonant insertion in other varieties of Meto (\srf{sec:ComSup})
		-- both processes which provide evidence for final empty C-slots for all word classes.}

However, probably the most serious issue with the prosodic
analysis is that there is no independent evidence that attributive
phrases actually have a structure in which prosodic
words share a higher prosodic category.
The only evidence for such a structure is the metathesis;
the very phenomenon we are trying to explain by positing such a structure.

This is contrary to metathesis before enclitic boundaries whereby the clitic host
unexpectedly bears stress, thus providing
evidence for a recursive prosodic word structure \sub{PrWd}[\sub{PrWd}[HOST]=cl],
with the internal prosodic word of the host linked to a higher prosodic word
containing both the host and the enclitic

\subsection{Summary}
Metathesis before attributive modifiers
can be analysed prosodically, as proposed in this section,
or as a morphological process as proposed in \srf{sec:UniAna}.
The main strengths and weaknesses of each analysis, as I currently
see them, are given in \qf{ex:MorpAna} and \qf{ex:ProsAna}.

\begin{exe}
	\ex{Morphological Analysis}\label{ex:MorpAna}
	\begin{xlist}
		\ex{Strengths:}
			\begin{xlist}
				\ex{accounts for VVC{\#} final nouns in the same way as other nouns}
				\ex{makes use of empty C-slots (posited based on other evidence)}
			\end{xlist}
		\ex{Weaknesses:}
			\begin{xlist}
				\ex{no explanation for lack of metathesis for C-final verbs in SVCs}
				\ex{inconsistent with diachronic development of nominal 
						M\=/forms whereby consonant deletion precedes metathesis (\srf{sec:LosFinCon})}
			\end{xlist}
	\end{xlist}
	\ex{Prosodic Analysis}\label{ex:ProsAna}
	\begin{xlist}
		\ex{Strengths:}
			\begin{xlist}
				\ex{consistent with the diachronic development of nominal M\=/forms}
				\ex{good explanation for lack of metathesis for C-final verbs in SVCs}
			\end{xlist}
	\end{xlist}
\newpage
	\sn{\begin{xlist}
		\exi{b.}{Weaknesses:}
			\begin{xlist}
				\ex{M\=/form of VVC{\#} final nouns is unexpected}
				\ex{empty C-slots introduce complications}
				\ex{requires positing otherwise unsupported prosodic structures}
			\end{xlist}
	\end{xlist}}
\end{exe}


Two additional facts must be born in mind
when deciding between the two analyses.
Firstly, adopting the prosodic analysis
is not the same as analysing metathesis as phonologically conditioned.
It is well worth repeating that the only phonological difference between
attributive and non-attributive phrases is often
the order of the final CV sequence of the first noun.

The examples involving cardinal and ordinal numerals
from the beginning of this chapter are repeated
in \qf{ex:NenoMeseq3} and \qf{ex:NeonMeseq3} below.
At the risk of labouring the point,
the stress of these phrases are identical,
the number of vowels of these phrases are identical,
and the intonation is identical.
The only difference is the metathesis.

\begin{multicols}{2}
	\begin{exe}
	\let\eachwordone=\textnormal
	\let\eachwordtwo=\itshape
		\ex{\glll {}				ˌnɛnɔ {} {} ˈmɛsɛʔ	{} \\
							\brac{NP} ne\tbr{no} \bracr{} \brac{Num} meseʔ \bracr{}\\
							{} day{\tbrU} {} {} one{\U} {}\\
				\glt \lh{\brac{NP}} `one day' {\emb{neno-meseq.mp3}{\spk{}}{\apl}}}\label{ex:NenoMeseq3}
		\ex{\glll	{} ˌnɛ.ɔn ˈmɛsɛʔ {} \\
							\brac{NP} ne\tbr{on} meseʔ \bracr{}\\
							{} day{\tbrM} one{\U} {}\\
				\glt \lh{\brac{NP}}`first day (i.e. Monday)' {\emb{neon-meseq.mp3}{\spk{}}{\apl}}}\label{ex:NeonMeseq3}
	\end{exe}
\end{multicols}

Instead, with the prosodic analysis, metathesis
itself is a realisation of the prosody.
In most languages prosodic structures
are realised by suprasegmental features such as stress and/or intonation.
Under the prosodic analysis Amarasi has an additional
feature which encodes prosodic structures:
the order of the segmental material in post-tonic syllables.

If the prosodic analysis is taken, then I would agree with
\cite{hika} that Amarasi metathesis is a prosodic cue for phrasing.
Though, I am not sure that I would agree with them that
``VC-metathesis appears to be rooted in phonological phrasing.''

The second point which should be born in mind
when deciding on the best analysis of metathesis in the syntax
is that this is only one aspect of Amarasi metathesis.
As I discuss in Chapter \ref{ch:DisMet}, metathesis also
occurs in the discourse to encode resolved and unresolved events.
Given that neither metathesis occurs in a specific phonological environment
(unlike the phonological metathesis discussed in Chapter \ref{ch:PhoMet})
it is reasonable to expect both kinds of metathesis to be analysed in a similar way.
If a prosodic analysis of metathesis in the syntax is adopted,
then we would expect that this can also be extended to the discourse,
and vice-versa.

It is this last point which has mainly led me to analyse
syntactically driven metathesis as a morphological process.
I have been unable to come up with a plausible prosodic analysis
which would cover metathesis in both the syntax and the discourse.
It may be that this shortcoming lies not with the data but with the analyst.
If this is the, case then I would urge those who might favour a prosodic
analysis of metathesis to examine the Amarasi data carefully
and propose an analysis which is faithful to this data.
In the end, my primary commitment is not to any particular analysis,
but to the data on which any analysis must be based.
