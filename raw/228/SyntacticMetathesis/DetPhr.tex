\subsection{Determiner phrase}\label{sec:DetPhr}
The head of the determiner phrase in Amarasi is filled by either a demonstrative or a determiner.
Syntactically conditioned M\=/forms do not occur before the determiner phrase.
The determiner phrase occurs after the number phrase and before the quantifier phrase.
The Amarasi demonstratives and determiners are given in \trf{tab:AmaDem} below.
They have the same four person values present in the genitive suffixes
(\srf{sec:GenSuf}, \srf{sec:GenSuf ch:SynMet}).
Demonstratives and determiners do not co-occur.

\begin{table}[h]
	\caption{Amarasi demonstratives and determiners}\label{tab:AmaDem}
	\centering
		\begin{threeparttable}[b]
		\begin{tabular}{clll}\lsptoprule
			Pers. &\tsc{dem}							&\tsc{det}			&Function \\ \midrule
			1			&\ve{ia, i{\j}a}\su{†}	&\ve{=ii}				&``near'' speaker\\
			2			&\ve{nana}/\ve{naan}		&\ve{=ana/=aan}	&``near'' addressee\\
			3			&\ve{nee}								&\ve{=ee}				&``near'' third person\\
			0			&\ve{naa}								&\ve{=aa}				&location ``near'' no-one (≈ obviative)\\
		\lspbottomrule
		\end{tabular}
			\begin{tablenotes}
				\item [†]
					The {\ia} form \ve{i\j a} is rare with only
					twelve attestions in my corpus compared with 267 of \ve{ia}. 
					It occurs mainly in the speech of older speakers.
			\end{tablenotes}
		\end{threeparttable}
\end{table}

Nouns preceding demonstratives occur in the U\=/form.
Demonstratives have two main functions.
Firstly, they can be used to introduce new participants into the discourse,
as shown in \qf{ex:120715-4, 0.05} below.
Secondly, they are used when the spatial, temporal, or
referential location of the previous nominal is in focus.
This is shown in \qf{ex:130906-1, 2.48} and \qf{ex:120923-2, 5.03} below.
The structure of the determiner phrase in \qf{ex:120923-2, 5.03}
is given in \qf{tr:120923-2, 5.03} below.

\begin{exe}
	\ex{\glll	ne\tbr{no} \tbr{naa} paha{\gap}ʔpi\tbr{na}-\tbr{n} \tbr{ia}, a|n-kobub on bare meseʔ.\\
						neno naa paha{\gap}ʔpina-n ia, {\a}n-kobub on bare meseʔ\\
						day{\tbrU} \tbr{\naa} land{\gap}below{\tbrU}-{\N} \tbr{\ia} {\a\n}-pile {\on} place one \\
			\glt	`In those days this world was piled up in one place.'
						\txrf{120715-4, 0.05} {\emb{120715-4-00-05.mp3}{\spk{}}{\apl}}}\label{ex:120715-4, 0.05}
	\ex{\glll	nehh, baab Saraʔ n-nao et pa\tbr{ni}-\tbr{n} \tbr{nee}.\\
						{} baba Saraʔ n-nao et pani-n nee\\
						{} MB/FZ Sarah \n-go {\et} across{\tbrU}-{\N} \tbr{\nee}\\
			\glt	Well, Aunt Sarah had gone across there.'
						\txrf{130906-1, 2.48} {\emb{130906-1-02-48.mp3}{\spk{}}{\apl}}}\label{ex:130906-1, 2.48}
	\ex{\glll	of pa\tbr{paʔ} \tbr{naan} na-papaʔ \sf{terus}. \\
						of papaʔ naan na-papaʔ \sf{terus}\\
						sure wound{\tbrU} {\naan} \na-wound constant\\
			\glt	`That wound will surely be a wound constantly.'
						\txrf{120923-2, 5.03} {\emb{120923-2-05-03.mp3}{\spk{}}{\apl}}}\label{ex:120923-2, 5.03}
		\ex{\begin{forest} where n children=0{tier=word}{}
			%[DP,[NP,[\br{N},[N,[\ve{papaʔ}\\wound{\U}]]]][D,[\ve{naan}\\{\naan}]]] %no roof
			[DP,[NP,[\ve{papaʔ}\\wound{\U}, roof]][D,[\ve{naan}\\{\naan}]]] %roof
		\end{forest}}\label{tr:120923-2, 5.03}
\end{exe}

Nominals occur in the U\=/form before demonstratives
because the latter are the head of their own phrase
and do not occur inside the nominal phrase.
Independent evidence for this analysis comes from the fact that
demonstratives frequently occur with no preceding nominal.
This behaviour is very common in my corpus,
with most demonstratives not following a nominal phrase.

Depending on the discourse pragmatics and syntactic structures of the entire sentence,
an independent demonstrative is interpreted as a locational adjunct, as in \qf{ex:120715-4, 0.55} below,
or as  a verbal object, as seen in \qf{ex:130920-1, 0.45}--\qf{ex:130825-6, 12.10-12.13} below.
The structure of the first part of example \qf{ex:130825-6, 12.13}
is given in \qf{tr:130825-6, 12.13} below.

\begin{exe}
	\ex{\gll	\sf{ahirɲa} ahh, n-aim \tbr{naan} baar\j=esa =m na-maikaʔ n-- \\
						in.the.end {} {\n}-look.for {\naan} place={\es} =and {\na}-settle\\
			\glt	`In the end, he looked there for a place and settled.'
						\txrf{120715-4, 0.55} {\emb{120715-4-00-55.mp3}{\spk{}}{\apl}}}\label{ex:120715-4, 0.55}
	\ex{\gll hai ima m-tea \tbr{ia}, ehh, n-reuk hitu n-kono kreʔo.\\
						{\hai} {\ima} {\m}-arrive {\ia} {} {\n}-hit seven {\n}-past little\\
			\glt	`We arrived here a little bit after it struck 7:00.'
						\txrf{130920-1, 0.45} {\emb{130920-1-00-45.mp3}{\spk{}}{\apl}}}\label{ex:130920-1, 0.45}
	\ex{\gll iin na-maikaʔ n-bi \tbr{nee}, a|n-sao nte-- a|n-sao n-bi \tbr{nee}.\\
						{\iin} {\na}-settle{\U} {\n}-{\bi} {\nee} {\a\n}-marry {} {\a\n}-marry {\n}-{\bi} {\nee}\\
			\glt	`He settled there and married agai--, married there.'
						\txrf{130821-1, 4.52} {\emb{130821-1-04-52.mp3}{\spk{}}{\apl}}}
	\ex{\begin{xlist}
		\ex{\gll 	siin ka= na-hini=n =fa n-eu hoo m-nao on \sf{Jakarta}.\\
							{\siin} {\ka}= {\na}-know={\einV} ={\fa} {\n}-{\eu} {\hoo} {\m}-go {\on} Jakarta\\
				\glt	`They didn't know you were going to Jakarta.' \txrf{130825-6, 12.10} {\emb{130825-6-12-10-12-13.mp3}{\spk{}}{\apl}}}\label{ex:130825-6, 12.10}
		\ex{\gll	hoo m-ʔain \tbr{naa}, oo? hoo m-ʔain =siin.\\
							{\hoo} {\m}-head.to{\M} {\naa} {\aaQ} {\hoo} {\m}-head.to{\M} ={\siin}\\
				\glt	`You were heading there, no? You were heading towards them.'
							\txrf{12.13}}\label{ex:130825-6, 12.13}
		\end{xlist}}\label{ex:130825-6, 12.10-12.13}
		\ex{\begin{forest} where n children=0{tier=word}{}
			[S,[NP,[\br{N},[N,[\ve{hoo}\\{\hoo}]]]][VP,[\br{V},[V,[\ve{m-ʔain}\\head.to]]][DP,[D,[\ve{naa}\\{\naa}]]]]]
		\end{forest}}\label{tr:130825-6, 12.13}
\end{exe}

Nouns take the U\=/form before demonstratives
because demonstratives are the head of a determiner phrase,
which in turn contains the nominal phrase.
Demonstratives are outside the nominal phrase
and outside the domain of metathesis.
