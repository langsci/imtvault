\section{The nominal word class}\label{sec:NomWorCla}
Content words (non-functors) in Amarasi fall into
two major word classes: nominals and verbs.
Some roots are specified as nominal roots,
some roots are specified as verbal roots and some 
roots are precategorial \citep{don08}, being specified as neither nominal nor verbal.
\trf{tab:AmaWorCla} lists the most salient morphosyntactic criteria
which allow us to distinguish between nominals and verbs in Amarasi.

\begin{table}[h]
	\caption[Amarasi word classes]{Amarasi word classes\su{†}}\label{tab:AmaWorCla}
	\centering
		\begin{threeparttable}\stl{0.4em}
			\begin{tabular}{rcccccccc}\lsptoprule
											&agr-&{\at}&{\b}&{\mak}&C{\#}{\ra}{\0}&\tsc{subj/obj}&=Det&=Num\\ \midrule
				Nominal 			&--&--&--&--&{✔}&{✔}&{✔}&{✔}\\
				Precategorial	&{✔}&{✔}&{✔}&{✔}&{✔}&{✔}&{✔}&{✔}\\
				Verb	 				&{✔}&{✔}&{✔}&{✔}&--&--&--&--\\ \lspbottomrule
			\end{tabular}
		\begin{tablenotes}
			\item[†]	agr-: take verbal agreement prefixes (\srf{sec:VerAgrPre}),
								{\at} can be nominalised with the circumfix \ve{a-{\ldots}-t} (\srf{sec:NomA--t}),
								{\b} can take the transitive suffix \ve{-b} (\srf{sec:TraSuf}), 
								{\mak} can take the reciprocal prefix \ve{ma(k)-} (\srf{sec:RecPre}),
								C{\#}{\ra}{\0} final consonant can be deleted to derive verbs (\srf{sec:BasVerDer}),
								\tsc{subj/obj} can be the subject or object of a verb,
								=Det can take definiteness marking determiners (\srf{sec:Det}),
								=Num can take number enclitics (\srf{sec:NumEnc sec:OthNomMod}).
		\end{tablenotes}
		\end{threeparttable}
\end{table}

In this section I discuss the four criteria in \trf{tab:AmaWorCla}
which allow us to identify a nominal word class:
verbal derivation (\srf{sec:BasVerDer}),
verbal arguments (\srf{sec:VerArg}), determiner modification (\srf{sec:Det})
and number enclitic modification (\srf{sec:NumEnc}).

There is no morphosyntactic basis for distinguishing
separate classes of nouns and adjectives.
All differences in the behaviour of these putative 
categories are straightforwardly explained by their semantics.
For instance, only adjective-like nominals have been attested modified by \ve{besi} `very'.
This can be explained by the fact that some
nominals, such as \ve{reʔuf} `bad', are gradable,
while other nominals, such as \ve{fatu} `stone', are not gradable in Amarasi.

When it is necessary to distinguish between these semantic categories,
I call nominals which refer to things \emph{thing nominals}
and nominals which describe such things \emph{property nominals}.
Many nominals do not belong clearly to either of these semantic categories.
Three such examples are: \ve{mnanuʔ} `long/length, deep/depth'
\ve{kase} `foreign(er)' and \ve{anaʔ} `small, baby'.

\subsection{Base for verbal derivation}\label{sec:BasVerDer}
Amarasi has a morphological process of subtraction which derives a verb from a nominal.
Under this process the final consonant of a nominal root is deleted.
Verbs derived by this process are usually intransitive.
Examples of verbs derived from nominals by word-final consonant deletion
are given in \qf{ex:VCNoun->VVerb} below.
Verbs are listed with the \tsc{3sg} prefix \ve{na-/n-}.

In many cases the deleted consonant is historically analysable as a suffix.
Thus, for instance, Amarasi \ve{mnasiʔ} `aged, old'
is cognate with Termanu `old' \it{lasi-k} and Dengka `old' \it{lasi-ʔ}
with a final suffix \it{-k/-ʔ} which is
a productive suffix in the languages of Rote.
However, such final consonants are no longer
analysable as suffixes in Amarasi.

Furthermore, such nominal/verb pairs include several
in which the final consonant of the noun is an inheritance from Proto-Malayo-Polynesian.
Examples include are *quza\tbr{n} > \ve{ura\tbr{n}} `rain' {\ra} \ve{na-ʔura} `rains',
*ma-diŋdi\tbr{ŋ} > \ve{mainiki\tbr{n}} `cold' {\ra} \ve{n-mainiki} `is cold',
and *tapi\tbr{s} > \ve{tai\tbr{s}} `sarong' {\ra} \ve{na-tai} `clothes s.o.'.

\begin{exe}
	\ex{{\ldots}VC{\#} nominal {\ra} {\ldots}V{\#} verb}\label{ex:VCNoun->VVerb}
		\sn{\stl{0.4em}\gw\begin{tabular}{llcll}
			`rain'					&\ve{ura\tbr{n}} 			&\ra& \ve{na-ʔura} 		& `rains'\\
			`cold' 					&\ve{mainiki\tbr{n}}	&\ra& \ve{n-mainiki}	& `is cold' \\
			`sea snail' 		&\ve{kbatu\tbr{s}}		&\ra& \ve{na-kbatu} 	& `gathers sea snails' \\
			`digging stick'	&\ve{ʔsua\tbr{k}}			&\ra& \ve{na-ʔsua} 		& `digs with a digging stick' \\
			`umbrella' 			&\ve{tenu\tbr{k}}			&\ra& \ve{n-tenu} 		& `shades' \\
			`sarong' 				&\ve{tai\tbr{s}}			&\ra& \ve{na-tai} 		& `(s/he) clothes s.o.' \\
			`dry' 					&\ve{meto\tbr{ʔ}}			&\ra& \ve{n-meto} 		& `is dry' \\
			`aged, old'			&\ve{mnasi\tbr{ʔ}}		&\ra& \ve{na-mnasi}		& `becomes old' \\
			`bad' 					&\ve{reʔu\tbr{f}}			&\ra& \ve{n-reʔu}			& `is broken/bad' \\
		\end{tabular}}%
\end{exe}

\subsection{Subject and object}\label{sec:VerArg}
Nominal phrases are eligible to be the subject or object of a verb.
Amarasi word order is subject verb object (SVO).
%Verbs agree with the subject (\S \ref{sec:VerAgrPre}, \srf{sec:ProArg}).
Any extended nominal phrase can be a subject or object in Amarasi
while there are no examples of verbs as objects or subjects in my entire corpus.
Two examples of a nominal as the subject of a clause are given in \qf{ex:120715-3, 0.33} below.

\begin{exe}
	\ex{\gll	\brac{SUBJ} \tbr{beʔi} \bracr{} na-suna =te,
						\brac{SUBJ} \tbr{naʔi} \bracr{} n-sapi ʔso-- ʔpanu ʔsonoʔ. \\
						{} PM {} \na-spin ={\te} {} PF {} \n-shave {} shell{\Mc} spoon \\
			\glt	\lh{\brac{SUBJ}}`While the grandmothers were spinning thread,
						the grandfathers \newline\hp{\brac{SUBJ}`}would cut coconut shells into spoons.'
						\txrf{120715-3, 0.33} {\emb{120715-3-00-33.mp3}{\spk{}}{\apl}}}\label{ex:120715-3, 0.33}
\end{exe}

Two examples of a nominal phrase with a single nominal in post-verbal position
as the object of the clause are given in
\qf{ex:130914-2, 0.46} and \qf{ex:130902-1, 4.32} below.

\begin{exe}
	\ex{\gll	n-naaʔ \brac{OBJ} {\tbr{benas} \bracr{}} he n-nao =t, afi--\\
						\n-hold {} machete {\he} \n-go ={\te} yesterday\\
			\glt	`He was holding a machete to go, yesterday{\ldots}'
						\txrf{130914-2, 0.46} {\emb{130914-2-00-46.mp3}{\spk{}}{\apl}}}\label{ex:130914-2, 0.46}
	\ex{\gll	neno nima =te, hai m-piir \brac{OBJ} \sf{\tbr{bupati}} \bracr{}  \\
						day five ={\te} {\hai} {\m}-elect {} regent\\
			\glt	`After five days we'll elect a regent.'
						\txrf{130902-1, 4.32} {\emb{130902-1-04-32.mp3}{\spk{}}{\apl}}}\label{ex:130902-1, 4.32}
\end{exe}

When the object nominal has already been introduced
in the discourse and/or is a known participant,
it is preceded by \ve{reʔ}.
Such uses of \ve{reʔ} are glossed {\reqt} `topic'.\footnote{
		The other function of \ve{reʔ} is as a general purpose relativiser, (glossed {\req}).
		In Koro{\Q}oto village \ve{reʔ} has the optional alternate form \ve{neʔ}
		and in Ro{\Q}is Amarasi it has the form \ve{heʔ}.}
Two examples of topical objects preceded by \ve{reʔ}
are given in \qf{ex:160326, 4.16-4.31} and \qf{ex:160326, 10.06} below,
each of which is extracted from a history of the village of \emph{Koro{\Q}oto}.
In example \qf{ex:160326, 4.16-4.31} the topical participant
is introduced as a subject in \qf{ex:160326, 4.16}.
It is repeated as subject in \qf{ex:160326, 4.25}
and when it is an object in \qf{ex:160326, 4.31} it is preceded by \ve{reʔ}.

\begin{exe}
	\ex{How the hamlet of Koro{\Q}oto got its name: \txrf{160326} {\emb{160326-04-16-04-31.mp3}{\spk{}}{\apl}}}\label{ex:160326, 4.16-4.31}
		\begin{xlist}
		\ex{\gll	neot=esa =te, siin n-took na-mfa{\tl}faun =ate \tbr{koorgw}=ees, a|n-kae.\\
							time={\es} ={\te} {\siin} \n-sit \na-{\prd}many ={\te} bird={\es} \a\n-cry\\
				\glt	`One time while they were all sitting together a bird cried.' \txrf{4.16}}\label{ex:160326, 4.16}
		\ex{\gll	\tbr{koro} ia n-kae =t n-ak: ``koorʔoot, koorʔoot, koorʔoot.''\\
							bird {\ia} \n-cry ={\te} \n-say \hphantom{``}koor{\Q}oot koor{\Q}oot koor{\Q}oot\\
				\glt	`This bird cried out: ``koor{\Q}oot, koor{\Q}oot, koor{\Q}oot''.' \txrf{4.25}}\label{ex:160326, 4.25}
		\ex{\gll	siin hai beʔi naʔi =siin n-aim \brac{OBJ} \tbr{reʔ} \tbr{koro} {ia. \bracr{}}\\
							{\siin} {\hai} PM PF ={\siinN} \n-look.for {} {\reqt} bird this\\
				\glt	`Those ancestors of ours looked for this bird.' \txrf{4.31}}\label{ex:160326, 4.31}
	\end{xlist}
\end{exe}

In example \qf{ex:160326, 10.06} the object of the locational verb \ve{n-bi}
is \ve{Koor{ʔ}oot} `Koro{\Q}oto' which has long since been established
as a highly topical participant in this story.

\begin{exe}
		\ex{\gll	\sf{na,} \sf{{\j}adi} noki-noki =te, na-tua {n-bi \hspace{3mm}\brac{OBJ}} \tbr{reʔ} {\tbr{Koorʔoot} \bracr{}} sero ʔroo.\\
							well so eventually ={\te} \na-live \n-{\bi} {\reqt} Koro{\Q}oto rather long\\
				\glt	`Eventually they'd been living in Koro{\Q}oto a while.'
							\txrf{160326, 10.06} {\emb{160326-10-06.mp3}{\spk{}}{\apl}}}\label{ex:160326, 10.06}
\end{exe}

Nominal phrases containing only a property nominal can also be verbal arguments.
Two examples of such nominal phrases as the object of a verb
are given in \qf{ex:120715-4, 0.45} and \qf{ex:140726, 0.00} below.\footnote{
	There is no morphosyntactic basis for separating
	the locational verbs \ve{n-bi} \tsc{\bi} (realis locative)
	and \ve{n-eu} \tsc{\eu} (dative) from the word class of verbs.}

\begin{exe}
		\ex{\gll a|n-moʔe =ma n-poo\j=ena n-bi \brac{OBJ} \tbr{metoʔ}. \bracr{}\\
						%	{\a}n-moʔe =ma n-poi=ena n-bi {} metoʔ {}\\
							{\a\n}-make =and {\n}-exit={\een} {\n}-{\bi} {} dry {}\\
				\glt	\lh{a|}`He made and went out onto a dry place.'
							\txrf{120715-4, 0.45} {\emb{120715-4-00-45.mp3}{\spk{}}{\apl}}}\label{ex:120715-4, 0.45}
	\ex{\gll	baisenu-t =ma ronaen n-eu \brac{OBJ} \tbr{mutiʔ} =ma {mnatuʔ \bracr{} }
						et muit ma-hine-ʔ =ma mnatuʔ neee.\\
					%	baisenu-t ma ronaen n-eu {} mutiʔ =ma mnatuʔ {} et mutiʔ ma-hineʔ ma mnatuʔ nehh\\
						look.up-{\at} =and greeting {\n}-{\eu} {} white and gold {\et} white{\M} {\ma}-know-{\ma} =and gold \tsc{pause}\\
			\glt	`Greetings and honour to (those like) silver and gold, wise silver and gold'
						(figurative for `wise and honoured dignitaries'.)
						\txrf{140726, 0.00} {\emb{140726-00-00.mp3}{\spk{}}{\apl}}}\label{ex:140726, 0.00}
\end{exe}

Other parts of the extended nominal phrase including numbers,
demonstratives, and quantifiers can also be the subject and object of a verb.
Examples are given in \srf{sec:OthNomMod}.

\subsubsection{Pronominal subjects and objects}\label{sec:ProArg}
Pronouns are a subclass of nominals in Amarasi.
They can be distinguished from other nominals
as they inflect for case: nominative or accusative.
Nominative pronouns are given in \trf{tab:NomPro}
and accusative pronouns in \trf{tab:AccPro}.
Nominative pronouns are used for subjects,
and accusative pronouns for objects and/or benefactives.\footnote{
		Accusative pronouns are also used as the second element in 
		a pronominal equative clause (\srf{sec:EquCla}).}
(The Ro{\Q}is \tsc{3sg} nominative pronoun is \ve{hiin}.)

\begin{table}[h]
	\caption{Amarasi pronouns}\label{tab:AmaPr}
	\begin{subtable}[b]{0.49\textwidth}
		\centering\caption{Nominative pronouns}\label{tab:NomPro}
			\begin{tabular}{rll} \lsptoprule
						& \tsc{sg}	&	\tsc{pl}	\\ \midrule
				1		& \ve{au}		& \ve{hai}	\\
				1,2	& 					& \ve{hiit}	\\
				2		& \ve{hoo}	& \ve{hii}	\\
				3		& \ve{iin}	& \ve{siin}	\\ \lspbottomrule
			\end{tabular}
	\end{subtable}
	\begin{subtable}[b]{0.49\textwidth}
		\centering\caption{Accusative pronouns}\label{tab:AccPro}
			\begin{tabular}{rll} \lsptoprule
						& \tsc{\hp{=}sg}	&	\tsc{\hp{=}pl}	\\ \midrule
				1		& \ve{=kau}	& \ve{=kai}		\\
				1,2	& 					& \ve{=kiit}	\\
				2		& \ve{=koo}	& \ve{=kii}		\\
				3		& \ve{=ee}	& \ve{=siin}	\\ \lspbottomrule
			\end{tabular}
	\end{subtable}
\end{table}

Two examples of nominative pronouns as the subject are given
in \qf{ex:120715-1, 0.30} and \qf{ex:120715-4, 3.05} below,
with the verbal agreement also indicated.
Two examples of an accusative pronoun as the object of a verb
are given in \qf{ex:130907-4, 2.31} and \qf{ex:130902-1, 3.38} below.

\begin{exe}
	\ex{\gll	\tbr{au} he \tbr{u}-toon n-ok kuan Nekmeseʔ.\\
						{\au} {\he} {\qu}-tell \n-{\ok} village Nekmese{\Q}\\
			\glt	`I want to talk about Nekmese{\Q} village.'
						\txrf{120715-1, 0.30} {\emb{120715-1-00-30.mp3}{\spk{}}{\apl}}}\label{ex:120715-1, 0.30}
	\ex{\gll	\tbr{hoo} \tbr{mu}-mnau fatu Brao=n konaʔ hiut?\\
						{\hoo} \muu-remember stone Brao={\ein} hole seven\\
			\glt	`Do you remember the Brao stones' seven holes?'
						\txrf{120715-4, 3.05} {\emb{120715-4-03-05.mp3}{\spk{}}{\apl}} }\label{ex:120715-4, 3.05}
	\ex{\gll	mama na-tuinaʔ =\tbr{kau} =ma,\\
						mum \na-follow ={\kau} =and \\
			\glt	`Mum agreed with me and{\ldots}'
						\txrf{130907-4, 2.32} {\emb{130907-4-02-32.mp3}{\spk{}}{\apl}}}\label{ex:130907-4, 2.31}
	\ex{\gll	ertee\j=ii n-pooʔ =\tbr{kai} =ma hai m-fena =m\\
						neighbourhood.head={\ii} \n-wake ={\kai} =and {\hai} \m-rise =and\\
			\glt	`The neighbourhood head woke us up and we got up'
						\txrf{130902-1, 3.38} {\emb{130902-1-03-38.mp3}{\spk{}}{\apl}}}\label{ex:130902-1, 3.38}
\end{exe}

The third person singular accusative pronoun is the vowel-initial enclitic \ve{=ee}.
Examples are given in \qf{ex:130907-3, 10.29} and \qf{ex:130825-6, 3.33} below.
This enclitic is also a nominal determiner,
marking the definiteness and topicality of a nominal phrase
(see \srf{sec:FunDet} for more details).

\begin{exe}
	\ex{\gll	na-sae-b=\tbr{ee} =m n-eek\j=\tbr{ee} n-nao n-bi Alor.\\
						\na-rise-\b={\eeV} =and \n-take={\eeV} \n-go \n-{\bi} Alor\\
			\glt	`(They) picked him up and took him to Alor.'
						\txrf{130907-3, 10.29} {\emb{130907-3-10-29.mp3}{\spk{}}{\apl}}}\label{ex:130907-3, 10.29}
	\ex{\gll	oras ia au ʔ-oopʔ=\tbr{ee} n-fain et au kuan.\\
						time {\ia} {\au} \q-pour={\eeV} \n-again {\et} {\au} village\\
			\glt	`Now I'm just pouring it back into my (own) village. '
						\txrf{130825-6, 3.33} {\emb{130825-6-03-33.mp3}{\spk{}}{\apl}}}\label{ex:130825-6, 3.33}
\end{exe}

One syntactic test which allows us to identify a word class of nominals
in Amarasi is that nominals can be the subject or object of a verb.

\subsection{Determiners}\label{sec:Det}
Another syntactic criterion which nominals fulfil
is that they can be followed by a determiner.
The Amarasi determiners are given in \trf{tab:AmaNomDet2} below.
They have the same four person values present in the genitive suffixes
(\srf{sec:GenSuf}, \srf{sec:GenSuf ch:SynMet}).

\begin{table}[ht]
	\caption{Amarasi determiners}\label{tab:AmaNomDet2}
	\centering
		\begin{tabular}{lll} \lsptoprule
			Form			&Gloss	& Use\\ \midrule
			\ve{=ii}		&{\ii}	& definite referent near/relevant to speaker\\
			\ve{=ana/=aan}	&{\aan}	&	definite referent near/relevant to addressee\\
			\ve{=ee}		&{\ee}	&	definite referent near/relevant to a third person\\
			\ve{=aa}		&{\aa}	&	definite referent near/relevant to no one (≈ obviative)\\
		\lspbottomrule
		\end{tabular}
\end{table}

All these determiners are vowel-initial enclitics,
and the stem to which they attach undergoes phonologically
conditioned metathesis, as discussed in Chapter \ref{ch:PhoMet}.
(Such phonologically conditioned M\=/forms are glossed `{\Mvv}'.)
These enclitics occur after definite topical nominals,
discussed further in \srf{sec:FunDet} below.

The enclitic \ve{=ee} can also attach
to a verb to mark a third person singular pronominal object.
Similarly, the enclitic \ve{=ii} can attach to a phrase to
raise the discourse prominence of that phrase.
The enclitics \ve{=aan} `{\aan}' and \ve{=aa} `{\aa}' have
only been attested attached to a nominal phrase.
An example of each is given in \qf{ex:130825-6, 6.31} and \qf{ex:120923-1, 7.15} below.

\begin{exe}
	\ex{\gll	Meok Seran aanh=\tbr{aan} naiʔ sekaagw=een?\\
						Meok Seran child={\aan} {\naiq} who={\een}\\
			\glt	`Who is Meok Seran's son, then?' \txrf{130825-6, 6.31} {\emb{130825-6-06-31.mp3}{\spk{}}{\apl}}}\label{ex:130825-6, 6.31}
	\ex{\gll	atoniʔ iin, n-pairoir iin muiʔt=\tbr{aa} =t iin n-hae \sf{{\j}adi}\\
						man {\iin} \n-prepare {\iin} animal={\aa} ={\te} {\iin} \n-tired so\\
			\glt	`Someone prepares his animal (then) he's tired, so {\ldots}'
						\txrf{120923-1, 7.15} {\emb{120923-1-07-15.mp3}{\spk{}}{\apl}}}\label{ex:120923-1, 7.15}
\end{exe}

Property nominals can also take any of the nominal determiners
given in \trf{tab:AmaNomDet2}.
Examples are given in \qf{ex:120923-1, 12.22} and \qf{ex:120715-4, 0.47} below.

\begin{exe}
	\ex{\gll	hoo m-ait\j=ee \sf{berarti} of hoo m-ait mu-faniʔ mapuutʔ=\tbr{ee}.\\
					%	hoo m-aiti=ee \sf{berarti} of hoo m-aiti mu-faniʔ maputuʔ=ee\\
						{\hoo} {\m}-take={\eeV} mean later {\hoo} {\m}-take {\muu}-repeat hot={\ee}\\
			\glt	`(If) you take it, it means that later you'll burn repeatedly.'
						(\emph{lit.} `take repeatedly the hot/heat')\txrf{120923-1, 12.22} {\emb{120923-1-12-22.mp3}{\spk{}}{\apl}}}\label{ex:120923-1, 12.22}
	\ex{\gll	n-poi n-bi meotʔ=\tbr{ee} onai =te\\
						{\n}-exit {\n}-{\bi} dry={\ee} like.that ={\te}\\
			\glt	`Having gone out onto the land like that {\ldots}'
						\txrf{120715-4, 0.47} {\emb{120715-4-00-47.mp3}{\spk{}}{\apl}}}\label{ex:120715-4, 0.47}
\end{exe}

\subsubsection{Function of determiners}\label{sec:FunDet}
Determiners in Amarasi occur attached to definite referents
which could be expected from the discourse context.
They cannot occur on referents with the pragmatic role
of focus (in the sense of \citealt[214]{la94}).
They have four person values,
and naturally pattern with equivalent pronouns,
as illustrated in examples \qf{ex:au niisk ii}--\qf{ex:in niisn aa} below.

\begin{multicols}{2}
	\begin{exe}
		\ex{\glll \tbr{au} niis-k=\tbr{ii}\\
							au nisi-k=ii \\
							\tbr{\au} tooth-\tsc{\k}=\tbr{\ii}\\
				\glt `my tooth'}\label{ex:au niisk ii}
		\ex{\glll \tbr{hoo} niis-m=\tbr{aan}\\
							hoo nisi-m=ana \\
							\tbr{\hoo} tooth-{\mg}=\tbr{\aan}\\
				\glt `your tooth'}\label{ex:hoo niism aan}
	\end{exe}
\end{multicols}
\begin{multicols}{2}
	\begin{exe}
		\ex{\glll \tbr{iin} niis-n=\tbr{ee}\\
							ini nisi-n=ee \\
							\tbr{\iin} tooth-{\N}=\tbr{\ee}\\
				\glt `her/his tooth'}\label{ex:in niisn ee}
		\ex{\glll \tbr{iin} niis-n=\tbr{aa}\\
							ini nisi-n=aa \\
							\tbr{\iin} tooth-{\N}=\tbr{\aa}\\
				\glt `someone's tooth'}\label{ex:in niisn aa}
	\end{exe}
\end{multicols}

The use of these determiners to mark expected definite nominals
is illustrated in \qf{ex:120715-4, 0.37-0.47} below.
The discourse of \qf{ex:120715-4, 0.37-0.47} is structured such that each clause
(except the first) is paralleled by the following clause.
The first part of each pair (\ref{ex:120715-4, 0.37-0.47, 0.37b},
\ref{ex:120715-4, 0.37-0.47, 0.40b}, \ref{ex:120715-4, 0.37-0.47, 0.45})
introduces a new participant,
with this participant then repeated marked with a determiner
in the second part of each pair (\ref{ex:120715-4, 0.37-0.47, 0.40},
\ref{ex:120715-4, 0.37-0.47, 0.43}, \ref{ex:120715-4, 0.37-0.47, 0.45b}).

\begin{exe}
	\ex{How Moo{\Q}-hitu made the world:  \txrf{120715-4} {\emb{120715-4-00-37-00-47.mp3}{\spk{}}{\apl}}}\label{ex:120715-4, 0.37-0.47}
	\begin{xlist}
		\ex{\gll	naʔ n-sanu n-fani kreʔo{\tl}kreʔo =ma\\
							then \n-descend \n-back {\frd}slow =and\\
				\glt	`Then (he) went back down slowly and' \txrf{}}\label{ex:120715-4, 0.37-0.47, 0.37} 
		\ex{\gll	n-fani n-bi iin \brac{} {\tbr{bara}-\tbr{n}. \bracr{NEW-\it{i}}}\\
							\n-return \n-{\bi} {\iin} {} place-{\N}\\
				\glt	`went back to his place.' \txrf{0.37}}\label{ex:120715-4, 0.37-0.47, 0.37b}
		\ex{\gll	n-fani n-bi iin \brac{} {\tbr{baar}-\tbr{n}=\tbr{ii}. \bracr{OLD-\it{i}}}\\
							return \n-{\bi} {\iin} {} place-{\N}=\tbr{\ii}\\
				\glt	`went back to his place.' \txrf{}}\label{ex:120715-4, 0.37-0.47, 0.40}
		\ex{\gll	iin \brac{} {baar-n=ee \bracr{OLD-\it{i}}} et oo\j=ee \brac{} {\tbr{nana}-\tbr{n}. \bracr{NEW-\it{j}}}\\
							{\iin} {} place-{\N}={\ee} {\et} water={\ee} {} inside-{\N}\\
				\glt	`His place was in the water.' \txrf{0.40}}\label{ex:120715-4, 0.37-0.47, 0.40b}
		\ex{\gll	n-bi{\tl}bi oo\j=ee \brac{} {\tbr{naan}-\tbr{n}=\tbr{ee} \bracr{OLD-\it{j}}} {onai =te},\\
							\n-{\frd}{\bi} water={\ee} {} inside-{\N}=\tbr{\ee} and.then\\
				\glt	`Having been in the water for a while,' \txrf{0.43}}\label{ex:120715-4, 0.37-0.47, 0.43}
		\ex{\gll	a|n-moʔe =ma n-poo\j=ena n-bi \brac{} {\tbr{metoʔ}. \bracr{NEW-\it{k}}}\\
							{\a\n}-make =and \n-exit={\een} {\n}-{\bi} {} dry\\
				\glt	\lh{a|}`(he) made and went out onto dry land.' \txrf{0.45}}\label{ex:120715-4, 0.37-0.47, 0.45}
		\ex{\gll	n-poi n-bi \brac{} {\tbr{meotʔ}=\tbr{ee} \bracr{OLD-\it{k}}} {onai =te,}\\
							\n-exit \n-{\bi} {} dry=\tbr{\ee} and.then\\
				\glt	`Having gone out onto the dry land,' \txrf{}}\label{ex:120715-4, 0.37-0.47, 0.45b}
		\ex{\gll	iin ka= n-muiʔ =fa \brac{} {bare \bracr{NEW-\it{l}}} he na-tua =m\\
							{\iin} {\ka}= \n-have ={\fa} {} place  {\he} \na-settle =and\\
				\glt	`he didn't have a place to live and, {\ldots}' \txrf{0.47}}\label{ex:120715-4, 0.37-0.47, 0.47}
	\end{xlist}
\end{exe}

The choice between different determiners is extremely subtle
and serves to signal different levels of discourse prominence among definite participants.
In \qf{ex:120715-4, 0.37-0.47} above
each second mention of \ve{nana-n} `inside'
in \qf{ex:120715-4, 0.37-0.47, 0.43} and \ve{metoʔ} `dry'
in \qf{ex:120715-4, 0.37-0.47, 0.45b} occur with
the third person determiner \ve{=ee}.
Each of these nouns is also only mentioned twice in this extract.
This is in contrast to \ve{bara-n} `place', which on its second mention
in \qf{ex:120715-4, 0.37-0.47} occurs with the first person determiner \ve{=ii}.
This noun is also the only noun which occurs three times in this extract
with its third mention in \qf{ex:120715-4, 0.37-0.47, 0.40b}
occurring with the third person determiner \ve{=ee}.\footnote{
		The same noun occurs in \qf{ex:120715-4, 0.37-0.47, 0.47},
		though with a different referent.
		The final vowel of \ve{bare} `place' is irregularly /a/
		when a genitive suffix is attached; \ve{bara-n} place-{\N}.}

The first person determiner raises the discourse prominence
of the participant it is attached to and signals that this
participant is slightly more important than other participants.
Such subtleties are further illustrated in \qf{ex:130825-6} below,
in which a single participant occurs with \ve{=ii} `{\ii}' on its
first mention and \ve{=aan} `{\aan}' on its second mention.

\begin{exe}
	\ex{Asking for the name of someone: \txrf{130825-6} {\emb{130825-6-02-06-02-09.mp3}{\spk{}}{\apl}}}\label{ex:130825-6}
	\begin{xlist}
		\ex{\gll	hoo feat-f=\tbr{ii} bi sekau?\\
							{\hoo} man's.sister-{\F}=\tbr{\ii} {\BI} who\\
				\glt	`Who is your sister?' \txrf{2.06}}\label{ex2:130825-6, 2.06}
		\ex{\gll	au aanh=ii naiʔ Lukas fee\j=\tbr{aan}?\\
							{\au} child={\ii} {\naiq} Lukas wife=\tbr{\aan}\\
				\glt	`My son Lukas's wife?' \txrf{2.09}}\label{ex:130825-6, 2.09}
	\end{xlist}
\end{exe}

By changing from the first person determiner in \qf{ex2:130825-6, 2.06}
to the second person determiner \ve{=aan} in \qf{ex:130825-6, 2.09}
the speaker moves this participant from his own ``space'' to the ``space''
of the addressee; it is knowledge the speaker wants to know
but which the addressee is presumed to have access to.

\subsection{Number enclitics}\label{sec:NumEnc}
Another characteristic of Amarasi nominals is that they
can be modified by either of the number enclitics
given in \trf{tab:AmaEncNum} below.
The syntactic structure of these number enclitics is discussed in \srf{sec:NumEnc sec:OthNomMod}.
Examples of a nominal followed by a number enclitic
are given in \qf{ex:130822-1, 0.56}--\qf{ex:130914-3, 0.58} below.
Broadly speaking, allomorph \ve{=ein} of the plural enclitic
is used after consonant-final stems and \ve{=n} after vowel-final stems
(see \srf{sec:PluEnc} for a full discussion).

\begin{table}[h]
	\caption{Amarasi number enclitics}\label{tab:AmaEncNum}
	\centering
		\begin{tabular}{lll} \lsptoprule
			Form			&Gloss	& Use\\ \midrule
			\ve{=ein}, \ve{=n}	&{\ein}	&	plural\\
			\ve{=esa/=ees}	&{\es}	&	indefinite singular; the numeral one (1)\\ \lspbottomrule
		\end{tabular}
\end{table}

\begin{exe}
	\ex{\gll	sbaekʔ=\tbr{ees} na-fua, sbaekʔ=\tbr{esa} msaʔ na-fua. \\
						branch={\es} \na-fruit branch={\es} also \na-fruit\\
			\glt	`A branch grew fruit, another branch also grew fruit.'
						\txrf{130822-1, 0.56}{\emb{130822-1-00-56.mp3}{\spk{}}{\apl}}}\label{ex:130822-1, 0.56}
	\ex{\gll	a|n-tui hii kaan-m=\tbr{ein} n-bi eanʔ=\tbr{ein} ehh?	\\
						\a\n-write {\hii} name-{\mg}={\ein} \n-{\bi} door={\ein} {\aaQ} \\
			\glt	\lh{a|}`Were your names were written on the doors?'
						\txrf{130825-7, 0.38} {\emb{130825-7-00-38.mp3}{\spk{}}{\apl}}}\label{ex:130825-7, 0.38}
	\ex{\gll	karu hoo m-serak =kau au huut=\tbr{n}=aan. \\
						if {\hoo} \m-take.apart ={\kau} {\au} head.louse={\ein}={\aan}.\\
			\glt	`if you part (my hair), (those) are my lice'
						\txrf{130914-3, 0.58}{\emb{130914-3-00-58.mp3}{\spk{}}{\apl}}}\label{ex:130914-3, 0.58}
\end{exe}

Examples of property nominals which are the head of a nominal phrase
modified by a number enclitic are rare.
This is probably due to the fact that property nominals
do not usually have a countable meaning.
Three examples from the Amarasi Bible translation
are given in \qf{ex:Luke 8:17}--\qf{ex:Genesis 24:22} below.
In examples \qf{ex:Luke 8:17} and \qf{ex:Genesis 33:11} the nominal
modified by \ve{=ein} is a property nominal derived from
a verbal root with the property circumfix \ve{ma-{\ldots}-ʔ} (\srf{sec:PropCir}).

\begin{exe}
	\ex{\gll	ma areʔ kana=n rasi reʔ ka= ma-hini-ʔ =fa oras ia, of ma-hiin-ʔ=\tbr{ein}.\\
						And every name={\ein} matter {\req} {\ka}= {\ma}-know-{\ma} ={\fa} time {\ia} later {\ma}-know-{\ma}={\ein}\\
			\glt	`And each matter which is not known now will later be known.' \xrf{Luke 8:17}}\label{ex:Luke 8:17}
	\ex{\gll	na-tuin baerʔ=ein naan siin ka= ma-ʔoos-ʔ=\tbr{ein} =fa. \\
						\na-because thing={\ein} {\naan} {\siin} {\ka}= {\ma}-price-{\ma}={\ein} ={\fa} \\
			\glt	`Because those things have no value.' \txrf{Genesis 33:11}}\label{ex:Genesis 33:11}
	\ex{\gll	rari =t niti mnaut nuaʔ=ein naan siin maʔfeen-k=\tbr{ein} \sf{sekel} boʔ=ees. \\
						finish ={\te} bracelet gold two={\ein} {\naan} {\siin} heavy-\k={\ein} shekel ten=one \\
			\glt	`Then those two gold bracelets weighed ten shekels.'
						(\emph{lit.} `their heavinesses were ten shekels') \txrf{Genesis~24:22}}\label{ex:Genesis 24:22}
\end{exe}

The plural enclitic \ve{=ein/=n} also occurs with verbs.
When it does so it marks that one of the arguments is plural.

\subsubsection{The polyfunctional form \it{ees}}\label{sec:PolFunEs}
The form \ve{esa/ees} has a number of functions in Amarasi,
not all of which are as an enclitic.
These are derived from or extensions of Proto-Malayo-Polynesian *əsa `one'.
It has the U\=/forms (=)\ve{esa} and the M\=/form (=)\ve{ees}.
However, the use of each form is not fully productive,
as is consistent with other functors which have U\=/forms and M\=/forms.
While at an abstract level the M\=/form must be posited as having
two underlying vowels \ve{ees} it is usually unstressed
and pronounced with a single vowel.

The most frequent use of this form is 
as a number enclitic attached to a noun phrase.
When it does so, it has a range of 
uses which range between the numeral `one (1)'
and a more semantically bleached indefinite marker.
Two examples in which the numeral meaning
of this enclitic is not prominent
are given in \qf{ex:120715-3, 0.46} and \qf{ex:130821-1, 3.43-3.47} below.

\begin{exe}
	\ex{\gll	fee{\gap}mnaisʔ=\tbr{ees} nema =ma n-ak: \\
						old.woman=\tbr{\es} {\nema} =and {\n}-say\\
			\glt	`An old woman came and said:{\ldots}'
						\txrf{120715-3, 0.46} {\emb{120715-3-00-46.mp3}{\spk{}}{\apl}}}\label{ex:120715-3, 0.46}
	\ex{\begin{xlist}
		\ex{\gll	naiʔ Soan reʔ ia iin am-neem-n=ii na-ʔko, \\
							{\naiq} Soan {\req} {\ia} {\iin} {\at}-come-{\N}={\ii} {\na}-{\qko} \\}
		\ex{\gll	paah kuan=\tbr{ees} kaan-n=ee Kuatunis.\\
							country village=\tbr{\es} name-{\N}={\ee} Kuatunis\\
				\glt	`This Soan came from a village called Kuatunis.'
							\txrf{130821-1, 3.47} {\emb{130821-1-03-47.mp3}{\spk{}}{\apl}}}
		\end{xlist}}\label{ex:130821-1, 3.43-3.47}
\end{exe}

Two examples in which the numeral function of \ve{=ees} is more prominent
are given in \qf{ex2:130821-1, 1.17} and \qf{ex2:130823-2, 0.49} below.

\begin{exe}
	\ex{\gll	{oka =t} tuaf nua n-fain nai =t, tuaf=\tbr{ees} na-maikaʔ n-ok =kiit funan nua, of hii ees m-oka=n.\\
						after.that person two {\n}-return already ={\te} person=\tbr{\es} {\na}-stay {\n}-{\ok} ={\kiit} moon two later {\hii} {\es} \m-{\ok}={\einV}\\
			\glt	`After that when two people have already gone back
						one person will stay with us for two months,
						later you'll be the ones with him.'
						\txrf{130821-1, 1.17} {\emb{130821-1-01-17.mp3}{\spk{}}{\apl}}}\label{ex2:130821-1, 1.17}
	\ex{\gll	taaʔ\j=\tbr{ees}=ii mutiʔ.\\
						branch=\tbr{\es}={\ii} white\\
			\glt	`One of these branches was white.'
						\txrf{130823-2, 0.49} {\emb{130823-2-00-49.mp3}{\spk{}}{\apl}}}\label{ex2:130823-2, 0.49}
\end{exe}

The form \ve{esa/ees} also has a number of non-enclitic uses as an independent word.
Unlike the enclitic uses, \ve{esa/ees} as an independent word 
begins with an automatic glottal stop as [ˈʔɛsɐ] or [ʔɛs]
as is expected for vowel-initial words (\srf{sec:GloStoIns}).
None of these uses trigger M\=/forms.
One of these uses is as the head of number phrase with the meaning `one',
as in examples \qf{ex:120715-1, 0.44-0.47} and \qf{ex:130906-1, 3.15} below.
In such uses it selects a single referent out of a range of possible referents.

\begin{exe}
	\ex{\begin{xlist}
		\ex{\gll	\tbr{ees} eta ʔTakaʔ, ʔTakaʔ.\\
							{\es} {\et} {\Q}Taka{\Q} {\Q}Taka{\Q}\\
				\glt	`One (of them) was at {\Q}Taka{\Q}, (that one is) {\Q}Taka{\Q}.'
							\txrf{120715-1, 0.44} {\emb{120715-1-00-44-00-47.mp3}{\spk{}}{\apl}}}
		\ex{\gll	\tbr{ees} et Kotos, Koorʔoto.\\
							{\es} {\et} Kotos Koro{\Q}oto\\
				\glt	`One (of them) was at Kotos, (that one is) Koro{\Q}oto.' \txrf{0.47}}
	\end{xlist}}\label{ex:120715-1, 0.44-0.47}
	\ex{\gll	\tbr{esa} n-poi n-teniʔ. \\
						{\es} {\n}-exit {\n}-again \\
			\glt	`One (of them) came out again.'
						\txrf{130906-1, 3.15} {\emb{130906-1-03-15.mp3}{\spk{}}{\apl}}}\label{ex:130906-1, 3.15}
\end{exe}

Probably as an extension of this use,
\ve{esa/ees} is used in a contrastive focus construction
in which it introduces participants
who are in contrast with other participants of the discourse.
Often this contrast is implicit,
and indeed in many instances the contrast function
is semantically bleached
and \ve{esa/ees} functions as a copula or relativiser.
Such uses of \ve{ees} are glossed as {\esc}.
Four examples are given in \qf{ex:120715-4, 4.07}--\qf{ex:13/09/14, p.96} below.

\begin{exe}
	\ex{\gll	\sf{{\j}adi,} iin naʔi iin beʔi \tbr{ees} n-- na-konaʔ reʔ fatu Brao=n. \\
						so {\iin} PF {\iin} PM {\esc} {} {\na}-hole {\reqt} stone Br.={\ein} \\
			\glt `So, his ancestors were the ones who made the holes in the Brao stones.'
						\xrf{120715-4, 4.07} {\emb{120715-4-04-07.mp3}{\spk{}}{\apl}}}\label{ex:120715-4, 4.07}
	\ex{\gll	au \tbr{ees} a-meup umi.\\
						{\au} {\esc} {\at}-work house\\
			\glt	`I'm the one building the house.' (\emph{lit.} `house worker')
						\txrf{Obs. 13/09/14, p.96}}\label{ex:13/09/14, p.96}
	\ex{\gll	au \tbr{ees} a-na{\tl}nao-t. au \tbr{ees} a-tok{\tl}took \sf{sidaŋ}.\\
						{\au} {\esc} {\at}-{\prd}go-{\at} {\au} {\esc} {\at}-{\prd}sit meeting\\
			\glt	`I was the one who went (\emph{lit.} goer). I was the one who attended the meetings (\emph{lit.} meeting sitter).'
						\txrf{130907-3, 8.40} {\emb{130907-3-08-40.mp3}{\spk{}}{\apl}}\vspace{-9pt}}\label{ex:130907-3, 8.40}
\end{exe}

When \ve{esa/ees} is used as a copula, it has the optional
plural form \ve{esa=n}, when the subject is plural.
Two examples of plural \ve{esa=n} are given in
\qf{ex:120715-4, 3.57} and \qf{ex:130905-1, 1.18}.

\begin{exe}
	\ex{\gll	iin naʔi iin beʔi \tbr{esa}=\tbr{n} reʔ ma-keen uun.\\
						{\iin} PF {\iin} PM {\esc}={\ein} {\req} {\ma}-weapon earlier\\
			\glt	`His ancestors were the ones who were at war.'
						\txrf{120715-4, 3.57} {\emb{120715-4-03-57.mp3}{\spk{}}{\apl}}}\label{ex:120715-4, 3.57}
	\ex{\gll	\sf{na,} uab=ein \tbr{esa}=\tbr{n} reʔ ia.\\
						well speech={\ein} {\esc}={\ein} {\req} {\ia}\\
			\glt	`Well, these are the things I wanted to say.' \newline
						(\emph{lit.} `Speeches are the ones who are here.')
						\txrf{130905-1, 1.18} {\emb{130905-1-01-18.mp3}{\spk{}}{\apl}}}\label{ex:130905-1, 1.18}
\end{exe}

Finally, there is a homophonous word [ʔɛs],
transcribed \ve{es}, which is a locative marker.
This \ve{es} is probably not cognate with the numeral \ve{esa/ees},
and is instead related to locative \ve{et}.
These forms appear to have no semantic difference.
Both are an imperfective locative.
Following the definition of \citet[16]{co76},
this locative focusses on the internal structure of the situation of 
being present in a certain location with the end
point of this situation (being present in a certain location)
either removed or not in focus.
The form \ve{et} is more common in my corpus with 77 attestations
compared to eight attestations of locative \ve{es}.
Two examples of locative \ve{es} are given in
\qf{ex:130821-1, 0.38} and \qf{ex:130925-1, 1.24} below.\footnote{
		In other varieties of Meto, including Amfo{\Q}an and Baikeno,
		the form \ve{es} is more common.
		In my Ro{\Q}is Amarasi data this locative
		has the form \ve{ek/et} with \ve{ek} most common.
		Ro{\Q}is \ve{ek} seems to function as a more
		general locative, rather than specifically imperfective.}

\begin{exe}
	\ex{\gll	aan moon\j=ees \tbr{es} nana-f ia, feʔ munif.\\
						child male={\es} \tbr{\et} inside-{\f} {\ia} still young\\
			\glt	`A young man inside here, (he's) still young.'
						\txrf{130821-1, 0.38} {\emb{130821-1-00-38.mp3}{\spk{}}{\apl}}}\label{ex:130821-1, 0.38}
	\ex{\gll	m-aam\j=ee =t, m-iit=ee \tbr{es} mee?\\
						{\m}-look.for={\eeV} ={\te} {\m}-see={\eeV} \tbr{\et} where\\
			\glt	`When you look for him, where will you find him?'
						\txrf{130925-1, 1.24} {\emb{130925-1-01-24.mp3}{\spk{}}{\apl}}}\label{ex:130925-1, 1.24}
\end{exe}

To summarise, the forms \ve{esa/ees} and similar \ve{es} have a number of functions.
When \ve{esa/ees} is a number enclitic it functions as the numeral `one'
or as an indefinite marker and it triggers M\=/forms.
In other situations it is an independent word and does not trigger M\=/forms.