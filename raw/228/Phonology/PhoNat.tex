\subsection{Phonotactic nativisation of loan words}\label{sec:PhoNat}
Amarasi roots conform to a strict set of structures based on the CVCVC foot.
Loanwords with other structures are adapted to conform to this structure.
One common disallowed shape in loans is final CCV{\#}.
Amarasi employs one of two strategies to conform such words to the CVCVC foot template.
The most common strategy is to metathesise C\sub{1}C\sub{2}V{\#} to C\sub{1}VC\sub{2}{\#}.
Examples are shown in \trf{tab:LoaMet}.\footnote{
		Another possible example is Dutch \it{mark} > \ve{marak} `brand, stamp',
		though this could simply show final epenthesis.}
There is also one clear example of a final CCV{\#}
being adapted by doubling the final vowel: \ve{n-saksii} `witness'
ultimately from Sanskrit \it{s\={a}kṣi} /saːkʂi/.\footnote{
		Another possible example of final vowel doubling is \ve{kantoor} `office'.
		This could be a borrowing directly from Dutch \it{kantoor}
		or it could have entered Amarasi via Malay \it{kantor}.}

\begin{table}[h]
	\centering\caption{Loanword metathesis}\label{tab:LoaMet}
		\begin{tabular}{llll} \lsptoprule
			 Meaning 				&	Root					&	 Source 				&	Donor 				\\ \midrule
			 `party'				&	\ve{fesat}		&	\emph{festa}		&	Portuguese		\\
			 `examine' 			&	\ve{parikas}	&	\emph{periksa}	&	via Malay			\\
			 `govern(ment)' &	\ve{prenat} 	&	\emph{prenta}		&	Kupang Malay	\\
			 `dance' 				&	\ve{ranas}		&	\emph{dansen}		&	Dutch					\\
			 `lights' 			&	\ve{ramup}		&	\emph{lampu}		&	via Malay			\\
			 `sign' 				&	\ve{tanar}		&	\emph{tanda}		&	Malay					\\
			 `tent' 				&	\ve{tenar}		&	\emph{tenda}		&	Portuguese		\\ \lspbottomrule
		\end{tabular}
\end{table}

\begin{table}[h]
	\caption{Reanalysis of loanwords with enclitic \ve{=aa}}\label{tab:ReaLoaEnc}
	\centering
			\begin{threeparttable}
		\begin{tabular}{llllll} \lsptoprule
			Gloss					&	Root								&	with \ve{=aa}		&	 Source 			&							&	Donor	\\ \midrule
			 `deer'				&	\ve{ruus}						&	\ve{ruus=aa}		&	\it{rusa}			&	/rusa/			&	via Malay	\\
			 `disaster'		&	\ve{siraak}					&	\ve{siraak=aa}	&	\it{cilaka}		&	/\tS ilaka/	&	via Malay	\\
			 `girlfriend'	&	\ve{binoon}\su{†}		&	\ve{binoon=aa}	&	\it{nona}			&	/nona/			&	via Malay	\\
			 `minister'		&	\ve{panriit}				&	\ve{panriit=aa}	&	\it{paṇḍita}	&	/paɳɖita/		&	ult. Sanskrit	\\
			 `chair' (Ro{\Q}is)	&	\ve{kanreer}	&	\ve{kanreer=aa}	&	\it{cadeira}	&	/kadeira/		&	Portuguese	\\
			 `church'			&	\ve{krei}						&	\ve{kree\j=aa}	&	\it{igreja}		&	/iɡreʒa/		&	Portuguese	\\
			 `table'			&	\ve{mei}						&	\ve{mee\j=aa}		&	\it{mesa}			&	/meza/			&	Portuguese	\\
			 `window'			&	\ve{enosneer}\su{‡}	&	\ve{sneer=aa}		&	\it{janela}		&	/ʒanɛla/		&	Portuguese	\\ \lspbottomrule
		\end{tabular}
			\begin{tablenotes}
				\item [†]	Initial \ve{bi} in \ve{binoon}
									is from the feminine article \ve{bi} also used
									before female proper names.
				\item [‡]	Initial \ve{eno} in \ve{enosneer} `window' is from \it{enoʔ} `door'.
									The derivation of the final element from Portuguese \it{janela} /ʒanɛla/
									involves antepenultimate vowel reduction and devoicing of
									the initial voiced fricative; thus
									*ʒanɛla > **sanela > **snela > **snera > **sneer=aa.
			\end{tablenotes}
		\end{threeparttable}
\end{table}

Finally, there are half a dozen or so loanwords
which have been re-analysed as containing
the nominal determiner \ve{=aa}, with this enclitic
then removed to form the root.
Examples are given in \trf{tab:ReaLoaEnc}.
While the determiner \ve{=aa} is not particularly
frequent in Kotos Amarasi, in other varieties of Meto
it is the most common nominal determiner.
