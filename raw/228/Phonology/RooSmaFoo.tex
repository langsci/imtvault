\subsection{Monosyllabic roots}\label{sec:SinSylRoo}
The root shapes discussed in the previous sections
constitute the vast bulk of all roots in my database
and the only shapes with which all but one lexical root occur.
There are, however, a number of functors which
contain only a single syllable.
Nearly half (28/64) of all functors in my database are likely monosyllabic.
Examples include the relativiser \ve{reʔ},
the clitic negators \ve{ka=} and \ve{=fa}
as well as conjunctions such as \ve{mes} `but' or \ve{=ma} `and'.

There is one lexical root in my database
which is monosyllabic: \ve{{\rt}ha} `eat soft food'.
However, this verb always takes vocalic subject agreement prefixes (\srf{sec:VerAgrPre}),
such as \ve{na-ha} `3-eat', providing it with an extra
syllable and thus constituting an entire foot.

In my PhD thesis \citep{ed16b}
some functors were analysed as monosyllabic
which I now analyse as containing a double vowel.
Notably, I analysed many of the vowel-initial enclitics
(see \trf{tab2:AmaVowIniEnc} on page \pageref{tab2:AmaVowIniEnc})
as containing a single vowel, such as \ve{=ees} `one', and \ve{=ii} `{\ii}'.
Similarly, I analysed the pronouns \ve{hoo} `{\hoo}', \ve{iin} `{\iin}',
\ve{hiit} `{\hiit}', \ve{hii} `{\hii}', and \ve{siin} `{\siin}' as containing only a single vowel.
I have since gathered comparative data from other varieties
of Meto which indicates that such functors have two underlying vowels.\footnote{
		My analysis of the pronouns as containing a single vowel
		was influenced by the practical orthography in which they are written with a single vowel:
		\ve{hoo} \it{<ho>} `{\hoo}', \ve{iin} \it{<in>} `{\iin}', \ve{hiit} \it{<hit>} `{\hiit}',
		\ve{hii} \it{<hi>} `{\hii}', and \ve{siin} \it{<sin>} `{\siin}'.
		However, for many of the vowel-initial enclitics, my analysis was despite
		the orthography in which most of these enclitics are written with two vowels.
		Examples include \ve{=ii} \it{<ii>} `{\ii}', \ve{=aan}
		\it{<aan>} `{\aan}', and \ve{=ee} \it{<ee>} `{\ee}'}

Thus, evidence that the vowel-initial enclitics have two vowels
comes from a process of vowel assimilation in some varieties of Meto,
such as Ro{\Q}is Amarasi (Buraen) \ve{nifu} `thousand' + \ve{=ees} `one' {\ra} \ve{niifb=\tbr{o}es}
and \ve{metoʔ} `dry, dryness' + \ve{=ii} `{\ii}' {\ra} \ve{meetʔ=\tbr{u}i}
(see \srf{sec:HisDev} for more discussion of these and similar forms)
Similarly, for the pronouns Kusa-Manea has \ve{ian} `{\iin}', \ve{hiat} `{\hiit}',
\ve{hei} `{\hii}', and \ve{sian} `{\siin}' which unambiguously have two vowels.
It possible that more comparative data will lead to
reanalysis of some of the remaining putative monosyllabic functors
as containing an underlying double vowel.