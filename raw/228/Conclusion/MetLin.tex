\section{Metathesis in linguistics}\label{sec:MetLin}
Previous works on metathesis have mainly viewed the phenomenon through a phonological lens.
This has led to much useful development of phonological
models and analyses which can account for metathesis.
In this work I have gone beyond a phonological account,
and have provided perhaps the first detailed study of both the form
\emph{and} function of metathesis in a single language, Amarasi.

Like some other languages with synchronic metathesis,
such as Rotuman, Leti, and Helong (see \srf{sec:MorMet}),
metathesis in Amarasi is not a unitary phenomenon.
Instead, we can identify three kinds of metathesis:
one kind of phonologically conditioned metathesis and two
kinds of morphological metathesis, one with a syntactic
function and one with a discourse function.

In the greater Timor region (and more widely) metathesis is
often used as a construct form marking attributive modification.
Other languages in which metathesis is a construct form
include Rotuman, Leti, Roma, Mambae, and Helong  (\srf{sec:MorMet}).
The large number of languages which have the same function for metathesis
is probably connected with the development of consonant-vowel metathesis.

Metathesis with a discourse function has been
reported for Luang, in which metathesis of verbs
occurs at certain points in the narrative structure, namely:
``Known information and mainline event information,
especially at peak points of the story.'' \citep[24]{tata15}.
However, under the analysis of \cite{tata15}, metathesis
in Luang is a phonological process which occurs
to join words into a single rhythm unit;
a phrase with only one stressed syllable.

Metathesis in Kwara'ae, in which both forms
are used in different speech registers,
could be construed as a kind of discourse metathesis.
Again, however, metathesis in Kwara'ae can be successfully analysed
as a result of the stress rules of the language \citep{he04}.

\largerpage
Neither syntactic nor discourse metathesis
in Amarasi can be reduced to being the accidental side affect of some other
phonological process (see \srf{sec:AltAppPhoMet}).
Instead, the two generalisations which allow
us to account for all the different phonological processes
found in the formation of M\=/forms
are an obligatory CVCVC foot (with empty C-slots)
and a rule of consonant-vowel metathesis; CV {\ra} VC /\'V{\gap}.

\cite{blga98} showed how a process of synchronic metathesis
can develop through a number of phonetically natural steps (see \srf{sec:OriMorMet})
and in \srf{sec:OriMetAma} I showed that there is evidence
that Amarasi developed its metathesis according to the pathway labelled \emph{compensatory metathesis} by \cite{blga98};
this is a kind of metathesis which originally arose in certain prosodically conditioned environments.

Metathesis in Amarasi is no longer restricted these environments.
Unlike Lunag or Kwara'ae, metathesis in Amarasi has
escaped from any original phonological constraints
and now runs throughout the whole language.
From the occurrence of isolated words
where nouns are cited in the U\=/form and verbs in the M\=/form,
right up to complex clause chaining phenomena such as tail-head linkage,
the single phenomenon the analyst encounters time and time again is metathesis.

Nearly all areas of Amarasi grammar interact with metathesis.
In Chapter \ref{ch:StrMetAma} I posited that the creation of a morphological metathesis paradigm
was part of the motivation for the imposition of the CVCVC foot to all words of the language
in order to provide the necessary machinery for consonant-vowel metathesis to operate,
and thereby allow each word to fill both cells of the morphological paradigm.
In this way metathesis has taken over the phonotactics of Amarasi
and become the central organising principle by which words are structured.

Phonotactics is not the only linguistic sub-domain to have been invaded by metathesis.
In Chapter \ref{ch:PhoMet} I showed that metathesis
interacts with prosodic structures by marking a clear
phonological boundary between two prosodic words.

In Chapter \ref{ch:SynMet} I showed that
it is metathesis which marks the structure of the noun and verb phrase
by marking the presence of an attributive modifier
which belongs to the same word class as the head of a phrase.

In Chapter \ref{ch:DisMet} I showed that in narratives
it is metathesis which advances the plot through tail-head linkage
and dependent coordination.
In multi-speaker discourse,
metathesis is the social glue which binds the conversation together.
Unmetathesised and metathesised forms are employed as question-answer pairs,
and signal the end of a conversational turn,
thereby carrying forward interactions between speakers.

Metathesis in Amarasi is not merely
an epiphenomenon or exotic curiosity.
Rather, it is a central feature around which other linguistic structures are organised.
In addition to being the key which unlocks the structure and genius
of the language, Amarasi metathesis is also the linguistic instantiation of two pervasive
ethnographic traits of the Amarasi people: identity and parallelism.