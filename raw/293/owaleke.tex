\documentclass[output=paper]{langscibook}
\ChapterDOI{10.5281/zenodo.5578842}

\title[A diachronic analysis of the socio-semantic features of Igbo names]
      {A diachronic analysis of the socio-semantic features of Igbo personal names}
\author{Esther Nwakaego Oweleke\affiliation{Department of Linguistics and Communication Studies, University of Port Harcourt}}
\abstract{This paper examines Igbo personal names from the perspectives of anthropological linguistics, socio-semantics, pragmatics and diachrony. It traces the evolution of name-giving within three major eras, pre-Christian (before the 1850s), early-Christian (from 1857--1960s) and Pentecostal (1960s to date) among the Igbo. The data for the study are 300 personal names sourced through interviews with both old and young members of the Igbuzo-Igbo community as well as from class registers of some nursery, primary and secondary schools in Port Harcourt. This paper reveals three interesting and important findings. First, indigenous Igbo names are culturally, semantically and pragmatically significant. Second, the principles and practice of name-giving have undergone some drastic changes through the eras. During the early-Christian and Pentecostal eras, the forces of Christianity, modernization and globalization have resulted in new names that reflect these changes. Third, indigenous names which belong to the core vocabulary of the language are seriously endangered. The paper concludes that if the indigenous names that are replete with meaning, and portray the values, identity and the beauty of the Igbo culture are allowed to die through lack of intergenerational transmission, then this aspect of the language and culture will also die.}


\begin{document}
\SetupAffiliations{mark style=none}
\maketitle


\section{Introduction}

In Igbo society, personal names (anthroponyms) are not mere resources for identifying human beings. They encode information about the socio-cultural values, beliefs and world views of the people. The events surrounding the birth of a child, the physical appearance of the child, the state of mind and experiences of the parents, time and place of birth, as well as the shared values and beliefs of the society are all encapsulated in the names given to a child. Thus, the name a child bears says a lot about him and his parents. The name \textit{Àgịlị̀ga }, for example, tells us that the child is the second of a twin birth and \textit{Àto}̣ the third of a triplet. \textit{Àdịmàbụ̀a} reveals that the father of the child is an only male child of his own parents. \textit{Nwaèlège} is given to a slim and slender female child and \textit{Àlị̀}àm to a fair complexioned child. If a child is given the name \textit{Nọ̀du}̣, it signifies that the parents have had other children before him who died before their naming ceremonies; that is, less than eight days after birth. \textit{Òkelekē} is reserved for a male child born on the Èke market day. 

Name giving is an aspect of the language which has attracted the attention of anthropologists, linguists and lately language documentation specialists. The practice and principles of naming have over the years provided a great resource for investigation in many African languages. Some Igbo writers like \citet{Ezeanya1967,Ubahakwe1981,Onukawa1999,Onukawa2000,Iwundu1994,Ezeude2000,Nwigwe2001,OkaforEtAl2008} and \citet{Onumajuru2013} among others have described and documented Igbo anthroponyms. They believe that Igbo names reflect the value systems of the people. \citet{Madu1996} notes that it is possible to understand, to a large extent the social, religious, historical and political issues in Igbo society through Igbo names. Consequently, Igbo names cannot be fully appreciated without adequate knowledge of the Igbo world view. In consonance with this view, Achebe (1975 as cited in \citealt{Onukawa2000}) explains that, to know how life has treated an Igbo man is to look at the names his children bear, because they reflect his hopes, fears, joys, sorrows, and even his grievances against others. Similarly, \citet{Ekwunife1989} observes that African names express African cosmology, social, economic, political, and religious values of the society, the tensions and struggles of man in society. From another perspective, \citet{Onyema2009} sees Igbo names as being responsible for forming conscience and consequently determining the character of the bearer. Personal names include the first or given name (the name given to a child at birth by his/her parents), middle name, and family name or surname (the name normally inherited and shared with other members of the family).

The process of name giving in Igbo has witnessed a dramatic change as many indigenous  names (especially given names) which reflect the core values, cultural identity and beauty of the Igbo language have been abandoned for the so called modern, ``sophisticated'' and ``stylish'', Christian, English, or French names, as well as  names that reflect the Pentecostal movement and ideology. Before the advent of colonialists and missionaries to the shores of Africa, all Igbo names (both first and family names) were indigenous to the culture of the people. 

The motivation for this study comes from a sarcastic expression used by some Enuani-Igbo speakers around the 1970’s: \textit{Nwaùkpò abụ̄họ nwatà} ‘Nwaukpo is not a child’. More specifically, it means, ‘a woman named \textit{Nwaukpo} is not a young woman’. Just before the Nigerian Civil War (1967--1970), most women who bore this name were already in their late thirties. The name was no longer found among children or young girls at that time. Thus, it readily revealed the age of the bearer. A woman in her thirties and unmarried in the African setting was stigmatized. The element, \textit{ùkpò} ‘throne’ or a ‘raised platform’ usually inside the house is a metonymy for kingship.  A contemporary replacement for the name is \textit{Adaeze} ‘Princess’. As significant as the name is, it is no longer borne by young girls. This is just one of the numerous Igbuzo indigenous names that are rarely given today as either a first or second name. 


This study traces the names and name-giving trends within the three periods specified. It describes the socio-semantic and pragmatic features of Igbo traditional names; highlights the social and semantic changes that have taken place within the period, and examines the factors responsible for the drastic change and the reasons for non-generational transmission of both given and family names. Earlier works in Igbo anthroponomastics (the study of personal names) have largely adopted a synchronic approach, concentrating on the description of the semantic, syntactic, morphological analyses, and socio-cultural dimensions of Igbo personal names. This study however, takes a diachronic approach. It focuses on the content of names and its influence on social-cultural, pragmatic and semantic shifts, and so, investigates the implications of the trend to language endangerment. Therefore, it is relevant to Igbuzo-Igbo language documentation and contributes to general Igbo studies. 

\subsection{The Igbo language and its speakers}

Igbo is the native language of the Igbo people, an ethnic group of South-Eastern Nigeria. It is spoken as a major and regional language in the states of Abia, Anambra, Ebonyi, Enugu and Imo. Igbo is a minor language in Rivers and Delta states, and it is spoken as a second language by many non-Igbo speaking people within and outside Nigeria. It is one of Nigeria’s three major languages; the others are Hausa and Yoruba. Igbo is spoken by 20--25 million people in Nigeria. It has numerous and divergent dialects that share a high degree of mutual intelligibility. Some of dialects include Owere, Onitsha, Orlu, Afikpo, Ngwa, Oguta, Ika, Ogwashi-Ukwu, and Igbuzo. \citet{Manfredi1989} identified seven dialect zones namely: Delta, Enyom Central, Plains, Oshimili, Omanbala, and Savana. The Igbuzo variety of the Enuani cluster has been chosen for the study. Igbo, like many Nigerian languages, is a tone language. It has two basic tones: high and low, as well as a downstep tone. The standard also referred to as Onwu Orthography of Igbo is used in this study for representation. The orthography consists of thirty six letters of the alphabet, made up of 28 consonants and 8 oral vowels, given in this normal order:

\ea
\begin{xlist}
\ex Orthographic symbols:\\
{$\langle$}a b ch d e (ẹ) f g gb gh gw h i ị j k kp kw l m n n̊ ny nw o ọ p r s sh t u ụ  v w y z{$\rangle$}
\ex Phonetic representation:\\\relax
[a b tʃ d e (ɛ) f g ɓ ɣ gʷ h i ɪ dʒ k ƥ kʷ  l m n ŋ ɲ ŋʷ o ɔ p r s ∫ t u ʊ v w j z]
\end{xlist}
\z


\noindent This is necessary to help non-native speakers with a linguistic background comprehend how to phonetically interpret the names. In this paper, high tones are left unmarked, low tones are marked with a grave accent [\,\`{}\,] and downstepped tones with
a macron [\,{\={}}\,] in line with the standard convention of Igbo orthography. Igbuzo dialect like many other Igbo varieties makes use of the ninth oral vowel ẹ/ɛ used in this paper.

\subsection{Methodology}\largerpage
For this study, 300 personal names of both old and young members of the Igbuzo-Igbo community were collected through group interviews with Igbuzo men and women, living in, and outside Igbuzo town. Lists of registered students of the Department of Linguistics and Communication Studies, University of Port Harcourt as well as those of some primary and nursery schools in Igbuzo and Port Harcourt were examined. Clarification on the meanings and contexts of some specific names especially the philosophical and proverbial types were further elicited from personal interviews with bearers and some older members of the community versed in the traditional meaning of the names. Data collection was supplemented by investigator-based introspection of a competent native speaker. The three major eras identified for the study are
Pre-Christian (before the 1850s),
Early-Christian (from 1857--1960s) and
Pentecostal (1960s to date). The Pre-Chris\-tian era is the period before the introduction of the white man’s God in the religious concepts of the Igbo. The study shows how these three different periods have influenced name giving in Igbo. Three sets of data were used to reflect the pattern of the changes. Both literal and non-literal interpretations of the names are presented.



\section{Some features of Igbo culture}
The culture of a people is the totality of their ways of life. It includes their beliefs, ideas, language, religion, practices, attitudes, social behaviours and even the social/religious objects shared by them. Other aspects of culture include music, dance, clothes and food. Nida (\citeyear{Nida1975} as cited in \citealt{Surukat2009}) classifies culture into ecology, material, religious, social and linguistic culture. Language is an aspect of a people’s culture. The words, idioms and proverbs found in a language are used for expressing the beliefs and practices of the people. Language is a custodian of the cultural realities of a people and a medium through which culture is generationally transmitted. This explains why indigenous Igbo names reflect the culture of the people.

Culture is dynamic and so is language; it is not static. As a society changes, so does its language. A language is complete for its indigenous culture, but as speakers make contact with new concepts, civilizations and languages, new terms emerge in the language to cope with new realities. No matter how homogenous a community/society is, culture changes since societies do not exist in isolation. The Igbo contact with Western culture is largely responsible for its change in cultural values.

\subsection{Religion and the Igbo culture}
A study of the religious culture reveals that Igbo traditional religion is polytheism. The people believe in and worship the supreme God, \textit{Chukwu }or \textit{Chi Ukwu} also known as \textit{Chineke} (the Creator of all things), \textit{Chukwu Òbièmè} (the Great achiever), \textit{Òsàebùlụwà} (the Owner and Sustainer of the universe) as well as other deities, and spiritual forces. They worship deities such as \textit{Àlà} -- the Earth goddess, river gods/goddesses, \textit{Anyanwụ̄} -- the Sun god, \textit{Àmadị Ọ̀hà} -- Thunder,  \textit{Igwe} -- the sky god, \textit{Àshì} and \textit{Chi} -- personal gods and a host of others. \textit{Chineke} or \textit{Chukwu} the Supreme God was considered to be too high up and too far from the people to worship directly, so they needed intermediary beings to reach him. This situation explained the strong affinity the Igbo had to certain gods or deities that influenced the names they gave their children in the traditional setting. The belief in ancestors also pervaded the ethos of the period. According to \citet{Aligwekwe1991}, ancestors were the deceased members of the community who from the land of the dead could influence their family members.

Igbo generally believe that each person has a personal god known as \textit{chi} which controls his or her destiny and is responsible for what makes the difference between individuals. \citet{Ezeanya1967} explains that \textit{chi} ``is a word used commonly to denote the good spirit which according to the belief of the Igbo people is assigned to everyone from conception." According to \citet{Onukawa1999} most Igbo scholars believe that the concept of \textit{chi} is central to the thought and belief of the Igbo. The concept is enigmatic and has been interpreted by different Igbo scholars as an individual’s fate, luck, guardian angel/spirit, personal life force or ‘individualized providence’ or even a man’s destiny god. In Igbo cosmology, \textit{chi} is believed to be responsible for an individual’s success or failure, abilities or weaknesses, fortune or misfortune and even his wealth or health. Traditionally, the concept of \textit{chi} is different from \textit{Chi-Ukwu} or \textit{Chukwu}, the Supreme Being. With the advent of Christianity, however, \textit{Chi} came to stand for the Supreme God -- \textit{Chi-Ukwu} or \textit{Chukwu} and consequently, Igbo Christians now use \textit{Chi} as the short form for \textit{Chukwu}.

The belief in reincarnation and \textit{ọ̀gbanje} is another aspect of the religious culture of the Igbo. It is the belief that a dead person has the ability to come back to life, or be born again into his family. They believe that their ancestors can come back to the world in the form of a new baby. An \textit{ọ̀gbanje} child is believed to have cycles of life and death; the same child can die and be born several times to the same mother. The views of reincarnation and \textit{ọ̀gbanje} are in opposition to Christianity.

Other cultural activities apart from religion include marriage ceremonies, title taking, initiations, burial ceremonies, childbirth, naming celebrations and other political activities. Child-naming is an important activity in Igbo land that it is done amidst celebration and great joy. In Igbuzo different sections of the community also engage in celebrating different festivals such as \textit{Ịgbā Ine}, \textit{Ịchụ̄ Ụlọ̀}, \textit{Ịgbā Ekwensu}, \textit{Ịgbā Iwū}, and \textit{Ìfèjiọkụ̄}, (thanksgiving ceremonies to mark the end of a planting season while awaiting harvest), \textit{Ịwāji} ‘New Yam festival’, \textit{Ịgbā mgba} ‘wrestling contest’, and  \textit{Òtùtù} ‘annual fishing contest’, among others. All these activities which define the Igbo are reflected in the names they bear. 

\section{Socio-semantic features of name-giving in the Pre-Christian era} 
This section focuses on the analysis of the religious, philosophical, metaphorical and contextual meanings of Igbo names within this period, before 1857. This analysis is adopted because it helps to establish the basic forms of the names before the changes. 

\subsection{Traditional names with religious meaning}\label{sec:owaleke:3.1}
Names in this era reflect the religious facts about the Igbo. The names are couch\-ed in concepts like  \textit{mmụ̄ọ} ‘spirit being’,  \textit{chi} ‘personal god/fate’,  \textit{àshì} ‘destiny’,  \textit{ànị̀} earth deity’, and  \textit{ǹzè} ‘deity/royalty’. A close examination of the names reveals the importance Igbo people attach to religion. It shows their relationship and dependence on spiritual begins (the Supreme Being, other deities and ancestral spirits). Before the advent of Christian missionaries in Nigeria, the Igbo did not believe in the Christian God even though they believed in the Supreme Being --  \textit{Chukwu}, \textit{Chineke}, \textit{Chukwu Obieme}, \textit{Òlìsà}, \textit{Olisaebuluwa}, etc. Indigenous names which reflect this belief abound in this era. A few examples from Igbuzo-Igbo will suffice. The phonetic transcriptions of the names have been included to aid pronunciation.

\vfill
\begin{table}[H]
\caption{\label{tab:owaleke:3.1} \textit{Chi} names}
\begin{tabularx}{\textwidth}{llQ}
\lsptoprule
Names          & Pronunciation      & Literal meaning                    \\
\midrule
Chijìndụ̀    & {[}tʃidʒìndʊ̀{]}  & ‘God is the owner/giver of life’   \\
Chiegbūnẹm   & {[}tʃieɓunɛm{]}    & ‘God should not kill me’           \\
Chinèdu       & {[}tʃinèdu{]}     & ‘God directs/leads’                \\
Chibụ̀ọgwụ̀ & {[}tʃibʊ̀ɔgʷʊ̀{]}  & ‘God the healer’                   \\
Anaènechī    & {[}anaènetʃi{]} & ‘Let’s look up to God (for mercy)’ \\
\lspbottomrule
\end{tabularx}
\end{table}
\vfill
\begin{table}[H]
\caption{\label{tab:owaleke:3.2} \textit{Mmụọ̄} names}
\begin{tabularx}{\textwidth}{llQ}  
\lsptoprule
Names          & Pronunciation      & Literal meaning            \\
\midrule
Mụọ̄dèbè   & {[}mʊɔ̄dèbè{]}  & ‘God/gods/ the spirit preserves’          \\
Mụọ̄sị̀mdị & {[}mʊɔ̄s̀md{]}  & ‘God/gods/the spirit asked me to be/stay’ \\
Mụọ̄gàha    & {[}mʊɔ̄gàha{]}   & ‘God/the spirit is my arbiter’            \\
Mụọ̄lòokwu  & {[}mʊɔ̄lòokwu{]} & ‘God/god/spirit has the final say’        \\
Ẹnẹmụ̄o     & {[}ɛnɛmʊɔ̄{]}     & ‘Let’s look up to God (god)’              \\
\lspbottomrule
\end{tabularx}
\end{table}
\vfill\pagebreak


\textit{Mmụọ̄} names could refer to the Supreme Being or any of the lesser gods or ancestors. The term is generic and also refers to ancestral spirits. \textit{Chi} and \textit{àshì} names are closely related. They can be interpreted as an individual’s personal god or guardian angel or force which determines the person’s fate. Besides, the information elicited from Igbuzo informants reveals that the concept of \textit{chi} is not synonymous with the Supreme Being in the traditional setting. 

\begin{table}[p]
\caption{\label{tab:owaleke:3.3} \textit{Àshì} names}
\begin{tabularx}{\textwidth}{llQ}
\lsptoprule
Names            & Pronunciation         & Literal meaning  \\
\midrule
Àshìkọdị̀     & {[}à∫ìkɔdɪ̀{]}      & ‘It is one’s destiny’         \\
Ikedị̀àshì & {[}ikedɪ̀à∫ì{]}     & ‘My destiny determines my achievement’     \\
Dìbeàshì       & {[}dìbeà∫i{]}       & ‘Accept your destiny’                      \\
Àshionyeèduẹ̄ & {[}à∫ìonyeèduɛ̄{]} & ‘One’s god determines his fate’            \\
Àshìèdo       & {[}à∫ìèdo{]}      & ‘One’s god designs or determines his fate’ \\
\lspbottomrule
\end{tabularx}
\end{table}

\begin{table}[p]
\caption{\label{tab:owaleke:3.4} \textit{Ànị̀ } names}
\fittable{\begin{tabular}{lll}
\lsptoprule
Names             & Pronunciation         & Literal meaning  \\
\midrule
Ànị̀kàmgbòlù & {[}ànɪ̀kàmɓòlù{]} & ‘I take refuge in the earth deity’            \\
Ànìèmekẹ       & {[}ànɪ̀èmekɛ{]}     & ‘The earth deity has done well’               \\
Ànị̀àgọlụm̄  & {[}ànɪ̀àgɔlʊm̄{]}   & ‘The earth deity speaks for me/is my arbiter’ \\
Kànịkwù        & {[}kànɪkwù{]}       & ‘Ani (the earth deity) has the final say’     \\
\lspbottomrule
\end{tabular}}
\end{table}

\begin{table}[p]
\caption{\label{tab:owaleke:3.5}\textit{Òlìsà/Òsà/Òsẹ̀ } names}
\begin{tabularx}{\textwidth}{llQ}
\lsptoprule
Names             & Pronunciation         & Literal meaning\\
\midrule
Òsàelòkẹ    &     [òsàelòkɛ]    &     ‘God has been very thoughtful’\\
Òlìsàdebe   &        [olisadebe]       &       ‘God preserves’\\
Òlìsàèmekẹ &     [òlìsàèmekɛ]   &  ‘God has done well’\\
Òlìsèàlụka   &    [òlìsɛ̀àlʊka]   &    ‘God has achieved much for me’\\
Ọ̀dị̀akaòsẹ̀ &      [ɔ̀dɪ̀akaòsɛ̀]  &       ‘It is in God’s hand. (God is my help) \\
\lspbottomrule
\end{tabularx}
\end{table}


\begin{table}[p]
\caption{\label{tab:owaleke:3.6}\textit{Ǹzè} names}
\fittable{\begin{tabular}{lll}
\lsptoprule
Names             & Pronunciation         & Literal meaning\\
\midrule
Ǹzèkwe       & {[}nzèkwe{]}     & ‘If Nze/deity agrees…’                             \\
Ǹzèegbùnẹ̀ & {[}nzèeɓùnɛ̀{]} & ‘Nze (Nze) should not harm, kill or cause sorrow)’ \\
Ǹzèkwuẹ     & {[}nzèkwuɛ{]}    & ‘Nze has spoken’                                   \\
Ǹzèèmekẹ   & {[}nzèèmekɛ{]}  & ‘Nze (deity or royalty) has done well’             \\
Nwanzè        & {[}ŋʷanzè{]}     & ‘child of a noble one’ (‘prince’)’  \\
\lspbottomrule
\end{tabular}}
\end{table}

Names with \textit{àshì} are prevalent in the Enuani-Igbo cosmology and sometimes can be used interchangeably with \textit{chi} names in Standard Igbo. \textit{Ànị̀} the earth deity is also highly revered in the Igbo society and given the status of a god. It is regarded as the being in charge of morality, agriculture, reproduction and fertility. The concept of \textit{Ǹzè} has dual interpretation; first as a kind of deity and secondly as referring to royalty. \textit{Òlìsà} is a traditional name for the supreme God, and has the short forms as \textit{Òsà} or \textit{Òsẹ̀ }. Examples of Ashi, Ani, Nze and Ose names are given in Tables \ref{tab:owaleke:3.3}–\ref{tab:owaleke:3.6}.


Names that express the Igbo belief in religious objects of worship, adoration and practice also abound in this era. Such objects associated with deities include: \\\textit{Òkwutẹ̄} ‘stone', \textit{mkpụlụ iyī} ‘pebbles’,  \textit{nkumẹ̀} ‘rock’, \textit{Anyanwu} ‘sun deity’, \textit{Ọ̀fọ̣} ‘symbol of innocence', \textit{ugwu} ‘hill’,  \textit{Àkpaàlà} ‘thunder deity’, \textit{Iyiocha} ‘clear/clean running river’ and some animals:  \textit{Agụ}̄ ‘lion’, \textit{Agụ Iyī} ‘crocodile', and \textit{eke ukwu} ‘python’. These are totems believed to have spiritual significance.\largerpage

Another significant aspect of naming in the traditional religious society which is no longer in vogue is that which revolves around the concepts of reincarnation, \textit{ọ̀gbanje } and ancestors.  These names were unique, interesting and very popular in the pre-Christian era. Some of the names in \tabref{tab:owaleke:3.7} are given to pacify or appease the children. However, some parents in indignation of the repeated deaths of the babies give them unpleasant and repulsive names that stigmatize and probably forbid them from going back to the spirit world. Such names include: \textit{Ùlìòmàgwụ} ‘day-old chicks for sacrifice are finished’, \textit{Ikpontụ̄} ‘refuse heap’ or \textit{Afàadị̀àzị̀ } ‘no more names to give’ or \textit{Ùdenè }‘vulture’.

\begin{table}
\caption{\label{tab:owaleke:3.7} Names portraying reincarnation and Ọ̀gbanje concepts }
\begin{tabularx}{\textwidth}{llQ}
\lsptoprule
Names                & Pronunciation                & Literal meaning\\
\midrule
Nnam̀dị       &    [nnàmdɪ̀]   &   ‘My father is alive’/ my father has come back’\\
Ọnọ̀chiẹ      &    [ɔnɔ̀tʃiɛ]    &  ‘One has come to replace another’\\
Àbịagọm̄     &   {[}àbiagɔm̄{]}  &  ‘I have come/returned’\\
Ọnwọ̀dị      &    [ɔŋʷɔ̀dɪ]	& ‘Let this one stay/live’\\
Ǹkẹàdị       &    [nkɛàdɪ]      &  ‘This one must be (live)’\\
Nwanọ̀luẹ    &  [ŋʷanɔ̀luɛ]  & ‘May this child live to a ripe old age’\\
\lspbottomrule
\end{tabularx}
\end{table}

Generally Igbo names are predominantly sentential in structure. They are nominals usually derived from phrases, especially noun and verb phrases and sentences (simple, complex and compound), and may be declarative, imperative or interrogative. They may also be in the affirmative or negative form.  There are rarely single-morpheme names. Most personal names are expressed in full sentences that are often times clipped, leftward or rightward, in a way that makes deciphering of meaning unclear to the hearer. Some names may be clipped from the right or left only or from both as shown in the examples (\ref{ex:owaleke:1}a--e) below. Clipping may involve segments, syllables, words or phrases. So, to decode the complete meanings of the names the sentences must be properly analysed to bring out the underlying structure and meaning. In the sentences, the structures enclosed in brackets are the ones clipped.

\ea\label{ex:owaleke:1}
\ea
Izuka:  Izu ka (na ofu nne) \\
\glt `Consultation/Solidarity is best achieved among maternal siblings.'\\
\ex
Ụgbaàjà: (Ụ̀kọ nwa bùtẹ̀)  ụgba àjà\\
\glt `The absence of a child is responsible for clay pots of ritual items.'\\
\ex
  Adiigwe: À(nyị) dị (na) ìgwè, ànyị̀ èmeliẹ dike\\
\glt `In great number, we defeat the valiant warrior.'\\
\ex
  Elofu:   (Chukwu adị̄) èlofù (onye o kè)   \\
\glt `God does not forsake the one he created.'\\
\ex
  Isitọa:   Isi tọ̀a (ò liẹ ezè) \\
\glt `If the head (a man) grows old, he will become the chieftain of his clan.'\\
\z
\z

\subsection{Philosophical names in the Pre-Christian era}
Igbo names given to children in this era were highly philosophical; they reflected the thoughts of the Igbo man about his worldviews. Thus, to decode the meaning of these names is to understand the traditional Igbo man’s philosophy of life. \tabref{tab:owaleke:3.8} shows a few examples.


\begin{sidewaystable}
\caption{\label{tab:owaleke:3.8} Names depicting philosophical meanings}
\begin{tabularx}{\textwidth}{llQQ}
\lsptoprule
{S/N} & {Igbuzo-Igbo Names}     & {Literal meaning}    & {Philosophical/Pragmatic Interpretations}\\\midrule
{1}   & {Ònyekàozùlù}      & {Is anyone all sufficient?}  & {Name given in recognition of God’s sovereignty and prerogative of distribution of resources/destiny.}\\
{2}   & {Ònyepụ̀nụkà}       & {Who can escape criticism?}   & {No man is infallible. We all have our weaknesses.}\\
{3}   & {Ònyejèbeòlìsẹ̀}   & {Who has ever visited God?}  & {Diverse talents are given by God. You do not determine your fate.}\\
{4}   & {Àkpàdòmònye}       & {Am I inconveniencing anyone?}   & {Response by a good and prosperous man who suffers extreme resentment from his relatives/neighbours.}\\
{5}   & {Kahàmụ̄amụ}         & {All that laugh/rejoice with you are not your friends.} & {Betrayal can come from one’s household or a trusted friend.}\\
{6}   & {Ọchụkènyì}         & {The one that pursues an elephant}  & {The one who pursues a great vision about life (given to one envisioned to have great fortune in life.}\\
{7}   & {Ụzụakpụnwa}         & {A blacksmith cannot make life.}   & {Man does not create life. To an Igbo, life is sacred and should not be destroyed.}\\
{8}   & {Ònyekàọnwụfụ̀jụ} & {Does death reject anyone?}    & {The name describes the inevitability of death.}\\
{9}   & {Ọnwụyàlị}         & {Death, please, let go…}   & {Given in a family where many had died before the child’s birth.}\\
{10}  & {Ọnwuạ̀chị}      & {Death rules as master.}    & {The Igbo recognizes the supremacy of death.}\\
\lspbottomrule
\end{tabularx}
\end{sidewaystable}


As evident from the table, philosophical names are usually couched in rhetorical questions and figurative expressions. Metaphors, similes, metonymy, personification, and synecdoche are special strategies frequently employed. The names express Igbo maxims, aphorisms and are indeed proverbial. Like proverbial expressions, they are amenable to three levels of interpretation -- literal, contextual and philosophical. Given their internal lexical structure, they are culture specific and so only those familiar with the culture are able to understand the semantic nuances, hidden associations and allusions in the names. Thus, deciphering such names require the contextual implication surrounding the birth of the child. This is also responsible for the many interpretations these names could be subjected to.  

Some Igbo names like the philosophical names are derived from proverbs that have undergone clipping. The clipped, short or surface forms are left when the other parts have been deleted. The proverbs, \textit{Nwa fẹ̀ẹ nnà a, ò bulu isi nkā} ‘If a child honours his father, he will grow to old age’ and \textit{Kà ànyị hà mụ amụ̄ abụhọ ọ̀sị̀} ‘All that laugh/rejoice with you are not your friends’ are shortened to \textit{Nwafẹe}̣ and\textit{Kaham }as names. The bearers and members of the community know the full forms and meanings of the proverbs/names. Other examples of truncated proverbs have been given in Examples (\ref{ex:owaleke:1}a--e) in \sectref{sec:owaleke:3.1}. Similarly, \citet{Musere1999} has a good account of this naming strategy among the Ganda people of Uganda. However, in Igbuzo, not all names are derived from proverbs or express the philosophical thoughts of the people.

The Igbo perceive \textit{ọnwụ} as a mysterious and enigmatic phenomenon in the reality of existence.  To express this view, most \textit{ọnwu̩} names as above employ personification, that is, in the names, \textit{ọnwu}̣ ‘death’ is personally addressed. Quite often, the names remind the people of the inevitability, universality and humbling nature of death. Alternatively, \textit{ọnwụ} is seen as merciless and vicious. Philosophical names in general find expressions in rhetorical questions because their answers are obvious and usually the only one possible. For exampleː Does death reject anybody?  Is anyone all sufficient? The obvious answer is ``No''. 


\subsection{Contextual relevance of traditional Igbo personal names}
Names that carry a high degree of contextual relevance characterized the pre-Christian era. In short, one can convincingly say that every Igbo traditional name had a covert or an overt contextual implication which told the story of the child, his family or his entire village. Context as a name-giving strategy refers to the physical features of the child, the place and time of birth, the sex of the child and the entire event surrounding the birth of the child. The four market days of \textit{Èke, Olìe, Àfọ̀ } and \textit{Ǹkwọ}, frequently formed very useful resources for naming both male and female children in Igbo land. The linguistic element \textit{òko, òkolo} or \textit{oke }in a name suggests that the bearer is a male and \textit{m̀gbo}̩ or \textit{àda} is used for females. Below are some names reflecting the four market daysː


\ea

Female\\
M̀gbọeke   `born on Èke day'\\
M̀gbọnkwọ̀    ‘born on Ǹkwọ day'\\
M̀gbọàfọ̀    ‘born on Àfọ̀ day’    \\      
M̀gbọolìe    ‘born on Olìè day’\\
Àdaòlìe        ‘born on Olìè day’\\   

\medskip

Male\\
Òkelekē  ‘born on Èke day’\\
Òkonkwọ    ‘born on Ǹkwo̩ day’\\
Òkoàfọ̀         ‘born on Àfọ̀ day’ \\ 
Òkolìè         ‘born on Olìè day'\\
Nwaeke        ‘born on Èke day’\\
\z

In addition, some Igbo personal names reflect the place or time of birth of a child. A name like \textit{Nwaọ̀bọ̀shị̀} ‘child of Ọ̀bọ̀shị̀’ or \textit{Òkọ̀bọ̀shị} ‘male child of Ọ̀bọ̀shị̀’ a river deity in Igbuzo is usually given to a child in appreciation or dedication to \textit{Ọ̀bọ̀shị̀}  goddess by her devotees or to a child born by the \textit{Ọ̀bọ̀shị̀} river. \textit{Nwaozili}  or \textit{Ozili} is a name given to a child born along a tarred road. The names \textit{Òkugbō or Nwandiògò } refer to ‘a male child born in the farm', whereas \textit{Òkeinē or Òkeụlọ̀ } is a name for a male child. The names \textit{Àdaụlọ̀/Àdaine} are give to a female born on \textit{Ine} and \textit{Ụlọ̀}, festivals in Ìgbuzo. Where a man finds respite from problems in his maternal family rather than among his kith and kin, he names his child \textit{Ikwunnekàt} for ‘mother’s birth place is supreme/my respite’, or \textit{Ìzùkànàòfunnē} ‘there is greater solidarity in maternal affiliation’, or \textit{Ǹnebụ̀ọ̀gọ̀} ‘mother is kindness’. A child named \textit{Nwaụ̄kọ} reveals that the child is born during famine or some scarcity. \textit{Egobi}̄  ‘obi’s jewel’ is usually borne by the king’s first daughter. To express physical appearance, Igbo use names such as \textit{Òkọcha/Nwaokọcha-} ‘a fair complexioned male’, \textit{Òkojiī/Òkonji/Nwanji/Òkonwanji} for a dark complexioned male child, \textit{Nwalùpuẹ} for a plump and robust child, and \textit{Mgbuude}, \textit{Àlìàm}, or \textit{Nwaèlège} for a slim, elegant and adorable child. These names are given as tradition stipulates on the eight day after birth.

Other names express the Igbo belief and views about longevity, strength and security, wealth, sustenance, bravery, respect for elders, filial affection, barrenness, and inestimable value for children, the importance attached to the male child above the female in a family, as well as names that express the practice of polygamy. All these are beautifully couched in the names children bore in Igboland in the pre-Christian period especially before the Nigerian Civil War. Names like \textit{Ìlobàbaechìnẹ̀ (Ìlobàa)}, \textit{Òbòdòechìnẹ̀}, \textit{Ụzọ̀echìnè},  \textit{Òbiechìnè}, and \textit{Afàmefùnẹ} are given to male children to express their parents’ desire and prayer for continuity or preservation of the family name/lineage. Names as these were given to celebrate the goodness of the Supreme Being. A child is of inestimable value in Igbo and as such a woman without a child is ridiculed and considered as worthless or cursed. Thus, \textit{Nwalọ̀kọ̀mobì (Nwalọ̀kọ̀)} ‘child comforts me’, \textit{Nwakàegō} ‘child is greater than wealth’, \textit{Nwanyèmume} ‘child is my strength and security’,  \textit{Nwaamaka} ‘child is beautiful’,  \textit{Nwaejìje} ‘child is my jewel’, \textit{Giikànwa} ‘what can be more valuable than  a child’, \textit{Nwabụ̀ǹdò} ‘a child is your shield/cover’, and \textit{Nwabụwà} ‘child is the essence of life’, are a few significant names common at that time which express the importance of child bearing in Igbo. However, women who had infertility challenges responded to the derision and scorn they suffered in the names they gave their children when they finally got blessed. Names like \textit{Ajọ̄okwuàgwụ} ‘ridicle, finaly came to an end’, \textit{Òkèdị̀nàchi} ‘my lot is in God’, \textit{Ọnụwàegbunẹm} ‘the reproach of my adversaries will not kill me’ and \textit{Ùwàkwuẹliẹ} ‘the world/enemies should take back their criticism’ are a few examples. As evident from the analyses, names in this era carry along with them high semantic-pragmatic import necessary for effective communication among members of the community.

\section{Name-giving in the Early-Christian era}
With the introduction of Christianity came monotheism and the change in faith, philosophy and the general world view of the Igbo. Apparently, the new religion was in opposition to the Igbo practice of polytheism. This crossroads gave rise to a partial acceptance of the Christian God. Thus, the Igbo imbibed Christianity but did not completely abandon their own traditional ways. They still practiced and worshipped their deities and ancestors in addition to the Christian God. The belief in ancestors, re-incarnation, medicine men, wearing of charms and amulets persisted. All of these events reshaped the practice and principles of the new sets of names that characterized the period. The naming system in this era therefore has witnessed significant social and semantic changes. 

\subsection{Social changes in naming-practices in the Early Christian era}
A number of significant social changes have taken place. First, the addition of the Christian names to the traditional names was witnessed. Most people, who had two Igbo traditional names, at baptism, took on European and Christian names (names of saints for the Roman Catholics, or Bible figures for non-Catholics). Thus, Nwanze Òwèlèmekẹ and Mgbuude Ofili became William Nwanze Òwèlè\-mekẹ and Bridget Mgbuude Ofili. With time too, the Igbo middle names were dropped for the English Christian names, e.g. Bridget Ofili. In some cases, both given and surnames were out-rightly replaced with English names in a way that it becomes difficult to identify the place of birth of the individual. 

Second, their new belief in \textit{Chi-Ukwu}, or, \textit{Chukwu} (the Christian God) now finds expressions in the names given to the children. With Christianity, \textit{Chukwu}\slash\textit{Chi} names became very popular. The concept of \textit{Chukwu} now denotes the Christian God and faith. \textit{Chi} names no longer refer to the bearer’s personal god, some force or destiny, but to the God Almighty, the father of Jesus Christ. \textit{Chi} is now taken as the short form of \textit{Chukwu} and so \textit{Chi} and \textit{Chukwu} names are now used interchangeably. Impressionistically, \textit{Chi}/\textit{Chukwu} names account for about 60-65 percent of Igbo names, both given and family names. A name common in this era was: \textit{Chukwudùmèbi} which translates to the biblical Emmanuel --‘God is with us’, and borne by most first born sons in many families. Others include:

\ea
\textit{Chinedu} ‘God is my shepherd or God guides’ \\
\textit{Chukwumụanya} ‘God does not sleep/slumber’\\
\textit{Ọlụchukwu} ‘The marvelous works of God’ \\
\textit{Uchechi} ‘God’s will is supreme’\\
\textit{Chukwudunzọ̄ } ‘God will fight for me’\\
\textit{Onyinyechi} ‘God’s gift (probably the gift of the Holy Spirit)’
\z

Third, the \textit{m̀mụọ̄, ànị̀, chi, àshì, òlìsà} and other names which represented deities, lesser gods, spirits, destiny and forces were substituted for \textit{Chukwu/Chi} names since they are used interchangeably in the early Christian era. Names that portrayed the traditional religious practices of the people, religious objects, reincarnation and the \textit{ọ̀gbanjē} concept were consequently dropped. More specifically, \textit{chi} names no longer refer to one’s personal god or destiny force, but to the Christian God. At this stage too, the names were no longer given to babies as first names, but were found only among the older members of the community. A few examples in \tabref{tab:owaleke:4.1} would suffice.

\begin{table}
\caption{\label{tab:owaleke:4.1}Substitution of traditional name for the Christian}
\begin{tabular}{llll}
\lsptoprule
{Traditional Names}             & {Early Christian Names} & {Meaning}                 &  \\ 
\midrule
{Mmụọ̄mà}                    & {Chukẁumà}            & {‘God is all knowing’}    &  \\ 
{Mmụọ̄zọ̀ba}                 & {Chizọ̀ba}             & {‘God is able to rescue’} &  \\ 
{Mmụọ̄ka}                     & {Chika/Chukwuka}        & {‘God is supreme’}        &  \\ 
{Òlìsàdebe}                  & {Chukwudebe}            & {‘God preserves’}         &  \\ 
{Anigbọ̀gụ}                   & {Chigbọ̀gụ}           & {‘God, our arbiter'}      &  \\ 
{Ànị̀èmekẹ/Òlìsàèmekẹ} & {Chukwuèmekẹ}         & {‘God has don well'}      &  \\
{Ònyebụàshì}                & {Ònyebụ̀chi}          & {‘Who is like unto God’}  &  \\ 
{Àshìèdu}                    & {Chinèdu}              & {‘God guides’}            &  \\
\lspbottomrule
\end{tabular}
\end{table}

Family names (that is, inheritable surnames that pass through many generations, not patronymic) were also dropped in favour of Christian names. The naming system consequently witnessed a significant shift in this era as a good number of individuals changed their family names because of the negative meanings they consider them to connote in contemporary times based on the change in religious values. Names like \textit{Nwaò̩bò̩shì̩} ‘a child of Ọ̀bọ̀shì deity’, \textit{Ànị̀èmekẹ}  ‘The Earth deity has done well’, \textit{Ọ̀fọegbù} -‘\textit{Ọ̀fọ} (deity) should not kill’, and \textit{Nwosū}  ‘child of \textit{òsu} caste’ have been dropped by their bearers. In addition to the surnames, first and second names were replaced. Consequently, names such as Nancy Immanuel, Mercy Johnson, David Harrison, Morris Jacobs, Michael Peters and Allen Moses and others that blur the socio-cultural identity and generational information about the bearer now abound. In the past you could determine the part of Igbo land that a person came from by their surnames. Such information has been completely obliterated by the new naming system in the Early-Christian era. The names, first, middle and last are now woven around the Christian doctrine and Western civilization.

\subsection{Semantic changes in the Early-Christian era}
Forces of Christianity, colonization and modernization all add up to influence the semantic changes that names have undergone. Christianity in Igbo has changed the traditional meaning of \textit{Chi}/\textit{Chukwu} names and eroded reference to other deities. First, I will examine the religious terms denoting the spirit beings:

\begin{table}
\caption{\label{tab:owaleke:4.2} Semantic changes in the early Christian era}
\begin{tabularx}{\textwidth}{ll>{\raggedright}p{\widthof{connotes extreme fear,}}Q}
\lsptoprule
{S/N} & {Word}    & {Old meaning}                       & {New meaning}\\\midrule
{1}   & {Chi/chì} & {‘personal god’, ‘fate’, ‘destiny’} & {the Christian God, every believer’s God, the supreme God}\\
{2}   & {Chukwu}  & {`supreme being’}                   & {`the Christian God’, `the father of Jesus Christ’}\\
{3}   & {àshì}    & {‘fate’, ‘destiny’}                 & {the Christian God, supreme God}\\
{4}   & {ǹzè}     & {‘a deity’}                         & {‘royalty’}\\
{5}   & {ànị}     & {‘a deity’, ‘goddess’}              & {‘community’, ‘place of origin’, `birth place’ ‘earth’, floor}\\
{6}   & {m̀mụọ̄}    & {a spirit being}                    & {‘Holy Spirit’, ‘evil spirit’}\\
{7}   & {ọnwụ}    & {connotes extreme fear, mystery}    & {a means to eternity}\\
\lspbottomrule
\end{tabularx}
\end{table}



In the traditional setting, \textit{Chi} and \textit{Chukwu} were not synonymous; \textit{Chi} specifically referred to one’s personal god or destiny and \textit{Chukwu} to the Supreme-Being. Similarly, the Christian God and \textit{Chukwu}, the Igbo supreme being were not equivalent terms. The \textit{mmụọ̄}, \textit{ànị̀}, \textit{chi}, \textit{àshì}, \textit{òlìsà} names now no longer denote spirits, destiny or lesser gods but the supreme God of the Christians. These terms too have become polysemous, capable of more than one interpretation depending on the religious value of the one using the term (whether Christian or non-Christian). So the different senses now exist in the Igbo lexicon.

Words reflecting the religious concepts of reincarnation and \textit{ọ̀gbanje} in the past have their original figurative meaning exchanged for the literal. \textit{Abị̄agọm }-- ‘I have come/returned’; \textit{Nọ̀sikē}  ‘Sit well’; \textit{Anàzị̀à}  ‘Don’t go anymore’. Their figurative meanings have become restricted. The concepts of \textit{ọ̀gbanje} and reincarnation have been debunked by medicine and science and Christianity, so new names that reflect the immediate realities of the time are now favoured. However, it is important to state that not all Igbo imbibed Christianity, and so, those who still practise traditional religion still favour the original senses.
 
Other words indigenous to the culture have become obsolete, less frequently used, or have changed their meanings, (that is, have their meanings extended or restricted). I give some examples in \tabref{tab:owaleke:4.3}.

\begin{table}
\caption{\label{tab:owaleke:4.3}Some indigenous Igbuzo words}
\begin{tabular}{llllll}
\lsptoprule
Word     & Gloss            & Word     & Gloss\\\midrule
lọ̀kọ̀    & ‘console’        & ùkpò    & ‘throne’         \\
bàa     & ‘father’         & ejìje   & ‘jewel’          \\
medụ    & ‘accomplish’     & nzọ̀lọ   & ‘consultation’   \\
gbòlu   & ‘protect/defend’ & ọbọ̀     & ‘vengeance’      \\
elò     & ‘thoughtfulness’ & enū     & ‘world’          \\
kàlụ    & ‘slander’        & mkpu    & ‘condolence’     \\
ngòli   & ‘enjoyment’      & ògò     & ‘farm’           \\
ùlìòm   & ‘chick’          & ụgbā    & ‘calabash plate’ \\
mkpàmị  & ‘appreciation’   & ikwu    & ‘kindred’        \\
ọ̀sị̀       & ‘goodness’  & èlège     & ‘slim’/’elegance’       \\
m̀gbọ.     & ‘girl’     & ìlo       & ‘lineage’               \\      
mgbuūde   & ‘lustre’   & gbo ọ̀gụ̀   & ‘arbiterate’            \\ 
ọnwọ̀      & ‘this’     & àjà       & ‘ritual items’          \\ 
Èke       & ‘first day of the week’ \\                               
\lspbottomrule
\end{tabular}
\end{table}


Words like \textit{ùkpò} ‘throne’, \textit{Nwukpò} ‘personal name’, \textit{m̀gbọ} ‘girl’, \textit{bàa} ‘father’, \textit{nzọ̀lọ} ‘consultation’, \textit{lọ̀kọ̀} ‘console’ and others are now obsolete and at the verge of extinction. Some others are more frequently used by the older members of the community. The younger generation use the newer equivalents. This situation has created temporal semantic variations in the language such as: \textit{ọnwọ̀} $\thicksim$ \textit{nkẹà} ‘this one’, \textit{ògò} $\thicksim$ \textit{ugbō} ‘farm’, \textit{ọ̀sị̀ $\thicksim$ ọ̀di mmā} ‘goodness’, \textit{ùlìòm $\thicksim$ nwa ọ̀kụkụ} ‘chick’, \textit{elò $\thicksim$ uchè} ‘thoughtfulness’ and others. Sets of existing pairs of temporal variants in the language have thus, created synonyms in the language.  

\subsection{Name-giving in the Pentecostal era}
With the introduction of Pentecostal movement and ideology (another form of Christianity) in the 1930s came a more drastic change in name-giving. The Pentecostal move offers a complete rejection of the traditional religious values of the Igbo and so, out-rightly opposes polytheism. Their mission emphasizes evangelism, miracle working, divine healing, conversion, dependence on the Holy Spirit, empowerment for the supernatural, passion for worship, praise and prayer and this has made the Pentecostals stand out from orthodox groups before them. Ironically, the doctrine downplays holiness and righteous living, and emphasizes prosperity of the body and so, Pentecostalism has been described as ``feel good Christianity''. The philosophy encourages positive confessions and rejects suffering, long suffering and persecution, and everything that is negative, as they argue that God is the author of all that is good. And so, such expressions as \textit{I am blessed}, \textit{It is well with me}, \textit{There shall be no loss}, \textit{I am highly favoured}, among others, quite often are common with Pentecostal adherents. A few examples of English and Igbo first names that reflect the ideology of this period are given in Tables~\ref{tab:owaleke:4.4} and~\ref{tab:owaleke:4.5}.

\begin{table}
\caption{\label{tab:owaleke:4.4}Typical English Pentecostal names}
\begin{tabular}{llll}
\lsptoprule
Amblessed     &    Glory          & Honour    &       Evidence  \\
\midrule
Everlasting   &   Prayer     &      Faithful    &        Miracle  \\        Anointed   &     Excellent     &     Praise      &      Melody  \\           Gracious    &   Precious     &      Warrior       &   Endwell   \\
Brightness & Favour & Prosper &           Godspresence      \\
Redeemed     & Worship    &     Marvelous      &   Trueson   \\
Breakthrough  & Godson    &     Rejoice     &   Winner \\
Steadfast    &   Goshen &     Miraculous  &          Delight \\
\lspbottomrule
\end{tabular}
\end{table}


\begin{table}
\caption{\label{tab:owaleke:4.5}Typical Igbo names in the Pentecostal period}
\begin{tabularx}{\textwidth}{lQlQ}
\lsptoprule
Somtochukwu  &‘Praise God with me’ &  Chukwuebuka  &   ‘God is awesome’ \\   
Busomma  & ‘Be so good’    &Chisom    &     ‘God is with me’\\
Chinaza & ‘God answers prayers’  & Chimamanda &‘My God will not fail me’ \\  Munachim   &‘My God and I’    &Chidera &    ‘Steadfastness of God’  \\       Chinaemezunkwa&‘God fulfils his promise’& Tobechukwu & ‘Begin to praise’ \\  Chinonyelum &‘God be with me' &Chukwunonso   &  ‘God is near’  \\         Ekenedilichukwu & ‘Thanks be to God’ & Ogochukwu & ‘God’s mercy/kindness’\\  
Ọlụchukwu      &     ‘God’s marvelous works’ & Kamsiyochukwu &  ‘As I implored of God’\\
\lspbottomrule
\end{tabularx}
\end{table}



\subsection{Social changes in the Pentecostal era}
The period introduced a more drastic change in name-giving. Names in this era, both English and Igbo reveal a great affinity to positive thinking and confession, and rejection of certain realities -- poverty, lack, sickness, and barrenness. Popular names in the Pre-Christian and Early Christian era such as \textit{Ndidi}, ‘patience’, \textit{Ndidiamaka} ‘patience is honourable’, \textit{Dibeashi} ‘accept whatever comes your way as the will of God’, and \textit{Nebeolisa} ‘look up to God’ are no longer in vogue. 

In addition, unlike naming patterns in the Christian and traditional era, Pentecostal names given in English no longer tell stories about the specific events surrounding the birth of the child and the experiences of the parents. Names such as \textit{Breakthrough, Brightness, Gospel, Beloved, Prayer, Praise, Steadfast}, and \textit{Everlasting} do not portray any unique social relevance to the child.

Clipping, a major strategy of name-giving is used here to make the names sound sophisticated, foreign and stylish. The names, therefore, do not conform in both style and structure with Igbo words as shown in \tabref{tab:owaleke:4.6}.

\begin{table}
\caption{\label{tab:owaleke:4.6} Clipped forms of Igbo names}
\begin{tabular}{llll}
\lsptoprule
Names & Clipped form & Names & Clipped form\\
\midrule
Somtochukwu &Somto, Somie &	Kamsiyochukwu &	Kamsy, Kamsie\\
Kosisochukwu &	Kosy & Chimamanda &	Mandy, Amanda\\
Ekenedilichukwu	& Eky &	Onyekachukwu &	Katchy, Katch \\
Chinonyelum	& Nolly &Chukwuemeka &	Mekus, Chucks\\
Chukwunonso	& Nonso &	Chukwuebuka	&Bukas, Ebus\\
Ọlụchukwu & Ollie & Uche & Utchay\\
\lspbottomrule
\end{tabular}
\end{table}

Clipping in this era has surpassed the earlier period and has become more sophisticated and complex due to the adoption of English spelling and pronunciation.  It is observed that the clipped or shortened forms leave the names meaningless and devoid of any socio-cultural associations. However, bearers assert that the forms make them feel good, more modern and attract more attention. Furthermore, data reveal that these clipped forms are registered officially. Clipping as a naming strategy has significantly changed the social-cultural and semantic import of naming practice and principles in Igbo.

\section{Conclusion}
Most works done in Igbo anthroponyms in the past decades have concentrated largely on the synchronic analysis of the socio-cultural import and linguistic features of personal names, but this paper has specifically examined Igbo personal names from a diachronic perspective. It highlighted the features that characterize the shifts in the practice of name-giving in the three periods chosen for the study. It reveals a number of interesting and significant facts about the evolving trends in the practice of name-giving in Igbo. In addition, the paper demonstrates that name-giving, an aspect of the Igbo language is seriously endangered as the traditional names that reflect the core vocabulary of the language are fast disappearing. Facts about the core vocabulary items show a drastic reduction in the use of indigenous names that are replete with meaning and portray the socio-cultural values of the people. A significant percentage of the vocabulary is now limited to the older members of the speech community. Quite a good number of these words however, have been documented in this study. The lack of intergenerational transmission coupled with facts of globalization is largely responsible for the situation.  We therefore warn that if the trend is not checked through documentation, a major aspect of the Igbo language and heritage would be lost.

{\sloppy\printbibliography[heading=subbibliography,notkeyword=this]}
\end{document}
