\documentclass[output=paper]{langscibook}
\ChapterDOI{10.5281/zenodo.5578830}

%This is where you put the authors and their affiliations
\author{Monica Alexandrina Irimia \affiliation{University of Modena and Reggio Emilia} and Patricia Schneider-Zioga \affiliation{University of California, Fullerton}}

%Insert your title here
\title{Differential marking in Kinande}  
\abstract{This work presents newly-discovered data involving differential marking in the Bantu language Kinande (JD42), which affects three classes: (i)  goals and sources; (ii) predicative possession; (iii) external possession. These patterns reveal important insights into the nature of differential marking in Bantu. First, typical differential marking features like animacy, definiteness, and person can involve a morphological difference on a \textit{dependent} nominal in Kinande, and is not only restricted to the better studied agreement/concord cross-indexing on the verb, which is familiar from work on other Bantu languages. Secondly, Kinande differential marking has non-trivial syntactic correlates and provides further support for nominal licensing inside \textit{v}P in Bantu languages. Thirdly, Kinande provides evidence for more than one structural source of differential marking, supporting at least three types of syntactic analyses for this phenomenon.}

\begin{document}
\maketitle

\section{Introduction}\label{sec:schneider:section1Intro}\largerpage[.525]

Bantu languages are at the center of a debate concerning the universality of nominal licensing. One position in this debate is that licensing is not a universal requirement on grammars (\citeauthor{HarfordPerez1985} \citeyear{HarfordPerez1985}, \citeauthor{Diercks2012parameterizing} \citeyear{Diercks2012parameterizing}, a.o.) and that Bantu languages exemplify the ``licensing not required'' option. A second position is taken by \citeauthor{Halpert2013} (\citeyear{Halpert2013}, \citeyear{Halpert2015}), who, focusing primarily on data from Zulu, argues that Bantu nominals require licensing in ways familiar from Case theory, but these requirements hold in domains that are different from those of Indo-European languages. She proposes that, in Bantu languages, familiar structural-Case-type licensing takes place within \textit{v}P and a different type of licensing takes place in other domains.\footnote{There are additional important aspects of the licensing debate that our paper is unable to address. One is the question of information structure in formally licensing nominals (\citeauthor{vanderWal2017} \citeyear{vanderWal2017}). The other is \citeauthor{CarstensMletshe2016}'s (\citeyear{CarstensMletshe2016}) and \citeauthor{Pietraszko2019}'s (\citeyear{Pietraszko2019}) discussions of the role of focus in the licensing of unaugmented expressions. }\\
\indent We add to this discussion by introducing our newly-discovered data involving differential marking in Kinande (JD42),\footnote{Kinande, with about 900,000 speakers per Ethnologue (2015), is spoken in eastern Democratic Republic of the Congo. Unless otherwise stated, data has been collected by Schneider-Zioga in collaboration with Phillip Ngessimo Mutaka.} which permit a two-fold theoretical contribution. On the one hand, the existence of differential marking in Kinande provides further support for nominal licensing inside \textit{v}P (following \citeauthor{OrmazabalRomero2013} \citeyear{OrmazabalRomero2013},  \citeauthor{Kalin2018} \citeyear{Kalin2018}, a.o., for other languages). On the other hand, we establish that typical differential marking features like animacy, definiteness, and person can involve a morphological difference on a dependent nominal in Bantu languages, and is not only restricted to the better studied agreement/concord cross-indexing on the verb, which is familiar from work on Bantu languages such as Chichewa (\citeauthor{BresnanMchombo1987} \citeyear{BresnanMchombo1987}) or Sambaa (\citeauthor{Riedel2009} \citeyear{Riedel2009}). Another important observation is that Kinande provides evidence for more than one structural source of differential marking. More precisely, the language exhibits more than one type of dependent differential marking, supporting at least three types of analyses for this phenomenon. 

\subsection{Differential marking}\label{sec:schneider:subsection1.1Differentialmarking}

Differential marking refers to splits in the morpho-syntactic encoding of arguments regulated by features such as animacy, definiteness, specificity, topicality, etc. (\citeauthor{Silverstein1976} \citeyear{Silverstein1976}, \citeauthor{boss98} \citeyear{boss98}, \citeauthor{laz01} \citeyear{laz01}, \citeauthor{Aissen2003} \citeyear{Aissen2003}, \citeauthor{Lopez2012} \citeyear{Lopez2012}, \citeauthor{OrmazabalRomero2013} \citeyear{OrmazabalRomero2013}, a.o.). With respect to differential marking in Bantu languages, the following authors can be noted: \citeauthor{Bentley1994} (\citeyear{Bentley1994}); \citeauthor{Duranti1979} (\citeyear{Duranti1979}); \citeauthor{HawkinsonHyman1974} (\citeyear{HawkinsonHyman1974}); \citeauthor{MorolongHyman1977} (\citeyear{MorolongHyman1977}); \citeauthor{MorolongHyman1977} (\citeyear{MorolongHyman1977}); \citeauthor{Mursell2018} (\citeyear{Mursell2018}); \citeauthor{SeidlDimitriadis1997} (\citeyear{SeidlDimitriadis1997}); and \citeauthor{vanderWal2015} (\citeyear{vanderWal2015}). \\
\indent A typical example of differential marking can be seen with animacy and spec\-i\-fic\-i\-ty-based differential object marking (DOM) in Swahili (Bantu; \citeauthor{Riedel2009} \citeyear{Riedel2009}: 42, 46, adapted here), via ``object agreement.'' Sentence \REF{ex:schneider:Swahilifirst} versus \REF{ex:schneider:Swahilisecond} illustrates that an animate, specifically [+human], object must co-occur with an object marker on the verb that expresses agreement in features with that object. Sentence \REF{ex:schneider:Swahilithird} demonstrates that non-human/inanimate objects are subject to a different requirement such that object agreement is simply optional when such arguments occur:\largerpage[-1]\pagebreak

\ea \label{ex:swahili}
DOM in Swahili, human $\&$ definite object, [head marking]
\begin{xlist}
\ex[]{ \label{ex:schneider:Swahilifirst}
\gll Ni-li-\textbf{mw}-ona  		mwana-we.\\
1\textsc{sg-pst-1om}-see  	1child-\textsc{poss.3sg}\\
\glt `I saw his child.' }
\ex[*]{ \label{ex:schneider:Swahilisecond}
\gll Ni-li-ona  		mwana-we.\\
1\textsc{sg-pst}-see  	1child-\textsc{poss.3sg}\\
\glt Intended: `I saw his child.' }
\ex[]{ \label{ex:schneider:Swahilithird}
\gll Ni-li-(zi)-ona         picha      hizo.\\
1\textsc{sg-pst-10om}-see  	10picture those\\
\glt `I saw those pictures.'}
\end{xlist}
\z

This pattern of differential marking makes use of head marking. That is, there is some indication of differential object marking on the head of the phrase that immediately contains the differentially marked object.\\
\indent A different example of differential object marking based on animacy can be found in languages such as Spanish (\citeauthor{boss91} \citeyear{boss91}, \citeauthor{Aissen2003} \citeyear{Aissen2003}, 
\citeauthor{Lopez2012} \citeyear{Lopez2012}, \citeauthor{OrmazabalRomero2013} \citeyear{OrmazabalRomero2013}, a.o.). Grammaticalized animacy in Spanish is encoded via a locative/dative preposition, with definite animate objects such as \textit{ni\~na} ‘girl’ in \REF{ex:schneider:Spanishfirst} being obligatorily introduced by a locative preposition and definite inanimate objects such as \textit{libro} ‘book’ in \REF{ex:schneider:Spanishsecond} not being so introduced:

\ea \label{ex:Spanish}
Spanish DOM, grammaticalized animacy, [dependent marking]

\begin{xlist}
\ex \label{ex:schneider:Spanishfirst}
\gll He 		encontrado   *(\textbf{a}) la    ni\~na.\\
have.\textsc{1sg} found \textsc{dat=dom} \textsc{def.f.sg} girl\\
\glt `I found the girl.' 

\ex \label{ex:schneider:Spanishsecond}
\gll He encontrado (*a) el libro.\\
have.\textsc{1sg} found \textsc{dat=dom} \textsc{def.m.sg} book\\
\glt `I found the book.' \hfill (\citeauthor{OrmazabalRomero2013} \citeyear{OrmazabalRomero2013}: ex. 1a, b)

\end{xlist}

\z

In this example from Spanish, we see that DOM is marked on the object which functions as an argument of the verb. In short, we see that differential marking can be expressed either through head or dependent marking. In Bantu languages, differential marking that is expressed via head-marking has been widely reported. Our work on Kinande makes a unique contribution toward the differential marking literature in Bantu in that it reveals cases of differential marking expressed as dependent marking. We will see the details of the dependent differential marking in the following discussion, where three distinct cases of dependent-marking differential marking are examined.   


\section{The data}\label{sec:schneider:section2Data}

\subsection{Pattern 1: Differential marking of animate goals}\label{sec:schneider:subsection2.1Differentialmarkingofanimategoals}

The first differential marking pattern we consider involves animacy-based differential marking of goals and sources. Specifically, pronouns and human names are differentially marked in Kinande when they function as thematic goals and sources. Studies such as that of \citeauthor{Aissen2003} (\citeyear{Aissen2003}), and others, locate pronouns and proper names as the most animate nominal expressions on a scale of animacy:

\ea \label{ex:AnimacyScale}
Animacy Scale

\begin{xlist}

\ex 1/2 \textgreater 3 \textgreater  proper name \textgreater human  \textgreater animate… \hfill (\citeauthor{Aissen2003} \citeyear{Aissen2003}, a.o.)

\end{xlist}
\z

This type of animacy-based differential marking requires the use of a certain form of the locative noun class marker just in case the goal is either a human proper name \REF{ex:schneider:uku}, or a pronoun \REF{ex:schneider:ko}.  Using noun class 17 as an example, a certain form of a locative class, namely \textit{uku}, must occur with human names and pronouns; a different form of locative noun class 17, \textit{oko}, marks all other goals, including the names of inanimate entities such as the Jordan River in \REF{ex:schneider:oko}:

\ea \label{ex:schneider:ukuandoko}
\begin{xlist}

\ex \label{ex:schneider:uku}
\gll Kandi  omúgulu  ba-h\~ika \textbf{ok\'o} ndeko, omundú  mw-á-h\~ika \textbf{uku} Y\'esu. \\
again 3time 2-arrive 17\textsc{loc} 9crowd,  1person \textsc{aff-3sg-}arrive  \textsc{uku} Jesus \\
\glt `When they came to the crowd, a man approached Jesus.' \\
(Matthew 17: 12)

\ex \label{ex:schneider:oko}
\gll Neryo Y\'esu… mw-á-h\~ika \textbf{oko} Yorodáni. \\
then Jesus \textsc{aff-3sg-}arrive  17\textsc{loc} 19Jordan \\
\glt `Then Jesus came to the Jordan.'  (Matthew 3: 13) 

[Nande Bible 1980 edition]

\ex \label{ex:schneider:ko}
\gll tú-lya-byá tw-a-hiká \textbf{kú}-b\'o,    bá-má-tú-bwira ba-ti.... \\
\textsc{1pl-tam-}be \textsc{1pl-tam-}arrive \textsc{uku}-2pro \textsc{2-tam-1pl-}told 2-say
 \\
\glt `When we reached them, they told us that....' 

\end{xlist}

\z 

The data in \REF{ex:schneider:Italian} illustrate this common differential marking patterns with goals cross-linguistically. In Italian, we see that whereas inanimate goals are marked with the directional preposition \textit{in} \REF{ex:schneider:Italianinanimate} animate goals are not. Instead they are marked with a different directional preposition which fuses with the definite morpheme, as seen in \REF{ex:schneider:Italiananimate}: 

\ea \label{ex:schneider:Italian}
Italian differential marking of goals-animacy
\begin{xlist}

\ex \label{ex:schneider:Italianinanimate}
\gll Vado    \textbf{in/*dalla}      citt\`a. \\
go.\textsc{1sg}  \textsc{dir/dir=dom.def.f.sg}    city \\
\glt `I go to the city.'

\ex \label{ex:schneider:Italiananimate}
\gll Vado  \textbf{dal/*in}  dottore/mio amico. \\
go.\textsc{1sg} \textsc{dir=dom.def.m.sg/dir}  doctor/my friend \\
\glt  `I go to the doctor/my friend.'\\
(see \citeauthor{FrancoManzini2017} \citeyear{FrancoManzini2017}, a.o. for discussion)

\end{xlist}

\z

Returning to Kinande, we note that the difference in morphological form we observe when goals and sources are involved corresponds to whether or not the nominal that is prefixed by the locative class marker is augmented or not. Augments are additional morphological material that is associated with most noun class markers and which usually affect the interpretation of the noun. Augments are related to reference, but have different interpretive effects in different Bantu languages (see \citeauthor{vandeVelde2019} \citeyear{vandeVelde2019} for overview). An augmentless nominal in Kinande is most typically associated with an interpretation of narrowest scope, similar to polarity items (see \citeauthor{Progovac1993} \citeyear{Progovac1993}). Augmented nominals, on the other hand, often get a definite or indefinite reading. However, more than the semantic notion of (in)definiteness is relevant to understanding augments as explored in the work of \citeauthor{Gambarage2019} (\citeyear{Gambarage2019}), \citeauthor{Halpert2015} (\citeyear{Halpert2015}), \citeauthor{HymanKatamba1993} (\citeyear{HymanKatamba1993}), \citeauthor{Progovac1993} (\citeyear{Progovac1993}),  and \citeauthor{vandeVelde2019} (\citeyear{vandeVelde2019}). The general shape of Kinande nominals is \textsc{(aug)-nc-root}, as illustrated in \tabref{tab:schneider:NominalstructureKinande}.

\begin{table}
\caption{Nominal structure in Kinande}
\label{tab:schneider:NominalstructureKinande}
 \begin{tabular}{ll}
  \lsptoprule
           \multicolumn{2}{c}{(Augment-)noun class marker-noun}\\
  Augmented nouns  &   Augmentless nouns \\
  \midrule
  \textbf{o}-mu-kali  &   mu-kali \\
  \textsc{aug-nc1-}woman ‘the/a woman’ & \textsc{nc1-}woman `any woman' \\\tablevspace
  
  \textbf{e}-ki-tabu & ki-tabu\\
  \textsc{aug-nc7-}book ‘the/a book’ & \textsc{nc7-}book `any book'\\
  \lspbottomrule
 \end{tabular}
\end{table}

When a nominal is placed in a locative class (class 17 and class 18), the augment of a definite or indefinite located nominal is not overtly expressed. Instead, with augmented nominals, the locative class forms 17 and 18 are \textit{oko} and \textit{omo} respectively. If locative nominals must be interpreted as augmentless, as, for example, when they are negative polarity items (NPIs), then the locative class forms surface as \textit{uku} and \textit{umu}. In short, the augmented or augmentless status of locative nominals can be read off of the form of the locative noun class marker. This can be clearly seen when locative negative polarity contexts are examined. When a nominal occurs in a context where it must remain augmentless, as in \REF{ex:schneider:augmentnpi} where the NPI interpretation of the nominal indicates it must be augmentless, the locative marker is expressed as \textit{uku}. Where the nominal is interpreted as having an augment because it is definite, for example, the locative marker is expressed as \textit{oko}, as in \REF{ex:schneider:augmentdef}. This pattern is summarized in \tabref{tab:schneider:Kindandelocatives}.

\ea \label{ex:schneider:augments}
\begin{xlist}

\ex \label{ex:schneider:augmentnpi}
\gll Maryá \textbf{\textit{sy}}-á-wíte \textbf{uku} kitábu.\\
Marya \textsc{neg-3sg}-have \textsc{uku} 7book\\
\glt  `Mary doesn't have \textbf{any} book.'

\ex \label{ex:schneider:augmentdef}
\gll Maryá \textbf{\textit{sy}}-á-wíte \textbf{oko} kitábu.\\
Marya \textsc{neg-3sg}-have \textsc{oko} 7book\\
\glt  `Mary doesn't have \textbf{the} book.'

\end{xlist}
\z 

\begin{table}
\caption{Kinande locatives}
\label{tab:schneider:Kindandelocatives}
 \begin{tabular}{ll}
  \lsptoprule
           Augmented form &  Non-augmented form \\
  \midrule
  \textbf{oko}  &  \textbf{uku}  \\
  \textsc{17loc + aug} & \textsc{17loc(-aug)} \\\tablevspace
  *oko-e-\textit{bi}-tabu & *uku-e-\textit{bi}-tabu\\
  *\textbf{oko}-e-\textit{bi}-tabu & \textbf{*uku}-e-\textit{bi}-tabu\\
  \textsc{[17loc+aug]-[aug]-nc8}-book  & \textsc{[17loc-aug]-[aug]-nc8}-book\\
  `to books' & `to (any) books'\\\tablevspace
  oko-\textit{bi}-tabu & uku-\textit{bi}-tabu \\
  \textbf{oko}-\textit{bi}-tabu & \textbf{uku}-\textit{bi}-tabu \\
  \textsc{[17loc+aug]-nc8}-book & \textsc{[17loc-aug]-nc8}-book\\
  `to books' & `to any books'\\
  \lspbottomrule
 \end{tabular}
\end{table}

Thus, we see from the examples in \REF{ex:schneider:ukuandoko} that Kinande goals (and sources) that are high in animacy are differentially marked via the augmentless form of the locative. The differential form is the one that is typically reserved to express an NPI-type, narrowest scope interpretation. But the nominals that are differentially marked here (human names and pronouns) do not have or do not need to have an NPI interpretation.\footnote{We note that there are contexts where human names are preceded by the augmented form of locative class markers (one such case will be seen in \REF{ex:schneider:examplesadjacency}) and that sometimes the augmented form is required. The following example illustrates such a case. Note that in this example the name introduced by \textit{oko} is no longer a goal, but rather refers to the surface upon which movement took place: 

\ea \label{ex:schneider:surfaceanimate}
\gll olumekeke 	mo-lu-ka-kululuk-ir-a 		\textbf{oko} 	Tekela\\
11baby  \textsc{aff-11-tam-}crawl-\textsc{appl-fv} 	\textsc{17loc} 	Thekla\\
\glt  `The baby crawled (all over) on Thekla.'
\z 

Note that this locative (in contrast to goal and source) use does not make use of differential marking because \textit{oko} also shows up with locative inanimates as the following example demonstrates:

\ea \label{ex:schneider:surfaceinanimate}
\gll olumekeke 	mo-lu-ka-kululuk-ir-a       \textbf{oko}  	musesa.\\
11baby \textsc{aff-11-tam-}crawl-\textsc{appl-fv}  \textsc{17loc} 3bed\\
\glt  `The baby crawled (all over) on the bed.'
\z} Following \citeauthor{Halpert2015} (\citeyear{Halpert2015}), we will argue that the augmentless form needs special licensing, a point that we return to later in \sectref{sec:schneider:section5}.

\subsection{Pattern 2: The differential marking of predicative possession}\label{sec:schneider:subsection2.2Differentialmarkingpredicativepossession}

The second differential marking pattern involves predicative possession. Predicative possession in Kinande uses a dedicated verb for ‘have’ (\textit{-wite}). Crucially, possessa are differentially marked; nominals encoding objects that can be held in the hand receive more complex morphological marking, namely the locative class 17 marker, compared to objects that are possessed in a more general way. The examples in \REF{ex:schneider:possessionandownership} contrast possession of a holdable object (\ref{ex:schneider:possession}, ‘book’) with a non-holdable object (\ref{ex:schneider:ownership}, ‘house’). Only holdable \textit{ekitabu} ‘book’ is prefixed with a locative class marker, although this is not interpreted as a locative here. 

\ea \label{ex:schneider:possessionandownership}
\begin{xlist}

\ex \label{ex:schneider:ownership}
\gll Kámbale a-wíte \textbf{(*oko)} enyúmba.\\
Kambale \textsc{3sg}-have \textsc{17loc} 9house\\
\glt  `Kambale has a house.'

\ex \label{ex:schneider:possession}
\gll Kámbale a-wíte \textbf{*(oko)} kitábu.\\
Kambale \textsc{3sg}-have \textsc{17loc} 7book\\
\glt  `Kambale has a book.'

\end{xlist}
\z

The following examples underscore the systematic morphological distinction that is made between holdable and general possession. In \REF{ex:schneider:ownershipnooko}, we see an example of general possession, where the possessum is not differentially marked. In \REF{ex:schneider:possesionandoko}, we see an example of holdable possession: 

\ea \label{ex:schneider:ownershipandoko}
\begin{xlist}
\ex \label{ex:schneider:ownershipnooko}
\gll A-wíte esyofarángâ.\\
\textsc{3sg}-have 10money\\
\glt  `He has money (in general).'

\ex \label{ex:schneider:possesionandoko}
\gll A-wíte \textbf{oko} farángâ.\\
\textsc{3sg}-has \textsc{17loc} 10money\\
\glt  `He has money (on him).'
\end{xlist}
\z 

Possession of an attribute or inalienable possession is also not differentially marked, as seen in the two examples in \REF{ex:schneider:attributeorinalienable} below:

\ea \label{ex:schneider:attributeorinalienable}
\begin{xlist}

\ex \label{ex:schneider:attribute}
\gll A-wíte  ekitumaíni\\
\textsc{3sg}-have 7hope\\
\glt  `She/he has hope.'

\ex \label{ex:schneider:inalienable}
\gll Maryá a-wity' ámeso aw\^uw\'ene.\\
Marya \textsc{3sg}-have 6eye 6beautiful\\
\glt  `Mary has beautiful eyes.'

\end{xlist}
\z

\citeauthor{SchneiderZiogaMutaka2019a} (\citeyear{SchneiderZiogaMutaka2019a}) observe that further examples suggest that Kinande distinguishes possession from ownership and propose that \textit{oko} marks possession in opposition to ownership. They give the examples in \REF{ex:schneider:okopossession} and \REF{ex:schneider:ownversusavailable}:

\ea \label{ex:schneider:okopossession}
\gll Nyi-ná-wíte \textbf{ok\'o}  mútoká    kw'   eyíhy\^a.\\
\textsc{1sg-ver}-have  \textsc{17oko} 3car  \textsc{17lk} 24outside\\
\glt  `I indeed have a car outside.'
\ex \label{ex:schneider:ownversusavailable}
\begin{xlist}

\ex \label{ex:schneider:own}
\gll A-ná-wíte  omútoka.\\
\textsc{3sg-ver}-have 3car\\
\glt  `He has (owns) a car.'

\ex \label{ex:schneider:available}
\gll A-ná-wíte \textbf{ok\'o} mútoka.\\
\textsc{3sg-ver}-have \textsc{17loc} 3car\\
\glt  `He has a car available.' \hfill (He might own it or just have it temporarily.)

\end{xlist}
\z

\subsection{Pattern 3: The differential marking of external possession}\label{sec:schneider:subsection2.3}

The third differential marking pattern involves a type of external possession. In this construction the possessor of a noun occurs in a position that is external to the noun phrase containing the possessed noun. Moreover, that possessor behaves like an argument of the verb. This construction is sometimes referred to as possessor raising, as if a movement/raising relation holds between the external possessor and a related sentence involving phrase-internal possession. This is illustrated below for inalienable possession in Kinande where \REF{ex:schneider:illustrateinternal} corresponds to phrase internal possession and \REF{ex:schneider:illustrateexternal} corresponds to the related possessor ``raising'' construction.\footnote{Note that an agreeing particle, called the linker, appears in the Kinande verb phrase in most cases when there are two or more XPs within the verb phrase. This particle heads a functional projection that is immediately dominated by \textit{v}P as will be evident in the trees we give illustrating structures where linkers are also involved. See \citeauthor{BakerCollins2006} (\citeyear{BakerCollins2006}) and \citeauthor{Schneider-Zioga2015a} (\citeyear{Schneider-Zioga2015a},  \citeyear{Schneider-Zioga2015b}) and \citeauthor{Schneider-ZiogaMutaka2015} (\citeyear{Schneider-ZiogaMutaka2015}) for discussion.}

\ea \label{ex:schneider:illustratepossession}
\begin{xlist}

\ex \label{ex:schneider:illustrateinternal}
Phrase-internal possession:\\
\gll Ná-kúrugut-a  [ omug\'ongo w-a Sárah ]\\
\textsc{1sg}-scrub-\textsc{fv} {} 3back 3-\textsc{assoc} Sarah\\
\glt  `I scrubbed Sarah's back.' (lit: I scrubbed the back of Sarah)

\ex \label{ex:schneider:illustrateexternal}
Possessor raising:\\
\gll Ná-kúrugut-ta [ Sáráh ] y' [ *(\textbf{oko-})mug\'ongo ] \\
\textsc{1sg}-scrubbed-\textsc{fv} {} Sarah {} \textsc{lk} {} \textsc{17loc}-3back {} \\
\glt  `I scrubbed Sarah's back.' (lit: I scrubbed Sarah on the back.)

\end{xlist}
\z

We observe that in external possession constructions, \textit{oko} marks the inalienable \textit{possessum}. Alienable and part/whole possessa also participate in external possession constructions, as seen in  \REF{ex:schneider:raisingalienable} and \REF{ex:schneider:partwholeraising}. In that case, the possessum is bare. Note, however, that alienable and part/whole external \textit{possessors} in that case are marked with \textit{oko}. 

\begin{itemize}
    \item Alienable possession:
    
\ea \label{ex:schneider:alienablepossession}
\begin{xlist}

\ex \label{ex:schneider:phrasealienable}
Phrase-internal possession: \\
\gll Ná-mat-ul-a [ ezípe y-a Sárah ]\\
\textsc{1sg}-fasten-\textsc{rev-fv} {} 9zipper 9-\textsc{assoc} Sarah {}\\
\glt  `I unfastened Sarah's zipper.'\\
(for example, on a dress she has, not necessarily wearing)

\ex \label{ex:schneider:raisingalienable}
Possessor raising:\\
\gll Ná-mat-ul-a [ *(\textbf{oko-})Sáráh ] kw' [ ezípe ]\\
\textsc{1sg}-fasten-\textsc{rev-fv} {} \textsc{17loc-}Sarah {} \textsc{lk} {} 9zipper {} \\
\glt  `I unfastened Sarah's zipper.'\\
(she is necessarily wearing the thing with the zipper)

\end{xlist}
\z

    \item Part/whole possession:
    
\ea \label{ex:schneider:partwholepossession}
\begin{xlist}

\ex \label{ex:schneider:partwholephrase}
Phrase-internal possession:\\
\gll M\'o-b-erír-y-e [ ebíring\'o by-' omútoka ]\\
\textsc{aff-}2-clean-\textsc{trans-fv} {} 8wheel 8-\textsc{assoc} 3car {} \\
\glt  `They cleaned the wheels of the car.'\\
(the wheels could be separate from the car)

\ex \label{ex:schneider:partwholeraising}
Possessor raising: \\
\gll M\'o-b-erír-y-e  [ *(\textbf{ok\'o-})mútoká ]  kw' [ ebíríngo ]\\
\textsc{aff-}2-clean-\textsc{trans-fv} {} \textsc{17loc-}3car {} \textsc{lk} {} 8wheel {} \\
\glt  `They cleaned the wheels of the car.'\\
(wheels are on the car)

\end{xlist}
\z
\end{itemize} 

A summary of the facts is given in \tabref{tab:schneider:SummaryDifferentialMarkingKinande}.

\begin{table}
\caption{Summary of differential marking patterns in Kinande}
\label{tab:schneider:SummaryDifferentialMarkingKinande}
 \begin{tabular}{lll}
  \lsptoprule
           Pattern I &  \textsc{goals \& sources} & \\
  \textit{uku/oko} & pronouns, humans & \textsc{uku}\\
  & others & \textsc{oko}\\
  \midrule
   Pattern II &  \textsc{predicative possession} & \\
   \textit{oko}/$\emptyset$ & possession & \textsc{oko} \\
   &  ownership & $\emptyset$\\
   \midrule
    Pattern III &  \textsc{external possession} & \\
    \textit{oko}/$\emptyset$ & inalienable &  \\
    & { } possessum marked with & \textsc{oko} \\
    & non-inalienable & \\
    % & { } alienable  &\\
    % & { } part/whole  &\\
    & { } possessum marked with & $\emptyset$ \\
    & { } possessor/whole marked with & \textsc{oko}\\
  \lspbottomrule
 \end{tabular}
\end{table}
  	
   


\section{Differential marking in morphology or syntax?} \label{sec:schneider:section3}


Differential marking has received renewed attention in recent formal accounts. Starting from differential object marking, there are two important lines of reasoning into its nature, namely a morphological one and a syntactic one. Under morphological analyses (\citeauthor{HalleMarantz1993} \citeyear{HalleMarantz1993}, \citeauthor{KeineMuller2008} \citeyear{KeineMuller2008}, \citeauthor{Keine2010} \citeyear{Keine2010}, \citeauthor{Glushan2010} \citeyear{Glushan2010}, a.o.), the special morphology that certain types of direct objects (such as the Spanish definite animate in \REF{ex:schneider:Spanishfirst}) receive is simply a matter of PF, without any deep syntactic roots. In other words, differentially marked objects and the non-differentially marked ones are seen as having the same syntax (e.g., undergoing licensing for Case, etc.). Formally, the special differential marking is implemented as the result of an \textit{Impoverishment} operation applying at the interface between syntax and morphology/PF, and which deletes certain types of features at PF, without affecting their syntax. For example, under \citeauthor{HalleMarantz1993}'s (\citeyear{HalleMarantz1993}) implementation, specifications such as animacy trigger the deletion of an (accusative) case feature on the relevant objects. Therefore, differential objects, although having been licensed as structural objects in the syntax, carry oblique morphology on the surface (i.e., the animate genitive of direct objects in Russian, oblique DOM in Romance and other languages, etc.). \\ \indent Although various morphological accounts use \textit{Impoverishment} in slightly different ways, they agree on an important prediction: the same syntactic configuration is exhibited by both differentially marked classes and the non-differentially marked ones, implying the same syntax. This is relevant for the discussion in this paper. What Kinande shows instead is that differential objects are syntactically distinct, and for this reason, syntactic accounts are better fitted to address the data at hand. Below, we illustrate various syntactic effects of differential marking, and then we proceed to the syntactic analyses. 


\section{Syntactic effects of differential marking} \label{sec:schneider:section4}\largerpage

In this section we explore the syntax of nominals in the three constructions we have identified as involving differential marking: i) sentences with goals/sources; ii) predicative possession; iii) external possession (so-called possessor raising). We will see that the differentially marked nominal is not simply morphologically distinct from the unmarked form. Instead, we notice that the differentially marked nominal displays a syntactic behavior that is distinct from that of an unmarked nominal.  

\subsection{Animate/inanimate goal (or source) distinction}\label{sec:schneider:subsection4.1}

We consider first differential marking with animate goals (or sources), which are marked by the augmentless \textit{uku} rather than the expected \textit{oko}. Here we will see that the distribution of \textit{oko} and \textit{uku} is sensitive to syntactic structure. For example, we note that, whereas highly animate goals take locative differential marking when the goal is the only dependent of the verb \REF{ex:schneider:noadverb}, differential marking becomes optional when an adverb is added in postverbal position, as in \REF{ex:schneider:adverb}: 

\ea \label{ex:schneider:exampleswithadverbs}
\begin{xlist}

\ex \label{ex:schneider:noadverb}
\gll Omundú mw-á-híka    \textbf{uku}/\textbf{*oko} Yesu. \\
1person  \textsc{aff-3sg-}arrive (-\textsc{aug})17\textsc{loc}/*17\textsc{loc} Jesus \\
\glt `Someone came to Jesus.' 

\ex \label{ex:schneider:adverb}
\gll Omundú mw-á-híka    \textbf{uku}/\textbf{\textsuperscript{?}oko} Y\'esú k' omotututu. \\
1person  \textsc{aff-3sg-}arrive (-\textsc{aug})17\textsc{loc}/17\textsc{loc} Jesus \textsc{lk'} 18morning\\
\glt `Someone came to Jesus in the morning.' 

\end{xlist}

\z

In contrast, inanimate goals are invariantly marked with \textit{oko} regardless of the syntactic structures in which they are found. Locative \textit{oko} is the only possibility when the goal is the sole dependent of the verb (\ref{ex:schneider:noadverbinanim}), and remains the only possibility when an adverb is added in postverbal position (\ref{ex:schneider:adverbinanim}):


\ea \label{ex:schneider:exampleswithadverbsAnim}
\begin{xlist}

\ex \label{ex:schneider:noadverbinanim}
\gll Omundú mw-á-híka    \textbf{*uku}/\textbf{\Checkmark oko} muyî. \\
1person  \textsc{aff-3sg-}arrive (-\textsc{aug})17\textsc{loc}/17\textsc{loc}  3village \\
\glt `Someone came to the village.' 

\ex \label{ex:schneider:adverbinanim}
\gll Omundú mw-á-híka    \textbf{*uku}/\textbf{\Checkmark oko} muyî kw’omotututu.. \\
1person  \textsc{aff-3sg-}arrive (-\textsc{aug})17\textsc{loc}/17\textsc{loc} 3village \textsc{lk}'18morning \\
\glt `Someone came to the village in the morning.' 

\end{xlist}

\z
 
Therefore, we see that nominals that can be differentially marked by \textit{uku}- have a different syntactic behavior. Furthermore, as an additional syntactic distinction, an animate \textit{uku}-marked goal (or source) must be adjacent to the verb. When it is not, it is marked by \textit{oko} rather than \textit{uku}. This is illustrated by the contrast between (\ref{ex:schneider:adjacent}) and (\ref{ex:schneider:procliticverb}), where if a direct object enclitic intervenes between the verb and the animate goal, \textit{uku} is not possible. If the direct object pronoun precedes the verb root instead, so that the animate goal/source is immediately adjacent to the verb, then \textit{uku} can and indeed must, mark the animate nominal. Likewise, if the object has undergone \textit{wh}-movement as in (\ref{ex:schneider:whintervening}), so that the animate goal/source is string adjacent to the verb, then \textit{uku} can and must, mark the animate nominal. 


\ea \label{ex:schneider:examplesadjacency}
\begin{xlist}

\ex \label{ex:schneider:adjacent}
\ldots verb] enclitic] oko/*uku \\
\gll N-ibá-ky' \textbf{oko/*uku} Mary\^a. \\
\textsc{1sg}-steal-\textsc{7cl} \textsc{17loc/(-aug)17loc} Marya \\
\glt `I stole it from Marya.'

\ex \label{ex:schneider:procliticverb}
\ldots procl\textsubscript{j}-verb] \underline{\hspace{0.55cm}}\textsubscript{j} uku/*oko \\
\gll Ná-ki-iba \textbf{*oko/\Checkmark uku} Mary\^a. \\
\textsc{1sg}-\textsc{7om}-steal \textsc{17loc/(-aug)17loc} Marya \\
\glt `I stole it from Marya.'

 
\ex \label{ex:schneider:whintervening}
\ldots\textsc{wh}\textsubscript{j}-verb] \underline{\hspace{0.55cm}}\textsubscript{j} uku/*oko \\
\gll \'Ekihi\textsubscript{j} ky\'o w-ibá \underline{\hspace{0.5cm}}\textsubscript{j} \textbf{*oko/\Checkmark uku} Mary\^a. \\
7what \textsc{7foc} \textsc{2sg-}steal {  } \textsc{17loc/(-aug)17loc} Marya \\
\glt `What did you steal from Marya?'


\end{xlist}

\z

\subsection{Possession vs ownership}\label{sec:schneider:subsection4.2}  

We have already seen semantic differences between differentially marked and non-differentially marked sentences involving predicative possession, with differentially marked sentences conveying the meaning of possession and non - differentially marked sentences conveying ownership. We further adopt the idea of   \citeauthor{SchneiderZiogaMutaka2019a} (\citeyear{SchneiderZiogaMutaka2019a}) that \textit{oko}-marked possessa, in contrast to non-marked possessa, can be analyzed as involving a small clause with the \textit{oko}-marked possessum serving as subject of a small clause which has an understood predicate that can be translated as ‘with [possessor]’. We observe a further clear syntactic distinction between \textit{oko}-marked and bare nominals when we consider word order possibilities. Specifically, we see that differentially marked nominals can either precede or follow an adverb in predicative possession constructions, as in \REF{ex:schneider:PossessumAdverb} and \REF{ex:schneider:AdverbPossessum}. In contrast, unmarked nominals have a fixed word order and must remain next to the verb, as illustrated in \REF{ex:schneider:PossessumAdverbNoDifMarking} and \REF{ex:schneider:AdverbPossessumNoDifMarking}.

\ea \label{ex:schneider:Possession}
\begin{xlist}

\ex \label{ex:schneider:PossessumAdverb}
\gll Tu-ná-wit\'y \textbf{oko} mukáti k\'o hano. \hspace{1cm} \textsc{\textsubscript{oko}possessum-adverb}  \\
\textsc{1pl-ver}-have \textsc{17loc} 3bread \textsc{lk} 16here \\ 
\glt `We do have bread here (with us).' 

\ex \label{ex:schneider:AdverbPossessum}
\gll Tu-ná-wit\'e han\'o h' \textbf{oko} mukáti. \hspace{0.9cm} \textsc{adverb-\textsubscript{oko}possessum} \\
\textsc{1pl-ver}-have  16here \textsc{lk} \textsc{17loc} 3bread\\ 
\glt `We do have here (with us) bread.'

\end{xlist}

\ex \label{ex:schneider:PossessionNoDifMarking}
\begin{xlist}

\ex \label{ex:schneider:PossessumAdverbNoDifMarking}
\gll Tu-ná-wit\'y enyúmbá y\'o hano. \hspace{1.7cm} \textsc{possessum-adverb} \\
\textsc{1pl-ver}-have 9house \textsc{lk} 16here \\ 
\glt `We do indeed have a house here.'


\ex \label{ex:schneider:AdverbPossessumNoDifMarking}
\gll \textsuperscript{*/?}Tu-ná-wit\'y han\'o  h' enyúmba. \hspace{1.6cm} \textsc{possessum-adverb}  \\
\textsc{1pl-ver}-have 16here  \textsc{lk} 9house  \\
\glt `We do indeed have a house here.'

\end{xlist}

\z


Further evidence for syntactic differences is given by the fact that bare possessa are sensitive to definiteness effects, but \textit{oko}-marked possessa are not. See \citeauthor{SchneiderZiogaMutaka2019a} (\citeyear{SchneiderZiogaMutaka2019a}) for details. 

\subsection{External possession}\label{sec:schneider:subsection4.3} 

\subsubsection{Inalienable possession}\label{sec:schneider:subsection4.3.1}

External possession constructions also evince a syntactic distinction between differentially marked \textit{oko}-possessa and non-marked possessa. Although usually the dependents of the verb in Kinande have flexible word order with respect to each other, \textit{oko}-marked possessa have a fixed word order, with possessum following possessor. This is demonstrated by the contrast in \REF{ex:schneider:ExternalPossession}. Moreover, while a possessor can passivize, as in \REF{ex:schneider:PossessorPassivization}, an \textit{oko}-marked possessum cannot passivize across the possessor in \REF{ex:schneider:NoOKOPossessumPassivization}. However, if no possessor intervenes between the (\textit{oko}-marked) possessum and the verb, passivization is possible as long as the differential marker \textit{oko}- is stranded and encliticizes to the verb. This is illustrated in \REF{ex:schneider:OKOPossessumPassivizationNoPossessor} versus \REF{ex:schneider:NoOKOPossessumPassivizationWithPossessor}:

\ea \label{ex:schneider:ExternalPossession}
\begin{xlist}

\ex \label{ex:schneider:OKOPossessumFollowsPossessor}
[\textsc{possessor} – \textsubscript{oko}\textsc{possessum}]\\
\gll m\'o-na-lak-ír-y-e [ Kámbal\'e ] y' [ \textbf{ok\'o} níndo ].  \\
\textsc{aff-1sg}-hit-\textsc{tam-trans} {} Kambale {} \textsc{lk} {} \textsc{17loc} 9nose {} \\
\glt `I punched Kambale in the nose.' 

\ex \label{ex:schneider:OKOPossessumCannotPrecedePossessor}
[\textsubscript{oko}\textsc{possessum} – \textsc{possessor}]\\
\gll *m\'o-na-lak-ír-y-e [ \textbf{ok\'o} níndo ] ko [ Kámbal\'e ].  \\
\textsc{aff-1sg}-hit-\textsc{tam-trans} {} \textsc{17loc} 9nose {} \textsc{lk} {} Kambale. {}  \\

\end{xlist}

\ex \label{ex:schneider:ExternalPossessionPassivization}
\begin{xlist}

\ex \label{ex:schneider:PossessorPassivization}
[\textsc{possessor}\textsubscript{j} {  } \textsc{V-pass} {  } \underline{\hspace{0.5cm}}\textsubscript{j} {  } \textsubscript{oko}\textsc{possessum}] \\
\gll Kámbal\'e\textsubscript{j} a-lak-i-báwa  \underline{\hspace{0.5cm}}\textsubscript{j} \textbf{ok\'o} níndo.  \\
Kambale \textsc{3sg-}hit-\textsc{trans-pass} {   } \textsc{17loc} 9nose\\
\glt `Kambale was hit on the nose.' 

\ex \label{ex:schneider:NoOKOPossessumPassivization}
[*\textsc{possessum}\textsubscript{j} {  } \textsc{v-pass}-ko {  } \textsc{possessor} {  } \underline{\hspace{0.5cm}}\textsubscript{j} ] \\
\gll * \'Eníndo\textsubscript{j} y-a-lak-i-báwá-\textbf{ko} Kámbale\\
{} 9nose \textsc{9-tam-}hit-\textsc{trans-pass-ko} Kambale \\

\ex \label{ex:schneider:OKOPossessumPassivizationNoPossessor}
[\textsc{possessum}\textsubscript{j} {  } \textsc{v-pass}-ko {  } \underline{\hspace{0.5cm}}\textsubscript{j} ]\\
\gll \'Eníndo y-a-lak-i-báwá-\textbf{k\^o}. \\
9nose \textsc{9-tam-}hit-\textsc{trans-pass-ko} \\
\glt `(His) nose was hit.' 

\ex \label{ex:schneider:NoOKOPossessumPassivizationWithPossessor}
[{*}\textsubscript{oko}\textsc{possessum}\textsubscript{j} {  } \textsc{v-pass} {  } \underline{\hspace{0.5cm}}\textsubscript{j} ] \\
\gll {*}\textbf{ok\'o} níndo y-a-lak-i-báwá. \\
\textsc{17loc} 9nose \textsc{9-tam-}hit-\textsc{trans-pass} \\
\glt `(His) nose was hit.'  

\end{xlist}

\z

\subsubsection{Part/whole and alienable possession}\label{sec:schneider:subsection4.3.2}

Recall that part/whole and alienable possession are distinct from inalienable possession in that the possessum is not differentially marked when this type of possession is involved. The possessor, in contrast, is marked by \textit{oko}- in these constructions but that is not relevant here as we are considering syntactic differences between a differentially marked possessum in external possession and a non-differentially marked possessum. A non-differentially marked possessum (the part in relation to the whole in the examples below) has freedom of word order within the verb phrase (as seen in \ref{ex:schneider:OKOPWholePart} and \ref{ex:schneider:PartOKOWhole}) and it can undergo passivization even in the presence of the intervening nominal corresponding to the whole, as in \REF{ex:schneider:PartPass}. The nominal corresponding to the whole is preceded by \textit{oko}, but unlike the differentially oko-marked inalienable possessum, does not have to strand \textit{oko} to undergo passivization in \REF{ex:schneider:OKOWholePass}, although this is optionally a possibility, as shown in \REF{ex:schneider:WholeVPassko}, 

\ea \label{ex:schneider:PartWhole}
\begin{xlist}

\ex \label{ex:schneider:OKOPWholePart}
[\textsubscript{oko}\textsc{whole {  } part}] \\
\gll m\'o-na-tuláng-ire [ \textbf{\textbf{oko}} nyúngú ] kw' [ omúk\'ono ].  \\
\textsc{aff-1sg}-broke-\textsc{tam} {} \textsc{17loc} 9pot {} \textsc{lk} {} 3handle {} \\
\glt `I broke the vessel's handle.'

\ex \label{ex:schneider:PartOKOWhole}
[\textsc{part}\textsubscript{j} {  } \textsubscript{oko}\textsc{whole} {  } \underline{\hspace{0.5cm}}\textsubscript{j}] \\
\gll m\'o-na-tuláng-ire [ omúk\'ono ] w' [ \textbf{oko} nyúngú ].  \\
\textsc{aff-1sg}-broke-\textsc{tam} {} 3handle {} \textsc{lk} {} \textsc{17loc} 9pot. {}  \\
\glt `I broke the vessel's handle.'

\end{xlist}

\ex \label{ex:schneider:PartWholePass}
\begin{xlist}

\ex \label{ex:schneider:PartPass}
[\textsc{part}\textsubscript{j} {  } \textsc{v-pass} {  } \textsubscript{oko}\textsc{whole} {  } \underline{\hspace{0.5cm}}\textsubscript{j}] \\
\gll \'Ebíring\'o by-\'eri-báwa \textbf{ok\'o} mútoka.  \\
8wheel 8-cleaned-\textsc{pass} \textsc{17loc} 3car\\
\glt `The wheels were cleaned on the car.' 

\ex \label{ex:schneider:OKOWholePass}
[\textsubscript{oko}\textsc{whole}\textsubscript{j} {  } \textsc{v-pass} {  } \underline{\hspace{0.5cm}}\textsubscript{j} {  } \textsc{part} ] \\
\gll \textbf{Ok\'o} mútoka kw-\'eri-báwa \'ebíring\'o  .  \\
\textsc{17loc} 3car 17-cleaned-\textsc{pass} 8wheel \\
\glt `On the car was cleaned the wheels.' 

\ex \label{ex:schneider:WholeVPassko}
[\textsc{whole}\textsubscript{j} {  } \textsc{v-pass}-ko {  } \underline{\hspace{0.5cm}}\textsubscript{j} {  } \textsc{part} ] \\
\gll Omútoka \'eri-báwa-\textbf{ko} \'ebíring\'o .  \\
3car 3.clean-\textsc{pass}-\textsc{ko} 8wheel \\
\glt Literally: `The car was cleaned on the wheels.' \\
`On the car was cleaned the wheels.'

\end{xlist}

\z

\noindent Alienable possessa, which also do not undergo differential marking, behave just like part/whole possessa. \\
\indent The data we have considered in this section clearly establish that in external possession, there is a difference in syntactic behavior when differentially marked possessa are compared to unmarked ones. Therefore, differential marking in this case cannot be a morphological phenomenon. We note for completeness that possessors also appear to be differentially marked. This will be addressed in the next section. 


\section{Differential marking in syntax}\label{sec:schneider:section5}

Since we see syntactic effects related to differential marking in all the cases we have considered, we have evidence that differential marking is a syntactic rather than morphological phenomenon in Kinande. In this section we show that each of the three constructions needs a separate type of syntactic account. Before proceeding with the discussion, we make an important clarification. As is well known, a very prominent account for differential (object) marking connects the special morphology with movement. It might appear that the type of Kinande differential marking found in predicative possession structures, discussed in \sectref{sec:schneider:subsection4.2} and (and further in \sectref{sec:schneider:subsection5.1Possessionsyntax} below), is dependent on movement. We noticed there that the \textit{oko}-marked argument exhibits higher word order flexibility than the bare nominal, which must be adjacent to V (apparently signalling a type of incorporation).\footnote{Another observation is that, in these contexts, differentially marked arguments and non-differentially marked arguments cannot be co-ordinated, indicating that they probably occupy different positions.} However, the data must be qualified and, in fact reviewed with attention. There is raising within the verb phrase in Kinande across all structures we studied, including predicative possession involving possession and ownership, whenever VP contains more than one XP. This raising is to the specifier of Linker Phrase, a functional projection that is below \textit{v}P and above VP or any applied phrase that were to occur. It is very clear that either all nominals can equally target the specifier of this phrase (signaled by the linker, head of this phrase, immediately following the raised phrase) and thus raise equally high, or in some circumstances differentially-marked expressions cannot raise. Therefore, much more needs to be said about differential marking beyond raising.

\subsection{Possession vs ownership: A (Case) licensing approach to differential marking with predicative possession in Kinande}\label{sec:schneider:subsection5.1Possessionsyntax}\label{subsec:schneider:subsection5.1LicensingAccount}

Given that raising per se is not sufficient, we propose an account which connects differential marking to those objects that undergo some types of (Case) licensing in the syntax. We start with predicative possession. Recall that Kinande distinguishes possession from ownership with a possessum being marked by \textit{oko}- when possession, rather than ownership, is involved. In addition, we followed \citeauthor{SchneiderZiogaMutaka2019a} (\citeyear{SchneiderZiogaMutaka2019a}) in analyzing possession in Kinande as involving a small clause structure (\sectref{sec:schneider:subsection4.2}), with the possessum as subject and a silent predicate meaning ‘with [possessor]'. Furthermore, we noted that definiteness effects are evident in Kinande when predicative possession conveys the meaning of ownership but are absent when possession is conveyed.\footnote{That predicative possession displays definiteness effects is well-known in the literature (\citeauthor{Partee1999} \citeyear{Partee1999}, and others). \citeauthor{Myler2016} (\citeyear{Myler2016}) notes a division between ownership and possession with respect to definiteness effects in the languages he studies, similar to what we observe for Kinande.}
A leading analysis of predicative possession is that it contains modal existential predicates. As widely discussed in the literature, such structures introduce ambiguity in the sense that existential predicates allow the presence of complements with a predicate nature, with which they can combine via predicate modification. This entails the need to disambiguate between existential readings of complement nominals and non-existential readings in predicative possession \citeauthor{Lopez2012} (\citeyear{Lopez2012}). The existential readings, normally connected to non-specificity, are composed when the nominals in complement position are predicates of type $\langle\text{e, t}\rangle$ and will be interpreted under existential closure (\citeauthor{Diesing1992} \citeyear{Diesing1992}). This type of licensing is illustrated in \figref{ex:schneider:TreeIncorporp}. The fact that nominals that are so interpreted must occur adjacent to the verb, as discussed in \sectref{sec:schneider:subsection4.2}, indicates that their licensing involves \mbox{(pseudo-)}incorporation.

The non-existential readings correspond to differential marking of the nominals. In these cases, nominals are arguments of type \textit{e}, contain structure that needs licensing, and must escape existential closure. The differentially marked possessum is the subject of a small clause, as previously discussed. It cannot be licensed via (pseudo) incorporation. It has been known at least since \citeauthor{Baker1988} (\citeyear{Baker1988}), that incorporation of subjects ``downward'' into the predicates that thematically introduce them is ungrammatical. Furthermore, for semantic reasons, the silent ‘with [possessor]’ secondary predicate must itself incorporate into the primary predicate. It is not possible for the possessum subject of the possession small clause to also incorporate into the primary predicate. Therefore, licensing by (pseudo-)incorporation is unavailable. In this situation, differential marking can license the expression, which can be a KP, via a VP external licensing mechanism, for example a functional projection which we indicate here as $\alpha$\textsuperscript{0} (\figref{ex:schneider:LicensingAlpha}). Investigation of the exact nature of this ancillary licensing mechanism lies outside the scope of this work. We conjecture that $\alpha$ is related to \citeauthor{Halpert2015}'s (\citeyear{Halpert2015}) VP external nominal licensing projection for whose existence in Zulu she provided extensive argumentation.

\begin{figure}
\begin{subfigure}[b]{.4\linewidth}\centering
\Tree 
 [....\textit{v} {\textit{v} } [.VP {\exists } [.VP
 {VP} [.NP\textsubscript{$\langle\text{e, t}\rangle$} ] ] ] ]
\caption{Licensing via adjacency to the verb\label{ex:schneider:TreeIncorporp}}
\end{subfigure}\begin{subfigure}[b]{.6\linewidth}\centering
\Tree 
[.$\alpha$P {KP\textsubscript{e}} [.$\alpha$ {$\alpha$} [.\textit{v} {\textit{v}} [.VP {V} [.PredP {$\langle\text{KP\textsubscript{e}}\rangle$} [.Pred
 {Pred} [.with\textsubscript{Poss} ] ] ] ] ] ] ]
\caption{Licensing via a VP external mechanism\label{ex:schneider:LicensingAlpha}}
\end{subfigure}
\caption{Nominal licensing strategies}
\end{figure}

\subsection{External possession}\label{sec:schneider:subsection5.2StructuresExternalPossession}

A second pattern of differential marking is observed in external possession contexts. As shown earlier, we have two classes here: (i) the \textit{oko}-marked inalienable possessum (example \ref{ex:schneider:illustrateexternal}, repeated in \ref{ex:schneider:illustrateexternal2}), and (ii) the non-\textit{oko}-marked alienable (and part/whole) possessum (example \ref{ex:schneider:raisingalienable}, repeated in \ref{ex:schneider:raisingalienable2}). 


\ea \label{ex:schneider:illustrateexternal2}
\begin{xlist}

Possessor raising (inalienable):\\
\gll Ná-kúrugut-ta [ Sáráh ] y' [ *(\textbf{oko-})mug\'ongo ] \\
\textsc{1sg}-scrubbed-\textsc{fv} {} Sarah {} \textsc{lk'} {} \textsc{17loc}-3back {} \\
\glt  `I scrubbed Sarah's back.' (lit: I scrubbed Sarah on the back.)

\end{xlist}

\ex \label{ex:schneider:raisingalienable2}
\begin{xlist}

Possessor raising (alienable):\\
\gll Ná-mat-ul-a [ *(\textbf{oko-})Sáráh ] kw' [ ezípe ]\\
\textsc{1sg}-fasten-\textsc{rev-fv} {} \textsc{17loc-}Sarah {} \textsc{lk} {} 9zipper {} \\
\glt  `I unfastened Sarah's zipper.'\\
(she is necessarily wearing the thing with the zipper)

\end{xlist}
\z

There are significant differences between the two classes of external possession. We consider first inalienable possession and recall that in predicative possession, inalienable possessa do not receive any differential marking (see \ref{ex:schneider:inalienable}). The possessa are licensed in that construction via (pseudo) incorporation into the verb. With external possession, we assume a POSS projection whose head, POSS, takes a KP possessum, which corresponds to the inalienable possessum, as illustrated in \figref{ex:schneider:InalienablePossTree}. There is no local predicate into which the inalienable possessum can incorporate for licensing purposes. Here, we hypothesize a projection $\alpha$, external to VP, which licenses the KP possessum. As pointed out above in \sectref{subsec:schneider:subsection5.1LicensingAccount}, the $\alpha$ projection appears to be related to \citeauthor{Halpert2015}'s (\citeyear{Halpert2015}) VP external nominal licensing projection, as it has very nearly the same effect, although a detailed exploration of the licensing process lies outside the scope of this current work.

We next consider alienable or part/whole possession. We observe that in external possession constructions, it is the case that the part or alienable possessum is necessarily understood as actually being located on the possessor in cases of part/whole or alienable possession (see, for example, \ref{ex:schneider:raisingalienable2}). We capture this interpretive fact by analyzing alienable – part/whole external possession as involving locative possession. Locative predicative possession does not use the usual \textit{have} verb one finds in Kinande but instead the \textit{be} copula is used and the external possessor, which is thematically locative, is morphologically marked by the locative \textit{oko} marker (see \citeauthor{SchneiderZiogaMutaka2019a} \citeyear{SchneiderZiogaMutaka2019a}). Locative predicative possession means approximately: “on X is Y”, where X is the whole/possessor of something alienable. Here are illustrative examples:\footnote{Note that in locative predicative possession, the \textit{oko}- locative class marker is ``stranded'' in post copular position. We take this property to indicate that the locative possessor was first merged below the copula, presumably in a locative phrase. }  

\ea \label{ex:schneider:Alienable}
\begin{xlist}

\ex \label{ex:schneider:Alienable1}
\gll omúti a-\textbf{n\'e}- \textbf{kw'} ehinyúnyu.  \\
3tree 3-be(\textsc{assert-}) \textsc{17loc} 19bird\\
\glt `The tree has birds (on it).' 

\ex \label{ex:schneider:Alienable2}
\gll etsúkudu  yi-\textbf{rí-ko} ebíring\'o.\\
9cart 9-be-\textsc{17loc} 8wheel \\
\glt `The cart has wheels (on it).' 

\end{xlist}
\z

Consonant with this, we propose that alienable and part/whole external possession involves a locative phrase, identified here as LOC, which takes a DP complement, and whose specifier contains a locative possessor, which is \textit{oko}- marked for thematic reasons. 

The two constructions, which correspond to inalienable and part\slash whole-alien\-able possession respectively, are given in Figures~\ref{ex:schneider:InalienablePossTree} and~\ref{ex:schneider:AlienablePossTree}.\largerpage

\begin{figure}
\caption{Inalienable possession\label{ex:schneider:InalienablePossTree}}
\Tree 
 [. ...$\alpha$ {$\alpha$} [.\textit{v}P {\textit{v}} [.LkP { } [.Lk' {Lk} [.VP {V} [.POSS {$\langle\text{DP\textsubscript{\textit{possessor}}}\rangle$} [.POSS {POSS} [.{$\langle\text{KP\textsubscript{\textit{possessum}}}\rangle$}
 {K\textsubscript{oko-}} [.DP ] ] ] ] ] ] ] ] ]
\end{figure}

\begin{figure}
\caption{Part/whole-alienable possession\label{ex:schneider:AlienablePossTree}}
\Tree 
[. ...\textit{v}P {\textit{v}} [.LkP {} [.Lk' {Lk} [.VP {V} [.LOC {\textit{oko-}DP\textsubscript{\textit{possessor}}} [.LOC
 {LOC} [.DP ] ] ] ] ] ] ]
\end{figure}


We suggest that the syntactic differences we see between these two types of external possession constructions with respect to movement within the verb phrase and passivization follow from the fact that in one construction, the possessum requires special licensing via \textit{oko}-marking, whereas in the other, this is not the case. The special licensing behaves very much like structural Case licensing in that the so-licensed nominal is frozen in the position where it was first licensed, much like an accusative Case-marked object, which cannot move to nominative subject position once it has been licensed. Following the Case analysis of Halpert and extending it to Kinande, non-differentially marked DPs are always able to receive licensing via the prefixing of an augment, which Halpert argues can self-license DPs in the Bantu languages she studied. Such nominals are essentially unrestricted in their distribution as the self-licensing carried out by the augment is independent of the position the so-licensed nominal occupies, in contrast to the licensing of the inalienable possessum KP which is marked by \textit{oko}-. Therefore, the inalienable possessum KP cannot reorder with the verb phrase as illustrated in \REF{ex:schneider:ExternalPossession} by targeting the specifier of linker phrase, assuming this is Case related movement.

In contrast, the DP possessum found in locative external possession has no problem targeting the specifier of linker phrase position (\ref{ex:schneider:OKOPWholePart} and  \ref{ex:schneider:PartOKOWhole}) as Case is available through the possibility of self-licensing. The possibility of passivization is also affected due to differential marking. The data in (\ref{ex:schneider:PossessorPassivization}--\ref{ex:schneider:NoOKOPossessumPassivizationWithPossessor}) indicated that passivization of an inalienable possessum is only possible if differential \textit{oko}- has cliticized or reanalyzed into the verb under adjacency. Under that condition, the nominal is free of its differential marking and is therefore able to move to the subject position. The intuition here is that the cliticization of \textit{oko}- to the passive verb prevents the possibility of licensing (by the $\alpha$ projection), through differential marking. This is reminiscent of early accounts in the clitic-doubling literature of the occurrence of a pronominal clitic in the extended verbal complex correlating with an unavailability of Case for the argument associated with the clitic. More specifically, the clitic was seen as “absorbing the Case assigning ability of the verb”. 
The differentially marked specifier of locative external possession, we analyze as a thematically and morphologically locative phrase that does not require any special licensing beyond the licensing needed by locative phrases in general.

Our understanding of the licensing needs of locative phrases in Kinande is in its early stages. We do observe that they have a broader distribution than differentially marked locatives, however. The examples involving passivization in (\ref{ex:schneider:PartPass}--\ref{ex:schneider:WholeVPassko}) illustrate that, when reflective of a thematic locative, the \textit{oko}- marker can optionally cliticize to the passive verb. We do not yet have an account of why this optionality is possible. We do however note that the fact that passivization of an entire locative phrase, without cliticizing the locative marker, is consistent with our proposal that locative morphology in part/whole-alienable possession is not for Case licensing of the locative marked nominal, but instead is necessitated by the meaning of the construction. 

\subsection{Animate goal/source licensing}\label{sec:schneider:subsection5.3AnimateGoalSourceLicensing}

Kinande shows yet a third type of differential marking. Higher animates that are sources or goals occur without an augment under certain syntactic conditions we will examine in this section. This means they will require require \textit{uku}- marking,\footnote{We use \textit{uku- }to stand for any class locative marker that indicates the nominal it marks is unaugmented.} instead of the expected \textit{oko-} in these contexts, as exemplified in \REF{ex:schneider:ukuandoko} and \REF{ex:schneider:noadverb}. However, the augmentless licensing possibility only holds when the goal or source is adjacent to the verb. There is an additional structural requirement that becomes relevant when the adjacency licensing mechanism is considered in more detail, as we will discuss shortly. Furthermore, recall that sources and goals that are lower in animacy have no such licensing mechanism available, as seen in \REF{ex:schneider:noadverbinanim} and \REF{ex:schneider:adverbinanim}.\\
\indent The question is what could explain the apparently ``augmentless'' nature of these higher animate expressions. We propose that this type of differential marking can be best explained as involving a licensing strategy beyond Case per se. Following \citeauthor{Irimia2018} (\citeyear{Irimia2018}, \citeyear{IrimiaToAppear}), this additional strategy is needed to license a \textsc{[person]} specification (\citeauthor{Richards2008} \citeyear{Richards2008}, a.o.) added to certain classes such as higher animates.\footnote{The intuition here is that the additional feature \textsc{[person]} requires a distinct type of licensing (licensing relativized to discourse). See also \citeauthor{Miyagawa2017} (\citeyear{Miyagawa2017}) for related discussion. } In these configurations, there are two features that need licensing – the uninterpretable Case feature and the additional \textsc{[person]}. \citeauthor{VanUrk2020} (\citeyear{VanUrk2020}) makes a similar proposal based on his investigation of an adjacency licensing mechanism that is relevant to higher animates (names and pronouns) in Fijian. Whether the actual relevant feature here is \textsc{[person]} or not requires additional research. It suffices for our purposes that some animacy-related feature plays a role in this differential marking. \\
\indent We note that higher animates without augments are syntactic predicates\footnote{Non-verbal predication in Kinande also requires the predicate to surface without an augment, as discussed in \citeauthor{Progovac1993} (\citeyear{Progovac1993}) and \citeauthor{Schneider-Zioga2018} (\citeyear{Schneider-Zioga2018}). } and as predicates can and do incorporate into the verbal complex. This incorporation essentially takes place under adjacency. Recall the data in \REF{ex:schneider:examplesadjacency}, repeated here as \REF{ex:schneider:examplesadjacencyRepeated}, which illustrate this generalization:

\ea \label{ex:schneider:examplesadjacencyRepeated}
\begin{xlist}

\ex \label{ex:schneider:adjacentRepeated}
\ldots verb] enclitic] oko/*uku\\
\gll N-ibá-ky' \textbf{oko/*uku} Mary\^a. \\
\textsc{1sg}-steal-\textsc{7cl} \textsc{17loc/(-aug)17loc} Marya \\
\glt `I stole it from Marya.' 

\ex \label{ex:schneider:procliticverbRepeated}
\ldots procl\textsubscript{j}-verb \underline{\hspace{0.5cm}}\textsubscript{j} uku/*oko \\
\gll Ná-ki-iba \textbf{*oko/\Checkmark uku} Mary\^a. \\
\textsc{1sg}-\textsc{7om}-steal \textsc{17loc/(-aug)17loc} Marya \\
\glt `I stole it from Marya.'
 
\ex \label{ex:schneider:whinterveningRepeated}
\ldots\textsc{wh}\textsubscript{j}-verb] \underline{\hspace{0.5cm}}\textsubscript{j} uku/*oko \\
\gll \'Ekihi\textsubscript{j} ky\'o w-ibá \underline{\hspace{0.5cm}}\textsubscript{j} \textbf{*oko/\Checkmark uku} Mary\^a. \\
7what \textsc{7foc} \textsc{2sg-}steal {  } \textsc{17loc/(-aug)17loc} Marya \\
\glt `What did you steal from Marya?' 


\end{xlist}

\z

As long as an expression which encodes higher animacy can incorporate into the verbal complex, it must - this appears to be the only way for its higher animacy feature(s) to be licensed. However, if it is too far away from the verbal complex (by which we mean not string adjacent), it does not incorporate. This tells us that there is an additional way for the higher animate goal/source to be licensed when licensing by adjacency is not possible. \\
\indent There appears to be only one possible position available for incorporation as illustrated in \REF{ex:schneider:adjacentRepeated}, where we see that the presence of an enclitic prevents the incorporation of the higher animate; but if nothing overt intervenes between the verb and the higher animate goal/source, as in the case of the object prefix preceding the verb stem \REF{ex:schneider:procliticverbRepeated} or the silent extraction site between the verb and the higher animate in \REF{ex:schneider:whinterveningRepeated}, the incorporation licensing mechanism comes into play. Recall that non-higher animates (common nouns of various types) in the same configurations, as in \REF{ex:schneider:noadverbinanim}, are not unaugmented and therefore cannot undergo incorporation. \\
\indent There is a more complex environment where adjacency is at play. It involves a configuration where a higher animate goal/source is in a specifier position within the larger \textit{v}P domain. This happens when there is a second XP within the verb phrase as in \REF{ex:schneider:adverb}, which we repeat here, with bracketing, as \REF{ex:schneider:adverbRepeated}:

\ea
 \label{ex:schneider:adverbRepeated}
\gll Omundú mw-á-híka    [\textbf{uku}/\textbf{\textsuperscript{?}oko} Y\'esú [ k' omotututu ]]. \\
1person  \textsc{aff-3sg-}arrive (-\textsc{aug})17\textsc{loc}/17\textsc{loc} Jesus {} 17\textsc{lk'} 18morning {} \\
\glt `Someone came to Jesus in the morning.' 
\z

Note that the higher animate here occupies the specifier of the linker phrase and is indeed string adjacent to the verb. However, although the putative \textsc{[person]} feature can be licensed via adjacency (incorporation into the verb), it is evident that it can also be licensed by occupying the specifier position within the verb phrase. Therefore, the adjacency licensing mechanism is optional in this case. We are in a preliminary stage of working out the exact details of this licensing mechanism and the full range of structures within the verb phrase and how they interact with the higher animacy feature, which we hypothesize is \textsc{[person]}; so open questions remain. However, we have succeeded in establishing that there is a distinction in licensing based on such a feature and have demonstrated that there is an additional mechanism at play here, namely one based on licensing via verbal+nominal adjacency. 

\section{Conclusions}\label{sec:schneider:section6Conclusion}

In this paper we have presented novel data involving dependent differential (object) marking in Kinande. We have supported several important conclusions. On the one hand, differential marking of dependents of the verb in Kinande shows non-trivial syntactic correlates, supporting a syntactic analysis of differential marking. On the other hand, we have also demonstrated that Kinande employs more than one mechanism for differential marking, all of which are consistent with various analyses proposed in the theoretical literature. We have shown that a movement-based approach is hard to extend to the types of differential marking discussed here. Instead, the problem reduces either to the presence of a Case feature which needs licensing (predicative and inalienable possession) or an additional \textsc{[person]} feature beyond Case per se (unaugmented higher animates). We have left a number of issues related to differential marking for future work but have certainly established that differential marking on dependent nominals is robust in at least one Bantu language. Finally, we have endeavored to establish that the study of differential marking offers new avenues of investigating the issue of nominal licensing in Bantu languages. 

\section*{Abbreviations}

Below are listed only those abbreviations that do not adhere to or are beyond the scope of the Leipzig Glossing Rules.

\begin{multicols}{2}
    \begin{tabbing}
        \textsc{assoc}+ \=differential object marking\kill
        \textsc{aff}   \>affirmative \\
        \textsc{anim}  \>animate \\
        \textsc{appl}  \>applied \\
        \textsc{assoc} \>associative \\
        \textsc{aug}   \>augmented \\
        \textsc{cl}    \>clitic \\
        \textsc{dir}   \>directional\\
        \textsc{dom}   \>differential object marking\\
        \textsc{fv}    \>final vowel\\
        \textsc{lk}    \>linker\\
        \textsc{nc}    \>noun class\\
        \textsc{om}    \>object marker\\
        \textsc{rev}   \>reversive\\
        \textsc{tam}   \>tense-aspect-mood\\
        \textsc{trans} \>transitivizer\\
        \textsc{ver}   \>verum focus\\
        1, 2, 3        \>noun classes
    \end{tabbing}
\end{multicols}


\section*{Acknowledgements}

We are grateful to have this opportunity to thank Philip Ngessimo Mutaka for his comments, insights, and endless patience in discussing all the Kinande data. This paper also benefited greatly from remarks by an anonymous reviewer and the patience and generosity of the editor of this volume. We sincerely appreciate the insightful comments from and illuminating discussions with Claire Halpert, Nancy Hedberg, Larry Hyman, and the participants of the 49th Annual Conference of African Linguistics held at the University of British Columbia. We also express our gratitude to the Creteling summer school, held annually since 2017 at the University of Crete in Rethymnon, Crete which provided the co-authors with the opportunity to mutually explore differential marking in Kinande and some of the related issues. Parts of this work have been supported by a research grant from the University of Modena and Reggio Emilia awarded to Monica Alexandrina Irimia. 

{\sloppy\printbibliography[heading=subbibliography,notkeyword=this]}

\end{document}
