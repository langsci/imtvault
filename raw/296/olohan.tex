\documentclass[output=paper,colorlinks,citecolor=brown]{langsci/langscibook}
\ChapterDOI{10.5281/zenodo.4450077}
\author{Maeve Olohan\affiliation{University of Manchester}}
\title{Post-editing: a genealogical perspective on translation practice}
\abstract{This paper develops a practice-theoretical conceptualization of post-editing, as an activity that increasingly forms part of translation practice. This contrasts with a prevailing conceptualization of post-editing as a practice in its own right, competing with or complementing translation practice. Adopting a genealogical perspective, I trace this particular evolution of the translation practice through some of the interdependent changes in the materials constituting the practice, the competences or know-how that transpire in the practice, and the meanings of the practice, in particular as constructed through the discourse of language service providers and the international standards that normatively regulate the practice. The paper concludes with some implications of this practice-theoretical approach for future research on post-editing.}

\begin{document}

\maketitle
\section{Introduction}

Machine translation (MT) is increasingly deployed by language service providers (LSPs) and translators. This means that some professional translators work more often with machine translation outputs, editing them to make them fit for purpose, an activity known as post-editing (henceforth PEMT but also referred to elsewhere as MTPE). 

In this paper I develop a practice-theoretical conceptualization of PEMT as one of several activities that make up the changing practice of translating, alongside other activities such as editing fuzzy matches retrieved from translation memory (TM). This contrasts with a view of PEMT as a practice in its own right that competes with or complements the practice of translating. I examine how PEMT is reconfiguring translation practice, through changes in constituent elements of the practice, including the tools and materials deployed, the competences and knowing that transpire in practice, and various understandings of the practice. By exploring this reconfiguration, we extend our genealogical understanding of translation, as a practice that changes over time. 
The account of translation practice put forward in this paper has been abstracted from my own workplace observations in several LSPs and formal and informal interviews with translators, translation project managers and LSP managers. Practitioner performances and articulations from particular sited practices are not further elaborated on here; rather, those understandings are theorized and contextualized through relevant practice-theoretical and translation studies scholarship to conceptualize a particular trajectory of the practice of translation.

The paper first outlines the contours of a practice-theoretical understanding of translation. It then examines the reconfigured practice in more detail, focusing on how PEMT-related changes in the practice are linked to changes in material elements, know-how and meanings participating in it. To conclude, I reflect on how empirical research can shed further light on PEMT activity in translation, by employing methods typical of ethnographic research but also complementing them by other methods that may be productive in studying ongoing changes in translation practice. 

\section{Understanding translation as a practice}

Practice theory refers to a range of theoretical approaches to the study of social practices, as elaborated by \citet{schatzki_social_1996,schatzki_site_2002}, \citet{reckwitz_toward_2002}, \citet{shove_dynamics_2012}, \citet{nicolini_practice_2012} and \citet{warde_practice_2016}, among others. Although there is some divergence in thinking and terminology, these contributions share some specific priorities and concerns. Crucially they place practices at the centre of their conceptualizations and analyses, conceiving the social world as a plenum of practices in which practices hang together in constellations or complexes \citep{schatzki_practice_2016}. This focus on practice entails key ontological and methodological shifts \citep{brauchler_introduction_2010,reckwitz_toward_2002}, moving away from research that seeks to explain social phenomena through individual actions, informed by rational choice theory. Similarly, a focus on practices also entails a shift away from systems-oriented thinking that looks for explanations in social systems and norms. \citet[250]{reckwitz_toward_2002} cautions against trivializing practice theory because much of its terminology appears to resemble our everyday descriptions of human behaviour. Terms such as \textit{activity}, \textit{material}, \textit{competence}, \textit{knowing}, \textit{meaning}, \textit{practical understanding}, \textit{general understanding}, \textit{rules} and, of course, \textit{practice} are used in those more technical senses below.

Definitions of practice vary. One of the most influential is Schatzki's \citeyearpar[2]{schatzki_introduction:_2001} depiction of practices as \enquote{embodied, materially mediated arrays of human activity centrally organized around shared practical understanding}. This definition emphasizes the human-centred nature of the activities but also the situated, embodied performances by practitioners. The activities are bodily \enquote{doings and sayings} that unfold in time and space \citep[89]{schatzki_social_1996}. The activities are mediated by materials, i.e., non-human entities, whether physical, biological, chemical or artefactual. Moreover, the activities are organized around shared practical understanding, i.e., knowing how to identify the doings and sayings that make up a practice and knowing how to carry them out. In addition to practical understanding, three other kinds of practice-organizing elements are proposed \citep{schatzki_site_2002}, namely general understandings, rules and teleo-affective structures. 

General understandings refer to the general sense or ethos of a practice, otherwise described as the \enquote{senses of worth, value, nature or place of things, which infuse and are expressed in people’s doings and sayings} \citep[16]{schatzki_primer_2012}. For example, in \citet[76--77]{olohan_translation_2021} I draw on corpus data to show that LSPs, in their promotional discourse, seek to convey a general sense of translating as being in the service of globalized trade. 

For Schatzki, rules are the explicitly normative formulations that are found in regulatory or legislative frameworks. Rules of this kind figure less in the organization of translating practice than in other professional practices, e.g., medicine or chartered accounting, although other mechanisms conveying what is considered an acceptable performance of translating include tests, accreditations, prizes, client feedback, etc. 

Finally, the teleo-affective structure brings together the teleological and the affective dimensions of practice organization. The first refers to the ends, projects and tasks that hierarchically order the activity \citep[80]{schatzki_site_2002}. These ends, projects and tasks function normatively, in the sense that they are the ends, projects and tasks that participants ought to realize as they produce what can be considered an acceptable performance of the practice. Practitioners complete tasks, which are part of projects, which serve ends. Freelance translating in different settings may pursue a number of ends, ranging, for example, from earning a living to disseminating a particular cultural product or to supporting a humanitarian cause. The affective dimension refers to the emotions and moods that are permitted, encouraged or considered acceptable or obligatory for participants to exhibit when carrying out a practice. Translating practice is generally not strongly ordered by affectivity, so there are often no obvious affect-related expectations; a translating practice may be deemed to be acceptable whether the practitioner is excited, bored or despairing in its performance, for example. 

For analytical purposes, it is often helpful to consider different constituent elements of practice in turn, as will be done in the next sections, but it is important to emphasize the interconnectedness of those elements, without which the practice would not exist. Reckwitz’s \citeyearpar[249]{reckwitz_toward_2002} definition of a practice helpfully highlights the interconnections between elements that produce a \enquote{routinized form of behaviour} consisting of \enquote{forms of bodily activities, forms of mental activities, \enquote{things} and their use, a background knowledge in the form of understanding, know-how, states of emotion and motivational knowledge}. A key consequence of this thinking is that a practice cannot be reduced to any single element \citep[249]{reckwitz_toward_2002}. This has implications for empirical investigations and research methods, as noted in the concluding section of this paper. 

Practice theory has shed light on practices in many domains, from everyday practices of eating \citep{warde_practice_2016} and consuming energy \citep{shove_conceptualizing_2015} to professional practices performed in workplaces such as hospitals \citep{nicolini_practice_2011}, schools \citep{kemmis_ecologies_2012} and engineering and construction sites \citep{buch_studying_2015}, to give just a few examples. Many of these practices have been theorized and empirically investigated by researchers working in sociology, organization studies and consumption studies but practice theory has made its way into many other academic disciplines too, for example, political science \citep{jonas_praxeological_2017} and media studies \citep{brauchler_theorising_2010}. 

Practice scholars pursue a range of research questions but are often interested in investigating the nature of specific practices, how practices emerge and evolve, as well as how they endure or fade \citep{shove_dynamics_2012,schatzki_social_2019}. An understanding of how practices interconnect with and are dependent on other practices is also highly relevant \citep{hui_nexus_2017,spaargaren_practice_2016}. Underlying these investigations is a distinction between practice as performance and practice as entity. Individual performances of a practice occur in specific times and spaces; they are \enquote{continual improvisations} along \enquote{more or less precise or fuzzy parameters} \citep[46]{warde_practice_2016}. The practice entity is the encapsulation or abstraction of what makes the performances recognizable as acceptable performances of the practice. A practice requires repeated performance to endure, and changes in performances may eventually lead to changes in the practice entity. Formulated in another way that is helpful for the purposes of this paper, practices are \enquote{open-ended, spatial-temporal sets of organized doings and sayings} and they can be extended through additional doings and sayings \citep[28]{schatzki_social_2019}. 

The value of theorizing translation as a practice and researching it empirically rests in the holistic perspective, compared with approaches to studying translation that may focus on translation product, cognitive process or practitioner. A practice-theoretical perspective encourages us to consider all of the various elements that make up the practice of translating, including the human body, material entities, know-how and meanings of the practice. In focusing on often overlooked material and embodied elements of practices, as well as the know-how that is enacted in practices, practice theory provides a productive framework for a dynamic and materially aware understanding of translation practice and for an examination of the emergence of new configurations of the practice \citep{olohan_translation_2021}. This paper examines aspects of how translation practice is being reconfigured through the integration of the PEMT activity, by considering the interdependence of the diverse elements that constitute the practice.

\section{The proliferation of MT and post-editing}

Since 2016 the dominant MT model has been neural machine translation (NMT), which displaced the predecessor model of statistical machine translation (SMT). NMT relies on machine learning performed via neural networks, and NMT developers, starting with Google and Microsoft, were quick to claim fairly substantial increases in translation quality for the new approach when compared with SMT \citep{wu_googles_2016}. Unless otherwise stated, MT refers here to NMT.

The concepts of pre-editing and post-editing emerged some decades ago in the era of older, rule-based MT technologies and were applied more often in the context of research systems than in commercial applications. Post-editing (PE), as currently understood, is defined in international standard ISO 18587:2017 as editing and correcting machine translation output \citep{british_standards_institution_iso_2017}. Distinctions between different PE modes will be made below but it is first useful to consider PEMT’s prevalence in today’s language services sector. There are no sector-wide measures of how widely PEMT is being performed in language services but some indicators attest to the ever-increasing deployment of MT by LSPs for their clients. For example, the \textit{Slator 2019 Language Industry Market Report} \citep[16]{faes_slator_2019} notes that MT is \enquote{well on its way to becoming the single most important productivity enhancement technology for human translators}. This report also acknowledges the use of MT by enterprises who thereby forego the intermediary services of LSPs; they refer to the market for \enquote{stand-alone} or \enquote{pure play} MT, described as MT without any human translation services, i.e., raw MT output that is used without post-editing by linguists \citep[16]{faes_slator_2019}.

Raw MT is considered useful for content that would otherwise be too ephem\-er\-al or too voluminous to be commissioned for human translation, while human involvement is usually preferred for the production of high visibility target-language content on which commercial reputations rest. User-generated content (e.g., customer reviews) and customer support are content types for which raw MT is considered acceptable in some situations. These preferences are confirmed by the European Commission’s survey on likely uptake of MT in small and medium enterprises \citep{directorate-general_for_communications_networks_content_and_technology_sme_2020}, where most respondents considered MT to be useful for understanding websites or social media, gathering information about or corresponding with companies or partners, and purchasing and selling products or services, including offering after-sales service. By contrast, respondents expressed a clear preference for human translation for activities relating to negotiating and signing contracts, resolving conflicts in commercial transactions, dealing with public administration in other countries and conducting marketing and promotional activities.

These survey responses of business representatives highlight the interdependence of the practice of risk management and the practice of translation. A concrete example of a company’s concern to avoid reputational damage from potentially low quality raw MT is offered by \citet{schmidtke_automatic_2019} in their account of the deliberations of Microsoft as it sought to introduce raw MT into software localization, having previously published some raw MT in technical and end-user support documentation for Microsoft Office. This is an example of one of three levels of NMT-related risk identified by \citet{canfora_risks_2020}, namely the damage that can be incurred by clients and end users from errors in the MT output. This is seen as an issue for NMT in particular because errors are not readily predictable and the output can resemble a convincing piece of target language discourse. Errors of accuracy can therefore be overlooked by post-editors or revisers. A second level at which risks have to be managed concerns the attribution of liability and accountability when NMT tools are used and damage is incurred. There is no legal clarity on this matter  as yet. While traditional legal notions of misconduct or negligence apply to human behaviour and not AI systems, they could potentially be applied to those who produce, own or use the AI system, thus also possibly extending to post-editors \citep[63]{canfora_risks_2020}. The third level at which risks are incurred and must be managed are those related to data security, a particular problem when NMT is used via free, online, generic MT services \citeyearpar[64]{canfora_risks_2020}.

The use of raw MT has not reduced demand for translations of a specified (high) quality for which human involvement, through PEMT activity, is generally required and expected. A European survey of 298 LSPs and 905 individual translators conducted in 2018 reported that more than half of companies and individuals were using MT in some form \citep{elia_2018_2018}. In a worldwide survey of 7,363 translators and interpreters at the end of 2019, almost all respondents (97\%) provide translation services, 72\% offer editing or proofreading, and PEMT is the next largest service offered, by 35\% of respondents \citep{pielmeier_state_2020}. However, 55\% of respondents report that they use MT, including on projects when the client does not request it, which means they are also post-editing as part of their own translation services. 23\% of the MT users find that they deliver better quality when they use MT, and 52\% say that MT speeds up their work \citep[45]{pielmeier_state_2020}. The aforementioned Slator report asserts that there is an increasing demand for \enquote{professional linguists who can interact with machine translation output}, given that LSPs’ corporate clients are looking for bespoke MT solutions tailored to their content, workflows and preferences \citep[22]{faes_slator_2019}. 

Approaches to PEMT activity are discussed in language services and academic research. The ISO 18587:2017 standard and numerous MT technology providers differentiate between full and light PE. Full PE is the \enquote{process of post-editing to obtain a product comparable to a product obtained by human translation}, while light PE is a \enquote{process of post-editing to obtain a merely comprehensible text without any attempt to produce a product comparable to a product obtained by human translation} \citep[2]{british_standards_institution_iso_2017}. This distinction, using similar or different terminology, is also made by MT promoters and developers, e.g., \citet{taus_mt_2015}, \citet{kantanmt_post-editing_nodate} and \citet{sdl_post-editing_nodate}. However, it may be misleading to suggest that there are two (or more) PE modes that are easily defined and recognized, and in demand in commercial practice, or that translators can easily switch between them. Light PE appears to be much less relevant in practice and, indeed, the ISO standard restricts its detailed prescriptions to full PE. Similarly, formulations that refer to the product of human translation as the aspirational goal of PEMT are common and perhaps understood as a shorthand but may be unhelpful, since they reflect unrealistic notions of all human translation being of invariably appropriate quality.      

Finally, another indicator for the increasing importance of PEMT is the general growth of research on the phenomenon. Much of the earlier research investigated PEMT as performed by students or novices or as a stand-alone activity, typically also in experimental settings. A relatively high proportion of studies also focus on MT research systems and are more concerned with performance or assessment of the technologies rather than PEMT \textit{per se} or as it occurs with commercial systems. However, there is growing interest in studying PEMT in the professional workplace \citep[e.g.,][]{gois_translator2vec:_2019,vardaro_translation_2019,macken_quantifying_2020} and in assessing the acceptability of PEMT for end users in typical usage settings \citep{girletti_preferences_2019}. Accounts of professional deployments of PEMT are also becoming more prevalent in the literature \citep{zaretskaya_optimising_2019,zaretskaya_raising_2019,kosmaczewska_application_2019,premoli_mtpe_2019,nunziatini_machine_2019}.

Having outlined a practice-theoretical framework and having established that PEMT is increasingly deployed in language services, we now examine in more detail the PEMT-related evolution of translating practice. To do this, we trace a selection of the changes in the constituent elements of the practice, namely its materials, competences and meanings, following \citet{shove_dynamics_2012}. Materials include artefacts such as software and hardware, other tools, devices and infrastructures, as well as the human body. Competence refers to practical understanding or know-how. Meanings bring together general understanding, teleo-affectivity and other elements that normatively organize the practice. 

\section{Changing materials}

Material entities of many kinds participate in practices, including humans, organisms, phenomena of nature and artifacts \citep[39]{schatzki_social_2019}. \citet{shove_matters_2017} distinguishes different roles that may be played by material entities in practices, namely as infrastructures, devices and resources. It is beyond the scope of this paper to consider all relevant material aspects of PEMT and translation practice, so we will consider one example of each of these three categories in turn, to illustrate how changes in materials shape changes in the practice.

The first role to be considered for material entities is that of infrastructure. These are understood as things in the background that are necessary for the practice to be performed but are not directly engaged with it \citep{shove_matters_2017}. For translation with or without MT, the infrastructure that is usually necessary for the practice to be performed includes buildings, lighting, heating, electrical power, the Internet and information and communications technologies, among other elements. An infrastructural addition that is specific to the PEMT activity is the NMT engine. As noted above, NMT relies on neural networks, and an NMT engine has been trained and tested on language data, usually in large quantities and for a specific language pair. In addition, it is often customized or fine-tuned by adding further smaller datasets comprising texts from a specific subject domain, in order to improve the quality of outputs when deployed for that domain. 

The rapid advances in NMT and other machine-learning technologies over the past five years are themselves partly attributed to the material changes in computer systems that came with the realization that neural networks can run relatively efficiently on graphical processing units (GPUs). GPUs are the computer processors designed for rendering graphics and games, and it was discovered that they outperform conventional processors (central processing units, or CPUs) for implementing and training neural networks. The capabilities of GPUs and subsequent enhancements, as well as the availability of large datasets for training MT engines, thus enabled significant developments in machine learning, including NMT, which, in turn, are changing the trajectory of the translation practice by reshaping some of the activities that constitute the practice. 

The building of NMT engines and their adaptation to domains is technically complex and beyond the capabilities of most LSPs and individual translators (see \citealt{gupta_improving_2019} and \citealt{silva_improving_2019} for descriptions of some of the processes involved). Thus, the viability of MT deployment for an LSP can be considered in terms of the computational infrastructures required. Some LSPs, like SDL and Tilde, develop MT systems for their own use in their language services businesses and also for sale to other LSPs or translators. However, most LSPs are dependent on buying an NMT service from a specialist provider, either as an off-the-shelf product or as a customized engine that the provider will build, test and perhaps maintain and host on their behalf. As noted by \citet[33]{faes_slator_2019}, increasing commercial deployment of NMT is being driven by some of the global, big tech companies: Microsoft, Google, Facebook, Amazon, IBM, SAP, Salesforce, Alibaba, Baidu, iFlytek and Sogou. These companies have invested very heavily in developing NMT, initially to help them to deliver their core businesses, but some then take advantage of the opportunity to sell the MT technology to smaller companies, either as a stand-alone service or as part of a wider suite of technological applications. MT technologies and services are also being sold to LSPs and linguists by another group of technology companies for which MT is their core business; these include DeepL, KantanMT, Omniscien Technologies, Systran and PROMT, among others. Thus, LSPs are often relieved of the material requirements to purchase and run specific hardware or software or to ensure data security and confidentiality on their own premises. However, the potential success of customized MT engines is dependent on LSPs being able to provide large corpora of source texts and translations for the language pair and subject domain so that the system can be appropriately trained, and they still need to be able to give clients the necessary assurances regarding data security for engines hosted by a third party.

The second role to consider for materials is as devices, i.e., things that are in the foreground of practices and participate directly in them \citep{shove_matters_2017}. Devices that are undergoing material changes as PEMT is integrated into translation practice include the translator’s desktop environment. The most typical deployment of MT is through an application programming interface (API) that connects the NMT service with computer-assisted translation (CAT) tools. Thus, the CAT environment combines resources from MT, translation memory (TM), and terminology management tools, and translators using MT work in their usual editor and follow workflows that are familiar from their non-PEMT practices. Typically, the TM software first retrieves, from its database, full matches (i.e. 100\%) and fuzzy matches (typically 75\% to 99\%) for segments of the source text that formally resemble source text segments already stored in the TM. Then, for those segments of text for which there are no full or fuzzy TM matches, an MT suggestion is generated and inserted into the editor, so that the translator is confronted with suggestions for all segments of text and generally proceeds to post-edit the MT suggestions and edit TM matches to produce a translation of the requisite quality (see \citealt{zaretskaya_optimising_2019}, \citealt{premoli_mtpe_2019} and \citealt{nunziatini_machine_2019} for descriptions of this process as implemented in different LSP settings).

The resources are handled in this way because an assumption is made that a fuzzy TM match is more useful to the translator than an MT suggestion, so the TM takes precedence and the MT is only provided where the TM can offer no assistance. However, as a study at TransPerfect shows \citep{zaretskaya_raising_2019}, when NMT engines are customized for the domain and the quality of the MT suggestions is high, it is desirable to give the MT suggestions priority over fuzzy TM matches. In those cases, as demonstrated for short segments of text (typically 4 to 6 words) in the TransPerfect research, the TM fuzzy matches required more editing than the MT suggestions (as measured by the post-edit distance, PED). 

Although the translation practice still happens in the familiar interface, it is changed materially by the change in quantity and type of data presented to the translator, and the material organisation of that data. ISO 18587:2017, the international standard for post-editing, makes an explicit, material distinction between translation and post-editing by describing PEMT as involving three texts: the source text, the MT output and the final target text, while translation only involves two \citep[5]{british_standards_institution_iso_2017}. In the working environment just outlined, the translator deals not only with MT output but also with TM matches, with some visual differentiation through colour coding and the addition of metadata. 

NMT systems operate on a sentence level and translation suggestions are proposed segment by segment, as is also the case with TM (where segments are typographically delimited and often equate to a sentence, heading, bullet point, etc.). However, as argued in \citet[51--54]{olohan_translation_2021}, since many texts follow a narrative structure, segment-based organisation of TM databases is at odds with the texts' narrative logic. Moreover, the algorithmic nature of NMT is at odds with both database and narrative logic. One manifestation of the MT’s algorithmic logic is its relative lack of transparency compared to TM suggestions. Translators prefer to have some information on provenance and on the nature of TM matches \citep{teixeira_impact_2014,cadwell_resistance_2018}. However, the inner workings of neural networks are inscrutable so it is virtually impossible for translators (or system developers) to predict MT outcomes and it is difficult to explain MT errors. These clashes in narrative, database and algorithmic logic underlying the material configuration of data may be at the heart of some of the frustrations experienced by translators working with TM and MT \citep[e.g.][]{moorkens_assessing_2017,leblanc_les_2014,cadwell_human_2016}. 

Suggestions for potential improvements that are not yet generally implemen\-ted in commercial MT applications include MT quality estimators that are meaningful in the context of the post-editing process, e.g., identification of segments that require revision, or estimates of post-editing efficiency, rather than abstract quality metrics \citep{stahlberg_neural_2019}. Other desired changes  are delivered, to some extent, by interactive and adaptive MT systems, where the MT suggestion is changed on the basis of what the user types, and the system also learns from the corrections made \citep{daems_interactive_2019,karimova_user-study_2018}. \citet[43]{pielmeier_state_2020} report that, of their 2,059 respondents to questions about MT use, 71\% agree with the statement \enquote{I prefer to work with adaptive MT like Lilt rather than raw MT output}. Lilt promotes its interactive, adaptive MT for use with \enquote{high-value content} in particular. It changes the material working environment of the translator further, in that fuzzy TM matches are no longer helpful, so the translator is working with MT suggestions for all segments.

The final role to consider for materials is as resources, i.e., things that are used up or consumed in the practice \citep{shove_matters_2017}. Translation practices consume resources, with or without the deployment of MT (see also \citealt{cronin_eco-translation_2017}). However, as might be concluded from the description of computer processors above, the building and training of NMT engines is considerably more resource-intensive than the compilation and use of TMs. Indeed, NMT engines not only consume more processing resources but also require longer training times than the previous SMT systems. In resource terms, the technology developers appear to be moving in two different directions. On the one hand, there are attempts to enable machine learning applications like NMT to use CPUs more efficiently so that they may be run on conventional PCs and mobile devices, to reduce both the need for specialized hardware and the training times \citep{devlin_sharp_2017}. SDL’s latest NMT product, for instance, the Enterprise Translation Server, is offered in both GPU and CPU modes. The benefit of running NMT with CPU is presented as lower infrastructure costs, although it entails compromises on speed or quality. On the other hand, MT research is also pulling in the opposite direction, towards massively multilingual NMT systems that require billions of words as data and very substantial computing power \citep{aharoni_massively_2019}.

\section{Changing competences}

Translation scholars have long been interested in competences, understood and articulated in a variety of different ways, ranging from Pym’s \citeyearpar{pym_redefining_2003} minimalist definition of translation competence to the complex, multi-dimensional models and competency frameworks proposed by the PACTE \citep{hurtado_albir_researching_2017}, TransComp \citep{gopferich_translation_2013} and EMT \citep{emt_board_european_2017} projects, among others. In \citet{olohan_knowing_2017} I argue that a focus on knowing-in-practice (i.e., knowing as it transpires in and through practice) is desirable because it pays due attention to the situated, embodied, relational, and materially mediated aspects of knowing, alongside the embrained knowing that is more traditionally accorded primacy in discussions of competence, training and education. In Schatzki’s terms, this is practical understanding, as introduced above, i.e. knowing how to perform the doings and sayings that constitute the practice and also recognizing when these are performed. Practical understanding is alternatively described as a \enquote{a battery of bodily abilities that results from, and also makes possible, participation in practices} \citep[9]{schatzki_introduction:_2001} Shove et al.’s understanding of competence similarly encompasses \enquote{skills, know-how and technique} \citeyearpar[15]{shove_dynamics_2012}.

\hspace*{-1mm}Multi-dimensional translation-related competency frameworks have generally been developed with professional practice in mind, and through consultation with practitioners and other relevant stakeholders. Such frameworks typically seek to formalize the practice by formulating an understanding of what it means to be competent that can serve as a competency standard. However, there is some variation in the practice that is being addressed. A framework such as the EMT’s is strongly focused on learning outcomes and arguably formalizes what it means to be competent in the learning practice rather than competent in the translation practice. Others, such as the PACTE framework, focus on capacities that professional translators should demonstrate, formulated as a list of tasks or activities that practitioners should be able to complete. In both kinds of cases, outcomes are foregrounded, with relatively less consideration of the performances from which those outcomes ensue. A practice-theoretical approach, by contrast, is interested in the situated, social, embodied and materially mediated nature of the knowing that makes participation in the practice both possible and appropriate. It also recognizes that there are different ways of carrying on a practice.

Where scholars have considered post-editing as a separate practice from translation, they nonetheless develop PEMT competency frameworks that are strikingly similar to those for translation \citep[e.g.,][]{nitzke_risk_2019}. These similarities have also been codified in the international standard for the post-editing of machine translation output, ISO 18587:2017, where a substantial focus is on competences. A comparison of the post-editing standard and the standard for translation services, ISO 17100:2015, reveals an almost identical description of competences, classified as translation competence; linguistic and textual competence in the source language and the target language; competence in research, information acquisition, and processing; cultural competence; technical competence, and domain competence \citep{british_standards_institution_iso_2015,british_standards_institution_iso_2017}. Likewise, the qualifications required by the standards are very similar. Translators are required to have a formal degree in translation or full-time professional experience in translating, or a combination of professional experience and a degree in another field. For post-editors, the formal degree simply needs to include significant translation training (so it can be a more general degree in language studies). The professional experience required can be in translating or post-editing. These prescriptions of substantially similar know-how for PEMT and translation provide further support for this paper’s argument that PEMT constitutes an additional activity that may take place as part of the translation practice, rather than a separate, recognizable practice in its own right. The overlaps extend to the standards’ expectations on the role of formal training in abstracting and codifying that know-how. 

\hspace*{-1mm}Despite these competence-related convergences, Slator’s \textit{Neural Machine Translation Report} \citep{slator_neural_2019} noted a growing demand for qualified post-editors and a growth in companies developing training courses to fill this demand. LSPs who expect their linguists to perform PEMT also frequently acknowledge the need for training in this activity. Transperfect, for example, provides training and a certification programme in PEMT for some thousands of freelance linguists \citep[137]{zaretskaya_optimising_2019}, and training was also required at TranslateMedia when post-editors switched from editing SMT to NMT \citep[170]{kosmaczewska_application_2019}.

The training that is delivered tends to address additional requirements that are given in ISO 18587:2017 in a section entitled ‘Professionalism’. Here it is stipulated that post-editors should have general knowledge of MT technology, basic understanding of common MT errors and a general knowledge of CAT tools \citep[8]{british_standards_institution_iso_2017}. This know-how is deemed important, not only for the execution of changes to the MT output but also because LSPs collect data on how MT is used and they usually require translators to report on frequently encountered errors. These reports are fed back to technology developers, to contribute to improving the MT engines. 

The standard also requires post-editors to have \enquote{the knowledge and ability to establish whether editing MT output makes sense, it terms of time and effort estimation}, and the \enquote{ability to follow instructions received} and \enquote{to focus on specific issues and make specific corrections as given} \citep[8]{british_standards_institution_iso_2017}. These aspects address key know-how from an LSP’s perspective, often linked to post-editing speeds and productivity. A decision-making process is mapped by \citet{nitzke_risk_2019}, who propose a decision tree to help users to decide whether or not to use MT and how to approach PE. Factors to be considered in making the decision include possible risks and benefits, resources needed and available, data sensitivity and security issues, quality of MT output produced and the client’s or end user’s quality requirements. Once MT has been deployed, the two-second rule \citep{graciet_translator_2018} encapsulates the rapid decision making required of translators about whether an MT suggestion is usable with editing, or whether the translator needs to produce a translation from scratch. 

\citet{blue_how_2016} posit that practices constitute the knowledge that they need to continue to exist, and that there are various mechanisms by which this happens. The translation practice takes some of its know-how from closely related practices, such as the practice of learning a language or the practice of writing literature or other genres. It cultivates other aspects of its know-how, for example for the PEMT activity, through know-how that is already embedded in material forms. With MT integrated materially into the familiar CAT interface, translators know how to interact with MT suggestions through their previous interactions with TM matches, since the TM’s segment-focused database logic is typically extended to the PEMT activity. At the same time, material differences that have an impact on knowing, as noted above, include differences in metadata available for TM matches and MT suggestions. 


\section{Changing meanings}
Meaning is used as an overarching term to encompass \enquote{symbolic meanings, ideas and aspirations} of a practice \citep[14]{shove_dynamics_2012}, alternatively thought of as forms of understanding, states of emotion and motivational knowledge \citep[249]{reckwitz_toward_2002}. Competences, as discussed above, relate to the practical know-how required by the practice and some codifications or prescriptions pertaining to that know-how. Here we consider other organizing elements of the practice, selectively focusing on changes in general understandings of PEMT activities in translation practice among LSPs and among translators, where general understandings are understood in Schatzkian terms \citep[e.g.,][]{schatzki_site_2002} as general senses of the nature of things which find expression in the doings and sayings of a practice. 

\citet{welch_how_2017} consider general understandings of practices as sometimes tacit in the background and sometimes discursively articulated. Discursive articulations related to the use of MT in language services are clearly shifting. MT, as a service offering, was much less visible in industry discourse just a couple of years ago than it is now. In their online promotional material of 2018, the world’s largest LSPs were mostly concerned with assuring clients that their texts would be translated by human translators to the highest levels of quality, using CAT tools (not MT) for productivity gains \citep[76]{olohan_translation_2021}. A small number of these LSPs still do not offer MT and do not acknowledge its existence but most of them now generally promote MT as bringing benefits to clients, usually due to the need to translate greater volumes faster. Some articulations that are representative of the largest LSPs (by revenue, as listed in CSA Research’s annual LSP rankings) are as follows: 


\begin{modquote}
Linguistic computing has come a long way over the decades, and in recent years, the quality and cost of machine translation (MT) solutions has harmonized with demand and time-to-market requirements.\\
\href{https://janusww.com/technology/mtmtpe/}{Janus} 
\end{modquote}

\begin{modquote}
To meet tight deadlines for large translation volumes while keeping a critical eye on the long-term costs, a machine translation may be a perfect alternative. \\
\href{https://www.yamagata-europe.com/en-gb/translation/machine-translation}{Yamagata}
\end{modquote}

\begin{modquote}
Welocalize language automation like machine translation (MT) delivers translation and content transformation faster across a larger volume of content without compromising quality. \\
\href{https://www.welocalize.com/content-transformation/}{Welocalize}
\end{modquote}

These and other LSPs offer post-editing as part of their customized MT services, as a means for clients to achieve a desired level of quality, related to specialist content in particular: 

\begin{modquote}
We select and onboard post-editors with linguistic and technical experience in your industry to edit the machine’s output to your desired level of quality.\\
\href{https://www.rws.com/technology/machine-translation/}{RWS} 
\end{modquote}

\begin{modquote}
The post-editing service complements machine translation. The translator, referred to as \enquote{post-editor} in this case, uses his [sic] knowledge to harmonise the pre-translated text in order to make it easier to understand and to respect the terminology used in your sector. \\
\href{https://www.acolad.com/uk/services/}{Acolad}
\end{modquote}

\begin{modquote}
With the aim of making Machine Translation (MT) work for each translation, \textbf{we always advise our clients} to use it under prior human supervision (training, personalization, adaptation) and/or subsequent human editing (human revision of the content produced by the translation machine.)\\
\href{https://www.linguaserve.com/en/multilingual-services/machine-translation-and-human-post-editing/}{Linguaserve} (emphasis in the original)
\end{modquote}


It should be noted that these LSPs tend to provide MT or PEMT as a distinct service offering or option, or as a standard approach for particular domains of activity. However, in most cases the promised PEMT end product is not depicted as qualitatively different from what they promise as the product of translation. The client is not to expect any discernible difference in their translations, regardless of the combinations of activities that produce them. An exception in this dataset is seen in the discursive articulation by Morningside Translations, which stresses the cost savings for high volumes but also explicitly tempers quality expectations:

\begin{modquote}
Machine translation is a powerful tool for lowering costs and accelerating turnaround times for high-volume document translation projects, though its quality is still far from being on par with human translation. […] It can help you get the \enquote{gist} of a document when subpar quality is sufficient. \\
\href{https://www.morningtrans.com/services/translations/document-translations/}{Morningside Translations}
\end{modquote}


Addressing the teleo-affective or motivational dimensions of the practice, LSP managers, when describing how they introduce PEMT activities to their workflows, often mention a reluctance on the part of their translators to be involved in PEMT projects \citep[e.g.,][]{premoli_mtpe_2019,kosmaczewska_application_2019}. In CSA Research’s large-scale survey \citep{pielmeier_state_2020}, 8,794 translators were asked to choose the task that they \enquote{would prefer to do when given the choice} and 89\% chose translation, while 8\% chose editing human translation and only 3\% chose editing machine translation. As with the introduction of translation memories a few decades ago, this reluctance is sometimes interpreted as a reluctance to embrace new technology but this is an overly simplistic interpretation; the same survey data shows that only 7\% of respondents are not very confident trying new language technology. Other motivational factors are therefore likely to be much more relevant. Focus group studies such as Cadwell et al.’s \citeyearpar{cadwell_resistance_2018} have uncovered several of these, including translators’ expectations of poor quality MT output, the potential degrading of their translation abilities or creativity through PEMT, and the prospect of MT eventually replacing human translators. It should be noted that the same translators also gave several reasons in favour of working with MT. Through real-time logging of translation workflows and a follow-up survey at the Directorate-General for Translation of the European Commission, \citet{macken_quantifying_2020} also identify some of the factors that motivate translators’ preferences for working with MT. These were mostly related to their impressions that they worked faster with MT than without; and, for most but not all translators, this was backed up by the researchers’ measurements. 

Attitudes of salaried translators in institutional environments, where some of the risk factors associated with MT use are managed by the institution, can be understandably different from freelancers, in sometimes precarious work situations. \citet{nunziatini_machine_2019} reports on an MT implementation in the financial services domain in which translators’ reluctance to engage in PEMT was overcome, to some extent, by continuing to pay the full word rate in the pilot phase of the implementation. The question of how translators are remunerated for PEMT should perhaps not be underestimated as playing a part in motivation; many language professionals are suffering from downward pressure on rates \citep[60]{pielmeier_state_2020} and this can be exacerbated by other practices in the sector. Finally, \citet{kosmaczewska_application_2019} note that translators’ initial reservations were overcome by their interest in continuing to work on their client’s content and to use their acquired experience, as they changed from a human translation to a PEMT workflow. These observations serve to highlight the need for translation research to consider more closely those other practices, such as the management of resources, when seeking to understand the complexities of the translation practice. 

\section{Conclusion}
Looking through a practice-theoretical lens, this paper has illustrated some of the changes in materials, competences and meanings that have recently reshaped and continue to transform the translation practice as it is expanded to include the activity of PEMT. The interdependencies of these elements has also come to the fore; changes in one element often bring about changes in others. In addition, thinking about translation in this way highlights the importance of connections between the translation practice and other practices, whether they are training NMT engines, buying and selling MT services, managing risk or balancing budgets in LSPs. 

I conclude by reflecting on how this reconfiguration of translation practice can be studied empirically. Generally, practice research relies heavily on real-time observations of situated practice performances, often in combination with qualitative, ethnographic interviews. These methods allow practices to be made visible, articulated and reflected on by practitioners and then mediated and theorized through research practices. \citet{nicolini_zooming_2009} advocates an approach that involves \enquote{zooming in} on the accomplishment of a practice in a particular setting and then \enquote{zooming out} to focus on the texture of the practices with which it is connected. Translation practices incorporating PEMT activities were initially studied predominantly in experimental settings and sometimes among students, novice translators or those with little prior exposure to the PEMT activity. Increasingly, they are being observed by translation researchers in real time in their everyday occurrences (see, for example, \citealt{macken_quantifying_2020}). There remains considerable scope for this kind of focus on a specific sited practice to be accompanied by a "zooming out" to the textures of connected practices.


Variations on ethnographic research \citep[see][]{katz_becoming_2019} that can be transposed to practice research are also worth considering for the study of translation. Iconic ethnography, for example, focuses on a small number of practitioners or settings considered particularly representative of a type. In the realm of PEMT and translation, an example would be the practice performed by translators designated as MT superusers or similar within LSPs, whose practice is held up as an example to others or who are responsible for instructing, guiding or supporting the practices carried out by others. Comparative analytical ethnography or multi-site ethnography, by contrast, focuses on teasing out the variations in translation practice as enacted at different sites. 

Alongside conventional methods of observation and ethnographic interviewing, novel methods for practice research not yet used by translation scholars but offering some potential include Nicolini’s \citeyearpar{nicolini_articulating_2009} \enquote{interview to the double}, a form of interview in which the practitioner gives an oral set of instructions to their hypothetical double, who will replace them in the workplace the next day but whose presence there should not be detected by others. This is intended to produce a detailed account of behaviour but is also likely to reveal the situated, normative influences on practices as the practitioner gives an insight into what is considered good practice, what should be done, said or prioritized, based on whose judgement, etc.

Many practice researchers do not see a role for quantitative data to be used alongside qualitative when studying practices but some studies have shown the benefits of a mixed-methods approach, for example, where time-use or diary records have been a useful source of information about the spatio-temporal organization of eating practices in past decades \citep{warde_changes_2007}. As seen in this paper, survey and focus group methods have been helpful in prompting practitioners to report on aspects of their practices. Those \textit{post-hoc} accounts can provide insights into the doings and sayings of specific, sited practices, especially for aspects such as motivations, expectations or preferences in practices. As illustrated by numerous researchers who have studied post-editing effort \citep[e.g.,][]{moorkens_correlations_2015,herbig_multi-modal_2019,macken_quantifying_2020}, quantitative metrics are invaluable for understanding temporal organization and sequencing of activities within the practice, and technical effort is typically also captured through quantitative data on editing actions. I content that a reflexive, mixed-methods approach to the translation practice is possible, when such quantitative methods are used in conjunction with qualitative studies of practice performances. The imperative on the practice researcher is to resist the temptation to study the practice by attending to just one of its constituent elements, and to seek to understand the interdependencies of constituent elements and the interwoven nature of practices. 

{\sloppy\printbibliography[heading=subbibliography,notkeyword=this]}

\end{document}
