\documentclass[output=paper,colorlinks,citecolor=brown]{langscibook} 
\author{Eleanor Ridge\affiliation{Massey University}\orcid{}}
\title{Morphosyntactic and functional asymmetries in Vatlongos discourse demonstratives}
\abstract{Vatlongos (also known as Southeast Ambrym, Oceanic, Vanuatu) has four demonstrative categories: three person-based distance distinctions (first-person proximal, second-person proximal, and distal) and a contrastive category. In spatial situational domains, and to refer to locative referents, first-person proximal, second-person proximal and distal categories are distinguished from each other. Discourse functions are largely structured around an opposition between forms based on the first-person proximal clitic \textit{ak} and the contrastive suffix \textit{-e}. The wider morphosyntactic distribution of the first-person proximal is reflected in its discourse functions, as the unmarked forms for anaphora and recognitional uses. The more restricted contrastive \textit{-e} forms also occur in contexts of negative affect. In the verbal forms this asymmetry is even more striking. The first-person proximal verbal demonstrative \textit{mak} is the general manner demonstrative, occurring both as a main verb and modifying other verbs in serial verb constructions. The verbal form based on contrastive \textit{-e}, \textit{mue}, is only used in hesitation, a specialisation that could arise from the role of contrastive \textit{-e} forms in discourse repair to modify placeholders.}
\IfFileExists{../localcommands.tex}{
  \input{../localpackages}
  %Copy this to localcommands.tex

\usepackage[english]{babel}
\usepackage{amsmath}
\usepackage{amssymb,amsfonts,textcomp}
\usepackage{array}
\usepackage{hhline}
\usepackage{hyperref}

\newenvironment{styleStandard}{}{}
\newenvironment{stylelsAbstract}{}{}
\newenvironment{stylelsSectioni}{}{}
\newenvironment{stylelsSectionii}{}{}
\newenvironment{stylelsBulletList}{}{}
\newenvironment{styleBibliographyi}{}{}
\newenvironment{listWWNumxxvleveli}{}{}
\newenvironment{listWWNumxxvlevelii}{}{}
\newenvironment{listWWNumxxvleveliii}{}{}
\newenvironment{listWWNumxxvleveliv}{}{}
\newenvironment{listWWNumixleveli}{}{}
\newenvironment{listWWNumixlevelii}{}{}
\newenvironment{listWWNumixleveliii}{}{}
\newenvironment{listWWNumixleveliv}{}{}

\newcommand\textstyleListLabelxvi[1]{#1}
\newcommand\labellistWWNumxxvleveli{\thelistWWNumxxvleveli.}
\newcommand\labellistWWNumxxvlevelii{\thelistWWNumxxvlevelii.}
\newcommand\labellistWWNumxxvleveliii{\thelistWWNumxxvleveliii.}
\newcommand\labellistWWNumxxvleveliv{\thelistWWNumxxvleveliv.}
\newcommand\labellistWWNumixleveli{[F0B7?]}
\newcommand\labellistWWNumixlevelii{\textstyleListLabelxvi{o}}
\newcommand\labellistWWNumixleveliii{[F0A7?]}
\newcommand\labellistWWNumixleveliv{[F0B7?]}

\newcounter{listWWNumxxvleveli}
\newcounter{listWWNumxxvlevelii}[listWWNumxxvleveli]
\newcounter{listWWNumxxvleveliii}[listWWNumxxvlevelii]
\newcounter{listWWNumxxvleveliv}[listWWNumxxvleveliii]
\newcounter{itemize}  
  \input{../localhyphenation} 
  \togglepaper[1]%%chapternumber
}{}

\begin{document}
\maketitle
\shorttitlerunninghead{Morphosyntactic and functional asymmetries in Vatlongos}

%formatting of tables to be repaired
%To be checked: How do I insert the hyperlink to Pangloss collection in Ridge 2018b (see References in this chapter and .bib-file)?
%orphan control
%Problem in the bibliography: von Prince occurs under P and de Vries and V > shouldn't it be under V and D, respectively? (Check style guide)

\section{Introduction}
\label{sec:ridge:1}

Vatlongos, also known as Southeast Ambrym, is an Oceanic (Austronesian) language spoken by around 3,000 speakers in Vanuatu. As well as in the southeast region of Ambrym Island, it is spoken by a community who relocated to Mele Maat, a settlement just outside the capital city Port Vila on Efate Island, in the 1950s. It has many features typical of Oceanic languages: SVO constituent order, head-marking, and subject cross-indexing prefixes and object pro-indexing suffixes.\footnote{This means subject prefixes optionally co-occur with a co-referential noun phrase, whereas object suffixes have complementary distribution with object noun phrase (see \citealt{Haspelmath2013}).}  It has relatively complex morphology, including extensive verb-initial consonant mutation and morphological dependencies in non-contiguous serial verb constructions.

The distribution of Vatlongos demonstratives in discourse show how extension into different functions can interact with the grammatical properties of morphemes involved, and the wider grammatical structures of the language.

A three-way person-based distance distinction between first-person proximal, second-person proximal and distal is used in the spatial situational domain, but is not maintained in the major discourse extensions of the demonstrative system. Instead, a two-way distinction between first-person proximal and contrastive forms is applied in discourse functions. The first-person proximal is the unmarked form for anaphora, while the contrastive is also used to code negative affect, indicating the speaker’s dislike or emotional distress about characters, times and places.

This difference in markedness is also evident in the different grammatical possibilities for the two forms. First-person proximal \textit{ak} is an enclitic that can also function as an independent phonological word, and can freely modify noun phrases including pronouns, or function as a pronoun. It is also lexicalised with the \textsc{3sg} pronoun \textit{xi} as \textit{xiak}, which can function as a temporal or locative adverb. The contrastive suffix \textit{-e} cannot appear independently, but must attach to a singular or plural pronoun, formally identical quantifiers which are diachronically related to the pronouns, or more rarely to the locative noun stem \textit{ig}. This means that as an adnominal, it occurs only in singular or plural noun phrases, and not in the dual or paucal. Like first-person proximal \textit{ak}, it can also function as a temporal or locative adverb when attached to the \textsc{3sg} pronoun or singular quantifier.

First-person proximal \textit{ak} is also a component of the demonstrative verb \textit{mak} ‘like this’, which is very frequent in the corpus and usually occurs as the subsequent verb in a serial verb construction, meaning ‘(be/do) like this’, ‘in this manner’. There is also a verbal form \textit{mue}, which probably originates from the same demonstrative paradigm based on contrastive \textit{-e}, although its use is restricted to a verbal hesitation form. Verbal demonstrative \textit{mue} is used as an inflected placeholder when a speaker is searching for a verbal lexeme.

The examples discussed in this chapter come from a {\textasciitilde}65,000 word corpus of audio- and video-recorded texts collected during fieldwork on Ambrym and in Mele Maat between 2014 and 2017, as part of a PhD project focusing on verbal constructions \citep{Ridge2019}. The token frequencies of different demonstrative forms are based on their distribution in the spontaneous (non-elicited) texts within the corpus, consisting of a {\textasciitilde}48,000-word subcorpus. Most of the texts in this subcorpus can be accessed via the Pangloss collection \citep{Ridge2018Deposit}, which displays audio and video time-aligned by pause unit, with a transcription, interlinear glossing, and translations in English and Bislama (the national language of Vanuatu).

A limitation of the corpus is that it is heavily skewed towards monologues and narratives. This means that some functions of demonstratives are underrepresented compared to everyday language use, especially spatial situational uses and discourse functions involving confirmation from an interlocutor. Some of these are pointed out below, and examples are taken from field notes of observed language use to supplement the corpus. However, the functions of demonstratives in relation to referent tracking and establishing topics are very clearly exemplified in these narrative texts.

This chapter focuses on the demonstrative forms that occur most frequently in the subcorpus (see \tabref{tab:ridge:1}). These five forms are all based on first-person proximal \textit{ak} and contrastive \textit{-e}, but other demonstrative distinctions and paradigms are outlined in \sectref{sec:ridge:2}. The high token frequency of these forms allows a detailed examination of their discourse functions, whereas comments on some of the other demonstrative forms are necessarily more speculative.

%Adjust the table so that the three lines Independent, 3sg/sg xi and Verb form groups
\begin{table}
\caption{Most frequent demonstrative forms in the Vatlongos corpus, their token frequency and major grammatical functions}
\label{tab:ridge:1}
\begin{tabularx}{\textwidth}{lQQ}
\lsptoprule
& \textbf{\textit{ak} first-person proximal} & \textbf{\textit{-e} contrastive}\\
\midrule
\textbf{Independent} & \textit{ak} (488 tokens)

adnominal

locative adverbial & {}-\\

\tablevspace
\textbf{\textsc{3sg/sg} \textit{xi}} & \textit{xiak} (251 tokens)

temporal adverbial

locative adverbial

pronominal

adnominal (in complex NPs and relative clauses) & {\textit{xie} (290 tokens)}

adnominal

temporal adverbial

locative adverbial

pronominal\\
\tablevspace
\textbf{Verb} & {\textit{mak} (328 tokens)}

subsequent verb in SVC

main verb in clause & {\textit{mue} (227 tokens)}

verbal hesitation form\\
\lspbottomrule
\end{tabularx}
\end{table}

Vatlongos is most closely related to Paamese, the language spoken on the islands located off the southeast coast of Ambrym, Paama and (historically) Lopevi. Paamese has a simpler demonstrative system, making a two-way distinction between a proximal and a distal. In addition to the expected spatio-temporal functions of those categories, the proximal is used for the “current topic of discussion” in the discourse, and the distal is used to refer back to a previously mentioned referent which is not the current topic of discussion (\citealt[226-229]{Crowley1982}). Paamese also has verbal demonstratives, and they are more transparently compositional than the Vatlongos equivalents. Crowley identifies a verbal root \textit{muko} which obligatorily occurs with one of the two demonstrative clitics.

The closest parallel to Vatlongos contrastive \textit{-e} is found in Daakaka (West Ambrym), where a clitic \textit{e} ‘the other (place/one)’ indicates contrast regardless of deictic distance (\citealt[175-176, 333]{Prince2015}). This form can cliticise to an agent pronoun, a local pro-adverb, and a verb meaning ‘like’, parallel to the distribution of Vatlongos \textit{-e}.

\sectref{sec:ridge:2} outlines the possible distinctions and forms of Vatlongos demonstratives in spatial situational uses, including the forms that are restricted to the spatial paradigm. \sectref{sec:ridge:3} focuses on the morphosyntactic distribution of the \textit{ak} and -\textit{e} series, while \sectref{sec:ridge:4} takes a closer look at the range of discourse functions performed by these forms. \sectref{sec:ridge:5} focuses on the verbal demonstrative forms, linking their functions to the role of the contrastive \textit{-e} series in discourse repair.


\section{Distinctions in Vatlongos demonstrative forms}
\label{sec:ridge:2}

\subsection{Introduction}
\label{sec:ridge:2.1}

Earlier descriptions of Vatlongos demonstratives only mention a three-way distinction between the forms shown in \tabref{tab:ridge:2}. This paradigm is fundamentally spatial, usually marking deictic reference in situational use, but also available for anaphoric reference to places when combined with the locative bound stem \textit{ig}.

%Adjust formatting of table
\begin{table}
\caption{The spatial paradigm in Vatlongos demonstratives}
\label{tab:ridge:2}
\begin{tabularx}{\textwidth}{lQl}
\lsptoprule
\textbf{Marker} & \textbf{Gloss} & \textbf{Example forms}\\
\midrule
{\textbf{\textit{ak}}} & {first-person proximal} & {\textit{ak, xiak, igak}}\\
{\textbf{\textit{xai}} } & {second-person proximal} & {\textit{xai, igaxai, iaxai}} \\
{\textbf{\textit{xor}}} & {distal} & {\textit{xor, igoxor, ioxor}}\\
\lspbottomrule
\end{tabularx}
\end{table}

\citegen{Parker1970} dictionary lists \textit{ak} ‘this, here’ and \textit{xiak} ‘now, here’, suggesting these forms had roughly similar functions to their contemporary usage. He also lists \textit{xai} ‘that (relatively near)’ and \textit{xor} ‘there, that (relatively far)’. Recent fieldwork has found that these spatial situational distinctions are person-based, with \textit{xai} marking that the referent is close to the addressee, rather than the purely distance-based medial category implied by Parker’s definitions.

Another major difference between Parker’s description and the description offered here is the addition of the contrastive series of demonstrative forms based on \textit{{}-e}. In the contrastive paradigm, \textit{{}-e} forms are contrasted with forms based on \textit{ak}, as shown in \tabref{tab:ridge:3}.

%adjust formatting of the table
\begin{table}
\caption{The contrastive paradigm in Vatlongos demonstratives}
\label{tab:ridge:3}
\begin{tabularx}{\textwidth}{lQQ}
\lsptoprule
\textbf{Marker} & \textbf{Gloss} & \textbf{Example forms}\\
\midrule
\textbf{\textit{ak}}  & first-person proximal, non-contrastive & \textit{ak, xiak, igak, mak}\\
{\textbf{\textit{{}-e}} } & contrastive, negative affect, hesitation marker & \textit{xie, ige, mue}\\
\lspbottomrule
\end{tabularx}
\end{table}

Like the spatial paradigm, this contrastive paradigm can be used for deictic reference in situational use, but additionally it can be used for discourse deixis, anaphora and recognitional use, functions described in \sectref{sec:ridge:4}. The participation of \textit{ak} forms in both the spatial and contrastive paradigm explains the higher frequency and wider extension of \textit{ak} forms, including into temporal domains. The wider functional range of \textit{ak} forms is also mirrored in their morphosyntactic behaviour described in \sectref{sec:ridge:3}.

The rest of this section will discuss less frequent forms that mark a spatial distinction: the independent forms, the forms based on the locative bound stem \textit{ig}, and the \textit{ia} forms which appear to have developed diachronically as phonologically reduced variants of the \textit{ig} forms. Apart from in the first-person proximal series, these forms are all less frequent in the subcorpus than those in \tabref{tab:ridge:1}, which are the main focus of this chapter, so the discussion here is relatively brief and speculative, but important to understanding how the discourse functions of the \textit{ak} and -\textit{e} forms fit into the wider Vatlongos demonstrative system.

\subsection{Spatial distinctions in independent forms}
\label{sec:ridge:2.2}

%repair Reference spilling into the margins in the following paragraph
Although they are listed in Parker’s dictionary, today the independent forms of the second-person proximal and distal demonstratives are fairly marginal (\tabref{tab:ridge:4}): there are 21 tokens of \textit{xai} in the corpus and 14 of \textit{xor}, and these tokens are associated with only a handful of older speakers. They seem to index a conservative style of speech, and are especially associated with the distinctive Endu-Vatlongos dialect, spoken in the northernmost village of Southeast Ambrym \citep{Ridge2018Contexts}. First-person proximal \textit{ak} is far more frequent and has spatial, temporal and discursive functions discussed in \sectref{sec:ridge:4}.

%adjust the formatting of the table
\begin{table}
\caption{The independent forms}
\label{tab:ridge:4}
\begin{tabularx}{\textwidth}{XlXr}
\lsptoprule
\textbf{Series} & \textbf{Independent form} & \textbf{Gloss} & \textbf{Token count}\\
\midrule
{Proximal 1} & {\textbf{\textit{ak}}} & {near speaker} & {488}\\
{Proximal 2} & {\textbf{\textit{xai}}} & {near addressee} & {21}\\
{Distal} & {\textbf{\textit{xor}}} & {distal} & {14}\\
{Contrastive} & {\textbf{\textit{–}}} &  & \\
\lspbottomrule
\end{tabularx}
\end{table}

\subsection{Spatial and contrastive distinctions with locative stem \textit{ig}}
\label{sec:ridge:2.3}

In the locative domain, all three spatial forms and the contrastive can be suffixed to the locative stem \textit{ig} ‘place’, a bound stem that can only occur with one of the demonstrative forms, as shown in \tabref{tab:ridge:5}.

\begin{table}
\caption{Forms based on locative bound stem \textit{ig}}
\label{tab:ridge:5}
\begin{tabularx}{\textwidth}{llQr}
\lsptoprule
\textbf{Series} & \textbf{\textit{ig} locative form} & \textbf{Gloss} & \textbf{Token Count}\\
\midrule 
{Proximal 1} & {\textbf{\textit{igak}}}  & {‘here, this place (near speaker)’} & {217}\\
{Proximal 2} & {\textbf{\textit{igaxai}}} & {‘there, that place (near addressee)’} & {7}\\
{Distal} & {\textbf{\textit{igoxor}}} & {‘there, that place (distal)’} & {38}\\
{Contrastive} & {\textbf{\textit{ige}}} & {‘that place (negative affect)’} & {21}\\
\lspbottomrule
\end{tabularx}
\end{table}

The first-person proximal form \textit{igak} is by far the most frequent with 221 tokens in the corpus, followed by distal \textit{igoxor} (38 tokens), and the fairly infrequent second-person proximal \textit{igaxai} (7 tokens). However, in everyday interactions, the second-person proximal and distal forms were observed to be more frequent than these figures would suggest. The maintenance of a three-way distance-based distinction in only locative forms (and not demonstrative determiners) has been observed for Neverver (Malekula, Vanuatu, \citealt{Barbour2012}). It appears that Vatlongos is headed in this direction given the marginal status of independent second-person proximal \textit{xai} and distal \textit{xor}.

Contrastive \textit{-e} can also be suffixed to the locative stem, but in this form the negative affect meaning of the suffix seems to be more important than the basic contrastive meaning. Of the 21 tokens of \textit{ige} in the corpus, 18 occur in accounts of emotionally distressing episodes, especially in personal experiences of natural disasters, or to describe places where characters are tricked or disrespected in traditional stories. The three remaining tokens are in the concluding utterances of narratives, in formulaic phrases meaning ‘the story finishes there’. This connotes a humble, apologetic tone at the end of a performance, and cooccurs with other hedging strategies like repetition of the adverb \textit{xos} ‘maybe’.

\ea{}
[20150419e\_h01m128\_68]\label{ex:ridge:1}\footnote{Example
    codes consist of the date of the recording event in YYYYMMDD format (e.g. 20150419), an identifying letter for each session recorded that day (e), an underscore, a letter indexing broad genre (c~=~conversation, h~=~history, n~=~narrative, p~=~procedural, t~=~formal speech), a number identifying the recording within the session (e.g. 01), a letter indicating the speaker community (e~=~Endu village, m~=~Mele Maat village, s~=~other villages of Southeast Ambrym), a speaker code (e.g. 128), an underscore, and a number identifying the pause unit within the recording (e.g. 68).
    }\\
\gll ma  xos  stori  navan    xos  bus      ig-e\footnotemark{}\\
     so  maybe   story  \textsc{1sg.poss}  maybe   \textsc{3sg.nfut}.finish  place-\textsc{contr}\\
\glt ‘So maybe my story ends there.’
\z
\footnotetext{Orthography aligns with IPA with the following exceptions: voiced stops are prenasalised \textrm{⟨}b\textrm{⟩} /\textsuperscript{m}b/, \textrm{⟨}d\textrm{⟩} /\textsuperscript{n}d/, \textrm{⟨}g\textrm{⟩} /\textsuperscript{ŋ}ɡ/; \textrm{⟨}v\textrm{⟩} can be realised as [v] or [β], the digraph \textrm{⟨}ng\textrm{⟩} represents /ŋ/. \textrm{⟨}j\textrm{⟩} represents the affricate /dʒ/ which only occurs in loan words from Bislama and English. Capitalisation and punctuation follow practices in English, the language of education for most Vatlongos speakers.}


{The \textit{ig} forms can function as locative adverbials, as in \REF{ex:ridge:1} and \REF{ex:ridge:2}, or pronominally as the argument of a verb. In \REF{ex:ridge:3}, \textit{igak} is the object of the verb: \textit{kamuet} ‘find’ is a morphologically transitive verb which must either be followed by an object noun phrase or take an object pro-indexing suffix.}

\ea
{\label{ex:ridge:2}[20150415a\_h02s125\_41]}\\
\gll lata-pat      igak\\
     \textsc{3pc.nfut}{}-sleep    place.\textsc{prox1}\\
\glt ‘They slept here.’
\z

\ea
{\label{ex:ridge:3}[20170413a\_h01m169\_21]} \\
\gll la-kamuet  igak\\
     \textsc{3pl.nfut-}find  place.\textsc{prox1}\\
\glt ‘They found this place.’
\z

Like the other forms based on first-person proximal \textit{ak}, \textit{igak} is also used for anaphoric functions, as long as the referent is a place. These anaphoric functions explain the much higher token frequency of \textit{igak} compared to the other \textit{ig} forms. In \REF{ex:ridge:4}, \textit{igak} is used as recognitional placeholder, followed by the place name Lamap (a village on Epi Island). The listener is expected to be familiar with this place, both because it was introduced six pause units earlier in the narrative, and because the history of the founding of Mele Maat is well-known in the community.

\ea
{\label{ex:ridge:4}[20150219b\_n01m001\_45]}\\
\gll la-be      kalavus  e  igak,    Lamap\\
     \textsc{3pl.nfut-nfut.}go\_to  prison    \textsc{loc}  place.\textsc{prox1}  Lamap\\
\glt ‘They went to prison in that place, Lamap.’
\z


\subsection{Second-person proximal and distal forms with \textit{ia}}
\label{sec:ridge:2.4}

The second-person proximal and distal forms based on a stem \textit{ia} are probably phonologically reduced variants of the locative \textit{ig} forms. \citet[8]{Parker1970} lists \textit{iaxai} and \textit{ioxor} as variants of \textit{igaxai} and \textit{igoxor}. However, in the contemporary corpus they have a different distribution from the locative \textit{ig} equivalents (\tabref{tab:ridge:6}).

%Table to be reformatted
\begin{table}
\caption{\label{tab:ridge:6}Forms based on stem \textit{ia}}
\begin{tabularx}{\textwidth}{llQr}
\lsptoprule
\textbf{Series} & \textbf{\textit{ia} form} & \textbf{Gloss} & \textbf{Token count}\\
\midrule
Proximal 1 & – &  & \\
Proximal 2 & \textbf{\textit{iaxai}} & {‘there, that place, that one (near addressee)’} & {15}\\
Distal & \textbf{\textit{ioxor}} & {‘there, that place (distal)’} & {19}\\
Contrastive & – &  & \\
\lspbottomrule
\end{tabularx}
\end{table}

Distal \textit{ioxor} has the greatest functional overlap with the locative form \textit{igoxor}: it occurs as a locative adverbial in eight of the 19 tokens in the corpus. The rest of the occurrences are adnominal, but all occur in complex NPs, and in six of these \textit{ioxor} occurs in a relative clause with an unambiguously locative meaning.

Of the 15 tokens of second-person proximal \textit{iaxai} in the corpus, only one example functions as a locative adverbial. The two pronominal tokens both refer to a person rather than a place, and in both examples function as predicates, rather than arguments of a verb. The 13 other tokens all modify nominals, and predominantly occur with relative clauses and in other complex noun phrases.

The rest of the chapter will focus on the most frequent \textit{ak} and \textit{-e} forms, which are most prominent in discourse functions.

\section{Morphosyntactic distribution of \textit{ak} and -\textit{e}}
\label{sec:ridge:3}

\subsection{First-person proximal \textit{ak} forms}
\label{sec:ridge:3.1}

The bare form of first-person proximal \textit{ak} and its variant \textit{ok} are mostly used adnominally, and usually attach to the right edge of a noun phrase, but can also function as temporal adverbials. I am describing \textit{ak} as a clitic, because its distribution is syntactically determined, but it can form a single prosodic word with the preceding word (\citealt{ZwickyPullum1983}; \citealt{SpencerLuís2012}). A single main stress occurs on \textit{ak}, as final closed syllables take primary stress in Vatlongos words. However, there are examples where \textit{ak} is an independent prosodic word, especially when used adverbially, which could be an argument for analysing it as a particle instead.

In the glosses below I am aligning it as a separate word to reflect its syntactic status, and using the clitic symbol (=) when it appears to form a single prosodic form with the preceding word. However, this is based on auditory impressions and in some cases, transcribers’ decisions about where to write word boundaries, rather than acoustic analysis. The frequent combinations of \textit{ak} with the third person singular pronoun or singular quantifier \textit{xi,} and the locative stem \textit{ig} discussed above, are lexicalised forms glossed as single words.

First-person proximal \textit{ak} follows a noun \REF{ex:ridge:5} or pronoun \REF{ex:ridge:6} heading a noun phrase.

\ea
{\label{ex:ridge:5}[20170331b\_n01s034\_30]}\\
\gll atou  =ak \\
     woman  \textsc{=prox1}\\
\glt ‘this woman’
\z

\ea
{\label{ex:ridge:6}[20141107b\_n01e020\_19]}\\
\gll xalu  =ak\\
     \textsc{3du}  \textsc{=prox1}\\
\glt ‘these two’
\z

In more complex noun phrases, \textit{ak} can follow a variety of word categories, including adjectives \REF{ex:ridge:7}, quantifiers \REF{ex:ridge:8}-\REF{ex:ridge:9}, and possessive classifiers \REF{ex:ridge:10}.

\ea
{\label{ex:ridge:7}[20170222f\_n01s153\_30]}\\
\gll {\ob}ses  {\ob}xa  val{\cb}{\cb}  ak{\cb}\\
     {\db}shellfish  {\db}\textsc{rel}  big  \textsc{prox1}\\
\glt ‘this big shellfish’
\z

\ea
{\label{ex:ridge:8}[20141220g\_n01s080\_03]}\\
\gll {\ob}{\ob}horamue  nalu{\cb}    xalu{\cb}  =ak\\
     {\db}{\db}boy    \textsc{3du.poss} \textsc{du} \textsc{=prox1}\\
\glt ‘these two boys of theirs’
\z

\ea
{\label{ex:ridge:9}[20170413e\_n01m030\_21]}\\
\gll {\ob}{\ob}holesok   sap{\cb}     xil{\cb}   =ak\\
     {\db}{\db}thing    different  \textsc{pl}  \textsc{=prox1}\\
\glt ‘these different things’
\z

\ea
{\label{ex:ridge:10}[20170221e\_n01o150\_04]}\\
\gll {\ob}horamue  nan{\cb}    =ak\\
     {\db}boy    \textsc{3sg.poss}  \textsc{=prox1}\\
\glt ‘this boy of his’
\z

Similarly, \textit{ak} can follow any word at the end of a preposition phrase, like the noun in \REF{ex:ridge:11}, or at the end of a relative clause, like the preposition in \REF{ex:ridge:12}. This shows that it is syntactic boundaries, rather than word class, that determines the position of \textit{ak}, supporting the analysis as the clitic.

\ea
{\label{ex:ridge:11}[20141106f\_n01e018\_07]}\\
\gll {\ob}vul   {\ob}te   oum{\cb}{\cb}   =ak\\
     {\db}hole  {\db}of  crab    \textsc{=prox1}\\
\glt ‘this crab hole’
\z

\ea
{\label{ex:ridge:12}[20170222f\_n01s153\_49-50]}\\
\gll {\ob}venu  {\ob}xa  di  gan      e  nesau{\cb}{\cb}  ak\\
     {\db}volcano  {\db}\textsc{rel}  \textsc{cont}  \textsc{3sg.nfut.}burn  \textsc{loc}  up  \textsc{prox1}\\
\glt ‘this volcano that’s burning up there’
\z

In all the examples so far, \textit{ak} occurs at the right edge of a noun phrase. However, within a noun phrase, it can be followed by a relative clause \REF{ex:ridge:13}, a coordinator \REF{ex:ridge:14}, or \textit{mun} ‘too, also’ \REF{ex:ridge:15}.

\ea
{\label{ex:ridge:13}[20150219b\_n01m001\_44]}\\
\gll {\ob}xil  =ak{\cb}    {\ob}xa  {\ob}tahal  Maat{\cb}{\cb}\\
     {\db}\textsc{3pl}  \textsc{=prox1}  {\db}\textsc{rel}  {\db}from  Maat\\
\glt ‘the ones from Maat’
\z

\ea
{\label{ex:ridge:14}[20141106f\_n01e018\_26]}\\
\gll {\ob}{\ob}{\ob}tut  atuli{\cb}  =ak{\cb}    xal  {\ob}mama  nan{\cb}{\cb}\\
     {\db}{\db}{\db}little girl  \textsc{=prox1} with  {\db}mum \textsc{3sg.poss}\\
\glt ‘this little girl and her mum’
\z

\ea
{\label{ex:ridge:15}[20141106f\_n01e018\_21]}\\
\gll {\ob}{\ob}tatal   =ak{\cb}     mun{\cb}  be      xalu\\
     {\db}{\db}snake  \textsc{=prox1}  too  \textsc{3sg.nfut}.go\_to  \textsc{3du}\\
\glt ‘The snake also went to them.’
\z

Finally, \textit{ak} occurs as a locative adverb modifying a clause, as in \REF{ex:ridge:16}.

\ea
{\label{ex:ridge:16}[20170413a\_h01m169\_95]}\\
\gll na-taa-ra    =ti  ak\\
     \textsc{1sg.nfut-neg-}stay  \textsc{=neg}  \textsc{prox1}\\
\glt ‘I wasn’t here.’
\z

The lexicalised form \textit{xiak} consists of \textit{ak} and the form \textit{xi}, which is both the third person singular pronoun and the singular quantifier. This form can occur in many of the same syntactic environments as independent \textit{ak}. It can occur adnominally at the right edge of a noun phrase. However, compared to bare \textit{ak}, \textit{xiak} is more likely to occur when the noun phrase also contains other modifiers \REF{ex:ridge:17}.

\ea
{\label{ex:ridge:17}[20170220e\_n01s147\_12]}\\
\gll vul  venu    xiak\\
     hole  volcano  \textsc{prox1}\\
\glt ‘this volcano crater’
\z

In its adnominal function, \textit{xiak} is more likely to follow the relative clause marker \textit{xa} than to adjoin directly to the noun phrase. This strategy is also preferred for most adjectives in Vatlongos. Unlike in main clauses, adjectives, noun phrases and demonstratives in relative clauses are rarely preceded by the copular verb \REF{ex:ridge:18}.

\ea
{\label{ex:ridge:18}[20141106f\_n01e018\_15]}\\
\gll mi-gur      oum  xa  xiak\\
     \textsc{3sg.nfut-nfut.}take  crab  \textsc{rel}  \textsc{prox1}\\
\glt ‘She took this crab.’
\z

It is also fairly common for \textit{xiak} to occur after a relative clause, referring back to the head of the noun phrase, see \REF{ex:ridge:19} and \REF{ex:ridge:20}. Adnominal demonstratives are often reanalysed as relative clause boundary markers cross-linguistically \citep[132-135]{Diessel1999Book}.

\ea
{\label{ex:ridge:19}[20170222d\_n01s152\_71]}\\
\gll ueili  {\ob}xa  tati  ba  te-pou-e    e  Saot-Is{\cb}   xiak\\
     pig   {\db}\textsc{rel}  dad   \textsc{go}  \textsc{3sg.pri}{}-carry-\textsc{3obj} \textsc{loc}  Southeast  \textsc{prox1}\\
\glt ‘the pig that Dad went and got from the Southeast’
\z

\ea
{\label{ex:ridge:20}[20170315b\_n01m159\_15]}\\
\gll iat  {\ob}xa  ma-ling    joj      daen{\cb}      xiak\\
     yard   {\db}\textsc{rel}   \textsc{1pl.nfut}{}-put   church \textsc{3sg.nfut}.stay\_in  \textsc{prox1}\\
\glt ‘the yard that we put the church in’
\z

These tendencies show that while there is overlap in the adnominal functions of \textit{ak} and \textit{xiak}, they tend to occur in slightly different syntactic environments. As might be predicted on the basis of its longer form, \textit{xiak} seems to have greater syntactic prominence, and may help hearers to process syntactic boundaries in complex noun phrases.

\textit{Xiak} is used as a locative adverbial more frequently than bare \textit{ak}. It can refer to a location that is near a real or fictional place of speech, or to a place that is already activated in the discourse, see \REF{ex:ridge:21} and \REF{ex:ridge:22}. In this use it also has functional overlap with \textit{igak}, discussed in \sectref{sec:ridge:2}.

\ea
{\label{ex:ridge:21}[20141117a\_n01m003\_50]}\\
\gll taa-ve        vonine-mem        =ti  xiak\\
     \textsc{3sg.nfut.neg-neg.cop}   place-\textsc{1pl.excl.poss}   \textsc{=neg} here.\textsc{prox}1\\
\glt ‘It’s not our place here.’
\z

\ea
{\label{ex:ridge:22}[20170124b\_t01e137\_15]}\\
\gll metilou  da      xiak\\
     nephew  \textsc{3sg.nfut.}stay  here.\textsc{prox}1\\
\glt ‘Nephew is here.’
\z

Unlike bare \textit{ak}, \textit{xiak} is also used as a temporal adverbial, see \REF{ex:ridge:23} and \REF{ex:ridge:24}. In the corpus, the locative adverbial uses can be difficult to distinguish from the temporal ones, but \textit{xiak} seems to be most frequently used with a temporal interpretation. In its temporal extensions it can refer to the time of real or fictional speech, or to the time frame that is under discussion in the discourse, which may be in the past, as in \REF{ex:ridge:25}.

\ea
{\label{ex:ridge:23}[20150419e\_h01m128\_24]}\\
\gll di     a-mmei    xiak?\\
     \textsc{cont}  \textsc{2sg.nfut-}come  now\\
\glt ‘Are you coming now?’
\z

\ea
{\label{ex:ridge:24}[20170217j\_n01s125\_19]}\\
\gll eai  ba  di  mi-nu      xiak\\
     sun  \textsc{go}  \textsc{cont}  \textsc{3sg.nfut}{}-dive  now\\
\glt ‘The sun is setting now.’
\z

\ea
{\label{ex:ridge:25}[20141117a\_n01m003\_17]}\\
\gll kavmen    mi-leh      moletin   xil tu     la-mmei     Epi   xiak\\
     government   \textsc{3sg.nfut}{}-take  person   \textsc{pl} already \textsc{3pl.nfut}{}-come   Epi   now\\
\glt ‘The government had already brought people to Epi then.’
\z

\textit{Xiak} can additionally be used as a pronominal form that can function as the argument of a verb. In \REF{ex:ridge:26}, \textit{xiak} is the subject of the verb and refers to a person present in the speech situation (the researcher), who is standing beside the speaker. In \REF{ex:ridge:27}, \textit{xiak} is the object of the instrumental preposition \textit{ni} and refers to a loan already mentioned in the autobiographical narrative.

\ea
{\label{ex:ridge:26}[20141028a\_c01m002\_07]}\\
\gll xiak    bitene      va-hur      sepin-ien  nar\\
     \textsc{3sg.prox1}   \textsc{3sg.nfut.}want   \textsc{3sg.ifut-}take   talk-\textsc{nmlz}   \textsc{1pl.incl.poss}\\
\glt ‘This one wants to learn our language.’
\z

\ea
{\label{ex:ridge:27}[20170413e\_n01m030\_19]}\\
\gll na-staat-ni    pisnis  navan    ni  xiak\\
     \textsc{1sg.nfut-}start-\textsc{tr} business  \textsc{1sg.poss}  \textsc{ins}  \textsc{3sg.prox1}\\
\glt ‘I started my business with this.’
\z


\subsection{Contrastive -\textit{e} forms}
\label{sec:ridge:3.2}

I analyse contrastive \textit{-e} as a suffix rather than a clitic because it is heavily restricted in the forms it can attach to, and its syntactic distribution is dependent on the host forms, rather than independently attaching to a syntactic constituent (\citealt{ZwickyPullum1983}; \citealt{SpencerLuís2012}). Contrastive \textit{-e} is suffixed to a subset of the pronouns and formally identical quantifiers, as well as the locative stem \textit{ig} discussed in \sectref{sec:ridge:2.3}. Unlike first-person proximal \textit{ak}, syntactic requirements do not ever directly determine the position of \textit{-e}: it can only occur in positions available to the forms it is suffixed to. This is a possible counterexample to Diessel’s (\citeyear[25]{Diessel1999Book}) hypothesis that bound demonstratives are always clitics rather than affixes, although semantically the suffixed forms do always modify or substitute for a phrase as he argues.
%Is there a more elegant solution for Diessel's (1999: 25) hypothesis? how do I integrate page numbers in \citegen? \citegen[25]{Diessel1999Book} does not work.

Contrastive \textit{-e} is most frequently suffixed to \textit{xi}, which is the third person singular pronoun or definite singular quantifier, or \textit{xil}, the third person plural pronoun or plural quantifier. The singular pronoun form is shown in \REF{ex:ridge:28}, as the object of the instrumental preposition \textit{ni}. The plural pronoun form is shown in \REF{ex:ridge:29}, functioning as the subject of the verb.

\ea
{\label{ex:ridge:28}[20141116b\_c01m\_26]}\\
\gll lu-bunit          ni  xi-e\\
     \textsc{3du.nfut-nfut.}cover\_laplap  \textsc{ins}  \textsc{3sg-contr}\\
\glt ‘They covered the laplap\footnote{Laplap is a food made from grated root vegetables or banana soaked in coconut milk and baked in leaves.} with that.’
\z

\ea
{\label{ex:ridge:29}[20141208a\_n01m045\_29]}\\
\gll xil-e    la-pangei    nou\\
     \textsc{3pl-contr}  \textsc{3pl.nfut-}spoil  \textsc{1sg}\\
\glt ‘They disrespected me.’
\z

The singular and plural quantifier forms are shown in \REF{ex:ridge:30} and \REF{ex:ridge:31} , modifying a single noun.

\ea
{\label{ex:ridge:30}[20141027a\_n01m001\_109]}\\
\gll tovolih    xi-e\\
     old\_woman  \textsc{sg-contr}\\
\glt ‘that old woman’
\z

\ea
{\label{ex:ridge:31}[20150310a\_h01s114\_08]}\\
\gll meseau  xil-e\\
     fish  \textsc{pl-contr}\\
\glt ‘those fish’
\z

These forms have the same distribution within the noun phrase as unmarked quantifiers, usually following any adjectives \REF{ex:ridge:32} or possessive classifiers \REF{ex:ridge:33}. However, the contrastive form of the singular quantifier is much more frequent than the singular quantifier alone, which only occurs twice with an unambiguous quantification function. It therefore serves as a host for the contrastive demonstrative suffix, rather than making an independent semantic contribution.

\ea
{\label{ex:ridge:32}[20170220g\_n01s148\_24]}\\
\gll mai  mieh  xi-e\\
     reef  white  \textsc{sg-contr}\\
\glt ‘that white reef’
\z

\ea
{\label{ex:ridge:33}[20141027a\_n01m001\_29]}\\
\gll holesok  salu    xil-e\\
     thing  \textsc{3du.poss}  \textsc{pl-contr}\\
\glt ‘those things of theirs’
\z

\textit{Xie} is especially common at the end of long relative clauses, and more frequent than \textit{xiak} in this environment \REF{ex:ridge:34}.

\ea
{\label{ex:ridge:34}[20141220g\_n01s080\_11]}\\
\gll ma-be         {\ob}{\ob}rute   {\ob}xa   xouk       o-bit       mal-naa-va =ti en{\cb}{\cb} xi-e{\cb}\\
     \textsc{1du.excl-nfut}.go\_to     {\db}{\db}place  {\db}\textsc{rel}   \textsc{2sg}     \textsc{2sg-nfut.}say   \textsc{1du.excl.fut-neg}{}-go \textsc{=neg}    in   \textsc{sg-contr}  \\
\glt ‘We went to that place that you told us not to go to.’
\z

Like \textit{xiak}, \textit{xie} is also often introduced by the relative clause marker \textit{xa}, where it is ambiguous between a quantifier and pronominal reading \REF{ex:ridge:35}.

\ea
{\label{ex:ridge:35}[20141107b\_n01e020\_31]}\\
\gll tatal  xa  xi-e\\
     snake  \textsc{rel}  \textsc{(3)sg-contr}\\
\glt ‘that snake’
\z

\textit{Xie} can also modify a clause as a temporal \REF{ex:ridge:36} or locative \REF{ex:ridge:37} adverb, which is evidence of lexicalisation, as in the case of \textit{xiak}. Although it can again be difficult to distinguish between temporal and locative meanings in individual examples in the corpus, the temporal meaning of the adverb appears to be more frequent.

\ea
{\label{ex:ridge:36}[20141208a\_n01m045\_42]} \\
\gll mu-lul      vatang  xie\\
     \textsc{3sg.nfut-}shake  a\_lot    then\\
\glt ‘It [the earthquake] shook a lot then.’
\z

\ea
{\label{ex:ridge:37}[20141219c\_n01s033\_21]}\\
\gll xale-n    daen      xie\\
     tail-\textsc{3sg.poss}  \textsc{3sg.nfut.}be\_in\_it  there\\
\glt ‘Its tail was in there.’
\z

The morphosyntactic distribution of contrastive \textit{xie} and \textit{xile} therefore appear to encompass the various functions of both \textit{ak} and \textit{xiak} in the first-person proximal series, while making an additional distinction for number, but are not able to modify dual and paucal referents. There is a single example in the corpus of \textit{-e} modifying the first-person plural inclusive pronoun \textit{xir}.

Finally, \textit{-e} can modify the locative stem \textit{ig}, as discussed in \sectref{sec:ridge:2.3}.


\subsection{Summary}
\label{sec:ridge:3.3}

The morphosyntactic distributions of these forms are summarised in \tabref{tab:ridge:7}. The greater syntactic independence and freer distribution of first-person proximal \textit{ak} is reflected in the wider discourse functions of this series.

%formatting of table to be adjusted
\begin{table}
\caption{Summary of morphosyntactic distributions of \textit{ak} and \textit{-e} forms}
\label{tab:ridge:7}
\begin{tabularx}{\textwidth}{lXl}
\lsptoprule
\textbf{Morphosyntactic distributions} & \textbf{\textit{ak} forms} & \textbf{\textit{-e} forms}\\
\midrule 
{adnominal} & \textit{ak, xiak} & \textit{xie, xile}\\
{after relative clause marker} & \textit{xiak} & \textit{xie, xile}\\
{after complex NP} & \textit{xiak} & \textit{xie, xile}\\
{locative adverbial} & \textit{ak, xiak, igak} & \textit{xie, ige}\\
{temporal adverbial} & \textit{xiak} & \textit{xie}\\
{pronominal} & \textit{xiak} & \textit{xie, xile}\\
\lspbottomrule
\end{tabularx}
\end{table}

\section{Discourse functions of \textit{ak} and -\textit{e}}
\label{sec:ridge:4}

\subsection{Introduction}
\label{sec:ridge:4.1}

The wider morphosyntactic distribution of \textit{ak} is mirrored in its use in semantically general, high frequency discourse contexts. \textit{Ak} is the default form for anaphora, with an important role in referent tracking, especially in establishing topics. It is frequent for second mention of new referents, and for switch topics. Contrastive \textit{-e} can occur in some of these discourse environments, but also when an explicit contrast is being made, or in contexts of negative psychological affect, for example if a switch topic is also an adversary in a narrative.

\textit{Ak} forms are associated with recognitional uses, for referents that the speaker asserts to be known to the hearer, while anticipating that the information given may not be sufficient (\citealt[230]{Himmelmann1996}; \citealt[105-109]{Diessel1999Book}). These are not well-exemplified in the corpus, but were observed during fieldwork.

In this section it is sometimes useful to include a translation of the section of discourse preceding or following an example, this is given in the translation line, before or after the quotation marks.

\subsection{First-person proximal \textit{ak} forms}
\label{sec:ridge:4.2}

If any demonstrative is used in coding a referent, \textit{ak} forms are the default for anaphora. This is especially likely when establishing a topic, or for switch topics when a referent is active in the discourse, but not currently under discussion. Continuing topics are usually coded only by subject-indexing verbal prefixes.

In \REF{ex:ridge:38} (which is the fuller context for \REF{ex:ridge:27}), \textit{xiak} is first used adnominally to modify the complex noun phrase ‘money that I got’, which refers back to a loan first mentioned two pause units earlier. Then \textit{xiak} is used again to refer to the money pronominally.

\ea
{\label{ex:ridge:38}[20170413e\_n01m030\_19]}\\
\gll pipin-ni      mani    taxa        na-gur-i      xiak.\\
     \textsc{3sg.nfut}.enough-\textsc{tr}   money   \textsc{rel}       \textsc{1sg.nfut-nfut}.take-\textsc{3obj}   \textsc{prox1}\\
\gll na-staat-ni    pisnis    navan    ni  xiak\\
     \textsc{1sg.nfut-}start-\textsc{tr}   business  \textsc{1sg.poss}  \textsc{ins}  \textsc{3sg.prox1}\\
\glt ‘It was as much as I could buy with the money that I got. I started my business with this.’
\z

Adnominal marking with \textit{ak} forms is also very frequent at second mention of a referent that remains prominent in the subsequent discourse. In \REF{ex:ridge:39} (which is the fuller context for \REF{ex:ridge:8}), the two boys are introduced in a noun phrase without any demonstrative. Anaphoric \textit{ak} modifies ‘their two boys’ on the second mention, which \citet[385]{Lichtenberk1996} calls “immediate anaphora after first mention”. They are then referred to with the third-person dual pronoun \textit{xalu}.

\ea
{\label{ex:ridge:39}[20141220g\_n01s080\_02-03]}\\
\gll lu-pes     horamue   nalu     lu-be       lu. \\
     \textsc{3du.nfut}{}-bear   boy   \textsc{3du.poss}   \textsc{3du.nfut-cop} two\\

\gll lu-pes       horamue   nalu     xalu     ak, \\
     \textsc{3du.nfut}{}-bear   boy     \textsc{3du.poss}   \textsc{du}   \textsc{prox1}\\
\gll lu-laxat       xalu\\
     \textsc{3du.nfut}{}-look\_after   \textsc{3du}\\
\glt ‘They had two boys. They had these two boys of theirs, they looked after them.’
\z

This is an example of a tail-head linkage structure (\citealt{Vries2005}), described as bridging constructions in a recent typology \citep{Guérin2019}. This is a common strategy for organising discourse in Vanuatu languages (\citealt[454]{Early1994}; \citealt[426-427]{Hyslop2001}; \citealt[324-325]{Thieberger2004}; \citealt[240]{Schneider2010}; \citealt[298-299]{Brotchie2009}; \citealt[376]{Jauncey2011}). A situation or referent that is new information at the end of one section of discourse is repeated as old information at the beginning of the next section of discourse, as background for further new information. In Vatlongos, adnominal \textit{ak} often occurs in the repeated stretch of discourse.

Example \REF{ex:ridge:40}, from the beginning of an account of tropical cyclone Pam in 2015, shows \textit{ak} modifying the ‘cyclone’ that has been introduced in the preceding clause, in a topicalised clause-initial noun phrase. The topicalised noun phrase precedes the independent pronoun in subject position \textit{xamem} ‘\textsc{1pl.excl}’, and is marked prosodically with rising intonation and a pause (indicated by a comma in the transcription line).

\ea
{\label{ex:ridge:40}[20150410a\_h01s121\_02-03]}\\
\gll na-sep     ulis   =ti   vatiang  xa      bas     xir.     \\
     \textsc{1sg.ifut-}say   about   =\textsc{part}  cyclone  \textsc{rel}  \textsc{3sg.nfut.}hit \textsc{1pl.incl}\\

\gll vatiang  ok,  xamem   di ...\\
     cyclone  \textsc{prox1}  \textsc{1pl.excl}  \textsc{cont}\\
\glt ‘I’m going to talk a bit about the cyclone that struck us. This cyclone, we were … [getting news about it].’
\z

The very high frequency of \textit{ak} in these anaphoric uses suggests it could be reanalysed as an article, in opposition to \textit{tei} ‘one’, which functions like an indefinite article. This is similar to the role of the most general deictic particle in Lewo (Epi, Vanuatu) \citep[225]{Early1994}.

Similarly, \textit{xiak} is the unmarked form for the locative and temporal adverb, and can modify fairly distant situations in space and time. While in examples \REF{ex:ridge:21} to \REF{ex:ridge:24} \textit{xiak} refers to the here and now of the speech event, it can also refer to times and places under discussion regardless of distance from the speech event, as in \REF{ex:ridge:25} where \textit{xiak} refers to a time several decades earlier.

Recognitional use of \textit{ak} was very frequently observed in casual conversation in the field, in contexts like \REF{ex:ridge:41}, when a speaker is explaining something and wants to make sure that the listener has understood the intended referent. The speaker is also suggesting that the listener does know about the little house in question.

\ea
{\label{ex:ridge:41}[field notes]}\\
\gll nim  hokkorong  tang  nesau  ak\\
     house small just  up  \textsc{prox1}\\
\glt ‘that little house up there’
\z

However, there are few clear-cut examples of this in the corpus. Their low incidence may be due to the presence of the researcher at most recording events: speakers have low expectations of the researcher-as-listener’s cultural knowledge and shared frame of reference, and are more likely to introduce a referent as new information rather than assuming that the whole audience has prior knowledge. Another difficulty in identifying examples of recognitional uses in the corpus is that many possible examples could instead be analysed as anaphora, as the referent has usually been mentioned at some earlier point in a longer narrative, a difficulty observed by \citet[236]{Himmelmann1996}. \citet[235]{Himmelmann1996} also observes that distal demonstratives are far more common in recognitional uses cross-linguistically, so the role of \textit{ak} in these uses is suggestive of how general and unmarked the Vatlongos first-person proximal is in its discourse extensions. This supports the analysis that, in these uses, \textit{ak} is in a paradigmatic relationship with the contrastive, rather than with the other spatial demonstrative series.

First-person proximal\textit{ ak} is also associated with what could be described as “coercive recognitional uses”, to assert that the hearer should be familiar with the referent, even if the speaker suspects that they are not. This was very frequently observed in speech directed to me by my host families, especially in relation to kinship networks. Example \REF{ex:ridge:42} is a typical example from observation – it was obvious that I did not know which aunt was being referred to, but the use of \textit{ak} made it clear that I should have known.

\ea
{\label{ex:ridge:42}[field notes]}\\
\gll tila  nam    ak\\
     aunt  \textsc{2sg.poss} \textsc{prox1}\\
\glt ‘this aunt of yours’
\z

The forms based on \textit{ak} and \textit{-e} forms have complementary roles in discourse repair, which are important to the distinction marked in the verbal demonstratives discussed in \sectref{sec:ridge:5}. \textit{Ak} forms often mark a constituent that has been provided after a hesitation. While a speaker is searching for a word, the nominal hesitation form \textit{na} can be prosodically lengthened for as long as it takes to find the target word. When the target word is spoken, it is then often marked with the \textit{ak} form, as in \REF{ex:ridge:43}.

\ea
{\label{ex:ridge:43}[20150305h\_h01o111\_12]}\\
\gll lu-pus    na  nim  ok\\
     \textsc{3du.nfut-}see  \textsc{hes}  house  \textsc{prox1}\\
\glt ‘They saw um, this house.’
\z


\subsection{Contrastive -\textit{e} forms}
\label{sec:ridge:4.3}

The primary function of the \textit{-e} forms is to mark an explicit or implicit contrast. Often the contrast is with a referent marked with first-person proximal \textit{ak}, but not necessarily. In \REF{ex:ridge:44}, a father is giving instructions to his sons to hunt on ‘this side’ (marked with \textit{ak}), contrasted with ‘our home’, which is not in the area they should hunt in.

\ea
{\label{ex:ridge:44}[20141220g\_n01s080\_11]}\\
\gll tim  saratel    xi-e\\
     home  \textsc{1pc.incl.poss} \textsc{sg-contr}\\
\glt ‘[You must go hunt on this side (\textit{ak}) because of] our home there.’
\z

In \REF{ex:ridge:28}, the contrast is implicit, rather than explicitly stated in the discourse. The pronominal use of \textit{xie} refers to leaves of the \textit{nelnel} plant, as opposed to \textit{hahau} leaves, which are usually used in preparing laplap. Earlier in the conversation, the speaker explained how no one went to collect the \textit{hahau} leaves.

Contrastive -\textit{e} forms are also used to index negative affect, especially for situations associated with emotional distress or disrespect, and referents that are figured as adversaries in a narrative. The role of negative affect in the use of demonstratives is often figured in terms of psychological distancing. For example, \citet[225]{Early1994} aligns “spatial, temporal and psychological location” in his description of demonstratives in Lewo (Epi, Vanuatu). However, in Vatlongos the demonstrative associated with these functions does not (at least synchronically) have a spatial primary meaning, so the metaphorical rationale for this polysemy is not one of physical distance.

\textit{Xie} often marks the “enemy” character in traditional narratives, frequently an old woman \REF{ex:ridge:30} or a snake \REF{ex:ridge:35}. It can be used more generally for troublemakers, as in \REF{ex:ridge:45}, referring to an ancestor who caused a volcanic crater to open by eating forbidden sugarcane.

\ea
{\label{ex:ridge:45}[20170220e\_n01s147\_09]}\\
\gll avu    xa  xi-e    mei      pol-ni      travol\\
     grandparent   \textsc{rel}   \textsc{(3)sg-contr} come \textsc{3sg.nfut.}make-\textsc{tr} trouble\\
\glt ‘That ancestor came and made trouble.’
\z

Contrastive \textit{-e} forms are closely linked to the breaking of social conventions, especially taboos and concepts of respect and obedience, as shown in several examples above. In \REF{ex:ridge:29}, \textit{xile} refers to a group of people – a village on another island – who tricked the protagonist into sleeping with his own mother. In \REF{ex:ridge:34}, two brothers are explaining to their father that they disobeyed his command, and \textit{xie} modifies the place that they were forbidden to hunt in. In \REF{ex:ridge:37}, \textit{xie} is used as a locative adverb, and refers to the rat’s bottom, a taboo body part.

As a temporal adverb, \textit{xie} is especially associated with times of fear and distress, as in \REF{ex:ridge:36}, hardship \REF{ex:ridge:46} or discomfort \REF{ex:ridge:47}.

\ea
{\label{ex:ridge:46}[20150219b\_n01m001\_08]}\\
\gll te-staat    e  fotitu  xie\\
     \textsc{3sg.pri}-start  \textsc{loc}  42  then\\
\glt ‘It [World War Two] started in 1942 then.’
\z

\ea
{\label{ex:ridge:47}[20150129c\_i01m095\_29]}\\
\gll ut   mi-ttin     ma  na-po        rat-i    xie\\
     place   \textsc{3sg.nfut}hot   so   \textsc{1sg.ifut-}carry   out-\textsc{3obj}   then\\
\glt ‘It’s hot so I’ll take it out now.’
\z

Neither the temporal distance of the time indexed by the temporal adverb, nor the relative tense marking of the clause, seem to influence the use of \textit{xie}. In the examples here it indexes a time a few weeks ago \REF{ex:ridge:36}, many decades ago \REF{ex:ridge:46}, and just after speech time \REF{ex:ridge:47}, and modifies verbs marked with the non-future, the prior, and the immediate future respectively.

However, \textit{xie} as a temporal adverb is especially likely to co-occur with strategies for marking completed situations. This could be an extension of the basic contrastive meaning of the \textit{-e} forms, as an emphasis on the boundaries of events allows them to contrast with other events in a sequence. In \REF{ex:ridge:48} and \REF{ex:ridge:49}, \textit{xie} co-occurs with the adverb \textit{turei} ‘already’. Example \REF{ex:ridge:49} also includes the use of the verb ‘finish’ which is a high frequency strategy for marking completive aspect in Vatlongos. Example \REF{ex:ridge:50} uses the combination of negative polarity and \textit{mu} ‘first’ to express ‘not yet’.

\ea
{\label{ex:ridge:48}[20141219c \_n01s033\_08]}\\
\gll ale  lanstaem  turei    xie,  va      ra-vanien \\
     so   lunchtime   already   then   \textsc{go} \textsc{1pl.incl.ifut}{}-eat\\
\glt ‘Okay, it’s lunchtime already now, let’s go eat.’
\z

\ea
{\label{ex:ridge:49}[20141212f\_n01s054\_14]}\\
\gll meu     na-pein-ni     xouk   turei       bus       xie \\
     megapode   \textsc{1sg.nfut}-paint-\textsc{tr}   \textsc{2sg}   already \textsc{3sg.nfut.}finish   then  \\
\glt ‘Megapode I’ve already finished painting you now.’
\z

\ea
{\label{ex:ridge:50}[20150219b\_n01m001\_84]}\\
\gll lut-naa-ve    pasta  =ti  mu  xie\\
     \textsc{3du.pri-neg-cop}   pastor \textsc{=neg}   first   then\\
\glt ‘They weren’t pastors yet then.’
\z

Contrastive \textit{-e} forms have a different role to play in discourse repair. Whereas first-person proximal \textit{ak} forms mark a successful repair after a hesitation, \textit{{}-e} forms modify placeholders which are used instead of more specific formulations, or because a speaker has decided not to continue searching for an intended lexical item. In \REF{ex:ridge:51}, \textit{xie} modifies the general nominal placeholder \textit{neta} ‘thing’ (which follows the bound noun \textit{relit} ‘egg’, obligatorily followed by a possessor). At no point in the narrative is this mysterious object, which later becomes the volcano, given a precise name, although it is compared to a chicken’s egg. In a different telling of the same narrative it is instead referred to as a shell.

\ea
{\label{ex:ridge:51}[20170220g\_n01s148\_38]}\\
\gll na  relit  neta  xa  xi-e\\
     \textsc{hes}  egg  thing  \textsc{rel}  \textsc{(3)sg-contr}\\
\glt ‘[They took out] um that thing’s egg.’
\z


\subsection{Summary}
\label{sec:ridge:4.4}

\tabref{tab:ridge:8} summarises the discourse functions described in this section. It is striking that \textit{ak} forms have many more discourse functions than contrastive \textit{-e} forms, in addition to their role in the spatial paradigm. This large functional extension suggests a reason for the greater morphosyntactic flexibility of \textit{ak}, which has more functions to perform in a variety of discourse contexts.

This section has shown the complementary roles of these two demonstrative series in the mechanics of discourse repair after a hesitation or false start. First-person proximal \textit{ak} is used for discourse repair, when a referent is repeated after a false start. Contrastive \textit{-e} is instead used with placeholders when the speaker is searching for a word. This distinction is important to understanding how the restricted functional distribution of the verbal demonstrative \textit{mue} discussed in \sectref{sec:ridge:5.4} could have arisen.

%adjust table formatting
\begin{table}
\caption{Summary of discourse functions of \textit{ak} and -\textit{e} forms}
\label{tab:ridge:8}
\begin{tabularx}{.8\textwidth}{Xcc}
\lsptoprule
\textbf{Discourse functions} & \textbf{\textit{ak} forms} & \textbf{\textit{-e} forms}\\
\midrule
{default anaphora} & {✔} & \\
{establishing topics} & {✔} & \\
{second mention} & {✔} & \\
{switch topics} & {✔} & \\
{contrastive topics} &  & {✔}\\
{repeated content in tail-head linkage} & {✔} & \\
{recognitional uses} & {✔} & \\
{coercive recognitional uses} & {✔} & \\
{successful repair after hesitation} & {✔} & \\
{placeholders for failed repairs} &  & {✔}\\
{negative affect} &  & {✔}\\
\lspbottomrule
\end{tabularx}
\end{table}

\section{Demonstrative verbs}
\label{sec:ridge:5}

\subsection{Introduction}
\label{sec:ridge:5.1}

In addition to the adnominal, adverbial and pronominal forms described so far, Vatlongos has a verbal manner demonstrative \textit{mak} based on first-person proximal \textit{ak}, and a verbal hesitation form \textit{mue} which could have developed from contrastive -\textit{e}, in line with the contrast between \textit{ak} and -\textit{e} forms involved in discourse repair (\tabref{tab:ridge:9}). There is also one token in the corpus of a verb \textit{maxai}, a second-person proximal \textit{xai} form.

Paamese, the most closely related language, also has verbal demonstratives, which \citet[229]{Crowley1982} analyses as comprising of a deictic verbal root \textit{muko}, and the demonstrative clitics, \textit{=ke} (proximal) and \textit{=neke} (distal). Unlike in Vatlongos, the verbal root can be separated from the demonstrative elements by an intervening negative-partitive suffix and the additive clitic, but \textit{muko} cannot occur without one of these two clitics.

%adjust formatting of the table
\begin{table}
\caption{Verbal demonstrative forms in Vatlongos}
\label{tab:ridge:9}
\begin{tabularx}{\textwidth}{lXlr}
\lsptoprule
\textbf{Series} & \textbf{Verb} & \textbf{Gloss} & \textbf{Token count}\\
\midrule 
Proximal 1 & \textbf{\textit{mak}} & {‘like this’} & {328}\\
Proximal 2 & \textbf{\textit{maxai}} & {‘like that’ (associated with addressee)} & {1}\\
Distal & \textbf{\textit{{}-}} &  & \\
Contrastive & \textbf{\textit{mue}} & {hesitation} & {227}\\
\lspbottomrule
\end{tabularx}
\end{table}

\subsection{First-person proximal \textit{mak}}
\label{sec:ridge:5.2}

The first-person proximal verbal form \textit{mak} functions as a manner demonstrative. It is very frequent, occurring roughly once every 150 words (328 tokens in the subcorpus). Its high frequency and broad range of functions mirrors the extension of \textit{ak} in identifying discourse referents, but instead refers to manner.

As a main verb it is especially common in procedural texts, referring to a manner being demonstrated non-verbally. In \REF{ex:ridge:52}, the speaker is demonstrating how to make simboro, grated root vegetables or banana rolled in cabbage leaves and cooked in coconut milk.

\ea
{\label{ex:ridge:52}[20141212g\_p01s046\_05]}\\
\gll mak,  di,  mak   tang\\
     \textsc{3sg.nfut.}like\_this  then  \textsc{3sg.nfut.}like\_this  just\\
\glt ‘Like this, then, just like this.’
\z

\textit{Mak} is also often used to introduce performances, as in \REF{ex:ridge:53} where it first introduces a rope drawing of a nut, and then the song that accompanies the drawing. When a performance is verbal, this can be thought of as discourse deixis, referring to the words of the song. Although in these two examples the direction of reference is cataphoric, \textit{mak} can also refer backwards in time to a completed performance.

\ea
{\label{ex:ridge:53}[20141222c\_p01s087]}\\
\gll tavu  mak:  […]    ale  sisien  nen    mak: […]\\
     bushnut \textsc{3sg.nfut.}like\_this {}  so   song   of\_it \textsc{3sg.nfut.}like\_this\\
\glt ‘The bushnut goes like this: [rope drawing]. Then the song of it goes like this: [song].’
\z

It is also used for anaphoric discourse deixis referring to situations expressed in the preceding discourse. In \REF{ex:ridge:54}, \textit{mak} refers to the preceding sentence \REF{ex:ridge:26}, i.e. that I want to learn Vatlongos, as an explanation for the speaker’s actions. The use of discourse deictic \textit{mak} followed by \textit{ma} ‘so, then’ is a fairly frequent strategy for expressing causal or purposive links between situations in the discourse.

\ea
{\label{ex:ridge:54}[20141028a\_c01m002\_08]}\\
\gll mak      ma  na-kes      xamim\\
     \textsc{3sg.nfut.}like\_this   so   \textsc{1sg.nfut-}call   \textsc{2pl}\\
\glt ‘It’s like this so I called you.’ (i.e. ‘That’s why I called you.’)
\z

This discourse deictic use can also refer to the speech of another interlocutor in a conversation, as in \REF{ex:ridge:55}. \textit{Mak} collocates with \textit{tang} ‘just’, often to express a general affirmation, which could be translated as ‘that’s it’ or ‘that’s the way’ in English. This collocation occurs 27 times in the corpus, representing nearly a tenth of tokens of \textit{mak}, and is probably even more frequent in casual conversation, which is underrepresented in the corpus.

\ea
{\label{ex:ridge:55}[20150129b\_i01m094\_15]}\\
\gll mu-tin      nahou    tang –\\
     \textsc{1pl.excl.dfut-}burn  garden  just\\
     
\gll a    bos,      mak      tang\\
     oh   \textsc{3sg.nfut.}good   \textsc{3sg.nfut.}like\_this   just\\
\glt (A:) ‘We’ll just burn the garden.’ – (B:) ‘Oh good, that’s the way.’
\z

However \textit{mak} most often occurs as the subsequent verb in a serial verb construction, modifying the manner of the situation expressed by the initial verb. \textit{Mak} always takes third person singular subject agreement marking in these constructions, which is the default for event-argument\footnote{I am following Aikhenvald’s (\citeyear[18-19]{Aikhenvald2006}) typological framework in using this term, rather than “ambient”, which is used by Crowley to describe the same argument structure configuration.} serialisation in both Vatlongos (\citealt{Ridge2019}:§6) and Paamese (see \citet[49]{Crowley1987} and (\citeyear[61]{Crowley2002}) on ambient core-layer serial verb constructions in Paamese). In Vatlongos serial verb constructions, the relative tense marking on the initial verb determines the marking on the subsequent verb. Matching relative tense marking is required in all the examples of serialised \textit{mak} here, because they are in affirmative polarity.

In serialisation, \textit{mak} can be used exophorically to refer to the external situation, as in \REF{ex:ridge:56}, from a conversation recorded while eating laplap on a Sunday.

\ea
{\label{ex:ridge:56}[20141116b\_c01m\_54]}\\
\gll xir,  evri  Sade    ra-ga      laplap  mak\\
     \textsc{1pl.incl} every Sunday   \textsc{1pl.incl.nfut}{}-eat   laplap \textsc{3sg.nfut.}like\_this\\
\glt ‘Us, every Sunday we eat laplap like this.’
\z

Serialised \textit{mak} also refers to non-verbal demonstrations, both in procedural demonstrations \REF{ex:ridge:57}, and for more spontaneous demonstrations and iconic gestures. In \REF{ex:ridge:58}, the speaker picks up a banana to demonstrate how a character in a narrative blocks a snake’s mouth with a nut.

\ea
{\label{ex:ridge:57}[20141212g\_p01s046\_04]}\\
\gll la-pis-i      mak\\
     \textsc{3pl.nfut-}roll-\textsc{3obj}  \textsc{3sg.nfut.}like\_this\\
\glt ‘They roll it like this.’
\z

\ea
{\label{ex:ridge:58}[20141220g\_n01s080\_52]}\\
\gll tati  natel    gur      huit  tavu    tei  mak\\
     dad   \textsc{3pc.poss}   \textsc{3sg.nfut.}take   fruit   bushnut one   \textsc{3sg.nfut}.like\_this\\
\glt ‘Their Dad took a bushnut like this.’
\z

Serialised \textit{mak} can also be used for discourse deixis. In \REF{ex:ridge:59}, the preceding stretch of discourse describes how the dwarf has been pulling the rat’s tail, and \textit{mak} refers back to that action as the manner in which the dwarf is trying to get the rat out of its hole.

\ea
{\label{ex:ridge:59}[20170331c\_n01s140\_60]}\\
\gll mi-sak    pis-i    va-leh      rat  asu  va-mak\\
     \textsc{3sg.nfut}{}-do   try-\textsc{3obj}   \textsc{3sg.ifut-}take   out   rat \textsc{3sg.ifut-}like\_this\\
\glt ‘[He pulled and pulled and pulled…] he tried to take out the rat like this.’
\z

Finally, \textit{mak} can also have a recognitional use. In \REF{ex:ridge:60}, the speaker uses \textit{mak} to refer to the way in which young people often just stay home, assuming the audience’s shared knowledge of the common social problem of young people hanging around without enough to do.

\ea
{\label{ex:ridge:60}[20170406a\_n01m164\_61]}\\
\gll yangfala    xil,  samtaem  ra-di      mak\\
     youth     \textsc{pl}   sometimes   \textsc{1pl.incl-nfut.}stay \textsc{3sg.nfut.}like\_this\\
\glt ‘Young people, sometimes we stay like this.’
\z

Because \textit{mak} has a bilabial onset, it usually does not take the non-future third person singular subject-indexing prefix \textit{mi}, instead taking a zero allomorph. This means that it most often appears in its unprefixed form, as in all the examples here except \REF{ex:ridge:59}. This bridging context seems to have allowed reanalysis as an invariant adverb. There are five unambiguous tokens in the corpus in tense and person contexts where prefixes are expected, but do not occur. However, in the non-future with a third person singular subject it is usually not possible to disambiguate the verbal and adverbial lexemes.


\subsection{Second-person proximal or contrastive \textit{maxai}}
\label{sec:ridge:5.3}

While there is no evidence in the corpus or observation of any verbal forms based on distal \textit{xor}, there is also one example of a verbal manner demonstrative \textit{maxai} in the corpus, which appears to be the second-person proximal \textit{xai} form. Example \REF{ex:ridge:61} is reported speech from the speaker’s sister-in-law, instructing her (the addressee of the reported speech) not to roll simboro in the manner she was doing it, but instead in the manner the sister-in-law demonstrates. \textit{Maxai} is the subsequent verb in the serial verb construction, modifying the initial verb phrase ‘roll simboro’. While both verbs take negative polarity immediate future prefixes, the negative clitic \textit{ti} only appears after the initial verb phrase, a strong diagnostic of serialisation, as \textit{naa-maxai} would be ungrammatical as independent clause without the negative clitic.

\ea
{\label{ex:ridge:61}[20141212g\_p01s046\_11]}\\
\gll a-naa-pis     siboro   =ti       naa-maxai         e   u-pis       siboro   i-xoni       ngan =ak\\
     \textsc{2sg.ifut}{}-\textsc{neg}{}-roll   simboro   \textsc{=neg} \textsc{3sg.ifut.neg}-like\_that.\textsc{prox2}   but   \textsc{2sg.dfut}{}-roll dumpling   \textsc{3sg.dfut}{}-like   one =\textsc{prox1}\\
\glt ‘You don’t roll simboro like that, you roll it like this one.’
\z


\subsection{Verbal hesitation form \textit{mue}}
\label{sec:ridge:5.4}

The verbal form \textit{mue} is probably based on contrastive \textit{-e}. Rather than functioning as a verbal manner demonstrative, this is a hesitation form used when a speaker is searching for a verbal lexeme, which is usually supplied by the speaker immediately afterwards. Alternatively a speaker might break off after \textit{mue}, and use a different sentence structure altogether in the following sentence. It is very frequent, occurring roughly once every 210 words in the corpus (227 tokens).

\textit{Mue} is fully inflected for the intended subject person-number and relative tense features, and can be modified by auxiliary verbs \REF{ex:ridge:64}.

\ea
{\label{ex:ridge:62}[20141208a\_n01m045\_25]}\\
\gll mue,    muis      vatang\\
     \textsc{3sg.nfut.hes}  \textsc{3sg.nfut.}cry   a\_lot\\
\glt ‘He um, he cried a lot.’
\z

\ea
{\label{ex:ridge:63}[20150223a\_n01m096\_59]}\\
\gll ral-mue,      rali-pol\\
     \textsc{1du.incl.dfut-hes}  \textsc{1du.incl.dfut}{}-work\\
\glt ‘We’ll um, we’ll work.’
\z

\ea
{\label{ex:ridge:64}[20150303d\_p01e016\_03]}\\
\gll di    ma-mue,    di        ma-ga-ni\\
     \textsc{cont}   \textsc{1pl.excl.nfut-hes} \textsc{cont}       \textsc{1pl.excl.nfut-nfut.}eat-\textsc{3obj}\\
\glt ‘We um, we eat it.’
\z

\textit{Mue} can stand in for both intransitive \REF{ex:ridge:62}, \REF{ex:ridge:63} and transitive \REF{ex:ridge:64} clauses. When a transitive clause is intended, it can also be followed by the nominal placeholder \textit{neta} ‘thing’, see \REF{ex:ridge:51}. This can be used whether the intended object will be indexed by the object pro-indexing suffixes \REF{ex:ridge:65}, or an independent noun phrase \REF{ex:ridge:66}.

\ea
{\label{ex:ridge:65}[20141027a\_n01m001\_18-19]}\\
\gll lu-mue    neta,  lu-pangas-i\\
     \textsc{3du.nfut-hes}  thing  \textsc{3du.nfut-}burn-\textsc{3obj}\\
\glt ‘They um what, they burnt it.’
\z

\ea
{\label{ex:ridge:66}[20170331c\_n01s140\_15]}\\
\gll asu  mue      neta,  mi-gur    vatupang\\
     rat   \textsc{3sg.nfut.hes}   thing \textsc{3sg.nfut-nfut.}take fire\_stick\\
\glt ‘The rat um what, took the fire stick.’
\z

\textit{Mue} has a bilabial initial and frequently appears as a bare stem in the non-future with a third person singular subject, as in \REF{ex:ridge:62} and \REF{ex:ridge:66}. Like \textit{mak}, there are signs that this has been a bridging context for reanalysis as a non-verbal form. There are eight tokens in the corpus where a verbal reading is unlikely, and \textit{mue} instead seems to be acting as a more general hesitation marker, with the syntactic distribution of an interjection.

Despite the formal similarity, and the evidence from Paamese of a deictic verbal root with an initial /mu/, it is difficult to assert a definite relationship between this form and the contrastive demonstrative suffix \textit{-e}, because it does not function as a typical verbal manner demonstrative like \textit{mak}. It is therefore important to ascertain a plausible path of reanalysis from the demonstrative uses of \textit{-e} to the function of a hesitation marker. \citet[234-235]{Himmelmann1996} observes that demonstratives are often used as fillers, and suggests that hesitation phenomena are closely associated with the recognitional uses of demonstratives. While \textit{ak} is the primary strategy for recognitional use in Vatlongos, the complementary roles of \textit{ak} and \textit{-e} forms in discourse repair suggest a path for the verbal \textit{-e} form to function as a hesitation marker. In examples like \REF{ex:ridge:51}, \textit{-e} modifies a placeholder for a lexical item that is currently inaccessible to the speaker, a close parallel to the role of verbal \textit{mue}.


\subsection{Summary}
\label{sec:ridge:5.5}

\tabref{tab:ridge:10} summarises the functions of the verbal demonstrative forms. Here the imbalance in functional load between the first-person proximal series and the other forms is even more striking than in the adnominal, adverbial and pronominal forms discussed in \sectref{sec:ridge:4}. There is a single example of a second-person proximal verbal form, and the contrastive series is restricted to a hesitation form.

%adjust table formatting
\begin{table}
\caption{\label{tab:ridge:10}Summary of functions of verbal demonstratives}
\begin{tabularx}{\textwidth}{Qccc}
\lsptoprule
\textbf{Functions} & \textbf{\textit{mak}} & \textbf{\textit{maxai}} & \textbf{\textit{mue}}\\
\midrule 
referring to physical demonstrations by speaker & ✔ &  & \\
referring to manner of situational context & ✔ &  & \\
referring to manner in the preceding discourse & ✔ &  & \\
modifying manner of another verb & ✔ &  & \\
recognitional use & ✔ &  & \\
referring to manner of actions of addressee &  & ✔ & \\
{hesitation} &  &  & {✔}\\
\lspbottomrule
\end{tabularx}
\end{table}

\section{Conclusion}
\label{sec:ridge:6}

There are four series of Vatlongos demonstratives: in addition to first-person proximal, second-person proximal and distal forms, there is a contrastive series that has not been previously described. These can be thought of as a person-based spatial paradigm, distinguishing first-person proximal from second-person proximal and distal series, and a contrastive paradigm where the first-person proximal series is distinguished from the contrastive series. The participation of the first-person proximal series in both paradigmatic distinctions is one reason for its greater frequency and wider discourse functions than the other demonstrative forms.

In discourse functions Vatlongos demonstratives are organised around a two-way distinction between forms based on the clitic \textit{ak}, and contrastive forms based on the suffix -\textit{e.} This chapter has described how asymmetries in the morphosyntactic status and distribution of these forms are reflected in their functional extensions. Whereas the first-person proximal clitic \textit{ak} can freely modify both noun phrases and clauses, contrastive \textit{-e} can only modify singular and plural noun phrases, usually in the third person, and its distribution is dependent on the host forms it attaches to, usually pronouns and quantifiers.

This asymmetry is mirrored in the more general and frequent contexts of use for the first-person proximal forms. First-person proximal \textit{ak} forms are the default for anaphora and recognitional demonstrative uses, while \textit{-e} forms are used for explicit contrast and negative affect. The negative affect meaning component is unusual in that it is not based on metaphorical distance: the primary meaning of these forms is contrastive rather than distance-based, so opposition seems to be a more relevant dimension for this connotation.

Vatlongos demonstrative verbs show an even starker asymmetry between these series. First-person proximal \textit{mak} is by far the most frequent, used as a general verbal manner demonstrative, especially in serial verb constructions. There is also one example of second-person proximal \textit{maxai}. \textit{Mue}, the form probably based on contrastive \textit{-e}, is restricted to a verbal hesitation form, reflecting the role of \textit{-e} forms in modifying nominal placeholders. The role of the \textit{ak} and -\textit{e} series in discourse repair in nominal and verbal contexts is summarised in \tabref{tab:ridge:11}.

%adjust table formatting
\begin{table}
\caption{The roles of \textit{ak} and \textit{-e} in discourse repair}
\label{tab:ridge:11}
\begin{tabularx}{\textwidth}{lQQ}
\lsptoprule
\textbf{Syntactic environment} & \textbf{\textit{ak} series} & \textbf{-\textit{e} series}\\
\midrule 
nominal & \textit{ak} modifies successful repair after hesitation & {\textit{xie} modifies placeholder for failed repair}\\
{verbal} & \textit{mak} refers to physical demonstrations & {\textit{mue} is a placeholder hesitation form during search for verbal lexeme}\\
\lspbottomrule
\end{tabularx}
\end{table}

In summary, Vatlongos discourse demonstratives show how distinctions in semantic extension, discourse functions, and markedness interact with the morphosyntactic status and distribution of demonstrative forms, and the wider syntactic structures available in a language.

\section*{Abbreviations}
In addition to Leipzig glossing rules, the following abbreviations are used: 

\noindent
\begin{tabularx}{.45\textwidth}{lQ}
\textsc{come} & auxiliary of prior motion towards deictic centre \\
\textsc{cont} & continuous (habitual or progressive)\\
\textsc{contr} & contrastive\\
\textsc{dfut} & distant future\\
\textsc{go} & auxiliary of prior motion away from deictic centre\\
\textsc{hes} & hesitation form\\
\end{tabularx}
\begin{tabularx}{.45\textwidth}{lQ}
\textsc{ifut} & immediate future relative tense\\
\textsc{nfut} & non-future relative tense\\
\textsc{part} & partitive\\
\textsc{pc} & paucal \\
\textsc{pri} & prior relative tense\\
\textsc{prox}1 & first-person proximal (near speaker)\\
\textsc{prox}2 & second-person proximal (near addressee)\\
\end{tabularx}

\sloppy\printbibliography[heading=subbibliography,notkeyword=this]
\end{document}
