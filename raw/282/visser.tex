\documentclass[output=paper,colorlinks,citecolor=brown]{langscibook}
\author{Eline Visser\affiliation{Lund University}}
\markuptitle{Tracking and recognitional use of Kalamang \textit{opa}: demonstrative of cognitive accessibility}{Tracking and recognitional use of Kalamang ``opa'': demonstrative of cognitive accessibility}

% \title{\texorpdfstring{Word formation and word history:\\ The case of
%  \textsc{capitalist} and \textsc{capitalism}}{Word formation and word
% history:  CAPITALIST and
% CAPITALISM}}

\renewcommand{\lsCollectionPaperFooterTitle}{Tracking and recognitional use of Kalamang \noexpand\textit{opa}: demonstrative of cognitive accessibility}


\abstract{The Papuan language Kalamang has an adnominal demonstrative \textit{opa}, which marks referents that are cognitively accessible either because they are common ground or because they are previously mentioned in the discourse. It is mainly an endophoric demonstrative which has tracking and recognitional uses, but it can be used exophorically as well. It can be used for both maintaining and reintroducing referents. The demonstrative \textit{opa} is co-lexified with a time adverb \textit{opa} `just now; earlier'. A grammaticalisation path cannot be established, but similar co-lexifications are found in Bunaq (Timor-Alor-Pantar) and Indonesian (Austronesian).}


\IfFileExists{../localcommands.tex}{
  \input{../localpackages}
  %Copy this to localcommands.tex

\usepackage[english]{babel}
\usepackage{amsmath}
\usepackage{amssymb,amsfonts,textcomp}
\usepackage{array}
\usepackage{hhline}
\usepackage{hyperref}

\newenvironment{styleStandard}{}{}
\newenvironment{stylelsAbstract}{}{}
\newenvironment{stylelsSectioni}{}{}
\newenvironment{stylelsSectionii}{}{}
\newenvironment{stylelsBulletList}{}{}
\newenvironment{styleBibliographyi}{}{}
\newenvironment{listWWNumxxvleveli}{}{}
\newenvironment{listWWNumxxvlevelii}{}{}
\newenvironment{listWWNumxxvleveliii}{}{}
\newenvironment{listWWNumxxvleveliv}{}{}
\newenvironment{listWWNumixleveli}{}{}
\newenvironment{listWWNumixlevelii}{}{}
\newenvironment{listWWNumixleveliii}{}{}
\newenvironment{listWWNumixleveliv}{}{}

\newcommand\textstyleListLabelxvi[1]{#1}
\newcommand\labellistWWNumxxvleveli{\thelistWWNumxxvleveli.}
\newcommand\labellistWWNumxxvlevelii{\thelistWWNumxxvlevelii.}
\newcommand\labellistWWNumxxvleveliii{\thelistWWNumxxvleveliii.}
\newcommand\labellistWWNumxxvleveliv{\thelistWWNumxxvleveliv.}
\newcommand\labellistWWNumixleveli{[F0B7?]}
\newcommand\labellistWWNumixlevelii{\textstyleListLabelxvi{o}}
\newcommand\labellistWWNumixleveliii{[F0A7?]}
\newcommand\labellistWWNumixleveliv{[F0B7?]}

\newcounter{listWWNumxxvleveli}
\newcounter{listWWNumxxvlevelii}[listWWNumxxvleveli]
\newcounter{listWWNumxxvleveliii}[listWWNumxxvlevelii]
\newcounter{listWWNumxxvleveliv}[listWWNumxxvleveliii]
\newcounter{itemize} 
  \input{../localhyphenation}
  \togglepaper[1]%%chapternumber
}{}

%To be repaired: UsherSchapper reference, change nodate > in preparation in in-line text reference and bibliography
%Some notes are still addressed to the author
%orphan control
%keep examples together, command \protectedex does not seem to work

\begin{document}
\maketitle
\shorttitlerunninghead{Tracking and recognitional use of Kalamang \textup{opa}}

\section{Introduction}
Kalamang, a Papuan language of the West Bomberai family spoken in the east of Indonesia (ISO 639-3 code: kgv), has an adnominal demonstrative \textit{opa}. It is typically used to refer to or establish common ground, or to indicate that the referent has been mentioned before. In \REF{exe:canam}, the referent \textit{canam} `man' is mentioned at minute 1:23. The same referent re-occurs about a minute later, at 2:25, and is then marked with \textit{opa} to indicate it is the same man as at 1:23.

\ea\label{exe:canam}
	\ea\label{exe:canamat}[stim24\_1:23]\\
	\gll ma 	canam-at kon-i 		koluk \\
	\textsc{3sg} 	man-\textsc{obj} one-\textsc{num.acc}	meet \\
	\glt	`She meets a man.' 
	\ex\label{exe:canamopac}[stim24\_2:25]\\
	\gll ma 	canam 	\textbf{opa} 	me 		koluk \\
	\textsc{3sg} 	man 		{\glopa} 	{\glme} 	meet \\
	\glt	`She meets that man (that I've mentioned before).'
	\z
	\z
	
With help of the last twenty years' advancements in the study of the semantics and pragmatics of demonstratives, I carry out an analysis of the functions of Kalamang adnominal \textit{opa}. \sectref{sec:backg} provides an overview of the study of demonstratives, including an outline of exophoric and endophoric uses of demonstratives, and discusses notions such as attention direction, shared knowledge and accessibility. \sectref{sec:overv} discusses the Kalamang demonstrative system, focusing on the proximal and distal roots \textit{wa} and \textit{me}. In \sectref{sec:morph}, the formal (i.e., morphosyntactic) behaviour of \textit{opa} is discussed, and it is shown that \textit{opa} is part of the adnominal demonstrative paradigm. \sectref{sec:funct} is the core of this chapter, and goes into detail about the behaviour of \textit{opa}. Adnominal demonstrative \textit{opa} is co-lexified with a time adverb \textit{opa} `just (now); earlier', which occurs at the edge of the clause. This co-lexification of time adverb and demonstrative, or similar functions, is also found in Indonesian and the Timorese language Bunaq. This is discussed in \sectref{sec:typ}. Conclusions are drawn in \sectref{sec:concl}.

All data is drawn from the Kalamang corpus, which is based on field work by the author. The corpus contains spontaneous conversations and narratives (abbreviated “conv" or “narr"), stimulus-based conversations such as director-matcher tasks and stimulus-based narratives such as the Pear Story (abbreviated “stim") and elicitation (abbreviated “elic"). Examples from recorded natural speech are coded according to the genre and number of the recording and the time stamp of the utterance (for example, [conv8\_2:03]), whereas elicited examples are coded according of the convention of [elic\_dem] (where the second part refers to the topic of elicitation). All sources can be found using these codes in the Language Archive of Lund University.\footnote{\url{https://corpora.humlab.lu.se/ds/asv/?4&openhandle=hdl:10050/00-0000-0000-0003-C3E8-1}. Recordings will be uploaded in the course of 2020.}


\section{Background}
\label{sec:backg}
Demonstratives are forms that refer to an entity (a referent or event) by locating it in space, time or discourse. Traditionally, demonstratives are thought of as basically encoding spatial distinctions \citep{AndersonKeenan1985,Lyons1977,Diessel1999Book,Dixon2003,Fillmore1997,Kemmerer1999},\footnote{Most of these authors recognise there is a bias in descriptive grammars towards analysing demonstrative systems as being spatial, e.g. \cite{AndersonKeenan1985,Lyons1977,Dixon2003,Diessel1999Book}.} with a difference between person-oriented and distance-oriented systems \citep[e.g.][]{Lyons1977}. Later work, like~\cite{Ozyurek1998} and~\cite{Burenhult2003}, offers broader and more detailed analyses of the discourse functions of demonstratives.

%@Author: Reconsider rephrasing the following sentence: A demonstrative is used for tracking if it is used for established referents, to help the addressee keep track of what happens to whom. 
Endophoric demonstratives commonly include demonstratives that are used for introducing, identifying and tracking referents in discourse. Endophoric demonstratives refer within the text, and are used to create textual cohesion \citep[33; 57-76]{HallidayHasan1976}. \cite{Himmelmann1996} distinguishes, among others, tracking and recognitional use of demonstratives. A demonstrative is used for tracking if it is used for established referents, to help the addressee keep track of what happens to whom. Demonstratives in recognitional use are employed for unestablished referents, but refer to specific shared knowledge between speaker and addressee. In other words, recognitional use signals to the addressee that the referent is common ground (“mutual knowledge, mutual beliefs, and mutual assumptions" \citep[127]{ClarkBrennan1991}). In this chapter, I use the terms “common ground" and “shared knowledge" interchangeably.

Exophoric demonstratives refer to entities in the speech situation \citep[33]{HallidayHasan1976}. Twenty years of research on exophoric demonstratives has shown that there is disagreement on the precise nature of demonstrative semantics. While \cite{Diessel2006} maintains that distance is the semantic core of demonstratives, others have suggested that attention status \citep{Ozyurek1998} or accessibility \citep{Burenhult2003} are the primary semantic properties encoded by demonstratives in some languages. In Jahai (Austro-Asiatic), some demonstratives can be analysed in terms of cognitive accessibility. These demonstratives refer to referents which have the addressee's current or previous attention or knowledge \citep[366]{Burenhult2003}, but they are strictly exophoric in use, relating to physical accessibility in the speech situation. Accessible referents are also likely to be reachable, approachable, perceptible and close. Although such physical characteristics play an important role in accessibility, \cite[31]{Levinson2018} notes that it ``is also conceptual and refers to whether a referent is or is not in the common ground or attention of speaker and addressee''. Accessibility is also noted by \cite{Hanks2005,Hanks2011} to be commonly marked by demonstratives. He stresses the non-physicality of the notion of accessibility, linking it to shared knowledge (``mutual knowledge'' in his words), memory, and prior discourse \citep{Hanks2005}, thus removing the term accessibility from the sphere of exophoric demonstratives and relating it to endophoric demonstratives, like \cite{Levinson2018}.


\section{Overview of Kalamang demonstratives}
\label{sec:overv}
Kalamang word order is SOV, with nominative-accusative alignment. Verbs are largely unmarked for tense, aspect and modality. The noun phrase is left-headed. \tabref{tab:demwc} lists the demonstrative roots and their syntactic uses.

\begin{table}
	\caption{Kalamang demonstratives and their syntactic use}
	\label{tab:demwc}
	\begin{tabular}{l ccc}
		\lsptoprule 
		form&pronominal&adnominal&identificational\\
		\midrule
		\textit{wa} proximal& $+$ & $+$ & $+$\\
		\textit{me} distal&  $+$ & $+$ & $+$\\
		\textit{owa} far distal&&$+$&$+$\\
		\textit{yawe} elevational `down'&&$+$&$+$\\
		\textit{osa} elevational `up'&&$+$&$+$\\
		\textit{opa} demonstrative &  & $+$&\\ 
		\lspbottomrule	
	\end{tabular}
\end{table}

Kalamang has two commonly used demonstrative roots: \textit{wa} for proximal reference, and \textit{me} for distal reference. These are exophoric demonstratives, referring to entities in the speech situation. A third root, far distal \textit{owa}, is much more limited in use. \textit{Wa} and \textit{me} can be used pronominally, adnominally and identificationally. Pronominal demonstratives occur instead of nouns, adnominal demonstratives co-occur with nouns, and identificational demonstratives occur in copular and non-verbal clauses \citep[4]{Diessel1999Book}. Adnominal demonstratives occur in the right periphery of the noun phrase. The same form is used in all functions, as illustrated in \REF{exe:demadd} to \REF{exe:demidd}. The difference between \textit{wa} and \textit{wat} and \textit{me} and \textit{met} is that of subject versus object form, respectively.


\ea Adnominal\label{exe:demadd}
\ea\label{exe:demad} [stim6\_11:45]\\
\gll ma enem \textbf{wat}-a tu\\
\textsc{3sg} woman \textsc{prox.obj-foc} hit\\	
\glt	`He hits this woman.' 
\ex  \label{exe:demadme} [conv12\_8:15]\\
\gll an pulor-ca \textbf{met} parua-ta\\
\textsc{1sg} betel.vine-\textsc{2sg.poss} \textsc{dist.obj} pluck-{\glta}\\	
\glt	`I plucked that betel vine of yours.' 
\z \z 

\ea Pronominal
\label{exe:demproo}
\ea \label{exe:dempro} [conv3\_1:59]\\
	\gll ki \textbf{wat} napaki-kin ye ge\\
	\textsc{2pl} \textsc{prox.obj} use-\textsc{irr} or not\\	
	\glt	`Are you going to use this or not?'
\ex \label{exe:demprome} [conv12\_20:42]\\
	\gll ma \textbf{met}-a sanggara\\
	\textsc{3sg} \textsc{dist.obj-foc} search\\	
	\glt	`She searches for that.'
\z\z 

\ea  Identificational\label{exe:demidd}
\ea \label{exe:demid} [stim15\_2:03]\\
\gll \textbf{wa} me yartep\\
\textsc{prox} {\glme} sardine\\	
\glt	`This is a sardine.' 

\ex \label{exe:demidme} [conv11\_4:25]\\
\gll naharen-un-a \textbf{me}\\
leftover-\textsc{3poss-foc} \textsc{dist}\\	
\glt	`That is his leftover.' 
\z\z
			
At the present stage of my work on Kalamang, the demonstrative system is analysed as distance-based. At several occasions, speakers have expressed that \textit{wa} is reserved for referents close to the speaker, and \textit{me} for referents at some distance from the speaker. However, this is mainly valid in a contrastive situation, where two referents are at different distances from the speaker and need to be contrasted (‘\textit{this} is my phone, not \textit{that}'). As an illustration, consider \REF{exe:kuskap}. The example quotes the speech of a monkey whose fur is slowly turning white. The demonstratives \textit{wa} and \textit{me} refer to the same referent: the monkey's head. Nevertheless, the speaker first chooses proximal \textit{wa} and then, after using \textit{wangga} ‘from here', chooses distal \textit{me}. It seems that the speaker wants to create a contrast between the point `here' (the neck, the point up to where the monkey has become white), and the monkey's head `that', where his black fur is still visible. There is no literal distance here, but by using \textit{me} a contrast between the neck and the head is created.

\ea \label{exe:kuskap} [narr19\_16:04]\\
\gll \textbf{wa} tok kuskap \textbf{wangga} sara \textbf{me} tok kuskap\\
\textsc{prox} still black from.here go.up \textsc{dist} still black\\
\glt `This is still black, going up from here, that is still black.' 
\z 

\textit{Wa} can also be used adnominally with the first person pronoun, to contrast between speaker and addressee or another referent. Example \REF{exe:anwa} shows the first person pronoun followed by proximal \textit{wa} and topic marker \textit{me}, where the use of the first person creates a distance between himself, who is married to a mermaid in the sea, and the other villagers, who are married to human beings on land.

\ea \label{exe:anwa} [narr26\_6:59]\\
\gll an \textbf{wa} me laut-ko an se kion\\
\textsc{1sg} \textsc{prox} {\glme} sea.\textsc{mly}-\textsc{loc} \textsc{1sg} \textsc{iam} married.male\\
\glt `As for me, in the sea I am already married.' 
\z 

Demonstratives \textit{wa} and \textit{me} have object forms \textit{wat} and \textit{met}, which include the object marker \textit{-at}. Both adnominal demonstratives modifying an object NP and pronominal demonstratives in object position are inflected with the object case marker, as illustrated in \REF{exe:demad} and \REF{exe:demadme} for adnominal \textit{wa} and \textit{me}, and \REF{exe:dempro} and \REF{exe:demprome} for pronominal \textit{wa} and \textit{me}. Modified nouns, such as \textit{enem} `woman' in \REF{exe:demad}, remain uninflected.

The roots \textit{wa} and \textit{me} are used to derive all kinds of related forms. With the lative and locative case suffixes \textit{-ka} and \textit{-ko}, respectively, the meanings `to/from here/there' and `here/there' are derived. These forms are fossilised as \textit{wangga/ mengga} and \textit{watko/metko}, respectively. Other related forms are adverbs \textit{wandi} `like this' and \textit{mindi} `like that'.
%If paragraph above is changed, blank space after wangga/ needs to be removed again

As mentioned earlier, there is a third demonstrative root \textit{owa} which is used for referents that are (relatively) far away. It is much more limited in use than \textit{wa} and \textit{me}, and is not used contrastively with \textit{wa} and \textit{me}. \textit{Owa} is often inflected with locative \textit{-ko} or lative \textit{-ka}, surfacing as a predicative form \textit{owatko} `over there' or \textit{owangga} `to/from over there'. An example is given in \REF{exe:owatko}, where the speaker has been at a distant market to sell goods.

	\ea\label{exe:owatko} [narr25\_0:45]\\	
	\gll an \textbf{owatko} eba esnem an-at kon sarie\\
	\textsc{1sg} \textsc{far.dist.loc} then man \textsc{1sg-obj} one chase\\	
	\glt	`I was over there but a man chased me.' 
	\z 

This limited use of \textit{owa} is the reason to gloss the forms \textit{wa}, \textit{me} and \textit{owa} as proximal, distal and far distal, rather than proximal, medial and distal. There is a clear opposition between \textit{wa} and \textit{me}, with \textit{owa} only being used in exceptional cases.

Kalamang also has two elevationals. \textit{Yawe} means `down' and \textit{osa} `up'. These can be used predicatively and adnominally. An adnominal example with \textit{yawe} `down' is given in \REF{exe:yaweba}. A predicative example with the locative form \textit{osatko} is given in \REF{exe:osatko}.

\ea \label{exe:yaweba} [narr22\_5:59]\\
\gll som \textbf{yawe}=ba daruar\\
person down=\textsc{foc} pull.out\\
\glt `The person down there pulled it out.' 
\z 

\ea \label{exe:osatko} [conv16\_3:16]\\
\gll in tok \textbf{osatko}\\
\textsc{1pl.excl} still up.\textsc{loc}\\
\glt `We were still up there.' 
\z 


\section{Morphosyntactic behaviour of \textit{opa}}
\label{sec:morph}
Used at the edge of a clause, \textit{opa} is a time adverb meaning `just (now)' or `earlier'. This is illustrated in \REF{exe:opain}.

	\ea 	\label{exe:opain} [stim21\_0:01]\\	
		Yeni \textbf{opa} ka kareng-at kona\\
		Y. just \textsc{2sg} frog-\textsc{obj} see\\	
		\glt`Yeni, did you see the frog just now?' 
	\z 

In this chapter I focus on the demonstrative \textit{opa}, which is used adnominally. A typical example is \REF{exe:opatyp}, where the noun \textit{balun} `his dog' is followed by \textit{opa}. 

	\ea	\label{exe:opatyp} [stim20\_2:42]\\
	\gll bal-un \textbf{opa} kome sara∼sara\\
		dog-\textsc{3poss} {\glopa} look up∼\textsc{red}\\	
	\glt	`His dog [that I mentioned before] looked up.' 
	\z 
	
Unlike the demonstratives \textit{wa} \textsc{prox} and \textit{me} \textsc{dist}, demonstrative \textit{opa} cannot be used pronominally, and cannot be inflected for case. Another difference between \textit{opa} on the one hand and \textit{wa} and \textit{me} on the other is that the latter two have longer forms (identical to the roots in use and meaning) with a prefix \textit{yu-} and a suffix \textit{-ne}, while \textit{opa} cannot take these affixes. \textit{Opa} thus stands apart morphosyntactically from \textit{wa} and \textit{me}, and there are some difficulties determining which slot \textit{opa} occupies. Because there is no clear proof that \textit{opa} occupies its own slot, and because of its demonstrative function, I consider \textit{opa} to be part of the demonstrative paradigm. These arguments are illustrated in the following.

It is difficult to determine whether \textit{opa} occupies the same slot as \textit{wa} and \textit{me}, or whether it occupies the slot before it. Demonstratives \textit{wa} and \textit{me} can be followed by a topic marker \textit{me}, a homonym of the distal demonstrative. This is illustrated in \REF{exe:lengwame} and \REF{exe:yuolmeme}.

	\ea 	\label{exe:lengwame} [narr40\_13:11]\\
	\gll in kelek-ko-i koyet leng \textbf{wa} \textbf{me} sontum saerak\\
		\textsc{1pl.excl} mountain-\textsc{loc-vlnk} finish village \textsc{prox} {\glme} person \textsc{neg.ext}\\	
	\glt	`After we had gone to the mountains, there were no people in this village.' 
	\z 

	\ea 	\label{exe:yuolmeme} [narr22\_1:03]\\
	\gll yuol \textbf{me} \textbf{me} nauwanona-te\\
		day \textsc{dist} {\glme} pack-\textsc{imp}\\	
	\glt	`That day you pack!' 
	\z 

Like the proximal and distal demonstrative, \textit{opa} can also be combined with \textit{me}, as in \REF{exe:opame}.

	\ea \label{exe:opame} [narr6\_0:23]\\
	\gll ror-at kuru mia terus kewe \textbf{opa} \textbf{me} pi he usar\\
		wood-\textsc{obj} bring come then house {\glopa} \textsc{top/dist} \textsc{1pl} already build\\
	\glt	`[We] bring the wood and then we already build that house.' 
	\z 

The question is whether one should analyse \textit{me} in \REF{exe:opame} as the distal demonstrative or as the topic marker.\footnote{Translation to Indonesian does not help: \textit{opa me} is usually translated as \textit{(yang) tadi itu} `the/that one earlier'. \textit{Itu} is the distal demonstrative, but can also be used as topic marker in Indonesian.} One way to find out whether there is a separate slot for \textit{opa} before the demonstrative slot is to look at combinations of \textit{opa} with other demonstrative forms. The combination of \textit{opa} with the proximal form \textit{wa} is not found; there are no occurrences in the corpus of adnominal use of \textit{opa wa}. There is one example with the combination \textit{opa wa}, which I analyse as biclausal \REF{exe:opawa}: time adverb \textit{opa} `earlier' closes the first clause, and the second clause consists of a pronominal proximal demonstrative \textit{wa} with the tag \textit{to} `right'. (The intonation of this utterance is inconclusive.)

	\ea \label{exe:opawa} [stim6\_5:09]\\
	\gll {\ob}kon me ma koup-te \textbf{opa}{\cb} \textbf{wa} to\\
		one {\glme} \textsc{3sg} hug-{\glte} earlier \textsc{prox} right\\
	\glt	`This one he was hugging earlier, here, right?' 
	\z 

The combination of \textit{opa} with \textit{yuwa}, a longer variant of \textit{wa}, has a specific meaning `earlier today', and is used at the beginning or end of a clause as a time adverb. \textit{Opa wa} cannot mean `earlier today', suggesting that \textit{yuwa} in \textit{opa yuwa} has lost its demonstrative meaning. There are no examples in the corpus where adnominal \textit{opa} is combined with another demonstrative, except for locative forms. However, because locative constructions are predicative, locative demonstratives take the slot of the verb. Constructions combining \textit{opa} and a locative, such as in \REF{exe:opametko}, can thus be analysed as N + Dem + Pred.

	\ea \label{exe:opametko} [narr26\_15:28]\\
	\gll supkaling\textsubscript{N} \textbf{opa}\textsubscript{Dem} \textbf{metko}\textsubscript{Pred}\\
		eel {\glopa} \textsc{dist.loc}\\
	\glt 	`That eel is there.' 
	\z 

I conclude that although it is not unequivocally proven that adnominal \textit{opa} occupies the same slot as the demonstratives \textit{wa} and \textit{me}, I certainly do not have enough proof to place it in its own slot. With its demonstrative functions, which will be illustrated in the next paragraph, I will therefore analyse \textit{opa} as being in the slot for demonstratives. \tabref{tab:prondem} summarises the demonstrative forms discussed in this paragraph.

%adjust formatting of table below
\begin{table}
	\caption{Pronominal and adnominal demonstratives (selection)}
	\label{tab:prondem}
		\begin{tabular}{l l @{\hspace{1cm}} l l}
		\lsptoprule
		\multicolumn{2}{c}{proximal}&\multicolumn{2}{c}{adnominal}\\
		\midrule
			\textit{wa} & \textsc{prox} 		&	\textit{wa} & \textsc{prox} \\
			\textit{me} & \textsc{dist} &			\textit{me} & \textsc{dist}\\
			&& \textit{opa} & {\glopa}\\
		\lspbottomrule
		\end{tabular}
\end{table}


\section{Functional behaviour of demonstrative \textit{opa}}
\label{sec:funct}
Demonstrative \textit{opa} is a common and versatile form.  After proximal \textit{wa}, \textit{opa} is the second most common demonstrative in the corpus. To illustrate the uses of \textit{opa}, I present a fragment from a four-minute narrative in \REF{exe:fishnarr}. This story is a re-narration of a video that the speaker has watched.\footnote{The video is computer-animated and without spoken language, displaying people dressed in Japanese fashion. It shows a woman who goes for a walk holding a vase. The vase is accidentally shot by a man with a bow, who later comes to her house to apologise. The video is available at \url{https://www.youtube.com/watch?v=iPrn_raZxEs.}} In \REF{exe:fishnarr}, it is translated into English, with all instances of \textit{opa} (in this case all combined with topic marker \textit{me})\footnote{About a quarter of the occurrences of \textit{opa} in the corpus are without topic marker \textit{me}. It is not clear to me what the difference between \textit{opa} and \textit{opa me} is, but note that both \textit{opa} and \textit{opa me} occur with all kinds of referents: names, pronouns, possessed nouns, and nouns. I am aware that the analysis of topic marker may be not be the best fitting in all instances; see, for example, \REF{exe:lek}, where two referents are marked with \textit{opa me} in the same clause.} left untranslated in place. The four referents that are referred to with \textit{opa}, namely the woman, the vase, the man and  the fish, are bold to ease their tracing.

\ea\label{exe:fishnarr} {[stim24\_0:00-1:25]}
\ea\label{exe:fishnarr:a} \gll \textbf{pas} \textbf{opa} me\\
woman {\glopa} {\glme}\\
\glt `That woman,'

\ex\label{exe:fishnarr:b} \gll karena pi komeret me\\
because \textsc{1pl.excl} see {\glme}\\
\glt `because we saw,'

\ex\label{exe:fishnarr:c} \gll \textbf{pas} me me mungkin berupa jim atau\\
woman \textsc{dist} {\glme} maybe.\textsc{mly} a.kind.of.\textsc{mly} ghost or.\textsc{mly}\\
\glt `that woman is maybe a kind of ghost or'

\ex\label{exe:fishnarr:d} \gll e sontum purba\\
\textsc{hes} person ancient.\textsc{mly}\\
\glt `an ancient person.'

\ex\label{exe:fishnarr:e} \gll terus ma \textbf{gusi-un} kon-i tanggo\\
then \textsc{3sg} vase-\textsc{3poss} one-\textsc{num.acc} hold\\
\glt `Then, she is holding one of her vases.'

\ex\label{exe:fishnarr:f} \gll kewe-un-ka ma marmar-te\\
house-\textsc{3poss-lat} \textsc{3sg} walk-{\glte}\\
\glt `From her house she walks ...'

\ex\label{exe:fishnarr:g} \gll ma kewe-un-ka kaluar-te \textbf{gusi-un} kon-i tanggo\\
\textsc{3sg} house-\textsc{3poss-lat} exit-{\glte} vase-\textsc{3poss} one-\textsc{num.acc} hold\\
\glt `She comes out of her house holding one of her vases.'

\ex\label{exe:fishnarr:h} \gll \textbf{gusi-un} \textbf{opa} me iriskapten ma marmar terus\\
vase-\textsc{3poss} {\glopa} {\glme} white \textsc{3sg} walk further.\textsc{mly}\\
\glt `That vase of hers is white. She walks further.'

\ex\label{exe:fishnarr:i} \gll pulpul pararuoren bunga∼bunga\\
butterfly fly flower.\textsc{mly}∼\textsc{red}\\
\glt `A butterfly flies, flowers ...'

\ex\label{exe:fishnarr:j} \gll ma uriap-un-ka marmar marmar marmar\\
\textsc{3sg} street-\textsc{3poss-lat} walk walk walk\\
\glt`She walks and walks along her street'

\ex\label{exe:fishnarr:k} \gll sampai mindai me bo-te raor-ko\\
until.\textsc{mly} like.that {\glme} go-{\glte} middle-\textsc{loc}\\
\glt`until she comes to the middle.'

\ex\label{exe:fishnarr:l} \gll ma \textbf{canam-at} kon-i koluk\\
\textsc{3sg} man-\textsc{obj} one-\textsc{num.acc} meet\\
\glt`She meets a man.'

\ex\label{exe:fishnarr:m} \gll \textbf{canam} me pusir-un-bon\\
man \textsc{dist} bow-\textsc{3poss-com}\\
\glt`That man has a bow.'

\ex\label{exe:fishnarr:n} \gll \textbf{canam} jenis sontum me ba weinun jim atau sontum jaman purba\\
man kind.of.\textsc{mly} person {\glme} but also ghost or.\textsc{mly} person era ancient\\
\glt `The man is a person but also a kind of ghost or ancient person.'

\ex\label{exe:fishnarr:o}	\gll a mindi ma kome-ta \textbf{sor} per ne-ko tiri∼tiri par-un-bon-ten\\
\textsc{hes} like.that \textsc{3sg} look-\textsc{vlnk} fish water in-\textsc{loc} swim∼\textsc{red} fin-\textsc{3poss-com-rel}\\
\glt `Like that she watches the fish in the water that are swimming with their fins.'

\ex\label{exe:fishnarr:p} \gll \textbf{sor} hias\\
fish decorated.\textsc{mly}\\
\glt `Multi-coloured fish.'

\ex\label{exe:fishnarr:q} \gll komet me \textbf{sor} \textbf{opa} tiri ran mian-ten\\
look {\glme} fish {\glopa} swim go come-\textsc{rel}\\
\glt `Looking ... those fish are swimming back and forth.'

\ex\label{exe:fishnarr:r} \gll mungkin ma met komet∼komet \textbf{gusi-un} \textbf{opa} me tan-un-ka tur\\
maybe.\textsc{mly} \textsc{3sg} \textsc{dist.obj} look∼\textsc{red} vase-\textsc{3poss} {\glopa} {\glme} hand-\textsc{3poss-lat} fall\\
\glt `Maybe [while] she is looking at that that vase of hers falls from her hands.'

\ex\label{exe:fishnarr:s} \gll \textbf{gusi-un} \textbf{opa} me tan-un-ka tur [...] \\
vase-\textsc{3poss} {\glopa} {\glme} hand-\textsc{3poss-lat} fall {}\\
\glt `Her vase falls from her hands.'

\ex\label{exe:fishnarr:t} \gll ma he marmar yecieni me raor-ko ma \textbf{canam} \textbf{opa} me koluk\\
\textsc{3sg} \textsc{iam} walk return {\glme} middle-\textsc{loc} \textsc{3sg} man {\glopa} {\glme} meet\\
\glt `She walks back to the middle, she meets that man.'
\z \z

The first use of \textit{opa} in this fragment, with `woman' in \REF{exe:fishnarr:a}, indicates to the addressee (me, the linguist) that she ``knows'' this woman, because she has also seen the video where the woman appears.\footnote{The same speaker starts his narration of the \textit{Frog Story} \citep{Mayer1969} the same way (`child \textit{opa} is sitting in his house'), but there he is narrating \textit{while} he sees the pictures of the story.} That is, `woman' is marked as being shared knowledge. This is an example of recognitional use of \textit{opa}.

The uses of \textit{opa} with \textit{gusi} `vase' in \REF{exe:fishnarr:h} and \REF{exe:fishnarr:s}, with \textit{sor} `fish' in \REF{exe:fishnarr:q} with \textit{canam} `man' in \REF{exe:fishnarr:t} are tracking uses, helping the addressee ``in keeping track of what is happening to whom'' \citep[226]{Himmelmann1996}. With `vase' in \REF{exe:fishnarr:h} and `fish' in \REF{exe:fishnarr:q}, \textit{opa} is used to immediately maintain the referent. `Man' in \REF{exe:fishnarr:t}, on the other hand, is referred to with \textit{opa} only after having been absent from the narrative for a while, i.e. to reintroduce a referent. This also counts for a later occurrence of `vase' in \REF{exe:fishnarr:s}.

A tracking example from another text is given in \REF{exe:semen}. The referent, \textit{semen} `concrete', is mentioned at minute 2:16, and its repeated use is marked with \textit{opa} at minute 5:27.


\ea \label{exe:semen}	
\ea {[narr7\_2:16]}\\
\gll mu he semen-at cetak\\
\textsc{3pl} already concrete-\textsc{obj} mould\\	
\glt	`They already mould the concrete.'
\ex {[narr7\_5:27]}\\
\gll mu he semen \textbf{opa} koyal-te di=ra-n\\
\textsc{3pl} already concrete {\glopa} mix-{\glte} \textsc{ter}-go.up-\textsc{n}\\	
\glt	`They already mixed the concrete and put it up.' 
\z 
\z 

In \REF{exe:desi}, the referent Desi is in the outside world, in fact just outside the window of where the speaker and addressee are sitting.

	\ea	\label{exe:desi} [conv11\_6:36]\\
	\gll Desi \textbf{opa} me yal∼yal-te yawe\\
		D. {\glopa} {\glme} paddle∼\textsc{red}-{\glte} down\\
	\glt 	`Desi is paddling down there.' 
	\z 

In the entire village of the speaker there is only one person named Desi, who, moreover, is the speaker's daughter. As such she does not use \textit{opa} to disambiguate which Desi she's talking about. Even though Desi has come up several times in the same conversation from which \REF{exe:desi} is taken, the speaker does not seem to employ a tracking use to indicate ``that Desi we've been talking about'', since there is only one Desi. Rather, she is foregrounding a referent to which both speaker and hearer have a relation. Here, the boundary between endophoric and exophoric reference is a bit vague. On the one hand one would like to say that this is recognitional use of \textit{opa}, referring to shared knowledge. On the other hand, the use of \textit{opa} while the referent is just outside the window is reminiscent of exophoric use of demonstratives, indicating entities within the speech situation. Perhaps the definition of recognitional demonstratives, given as being a kind of endophoric demonstrative, should be widened to include exophoric reference as well, at least in Kalamang. See also \REF{exe:pulor}, where the referent \textit{pulor} `betel leaf' is marked with \textit{opa} in a tracking kind of use (having been referred to by the other speaker a few turns earlier), but at the same time it is an exophoric demonstrative as the betel leaves are somewhere in the room where speaker and addressee are located. The speaker even points to the vague direction where she thinks the betel leaves are located.

\ea \label{exe:pulor} [conv12\_8:51]\\
\gll pulor \textbf{opa} ka tu-ta me tama-ba-kadok os-kadok\\
betel.leaf {\glopa} \textsc{2sg} pound-\textsc{ta} {\glme} where-\textsc{foc}-side beach-side\\
\glt 	`That betel leaf that you were pounding, on which side is it, on the beach-side?' 
\z 

The reason the speaker does not opt for a basic spatial demonstrative (proximal \textit{wa} or distal \textit{me}) is that she wants to indicate that she is referring to the betel leaves that have been mentioned before. The speaker also refers to the common ground, by specifying that the addressee has been pounding the betel leaves she is looking for.

Example \REF{exe:hadi} illustrates nicely how common ground can be established, or perhaps rather reinforced, in discourse by using \textit{opa}.

	\ea \label{exe:hadi} [narr14\_3:14]\\
	\gll inier \textbf{opa} [...] Hadi \textbf{opa} to\\
		\textsc{2du} {\glopa} [...] H. {\glopa} right \\
	\glt 	`We two, with Hadi, right.' 
	\z 


At first, the speaker marks \textit{inier} `we two' with \textit{opa}, assuming the addressee knows who she means. Then she starts to doubt this (in fact, there are a few repairs between the first \textit{opa} and \textit{Hadi}): does the addressee know both referents? One is the speaker herself, and the other one is her brother Hadi. She specifies this by saying \textit{Hadi opa} `that Hadi whom you know', indicating (again) that there is common ground, that the addressee knows this Hadi. Hadi has not been mentioned in the story from which \REF{exe:hadi} is taken. Again, this example can be analysed as recognitional use of \textit{opa}.

As a last characteristic of \textit{opa}, note that it can be used with two referents in the same clause. Example \REF{exe:lek} is taken from a narration of the Pear Story \citep{Chafe1975}. Both `goat', which was recently introduced in the narrative, and `avocados', which has been an important referent in the story from the beginning, are referred to with \textit{opa}.

\ea \label{exe:lek} [stim31\_0:51]\\
\gll lek \textbf{opa} me afokat \textbf{opa} me kona\\
goat {\glopa} {\glme} avocado {\glopa} {\glme} see\\
\glt 	`That goat sees those avocados.' 
\z 

This double use of \textit{opa} shows that an addressee can choose to foreground two referents at the same time that the speaker can trace back to earlier mentioned referents. Again, this is an example of tracking use.

Summarising, the demonstrative \textit{opa} has both tracking and recognitional uses \citep{Himmelmann1996}. It can be used for both maintaining and reintroducing referents. Typically, \textit{opa} is used as an endophoric demonstrative, but sometimes it can also be used exophorically. The overarching characteristic of these uses of \textit{opa} is cognitive accessibility in the sense of \cite{Hanks2005}, covering both shared knowledge and previous discourse. In all examples in this section, by using \textit{opa} the speaker signals that the referent is accessible within the discourse because it has been mentioned before (tracking use) or accessible from the addressee's knowledge because the addressee has engaged with this referent in real life outside the discourse (recognitional use).

This contrasts with the use of proximal \textit{wa} and distal \textit{me}, where the main factor in the choice of demonstrative is decided by space. There are no examples of these demonstratives in the corpus where they are used to highlight that the referent is (or is not) common ground, or that the referent has (or has not) been mentioned before.


\section{Co-lexification of demonstrative and adverb}
\label{sec:typ}
Kalamang has two lexemes \textit{opa}, which occur in different syntactic positions. At clause edges, it is a time adverb that translates as `just now' or `earlier'. Adnominally, it is a demonstrative. The analysis of demonstrative \textit{opa} as presented here, namely as a marker of cognitive accessibility, helps to understand how these two lexemes are related and how co-lexification might have arisen. What has happened recently is accessible within the minds of those who have witnessed it. It is easy to imagine a grammaticalisation path where the time adverb has come to be applied in a more abstract way as an adnominal demonstrative. In its tracking sense, pointing back to earlier mentioned referents, the use of \textit{opa} is still quite literal. In its recognitional use, there seems to have taken place a generalisation from a use of \textit{opa} with referents that are accessible because they have been mentioned earlier to a use of \textit{opa} with referents that are accessible because they are part of the common ground. However, there is no diachronic proof to support this grammaticalisation path from time adverb to demonstrative.

Kalamang is not the only language displaying the co-lexification of these or very similar functions. Indonesian (Austronesian) uses \textit{tadi} as an adjunct of time meaning `just recently' or `a while ago', but also adnominally ``to refer back to someone or something recently mentioned or recently present'' \citep[135]{Sneddon2012}. The Indonesian-English dictionary by \citet{EcholsEtAl1989} gives both `a while ago' and `aforementioned' as translations of \textit{tadi}. These meanings are illustrated in \REF{exe:tadi} and \REF{exe:tadi2}. Both glossing and translations are mine in these examples.

\ea \label{exe:tadi}
\langinfo{Indonesian}{Austronesian}{\citealt[540]{EcholsEtAl1989}}\\
\gll tadi ia di.sini sekarang tidak\\
a.while.ago \textsc{3sg} here now \textsc{neg}\\
\glt `He was here a while ago; now he is not.'
\z

\ea \label{exe:tadi2}
\langinfo{Indonesian}{Austronesian}{Indonesian translation of Paolo Coelho's \textit{The Alchemist}; source: web}{\footnotemark}\\
\gll anak tadi mulai naik turun tangga∼tangga istana\\
	child aforementioned start go.up go.down stairs∼\textsc{red} palace\\
	\footnotetext{\url{https://ndiethesmartass.wordpress.com/2010/01/22/rahasia-kebahagiaan/}}
	\glt `The aforementioned child started to go up and down the stairs of the palace.'
\z 		


Kalamang belongs to the three-language family West Bomberai. One other language in the family, Mbaham, has a demonstrative that seems to have a very similar function to demonstrative \textit{opa}. In a very brief Indonesian grammar sketch, three demonstratives are listed \citep{Flassy1984}: \textit{nggismo}/\textit{kismo} seems to be proximal (translated as \textit{ini} `this'), \textit{unggya} seems to be a distal (translated as \textit{itu} `that'), and a third form \textit{nteit} is translated as \textit{tadi (itu)} `earlier'. Unfortunately, no glossed examples are given. It is unclear whether \textit{nteit} is also an adverb of time. This could be an area of further study in that language.

%Repair UsherSchapper reference, change nodate > in preparation
Kalamang is possibly related to other languages of western New Guinea that have been hypothesised to belong to the Trans New Guinea macro-family. This includes Bunaq, a Timor-Alor-Pantar language of Timor \citep{UsherSchapperND}. Bunaq has a word \textit{mete} (\textsc{now}) that is functionally similar to both uses of Kalamang \textit{opa}. \textit{Mete} is analysed as a temporal/discourse locational \citep[285]{Schapper2009}.\footnote{Locationals form a distinct word class from determiners (which include demonstratives) in Bunaq, because there are some syntactic and semantic differences between the two. Both classes, though, are used for locating and identifying referents. The distribution of Bunaq locational \textit{mete} and Kalamang demonstrative \textit{opa} does not seem to be very different.} When used independently, it refers to the present. Within the NP it is most commonly used to redirect attention to an established referent. It can also be used as a topic shifter, or, when used with a new referent, as a marker of a new stage in discourse. \cite{Schapper2009} gives the following example from a narrative to illustrate the marking of an established referent. The referent, the monkey, has been absent for two utterances, and is reintroduced in \REF{exe:mete} with \textit{mete}. The NP is bracketed.

\ea \label{exe:mete}
\langinfo{Bunaq}{Timor-Alor-Pantar}{\citealt[298]{Schapper2009}}\\
\gll hilaq [\textbf{mete} orel himo] g-ini g-ie\\
surprise \textsc{now} monkey \textsc{contr.an} \textsc{3an-caus} \textsc{3-poss}\\
\glt `What a surprise, it was this monkey just now who had made (the child) his.'
\z 		

Bunaq adnominal \textit{mete} functions in a very similar way to Kalamang adnominal \textit{opa}, while also being used as an indicator of time.\footnote{Note that Bunaq also has the words \textit{meten} `before, past' and \textit{metensi} `just now'. It is unclear, however, whether these are related to \textit{mete} (\textsc{now}).}

The data from Kalamang, Indonesian and Bunaq suggests that the co-lexification of a referent tracker (a word of unknown word class in Indonesian, a locational in Bunaq, a demonstrative in Kalamang) and a time indicator is a phenomenon that might be worth investigating further, especially diachronically, to understand whether a grammaticalisation path can be discerned.

\section{Conclusion}
\label{sec:concl}
This chapter presented a careful analysis of the functions of Kalamang demonstrative \textit{opa}. In contrast to proximal \textit{wa} and distal \textit{me}, which are spatial exophoric demonstratives that can be used pronominally and adnominally, \textit{opa} is mainly an endophoric demonstrative that is only used adnominally. It is used for tracking referents that have been mentioned before, as well as for indicating that the referent is part of the common ground between speaker and addressee (recognitional use). Demonstrative \textit{opa} thus signals to the addressee that the referent is cognitively accessible. Interestingly, demonstrative \textit{opa} sometimes occurs outside strict endophoric use, referring to entities in the speech situation. In its tracking use, it can be used to maintain referents or to reintroduce them.

Demonstrative \textit{opa} is homonymous with a time adverbial \textit{opa} `just now; earlier'. There is no proof for a grammaticalisation path from one to the other, but it should be noted that Indonesian and Bunaq (Timor-Alor-Pantar) have co-lexified words with similar functions to Kalamang \textit{opa}, suggesting a possible area for further study.

\section*{Abbreviations}
The chapter follows the Leipzig Glossing Rules. Additional abbreviations include:

\begin{tabularx}{.45\textwidth}{lQ}
	\textsc{an} & animate\\
	\textsc{contr} & contrastive\\
	\textsc{ext} & existential\\
	\textsc{far.dist} & far distal  \\
	\textsc{iam} & iamitive \\
	\textsc{hes} & hesitation\\	
	\textsc{lat} & lative\\
	\textsc{mly} & Malay loan\\
	\textsc{n} & optional verb class marker of class `\textsc{n}'\\
\end{tabularx}
\begin{tabularx}{.45\textwidth}{lQ}
	\textsc{nfin} & non-final verb\\
	\textsc{now} & temporal/discourse locational\\
	\textsc{num} & number\\
	\textsc{red} & reduplication\\	
	\textsc{ta} & unanalysed suffix \textit{-ta} \\	
	\textsc{te} & unanalysed suffix \textit{-te} \\
	\textsc{ter} & terminative\\
	\textsc{vlnk} & verb linker \\
\end{tabularx}

\section*{Acknowledgements}
Thanks to Niclas Burenhult, \r{A}shild N{\ae}ss and anonymous reviewers for their useful comments.

\sloppy\printbibliography[heading=subbibliography,notkeyword=this]
%\printbibliography[heading=subbibliography,notkeyword=this]

\end{document}
