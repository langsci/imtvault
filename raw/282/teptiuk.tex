\documentclass[output=paper,colorlinks,citecolor=brown]{langscibook} 
\author{Denys Teptiuk\affiliation{University of Tartu}\orcid{}}
\title{Manner deictics in quotative indexes of Finno-Ugric}
\abstract{This chapter discusses the quotative use of manner deictics in computer-mediated communications of five languages representing three Finno-Ugric branches: Finnic, Permic and Hungarian. The aim of the study is to (i) define the functional properties of manner deictics in quotative indexes (QIs) of the languages in focus, (ii) demonstrate in what types of QIs they appear, and (iii) determine possible functional similarities in the distribution of the markers between the languages. It is shown that manner deictics can be employed as cataphoric (Finnish \textit{näin}, \textit{niin}, \textit{sillee}(\textit{n}); Estonian \textit{nii}; Hungarian \textit{úgy}; Udmurt \textit{taźy}) and anaphoric (Udmurt \textit{oźy}) markers, or can function as both (Komi \textit{taďź}(\textit{i}), \textit{siďź}; Hungarian \textit{így}). Furthermore, some manner deictics (Finnish \textit{sillee}(\textit{n}), Hungarian \textit{így}) introduce mimetic expressions that can be interpreted as quasi-quotations. In the conclusion, cross-linguistic parallels in the use of manner deictics in quotative constructions are pointed out.}
\IfFileExists{../localcommands.tex}{
 \input{../localpackages}
 %Copy this to localcommands.tex

\usepackage[english]{babel}
\usepackage{amsmath}
\usepackage{amssymb,amsfonts,textcomp}
\usepackage{array}
\usepackage{hhline}
\usepackage{hyperref}

\newenvironment{styleStandard}{}{}
\newenvironment{stylelsAbstract}{}{}
\newenvironment{stylelsSectioni}{}{}
\newenvironment{stylelsSectionii}{}{}
\newenvironment{stylelsBulletList}{}{}
\newenvironment{styleBibliographyi}{}{}
\newenvironment{listWWNumxxvleveli}{}{}
\newenvironment{listWWNumxxvlevelii}{}{}
\newenvironment{listWWNumxxvleveliii}{}{}
\newenvironment{listWWNumxxvleveliv}{}{}
\newenvironment{listWWNumixleveli}{}{}
\newenvironment{listWWNumixlevelii}{}{}
\newenvironment{listWWNumixleveliii}{}{}
\newenvironment{listWWNumixleveliv}{}{}

\newcommand\textstyleListLabelxvi[1]{#1}
\newcommand\labellistWWNumxxvleveli{\thelistWWNumxxvleveli.}
\newcommand\labellistWWNumxxvlevelii{\thelistWWNumxxvlevelii.}
\newcommand\labellistWWNumxxvleveliii{\thelistWWNumxxvleveliii.}
\newcommand\labellistWWNumxxvleveliv{\thelistWWNumxxvleveliv.}
\newcommand\labellistWWNumixleveli{[F0B7?]}
\newcommand\labellistWWNumixlevelii{\textstyleListLabelxvi{o}}
\newcommand\labellistWWNumixleveliii{[F0A7?]}
\newcommand\labellistWWNumixleveliv{[F0B7?]}

\newcounter{listWWNumxxvleveli}
\newcounter{listWWNumxxvlevelii}[listWWNumxxvleveli]
\newcounter{listWWNumxxvleveliii}[listWWNumxxvlevelii]
\newcounter{listWWNumxxvleveliv}[listWWNumxxvleveliii]
\newcounter{itemize}  
 \input{../localhyphenation} 
 \togglepaper[1]%%chapternumber
}{}

\begin{document}
\maketitle 

%still to be done:
%Adjust table formatting
%Orphan control
%Preposition abbreviations table

\section{Introduction}\label{sec:teptiuk:1}

Recent cross-linguistic studies have shown that comparative/similative \textit{like}, demonstrative deictic \textit{so} and quantifying elements \textit{just} and \textit{all}, as well as motion (\textit{go}) and action (\textit{do}) verbs, can grammaticalise into quotative markers (see \citealt[xii-xiv]{BuchstallerVanAlphen2012}; \citealt[§ 5.1.2-5.1.5]{Güldemann2008}). Quotative markers with demonstrative semantics are found in many typologically diverse languages. Manner deictics are either used in quotative constructions together with reportative verbs, i.e. speech or epistemic verbs, or they are used alone to point out the presence of a quote \citep[321, 350]{Güldemann2008}. Their use is explained by the function of quotations as a type of demonstration embedded in language use, i.e. “a mimetic reenactment of a non-immediate state of affairs” \citep[320]{Güldemann2008}. By producing an utterance from a different temporal or spatial situation, the reporter demonstrates the situation to an audience (\citealt[802]{ClarkGerrig1990}; see \citealt{Clark2016} on speech acts as demonstrations). The endophoric use of demonstratives in quotative constructions is usually associated with a cataphoric reference, which “relate[s] to stretches of following discourse” (\citealt[297]{KönigUmbach2018}; see also \citealt{Güldemann2008}). Although cataphoric reference seems to be the most common attested extended use of manner deictics in the world’s languages, this statement might be too general. Consider, for example, \REF{ex:teptiuk:1} from Hungarian, in which the manner deictic \textit{így} ‘so’ is used anaphorically, pointing at the just-produced quote.

\ea\label{ex:teptiuk:1}
Hungarian (MNSz)\footnote{See \sectref{sec:teptiuk:3} for details on the type of data used in the study. The abbreviated sources in parentheses in the first line of every example are references to the list of electronic resources given in the appendix.}\\
\gll \|“Nagy pénz, kis foci.”\|\textsubscript{RD} – vagy \textbf{mégsem} \textbf{így} \textbf{mondta} \textbf{volna} Puskás?\\
{\db}{\db}big money small football {} or still.also.\textsc{neg} so say.\textsc{pst}.3\textsc{sg}.\textsc{def} be\textsc{.cond.3sg} \textsc{pn}\\
\glt ‘\|“Big money, small football.”\|\textsubscript{RD} – as Puskás would have said (lit. or \textbf{wouldn’t} \textbf{have} Puskás \textbf{said} \textbf{so}?)’\footnote{In the examples, quotative indexes are marked in bold, and reported discourse is enclosed in double vertical bars.}
\z

Hence, there is a reason to look more closely at the direction of endophoric reference of manner deictics in quotative constructions. For this purpose, I conducted a contrastive study on the quotative use of manner deictics in computer-mediated communications of five distantly related Finno-Ugric languages. The choice of languages is not accidental. They belong to three different branches of the language family, i.e. Finnic (represented by Finnish and Estonian), Permic (Komi and Udmurt), and Hungarian, and to three different geographical areas: Northern Europe (Finnish and Estonian), Central Europe (Hungarian) and Russia (Komi and Udmurt). Despite their relatedness and typological closeness, the languages did not have contact with each other for centuries, with the exception of Finnish and Estonian, and they possess individual features that developed independently or through contact with languages in their respective areas. Hence, I suspect these languages represent different typological patterns in the use of manner deictics in quotative constructions which can be determined and further applied cross-linguistically. By taking a closer look at the distribution of manner deictics in the five Finno-Ugric languages, I aim to determine their functional properties in quotative constructions, e.g. do manner deictics introduce only particular types of reported speech and thought or can they function as general mimetic markers? Do structural features have an impact on their functions? Is there any correspondence between their use inside and outside of quotative constructions? Although descriptive grammars (e.g. \citealt{EreltMetslang2017} on Estonian; \citealt{HakulinenEtAl2004} on Finnish) and previous studies provide basic descriptions of manner deictics in individual languages and even touch upon their quotative use (e.g. \citealt{Keevallik2005} on Estonian; \citealt{Kiefer2016} on Hungarian), these and other questions in relation to their quotative use still remain unexplored.

The chapter is organised as follows. In \sectref{sec:teptiuk:2}, I provide the terminological framework for this study. In \sectref{sec:teptiuk:3}, my methodology and database are described. \sectref{sec:teptiuk:4} presents the results of the study of Finnish and Estonian (\sectref{sec:teptiuk:4.1}), Komi and Udmurt (\sectref{sec:teptiuk:4.2}), and Hungarian (\sectref{sec:teptiuk:4.3}). Finally, \sectref{sec:teptiuk:5} summarises the main findings and highlights cross-linguistic similarities in the quotative use of manner deictics based on the determined typological patterns.

\section{Terminological framework}\label{sec:teptiuk:2}

In my investigation on manner deictics in five Finno-Ugric languages, I adapt Güldemann’s framework of reported discourse (henceforth RD), which is defined as follows:

\begin{quote}
Reported discourse is the representation of a spoken or mental text from which the reporter distances him-/herself by indicating that it is produced by a source of consciousness in a pragmatic and deictic setting that is different from that of the immediate discourse \citep[6]{Güldemann2008}.
\end{quote}

Güldemann prefers “discourse” as “the representation of spoken or mental text” over the more traditional “speech”, since RD “is not restricted to real instances of speech” and may also include “texts that were never actually uttered like so-called ‘internal speech’, or in general any representation of cognitive acts or states” \citep[7]{Güldemann2008}.

According to \citet[10]{Güldemann2008}, RD together with the elements introducing it form a complex whole labelled as an “RD-construction”. RD-constructions canonically consist of two major constituents: RD and “quotative index” (henceforth QI). In \REF{ex:teptiuk:2}, the clause \textit{I said to him} is a QI followed by the RD “\textit{Your party won’t be pleased …}”.

\ea\label{ex:teptiuk:2}
{English (Daily Mail)}\\
 \textbf{I} \textbf{said} \textbf{to} \textbf{him} \|“Your party won’t be pleased ...”\|\textsubscript{RD}\\
 \z

\citet[11]{Güldemann2008} defines a QI as “a segmentally discrete linguistic expression which is used by the reporter for the orientation of the audience to signal in his/her discourse the occurrence of an adjacent representation of reported discourse”. As a “linguistic expression”, QIs can represent structures of different complexities, consisting of “just a gram (…), an independent function word, a phrase, a full sentential syntagm (…), or even a clause with more than one predicate” \citep[11]{Güldemann2008}.\footnote{In some situations, QIs can remain verbally unexpressed. Instead, suprasegmental features of intonation, dynamics and pitch can be used as the sole means to contrast a quote with its surrounding contexts. Since I use non-standard written texts as a corpus for this study (see \sectref{sec:teptiuk:3}) and focus on the use of manner deictics in quotative constructions, I exclude verbally unexpressed QIs from the current investigation.} Example \REF{ex:teptiuk:2} contains only one of various possible QI forms, namely a speech verb (\textit{said}) describing the event behind the RD (the quotation of speech), and NPs encoding participants, i.e. the original speaker (\textit{I}) and an addressee (\textit{to him}). However, although these constituents are probable and relatively frequent they are not indispensable elements of the QI. Even speech verbs are not universal components of QIs per se. Consider \REF{ex:teptiuk:3}, for instance, where instead of a speech or epistemic verb (henceforth labelled reportative verb), the reporter uses the motion verb \textit{go} and the combination of the equational verb \textit{be} with the similative marker \textit{like} for the presentation of RD.

\ea\label{ex:teptiuk:3}
{English (Twitter)}\\
… and \textbf{he} \textbf{goes} \|I am the police bitch\|\textsubscript{RD} and starts touching the register \textbf{I’m} \textbf{like} \|oh this mf’er didn’t just do that.\|\textsubscript{RD}\\
\z
Manner deictics as non-reportative elements are expected to co-occur primarily with reportative elements in QIs, as in \REF{ex:teptiuk:1}. However, their co-occurrence with other (grammaticalised) elements is also investigated here. Furthermore, it is also of interest to see whether they undergo changes in the quotative domain and grammaticalise into genuine quotative markers that are not bound to reportative elements. Therefore, I pay attention to the use of manner deictics in different constructions and point out additional meanings and functions that can be observed in their use in less or more complex QIs.

\section{Methodology and data}\label{sec:teptiuk:3}

In this study, data originating from social network sites (SNS) is used as a database. My choice of SNS data is motivated by the following factors. First, the focus is on QIs that are mainly used in non-standard varieties of the studied languages, which typically exceed the limited amount of QIs used in standardised texts. Second, previous studies show that “[t]he informal characteristics of SNS enables the usage of generally oral forms such as slang and dialects in a written context” \citep[144]{Pischlöger2014}. As for the minority languages Komi and Udmurt, “the relaxed atmosphere on SNS allows language use which is typical for oral communication and otherwise frowned upon in other (especially written) contexts by language purists” \citep[144]{Pischlöger2014}. Thus, Komi and Udmurt speakers use language online that not only includes variants that are commonly mixed with the dominant Russian language but also a mixture of dialects and styles that are peculiar to colloquial speech (see \citealt{Pischlöger2016}; \citealt{Edygarova2013}, \citeyear{Edygarova2014}). Thus, despite the presence of emoticons, different orthographic symbols and nonstandard shortenings, the language on SNS can be considered a written approximation of spoken language, combining the features of colloquial speech and standard writing in one text \citep{HelasvuoEtAl2014}.

For data collection, I studied the occurrence of quotations in different new media sources. Since Komi and Udmurt are endangered languages with a smaller amount of online material compared to Finnish, Estonian and Hungarian, I also investigated available text collections for the Permic languages, i.e. \citet{Uotila1985,Uotila1989} for Komi and \citet{Kelmakov1981,Kelmakov1990} for Udmurt. The collections provide transcribed oral narratives by Komi and Udmurt speakers from various dialectal groups. This material supplemented the data from new media sources if the latter did not provide sufficient evidence. For Udmurt, the material predominantly originates from the group \textit{Jumshan57}\footnote{\url{https://vk.com/knyazpozdey} (accessed last August 1, 2019).} on the Russian SNS \textit{vk.com}. I studied approximately 100 of the 249 pages (ca. 200 blog posts) of the material available. In addition, I used the blog page \textit{KYLZY Jopte}\footnote{\url{https://vk.com/udmurt_ept} (accessed last August 1, 2019).} (containing 35 pages and 696 blog posts at the time of investigation) as well as other groups and pages. The choice of groups was motivated by four factors: (i) the number of entries, (ii) the number of group members, (iii) the use of Udmurt (exclusive or parallel to Russian), and (iv) the dominance of unedited texts. Thus, I preferred unofficial pages consisting of live conversations in comment sections and unedited blog entries to official pages of media resources, non- and governmental organisations, etc. As a result, I could also trace the systematic appearance of vernacular variants mixed with Russian in live discussions in comment sections, which would have been impossible with standardised texts. Where additional material was required, data from the Blog subcorpus of the Udmurt corpus (henceforth Blog subcorpus)\footnote{\url{http://web-corpora.net/UdmurtCorpus} (August 1, 2019). The Blog subcorpus consists of ca. 6\% of the whole corpus (7.3 million tokens).} was used to further investigate the use of manner deictics in QIs.\footnote{At the time of data collection, the Volga-Kama Udmurt corpora containing the new media subcorpus (\url{http://volgakama.web-corpora.net}, accessed August 1, 2019) were not available.} The Blog subcorpus contains approximately 160 examples of the proximal manner deictic \textit{taźy} and approximately 1400 examples of the distal manner deictic \textit{oźy} (see \sectref{sec:teptiuk:4.2} on the use of Udmurt manner deictics). Where the number of examples was fairly low (as in the case of \textit{taźy}), I checked every instance separately. In the case of a higher number of examples, I checked the collocation of this marker with the reportative verbs \textit{šuyny} ‘say’ and \textit{malpany} ‘think’ in different finite forms, e.g. ‘(I) say/said thus’, ‘(she/he) thinks/thought so’, etc. With the combination of the new media sources and the online corpora I collected approximately 40 examples of manner deictics in quotative constructions and 20 examples of their use outside the quotative domain.

Taking the amount and quality of Udmurt data as a reference point, I studied a similar number and type of pages on \textit{vk.com} and \textit{blogspot.com} for Komi. However, at the end of the investigation, my corpus contained only three examples of the manner deictic \textit{taďź(i)} in quotative constructions. Therefore, I used the corpus of the Komi language (henceforth Komi corpus)\footnote{The corpus of the Komi language (Russian: \textit{Korpus komi jazyka}) is  available at: \href{http://komicorpora.ru/}{http://komicorpora.ru} (July 1, 2019). An exact number of tokens in the corpus is not specified.} as a supplementary source to make generalisations about the use of manner deictics in QIs. The corpus contains oral and written texts of various genres (fiction, journalistic texts, educational and scientific literature, official correspondence, etc.), excluding new media texts.

For Finnish, I used the Corpus of Internet Communications,\footnote{Finnish: \textit{Internet-keskusteluaineistoja}.} consisting of data from the forums \textit{Suomi24} and \textit{Ylilauta},\footnote{\url{https://korp.csc.fi} (July 1, 2019). The corpus consists of ca. 6.9 billion tokens.} as my main material. I browsed the corpus for collocations of the manner deictics \textit{näin}, \textit{niin}, \textit{noin}, \textit{tälläin} ({\textasciitilde} \textit{tälläi}), \textit{t}(\textit{u})\textit{olloin} ({\textasciitilde} \textit{tollai}) and \textit{sillee}(\textit{n}) ({\textasciitilde} \textit{sillai}) with the reportative verbs \textit{sanoa} ‘say’ and \textit{ajatella} ‘think’, and the equational verb \textit{olla} ‘be’. Based on my previous investigations \citep{Teptiuk2019}, these verbs were expected to be the most probable components of Finnish QIs. I checked the first 100 examples of every query for the quotative use of these collocations when the number of queries was too high.

In addition, I used the Google search engine, which allowed me to test these collocations on web pages outside the digital corpus when the number of collocations was low. I checked the first ten pages of the Google search results, with each page containing ten links. Since the search results are less accurate the further one proceeds, I considered this amount to be enough to investigate the use of manner deictics in QIs (ca. 100 search results for one tested variant of a construction). In order to make the search more effective, I placed the studied material into quotation marks and checked the different collocations of reportative verbs and manner deictics in their different grammatical forms. Pages that did not fall under the category of computer-mediated communications (CMC), e.g. edited newspaper articles or science fiction texts (see \citealt{Crystal2001}), were not taken into account as primary sources of examples. I used data outside CMC only when primarily suitable sources did not yield any relevant results. In all, I collected approximately 50 examples of Finnish manner deictics in quotative constructions.

For Estonian and Hungarian, I used both suitable corpus data (Estonian: New media subcorpus of the Estonian Reference Corpus,\footnote{\url{http://www.cl.ut.ee/korpused/segakorpus/} (July 1, 2019); the subcorpus contains ca. 21 million words.} henceforth  NMS; Hungarian: Personal subcorpus of the Hungarian National Corpus,\footnote{\url{http://corpus.nytud.hu/mnsz} (July 1, 2019); the subcorpus contains 18.6 million words of discussion on internet forums, mainly deriving from the Hungarian internet portal, \textit{index.hu}, and several forums from \textit{Subcarpathia} (RO, UA).} henceforth MNSz) and the independently collected material obtained through Google searches. I checked for the collocations of manner deictics with reportative verbs within the corpora and Google searches. For Estonian, collocations with the equational verb \textit{olema} ‘be’ were also considered (see \citealt{Teptiuk2019} for more details on Estonian QIs). Altogether, I collected approximately 20 examples for Estonian and approximately 30 examples for Hungarian.

All examples are provided here with translations and glosses. Minor spelling mistakes are corrected, but punctuation errors are not. Since Komi and Udmurt use different symbols to mark identical sounds, I present these in the transcriptions to avoid confusion while transliterating the Permic examples. Russian code-switches are presented in transliteration and enclosed in curly brackets \{ \} in the glosses.

\section{Manner deictics as quotatives in Finno-Ugric languages}\label{sec:teptiuk:4}
\subsection{Manner deictics in quotative indexes in Finnish and Estonian}\label{sec:teptiuk:4.1}

Finnish has a tripartite system of manner deictics: (i) speaker-proximal \textit{näin}, (ii) hearer-proximal \textit{noin}, and (iii) distal \textit{niin} (\citealt[147]{König2017}; also see \citealt[§668]{HakulinenEtAl2004}). The manner deictics historically derive from the instructive case forms of the plural demonstratives \textit{nämä,} \textit{nuo} and \textit{ne}. Besides their basic functions as manner deictics, they are used as intensifiers \citep[§792]{HakulinenEtAl2004}, or, in the case of \textit{niin}, as a correlative pronoun \citep[§1160, §1172]{HakulinenEtAl2004}.

In addition, one can find another set of manner deictics used in colloquial Finnish \citep[§721]{HakulinenEtAl2004} with the same distinction in the deictic domain: \textit{tällee}(\textit{n}), \textit{t}(\textit{u})\textit{ollee}(\textit{n}) and \textit{sillee}(\textit{n}). These forms historically derive from the singular demonstratives \textit{tämä}, \textit{tuo} and \textit{se} in the allative case (-\textit{lle}) and the 3\textsuperscript{rd} singular possessive suffix (\textit{-Vn}). \textit{Tällee}(\textit{n}) and \textit{t}(\textit{u})\textit{ollee}(\textit{n}) are rarely mentioned in previous descriptions and probably are used less often in contemporary Finnish. \textit{Sillee}(\textit{n}) has the same basic demonstrative functions as \textit{niin} \citep[§1160]{HakulinenEtAl2004}. Additionally, it is used as a discourse particle in self-repairs and during speech planning \citep[§861]{HakulinenEtAl2004}. \tabref{tab:teptiuk:1} summarises manner deictics in Finnish.

%Adjust table formatting
\begin{table}
\begin{tabularx}{\textwidth}{XXl}
\lsptoprule
\textbf{Deictic differentiation} & \textbf{Standard} & \textbf{Colloquial}\\
\midrule
Speaker-proximal & \textit{näin} & \textit{tällee}(\textit{n}) {\textasciitilde} \textit{tällai}\\
Hearer-proximal & \textit{noin} & \textit{t}(\textit{u})\textit{ollee}(\textit{n}) {\textasciitilde} \textit{tollai}\\
Distal & \textit{niin} & \textit{sillee}(\textit{n}) {\textasciitilde} \textit{sillai}\\
\lspbottomrule
\end{tabularx}
\caption{Manner deictics in Finnish}\label{tab:teptiuk:1}
\end{table}

In quotative constructions, only \textit{näin}, \textit{niin} and \textit{sillee}(\textit{n}) appear systematically. Hearer-proximal \textit{noin} is not observed in the quotative domain in 100 randomly selected examples testing its collocation with the reportative verbs \textit{sanoa} ‘say’ and \textit{ajatella} ‘think’ or the equational verb \textit{olla} ‘be’ (see \sectref{sec:teptiuk:3}). As for the manner deictics \textit{tällee}(\textit{n}) or \textit{t}(\textit{u})\textit{ollee}(\textit{n}), their quotative use is marginal, accounting for only a couple of occurrences in the Finnish new media corpus. Therefore, I exclude them from further discussion and concentrate on \textit{näin}, \textit{niin} and \textit{sillee}(\textit{n}).

In standard Estonian, in turn, only \textit{nii}, the cognate of Finnish distal \textit{niin}, is used regularly in quotative constructions. Outside the quotative domain, the manner deictic \textit{nii} also serves as an intensifier, a causal conjunction and a correlative word \citep[109]{Keevallik2005}. The deictic distinctions of manner and other demonstratives are neutralised at the level of the literary standard. Other manner deictics, e.g. the synonymous adverbs \textit{nõnda,} \textit{sedamoodi,} \textit{selliselt}, meaning ‘this way, so’, appear infrequently as synonyms to \textit{nii} in quotative constructions. The contrastive manner deictic \textit{naa} ‘that (other) way’ does not appear in QIs at all. Therefore, only \textit{nii} is included in the discussion.

Finnish and Estonian manner deictics are used with cataphoric reference. They are part of the preposed QI, which includes reportative verbs. The manner deictic points at a following stretch of RD, as in \REF{ex:teptiuk:4} and \REF{ex:teptiuk:5}. In constructions with reportative verbs, manner deictics are not necessary parts of the QI, and even if one omits them from the QI-clause, the RD is still understood as such. Hence, the main quotative function is carried out by reportative verbs, while the manner deictic is secondary.

\ea\label{ex:teptiuk:4} {Finnish}\\
\ea {(Internet-keskusteluaineistoja)}\\
\gll \textbf{…} \textbf{sanoi} \textbf{näin} \|“tykkään susta  paljon, mutta en sillai.”\|\textsubscript{RD}\\
{} say.\textsc{pst.3sg} so\footnotemark{} {\db}{\db}like.\textsc{prs.1sg} \textsc{2sg.ela} a.lot but \textsc{neg.1sg} thus\\
\glt ‘... he said (lit. \textbf{said} \textbf{so}) \|“I like you a lot, but not that way.”\|\textsubscript{RD}’
\footnotetext{For the sake of convenience, here and in the glosses of other examples, I translate the proximal manner deictics as ‘so’ and the distal as ‘thus’, even though this does not reflect the actual usage of these terms in English.}
\ex {(Internet-keskusteluaineistoja)}\\
\gll Hän \textbf{ajatteli} \textbf{niin}, \textbf{että} \|aika on neljäs 
ulottuvuus~…\|\textsubscript{RD}\\
 \textsc{3sg} think.\textsc{pst.3sg} thus \textsc{comp} {\db}time be.\textsc{prs.3sg} four.\textsc{ord} dimension\\
\glt ‘He thought (lit. \textbf{thought} \textbf{thus}) that \|time is the fourth dimension~…\|\textsubscript{RD}’
\ex {(Demi)}\\
\gll ... \textbf{sanoin} \textbf{sillee} \textbf{et} \|sä tykkäät must …\|\textsubscript{RD}\\
 {} say.\textsc{pst}.1\textsc{sg} thus \textsc{comp} {\db}2\textsc{sg} like.\textsc{prs.2sg} 1\textsc{sg}.\textsc{ela}\\
\glt ‘… I said (lit. \textbf{said} \textbf{thus} \textbf{that}) \|you like me ...\|\textsubscript{RD}’
\z
\z

\ea\label{ex:teptiuk:5}
{Estonian ({Eestimaa Loomakaitse Liit})}\\
\gll Kaija ise \textbf{ütles} \textbf{nii}: \|“Sellist hirmu pole enam ammu ühegi kassi silmis näinud ...”\|\textsubscript{RD}\\
 \textsc{pn} self say.\textsc{pst}.3\textsc{sg} so {\db}{\db}such.\textsc{ptv} fear.\textsc{ptv} \textsc{neg} more long.ago one.\textsc{gen}.\textsc{ptcl} cat.\textsc{gen} eye.\textsc{pl}.\textsc{ine} see.\textsc{pp}\\
\glt ‘Kaija herself said (lit. \textbf{said} \textbf{so}): \|“I haven’t seen such fear in the eyes of a single cat for a long time ...”\|\textsubscript{RD}’
\z

Besides reportative verbs, Finnish \textit{sillee}(\textit{n}) and Estonian \textit{nii} are also used with the equational verbs \textit{olla} ‘be’ \REF{ex:teptiuk:6} and \textit{olema} ‘be’ \REF{ex:teptiuk:7}, respectively. Finnish \textit{näin} appears in such constructions in only a few instances, and \textit{niin} not at all. Since the use of equational verbs in QIs is a typical strategy in colloquial speech, the more colloquial \textit{sillee}(\textit{n}) is preferred over other manner deictics in such a construction. The same seems to be the case in Estonian. Whereas one can find examples in which \textit{nõnda} co-occurs with a reportative verb, combinations with \textit{olema} ‘be’ are unattested.

\ea\label{ex:teptiuk:6} {Finnish} 
\ea {(lansiuusimaa.fi)} \label{ex:teptiuk:6a}\\
\gll … se \textbf{oli} \textbf{sillee} \textbf{et} \|ou nou\|\textsubscript{RD}, ja minä \textbf{olin} \textbf{sillee} \textbf{tyyliin}, \|onks pakko\|\textsubscript{RD} … \\
 {} \textsc{dem} be.\textsc{pst}.3\textsc{sg} thus \textsc{comp} {\db}oh no and 1\textsc{sg} be.\textsc{pst}.1\textsc{sg} thus style.\textsc{ill}/like {\db}be.\textsc{prs.3sg.qp} obligatory\\
\glt ‘… (s)he was like (lit. \textbf{was thus that}) \|oh, no\|\textsubscript{RD}, and I was like (lit. \textbf{was thus like}) \|is it obligatory\|\textsubscript{RD} …’
\ex {(ask.fm)} \label{ex:teptiuk:6b}\\
\gll Aa \textbf{olin} \textbf{sillee} \textbf{et} \|WATAFAK\|\textsubscript{RD}.\\
 \textsc{interj} be.\textsc{pst}.1\textsc{sg} thus \textsc{comp} {\db}what.a.fuck\\
\glt ‘Aa, I was like (lit. \textbf{was} \textbf{thus} \textbf{that}) \|WATAFAK\|\textsubscript{RD}.’
\z
\z

\ea\label{ex:teptiuk:7} {Estonian}\\
\ea {(NMS)} \label{ex:teptiuk:7a}\\
\gll ... isa \textbf{oli} kõrval \textbf{nii} \textbf{et} \textbf{nagu} \|mis sa siis ikka kihutad\|\textsubscript{RD} … \\
{} father be.\textsc{pst}.3\textsc{sg} nearby so \textsc{comp} like {\db}what 2\textsc{sg} then still rush.\textsc{prs.}2\textsc{sg}\\
\glt ‘... father \textbf{was} next to me like (lit. \textbf{so} \textbf{that} \textbf{like}) \|why are you still rushing\|\textsubscript{RD} …’
\ex {(Müürileht)} \label{ex:teptiuk:7b}\\
\gll … ma \textbf{olin} \textbf{nii}, \textbf{et} \|“oh my god, päriselt!”\|\textsubscript{RD}\\
{} {1}\textsc{sg} {be.}\textsc{pst}{.1}\textsc{sg} {so} \textsc{comp} {\db}{\db}\textsc{interj} my god seriously\\
\glt ‘… I was like (lit. \textbf{was} \textbf{so} \textbf{that}) \|“oh my god, seriously!”\|\textsubscript{RD}’
\z
\z

The substitution of a reportative verb with a ‘be’-verb frequently leads to the loss of difference between quotations of speech and thought. The equational verb does not specify the type of event behind the RD and only establishes a predicative structure in the QI. I label this process as “event-neutralisation”, i.e. a process which permits different interpretations of the presented quote. In \REF{ex:teptiuk:6a} from Finnish, the type of reported event behind the RD can be deduced from the sequence of two quotes belonging to two different speakers. It is unlikely that the reporter first quotes another speaker’s thoughts and then his/her own thoughts. Instead, (s)he is most likely to represent a dialogue between him-/herself and another speaker. In \REF{ex:teptiuk:7a} from Estonian, the reporter represents the words his father addressed to him, i.e. a quotation of speech. In contrast, \REF{ex:teptiuk:6b} and \REF{ex:teptiuk:7b} illustrate self-quotations that are not explicitly assigned to a concrete addressee. Here the supporting context does not help to distinguish between a quotation of speech or thought.

An even higher degree of event-neutralisation can be observed in Estonian, which allows for the ellipsis of the NP encoding the original speaker, as in \REF{ex:teptiuk:8}. Thus, instead of an actual utterance produced by some speaker in a different setting, the reporter presents a quote that (s)he considers emblematic for the described circumstances. The omission of the NP expressing the author of the RD makes this reading the most obvious one.

\ea\label{ex:teptiuk:8} {Estonian (NMS)}\\
\gll Tegelt pärnus \textbf{on} \textbf{nii} \textbf{et} \|vabandust aga ma unustasin oma pileti koju\|\textsubscript{RD} …\\
 basically \textsc{pn.ine} be.\textsc{prs}.3\textsc{sg} so \textsc{comp} {\db}sorry but 1\textsc{sg} forget.\textsc{pst.}1\textsc{sg} own ticket.\textsc{ptv} home.\textsc{ill}\\
\glt ‘Basically, in Pärnu it’s (lit. \textbf{it’s} \textbf{so} \textbf{that}) \|sorry, but I forgot my ticket at home\|\textsubscript{RD} ...’
\z

Even though homomorphic constructions do not appear in Finnish, hypothetical quotes can also be introduced by \textit{sillee(n)}. In \REF{ex:teptiuk:9}, \textit{sillee}(\textit{n}) co-occurs with the noun \textit{viesti} ‘message’, encoding the source of the RD. Despite the structural differences between \REF{ex:teptiuk:8} and \REF{ex:teptiuk:9}, in both cases the reporters enact fictional discourse through a demonstration for dramatic purposes.

\ea\label{ex:teptiuk:9} {Finnish (Kaksplus)}\\
\gll ... laittaa kavereille \textbf{viestin} \textbf{sillee} \textbf{et} \|by the way mulle synty viime yönä tyttö / poika mitoilla se ja se\|\textsubscript{RD} jne.\\
{} put.\textsc{prs}.3\textsc{sg} friend.\textsc{pl}.\textsc{all} message.\textsc{gen}  thus \textsc{comp} {\db}by the way  1\textsc{sg}.\textsc{all} be.born.\textsc{pst}.3\textsc{sg} last night.\textsc{ess} girl {} boy measure.\textsc{pl}.\textsc{ade} \textsc{dem} and \textsc{dem} etc.\\
\glt ‘… sends \textbf{the message} to friends saying (lit. \textbf{thus that}) \|by the way last night I gave birth to a girl/boy who measures this and that,\|\textsubscript{RD} etc.’
\z

In addition to representations of factual and fictional quotes depicting verbal or mental processes, \textit{sillee}(\textit{n}) introduces a mimetic expression \REF{ex:teptiuk:10}. Besides the representation of enacted human verbal behavior, QIs can also introduce non-linguistic sound imitations, representational gestures or ideophones \citep[275-295]{Güldemann2008}; see \sectref{sec:teptiuk:4.3} for a similar instance in Hungarian. The mimetic expression in \REF{ex:teptiuk:10} can be interpreted as a quasi-quotation, where instead of using verbal means, the reporter expresses his/her surprise with the emoticon {\textit{0.o}} depicting eyes wide open and two question marks.

\ea\label{ex:teptiuk:10} {Finnish (Demi)}\\
\gll Mä \textbf{olin} \textbf{silleen} \textbf{et} \|0.o??\|\textsubscript{RD}\\
 \textsc{1sg} be\textsc{.pst.1sg} thus \textsc{comp} {\db}\textsc{mim.mir}\\
\glt ‘I was (lit. \textbf{was thus that}) \|0.o??\|\textsubscript{RD}’
\z

Notably, similar functions can be observed online for manner deictics in other languages. See, for example, the German manner deictic \textit{so} in \REF{ex:teptiuk:11}, which is used with the same emoticon expressing the reporter’s surprise.

\ea\label{ex:teptiuk:11} {German (XHardware)}\\
\gll Ich \textbf{so} … \|0.o\|\textsubscript{RD} …\\
 1\textsc{sg} so {} {\db}\textsc{mim.mir}\\
\glt ‘I was (lit. I \textbf{so}) … \|0.o\|\textsubscript{RD} ...’\\
 \z

To sum up, although the Finnish manner deictics can appear in constructions where its Estonian counterpart \textit{nii} is not observed, and vice versa, all function as cataphoric markers. The Finnish manner deictic \textit{niin} is particularly interesting in this regard since it was previously reported to be anaphoric \citep[160]{König2017}. My data shows, however, that \textit{niin} is exclusively attested as a cataphoric quotative marker, and in this it does not differ from the other Finnish manner deictics used in homomorphic constructions.

Although the majority of the examples given here clearly point to a preference for the use of QIs with manner deictics in direct RD, they can also introduce indirect RD. In the non-standard written data considered here, direct RD is usually preferred over indirect. Speakers online often attempt to demonstrate their own or someone else’s previously produced or fictional speech and thought rather than to simply describe them. However, as shown for Hungarian in \sectref{sec:teptiuk:4.3}, the preference for direct or indirect RD can be a crucial factor in the choice of manner deictics in quotative constructions. \tabref{tab:teptiuk:2} summarises the use of manner deictics in quotative constructions in Finnish and Estonian.

\begin{table}
\begin{tabularx}{\textwidth}{llQQ}
\lsptoprule
& \textbf{Reference} & \textbf{Introduction of mimetic expressions} & \textbf{Event neutralisation}\\
\midrule
Finnish & & & \\
\midrule
\textit{näin} & cataphoric & no & + equational verb (marginal) \\
\textit{niin} & cataphoric & no & no\\
\textit{sillee}(\textit{n}) & cataphoric & yes & + equational verb\\ 

\tablevspace
Estonian & & & \\
\midrule
\textit{nii} & cataphoric & no & + equational verb\\ 
\lspbottomrule
\end{tabularx}
\caption{Manner deictics in QIs in Finnish and Estonian}\label{tab:teptiuk:2}
\end{table}

\subsection{ Manner deictics in quotative indexes in Permic}\label{sec:teptiuk:4.2}

Komi and Udmurt have a basic bipartite system of manner deictics: proximal vs. distal. In Udmurt, proximal \textit{taźy} and distal \textit{oźy} do not only indicate a spatial contrast in exophoric function but also have a specific anaphoric function. According to Svetlana Edygarova (p.c.), proximal \textit{taźy} in Udmurt discourse is associated with new information and occupies a pre-focused position. In contrast, distal \textit{oźy} mainly refers to already known or previously mentioned information. In addition, it appears in several idiomatic constructions, e.g. the anaphoric expression \textit{vot oźy} ‘so it is, so it goes’, or as a confirmative particle meaning ‘yes’. Hence, a similar functional division in the quotative domain is expected.

As for Komi, proximal \textit{taďź(i)}\footnote{According to the Komi-Russian dictionary \citep{BeznosikovaEtAl2000}, the form \textit{taďź} is a shortened variant of \textit{taďźi}.} and distal \textit{siďź} are formally distinguished and based on the same stems as other demonstratives, e.g. \textit{tajö} ‘this’, \textit{sijö} ‘that, (s)he’. However, they preserve little functional difference in contemporary language, if any at all, aside from their distribution in several lexicalised expressions, e.g. \textit{siďź bośtny} ‘to take for free (lit. thus)’, \textit{kyďź taďź} ‘why did it happen this way (lit. how so)’.\footnote{An identical idiomatic expression is found in Russian, \textit{kak tak}, which is probably the source of the Komi expression.} Previously, there seemed to be a dialectal difference in the distribution of manner deictics among Komi dialects. In \citeauthor{Uotila1985}’s text collections depicting language use from the first half of the 20\textsuperscript{th} century, proximal \textit{taďź(i)} does not appear at all in dialects of the Komi-Permyak literary standard \citep{Uotila1985}; all recorded speakers only use the distal counterpart \textit{siďź}. In dialects of the Komi-Zyrian literary standard \citep{Uotila1989}, one can similarly observe a preference for distal \textit{siďź}. It is used predominantly anaphorically, referring to the manner or events previously described in discourse. Proximal \textit{taďź(i)} is used only once cataphorically; in all other instances, although they are not numerous, it appears in the anaphoric function, similarly to distal \textit{siďź}. Besides these basic manner deictics, Komi also has focused markers that are formed with the prefixed particle \textit{e-} attached to the basic stem of manner deictics: \textit{etaďź}, \textit{esiďź}. \citet{Fedjuneva2009} argues that focused manner deictics appeared in the language as a relatively recent innovation under the influence of Russian, e.g. \textit{tot} ‘that one’ vs. \textit{\textbf{è}-tot} ‘this one’, \textit{tak} ‘so’ vs. \textit{\textbf{è}-tak} ‘thus (in contrast to so)’. This claim is supported by the lack of corresponding markers in Udmurt and by similarities in the use of demonstratives with the \textit{e-}element in Russian and Komi \citep[95-96]{Fedjuneva2009}.

In quotative constructions of both Udmurt and Komi, manner deictics are used as additional elements in combination with reportative verbs. The focused deictics of Komi are not used in QIs.

In Udmurt quotative constructions, the above-mentioned distinction between the proximal and distal form can also be observed. Most commonly, proximal \textit{taźy} is used cataphorically \REF{ex:teptiuk:12a} in QIs preceding RD, while \textit{oźy} appears as an anaphoric marker following the quote \REF{ex:teptiuk:12b}.

\ea\label{ex:teptiuk:12} {Udmurt}\\
\ea\label{ex:teptiuk:12a} {(vk.com/udmurt\_ept)}\\
\gll \textbf{Veraśke} \textbf{taźy} čömyś: \|“Esli ty ne na Internete {…}”\|\textsubscript{RD}\\
 say.\textsc{prs}.3\textsc{sg} so often {\db}{\db}\{if\} \{2\textsc{sg}\} \{\textsc{neg}\} \{on\} \{internet.\textsc{prep}\}\\
\glt ‘He often says (lit. \textbf{says so}): \|“If you are not on the internet …”\|\textsubscript{RD}’\footnote{The quote depicts Russian speech produced by a non-native speaker.}
\ex\label{ex:teptiuk:12b} {(Blog subcorpus)}\\
\gll \|“Mon pićiges!”\|\textsubscript{RD} – aćiz śaryś \textbf{oźy} \textbf{vera}.\\
{\db}{\db}\textsc{1sg} small.\textsc{compar} {} self.3\textsc{sg} about thus say.\textsc{prs}.3\textsc{sg}\\
\glt ‘\|“I am smaller!”\|\textsubscript{RD} – she says (lit. \textbf{thus} she \textbf{says}) about herself.’
\z
\z

However, one can also find instances where the manner deictics are used in the opposite way; consider the anaphoric use of proximal \textit{taźy} in \REF{ex:teptiuk:13}.

\ea\label{ex:teptiuk:13} {Udmurt (vk.com/udmurt\_ept)}\\
\gll \|“Anaj, nu eščё {čut’-čut’} …”\|\textsubscript{RD} – \textbf{taźy} \textbf{šuysal} dyr mon …\\
{\db}{\db}mom \{\textsc{ptcl}\} \{more\} \{a.bit\} {} {} so say.\textsc{cond.1sg} maybe \textsc{1sg}\\
\glt ‘\|“Mom, give me a bit of time [and I will get up, put a kettle on the stove, please]”\|\textsubscript{RD} – that’s what I would probably say (lit. \textbf{so I would say} maybe) [if I was at home].’
\z

Example \REF{ex:teptiuk:13} reflects a less systematic use of manner deictics. An investigation of the speech of the blogger on \textit{vk.com/udmurt\_ept} also shows that he deviates from the established use of markers outside the quotative domain. In his speech (and that of few other speakers), proximal \textit{taźy} (rather than distal \textit{oźy}) appears in anaphoric reference \REF{ex:teptiuk:14}. Furthermore, proximal \textit{taźy} is used instead of distal \textit{oźy} in the anaphoric expression \textit{vot oźy} ‘so it goes’, which is actually a fixed idiomatic expression in contemporary Udmurt.

\ea\label{ex:teptiuk:14} {Udmurt (vk.com/udmurt\_ept)}\\
\gll {mon ebašil školae 11}\\
 1\textsc{sg} \{fucking.go.\textsc{pst}.\textsc{m\}} school.\textsc{ill} 11\\
\gll {ar i \textbf{taźy} mon}\\
 year \{and\} so 1\textsc{sg}\\
\gll otmečaju soje bydtemme!???\\
 \{celebrate.\textsc{prs}.1\textsc{sg\}} \textsc{dem}.\textsc{acc} end.\textsc{ptcp}.\textsc{acc}1\textsc{sg}\\
\glt {‘[We were sitting and drinking tea with classmates. Walked around a bit. We were watching movies all night. (...)] I was fucking going to the school for 11 years and \textbf{so} I celebrate it, my graduation!???’}
\z

Similarly, the distal \textit{oźy} is attested as a cataphoric marker pointing at the following quote \REF{ex:teptiuk:15}.

\ea\label{ex:teptiuk:15} {Udmurt (Blog subcorpus)}\\
\ea\label{ex:teptiuk:15a}
\gll Vyny tužo maly ke \textbf{oźy} \textbf{šuiz}: \|“Oh, mar ke so  tuž kurdyt ...”\|\textsubscript{RD} \\
 younger.brother.1\textsc{sg} also why \textsc{indef} thus say.\textsc{pst.3sg} {\db}\textsc{interj} what \textsc{indef} \textsc{dem} very {\db}scary\\
\glt ‘My younger brother also for some reason said (lit. \textbf{said} \textbf{thus}): \|“Wow, she’s somewhat very scary ...”\|\textsubscript{RD}’
\ex\label{ex:teptiuk:15b}
\gll Vitaľij Agabajev śaryś \textbf{oźy} \textbf{šuysal}: \|so odigez geńiaľnoj arťist val.\|\textsubscript{RD} \\
\textsc{pn} \textsc{pn} about thus say.\textsc{cond.1sg} {\db}\textsc{3sg} one.3\textsc{sg} genius.\textsc{adj} artist be.\textsc{pst.3sg}\\
\glt ‘\textbf{I could} say this (lit. \textbf{say thus}) about Vitaliy Agabaev: \|he was one of the genius artists.\|\textsubscript{RD}’
\z
\z

Two different explanations can be proposed for the appearance of {\textit{oźy}} in preposed QIs instead of in the postposed position, which can be considered more common. One scenario suggests that {\textit{oźy}} is still anaphoric and refers to information previously mentioned in the context and later repeated as a quote. In \REF{ex:teptiuk:15a}, this claim may be supported by the presence of the adverb \textit{tužo} ‘also’. Thus, one could assume that the reporter quotes her brother, anaphorically referring to an identical utterance that has been previously produced by another speaker in a different context, i.e. ‘my brother made an utterance identical/similar to someone else’s’; see \REF{ex:teptiuk:16} for the collocation of \textit{oźy} and the focus particle \textit{ik} in a similar context. With respect to \REF{ex:teptiuk:15b}, one could assume that the reporter refers to an already familiar opinion about the famous artist. However, in both cases the context does not explicitly support this explanation. Alternatively, \REF{ex:teptiuk:15} may simply present a less systematic use of \textit{oźy}. Thus, similar to proximal \textit{taźy} \REF{ex:teptiuk:13}, some idiolects may reflect asymmetry with its more conventionalised use as an anaphoric marker.

Separately, one can also observe the collocation of \textit{oźy} with the particle \textit{ik} in preposed QIs \REF{ex:teptiuk:16}. The particle \textit{ik} functions as a focus particle or as a marker signalling the repetition of an element of a situation; cf. the label \textit{marker povtora elementa situacii} in \citet[445]{Zubova2016} and the translation ‘the same’ in \citet{Arkhangelskiy2014}. According to \citet[445-446]{Zubova2016}, in the Beserman dialect of Udmurt, the particle \textit{ik} is frequently in collocation with anaphoric elements. In QIs, the collocation of \textit{oźy} and \textit{ik} is used to present RDs already mentioned in a different form in discourse. Thus, in \REF{ex:teptiuk:16} the reporter confirms that somebody was indeed chased away.

\ea\label{ex:teptiuk:16} {Udmurt (vk.com/knyazpozdey)}\\
\gll {\textbf{Oźy} \textbf{ik} \textbf{šuizy}: \|“Myn tatyś!”\|\textsubscript{RD}}\\
 {thus \textsc{ptcl} say.\textsc{pst.3pl} {\db}{\db}go.\textsc{imp.2sg} here.\textsc{ela}}\\
\glt ‘\textbf{They} \textbf{also} said (lit. \textbf{said} \textbf{thus}): \|“Go away from here!”\|\textsubscript{RD}’
\z

In sum, in contemporary Udmurt, the manner deictics are used in quotative constructions as follows: Proximal \textit{taźy} is primarily a cataphoric marker, referring to following quotes; distal \textit{oźy} is mainly used anaphorically following quotes. The appearance of the distal manner deictic \textit{oźy} as a part of a preposed QI is quite common when \textit{oźy} collocates with the additive particle \textit{ik}. Thus, \textit{oźy ik} serves as a reference to a previously described event repeated as a quote.

Some less frequent uses of the manner deictics that do not fall under the above rules may reflect the decreasing linguistic intuition of Udmurt speakers under the influence of Russian, which uses the proximal manner deictic \textit{tak} both for anaphoric and cataphoric reference \REF{ex:teptiuk:17}. The distal (\textit{sjak}) and contrastive manner deictic (\textit{ètak}) do not appear systematically in Russian QIs.

\ea\label{ex:teptiuk:17} {Russian}\\
\ea\label{ex:teptiuk:17a} {(Woman.ru)}\\
\gll \`{E}to on \textbf{tak} \textbf{skazal}, \textbf{čto} \|žit’ vmeste budete\|\textsubscript{RD} …\\
 \textsc{dem} he so say.\textsc{pst}.\textsc{m} \textsc{comp} {\db}live.\textsc{inf} together be.\textsc{fut}.2\textsc{pl}\\
\glt ‘This is him who said (lit. \textbf{said so that}) \|you will live together\|\textsubscript{RD} …’
\ex\label{ex:teptiuk:17b} {(Vremja Novostej Online)}\\
\gll \|“Neobxodimo doždat’sja kopii protokola …”\|\textsubscript{RD}  \textbf{tak} \textbf{otvetil} na vopros … Geral’d {[sic!]} Jaroš.\\
{\db}{\db}necessary wait.\textsc{inf} copy.\textsc{gen} record.\textsc{gen} {} so answer.\textsc{pst}.\textsc{m} on question {} \textsc{pn} {} \textsc{pn}\\
\glt ‘\|“It’s necessary to wait for the copy of the record [and then conduct corresponding steps]”\|\textsubscript{RD} – \textbf{so answered} Gerald [sic!] Jarosch to a question …’
\z
\z

In Komi quotative constructions with manner deictics, it is observed that only the proximal marker \textit{taďź(i)} ‘so’ appears with reportative verbs, \REF{ex:teptiuk:18a} and \REF{ex:teptiuk:18c}, and inchoative verbs \REF{ex:teptiuk:18b} in forming QIs. In the data, such combinations are not frequent and account for only three instances. However, this figure should not be taken as definitive and may only show less frequent quotative use of manner deictics in Komi new media texts. It is noteworthy that the proximal marker only appears in Komi QIs in internet communications, whereas distal \textit{siďź} ‘thus’ can be used in quotative constructions outside the new media genre \REF{ex:teptiuk:18}.

\ea\label{ex:teptiuk:18} {Komi}\\
\ea\label{ex:teptiuk:18a} (Alёna Tuvsovja)\\
\gll … \textbf{mövpyšti} \textbf{taďź}: \|“Metög na udžalyśjas śuröny~…”\|\textsubscript{RD}\\
{} think.\textsc{pst.1sg} so {\db}{\db}1\textsc{sg.abe} \textsc{ptcl} worker.\textsc{pl}  get.found.\textsc{prs.3pl}\\
\glt ‘[Like most of the people,] \textbf{I} thought (lit. \textbf{thought so}): \|“The workers will be found without me ...”\|\textsubscript{RD}’

\ex\label{ex:teptiuk:18b} (vk.com/biarmian)\\
\gll A \textbf{zavoďitćö} \textbf{taďźi}: \|“Sövetsköj obščestvennost’ …”\|\textsubscript{RD}\\
and begin.\textsc{prs.3sg} so {\db}{\db}Soviet \{community\}\\
\glt {‘And \textbf{it} begins (lit. \textbf{begins} \textbf{so}): \|“The Soviet community ...”\|\textsubscript{RD}’}

\ex\label{ex:teptiuk:18c} (Tuś)\\
\gll \|“Nevažno v kakom krutom vuze ty učiš’sja, važno sumet’ sebja realizovať”\|\textsubscript{RD} 11-öd klassyn velödčigön \textbf{taďźi} \textbf{menym} \textbf{viśtalis} gimnazijasa radejtana velödyśjasyś öťi Alla Aľeksandrovna Taskajeva. \\
{\db}{\db}\{unimportant.\textsc{adv}\} \{in\} \{which.\textsc{prep}\} \{cool.\textsc{prep}\} \{university.\textsc{prep}\} \{2\textsc{sg}\} \{study.\textsc{prs.2sg}\} \{important.\textsc{adv}\} \{manage.\textsc{inf}\} \{self.\textsc{acc}\} \{fulfill\textsc{.inf}\} 11-\textsc{ord} class.\textsc{ine} study.\textsc{cv.instr} so  1\textsc{sg.dat} tell.\textsc{pst.3sg} gymnasium.\textsc{adj} favorite teacher.\textsc{pl.ela} one \textsc{pn} \textsc{pn} \textsc{pn}\\
\glt ‘\|“It isn’t important what kind of cool university you are studying at, it is important to manage to fulfil yourself,”\|\textsubscript{RD} so I was told (lit. \textbf{so told me}) when I studied in the 11\textsuperscript{th} class of one of the gymnasium’s favourite teachers, [the one (called)] Alla Aleksandrovna Taskayeva.’
\z
\z

Since \textit{taďź(i)} is only rarely employed in my data, I turned to the available text collections \citep{Uotila1985,Uotila1989} and the digital corpus that assembles texts not belonging to the new media genre (see \sectref{sec:teptiuk:3}) to see whether they show different results. As mentioned at the beginning of this subsection, in Komi-Permyak dialects \citep{Uotila1985}, only distal \textit{siďź} is used. It also appears in quotative constructions referring cataphorically to the following quote \REF{ex:teptiuk:19}. In Komi-Zyrian texts \citep{Uotila1989}, both markers are used but neither appears in quotative constructions.

\ea\label{ex:teptiuk:19} Komi-Permyak (\citealt[40]{Uotila1985}; glossing and translation mine)\\
\gll a sar’ \textbf{viśtalis} \textbf{siďź}: \|“on-kö aďďźy ćuńkyčlö, me tenö vija.”\|\textsubscript{RD}\\
 and tzar tell.\textsc{pst.3sg} thus {\db}{\db}\textsc{neg.2sg-ptcl.cond} see.\textsc{cn} ring.\textsc{dat} 1\textsc{sg} 2\textsc{sg.acc} kill.\textsc{prs.1sg}\\
\glt ‘And the tzar said (lit. \textbf{said thus}): \|“If you don’t find the ring, I will kill you.”\|\textsubscript{RD}’
\z

Available texts from the digital corpus show that both manner deictics are used in QIs. However, distal \textit{siďź} is more frequent than proximal \textit{taďź(i)}. As for the referential function, both markers can be used either anaphorically or cataphorically, as in \REF{ex:teptiuk:20}. Distal \textit{siďź} is more common in cataphoric functions \REF{ex:teptiuk:20a}; it is only attested twice in anaphoric functions \REF{ex:teptiuk:20b}. In contrast, proximal \textit{taďź(i)} is predominantly used anaphorically \REF{ex:teptiuk:20c}; only some instances reflect a cataphoric use \REF{ex:teptiuk:20d}.

\ea\label{ex:teptiuk:20} {Komi (Komi corpus)}\\
\ea\label{ex:teptiuk:20a}
\gll ... i \textbf{siďź} \textbf{šuöny}: \|byťťökö seni sijö götyr pyďďi olö.\|\textsubscript{RD}\\
{} and thus say.\textsc{prs.3pl} {\db}as.if there 3\textsc{sg} wife instead live.\textsc{prs.3sg}\\
\glt ‘… they say (lit. \textbf{say thus}): \|as if she lives there instead of the wife.\|\textsubscript{RD}’

\ex\label{ex:teptiuk:20b}
\gll \|A menym kolö korśny arlyda ńin mortös\|\textsubscript{RD} – \textbf{siďź} \textbf{dumajtis} \'{N}ikolaj … \\
{\db}but 1\textsc{sg}.\textsc{dat} must.\textsc{prs}.3\textsc{sg} find.\textsc{inf}  {in.years.}\textsc{adj} already person.\textsc{acc}{1}\textsc{sg} {} {thus}  {think.}\textsc{pst}{.3}\textsc{sg} \textsc{pn}\\
\glt ‘\|But I have to find a person who is already elderly,\|\textsubscript{RD} – thought (lit. \textbf{thus thought}) Nikolay ...’\\

\ex\label{ex:teptiuk:20c}
 \gll \|“Bur olömsö oškyšta, (…)”\|\textsubscript{RD} – \textbf{taďźi} \textbf{šuö} poet.\\
{\db}{\db}good life.\textsc{acc.3sg} praise.\textsc{prs.1sg} {} {} so say.\textsc{prs}.3\textsc{sg} poet\\
\glt ‘\|“Good life I praise, [bad life I wipe away],”\|\textsubscript{RD} – \textbf{so said} the poet.’

\ex\label{ex:teptiuk:20d}
\gll Tajö sijö \textbf{taďźi} \textbf{šuö}: \|“Nyvka, a komandovajtö!”\|\textsubscript{RD}\\
 \textsc{dem}.\textsc{prox} \textsc{3sg} so say.\textsc{prs}.3\textsc{sg} {\db}{\db}girl but give.orders.\textsc{prs.3sg}\\
\glt ‘He said (lit. \textbf{said} it \textbf{so}): \|“A girl, but she gives orders!”\|\textsubscript{RD}’
\z
\z

Despite the lack of a representative number of examples in my corpus, other materials show that manner deictics can refer indiscriminately to preceding and following RDs in contemporary Komi QIs. This use of manner deictics resembles the use of \textit{tak} in Russian and could be motivated by Russian influence. Different Komi speakers may associate different indigenous markers with Russian \textit{tak} \REF{ex:teptiuk:17} and use them according to the Russian model. Such pattern replications are frequently observed in contemporary Komi as a way to preserve Komi language use through the choice of indigenous markers in constructions modelled on Russian or of language features that could be common to both languages (see \citealt{Leinonen2006,Leinonen2009}). Thus, Komi-Permyak speakers may turn to the distal marker as the closest equivalent to Russian \textit{tak}, while in Komi-Zyrian one can observe the use of distal \textit{siďź} more frequently; note, however, that proximal \textit{taďź}(\textit{i}) also appears as the only option in several idiolects, as reflected in my data \REF{ex:teptiuk:18}.

The use of Udmurt and Komi manner deictics is summarised in \tabref{tab:teptiuk:3}.

%adjust table formatting
\begin{table}
\begin{tabularx}{\textwidth}{lQll}
\lsptoprule
& \textbf{Primary endophoric meaning} & \textbf{Reference} & \textbf{Position of QIs}\\
\midrule
Udmurt & & & \\
\midrule 
\textit{taźy} & new information & primarily cataphoric & pre- and post-posed\\
\textit{oźy} & known information & primarily anaphoric & pre- and post-posed\\

\tablevspace
Komi & & & \\
\midrule
\textit{taďź}(\textit{i}) & no specialisation & cata- and anaphoric & pre- and post-posed\\ \textit{siďź} & no specialisation & cata- and anaphoric & pre- and post-posed\\ 
\lspbottomrule
\end{tabularx}
\caption{Manner deictics in QIs in Udmurt and Komi}\label{tab:teptiuk:3}
\end{table}


\subsection{ Manner deictics in quotative indexes in Hungarian}\label{sec:teptiuk:4.3}

Hungarian has a bipartite system of manner deictics, contrasting proximal \textit{így} (1) and distal \textit{úgy}. These markers have adopted distinct functions in the quotative domain. Proximal \textit{így} is used as a general mimetic marker introducing a demonstration into discourse (Beáta Gyuris, p.c.) \REF{ex:teptiuk:21a}. In turn, distal \textit{úgy} functions as a marker pointing at the manner of action. Hence, instead of demonstrations, the marker is usually followed by a comparison with another action (Beáta Gyuris, p.c.), as in \REF{ex:teptiuk:21b}.\footnote{I am not aware of studies addressing this functional division of manner deictics in contemporary Hungarian.}

\ea\label{ex:teptiuk:21} Hungarian\\
\ea\label{ex:teptiuk:21a} (Beáta Gyuris, p.c.)\\
\gll \textbf{Így} \textbf{csinálom}: \{demonstration\}.\\
 so do\textsc{.prs.1sg.def}\\
\glt ‘I \textbf{do} \textbf{it} \textbf{like} \textbf{this}: \{demonstration\}.’

\ex\label{ex:teptiuk:21b} (gyakorikerdesek.hu)\\
\gll Én is \textbf{úgy} csinálom \textbf{mint} \textbf{te}.\\
 1\textsc{sg} also thus do.\textsc{prs.1sg.def} like 2\textsc{sg}\\
\glt ‘I also do it \textbf{like} \textbf{you}.’
\z
\z

Similarly to Udmurt, the functions of the manner deictics outside the quotative domain affect their use in quotative constructions. This effect is reflected in two aspects: (i) the types of RD, and (ii) the position of the QI within the RD-construction.

As a mimetic marker, \textit{így} is used only with direct RD that “reports and demonstrates what has been uttered” \citep[338]{Dömötör2001}. In mimetic expressions, demonstrations are carried out by movements and gestures. In quotative constructions, RD is a demonstration of somebody’s words \REF{ex:teptiuk:22a} or mental activities \REF{ex:teptiuk:22b} (see \citealt{ClarkGerrig1990} on quotations as demonstrations). Note that, in Hungarian, even direct RD can be preceded by the complementiser \textit{hogy} \REF{ex:teptiuk:22c}. Thus, unlike in many SAE languages, the presence of the complementiser \REF{ex:teptiuk:22b} is not a sufficient criterion for distinguishing between indirect and direct RD.

\ea\label{ex:teptiuk:22} Hungarian\\
\ea\label{ex:teptiuk:22a} (MNSz)\\
\gll a rapper rövid beszédében \textbf{így} \textbf{mutatta} \textbf{be} magát: \|“cigány vagyok, zsidó vagyok …”\|\textsubscript{RD}\\
\textsc{def} rapper short talk.3\textsc{sg}.\textsc{ine} so show.\textsc{prs.3sg.def} into self.\textsc{acc} {\db}{\db}Gypsy be.\textsc{prs.1sg} Jew be.\textsc{prs.1sg}\\
\glt ‘The rapper \textbf{introduced} himself in his short speech by saying (lit. \textbf{so}): \|“I am Gypsy, I am Jew …”\|\textsubscript{RD}’

\ex\label{ex:teptiuk:22b} (MNSz)\\
\gll Én is \textbf{így} \textbf{gondoltam} \textbf{hogy} \|nem túl jó választás nekem a 29er …\|\textsubscript{RD}\\
 1\textsc{sg} also so think.\textsc{pst}.1\textsc{sg} \textsc{comp} {\db}\textsc{neg} very good choice \textsc{dat}.1\textsc{sg} \textsc{def} 29er\\
\glt ‘I also thought (lit. \textbf{thought it so that}) \|the 29er is not a very good choice for me ...\|\textsubscript{RD}’

\ex\label{ex:teptiuk:22c} (Google books)\\
\gll ... azt \textbf{mondta}, \textbf{hogy} \|szeretlek, drága csillagom.\|\textsubscript{RD}\\
{} \textsc{dem.acc} say.\textsc{pst.3sg.def} \textsc{comp} {\db}love.\textsc{prs.1sg.def.2sg} expensive star.\textsc{1sg}\\
\glt ‘... (he) said (lit. \textbf{said} \textbf{that}) \|I love you, my dear star.\|\textsubscript{RD}'
\z
\z

As in Finnish \REF{ex:teptiuk:10}, some of the Hungarian mimetic expressions introduced by a QI with \textit{így} can be interpreted as quasi-quotations. In \REF{ex:teptiuk:23}, the reporter puts a person’s moves and gestures into words.

\ea\label{ex:teptiuk:23} Hungarian (MNSz)\\
\gll … a kezével \textbf{így} \textbf{mutatta}, \textbf{hogy} \|ha nem, akkor ez a táska bumm.\|\textsubscript{RD}\\
{} \textsc{def} hand.3sg.\textsc{com} so show.\textsc{pst.3sg.def} \textsc{comp} {\db}if \textsc{neg} then \textsc{dem} \textsc{def} bag \textsc{ideo}\\
\glt ‘… with his hand he showed (lit. \textbf{showed so}) \textbf{that} \|if not, then this bag will go \textit{boom}.\|\textsubscript{RD}’
\z

Functionally, distal \textit{úgy} introduces the content of a proposition expressed as a quote, rather than a mimetic quote. Thus, QIs with \textit{úgy} do not show any restriction to one type of RD and can be used with either direct \REF{ex:teptiuk:24a} or indirect \REF{ex:teptiuk:24b} quotations.

\ea\label{ex:teptiuk:24} Hungarian\\
\ea\label{ex:teptiuk:24a} (MNSz)\\
\gll ... \textbf{úgy} \textbf{mondta}: \|látom nagyon istenfélők vagytok\|\textsubscript{RD} …\\
 {} thus say.\textsc{pst}.3\textsc{sg.def} {\db}see.\textsc{prs}.1\textsc{sg.def} very God.fearing.\textsc{pl} be.\textsc{prs.2pl}\\
\glt ‘... said (lit. \textbf{said it thus}): \|I see you are very God-fearing\|\textsubscript{RD} ...’

\ex\label{ex:teptiuk:24b} (nepmese.hu)\\
\gll \textbf{Úgy} \textbf{mondta}, \textbf{hogy} \|kell neki a száz forint~…\|\textsubscript{RD}\\
 thus say.\textsc{prs.3sg.def} \textsc{comp} {\db}need.\textsc{prs.3sg} \textsc{dat.3sg} \textsc{def} 100 forint\\
\glt ‘[He] \textbf{thus said that} \|he needs the one hundred forints\|\textsubscript{RD} ...’
\z
\z

In principle, indirect quotations as in \REF{ex:teptiuk:24b} do not contradict the statement that \textit{úgy} introduces only the content of a quote, differing from the original utterance at least in deictic orientation, i.e. ‘he needs the one hundred forints’ vs. ‘I need the one hundred forints’. In addition, one can expect that the choice of original words might have differed from those reported, e.g. ‘Give me, please, one hundred forints’ or ‘Could you lend me the one hundred forints’. As \citet[338]{Dömötör2001} points out, “[w]hile direct speech [equivalent to direct RD] reports and demonstrates what has been uttered, indirect speech [equivalent to indirect RD] renounces this demonstration (…) [indicating] that the author has understood the utterance and based on this, he reformulates its content”. Yet, this statement is problematic with direct quotations such as \REF{ex:teptiuk:24a}, which resemble a demonstration of a person’s utterance rather than a mere depiction of its content. A closer look at the syntactic distribution of the manner deictics reveals that the proximal \textit{így} is not restricted to one position within the RD-construction: it can appear preceding \REF{ex:teptiuk:22}–\REF{ex:teptiuk:23} or following \REF{ex:teptiuk:25a} the RD, or it can split the RD into parts \REF{ex:teptiuk:25b}. In turn, QIs with the distal \textit{úgy} are restricted to the pre-RD position.

\ea\label{ex:teptiuk:25} Hungarian (MNSz)\\
\ea\label{ex:teptiuk:25a}
\gll \|De egy tehénnel tette\|\textsubscript{RD} – \textbf{így} \textbf{a} \textbf{riporter}.\\
{\db}but \textsc{indef} cow.\textsc{com} do.\textsc{prs.3sg.def} {} so \textsc{def} reporter\\
\glt ‘\|But he did it with a cow\|\textsubscript{RD} – \textbf{so} [said] \textbf{the} \textbf{reporter}.’

\ex\label{ex:teptiuk:25b}
\gll \|Á nem,\|\textsubscript{RD} (\textbf{így} \textbf{ő}), \|hisz mi immár kétezer éve imádkozunk Jeruzsálemért.\|\textsubscript{RD}\\
{\db}\textsc{interj} \textsc{neg} so 3\textsc{sg} {\db}believe.\textsc{prs.3sg} 1\textsc{pl} now two.thousand year.3\textsc{sg} pray.\textsc{prs.1pl} Jerusalem.\textsc{caus}\\
\glt \|‘Ah no,\|\textsubscript{RD} (\textbf{so he} [said]), \|we have already prayed for Jerusalem for two thousand years.\|\textsubscript{RD}’
\z
\z

As shown in \REF{ex:teptiuk:25}, QIs with the proximal \textit{így} do not always contain a speech or epistemic verb. Event-neutralised QIs (\sectref{sec:teptiuk:4.1}) consists merely of an NP referring to the original speaker and the manner deictic, as in \REF{ex:teptiuk:25a} and \REF{ex:teptiuk:25b}. In contrast, QIs with the distal \textit{úgy} must contain reportative verbs, otherwise the whole construction is considered ungrammatical. Hence, in the case of proximal \textit{így}, the RD-construction can be considered a subtype of mimetic construction, i.e. ‘X is/was like this: \{demonstration\}’, while in the case of distal \textit{úgy}, the RD-construction is a reproduction of somebody’s words or thoughts: ‘X says/thinks something like that: \{quote\}’.

Previous studies have already pointed out the additional meanings expressed by the manner deictics in Hungarian quotative constructions. \citet[83]{Kiefer2016} indicates that “if the exact wording of the reported utterance is at stake, the reported utterance is repeated but the reporting clause contains the [proximal] adverbial particle \textit{így} ‘so, thus’”. \citet[607]{Körtvély2016}, in turn, mentions that “[n]ative speakers of Hungarian estimate both versions [of constructions with and without the distal \textit{úgy}] as quasi equivalent in their meaning; however, some of them consider the contents of the \textit{úgy-}type as less certain”.

To complement \citeauthor{Kiefer2016}’s statement, my investigation shows that QIs with \textit{így} can also introduce hypothetical quotes, as in \REF{ex:teptiuk:26}, where the reporter presents a quote purportedly said by the world-famous Hungarian football player, Ferenc Puskás.

\ea\label{ex:teptiuk:26} {Hungarian (MNSz)}\\
\gll \|“Nagy pénz, kis foci.”\|\textsubscript{RD}  –  vagy \textbf{mégsem} \textbf{így} \textbf{mondta} \textbf{volna} Puskás?\\
{\db}{\db}big money small football {} or still.also.\textsc{neg} so say.\textsc{pst.3sg.def} be\textsc{.cond.3sg} \textsc{pn}\\
\glt ‘\|“Big money, small football.”\|\textsubscript{RD}  – as Puskás would have said (lit. or \textbf{wouldn’t have} Puskás \textbf{said so})?’
\z

Note that Puskás’s actual utterance \textit{Kis} \textit{pénz} \textit{–} \textit{kis} \textit{foci,} \textit{nagy} \textit{pénz} \textit{–} \textit{nagy} \textit{foci} ‘Small money – small football, big money – big football’,\footnote{This utterance is frequently attributed to Puskás, although there is no actual proof that he has ever said it.} differs from the one presented in \REF{ex:teptiuk:26}. Thus, instead of presenting the original utterance, the reporter modifies it according to his/her aims. Since \textit{így} can introduce hypothetical quotes, I suggest revising \citeauthor{Kiefer2016}’s claim: instead of associating the use of the proximal manner deictic \textit{így} with the exactness of the presented quote, I propose associating it with the directness of the quote. Of course, direct quotes introduced by the proximal manner deictic may be represented by verbatim quotations. However, this condition is not necessarily true for the whole category of direct RD.

As for \citeauthor{Körtvély2016}’s claim, it is to be expected that indirect quotes introduced by QIs with \textit{úgy} can be perceived as approximately reproduced and signal the reporter’s lack of commitment to the content of quote. First, \textit{úgy} introduces the content of quote only, which leads to the difference between the original utterance and its reproduction. Second, several features of RD might be blurred due to syntactic adjustment, while presenting indirect RD. Although both conditions are found in the use of \textit{úgy}, the reporter’s lack of commitment is hardly foregrounded in all RD-constructions where the distal manner deictic is used. As a result, I step back from the analysis proposed above and propose considering both meanings assigned for the manner deictics secondary, rather than universally applicable to their use in the quotative domain.

The main features drawing differences between proximal \textit{így} and distal \textit{úgy} are summarised in \tabref{tab:teptiuk:4}.

%Adjust table formatting
\begin{table}
\fittable{
\begin{tabular}{lll}
\lsptoprule
& \textbf{Proximal \textit{így}} & \textbf{Distal \textit{úgy}}\\
\midrule 
Primary meaning & demonstration of RD & representation of\\
& & content of RD\\
RD type & direct & direct, indirect\\
Reference & cata- and anaphoric & cataphoric\\
Position of QIs & pre-, intra- and postposed & preposed\\
Event-neutralisation & ellipsis of the verb & does not occur\\
\lspbottomrule
\end{tabular}
}
\caption{Features of manner deictics in Hungarian QIs}\label{tab:teptiuk:4}
\end{table}

\section{Summary and discussion}\label{sec:teptiuk:5}

This chapter has shown that manner deictics follow different patterns of distribution in QIs in related languages. According to their referential function, three main types of manner deictics are distinguished: (i) cataphoric, (ii) anaphoric, and (iii) both cataphoric and anaphoric deictics; cf. the summary in \tabref{tab:teptiuk:5}. In addition to their referential function, manner deictics show preferences for the direct or indirect types of RD and for the pre-, intra- or postposed position in the RD-construction.

%Adjust width of columns to content
\begin{table}
\begin{tabularx}{\textwidth}{Xl}
\lsptoprule
\textbf{Referential functions} & \textbf{Manner deictics}\\
\midrule 
Cataphoric & Finnish \textit{näin}, \textit{niin}, \textit{sillee}(\textit{n})\\
& Estonian \textit{nii}\\
& Hungarian \textit{úgy}\\
& Udmurt \textit{taźy}\\
\tablevspace
 Anaphoric & Udmurt \textit{oźy}\\
\tablevspace
Cata- and anaphoric & Komi \textit{taďź(i)}, \textit{siďź}\\
& Hungarian \textit{így}\\
\lspbottomrule
\end{tabularx}
\caption{Referential functions of manner deictics in quotative constructions of selected Finno-Ugric languages}\label{tab:teptiuk:5}
\end{table}

Among the five Finno-Ugric languages, several patterns of distribution can be identified based on the above criteria. In languages with a proximal/distal pair of manner deictics, their meanings and functions outside the quotative domain predetermine their use in QIs. Thus, the Hungarian proximal manner deictic \textit{így} as a general mimetic marker introduces direct RD only. Distal \textit{úgy}, which outside the quotative domain can only refer to the manner of action but not demonstrate it, is used as a marker pointing at the representation of the content of a person’s words or thoughts. Therefore, it can introduce both direct and indirect RD.

In Udmurt, the distribution follows from the meanings of the manner deictics. Proximal \textit{taźy} introduces new information and, in RD-constructions, typically appears in preposed QIs. Distal \textit{oźy}, in turn, refers to already-known information and is used in postposed QIs.

In contrast, in Komi, the proximal and distal manner deictics are interchangeable outside the quotative domain. In the quotative domain, their use seems influenced by the genre in which they are used: distal \textit{siďź} is used more frequently in conventional written texts, while in my new media material only proximal \textit{taďź(i)} is attested in quotative constructions.

In Finnish, proximal and distal manner deictics are exclusively used with cataphoric reference in quotative constructions. This is all the more interesting as the distal manner deictic \textit{niin} is also used as an anaphoric marker outside the quotative domain \citep[160]{König2017}. The Estonian cognate \textit{nii} has been shown to be used only as a cataphoric marker, which confirms \citeauthor{Keevallik2005}’s (\citeyear[116-117]{Keevallik2005}) earlier findings. In colloquial speech, Finnish \textit{sillee(n)} and Estonian \textit{nii} co-occur with reportative verbs or with equational verbs in QIs. QIs with equational verbs can equally introduce quotations of speech and thought, while speech or epistemic verbs restrict the interpretation to one type of RD. Manner deictics can appear in contexts where the RD is not attributed to a concrete speaker and is therefore hypothetical. Usually, the QI introducing hypothetical quotes is structurally less complex.

In addition, two manner deictics, Finnish \textit{sillee(n)} and Hungarian \textit{így}, are also observed introducing mimetic expressions that can be interpreted as quasi-quotations. It is not surprising that the demonstratives are employed as quotatives with mimetic expressions. \citet[521]{Güldemann2008} mentions that many quotative markers have been initially used in indexing mimesis and later on were grammaticalised into exclusively quotative markers or are still employed in their initial function parallel to their relatively new quotative use.

%Check what Diessel writes about Reesink 1993
The observations made for Finno-Ugric languages have parallels in other languages of the world. For example, an exclusively cataphoric manner deictic is found in Usan (Papuan) where “the quote introduction has the cataphoric adverbial \textit{ete} [\textit{e-t-e} ‘this/here’ + postposition \textit{-t} ‘for/at/on/etc.’ + ‘this/here’]: \textit{wo} \textit{ete} \textit{qamar} ‘he said thus’” \citep[218]{Reesink1993}. Among the markers that can be used both cataphorically and anaphorically in QIs, I have already mentioned the Russian manner deictic \textit{tak}. Similarities can be pointed out in the use of German \textit{so} and French \textit{ainsi}, both meaning ‘so’, that appear preceding and following the RD (\citealt{König2017}: 160; \citealt{KarssenbergLahousse2018}). Similarly, in Usan QIs, the manner deictic \textit{ende} ‘this/here’ + ‘given/a particular one’ + postposition \textit{-t} ‘for/at/on/etc.’ + ‘this/here’ can be used both cata- and anaphorically \citep[218]{Reesink1993}.

I am not aware of other languages in which the proximal form of a proximal/distal manner deictics pair is used for direct RD/new information, while the distal form is used for indirect RD/old information, as has been shown for Hungarian and Udmurt, respectively. However, one could expect to find similar patterns in quotative systems of languages not considered here, which is a direction for future research.

\section*{Abbreviations}

\begin{tabularx}{.45\textwidth}{lQ}
\textsc{abe} & abessive case\\
\textsc{acc} & accusative case\\
\textsc{ade} & adessive case\\
\textsc{adj} & adjective\\
\textsc{adv} & adverb\\
\textsc{all} & allative case\\
\textsc{caus} & causative case\\
\textsc{cmc} & computer-mediated communication\\
\textsc{cn} & connegative\\
\textsc{com} & comitative case\\
\textsc{comp} & complementiser\\
\textsc{compar} & comparative\\
\textsc{cond} & conditional mood\\
\textsc{cv} & converb\\
\textsc{dat} & dative case\\
\textsc{def} & definite\\
\textsc{dem} & demonstrative\\
\textsc{ela} & elative case\\
\textsc{ess} & essive case\\
\textsc{fut} & future tense\\
\textsc{gen} & genitive case\\
\textsc{ideo} & ideophone\\
\textsc{ill} & illative case\\
\textsc{imp} & imperative mood\\
\end{tabularx}
\begin{tabularx}{.45\textwidth}{lQ}
\textsc{indef} & indefinite\\
\textsc{ine} & inessive case\\
\textsc{inf} & infinitive\\
\textsc{instr} & instrumental case\\
\textsc{interj} & interjection\\
\textsc{m} & masculine gender\\
\textsc{mim} & mimetic\\
\textsc{mir} & mirative\\
\textsc{neg} & negative\\
\textsc{ord} & ordinal\\
\textsc{prox} & proximal\\
\textsc{pl} & plural\\
\textsc{pn} & proper noun\\
\textsc{pp} & past participle\\
\textsc{prep} & prepositional case\\
\textsc{prs} & present tense\\
\textsc{pst} & past tense\\
\textsc{ptcl} & particle\\
\textsc{ptcp} & participle\\
\textsc{ptv} & partitive case\\
\textsc{qi} & quotative index\\
\textsc{qp} & question particle\\
\textsc{rd} & reported discourse\\
\textsc{sg} & singular\\
\textsc{sns} & social network sites\\
\end{tabularx}

\section*{Data sources}

The data sources were last accessed in the period from 1 February to 1 August 2019.

\subsection*{Electronic corpora}
%Avoid URLs in the margins
\textit{Blog subcorpus = Blog subcorpus of Udmurt corpus}: \url{http://web-corpora.net/UdmurtCorpus/search/}\\
\textit{Internet-keskusteluainestoja} [The databases of internet communications]:  \url{https://korp.csc.fi}\\
\textit{Komi corpus = Korpus komi jazyka} [The corpus of the Komi language]: \url{http://komicorpora.ru}\\
\textit{MNSz = Magyar Nemzeti Szövegtár. Személyes alkorpusz} [The Hungarian National Corpus. Personal Subcorpus]: \url{http://corpus.nytud.hu/mnsz/}\\
\textit{NMS = Eesti keele koondkorpus. Uus media} [Estonian Reference Corpus. New media subcorpus]: \url{https://www.cl.ut.ee/korpused/segakorpus/uusmeedia/}\\

\subsection*{Internet sources}

\textit{Alёna Tuvsovja}: \url{http://tuvsovja.blogspot.com/2014/09/blog-post_6.html}\\
\textit{ask.fm}: \url{https://ask.fm/Joonas_Tuloneen/answers/120680473973}\\
\textit{Daily Mail}: \url{https://www.dailymail.co.uk/news/article-3030136/Abbott-deserved-credit-McGregor.html}\\
\textit{Demi}: \url{https://www.demi.fi/keskustelu/suhteet-aaah-mita-ma-teen}; \url{https://www.demi.fi/keskustelu/suhteet-miten-unohan-jatkan}\\
\textit{Eestimaa Loomakaitse Liit}: \url{http://loomakaitse.eu/lapsed-kui-te-seda-kassipoega-ei-vota-siis-ma-tapan-ta-ara/}\\
\textit{Google books}: \url{https://books.google.ee/books?id=xQbWDwAAQBAJ&lpg=PT415&dq=%22azt%20mondta%20hogy%20szeretlek%20dr%C3%A1ga%20csillagom%22&pg=PP1#v=onepage&q&f=false}\\
\textit{gyakorikerdesek.hu}: \url{https://www.gyakorikerdesek.hu/gyerekvallalas-neveles__babak__1253137-halozsak-}\\
\textit{Kaksplus}: \url{https://kaksplus.fi/threads/alkionsiirroista-plussanneet.2458857/page-11}\\
\textit{lansiuusimaa.fi}: \url{https://www.tallinna24.ee/blogit/satunnainen-herrasmies}\\
\textit{Müürileht}: \url{https://www.muurileht.ee/skoda-mazda-ja-ritmos-liquidos/}\\
\textit{nepmese.hu}: \url{https://www.nepmese.hu/index.php/mesetar/mesek/adj-uram-isten-szaz-forintot}\\
\textit{Tuś}: \url{http://tusjuk.blogspot.com/2015/}\\
\textit{Twitter}: \url{https://twitter.com/allthingsmollie/status/690042054936772608}\\
\textit{vk.com/biarmian}: \url{https://vk.com/biarmian}\\
\textit{vk.com/knyazpozdey}: \url{https://vk.com/knyazpozdey}\\
\textit{vk.com/udmurt\_ept}: \url{https://vk.com/udmurt_ept}\\
\textit{Vremja Novostej Online} [Time of News Online]: \url{http://www.vremya.ru/2009/74/5/228155.html}\\
\textit{Woman.ru}: \url{http://www.woman.ru/relations/men/thread/4996059/}\\
\textit{XHardware}: \url{http://extreme.pcgameshardware.de/mainboards-und-arbeitsspeicher/257287-ram-speed.html}

\sloppy\printbibliography[heading=subbibliography,notkeyword=this]\end{document}
