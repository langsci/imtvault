\chapter{Morphology}
\label{sec:WordForm}

This chapter covers two broad aspects of Gyeli morphology. In the first part, I outline the  forms and types of bound morphemes. These serve as ingredients to form words either through inflection,  derivation, or composition. I follow \citeapo{haspelmath2010} textbook definitions of these terms. Inflection is the morphological process of producing word forms of a lexeme. Inflectional morphemes in Gyeli express grammatical categories such as agreement, tense, mood, negation, and objecthood. As such, inflectional morphemes do not change the lexeme's part of speech. Many of the inflectional morphemes are  syntactically required and thus appear obligatorily. Additionally, their attachment is fully productive and predictable. Inflection is discussed along with the morpheme types in part one of this chapter.

In contrast, derivational affixes create new lexemes that belong to the same word family. A derived lexeme can belong either to the same word class or a different one than its source lexeme. Derivational morphemes are syntactically optional. Also, it is lexically specified which lexeme can take which derivational affix. As such, attachment of derivational affixes is less predictable. Finally, composition is a type of word formation that combines lexemes from different word families. In Gyeli, compounds typically include two lexical morphemes. Derivation and composition processes are discussed in the second part of this chapter.

\section{Morpheme types}
\label{sec:MorphType}

In this section, I give an overview of the types of affixation morphemes found in the Gyeli language. I limit the discussion to overt non-root morphemes. That is, all morphemes discussed in this section are overt,\footnote{I do not consider null-forms here that are found in some nouns and agreement targets. To be consistent with noun class and agreement marking, however, I do represent them in glosses.}  bound, and grammatical.  Thus, lexical roots are not discussed here, but in \chapref{sec:POS}. The same holds for non-overt morphemes, such as portmanteau morphemes like, for instance, the subject-tense-aspect-mood-polarity (\textsc{stamp}) marker and certain copulas. These portmanteau morphemes are free and occur as words in their own right, as presented in \chapref{sec:POS}. 

I organize the presentation through the opposition between derivational and inflectional morphemes with a distinction between those that precede the lexical root (prefixes) and those that follow the root (suffixes). Gyeli has a total of 44 affixes, 15 (34.1\%) of which are derivational, as listed in \tabref{Tab:AffixDeriv}, and 29 (65.9\%) of which are inflectional affixes, as shown in \tabref{Tab:AffixInflect}.\footnote{The derivational forms in parentheses, {\itshape nà}- and -{\itshape ɛ̀dɛ̀}, are minor and non-productive forms.}

\begin{table}
\begin{tabular}{lll lll}
 \lsptoprule
  \multicolumn{3}{c}{Prefixes} &  \multicolumn{3}{c}{Suffixes} \\
\cmidrule(lr){1-3}\cmidrule(lr){4-6}
		Forms & Function & {\pOS}         & Forms & Function & {\pOS}       \\
\midrule
{\itshape ná}- & {\SIM}   	&  {\N}, {\ADJ}	  & (-{\itshape ɛ̀dɛ̀}) & {\NOM}	   & {\N}	\\
({\itshape nà}-)		& adverbializer	&  {\ADV}  		&  -{\itshape a} & {\NOM}& {\N}\\
	& 	& 			& -H & {\NOM}	& {\N} 		\\
		& 	& 			& -{\itshape á}/-{\itshape â} & {\NOM} & {\N} 		\\
		& 	& 			& -{\itshape a} & {\EXT}	& {\V} 		\\
		& 	& 			& -{\itshape ala} & {\EXT}	& {\V} 		\\
		& 	& 			& -{\itshape ɛlɛ} & {\EXT}	& {\V} 		\\
		& 	& 			& -{\itshape ɛga}/-{\itshape aga} & {\EXT} & {\V} \\
		& 	& 			& -{\itshape ɛsɛ} & {\EXT}	& {\V} 		\\
		& 	& 			& -{\itshape ɔwɔ} & {\EXT}	& {\V} 		\\
		& 	& 			& -{\itshape bɔ}/-{\itshape wɔ} & {\EXP}	& {\V} 		\\
		& 	& 			& -{\itshape kɛ}/{\itshape gɛ} & {\EXP}	& {\V} 		\\
		& 	& 			& -{\itshape lɛ} & {\EXP}	& {\V} 		\\
 \midrule 
\raggedleft {\bfseries Total: 2} &  {\bfseries 2}& {\bfseries 3} &  {\bfseries 13} &    {\bfseries 3} &  {\bfseries 2}\\
 \lspbottomrule
 \end{tabular}
\caption{Frequency of derivational affixes by form and function}
\label{Tab:AffixDeriv}
\end{table} 


\tabref{Tab:AffixDeriv} and \tabref{Tab:AffixInflect}  present the functions associated with each affix, either for derivational affixes{\textemdash}for instance, as a nominalization or verb extension suffix{\textemdash} or for inflectional affixes such as noun class prefixes,  and the part(s) of speech each affix derives.
While derivational morphemes are mostly suffixes which attach to verbs, inflectional morphemes are mostly prefixes that are linked to noun class and agreement marking.\footnote{Noun class and agreement prefixes often have alternate forms that are phonologically conditioned. In \tabref{Tab:AffixInflect}, I count a form and its alternate as only one form in order to not artificially increase the number of forms.} 



\begin{table}
\fittable{\begin{tabular}{lll lll}
 \lsptoprule
  \multicolumn{3}{c}{Prefixes} &  \multicolumn{3}{c}{Suffixes} \\
\cmidrule(lr){1-3}\cmidrule(lr){4-6}
		Forms & Function & {\pOS}         & Forms & Function & {\pOS}       \\
\midrule
    {\itshape m}- & n.cl. , {\AGR}	& {\N}, {\MOD}	&  -{\itshape lɛ} & {\NEG}	& {\V}	\\
   {\itshape n}- &  n.cl., {\AGR}	& {\N}, {\MOD} 	&  -{\itshape gà} & {\CONTR}	&  {\SBJ} 	\\
	{\itshape ba}-/{\itshape b}- & n.cl. & {\N}        & -{\itshape o} &  {\VOC} & 	{\N}, {\ADV}	\\
	{\itshape mi}-		& n.cl. & {\N}		&   -H & {\TM}	&  \textsc{stamp}, {\V} 	\\
	{\itshape le}-/{\itshape d}- & n.cl. & {\N} 	&  & 	&  \\
	{\itshape ma}-/{\itshape m}-		& n.cl. & {\N}		&  & 		&			\\
	{\itshape be}-		& n.cl. & {\N} 			& &  		& 			\\ 
	{\itshape w}-		& {\AGR} & {\OBJ}, {\POSS}, {\MOD}	& &  		& 	 		\\
	{\itshape nú}-		& {\AGR} & {\ANA}	& &  		& 			\\
	{\itshape bà}-/{\itshape b}-	& {\AGR} & {\OBJ},  {\MOD} 	& &  		& 	\\
	{\itshape bá}-		& {\AGR} & {\POSS}, {\ANA}, {\NUM}, {\GEN} 			& &  		& 			\\
	{\itshape wɔ́}-		& {\AGR} & {\ANA}	& &  		& 			\\
	{\itshape mì}-/{\itshape my}-		& {\AGR} & {\OBJ},  {\MOD} 	& &  	&	\\
	{\itshape mí}-/{\itshape my}-	& {\AGR} & {\POSS}, {\ANA},   {\NUM}, {\GEN}	& &  		& 		\\
	{\itshape lè}-/{\itshape l}-		& {\AGR} & {\OBJ},  {\MOD}	& &  	& 	\\
	{\itshape lé}-/{\itshape l}- 	& {\AGR} & {\POSS},  {\ANA}	& &  		& 			\\
	{\itshape mà}-/{\itshape m}-		& {\AGR} & {\OBJ}, {\MOD} & &  	& 	\\
	{\itshape má}-		& {\AGR} & {\POSS}, {\ANA}, {\NUM}, {\GEN} 		& &  		& 	 		\\
	{\itshape y}-		& {\AGR} & {\OBJ}, {\MOD}	& &  		& 		\\
	{\itshape yí}-		& {\AGR} & {\POSS}, {\ANA}	& &  		& 			\\
	{\itshape bì}-/{\itshape by}- & {\AGR} & {\OBJ}, {\MOD} & &  		& 	\\
	{\itshape bí}-		& {\AGR} & {\POSS}, {\ANA}, {\NUM}, {\GEN} 			& &  		& 			\\
	{\itshape ny}-		& {\AGR} & {\OBJ}, {\POSS}, {\MOD}	& &  		& 			\\
	{\itshape nyí}-		& {\AGR} & {\ANA}	& &  		& 			\\
	H-		& {\OBJ}.{\LINK} & {\N} 			& &  		& 			\\
 \midrule 
\raggedleft {\bfseries Total: 25} &  {\bfseries 3}	&  {\bfseries 7}&  {\bfseries 4} &   {\bfseries 4} &  {\bfseries 5}	\\
 \lspbottomrule
 \end{tabular}}
\caption{Frequency of inflectional affixes by form and function}
\label{Tab:AffixInflect}
\end{table} 


With regard to the ratio of prefixes to suffixes across derivational and inflectional affixes, prefix forms are higher in number than suffixes. The 27 prefixes constitute 61.4\% of all affixes, while there are only 17 suffixes. Many inflectional prefix forms are segmentally identical, for example {\itshape mi}-, {\itshape mì}-, and {\itshape mí}-. They differ, however, in their tonal specification and thus must  be formally distinguished.


Prefixes map onto a higher number of functions than suffixes.  Most prefixes encode agreement and/or noun class.\footnote{The relation between noun class, agreement class, and grammatical number is discussed in \sectref{sec:Gender}.} In addition to these most frequent functions, there is also an object-linking H tone and the derivational prefixes {\itshape ná}- and {\itshape nà}-. 
Most suffix forms are functionally derivational extension or expansion morphemes.\footnote{See \sectref{sec:EXtp} for the difference between extension and expansion suffixes.} Other derivational suffixes serve as nominalization morphemes. Inflectional suffixes include a negator, contrastive and vocative markers, and also an H tone suffix that marks various tense and mood categories. 

Cross-linguistically, it is not typical that contrastive or vocative suffixes appear as inflectional morphemes. In Gyeli, they differ from the other inflectional affixes since they are not obligatory. In order to make this distinction, I  call them ``markers''. I still consider them as inflectional morphemes, however, for two reasons. First, unlike the derivational affixes, they do not form new lexemes, i.e., they do not have an entry in the lexicon. Second, their attachment is completely predictable, unlike derivational affixes. For example, every subject and object pronoun can take the contrastive marker -{\itshape gà} (but not every verb can take a causative derivational suffix). 

Nine different parts of speech take some form of prefix,  two with derivational and seven with inflectional prefixes, but most prefixes attach to nouns. Prefixes are generally restricted to the domain of the noun phrase. The two derivational prefixes occur with nouns, adjectives, and adverbs. Inflectional agreement prefixes attach to the diverse category of nominal modifiers (\sectref{sec:NAdjuncts}), object and possessor pronouns as well as anaphoric and genitive markers and numerals.  In contrast, suffixes span five different word classes within both noun and verb phrases. Nominalization and vocative  suffixes attach to nouns. Extension, expansion, negation, and tense-mood suffixes attach to verbs. The remainder of suffixes attach to subject pronouns, adverbs, and the subject-tense-aspect-mood-polarity (\textsc{stamp}) marker. 

In the following, I will briefly outline the various categories of prefixes and suffixes grouped by function, focusing on their form patterns. Depending on the complexity of forms, some discussions might be more detailed, for instance on nominalization suffixes (\sectref{sec:NOMSuff}). In contrast, other forms are straightforward, for instance verbal expansions, but their semantics are more complex. In this case, I keep the formal discussion short and elaborate further in \sectref{sec:FormProcess}.  




\subsection{Prefixes}
\label{sec:Prefix}

A noun stem can maximally take three prefixes, as illustrated in \REF{Prefix-extent}.

\ea  \label{Prefix-extent} \textsc{object-linking H tone -- noun class -- similative -- stem}
\z

\noindent The prefix that is closest to the stem is the similative marker {\itshape ná}-. This can be preceded by a plural noun class prefix and an object-linking H tone. 

Gyeli has four different functional types of prefixes: the derivational prefixes {\itshape ná}- (the similative marker) and {\itshape nà}-, and  the inflectional noun class, agreement, and object-linking H tone prefixes. 



\subsubsection{Derivational prefixes} 
\label{sec:SIM}

Gyeli has two derivational prefixes, {\itshape ná}- and {\itshape nà}-, which are segmentally identical, but differ tonally and functionally.
The derivational similative marker {\itshape ná}- forms a functional category on its own expressing the meaning `like'. The prefix is related to the free morpheme {\itshape ná} which serves as a similative marker in noun + noun naming constructions, as discussed in \sectref{sec:SIMword}. Words with the prefix {\itshape ná}- are (historically) derived from  either a verb or noun or are synchronically opaque.  The application of {\itshape ná}- results in common nouns, proper names,  and some adjectives, as shown for each type in \REF{SIMNouns}.

\ea \label{SIMNouns}
\ea  {\bfseries ná-gyàlɛ́} `breastfeeding woman [lit.\ like nursing period]' (common noun)
\ex {\bfseries Ná-nzɛ̂} (Nanzé) `female name [lit.\ like panther]' (proper name)
\ex {\bfseries ná-vyû} `black [lit.\ like blackened]' (adjective)
\z
\z

The {\itshape ná}- similative marker is the most lexicalized prefix in the language since its use is not productive. Instead,  it is lexically specified which nouns and adjectives occur with this marker.  Especially in the case of nouns with the {\itshape ná}- prefix, one could even argue that the prefix is synchronically frozen to the lexical stem since, in many instances, the meaning of the lexical stem is opaque.  There are several reasons, however, why I consider {\itshape ná}- a prefix and not part of the lexical stem.  First, nouns with the {\itshape ná}- prefix are structurally different from other common nouns. If one counted {\itshape ná}- as part of the nominal stem, some of these stems would have a syllable length of four syllables. As discussed in \sectref{sec:SyllN}, however, the maximum syllable length in stems is three syllables (and even this is dispreferred, accounting for only 10\% of the nouns in the database).  Second,  the {\itshape ná}- prefix occurs quite frequently and regularly, especially in the derivation of female names from male names and in adjectives. This suggests that there is a formal pattern (rather than just a random CV syllable shape).   Third,  there is a clear function attributed to {\itshape ná}-, namely that of expressing similarity, as shown in the examples in \REF{SIMNouns}.  Derivation with the prefix {\itshape ná}- is discussed in greater detail in \sectref{sec:NOMSIM}.

The other derivation morpheme {\itshape nà}- is a minor and little productive prefix that derives adverbs from nouns. There are only two known instances, illustrated in \sectref{sec:ADVderiv}. As this prefix has a different phonological form and a distinct derivation function from the similative prefix {\itshape ná}-, I analyze it as a prefix on its own.



\subsubsection{Noun class prefixes}
\label{sec:NCPre}

Noun class prefixes are inflectional morphemes that attach only to common nouns (but not proper names). There are eleven different overt forms which can be grouped into six underlying categories, based on phonological conditioning. The forms and their alternates are listed in \tabref{Tab:NCmorph}.\footnote{The prefix {\itshape be}- does not have a listed alternate form because there is no known instance of a noun using this prefix and having a stem-initial vowel.}

\begin{table}
\begin{tabular}{lll}
 \lsptoprule
Noun class form & Alternate form  & Phonological condition of alternate \\
 \midrule
m- & n- &  assimilation of place of articulation \\
ba- & b- & before stem-initial vowel \\ 
mi- & m- & before stem-initial vowel \\
le- & d- &  before stem-initial vowel \\
ma- & m- & before stem-initial vowel \\
be- & -- & \\
 \lspbottomrule
\end{tabular}
\caption{Noun class prefix forms}
\label{Tab:NCmorph}
\end{table} 

\noindent  Noun class prefixes fill the second of three possible prefix slots in nouns, potentially preceded by the object-linking H tone (see \sectref{sec:OBJTone}) and followed by a similative marker (see \sectref{sec:SIM}). 

It is an inherent property of each noun which noun class prefix(es) the noun can take. Some noun forms do not take any overt prefix at all. Since noun class prefixes are part of the gender and agreement system which operates on a morphosyntactic rather than solely morphological level,  these prefixes  are discussed in greater detail in \sectref{sec:NC} where their forms are organized according to noun and agreement classes. In the context of gender and agreement, I also view a null-form as a category, but since it is not overt, I do not list it as a morpheme in this section.




\subsubsection{Agreement prefixes}
\label{sec:AGRPre}

Like noun class prefixes, agreement prefixes are inflectional bound morphemes. They attach to six different agreement targets: nominal modifiers, object and possessor pronouns, anaphoric markers, numerals, and genitive markers.\footnote{These word classes are not the only agreement targets in Gyeli, but they are the parts of speech that mark agreement by means of a prefix. Other agreement targets have free forms which are described as parts of speech in \chapref{sec:POS}; all agreement targets are listed in \sectref{sec:Nprop} and \sectref{sec:AGRtargets}.}  In contrast to nouns, agreement targets have only one prefix slot. Gyeli has 19 agreement prefix forms, as listed in \tabref{Tab:AGRprefx}. All except two agreement prefixes mark only one agreement class. For instance, the CV prefix {\itshape ba}- and its phonological ({\itshape b}-) and tonal ({\itshape bà}- and {\itshape bá}-) variants always encode agreement class 2.


\begin{table}
\begin{tabularx}{\textwidth}{XXl}
 \lsptoprule
 Prefix form & {\pOS}  & Agreement class \\
 \midrule
    {\itshape m}-	 & {\MOD}	& 1, 3, 9		\\
	{\itshape w}-	 & {\OBJ}, {\POSS}, {\MOD}	& 1, 3 \\
	{\itshape n}-	 & {\MOD}	& 1		\\
	{\itshape nú}-	 & {\ANA}	& 1		\\
	{\itshape bà}-/{\itshape b}- & {\OBJ},  {\MOD} 	& 2	\\
	{\itshape bá}-	 & {\POSS}, {\ANA}, {\NUM}, {\GEN} & 2	\\
	{\itshape wɔ́}-  & {\ANA}	& 3	\\
	{\itshape mì}-/{\itshape my}- & {\OBJ},  {\MOD} 	& 4	\\
	{\itshape mí}-/{\itshape my}- & {\POSS}, {\ANA}, {\NUM}, {\GEN}	& 4	\\
	{\itshape lè}-/{\itshape l}- & {\OBJ},  {\MOD}	& 5	\\
	{\itshape lé}-/{\itshape l}- 	 & {\POSS},  {\ANA}	& 5 \\
	{\itshape mà}-/{\itshape m}-	 & {\OBJ}, {\MOD} & 6	\\
	{\itshape má}-	 & {\POSS}, {\ANA}, {\NUM}, {\GEN} 	& 6	\\
	{\itshape y}- & {\OBJ}, {\MOD}	& 7	\\
	{\itshape yí}-	 & {\POSS}, {\ANA}	& 7	\\
	{\itshape bì}-/{\itshape by}-  & {\OBJ}, {\MOD} & 8	\\
	{\itshape bí}-	 & {\POSS}, {\ANA}, {\NUM}, {\GEN} & 8	\\
	{\itshape ny}- & {\OBJ}, {\POSS}, {\MOD}	& 	9 \\
	{\itshape nyí}- & {\ANA}	& 9	\\ 
  \lspbottomrule
\end{tabularx}
\caption{Agreement prefixes across agreement targets and agreement classes}
\label{Tab:AGRprefx}
\end{table} 

Most agreement classes have two sets of agreement prefixes distinguished by an L versus an H tone.  The L tone forms and the consonantal prefix {\itshape y}- attach to object pronouns and nominal modifiers. The H tone prefixes generally mark possessor pronouns and anaphoric markers. The plural classes 2, 4, 6, and 8 also mark numerals and the genitive marker with an H tone prefix.


There are two prefixes which mark several agreement classes across different parts of speech. As the distribution does not apply to entire word classes, most notably nominal modifiers, the details of agreement prefixes marking several agreement classes with specific agreement targets are given in \tabref{Tab:AGRmorph}. 

\begin{table}
\fittable{\begin{tabular}{l ll llll}
 \lsptoprule
\textsc{agr} prefix & \multicolumn{2}{c}{Pronouns} & \multicolumn{4}{c}{Nominal modifiers}  \\
\cmidrule(lr){1-1}\cmidrule(lr){2-3}\cmidrule(lr){4-7}
 & {\OBJ} & {\POSS} & -{\itshape vúdũ̂} `1' & -{\itshape fúsì} `different' & -{\itshape ɛ́sɛ̀} `all' & -{\itshape ɔ́nɛ́gá} `other' \\
 \midrule
m- &  - & - & cl. 1, 3, 9 & cl. 1 & - & -   \\
w- & cl. 3 &  cl. 1, 3 & - & - & cl. 1, 3 & cl. 3  \\
\lspbottomrule
\end{tabular}}
\caption{Prefixes marking multiple agreement classes}
\label{Tab:AGRmorph}
\end{table}



\subsubsection{The object-linking H tone}
\label{sec:OBJTone}


Some morphemes in Gyeli are not segmental, but solely tonal. This is the case for the H tone that attaches to the left of common nouns in certain contexts.\footnote{Proper names do not take an object-linking H tone, which is tied to the fact that proper names do not usually take noun class prefixes. Therefore, the object-linking H tone does not have a toneless TBU to attach to.} In terms of its function, this H tone prefix marks a noun as the object. Where there is more than one object, only the object closest to the verb is marked by the object linker.  As such, it is an inflectional morpheme that is obligatorily in this environment. I call this prefix ``object-linking H tone'' and gloss it as `{\OBJ}.{\LINK}', as shown in \REF{OBJ40}.

\ea \label{OBJ40}
  \glll nkɛ̀ nyì nzí sílɛ̃́ɛ̃̀ {\bfseries bé}déwò. \\
          nkɛ̀ nyi nzí sílɛ̃́ɛ̃̀ {\bfseries H}-be-déwò \\
          $\emptyset$9.field 9 {\PROG}.{\PST} finish.{\COMPL} {\bfseries {\OBJ}.{\LINK}}-be8-food   \\
    \trans `The field was already running out of food.'
\z

The object-linking H tone only appears on otherwise toneless CV- shape noun class prefixes, but is not realized on null-form or consonantal noun class prefixes.\footnote{Object nouns with null-form and consonantal noun class prefixes are completely unchanged; no downstep phenomena could be observed.}
More examples of the object-linking H tone and information on its function in marking grammatical relations is provided in \sectref{sec:HLinker}.







\subsection{Suffixes}
\label{sec:Suffix}

Gyeli  suffixes can be categorized into seven different functions: nominalization, extension and expansion, negation, contrast, direct address, and a tense-mood marking. I will outline each of these types in the following, discussing extension and expansion suffixes together since their function is the same (but they differ in the level of productivity).
Derivational suffixes are outlined only briefly in this section before their fuller discussion in \sectref{sec:FormProcess}, while inflectional suffixes are outlined in greater length here.






\subsubsection{Nominalization suffixes}
\label{sec:NOMSuff}

Gyeli has four nominalization suffixes, as shown in \tabref{Tab:Nomsuff}. -{\itshape a}, the tonal morpheme -H, and -{\itshape ɛ̀dɛ}  derive full deverbal nouns.  -{\itshape a} and -{\itshape ɛ̀dɛ̀} are in complementary distribution, with -{\itshape ɛ̀dɛ̀} occurring in deverbal nouns of gender 1/2 and -{\itshape a} everywhere else. -{\itshape ɛ̀dɛ̀} is, however, occurs infrequently, as most deverbal nouns in gender 1/2 do not take nominalization suffixes. 

\begin{table}
\small
\begin{tabularx}{\textwidth}{l lll}
 \lsptoprule
Nominalization suffix & {\pOS} & Productivity & Co-occurrence \\
\midrule
-{\itshape a}           & full deverbal noun & medium &  with -H \\
-H                      & full deverbal noun & medium & with -{\itshape a}\\
-{\itshape ɛ̀dɛ̀}   & full deverbal noun& low & none \\
-{\itshape á}/-{\itshape â} & nominalized past participle & high & none \\
\lspbottomrule
\end{tabularx}
\caption{Nominalization suffixes}
\label{Tab:Nomsuff}
\end{table} 

The fourth suffix, -{\itshape á}/-{\itshape â}, is productively used in the derivation of nominalized past participles (\sectref{sec:NOMPart}), as in \REF{nomp}. The tonal pattern of an H tone or a falling tone is phonologically determined by the tone on the lexical root. An H tone lexical root takes a falling -{\itshape â} nominalization suffix, while an L tone verb stem takes an H suffix -{\itshape á}.  This makes it a distinct pattern from the -H and -{\itshape a} suffixes in full deverbal nouns, whose tonal patterns seem lexically specified and not conditioned by phonology. 

\ea  \label{nomp}
\ea  tsíbɔ `grind' >  n-tsíb-{\bfseries â} `ground (thing)'
\ex tálɛ `begin' >  n-tál-{\bfseries â} `begun (thing)'
\ex gyàga `buy' >  n-gyàg-{\bfseries á} `bought (thing)'
\ex jì `open' >  n-jìy-{\bfseries á} `opened (thing)'
\z
\z


In contrast to nominalized past participles, not all full deverbal nouns take a nominalization suffix. In fact, they frequently do not use any nominalization suffix at all. In these cases, the noun class marker that is used with the verb stem is the only nominal marker, as shown in \REF{NOMb}. The toneless TBUs of the verb stem are specified for L tones in the nominal forms. 

\ea  \label{NOMb}
\ea  bwàlɛ `be born' > ma-bwàlɛ̀ `birth'
\ex gyɛ̀'ɛlɛ `pray' >  ma-gyɛ̀'ɛ̀lɛ̀ `prayer'
\ex dɔ̀ `negotiate' >  ma-dɔ̀ `negotiation'
\z
\z

It is lexically specified which nouns take a nominalization suffix and also if the noun will take both the suffix -{\itshape a} and the tonal suffix -H, as shown in \REF{NOMa}, or just one of them. 

\ea  \label{NOMa}
\ea  tálɛ `begin' >  ma-tál-{\bfseries á} `beginning'
\ex dígɛ `look' >  ma-díg-{\bfseries á} `vision'
\ex dìlɛ `bury' >  ma-dìl-{\bfseries á} `funeral'
\ex líbɛlɛ `show' >  ma-líbɛ́l-{\bfseries á} `appearance, showing'
\ex tfúdɔ `pinch' >  tfúd-{\bfseries á} `pinch (n.)'
\ex tsìlɔ `write' >  n-tsìl-{\bfseries á} `hand writing'
\z
\z

While the suffixes -{\itshape a} and -H most often occur together, there are also cases where only -H or only -{\itshape a} is used, as shown in \REF{NOMx}.

\ea  \label{NOMx}
\ea tɛ̀mbɔwɔ `set (sun)' >  ma-tɛ̀mbɔ́wɔ́ `sunset'
\ex sɔ̀si  `be happy' > ma-sɔ̀sí `joy'
\ex dígɛ `look, watch' >  ma-dígà `vision'
\z
\z

There are some constraints on the occurrence of -{\itshape a}. In terms of phonological constraints, the suffix only attaches to stems that end in /ɛ/ or /ɔ/, as shown in \REF{NOMa}.\footnote{Nouns derived from verbs with other final vowels such as /i/, /o/, and /a/ never undergo vowel change.} As for its distribution over various genders, -{\itshape a} is only used in non-agentive deverbal nouns, excluding nominalizations in gender 1/2 (\sectref{sec:NOM12}). 


Nominalized forms in gender 1/2 rarely take any nominalization suffixes. The only nominalization suffix in this gender is -{\itshape ɛ̀dɛ̀}, which is in complementary distribution with -{\itshape a}. -{\itshape ɛ̀dɛ̀} is not found in other genders and/or with non-agentive nouns. 
There are only two known occurrences of the -{\itshape ɛ̀dɛ̀} suffix in the lexical database, which are listed in \REF{NOMede}.

\ea  \label{NOMede}
\ea  gyámbɔ `cook' >  n-gyámb-{\bfseries ɛ̀dɛ̀} `cook (n.)'
\ex gyímbɔ `dance' >  n-gyímb-{\bfseries ɛ̀dɛ̀} `dancer'
\z
\z

The -{\itshape ɛ̀dɛ̀} suffix might be a more marked form for agentive nouns in order to disambiguate between other nominalized forms. An example is the verb {\itshape gyímbɔ} `dance'  and the derived noun {\itshape n-gyímbɔ̀} `sorcerer'. In this instance, -{\itshape ɛ̀dɛ̀} might be used to distinguish   {\itshape n-gyímbɔ̀} `sorcerer' from {\itshape n-gyímb-ɛ̀dɛ̀} `dancer'.
\sectref{sec:NOM} gives a more detailed account of nominalization processes.









\subsubsection{Extension and expansion suffixes}
\label{sec:EXtp}

Extension and expansion suffixes are derivational suffixes which derive verbs from other verbs, changing their valency. The difference between extension and expansion lies in the suffix's relative level of productivity. Extension morphemes are synchronically productive, while expansion morphemes are not. Gyeli has six extension and three expansion morphemes, as listed in \tabref{Tab:AffixDeriv}. Each of them is discussed in detail in \sectref{sec:VDeriv}.

\subsubsection{Negation suffix -{\itshape lɛ}}
\label{sec:NEGSuff}

There are two suffixes -{\itshape lɛ}, which are homophonous but have two distinct functions. One is a derivational expansion suffix (\sectref{sec:DiaEx}) and the other is an inflectional negation suffix. Both suffixes can co-occur, as shown in \REF{EXPNEG}. In these instances, the negation suffix follows the expansion suffix, as expansion suffixes generally follow the verb root, while the negation suffix attaches to the verb stem.  

\ea \label{EXPNEG}
  \gll  tɛ́-lɛ-lɛ \\
      place-\textsc{exp}-{\NEG} \\
\trans `to not place something'
\z

As a negation suffix, -{\itshape lɛ} productively attaches to all verb stems in the present tense, as exemplified in \REF{negle}. Tonal changes depend on the verb's stem tones and are discussed in detail in \sectref{sec:NEGPRES}.

\ea  \label{negle}
\ea  gyámbɔ `cook' $\rightarrow$ gyámbɔ́-{\bfseries lɛ́} `not cook'
\ex kòla `add' $\rightarrow$ kólà-{\bfseries lɛ̀} `not add'
\z
\z

\noindent In other tenses, auxiliary negation verbs (\sectref{sec:AUX}) are used that contain the suffix -{\itshape lɛ}. 







\subsubsection{Contrastive marker -{\itshape gà}}
\label{sec:CONTRS}

The morpheme -{\itshape gà} is an inflectional suffix that attaches to subject pronouns, as shown in \REF{122t}, and to object pronouns, as in \REF{122s}. 

\ea \label{122t}
  \glll wɛ́ kɛ́ nà nyɛ̂ nkɔ̃̀wáká {\bfseries nyɛ̀gà} à nzíí wɛ̂ vã́ã̀kɛ́ sâ mpù \\
         wɛ-H kɛ̀-H nà nyɛ̂ nkɔ̃̀wáká nyɛ̀-gà a nzíí wɛ̂ vã́ã̀kɛ́ sâ mpù \\
         2\textsc{sg}-\textsc{prs} go-{\R} {\COM} 1 equal.sharing 1.{\SBJ}-{\CONTR} 1 {\PROG}.\textsc{prs} 2\textsc{sg}.{\OBJ} go[Bulu] do like.this  \\
    \trans `You go with him equally sharing, he tries to trick you [lit. he is going to do you like this].'
\z

\ea \label{122s}
\glll mɛ́ nyɛ́ {\bfseries wɛ̀gà} \\
mɛ-H nyɛ̂-H wɛ̂-gà \\
1\textsc{sg}-\textsc{prs} see-{\R} 2.\textsc{sg}-{\CONTR} \\
\trans  `I see YOU/you, too. 
\z

-{\itshape gà} serves to track referents and, in terms of information structure, indicates a switch of topics, as explained in \sectref{sec:insituTop}.
The suffix appears to be derived from the nominal modifier -{\itshape ɔ́(nɛ́)gá} `other', as discussed in \sectref{sec:other}. 

\subsubsection{Vocative marker -{\itshape o}}
\label{sec:VOCSuff}

All proper names can take the vocative suffix -{\itshape o}, for instance as in {\itshape Mìnsêm-o} or {\itshape Màmá-o}. The suffix attaches to the noun without undergoing assimilation; thus a final vowel of the noun stem does not delete. The tone of the suffix depends on speaker proximity. If the addressee is close to the speaker, the suffix has an L tone, if the addressee is further away, it has an H tone. 
The vocative suffix is not exclusively restricted to proper names, but can also be used with common nouns. These occurrences are, however, limited to common nouns expressing a relation that can be used as address, such as {\itshape nyá-ò} `mother' and {\itshape tá-ò} `father'.  The vocative can also attach to the locative adverb {\itshape wɛ̂} `there', as shown in \REF{VOC}, where it also combines with the distal H tone.

\ea \label{VOC}
  \glll mùdì kí tàtɔ̀ wú{\bfseries ó} \\
       m-ùdì kí tàtɔ wû-o-H \\
        \textsc{n}1-person {\NEG} scream there-{\VOC}-{\DIST} \\
    \trans `Nobody scream over there!'
\z

\subsubsection{Tense-mood H tone suffix}
\label{sec:TMHSuff}


An H tone suffix attaches to the subject-tense-aspect-mood-polarity (\textsc{stamp}) mar\-ker and verbs in certain tense-mood categories. The \textsc{stamp} marker takes the H tone suffix to mark \textsc{present} and \textsc{subjunctive}, while verbs take the H tone suffix to encode \textsc{recent past} and \textsc{remote past}.
These processes are described in detail in \sectref{sec:GramTM}.






\section{Derivation and compounding}
\label{sec:FormProcess}


Having discussed the different morpheme types and their distribution, I now turn to describing the language's word formation processes. This includes nominalization, verbal derivation, and compounding.


\largerpage[-2]
\subsection{Nominalization}
\label{sec:NOM}


Nominalization is a word formation process in which nouns are formed from lexemes of other word classes. In Gyeli, the source word class for nominalization is generally restricted to verbs, at least for the derivation processes that are synchronically transparent.\footnote{In nominalizations with the similative marker {\itshape ná}-, the derivation process is rather opaque so that the derivational source of most derived forms is synchronically not recognizable, as discussed in \sectref{sec:NOMSIM}.}

Formally, there are several means to derive  a derived noun:
\begin{enumerate}[label=(\roman*)]
\itshapeem prefixation of a noun class prefix 
\itshapeem prefixation of the similative marker {\itshape ná}-
\itshapeem suffixation of -{\itshape a}/-{\itshape ɛ̀dɛ̀} and/or -H
\itshapeem suffixation of -{\itshape á}/-{\itshape â}
\end{enumerate}

\noindent Based on how these means are systematically used and combined, three different types of nominalized forms can be created.  First, there are those which are full nouns, assigned to a gender. Their prefixation pattern is based on assigned gender. A subset of these also take nominalization suffixes, namely -{\itshape ɛ̀dɛ̀} in gender 1/2 and -{\itshape a} in all other genders. In all genders except 1/2, the nominalization suffix can be a tonal morpheme -H, which can attach to the stem directly or occur in combination with the suffix -{\itshape a}. Second, there are defective nouns, which are nominalized participles.  These always manifest prefixation of a nasal prefix N- and  suffixation of -{\itshape á}/-{\itshape â}.   Third, there are  derived forms with {\itshape ná}-, producing nouns and adjectives. These always manifest prefixation of {\itshape ná}-, but never segmental nor tonal suffixation. 


What all three nominalization types have in common is that they take some sort of prefix. Full deverbal nouns are assigned to different genders, including genders 1/2, 3/4, 5/6, 6, 7/8, and 8. Depending on the gender they are assigned to, affixation of a noun class prefix is predictable. For instance, full deverbal nouns in gender 1/2 will always take a nasal noun class prefix in the singular and the noun class prefix {\itshape ba}- in the plural. Nominalized past participles always take a homorganic nasal prefix while nouns derived with the similative always take the {\itshape ná}- prefix.

In contrast to prefixation, suffixation is more diverse across the different types of nominalization.  Nouns derived with {\itshape ná}- never take a suffix, while  nominalized past participles predictably take the suffix -{\itshape á}/-{\itshape â}, depending on the tonal pattern of the verb stem, which determines the tonal pattern of the suffix. L tone stems trigger the -{\itshape á}  suffix, while H  tone stems result in the  -{\itshape â} suffix.
Full deverbal nouns only sometimes take a suffix, which can be either a segmental or a tonal suffix or a combination of both. As explained in \sectref{sec:NOMSuff}, the suffixes -{\itshape ɛ̀dɛ̀} and -{\itshape a} occur in deverbal nouns of different genders. Their attachment seems lexically specified.
The tonal suffix -H occurs in full deverbal nouns of all genders except for gender 1/2.  The -H suffix spreads across all toneless TBUs of a verb, namely all syllables after the first one (see \sectref{sec:toneless}). In deverbal nominalization, all the tones become lexicalized, i.e. there are no toneless TBUs in noun stems. The verbal toneless units lexicalize either as an L, as in \REF{NomTBU1} or an H, as in \REF{NomTBU2}.

\ea \label{NomTBU}
\ea \label{NomTBU1} ma-bwàl{\bfseries ɛ̀ }`birth' < bwàlɛ `give birth'
\ex\label{NomTBU2} ma-sɔ̀s{\bfseries í }`happiness' < sɔ̀si `be happy'
\z
\z


Given the variability in segmental and tonal suffixation with full deverbal nouns, I will present each affix according to the category it derives. I first present full deverbal nouns that are assigned to gender 1/2, 3/4, 5/6, 6, 7/8, or the transnumeral gender 8.    
(For more information on genders, see \sectref{sec:genders}.) Gender assignment seems largely meaning driven. For instance, deverbal agentive nouns are assigned to gender 1/2 while event nouns are found in the transnumeral gender 6. Generally, deverbal nouns are found in all major genders except for gender 9/6. I then discuss nominalized participles as a type of defective noun.
Forms derived with the prefix {\itshape ná}- include both nouns and adjectives, and are discussed separately in section  \sectref{sec:NOMSIM}. 
I also treat this type of nominalization separately because (i) nouns with {\itshape ná}- only use limited nominalization means, excluding suffixation and tone change,  and (ii) their derivational source is significantly more opaque than that of other derived nouns.









\subsubsection{Deverbal agentive nouns in gender 1/2}
\label{sec:NOM12}

Deverbal nouns in gender 1/2 semantically designate a human or other animate entity as an agent.
These agentive nouns typically describe the `doer' of an action. As animate entities, they are countable in Gyeli and thus always come with a plural form of the {\itshape ba} noun class, as described in \sectref{sec:NC}.\footnote{Nouns for humans are also found in other genders in Gyeli, but gender 1/2 is the human class in Proto-Bantu and many other Bantu languages synchronically. Also, in Gyeli most humans are assigned to gender 1/2.}

All deverbal nouns in gender 1/2 take a nasal prefix in the singular and the prefix {\itshape ba}- in the plural. The systematic attachment of a nasal prefix in the singular is remarkable since most nouns of agreement class 1 do not take any prefix at all (see \sectref{sec:NC}).
The type of nasal prefix in class 1 depends on the phonological properties of the noun's stem-initial consonant (\sectref{sec:NPlaceAss}). If the stem starts with a bilabial consonant, the nasal will be a labial nasal /m/ as in \REF{prem}.


\ea \label{prem} m- prefix
\ea  m-bɛ́dɔ̀ `climber' < bédɔ `climb'
\ex m-bwàlɛ̀  `parent' <  bwàlɛ `be born'
\z
\z

\noindent On the other hand, if the consonant is an alveolar consonant, it will be an alveolar nasal /n/ as in \REF{pren}.

\ea \label{pren} /n/- prefix
\ea  n-sálɛ̀ `maker' < sálɛ `make (v.t.)'
\ex n-dìlɛ̀ `undertaker' <  dìlɛ `bury'
\ex  n-jì `opener' < jì `open'
%\ex add other disyllabic               
\z
\z

\noindent Finally, if the consonant is a velar, as in \REF{preN}, the nasal will be a velar nasal /ŋ/.\footnote{In general orthography, however, I do not distinguish alveolar and velar nasals, as explained in \chapref{sec:Phon}.}

\ea \label{preN} /ŋ/- prefix
\ea  n-gyàgà `buyer' <  gyàga `buy'
\ex  n-kòlɛ̀ `helper' <  kòlɛ `help'
\ex n-kwã́ã̀lɛ̀ `spy (n.)' <  kwã́ã̀lɛ `spy (v.)'
\z
\z



Most deverbal nouns in gender 1/2 do not take any nominalization suffix, but retain the original verb stem, as shown in \REF{BiN} with the examples displaying different final vowels of /a/, /ɛ/, and /ɔ/.


\ea \label{BiN}
\ea  n-gyàgà `buyer' < gyàga `buy'
\ex n-kòlɛ̀ `helper'   < kòlɛ `help'
\ex n-tsìlɔ̀ `writer' < tsìlɔ `write' % check if n-tsílɔ̀   
\ex n-jíbɔ̀ `sb. who closes' < jìbɔ `close'
\ex n-gyìmbɔ̀ `sorcerer'   < gyìmbɔ `dance'      
\z
\z

All known deverbal nouns in gender 1/2 that do not take a nominalization suffix are disyllabic. In the examples in \REF{BiN}, this is obvious since the verb stem is disyllabic as well. There are, however, also cases where a disyllabic version of a monosyllabic verb is, at least synchronically, not used in the language, as in \REF{Napp}. The derived noun is still disyllabic, receiving the non-productive extension -{\itshape lɛ} which is discussed in \sectref{sec:DiaEx}. Trisyllabic derived nouns without an extension suffix are not known.

\ea \label{Napp}
\ea   n-dèlɛ̀ `eater' < ?dèlɛ `eat (?)' < dè `eat'
\ex n-kɛ̀lɛ̀ `walker' < ?kɛ̀lɛ̀ `walk (?)' <    kɛ̀ `walk'
\z
\z

\noindent Another opaque exception to the general retention of the verb stem is \REF{Napp2}. Not only is the derivation process not clear, also the final vowel of the noun changes to /i/. There are no other nouns that follow this pattern.

\ea \label{Napp2}
 n-jíbí `thief'  < ? <  jíwɔ `steal'  
\z


When suffixation of deverbal nouns in gender 1/2 occurs, it is always with -{\itshape ɛ̀dɛ̀}, but never -{\itshape a} or -H). Examples of this are given in \REF{ede}.

\ea \label{ede}
\ea  n-gyámbɛ̀dɛ̀ `cook (n.)' < gyámbɔ `cook'
\ex n-gyìmbɛ̀dɛ̀ `dancer'     < gyìmbɔ `dance'       
\z
\z




\subsubsection{Deverbal nouns in gender 3/4}
\label{sec:NOM34}

Deverbal nouns in gender 3/4 are less frequent than those in gender 1/2 or 6. They are, however, formally very similar to nominalized past participles, discussed in \sectref{sec:NOMPart}. All of them take a nasal prefix (in class 3), they all take the nominalization suffix -{\itshape a}, and disyllabic nouns also take the tonal nominalization suffix -H on the final vowel, as shown in \REF{34dNa}. This pattern is distinct from nominalized past participles, since lexical stems with an H tone take the -H suffix. Nominalized past participles surface with a final HL if the stem is H. 

\ea \label{34dNa}
\ea  n-tsìl-á `hand writing' < tsìlɔ `write'
\ex n-sàl-á `crevice' < sàlɔ `cut lengthwise'
\ex  n-lvúm-á `fork' <  lvúmɔ `sting'
\z
\z

In contrast to nominalized past participles, deverbal nouns in gender 3/4 are full nouns including a plural form with the noun class prefix {\itshape mi}- in addition to the singular form with a nasal prefix. They occur in all nominal environments, as described in \sectref{sec:N}, while nominalized participles do not.

Unlike deverbal agentive nouns of gender 1/2, deverbal nouns in gender 3/4 are not restricted to a disyllabic pattern. As \REF{34dNb} shows, there are also instances of mono- and trisyllabic derived nouns. In these cases, the change to an H tone on the final vowel does not apply.
 
\ea \label{34dNb}
\ea  n-lâ `story' < lâ `tell'
\ex n-sá'àwà `repeated movement (e.g.\ leaves)' < sá'àwa `move repeatedly, fidget'
\z
\z

\subsubsection{Deverbal nouns in gender 5/6}
\label{sec:NOM56}

Deverbal nouns in gender 5/6 seem to be rare, just like those in gender 3/4. They all take the gender's noun class prefixes, {\itshape le}- in the singular class 5 and {\itshape ma}- in the plural class 6. There are no known instances of segmental nominalization suffix attachment and nouns generally retain the final vowel of the verb, as shown in \REF{56dNa}.

\ea \label{56dNa}
\ea  le-jìlɔ̀ `weight' <  jìlɔ `be heavy'
\ex le-dã̀ `pond, source, well'  <  dã̀ `draw water'
\z
\z

\noindent An exception to the final vowel is presented in \REF{56exc} where the derivation path is opaque. The final vowel of the synchronically existing verb and the derived noun do not match.

\ea \label{56exc} le-sù'ù `waterfall' < ?sù'ù `pour (?)'  < sùbɛ `pour out'
\z

Deverbal nouns in gender 5/6 are either bi- or trisyllabic with the noun class prefix and a mono- or disyllabic verb stem. There are instances where the verb stem is trisyllabic, as in \REF{56dNb}, but in the derived noun, the first and second verb syllables are merged.

\ea \label{56dNb} le-fwálá `end, border, summit' <  fúala `end ({\RECIP})'
\z

\noindent The example in \REF{56dNb} presents the only known instance of a tonal nominalization suffix -H in this gender; all other examples surface with a final L tone.

\largerpage
\subsubsection{Deverbal event nouns in gender 6}
\label{sec:NOM6}

A vast number of deverbal nouns are assigned to the transnumeral gender 6. Semantically, deverbal nouns in this gender represent an event, as  examples in \REF{eventN1} through \REF{eventN4} show (with the exception of  {\itshape ma-nyâ} `(breast) milk' which is in this gender for its status as a liquid mass noun).
Formally, all deverbal nouns in this gender take the noun class prefix {\itshape ma}- and are uncountable, lacking a singular counterpart in class 5.  They differ, however, with respect to suffixation of the segmental nominalization suffix -{\itshape a} and the tonal suffix -H. Since these nominalization suffixes occur independently of each other, there are four different classes of deverbal event nouns in class 6. \REF{eventN1} shows those that do not take any nominalization suffix, surfacing with a lexicalized final L tone in the noun stem.

\ea \label{eventN1}
\ea  ma-sâ `game, playing'  < sâ `make, do'
\ex ma-bwã̂sà `thoughts'< bwã̂sa `think'
\ex ma-nyànò `pain' < nyàno `hurt'
\ex ma-nyâ `(breast)milk' < nyâ `suckle, lick'
\ex ma-gyɛ̀'ɛ̀lɛ̀ `prayer' < gyɛ̀'ɛlɛ `pray'
\ex ma-dɔ̀ `negotiation, discussion' < dɔ̀ `negotiate, discuss'
\ex ma-bwàlɛ̀ `birth' < bwàlɛ `be born' < bwà `give birth'
\z
\z

\noindent Example \REF{eventN2} shows affixation with the segmental nominalization suffix -{\itshape a}, but without -H.

\ea \label{eventN2} ma-díg-à `vision' < dígɛ `look, watch'
\z

Vice versa, there are nouns that take only the tonal nominalization suffix, as in \REF{eventN4}.

\ea \label{eventN4}
\ea  ma-pámó `appearance, rise' < pámo `appear'
\ex ma-tɛ̀mbɔ́wɔ́ `sun set' < tɛ̀mbɔwɔ `set (sun)'
\ex ma-sɔ̀sí `joy'  < sɔ̀si `be happy'
\z
\z

The fourth class comprises those nouns that take both the segmental nominalization suffix -{\itshape a} and tonal nominalization suffix -H, as in \REF{eventN3}. \REF{eventN3c} further illustrates that it is possible to derive four-syllable nouns (including the noun class marker).

\ea \label{eventN3}
\ea  ma-tál-á `beginning' < tálɛ `begin'
\ex ma-dìl-á `funeral' < dìlɛ `bury' 
\ex\label{eventN3c} ma-líbɛ́l-á `showing, appearance (of moon') < líbɛlɛ `show'
\z
\z


\largerpage

\subsubsection{Deverbal nouns in gender 7/8}
\label{NOM78}



Gender 7/8 also hosts deverbal nouns. They take the noun class markers of their classes, namely $\emptyset$ for class 7 and {\itshape be}- in class 8. All examples presented here have a plural form, even abstract nouns such as {\itshape tfúgà, be-tfúgà} `suffering, sufferings' or {\itshape kwàlɛ́} `love', {\itshape be-kwàlɛ́} `love (for different things/people)'.

Within deverbal nouns of gender 7/8, there are several formal subclasses, determined by the presence or absence of a segmental and/or tonal nominalization suffix. Examples in \REF{78N1} neither take the segmental suffix -{\itshape a} nor the tonal suffix -H, but are formally identical to the verb they are derived from.

\ea \label{78N1}
\ea  sálɛ̀ `work (n.)' < sálɛ `make, do (v.t.)'
\ex tfúgà `suffering' < tfúga `suffer'
\z
\z

\noindent In contrast, \REF{78N2} exhibits cases where the  tonal suffix -H is used without the segmental suffix -{\itshape a}.

\ea \label{78N2}
\ea  sá `thing' < sâ `make, do'
\ex kwàlɛ́ `love (n.)' < kwàlɛ̀ `love (v.)'
\z
\z

\noindent In \REF{78N3}, both -H  and the segmental nominalization suffix -{\itshape a} attach.

\ea \label{78N3} tfúd-á `pinch (n.)' < tfúdɔ `pinch (v.)'
\z

Finally, there are cases where the derivation process is synchronically not clear. In \REF{78N4a}, the source of {\itshape ndɛ̀} that is attached to {\itshape kɛ̀} `walk' is unknown. In \REF{78N4b}, it seems that there might have been another verb form from which the noun has been derived, but which does not exist synchronically anymore.

\ea \label{78N4}
\ea \label{78N4a} kɛ̀ndɛ̀ `walk (n.)' <  kɛ̀ `walk (v.)'
\ex\label{78N4b} lɔ̀gɔ̀ `curse (n.)' < ? <  lùà `curse (v.)' 
\z
\z


\subsubsection{Deverbal nouns in gender 8}
\label{NOM8}

There are also nominalized forms in inquorate genders (\sectref{sec:MinGen}) such as gender 8, which lack a singular counterpart, as in \REF{eventmass}. This is based on the status of gender 8 nouns as mass nouns.

\ea \label{eventmass}
\ea be-déwɔ̀ `food' < dè `eat'
 \ex be-jíì `anger' < jíga `be angry'
 \ex be-kílì `slyness, guile' < kílɔwɔ `be vigilant'
\z
\z









\subsubsection{Nominalized past participles}
\label{sec:NOMPart}

The nominalized past participle is the most productive type of derivation, more productive than  full deverbal nouns or derived verbs which are discussed in \sectref{sec:VDeriv}. In the database of 377 verbs, 325 (86\%) allow for a nominalized participle.\footnote{Frequencies of derived verbs such as reciprocal, passive, or causative are provided in \tabref{Tab:SumVext} in \sectref{sec:VDeriv}.} It seems that the only restriction includes verbs of saying or intransitive verbs such as {\itshape dyúà} `swim' or {\itshape sìsɔ} `be happy'. Grammatical properties of nominalized past participles, such as their syntactic restriction to the predicate position in copula constructions, as well as their status as nouns in terms of parts of speech are discussed in \sectref{sec:NounPart}. Semantically, they encode resultativity, as shown in \REF{NomPart1}.

\ea \label{NomPart1}
\ea  n-kòl-á `helped person/thing' < kòla `help (v.)'
\ex n-dvùb-á `soaked person/thing' < dvùba `soak'
\ex n-gyámb-â `cooked person/thing' < gyámbɔ `cook (v.)'
\ex  n-tfúmb-â `wrinkled thing' < tfúmba `wrinkle (v.)'
\z
\z



The derivation of nominalized participles involves prefixation of a nasal and suffixation of -{\itshape a}. Unlike the nominalization suffix with full deverbal nouns, the tonal pattern of the nominalized past participle suffix is determined by the tone pattern of the lexical stem, as shown in \REF{NomPart2} for disyllabic verbs. If the stem tone is L, the  suffix will take an H tone. If the stem tone is H, the suffix will take an HL tone.

\ea \label{NomPart2}
\ea  n-dvùb-{\bfseries á} `soaked person/thing' < dvùbɔ `soak'
\ex m-bɔ̀g-{\bfseries á} `enlarged person/thing' < bɔ̀gɛ `enlarge'
\ex n-jímb-{\bfseries â} `lost person/thing' < jímbɛ `lose'
\ex n-sɛ́l-{\bfseries â} `peeled thing' < sɛ́lɔ `peel'
\z
\z

In fact, two syllables is the minimum requirement of length for nominalized past participles. In this, it differs from full deverbal nouns such as {\itshape n-jì} `eater' which is derived from {\itshape dè} `eat'. The nominalized participle form, however, is {\itshape n-jìy-á} `(be) open', as shown in \REF{NomPart5}.  Monosyllabic verb stems keep their final vowel in the first syllable and attach the suffix -{\itshape a}  as the second syllable, inserting an epenthetic consonant between the two vowels. The potential epenthetic segments mainly include {\itshape y}, {\itshape w}, and {\itshape ng}, which each occur in about a third of the monosyllabic verbs; there are a few exceptional cases which take {\itshape l}, {\itshape s}, or {\itshape n}. Only the insertion of {\itshape ng} as epenthetic consonant is mostly predictable.\footnote{There are a few exceptions, e.g.\ {\itshape  má'à} `accuse' is not derived with {\itshape ng}, but with {\itshape g} in {\itshape mágâ} `(be) accused', despite the nasal. The glottal stop seems to have more weight than the nasal, but other exceptions exist as well that do not appear to have an obvious explanation, for instance {\itshape nyàg-á} `(be) defecated' as derived from {\itshape nyàà} `defecate'.} It occurs in verbs that start with a nasal consonant and/or that have a nasalized vowel, as shown in \REF{NomPart3}.

\ea \label{NomPart3}
\ea  ndà{\bfseries ng}-á `crossed thing' < ndà `cross'
\ex n-là{\bfseries ng}-á `passed person/thing' < lã̀ `pass'
\ex n-lá{\bfseries ng}-â `read thing' < lã̂ `read'
\ex nyí{\bfseries ng}-â `entered thing' < nyî `enter'
\z
\z

The insertion of {\itshape g} is predictable if the monosyllabic verb contains a glottal stop. There are, however, many instances of {\itshape g} insertion which are not predictable, for instance in {\itshape n-tsìg-á} `(be) alive', derived from {\itshape tsìɛ̀} `live', as opposed to {\itshape n-tsíy-â} `(be) cut' which is derived from {\itshape tsíɛ̀} `cut'.

\ea \label{NomPart4}
\ea  n-kwà{\bfseries g}-á `ground thing' < kwà `grind'
\ex n-dvù{\bfseries g}-á `hurt person/thing' < dvùɔ̀ `hurt'
\ex n-ká{\bfseries g}-â `rolled up thing' < ká'à `roll up'
\ex m-pá{\bfseries g}-â `dug out person/thing' < pá'à `dig out'
\z
\z

\noindent Further examples of {\itshape y} insertion are given in \REF{NomPart5}.

\ea \label{NomPart5}
\ea  m-wɛ̀{\bfseries y}-á `dead person/thing' < wɛ̀ `die'
\ex n-jì{\bfseries y}-á `opened person/thing' < jì `open'
\ex n-kwé{\bfseries y}-â `fallen person/thing' < kwê `fall'
\ex m-vɛ́{\bfseries y}-â `given person/thing' < vɛ̂ `give'
\z
\z


Finally, nominalized past participles can also have three syllables. In this case,  the tonal pattern is exceptional in that the suffix does not change according to the lexical stem tone, but is the same for all derived forms: the second TBU surfaces as H and the third TBU surfaces as HL, as shown in \REF{NomPart6}.

\ea \label{NomPart6}
\ea  m-bèl{\bfseries án-â} `used person/thing' < bèlanɛ `use'
\ex n-lèb{\bfseries ál-â} `followed person/thing' < lèbɛlɛ `follow'
\ex n-súm{\bfseries ál-â} `greeted person' < súmɛlɛ `greet'
\ex m-víy{\bfseries ál-â} `touched person/thing' < víyala `touch'
\z
\z




\subsection{Derivation with similative {\itshape ná}-}
\label{sec:NOMSIM}

The similative prefix {\itshape ná}- derives common and proper names as well as adjectives. In this, it differs from other nominalization markers discussed in \sectref{sec:NOM} which only derive common nouns. Formally, derivation with {\itshape ná}- functions the same way for adjectives, common, and proper names.\footnote{While in most cases the derivational source is synchronically opaque, it still does not look as if there is any final vowel change to -{\itshape á} or tone change of the final vowel, as often found in deverbal nominalization.} In all cases, the only derivation marker is the prefix {\itshape ná}-. 

Derivation with the similative marker {\itshape ná}- is more diverse in its derivational source than nominalization processes discussed in \sectref{sec:NOM}. In most cases, the der\-i\-va\-tional source is, in fact, synchronically opaque.\footnote{See \sectref{sec:SIM} for why {\itshape ná}- should still be viewed as a derivational morpheme.} There are some clear cases, however, where the derivational source is a noun, as for instance in the proper name {\itshape Ná-nzɛ̌} which is derived from  {\itshape nzɛ̌} `leopard'.
There are also derived forms which likely arose from diachronic stative verbs, which are,  however, not used presently anymore, as with the adjectives in \REF{naSIM1}. Especially the cross-linguistically uncommon ``lightened'' and ``darkened'' color categories suggest a change of state and make a verbal source likely.

\ea \label{naSIM1} {\itshape ná}- with adjectives
\ea  ná-vyû(vyû) `black [lit.\ like blackened]'
\ex ná-bè(bè) `red [lit.\ like reddened]' 
\ex ná-mbàmbàlà `white [lit.\ like whitened]' 
\ex ná-yɛ̂(yɛ̂) `lightened color [lit.\ like bleached out]'
\ex ná-pfû(pfû) `darkened color [lit.\ like darkened]'
\z
\z

\noindent Further evidence for a verbal derivation source comes from \citet[382]{cheucle2014}, who analyzes the Proto-A80 particle °{\itshape na}- as a deverbal morpheme.\footnote{According to her data, °{\itshape na}- is synchronically a lot more productive in Bekwel (A85). Also, colors in Bekwel are preceded by this morpheme. \citet[138]{cheucle2014} views Bekwel color terms as nouns while the potential verbal source seems unclear.}

Nouns derived with {\itshape ná}- include both common and proper names. As for {\itshape ná}- derived common nouns, they all belong to gender 1/2 and their similative prefix can be preceded by the plural noun class prefix {\itshape ba}-, as shown in \REF{naSIM2}. As a CV- shape noun class prefix, {\itshape ba}- also then allows for the attachment of the object-linking H tone, as discussed in \sectref{sec:OBJTone}.  In contrast, singular noun forms with the similative marker never take a noun class prefix or  object-linking H tone. This is as expected, since the first syllable is already specified for tone.  Semantically, common nouns derived with {\itshape ná}- consist mostly of animals, especially insects.

\ea \label{naSIM2} {\itshape ná}- with common nouns
%\ex ná-bànkúdí, ba-nábànkúdí `female Agama lizard'
%\ex ná-nkàálɛ́, ba-ná-nkyàálɛ́ `termite mound'
\ea ná-búnjã̂, ba-ná-búnjã̂ `bed bug'
%\ex ná-yûyû, ba-ná-yûyû `vertigo'
%\ex ná-kúlúú, ba-ná-kúlúú `forest tortoise ({\itshape Kinixys homeana})'
\ex ná-mìnsógɛ̀, ba-ná-mìnsógɛ̀ `palm rat'
\ex ná-mángɔ̀(mángɔ̀), ba-ná-mángɔ̀(mángɔ̀) `male Agama lizard'
\ex ná-yûyû, ba-ná-yûyû `vertigo'
\z
\z

\noindent With proper names, {\itshape ná}- only occurs in female names, deriving them from male names, as illustrated in \REF{naSIM3}.

\ea \label{naSIM3}  {\itshape ná}- with proper names
\ea  Ná-ngyɛ́mbá (female name) > Ngyɛ́mbá (male name)
\ex Ná-ntùngù (female name)  > Ntùngù (male name)
\ex Ná-yímá (female name) > Yímá (male name)
\ex Ná-bàmù (female name) > Bàmù (male name)
\z
\z

In terms of frequency, the prefix {\itshape ná}- is found with eight common nouns in the 875-entry noun database, which is less than 1\%. The similative marker is relatively more widespread among proper names, with 16 occurrences{\textemdash}one third of a sample of about 50 female proper names. The similative marker occurs with half of the 12 adjectives. These include all five color terms as well as {\itshape ná-tĩ̂} `straight'. 





\subsection{Adverbal derivation with {\itshape nà}-}
\label{sec:ADVderiv}

The prefix {\itshape nà}-  derives adverbs, as shown in \REF{naSIM4}. These are the only two examples contained in the lexical database, but the class of adverbs is small in the first place. 

\ea \label{naSIM4}  {\itshape nà}- with adverbs
\ea  nà-mɛ́nɔ́ `tomorrow' > mɛ́nɔ́ `morning'
\ex nà-kùgúù `yesterday' > kùgúù `evening'
\z
\z

The adverbializing derivation prefix differs from derivations with the similative prefix {\itshape ná}- phonologically in its tonal pattern and functionally in that {\itshape nà}- derives adverbs. The derivational source in adverbal derivation is always a noun.



 













\subsection {Verbal derivation}
\label{sec:VDeriv}

%[???] auxiliaries cannot be derived or exceptions?

Bantu languages are known for their multitude of productive verb extensions, also known under the term  ``verbal derivation''. These suffixes  bring about a valence change from intransitive to transitive verbs and may generally include such categories as applicatives, causatives, reversives, or reciprocals.

\tabref{Tab:SumVext} summarizes verbal derivation morphemes in Gyeli, including both extensions and expansions, while \tabref{Tab:SumVextex} gives examples for each one. \citet{nurse08} defines {\itshape extensions} as verbal ``productive derivational suffixes'' that ``change the valency and meaning of [verb] roots'' (p.\ 311). In Gyeli, they comprise the forms -{\itshape ala}, -{\itshape a}, -{\itshape ɛsɛ}, -{\itshape ɛlɛ}, -{\itshape ɛga}, and -{\itshape ɔwɔ}. In contrast, {\itshape expansion} suffixes -{\itshape kɛ}, -{\itshape lɛ},\footnote{It is not clear whether this suffix is related to the applicative. As shown in \sectref{sec:DiaEx}, there are instances of valency increase, as expected for the applicative, but also cases where the opposite happens. Also, there does not seem to be a phonological rule according to which the expansion suffix could have been reduced from the applicative form. Given the inconclusive data on a potential relation between -{\itshape lɛ} and the applicative suffix -{\itshape ɛlɛ}, I consider -{\itshape lɛ} as a form in its own.} and -{\itshape bɔ} are not productive synchronically. They are low in number and it is difficult to match their form onto a specific function (other than being related to valency).


\begin{table}
\begin{tabular}{lllr}
\lsptoprule
Status & Form & Category label & \# verbs \\  \midrule
\multirow{6}{*}{extensions} & -ala & {\scshape reciprocal} & 270  \\
& -a & {\scshape passive} & 105 \\
& -ɛsɛ & {\scshape causative} & 89 \\
& -ɛlɛ & {\scshape applicative} & 34 \\
& -ɛga & {\scshape autocausative middle voice} & 28  \\
& -ɔwɔ & {\scshape positional middle voice} & 5 \\  \midrule
  & -kɛ & {\scshape ???} &  10 \\
{expansions} &  -lɛ & {\scshape ???} & 6 \\
 & -bɔ & {\scshape ???} & 1  \\
 \lspbottomrule
\end{tabular}
\caption{Verbal derivation morphemes}
\label{Tab:SumVext}
\end{table} 


While historically the derivational system was most likely more productive, it is synchronically determined in the lexicon whether a verb takes verb extensions and, if so, which.  There is no verb that takes all possible extensions. Also, there seems to be a general tendency to reduce verb extensions. For instance, the applicative and causative are currently merging into one transitivizing category, blurring semantic distinctions. 

Gyeli verb roots usually take one derivational suffix at a time, a restriction which appears related to the maximum stem length of three syllables (\sectref{sec:SyllV}). There are a few exceptions, however. Within the limits of a maximum of three syllables, a verb may combine two extensions/expansions. This is, for instance, the case with passives formed from other extensions such as the causative, applicative, or positional middle voice (\sectref{sec:PASS}). Another exception to the trend of allowing only one derivation morpheme concerns the causative that may show (remnants of) combination with the applicative, \REF{CAUSAPP}, or the expansion morpheme -{\itshape lɛ}, \REF{CAUSle}, again respecting the three syllable maximum of the verb stem. Examples such as \REF{CAUSAPP} are rare. One could likewise assume that  -{\itshape s}- in \REF{CAUSAPP} is an epenthetic consonant, as discussed in \sectref{sec:StructVerb}. Since /s/ as an epenthetic consonant is rare as well, however, it is possible that all of these instances stem from an original causative morpheme. Synchronically, this cannot be determined with certainty. Combinations of causative and applicative morphemes in Gyeli follow the reconstructed causative-applicative order for Bantu, as discussed by \citet{good2005}.


\ea \label{CAUSAPP}
 \gll  kà-s-ɛlɛ  \\
         catch-{\CAUS}-\textsc{\APPL} \\
    \trans `light sth. (make sth. catch fire)'
\z

\noindent In combinations of the causative and the expansion -{\itshape lɛ}, in contrast, the expansion morpheme precedes the causative suffix, as shown in \REF{CAUSle}.  Synchronically, it is not clear what this expansion does or what its semantic function is, as I discuss in more detail in \sectref{sec:DiaEx}. In \REF{CAUSle}, -{\itshape lɛ} may indicate a perfective reading:\footnote{While there is definitely a difference in valency involved as well, {\itshape bwà-lɛ} `be born' does not match the passive forms discussed in \sectref{sec:PASS}.} {\itshape bwà} `give birth' >  {\itshape bwà-lɛ} `be born' >  {\itshape bwà-l-ɛsɛ} `make give birth'.

\ea \label{CAUSle}
 \gll  bwà-l-ɛsɛ  \\
         catch-lɛ-{\CAUS} \\
    \trans `make give birth (e.g.\ acting as midwife)'
\z

\noindent Some verbs lacking the disyllabic expansion form with -{\itshape lɛ}, still use /l/  as an epenthetic consonant in the causative form, for instance in {\itshape bâ} `marry' >  {\itshape bál-ɛsɛ} `make marry' (but having no form {\itshape bálɛ}). In verb forms that take two different epenthetic consonants with different derivation morphemes, one of the consonants is often /l/, which may have its origin in the expansion morpheme -{\itshape lɛ}. Extensions derived from the -{\itshape lɛ} form include passive and applicative, for example in {\itshape bû} `destroy' >  {\itshape búl-a} `destroyed', while the reciprocal is formed with /y/ {\itshape búy-ala} `destroy each other'. As stated above, however, this observation does not translate into any synchronic rule and is currently lexically specified.

 As \tabref{Tab:SumVext} shows, extension forms highly vary in the number of verbs they combine with, which may have different causes. While categories such as causative and applicative seem to have become reduced, other extensions such as -{\itshape ɔwɔ} and -{\itshape ɛga} are restricted semantically. -{\itshape ɔwɔ} as a positional category, for instance, only combines with semantically compatible verb roots.\footnote{It should also be mentioned that the numbers given in the table should not be taken as absolute. For one, despite my attempt to elicit the entire paradigm of possible extended verb forms, there is the possibility that the speaker could not think of any appropriate context and rejected a possible extended verb form on these grounds, while another speaker would have accepted a potential form. So there may actually be more forms.}


\begin{table}
\fittable{\begin{tabular}{l ll @{~}l@{~} ll}
  \lsptoprule
 Category  & \multicolumn{5}{l}{Example} \\  \midrule
 {\scshape reciprocal}  & lúnd-ala & `fill one another'& <  & lúndɔ & `fill (v.i.)' \\
 {\scshape passive}  & lúnd-a & `be filled' & <  & lúndɔ & `fill (v.i.)' \\
 {\scshape causative} & lúnd-ɛsɛ & `make sth. full'  & <  & lúndɔ & `fill (v.i.)' \\
 {\scshape applicative} &  lúnd-ɛlɛ & `fill sth.' & <  & lúndɔ & `fill (v.i.)'  \\
 {\scshape autocausative}  & vìd-ɛga & `turn (by itself)' & <  & vìdɛ & `turn sth.' \\
 {\scshape positional}  & kɛ̀l-ɔwɔ & `assume hanging position' & < & kɛ̀lɛ & `hang sth.' \\  \midrule
 {\scshape -kɛ}  & jí-kɛ & `burn sth.' & <  & jíyɛ & `burn (v.i.)' \\
 {\scshape -lɛ} & bwà-lɛ & `be born' & <  & bwà & `give birth' \\ 
 {\scshape -bɔ/wɔ}  &   jì-bɔ & `close' & <  & jì & `open' \\
 \lspbottomrule
\end{tabular}}
\caption{Examples of verbal derivation morphemes}
\label{Tab:SumVextex}
\end{table} 


\noindent Another issue concerns verb forms that have an extension or expansion but no synchronic underived form. I treat them as underived forms here, i.e.\ I do not count them as extensions in the table in order to be consistent across categories. While it is easy to recognize, for instance, a causative or applicative form, it is much harder for possible expansions such as -{\itshape kɛ}.  As indicated in \tabref{Tab:SumVext}, there are ten instances of this morpheme serving as an expansion to an underived form. There are, however, five instances in my database where a -{\itshape kɛ} ending appears as an apparent underived form itself, taking yet its own extension morphemes. Synchronically, it is not possible to determine whether this -{\itshape kɛ} carries any morphological function or whether it is simply a random lexical form.
\tabref{Tab:SumVextex} provides examples of each extension and expansion category, including the underived verb form.

In the following, I will describe each derivation morpheme and its semantic functions in a decreasing order of frequency. As discussed in \sectref{sec:Tonology}, all derivation morphemes are underlyingly toneless. Therefore they are represented without tonal marking here.


\subsubsection{Reciprocal \textit{-ala}}
\label{sec:REC}

The verb extension -{\itshape ala} is by far the most frequent in Gyeli. Out of 377 verbs in the database, 270 (71.6\%) allow for this extension, which I label as reciprocal. Further, there are eight occurrences of verb stems ending in -{\itshape ala} that do not have an underived form.

In terms of the extension's semantic function, it has mostly a reciprocal meaning, as the examples in \REF{RECIP} show, which express ``mutuality''.


\eabox{\label{RECIP}\begin{tabular}{@{}lllll@{}}
dvùɔ̀ & `hurt (v.i.)' & >  & dvùg-ala & `hurt one another' \\
dyúwɔ & `hear' & > & dyúw-ala & `understand each other' \\
gyíwɔ & `call' & > &  gyíw-ala & `call each other' \\
kwàlɛ & `love' & > & kwàl-ala & `love each other' \\
tsíndɔ & `push' & >  & tsínd-ala & `push each other' \\
bâ & `marry' & >  & bán-ala & `marry each other' \\
kɛ̃̀ & `shave' & >  &  kèng-ala & `shave each other' \\
\end{tabular}}


Beyond this reciprocal meaning, there are many instances of verbs whose semantics do not allow for a reciprocal use. In these cases, the extension -{\itshape ala} has a ``togetherness'' reading, as shown in \REF{togetherness}.

\eabox{\label{togetherness}\begin{tabular}{@{}lllll@{}}
nyùlɛ & `drink' & >  & nyùl-àlà & `drink together' \\
kɔ́sɛ & `cough' & >  & kɔ́s-ala & `cough together' \\
pámɔ & `show up' & >  & pám-ala & `show up together' \\
tɛ́bɔ & `get up' & >  & tɛ́b-ala & `get up together' \\
bwà & `become big' & >  & bɔ̀g-ala & `become big together' \\
kwê & `fall' & >  & kwéy-ala & `fall together' \\
nyî & `enter' &  >  & nyíng-ala & `enter together' \\
\end{tabular}}

It is possible that verbs which do allow a reciprocal meaning may get a ``togetherness'' reading, depending on the context. This, however, needs further investigation. 
It is not possible to get a reciprocal causative reading, for instance `make each other fall' for {\itshape kwéy-ala}, as the causative extension of the verb, {\itshape kù-ɛsɛ} `make fall' cannot combine with the reciprocal extension. Instead, a coordination construction with a morphological causative can be used, as in \REF{CAUSx1}, or a complement clause, as in \REF{CAUSx2}.\footnote{Although \REF{CAUSx2} can be interpreted with a reciprocal meaning, the referent of {\itshape nɔ́nɛ́gá} `other' is ambiguous, since it is not necessarily co-referential with the subject of the main clause.} 

\ea \label{CAUSx}
  \ea  \label{CAUSx1}
  \glll nyɛ̀gà á kùɛ́sɛ́ nɔ́nɛ́gá nà nyɛ̀gà á kùɛ́sɛ́ nɔ́nɛ́gá \\
   nyɛ̀-gà a-H kù-ɛsɛ-H nɔ́-nɛ́gá nà nyɛ̀-gà a-H kù-ɛsɛ nɔ́nɛ́gá \\
    1.{\SBJ}-{\CONTR} 1-{\PRS} fall-{\CAUS}-{\R} 1-other {\CONJ} 1.{\SBJ}-{\CONTR} 1-{\PRS} fall-{\CAUS}-{\R} 1-other\\
   \trans `One makes the other fall and that one makes the other fall.'
\ex\label{CAUSx2}
 \glll bá sá nâ nɔ́nɛ́gá á kwê \\
      ba-H sâ-H nâ nɔ́-nɛ́gá a-H kwê \\
     2-{\PRS} do-{\R} {\COMP} 1-other 1-{\PRS} fall \\
    \trans `They make that the other falls.'
\z
\z




\subsubsection{Passive \textit{-a}}
\label{sec:PASS}

I will discuss the contrast between active and passive constructions following \posscitet{siewierska2013} defining criteria for passive constructions which I illustrate in \REF{passdef}.

\ea \label{passdef}
\ea  \label{passdef1}
  \glll  bùdì bá tsìlɔ́ békálàdɛ̀.\\
	b-ùdì ba-H tsìlɔ-H H-be-kálàdɛ̀ \\
             ba2-person 2-\textsc{prs} write-{\R} {\OBJ}.{\LINK}-be8-book  \\
    \trans `People write books.'
\ex\label{passdef2}
 \glll  bèkálàdɛ̀ bé tsìl{\bfseries á} (nà bùdì). \\
	be-kálàdɛ̀ be-H tsìl-a-H nà b-ùdì \\
         be8-book 8-\textsc{prs} write-{\bfseries {\PASS}}-{\R} {\COM} ba2-person  \\
    \trans `Books are written (by people).'
\z
\z


\noindent  \REF{passdef1} is the active, while \REF{passdef2} is the contrasting passive construction. According to \citet{siewierska2013}, ``the subject of the active corresponds to a non-obligatory oblique phrase of the passive or is not overtly expressed,'' which is the case for the subject {\itshape bùdì} in \REF{passdef1}. Another characteristic of passive constructions is that their subjects correspond to the direct object in the active counterpart, as with {\itshape bèkálàdɛ̀} `books'. Siewierska also points out that passive constructions are pragmatically more restricted than active constructions, which is true in Gyeli as well. Finally, she notes that passive constructions receive a special morphological marking of the verb. In the case of Gyeli, this is a final vowel -{\itshape a}, in most cases, as will be discussed below.

Generally, passive forms are far less frequent than reciprocals, with only 105 attested instances (27.9\% of the verbs in the database). Speakers appear to prefer the active form with the impersonal third person plural of class 2 and are forced to use this for the majority of verbs which do not have a passive form.  Morphological marking of the passive on the verb in Gyeli differs phonologically, depending on the syllable number of the verb form the passive is derived from. Passives from mono- and disyllabic roots differ from trisyllabic ones. I will discuss both in turn.

The passive in Gyeli is formed by the extension -{\itshape a}, resulting in a disyllabic verb stem if it is derived from a mono- or disyllabic verb root, as shown in \REF{passive1}.


\eabox{\label{passive1}\begin{tabular}{@{}lllll@{}}
kwàlɛ & `love' & >  & kwàl-a & `be loved' \\
bvúɔ̀ & `break sth.' & >  & bvúg-a & `be broken' \\
jì & `open' & >  & jìy-a & `be open' \\
dyû & `kill' & >  & dyúw-a & `be killed' \\
jíwɔ & `steal' & >  & jíy-a & `be stolen' \\
vìdɛ & `turn sth.' & >  & vìd-a & `be turned' \\
bàwɛ & `carry sth.' & >  &  bàw-a & `be carried' \\
\end{tabular}}


All these instances have an underived form. There are, however, 36 disyllabic verbs ending in -{\itshape a} which are underived, non-passive forms. Examples are given in \REF{passive2}. In fact, these verbs cannot be passivized nor do they have a passive meaning. Expressing passive meaning as in \REF{passive1} is not possible for them since their ending is identical with the passive suffix.


\eabox{\label{passive2}\begin{tabular}{@{}ll@{}}
gyàga & `buy'  \\
kòla & `add'  \\
kìya & `give'  \\
bwàndya & `despise' \\
\end{tabular}}

For other disyllabic verb stems ending in -{\itshape a} which do not have an underived form, agentivity is less specified. The examples in  \REF{passive3} can be thought of as having a non-specified agent while the subject takes the semantic role of an experiencer.


\eabox{\label{passive3}\begin{tabular}{@{}ll@{}}
vòwa & `wake up'  \\
wùsa & `forget'  \\
káka & `shiver'  \\
kánda & `crack (v.i.; e.g.\ bottle or glass)' \\
sìya & `wash, bathe sb./oneself'  \\
\end{tabular}}

Finally, a few disyllabic passive forms take a final -{\itshape ɛ} rather than the usual passive -{\itshape a} extension, as shown in \REF{passive1a} which lists all known instances.

\eabox{\label{passive1a}\begin{tabular}{@{}lllll@{}}
bwè & `catch' & >  & bùl-ɛ & `be caught' \\
sàlɔ & `cut lengthwise' & >  & sàl-ɛ & `be cut lengthwise' \\
tìnɔ & `harvest tubers' & >  & tìl-ɛ & `be harvested (tubers)' \\
\end{tabular}}

\noindent These exceptions are specified in the lexicon rather than stemming from a predictable morphophonological rule.


In a few rare cases, the passive can also be formed from trisyllabic stems, i.e.\ from verbs which already have an extension such as the causative, applicative, or positional middle voice. In these cases, not only the final vowel changes to -{\itshape a}, but also that of the second syllable, as shown in \REF{passive1b}. The passive forms that are derived from applicatives -{\itshape ɛlɛ} are identical with the reciprocal forms. I do not mark morpheme breaks with a hyphen for these passive forms since morpheme boundaries are not clear-cut. Rather, an extension morpheme such as -{\itshape awa} has to be considered a portmanteau morpheme, encoding both the passive via the vowels /a/\footnote{The occurrence of /a/ in the second and third syllable can be viewed as an instance of vowel harmony where the first extension vowel harmonizes with the last one.} and the positional via the consonant /w/.

\eabox{\label{passive1b}\begin{tabular}{@{}lllll@{}}
bál-ɔwɔ & `bend down' & >  & bálawa & `be bent down' \\
bén-ɛlɛ & `raise, lift sth.' & >  & bénala & `be lifted (lift each other)' \\
bùm-ɛlɛ & `hit (nail)' & >  & bùmala & `be hit (hit each other)' \\
dyɔ̀l-ɛsɛ & `make laugh' & >  & dyɔ̀lasa & `be made to laugh' \\
pín-ɛsɛ & `squeeze' & >  & pínasa & `be squeezed' \\
\end{tabular}}

Historically, the passive extension is likely to have developed from the middle voice suffix -{\itshape aga} which is still used in Mabi as passive. In Gyeli, the velar stops got lost and the vowel contracted. In careful speech, the final -{\itshape a} is sometimes still lengthened, for instance in {\itshape gyàmbaa} `be cooked' which is derived from {\itshape gyámbɔ} `cook',  but in fast speech and most lexemes, it surfaces as a short vowel.

The use of passive verbs is rather restricted, nevertheless. For one, many underived verbs do not allow for passivization, even though this would semantically be possible. Also, in terms of text frequency, even verbs that do have a passive form are rarely used.\footnote{The passive forms discussed in this section were mainly collected by elicitation.} In natural speech, the Bagyeli prefer to use an active construction with a class 2 (3\textsuperscript{rd} person plural) subject as an agent, which remains semantically unspecified, as in \REF{Passba}.

\ea \label{Passba}
 \glll bá gyàgá má-ntúà  \\
	ba-H gyàga-H H-ma-ntúà \\
         2-\textsc{prs} buy-{\R} {\OBJ}.{\LINK}-ma6-mango  \\
    \trans `They buy the mangoes (= the mangoes are bought).'
\z

\noindent See also \sectref{sec:IS} for a more detailed discussion of information structure.


The passive appears to be related to two other derivation forms: the autocausative and the nominalized past participle. The passive could be the shortened form of the autocausative -{\itshape aga}, discussed in \sectref{sec:AutoCaus}. As explained there, -{\itshape aga} is the regular passive suffix in Mabi. In Gyeli, it appears to have split into two categories: the passive and the autocausative. This can be seen in a few instances where the passive suffix is a lengthened vowel, as in \REF{PASS1}. It seems lexically specified whether a verb can take the lengthened passive form. In any case, the lengthened suffix is in free variation with the default short form.

\eabox{\label{PASS1}\begin{tabular}{@{}lllll@{}}
kfúdɛ & `cover' & >  & kfúd-a(a) & `be covered' \\
wàwɛ & `spread' & >  & wàw-a(a) & `be spread' \\
gyámbɔ & `cook' & >  & dyúg-a(a) & `be cooked' \\
kwɛ̀lɔ & `cut down' & >  & kwɛ̀l-a(a) & `be cut down' \\
\end{tabular}}

\noindent In a likely scenario, the consonant /g/ has been deleted from -{\itshape aga}, developing into a lengthened passive form which still exists in a few lexemes while the synchronic default form is a short vowel.

Semantically, the shift from autocausative middle voice to passive seems natural. In both cases, the agent is not overtly expressed. The main difference seems to concern the attribution of agentivity. In the autocausative, the subject has a certain degree of agentivity, while, in the passive, the subject is clearly the patient.  Given the distinct functions of passive and autocausative, quite a few verbs take both extensions. This is true for all examples in \REF{PASS1}; others are listed in \appref{sec:AppendixI}.

The passive form is also related to the nominalized past participle described in \sectref{sec:NOMPart}. The difference between the two  is both structural and semantic. The passive verb form is preceded by a \textsc{stamp} marker, as in \REF{PASS}, while the nominalized past participle requires the \textsc{stamp} copula (\sectref{sec:COP}) that agrees with the subject, as shown in \REF{NomPart}.

\ea \label{PASS}
 \glll yí kɛ̀là  \\
	yi-H kɛ̀l-a \\
         7-\textsc{prs} hang-{\PASS}  \\
    \trans `It is being hung.'
\ex \label{NomPart}
 \glll yíì nkɛ̀lá  \\
         yíì n-kɛ̀l-a-H \\
	{\COP} {\NOM}-hang-{\PASS}-{\NOM} \\
    \trans `It has been hung [lit. It is a `hung-up one'].'
\z

\noindent The meaning difference between the two constructions is in fact aspectual. The passive construction views an event as ongoing and is incompatible with completive aspect. In contrast,  the nominalized past participle form is more resultative and completive.


\subsubsection{Causative \textit{-ɛsɛ}}
\label{sec:CAU} 

The causative extension morpheme -{\itshape ɛsɛ} increases the verb's valency, turning intransitive verbs into transitive and transitive verbs into ditransitive ones. \citet{song2013} defines causative constructions as denoting complex situations
\begin{quote}
consisting of two component events [...]: (i) the {\bfseries causing event}, in which the {\bfseries causer} does or initiates something; and (ii) the {\bfseries caused event}, in which the {\bfseries causee} carries out an action, or undergoes a change of condition or state as a result of the causer’s action.
\end{quote}
This definition becomes clearer when looking at \REF{Causdef} where the causer, {\itshape Màmbì} performs a causing event (teaching) and the causee, Ada, experiences the caused event (learning English). As a consequence, the causee, {\itshape Àdà}, does something, namely learning English which is the caused event.

\ea \label{Causdef}
 \glll Màmbì á gyíkɛ́sɛ́ Àdà ngɛ̀lɛ́nɛ̀  \\
	Màmbì a-H gyík-ɛsɛ-H Àdà ngɛ̀lɛ́nɛ̀ \\
         $\emptyset$1.{\PN} 1-\textsc{prs} learn-{\CAUS}-{\R} $\emptyset$1.{\PN} $\emptyset$1.English  \\
    \trans `Mambi teaches Ada English (lit. makes Ada learn English).'
\z

This type of morphological causative, as opposed to lexical and syntactic causatives (see \citealt[3]{song96}), is marked on the verb by a suffix. The morphological causative is not the only causative construction found in Gyeli. Also syntactic causatives using the verb {\itshape sâ} `make' plus the complementizer {\itshape nâ}, as in \REF{CAUS1}, are quite common.

\ea \label{CAUS1}
  \glll     mɛ́ nzíí sâ nâ wɛ́ dyɔ̀\\
	mɛ-H nzíì-H sâ nâ wɛ-H dyɔ̀ \\
              1\textsc{sg}-\textsc{prs} {\PROG}-{\R} make {\COMP} 2\textsc{sg}-\textsc{prs} laugh  \\
    \trans `I make you laugh.'
\z

The morphological causative in Gyeli is formed by the suffix -{\itshape ɛsɛ}. 89 verbs in the database (23.6\%) have a causative suffix. There are another 6 verbs with a causative ending which do not  have an underived form. Examples are provided in \REF{CAUS}.

\eabox{\label{CAUS}\begin{tabular}{@{}lllll@{}}
gìyɔ & `cry' & >  & gìl-ɛsɛ & `make cry' \\
gyímbɔ & `dance' & >  & gyímb-ɛsɛ & `make dance' \\
dyúwɔ & `hear, perceive' & >  & dyúg-ɛsɛ & `make feel' \\
nyâ & `suckle, lick' & >  & nyáng-ɛsɛ & `breast-feed' \\
mìno & `swallow' & >  & mìn-ɛsɛ & `make swallow' \\
jíyɔ & `burn (v.i.)' & >  & jíg-ɛsɛ & `make angry' \\
lùnga & `grow (v.i.)' &  >  & lùng-ɛsɛ & `raise, make grow' \\
gyíkɛ & `learn' & >  & gyík-ɛsɛ & `teach' \\
\end{tabular}}

Some medial consonants of underived verb forms are subject to change in verbal derivation. This is precisely the case with epenthetic consonants such as /w/ (between /u/ and /ɔ/) and /y/ (between /i/ and /ɔ/) which may be replaced by another consonant in the derived forms. In this respect, disyllabic underived verbs behave parallel to monosyllabic roots, as discussed in \sectref{sec:StructVerb} for stem-final vowels.


While in the great majority of cases, the suffix -{\itshape ɛsɛ} expresses causativity, there are a few cases where the semantic lines between causative and applicative are blurred, as for instance with the verb {\itshape dvùbɔ} `dip, soak'.  For these, the underived verb can be used, as in \REF{causative1}, or the causative, as in \REF{causative2}, or the applicative form, as in \REF{causative3}. The causative form in \REF{causative2} is semantically closer to an applicative. Thus, the sentence in \REF{causative2} with another verb such as {\itshape nyî} `enter', as in \REF{causative4}, will take the applicative form. With {\itshape dvùbɔ} ` dip', however, the applicative has undergone semantic shift, denoting a different type of action, as shown in \REF{causative3}, and the causative takes over the applicative semantics. 


\ea \label{causative}
\ea  \label{causative1}
  \glll  mɛ́ dvùbɔ́ pɛ̀mbɔ̀ (ɛ́ kɔ̀fí) \\
	mɛ-H dvùbɔ-H pɛ̀mbɔ̀ ɛ́ kɔ̀fí \\
              1\textsc{sg}-\textsc{prs} dip-{\R} $\emptyset$1.bread {\LOC} $\emptyset$7.coffee \\
    \trans `I dip the bread in coffee.'
\ex\label{causative2}
 \glll  mɛ́ dvùbɛ́sɛ́ wɛ̂ màjíwɔ́ \\
	mɛ-H dvùb-ɛsɛ-H wɛ̂ ma-jíwɔ́ \\
         1\textsc{sg}-\textsc{prs} dip-{\CAUS}-{\R} 2\textsc{sg} ma6-water  \\
    \trans `I dip you in water.'
\ex\label{causative3}
 \glll  mɛ́ dvùbɛ́lɛ́ béká  \\
	mɛ-H dvùb-ɛlɛ-H H-be-ká \\
         1\textsc{sg}-\textsc{prs} dip?-\textsc{appl}-{\R} {\OBJ}.{\LINK}-be8-grass \\
    \trans `I weed the grass [with a rake].'
\z
\ex \label{causative4}
 \glll  mɛ́ nyíngɛ́lɛ́ wɛ̂ màjíwɔ́ \\
	mɛ-H nyíng-ɛlɛ-H wɛ̂ ma-jíwɔ́ \\
         1\textsc{sg}-\textsc{prs} enter-{\CAUS}-{\R} 2\textsc{sg} ma6-water  \\
    \trans `I insert you into water.'
\z

The distribution and frequency of the underived versus the causative form needs further investigation. The occurrence of comparable cases in the corpus is so rare that no generalizations can be made at this point.


\subsubsection{Applicative \textit{-ɛlɛ}}
\label{sec:APP}


The extension -{\itshape ɛlɛ}  is significantly rarer in Gyeli than the causative -{\itshape ɛsɛ}, with only 34 (9\%) instances in the database. Further, there are no verbs ending in -{\itshape ɛlɛ} that have no underived form. I refer to the -{\itshape ɛlɛ} suffix as ``applicative'', a category that is commonly found in Bantu languages.

Morphosyntactically, the applicative changes the verb's valency by increasing ``the number of object arguments selected by the predicate [...] by one with respect to the basic construction'' (Polinsky 2013). \citet[278]{peterson97} specifies that, in applicative constructions:
\begin{quote} thematically peripheral objects are treated in a more core or direct object manner, and in terms of discourse, they often have higher relative topicality in applicative constructions as compared to when they occur in non-applicative constructions. \end{quote}

\noindent Gyeli forms applicatives both from intransitive \REF{APPintr} and transitive \REF{APPtr} verbs, which seems to be the typical case in Bantu languages, according to \citet{polinsky2013}.

\eabox{\label{APPintr}\begin{tabular}{@{}lllll@{}}
nyùmbɔ & `smell (v.i.)' & >  & nyùmb-ɛlɛ & `smell sth.' \\
swásɔ & `dry (v.i.))' & >  & swás-ɛlɛ & `dry sth.' \\
bédɔ & `go up' & >  & béd-ɛlɛ & `mount sth.' \\
lúndɔ & `fill oneself' & >  & lúnd-ɛlɛ & `fill sth.' \\
sɔ́'ɔ̀ & `continue' & >  & sɔ́s-ɛlɛ & `continue with sth.' \\
jímbɛ & `get lost' & >  & jímb-ɛlɛ & `lose sth.' \\
bámɔ & `scold (v.i.)' &  >  & bám-ɛlɛ & `scold sb.' \\
dyũ̂ & `be hot' & >  & dyúng-ɛlɛ & `heat sth.' \\
\end{tabular}}

\noindent Further, \citet{polinsky2013} distinguishes applicative constructions in terms of the semantic role of the applied object, pointing out that Bantu languages typically licence benefactive and other semantic roles. This is also true for Gyeli. Benefactive contexts usually arise with applicatives formed from transitive verbs, for instance as shown in \REF{APPtr} for {\itshape gyámbɔ} `prepare'. In these cases, a second object is added which often takes the role of a benefactive or an instrumental.

\eabox{\label{APPtr}\begin{tabular}{@{}lllll@{}}
lúmɛ & `send' & >  & lúm-ɛlɛ & `send to sb.'\\
gyámbɔ & `prepare' & >  & gyámb-ɛlɛ & `prepare for sb.'\\
dyúwɔ & `hear, perceive' & >  & dyúw-ɛlɛ & `listen'\\
vísɔ & `cover' & >  & vís-ɛlɛ & `cover sth.'\\
kfùbɛ & `provoke' & >  & kfùb-ɛlɛ & `provoke sb.'\\
vídɛ & `turn sth.' & >  & víd-ɛlɛ & `turn sth.'\\
\end{tabular}}

Applicatives which are derived from intransitive verbs typically do not have a benefactive reading. In fact, they differ significantly in the distribution of semantic roles across arguments from applicatives that are derived from transitive verbs. The subject of the intransitive verb, which has the role of an undergoer, is expressed as the object in the applicative form, as shown in \REF{AppCau}. In many of these instances, the applicative forms have a causative meaning.

\eabox{\label{AppCau}\begin{tabular}{@{}lllll@{}}
vásɛ & `rise (dough)' & >  & vás-ɛlɛ & `make (dough) rise' \\
vè'è & `try on clothes' & >  & vè'-ɛlɛ & `make sb. try clothes on' \\
kɔ́sɛ & `cough' & >  & kɔ́s-ɛlɛ & `make cough' \\
\end{tabular}}

In contemporary speech, the applicative and the causative seem to be merging into one category, with the applicative most likely becoming lost, given its lower frequency in comparison to the causative.  It is rare that a verb has both an applicative and a causative form. In my database, I  found only five instances where a verb takes both -{\itshape ɛsɛ} and -{\itshape ɛlɛ}. In the majority of cases, a verb has a causative, but no applicative form.


It is not surprising that these two categories are merging since, semantically, there is some overlap between them. For instance, the applicative form {\itshape nyíngɛlɛ} `insert', derived from {\itshape nyî} `enter',  may be viewed as adding an applied object to the underived verb form. On the other hand, semantically, it can also be thought of as a causative context in the sense of `making sth.\ enter'. The same is true for {\itshape dyû} `be hot' which has an applicative form {\itshape dyúng-ɛlɛ} `heat sth.' Again, an object is added to an otherwise intransitive verb, resulting in a reading of `applying heat to sth.' At the same time, semantically, it can also be thought of as `make sth. hot'.\footnote{\citet{bostoen2011} report a similar syncretism of applicative and causative for Mbuun (Bantu B87). According to them, however, the syncretism in Mbuun is based on phonological rather than semantic grounds.}

Just like the causative,  the applicative extension has a periphrastic alternative to convey a same or similar meaning, as shown in \REF{APPperi}.

\ea \label{APPperi}
  \glll     mɛ́ gyá gyá mpá'à wɔ̂ \\
	mɛ-H gyâ-H gyá mpá'à w-ɔ̂ \\
    1\textsc{sg}-\textsc{prs} sing-{\R} $\emptyset$7.song $\emptyset$3.side 3-2\textsc{sg}.{\POSS}   \\
    \trans `I sing a song for you.'
\z

\subsubsection{Autocausative middle voice \textit{-ɛga/-aga}}
\label{sec:AutoCaus}

The extension -{\itshape ɛga}/-{\itshape aga} appears 28 times in the verb database, which means that 7.4\% of the verbs allow this extension. Further, there are four verbs with this extension which have no synchronic underived form. 

In contrast to other extensions, this derivation has two variant suffixes: -{\itshape ɛga} and -{\itshape aga}, with each verb being specified for one or the other. The choice for one of the two suffix forms seems to be lexically specified rather than depending on phonological rules. Even though there is a tendency that -{\itshape aga} is used after the glide /j/ ({\textlangle}y{\textrangle}  in orthography) as well as after /m/ or /mb/, there are also a few cases where -{\itshape ɛga} appears after these consonants. Given that their form is very similar while the function is the same, I consider these two suffixes as belonging to the same category.  It is possible that the form -{\itshape aga} has its origin in the neighboring language Mabi where the suffix is used productively for passive formation. This, however, does not explain why -{\itshape ɛga} is used for some and -{\itshape aga} for other verbs and how the existing distribution comes about. In terms of frequency, -{\itshape ɛga} is found more often than its variant -{\itshape aga}, the latter appearing only nine times in contrast to -{\itshape ɛga} with 19 times.

The suffix variants -{\itshape ɛga} and -{\itshape aga} constitute one of two middle voice categories in Gyeli. I distinguish, in terms of terminology, the autocausative middle voice extension -{\itshape ɛga}/-{\itshape aga} from the ``positional'' middle voice suffix -{\itshape ɔwɔ}, discussed in \sectref{sec:PosVerbs}. Unlike valency-increasing extensions, such as the applicative or causative, the middle voice constitutes a category ``intermediate  in  transitivity  between  one-participant  and  two-participant  events'', as defined by \citet[3]{kemmer93}.\footnote{Note that \citet{kemmer93} primarily defines the middle voice as a semantic category which, in some languages, receives formal marking. I deviate from this notion in that I consider middle voice categories in Gyeli as formal categories which map onto certain functions.} In Gyeli, the autocausative middle voice typically denotes one-participant events. It requires only one argument (the subject), having a valency decreasing effect. 
The autocausative, as exemplified in \REF{Autocaus}, is accordingly intransitive, derived from transitive verbs. Semantically, the subject of autocausative verbs incorporates the roles of both agent and undergoer, while syntactically the agent remains under-specified. Often, a certain self-causation is implied in such events which I translate as `by itself'.


\eabox{\label{Autocaus}\begin{tabular}{@{}lllll@{}}
vìdɛ & `turn (v.t.)' & >  & vìd-ɛga & `turn (by itself)' \\
wàwɛ & `spread sth.' & >  & wàw-ɛga & `spread (by itself)' \\
jìna & `dive' & >  & jìn-ɛga & `sink (intr), melt (intr)' \\
kfúdɛ & `cover sth.' & >  & kfúd-ɛga & `cover (by itself)' \\
lɛ̀ndo & `flow' & >  & lɛ̀nd-ɛga & `flow (by itself) \\
lɛ́gɛ & `singe' & >  & lɛ́g-ɛga & `singe (by itself) \\
tfúmbɔ & `wrinkle sth.' & >  & tfúmb-aga & `get wrinkled (by itself)' \\
líyɔ & `clear land' & >  & líy-aga & `clear (by itself)' \\
\end{tabular}}

Cross-linguistically, there seems to be a strong relation between middle voice and reflexive constructions. \citet{
kemmer93}, for example, demonstrates that middle marking often evolves from reflexive constructions. Speakers indeed tend to translate autocausative middle voice forms with a French reflexive construction using {\itshape se}, for example {\itshape tfúmb-aga} `get wrinkled (by itself)' would be translated as {\itshape se plier} in French. Nevertheless, I argue that the autocausative in Gyeli constitutes a basic system which is not derived from reflexive constructions. This view is parallel to \posscitet{maldonado2009} observation on South American languages, where middle voice also constitutes a basic system independent of reflexives. 

In comparison to the autocausative suffix, Bantu reflexives are canonically expressed by an affix preceding the stem, which \citet[109]{meeussen67} calls an ``infix'' and  reconstructs as *-\c{í}-  (-j\c{í}-? -j\c{í}\c{i}-?)  for Proto-Bantu. Such a prefix is not found in Gyeli. Reflexivity in Gyeli is rather expressed by object pronouns plus {\itshape mɛ́dɛ̀} `self' as in \REF{reflex} or, in other cases, verbs carry reflexive meaning lexically as in {\itshape síya} `wash (oneself)'. 


\ea \label{reflex}
  \glll     mɛ́ nyɛ́ mɛ̂ mɛ́dɛ̀ \\
	mɛ-H nyɛ̂-H mɛ̂ mɛ́dɛ̀ \\
              1\textsc{sg}-\textsc{prs} see-{\R} 1\textsc{sg}.{\OBJ} self   \\
    \trans `I see myself.'
\z

Given these constructions which differ formally very much from the autocausative, there is no obvious reason to assume that they are related or even that the autocausative has evolved from the reflexive. On the other hand, the autocausative is structurally more similar to the passive in Mabi, which has the extension -{\itshape aga} or may even be related to the passive extension -{\itshape a(a)} in Gyeli itself. This relationship is discussed in more detail in \sectref{sec:PASS}


\subsubsection{Positional middle voice \textit{-ɔwɔ}}
\label{sec:PosVerbs}

The extension -{\itshape ɔwɔ} constitutes the second type of middle voice category in Gyeli. -{\itshape ɔwɔ} is the least frequent verb extension in Gyeli with a total of 15 occurrences, 11 of which are part of the 377-verb database while four have not been considered for this database, as they were discovered later in the project. Out of the 11 occurrences within the database, only six (1.6\%) are used productively in the sense that they have synchronically an underived verb form.  

I label this category as ‘positional middle voice’ since almost all verbs with this extension describe the event of assuming a position, as illustrated in \REF{Positional}.\footnote{The one known exception to posture reference is the verb {\itshape bwèd-ɔwɔ} `be tasty/sweet'.}

\eabox{\label{Positional}\begin{tabularx}{\textwidth - \widthof{(71)~~~~}}{@{}ll @{~}l@{~} lQ@{}}
kɛ̀lɛ & `hang sth.' & >  & kɛ̀l-ɔwɔ & `assume a hanging position' \\
kfúdɛ & `cover sth.' & >  & kfúd-ɔwɔ & `lie down by covering head with arms' \\
kwádɔ & `twist sth.' & >  & kwád-ɔwɔ & `assume a crooked position' \\
ngwáwɔ & `bend sth.' & >  & ngwáng-ɔwɔ & `bend (v.i.)' \\
pwásɔ & `flatten sth.' & >  & pwás-ɔwɔ & `assume a flattened position, stretch out' \\
\end{tabularx}}

\noindent The same is true for verbs of this ending which do not seem to have a synchronic underived form, as exemplified in \REF{Positional2}.


\eabox{\label{Positional2}\begin{tabular}{@{}ll@{}}
bál-ɔwɔ & `bend down' \\
kwàng-ɔwɔ & `lie down on side' \\
gyí-ɔwɔ & `lean back' \\
pwàngy-ɔwɔ & `lie down stretched out (French: {\itshape s'allonger})'  \\
sɛ̀ngy-ɔwɔ & `assume inclined position' \\
\end{tabular}}

\citet[75]{schadeberg2003} uses the term ``positional'' for a stative category that talks about ‘assuming a position’ or ‘being in a position’. He reconstructs °{\itshape -am-} as the positional extension for Proto-Bantu which differs significantly in the segmental material {\itshape -ɔwɔ} in Gyeli. Nevertheless, both forms seem to carry the same meaning.

\citet[76]{schadeberg2003} does not consider the derivation °{\itshape -am-} in PB as middle voice. He mentions, however, that this extension is known to have become a passive suffix in certain Bantu languages of zone C. For languages such as Gyeli and Mabi, it seems that passive forms are more related to the autocausative middle voice category, as described in \sectref{sec:AutoCaus} and \sectref{sec:PASS}.


A few positional forms can further be derived to passive forms by substituting the two final vowels /ɔ/ by the passive vowel /a/, as shown in \REF{PosPass}.\footnote{Passive forms of the positional middle voice were not given for all positional verb forms. Given that passive forms are generally restricted and less frequent than logically possible, it seems that the same is true for passives of positional forms rather than assuming that these are gaps in the data, which in particular instances might be the case.}


\eabox{\label{PosPass}\begin{tabular}{@{}lllll@{}}
bál-ɔwɔ & `bend down' & → & bál-awa & `be bent down' \\
pwàs-ɔwɔ & `stretch out' & → & pwás-awa & `be stretched out' \\
\end{tabular}}

The two middle voice categories, the autocausative and the positional, differ not only in their extension forms, but also in their distribution of admissible subjects, and in their semantics. Subjects of the positional middle voice are typically human, at least animate, while the autocausative allows both animate and inanimate subjects. Very often, however, subjects of autocausative verb forms are inanimate, given that they incorporate the role of an undergoer, which for many transitive verbs such as {\itshape kfúdɛ} `cover' or {\itshape lɛ́gɛ} `singe' is typically inanimate.

In terms of semantics, the agent in autocausative forms is underspecified, implying a certain self-causation which is possibly more metaphorical than real. For instance, when using the form {\itshape wàw-ɛga} `spread (by itself)' with a subject such as `seeds', this is generally understood as `the seeds spread by themselves'. In reality, they are probably spread by the wind or some other agent such as animals which is not salient enough to deserve mentioning. Thus, the subject can be treated as the agent, even though this might not be the case in the world. In contrast, the agent of positional verb forms is always identical with the subject.


A verb can have both middle voice forms. Given the low frequency of forms of both middle voice categories, there are not many examples, but one is the verb {\itshape kwádɔ} `twist' which has both the autocausative {\itshape kwád-ɛga} `get twisted, twist by itself' and the positional {\itshape kwád-ɔwɔ} `assume a twisted, curved position'.  The autocausative typically has an inanimate subject, for instance a rope or a net, while the positional form has a human subject. Further, this verb has a passive form {\itshape kwád-a} `be twisted'. \tabref{Tab:VerbScale} shows the whole range of possible agent specifications in Gyeli.


\begin{table} 
\fittable{%
\begin{tabular}{llll}
\lsptoprule
Transitive → &   Positional → &    Autocausative → &    Passive \\  \midrule
two participants &  agent={\SBJ} & agent={\SBJ} implied &  agent=non-{\SBJ} \\  \midrule
kwádɔ &  kwádɔwɔ & kwádɛga  & kwáda  \\
`twist sth.' &  `assume twisted position' &  `get twisted' &  `be twisted' \\
\lspbottomrule
\end{tabular}}
\caption{Scale of decreasing expression of agentivity}
 \label{Tab:VerbScale}
\end{table}



\subsubsection{Expansions}
\label{sec:DiaEx} 

Expansions, in contrast to extensions, are not productive. They are low in frequency and do not have an obvious core function. Gyeli has three expansion suffixes which I will discuss in turn.


The expansion suffix -{\itshape kɛ} or its weakened form -{\itshape gɛ} is found ten times in the database as clearly derived from another verb form that is presently used in the language. There are another five verbs in the database with this ending, all of which are transitive, but which do not have an underived intransitive form.

The addition of this suffix has different effects on different verbs. In most instances, the suffix -{\itshape kɛ} is valency increasing, turning an intransitive verb into a transitive one, as shown in \REF{ketr}.\footnote{Some verbs with a sequence of /wa/ or /ua/ in their underived form change to /ɔ/ in the derived form, as with {\itshape bwà} `become big' changing to {\itshape bɔ̀kɛ} `make big'. Whether this change happens is lexically specified and not a general phonological rule since there are verbs with the same sequences which do not change to /ɔ/, for example {\itshape bwà} `be born' having the derived form {\itshape bwà-lɛ} `be born'.} %[ADD {\TO} PHONOLOGY SECTION]}


\eabox{\label{ketr}\begin{tabular}{@{}lllll@{}}
bwà & `become big' & >  & bɔ̀-kɛ & `make sth. big' \\
kàgɔ & `promise (v.i.)' & >  & kà-gɛ & `promise (v.t.)' \\
lṹã̀ & `whistle' & >  & lɔ̃́n-gɛ & `whistle sth.' \\
tɛ́'ɛ̀ & `be soft' & >  & tɛ́-gɛ & `soften sth.' \\
tɔ̀à & `boil (v.i.)' & >  & tɔ̀-kɛ & `boil sth.' \\
bô & `lie down (v.i.)' & > & bú-gɛ & `lie sth. down' \\
\end{tabular}}


In at least one case, the inverse happens and the expansion -{\itshape kɛ} serves as a valency decreasing suffix, as in \REF{keintr}.

\eabox{\label{keintr}\begin{tabular}{@{}lllll@{}}
bvúɔ̀ & `break sth.' & → & bvú-kɛ & `break (v.i.)' \\
\end{tabular}}


For the majority of instances where the suffix -{\itshape kɛ} has a valency increasing effect, this is semantically linked to a causative meaning, for instance in examples such as {\itshape bɔ̀-kɛ} `make big' or {\itshape tɛ́-gɛ} `soften sth.'. The -{\itshape kɛ} expansion is, however, distinct from the standard causative -{\itshape ɛsɛ}, and not an allomorph, as some verb roots can take either suffix. 
For instance, the verb {\itshape jíyɛ} `burn (v.i.)', as shown in \REF{kecaus}, allows -{\itshape kɛ} as a valency-increasing expansion. Also, the causative form {\itshape jí-g-ɛsɛ} is found with the figurative meaning `make sb. angry'.


\eabox{\label{kecaus}\begin{tabular}{@{}lllll@{}}
jíyɛ & `burn (v.i.)' & >  & jí-gɛ & `burn (v.t.)' \\
 &  & > & jí-g-ɛsɛ & `make sb.\ angry' \\
dvùɔ̀ & `hurt (v.i.)' & >  & dvù-gɛ & `hurt (v.t.)' \\
 &  & >  & dvù-g-ɛsɛ & `make sb. hurt' \\
\end{tabular}}

An alternative analysis to the suffixes -{\itshape kɛ/gɛ} and -{\itshape lɛ} would be to assume an expansion -{\itshape ɛ} which takes different epenthetic vowels /g/ and /l/, as described in \sectref{sec:StructVerb}. Under this view, /g/ in {\itshape jíg-ɛ} `burn sth.'  would be treated as a root-final epenthetic consonant. Given the tendency for a distinct causative function with the expansion -{\itshape kɛ/gɛ}, which is not found with -{\itshape lɛ}, I analyze -{\itshape gɛ/kɛ} and -{\itshape lɛ} as distinct expansion morphemes rather than assuming one expansion -{\itshape ɛ} with different epenthetic consonants.




Another non-productive suffix is -{\itshape lɛ}, which has only 6 derived forms in the database. -{\itshape lɛ} is a frequent ending of disyllabic verbs, however; 21 underived disyllabic verbs end in this syllable. It is, however, uncertain whether this is a phonologically wide-spread syllable in verbs or whether historically there was a productive extension morpheme -{\itshape lɛ}.

As with the suffix -{\itshape kɛ/gɛ}, it is difficult to pinpoint -{\itshape lɛ}'s function. Often, it seems to be valency-increasing, transitivizing an intransitive verb form, as in \REF{letr}.


\eabox{\label{letr}\begin{tabular}{@{}lllll@{}}
vû & `leave' & >  & vú-lɛ & `get rid of sth.' \\
jí(yɔ) & `sit, live' & >  & jí-lɛ & `seat sb.' \\
tɛ́-bɔ & `rise' & >  & tɛ́-lɛ & `place sth. upright' \\
\end{tabular}}

In other cases, however, the -{\itshape lɛ} suffix more seems to have a passivizing function, as in \REF{lepass}. Usually, passivization is achieved by the passive morpheme -{\itshape a}. In these two cases, however, no such form is available and rather the -{\itshape lɛ} suffix is used.


\eabox{\label{lepass}\begin{tabular}{@{}lllll@{}}
bwà & `give birth' & >  & bwà-lɛ & `be born' \\
tìnɔ & `harvest tubers' & >  & tì-lɛ & `be harvested' \\
\end{tabular}}

Given these different uses of -{\itshape lɛ}, it is not possible to provide a unified category label for this expansion.




Finally, another frequent suffix is the expansion -{\itshape wɔ/bɔ}, used with disyllabic verbs. With only two derived forms and eight verbs without a corresponding base form, the database provides few examples. This, again, makes it difficult to make generalizations about its function. It is tempting to assume a reversive category when considering \REF{reverse}.


\eabox{\label{reverse}\begin{tabular}{@{}lllll@{}}
jì & `open sth.' & >  & jì-bɔ & `close sth.' \\
\end{tabular}}

Other examples, however, do not support this hypothesis, but rather suggest that in some cases at least, -{\itshape bɔ/wɔ} has a detransitivizing effect, as in \REF{wointr}.\footnote{In the two first cases, it is hard to specify which form is the derived and which is the underived form since both verbs have an expansion morpheme, but there is no monosyllabic form without a derivation morpheme.}


\eabox{\label{wointr}\begin{tabular}{@{}lllll@{}}
sɔ̀-lɛ & `hide sth.' & >  & swà-wɔ & `hide (v.i.)' \\
tɛ́-lɛ & `place sth.' & >  & tɛ́-bɔ & `rise' \\
láà & `tell sth.' & >  & là-wɔ & `speak' \\
\end{tabular}}


\largerpage[2]
\subsection{Zero-derivation}
\label{sec:ZeroDer}

Zero-derivation is found in only a few domains.
Almost all postpositions are zero-derived from nouns, as shown in \tabref{Tab:ZeroDeriv1}.\footnote{The only unclear case is the postposition {\itshape dé} `in' for which a possible nominal source is synchronically not known.} Postpositions and their source noun do not differ in form, but in their morphosyntactic behavior and distribution, as explained in \sectref{sec:LOCgen}.

\begin{table}
\begin{tabular}{lll}
\lsptoprule
Lexeme & Postposition & Nominal source \\  \midrule
sí & `under, down' & `ground' \\
 dyúwɔ̀ &  `up, on top' & `sky' \\
tɛ́mɔ́ &  `between' & `middle' \\
písɛ̀ &   `behind' & `behind, back (n.)' \\
sɔ̂ & `in front, before' & `front (n.)' \\
\lspbottomrule
\end{tabular}
\caption{Derivation of postpositions}
\label{Tab:ZeroDeriv1}
\end{table}

In the absence of any derivational marking, one might object that it is difficult to pinpoint the grammaticalization path from noun to postposition or vice versa. The phenomenon that locative adpositions are derived from body-part and environmental landmark nouns, however, has been observed by, for instance, \citet[215]{kiessling2008} for African languages and \citet{bowden1992} for Oceanic languages. It is rather noteworthy that, in Gyeli, these expressions are grammaticalized as postpositions instead of prepositions, as would be expected for Bantu languages \citep{dryer2013b}.

\hspace*{-1mm}Another potential case of zero-derivation includes the quantifier {\itshape bvùbvù} `many' and its nominal counterpart {\itshape bvúbvù} `multitude' (cl. 9). In this case, however, there is a difference in the tonal pattern. Since this is the only example, it is not clear, however, if the tonal difference marks derivation or happened by chance. It is further not clear whether the noun is the source or the derived form.









\subsection{Compounding}
\label{sec:Compound}

In comparison to derivation, compounding is a less productive word formation strategy.
Gyeli has two types of compound nouns which differ in their derivation source and complexity. Most compounds are formed from a nominalized verb and its nominal complement. A few compounds are derived from two underived nouns. Both types are discussed in the following sections.

\subsubsection{Deverbal noun-noun compounds}
\label{sec:VNCompound}


The most productive type of compounding is comprised of a nominalized verbal root and a noun, as illustrated in \REF{CompoundTemp}. Most nominal compounds semantically designate an agent, as shown in \REF{CompoundVN1a}. Accordingly, the verbal root is nominalized as a deverbal noun of gender 1/2, as described in \sectref{sec:NOM12}.\footnote{A more detailed discussion of compounding in Bantu, especially in Bemba, is provided in \citet{basciano2011}.}

\ea \label{CompoundTemp} [N\textsubscript{deverbal} + N]\textsubscript{N}
\z

The noun that follows the nominalized verb is the verb's direct argument that cannot be omitted, as the nominalized verb of these constructions on its own is ungrammatical. The complement noun, however, is ``not necessarily [an object] in the traditional syntactic sense'' \citep{schadeberg2003}. The tonal pattern of a deverbal compound, as illustrated in \REF{CompoundVN1a}, differs from the patterns found in a verb phrase between verb and object, as discussed in \sectref{sec:SynH} and \sectref{sec:HLinker}. In a VP, the noun class prefix of the nominal argument takes an object-linking H tone and the final vowel of the verb takes an H tone in realis categories.  In compounds, all these TBUs surface with an L tone.

\ea \label{CompoundVN1a}
\ea  mbòmɛ̀-{\bfseries mà}pɔ̂ `messenger' \\ < bòmɛ `bark, announce' + ma-pɔ̂ `news'
\ex ntsíɛ̀-{\bfseries bè}nyàgà `butcher' \\ < tsíɛ̀ `cut' + be-nyàgà `cows'
\ex nlɔ́lɛ̀-{\bfseries mì}nkɔ̀lɛ́  `weaver, tailor' \\ <  lɔ̂ `sew, weave' + mi-nkɔ̀lɛ́ `threads'
\ex ngyàgɛ̀sɛ̀-{\bfseries bè}sâ `vendor, merchant' \\ < gyàg-ɛsɛ `make buy' + be-sâ `things'
\ex mbwálɛ̀sɛ̀-{\bfseries b}ùdì `midwife' \\ < bwà(l)-ɛsɛ `make give birth' + b-ùdì `people'
\ex nlímbɔ́-{\bfseries mà}mbɔ̀ `connoisseur, educated person' \\ <  límbo `know' + ma-mbɔ̀ `things'
\ex nsálɛ̀-{\bfseries mà}nkɛ̃̂ `farmer' < sá-lɛ `do (v.t.)' + ma-nkɛ̃̂ `fields'
\z
\z

The tonal difference between objects in a VP and complement nouns in a compound can be explained by the compounds' lexicalization history. Rather than stemming from a nominalized VP, these compounds have their origin in a noun + noun attributive construction, as discussed in \sectref{sec:CONC}, whose first constituent is a deverbal agentive noun. This is in line with \citet[87]{schadeberg2003} who points out that compound ``nouns may originate from a genitival (connective) [attributive] construction,'' which then becomes lexicalized as a noun, as shown in \REF{CompoundVN}.

\ea \label{CompoundVN}
\glll   °mbòmɛ̀ wà màpɔ̂  > mbòmɛ̀ $\emptyset$ màpɔ̂  >  mbòmɛ̀-màpɔ̂  \\
m-bòmɛ̀ wà ma-pɔ̂ \\
	\textsc{n}1-announce 1:{\ATT} ma6-news \\
\glt `messenger [lit. announcer of news $\rightarrow$ news-announcer]'
\z

\noindent Even in many synchronic attributive constructions, the attributive marker can optionally be omitted, as discussed in \sectref{sec:CONOM}. In deverbal compounds, the omission of the attributive marker is no longer optional, but has become lexicalized. This lexicalization path explains why the prefix of the complement has an L tone rather than an object-linking H tone. Since the preceding attributive marker {\itshape wà} has an L tone, the following prefix surfaces as L as well (in contrast to the plural version shown in \REF{CompoundVN1b}).   Another piece of evidence for lexicalization from an attributive construction comes from the plural formation of these compounds explained below.


There are two types of  compounds, which differ in the number value of the argument nominal. In \REF{CompoundVN1a}, all argument nouns are plural, marked by the plural noun class prefixes in bold.
The number of the argument nominal has an impact on the plural formation of the compound noun. If the argument noun has a plural prefix, as in \REF{CompoundVN1a}, its plural counterpart does not constitute a compound noun, but a noun + noun attributive construction. \REF{CompoundVN1b} shows the plural forms of the examples in \REF{CompoundVN1a}. They are comprised of the plural nominalized verb, the plural argument noun and an attributive marker agreeing with the first noun that links the two constituents.\footnote{I represent the noun class prefix of the nominalized verb as toneless which will take its surface tone from its syntactic environment. While the CV- noun class prefix of the second constituent is underlyingly toneless as well, it surfaces with an H tone which it acquires through high tone spreading from the preceding attributive marker.}

\ea \label{CompoundVN1b}
\ea  ba-bòmɛ̀ bá má-pɔ̂ `messengers'
\ex ba-tsíɛ̀ bá bé-nyàgà `butchers'
\ex ba-lɔ́lɛ̀ bá mí-nkɔ́lɛ̀ `weavers, tailors'
\ex ba-gyàgɛ̀sɛ̀ bá bé-sâ `vendors, merchants'
\ex ba-bwálɛ̀sɛ̀ bá b-ùdì `midwives'
\ex ba-límbɔ́ bá má-mbɔ̀ `connoisseurs, educated people'
\ex ba-sálɛ̀ bá má-nkɛ̃̂ `farmers'
\z
\z

\noindent The structural difference between singular compound nouns and their non-com\-pound plural counterparts is due to their different stages in lexicalization. As described in \sectref{sec:CONOM}, attributive markers can be omitted from noun + noun constructions under certain morphophonological and semantic conditions. Two plural noun constituents and a CV- shape noun class prefix on the second constituent, however, inhibit the omission of the attributive marker,  explaining why the singular form is more lexicalized than its plural counterpart.

The second and less frequent type of deverbal compounds has a singular or transnumeral argument noun, as illustrated in \REF{CompoundVN2a}.

\ea \label{CompoundVN2a}
\ea  nkẽ̀-nlô `gecko'\footnote{It is believed that geckos eat people's hair while they are sleeping.} \\ < kẽ̀ `shave' + nlô `head'
\ex mbúlɔ̀-mã̂ `fisherman' \\ < búlɔ `fish (v.)' + mã̂ `sea'
\z
\z

\noindent In these cases, the plural counterpart remains a compound as well, as shown in \REF{CompoundVN2b}. Rather than transforming into a noun + noun attributive construction, the compound only takes a plural noun class prefix for the nominalized verb while the second constituent remains unchanged. It thus appears that compounds with singular second constituents are more lexicalized than those with plural second constituents.

\ea \label{CompoundVN2b}
\ea  ba-nkẽ̀-nlô `geckos'
\ex ba-búlɔ̀-mã̂ `fishermen'
\z
\z

As mentioned above, most compounds of the [VN] type constitute agent nouns. The only exception to this pattern I found is given in \REF{CompoundVN3}. Though it is still in gender 1/2, it lacks the nasal prefix in the singular.

\ea \label{CompoundVN3} tsíɛ̀-sámɛ̀, ba-tsíɛ̀-sámɛ̀ `circumcision' \\ < tsíɛ̀ `cut' + nsámbɔ̀ `penis'
\z

\noindent Having a singular second constituent, the plural form remains a compound noun. The phonologically changed form of the argument nominal suggests that this compound is further along the lexicalization path.

\subsubsection{Underived noun-noun compounds}
\label{sec:NNCompound}

The second category of nominal compounds take the structure of noun + noun compounds. They differ from deverbal compounds in that their constituents are not derived.  The most common lexical items involved in [NN] compounds include {\itshape mwánɔ̀} `child'  as a diminutive marker, as shown in \REF{CompoundNN1}. Semantically, the diminutive can refer both to the small size of a referent or a small amount.

\ea \label{CompoundNN1}
\ea  mwánɔ̀-mùdã̂ `girl' mwánɔ̀ `child' + mùdã̂ `woman'
\ex mwánɔ̀-mùdũ̂ `boy' < mwánɔ̀ `child' + mùdũ̂ `man'
\ex mwánɔ̀-nlàwɔ́ `twig' < mwánɔ̀ `child' + nlàwɔ́ `branch'
\ex mwánɔ̀-sâ `little something' < mwánɔ̀ `child' + sâ `thing'
\z
\z

Pluralization of such compounds requires both constituents to occur in their plural form, as shown in \REF{CompoundNN1a}.

\ea \label{CompoundNN1a}
\ea  bwánɔ̀-bùdã̂ `girls'
\ex bwánɔ̀-bùdũ̂ `boys'
\ex bwánɔ̀-mìnlàwɔ́ `twigs'
\ex bwánɔ̀-besâ `little things' 
\z
\z

In diminutive compounds, the second constituent serves as the syntactic and semantic head. As such, agreement targets agree with the second constituent and not with the first, as shown in \REF{CompoundNNAGR}.

\ea \label{CompoundNNAGR}
\ea \label{CompoundNNAGR1}
 \glll  bwánɔ̀-békúmbé {\bfseries bé} bà njí nà {\bfseries byɔ̂} {\bfseries bé} tɛ́lɛ́ mà{\bfseries bé}  \\
         b-wánɔ̀-be-kúmbé bé ba njì-H nà by-ɔ̂ be-H tɛ́lɛ-H mà-bé \\
          ba2-child-be8-tin 8:{\ATT} 2.{\PST}1 come-{\R} {\COM} 8-{\OBJ} 8-\textsc{prs} stand-{\R} here-8   \\
    \trans `The few tin roofs that they brought stand here.'
\ex[*]{\label{CompoundNNAGR2}
 \glll  bwánɔ̀-békúmbé {\bfseries bá} bà njí nà {\bfseries bɔ̂} {\bfseries bá} tɛ́lɛ́ mà{\bfseries bá}  \\
         b-wánɔ̀-be-kúmbé bá ba njì-H nà b-ɔ̂ ba-H tɛ́lɛ-H mà-bá \\
         ba2-child-be8-tin 2:{\ATT} 2.{\PST}1 come-{\R} {\COM} 2-{\OBJ} 2-\textsc{prs} stand-{\R} here-2   \\
    \trans `The few tin roofs that they brought stand here.'}
\z
\z


Underived noun-noun compounds other than diminutives seem to describe an inherent property, such as gender or size, as shown in \REF{CompoundNN2}. As with deverbal [NN] compounds, these compounds appear to originate in attributive constructions.

\ea \label{CompoundNN2}
\ea  sɔ́-mùdã̂ `female friend' < sɔ́ `friend' + mùdã̂ `woman'
\ex kfúbɔ̀-dyá `tall chicken' < kfúbɔ̀ `chicken' + dyá `length'
\z
\z

\noindent There seems to be a lexicalization scale from attributive constructions which require the attributive marker, as described in \sectref{sec:CONC}, those which optionally omit the attributive marker, and finally those constructions which have lexicalized separately as compounds without the attributive marker, as in \REF{CompoundNN2}. \REF{CompoundNN3} shows corresponding attributive constructions. I only view the latter type as compounds. Since examples with such a meaning contrast are hard to find, examples of these compounds are few in number.

\ea\label {CompoundNN3}
\ea  \gll
sɔ́ wà m-ùdã̂  \\ 
$\emptyset$1.friend 1:{\ATT} \textsc{n}1-woman \\
\trans `the friend of the woman'
\ex
\gll kfúbɔ̀ wà dyá \\
   $\emptyset$1.chicken 1:{\ATT} $\emptyset$1.length \\
\trans `the remote chicken'
\z
\z

Impressionistically, [NN] compounds in \REF{CompoundNN1} differ structurally from the diminutive compounds in \REF{CompoundNN2} with respect to their headedness. In the diminutives, the semantic and syntactic head is the second constituent, while in the other compounds, the first constituent functions as the head. The left-headed pattern might be expected from the compounds' origin in the noun + noun attributive construction. Given the limitation of examples, it is not possible at this point to explain how diminutives developed to be right-headed.


I conclude this chapter with a note on another derivation type common across Bantu languages, namely noun-to-noun derivation. As \citet[82]{schadeberg2003} describes, noun-to-noun derivation is commonly achieved by shifting nouns to different genders. I have not  observed this in my Gyeli data. Instead, Gyeli has different lexical stems or diminutive compounds with {\itshape mwánɔ̀} `child' {\sectref{sec:NomQUANT}} to encode size differences that may be expressed by different genders in other Bantu languages.
