\chapter*{Acknowledgments}
\label{sec:acknowledgments}
\addcontentsline{toc}{chapter}{Acknowledgments}

This grammar would not have been possible without the many Gyeli speakers I have worked with over the years and who patiently taught me about their language. I am especially grateful to the people of Ngolo, and in particular to Mama David, Ada Joseph, Mambi, Nandtoungou, Nze, Tsimbo, Nkolo Dorothée, Segyua, ``Délégué'' Bikoun (\textdagger), Tata, Pfunda, Mimbeh, and Aminu.

Thanks also to my Kwasio assistants and friends Bimbvoung Emmanuel Calvin, Djiedjhie François, and Nouangama Severin who not only helped with interpreting, translations, and annotations, but who also made my life in the field so much easier and more enjoyable. Thanks for keeping me safe and taking care of me when I was sick with malaria or {\itshape chikungunya} or after road accidents.
I am also particularly grateful to my fellow team members Daniel Duke and Emmanuel Ngue Um and our cameraman Christopher Lorenz.

This grammar started out as my PhD project at the Institute for Asian and African Studies at Humboldt Universität zu Berlin. I would like to thank my advisors Tom Güldemann and Maarten Mous for their helpful feedback throughout the course of writing my dissertation and beyond when revising it for publication.  I have discussed many aspects of this grammar with various people over the last years. I particularly thank Viktoria Apel, Pierpaolo Di Carlo, Bernard Comrie, Ines Fiedler, Hana Filip, Jeff Good, Larry Hyman, Lutz Marten, Joyce McDonough, and Murray Schellenberg, as well as my ``academic homes'', my departments in Berlin and Rochester. 

Over the years, I have received several grants that enabled me to work with the Bagyeli.  The DoBeS (Documentation of Endangered Languages) grant 84976 and a generous extension phase 87014  by the VolkswagenFoundation financed my PhD position and fieldwork.  I am grateful for the opportunity the grants gave me and for all the assistance, especially by Vera Szöllosi-Brenig. A special note of thanks goes to Paul Trilsbeek who has been providing continuous assistance in archiving the Gyeli data.  After my PhD, Jürgen Bohnemeyer invited me to collaborate in his NSF \#1535846 project ``Causality across languages'' (2015--2022), which funded further fieldwork in 2017, for which I am grateful.

I couldn't have had a better experience while publishing the grammar with Language Science Press. I thank Martin Haspelmath and nine anonymous reviewers for their constructive and kind comments, which certainly improved the quality of the description, as well as the proofreaders for their valuable time. It was a pleasure to work with Sebastian Nordhoff and Felix Kopecky, who always provided prompt and efficient support with technical and typesetting matters. Sebastian also did an outstanding job reworking all the maps in \chapref{sec:Introduction}. 

Last, but not least, I am very grateful to my family and friends who supported me in the field and took active interest in all the news I brought from Cameroon. Special thanks to my wonderful husband Scott for his patience with the long absences that fieldwork makes necessary, for sharing my excitement and worries, and for proofreading various versions of this grammar.
