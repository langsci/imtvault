

% \renewcommand{\thechapter}{\Roman{chapter}}
%
% \setcounter{chapter}{2}
%
% \setcounter{section}{0}

% \chapter*{Appendix II: Texts} \addcontentsline{toc}{chapter}{Appendix II: Texts}
\chapter{Texts}
\label{sec:AppendixII}
\label{sec:AppII}
% \chaptermark{Appendix II: Texts}


This appendix contains the annotated Gyeli text corpus which is comprised of three texts of different genres. The first one, {\itshape The healer and the antelope}, is an autobiographical narration; the second one, the {\itshape Nzambi Story}, a folktale; and the third a conversation with multiple speakers in the village Ngolo. 

Each text is split up into intonation phrases. Since intonation phrases are not always clear-cut, especially in fast natural speech, I relied on two principles in determining intonation phrases: pauses and speaker intuition. As a first parsing principle, I took pauses as indications for intonation phrases. Later on, text annotation was done with the help of a language consultant who would naturally break the text up into phrases as he repeated the recordings during transcription.
Intonation phrases do not always match grammatical sentences. 

Each intonation phrase has four annotation lines. The first represents the surface form on the word level. The second line shows the underlying form on the morpheme level, including tonal changes. All vowels are marked for tone. Tone bearing units without tonal marking in the second line are underlyingly toneless. Transcription lines do not contain punctuation marks as those are conventions for written, but not natural, spoken language. The third line is the gloss and the fourth the translation. Code-switching to, for instance, Kwasio or French, is indicated in the gloss line with the language name in square brackets for non-Gyeli elements. If a whole phrase is in a language other than Gyeli, for instance in Bulu, only the surface form is indicated, but not the underlying form. Square brackets in the translation line serve as explanations.





\section{The healer and the antelope}
\label{sec:Antelope}

The story about the healer who turned into an antelope is an autobiographical narrative by Ada Joseph, who was about 30 years old at the time of recording. The narrative was recorded in May 2011 in Nziou, a village close to Kribi. This anecdote came up during lunchtime small talk with the Mabi speakers Djiedjhie François and Bimbvoung Emmanuel Calvin, and me.  Ada agreed to tell it again for the recording. 

The narrative is about an old man that Ada knew from his village when he was a teenager. This man was a healer and became sick himself. Since he did not want to die, he turned himself into an antelope and fled into the forest. The villagers were worried about this and tried to kill the  antelope, but they never found it.


\begin{exe}[(A34)]
\exA\label{a01}
  \glll    yɔ́ɔ̀ yá táàlɛ̀  \\
          yɔ́ɔ̀ ya-H táàlɛ̀  \\
            so 1\textsc{pl}-\textsc{prs} begin     \\
    \trans `So, we begin.'
\exA\label{a02}
  \glll   yɔ́ɔ̀ ngã̀ nû à bɛ́ ngã̂  \\
          yɔ́ɔ̀ ngã̀ nû a bɛ̀-H ngã̂ \\
         so $\emptyset$1.healer 1.{\DEM}.{\PROX} 1.{\PST}1 be-{\R} $\emptyset$1.healer   \\
    \trans `So, this healer was a healer.'
\exA\label{a03}
  \glll  ngã̂    \\
          ngã̂   \\
          $\emptyset$1.healer       \\
    \trans `A healer.'
\exA\label{a04}
  \glll  à jìlɛ́ mâ   \\
          a jìlɛ-H mâ   \\
         1.{\PST}1 stay-{\R}  {\COMPL}       \\
    \trans `He was there.'
\exA\label{a05}
  \glll  à njâ dyùmɔ́ bùdàà dyùmɔ́ bùdàà dyùmɔ́ bùdàà dyùmɔ́ bùdì  \\
           a nji-H a dyùmɔ-H b-ùdì a dyùmɔ-H  b-ùdì a dyùmɔ-h b-ùdì   \\
          1.{\PST}1 come-{\R} 1.{\PST}1 heal-{\R} ba2-person 1.{\PST}1 heal-{\R} ba2-person 1.{\PST}1 heal-{\R} ba2-person  \\
    \trans `He came, he was healing people (4x).'
\exA\label{a05a}
  \glll  à múà mɛ́dɛ́ nyá mùdì \\
          a múà mɛ́dɛ́ nyá m-ùdì   \\
         1 be.almost self real {\N}1-person    \\
    \trans `He was himself a real [old] man.'
 
\exA\label{a06} 
  \glll  à dyùmɔ́ bùdì à dyùmɔ́ bùdì à múà mɛ́dɛ́ nyá mùdì pɔ́nɛ́ ntúlɛ́    \\
           a dyùmɔ-H b-ùdì a dyùmɔ-H b-ùdì a múà    mɛ́dɛ́ nyá m-ùdì pɔ́nɛ́ ntúlɛ́  \\
           1.{\PST}1 heal-{\R} ba2-person 1.{\PST}1 heal-{\R} ba2-person 1.{\PST}1 be.almost self real {\N}1-person $\emptyset$7.truth $\emptyset$3.old \\
    \trans `He was healing people, he was healing people. He himself was an old man.'
 
\exA\label{a07}
  \glll  nyɛ̀ táàlɛ́ bábɛ̀    \\
          nyɛ táàlɛ-H bábɛ̀  \\
          1.{\PST}1 begin-{\R} $\emptyset$7.illness       \\
    \trans `He started to be sick.'
 
\exA\label{a08} 
  \glll  gbĩ́-gbĩ̀-gbĩ́-gbĩ̀-gbĩ́  à múà nà bábɛ̀ tí wúmbɛ̀ wɛ̀\\
            gbĩ́-gbĩ̀-gbĩ́-gbĩ̀-gbĩ́  a múà nà bábɛ̀ tí wúmbɛ wɛ̀\\
         {\IDEO}:roaming 1.{\PST} be.almost {\COM} $\emptyset$7.illness {\NEG} want die\\
    \trans `[imitation of the disease roaming in his body] He was about to be sick, not wanting to die.'
 
\exA\label{a09}
  \glll   bá sàgà ɛ́ kfùmàlà mɛ̀ múà ndáà mùdì   \\
         ba-H sàga ɛ́ kfùmala mɛ múà ndáà m-ùdì\\
         2-\textsc{prs} be.surprised {\LOC} find 1\textsc{sg}.{\SBJ}  be.almost also {\N}1-person\\
    \trans `They are surprised to find that I was a grown up person [says the storyteller about his own age at the point when the story took place].'
 
\exA\label{a10}
  \glll mɛ̀ múà pɔ́nɛ́ wá yìmbá ntɛ́ wû  \\
          mɛ múà pɔ́nɛ́ wá yìmbá ntɛ́ wû       \\
         1\textsc{sg} be.almost $\emptyset$7.truth 3:{\ATT}  $\emptyset$7.age $\emptyset$3.size there \\
    \trans `I was really about the age of this size there [makes a gesture with hand showing his height].'
 
\exA\label{a11}
  \gll   álè   \\
           go.{\IMP}.{\PL}       \\
    \trans `[French] so. . .'
 
\exA\label{a12}
  \glll   yá sàgà àà ndáwɔ̀ dé tù nyɛ̀ mɛ́dɛ́ támé   \\
           ya-H sàga àà ndáwɔ̀ dé tù nyɛ mɛ́dɛ́ támé      \\
          1\textsc{pl}-\textsc{prs} be.surprised 1.{\COP} $\emptyset$9.house {\LOC} inside 1.{\SBJ} self alone \\
    \trans `We are surprised. He is in his house all by himself.'
 
\exA\label{a13}
  \glll   ḿh-m̀h-m̀-m̀-ḿh   \\
          ḿh-m̀h-m̀-m̀-ḿh   \\
          {\IDEO}:self.talk \\
    \trans `[imitation of healer's self talk and noises he makes in the house].'
 
\exA\label{a14} 
  \glll   yá sàgà mɛ́nɔ́ wɛ̂ nyɛ́ɛ̀ mápà má njìbù má bwámɔ́ ndáwɔ̀ dé tù \\
          ya-H sàga mɛ́nɔ́ wɛ̂ nyɛ́ɛ̀ H-ma-pà má njìbù ma-H bwámɔ-H ndáwɔ̀ dé tù     \\
        1\textsc{pl}-\textsc{prs} be.surprised $\emptyset$7.morning in see.{\SBJV}  {\OBJ}.{\LINK}-ma6-paw 6:{\ATT}  $\emptyset$1.antelope 6-\textsc{prs} come.out-{\R} $\emptyset$9.house {\LOC} inside  \\
    \trans `We are surprised in the morning to see [hoof] traces of an  antelope which come out of the house,'
 
\exA\label{a15}
  \glll   kɛ̀ dɛ́ndì \\
         kɛ̀ d-ɛ́ndì \\
          go le5-courtyard \\
    \trans `going into the courtyard.'
 
\exA\label{a16}
  \glll  kɛ̀ dígɛ̀ mpù \\
       kɛ̀ dígɛ mpù \\
          go look like.this       \\
    \trans `Going looking like this,'
 
\exA\label{a17}
  \glll  mùdì nú bɛ́lɛ́ \\
        m-ùdì nú bɛ́-lɛ́ \\
         {\N}1-person 1.{\DEM}.{\DIST} be-{\NEG} \\
    \trans `nobody is there.'
 
\exA\label{a18}
  \glll  ndùù à vìdégáà njìbù   \\
          ndùù a vìdeg-áà njìbù \\
           so[French]  1.{\PST}1 turn-{\PRF} $\emptyset$1.antelope    \\
    \trans `So, he has already turned into an antelope.'
 
\exA\label{a19}
  \glll  à múà á kɛ́ jìí dé tù    \\
          a múà a-H kɛ̀-H jìí dé tù       \\
       1.{\PST} be.almost 1-\textsc{prs} go-{\R} $\emptyset$7.forest {\LOC} inside \\
    \trans `He was about to go into the forest,'
 
\exA\label{a20}
  \glll   nà ndzǐ gyâ   \\
          nà ndzǐ gyâ   \\
           {\COM} $\emptyset$9.path $\emptyset$7.length \\
    \trans `on the long path.'
 
\exA\label{a21}
  \glll   kɛ́ jìí dé tù \\
         kɛ̀-H jìí dé tù  \\
         go-{\R} $\emptyset$7.forest {\LOC} inside        \\
    \trans `[He] goes into the forest,'
 
\exA\label{a22}
  \glll  nà ndzǐ gyâ    \\
      nà ndzǐ gyâ   \\
          {\COM} $\emptyset$9.path $\emptyset$7.length       \\
    \trans `on the long path.'
 
\exA\label{a23}
  \glll   kɛ́ jìí dé tù  \\
         kɛ̀-H jìí dé tù  \\
         go-{\R} $\emptyset$7.forest {\LOC} inside        \\
    \trans `[He] goes into the forest,'

\exA\label{a24}
  \glll   nà ndzǐ gyâ \\
          nà ndzǐ gyâ \\
      {\COM} $\emptyset$9.path $\emptyset$7.length   \\
    \trans `on the long path.'
 
\exA\label{a24a}
  \glll   nà pándɛ̀ vâ bùdì báà bɛ̀ \\
          nà pándɛ̀ vâ b-ùdì báà bɛ \\
      {\CONJ} arrive here ba2-person 2.{\COP} there[Kwasio]   \\
    \trans `And having arrived here, these people are there.'
 
\exA\label{a24b}
  \glll   á lèmbó nâ bùdì báà bá múà búɛ̀lɛ̀ nâ bá dyúù nyɛ̂  \\
          a-H lèmbo-H nâ b-ùdì báà ba-H múà búɛlɛ nâ ba-H dyùù nyɛ  \\
1-\textsc{prs} know-{\R} {\COMP} ba2-person 2.{\DEM}.{\PROX} 2-\textsc{prs} be.almost fish {\COMP} 2-\textsc{prs} kill.{\SBJV}  1.{\OBJ}   \\
    \trans `He knows that these people are about to look [lit. fish] for him in order to kill him.'
 
\exA\label{a25}
  \glll   nâ bá dyúù nyɛ̂ vɛ̀ɛ̀ mùdì nyɛ̀ jã́ã̀sà   \\
           nâ ba-H dyùù nyɛ̂ vɛ̀ɛ̀ m-ùdì nyɛ̂ jã́ã̀sà      \\
         {\COMP} 2-\textsc{prs} kill.{\SBJV}  1.{\OBJ} only {\N}1-person 1.{\SBJ}  disappear \\
    \trans `So that they kill him. Suddenly the person [the healer] disappears,'
 
\exA\label{a26}
  \glll    nà kɛ́ jìí dé tù nà ndzǐ pámò dẽ̂  \\
          nà kɛ̀-H jìí dé tù nà ndzǐ pámò dẽ̂   \\
         {\CONJ} go-{\R} $\emptyset$7.forest {\LOC} inside {\COM} $\emptyset$9.path arrive today \\
    \trans `and [he] goes in the forest on the path till today,'
 
\exA\label{a27}
  \glll  tí nyɛ̂ nyɛ̂ \\
        tí nyɛ̂ nyɛ̂\\
           {\NEG} see 1.{\OBJ}\\ 
    \trans `without being seen.'
\end{exe}



\pagebreak


\section{Nzambi story}
\label{sec:Nzambi}

The {\itshape Nzambi Story} is a well-known folktale among the Bagyeli. It was recorded on video in August 2012 in the Gyeli village Ngolo.  Tata is the main narrator. He stood in the middle of the village under the big tree, while the rest of the village is gathered around him and commented on both the story and the recording.

The folktale is about two friends, both called {\itshape Nzambi}, which means `God'. One of them grows breadfruit, the other palm nuts. The Nzambi growing breadfruit marries the daughter of his friend and they have a child. When the palm trees are not producing enough fruit, the family suffers hunger. Therefore, Nzambi of the palm nuts sends his wife to the his friend, Nzambi of the breadfruit, to ask for food. Nzambi of the breadfruit agrees to give food to the wife, but keeps their child in return and eats it. When Nzambi of the palm nuts learns about this, he goes to see his friend and ask him why he did this. The breadfruit grower admits that he ate the child and pretends that he also ate his own children by showing him monkey skulls. He then suggests that the palm nut grower should also eat his children as this would turn them into white people. Instead of heeding this advice, the palm nut grower takes revenge on his friend by locking the breadfruit grower's family in a house, which he then burns down. He then has mice eat the remains of the burned bodies. When the breadfruit grower Nzambi returns home and finds his whole family dead, he is devastated.

\noindent Tata:

\begin{exe}[(N234)]
\exN\label{n01}
  \glll    jíyɔ̀ \\
           jíyɔ \\
             sit.down \\
    \trans `Sit down [introductory words to a story].'
 
\exN\label{n02}
  \glll     yɔ́ɔ̀ Nzàmbí núù jì  \\
           yɔ́ɔ̀ Nzàmbí núù jì. \\
              so $\emptyset$1.{\PN} 1.{\DEM}.{\PROX} sit   \\
    \trans `So, there is this [person called] Nzambi.'
 
\exN\label{n03}
  \glll   Nzàmbí jìlɛ́ mà \\
         Nzàmbí jìlɛ-H mà \\
            $\emptyset$1.{\PN} sit-{\R} {\COMPL}[Kwasio]     \\
    \trans `Nzambi is already there.'
\end{exe}

\noindent Aminu to cameraman:

\begin{exe}[(N234)]
\exN\label{n04} 
  \glll   wɛ̀ nzíí bàlɛ̀ bébã́ã̀ \\
      wɛ nzíí bàlɛ H-be-bã́ã̀ \\
            2\textsc{sg} {\PROG}.\textsc{prs} keep {\OBJ}.{\LINK}-be8-word \\
    \trans `You are recording [lit. keeping the words].'
\end{exe}

\noindent Tata:

\begin{exe}[(N234)] 
\exN\label{n05} 
  \glll  wɛ̀ nzíí bàlɛ̀ mpà \\
       wɛ nzíí bàlɛ mpà \\
           2\textsc{sg} {\PROG}.\textsc{prs} keep good\\
    \trans `You [addressing cameraman] are recording well.'
     
\exN\label{n06}
  \glll  yɔ́ɔ̀ Nzàmbí núù jì  \\
         yɔ́ɔ̀ Nzàmbí núù jì \\
          so $\emptyset$1.{\PN} 1.{\DEM}.{\PROX} sit \\
    \trans `So there is this [person called] Nzambi.'
\end{exe}

\noindent Aminu:

\begin{exe}[(N234)]
\exN\label{n07}
  \glll  bwáá lã́ bɔ̂  \\
         bwáa-H lã-H b-ɔ̂ \\
            2\textsc{pl}-\textsc{prs} tell-{\R} 2-{\OBJ}     \\
    \trans `You tell them!'
\end{exe}

\noindent Tata:

\begin{exe}[(N234)] 
\exN\label{n08}
  \glll  Nzàmbí jĩ́ĩ̀ à lwɔ́ mɔ̀ kwádó \\
         Nzàmbí jĩ́ĩ̀ a lwɔ̂-H mɔ̀ kwádó \\
             $\emptyset$1.{\PN} sit.{\COMPL} 1.{\PST}1  build-{\R} {\COMPL} $\emptyset$7.village    \\
    \trans `Nzambi is there, he has already built a village,'
 
\exN\label{n09}
  \glll  bá nà mùdã̂ wɛ̂ \\
         bá nà m-ùdã̂ w-ɛ̂\\
           2.{\SBJ}  {\COM} 1-woman 1-{\POSS}.3\textsc{sg}  \\
    \trans `they [him] and his wife.'
 
\exN\label{n10}
  \glll  bàNzàmbí bábáà \\
        ba-Nzàmbí bá-báà, \\
           ba2-{\PN} 2-two  \\
    \trans `Two Nzambis,'
 
\exN\label{n11}
  \glll  nɔ́gá gyã́ã̀ nkɛ̀ nɔ́gá gyã́ã̀ mbyɛ̂ \\
         nɔ́-gá gyã́ã̀ nkɛ̀ nɔ́-gá gyã́ã̀ mbyɛ̂ \\
           1-other $\emptyset$1.side $\emptyset$3.low 1-other $\emptyset$1.side $\emptyset$3.high  \\
    \trans `one downstream, the other upstream.'
 
\exN\label{n12}
  \glll  ɛ́ mpù bá kí nâ jíwɔ́ mbyɛ̂ nà jíwɔ́ nkɛ̀\\
         ɛ́ mpù ba-H ki-H nâ jíwɔ́ mbyɛ̂ nà jíwɔ́ nkɛ̀\\
           {\LOC} like.this 2-\textsc{prs} say-{\R} {\COMP} $\emptyset$7.river  $\emptyset$3.high {\CONJ} $\emptyset$7.river $\emptyset$3.low\\
    \trans `Like this, they say that up the river and down the river. . .'
 
\exN\label{n13} 
  \glll  yɔ́ɔ̀ bàNzàmbí bá tè bá jì  \\
         yɔ́ɔ̀ ba-Nzàmbí bá tè ba-H jì \\
           so ba2-{\PN} 2:{\ATT}  there 2-\textsc{prs} sit   \\
    \trans `So the Nzambis there are settled.'
 
\exN\label{n14}
  \glll  yɔ́ɔ̀ Nzàmbí nɔ́gá núù bɛ́ Nzàmbí wà gyí \\
        yɔ́ɔ̀ Nzàmbí nɔ́-gá núù bɛ̀-H Nzàmbí wà gyí \\
          so $\emptyset$1.{\PN} 1-other 1.{\PST}2 be-{\R}  {\PN} 1:{\ATT}  what   \\
    \trans `So this other Nzambi was which Nzambi?'
 
\exN\label{n15}
  \glll  mɛ́ líbɛ̀lɛ̀ Nzàmbí  wà lèlɛ́ndí \\
         mɛ-H líbɛlɛ Nzàmbí  wà le-lɛ́ndí \\
           1\textsc{sg}-\textsc{prs} show $\emptyset$1.{\PN} 1:{\ATT}  le5-palm.tree  \\
    \trans `I show [gesture], the Nzambi of the palm tree.'

\exN\label{n16}
  \glll  nɔ́nɛ́gá nyɛ̀gà \\
        nɔ́-nɛ́gá nyɛ̀-gà\\
           1-other 1.{\SBJ}-{\CONTR} \\
    \trans `The other one,'
 
\exN\label{n17}
  \glll  wà lè-bɔ́ɔ̀ \\
        wà le-bɔ́ɔ̀. \\
           1:{\ATT}  le5-breadfruit.tree  \\
    \trans `the one of the breadfruit tree.'
 
\exN\label{n18}
  \glll  yɔ́ɔ̀ bàNzàmbí bá tè bà bwàá sɔ́ \\
         yɔ́ɔ̀ ba-Nzàmbí bá tè ba bwàà-H sɔ́, \\
            so ba2-{\PN} 2:{\ATT}  there 2.{\PST}1 have-{\R} $\emptyset$1.friend  \\
    \trans `So, the Nzambis there became friends,'
 
\exN\label{n19}
  \glll  nâ bá jíì \\
         nâ ba-H jíì \\
           {\COMP} 2-\textsc{prs} sit.{\SBJV}   \\
    \trans `so that they stay,'
 
\exN\label{n20}
  \glll  ɛ́ nû pɛ̀  ɛ́ nû pɛ̀ \\
          ɛ́ nû pɛ̀  ɛ́ nû pɛ̀ \\
            {\LOC} 1.{\DEM}.{\PROX} there {\LOC} 1.{\DEM}.{\PROX} there   \\
    \trans `one there and one there.'
 
\exN\label{n21}
  \glll   bàNzàmbí bá tè bá jìlɛ́ mà \\
          ba-Nzàmbí bá tè ba-H jìlɛ-H mà \\
              ba2-{\PN} 2:{\ATT}  there 2-\textsc{prs} sit-{\R} {\COMPL}[Kwasio]   \\
    \trans `The Nzambis there live there already.'
 
\exN\label{n22}
  \glll  yɔ́ɔ̀ bá kí nâ ɛ́ɛ́kɛ̀ \\
            yɔ́ɔ̀ ba-H ki-H nâ ɛ́ɛ́kɛ̀ \\
         so 2-\textsc{prs} say-{\R} {\COMP} {\EXCL} \\
    \trans `So they say that, ``{\itshape ɛ́ɛ́kɛ̀} [exclamation of surprise]!''.'
 
\exN\label{n23}
  \glll  mwánɔ̀ wɛ̂ mùdã̂ wà nû \\
         m-wánɔ̀ w-ɛ̂ m-ùdã̂ wà nû \\
            {\N}1-child 1-{\POSS}.3\textsc{sg} {\N}1-woman 1:{\ATT}  1.{\DEM}.{\PROX} \\
    \trans `His child [is] the wife of this one [pointing to imaginary breadfruit Nzambi].'
 
\exN\label{n24}
  \glll  à bwã̀ã̀ \\
         a bwã̀ã̀ \\
          1.{\PST}1 give.birth \\
    \trans `She has given birth.'
 
\exN\label{n25}
  \glll nyɛ̀gà váà nyɛ̀gá tsíyɛ́ sâ nà màlɛ́ndí, màlɛ́ndí máà mɔ́gà \\
         nyɛ-gà váà nyɛ-gá tsíyɛ́ sâ nà ma-lɛ́ndí, ma-lɛ́ndí máà m-ɔ́-gà \\
          1.{\SBJ}-{\CONTR} here 1.{\SBJ}-{\CONTR} live-{\R} only {\COM} 6-palm.tree 6-palm.tree 6:DEM 6-{\OBJ}-{\CONTR} \\
    \trans `Him here, he lives only from palm trees, these palm trees.'
 
\exN\label{n26} 
  \glll  má dvúmɔ́lɛ́ mbvú mbì mbvû \\
        ma-H dvúmɔ́-lɛ́ mbvú mbì mbvû \\
           6-\textsc{prs} produce-{\NEG}  $\emptyset$3.year like[Kwasio] $\emptyset$3.year \\
    \trans `They don't produce [fruit] every year,'
 
\exN\label{n27} 
  \glll  màlɛ́ndí máà vɛ̀ɛ̀ kwè mímpìndí  \\
         ma-lɛ́ndí máà vɛ̀ɛ̀ kwè H-mi-mpìndí \\
           ma6-palm.tree 6.{\DEM}.{\PROX} only fall {\OBJ}.{\LINK}-mi4-unripeness \\
    \trans `these palm trees only produce unripe [fruit].'
 
\exN\label{n28}
  \glll  Nzàmbí à bwã̀ã́ mwánɔ̀ \\
          Nzàmbí a bwã̀ã-H m-wánɔ̀ \\
             $\emptyset$1.{\PN} 1.{\PST}1 give.birth-{\R} {\N}1-child \\
    \trans `Nzambi has given birth to a child.'
 
\exN\label{n29}
  \glll  yɔ́ɔ̀ Nzàmbí nyɛ̀gà à kɛ̃́ɛ̃̀ dígɛ̀ mísì \\
         yɔ́ɔ̀ Nzàmbí nyɛ-gà a kɛ̃́ɛ̃̀ dígɛ m-ísì \\
           so $\emptyset$1.{\PN} 1.{\SBJ}-{\CONTR} 1.{\PST}1 go.{\COMPL} watch ma6-eye  \\
    \trans `So this Nzambi went and thought very hard [lit. he watched with his eyes].'
 
\exN\label{n30}
  \glll  nyɛ̀ nâ ɔ́ɔ́ɔ́ mùdã̂ \\
          nyɛ nâ ɔ́ɔ́ɔ́ m-ùdã̂ \\
            1.{\SBJ}  {\COMP} {\EXCL} {\N}1-woman  \\
    \trans `He: ``Oh, wife,'
 
\exN\label{n31}
  \glll  bàmbɛ́ kɛ̂ jíì mbúmbù mwánɔ̀ sá yí dè  \\
        bàmbɛ́ kɛ̂ jíì mbúmbù m-wánɔ̀ sá yí dè \\
           sorry  go.{\IMP}  ask $\emptyset$1.namesake {\N}1-child $\emptyset$7.thing 7:{\ATT}  eat  \\
    \trans `excuse me, go and ask my namesake [the other Nzambi] for a little to eat,'
 
\exN\label{n32} 
  \glll   ɛ́ pɛ̀ nâ á njíyɛ̀ mɛ̂ nà yɔ̂ \\
         ɛ́ pɛ̀ nâ a-H njíyɛ mɛ̀ nà y-ɔ̂ \\
         {\LOC} there {\COMP} 1-\textsc{prs} come.{\SBJV}  1\textsc{sg}.{\OBJ} {\COM} 7-{\OBJ}  \\
    \trans `over there, so that she [his wife] bring me that [food].'
 
\exN\label{n33}
  \glll  mɛ̀ múà wɛ̀ nà nzà \\
        mɛ múà wɛ̀ nà nzà \\
          1\textsc{sg} be.almost die {\COM} $\emptyset$9.hunger  \\
    \trans `I'm about to die from hunger''.'
 
\exN\label{n34} 
  \glll  yɔ́ɔ̀ mùdã̂ nùù tè \\
         yɔ́ɔ̀ m-ùdã̂ nùù tè, \\
          so {\N}1-woman 1.{\COP} there \\
    \trans `So the woman is there [she leaves],'
 
\exN\label{n35}
  \glll  kíyà mwánɔ̀ ndzɛ̀ngɔ̀  \\
        kíya m-wánɔ̀ ndzɛ̀ngɔ̀ \\
          carry {\N}1-child inclined    \\
    \trans `carrying the child on her side [instead of on her back],'
 
\exN\label{n36}
  \glll nkwé nkô \\
         nkwé nkô \\
          $\emptyset$3.basket $\emptyset$3.back \\
    \trans `the basket on the back.'
 
\exN\label{n37}
  \glll  wɔ́ɔ́ɔ́ɔ́ɔ́ɔ́ gbĩ̀m  \\
           wɔ́ɔ́ɔ́ɔ́ɔ́ɔ́ gbĩ̀m  \\
            {\IDEO}:moving {\IDEO}:surface.impact  \\
    \trans `[depiction of moving by foot or motorbike and imitating sound of putting basket down]'
 
\exN\label{n38}
  \glll  áá gyí wɛ́ lɔ́ njì gyɛ́sɔ̀ \\
        áá gyí wɛ-H lɔ́ njì gyɛ́sɔ \\
           {\EXCL} what 2\textsc{sg}-\textsc{prs} {\RETRO}  come look.for  \\
    \trans `[Breadfruit Nzambi talking] Ah, what have you just come to look for?'
 
\exN\label{n39} 
  \glll  nyɛ̀ náà mùdì wã́ã̀ mɛ̀ wɛ̃́ɛ̃̀ nà nzà \\
         nyɛ náà m-ùdì w-ã́ã̀ mɛ wɛ̃́ɛ̃̀ nà nzà \\
           1.{\SBJ}  {\COMP} {\N}1-person 1-{\POSS}.1\textsc{sg} 1\textsc{sg} die.{\COMPL} {\COM} $\emptyset$9.hunger\\
    \trans `She [says]: ``My person, I'm dead hungry.'
 
\exN\label{n40}
  \glll nkɛ̀ nyì nzí sílɛ̃́ɛ̃̀ bédéwò \\
          nkɛ̀ nyi nzí sílɛ̃́ɛ̃̀ H-be-déwò \\
          $\emptyset$9.field 9 {\PROG}.{\PST}  finish.{\COMPL} {\OBJ}.{\LINK}-be8-food   \\
    \trans `The field has already run out of food.'
 
\exN\label{n41}
  \glll bèdéwò bíndɛ̀ byɔ̀ mɛ́ lɔ́ njì lɛ́bɛ̀lɛ̀ bédéwò bà wɛ̂\\
        be-déwò bí-ndɛ̀ by-ɔ̀ mɛ-H lɔ́ njì lɛ́bɛlɛ H-be-déwò bà wɛ̀\\
           be8-food 8-{\ANA} 8-{\OBJ} 1-\textsc{prs} {\RETRO}  come  follow be8-food {\AP} 2\textsc{sg}.{\OBJ}\\
    \trans `This food, I have come to look for the food at your place.'
 
\exN\label{n42}
  \glll  náà ká wɛ̀ múà wáà vɔ́lɛ̀ mɛ̂ \\
        náà ká wɛ múà wáà vɔ́lɛ mɛ̀ \\
          {\COMP} if 2\textsc{sg} be.almost 2\textsc{sg}.{\FUT}[Kwasio] help 1\textsc{sg}.{\OBJ} \\
    \trans `If you can help me. . .'
 
\exN\label{n43}
  \glll  nzà nyíì mɛ̂ mɔ̂ \\
          nzà nyíì mɛ̂ mɔ̂ \\
           $\emptyset$9.hunger 9.{\COP} 1\textsc{sg}.{\OBJ} $\emptyset$3.stomach  \\
    \trans `I'm hungry [lit. I have hunger in my stomach].'
 
\exN\label{n44}
  \glll nágyàlɛ́ wà mùdã̂ \\
        nágyàlɛ́ wà m-ùdã̂ \\
           $\emptyset$1.breastfeeding 1:{\ATT}  {\N}1-woman \\
    \trans `[I am a] breastfeeding woman.'
 
\exN\label{n45}
  \glll yɔ́ɔ̀ mɛ́ lɔ́ njì gyɛ́sɔ̀ sá yí dè \\
       yɔ́ɔ̀ mɛ-H lɔ́ njì gyɛ́sɔ sá yí dè \\
         so 1\textsc{sg}-\textsc{prs} {\RETRO}  come search $\emptyset$7.thing 7:{\ATT}  eat    \\
    \trans `So I just came to look for something to eat''.'
 
\exN\label{n46}
  \glll  yɔ́ɔ̀ Nzàmbí á kí náà ɛ́ɛ̀ \\
         yɔ́ɔ̀ Nzàmbí a-H kì-H náà ɛ́ɛ̀ \\
          so $\emptyset$1.{\PN} 1-\textsc{prs} say-{\R} {\COMP} yes \\
    \trans `So Nzambi says, ``yes,'
 
\exN\label{n47}
  \glll bã́ yɔ́ɔ̀ yíì tè \\
        bã́ y-ɔ́ɔ̀ yíì tè \\
         $\emptyset$7.word 7-{\POSS}.2\textsc{sg}  7.{\COP} there \\
    \trans `I understand you [lit. your speech is there].'
 
\exN\label{n48} 
  \glll  ndí vɛ̀dáà \\
         ndí vɛ̀dáà \\
          but but[Bulu]  \\
    \trans `But still,'
 
\exN\label{n49} 
  \glll  yíì mùdà nlã̂ \\
        yíì mùdà nlã̂ \\
          7.{\COP} big $\emptyset$3.story \\
    \trans `this is a big deal''.'
 
\exN\label{n50}
  \glll yɔ́ɔ̀ Nzàmbí kí nâ bõ̀  \\
        yɔ́ɔ̀ Nzàmbí kì-H nâ bõ̀  \\
          so $\emptyset$1.{\PN} say-{\R} {\COMP} good[French] \\
    \trans `So Nzambi says ``Good.'
 
\exN\label{n51}
  \glll mùdã̂ kɛ̂ nà nyɛ̀ mánkɛ̃̂ \\
       m-ùdã̂ kɛ̀-H nà nyɛ̀ ɛ́-ma-nkɛ̃̂ \\
         {\N}1-woman go-{\R} {\COM} 1.{\OBJ} {\LOC}-ma6-field  \\
    \trans `Woman [his wife], go to the fields with her,'
 
\exN\label{n52}
  \glll kánâ m̀m̂ \\
      kánâ m̀m̂  \\
         or no   \\
    \trans `or no.'
 
\exN\label{n53}
  \glll   wɛ̀ mɛ́dɛ́ pã̂ lígɛ̀ yá nà nyɛ̀ yá kɛ́ mánkɛ̃̂\\
         wɛ mɛ́dɛ́ pã́ lígɛ ya-H nà nyɛ ya-H kɛ̀-H H-ma-nkɛ̃̂\\
           2\textsc{sg}.{\SBJ}  self do.first stay 1\textsc{pl}-\textsc{prs} {\COM} 1.{\OBJ}  1\textsc{pl}-\textsc{prs} go-{\R} {\OBJ}.{\LINK}-6-field\\
    \trans `You [his wife] stay first, we and her, we go to the fields''.'
 
\exN\label{n54} 
  \glll yɔ́ɔ̀ bá téé kɛ̀ndɛ̀ \\
        yɔ́ɔ̀ ba-H téè-H kɛ̀ndɛ̀ \\
         so 2-\textsc{prs} start.walking-{\R} $\emptyset$7.walk   \\
    \trans `So they go on a walk,'
 
\exN\label{n55}
  \glll bà mùdã̂ wà nû \\
       bà m-ùdã̂ wà nû \\
          {\AP} {\N}1-woman 1:{\ATT}  1.{\DEM}.{\PROX}  \\
    \trans `they with this woman.'
 
\exN\label{n56}
  \glll wɔ́ɔ́ɔ́ɔ́ɔ́ pámò mánkɛ̃̂ \\
        wɔ́ɔ́ɔ́ɔ́ɔ́ pámo H-ma-nkɛ̃̂ \\
        {\IDEO}:moving arrive {\OBJ}.{\LINK}-ma6-field    \\
    \trans `[depiction of moving] Having arrived in the fields,'
 
\exN\label{n57}
  \glll Nzàmbí màbɔ́ɔ̀ nkwɛ́ɛ̀ dé nâ vɔ́sì \\
        Nzàmbí ma-bɔ́ɔ̀ nkwɛ́ɛ̀ dé nâ vɔ́sì \\
          $\emptyset$1.{\PN} ma6-breadfruit $\emptyset$3.basket {\LOC} {\COMP} {\IDEO}:pouring\\
    \trans `Nzambi pours the breadfruit into the basket.'
 
\exN\label{n58}
  \glll yɔ́ɔ̀ Nzàmbí á nɔ̀ɔ́ mábɔ́ɔ̀ mándɛ̀\\
         yɔ́ɔ̀ Nzàmbí a-H nɔ̀ɔ̀-H H-ma-bɔ́ɔ̀ má-ndɛ̀\\
          so $\emptyset$1.{\PN} 1-\textsc{prs} take-{\R} {\OBJ}.{\LINK}-ma6-breadfruit 6-{\ANA}\\
    \trans `So Nzambi takes those breadfruit.'
 
\exN\label{n59}
  \glll nyɛ̀ nâ bõ̀ \\
       nyɛ nâ bõ̀ \\
          1.{\SBJ}  {\COMP} good[French]  \\
    \trans `He says ``Good,'
 
\exN\label{n60}
  \glll ɔ̀ múà gyɛ́sɔ̀ nâ wɛ́ kɛ̀ \\
       ɔ múà gyɛ́sɔ nâ wɛ-H kɛ̀ \\
         2\textsc{sg}[Kwasio] be.almost search {\COMP} 2\textsc{sg}-\textsc{prs} go   \\
    \trans `you are about to want to leave.'
 
\exN\label{n61}
  \glll sílɛ̂ nà mɛ̀ kɛ̀ndɛ̀ vúdũ̂ \\
         sílɛ̂ nà mɛ̀ kɛ̀ndɛ̀ vúdũ̂ \\
         finish.{\IMP}  {\COM} 1\textsc{sg} $\emptyset$7.time one   \\
    \trans `Finish [this] with me in one go.'
 
\exN\label{n62}
  \glll mɛ̀ɛ́ jílɛ́ wɛ̂ bvùbvù \\
       mɛ̀ɛ́ jí-lɛ́ wɛ̀ bvùbvù \\
         1\textsc{sg}.\textsc{prs}.{\NEG} ask-{\NEG} 2\textsc{sg}.{\OBJ} much \\
    \trans `I don't ask you for much.'
 
\exN\label{n63}
  \glll  vɛ̂ mɛ̂ sâ mwánɔ̀ wɔ́ɔ̀ wà wɛ̀ bùdɛ́ nû\\
         vɛ̂ mɛ̀ sâ m-wánɔ̀ w-ɔ́ɔ̀ wà wɛ bùdɛ-H nû\\
          give.{\IMP}  1\textsc{sg}.{\OBJ} only {\N}1-child 1-{\POSS}.2\textsc{sg} 1:{\ATT}  2\textsc{sg} have-{\R} 1.{\DEM}.{\PROX}\\
    \trans `Give me only your child that you have here.'
 
\exN\label{n64}
  \glll  mɛ́ lígɛ́ nyɛ̂ dè \\
        mɛ-H lígɛ-H nyɛ̂ dè \\
         1\textsc{sg}-\textsc{prs} stay-{\R} 1.{\OBJ} eat   \\
    \trans `I stay to eat it,'
 
\exN\label{n65}
  \glll  nà màbɔ́'ɔ̀ máà \\
         nà ma-bɔ́'ɔ̀ máà \\
         {\COM} ma6-breadfruit 6:DEM.{\PROX} \\
    \trans `with these breadfruit.'
 
\exN\label{n66}
  \glll wɛ́ nɔ̀ɔ́ mábɔ́'ɔ̀ máà \\
       wɛ-H nɔ̀ɔ̀-H H-ma-bɔ́'ɔ̀ máà \\
         2\textsc{sg}-\textsc{prs} take-{\R} {\OBJ}.{\LINK}-ma6-breadfruit 6:DEM.{\PROX} \\
    \trans `You take these breadfruit.'
 
\exN\label{n67}
  \glll wɛ̀gà wɛ́ kɛ́ nà mɔ̂ \\
       wɛ-gà wɛ-H kɛ̀-H nà m-ɔ̂ \\
          2\textsc{sg}.{\SBJ}-{\CONTR} 2\textsc{sg}-\textsc{prs} go-{\R} {\COM} 6-{\OBJ}  \\
    \trans `As for you, you take them [the breadfruit] away.'
 
\exN\label{n68}
  \glll  mɛ̀gà mɛ́ lígɛ́ dè mwánɔ̀ wɔ́ɔ̀ \\
        mɛ-gà mɛ-H lígɛ-H dè m-wánɔ̀ w-ɔ́ɔ̀ \\
          1.{\SBJ}-{\CONTR} 1\textsc{sg}-\textsc{prs} stay-{\R} eat ma1-child 1-{\POSS}.2\textsc{sg}  \\
    \trans `As for me, I stay and eat your child,'
 
\exN\label{n69}
  \glll nà màbɔ́'ɔ̀ \\
        nà ma-bɔ́'ɔ̀ \\
         {\COM} ma6-breadfruit \\
    \trans `with breadfruit.'
 
\exN\label{n70}
  \glll sílɛ̀ \\
        sílɛ \\
         finish   \\
    \trans `That's it!''.'
 
\exN\label{n71}
  \glll  ɛ́ɛ́kɛ̀ mùdã̂ à gyɛ̃́ɛ̃̀ à gyɛ̃́ɛ̃̀ áh mùdì wã́ã̀ wɛ́ sá mɛ̂ ná \\
        ɛ́ɛ́kɛ̀ m-ùdã̂ a gyɛ̃́ɛ̃̀ a gyɛ̃́ɛ̃̀ áh m-ùdì w-ã́ã̀ wɛ-H sâ-H mɛ̂ ná \\
          {\EXCL} {\N}1-woman 1.{\PST}1 cry.{\COMPL} 1.{\PST}1 cry.{\COMPL} {\EXCL} {\N}1-person 1-{\POSS}.1\textsc{sg} 2\textsc{sg}-\textsc{prs} do-{\R} 1\textsc{sg}.{\OBJ} how \\
    \trans `Oh, the woman cried and cried, ``Ah, my person, what do you do to me?''.'
 
\exN\label{n72}
  \glll  yɔ́ɔ̀ Nzàmbí kí náà mɛ̀ bwàá wɛ̂ tsíyɛ̀ lèkɛ́lɛ̀ dẽ́ nâ mɛ́ lígɛ́ dè mwánɔ̀ wɔ́ɔ̀ \\
       yɔ́ɔ̀ Nzàmbí kì-H náà mɛ bwàà-H wɛ̂ tsíyɛ le-kɛ́lɛ̀ dẽ́ nâ mɛ-H lígɛ-H dè m-wánɔ̀ w-ɔ́ɔ̀ \\
         so $\emptyset$1.{\PN} say-{\R} {\COMP} 1\textsc{sg}.{\PST}1 have-{\R} 2\textsc{sg}.{\OBJ} cut le5-speech today {\COMP} 1\textsc{sg}-\textsc{prs} stay-{\R} eat {\N}1-child 1-{\POSS}.2\textsc{sg}\\
    \trans `So Nzambi says, ``I'm not listening to you today [lit. I have cut your word]. I stay and eat your child,'
 
\exN\label{n73}
  \glll nà màbɔ́'ɔ̀ \\
        nà ma-bɔ́'ɔ̀ \\
        {\COM} ma6-breadfruit\\
    \trans `with breadfruit.'
 
\exN\label{n74}
  \glll lèkáà lé tè'ètè yá mwánɔ̀ yíì \\
          le-káà lé tè'ètè yá m-wánɔ̀ yíì \\
          le5-kind 5:{\ATT}  $\emptyset$7.tenderness 7:{\ATT}  {\N}1-child 7.{\DEM}.{\PROX}  \\
    \trans `The kind of tenderness of this child,'
 
\exN\label{n75}
  \glll yíì mpà yɔ̃́ɔ̃̀ wɛ́ kã́ yɔ̂ dúmbɔ́ \\
       yíì mpà yɔ̃́ɔ̃̀ wɛ-H kã̂-H y-ɔ̀ dúmbɔ́ \\
         7.{\COP} good $\emptyset$7.time 2\textsc{sg}-\textsc{prs} wrap-{\R} 7-{\OBJ} $\emptyset$7.package\\ 
    \trans `is good when you wrap it in a [leaf] wrap.'
 
\exN\label{n76}
  \glll mɛ̀ɛ̀ yɔ̂ dè \\
         mɛ̀ɛ̀ y-ɔ̂ dè \\
        1\textsc{sg}.{\FUT} 7-{\OBJ} eat  \\
    \trans `I will eat it''.'
 
\exN\label{n77}
  \glll yɔ́ɔ̀ Nzàmbí kí náà bõ̀ \\
       yɔ́ɔ̀ Nzàmbí kì-H náà bõ̀ \\
          so $\emptyset$1.{\PN} say {\COMP} good[French]  \\
    \trans `So Nzambi says ``Good,'
 
\exN\label{n78}
  \glll ká wɛ̀ɛ́ wúmbɛ́lɛ́ ndáà \\
        ká wɛ̀ɛ́ wúmbɛ-lɛ́ ndáà \\
         if 2\textsc{sg}.\textsc{prs}.{\NEG} want-{\NEG} also   \\
    \trans `if you don't want [this] either,'
 
\exN\label{n79}
  \glll mɛ́ nɔ̀ɔ́ nkwɛ̂ wá mábɔ́'ɔ̀ \\
        mɛ-H nɔ̀ɔ̀-H nkwɛ̂ wá H-ma-bɔ́'ɔ̀ \\
         1\textsc{sg}-\textsc{prs} take-{\R} $\emptyset$3.basket 3:{\ATT}  {\OBJ}.{\LINK}-ma6-breadfruit\\
    \trans `I take the basket with the breadfruit.'
 
\exN\label{n80}
  \glll wɛ́ kɛ́ wɛ̀ nà nzà nyɔ́ɔ̀ ɛ́ pɛ̀ wɛ́ kɛ́ wɛ̀ nà nyɔ́ɔ̀ \\
       wɛ-H kɛ̀-H wɛ̀ nà nzà ny-ɔ́ɔ̀ ɛ́ pɛ̀ wɛ-H kɛ̀-H wɛ̀ nà ny-ɔ́ɔ̀ \\
        2\textsc{sg}-\textsc{prs} go-{\R} die {\COM}  $\emptyset$9.hunger 9-{\POSS}.2\textsc{sg} {\LOC} there 2\textsc{sg}-\textsc{prs} go-{\R} die {\COM} 9-OBJ\\
    \trans `Your are going to die of your hunger there, you are going to die of it''.'
 
\exN\label{n81}
  \glll yɔ́ɔ̀ mùdã̂ dígɛ́ mísì ndẽ́ẽ́ẽ́ \\
        yɔ́ɔ̀ m-ùdã̂ dígɛ-H m-ísì ndẽ́ẽ́ẽ́ \\
        so {\N}1-woman watch-{\R} ma6-eye {\IDEO}:staring \\
    \trans `So the woman thinks [lit. looks with her eyes].'
 
\exN\label{n82}
  \glll nyɛ̀ nâ tɔ̀sâ \\
          nyɛ nâ tɔ̀sâ \\
       1.{\SBJ}  {\COMP} nothing  \\
    \trans `She [says]: ``No!''.'
 
\exN\label{n83}
  \glll yɔ́ɔ̀ mùdã̂ tɔ́kɛ́ mwánɔ̀ kàlànɛ̀ nyɛ̂ \\
       yɔ́ɔ̀ m-ùdã̂ tɔ́kɛ-H m-wánɔ̀ kàlanɛ nyɛ̂ \\
       so {\N}1-woman collect-{\R} {\N}1-child hand.over 1.{\OBJ}\\
    \trans `So the woman picks up the child, handing it over to him.'
 
\exN\label{n84}
  \glll Nzàmbí nyɛ̀ nâ nkɛ̀ \\
        Nzàmbí nyɛ nâ nkɛ̀ \\
       $\emptyset$1.{\PN}  1.{\SBJ} {\COMP} go.\textsc{hort}  \\
    \trans `Nzambi [says]: ``Let's go''.'
 
\exN\label{n85}
  \glll wɔ́ɔ́ɔ́ɔ́ bɔ́ pámò \\
        wɔ́ɔ́ɔ́ɔ́ bɔ-H pámo \\
       {\IDEO}:moving 2-\textsc{prs}[Kwasio] arrive  \\
    \trans `[depiction of motor sound] They arrive.'
 
\exN\label{n86}
  \glll Nzàmbí nyɛ̀ nâ ɛ́ mùdã̂ wã̂ \\
        Nzàmbí nyɛ nâ ɛ́ m-ùdã̂ w-ã̂ \\
       $\emptyset$1.{\PN}  1.{\SBJ}  {\COMP} {\LOC} {\N}1-woman 1-{\POSS}.1\textsc{sg}  \\
    \trans `Nzambi [says]: ``My wife,'
 
\exN\label{n87}
  \glll mwánɔ̀ wɛ́ɛ̀ nyɛ̀ nû \\
        m-wánɔ̀ w-ɛ́ɛ̀ nyɛ̀ nû \\
        {\N}1-child 1-{\POSS}.3\textsc{sg} 1.{\SBJ} 1.{\DEM}.{\PROX} \\
    \trans `this is her child.'
 
\exN\label{n88}
  \glll mɛ́ lɔ́ nɔ́ɔ̀ mwánɔ̀ púù yá mábɔ́'ɔ̀ mâ\\
        mɛ-H lɔ́ nɔ́ɔ̀ m-wánɔ̀ púù yá ma-bɔ́'ɔ̀ mâ\\
        1\textsc{sg}-\textsc{prs} {\RETRO}  take {\N}1-child $\emptyset$7.reason 7:{\ATT}  ma6-breadfruit 6.{\DEM}.{\PROX}\\
    \trans `I have just taken the child for these breadfruit.'

\exN\label{n89}
  \glll kálɛ̀ mɛ̀ báà kì nâ bá dúù bɛ̀ bédéwɔ̀\\
         kálɛ̀ mɛ̀ báà kì nâ ba-H dúù bɛ̀ H-be-déwɔ̀\\
       {\NEG}.{\FUT} 1\textsc{sg}.{\SBJ} 2.{\FUT} say {\COMP} 2-\textsc{prs} must.not.{\SBJV}  grow {\OBJ}.{\LINK}-be8-food\\
    \trans `It's not me who will say that they must not grow food [it's not my fault that they don't have food]''.'
 
\exN\label{n90}
  \glll yɔ́ɔ̀ mùdã̂ nú kɛ̀ \\
        yɔ́ɔ̀ m-ùdã̂ nû-H kɛ̀ \\
      so {\N}1-woman 1-\textsc{prs} go \\
    \trans `So the woman goes.'
 
\exN\label{n91}
  \glll  ndɛ̀náà pámò lébũ̂ àá gyì \\
         ndɛ̀náà pámo H-le-bũ̂ àá gyì \\
        like.this arrive {\OBJ}.{\LINK}-le5-river.bank 1.{\INCH} cry \\
    \trans `Having arrived like this [without the child] at the river bank she starts to cry.'
 
\exN\label{n92}
  \glll àá gyì àá gyì dyúmò njì nyɛ̂ nɔ̀ɔ̀ \\
       àá gyì àá gyì dyúmò njì nyɛ̂ nɔ̀ɔ̀ \\
       1.{\INCH} cry 1.{\INCH} cry $\emptyset$1.spouse come 1.{\OBJ} take  \\
    \trans `She starts to cry, she starts to cry, the husband comes to fetch her.'
 
\exN\label{n93}
  \glll ɛ́ ná mwánɔ̀ nùù vɛ́ \\
       ɛ́ ná m-wánɔ̀ nùù vɛ́ \\
       {\LOC} how {\N}1-child 1.{\COP} where  \\
    \trans `What! Where is the child?'
 
\exN\label{n94}
  \glll nyɛ̀ nâ sɔ́ wɔ́ɔ̀ nɔ̀ɔ́ mɔ̀ mwánɔ̀ \\
        nyɛ nâ sɔ́ w-ɔ́ɔ̀ nɔ̀ɔ̀-H mɔ̀ m-wánɔ̀ \\
      1.{\SBJ}  {\COMP} $\emptyset$1.friend 1-{\POSS}.2\textsc{sg} take-{\R} {\COMPL} 1-child   \\
    \trans `She [says]: ``Your friend has taken the child.'
 
\exN\label{n95} 
  \glll à kɛ̃́ɛ̃̀ nyɛ̂ dè \\
       a kɛ̃́ɛ̃̀ nyɛ̂ dè \\
       1.{\PST}1 go.{\COMPL} 1.{\OBJ} eat \\
    \trans `He has left to eat it''.'
 
\exN\label{n96}
  \glll yɔ́ɔ̀ á lɔ́ kì náà \\
        yɔ́ɔ̀ a-H lɔ́ kì náà \\
        so 1-\textsc{prs} {\RETRO}  say {\COMP} \\
    \trans `So he just said that:'
 
\exN\label{n97}
  \glll ɛ́ mpù wɛ̀ɛ́ gyángyálɛ́ bédéwɔ̀ \\
       ɛ́ mpù wɛ̀ɛ́ gyángya-lɛ́ H-be-déwɔ̀ \\
        {\LOC} like.this 2\textsc{sg}.\textsc{prs}.{\NEG} work-{\NEG} {\OBJ}.{\LINK}-be8-food \\
    \trans ``This is not how you work for your food''.'
 
\exN\label{n98}
  \glll yɔ́ɔ̀ nyɛ̀gá nɔ̀ɔ̀ mwánɔ̀ \\
        yɔ́ɔ̀ nyɛ-gá nɔ̀ɔ̀ m-wánɔ̀ \\
       so 1.{\SBJ}-{\CONTR} take {\N}1-child  \\
    \trans `So the other Nzambi, after having taken the child,'
 
\exN\label{n99}
  \glll á lígɛ́ nyɛ̂ dè \\
        a-H lígɛ-H nyɛ̂ dè \\
       1-\textsc{prs} stay-{\R} 1.{\OBJ} eat  \\
    \trans `stays to eat it,'
 
\exN\label{n100}
  \glll nà màbɔ́'ɔ̀ mɛ́ɛ́ \\
       nà ma-bɔ́'ɔ̀ m-ɛ́ \\
       {\COM} ma6-breadfruit 6-{\POSS}.3\textsc{sg} \\
    \trans `with his breadfruit.'
 
\exN\label{n101}
  \glll yɔ́ɔ̀ Nzàmbí wà nû kɛ́ dígɛ̀ mpù nâ kɛ́ \\
       yɔ́ɔ̀ Nzàmbí wà nû kɛ̀-H dígɛ mpù nâ kɛ́ \\
       so $\emptyset$1.{\PN}  1:{\ATT}  1.{\DEM}.{\PROX} go-{\R} look like.this {\COMP} {\EXCL}\\
    \trans `So this Nzambi goes and looks like this: ``Ey!'
 
\exN\label{n102}
  \glll mbúmbù wã̂ wɛ́ kúmbɔ́ mɛ̂ sá mpù \\
          mbúmbù w-ã̂ wɛ-H kúmbɔ-H mɛ̂ sá mpù \\
        $\emptyset$1.namesake 1-{\POSS}.1\textsc{sg} 2\textsc{sg}-\textsc{prs} arrange-{\R} 1\textsc{sg}.{\OBJ} $\emptyset$7.thing like.this \\
    \trans `My namesake, you really do this to me.'
 
\exN\label{n103}
  \glll ɛ́ mwánɔ̀ wã̂ dyúwɔ̀ \\
       ɛ́ m-wánɔ̀ w-ã̂ dyúwɔ̀ \\
        {\EXCL} {\N}1-child 1-{\POSS}.1\textsc{sg} on \\
    \trans `Hey, about my child!''.'
 
\exN\label{n104}
  \glll [clicking] yɔ́ɔ̀ wà núndɛ̀ dígɛ́ mísì  \\
     [clicking] yɔ́ɔ̀ wà nú-ndɛ̀ dígɛ-H m-ísì \\
      [clicking] so 1:{\ATT}  1-{\ANA} look-{\R} ma6-eye \\
    \trans `[sound of disappreciation] So this one thinks [lit. looks with his eyes],'
 
\exN\label{n105}
  \glll ndẽ́ẽ́ẽ́ẽ́ nyɛ̀ nâ tɔ̀sâ \\
       ndẽ́ẽ́ẽ́ẽ́ nyɛ nâ tɔ̀sâ \\
      {\IDEO}:staring 1.{\SBJ}  {\COMP} nothing   \\
    \trans `[depiction of staring] He [says]: ``No'
 
\exN\label{n106}
  \glll yíì pẽ̀'ẽ̀ nyà mwánɔ̀ mùdũ̂ mɛ́ pã́ã́ ná nyɔ̂ vɛ̀ \\
       yíì pẽ̀'ẽ̀ nyà m-wánɔ̀ m-ùdũ̂ mɛ-H pã́ã̀-H ná ny-ɔ̂ vɛ̀ \\
      7.{\COP} $\emptyset$9.wisdom 9:{\ATT}  {\N}1-child {\N}1-male  1\textsc{sg}-\textsc{prs} do.first-H again 9-{\OBJ} give  \\
    \trans `Every child knows this [lit. This is the wisdom of a boy], I will take revenge on him''.'
 
\exN\label{n107}
  \glll yɔ́ɔ̀ Nzàmbí wà núú nyî \\
       yɔ́ɔ̀ Nzàmbí wà núú nyî \\
        so $\emptyset$1.{\PN}  1:{\ATT}  1.{\DEM}.{\DIST} enter \\
    \trans `So that Nzambi comes in.'
 
\exN\label{n108}
  \glll bóŋ \\
        bóŋ \\
        good[French] \\
    \trans ``Good.'
 
\exN\label{n109}
  \glll mɛ́ lámbó Nzàmbí wà nû \\
       mɛ-H lámbo-H Nzàmbí wà nû \\
        1\textsc{sg}-\textsc{prs} trap-{\R} $\emptyset$1.{\PN}  1:{\ATT}  1.{\DEM}.{\PROX} \\
    \trans `I trap this Nzambi,'
 
\exN\label{n110} 
  \glll nà mɛ́ wúmbɛ́ lèmbò ɛ́ mpù à bùdɛ́ mɛ̂ \\
          nà mɛ-H wúmbɛ-H lèmbo ɛ́ mpù a bùdɛ-H mɛ̂ \\
       {\CONJ} 1\textsc{sg}-\textsc{prs} want-{\R} know {\LOC} like.this 1 have-{\R} 1\textsc{sg}.{\OBJ}  \\
    \trans `and I want to know what he thinks of this story''.'
 
\exN\label{n111}
  \glll yɔ́ɔ̀ Nzàmbí wà nû kɛ̃́ɛ̃̀ bwã̀sà nyɛ̀ nâ \\
         yɔ́ɔ̀ Nzàmbí wà nû kɛ̃́ɛ̃̀ bwã̀sa nyɛ nâ \\
       so $\emptyset$1.{\PN}  1:{\ATT}  1.{\DEM}.{\PROX} go.{\COMPL} think 1.{\SBJ} {\COMP}  \\
    \trans `So this Nzambi has gone to think, he [says]:'
 
\exN\label{n112}
  \glll sá mɛ́dɛ́ mɛ̀ nzí sâ yî \\
        sá mɛ́dɛ́ mɛ nzí sâ yî \\
        $\emptyset$7.thing self 1\textsc{sg} {\PROG}.{\PST}  do 7.{\DEM}.{\PROX} \\
    \trans ``This is all my fault [for sending the wife].'
 
\exN\label{n113}
  \glll mɛ́ pã́ ná kɛ̀ dígɛ̀ mùdì wà nû ɛ́ pɛ́ɛ́\\
        mɛ-H pã̂-H ná kɛ̀ dígɛ m-ùdì wà nû ɛ́ pɛ́-ɛ́\\
        1\textsc{sg}-\textsc{prs} do.first-H again go see {\N}1-person 1:{\ATT}  1.{\DEM}.{\PROX} {\LOC} over.there.{\DIST}\\
    \trans `I go first again to see this person over there''.'
 
\exN\label{n114}
  \glll yɔ́ɔ̀ Nzàmbí njí mpù bã̂ã̂ã̂ã̂ njì dígɛ̀ mpù \\
        yɔ́ɔ̀ Nzàmbí njî-H mpù bã̂ã̂ã̂ã̂ njì dígɛ mpù \\
        so $\emptyset$1.{\PN} come-{\R} like.this {\IDEO}:walking.far come look like.this \\
    \trans `So Nzambi comes like this [depiction of walking a long distance], comes looking like this.'
 
\exN\label{n115}
  \glll nyɛ̀ nâ kɛ́ɛ́ɛ́ɛ́ \\
        nyɛ̀ nâ kɛ́ɛ́ɛ́ɛ́ \\
       1.{\SBJ}  {\COMP} {\EXCL}  \\
    \trans `He [says]: ``What!'
 
\exN\label{n116}
  \glll mbúmbù \\
       mbúmbù \\
        $\emptyset$1.namesake \\
    \trans `Namesake!'
 
\exN\label{n117}
  \glll mɛ́ lɔ́ njì bàgyɛ̃̂ bà wɛ̂ \\
       mɛ-H lɔ́ njì ba-gyɛ̃̂ bà wɛ̂ \\
       1\textsc{sg}-\textsc{prs} {\RETRO}  come ba2-stranger {\AP} 2\textsc{sg}  \\
    \trans `I just came as a guest to you.'
 
\exN\label{n118}
  \glll ndíííí \\
        ndí-\textsc{length} \\
        but \\
    \trans `But. . .'
 
\exN\label{n119}
  \glll njìmɔ̀ wá sá njìníì \\
        njìmɔ̀ wá sá njìníì \\
        $\emptyset$3.entire 3:{\ATT}  $\emptyset$7.thing different \\
    \trans `the whole thing is different.'
 
\exN\label{n120}
  \glll mɛ́ lɔ́ njì gyɛ́sɔ̀ bà wɛ̂ \\
       mɛ-H lɔ́ njì gyɛ́sɔ bà wɛ̂ \\
       1\textsc{sg}-\textsc{prs} {\RETRO}  come search {\AP} 2\textsc{sg}  \\
    \trans `I just came to search at your place''.'
 
\exN\label{n121}
  \glll ɛ́ɛ́ɛ́ Nzàmbí kí nâ ɛ́ɛ́ \\
       ɛ́ɛ́ɛ́ Nzàmbí kì-H nâ ɛ́ɛ́ \\
       {\EXCL} $\emptyset$1.{\PN} say-{\R} {\COMP} yes  \\
    \trans `Hey, Nzambi says: ``Yes,'
 
\exN\label{n122} 
  \glll bèsá bíndɛ̀ byɛ́sɛ̀ béè ndáà \\
       be-sá bí-ndɛ̀ by-ɛ́sɛ̀ béè ndáà \\
        be8-thing 8-{\ANA} 8-all 8.{\COP} also \\
    \trans `all those things are also there [way of introducing a problem].'
 
\exN\label{n123}
  \glll bèsá bíndɛ̀ byɛ́sɛ̀ béè ndáà \\
        be-sá bí-ndɛ̀ by-ɛ́sɛ̀ béè ndáà \\
        be8-thing 8-{\ANA} 8-all 8.{\COP} also \\
    \trans `All these things are also there [way of introducing a problem].'
 
\exN\label{n124}
  \glll ndí mɛ̀ɛ́ sálɛ́ wɛ̂ bvùbvù ndí vɛ̀dáà mɛ́ dyúwɔ́ nâ \\
        ndí mɛ̀ɛ́ sâ-lɛ́ wɛ̂ bvùbvù ndí vɛ̀dáà mɛ-H dyúwɔ-H nâ \\
       but 1\textsc{sg}.\textsc{prs}.{\NEG} do-{\NEG} 2\textsc{sg}.{\OBJ} much but but[Bulu] 1\textsc{sg}-\textsc{prs} understand-{\R} {\COMP}   \\
    \trans `But I don't do you wrong, but I understand that,'
 
\exN\label{n125}
  \glll wɛ́ɛ̀ dé mwánɔ̀ nɔ́ɔ̀ \\
       wɛ́ɛ̀ dè-H m-wánɔ̀ nɔ́ɔ̀ \\
      2.{\PST}2 eat-{\R} {\N}1-child no[French] \\
    \trans `you  have eaten
     the child, haven't you?''.'
 
\exN\label{n126}
  \glll nyɛ̀ nâ mɛ́ɛ̀ dé pɔ́nɛ́ nà màbɔ́'ɔ̀ \\
        nyɛ nâ mɛ́ɛ̀ dè-H pɔ́nɛ́ nà ma-bɔ́'ɔ̀ \\
      1.{\SBJ}  {\COMP} 1\textsc{sg}.{\PST}2 eat-{\R} $\emptyset$7.truth {\COM} ma6-breadfruit   \\
    \trans `He [says]: ``I really ate [it] with breadfruit.'
 
\exN\label{n127} 
  \glll mɛ̀gà mɛ́ɛ̀ dyúwɔ́ nzã́ã̀ dúwɔ̀ lé tè \\
       mɛ-gà mɛ́ɛ̀ dyúwɔ-H nzã́ã̀ d-úwɔ̀ lé tè \\
         1\textsc{sg}.{\SBJ}-{\CONTR} 1\textsc{sg}.{\PST}2 feel-{\R} $\emptyset$7.appetite le5-day 5:{\ATT}  there \\
    \trans `As for me, I had a craving [for meat] that day.'
 
\exN\label{n128}
  \glll mɛ̀ kí bɛ̀ nà tsídí \\
       mɛ kí bɛ̀ nà tsídí \\
       1\textsc{sg}.{\PST}1 {\NEG}[Kwasio] be {\COM} $\emptyset$1.meat  \\
    \trans `I didn't have any meat''.'
 
\exN\label{n129}
  \glll á kfùmálá mpù Nzàmbí lúndɛ́lɛ̃́ɛ̃̀ ɛ́ mpù \\
        a-H kfùmala-H mpù Nzàmbí lúndɛ́lɛ̃́ɛ̃̀ ɛ́ mpù \\
        1-\textsc{prs} find-{\R} like.this $\emptyset$1.{\PN} fill.{\COMPL} {\LOC} like.this \\
    \trans `He discovers [inside the house] like this, Nzambi has filled [the house] like this [with skulls].'
 
\exN\label{n130}
  \glll kɛ́ mbúmbù bwánɔ̀ bà sílɛ̃́ɛ̃̀ kɛ̀ vɛ́ \\
       kɛ́ mbúmbù b-wánɔ̀ ba sílɛ̃́ɛ̃̀ kɛ̀ vɛ́ \\
        {\EXCL} $\emptyset$1.namesake ba2-child 2.{\PST}1 finish.{\COMPL} go where \\
    \trans ``Ey namesake, where have all the children gone?''.'
 
\exN\label{n131}
  \glll  nyɛ̀ nâ kɛ́ɛ́ɛ́ bwánɔ̀ bã̂ mɛ̀ sílɛ̃́ɛ̃̀ bɔ̂ dyùù\\
        nyɛ nâ kɛ́ɛ̀ b-wánɔ̀ b-ã̊ mɛ sílɛ̃ɛ̃̀ b-ɔ̂ dyùù\\
        1.{\SBJ}  {\COMP} {\EXCL} ba2-child 2-{\POSS}.1\textsc{sg} 1\textsc{sg} finish.{\COMPL} 2-{\OBJ} kill\\
    \trans `He [says]: ``Ha, my children, I have already killed them all.'
 
\exN\label{n132}
  \glll ngáà wɛ́ nyɛ́ mpù \\
       ngáà wɛ-H nyɛ̂-H mpù \\
       Q(tag) 2\textsc{sg}-\textsc{prs} see-{\R} like.this  \\
    \trans `Right, you see that?'
 
\exN\label{n133}
  \glll bèkɔ́kɔ́ bé nlô bé tè \\
       be-kɔ́kɔ́ bé nlô bé tè \\
        be8-hollowness 8:{\ATT}  $\emptyset$3.head 8:{\ATT}  there \\
    \trans `The skulls there,'
 
\exN\label{n134}
  \glll béè tè \\
        béè tè \\
        8.{\COP} there\\
    \trans `are there,'
 
\exN\label{n135}
  \glll mìnlô mí bákímì \\
        mi-nlô mí ba-kímì \\
       mi4-head 4:{\ATT}  ba2-monkey  \\
    \trans `monkey heads''.'
 
\exN\label{n136}
  \glll kó mbúmbù nyɛ̀ nzí lèmbò dyùù bɔ̂ fàmíì bá bùdì ná \\
       kó mbúmbù nyɛ nzí lèmbo dyùù b-ɔ̂ fàmíì bá b-ùdì ná \\
       {\EXCL} $\emptyset$1.namesake 1.{\SBJ}  {\PROG}.{\PST}  know kill 2-{\OBJ} $\emptyset$1.family 2:{\ATT}  ba2-person how \\
    \trans ``Oh namesake, how could he kill them, the family of people?''.'
 
\exN\label{n137}
  \glll nyɛ̀ nâ ó \\
        nyɛ nâ ó \\
        1.{\SBJ}  {\COMP} {\EXCL} \\
    \trans `He [says]: ``Oh,'
 
\exN\label{n138}
  \glll mbúmbù \\
       mbúmbù \\
       $\emptyset$1.namesake  \\
    \trans `Namesake!'
 
\exN\label{n139}
  \glll ɛ́ yɔ́ɔ̀ wà mwánɔ̀ mùdũ̂ sá màmbò má mwánɔ̀ mùdũ̂ \\
      ɛ́ yɔ́ɔ̀ wà m-wánɔ̀ m-ùdũ̂ sâ-H m-àmbò má m-wánɔ̀ m-ùdũ̂ \\
       {\LOC} so 2\textsc{sg}[Bulu] {\N}1-child {\N}1-man do-{\R} ma6-thing 6:{\ATT}  {\N}1-child {\N}1-man  \\
    \trans `So you behave like a boy.'
 
 \largerpage
\exN\label{n140}
  \glll mɛ̀ nzí wúmbɛ̀ nâ bwánɔ̀ bã̂ bá bwámóò ɛ́ mpù mìntángánɛ́ békúdɛ́ bé mpâ \\
    mɛ nzí wúmbɛ nâ b-wánɔ̀ b-ã̂ ba-H bwámóò ɛ́ mpù mi-ntángánɛ́ H-be-kúdɛ́ bé mpâ \\
    1\textsc{sg}.{\PST}1 {\PROG}.{\PST}  want {\COMP} ba2-child 2-{\POSS}.1\textsc{sg} 2-\textsc{prs} become.{\SBJV}  {\LOC} like.this mi4-white.person {\OBJ}.{\LINK}-be8-skin 8:{\ATT}  good \\
    \trans `I had wanted my children to get fair skin like white people''.'
 
\exN\label{n141} 
  \glll Nzàmbí kí nâ bõ̀ \\
        Nzàmbí kì-H nâ bõ̀ \\
       $\emptyset$1.{\PN} say-{\R} {\COMP} good[French]  \\
    \trans `Nzambi says: ``Good,'
 
\exN\label{n142}
  \glll mɛ̀ dyúwɔ́ mɔ̀ \\
       mɛ dyúwɔ-H mɔ̀ \\
        1\textsc{sg}.{\PST}1 understand-{\R} {\COMPL} \\
    \trans `I have understood''.'
 
\exN\label{n143}
  \glll yɔ́ɔ̀ Nzàmbí kí nâ bõ̀ mɛ̀ nìyɛ́ mɔ̀\\
        yɔ́ɔ̀ Nzàmbí kì-H nâ bõ̀ mɛ nìyɛ-H mɔ̀\\
        so $\emptyset$1.{\PN} say-{\R} {\COMP} good[French] 1\textsc{sg}.{\PST}1 return-H {\COMPL}\\
    \trans `So Nzambi says: ``Good, I am returning home''.'
 
\exN\label{n144} 
  \glll nyɛ̀ nâ mbúmbù nlâ wùú gyálɛ́ \\
         nyɛ nâ mbúmbù nlâ wùú gyà-lɛ́ \\
        1.{\SBJ}  {\COMP} {\N}1-namesake $\emptyset$3.story 3.\textsc{prs}.{\NEG} be.long-{\NEG} \\
    \trans `He [says]: ``Namesake, it is easy [lit. the story isn't long].'
 
\exN\label{n145}
  \glll sílɛ̂ dyùù fàmí wɔ́ɔ̀ wà bùdì wɛ̀ɛ̀ nyɛ̂ \\
       sílɛ̂ dyùù fàmí w-ɔ́ɔ̀ wà b-ùdì wɛ̀ɛ̀ nyɛ̂ \\
       finish.{\IMP}  kill $\emptyset$1.family 1-{\POSS}.2\textsc{sg} 1:{\ATT}  ba2-person 2\textsc{sg}.{\FUT} see  \\
    \trans `Kill your whole family of people, you will see.'
 
\exN\label{n146}
  \glll bwánɔ̀ bɔ́ɔ̀ báà bwámò míntángánɛ́  \\
        b-wánɔ̀ b-ɔ́ɔ̀ báà bwámo H-mi-ntángánɛ́  \\
        ba2-child 2-{\POSS}.2\textsc{sg} 2.{\FUT} become {\OBJ}.{\LINK}-mi4-white.person \\
    \trans `Your children will become white people.'
 
\exN\label{n147}
  \glll gyí mɛ́dɛ́ wɛ́ kɛ́ nà vũ̀ũ̀ wɛ̂ \\
        gyí mɛ́dɛ́ wɛ-H kɛ̀-H nà vũ̀ũ̀ wɛ̂ \\
        what self 2\textsc{sg}-\textsc{prs} go-{\R} {\CONJ} worry there \\
    \trans `What do you go and worry about there?''.'
 
\exN\label{n148}
  \glll yɔ́ɔ̀ Nzàmbí wà núú nìyɛ̀ \\
        yɔ́ɔ̀ Nzàmbí wà núú nìyɛ \\
         so $\emptyset$1.{\PN} 1:{\ATT}  1.{\DEM}.{\DIST} return\\
    \trans `So that Nzambi returns [and goes to the family of the Nzambi who has eaten his child].'
 
\exN\label{n149}
  \glll ɛ́kɛ̀ Nzàmbí wà nú áà sàlɛ́ bɛ̀ nà bã̂ línáá pámò \\
      ɛ́kɛ̀ Nzàmbí wà nú áà sàlɛ́ bɛ̀ nà bã̂ líní a-H pámo \\
        {\EXCL} $\emptyset$1.{\PN} 1:{\ATT}  1.{\DEM}.{\DIST} 1.{\PST}2 {\NEG}.{\PST}  be {\COM} $\emptyset$7.word when 1-\textsc{prs} arrive  \\
    \trans `Oh! That Nzambi had no words as soon as he arrived [he went mad].'
 
\exN\label{n150}
  \glll nyɛ̀ nâ álè \\
       nyɛ nâ álè \\
       1.{\SBJ}  {\COMP} go[French]  \\
    \trans `He [says]: ``Ok [French: {\itshape Allez!}].'
 
\exN\label{n151}
  \glll nyáà ngà sílɛ́ nyî ndáwɔ̀ dé tù \\
       nyáà ngà sílɛ́-H nyî ndáwɔ̀ dé tù \\
       shit.{\IMP}  {\PL}  finish-{\R} enter $\emptyset$9.house {\LOC} inside  \\
    \trans `Piss off [talking to the other Nzambi's family], everybody go into the house!'
 
\exN\label{n152}
  \glll sílɛ̂ ngà nyî vâ \\
       sílɛ̂ ngà nyî vâ \\
        finish.{\IMP}  {\PL}  enter here \\
    \trans `Enter all here''.'
 
\exN\label{n153}
  \glll á lúndɛ́lɛ́ bɔ̂ lèkàá lé ndáwɔ̀ nyî nâ bɛ́ɛ̀ vyâ \\
       a-H lúndɛlɛ-H b-ɔ̂ le-kàá lé ndáwɔ̀ nyî nâ bɛ́ɛ̀ vyâ \\
       1-\textsc{prs} fill-{\R} 2-{\OBJ} le5-kind 5:{\ATT}  $\emptyset$9.house 9.{\DEM}.{\PROX} {\COMP} be.{\SBJV}  full \\
    \trans `He fills them in this kind of house so that it [house] be full.'
 
\exN\label{n154}
  \glll áà sílɛ́ kɛ̀ nà dvùwɔ́ dyúwɔ̀ \\
     áà sílɛ-H kɛ̀ nà dvùwɔ-H dyúwɔ \\
        1.{\PST}2 finish-{\R} go {\CONJ} stuff-{\R} $\emptyset$7.top \\
    \trans `He has gone and stuffed the top [with straw],'
 
\exN\label{n155} 
  \glll nâ tã̂ \\
       nâ tã̂ \\
       {\COMP} tight  \\
    \trans `tight.'
 
\exN\label{n156}
  \glll yɔ́ɔ̀ Nzàmbí dígɛ́ mísì ɛ́ mpù \\
      yɔ́ɔ̀ Nzàmbí dígɛ-H m-ísì ɛ́ mpù \\
       so $\emptyset$1.{\PN} look-{\R} ma6-eye {\LOC} like.this  \\
    \trans `So Nzambi looks with the eyes like this [speaker imitated Nzambi how he is visually checking the house].'
 
\exN\label{n157}
  \glll nzá nzíí mɛ̂ nyɛ̂ \\
    nzá nzíí mɛ̂ nyɛ̂ \\
         who {\PROG}.\textsc{prs} 1\textsc{sg}.{\OBJ} see\\
    \trans ``Who is seeing me?'
 
\exN\label{n158} 
  \glll ah mbúmbù wɛ̀ wɛ́ tɛ́lɛ́ núndɛ̀ \\
      ah mbúmbù wɛ wɛ-H tɛ́lɛ-H nú-ndɛ̀ \\
        {\EXCL} $\emptyset$1.namesake 2\textsc{sg}.{\SBJ}  2\textsc{sg}-\textsc{prs} stand-{\R} 1-{\ANA} \\
    \trans `Ah namesake, is it you who is standing there?'
 
\exN\label{n159}
  \glll nyàá jìwɔ̀ jìwɔ̀ jìwɔ̀ wɛ̀ \\
      nyàà-H jìwɔ jìwɔ jìwɔ wɛ \\
        shit-{\R} close close close 2\textsc{sg} \\
    \trans `Shit, close, close, close you!'
 
\exN\label{n160}
  \glll nà mùdã̂ wɔ́ɔ̀ wɛ́ɛ̀ bɛ́sɛ̀ báà tù wû \\
      nà m-ùdã̂ w-ɔ́ɔ̀ wɛ́ɛ̀ b-ɛ́sɛ̀ báà tù wû \\
        {\COM} {\N}1-woman 1-{\POSS}.2\textsc{sg} {\EXCL} 2-all 2.{\COP} inside there \\
    \trans `With your wife, so all are inside there''.'
 
\exN\label{n161}
  \glll lígɛ̀ sâ Nzàmbí nyɛ̀ mɛ́dɛ́ \\
    lígɛ sâ Nzàmbí nyɛ̀ mɛ́dɛ́ \\
       stay only $\emptyset$1.{\PN} 1.{\SBJ} self  \\
    \trans `Only Nzambi [the victim of the revenge] himself stays [outside].'
 
\exN\label{n162}
  \glll yɔ́ɔ̀ Nzàmbí sá mpù \\ 
        yɔ́ɔ̀ Nzàmbí sâ-H mpù \\
        so $\emptyset$1.{\PN} do-{\R} like.this \\
    \trans `So Nzambi does like this.'
 
\exN\label{n163} 
  \glll à kɛ̃́ɛ̃̀ nyî pɛ̀ dyúwɔ̀ à dígɛ̃́ɛ̃̀ à díg-â dígɛ̃́ɛ̃̀\\
        a kɛ̃́ɛ̃̀ nyî pɛ̀ dyúwɔ̀ a dígɛ̃́ɛ̃̀ a dígɛ̃́ɛ̃̀ a dígɛ̃́ɛ̃̀\\
       1.{\PST}1 go.{\COMPL} enter there on.top 1.{\PST}1  watch.{\COMPL} 1.{\PST}1  watch.{\COMPL} 1.{\PST}1 watch.{\COMPL}\\
    \trans `He went inside there on top and watched and watched and watched.'
 
\exN\label{n164}
  \glll kì nâ nzá nyɛ́ mɛ̂ \\
      kì nâ nzá nyɛ̂-H mɛ̂ \\
       say {\COMP} who see-{\R} 1\textsc{sg}.{\OBJ}  \\
    \trans `[He] says: ``Who sees me?''.'
 
\exN\label{n165}
  \glll yá nyɛ́-lɛ́ yá nyɛ́-lɛ́ wɔ́ɔ̀ \\
    ya-H nyɛ́-lɛ́ ya-H nyɛ́-lɛ́ wɔ́ɔ̀ \\
         1\textsc{pl}-\textsc{prs} see-{\NEG} 1\textsc{pl}-\textsc{prs} see-{\NEG} 2\textsc{sg}.{\OBJ}[Kwasio] \\
    \trans ``We don't see, we don't see you''.'
 
\exN\label{n166}
  \glll nyɛ̀ nâ àwâ \\
     nyɛ nâ àwâ \\
       1.{\SBJ}  {\COMP} thanks  \\
    \trans `He [says]: ``Thanks''.'
 
\exN\label{n167}
  \glll nyàá sùbɔ̀ èsã̂s ɛ́ dyúwɔ̀ \\
        nyàá sùbɔ èsã̂s ɛ́ dyúwɔ̀ \\
       1.{\INCH} pour $\emptyset$1.fuel {\LOC} $\emptyset$7.top  \\
    \trans `He starts pouring fuel on top.'
 
\exN\label{n168}
  \glll wùùùù wùùùù \\
      wùùùù wùùùù \\
        {\IDEO}:pouring {\IDEO}:pouring \\
    \trans `[depiction of pouring].'
 
\exN\label{n169}
  \glll álè  \\
       álè  \\
        ok[French] \\
    \trans `Ok [French: {\itshape Allez}],'
 
\exN\label{n170}
  \glll kɔ́ɔ̀ nɔ̀ɔ̀ brìkɛ̂ wɛ̂ \\
        kɔ́ɔ̀ nɔ̀ɔ̀ brìkɛ̂ w-ɛ̂ \\
      SEQU  take $\emptyset$1.lighter[French] 1-{\POSS}.3\textsc{sg} \\
    \trans `then takes his lighter,'
 
\exN\label{n171}
  \glll vɛ̀ɛ̀ bɛ́dɛ̀ \\
       vɛ̀ɛ̀ bɛ́dɛ \\
       only light  \\
    \trans `just lighting [the house].'
 
\exN\label{n172}
  \glll tèèè uf \\
       tèèè uf \\
       {\IDEO}:waiting {\IDEO}:ignition  \\
    \trans `[depiction of waiting and then the flame].'
 
\exN\label{n173} 
  \glll mùdì kí tàtɔ̀ wúó \\
       m-ùdì kí tàtɔ wú-o-H \\
        {\N}1-person {\NEG}.{\IMP}  scream there-{\VOC}-{\DIST} \\
    \trans ``Nobody scream over there!''.'
 
\exN\label{n174}
  \glll áá nyáò áá táò \\
       áá nyá-ò áá tá-ò \\
       {\EXCL} {\N}1-mother-{\VOC} {\EXCL} {\N}1-father-{\VOC}  \\
    \trans ``Oh mother, oh father!''.'
 
\exN\label{n175}
  \glll nâ wɔ̀m mùdì núú jí nâ wɔ̀m\\
       nâ wɔ̀m m-ùdì núú jì-H nâ wɔ̀m\\
        {\COMP} {\IDEO}:silence {\N}1-person 1.{\DEM}.{\DIST} stay-{\R} {\COMP} {\IDEO}:silence\\
    \trans ``Be there silence, that person stay silent!'
 
\exN\label{n176}
  \glll màà mâ \\
      m-àà mâ \\
        ma6-thing 6.{\DEM}.{\PROX} \\
    \trans `These things. . .'
 
\exN\label{n177}
  \glll ɛ́ mùdì nɔ́gá núù lígɛ́ vâ \\
       ɛ́ m-ùdì nɔ́-gá núù lígɛ-H vâ \\
       {\LOC} {\N}1-person 1-other 1.{\DEM}.{\PROX} stay-{\R} here  \\
    \trans `Is there any person left here?''.'
 
\exN\label{n178}
  \glll lèkfúdɛ̀  \\
        le-kfúdɛ̀  \\
       le5-idiot  \\
    \trans `Idiot!'
 
\exN\label{n179} 
  \glll à bwàá yɛ́ɛ́ kɛ́ jì mpù \\
       a bwàà-H yɛ́ɛ́ kɛ̀-H jì mpù \\
        1 have-{\R} then? go-{\R} stay like.this \\
    \trans `He [the other Nzambi] has gone and stood like this.'
 
\exN\label{n180}
  \glll nyɛ̀ nâ mɛ̀ɛ́ bɛ́lɛ́ wû \\
       nyɛ nâ mɛ̀ɛ́ bɛ̀-lɛ wû \\
        1.{\SBJ}  {\COMP} 1\textsc{sg}.\textsc{prs}.{\NEG} be-{\NEG} there \\
    \trans `He: ``I'm not there''.'
\end{exe}

\noindent Nze: 
\begin{exe}[(N234)] 
\exN\label{n181}
  \glll yà! \\
         yà \\
        yes[German] \\
    \trans `Yes!'
\end{exe}

\noindent Tata:

\begin{exe}[(N234)]
\exN\label{n182}
  \glll mìntángánɛ́ mí múà vìdɛ̀gà dé \\
       mi-ntángánɛ́ mi-H múà vìdɛga dé \\
       mi4-white.person 4-\textsc{prs} be.almost turn {\LOC}  \\
    \trans `They are about to turn into white people.'
 
\exN\label{n183}
  \glll bõ̀ mpɔ̀ngɔ̀ sílɛ̃́ɛ̃̀ \\
        bõ̀ mpɔ̀ngɔ̀ sílɛ̃́ɛ̃̀ \\
      Ok[French] $\emptyset$7.generation finish.{\COMPL} \\
    \trans `Ok, the generation has been wiped out,'
 
\exN\label{n184}
  \glll nà béè bànáyɛ̂yɛ̂ \\
         nà béè ba-náyɛ̂yɛ̂ \\
        {\CONJ} 2\textsc{pl}.{\COP} ba2-bleached.out  \\
    \trans `and you are bleached out [white].'
 
\exN\label{n185}
  \glll ɛ́ mpù mbúmbù núú láá mɛ̂ nâ \\
        ɛ́ mpù mbúmbù núú láà-H mɛ̂ nâ \\
        {\LOC} like.this $\emptyset$1.namesake 1.{\DEM}.{\DIST} tell-{\R} 1\textsc{sg}.{\OBJ} {\COMP} \\
    \trans `Like this, that namesake tells me that,'
 
\exN\label{n186}
  \glll báà sâ nâ lèfû lèvúdũ̂ \\
          báà sâ nâ le-fû lè-vúdũ̂ \\
       2.{\FUT} do {\COMP} le5-day 5-one  \\
    \trans `they will make that one day,'
 
\exN\label{n187}
  \glll báà dyâ wû \\
       báà dyâ wû \\
       2.{\FUT} sleep there  \\
    \trans `they will sleep there.'
 
\exN\label{n188}
  \glll wɛ́ dyúwɔ́ mpù bàmìntùlɛ̀ bɔ́gá bá tsígɛ̀ tsùk-tsùk-tsùk\\
         wɛ-H dyúwɔ-H mpù ba-mìntùlɛ̀ bɔ́-gá ba-H tsígɛ tsùk-tsùk-tsùk\\
        2\textsc{sg}-\textsc{prs} hear-{\R} like.this ba2-mouse 2-other 2-\textsc{prs} take.off {\IDEO}:rustling\\
    \trans `You hear how the mice take off [depiction of noise of mice].'
 
\exN\label{n189}
  \glll àà nàmɛ́nɔ́ bwáà dè nàmɛ́nɔ́ \\
        àà nàmɛ́nɔ́ bwáà dè nàmɛ́nɔ́ \\
       {\EXCL} tomorrow 2\textsc{pl}.{\FUT} eat tomorrow  \\
    \trans ``Ah, tomorrow you will eat [speaking to the mice], tomorrow.'
 
\exN\label{n190}
  \glll bwáà pã́ã̀ ngâ dyà nà pówàlà wû \\
        bwáà pã́ã̀ ngâ dyà nà pówàlà wû \\
        2\textsc{pl}.{\FUT} do.first {\PL}  sleep {\COM} $\emptyset$7.calm there \\
    \trans `You [the mice] will first sleep quietly there.'
 
\exN\label{n191}
  \glll bé dúú vũ̀ũ̀ \\
      be-H dúu-H vũ̀ũ̀ \\
        2\textsc{pl}-\textsc{prs} must.not-{\R} worry \\
    \trans `Don't worry.'
 
\exN\label{n192} 
  \glll bèdéwɔ̀ bínɛ́ mɛ̀ nzíí byɔ̂ gyámbɔ̀ \\
      be-déwɔ̀ b-ínɛ́ mɛ nzíí by-ɔ̂ gyámbɔ̀ \\
       be8-food 8-{\POSS}.2\textsc{pl} 1\textsc{sg} {\PROG}.\textsc{prs} 8-{\OBJ} prepare  \\
    \trans `Your food, I am preparing it''.'
\end{exe}

\noindent Nze:

\begin{exe}[(N234)]
\exN\label{n193}
  \glll yééééé \\
        yééééé \\
        {\EXCL} \\
    \trans `[sound of disappreciation]!'
\end{exe}

\noindent Tata:

\begin{exe}[(N234)]
\exN\label{n194} 
  \glll wùf-wùf \\
        wùf-wùf \\
        {\IDEO}:pitter-patter \\
    \trans `[depiction of sound when mice are walking].'
 
\exN\label{n195}
  \glll bàmìntùlɛ̀ bá lèmbó nâ màmbò má bvùlɛ́ \\
        ba-mìntùlɛ̀ ba-H lèmbo-H nâ m-àmbò má bvùlɛ́ \\
     ba2-mouse 2-\textsc{prs} know-{\R} {\COMP} ma6-thing 6:{\ATT}  $\emptyset$8.night    \\
    \trans `The mice know that these are things of the night.'
 
\exN\label{n196} 
  \glll bá múà gyɛ́sɔ̀ bédéwɔ̀ byáwɔ́ \\
         ba-H múà gyɛ́sɔ H-be-déwɔ̀ by-áwɔ́ \\
      2-\textsc{prs} be.almost search {\OBJ}.{\LINK}-be8-food 8-{\POSS}.3\textsc{pl}  \\
    \trans `They are about to look for their food.'
 
\exN\label{n197}
  \glll ùwù-ùwù bàmìntùlɛ̀ báà wû \\
        ùwù-ùwù ba-mìntùlɛ̀ báà wû \\
        {\IDEO}:rustling ba2-mouse 2.{\COP} there \\
    \trans `[depiction of sound of mice] The mice are there.'
 
\exN\label{n198}
  \glll Nzàmbí nzí kàmbɔ̀ \\
        Nzàmbí nzí kàmbɔ \\
        $\emptyset$1.{\PN} {\PROG}.{\PST}  defend \\
    \trans `Nzambi was defending [the house, in vain].'
 
\exN\label{n199}
  \glll àá bámálá tɔ́bá mpfùmɔ̀ nà pámò mɛ́nɔ́ \\
       àá bámala-H tɔ́bá mpfùmɔ̀ nà pámo mɛ́nɔ́ \\
       1.{\INCH} scold-{\R} since  $\emptyset$3.midnight {\CONJ} arrive $\emptyset$7.morning \\
    \trans `He is starting to scold from midnight until the morning.'
 
\exN\label{n200} 
  \glll à tɛ́lɛ́ sâ dɛ́ndì témɔ́ \\
         a tɛ́lɛ-H sâ d-ɛ́ndì témɔ́ \\
        1.{\PST}1 stand-{\R} only le5-courtyard middle \\
    \trans `He just stood in the middle of the courtyard.'
 
\exN\label{n201}
  \glll mɛ́nɔ́ wɛ̀ɛ̀ nyɛ̂ nâ mbúmbù nzíí kì nâ \\
        mɛ́nɔ́ wɛ̀ɛ̀ nyɛ̂ nâ mbúmbù nzíí kì nâ \\
       $\emptyset$7.morning 2\textsc{sg}.{\FUT} see {\COMP} $\emptyset$1.namesake {\PROG}.\textsc{prs} say {\COMP}  \\
    \trans `In the morning you will see that namesake is saying that,'
 
\exN\label{n202}
  \glll bímbú lékàá lé wùlà yá Nadine lɔ́ sɛ́mbɔ̀ vâ \\
         bímbú le-kàá lé wùlà yá Nadine lɔ́ sɛ́mbɔ vâ \\
        $\emptyset$7.amount le5-kind 5:{\ATT}  $\emptyset$7.time 7:{\ATT}  $\emptyset$1.{\PN} {\RETRO}  arrive here \\
    \trans `the amount of time that Nadine just arrived here, [when Nadine just arrived here]'
 
\exN\label{n203}
  \glll  Nzàmbí vɛ̀ɛ̀kɛ́ yɔ́ɔ̀ mbɛ̀ \\
        Nzàmbí vɛ̀ɛ̀kɛ́ yɔ́ɔ̀ mbɛ̀ \\
        $\emptyset$1.{\PN} go[Bulu] open[Bulu] $\emptyset$3.door  \\
    \trans `Nzambi just goes open the door.'
\end{exe}

\noindent Mambi:

\begin{exe}[(N234)]
\exN\label{n204} 
  \glll vɛ̀ɛ̀ vɛ̀ɛ̀ vɛ̀ɛ̀ vɛ̀ɛ̀ kíngɛ̀lɛ̀ kíngɛ̀lɛ̀ kíngɛ̀lɛ̀ \\
         vɛ̀ɛ̀ vɛ̀ɛ̀ vɛ̀ɛ̀ vɛ̀ɛ̀ kíngɛlɛ kíngɛlɛ kíngɛlɛ \\
        only only only only become.stiff become.stiff become.stiff \\
    \trans `Only, only, only, only stiff, stiff, stiff.'
\end{exe}

\noindent Tata:

\begin{exe}[(N234)] 
\exN\label{n205}
  \glll bènké'é \\
        be-nké'é \\
        be8-scream \\
    \trans `Screams.'
 
\exN\label{n206}
  \glll Nzàmbí nké'é yá Nzàmbí núù vɛ̀ vâ\\
         Nzàmbí nké'é yá Nzàmbí núù vɛ̀ vâ \\
      $\emptyset$1.{\PN} $\emptyset$7.scream 7:{\ATT}  $\emptyset$1.{\PN} 1.{\DEM}.{\PROX} give here   \\
    \trans `Nzambi, the scream that Nzambi gave here.'
 
\exN\label{n207}
  \glll à nzíí kìyà nké'é \\
          a nzíí kìya nké'é \\
        1 {\PROG}.\textsc{prs} give $\emptyset$7.scream \\
    \trans `He is screaming.'
 
\exN\label{n208}
  \glll ká á dígɛ́ nâ [gesture] á nyɛ́ mbúmbù wɛ́ɛ̀ á pámò \\
        ká a-H dígɛ-H nâ [gesture] a-H nyɛ̂-H mbúmbù w-ɛ̂ a-H pámo \\
       when 1-\textsc{prs} look-{\R} {\COMP} [gesture]  1-\textsc{prs} see-{\R} $\emptyset$1.namesake 1-{\POSS}.3\textsc{sg} 1-\textsc{prs} arrive \\
    \trans `When he looks like [gesture], he sees his namesake who arrives.'
\end{exe}

\noindent Aminu:

\begin{exe} 
\exN\label{n209}
  \glll mbúmbù wà lèbɔ́'ɔ̀ \\
         mbúmbù wà le-bɔ́'ɔ̀ \\
        $\emptyset$1.namesake 1:{\ATT}  le5-breadfruit \\
    \trans `The namesake of the breadfruit.'
\end{exe}

\noindent Tata:

\begin{exe}[(N234)]
\exN\label{n210}
  \glll àá à pámṍõ̀ \\
       àá a pámṍõ̀ \\
       {\EXCL} 1.{\PST}1 arrive.{\COMPL}  \\
    \trans `Yes, he has arrived,'
 
\exN\label{n211}
  \glll wà màlɛ́ndí\\
       wà ma-lɛ́ndí \\
         1:{\ATT}  ma6-palm.tree \\
    \trans `of the palm trees.'
 
\exN\label{n212}
  \glll yɔ́ɔ̀ á sɛ́mbɔ̀ \\
          yɔ́ɔ̀ a-H sɛ́mbɔ \\
         so 1-\textsc{prs} arrive \\
    \trans `So he arrives.'
 
\exN\label{n213}
  \glll mbúmbù ɛ́ ná \\
        mbúmbù ɛ́ ná \\
        $\emptyset$1.namesake {\LOC} how  \\
    \trans `Namesake, how is it?'
 
\exN\label{n214} 
  \glll mbúmbù lèbvúú léè nlémò dé \\
         mbúmbù le-bvúú léè nlémò dé \\
        {\N}1.namesake le5-anger 5.{\COP} $\emptyset$3.heart {\LOC}  \\
    \trans `The namesake is angry [lit. has anger in his heart].'
 
\exN\label{n215}
  \glll mɛ̀ɛ́ lémbòlɛ̀ bàsɔ̃́ bɔ́ɔ̀ ɛ́ mpù báà \\
       mɛ̀ɛ́ lémbo-lɛ̀ bà-sɔ̃́ b-ɔ́ɔ̀ ɛ́ mpù báà \\
       1\textsc{sg}.\textsc{prs}.{\NEG} know-{\NEG} ba2-father 2-{\POSS}.2\textsc{sg} {\LOC} like.this 2.{\COP} \\
    \trans ``I don't know how your fathers are.'
 
\exN\label{n216}
  \glll mɛ̀ɛ́ lémbòlɛ̀ ɛ́ mpù báà ndáwɔ̀ dé tù dénè\\
      mɛ̀ɛ́ lémbo-lɛ̀ ɛ́ mpù báà ndáwɔ̀ dé tù dénè\\
        1\textsc{sg}.\textsc{prs}.{\NEG} know-{\NEG} {\LOC} like.this 2.{\COP} $\emptyset$9.house {\LOC} inside today[Bulu]\\
    \trans `I don't know how they are in the house today''.'
\end{exe}

\noindent Ada:

\begin{exe}[(N234)] 
\exN\label{n217} 
  \glll nâ wɛ̀ sílɛ̃́ɛ̃̀ nyàà dyùù mpɔ̀ngɔ̀ yá bùdì\\
        nâ wɛ sílɛ̃́ɛ̃̀ nyàà dyùù mpɔ̀ngɔ̀ yá b-ùdì\\
         {\COMP} 2\textsc{sg}.{\PST}1 finish.{\COMPL} shit kill $\emptyset$7.generation 7:{\ATT}  ba2-person\\
    \trans `That you have completely killed a generation of people!'
\end{exe}

\noindent Tata:

\begin{exe}[(N234)]
\exN\label{n218}
  \glll bá lɔ́ sâ ná \\
        ba-H lɔ́ sâ ná \\
         2-\textsc{prs} {\RETRO}  do how \\
    \trans `How did they do [that]?'
 
\exN\label{n219}
  \glll bùdì bà sílɛ̃́ɛ̃̀ mɛ̂ wɛ̀ ndáwɔ̀ tù vâ \\
        b-ùdì ba sílɛ̃́ɛ̃̀ mɛ̂ wɛ̀ ndáwɔ̀ tù vâ \\
       ba2-person 2.{\PST}1 finish.{\COMPL} 1\textsc{sg}.{\OBJ} die $\emptyset$9.house inside here  \\
    \trans `The people have all died here inside the house.'
 
\exN\label{n220}
  \glll ɛ́ mpù wɛ̀ nzí mɛ̂ láà \\
       ɛ́ mpù wɛ nzí mɛ̂ láà \\
       {\LOC} like.this 2\textsc{sg} {\PROG}.{\PST}  1\textsc{sg}.{\OBJ} tell   \\
    \trans ``You were telling me like this.'
 
\exN\label{n221}
  \glll kánâ mɛ̀ kɔ̀bɛ́ ndáà tsì \\
        kánâ mɛ kɔ̀bɛ-H ndáà tsì \\
        or 1\textsc{sg}.{\PST}1 break-{\R} also $\emptyset$7.interdiction  \\
    \trans `Or I also broke the prohibition,'
 
\exN\label{n222}
  \glll mɛ̀ɛ́ lémbólɛ́ \\
        mɛ̀ɛ́ lémbo-lɛ \\
        1\textsc{sg}.\textsc{prs}.{\NEG} know-{\NEG}  \\
    \trans `I don't know''.'
 
\exN\label{n223}
  \glll yɔ́ɔ̀ Nzàmbí kí nâ mbúmbù \\
        yɔ́ɔ̀ Nzàmbí kì-H nâ mbúmbù \\
         so $\emptyset$1.{\PN} say-{\R} {\COMP} $\emptyset$1.namesake \\
    \trans `So Nzambi says: ``Namesake,'
 
\exN\label{n224}
  \glll jíì sí vâ \\
        jíì sí vâ \\
         sit.{\IMP}  down here \\
    \trans `sit down here''.'
 
\exN\label{n225}
  \glll nóò \\
        nóò \\
         {\EXCL} \\
    \trans `No!'
 
\exN\label{n226}
  \glll béè bùdì bá vúdũ̂ ndí bwáá gyɛ́sɔ́ mápè'è\\
        béè b-ùdì bá vúdũ̂ ndí bwáa-H gyɛ́sɔ-H H-ma-pè'è\\
         2\textsc{pl}.{\COP} ba2-person 2:{\ATT}  one but 2\textsc{pl}-\textsc{prs} search-{\R} {\OBJ}.{\LINK}-ma6-wisdom\\
    \trans `You are the same people, but you are looking for wisdom.'
\end{exe}

\noindent Aminu:

\begin{exe} 
\exN\label{n227}
  \glll ɛ̀hɛ̂ \\
       ɛ̀hɛ̂ \\
        {\EXCL}  \\
    \trans `Exactly!'
\end{exe}

\noindent Tata:

\begin{exe}[(N234)] 
\exN\label{n228}
  \glll wɛ̀ lèmbṍõ̀ sâ bányá màmbò nâ ká mɛ́ lúmɔ́ wɛ̂ nláà nâ \\
       wɛ lèmbṍõ̀ sâ H-ba-nyá m-àmbò nâ ká mɛ-H lúmɔ-H wɛ̂ nláà nâ \\
        2\textsc{sg}.{\PST}1 know.{\COMPL}  do {\OBJ}.{\LINK}-ba2-important ma6-thing {\COMP} if 1\textsc{sg}-\textsc{prs} send-{\R} 2\textsc{sg}.{\OBJ} $\emptyset$3.message {\COMP} \\
    \trans ``You know to do the important things that if I send you the message that,'
 
\exN\label{n229}
  \glll mbúmbù kòlɛ̂ mɛ̂ \\
       mbúmbù kòlɛ̂ mɛ̀ \\
         $\emptyset$1.namesake, help.{\IMP}  1\textsc{sg}.{\OBJ} \\
    \trans `namesake, help me,'
 
\exN\label{n230}
  \glll ɛ́ tè wɛ̀gà wɛ́ njí sâ \\
        ɛ́ tè wɛ̀-gà wɛ-H njì-H sâ \\
        {\LOC} there 2\textsc{sg}.{\SBJ}-{\CONTR} 2\textsc{sg}-\textsc{prs} come-{\R} do  \\
    \trans `and there you, you come to make,'
 
\exN\label{n231}
  \glll mbvúndá ɛ́ ndzǐ vâ \\
       mbvúndá ɛ́ ndzǐ vâ \\
        $\emptyset$9.trouble {\LOC} $\emptyset$9.path here  \\
    \trans `trouble on the way here.'
 
\exN\label{n232} 
  \glll ndí wɛ́ lèmbó nâ mbvúndá nyíì bvúdà nà mbvúndá\\
        ndí wɛ-H lèmbo-H nâ mbvúndá nyíì bvúda nà mbvúndá\\
         but 2\textsc{sg}-\textsc{prs} know-{\R} {\COMP} $\emptyset$9.trouble 9.{\FUT} fight {\COM} $\emptyset$9.trouble\\
    \trans `But you know that violence will create more violence''.'
 
\exN\label{n233}
  \glll Nzàmbí wà nû kɔ́ɔ̀ kìyà léwê \\
         Nzàmbí wà nû kɔ́ɔ̀ kìya H-le-wê \\
        $\emptyset$1.{\PN} 1:{\ATT}  1.{\DEM}.{\PROX} only give {\OBJ}.{\LINK}-le5-cry  \\
    \trans `This Nzambi only gives a cry.'
 
\exN\label{n234}
  \glll bààm \\
         bààm \\
         {\IDEO}:finishing \\
    \trans `[finish].'
 
\exN\label{n235}
  \glll Nzàmbí gyĩ́ĩ̀ \\
        Nzàmbí gyĩ́ĩ̀ \\
         $\emptyset$1.{\PN} cry.{\COMPL} \\
    \trans `Nzambi has cried.'
 
\exN\label{n236}
  \glll à gyĩ́ĩ̀ \\
         a gyĩ́ĩ̀ \\
         1.{\PST}1 cry.{\COMPL} \\
    \trans `He has cried.'
 
\exN\label{n237}
  \glll lèkfúdɛ̀ à nzí bíyɔ̀ nlô pɛ́ɛ́ \\
      le-kfúdɛ̀ a nzí bíyɔ nlô pɛ́-ɛ́ \\
         le5-idiot 1 {\PROG}.{\PST}  hit $\emptyset$3.head there-{\DIST} \\
    \trans `The idiot was hitting his head there.'
 
\exN\label{n238}
  \glll áà bɛ́ à bó nà màbádò nyúlɛ̀ \\
        áà bɛ̀-H a bô-H nà ma-bádò nyúlɛ̀ \\
        1.{\PST}2 be-{\R} 1.{\PST}1 lie-{\R} {\COM} ma6-open.wound $\emptyset$9.body  \\
    \trans `He was being lying with open wounds on the body.'
 
\exN\label{n239}
  \glll nyɛ̀ nâ yáà mɛ́ láà \\
      nyɛ nâ yáà mɛ-H láà \\
        1.{\SBJ}  {\COMP} yes[German] 1\textsc{sg}-\textsc{prs} say  \\
    \trans `He [says]: ``Yes, I say'
 
\exN\label{n240}
  \glll nâ sá wɛ́ sá nɔ́gá mùdì \\
        nâ sá wɛ-H sâ-H nɔ́-gá m-ùdì \\
        {\COMP} $\emptyset$7.thing 2\textsc{sg}-\textsc{prs} do-{\R} 1-other {\N}1-person  \\
    \trans `the thing that you do to another person,'
 
\exN\label{n241}
  \glll àà yɔ̂ wɛ̂ nyè \\
         àà y-ɔ̂ wɛ̂ nyè \\
         1.{\FUT} 7-{\OBJ} 2\textsc{sg} return \\
    \trans `he will return to you''.'
 
\exN\label{n242}
  \glll yɔ́ɔ̀ Nzàmbí wà nû \\
         yɔ́ɔ̀ Nzàmbí wà nû \\
         so $\emptyset$1.{\PN} 1:{\ATT}  1.{\DEM}.{\PROX} \\
    \trans `So this Nzambi,'
 
\exN\label{n243}
  \glll sá á sá nɔ́nɛ́gá \\
      sá a-H sâ-H n-ɔ́nɛ́gá \\
        $\emptyset$7.thing 1-\textsc{prs} do-{\R} 1-other  \\
    \trans `the thing that he does to the other,'
 
\exN\label{n244} 
  \glll yɔ́ɔ̀ nyɛ̀gà á nyé nyɛ̂ \\
        yɔ́ɔ̀ nyɛ̀-gà a-H nyè-H nyɛ̂ \\
         so 1.{\SBJ}-{\CONTR} 1-\textsc{prs} return-{\R} 1.{\OBJ} \\
    \trans `so the other returns [it] to him,'
 
\exN\label{n245}
  \glll ngvùndɔ̀ nyà tè \\
       ngvùndɔ̀ nyà tè \\
        $\emptyset$9.vengeance 9:{\ATT}  there  \\
    \trans `that vengeance.'
 
\exN\label{n246}
  \glll ɛ́ vâ màlíyɔ̀ má fúgɛ̀ \\
       ɛ́ vâ ma-líyɔ̀ ma-H fúgɛ \\
         {\LOC} here ma6-clearing 6-\textsc{prs} end \\
    \trans `Here, the killing [lit. clearing] ends.'
 
\exN\label{n247}
  \glll ɛ́ vâ màkwɛ̀lɔ̀ má fúgɛ̀ \\
         ɛ́ vâ ma-kwɛ̀lɔ̀ ma-H fúgɛ \\
         {\LOC} here ma6-felling 6-\textsc{prs} end \\
    \trans `Here, the massacre [lit. felling] ends,'
 
\exN\label{n248}
  \glll vɛ̀ɛ̀ vâ \\
        vɛ̀ɛ̀ vâ \\
       only here   \\
    \trans `only here.'
 
\exN\label{n249}
  \glll kàndá wɛ́ ndɛ̀ \\
        kàndá wɛ́ ndɛ̀ \\
        $\emptyset$7.proverb {\ID} {\ANA}  \\
    \trans `This is the story.'
 
\exN\label{n250}
  \glll bàmpámbó bá líyɛ̀ líyɛ̀ \\
       ba-mpámbó ba-H líyɛ líyɛ \\
         ba2-ancestor 2-\textsc{prs} leave leave \\ 
    \trans `The ancestors leave [the story to us],'
 
\exN\label{n251}
  \glll nâ yá tã́ã̀tà békàndá bé tè \\
        nâ ya-H tã́ã̀-tà H-be-kàndá bé tè \\
        {\COMP} 1\textsc{pl}-\textsc{prs} tell-tell.{\SBJV}  {\OBJ}.{\LINK}-be8-proverbs 8:{\ATT}  there  \\
    \trans `so that we tell the stories there.'
 
\exN\label{n252}
  \glll byɔ̂ wɛ́ bíndɛ̀ \\
       by-ɔ́ wɛ́ bí-ndɛ̀ \\
         8-{\OBJ} {\ID} 8-{\ANA} \\
    \trans `That's them.'
 
\exN\label{n253}
  \glll byɔ̂ bé vɛ́ bíì màpè'è \\
        by-ɔ̂ be-H vɛ̀-H bíì ma-pè'è \\
        8-{\OBJ} 8-\textsc{prs} give-{\R} 1\textsc{pl}.{\OBJ} ma6-wisdom  \\
    \trans `They give us wisdom.'
\end{exe}

\noindent Aminu:

\begin{exe}[(N234)]
\exN\label{n254}
  \glll ká kɛ̃́ɛ̃́sɔ́ yí wúmbɛ́ wɛ̂ dyɔ̀dɛ̀ \\
        ká kɛ̃́ɛ̃́sɔ́ yi-H wúmbɛ-H wɛ̂ dyɔ̀dɛ \\
         if $\emptyset$7.égal 7-\textsc{prs} want-{\R} 2\textsc{sg}.{\OBJ} deceive \\
    \trans `If somebody wants to deceive you,'
 
\exN\label{n255}
  \glll wɛ́ kílɔ̀wɔ̀ \\
        wɛ-H kílɔwɔ \\
         2\textsc{sg}-\textsc{prs} be.vigilant \\
    \trans `you are vigilant.'
 
\exN\label{n256}
  \glll wɛ́ kí nâ ɛ́y \\
       wɛ-H kì-H nâ ɛ́y \\
        2\textsc{sg}-\textsc{prs} say-{\R} {\COMP} {\EXCL}  \\
    \trans `You say: ``Hey!''.'
\end{exe}

\noindent Djiedjhie:

\begin{exe}[(N234)]
\exN\label{n257}
  \glll yí bálɛ́ gyà \\
        yi-H bálɛ-H gyà \\
        7-\textsc{prs} surpass-{\R} $\emptyset$7.length   \\
    \trans `This is too long.'
\end{exe}


\pagebreak

\section{Conversation in the village Ngolo}
\label{sec:Conversation}

This text is a guided conversation between several speakers in the village Ngolo. It was recorded on video in May 2011 and is the first official conversation the DoBeS team had with the Bagyeli in Ngolo.  First, the chief Nze introduces himself and the village and states that they wish to have tin roofs instead of raffia roofs. He further complains that people from NGOs come and go, but that they are not really helpful. Occasionally, Nze is interrupted by Severin in Ngumba (northern Kwasio dialect) who serves as an interpreter and loosely guides the conversation. The topic then shifts to the construction of the port and its impact on the people of Ngolo, who fear that roads will be built and, as a consequence, their houses and plants will be destroyed. After Nze talks about his plans to move to his former settlement further in the forest, Severin encourages Mambi (a young man in his early twenties) to talk about himself. Mambi explains the problems they encounter with their Bulu neighbors. According to him, the Bulu contest their land rights, quarrel about money with them and threaten them with physical violence.
Nze shortly talks about his marital status, i.e.\ that he is married and has two children before Mambi continues about their wish to obtain electricity in the village. The third speaker in the conversation is Mama, about 17 years old, who introduces himself as an orphan, having lost his father while his mother lives in another village. Then, Mambi and Nze talk again about the future of their village, their desire to obtain tin-roofed houses, and the problems with the Bulu.

\noindent Nze:

\begin{exe}[(C234)] 
\exC\label{01}
  \glll     mɛ́ wúmbɛ́ lɛ́ɛ̀ nà bɔ̂\\
           mɛ-H wúmbɛ-H lɛ́ɛ̀ nà bɔ̂ \\
              1\textsc{sg}-\textsc{prs} want-{\R} talk[Kwasio] {\COM} 2.{\OBJ}   \\
    \trans `I want to talk with them.'
 
\exC\label{02}
  \glll     yí ntɛ́gɛ̀lɛ̀ ɔ̀ dyúwɔ́ mɔ̀ \\
            yi-H ntɛ́gɛlɛ ɔ dyúwɔ-H mɔ̀ \\
             7-\textsc{prs} disturb  2\textsc{sg}[Kwasio] hear-{\R} {\COMPL}  \\
    \trans `It disturbs, have you understood?'
 
\exC\label{03}
  \glll      yí ntɛ́gɛ̀lɛ̀ vɛ̀dáà mɛ́ sùmbɛ́lɛ́ bê \\
           yi-H ntɛ́gɛlɛ vɛ̀dáà mɛ-H sùmbɛlɛ-H bê \\
              7-\textsc{prs} disturb but[Bulu] 1\textsc{sg}-\textsc{prs} greet[Kwasio]-{\R} 2\textsc{pl}.{\OBJ} \\
    \trans `That disturbs, but I greet you.'
 
\exC\label{04}
  \glll     mɛ́ sùmɛ́lɛ́ bê ndɛ̀náà \\
            mɛ-H sùmɛlɛ-H bê ndɛ̀náà \\
              1\textsc{sg}-\textsc{prs} greet-{\R} 2\textsc{pl}.{\OBJ} like.that   \\
    \trans `I greet you like this.'
 
\exC\label{05}
  \glll  jínɔ̀ lɛ́ kwàdɔ̀ yã̂ yíì Ngòló \\
         j-ínɔ̀ lɛ́ kwàdɔ̀ y-ã̂ yíì Ngòló \\
    le5-name 5:{\ATT}  $\emptyset$7.village 7-{\POSS}.1\textsc{sg} 7.{\COP} $\emptyset$3.{\PN}  \\
    \trans `The name of my village is Ngolo.'
 
\exC\label{06}
  \glll  pándɛ̀ té nà té mɛ̀ jínɔ̀ ná Nzɛ̀ \\
          pándɛ té nà té mɛ j-ínɔ̀ ná Nzɛ̀ \\
              arrive  $\emptyset$7.position {\CONJ} $\emptyset$7.position 1\textsc{sg} le5-name SIM $\emptyset$1.{\PN}   \\
    \trans `Having arrived immediately, my name is Nze.'
 
\exC\label{07}
  \glll     kfúmà wà Nkóòlóng \\
          kfúmà wà Nkóòlóng \\
              $\emptyset$1.chief 1:{\ATT}  $\emptyset$3.{\PN} [Bulu] \\
    \trans `The chief of Ngolo [uses exonym].'
 
\exC\label{08}
  \glll     kfúmà wà Nkóòlóng Nzɛ̀ \\
           kfúmà wà Nkóòlóng Nzɛ̀ \\
              $\emptyset$1.chief 1:{\ATT}  $\emptyset$3.{\PN} [Bulu] $\emptyset$1.{\PN} \\
    \trans `The chief of Ngolo, Nze.'
\end{exe}

\noindent Mambi:

\begin{exe}[(C234)]  
\exC\label{09}
  \glll     nyɛ̀ wɛ́ nû \\
           nyɛ wɛ́ nû \\
              1.{\SBJ}  {\ID} 1.{\DEM}.{\PROX} \\
    \trans `This is him [Nze].'
 
\exC\label{10} 
  \glll     á páàngɔ́ tálɛ̀ sílɛ̀ mɛ̀ nzíí ná kɛ̀ \\
            a-H páàngɔ-H tálɛ sílɛ mɛ nzíí ná kɛ̀\\
              1-\textsc{prs} do.first[Kwasio]-{\R} begin finish 1\textsc{sg} {\PROG}.\textsc{prs} again go \\ 
    \trans `He starts first to finish [speaking], I'm continuing again [will then speak].'
\end{exe}

\noindent Nze:

\begin{exe}[(C234)]  
\exC\label{11}
  \glll     áà mɛ̀ nzíí ná làwɔ̀ ná \\
            áà mɛ nzíí ná làwɔ ná \\
              yes 1\textsc{sg} {\PROG}.\textsc{prs} still talk still \\
    \trans `Yes, I am still talking.'
 
\exC\label{12}
  \glll     gyí bí yá tfúgà yá tfúgá nà gyí \\
             gyí bí ya-H tfúga ya-H tfúga-H nà gyí\\
              what 1\textsc{pl}.{\SBJ}  1\textsc{pl}-\textsc{prs} suffer 1\textsc{pl}-\textsc{prs} suffer-{\R} {\COM} what \\
    \trans `What do we suffer, we suffer from what?.'
 
\exC\label{13}
  \glll     yá tfúgá nà ngùndyá mpángì \\
            ya-H tfúga-H nà ngùndyá mpángì \\
              1\textsc{pl}-\textsc{prs} suffer-{\R} {\COM} $\emptyset$9.raffia $\emptyset$7.bamboo \\
    \trans `We suffer from the straw, the bamboo.'
 
\exC\label{14}
  \glll    ká yí nyí mɛ̂ mbɔ̀ mpángì yí kùgá nâ nyíì wɛ̀ mbɔ̀ \\
           ká yi-H nyî-H mɛ̂ m-bɔ̀ mpángì yi-H kùga-H nâ nyíì wɛ m-bɔ̀ \\
             when 7-\textsc{prs} enter-{\R} 1\textsc{sg}.{\OBJ} N3-arm $\emptyset$7.bamboo 7-\textsc{prs} can-{\R} {\COMP} enter.{\SBJV}  2\textsc{sg} N3-arm  \\
    \trans `When it goes into my arm . . . the bamboo can sting your arm.'
 
\exC\label{15}
  \glll     yáà fúàlà bígɛ̀ yɔ̃̂ yá vɛ́ \\
          yáà fúala bígɛ yɔ̃̂ yá vɛ́ \\
              1\textsc{pl}.{\FUT} end develop $\emptyset$7.time[Bulu] 7:{\ATT}  which \\
    \trans `When will we end up developing?'
 
\exC\label{16}
  \glll     yá vyã́ã́ kɛ̀ nà kwã̂ mángùndyá wɛ̀ nà ngvùlɛ̀ kɛ̀ sɔ́lɛ̀gà wû nà njí kù ɛ́ sì \\
            ya-H vyã́ã̀-H kɛ̀ nà kwã̂ H-ma-ngùndyá wɛ nà ngvùlɛ̀ kɛ̀ sɔ́lɛga wû nà njì-H kù ɛ́ sì\\
              1\textsc{pl}-\textsc{prs} do.but-H go {\CONJ} cut {\OBJ}.{\LINK}-ma6-raffia 2\textsc{sg} {\COM} $\emptyset$9.strength go fall there {\COM} come-{\R} fall[Kwasio] {\LOC} $\emptyset$9.ground\\
    \trans `We do nothing but go and cut raffia, you are strong to go [and climb a raffia palm tree], tumbling and falling to the ground there.'
 
\exC\label{17}
  \glll    mɛ́ bvú nâ nkwálá wúù tfùndɛ́ mɛ̂ vâ \\
           mɛ-H bvû-H nâ nkwálá wúù tfùndɛ-H mɛ̀ vâ \\
              1\textsc{sg}-\textsc{prs} think-{\R} {\COMP} $\emptyset$3.machete 3.{\PST}2 miss-{\R} 1\textsc{sg}.{\OBJ} here \\
    \trans `I think that the machete had injured [missed to seriously harm] me here.'
 
\exC\label{18}
  \glll    ngùndyá mɛ́ kɛ́ sɔ́lɛ̀gà ngùndyá dyúwɔ̀ \\
          ngùndyá mɛ-H kɛ̀-H sɔ́lɛga ngùndyá dyúwɔ̀ \\
              $\emptyset$9.raffia 1\textsc{sg}-\textsc{prs} go-{\R} chop $\emptyset$9.raffia on.top \\
    \trans `The raffia, I go to chop the raffia on top.'
 
\exC\label{19}
  \glll     áá bíì màndáwɔ̀ má zì yáà mɔ̂ fúàlà bwɛ̂ lèwùlà lé vɛ́\\
          áá bíì ma-ndáwɔ̀ má zì yáà m-ɔ́ fúala bwɛ̂ le-wùlà lé vɛ́\\
              {\EXCL} 1\textsc{pl}.{\OBJ} ma6-house 6:{\ATT}  $\emptyset$7.tin[Bulu] 1\textsc{pl}.{\FUT} 6-{\OBJ} end receive le5-hour 5:{\ATT}  which \\
    \trans `Ah, us, tin houses, when will we receive them?'
 
\exC\label{20}
  \glll     mà bɛ́ vɛ́ \\ 
            ma bɛ̀-H vɛ́ \\
              6.{\PST}1 be-{\R} where \\
    \trans `Where were they?'
 
\exC\label{21}
  \glll     mɛ́ bvú nâ bàmó tè yɔ́ɔ̀ wɛ́ yî \\
          mɛ-H bvû-H nâ bàmó tè y-ɔ́ɔ̀ wɛ́ yî \\
              1\textsc{sg}-\textsc{prs} think-{\R} {\COMP} $\emptyset$7.scar there 7-{\OBJ} {\ID} 7.{\DEM}.{\PROX} \\
    \trans `I think, the scar there is this.'
\end{exe}


\begin{exe}[(C234)] 
\exC\label{22}
  \glll     bwà nzíí kàlànɛ̀ \\
            bwa nzíí kàlanɛ \\
              2\textsc{pl} {\PROG}.\textsc{prs} transmit \\
    \trans `Are you translating?'
 
\exC\label{23}
  \glll     yá lɔ́ fúàlà nà mɛ̀ lɔ́ làwɔ̀ \\
           ya-H lɔ́ fúala nà mɛ lɔ́ làwɔ \\
              1\textsc{pl}-\textsc{prs} {\RETRO}  end {\CONJ} 1\textsc{sg} {\RETRO}  talk \\
    \trans `We just finished and I just spoke.'
 
\exC\label{24}
  \glll     nlã̂ wá zì ndáwɔ̀ nyà zì nyíì mɛ̂ vɛ́ \\
          nlã̂ wá zì ndáwɔ̀ nyà zì nyíì mɛ̀ vɛ́ \\
              $\emptyset$3.story 3:{\ATT}  $\emptyset$7.tin $\emptyset$9.house 9:{\ATT}  tin 9.{\COP} 1\textsc{sg}.{\OBJ} where \\
    \trans `The problem with the tin, where is the tin (roofed) house for me?'
 
\exC\label{25}
  \glll     fàmí wã̂ nyɛ̀ngwɛ́sɛ̀ nâ á bígɛ́ɛ̀ \\
           fàmí w-ã̂ nyɛ̀-ngwɛ́sɛ̀ nâ a-H bígɛ́ɛ̀ \\
              $\emptyset$1.family 1-{\POSS}.1\textsc{sg} 9-entire {\COMP} 1-\textsc{prs} develop.{\SBJV}  \\
    \trans `My whole family, may it develop.'
 
\exC\label{26}
  \glll     wúù vɛ́ \\
         wúù vɛ́ \\
              3.{\COP} where \\
    \trans `Where is it [the story of the tin]?'
 
\exC\label{27}
  \glll     Nkóòlòng nâ wú bígɛ́ɛ̀ \\
        Nkóòlòng nâ wu-H bígɛ́ɛ̀ \\
              $\emptyset$3.{\PN} [Bulu] {\COMP} 3-\textsc{prs} develop.{\SBJV}  \\
    \trans `Nko'olong [name of the village], may it develop.'
 
\exC\label{28}
  \gll   ɔ̀bâj ɔ̀bâj ɔ̀bâj \\
              [straw straw straw]Bulu \\
    \trans `Straw, straw, straw.'
 
\exC\label{29}
  \gll    mɛ́ ngà kɛ́ sɔ́tàn ɛ̀lɛ̀ yɔ́ßɛ̀tɛ̀  \\
              [1\textsc{sg} build go jump tree top]Bulu \\
    \trans `I build and jump up on the tree.'
 
\exC\label{30}
  \gll     fá à ngà bálɛ̀ màvá \\
             [machete 3\textsc{sg} 1\textsc{sg} hurt here]Bulu \\
    \trans `The machete injured me here.'
 
\exC\label{31}
  \glll     yɔ́ɔ̀ mɛ́ wúmbɛ́ mándáwɔ̀ má zì má tɛ́wɔ́'ɔ̀ mɛ̂ vâ ndá zì \\
            yɔ́ɔ̀ mɛ-H wúmbɛ-H H-ma-ndáwɔ̀ má zì ma-H tɛ́wɔ̀ɔ̀ mɛ vâ ndá zì \\
              so 1\textsc{sg}-\textsc{prs} want-{\R} {\OBJ}.{\LINK}-ma6-house 6:{\ATT}  $\emptyset$7.tin 6-\textsc{prs} put.{\SBJV}  1\textsc{sg}.{\OBJ} here {\ATT}[Bulu] $\emptyset$7.tin[Bulu]\\
    \trans `So I want tin (roofed) houses that they be put here for me, of tin.'
 
\exC\label{32}
  \gll   má kì má yánɛ́ bî ndà zì jálɛ́ tèvá \\
             [1\textsc{sg} too 1\textsc{sg} have houses {\ATT} tin village {\ATT} here]Bulu\\
    \trans `Me too, I have tin (roofed) houses in the village here.'
\end{exe}

\noindent Severin:

\begin{exe}[(C234)]  
\exC\label{33}
  \glll    làwɔ̂ bágyɛ̀lì \\
           làwɔ̂ H-ba-gyɛ̀lì \\
            speak.{\IMP}  {\OBJ}.{\LINK}-2-Gyeli \\
    \trans `Speak Gyeli!'
\end{exe}

\noindent Nze:

\begin{exe}[(C234)]  
\exC\label{34} 
  \glll  mɛ́ làwɔ́ náà màndáwɔ̀ má zì má kùgáà mɛ̂ vâ \\
         mɛ-H làwɔ-H nâ ma-ndáwɔ̀ má zì ma-H kùgáà mɛ vâ \\
            1\textsc{sg}-\textsc{prs} say-{\R} {\COMP} ma6-house 6:{\ATT}  $\emptyset$7.tin 6-\textsc{prs} be.enough.{\SBJV}  1\textsc{sg}.{\OBJ} here \\
    \trans `I say that there should be enough tin (roofed) houses here for me.'
 
\exC\label{35}
  \glll   bàgyɛ̀lì bá sɔ́ bà sílɛ̃́ɛ̃̀ bígɛ̀ \\
           ba-gyɛ̀lì bá sɔ́ ba sílɛ̃́ɛ̃̀ bígɛ\\
           2-Gyeli 2:{\ATT}  $\emptyset$1.friend 2.{\PST}1 finish.{\COMPL} develop  \\
    \trans `The fellow Bagyeli have already all developed.'
 
\exC\label{36}
  \glll   bí bɔ́ɔ̀ yá bígɛ́ mpá'à wá vɛ́ \\
           bí b-ɔ́ɔ̀ ya-H bígɛ-H mpá'à wá vɛ́ \\
           1\textsc{pl}.{\SBJ}  2-other 1\textsc{pl}-\textsc{prs} develop-{\R} $\emptyset$3.side 3:{\ATT}  which  \\
    \trans `How will we others develop?'
 
\exC\label{37} 
  \glll   mɛ́ kɛ́ dvùmɔ̀ nkùndyá dyúwɔ̀ \\
         mɛ-H kɛ̀-H dvùmɔ nkùndyá dyúwɔ̀ \\
            1\textsc{sg}-\textsc{prs} go-{\R} fall $\emptyset$9.raffia on.top \\
    \trans `I go fall from the raffia palm up there,'
 
\exC\label{38} 
  \glll   kɛ̀ kwã̂ ngùndyá mbvúɔ̀ nzíí nɔ̀ \\
          kɛ̀ kwã̂ ngùndyá mbvúɔ̀ nzíí nɔ̀ \\
           go cut  $\emptyset$9.raffia $\emptyset$1.rain {\PROG}.\textsc{prs} rain \\
    \trans `going cutting the raffia when it's raining.'
 
\exC\label{39}
  \glll   ngà wɛ́ nyɛ́ nyɛ̂  \\
         ngà wɛ-H nyɛ̂-H nyɛ̂ \\
            Q(tag) 2\textsc{sg}-\textsc{prs} see-{\R} see \\
    \trans `Right, you see [that] often.'
 
\exC\label{40}
  \glll   ngùndyá tè nyɔ́ bɛ́ nyî \\
          ngùndyá tè ny-ɔ́ bɛ̀-H nyî \\
           $\emptyset$9.raffia there 9-{\OBJ} be-{\R} 9.{\DEM}.{\PROX}  \\
    \trans `The raffia there, that is it.'
 
\exC\label{41}
  \glll   ndí mɛ̀ mɛ̀ yà bà fàmí wã̂ yáà bígɛ̀ yɔ̃́ɔ̃̀ yá vɛ́ ɛ́ yã̂ kwádɔ́ nâ yíì vàágɔ̀ \\
         ndí mɛ mɛ ya bà fàmí w-ã̂ yáà bígɛ yɔ̃́ɔ̃̀ yá vɛ́ ɛ́ y-ã̂ kwádɔ́ nâ yíì vàágɔ̀ \\
            but 1\textsc{sg} 1\textsc{sg} 1\textsc{pl} {\AP} $\emptyset$1.family 1-{\POSS}.1\textsc{sg} 1\textsc{pl}.{\FUT} develop $\emptyset$7.time[Bulu] 7:{\ATT}  which {\LOC} 7-{\POSS}.1\textsc{sg}  $\emptyset$7.village {\COMP} 7.{\COP} animated \\
    \trans `But I, I, we, my family, when will we develop, so my part of the village be lively?'
 
\exC\label{42}
  \glll   mɛ̀ bɛ́ ngyɛ̃̂ Ngvùmbɔ̀ \\
          mɛ bɛ̀-H n-gyɛ̃̂ Ngvùmbɔ̀ \\
           1\textsc{sg}.{\PST}1 be-{\R} {\N}1-guest $\emptyset$1.{\PN}  \\
    \trans `I was a guest of the Ngumba.'
 
\exC\label{43} 
  \glll  mɛ̀ nyɛ́ kwádɔ́ yî Kúndúkùndù \\
         mɛ nyɛ̂-H kwádɔ́ yî Kúndúkùndù \\
            1\textsc{sg}.{\PST}1 see-{\R} $\emptyset$7.village 7.{\DEM}.{\PROX} $\emptyset$7.{\PN} \\
    \trans `I saw this village, Kundukundu.'
 
\exC\label{44}
  \glll   vɛ̀ɛ̀ màndáwɔ̀ má zì mɔ̂ nà mɔ̂ \\
          vɛ̀ɛ̀ ma-ndáwɔ̀ má zì m-ɔ́ nà m-ɔ́ \\
           only ma6-house 6:{\ATT}  $\emptyset$7.tin 6-{\OBJ} {\COM} 6-{\OBJ}  \\
    \trans `Only tin (roofed) houses, each of them.'
 
\exC\label{45} 
  \glll   mɛ̀gà ɛ́ɛ̀ yã̂ kwádɔ́ yɔ́gà \\
          mɛ̀-gà ɛ́ɛ̀ y-ã̂ kwádɔ́ y-ɔ́-gà \\
            1\textsc{sg}.{\SBJ}-{\CONTR} {\EXCL} 7-{\POSS}.1\textsc{sg} $\emptyset$7.village 7-{\OBJ}-{\CONTR} \\
    \trans `As for me, right, my [part of the] village too!'
 
\exC\label{46} 
  \glll  wɛ̀gà wɛ̀ njí dyɔ̀dɛ̀ bùdì \\
         wɛ̀-gà wɛ njì-H dyɔ̀dɛ b-ùdì \\
          2\textsc{sg}.{\SBJ}-{\CONTR} 2\textsc{sg}.{\PST}1 come-{\R} deceive ba2-person   \\
    \trans `As for you, you came to deceive people.'
 
\exC\label{47} 
  \glll   mínɔ̀ má bùdì mà kɛ̃́ɛ̃̀ máà vɛ́ \\
          m-ínɔ̀ má b-ùdì ma kɛ̃́ɛ̃̀ máà vɛ́ \\
            ma6-name 6:{\ATT}  ba2-person 6.{\PST}1 go.{\COMPL} 6.{\COP} where \\
    \trans `The people's names have gone, where are they? [strangers come once, but do not return again].'
 
\exC\label{48} 
  \glll  lèbvúú  lé tè lɔ́ɔ̀ yá bùdɛ́ lɛ̂ \\
         le-bvúú  lé tè lɔ́ɔ̀ ya-H bùdɛ-H lɛ̂ \\
         le5-anger 5:{\ATT}  there 5.{\COP} 1\textsc{pl}-\textsc{prs} have-{\R} 5.{\DEM}.{\PROX}    \\
    \trans `The anger there it is that which we have.'
 
\exC\label{49}
  \glll   vɛ̀ɛ̀ nàmɛ́nɔ́ nàmɛ́nɔ́ nà pámò dẽ̀ \\
       vɛ̀ɛ̀ nàmɛ́nɔ́ nàmɛ́nɔ́ nà pámo dẽ̀ \\
            only tomorrow tomorrow {\COM} arrive today \\
    \trans `Only tomorrow, tomorrow, until today. [only heard promises, but never any actions]'
\end{exe}

\noindent Severin in Ngumba:

\begin{exe} 
\exC\label{50}
  \gll  bùrè bvùbvù bɔ́ ʃí nzì wâ  \\
           people many 2 {\PROG} come here  \\
    \trans `Are many people coming here?'
\end{exe}

\noindent Nze:

\begin{exe}[(C234)]  
\exC\label{51}
  \glll   éè bvùbvù pílì mɛ́ làwɔ́ mpù mɛ̀ɛ́ válɛ́ làwɔ̀ \\ 
           éè bvùbvù pílì mɛ-H làwɔ-H mpù mɛ̀ɛ́ vá-lɛ́ làwɔ \\
           yes many when 1\textsc{sg}-\textsc{prs} speak-{\R} like.this 1\textsc{sg}.\textsc{prs}.{\NEG} tolerate-{\NEG} speak  \\
    \trans `Yes, many. When I speak like this, I'm not lying [lit.  I don't tolerate to talk].'
 
\exC\label{52}
  \glll   yíì nâ báà bvùbvù \\
         yíì nâ báà bvùbvù \\
            7.{\COP} {\COMP} 2.{\COP} many \\
    \trans `It is true that they are many.'
 
\exC\label{53}
 \glll  bwánɔ̀ békúmbé bé bà njí nà byɔ̂ bé tɛ́lɛ́ màbé\\
         b-wánɔ̀ be-kúmbé bé ba njì-H nà by-ɔ̂ be-H tɛ́lɛ-H mà-bé\\
          ba2-child be8-tin 8:{\ATT}  2.{\PST}1 come-{\R} {\COM} 8-{\OBJ} 8-\textsc{prs} stand-{\R} here-8\\
    \trans `The few tin roofs that they brought stand here.'
 
\exC\label{54} 
  \glll  màndáwɔ̀ má tɛ́lɛ́ màmá \\
         ma-ndáwɔ̀ ma-H tɛ́lɛ-H mà-má \\
           ma6-house 6-\textsc{prs} stand-{\R} here-6  \\
    \trans `Houses stand here.'
 
\exC\label{55}
  \glll   bèsàndyá lèwúmɔ̀ nà bétánɛ̀ \\
           be-sàndyá lè-wúmɔ̀ nà bé-tánɛ̀ \\
           be8-raffia.mat le5-ten {\CONJ} 8-five  \\
    \trans `Fifteen raffia mats,'
 
\exC\label{56}
  \glll   byɔ̀ bé tɛ́lɛ́ bé \\
         by-ɔ̀ be-H tɛ́lɛ-H (mà-)bé. \\
           8-{\OBJ}  8-\textsc{prs} stand-{\R} 8 \\
    \trans `They stand here.'
 
\exC\label{57} 
  \glll   bèkúmbɛ́ báà njì nà byɔ̂ nà báà njì lwɔ̃̂ mándáwɔ̀\\
          be-kúmbɛ́ báà njì nà byɔ̂ nà báà njì lwɔ̃̂ H-ma-ndáwɔ̀\\
           be8-roof  2.{\FUT} come {\COM} 8 {\CONJ} 2.{\FUT} come build {\OBJ}.{\LINK}-ma6-house\\
    \trans `They will bring roofs and they will come and build houses.'
 
\exC\label{58} 
  \glll   bímbú lɛ́ fàmí wã̂ wà mɛ̀ bùdɛ́ mà\\
           bímbú lɛ́ fàmí w-ã̂ wà mɛ bùdɛ-H mà\\
           $\emptyset$5.amount 5:{\ATT}  $\emptyset$1.family 1-{\POSS}.1\textsc{sg} 1:{\ATT}  1\textsc{sg}.{\PST}1 have {\COMPL}[Kwasio]\\
    \trans `The size of my family that I have gotten. . .'
 
\exC\label{59}
  \glll  ndáwɔ̀ tè ká mɛ́ lã́ tè  \\
         ndáwɔ̀ tè ká mɛ-H lã̀-H tè  \\
           $\emptyset$9.house there when 1\textsc{sg}-\textsc{prs} pass-{\R} there  \\
    \trans `The house there, when I pass there. . .'
 
\exC\label{60} 
  \glll  ɛ́ pɛ́ɛ́ mɛ̀ɛ̀ lwɔ̃̂ nyá ndáwɔ̀ \\
         ɛ́ pɛ́-ɛ́ mɛ̀ɛ̀ lwɔ̃̂ nyá ndáwɔ̀ \\
          {\LOC} there-{\DIST} 1\textsc{sg}.{\FUT} build real $\emptyset$9.house \\
    \trans `I will build a real house over there.'
 
\exC\label{61}
  \glll  ɛ́ pɛ́ɛ́ mɛ̀ɛ̀ jìyɔ̀ \\
         ɛ́ pɛ́-ɛ́ mɛ̀ɛ̀ jìyɔ \\
          {\LOC} there-{\DIST} 1\textsc{sg}.{\FUT} stay \\
    \trans `I will live over there, here I heard that here it [they] will come and destroy all.'
 
\exC\label{62}
  \glll  ɛ́ vâ mɛ̀ dyùwɔ́ nâ ɛ́ vâ yíì sílɛ̀ njì búlɛ̀\\
        ɛ́ vâ mɛ dyùwɔ-H nâ ɛ́ vâ yíì sílɛ njì búlɛ\\
         {\LOC} here 1\textsc{sg}.{\PST}1 hear-{\R} {\COMP} {\LOC} here 7.{\FUT} finish come destroy\\
    \trans `I heard that everything here will become destroyed.'
 
\exC\label{63}
  \glll  bímbú lɛ́ mámbɔ̀ngɔ̀ máà mɛ̂ vâ \\
        bímbú lɛ́ ma-mbɔ̀ngɔ̀ máà mɛ̀ vâ \\
           $\emptyset$5.amount 5:{\ATT}  ma6-plant 6.{\COP} 1\textsc{sg}.{\OBJ} here  \\
    \trans `I have many plants here.'
 
\exC\label{64}
  \glll  mɛ́ kɛ́ jìyɔ̀ vɛ́ yá bà fàmí wã̂ \\
          mɛ-H kɛ̀-H jìyɔ vɛ́ ya-H bà fàmí w-ã̂ \\
            1\textsc{sg}-\textsc{prs} go-{\R} stay where 1\textsc{pl}-\textsc{prs} {\AP} $\emptyset$1.family 1-{\POSS}.1\textsc{sg} \\
    \trans `Where will I live, we with my family?'
\end{exe}

\noindent Severin in Ngumba:

\begin{exe}[(C234)] 
\exC\label{65}
  \gll   bã̂ njè bû wáá \\
           2.{\FUT} arrive break here  \\
    \trans `Will they come to destroy the place here?'
\end{exe}

\noindent Nze:

\begin{exe}[(C234)]  
\exC\label{66} 
  \glll  mɛ́ dyúwɔ́ nâ mpàgó wá pɔ́dɛ̀ lã́ vâ \\
        mɛ-H dyúwɔ-H nâ mpàgó wá pɔ́dɛ̀ lã̀-H vâ \\
            1\textsc{sg}-\textsc{prs} hear-{\R} {\COMP} $\emptyset$3.street 3:{\ATT}  $\emptyset$1.port pass-{\R} here \\
    \trans `I hear that the road to the port passes [will pass] here. '
 
\exC\label{67}
  \glll  mɛ̀ɛ̀ kálɛ̀ ná bɛ̀ nà jí ɛ́ vâ \\
        mɛ̀ɛ̀ kálɛ̀ ná bɛ̀ nà jí ɛ́ vâ \\
           1\textsc{sg}.{\FUT} {\NEG}.{\FUT} still be {\COM} $\emptyset$7.place {\LOC} here  \\
    \trans `I won't have a place here anymore.'
 
\exC\label{68}
  \glll  mɛ̀ɛ̀ jíbì nyɛ̀ mɛ́ kɛ́ ɛ́ pɛ̀ búùlɛ̀ \\
        mɛ̀ɛ̀ jíbì nyɛ̀ mɛ-H kɛ̀-H ɛ́ pɛ̀ búùlɛ̀ \\
           1\textsc{sg}.{\FUT} first return 1\textsc{sg}-\textsc{prs} go-{\R} {\LOC} there $\emptyset$7.old.settlement  \\
    \trans `I will first return, I go over there to the old settlement.'
 
\exC\label{69}
  \glll   ɛ́ pɛ̀ mɛ́ɛ̀ tɛ́ \\
          ɛ́ pɛ̀ mɛ́ɛ̀ tɛ̂-H \\
           {\LOC} there 1\textsc{sg}.{\PST}2 found-PST \\
    \trans `Over there I had originally settled.'
 
\exC\label{70}
  \glll  áà kɛ́ndɛ́ gyà \\
          áà kɛ́ndɛ́ (yá) gyà \\
          {\EXCL} $\emptyset$7.walk 7:{\ATT} $\emptyset$7.distance   \\
    \trans `Oh, it's a long walk.'
 
\exC\label{71} 
  \glll   báà tfùbɔ̀ ndáà \\
        báà tfùbɔ ndáà \\
           2.{\FUT} pierce also  \\
    \trans `They will cut [a road there] too,'
 
\exC\label{72} 
  \glll   báà tfùbɔ̀ báà tfùbɔ̀ \\
           báà tfùbɔ̀ báà tfùbɔ̀ \\
           2.{\FUT} pierce 2.{\FUT} pierce  \\
    \trans `they will cut, they will cut.'
 
\exC\label{73}
  \glll  mpàgó wá nùmbà wúù \\
           mpàgó wá nùmbà wúù \\
          $\emptyset$3.road 3:{\ATT}  $\emptyset$1.logger there   \\
    \trans `The road of the loggers there.'
 
\exC\label{74}
  \glll   tè mɛ̀ɛ̀ jíbì kɛ̀ lwɔ̃̂ tè \\
          tè mɛ̀ɛ̀ jíbì kɛ̀ lwɔ̃̂ tè \\
           there 1\textsc{sg}.{\FUT} first go build there  \\
    \trans `There, I will first go to build [his house] there.'
 
\exC\label{75}
  \glll  àmú vâ mɛ̀ɛ́ bɛ́lɛ́ nà sí ɛ́ vâ \\
         àmú vâ mɛ̀ɛ́ bɛ́-lɛ́ nà sí ɛ́ vâ \\
           because[Bulu] here 1\textsc{sg}.\textsc{prs}.{\NEG} be-{\NEG} {\COM} $\emptyset$9.ground {\LOC} here  \\
    \trans `Because here I don't have any land.'
 
\exC\label{76} 
  \glll  ɛ́ vâ mɛ̀ɛ́ bɛ́lɛ́ nà sí vâ \\
         ɛ́ vâ mɛ̀ɛ́ bɛ́-lɛ́ nà sí vâ \\
          {\LOC}  here 1\textsc{sg}.\textsc{prs}.{\NEG} be-{\NEG} {\COM} $\emptyset$9.ground here  \\
    \trans `Here I don't have any property.'
 
\exC\label{77}
  \glll  wɛ́ dyúwɔ́ nâ mɛ̀ nzíí kɛ̀ nà kwɛ̀lɔ̀ málɛ́ndí tè ɛ́ vâ \\
         wɛ-H dyúwɔ-H nâ mɛ nzíí kɛ̀ nà kwɛ̀lɔ H-ma-lɛ́ndí tè ɛ́ vâ \\
           2\textsc{sg}-\textsc{prs} hear-{\R} {\COMP} 1\textsc{sg} {\PROG}.\textsc{prs} go {\CONJ} fell {\OBJ}.{\LINK}-6-palm.tree there {\LOC} here \\
    \trans `Do you hear that I'm going to fell these palm trees here?'
 
\exC\label{78}
  \glll   mɛ̀ nzíí kɛ̀ nà vúlɛ́ lévúdũ̂ nà lèvúdũ̂ mɛ́ táálɛ́ sílɛ̀ nyùlɛ̀ \\
          mɛ nzíí kɛ̀ nà vúlɛ-H H-le-vúdũ̂ nà le-vúdũ̂ mɛ-H táálɛ-H sílɛ nyùlɛ \\
           1\textsc{sg} {\PROG}.\textsc{prs} go {\CONJ} take.away-{\R} {\OBJ}.{\LINK}-le5-one {\COM} le5-one 1\textsc{sg}-\textsc{prs} begin-{\R} finish drink \\
    \trans `I'm taking down [palm trees] one by one, I start to drink [them] up [make palm wine out of them].'
 
\exC\label{79}
  \glll   m̀m̀ ndɛ̀náà lèkɛ́lɛ̀ léndɛ̀ lɛ́ɛ̀ nâ \\
          m̀m̀ ndɛ̀náà le-kɛ́lɛ̀ lé-ndɛ̀ lɛ́ɛ̀ nâ \\
         {\EXCL} like.this  le5-word 5-{\ANA} 5.{\COP} {\COMP} \\
    \trans `Yes, like this. The word is that. . .'
\end{exe}

\noindent other speaker:

\begin{exe}[(C234)]  
\exC\label{80}
  \glll  nà mìmbàngá nà màsá nà bègyí nà bègyí\\
       nà mi-mbàngá nà ma-sá nà be-gyí nà be-gyí\\
           {\CONJ} mi4-coconut.tree {\CONJ} ma6-prune {\CONJ} be8-what {\CONJ} be8-what\\
    \trans `both the coconut trees and the African plum trees and so on and so forth,'
 
\exC\label{81} 
  \glll   byɛ́sɛ̀ béè sílɛ̀ ntàmànɛ̀  \\
      by-ɛ́sɛ̀ béè sílɛ ntàmanɛ \\
           8-all 8.{\FUT} finish ruin \\
    \trans `they will all be ruined.'
\end{exe}

\noindent Nze:

\begin{exe}[(C234)]  
\exC\label{82}
  \glll  màsá mâ vâ kɛ́ nà ntàmànɛ̀ nà màbɔ́'ɔ̀ tu tu tu ngùɔ́ \\
          ma-sá mâ vâ kɛ̀-H nà ntàmanɛ nà ma-bɔ́'ɔ̀ tu tu tu ngùɔ́ \\
          ma6-African.plum 6.{\DEM}.{\PROX} here go-{\R} {\CONJ} ruin {\CONJ} ma6-breadfruit all[French] all[French] all[French] $\emptyset$7.sugar.cane\\ 
    \trans `These African plum trees will be ruined and the breadfruit trees, everything, the sugar cane.'
 
\exC\label{83}
  \glll   mɛ̀ bìyɛ́ làwɔ̀ nâ àà bwánɔ̀ bã̂ \\
           mɛ bìyɛ-H làwɔ nâ àà b-wánɔ̀ b-ã̂ \\
          1\textsc{sg} in.vain? speak {\COMP} {\EXCL} ba2-child 2-{\POSS}.1\textsc{sg} \\
    \trans `I say in vain: ``ah, my children. . .''.'
 
\exC\label{84} 
  \glll  yɔ́ɔ̀ mɛ̀ jìlɛ́ kwádɔ́ yî \\
         yɔ́ɔ̀ mɛ jìlɛ-H kwádɔ́ yî \\
          so 1\textsc{sg}.{\PST}1 stay-{\R} $\emptyset$7.village 7.{\DEM}.{\PROX}  \\
    \trans `So I stayed in this village.'
\end{exe}

\noindent Severin in French asking about Mambi:

\begin{exe}[(C234)]  
\exC\label{85}
  \gll  C'est qui là  \\
          it.is who there  \\
    \trans `Who is this there?'
\end{exe}

\noindent Nze:

\begin{exe}[(C234)] 
\exC\label{86} 
  \glll  ntɛ́mbɔ́ wã̂ wɛ́ nû \\
       ntɛ́mbɔ́ w-ã̂ wɛ́ nû \\
          $\emptyset$1.younger.sibling 1-{\POSS}.1\textsc{sg} {\ID} 1.{\DEM}.{\PROX}  \\
    \trans `This is my little brother.'
\end{exe}

\noindent Mama:

\begin{exe}[(C234)]  
\exC\label{87} 
  \glll  ntùmbà wã̂ wɛ́ nû \\
       ntùmbà w-ã̂ wɛ́ nû \\
         $\emptyset$1.older.brother 1-{\POSS}.1\textsc{sg} {\ID} 1.{\DEM}.{\PROX}  \\
    \trans `This is my big brother.'
\end{exe}

\noindent Nze:

\begin{exe}[(C234)]  
\exC\label{88} 
  \glll  mwánɔ̀ wã̂ ndáà wɛ́ nù \\
      m-wánɔ̀ w-ã̂ ndáà wɛ́ nù \\
         {\N}1-child  1-{\POSS}.1\textsc{sg} also {\ID} 1.{\DEM}.{\PROX}  \\
    \trans `This is also my child.'
\end{exe}

\noindent Djiedjhie:

\begin{exe}[(C234)]  
\exC\label{89}
  \glll pã̂ bígɛ̀  \\
        pã̂ bígɛ̀.  \\
         do.first.{\IMP}   develop \\
    \trans `Speak first.'
\end{exe}

\noindent Mambi:

\begin{exe}[(C234)]  
\exC\label{90}
  \glll bõ̀ mwa mɛ́ɛ́ béè alónzì vâ tè nà bèyá njí nyɛ̂ bágyèlì \\
      bõ̀ mwa mɛ́ɛ́ béè alónzì vâ tè nà bèya-H njì-H nyɛ̂ H-ba-gyèlì \\
         good[French] 1\textsc{sg}.\textsc{emph}[French] 1\textsc{sg}.{\COP} 2\textsc{pl}.{\COP} come.on[French] here there {\CONJ} 2\textsc{pl}-\textsc{prs} come-{\R} see {\OBJ}.{\LINK}-ba2-Gyeli   \\
    \trans `Good, me, I'm, you are, {\itshape allons-y}, here that you come to see the Bagyeli.'
\end{exe}

\noindent Severin in French:

\begin{exe}[(C234)]  
\exC\label{91} 
  \gll  C'est toi qui \\
         it.is 2\textsc{sg} who   \\
    \trans `Who are you?'
\end{exe}

\noindent Mambi:

\begin{exe}[(C234)]  
\exC\label{92} 
  \glll mɛ̀ jínɔ̀ ná Màmbì mɛ̀ɛ́ bɛ́lɛ́ nà mùdã̂  \\
      mɛ j-ínɔ̀ ná Màmbì mɛ̀ɛ́ bɛ́-lɛ́ nà m-ùdã̂  \\
         1\textsc{sg} le5-name SIM $\emptyset$1.{\PN} 1\textsc{sg}.\textsc{prs}.{\NEG} be.{\NEG} {\COM} {\N}1-woman   \\
    \trans `My name is Mambi, I don't have a wife.'
 
\exC\label{93}
  \glll  mɛ̀ pálɛ́ lìí bâ \\
          mɛ pálɛ́ lìí bâ \\
           1\textsc{sg}.{\PST}1 {\NEG}.{\PST}  yet marry \\
    \trans `I am not yet married.'
\end{exe}

\noindent Nze:

\begin{exe}[(C234)] 
\exC\label{94}
  \glll  à pálɛ́ lìí bâ \\
      a pálɛ́ lìí bâ \\
          1.{\PST}1 {\NEG}.{\PST}  yet married  \\
    \trans `He is not yet married.'
\end{exe}

\noindent Mambi:

\begin{exe}[(C234)]  
\exC\label{95} 
  \glll mɛ̀ jínɔ̀ ná Màmbì Màmbì  \\
        mɛ j-ínɔ̀ ná Màmbì Màmbì  \\
          1\textsc{sg} le5-name SIM $\emptyset$1.{\PN} $\emptyset$1.{\PN}  \\
    \trans `My name is Mambi, Mambi.'
\end{exe}

\noindent Nze:

\begin{exe}[(C234)] 
\exC\label{96} 
  \glll  mɛ̀ bùdɛ́ bwánɔ̀ bábáà \\
        mɛ bùdɛ-H b-wánɔ̀ bá-báà \\
         1\textsc{sg} have-{\R} ba2-child 2-two   \\
    \trans `I have two children.'
\end{exe}

\noindent Mambi:

\begin{exe}[(C234)] 
\exC\label{97}
  \glll pílì bèyá lɔ́ njì ɛ́ vâ tɛ́ɛ̀ dẽ́  \\ 
       pílì bèya-H lɔ́ njì ɛ́ vâ tɛ́ɛ̀ dẽ́ \\
         when 2\textsc{pl}-\textsc{prs} {\RETRO}  come {\LOC} here now today  \\
    \trans `When you just arrived here now today,'
 
\exC\label{98} 
  \glll  nâ bèyá njí nyɛ̂ bá-gyèlì vwálà \\
      nâ bèya-H njì-H nyɛ̂ H-ba-gyèlì vwálà \\
          {\COMP} 2\textsc{pl}-\textsc{prs} come-{\R} see {\OBJ}.{\LINK}-ba2-Gyeli there.it.is[French]  \\
    \trans `so that you come to see the Bagyeli, there they are.'
 
\exC\label{99}
  \glll  bí bɔ́gà yá wúmbɛ́ ndáà mínsáyá mí màmbò bèyá sá bî myɔ̂ kí bɛ̀ mí mpà \\
         bí b-ɔ́gà ya-H wúmbɛ-H ndáà H-mi-nsáyá mí m-àmbò bèya-H sâ-H bî my-ɔ̂ kí bɛ̀ mí mpà \\
          1\textsc{pl}.{\SBJ}  2-other 1\textsc{pl}-\textsc{prs} want-{\R} also {\OBJ}.{\LINK}-mi4-deed 4:{\ATT}  ma6-thing 2\textsc{pl}-\textsc{prs} do-{\R} 1\textsc{pl}.{\OBJ} 4-{\OBJ} {\NEG}[Kwasio] be 4:{\ATT} good \\
    \trans `Us, the others, we want also the deeds of things that you do us, they are not good.'
 
\exC\label{100} 
  \glll ká bèyá bùdɛ́ másà wùnɛ́ \\ 
        ká bèya-H bùdɛ-H másà w-ùnɛ́ \\
         if 2\textsc{pl}-\textsc{prs} have-{\R} $\emptyset$1.boss 1-{\POSS}.2\textsc{pl}   \\
    \trans `If you have your boss,'
 
\exC\label{101}
  \glll  ká másà wùnɛ́ njì yá láá másà wùnɛ́ nâ mìnsáyá mí bèyá sâ mí bɛ́lɛ́ mpà vúdũ̂ wɛ́ yí-ndɛ̀ \\
         ká másà w-ùnɛ́ njì ya-H láà-H másà w-ùnɛ́ nâ mi-nsáyá mí bèya-H sâ mi-H bɛ́-lɛ́ mpà vúdũ̂ wɛ́ yí-ndɛ̀\\
         if $\emptyset$1.boss 1-{\POSS}.2\textsc{pl} come 1\textsc{pl}-\textsc{prs} tell-{\R} $\emptyset$1.boss 1-{\POSS}.2\textsc{pl} {\COMP} mi4-deed 4:{\ATT}  2\textsc{pl}-\textsc{prs} do 4-\textsc{prs} be-{\NEG} good one {\ID} 7-{\ANA} \\
    \trans `If your boss comes we will tell him that the things that you do are not good, that is the first thing.'
 
\exC\label{102} 
  \glll  yá mbàà yá mbàà yíì nâ kɔ́ɔ̀ mpù ɛ́ Nzìwù lɔ́ táálɛ̀ làwɔ̀ nâ bõ̀ \\
         yá mbàà yá mbàà yíì nâ kɔ́ɔ̀ mpù ɛ́ Nzìwù lɔ́ táálɛ làwɔ nâ bõ̀ \\
         7:{\ATT}  second 7:{\ATT}  second 7.{\COP} {\COMP} still like.this {\LOC} $\emptyset$1.{\PN} {\RETRO}  begin talk {\COMP} good[French]   \\
    \trans `The second, the second is that still as Nze just began to say that, good,'
 
\exC\label{103}
  \glll  kwádɔ́ yá Ngòló yá jìlɛ́ màyì \\
        kwádɔ́ yá Ngòló ya-H jìlɛ-H mà-yì \\
          $\emptyset$7.village 7:{\ATT}  $\emptyset$3.{\PN} 1\textsc{pl}-\textsc{prs} seat-{\R} here-7  \\
    \trans `The village Ngolo, we have found it here.'
 
\exC\label{104}
  \glll  yáà ndáà vâ dísù bvúlɛ̀ bá vèlásá bíì nà kwádɔ́ yî \\
        yáà ndáà vâ dísù bvúlɛ̀ ba-H vèlasa-H bíì nà kwádɔ́ yî \\
          1\textsc{pl}.{\COP} also here first.off[Bulu] ba2.Bulu 2-\textsc{prs} contest-{\R} 1\textsc{pl}.{\OBJ} {\COM} $\emptyset$7.village 7.{\DEM}.{\PROX}  \\
    \trans `We are also here, first off, the Bulu contest our [ownership of] this village.'
 
\exC\label{105} 
  \glll bvúlɛ̀ bá ntɛ́gɛ́lɛ́ ndáà bíyɛ̀ \\
       bvúlɛ̀ ba-H ntɛ́gɛlɛ-H ndáà bíyɛ̀ \\
         ba2.Bulu 2-\textsc{prs} bother-{\R} also 1\textsc{pl}.{\OBJ}   \\
    \trans `The Bulu bother us, too.'
 
\exC\label{106}
  \glll bvúlɛ̀ bà bùdɛ́ nâ ká wɛ̀ ngyɛ̀lì wɛ̀ bùdɛ́ tsídí wɔ̂ bá sɛ̀ngɛ́ nyɛ̂ sí \\
        bvúlɛ̀ ba bùdɛ-H nâ ká wɛ n-gyɛ̀lì wɛ bùdɛ-H tsídí w-ɔ̂ ba-H sɛ̀ngɛ-H nyɛ̂ sí \\
          ba2.Bulu 2 have-{\R} {\COMP} if 2\textsc{sg} {\N}1-Gyeli 2\textsc{sg} have-{\R} $\emptyset$1.animal 1-{\POSS}.2\textsc{sg} 2-\textsc{prs} lower-{\R} 1.{\OBJ} down   \\
    \trans `The Bulu say that if you, Gyeli, you have your animal [for sale], they lower it [its price].'
 
\exC\label{107}
  \glll  bɛ́ɛ́ wɛ̀ nzíí dyúwɔ̀ mɛ̂ vwálà bõ̀ \\
        bɛ́ɛ́ wɛ nzíí dyúwɔ mɛ̂ vwálà bò\\
        right 2\textsc{sg} {\PROG}.\textsc{prs} hear 1\textsc{sg}.{\OBJ} ok[French] good[French]   \\
    \trans `Right, you hear me? Ok, good. . .'
 
\exC\label{108}
  \glll  yá nà yí báàlá nâ bèdɔ̀wɔ̀ nà bvúlɛ̀ báà nâ wɛ̀, sílɛ̂ kɛ̀ sâ sálɛ́ \\
        yá nà yi-H báàla-H nâ bèdɔwɔ nà bvúlɛ̀ báà nâ wɛ sílɛ̂ kɛ̀ sâ sálɛ́ \\
         7:{\ATT}  fourth 7-\textsc{prs} repeat-{\R} {\COMP} hang.on? {\COM} ba2.Bulu 2.{\COP} {\COMP} 2\textsc{sg} finish.{\IMP} go do $\emptyset$7.work  \\
    \trans `The fourth thing about the Bulu is that they say, ``you, go and finish [all] the work''.'

% bèdɔ̀wɔ̀ s'accrocher

\exC\label{109}
  \glll ká wɛ́ sílɛ́ kɛ̀ sâ sálɛ́ mɛ́ pílì wɛ́ kɛ́ nâ  wɛ́ kɛ́ jíì mònɛ́ wɔ̂ á làwɔ́ wɛ̂ nyùmbò \\
        ká wɛ-H sílɛ-H kɛ̀ sâ sálɛ́ mɛ́ pílì wɛ-H kɛ̀-H nâ wɛ-H kɛ̀-{\R} jíì mònɛ́ w-ɔ̂ a-H làwɔ-H wɛ̂ nyùmbò\\
         if 2\textsc{sg}-\textsc{prs} finish-{\R} go do work.7 but[French] when 2\textsc{sg}-\textsc{prs} go-{\R} {\COMP} 2\textsc{sg}-\textsc{prs} go-{\R} ask $\emptyset$1.money 1-{\POSS}.2\textsc{sg} 1-\textsc{prs} tell-{\R} 2\textsc{sg} $\emptyset$3.mouth   \\
    \trans `If you go do all the work [for a Bulu person], but when you go and ask for your money, he [the Bulu person] frowns at you.'
 
\exC\label{110}
  \glll  nyɛ̀ náà à múà wɛ̂ bíyɔ̀ \\
        nyɛ nâ a múà wɛ̀ bíyɔ \\
           1.{\SBJ}  {\COMP} 1 be.almost 2\textsc{sg}.{\OBJ} hit \\
    \trans `He [says] that he is about to beat you.'
 
\exC\label{111}
  \glll  nyɛ̀ náà à múà wɛ̂ bíyɔ̀ dẽ́ \\
        nyɛ nâ a múà wɛ̀ bíyɔ dẽ́\\
           1.{\SBJ}  {\COMP} 1 be.almost 2\textsc{sg}.{\OBJ} hit today \\
    \trans `He [says] that he is about to beat you today,'
 
\exC\label{112}
  \glll  nkàmɔ̀ nà mònɛ́ wɔ̂ dyúwɔ̀ \\
        nkàmɔ̀ nà mònɛ́ w-ɔ̂ dyúwɔ̀ \\
         $\emptyset$9.reason {\COM} $\emptyset$1.money 1-{\POSS}.2\textsc{sg} on.top  \\
    \trans `because of your money.'
 
\exC\label{113} 
  \glll pílì wɛ́ kɛ́ nâ wɛ́ kɛ́ tɔ́kɛ̀ mwánɔ̀ sáyà bvúlɛ̀ à bùdɛ́ lébvúú nà mɛ̂ \\
        pílì wɛ-H kɛ̀-H nâ wɛ-H kɛ̀-H tɔ́kɛ m-wánɔ̀ sáyà bvúlɛ̀ a bùdɛ-H H-le-bvúú nà mɛ̂ \\
           when 2\textsc{sg}-\textsc{prs} go-{\R} {\COMP} 2\textsc{sg}-\textsc{prs} go-{\R} collect {\N}1-child $\emptyset$7.thing ba2.Bulu 1 have-H {\OBJ}.{\LINK}-le5-anger {\COM} 1\textsc{sg}.{\OBJ} \\
    \trans `When you go to gather a small thing, the Bulu is angry with me.'
 
\exC\label{114} 
  \glll  mɛ̀ nzí dyâ vâ kùgúù dẽ̀ màfû mábáà \\
     mɛ nzí dyâ vâ kùgúù dẽ̀ ma-fû má-báà \\
          1\textsc{sg} {\PROG}.{\PST}  lie.down here $\emptyset$7.evening today ma6-day 6-two  \\
    \trans `I was sleeping here in the evening two days ago.'
 
\exC\label{115}
  \glll  mɛ̀ bɛ́ nà mùdã̂ wà mí də mil\\
        mɛ bɛ̀-H nà m-ùdã̂ wà m-í də  mil\\
          1\textsc{sg}.{\PST}1 be-{\R} {\COM} {\N}1-woman  1:{\ATT}  {\N}1-non-Pygmy two[French] thousand[French]\\
    \trans `I owed a Bantu farmer woman two thousand [Cameroon Francs].'
 
\exC\label{116}
  \glll ɛ́ vâ ndáwɔ̀ vâ mùdã̂ wà mí àà njì dúwɔ̀ lévúdũ̂ \\
       ɛ́ vâ ndáwɔ̀ vâ m-ùdã̂ wà m-í àà njì d-úwɔ̀ lé-vúdũ̂ \\
          {\LOC} here $\emptyset$9.house here {\N}1-woman 1:{\ATT}  {\N}1-non-Pygmy 1.{\FUT} come le5-day 5-one \\
    \trans `This house over here, the Bantu farmer woman will come the same day,'
 
\exC\label{117}
  \glll  ɛ́ pɛ̀ njì jíì mònɛ́ wɛ́ɛ̀ ɛ́ pɛ̀ njì jíì\\
         ɛ́ pɛ̀ njì jíì mònɛ́ w-ɛ̂ ɛ́ pɛ̀ njì jíì\\
        {\LOC} over.there come ask $\emptyset$1.money 1-{\POSS}.3\textsc{sg} {\LOC} over.there come ask\\
    \trans `there in order to come ask for her money, there to come ask.'
 
\exC\label{118}
  \glll  yɔ́ɔ̀ mɛ́ tɔ́kɛ́ mɔ̀nɛ́ wɛ̂ vɛ̀ nyɛ̂ \\
         yɔ́ɔ̀ mɛ-H tɔ́kɛ-H mɔ̀nɛ́ w-ɛ̂ vɛ̀ nyɛ̂ \\
         so 1\textsc{sg}-\textsc{prs} collect-{\R} $\emptyset$1.money 1-{\POSS}.3\textsc{sg} give 1.{\OBJ}  \\
    \trans `So I collect her money [and] give [it to] her,'
 
\exC\label{119}
  \glll  nâ ndɛ̀náà  yíì mpà \\
      nâ ndɛ̀náà  yíì mpà \\
          {\COMP} like.this 7.{\COP} good  \\
    \trans `that like this it be good.'
 
\exC\label{120}
  \glll  bõ̀ pílì yí báàlá nà bɛ̀ ndɛ̀náà ndɛ̀náà ndáà ná\\
        bõ̀ pílì yi-H báàla-H nà bɛ̀ ndɛ̀náà ndɛ̀náà ndáà ná\\
          good[French] when 7-\textsc{prs} repeat-{\R} {\CONJ} be like.that like.that also still\\
    \trans `So, when it continues and is still like this and like that.'
 
\exC\label{121}
  \glll bvúlɛ̀ bà bùdɛ́ mà sá yíì ná vúdũ̂ \\
         bvúlɛ̀ ba bùdɛ-H mà sá yíì ná vúdũ̂ \\
         ba2.Bulu 2 have {\COMPL}[Kwasio] $\emptyset$7.thing 7.{\COP} again one   \\
    \trans `There is one more thing about the Bulu.'
 
\exC\label{122} 
  \glll wɛ́ kɛ́ nà nyɛ̂ nkɔ̃̀wáká nyɛ̀gà à nzíí wɛ̂ vã́ã̀kɛ́ sâ mpù \\
         wɛ-H kɛ̀-H nà nyɛ̂ nkɔ̃̀wáká nyɛ̀-gà a nzíí wɛ̂ vã́ã̀kɛ́ sâ mpù \\
         2\textsc{sg}-\textsc{prs} go {\COM} 1.{\OBJ} equal.sharing 1.{\SBJ}-{\CONTR} 1 {\PROG}.\textsc{prs} 2\textsc{sg}.{\OBJ} go[Bulu] do like.this  \\
    \trans `You go with him equally sharing, he tries to trick you [lit. he is going to do you like this].'
 
\exC\label{123}
  \glll  pílì yí múà ndáwɔ̀ nyà mànyɔ̀ ndɛ̀náà \\
      pílì yí múà ndáwɔ̀ nyà ma-nyɔ̀ ndɛ̀náà \\
         when 7 be.almost $\emptyset$9.house 9:{\ATT}  ma6-drink like.this   \\
    \trans `When it is at a bar like this,'
 
\exC\label{124}
  \glll  á kí náà à múà njì bvúdà nà wɛ̂ \\
       a-H kì-H nâ a múà njì bvúda nà wɛ̂ \\
         1-\textsc{prs} say-{\R} {\COMP} 1 be.almost come quarrel {\COM} 2\textsc{sg}.{\OBJ}   \\
    \trans `he says that he is about to come quarrel with you.'
 
\exC\label{125} 
  \glll  pílì mwánɔ̀ bàgyɛ̀lì àà nyɛ̂ kɛ̀ bíyɔ̀ \\
       pílì m-wánɔ̀ ba-gyɛ̀lì àà nyɛ̂ kɛ̀ bíyɔ\\
          when {\N}1-child ba2-Gyeli 1.{\FUT} 1.{\OBJ} go hit  \\
    \trans `At times the Gyeli child, he will go hit it,'
 
\exC\label{126} 
  \glll  kɛ̀ nyɛ̂ bíyɔ̀ mpù \\
         kɛ̀ nyɛ̂ bíyɔ mpù \\
          go 1.{\OBJ} hit like.this  \\
    \trans `hit it like this.'
 
\exC\label{127} 
  \glll  báà nâ bìsɔ́mɔ̀nɛ̀ bìsɔ́mɔ̀nɛ̀ bé nyì \\
        báà nâ bi-sɔ́mɔ̀nɛ̀ bi-sɔ́mɔ̀nɛ̀ be-H nyì \\
         2.{\COP} {\COMP} be8-complaint be8-complaint 8-\textsc{prs} enter   \\
    \trans `it is their fault that again and again complaints start.'
 
\exC\label{128}
  \glll  dɔ̃̀ pɛ̀ tsíyɛ̀ pɔ́nɛ́ lékɛ́lɛ̀ bvúlɛ̀ bá ntɛ́gɛ́lɛ́ bíì ɛ́ vâ \\
         dɔ̃̀ pɛ̀ tsíyɛ pɔ́nɛ́ le-kɛ́lɛ̀ bvúlɛ̀ ba-H ntɛ́gɛlɛ-H bíì ɛ́ vâ \\
         so[French] there cut $\emptyset$7.truth le5-word ba2.Bulu 2-\textsc{prs} bother-{\R} 1\textsc{pl}.{\OBJ} {\LOC} here   \\
    \trans `So, to say the truth, the Bulu bother us here.'
 
 \largerpage
\exC\label{129}
  \glll kwádɔ́ yá wɛ́ nyɛ̂ yá jìlɛ́ mà wá yî \\
          kwádɔ́ yá wɛ-H nyɛ̂ ya-H jìlɛ-H mà wá yî \\
         $\emptyset$7.village 7:{\ATT}  2\textsc{sg}-\textsc{prs} see 1\textsc{pl}-\textsc{prs} place-{\R} {\COMPL}[Kwasio] here[Kwasio] 7   \\
    \trans `The village that you see, we have found it here.'
 
\exC\label{130}
  \glll bvúlɛ̀ bá ntɛ́gɛ́lɛ́ bíì kwádɔ́ yá wɛ́ nyɛ̂ yá jìlɛ́ mà wá yî \\
        bvúlɛ̀ ba-H ntɛ́gɛlɛ-H bíì kwádɔ́ yá wɛ-H nyɛ̂ ya-H jìlɛ-H mà wá yî \\
          ba2.Bulu 2-\textsc{prs} bother-{\R} 1\textsc{pl}.{\OBJ} $\emptyset$7.village 7:{\ATT}  2\textsc{sg}-\textsc{prs} see 1\textsc{sg}-\textsc{prs} seat-{\R} {\COMPL}[Kwasio] here[Kwasio] 7  \\
    \trans `The Bulu bother us. The village that you see, we have found it  here.'
\end{exe}

\noindent Severin in Ngumba:

\begin{exe}[(C234)] 
\exC\label{131}
  \gll bùdì bɔ́nɛ̀gà bɔ́ pɛ̂ mbíɛ̀ bɔ́ lɛ́ɛ̀ náà mí bɔ́ kwàlɛ́ b-ùdã̂ b-ɔ̀ɔ̀ \\
         ba2-person 2-other 2 there $\emptyset$3.high 2.\textsc{prs} say {\COMP} 2.non.Pygmy 2.\textsc{prs} love ba2-woman 2-{\POSS}.2\textsc{sg} \\
    \trans `The other people there upstream say that the Bulu love your women.'
\end{exe}

\noindent Mambi:

\begin{exe}[(C234)] 
\exC\label{132}
  \glll  vwálà wɛ̀ɛ̀ njǐ nà njǐ wɛ̀ɛ̀ njǐ nà njǐ\\
        vwálà wɛ̀ɛ̀ njǐ nà njǐ wɛ̀ɛ̀ njǐ nà njǐ\\
          ok[French] 2\textsc{sg}.{\COP} $\emptyset$9.path {\CONJ} $\emptyset$9.path 2\textsc{sg}.{\COP} $\emptyset$9.path {\CONJ} $\emptyset$9.path\\
    \trans `Exactly, you are on the right track.'
 
\exC\label{133} 
  \glll  dɔ̃̀ bɛ̀yá lɔ́ kɛ̀ nà bɛ̀yà nzíí pándɛ̀ \\
          dɔ̃̀ bɛ̀ya-H lɔ́ kɛ̀ nà bɛ̀ya nzíí pándɛ \\
          so[French] 2\textsc{pl}-\textsc{prs} {\RETRO}  go {\CONJ} 2\textsc{pl} {\PROG}.\textsc{prs} arrive \\
    \trans `So, you just came and you are arriving,'
 
\exC\label{134}
  \glll  bɛ̀yá nzíyɛ̀ bíyɛ̀ kfùmàlà \\
         bɛ̀ya-H nzíyɛ̀ bíyɛ̀ kfùmala \\
         2\textsc{pl}-\textsc{prs}  come.{\SBJV}  1\textsc{pl}.{\OBJ} find  \\
    \trans `you may come to meet us.'
 
\exC\label{135} 
  \glll  bùdì bɛ́sɛ̀ bà nzíí kɛ̀ nà kɛ́ dẽ́ bèjìí dé tù\\
       b-ùdì b-ɛ́sɛ̀ ba nzíí kɛ̀ nà kɛ̀-H dẽ́ be-jìí dé tù\\
         ba2-person 2-all 2 {\PROG}.\textsc{prs} go {\CONJ} go-{\R} today be8-forest {\LOC} inside\\
    \trans `All the people are going into the forest today.'
 
\exC\label{136}
  \glll  dɔ̃̂ bɛ̀yá nzíyɛ̀ bíyɛ̀ kfùmàlà \\
        dɔ̃̂ bɛ̀ya-H nzíyɛ̀ bíyɛ̀ kfùmala \\
        so[French] 2\textsc{pl}-\textsc{prs}  come.{\SBJV}  1\textsc{pl}.{\OBJ} find   \\
    \trans `So, you may come to meet us.'
 
\exC\label{137}
  \glll  bɔ́nɛ́gá báà ná jìí dé tù \\
        b-ɔ́nɛ́gá báà ná jìí dé tù \\
          2-other 2.{\COP} still $\emptyset$7.forest {\LOC} inside  \\
    \trans `The others are still in the forest.'
 
\exC\label{138}
  \glll  bɛ̀yá nzíyɛ̀ bíyɛ̀ kfùmàlà vâ \\
         bɛ̀ya-H nzíyɛ̀ bíyɛ̀ kfùmala vâ \\
         2\textsc{pl}-\textsc{prs}  come.{\SBJV}  1\textsc{pl}.{\OBJ} find here   \\
    \trans `You may come to meet us here.'
 
\exC\label{139} 
  \glll  dɔ̃̀ bí yá táálɛ́ bê yàlànɛ̀ àà \\
        dɔ̃̀ bí ya-H táálɛ-H bê yàlanɛ àà \\
       so[French] 1\textsc{pl}.{\SBJ}1\textsc{pl}-\textsc{prs} begin-{\R} 2\textsc{pl}.{\OBJ} respond[Bulu] {\EXCL}   \\
    \trans `So we start to respond to you, mhm.'
\end{exe}

\noindent Severin in Ngumba:

\begin{exe}[(C234)]  
\exC\label{140} 
  \gll  wɛ̀ sí lɛ́ɛ̀ náà ɔ̀ bírì bùrã̂ bɔ̀ nìà \\
          2\textsc{sg} {\PROG}.{\PST}  say {\COMP} 2\textsc{sg} have ba2-woman 2:{\ATT}  how.many  \\
    \trans `You said you have how many wives?'
\end{exe}

\noindent Nze:

\begin{exe}[(C234)] 
\exC\label{141} 
  \glll  nà mɛ̂ \\
         nà mɛ̂ \\
          Q 1\textsc{sg}  \\
    \trans `Me?'
\end{exe}

\noindent Mambi:

\begin{exe}[(C234)] 
\exC\label{142}  
  \glll à bùdɛ́ mà mùdã̂ \\
        a bùdɛ-H mà m-ùdã̂ \\
          1 have-{\R} {\COMPL}[Kwasio] {\N}1-woman  \\
    \trans `He already has a wife.'
\end{exe}

\noindent Nze:

\begin{exe}[(C234)]  
\exC\label{143} 
  \glll  mɛ̀ bùdɛ́ mà mùdã̂ mvúdũ̂ \\
         mɛ bùdɛ-H mà m-ùdã̂ m-vúdũ̂ \\
         1\textsc{sg} have-{\R} {\COMPL}[Kwasio]  {\N}1-woman 1-one  \\
    \trans `I have already one wife.'
 
\exC\label{144} 
  \glll  bwánɔ̀ mpù [gesture showing 2] \\
          b-wánɔ̀ mpù \\
         ba2-child like.this   \\
    \trans `that many children [gesture showing 2].'
 
\exC\label{145} 
  \glll  bwánɔ̀ bá bùdã̂ bábáà èè nà mwánɔ̀ wà mùdã̂ nláálɛ̀ ndáà ná \\
         b-wánɔ̀ bá b-ùdã̂ bá-báà èè nà m-wánɔ̀ wà m-ùdã̂ nláálɛ̀ ndáà ná \\
         ba2-child 2:{\ATT}  ba2-woman 2-two {\EXCL} {\CONJ} {\N}1-child 1:{\ATT}  {\N}1-woman three also again   \\
    \trans `Two girls, yes, and also again a third girl.'
 
\exC\label{146} 
  \glll  mm ndí nyɛ̀gà à ndáà lèbá ɛ́ pɛ̀ \\
        mm ndí nyɛ̀-gà a ndáà le-bá ɛ́ pɛ̀ \\
          {\EXCL} but 1.{\SBJ}-{\CONTR} 1 also le5-marriage {\LOC} there  \\
    \trans `Mhm, but the other one has gotten also married over there.'
 
\exC\label{147} 
  \glll  à kɛ́ bwálɛ̀ nà eeehhh  \\
          a kɛ́ bwálɛ nà eeehhh \\
          1.{\PST}1 go be.born {\CONJ} {\EXCL}  \\
    \trans `She was born elsewhere and eehmmm. . .'
 
\exC\label{148} 
  \glll ntɛ́mbɔ̀ wà mùdã̂ wã̂ nyɛ̀ wɛ́ bùdɛ́ mwánɔ̀ wà mùdã̂ mvúdũ̂ \\
       ntɛ́mbɔ̀ wà m-ùdã̂ w-ã̂ nyɛ wɛ́ bùdɛ-H m-wánɔ̀ wà m-ùdã̂ m-vúdũ̂ \\
        $\emptyset$1.younger.sibling 1:{\ATT}  {\N}1-woman 1-{\POSS}.1\textsc{sg} 1.{\SBJ}  {\ID} have-{\R} {\N}1-child 1:{\ATT}  {\N}1-woman 1-one    \\
    \trans `It's my wife's younger sister who has one girl.'
 
\exC\label{149} 
  \glll  kwádɔ́ yáwɔ̀ yɔ̂ wɛ́ yî \\
          kwádɔ́ y-áwɔ̀ yɔ̂ wɛ́ yî \\
         $\emptyset$7.village 7-{\POSS}.3\textsc{pl} 7 {\ID} 7.{\DEM}.{\PROX}   \\
    \trans `Their village is this one.'
 
\exC\label{150}
  \glll  ká wɛ́ nyɛ́ mɛ̂ jíì sâ vâ nâ bá nzíyɛ̀ bá nzíyɛ̀ jìyɔ̀ \\
       ká wɛ-H nyɛ̂-H mɛ̂ jíì sâ vâ nâ ba-H nzíyɛ̀ ba-H nzíyɛ̀ jìyɔ \\
         if 2\textsc{sg}-\textsc{prs} see-{\R} 1\textsc{sg}.{\OBJ} stay only here {\COMP} 2-\textsc{prs} come.{\SBJV}  2-\textsc{prs} come.{\SBJV}  stay   \\
    \trans `When you see me just staying here, so that they come, they come to stay.'
\end{exe}

\noindent Mambi:

\begin{exe}[(C234)]  
\exC\label{151} 
  \glll  yá wúmbɛ́ ndáà náà bí bɔ́gà yá pángɔ́ bɛ̀ \\
         ya-H wúmbɛ-H ndáà nâ bí b-ɔ́gà ya-H pángɔ-H bɛ̀ \\
        1\textsc{pl}-\textsc{prs} want-{\R} also {\COMP} 1\textsc{pl}.{\SBJ}  2-other 1\textsc{pl}-\textsc{prs}  do.first[Kwasio]-{\R} be  \\
    \trans `We also want that we others first have. . .'
 
\exC\label{152} 
  \glll nà kùrã̂ ndáà \\
       nà kùrã̂ ndáà \\
         {\COM} $\emptyset$7.electricity also   \\
    \trans `also electricity.'
 
\exC\label{153} 
  \glll  ónóò bí bɔ́gà yá pã́ jî bényámɛ̀ ná \\
         ónóò bí b-ɔ́gà ya-H pã̂-H jî H-be-nyámɛ̀ ná \\
         {\EXCL} 1\textsc{pl}.{\SBJ}  2-other 1\textsc{pl}-\textsc{prs} do.first-{\R} stay {\OBJ}.{\LINK}-be8-poor still \\
    \trans `Ohhh, we others will first stay still poor.'
 
\exC\label{154}
  \glll  yá bɛ́lɛ́ nà kùrã̂ \\
         ya-H bɛ̀-lɛ́ nà kùrã̂ \\
          1\textsc{pl}-\textsc{prs} be-{\NEG} {\COM} $\emptyset$7.electricity   \\
    \trans `We have no electricity.'
 
\exC\label{155} 
  \glll  mɛ́ dyúwɔ́ nâ mìntángánɛ́ mí nzíí njì mí nzíí njì \\
       mɛ-H dyúwɔ-H nâ mi-ntángánɛ́ mí nzíí njì mí nzíí njì \\
         1\textsc{sg}-\textsc{prs} hear-{\R} {\COMP} mi4-white.person 4 {\PROG}.\textsc{prs} come 4 {\PROG}.\textsc{prs} come   \\
    \trans `I hear that white people are coming and coming.'
 
\exC\label{156} 
  \glll  mìntángánɛ́ métì mí sá náà  \\
         mi-ntángánɛ́ mé-tì mi-H sâ-H nâ  \\
          mi4-white.person 4-DEM[Bulu] 4-\textsc{prs} do-{\R} {\COMP}   \\
    \trans `The white people make that,'
 
\exC\label{157} 
  \glll  bàmɔ̀nɛ́ bá vɛ́ bɔ̂ ɛ́ pɛ̀ sɔ́'ɔ̀ wû \\
        ba-mɔ̀nɛ́ ba-H vɛ̀-H b-ɔ̂ ɛ́ pɛ̀ sɔ́'ɔ̀ wû \\
          ba2-money 2-\textsc{prs} give-{\R} 2-{\OBJ} {\LOC} there before there  \\
    \trans `the money they give them there [in Europe] before. . .'
 
\exC\label{158} 
  \glll  bí bɔ́gà yá wúmbɛ́ ndáà pã̂ nyɛ̂ sâ bá gyíbɔ́ ngyùlɛ̀ wá kùrã̂ \\
         bí bɔ́-gà ya-H wúmbɛ-H ndáà pã̂ nyɛ̂ sâ ba-H gyíbɔ-H ngyùlɛ̀ wá kùrã̂ \\
          1\textsc{pl}.{\SBJ}  2-other 1\textsc{pl}-\textsc{prs} want-{\R} also do.first see $\emptyset$7.thing 2-\textsc{prs} call-{\R} $\emptyset$3.light 3:{\ATT}  $\emptyset$7.electricity[French]  \\
    \trans `We others, we also want to first see the thing they call the light of electricity.'
 
\exC\label{159}
  \glll  wú bɛ́ mà bî ndáwɔ̀ dé tù \\
         wú bɛ̀-H mà bî ndáwɔ̀ dé tù \\
         3 be-{\R} {\COMPL}[Kwasio] 1\textsc{pl}.{\OBJ} $\emptyset$9.house {\LOC} inside   \\
    \trans `That it was already in our houses!'
 
\exC\label{160} 
  \glll   màndáwɔ̀ má báà lwɔ̂ \\
          ma-ndáwɔ̀ má báà lwɔ̂ \\
          ma6-houses 6:{\ATT}  2.{\FUT} build  \\
    \trans `The houses that they will build,'
 
\exC\label{161} 
  \glll má bá lwɔ́ bî \\
         má ba-H lwɔ̂-H bî \\
          6:{\ATT}  2-\textsc{prs} build-{\R} 1\textsc{pl}  \\
    \trans `that they build for us.'
 
\exC\label{162}
  \glll  mɛ̀ɛ́ bɛ́lɛ́ mùdì wà lèkɛ́lɛ̀ \\
         mɛ̀ɛ́ bɛ́-lɛ́ m-ùdì wà le-kɛ́lɛ̀ \\
          1\textsc{sg}.\textsc{prs}.{\NEG} be-{\NEG} {\N}1-person 1:{\ATT}  le5-word  \\
    \trans `I'm not a person of many words.'
\end{exe}

\noindent Severin in Ngumba:

\begin{exe}[(C234)]  
\exC\label{163}
  \gll wɛ̀ wɛ́ yíì nzɛ́ gyí ywɛ̀ límbó màmbì mɔ́-míyà bɔ́ ʃíí sâ \\
         2\textsc{sg}.{\SBJ} 2\textsc{sg} 7.{\COP} who what 2\textsc{sg} know ma6.thing 6-all 2.\textsc{prs} {\PROG} do  \\
    \trans `Who are you? What do you know about all the things they do?'
\end{exe}

\noindent Nze to Mama:

\begin{exe}[(C234)]  
\exC\label{164} 
  \glll wɛ́ làwɔ́ tɛ́ɛ̀\\
       wɛ-H làwɔ-H tɛ́ɛ̀ \\
         2\textsc{sg}-\textsc{prs} talk-{\R} now   \\
    \trans `You speak now.'
\end{exe}

\noindent Mama:

\begin{exe}[(C234)]  
\exC\label{165} 
  \glll èè mɛ̀ jínɔ̀ ná Màmà \\
         èè mɛ̀ j-ínɔ̀ ná Màmà \\
       yes 1\textsc{sg} le5-name SIM $\emptyset$1.{\PN}  \\
    \trans `Yes, my name is Mama.'
 
\exC\label{166} 
  \glll yíì pɔ́nɛ́ kɔ́ɔ̀ lèváá lɛ̀vúdũ̂ nâ bí bá ntɛ́gɛ́lɛ́ bágyɛ̀lì \\
       yíì pɔ́nɛ́ kɔ́ɔ̀ le-váá lɛ̀-vúdũ̂ nâ b-í ba-H ntɛ́gɛlɛ-H H-ba-gyɛ̀lì \\
 7.{\COP} $\emptyset$7.truth still le5-thing 5-one {\COMP} ba2-non.Bagyeli 2-\textsc{prs} bother-{\R} {\OBJ}.{\LINK}-ba2-Gyeli \\
    \trans `It is true, still the same thing that the non-Bagyeli bother the Bagyeli.'
 
\exC\label{167} 
  \glll  mɛ̀ɛ̀ vâ sã̂ wã́ à wɛ́ mɛ̀ɛ̀ mwánɔ̀ nyùlɛ̀\\
        mɛ̀ɛ̀ vâ sã̂ w-ã́ a wɛ̀-H mɛ̀ɛ̀ m-wánɔ̀ nyùlɛ̀\\
          1\textsc{sg}.{\COP} here $\emptyset$1.father 1-{\POSS}.1\textsc{sg} 1.{\PST}1 die-{\R} 1\textsc{sg}.{\COP} {\N}1-child orphan\\
    \trans `I'm here, my father has died, I'm an orphan.'
\end{exe}

\noindent Nze:

\begin{exe}[(C234)]  
\exC\label{168} 
  \glll  èé lûngà yá sã́ wɛ́ɛ̀ yɔ́ɔ̀ yíì \\
          èé lûngà yá sã́ w-ɛ̂ y-ɔ́ɔ̀ yíì \\
           {\EXCL} $\emptyset$7.grave 7:{\ATT}  $\emptyset$1.father 1-{\POSS}.3\textsc{sg} 7-{\OBJ} 7.{\COP}\\
    \trans `Right, his father's grave is over there.'
\end{exe}

\noindent Mama:

\begin{exe}[(C234)] 
\exC\label{169} 
  \glll  lûngà yá sã́ wã̂ yɔ́ bɛ́ yíí \\
         lûngà yá sã́ w-ã̂ y-ɔ́ bɛ̀-H yíí  \\
          $\emptyset$7.grave  7:{\ATT}  $\emptyset$1.father 1-{\POSS}.1\textsc{sg} 7-{\OBJ} be-{\R} 7.{\DEM}.{\DIST} \\
    \trans `My father's grave is over there.'
 
\exC\label{170}
  \glll  bwánɔ̀ bá kálɛ́ bã̂ bɔ́ bá kɛ́ sílɛ̀ pándɛ̀ \\
          b-wánɔ̀ bá kálɛ́ b-ã̂ b-ɔ́ ba-H kɛ̀-H sílɛ pándɛ \\
         ba2-child 2:{\ATT}  $\emptyset$1.older.sister 2-{\POSS}.1\textsc{sg} 2-{\OBJ}[Kwasio] 2-\textsc{prs} go-{\R} finish arrive  \\
    \trans `The children of my older sister, they all arrive.'
\end{exe}

\noindent Nze:

\begin{exe}[(C234)]  
\exC\label{171} 
  \glll  yáà nyɛ̀ wɛ́ nû \\
         yáà nyɛ wɛ́ nû \\
         {\EXCL} 1.{\SBJ}  {\ID} 1.{\DEM}.{\PROX}  \\
    \trans `Yes, this is him.'
\end{exe}

\noindent Mama:

\begin{exe}[(C234)] 
\exC\label{172} 
  \glll  nyã́ã̀ wã̂ núú Ntàbɛ̀tɛ́ndá pɛ̀ \\
         nyã́ã̀ w-ã̂ núú Ntàbɛ̀tɛ́ndá pɛ̀ \\
          $\emptyset$1.mother 1-{\POSS}.1\textsc{sg} 1.{\DEM}.{\DIST} $\emptyset$3.{\PN} there  \\
    \trans `My mother is over there in Ntabetenda [name of village].'
 
\exC\label{173} 
  \glll à nzí kɛ̀ létsíndɔ́ lé ntùmbà wã̂ \\
        a nzí kɛ̀ H-le-tsíndɔ́ lé n-tùmbà w-ã̂ \\
         1 {\PROG}.{\PST}  go {\OBJ}.{\LINK}-le5-funeral.ceremony 5:{\ATT}  {\N}1-older.brother 1-{\POSS}.1\textsc{sg}   \\
    \trans `She was going to my older brother's funeral ceremony.'
\end{exe}

\noindent Nze: 

\begin{exe}[(C234)]  
\exC\label{174} 
  \glll nɔ́gá à nzí wɛ̀ wû \\
        n-ɔ́gá a nzí wɛ̀ wû \\
         1-other 1 {\PROG}.{\PST}  die there   \\
    \trans `That one died over there.'
\end{exe}

\noindent Mama:

\begin{exe}[(C234)] 
\exC\label{175} 
  \glll nɔ́nɛ́gá à nzí wɛ̀ wû \\
         nɔ́-nɛ́gá a nzí wɛ̀ wû \\
         1-other 1 {\PROG}.{\PST}  die there \\
    \trans `That one died over there.'
 
\exC\label{176}
  \glll yɔ́ɔ̀ pɔ̀nɛ̀ vɛ̀ɛ̀ mpù \\
        yɔ́ɔ̀ pɔ̀nɛ̀ vɛ̀ɛ̀ mpù \\
         so $\emptyset$7.thruth still like.this \\
    \trans `It is still true like this.'
 
\exC\label{177}
  \glll bɔ́nɛ́gá bá lɔ́ sílɛ̀ làwɔ̀ nâ bvúlɛ̀ bá ntɛ́gɛ́lɛ́ bágyɛ̀lì\\
      bɔ́-nɛ́gá ba-H lɔ́ sílɛ làwɔ nâ bvúlɛ̀ ba-H ntɛ́gɛlɛ-H H-ba-gyɛ̀lì\\
        2-other 2-\textsc{prs} {\RETRO}   finish speak {\COMP} ba2.Bulu 2-\textsc{prs} bother-{\R} {\OBJ}.{\LINK}-ba2-Gyeli\\
    \trans `The others have just said that the Bulu bother the Bagyeli,'
 
\exC\label{178}
  \glll kɛ̀ nà kwàlɛ̀ bùdã̂ kɛ̀ nà kwàlɛ̀ bùdã̂ bá bágyɛ̀lì\\
        kɛ̀ nà kwàlɛ b-ùdã̂ kɛ̀ nà kwàlɛ b-ùdã̂ bá ba-gyɛ̀lì\\
         go {\CONJ} love ba2-woman go {\CONJ} love ba2-woman  2:{\ATT}  ba2-Gyeli\\
    \trans `coming and loving the women, coming and loving the women of the Bagyeli.'
\end{exe}

\noindent Severin in Ngumba:

\begin{exe}[(C234)]  
\exC\label{179} 
  \gll  jínásá náà wà pɛ́lí lí bɛ̀ nà m-ùrã̂ \\
          mean {\COMP} 2\textsc{sg} {\NEG}.{\PST}  yet be {\COM} 1-woman \\
    \trans `That means that you haven't been yet with a woman?'
\end{exe}

\noindent Nze:

\begin{exe}[(C234)]  
\exC\label{180}
  \glll àà mwã̂ ntùà àà mwã̂ ntúà \\
       àà m-wã̂ ntùà àà m-wã̂ ntúà. \\
         1.{\COP} {\N}1-child small 1.{\COP} {\N}1-child small\\
    \trans `He is a small child, he is a small child.'
\end{exe}

\noindent Mama:

\begin{exe}[(C234)]  
\exC\label{181}
  \glll mɛ̀ɛ̀ nyá mùdì nà nyɛ̂ \\
        mɛ̀ɛ̀ nyá m-ùdì nà nyɛ̂ \\
        1\textsc{sg}.{\COP} real {\N}1-person {\CONJ} 1.{\OBJ}  \\
    \trans `I'm an adult and him [Mambi].'
 
\exC\label{182}
  \glll  yà pálɛ́ bɛ̀ nà bùdã̂ \\
      ya pálɛ́ bɛ̀ nà b-ùdã̂ \\
        1\textsc{pl} {\NEG}.{\PST}  be {\COM} ba2-woman  \\
    \trans `We did not have any women.'
\end{exe}

\noindent Severin in Ngumba:

\begin{exe}[(C234)] 
\exC\label{183} 
  \gll  ɔ́ kɛ́ nà lywɛ́lɛ́ b-ùdã̂ bì-jìnáà \\
        2\textsc{sg}.\textsc{prs} go {\CONJ} show ba2-woman be8-finger  \\
    \trans `You go and hit on women [lit. show women with fingers]?'
\end{exe}

\noindent Mama:

\begin{exe}[(C234)]  
\exC\label{184}
  \glll mɛ̀ fúgɛ̃́ɛ̃̀ \\
        mɛ fúgɛ̃́ɛ̃̀ \\
         1\textsc{sg}.{\PST}1 finish.{\COMPL} \\
    \trans `I have finished.'
\end{exe}

\noindent Nze:

\begin{exe}[(C234)]  
\exC\label{185} 
  \glll á kí náà à sílɛ́ mà \\
        a-H kì-H nâ a sílɛ-H mà \\
         1-\textsc{prs} say-{\R} {\COMP} 1.{\PST}1 finish {\COMPL}[Kwasio] \\
    \trans `He says that he has finished.'
\end{exe}

\noindent Mambi:

\begin{exe}[(C234)] 
\exC\label{186}
  \glll dɔ̃̀ bã̀ yíì nâ bí yá wúmbɛ́ nâ nyá màmbɔ̀ máà mpà \\
      dɔ̃̀ bã̀ yíì nâ bí ya-H wúmbɛ-H nâ nyá m-àmbɔ̀ máà mpà \\
        so[French] $\emptyset$7.word 7.{\COP} {\COMP} 1\textsc{pl}.{\PN}  1\textsc{pl}-\textsc{prs} want-{\R} {\COMP} real ma6-thing 6.{\COP} good  \\
    \trans `So, the truth is that we want the important things to be good,'
 
\exC\label{187}
  \glll màndáwɔ̀ má zì \\
       ma-ndáwɔ̀ má zì \\
        ma6-house 6:{\ATT}  $\emptyset$7.tin[Bulu]  \\
    \trans `tin-roofed houses,'
 
\exC\label{188}
  \glll nà nà kùrã̂ màndáwɔ̀ \\
       nà nà kùrã̂ ma-ndáwɔ̀ \\
        {\CONJ} {\CONJ} $\emptyset$7.electricity ma6-house  \\
    \trans `and, and electricity in the houses.'
 
\exC\label{189}
  \glll kí dyúwɔ̀ nâ bà lwɔ́ ndáwɔ̀ vúdũ̂ ndí màndáwɔ̀ \\
        kí dyúwɔ nâ ba lwɔ̂-H ndáwɔ̀ vúdũ̂ ndí ma-ndáwɔ̀ \\
         {\NEG} understand {\COMP} 2.{\PST}1 build-{\R} $\emptyset$9.house one but ma6-houses \\
    \trans `Not understanding that they [white people] built one house, but houses,'
 
\exC\label{190} 
  \glll mùdì nyɛ̀ ngwɛ̂ màndáwɔ̀ \\
         m-ùdì nyɛ ngwɛ̂ ma-ndáwɔ̀ \\
         {\N}1-person 1 all[Kwasio] ma6-house \\
    \trans `every person [their] houses.'
 
\exC\label{191}
  \glll nà bí bɛ́sɛ̀ kɔ́ɔ̀ kùrã̂  bɛ̀ dé tù \\
       nà bí b-ɛ́sɛ̀ kɔ́ɔ̀ kùrã̂  bɛ̀ dé tù \\
        {\COM} 1\textsc{pl}.{\SBJ}  2-all still $\emptyset$7.electricity  be {\LOC} inside \\
    \trans `with all of us just electricity be inside.'
 
\exC\label{192}
  \glll bã̂ yã̂ màfwálá wɛ́ yíndɛ̀ \\
        bã̂ y-ã̂ ma-fwálá wɛ́ yí-ndɛ̀ \\
         $\emptyset$7.word 7-{\POSS}.1\textsc{sg} ma6-end {\ID} 7-{\ANA} \\
    \trans `My last word is this.'
\end{exe}

\noindent Severin in Ngumba:

\begin{exe}[(C234)] 
\exC\label{193} 
  \gll bíyɔ̀ bí lɛ́ɛ̀ náà sí nyà bɛ́-lɛ́, dí bíyà lwò yɛ́ \\
        2\textsc{pl}.{\SBJ}  2\textsc{pl}.\textsc{prs} say {\COMP} $\emptyset$9.land 9 be-{\NEG} but 2\textsc{pl} build where  \\
    \trans `You say that you don't have any land, but where do you build?'
\end{exe}

\noindent Mambi:

\begin{exe}[(C234)]  
\exC\label{194} 
  \glll báà bù mpàgó pílì pɔ̀dɛ̀ àà lã̀ \\
      báà bù mpàgó pílì pɔ̀dɛ̀ àà lã̀ \\
        3.{\FUT} break  $\emptyset$3.road when $\emptyset$1.port 1.{\FUT} pass \\
    \trans `They will build a road when the port is built.'
 
\exC\label{195}
  \glll à múà njì lã̀, báà bù mpàgó \\
       a múà njì lã̀ báà bù mpàgó \\
        1 be.almost come pass 2.{\FUT} break $\emptyset$3.road  \\
    \trans `It [the port] is about to come [and the road to it will] pass [by here], they will build the road.'
 
\exC\label{196}
  \glll báà bù mpàgó pílì pɔ́dɛ̀ àà vâ njì tsíyɛ̀ vâ \\
      báà bù mpàgó pílì pɔ́dɛ̀ àà vâ njì tsíyɛ̀ vâ \\
         2.{\FUT} break $\emptyset$3.road when $\emptyset$1.port 1.{\COP} here come cut here \\
    \trans `They will build a road when the port is here, coming and cutting [the forest] here.'
 
\exC\label{197}
  \glll bá báà bù mpàgó \\
       bá báà bù mpàgó \\
        2.{\SBJ}  2.{\FUT} break $\emptyset$3.road  \\
    \trans `They will build a road.'
\end{exe}

\noindent Nze:

\begin{exe}[(C234)]  
\exC\label{198} 
  \glll mɛ̀ɛ̀ kɛ̀ búùlɛ̀ yã̂ \\
        mɛ̀ɛ̀ kɛ̀ búùlɛ̀ y-ã̂ \\
        1\textsc{sg}.{\FUT} go $\emptyset$7.old.camp 7-{\POSS}.1\textsc{sg}  \\
    \trans `I will go to my old settlement.'
\end{exe}

\noindent Mambi:

\begin{exe}[(C234)] 
\exC\label{199} 
  \glll èhè báà bù mpàgó nà pámò pɛ̀ Kyíɛ̀ngɛ̀ \\
        èhè báà bù mpàgó nà pámo pɛ̀ Kyíɛ̀ngɛ̀ \\
        {\EXCL} 2.{\FUT} break $\emptyset$3.road {\COM} arrive over.there $\emptyset$7.{\PN}  \\
    \trans `Yes, they will build a road up to Kienge [river and name for Kribi].'
 
\exC\label{200}
  \glll bá nà ngvùlɛ̀ bíyɛ̀ sílɛ̀ lwɔ̃̂ mándáwɔ̀ \\
        bá nà ngvùlɛ̀ bíyɛ̀ sílɛ lwɔ̃̂ H-ma-ndáwɔ̀ \\
        2 {\COM}  $\emptyset$9.strength 1\textsc{pl}.{\OBJ} finish build {\OBJ}.{\LINK}-ma6-house \\
    \trans `They have the strength to build us all houses.'
 
\exC\label{201}
  \glll wɛ̀ dyúwɔ́ mɔ̀ \\
       wɛ dyúwɔ-H mɔ̀ \\
      2\textsc{sg}.{\PST}1 hear-{\R} {\COMPL}   \\
    \trans `Have you understood?'
 
\exC\label{202}
  \glll báà sílɛ̀ bî kúmbà lwɔ̃̂ mándáwɔ̀ \\
        báà sílɛ bî kúmba lwɔ̃̂ H-ma-ndáwɔ̀ \\
        2.{\FUT} finish 1\textsc{pl}.{\OBJ} arrange build {\OBJ}.{\LINK}-ma6-house  \\
    \trans `They will arrange for us to build houses.'
\end{exe}

\noindent Mambi:

\begin{exe}[(C234)]  
\exC\label{203} 
  \glll  bá kɛ́ ndáà nà tɛ́lɛ́ mákùndù má kùrã̂  kɛ́-kɛ́-kɛ́-kɛ́-kɛ́\\
         ba-H kɛ̀-H ndáà nà tɛ́lɛ-H H-ma-kùndù má kùrã̂  kɛ́-kɛ́-kɛ́-kɛ́-kɛ́ \\
        2-\textsc{prs} go-{\R} also {\CONJ} put-{\R} {\OBJ}.{\LINK}-ma6-clay.house 6:{\ATT}  $\emptyset$7.electricity {\IDEO}:repeated.placement \\
    \trans `They also go and put up clay houses with electricity, [depiction of putting the electricity poles along the road].'
 
\exC\label{204}
  \glll wɛ̀ dyúwɔ́ mɔ̀ \\
       wɛ dyúwɔ-H mɔ̀ \\
        2\textsc{sg}.{\PST}1 hear-{\R} {\COMPL}  \\
    \trans `Have you understood?'
\end{exe}

\noindent Nze:

\begin{exe}[(C234)] 
\exC\label{205} 
  \glll ɛ́ pɛ̀ bà sílɛ́ bî lwɔ̃̂ mándáwɔ̀ ɛ́ pɛ̀ \\
        ɛ́ pɛ̀ ba sílɛ-H bî lwɔ̃̂ H-ma-ndáwɔ̀ ɛ́ pɛ̀  \\
        {\LOC} there 2.{\PST}1 finish-{\R} 1\textsc{pl}.{\OBJ} build {\OBJ}.{\LINK}-ma6-house {\LOC} there  \\
    \trans `There, they have finished building us houses there.'
 
\exC\label{206}
  \glll ɔ̀ dyúwɔ́ mɔ̀ \\
        ɔ dyúwɔ-H mɔ̀ \\
        2\textsc{sg}.{\PST}1[Kwasio] understand-{\R} {\COMPL}  \\
    \trans `Have you understood?'
 
\exC\label{207}
  \glll mm nâ yí kádɔ́ nâ mùdì bɛ̀ tí njì nà yímbɔ̀ mhm \\
        mm nâ yi-H kádɔ-H nâ m-ùdì bɛ̀ tí njì nà yímbɔ̀ mhm \\
        {\EXCL} {\COMP} 7-\textsc{prs} be.plenty-{\R} {\COMP} {\N}1-person be go[?] come {\CONJ} visit {\EXCL} \\
    \trans `Mhm, so that there will be plenty [of electricity] so that people come for a visit [which they don't at the moment because there is no electricity]. Mhm.'
 
\exC\label{208}
  \glll pã̂ mɛ̂ láà tè  \\
      pã̂ mɛ̂ láà tè \\
        do.first.{\IMP}  1\textsc{sg}.{\OBJ} tell there  \\
    \trans `Tell me how they would come.'
 
\exC\label{209}
  \glll bímbú lɛ́ mámbòngò mâ wɛ̀ mɛ́dɛ́ dígɛ̂ mɛ́dɛ́ \\
         bímbú lɛ́ ma-mbòngò mâ wɛ mɛ́dɛ́ dígɛ̂ mɛ́dɛ́ \\
       $\emptyset$5.amount 5:{\ATT}  ma6-plant 6.{\DEM}.{\PROX} 2\textsc{sg}.{\SBJ}  self look.{\IMP}  self   \\
    \trans `The number of these plants, take a look yourself [these plants will all be destroyed],'
 
\exC\label{210} 
  \glll nâ á dyúwɔ́ bágyɛ̀lì \\
         nâ a-H dyúwɔ-H H-ba-gyɛ̀lì \\
         {\COMP} 1-\textsc{prs} understand-{\R} {\OBJ}.{\LINK}-ba2-Gyeli \\
    \trans `so that she [Nadine] understands the Bagyeli.'
\end{exe}

\noindent Mambi:

\begin{exe}[(C234)]  
\exC\label{211} 
  \glll bã̀ wɛ́ làwɔ̀ bá dyúwɔ́ sâ yɛ́sɛ̀\\
         bã̀ wɛ-H làwɔ ba-H dyúwɔ-H sâ y-ɛ́sɛ̀ \\
         $\emptyset$7.word 2\textsc{sg}-\textsc{prs} speak 2-\textsc{prs} understand-{\R} $\emptyset$7.thing 7-every\\
    \trans `The words that you speak, they understand everything. [not the language, but what is promised]'
 
\exC\label{212} 
  \glll bíì bá dyúwɔ́ lɛ́kɛ́lɛ̀ lé wɛ́ làwɔ̀\\
        bíì ba-H dyúwɔ-H H-lɛ-kɛ́lɛ̀ lé wɛ-H làwɔ\\
         1\textsc{pl}.{\SBJ}  2-\textsc{prs} understand {\OBJ}.{\LINK}-le5-language 5:{\ATT}  2\textsc{sg}-\textsc{prs} speak\\
    \trans `We, they understand the language that you speak.'
\end{exe}

\noindent Mama:

\begin{exe}[(C234)] 
\exC\label{213} 
  \glll wɛ́ nyɛ́ mbɛ́ yá bá njí líbɛ̀lɛ̀ yíndɛ̀ \\
        wɛ-H nyɛ̂-H mbɛ́ yá ba-H njì-H líbɛlɛ yí-ndɛ̀ \\
        2\textsc{sg}-\textsc{prs} see-{\R} $\emptyset$7.thing 7:{\ATT}  2-\textsc{prs} come-{\R} show 7-{\ANA}  \\
    \trans `You see the thing [camera] that they came to show there.'
 
\exC\label{214}
  \glll wɛ́ tɛ́bɔ́ númbá vúdũ̂ \\
       wɛ-H tɛ́bɔ-H númbá vúdũ̂ \\
        2\textsc{sg}-\textsc{prs} put-{\R} $\emptyset$7.place one  \\
    \trans `Stay in the same place! [don't move because of the camera]'
\end{exe}

\noindent Nze:

\begin{exe}[(C234)]  
\exC\label{215} 
  \glll  mɛ́ mɛ̀ bùdɛ́ nâ ɛ́ pɛ̀  \\
       mɛ́ mɛ bùdɛ-H nâ ɛ́ pɛ̀ \\
         but[French] 1\textsc{sg} have-{\R} {\COMP} {\LOC} over.there \\
    \trans `But I say that over there,'
 
\exC\label{216}
  \glll ɛ́ wû bèyá lwɔ̃́ kwádɔ́ yã̂ ɛ́ wû \\
        ɛ́ wû bèya-H lwɔ̃̂-H kwádɔ́ y-ã̂ ɛ́ wû \\
         {\LOC} there 2\textsc{pl}[Kwasio]-\textsc{prs} build-{\R} $\emptyset$7.village 7-{\POSS}.1\textsc{sg} {\LOC} there\\
    \trans `there you build my village over there.'
 
\exC\label{217} 
  \glll   kwádɔ́ yã̂ màndáwɔ̀ má zì \\
      kwádɔ́ y-ã̂ ma-ndáwɔ̀ má zì \\
         $\emptyset$7.village 7-{\POSS}.1\textsc{sg} ma6-house 6:{\ATT}  $\emptyset$7.tin[Bulu] \\
    \trans `My village, tin houses.'
\end{exe}

\noindent Délégué:

\begin{exe}[(C234)]  
\exC\label{218} 
  \glll  vwálà bùgù yɛ́sɛ̀ \\
        vwálà bùgù y-ɛ́sɛ̀ \\
        there.it.is[French] $\emptyset$7.place 7-all  \\
    \trans `There it is, the entire place.'
\end{exe}

\noindent Nze:

\begin{exe}[(C234)]  
\exC\label{219} 
  \glll mɛ̀ bùdɛ́ nâ á lwɔ́ngɔ́ mɛ̂ màndáwɔ̀\\
        mɛ bùdɛ-H nâ a-H lwɔ́ngɔ-H mɛ̂ ma-ndáwɔ̀ \\
       1\textsc{sg} have-{\R} {\COMP} 1-\textsc{prs} build[Kwasio]-{\R} 1\textsc{sg}.{\OBJ} ma6-house   \\
    \trans `I say that she [Nadine] builds me houses,'
 
\exC\label{220} 
  \glll  búùlɛ̀ yá Ngɔ̀lɔ́ Ngɔ̀lɔ́ Ngɔ̀lɔ́ \\
          búùlɛ̀ yá Ngɔ̀lɔ́ Ngɔ̀lɔ́ Ngɔ̀lɔ́ \\
         $\emptyset$7.old.camp 7:{\ATT}  $\emptyset$3.{\PN} $\emptyset$3.{\PN} $\emptyset$3.{\PN} \\
    \trans `at the old settlement of Ngolo, Ngolo, Ngolo.'
 
\exC\label{221} 
  \glll mɛ̀ɛ̀ lwɔ́ngɔ̀ mándáwɔ̀ Ngɔ̀lɔ́ zì nà zì\\
        mɛ̀ɛ̀ lwɔ́ngɔ H-ma-ndáwɔ̀ Ngɔ̀lɔ́ zì nà zì \\
        1\textsc{sg}.{\FUT} build[Kwasio] {\OBJ}.{\LINK}-ma6-house $\emptyset$3.{\PN} $\emptyset$7.tin[Bulu] {\COM} $\emptyset$7.tin[Bulu]  \\
    \trans `I will build houses in Ngolo, each with tin (roofs).'
 
\exC\label{222} 
  \glll  yɔ́ɔ̀ mɛ́ wúmbɛ́ wû \\
        y-ɔ́ɔ̀ mɛ-H wúmbɛ-H wû \\
         7-{\OBJ} 1\textsc{sg}-\textsc{prs} want-{\R} there \\
    \trans `That is what I want there.'
 
\exC\label{223} 
  \glll  àmú mɛ̀ɛ́ bɛ́lɛ́ nà sí vâ \\
         àmú mɛ̀ɛ́ bɛ́-lɛ́ nà sí vâ \\
         because[Bulu] 1\textsc{sg}.\textsc{prs}.{\NEG} be-{\NEG} {\COM} $\emptyset$9.ground here \\
    \trans `Because I don't have any land here.'
 
\exC\label{224} 
  \glll  bã̀ yá bwánɔ̀ bá lɔ́ làwɔ̀ yíì tè \\
         bã̀ yá b-wánɔ̀ ba-H lɔ́ làwɔ yíì tè \\
         $\emptyset$7.word 7:{\ATT}  ba2-child 2-\textsc{prs} {\RETRO}  speak 7.{\COP} there \\
    \trans `What the children just said is true [lit. their word is there].'
 
\exC\label{225} 
  \glll  mɛ̀ɛ́ bɛ́lɛ́ nà sí vâ \\
         mɛ̀ɛ́ bɛ́-lɛ́ nà sí vâ \\
       1\textsc{sg}.\textsc{prs}.{\NEG} be-{\NEG} {\COM} $\emptyset$9.ground here   \\
    \trans `I don't have any land here.'
 
\exC\label{226}
  \glll mɛ̀ɛ̀ vâ mpínásâ\\
       mɛ̀ɛ̀ vâ mpínásâ \\
        1\textsc{sg}.{\COP} here squeezed  \\
    \trans `I'm squeezed here.'
 
\exC\label{227} 
  \glll dɔ̃̀ sí nyã̂ nyíì búùlɛ̀ yá Ngɔ̀lɔ́\\
        dɔ̃̀ sí ny-ã̂ nyíì búùlɛ̀ yá Ngɔ̀lɔ́ \\
        so[French] $\emptyset$9.ground 9-{\POSS}.1\textsc{sg} 9:COP $\emptyset$7.old.camp 7:{\ATT}  $\emptyset$3.{\PN}  \\
    \trans `So, my land is the old settlement of Ngolo.'
\end{exe}

\noindent Mambi:

\begin{exe}[(C234)]  
\exC\label{228} 
  \glll lé yá wɛ́ nyɛ̂ bá gyíbɔ́ ngàlɛ́ yíì \\
        lé yá wɛ-H nyɛ̂ ba-H gyíbɔ-H ngàlɛ́ yíì \\
       $\emptyset$7.tree 7:{\ATT}  2\textsc{sg}-\textsc{prs} see 2-\textsc{prs} call-{\R} $\emptyset$7.tree.species 7.{\COP}  \\
    \trans `The tree that you see that they call `ngàlɛ́' is that.'
 
\exC\label{229} 
  \glll bá lã́ pámò vâ tɛ́ɛ̀ bà kwɛ̀lɔ̃́ɔ̃̀ yɔ̀ kílɛ̀ dyúwɔ̀  tsíyà \\
      ba-H lã̀-H pámo vâ tɛ́ɛ̀ ba kwɛ̀lɔ̃́ɔ̃̀ y-ɔ̀ kílɛ̀ dyúwɔ̀  tsíyà \\
       2\textsc{sg}-\textsc{prs} pass-{\R} arrive here now 2\textsc{sg}.{\PST}1 cut.{\COMPL} 7-{\OBJ} {\NEG}[Kwasio] hear $\emptyset$1.question  \\
    \trans `They pass and arrive here now, they cut it already, without asking [lit. not hearing a question].'
 
\exC\label{230} 
  \glll  yɔ́ɔ̀ yɔ́ɔ̀ mɛ̀ jìlɛ́ mà \\
       y-ɔ́ɔ̀ y-ɔ́ɔ̀ mɛ jìlɛ-H mà \\
      7-{\OBJ} 7-{\OBJ} 1\textsc{sg}.{\PST}1 place-{\R} {\COMPL}[Kwasio]    \\
    \trans `This, this I have placed [there].'
\end{exe}

\noindent Djiedjhie in Mabi:

\begin{exe}[(C234)] 
\exC\label{231}  
  \gll pfúmá m-í lɛ́ɛ́ mɛ̂ náà bíì tí wúmbɛ̀ sá bì-sálɛ̀ bɔ̀ pwã̂ bì-dólò bí bíná dólò ywɛ̂ bí wúmbɛ́ sá náà líní bí sá bì-sálɛ̀ bɔ́ kíyá bî mà-nyùà  \\
        chief {\N}1-non.Pygmy say 1\textsc{sg}.{\SBJ}  {\COMP} 2\textsc{pl}.{\SBJ}  {\NEG} want do be8-work 2 pay be8-money 2\textsc{pl} refuse $\emptyset$7.money 7.{\POSS}.3\textsc{sg} 2\textsc{pl} want $\emptyset$7.thing {\COMP} when 2\textsc{pl} do bi8-work 2 give 2\textsc{pl} ma6-drink \\
    \trans `The chief of the farmers [Bulu] told me that you don't want to be paid money when you work, you refuse their money, you want that when you work you be given alcohol.'
\end{exe}

\noindent Mambi:

\begin{exe}[(C234)]  
\exC\label{232}  
  \glll  àà kfúmá ndɛ̀ wà Nlúnzɔ̀ \\
        àà kfúmá ndɛ̀ wà Nlúnzɔ̀ \\
        ECXL $\emptyset$1.chief {\ANA}  1:{\ATT}  $\emptyset$1.{\PN}  \\
    \trans `Ah, that chief from Nlunzo!'
\end{exe}

\noindent Nze:

\begin{exe}[(C234)] 
\exC\label{233}
  \glll àà  á sɔ́'ɔ̀ \\
       àà  a-H sɔ́'ɔ̀ \\
         {\EXCL} 1-\textsc{prs} quit  \\
    \trans `Ah, may he quit!'
\end{exe}

\noindent Mambi:

\begin{exe}[(C234)] 
\exC\label{234} 
  \glll  yɛ̀ngɛ̀-yɛ̀ngɛ̀ nâ bùdì bá ndyándyá wû kàlɛ̀gà bíyɛ̀ pándɛ̀ dígɛ̀ bíyɛ̀ vâ yà bùdɛ́ vâ nâ pílì wɛ́ kɛ́ bésàlɛ́ bèjɔ́wɔ̀ bé kùgúù nà bé lévídósí\\
      yɛ̀ngɛ̀-yɛ̀ngɛ̀ nâ b-ùdì ba-H ndyándya-H wû kàlɛga bíyɛ̀ pándɛ dígɛ bíyɛ̀ vâ ya bùdɛ-H vâ nâ pílì wɛ-H kɛ̀-H H-be-sàlɛ́ be-jɔ́wɔ̀ bé kùgúù nà bé le-vídósí\\
       especially {\COMP} ba2-person 2-\textsc{prs} work-{\R} there stop.over 1\textsc{pl}.{\OBJ} arrive watch 1\textsc{pl}.{\OBJ} here 1\textsc{pl} have-{\R} here {\COMP} when 2\textsc{sg}-\textsc{prs} go-{\R} {\OBJ}.{\LINK}-be8-work be8-day.labor 8:{\ATT}  $\emptyset$7.evening {\CONJ} 8:{\ATT}  le5-morning\\
    \trans `Especially people who work there stop over, come to see us here, we say that when you go to do day labor in the evening and in the morning,'
 
\exC\label{235} 
  \glll  dɔ̃̀ wɛ̀ bùdɛ́ ná bàfû wɛ́ yànɛ́ gyàgà bɔ̂ \\
       dɔ̃̀ wɛ bùdɛ-H ná ba-fû wɛ-H yànɛ-H gyàga b-ɔ̂ \\
        so[French] 2\textsc{sg} be-{\R} again ba2-fish 2\textsc{sg}-\textsc{prs} must-{\R} buy 2-{\OBJ}  \\
    \trans `so, you have fish again, you have to buy them.'
 
\exC\label{236} 
  \glll wɛ́ símásá ndáà sìgá \\
        wɛ-H símasa-H ndáà sìgá \\
        2\textsc{sg}-\textsc{prs} regret-{\R} also $\emptyset$1.cigarette \\
    \trans `You also regret [that you don't have] a cigarette [because you cannot afford it].'
 
\exC\label{237}
  \glll  wɛ́ símásá ndáà ŋwándɔ́ \\
       wɛ-H símasa-H ndáà ŋwándɔ́ \\
       2\textsc{sg}-\textsc{prs} regret-{\R} also $\emptyset$3.manioc.stick   \\
    \trans `You also regret [not to have] manioc sticks [local starchy food].'
 
\exC\label{238} 
  \glll  wɛ́ yànɛ́ ná gyàgà ndísì \\
     wɛ-H yànɛ-H ná gyàga ndísì \\
        2\textsc{sg}-\textsc{prs} must-H again buy $\emptyset$3.rice  \\
    \trans `You must again buy rice,'
 
\exC\label{239} 
  \glll pílì wɛ́ lèmbó nâ bùdì bá ndáwɔ̀ bvùbvù \\
      pílì wɛ-H lèmbo-H nâ b-ùdì bá ndáwɔ̀ bvùbvù \\
        when 2\textsc{sg}-\textsc{prs} know-{\R} {\COMP} ba2-person 2:{\ATT}  $\emptyset$9.house many  \\
    \trans `when you know that there are many people at home.'
\end{exe}

\noindent Nze:

\begin{exe}[(C234)] 
\exC\label{240} 
  \glll  ɛ̀sɛ́ béé ndáà bèyá làwɔ́ fàlà \\
       ɛ̀sɛ́ béé ndáà bèya-H làwɔ-H fàlà \\
        is.it[French] 2\textsc{pl}.{\SBJ}  also 2\textsc{pl}[Kwasio]-\textsc{prs} speak-{\R} $\emptyset$1.French  \\
    \trans `Isn't it, you also, you speak French.'
 
\exC\label{241} 
  \glll  mɛ̀ɛ́ láwɔ̀lɛ̀ fàlà \\
        mɛ̀ɛ́ láwɔ-lɛ̀ fàlà \\
         1\textsc{sg}.\textsc{prs}.{\NEG} speak-{\NEG} $\emptyset$1.French \\
    \trans `I don't speak French.'
 
\exC\label{242}
  \glll nzá núù dè nzá núù nyímɛ̀ \\
         nzá núù dè nzá núù nyímɛ \\
       who 1.{\FUT} eat who 1.{\FUT} refuse   \\
    \trans `Who will eat, who will stay hungry [lit. refuse].'
\end{exe}

\noindent Mambi:

\begin{exe}[(C234)]  
\exC\label{243} 
  \glll  pílì wɛ́ kɛ́ gyàgà báfû bábáà \\
        pílì wɛ-H kɛ̀-H gyàga H-ba-fû bá-báà \\
        when 2\textsc{sg}-\textsc{prs} go-{\R} buy {\OBJ}.{\LINK}-ba2-fish 2-two  \\
    \trans `When you go to buy two fish. . .'
 
\exC\label{244}
  \glll ká bá kɛ́ wɛ̂ vɛ̀ bébwúyà bébáà nà màwú mátánɛ̀ \\
      ká ba-H kɛ̀-H wɛ̂ vɛ̀ H-be-bwúyà bé-báà nà ma-wú má-tánɛ̀ \\
        if 2-\textsc{prs} go-{\R} 2\textsc{sg}.{\OBJ} give {\OBJ}.{\LINK}-be8-hundred 8-two {\CONJ} ma6-ten 6-five  \\
    \trans `If they go give you 250 [Francs],'
\end{exe}

\noindent Nze:

\begin{exe}[(C234)] 
\exC\label{245} 
  \glll wɛ́ sá tè ná  \\
        wɛ-H sâ-H tè ná \\
        2\textsc{sg}-\textsc{prs} do-{\R} there how \\
    \trans `how do you manage there? [because it's very little money]'
 
\exC\label{246} 
  \glll  mhm mɛ̀ Nzìwù wɛ̂\\
         mhm mɛ Nzìwù wɛ́ \\
       {\EXCL} 1\textsc{sg} $\emptyset$1.{\PN}  {\ID}  \\
    \trans `Mhm, I'm Nziwu.'
\end{exe}

\noindent Mambi:

\begin{exe}[(C234)]  
\exC\label{247} 
  \glll wɛ́ ná báàlá nà nyɛ́ fí nà wɛ́ ndyándyá ná sálɛ́ ɛ́ pɛ̀ nà wɛ́ kòlá ná mɔ̀nɛ́ nû \\
      wɛ-H ná báàla-H nà nyɛ̂-H fí nà wɛ-H ndyándya-H ná sálɛ́ ɛ́ pɛ̀ nà wɛ-H kòla-H ná mɔ̀nɛ́ nû \\
         2\textsc{sg}-\textsc{prs} again repeat-{\R} {\CONJ} see-{\R} different {\CONJ} 2\textsc{sg}-\textsc{prs} work-{\R} again $\emptyset$7.work {\LOC} over.there {\CONJ} 2\textsc{sg}-\textsc{prs} add-{\R} again $\emptyset$1.money 1.{\DEM}.{\PROX} \\
    \trans `You repeat [it] again and try something else [find other work] and you work there again and you add this money again [same amount of 250 Francs].'
\end{exe}

\noindent Nze:

\begin{exe}[(C234)]  
\exC\label{248} 
  \glll yɔ́ɔ̀ nû àá láwɔ̀lɛ̀\\
       yɔ́ɔ̀ nû àá láwɔ-lɛ̀ \\
         so 1.{\DEM}.{\PROX} 1.\textsc{prs}.{\NEG} speak-{\NEG} \\
    \trans `So this one doesn't speak [teasing Délégué who is deaf-mute, but the joke is that he doesn't speak because he is guilty of having himself been exploited].'
 
\exC\label{249}
  \glll  kɔ́ɔ̀ nyɛ́gà á làwɔ́ ndáà\\
         kɔ́ɔ̀ nyɛ́-gà a-H làwɔ-H ndáà \\
         only  1.{\SBJ}-{\CONTR} 1-\textsc{prs} speak-{\R} also \\
    \trans `As for him, he would also say something. [teasing: if he wasn't guilty, he would also speak and protest].'
 
\exC\label{250} 
  \glll  mhm dzámɛ́ ngá nyɛ̂ \\
       mhm dzámɛ́ ngá nyɛ̂ \\
         {\EXCL} excuse {\PL}  1.{\OBJ} \\
    \trans `Mhm, excuse him [teasing: excuse him for accepting the poorly paid work].'
 
\exC\label{251}
  \glll bí bɛ̂ yá lɔ́ làwɔ̀ \\
      bí b-ɛ̂ ya-H lɔ́ làwɔ \\
        1\textsc{pl}.{\SBJ}  2-all[Kwasio] 1\textsc{pl}-\textsc{prs} {\RETRO}  speak \\
    \trans `We all just spoke.'
 
\exC\label{252} 
  \glll yà bɛ́ bùdì báláálɛ̀ \\
       ya bɛ̀-H b-ùdì bá-láálɛ̀ \\
        1\textsc{pl}.{\PST}1 be-{\R} ba2-person 2-three  \\
    \trans `We were three people.'
\end{exe}

\noindent Mambi:

\begin{exe}[(C234)]  
\exC\label{253} 
  \glll  kɔ́ɔ̀ sílɛ̀ \\
        kɔ́ɔ̀ sílɛ \\
       just finish   \\
    \trans `Just finish.'
\end{exe}
