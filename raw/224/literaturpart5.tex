\WeitereLiteratur{

\paragraph*{Einführungen}

\begin{sloppypar}

Grundlegend zur Einführung in die Schreibprinzipien des Deutschen ist der \textit{Grundriss} \citep[Kapitel~8]{Eisenberg2013a}.
Eine kurze Einführung in die Orthographie ist \citet{Fuhrhop2015}, und \citet{FuhrhopPeters2013} verbindet eine ausführliche Einführung in die Phonologie mit einer ausführlichen Einführung in die theoretische Graphematik.
Speziell zur Interpunktion gibt \cite{Bredel2011} einen kompakten Überblick, teilweise mit deutlich anderen Erklärungsansätzen als hier.
\citet{Maas1992} liefert eine sehr ausführliche deskriptive Bestandsaufnahme zur deutschen Orthographie vor der Reform von 1996. 
In den angesprochenen Bereich der Eigennamen führt \citet{NueblingEa2012} ein.
Der hier vertretene Fremdwortbegriff (einschließlich des Begriffs des Kernwortschatzes) ist kompatibel zu \citet{Eisenberg2012}.
Einen Überblick über die Diskussionen hinter der letzten großen Reform der deutschen Orthographie bietet \citet{AugstEa1997}.
Einen Überblick über Schriftsysteme im Allgemeinen sowie die Geschichte der Schrift findet man in \citet{Duerscheid2012a} und \citet{Coulmas1989}.

\paragraph*{Weiterführende Lesevorschläge}
\citet{Gallmann1995} zur Substantivgroßschreibung;
\citet{Eisenberg1981} zu Problemfällen der Großschreibung mit besonderer Berücksichtigung der Eigennamen;
\citet{SchaeferSayatz2014} sowohl zu Klitisierungsphänomenen und deren Verschriftung als auch zur empirisch fundierten Erforschung von Gebrauchsschreibungen;
\citet{Jacobs2005} zur Getrennt- und Zusammenschreibung;
\citet{Buchmann2015} zu den Wortzeichen;
\citet{Primus1993} zum Komma;
\citet{Primus2008} zu einer klassischen Domäne des Gedankenstrichs.
\end{sloppypar}

}
