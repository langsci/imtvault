\WeitereLiteratur{

\paragraph*{Phonetik}

Eine sehr ausführliche Einführung in die artikulatorische Phonetik ist \citet{Laver1994}.
Einführende Darstellungen der deutschen Phonetik finden sich \zB in \citet{RuesEa2009} und \citet{Wiese2010}.
Eine ausführliche Beschreibung der deutschen Standardvarietäten (Deutschland, Österreich, Schweiz), der wir hier überwiegend gefolgt sind, gibt \citet{KrechEa2009}.
Ein weiteres Nachschlagewerk mit kleinen Unterschieden in der Darstellung zu \citet{KrechEa2009} ist \citet{Mangold2006}.

\paragraph*{Phonologie}

\label{abs:001}

Der hier zur Phonologie besprochene Stoff findet sich mit teilweise erheblichen Abweichungen in der Darstellung \zB in \citet{Hall2000} und \citet{Wiese2010}.
In eine grammatische Gesamtbeschreibung eingebunden sind Kapitel~3 und~4 im \textit{Grundriss} \citep{Eisenberg2013a}.
Eine Einführung, die das Phonemkonzept verwendet, ist \citet{Ternes2012}.
Als anspruchsvolle Gesamtdarstellung der deutschen Phonologie kann \citet{Wiese2000} verwendet werden.
Ein gut lesbarer Artikel zur hier nicht besprochenen phonetischen Motivation der Phänomene an der Silbengrenze ist \citet{Maas2002}. 

}
