\chapter{Communication, Frege's puzzle, and reference} \label{ch:centrality of communication}

We have now traversed the whole path of Equilibrium Semantics. Looking back, it is possible to see many different dimensions of the problem of meaning -- philosophical, linguistic, psychological, computational, and even humanistic, encompassing many aspects of the human sciences as discussed especially in Chapters~\ref{ch:beyond illocutionary meaning} and \ref{ch:beyond language}. Throughout, I have tried to demonstrate that the idea of equilibrium pervades all of semiosis.

%I hope it fulfills the promise made at the start to connect Equilibrium Semantics to a variety of disciplines, not only philosophy, the various subfields of linguistics, psychology, and artificial intelligence, but also the cognitive and social sciences, including especially economics.

Over the course of its modern history, many have felt the intuitive appeal of a use-based or communication-based foundation for semantics but the content of this Wittgensteinian\ia{Wittgenstein, Ludwig@Wittgenstein, Ludwig} insight has always been obscure, even mysterious. What can one possibly do with \emph{use}? It seems like such a structureless, anti-theoretical idea. Indeed, even connecting use with communication appears ineffectual because the latter looks like it has little to offer beyond Austinian\ia{Austin, J. L.@Austin, J. L.} and Gricean\ia{Grice, Paul@Grice, Paul} informalities. And so the relative -- but viciously circular and question-begging\footnote{See the quote by Dummett\ia{Dummett, Michael@Dummett, Michael} in \sectref{sec:classic example}.} -- safety of a Fregean\ia{Frege, Gottlob@Frege, Gottlob} and logicist\is{logicism} truth-based\is{truth} foundation has continued to beckon.

In my view, neither use-based unsystematicity nor truth-based systematicity offer a scientific approach to semantics. The only remaining possibility is to follow Grice, whose singular attempt to understand communication in the small and in the large was systematic but informal. I have tried to present a very thorough critique of practically his entire oeuvre together with that of his many followers whether they are orthodox Griceans, neo-Griceans, or post-Griceans.

%\footnote{See mainly \chapref{ch:Grice} but also Parts~\textrm{IV} and \textrm{V}.}

Equilibrium Semantics may be seen as a mathematical alternative to Grice\ia{Grice, Paul@Grice, Paul} that organizes the raw material of semantics quite differently. It rests on four fundamental ideas: reference, use, indeterminacy,\is{indeterminacy} and equilibrium, the first taken from ideal language philosophy which was mainly truth-based, the second taken from ordinary language philosophy which was mainly use-based, and the last two supplied by my framework itself.\footnote{I have discussed these four ideas in a general way in some detail in my previous book and so will not do so here.} These are developed into Communication Games and Language Games and into four Constraints, Semantic, Syntactic, Phonetic, and Flow, based on the mathematical frameworks of situation theory and game theory. It offers in principle a unified approach to all of semantics in a way that is philosophically sound, mathematically solid, computationally tractable, and empirically adequate.


\section{Four philosophical benefits}

As I mentioned in \chapref{ch:why communication is central}, there are at least four advantages to considering meaning via communication as Equilibrium Semantics does. 

%in addition to the detailed understanding of communication itself. 

The first is that because it tackles the main question of semantics -- how language acquires meaning -- directly, the framework's many results impose a top-down \emph{constraint} on possible answers to its various subquestions involving particular linguistic constructions. Indeed, the subtasks of semantics devolve into building alternative theories \emph{within} the framework for the same phenomena that can be straightforwardly compared because they share a common foundation. In the same way, theories within the framework for different phenomena can be synthesized and integrated owing to a common foundation. I tried to demonstrate this latter kind of integration in \citet[Chapter~6]{parikh:le} where I developed accounts of descriptions, names, and generalized quantifiers that all shared the same basic assumptions about how noun phrases work.

The only other possibility is to be bottom-up and piecemeal and hope that the proliferation of foundational assumptions will somehow get reconciled. It should be obvious that this is a hopeless task. In fact, many theorists end up relying on foundational Gricean assumptions, not pausing sufficiently to ask whether relegating certain things to ``pragmatics'' and to the Gricean maxims is a sound move.\footnote{Even \citet{kripke:srsr} is guilty of this.} If my critique of Grice is correct then this recourse is no longer available.

%when it is time to address the main question

%For example, it is often taken for granted that Gricean implicatures can be computed via his maxims.

Thus, if Equilibrium Semantics is accepted as a constraint on the formation of theories of meaning, it will be much easier to tackle the challenges more complex constructions pose with an elaboration of uniform tools so that the many subquestions of semantics all cohere into an integrated solution to its overarching problem of the relation of language to reality and knowledge. I illustrate this in the next section by returning to Frege's puzzle of informative identities that was posed in \sectref{sec:classic example}.

As we saw, Dummett\ia{Dummett, Michael@Dummett, Michael} gives a further reason to pursue meaning via communication: to sidestep the interdependence of truth and meaning. Those relying on truth and truth-conditional semantics or pragmatics, the most popular approach today, never show how the truth conditions are derived from first principles. A moment's thought will reveal that it is the theorist who already understands the example being discussed who provides the truth conditions. But how do the interlocutors arrive at them? The truth conditions specify the meaning of an utterance but this very meaning is also presupposed in identifying the truth conditions. 

My approach has described use directly and derived meaning from use. Once this is done, it is possible to use truth conditions or infon-based contents, the latter representing more fine-grained information conditions. This procedure dovetails with the first advantage listed above because when the content of an utterance is assumed, it is assumed \emph{relative} to a background framework like Equilibrium Semantics. In such cases, one knows whereof one speaks because the assumptions can be made good by simply deriving the content from first principles when required. It is only then that the relevant theory can be said to be complete.

%because truth conditions cannot differentiate between a content and the same content augmented with a necessary truth like $2 + 2 = 4$

The third philosophical benefit is to reduce meaning to communication and then rely on Dretske's further reductions of mental representations to physical facts. This allows us to see meaning as part of the natural world. I believe I have developed this reduction fairly completely throughout the book. With the truth-conditional strategy, this possibility remains murky at best. \citet[170]{lewis:gs} writes:

\begin{quote}
I distinguish two topics: first, the description of possible languages or grammars as abstract semantic systems whereby symbols are associated with aspects of the world; and, second, the description of the psychological and sociological facts whereby a particular one of these abstract semantic systems is the one used by a person or population. Only confusion comes of mixing these two topics.
\end{quote}

As I have tried to show, especially in \partref{part:V}, these two topics are inextricably linked because Language Games depend on interlocking Communication Games and the conventional meanings of Communication Games result from their participation in Language Games. That is, macro-semantics and micro-semantics cannot be separated from each other. Indeed, it could not be otherwise because language and meaning are just one large social institution that hangs together \emph{only} because its different parts and levels cohere. This is precisely what the burden of this book has been, to show how the main problem of semantics, how language acquires meaning in the small and in the large, can be solved.

%although one can certainly abstract from various aspects of meaning when addressing other aspects of meaning given that one has some idea of how to fill in the missing details and given that one's analyses are compatible with these details.

Lastly, I have argued that the \isi{vagueness} of language is an important \emph{natural} source of its normativity,\is{normativity} arising as it does from ``the great variation of things in the world combined with the limitations on our concepts'' as quoted from \citet[21]{murphy:bbc} in \chapref{ch:vagueness}. Thus, this elusive property, too, becomes clear in one of its aspects via understanding the communication of vague utterances.

%I now turn to Frege's puzzle.


\section{Solving Frege's puzzle} \label{sec:solving Frege's puzzle}

I will develop just the part of Equilibrium Semantics required for the solution. I omit the four different uses of names, referential, attributive, generic, and predicative, referred to earlier and analyzed in detail in \citet[Chapter~6]{parikh:le}. Apart from the particular theory I offer, I want to emphasize that Equilibrium Semantics is a framework and as such is compatible with other possible accounts although it also rules out some. This is the sense in which it provides a constraint on semantic theories. 

%I have both of these goals, the development of a particular solution and the illustration of the framework as a constraint, in mind here.

It is often not realized that the cognitive significance of identity statements involves \emph{two} items of information. In the case of Hesperus and Phosphorus, not only does the addressee learn that the names \Expression{Hesperus} and \Expression{Phosphorus} co-refer but also that one is the heavenly body seen in the evening and the other is the heavenly body seen in the morning and that they are the same.\footnote{It seems even Frege himself may not have realized this because his \emph{Begriffsschrift} solution attends to the first item involving names and his \emph{On Sense and Reference} solution attends to the second item involving modes of presentation. Subsequent scholarship, as far as I know, has also attended to one or the other but never to both in a single solution.} Most people subscribe instinctively to a direct reference theory where each name is just a label for the referent. And then neither item of information is available in the trivial content of the utterance and its cognitive significance becomes a mystery.

%The second item involving the way the addressee accesses the two names may be trivial if she has never heard either name before.

Direct reference theories are also held by sophisticated contemporary theorists who argue for quite counterintuitive consequences of their theory such as Hesperus's identity with Phosphorus being a metaphysically necessary truth that is nevertheless not epistemologically a priori as it is an empirical discovery.

I now solve Frege's puzzle in a way that survives challenges posed by more complex constructions and by other constraints such as \citegen{putnam:mm} arguments about the nature of meaning.\footnote{I do not discuss how my solution offers a way of solving Putnam's problems about meaning but anyone familiar with them should be able to see how. Indeed, I feel my solution is more satisfactory than his because he includes the referent in the conventional meaning or intension of the term, which is a bit artificial. If it is kept separate in a resource situation and both the conventional meaning and resource situation jointly determine the referent then we get a more natural solution.}


\subsection{Preliminaries}


\subsubsection{Causal chains}

\citet{kripke:nn} and others developed the notion of a \isi{causal chain} to account for how the reference of a name occurs. The idea is that someone in a linguistic community starts using a name to refer to an entity, say a heavenly body, and then this use spreads through the community. \citet[Chapter~7]{kp:cp} call the structure of references a network. A \isi{causal chain} can in fact be rigorously defined as following a certain path within the network of Communication Games called a Language Game.\footnote{Incidentally, so can \citegen{putnam:mm} division of linguistic labor.} In general, there will be multiple causal chains for any name and the network that results will be a subgraph of a Language Game because not all parts of the Language Game will involve the name. One consequence of such a rigorous definition is that it allows us to construct the infon $\soa{R^c;\ a;\ N;\ u}$ where $a$ is the referent and source of the chain, $N$ is the name that refers in the situation $u$, and $R^c$ is the relation linking $a$ to $N$ in $u$. There will be cases, however, where there is no real referent $a$ and then $a$ will be replaced either by a corresponding fictional object drawn from a fictional universe or by the contradictory infon $\mathbf{0}$. 

%I will not be concerned with such cases as my interest here is more circumscribed.

%In fact, it is not just names that involve causal chains. So do common nouns like \Expression{water} and \Expression{tiger}. That is, there are multiple chains originating in samples of water and instances of tigers linked to various uses of these terms. I will say more about these below but for now I will stick to names.


\subsubsection{Resource situations and modes of presentation}

The idea of resource situations comes from \citet{bp:sa}. At least one resource situation $r_u$ is associated with every noun phrase via $u$. Every resource situation associated with a name $N$ contains an infon like $\soa{R^c;\ a;\ N;\ u}$ \emph{by definition}. That is, $r_u(N)$ is a resource situation induced by $u$ only if $r_u(N) \vDash \soa{R^c;\ a;\ N;\ u}$ for some $R^c$, $a$ relative to $N$ and $u$. Not all resource situations involve such \is{causal chain}causal chains; some may be perceptual or may depend on an attribute.

%It is convenient to drop the $N$ from $r_u(N)$ and retain just $r_u$ as the name is understood from the context.

%Not all resource situations involve such causal chains, only the ones that matter here. With some noun phrases, resource situations may be perceptual or may depend on an attribute.

\citet{frege:sr} conflated the distinct notions of sense and mode of presentation. Sense is analogous to conventional meaning, which for names is just the property $P^N = \hbox{\emph{named N}}$. The mode of presentation is what enables a speaker or addressee to fix a reference and is analogous to a resource situation.\footnote{I myself conflated these two notions in \citet[291]{parikh:le} and posited the hybrid notion of an indexical property which was a property that contained a link to an external object. But such a combination is not required as the link can be housed within a resource situation. If we wish to avoid \citegen[pages 68--70]{kripke:nn} noncircularity condition even for the property taken by itself, we can reinstate the hybrid notion. See footnote~\ref{foot:kripke} in \chapref{ch:picture of communication}.\ia{Kripke, Saul A.@Kripke, Saul A.}}

\subsubsection{The extension function}

In \sectref{sec:information} I had defined the following function:

\begin{equation}
e(P,s) = \left\{\begin{array}{@{}l@{~}l@{}}
a & \mbox{if there is exactly one object $a$ having $P$ in $s$} \\
\{x \mid s \vDash \soa{P;\ x}\} & \mbox{otherwise}
\label{eq:eps}			\end{array}
		\right.
\end{equation}
~\\
In the context of \is{causal chain}causal chains and resource situations for names, I substitute $P^N$ and $r_u$ for $P$ and $s$:

\begin{equation}
e(P^{\hbox{\emph{N}}},r_u) = \left\{\begin{array}{@{}l@{}}
a,~\text{if there is exactly one object $a$ having $P^{\hbox{\emph{N}}}$ in $r_u$} \\ 
%& \mbox{and $r_u \vDash \soa{R^c;\ a;\ N\ u}$} \\
\{x \mid r_u \vDash \soa{P^{\hbox{\emph{N}}};\ x}\},~\mbox{otherwise}
\label{eq:eps}			\end{array}
		\right.
\end{equation}
~\\
$e(P^{\hbox{\emph{N}}},r_u)$ will provide the content of names like \Expression{Hesperus} and \Expression{Phosphorus}:\linebreak $e(P^{\Expression{Hesperus}},r_u) = \hbox{Venus}$ and $e(P^{\Expression{Phosphorus}},r'_u) = \hbox{Venus}$ where $r_u$, $r'_u$ are the two corresponding resource situations.\footnote{I am omitting certain complications involving conditioned infons that are considered in \citet{parikh:le} as the present definitions suffice for our purposes.}

\subsubsection{Functions treated intensionally and extensionally}

Generally, functions are treated extensionally. For example, $f(x) = x + 2$ is taken to be equal to $g(x) = x + 4 - 2$ because the ``external'' values both functions yield are the same. However, they are not intensionally\is{intensionality} equal because their other properties  such as the time taken to compute their outputs may differ. So functions taken intensionally may be different objects even though extensionally they are the same.

In the same way, we can treat the extension function above intensionally or extensionally. Thus, $e(P^{\Expression{Hesperus}},r_u)$ and $e(P^{\Expression{Phosphorus}},r'_u)$ are different objects intensionally because their first and second arguments differ. Specifically for our purposes, the \is{causal chain}causal chains their second arguments contain are different. Extensionally, however, they are the same and just have the value Venus. 


\subsection{Setting up the analysis}

Now consider an utterance situation $u$ involving a conversation about astronomy where $\cal A$ says ``Hesperus is Phosphorus'' to $\cal B$. As our interest is in solving Frege's puzzle, we can abstract from many features of the Communication Game 
$\Gamma_u$ and focus just on the locutionary Semantic Constraint. We can also ignore the mythological meanings of ``Hesperus'' and ``Phosphorus'' as they would be eliminated by the Flow Constraint.

This may seem like a trivial step as theorists do it almost unconsciously but the difference is that now one knows precisely what is being abstracted from so that the missing details can be filled in if required. It is just this kind of awareness that has led many researchers to consider utterances instead of sentences as they realized that certain contextual aspects of utterances could not be ignored.

\subsection{Semantic Constraint}

Let the sentence uttered be $\varphi = \varphi_1\varphi_2\varphi_3$ where $\varphi_1 = \hbox{\Expression{Hesperus}}$, $\varphi_2 = \hbox{\Expression{is}}$, and $\varphi_3 = \hbox{\Expression{Phosphorus}}$. Also, $R^{=}$ is just the relation of equality.

\ea \Expression{Hesperus}:\\
Referential Use: $\varphi_1 \longrightarrow P^{\varphi_1} \stackrel{u}\longrightarrow e(P^{\varphi_1},r_u) = \hbox{Venus} = \sigma_1$  
\z
\ea \Expression{is}:\\
Predicative Use: $\varphi_2 \longrightarrow P^{\varphi_2} = R^{=} \stackrel{u}\longrightarrow \soa{R^{=}} = \sigma_2$\footnote{Actually, there are two other uses of \Expression{is} that I discuss in \citet{parikh:le}, the so-called auxiliary and membership uses. The current predicative use just picks out the equality relation. However, it is conditioned by the requirement that it refer to a time period overlapping with the time of utterance and also belonging to a finite time interval.}
\z
\ea \Expression{Phosphorus}:\\
Referential Use: $\varphi_3 \longrightarrow P^{\varphi_3} \stackrel{u}\longrightarrow e(P^{\varphi_3},r'_u) = \hbox{Venus} = \sigma_3$
\z



%\noindent \Expression{is}:
%
%\begin{itemize}
%
%\item Auxiliary Use: $\varphi_2 \longrightarrow P^{\varphi_2} \stackrel{u}\longrightarrow 
%\soa{t \mid (t \parallel t_u) \wedge t, t_u \in [t_{0},t_{\infty}]} = \sigma_2$
%
%\end{itemize}
%
%\noindent As we saw in Section~2.3.2, $t \parallel t_u$ means that the action conveyed by the verb temporally overlaps with the time of utterance. The only constraint on $t_0$ and $t_{\infty}$ is that $t_u$ belong to the interval $[t_0, t_{\infty}]$. $\sigma_2$ is defined here just as $\sigma_1$ and $\sigma'_1$ were earlier. 

\subsection{The analysis}
As we have abstracted from the different possible uses of names and also from the lexical ambiguities in $\varphi$, the Flow Constraint involves trivial games and yields the following locutionary content for the utterance: \[ \sigma^{\ell} = \soa{R^{=};\ e(P^{\varphi_1},r_u);\ e(P^{\varphi_3},r'_u)} \]

Since we can treat the arguments of this infon both intensionally and extensionally, it follows immediately that the content carries nontrivial information because only the extensional equality holds. Indeed, both items of information that make up the utterance's cognitive significance are conveyed. There is no particular reason why the different \is{causal chain}causal chains in $r_u$ and $r'_u$ should involve the same object Venus. This requires, in fact, an empirical discovery. 

In an utterance of ``Hesperus is Phosphorus'' it is usually the intensional content that is intended as the extensional content is trivial. The elimination of the latter would also be effected by the Flow Constraint.

If we consider the sentence ``Hesperus is Hesperus'' instead then its content would be: \[ \tau^{\ell} = \soa{R^{=};\ e(P^{\Expression{Hesperus}},r_u);\ e(P^{\Expression{Hesperus}},r_u)} \]

\noindent where $\varphi_1 = \varphi_3 = \hbox{\Expression{Hesperus}}$ and the two resource situations are now the same. This equality is trivial as expected and, indeed, it holds both extensionally and intensionally. 

%Thus, we have solved a simple version of Frege's puzzle. 

%Of course, Frege himself solved this particular puzzle with his notions of sense and reference. But, as I mentioned earlier, this is really a family of puzzles involving all kinds of noun phrases including pronouns as well as more complex constructions, which his solution fails to solve. 

Suppose someone says, ``Woodchucks are groundhogs'' in some situation $u$. Now this is a case where both terms have the same conventional meaning because the property of being a woodchuck is the same as the property of being a groundhog. That is, their senses are the same so Frege's solution will not work. But the resource situations will be different and contain different \is{causal chain}causal chains terminating in the same kind of rodent, and this allows nontrivial information to be learned from the identity.\footnote{I would like to thank Ed Zalta for extended conversations about this topic over the years. See \citet[Chapter~VI]{zalta:ao} and \citet[Chapters~9--12]{zalta:ilmi}.}

%I cannot compare my solution with other solutions, whether of the Fregean kind or of the direct reference kind, as that would take us far afield. One merit of my approach is that the distinct resource situations with distinct causal chains and other information in them capture the full cognitive significance of an utterance of ``Hesperus is Phosphorus.'' 

As I said at the start of this section, one item of information is that the first is the heavenly body seen in the evening and the second is the heavenly body seen in the morning and they are the same. The first part of this information is contained in $r_u$ both via the \is{causal chain}causal chain $\soa{R^c_1;\ \hbox{Venus};\ \hbox{\Expression{Hesperus}};\ u}$ and via background information about when Hesperus is observed (i.e.\ in the evening). This latter fact is also embedded in the relation $R^c_1$ because the causal link involves Hesperus in the evening. The second part of this information is contained in $r'_u$ via the different \isi{causal chain} $\soa{R^c_2;\ \hbox{Venus};\ \hbox{\Expression{Phosphorus}};\ u}$ and via different other facts such as when Phosphorus is observed (i.e.\ in the morning). And the third part, the identity relation, is of course contributed by \Expression{is}. So one nontrivial significance of ``Hesperus is Phosphorus'' is clearly available to the speaker and addressee. (The same kind of difference occurs in the resource situations for woodchucks and groundhogs. One term may have been encountered in the field and the other in a dictionary. And so on.) The second item involving the coreference of the two names is also readily available from the distinct causal chains. So my solution is able to account for both items of cognitive significance.

Many proposals exhibit some difference between ``Hesperus is Phosphorus'' and ``Hesperus is Hesperus,'' either that the two names co-refer or that the two modes of presentation result in the same object, without, however, deriving both and so miss the former's full cognitive significance. 

%This, therefore, leaves the puzzle only partially solved. 

%by such efforts as the full source of all the new information revealed remains unexplained.

Imagine something like an Agatha Christie\ia{Christie, Agatha} murder mystery in which the characters' identities have been scrambled and in solving the mystery, Hercule Poirot grandly announces, ``Smith isn't Smith but Jones is Jones.'' Here, the first conjunct isn't contradictory but informative and the second conjunct isn't trivial but also informative just like ``Hesperus is Phosphorus.'' My approach easily handles such utterances because the relevant resource situations for each occurrence of the two names are different and contain different facts. Most other solutions, whether Fregean or of the direct reference kind, would find them hard to account for. 


\subsection{Wider considerations}
Direct reference theorists and others who treat the contents of identity statements only extensionally are led to conflate the trivial and informative contents mentioned above, and from there, inescapably, to the startling conclusion that they are synthetic necessary truths known a posteriori. \citegen{kripke:nn} discovery that it is important to distinguish the metaphysical notion of necessity from the epistemological notion of aprioricity and also from the linguistic notion of analyticity is certainly valid. In general, though, as Kripke shows himself, whether or not there are contingent a priori truths or necessary a posteriori truths is an open question and one that should have little to do with how we use our language; it should have to do with how the world is and with how agents might learn its many truths. For example, we may discover through computational means that a certain large number is a prime: this is then a necessary a posteriori truth.

Perhaps our most important means of acquiring information is through linguistic communication; not only that, much of the information we have could not be had without language.  But the closer scrutiny initiated by Kripke himself should caution us against taking one of the relatively simple uses of language as evidence for rather counterintuitive claims about what are essentially metaphysical and epistemological concepts.

Possibly, mathematical identities such as $7 + 5 = 12$ can also be seen in this dual intensional-extensional way and then their being informative would follow. Of course, they still remain necessary truths unlike ``Hesperus is Phosphorus'' and the question of how necessary
truths can be informative -- like learning that a large number is a prime -- is a different one.  

%But the first step to answering it is to make clear that the content of mathematical utterances is at least not trivial. 

%This way of looking at the uses
%of noun phrases (whether in mathematics or elsewhere) brings out the difference between necessary and a priori truths even more sharply.

\section{Defining reference}
I briefly point out an immediate consequence of the framework: it becomes easy to define reference.\is{reference} For Equilibrium Semantics the notion is broader than the standard notion because the content of any legitimate expression that is uttered is a referent. In an utterance of ``Bill ran,'' both ``Bill'' and ``ran'' refer.

There are two natural notions of reference, ``subjective reference'' corresponding to speaker meaning and ``reference'' to communication. The former allows a referent not to be communicated whereas the latter involves the successful transmission of the referent to the addressee. The expression $\alpha$ below can be subsentential or a whole sentence and $\sigma$ is a partial or full infon.

%I first reproduce the definitions for speaker meaning and communication from \sectref{sec:speaker meaning and word meaning} for the reader's convenience.
%
%\begin{definition}
%
%$\cal A$ means $p$ by uttering $\varphi$ in $u$ if and only if there is an addressee $\cal B$ such that $\cal A$ intends to convey $p$ to $\cal B$ in $u$ (a proper part of which intention may be implicit) and the solution to the game $G^{\cal A}_u(\varphi)$ between $\cal A$ and $\cal B$ that $\cal A$ considers in uttering $\varphi$ in $u$ and (nonconsciously) believes to be common knowledge between them is $p$.
%
%%\label{def:means 2}
%\end{definition}
%
%\begin{definition}
%
%$\cal A$ communicates $p$ to $\cal B$ by uttering $\varphi$ in $u$ if and only if $\cal A$ intends (possibly partly implicitly) to convey $p$ to $\cal B$ in $u$, $\cal B$ intends (possibly partly implicitly) to interpret $\cal A$'s utterance of 
%$\varphi$ in $u$, and the three games $G^{\cal A}_u(\varphi)$, $G^{\cal B}_u(\varphi)$, $G_u(\varphi)$ induced thereby are equal and common knowledge and their solution is $p$.
%%\label{def:communicates 3}
%\end{definition}

%These definitions lead to the corresponding definitions for what I will call ``subjective reference'' and ``reference.'' 



\begin{definition}
$\cal A$ subjectively refers to $\sigma$ by uttering $\alpha$ in $u$ if and only if there is a situation $c$ such that $\cal A$ means $p = (c \vDash \sigma)$ by uttering $\alpha$ in $u$.
\label{def:subjectively refers}
\end{definition}


\begin{definition}
$\cal A$ refers to $\sigma$ by uttering $\alpha$ in $u$ if and only if there is a situation $c$ and an addressee $\cal B$ such that $\cal A$ communicates $p = (c \vDash \sigma)$ to $\cal B$ by uttering $\alpha$ in $u$.
\label{def:refers}
\end{definition}

\noindent Here, $p = (c \vDash \sigma)$ is a partial or full proposition based on the partial or full infon $\sigma$. When $\sigma = \soa{a} = a$ where $a$ is an individual, we get the familiar case of singular reference. 

%As should be evident, reference is essentially speaker meaning or communication depending on one's interest. 

%I will not tarry to discuss how these definitions are to be interpreted in various situations, especially those involving error.

%\section{Communication and Content}
%
%\begin{center}
%
%`The time has come,' the Walrus said, \\ 
%      `To talk of many things: \\
%Of shoes --- and ships --- and sealing-wax --- \\
%      Of cabbages --- and kings --- \\
%And why the sea is boiling hot --- \\
%      And whether pigs have wings.' 
%
%\end{center}
