\chapter{Semantic change} \label{ch:semantic change}

I referred in \sectref{sec:macro-semantics} to the fact that not only do conventional meanings emerge through communication, they also evolve over the centuries. Such semantic change is part of the broader changes that occur in all aspects of language. My goal here is to sketch the mechanism by which such changes occur and show how it fits in with the rest of Equilibrium Semantics. As such, it merely scratches the surface of the large field of historical linguistics and, in particular, historical semantics and historical pragmatics, an instance of which may be found in \citet{td:rsm} for example.

An important constraint on almost all language change is that it must occur during communication. A speaker may utter a sentence to communicate some information to an addressee. Usually, the utterance merely reinforces standard ways of speaking (or writing). Occasionally, one or more aspects of the utterance -- either semantic, syntactic, or phonetic -- are altered. For example, a single word of the sentence uttered may involve a broadened meaning. The addressee then has to comprehend the utterance despite the alteration. If she is successful, the relevant change gets established between the two agents and can be repeated between them. Such a change may then also be transmitted to other agents by either of the two agents via similar utterances and may thereby spread through the linguistic community and become a change in the language.

Uttering sentences thus serves at least two ends: the first, generally conscious and intended, is to enable the flow of information\footnote{I mean to include all kinds of performative utterances of the kind \citet{austin:htdtww} introduced under this broad description.} from speaker to addressee, and the second, generally unconscious and unintended, is to either reinforce or change aspects of language. Because communication and language change must occur together, the constraints on the former also help to constrain the latter.

%The way this second result of communication arises does not seem to have been studied much at the level of its fine structure. A close look at the latter provides a clue to how languages change.

A key constraint that makes change possible was called \emph{overdetermination} in \citet[Section~6.1.5]{parikh:le}. Overdetermination in this second sense\footnote{The first sense was discussed in \sectref{sec:a simple language game} in the context of Language Games.} is the presence of an element in an utterance that provides information about another element in that utterance. For example, in the sentence, ``The dog didn't bark,'' the verb phrase \Expression{didn't bark} constrains the noun phrase to refer to something that can bark -- such as a dog or, less likely, a seal -- in most ordinary utterance situations. Even if the word \Expression{dog} had been eclipsed by some noise, the addressee would still have understood the utterance as conveying the information that the \emph{dog} didn't bark because in such everyday situations it is only dogs that bark.\footnote{Consider the sentence ``His boss didn't bark at him.'' This and similar examples show that overdetermination works only when the utterance situation is also sufficiently constraining to rule out other alternatives. \label{foot:bark example}} We can say here that the verb phrase overdetermined the meaning of the noun phrase given the context. In general, the elements that do the overdetermining can be semantic, syntactic, phonological, or contextual.\footnote{This notion of overdetermination is a generalization of the idea of selectional restriction.}

The second constraint that enables change is the mechanism through which overdetermined contents are transmitted, namely, \emph{modulation} or \emph{free enrichment} or \emph{implicature}. 

%I will draw upon the analyses of these phenomena in \partref{part:IV}. 

%These two constraints that make an initial change in language possible. This initial change may then spread through the community if other conditions obtain.

I will propose a model of language change based on overdetermination and modulation and on the propagation of an initial change through the community by looking at an example of semantic broadening. Many ideas in the theory of language change (e.g.\ \citealt{cb:hl}) appear to be qualitative and large-scale, that is, they do not examine language change in the small and then relate it to the community-wide scale. In my view, only this kind of communicative cum propagative process can really explain it. Large-scale evolutionary models by themselves abstract too much from the essence of the process. A better way is to see the evolution as a sequence of small steps, each of which is modeled in detail.


\section{Semantic broadening: The communicative aspect} \label{sec:Semantic Broadening: The Communicative Aspect}

The modern English word \Expression{dog} derives from the middle English word \Expression{dogge} (which comes from the old English word \Expression{docga}). Earlier, the word referred to a powerful breed of dogs (such as Great Danes) that originated in England. It was later semantically broadened (and transformed morphologically). This change must have come about through communicative interactions and I now look at the details of the semantic transformation. I abstract from the morphological change and so will use the word \Expression{dog} to refer to both its narrower and broader meanings.

%That is, many speakers may have attempted to broaden the meaning in roughly the same way but this particular speaker actually succeeded in doing so by efficaciously conveying the broader meaning to his addressee.

Presumably, the word was used by some particular speaker in a sentence to \emph{communicate} the broader meaning for the first time in the history of English. We can reconstruct the kind of original situation in which the change occurred and was communicated successfully by the following thought experiment. Middle English existed from roughly 1150 CE to 1470 CE and the paradigm situation I consider is relatively more modern. A more authentic example would require delving into the relevant sociology of late medieval households.

Suppose $\cal A$ and $\cal B$ have a poodle\footnote{Apparently, the poodle was imported into Great Britain from France in the 1870s but treat the time as the 15th century.} and they engage in the following dialogue.

\begin{quote}
$\cal B$: When did you get home last night?\\
$\cal A$: Around midnight. The dog didn't bark.
\end{quote}

It is not necessary to know \emph{why} $\cal A$ used \Expression{dog} to refer to their poodle. Maybe it was meant affectionately. Be that as it may, in the scenario sketched, the word was used to refer to their poodle and not to some larger dog which would have been the then-standard use.

Consider the second sentence $\cal A$ utters: ``The dog didn't bark.'' ($\omega = \omega_1\omega_2\omega_3\omega_4$)

There are two possibilities for $\omega_2 = \Expression{dog}$. The first is the old conventional meaning of $\omega_2$, the property $P^{\omega_2}_{1}$ of being a Great Dane, say. The second is the new conventional meaning $P^{\omega_2}_{2}$ the  utterance somehow succeeds in establishing. Initially, the addressee considers just the first of these.

Now, first the locutionary Semantic and Flow Constraints are activated for $\omega_2$ and the other words. This leads to a problem as there is no large dog in the resource situation accompanying the utterance. This triggers the illocutionary Semantic Constraint which results in finding $P^{\omega_2}_{2}$ and the corresponding referent by using overdetermination, and then a mixed Flow Constraint resolves any ambiguities that remain.

How does $\cal B$ decide whether the new conventional meaning $P^{\omega_2}_{2}$ is the property of being a dog (in its broadened modern sense) or the property of being a poodle? Here, it is arguable that there is a linguistic constraint that involves a preference -- in \emph{some} situations -- for semantic broadening (from Great Danes to any dog) over a shift to a parallel and coordinate property (from Great Danes to poodles). In the first case of broadening, the earlier meaning is superseded by the new meaning; in the second case, the earlier meaning remains and the new meaning becomes a second meaning of the word. The latter is costlier in general and so the first option is preferred in \emph{certain} situations.\footnote{Consider the (semantically related) term ``bitch.''  It derives from the old English word ``bicce'' and originated as a neutral term for a female dog, a meaning it still has. But when it was used to refer to a human, apparently around the fifteenth century, with an intended meaning like \emph{bad woman} or \emph{virago}, that meaning stuck. It was not broadened to something like a female mammal. See the relevant entry in the Oxford Dictionary of English Etymology. My thanks to Tom Wasow for this example.\ia{Wasow, Thomas}}

%It is not clear at this stage how to characterize the range of situations over which this sort of preference operates.

At this stage, we can say that in the communication between $\cal A$ and $\cal B$ a new conventional and therefore referential meaning get conveyed via modulation. There is no guarantee that the new conventional meaning will stick, even just between the two interlocutors. Indeed, most new meanings are short-lived. But any initial change has the potential to be propagated through the community and become a lasting change.


\section{Semantic broadening: The propagative aspect}

Recently, evolutionary models that abstract from the details of the propagation and study how suitable fractions of the population converge to a steady state based on positive or negative reinforcements have become popular.\footnote{See \citet{yang:pslc} and the references therein.} In my view, such models are too abstract, and assumptions involving reinforcements function more or less as mathematical fictions that enable the derivation of desirable empirical results. An alternative approach that builds on the preceding model of language change in the small is as follows.

After $\cal A$ has successfully communicated the new meaning to $\cal B$ above, the change may last in either person's mind or it may fade away. So we may assume there is some probability $q$ of a change being communicated and lasting between the two interlocutors given that a conversation takes place between them. If the change lasts, then either of them can pass it on to a third party via a similar modulation. And so on.

Start with a finite community of $n$ persons 1, 2, 3, \ldots , $n$.\footnote{Thus, $n$ does double duty: it serves as a name of a member and also indicates the total number of members.} Then let $1$ have some altered meaning for a word in mind at time $t = 0$.\footnote{The member $1$ plays the same role of originator as $\cal A$ did in the previous section.} The people in the community who accept the new meaning are collectively called the change set. Thus, at $t = 0$ the change set is just \{1\}. Assume that conversations take place with probability $p > 0$ between some pairs $j$ and $k$ and with probability zero for other pairs. If a communication is successful and the change is lasting, it cannot be reversed. This gives us a sequence of random graphs $G_t(n,p)$ at each subsequent stage $t = 1,2,3, \ldots$ where the $n$ vertices represent the $n$ members and an edge between any two vertices represents a conversation between the two corresponding persons and occurs independently for some pairs of vertices with probability $p > 0$. I will assume that each graph in the sequence is connected in the same way, that is, there is a path between any two nodes where a path is a sequence of distinct nodes with edges between successive nodes having $p > 0$.\footnote{A connected graph is usually just one where there is a path between any two vertices. In the case of a random graph, one has to add the requirement that $p > 0$ over the edges constituting a path. Moreover, in our case, the connections must be the same over time in the sequence of random graphs $G_t(n,p)$. I have deliberately used a slightly more restrictive notion of path as it is more convenient for our purposes. It is possible to work these random graphs into the Language Games considered earlier as the two are related but this formulation is simpler. Keep in mind that we know empirically that there are on average six degrees of separation between any two people on the planet.} If at least one of the vertices belongs to the change set prior to the time of communication and the communication succeeds and the change is lasting, then the other vertex is added to the change set. In this way, the change set keeps expanding and hopefully converges to the entire community of $n$ vertices.

%I had initially considered a random graph in which everyone talks to everyone directly with positive probability but this kind of strong connectedness is unrealistic in any community of reasonable size. The weaker assumption made here of a connected graph seems to more accurately model a linguistic community. After all,   

The basic question is: what are the conditions on $n$, $p$, and $q$ under which this happens? This is not the most general formulation of the situation but it suffices to state a simple version of the problem.

Let $C$ denote the entire community $\{1, 2, 3, \ldots , n\}$ and $C_t$ denote the change set at the end of period $t$. We would like to ask under what conditions $\lim_{t \rightarrow \infty} C_t = C$.

If $u$, $v$ in $C$ are two distinct vertices, let $u \sim v$ represent an edge between $u$ and $v$ or, in other words, a conversation between them. Then the probability of a conversation between $u$ and $v$ is $P(u \sim v) = p$ as specified by the statement of the problem. Also, if $E$ is the event of a communication between $u$ and $v$ whose effect persists, then $P(E \mid u \sim v) = q$ again as specified. 

As a result, the probability of a successful conversational event with a lasting communication at any given instant is given by $P(u \sim v\ \cap\ E) = P(u \sim v)P(E \mid u \sim v) = pq$. Thus, the probability of the complementary single failure at any instant is $P([u \sim v\ \cap\ E]') = (1 - pq)$.

Now let $k \in C$ be an arbitrary member distinct from $1$. Because the graphs are connected, there is always a path between $1$ and $k$. If we are able to compute the probability of $k$ entering $C_t$, then we would be able to see what happens to it in the limit as $t \rightarrow \infty$. Let this probability be denoted by $P(k \in C_t)$. 

First consider a path $(1, k)$ of length $l = 1$ where the vertices $1$ and $k$ are adjacent. If $k$ fails to be in $C_t$ given a path of length $l = 1$, this must have the probability of a single failure $P([u \sim v\ \cap\ E]')$ repeating itself for instants at $1, 2, 3, \dots , t$. This allows us to write: \[ P(k \not\in C_t \mid (1, k))\ =\ P([u \sim v\ \cap\ E]')^t\ =\ (1 - pq)^t \]

The probability of the complementary event, that is, the probability of success over this path, is: \[ P(k \in C_t \mid (1, k))\ =\ 1 - (1 - pq)^t \]

Next consider a path of length $l = 2$ and let the intervening node be $j$. That is, let the particular path be $(1, j, k)$ where $j \in C$ is different from $1$ and $k$. The key thing to be aware of here is that if $k$ is to become a member of $C_t$, then it can do so only after $j$ has first become a member of the change set. Let $t_0 = 0$, let $t_1$ be the time at which $j$ succeeds, and let $t_2 = t$. Then, it is possible to write the following: \[ P(k \in C_t \mid (1, j, k))\ =\ [1 - (1 - pq)^{t_1 - t_0}][1 - (1 - pq)^{t_2 - t_1}] \]

\noindent which is nothing but the success of $j$ at $t_1$ followed by the success of $k$ at $t_2$.

Now suppose a particular path of arbitrary length $l$ is $(1, j_1, \ldots , j_{l - 1}, k)$, where each element of the sequence is a distinct member of $C$. Then for $k$ to become a member of the change set, it is necessary that each $j_m$ has to have become a member of the change set first and in the right order, starting with $j_1$ and ending with $j_{l - 1}$. If we let $t_1, t_2, \ldots , t_{l - 1}$ be the successive times at which the corresponding nodes $j_1, j_2, \ldots , j_{l - 1}$ become members of the change set, and if we let $\Delta t_{m} = t_m - t_{m - 1}$ for $m = 1, 2, \ldots , l$ with $t_0 = 0$ and $t_l = t$, the probability of success over this path is given by:
\begin{equation}
P(k \in C_t \mid (1, j_1, \ldots , j_{l - 1}, k))\ =\ \prod_{m = 1}^{l} [1 - (1 - pq)^{\Delta t_{m}}]
\label{eq:connected}
\end{equation}

As $\Delta t_{m} \rightarrow \infty$ for all $m$ from $1$ to $l$, the conditional probability of success tends to 1 as long as $pq > 0$, that is, as long as $p > 0$ and $q > 0$. 

If there is more than one path between $1$ and $k$, the (unconditional) probability of success $P(k \in C_t)$ can only be greater. That is: \[ P(k \in C_t \mid (1, j_1, \ldots , j_{l - 1}, k))\ \leq\ P(k \in C_t)\ \leq\ 1 \] 

\noindent which implies that:
\begin{equation}
\lim_{t \rightarrow \infty} P(k \in C_t)\ =\ 1
\end{equation}

\noindent as well by the so-called squeeze theorem from elementary calculus.

Since $k$ was arbitrarily chosen, this shows that: \[ \lim_{t \rightarrow \infty} C_t\ =\ C \] 

\noindent as long as $p > 0$ and $q > 0$. The convergence is independent of $n$, the size of the community. 

In other words, as long as a path exists between the originator and every member of the community and the probability of a lasting communication is positive, all members will eventually fall into the change set. That is, an initial change between two members becomes a change in the language. This result fits the example I have considered as the narrow meaning of \Expression{dog} has completely disappeared. 

%Also, given the assumptions I have made, it is exactly what one would expect intuitively. 

There are many ways of relaxing the simplifications assumed earlier. Many potential chains of change could be broken by an errant member, that is, $q$ is likely to vary from pair to pair rather than be constant across the community. Also, one could assume there are two originators of two distinct meanings for a word and study what happens in a community whose graph forms two relatively isolated but connected components. This may be how linguistic variation occurs, creating distinct communities and languages over time when a sufficient number of changes accumulate in one component of a community. \citegen{labov:plcsf} detailed study of social factors influencing linguistic change may also suggest other ways to expand the model.

%of both refining and generalizing the relatively abstract graphs presented here.

%What I have shown is a way to model pairwise conversations in a community that lead to interesting consequences for the whole community. Presumably, this is how language change diffuses in a population from the small to the large.


\section{Generalizing the model}

%I have shown earlier how communication may be viewed as conveying not just semantic contents but also syntactic and phonetic contents. 

This account of semantic broadening can be adapted to other kinds of language change. For example, a syntactic change involving the present continuous (e.g.\ ``I am liking your shoes'') would initially appear odd but would, via syntactic modulations, recover the intended meaning that \emph{I like your shoes}.\footnote{The example is an instance of a variant of Indian English and is influenced by the corresponding syntax of many Indian languages.} 

%If the change in question sticks between the originating interlocutors, the new syntactic construction may then spread to other members of the community and become a change in the language.

I have added a philosophical and pragmatic dimension to the problem of language change by connecting it to a theory of communication. Does it account for all types of language change? Probably not. Even if it does apply to, say, phonetic change, it must operate at a less conscious level than our semantic example.

%than the kind of example of semantic change I have considered.

I have also tried to propose a new graph-theoretic model that builds on the idea of change occurring step by communicative step in a community. It appears to offer certain advantages over large-scale evolutionary models involving segments of populations by sticking more closely to the knitting, as it were.

As both these aspects can be generalized in many directions, they together amount to an outline of a framework for the study of language change.

