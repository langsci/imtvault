\chapter{A picture of communication} \label{ch:picture of communication}

Communication in the small and in the large is a roughly \emph{circular} process in a manner reminiscent of Dilthey's\ia{Dilthey, Wilhelm} famous hermeneutic circle.\is{hermeneutic circle} In the small, I have shown in \emph{Language and Equilibrium} how the interdependence of speaker and addressee as well as the mechanism of conveying and interpreting meaning make communication circular. In this book, I develop this theme further. In the large, this circularity is amplified because communication in society is ongoing and because such societal communication consists of smaller, interlocking conversations. That is, the conversations are linked with one another via the shared conventional meanings of words which emerge, in turn, from these conversations. The smaller model focuses on dialogues between two (or more) participants and describes what I call \emph{micro-semantics}; the larger model focuses on several such interconnected micro-semantic models and describes what I call \emph{macro-semantics}.

\section{Micro-semantics} \label{sec:micro-semantics}

I start by showing a snapshot of the intricate process that occurs in micro-seman\-tics in an utterance situation: 

\begin{enumerate}[itemsep=0pt]
\item[] \underline{Utterance Situation}
\item[] Setting Game
\item[\functionarrow] $\cal A$'s wish to elicit some response from $\cal B$
\item[\functionarrow] Content Selection Game
\item[\functionarrow] $\cal A$'s equilibrium content
\item[\functionarrow] Generation Game
\item[\functionarrow] $\cal A$'s equilibrium utterance
\item[\functionarrow] Interpretation Game
\item[\functionarrow] $\cal B$'s equilibrium content
\item[\functionarrow] Content Selection Game
\item[\functionarrow] $\cal B$'s equilibrium response
\item[\functionarrow] Back to the Setting Game
\end{enumerate}

I now proceed to explain these terms in more detail.

\subsection{The Setting Game}

A widely prevalent picture of communication starts with something the speaker wishes to convey. This is only partly correct. A more accurate starting point is that the speaker and addressee are involved in some interaction and this utterance situation or \emph{setting} induces the speaker to affect the addressee either by conveying information to her and triggering a corresponding belief or by eliciting information from her or, more generally, by getting her to act or react in some way. It may happen  that the speaker does not care about the addressee's response but even here the conveying and reception of information is still desired. 

This setting will generally be a game or decision problem of some kind. For example, $\cal A$ and $\cal B$ may be discussing where to eat and $\cal A$ may wish to influence $\cal B$'s choice. Such a situation, involving multiple options and involving possibly differing preferences for them, may take many forms, one of which may be the  agonistically dubbed game \is{Battle of the Sexes}\emph{Battle of the Sexes}.\footnote{Practically every textbook on game theory considers this game, which is almost as widely known as the \isi{Prisoner's Dilemma}.} Its normal form typically looks as shown in Figure~\ref{fig:battle of the sexes in normal form}.

\begin{figure}
\begin{tikzpicture}
    \matrix (coordgame) [inner ysep=0pt,draw, matrix of math nodes, every node/.style={text width=3cm, align=center,anchor=base, inner ysep=.5\baselineskip}] {
        (3,2) & (0,0) \\
        (0,0) & (2,3) \\ 
    };
    \draw (coordgame.north) -- (coordgame.south);
    \draw (coordgame.west) -- (coordgame.east);
    \node[above=.5\baselineskip of coordgame-1-1,anchor=base] {Restaurant one};
    \node[above=.5\baselineskip of coordgame-1-2,anchor=base] {Restaurant two};
    \node[left=1em of coordgame-1-1.base west,anchor=base east,align=right,overlay] {Restaurant one};
    \node[left=1em of coordgame-2-1.base west,anchor=base east,align=right,overlay] {Restaurant two};
\end{tikzpicture}
% % % \begin{tabular}{l|c|c|}
% % % \multicolumn{1}{l}{} & \multicolumn{1}{c}{Restaurant one} & \multicolumn{1}{c}{Restaurant two} \\  \cline{2-3}
% % % Restaurant one  & $(3,2)$ & $(0,0)$ \\\cline{2-3}
% % % Restaurant two  & $(0,0)$ & $(2,3)$ \\\cline{2-3}
% % % \end{tabular}
\caption{A Battle of the Sexes game in normal form}\label{fig:battle of the sexes in normal form}
\end{figure} 

The payoffs in such a Setting Game reflect the explicit or implicit goals and subgoals of the agents. Someone may have said something earlier like ``Why don't we eat out tonight?" which makes their joint goal explicit. Or the goal may have evolved  implicitly. These goals are similar to what \citet{roberts:is} calls ``domain goals.'' They may be either cooperative or conflictual.

The Setting Game is crucial for a theory of communication because it contains information, especially about the agents' goals, some of it shared, that constrains the future actions of both participants and helps in explaining them. There is often a hierarchy of goals in the utterance situation. For example, their joint goal may be to eat out but this leads to $\cal A$'s subgoal to influence $\cal B$'s choice of restaurant based on the payoffs in the Setting Game which reflect their preferences for different restaurants. 

%I will show later that the reasoning agents have to undertake often involves all of these interconnected goals or preferences.

The Setting Game takes place in the utterance situation and results in $\cal A$'s wish to elicit some response from $\cal B$ as indicated in the snapshot above. This desire may lead $\cal A$ to select some content to convey to $\cal B$.


\subsection{The Content Selection Game}

Based on how $\cal A$ wishes to affect $\cal B$,\footnote{Other ``external'' factors such as the presence of eavesdroppers may also constrain $\cal A$'s choices.} he considers the suitability of various \emph{contents}---\emph{not} utterances yet. As human beings are only partly rational, a speaker may not deliberate on all the relevant choices, only a few or possibly just one. $\cal A$'s pondering possible contents and the responses each would elicit from $\cal B$ results in a kind of game that seems superficially like what is called a \is{game!signaling}\emph{signaling} game.\footnote{A signaling game is a particular kind of game among many different types of games studied in game theory. It starts with a random move by ``Nature,'' that is, a kind of abstract device like the flip of a coin, which results in the first player, the Sender, acquiring some private information that the other player, the Receiver, does not know. The task of the Sender is to convey this private information to the Receiver by choosing one of a set of signals or utterances. The task of the Receiver is to take some action that affects both players based on the signal received. The first signaling games were invented by \citet{lewis:c}. They were later studied by \citet{spence:jms}, \citet{cs:sit}, and \citet{kreps:oeb}.

Some in the game-theoretic linguistics community combine features of signaling games with my former partial information games.\is{game!partial information} For example, David Lewis\ia{Lewis, David@Lewis, David} was interested in the emergence of meaning and so he used initially meaningless utterances. Those writing since 2003, such as Robert van Rooij,\ia{van Rooij, Robert@van Rooij, Robert} Gerhard J\"{a}ger,\ia{J\"{a}ger, Gerhard@J\"{a}ger, Gerhard} and Anton Benz,\ia{Benz, Anton@Benz, Anton} were  analyzing implicature and so could not assume initially meaningless utterances. So they adopted the idea from my games that the speaker's intentions constrain the range of sentences available, and related assumptions. I have compared signaling games and my games in several places and so I discuss such issues briefly in this book. The important point is not how different my games are from signaling games but whether the substantive structures employed are useful to compare directly or not. Besides, these theorists employ an orthodox Gricean\ia{Grice, Paul@Grice, Paul} framework, something I criticize in detail in \chapref{ch:Grice}. I hope readers from this community will find it useful to relate the substantive structures and ideas I offer to their frameworks.} But it will soon become clear why Content Selection Games are in fact very different from signaling games. For the moment, notice that in the latter the first player, the so-called sender, has to choose among a set of possible signals he can send. Signals are just utterances but in the Content Selection Game the sender has to decide among a set of \emph{contents}, so while our game may \emph{look} similar to a signaling game, it is different in substance.

I will call this structure the \emph{Content Selection Game} because in it the primary task for the speaker is to select the best content from his choices and the primary task for the addressee is to select her best response from her choices. Concretely, if $\cal A$ wants to influence $\cal B$'s choice of a restaurant, he may have to choose between a content like \emph{should we go to Bar Boulud?} or like \emph{should we go to Bar Boulud? we haven't been there in a while}. The latter are \emph{contents}, not sentences, and italics will be used to refer to particular contents throughout.\footnote{One can also use the infons of situation theory to represent such contents.} Once $\cal B$ learns what $\cal A$ wishes to convey, she would have a variety of choices, ranging from a simple \emph{yes} or \emph{no} to other possibilities such as \emph{how about Bouley instead?} Just as the addressee cannot fully anticipate what the speaker might convey, the speaker cannot fully anticipate how the addressee might respond. So this game is typically one with unaware players without common knowledge of the game.\footnote{See \citet[Sections~5.3 and 6.5]{parikh:ul} and, for a formal account, \citet{hr:egpup,hr:gscgpup}.\label{foot:halpern}} Despite this, I will assume common knowledge for such games as it is simpler to do so.

$\cal A$ then solves the Content Selection Game and identifies its equilibrium content. This search for his optimal action can be a tricky affair because this game can be conflictual. The interests of the two parties may not be perfectly aligned. 

%For now, I note that the content $\cal A$ selects will be the best one available to evoke the desired response in $\cal B$ given the latter's own goals.

%\footnote{The words ``best'' and ``optimal'' are meant to be synonymous with ``equilibrium.'' It is common to distinguish between an optimal choice and an equilibrium choice in game theory because an optimal choice, if evaluated in the absence of what other players may do, can diverge from the equilibrium. I just mean the best a player can do \emph{given} the other players' (best) choices. \label{foot:optimal}}

The way the addressee's choice influences the choice of content (and vice versa) has seldom been considered explicitly in semantics. Theories always start with a sentence the speaker has uttered as given. It is by attending to this game in the context of the whole communicative process that it becomes doubtful if the Gricean (\citeyear{grice:lc}) maxims work as they are meant to. A key reason is the presence of conflict; another is that in addition to conveying information, a speaker typically wishes to affect his relationship with the addressee in some way, either maintaining or altering it via the \emph{effect} the utterance has on her.\footnote{\citet{al:sc} address somewhat similar issues.}

Relating the foregoing to the snapshot of communication above, the Content Selection Game arises from $\cal A$'s desire to evoke a response from $\cal B$ and results in the selection of an equilibrium content to convey to $\cal B$.

\subsection{The Generation Game and the Interpretation Game}

If telepathy were possible, $\cal A$ would directly transmit this choice to $\cal B$ and that would be the end of his effort. Unfortunately, he now has to find the right words to convey his chosen content. This involves considering different combinations of words based on their conventional meanings. For example, two options he may contemplate are \Expression{Should we go to Bar Boulud?} and \Expression{Should we go to Bar Boulud? We haven't been in a while.} Throughout, small capitals or quotation marks will be used for expressions.

Stepping back momentarily, $\cal A$ and $\cal B$ will generally have a language $\cal L$ they share. Sharing a language is a complex matter. For now, assume it is the linguistic community's nonconscious common knowledge of the conventional meanings of the words of the language as well as of its grammar and phonetic system. So $\cal A$ will look for a sentence or two in $\cal L$ that will express the content he has identified given the utterance situation he is in. If $\cal L$ is a very simple (artificially constructed) language that has neither ambiguous nor structured expressions, his task will be much simpler in one sense although an overly simple language may not be sufficiently rich for all that he wants to convey. When $\cal L$ is a full-blown natural language like English or Gujarati,\ia{Gujarati} things get a bit complicated because there are three dimensions at play simultaneously. One is the conventional meanings of the words, a second is the internal structure of the sentence, that is, their syntax, and the third is their phonetic properties. Assuming $\cal A$ is speaking and not writing, the content of a possible utterance will be determined by all three dimensions.

Since languages can be simple or complex, it should be evident that syntax and phonology are merely a wrinkle, a complicating 
factor -- like friction in Newtonian\ia{Newton, Isaac} mechanics -- in the primary process of communication. The presence of syntax is highly interesting but contingent. It is the meaning (and effect) of utterances that are central, not their structure or phonetic patterns. Yet, linguists have focused primarily on these latter aspects and this is perhaps one reason why the science of communication has languished, including an account of how the \emph{optimal} parse or set of words in an utterance is conveyed.

At a simplified level, $\cal A$ thinks about uttering a sentence such as \Expression{Should we go to Bar Boulud? We haven't been in a while} by mentally trying out how $\cal B$ would interpret it if it were uttered. This interpretation process is part of his decision-making and is a \is{game!partial information}game of partial information.\footnote{These are games that were first developed in my doctoral dissertation \emph{Language and Strategic Inference} (\citeyear{parikh:diss}) at Stanford University in the context of modeling communication and meaning. They have been elaborated a great deal since then.} In other words, $\cal A$, so to speak, lines up alternative utterances, and then imagines the games of partial information each of them leads to, and, based on how $\cal B$ would interpret each utterance, he selects the utterance that will best convey the content he has identified in the Content Selection Game. This complex structure is called the \emph{Generation Game} because in it $\cal A$ generates his best choice of utterance.

This is a simplified description of generation because I have made $\cal A$ consider whole sentences: I have conveniently omitted how the speaker assembles the parts of possible sentences to create whole utterances and also how the speaker decides how to pronounce the sounds corresponding to the words. I will say more about this later but it is a part of the model of communication that remains to be worked out fully.

We can now see fully why the Content Selection Game is not really a \label{page:signaling} signaling game\is{game!signaling} at all. Since telepathy is not possible, $\cal A$ cannot convey his chosen content to $\cal B$ without playing the Generation Game. So the Content Selection Game is \emph{linked} to the Generation Game. And we will see below that the Generation Game is itself linked to another game because it isn't complete either. These linkages, together with the fact that there are no signals in it, make the Content Selection Game very different from a standard signaling game.

The games of partial information that are embedded in the Generation Game are called speaker games \label{page:speaker game} because they are the speaker's model of the linguistic interaction between $\cal A$ and $\cal B$. The speaker considers one set of speaker games for every utterance in the Generation Game but once he has selected his optimal utterance, he utters it, and this induces a corresponding set of partial information games on $\cal B$'s side, called addressee games. These are $\cal B$'s model of their linguistic interaction. $\cal A$ has multiple sets of speaker games to solve whereas $\cal B$ has just one set because $\cal A$ utters just the one best sentence from his choices. She has no idea about the other locutions $\cal A$ turns over in his mind. $\cal B$'s single set of addressee games is best referred to as the \emph{Interpretation Game}. These games, too, have the three dimensions of meaning, structure, and sound.

As the Generation Game and Interpretation Game are $\cal A$'s and $\cal B$'s subjective models of their interaction, there is also a structure that is their \emph{actual} objective interaction and this is also a game. In other words, once $\cal A$ utters a sentence publicly, the objective game comes into being along with $\cal B$'s model of it. For those who might find this third objective structure odd, think first of a single agent and their subjective representation of some part of the world in which they have to act. In this simpler scenario, there is the single agent's mental representation and an objective situation. An agent may look at the dark clouds in the sky and decide to carry an umbrella. When explaining their behavior and its outcome, a theorist needs access to both the actual state of the sky and the agent's mental representation of it. When communication is considered, there are two agents so there are two subjective representations along with the objective situation. In theorizing about communication, we need access to all three structures to understand the most general cases.

The Generation Game and Interpretation Game as well as the third objective game involve \emph{sets} of games because there are games corresponding to each word, each phrase, and the whole sentence on both sides for each sentence considered. So, in the example sentence \Expression{Should we go to Bar Boulud?} there would be as many as five lexical games just for the semantic aspect of the interpretation. There would also be syntactic games and phonetic games. All of this will be dealt with in detail in \partref{part:III}.

When the ideal conditions for communication are met, the speaker games in the Generation Game are identical to the addressee games in the Interpretation Game and both are also identical to the relevant objective games and the interlocutors have common knowledge of all these games. This will also become clearer in \partref{part:III}. After $\cal A$ has completed his utterance and the Interpretation Game and objective game have emerged, it is $\cal B$'s turn to act. She solves the game to find the best interpretation of the utterance. If all goes well, she is now in possession of the content $\cal A$ wanted to communicate to her and she then has to choose her best response to this content in the Content Selection Game.

To sum up, once $\cal A$ has selected the optimal content he wants to convey to $\cal B$, he considers the Generation Game which enables him to convert this content into an equilibrium utterance for him to produce in the utterance situation. Reciprocally, once $\cal A$ utters his optimal sentence, $\cal B$ plays the Interpretation Game and attempts to infer the equilibrium content that $\cal A$ is conveying. Based on this inference, she now selects her best response in the Content Selection Game. And then both players resume the Setting Game.

\subsection{The Communication Game} 

I once again display the process of communication schematically to help fix the sequence:
\begin{enumerate}[itemsep=0pt]
\item[] \underline{Utterance Situation} 
\item[] Setting Game
\item[\functionarrow] $\cal A$'s wish to elicit some response from $\cal B$
\item[\functionarrow] Content Selection Game
\item[\functionarrow] $\cal A$'s equilibrium content
\item[\functionarrow] Generation Game
\item[\functionarrow] $\cal A$'s equilibrium utterance
\item[\functionarrow] Interpretation Game
\item[\functionarrow] $\cal B$'s equilibrium content
\item[\functionarrow] Content Selection Game
\item[\functionarrow] $\cal B$'s equilibrium response
\item[\functionarrow] Back to the Setting Game
\end{enumerate}
Everything starting and ending with the Setting Game occurs inside the utterance situation. It is unfortunate that there are so many different games and so much terminology, but the reader will immediately see that communication does in fact involve all of these steps and will also see the interdependence among these games. Very simply, the interlocutors are trying to decide where to eat and so $\cal A$ chooses to make a suggestion to $\cal B$ to which $\cal B$ chooses a response after which they return to their decision-making. This simple set of acts has the rather complex structure sketched above. In reality, perhaps the whole process happens in a more seamless way, but it is helpful to separate its parts analytically to understand the process better.\largerpage[-1]

Further, even this quite complex picture is a simplification because the clean distinction between $\cal A$'s content and utterance selection embodied in the Content Selection Game and the Generation Game is often an idealization.\label{page:idealization} A poet may care more about a word and its sound than its meaning and want the former to affect the addressee more than the latter and so may not first identify a meaning and then a word but in the reverse order or even together. Or they may struggle to articulate an inchoate thought and realize what their meaning is only through their words. So, in the most general picture, the model would merge some of these parts, especially the Content Selection Game and the Generation Game, possibly by considering ordered pairs of contents and sentences. We will keep them separate, however, for clarity and to simplify the model.

As we have seen, there are four large \emph{interlocking} games:

\begin{enumerate}
\item Setting Game
\item Content Selection Game
\item Generation Game
\item Interpretation Game
\end{enumerate}

I have built upon an innovation in my previous book involving separate but interdependent games. In that book, the linked games considered were just the games of partial information on the speaker's and addressee's sides. Here, the idea has been extended to the whole communicative process. I call the entire structure comprising these four games the \emph{Communication Game} because it really is a single structure with interdependent parts. The Setting Game was described in some detail in \emph{Language and Equilibrium} (\citeyear[Chapter~3]{parikh:le}) so I will treat it summarily here. The other three games will be discussed in Parts~\textrm{III} and \textrm{IV}.

Ever since the founding of modern semantics by Frege in the late nineteenth century, communication has been understood in a one-sided way, focused primarily on the perspective of the speaker. Arguably, the role of communication in semiosis was left implicit at best. It was the ordinary language philosophers such as the later \citet{wittgenstein:pi},\ia{Wittgenstein, Ludwig@Wittgenstein, Ludwig} \citet{austin:pp, austin:htdtww},\ia{Austin, J. L.@Austin, J. L.} and \citet{grice:sitwow} who brought it to the fore but, though they grasped some of the complexities of the two-sided nature of the interaction, they appear to have missed the full structure articulated above. Their followers have seldom grappled with this ampler vision of communication and have confined themselves to elaborations and revisions of the original proposals.

In particular, all of Grice's ideas -- communication as a form of rational activity, the conversational maxims, the analyses of speaker meaning and word meaning, and the distinction between a largely conventional semantics and an inferential pragmatics -- fall short. Part of his goal was to naturalize ``intentionality'' or the aboutness of language by reducing linguistic meaning to speaker meaning and the latter to speaker intentions. Intentions can then presumably themselves be reduced to a digital notion of information as \citet{dretske:kfi} has suggested or to an agent's actions as \citet{stalnaker:inquiry} favors. This led Grice\ia{Grice, Paul@Grice, Paul} to analyze speaker meaning but he missed other key concepts in the process of communication and the concomitant emergence of meaning. I look closely at some of the limitations of Gricean ideas in \chapref{ch:Grice}.

This completes my picture of micro-semantics. It is not difficult to extend it from a single utterance to discourse and dialogue but I will not pursue these natural extensions here.\footnote{See \citet{cp:ga} for a preliminary account of discourse anaphora.}

A key assumption of micro-semantics is that the conventional meanings of words{\interfootnotelinepenalty=10000\footnote{In my framework, only words have conventional meanings. Longer expressions such as phrases and whole sentences do not, although conventional meanings can certainly be assigned to them if required for other reasons. The contents conveyed by utterances of longer expressions can be computed by using just the conventional meanings of words. Of course, idiomatic expressions (e.g.\ \Expression{kicked the bucket}) and such multi-word expressions that are not compositional would be treated differently but I do not consider such cases in this book. There are also sentences like \Expression{Can you pass the salt?} that involve conventional interpretations. I treat such situations as instances of modulation -- see \chapref{ch:modulation}. \label{foot:conventional meanings}}} are fixed and given and are exploited by speakers and addressees in communicating referential meanings, much as Fregean\ia{Frege, Gottlob@Frege, Gottlob} senses are used to convey references,\footnote{See \citet{frege:sr} and \sectref{sec:classic example}.} the former being generalized and refined versions of the latter.\footnote{While this will become fully clear in \partref{part:III}, it might help readers who think of conventional meanings differently to see the following schema: word $\longrightarrow$ conventional meaning $\longrightarrow$\hspace{-1.1em}\raisebox{1ex}{\scriptsize $u$}\hspace{.55em} referential meaning. Only the referential meanings of words enter into the content of the whole sentence uttered in the utterance situation $u$. Conventional meanings are a steppingstone to referential meanings and are more or less shared by the interlocutors.} This makes it possible to give a fairly complete account of communication and meaning in the small. Nothing else needs to be assumed besides the partial rationality of agents,\footnote{I discuss informally what I mean by ``partial rationality'' in \chapref{ch:Grice}. No clear definition of this notion has yet emerged and so I use it to indicate various sets of assumptions that fall short of full rationality. Other similar terms are ``bounded rationality,'' ``resource-limited rationality,'' and ``behavioral rationality.''\is{rationality}} a language and its grammar, and an ontology. But this leaves unfinished the explanation of how semiosis gets off the ground, how language acquires meaning in the first place, the main problem of semantics. Conventional meaning itself must emerge through communication because there is nothing else, no other possible source. To explain this origination, it becomes necessary to shift from a single utterance to society-wide conversations, from micro-semantics to macro-semantics.

\section{Macro-semantics} \label{sec:macro-semantics}

In macro-semantics, the assumption that conventional meaning\is{meaning!conventional|(} is fixed and \emph{externally} given is relaxed and other \emph{cross}-conversational constraints are identified. In other words, it isn't possible to account for semiosis by looking at just one utterance. \emph{Multiple} exchanges are required, either between the same two interlocutors over time or within a whole community. One can abstract from time and assume each member of a population is engaged in multiple conversations with other members synchronically. Each such utterance is described by a Communication Game. But now, instead of using a fixed set of conventional meanings for each word, all we ask is that \emph{whatever} the conventional meanings may turn out to be, they are \emph{consistent} across users. The same agent will not use a word with different conventional meanings every time they speak or interpret. In other words, the constraint of consistency entails that both parties to a conversation use the same conventional meanings not only in the exchange at hand but in \emph{all} their communications. This severely limits the range of conventional meanings a word can acquire and makes it possible to derive not just referential contents via the Communication Game but also conventional meanings for each of the words. I will assume a toy language of one-word sentences as that sidesteps syntax, and assume there is no ambiguity, that is, there is one conventional meaning per word. Such a model is enough to demonstrate the basic idea behind macro-semantics. Working out the mathematical details of such a \emph{meaningful} equilibrium for language presupposes a solution to the micro-semantic problem. An outline of such a model exists in \citet{parikh:note} and I introduce a more complete version in \partref{part:V}. The society-wide collection of interlocking Communication Games, linked via the Consistency Condition on conventional meanings, is called a \emph{Language Game}, a highly suggestive but vague metaphorical idea of \citegen{wittgenstein:pi} that is given a precise definition here. There can be many different kinds of Language Games based on the kinds of networks that are possible in society. This suggests that \citegen{lewis:c} account of convention\is{convention} captures conditions that are sufficient but not necessary. That is, my network model broadens Lewis's more restrictive notion of convention as shown in \chapref{ch:convention}.

%\footnote{See, for example, Boyd et al.~\cite[Section~7]{neal:do}.}

Many generalizations of this picture are possible, not least introducing syntax and allowing ambiguity. These steps are difficult since ambiguity involves not just multiple conventional meanings per word but other complexities such as what is called modulation.\footnote{See \citet{cohen:pa, cohen:ci}. An example of modulation is the use of a phrase like ``the stone lion'' where the content of ``lion'' has to be modulated to be made compatible with ``stone.'' I treat this phenomenon in detail in \chapref{ch:modulation}. \label{foot:modulation}} Another generalization is to note that people seldom share exactly the same conventional meanings so the notion of consistency has to be weakened. This also means the description of what it is to share a language in \sectref{sec:micro-semantics} has to be relaxed. Common knowledge of conventional meanings is too stringent. We can make the meaningful equilibrium dynamic by making utterances sequential rather than simultaneous to study meaning change. Another direction is to \emph{derive} syntax and phonology from communication. These are all theoretical ideas and could be hard to work out, especially in computationally effective ways that enable empirically testable hypotheses to be developed.

The conventional meanings of a single word such as ``dog'' result from countless conversations and countless conversations are affected by these conventional meanings in a diachronic way. Indeed, it isn't clear how stable conventional meanings really are because it is through these very encounters that they also \emph{change}, as becomes evident if one traces the word's evolution from the Old and Middle English ``dogge'' to its modern form and meanings. I address meaning change in \chapref{ch:semantic change}.

Such macro-semantic models allow us to answer a question that is rarely posed in semantics. It is often said that language is a social institution but what this consists in is seldom made explicit. One dimension of its being an institution is the conventional meanings of words as they are shared by the community and make communication possible. This is such a truism that it is seldom asked what purpose this conventionality serves.\footnote{Direct reference theories deriving from \citet{kripke:nn} err when they cut out conventional meanings, viewed as intermediate between words and referential meanings, for certain classes of words. A name N, for example, can be said to have a conventional meaning, the property of being named `N', as \citet{kneale:un} and \citet{burge:rpn} have suggested, with the added proviso that this property together with a resource situation involving a \isi{causal chain} of the kind Kripke has proposed determine the referent. My account retains an intermediate conventional meaning, thus avoiding direct reference, and also avoids Kripke's criticisms, including his noncircularity condition. See \citet[20--21, 68--70]{kripke:nn} and also \sectref{sec:solving Frege's puzzle}. \label{foot:kripke}} \citet[Sections~9.6--9.7]{recanati:lm}, based on his interpretation of \citet{wittgenstein:pi} and \citet{austin:t}, entertains the possibility of eliminating conventional meanings altogether while still retaining an intermediate mechanism of abstraction from past uses of words to derive the referential meanings of present uses. One answer to this ``Meaning Eliminativism'' is that relatively stable, quasi-public conventional meanings facilitate communication by simplifying the choices agents have to make, not to mention the reduction in the burden on memory that results. Likewise, the purpose grammar serves is to reduce effort because there would be many more choices to consider otherwise. Syntax slots words in ways that automatically preclude many possibilities. These are two key elements that make language a social institution. Indeed, the standardization of the various devices themselves, whether words, images, or gestures, also eases communication immeasurably by reducing the possible alternatives. Whether the \emph{existence} of such factors results from our rationality remains an open question but at least it can be so derived mathematically. 

%I will say more about this in \partref{part:V}.

As I have said before, my inspiration for this network model of macro-seman\-tics is a large analogy between the social institution of linguistic communication and a market economy, which is similarly ongoing and circular both with respect to the equation of supply and demand within a single market as well as within the wider interactions among different markets. Comparing the idea of meaningful equilibrium above with the idea of general economic equilibrium,\footnote{See \citet{ad:ecece}.} utterance situations are like markets, words are like commodities, conventional meanings are like the prices of commodities, and referential meanings are like the quantities of commodities bought or sold. Just as prices and quantities emerge from a general economic equilibrium so conventional meanings and referential meanings emerge from a meaningful equilibrium. Language change is roughly similar to the way in which the prices of goods in a market economy change. And just as a price system spares us the trouble of bargaining every time we buy or sell goods, so conventional meanings and grammar and standardization of the communicative devices we use ease our effort.\footnote{See \citet{rubinstein:el} for other interesting but quite different connections between economics and language.} 

A network model of this sort can be deployed at two levels. The first is a more abstract level where the empirical inputs are just assumed as given in order to show that conventional and referential meanings pop out of the model as expected. At this level, one does not attempt to derive the fact that the conventional meaning of \Expression{desk} is, in fact, the property of being \emph{something made for writing}, just that \Expression{desk} acquires \emph{some} conventional meaning. This is the level I will look at in \partref{part:V} as it is the appropriate task for an initial theoretical study. But once such models have been developed in sufficient generality and detail, they can be examined empirically with historical linguists supplying the concrete initial inputs. This is exactly how the work in building models of general economic equilibrium has proceeded, with the early models establishing the theoretical existence of equilibrium prices and quantities and later models applying these early results to particular empirical economies. My purpose in this book is to start off the theoretical task with a relatively simple but revealing model. Later, it can be developed more fully and applied to specific languages. Such work will depend on what empirical data is available and how far back into the mists of time and into the origins of particular languages the data goes.

This completes my sketch of macro-semantics. I believe it goes significantly beyond the explanation envisaged by \citet{grice:umsmwm} and even \citet{lewis:c, lewis:gs, lewis:ll}. Together, micro-semantics and macro-semantics make up my framework for the study of language that I call Equilibrium Semantics or Equilibrium Linguistics because equilibrium\is{equilibrium} plays such a central role in it, not only as an idea and image but also as a practical computational tool. Equilibrium is a deep and widely used notion in the physical and social sciences, and is a part of the fabric of language and communication as well, not a \emph{deus ex machina} imposed from the outside.\is{meaning!conventional|)}

To summarize, communication occurs in the small and in the large. In the small, it is an interaction between two agents where an utterance situation, owing to the concerns and goals of the agents in it, induces one agent, the speaker, to elicit a response from another agent, the addressee, by conveying a content. Based on how the addressee might react, the speaker identifies the content and converts it into an utterance involving a sentence with a sound pattern, structure, and meaning. Upon receiving the utterance, the addressee has to interpret it based on her model of the speaker, using the information the sentence provides as well as information from the utterance situation. Once she interprets it, and identifies the content, she can respond accordingly. In the large, there are multiple conversations in society that are linked to one another by an intricate structure. Both conventional and referential meanings arise through communication. As I have emphasized, communication in the small and in the large, what I call micro-semantics and macro-semantics, consists of \emph{circular} processes at multiple levels, not in a ``vicious'' way, but in a way that makes modeling them quite difficult. It appears to require the special resources of game theory and is not amenable to the kinds of tools philosophers and linguists ordinarily use -- logic and semi-technical English.

Those familiar with the literature in economics on communication and cheap talk,\is{game!cheap talk} especially \citet{cs:sit} and \citet{farrell:ctce, farrell:mcctg}, may now appreciate the key difference in my approach from theirs. First, both sets of authors consider just the micro-semantic level. Second, Crawford and Sobel assume there is no exogenously given meaning and Farrell assumes that  literal meaning is entirely exogenous. These are two extremes and the truth lies somewhere in between. I discuss the former in \sectref{sec:solving generation games} but suffice it to say here that for them meaning is entirely unconstrained and determined endogenously at the level of a single utterance. For Farrell, on the other hand, meaning is fully constrained because it is externally given. The latter begs the question because the problem is to show how meaning is determined. For me, at the level of micro-semantics, meaning is partly exogenous (i.e.\ conventional meanings are held fixed) and partly endogenous (i.e.\ literal and implied meanings are determined internally). At the level of macro-semantics, nothing is fixed, everything is endogenous, but an interplay among multiple utterance situations in society is required. Crawford and Sobel collapse these two levels. And Farrell just assumes the answer. There are other important differences too, mainly in the assumption of costless talk, which I discuss in \partref{part:III}.\\\\

\noindent There are a number of moving parts in my framework. Different readers will undoubtedly find some of these more acceptable than others, given their own theoretical commitments and interests. I urge them to pick and choose, retaining what suits their tastes or reformulating some of the ideas in new ways. I have tried to identify and address a large swath of the problem of communication and meaning. Once the elements of the framework are grasped, it will be easy to tinker with it, replacing or improving one or other part, or even adapting a substantive idea to a different framework. I invite the reader to be a co-creator in just this way.

From the foregoing account, it should be evident that communication is a complex and many-splendored thing. Viewed broadly, it is what enables symbols and language to emerge and, through them, human civilization itself. And yet, as late as the twenty-first century, we do not have a firm grasp of the science of communication.\footnote{\citegen{shannon:mtc} celebrated theory of communication is really a theory of information \emph{transmission}. Transmitting information, say from one hard disk to another through a channel, is very different from communication, although the former occurs within the latter. Incidentally, transmitting information and conveying information are one and the same thing.} Some have even felt such a theory of \emph{performance} is an impossibility. Perhaps this is because understanding communication turns out to require a number of novel ideas.
