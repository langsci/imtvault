\chapter{Grice and conversation} \label{ch:Grice}\largerpage[1]
While the later Wittgenstein\ia{Wittgenstein, Ludwig@Wittgenstein, Ludwig} was the first to bring communication into semantics by tying meaning to use and while Austin\ia{Austin, J. L.@Austin, J. L.} produced some profound insights into meaning and use, it was Grice\footnote{An overview of \citegen{grice:sitwow} writings can be found in \citet{neale:pgpl}.} who pioneered a systematic approach to deriving meaning \emph{from} use. Unsurprisingly, therefore, his work has been enormously influential not only among those who make communication the basis of meaning but even among more traditional researchers still attached to logicism for whom communication remains something of an afterthought.\footnote{I discuss the shortcomings of logicism\is{logicism} in some detail in \emph{Language and Equilibrium} (\citeyear[Chapter~1]{parikh:le}). Key figures include Frege,\ia{Frege, Gottlob@Frege, Gottlob} Russell,\ia{Russell, Bertrand@Russell, Bertrand} and, more recently, Montague.\ia{Montague, Richard} See the references in Chapter~1 of my book.} Indeed, he managed to straddle the two traditions of ideal language and ordinary language philosophy by dividing the subject into two parts, one largely conventional and consonant with logic and the other largely inferential and based on rationality, even though communication ultimately underlay both for him. 

There are four elements in \citegen{grice:sitwow} synthesis of communication and meaning: the idea that rational agency applies to communication as it does to other actions, a theory of conversation based on this idea, a reduction of various semantic notions such as speaker meaning and word meaning that are involved in communication to the intentions of speakers, and a division of the study of meaning into a largely conventional semantic component and a largely inferential pragmatic component, the former preceding the latter in interpretation with both preceded by syntax in turn. 

What was once an extremely fresh set of insights has now been shown to have serious defects, many of which have been pointed out by Grice's followers who may be divided broadly into orthodox Griceans, neo-Griceans, and post-Griceans. In my  dissertation, \emph{Language and Strategic Inference} (\citeyear{parikh:diss}), and in my first book, \emph{The Use of Language} (\citeyear{parikh:ul}), I myself sought to set some aspects of Grice's thought on firmer ground by using game theory and situation theory. But the inner dynamics of this process undermined these very ideas and an entirely new framework has emerged, leading to quite different construals of the four elements listed above. Not least among these is a mathematical rendering of the underlying phenomena.


\section{Communication as rational activity} \label{sec:communication as rational activity}

\citegen{grice:mr} explanations were couched in a philosophical psychology that had two shortcomings. 

First, his adherence to a belief-desire model based on rationality was at such a high level of generality that it appears that he only accounted for the optimality of \emph{single} actions -- essentially against a background of \emph{no} action rather than within the full-fledged context of \emph{choice} (i.e.\ a whole set of explicitly identified alternative actions) as studied in game theory. In a nutshell, he did not translate desires into preferences over situations and, indirectly, over alternative actions. In addition, this fledgling belief-desire model was one-sided rather than two-sided and did not account for both speaker and addressee.

\begin{quote}

This is that for any particular object X and for any feature F and for any activity or type of behavior A, if the creature C believes that the object X both has the feature F and is nearby, or within reach, and that things of type F are suitable for activity A, then the creature wants to A with respect to the object X. In other words, the law harnesses the object to the type of activity. \citep[285]{grice:mr}

\end{quote}

In this passage, only one behavior A is identified. The problem here and elsewhere in Grice's writings is that he never \emph{explicitly} discusses a choice between two or more actions, say A and B, whether these actions are utterances or interpretations. If he had done so, it would have compelled him to consider a more choice-theoretic perspective. He goes on to say:

\begin{quote}

The laws I have mentioned are vulgar laws. The kind of theory in which I think of them as appearing would not be a specialist or formalized psychological theory, if indeed there are such things; I am perhaps not very comfortable with the word ``theory'' being applied to it. It would be the rough kind of system with which we all work, and the laws in it are to be thought of as corrigible, modifiable, and \emph{ceteris paribus} in character. \citep[285]{grice:mr}

\end{quote}

This quote indicates, first, that he was probably not aware of decision and game theory in 1976 and 1980 when this late paper of his first appeared and so could not envisage the possibility that there might be a formalized theory for ``the rough kind of system with which we all work.'' The second is his use of \emph{ceteris paribus} in describing such laws which shows that he did have an \emph{implicit} notion of choice that he never fully realized explicitly. Later in the same essay, he even uses the word ``optimal'' for the first time in his writings but the discussion veers off into other territory and does not address the choice of utterance or interpretation and how such choices might be optimal. Since this paper represents his mature thinking on the topic, we must take it as final. It seems that he was very much on the threshold of choice-theoretic thinking but as choice theory itself burgeoned only in the 1970s, the two passed each other like strangers in the night.\footnote{I had started work on my dissertation at Stanford University in 1985 and had tried to contact him at U.C. Berkeley where he was based, but he was already unwell by then and was racing to complete his volume so he demurred. Sadly, I lost the opportunity to meet him and also to tell him about the possibility of using game theory to study language and meaning.}

Second, Grice introduced the all-important idea of \emph{interaction} between speaker and addressee in a slightly more structured way than \citet{wittgenstein:pi} or \citet{austin:pp} by seeing it through the lens of strategic reasoning. But this interaction and reasoning remained  \emph{epistemic} -- he knows that she knows that he knows and so on -- rather than \emph{practical} -- given what he knows, if he does this, she could do that, and so he should do something else instead. Here is a typical statement of Grice's that purports to account for implicature:

\begin{quote}

He said that P; he could not have done this unless he thought that Q; he knows (and knows that I know that he knows) that I will realise that it is necessary to suppose that Q; he has done nothing to stop me thinking that Q; so he intends me to think, or is at least willing for me to think, that Q. \citep[30--31]{grice:landc}

\end{quote}

\noindent All his statements are similarly phrased where the reference to \emph{alternative} actions and their desirability -- what the speaker could have said but did not -- remains implicit: ``he has done nothing to stop me thinking that Q'' where the main conclusion Q has already been arrived at \emph{without} any explicit consideration of choice -- ``he could not have done this unless he thought that Q.''

Thus, the two dimensions of rationality Grice enlisted -- a one-sided belief-desire model of action and two-sided epistemic strategic reasoning -- were not connected to each other and failed to explicitly compare the two-sided desirability of alternative actions. Moreover, he applied this limited sense of rationality only to the derivation of implicature and left it open how literal meaning was to be determined given its many sources of ambiguity: lexical and structural ambiguity, reference resolution and saturation, modulation, free enrichment, and other such phenomena that he mentioned only in passing if at all.

%The work of orthodox Griceans (e.g.\ Schiffer~\cite{schiffer:m}) and the neo-Griceans (e.g.\ Horn~\cite{horn:sploe, horn:qr}, Levinson~\cite{levinson:pragmatics, levinson:pm}) remains laregly in this vein. Perhaps only some post-Griceans (e.g.\ Sperber and Wilson

Although he never used the term, perhaps his account was \emph{abductive},\is{abduction} that is, it involved an inference to the best explanation, but choice is present even in abduction, and in all the examples he considered, a \emph{choice} among alternative explanations based on the agent's preferences never arose.\footnote{See C. S. \citet[151]{peirce:ai} for the first use of the term ``abduction.'' It is possible William \citet{whewell:pis} originated the idea. Here is a quote from \citet[727--728]{hobbs:anlu}: ``In deduction, from $P$ and $P \Rightarrow Q$, we conclude $Q$. In induction, from $P$ and $Q$, or more likely a number of instances of $P$ and $Q$ together with other considerations, we conclude $P \Rightarrow Q$. Abduction is the third possibility. From an observable $Q$ and a general principle $P \Rightarrow Q$, we conclude that $P$ must be the underlying reason that $Q$ is true. We assume $P$ because it explains $Q$.

Of course, there may be many such possible $P$'s, some contradictory with others, and therefore any method of abduction must include a method for evaluating and choosing among alternatives.'' Indeed, as I will show in \sectref{sec:cl}, the choice-based approach of Equilibrium Semantics is just a species of abduction in which the choice structure is far more sophisticated and detailed than that provided by Hobbs. \label{foot:abduction}} In other words, despite superficial similarities, his framework was never intended to be \emph{choice-theoretic} and his sense of strategic rationality was always epistemic rather than based on a two-sided belief-desire model as in game theory.

%But the first quote above \emph{is} indicative of a watered-down belief-desire framework and so it may just be that Grice conflated this model with an abductive one. A full-fledged belief-desire setup (i.e.\ a game-theoretic framework of the kind I have developed) could be expressed \emph{roughly} in ``abductive'' Gricean language as follows:
%
%\begin{quote}
%
%He has uttered x by which he could have meant either p or q; if he had meant p he would have uttered x but if he had meant q he would have uttered y; so he must have meant p. 
%
%\end{quote}
%
%This reasoning implicitly involves beliefs and desires or probabilities and utilities on the side of both speaker and addressee and is applicable to what is said as well as to implicature, that is, to locutionary and illocutionary meaning.\footnote{Indeed, it captures essentially the equilibrium of a game of partial information. But it does so only \emph{roughly} because a \emph{strategic} inference cannot be reduced to an abductive inference without significantly broadening the latter notion.}


Indeed, perhaps because it was logicians who took up the study of meaning and not social scientists, the almost exclusive emphasis in the entire field has been on the epistemic dimension of agency. Even when the ordinary language philosophers introduced a way of seeing utterances as actions, it was this epistemic level that persisted. Knowledge and belief were simply brought into practical reasoning but without all the accompaniments of practical action such as choices and preferences and utilities \emph{even} in the domain of pragmatics. This is possibly why game theory has been such a latecomer to the field.\footnote{Kant's \emph{Critique of Pure Reason}\ia{Kant, Immanuel} and \emph{Critique of Practical Reason} concern these two dimensions of agency. The trouble arises when one tries to use the epistemic notion to do work that can only be done by the practical notion. The first is generally regarded as falling under metaphysics and epistemology and the second as falling under ethics and political philosophy. Philosophy of language in modern times was pursued primarily by logicians and so has long been practiced as if it were more like the first domain than the second. But it is arguable that even metaphysics and epistemology, let alone philosophy of language, have an inescapable practical dimension. See \citet[Section~7.6]{parikh:le} and \sectref{sec:information} of this book where I argue that \citegen{quine:wo} view of radical indeterminacy fails to take the \emph{optimal actions} of individuating agents into account. To its credit, choice theory -- including both decision theory and game theory -- unifies these two aspects of human agency. See \citet{myerson:gt}.}

As I stressed in my dissertation,\footnote{See \emph{Language and Strategic Inference} (\citeyear{parikh:diss}).} ambiguity applies not just to the lexical and structural aspects of a sentence but much more broadly to the presence of alternative interpretive possibilities as such, and then a framework for choosing among such alternative interpretations arising from various sources becomes indispensable. It is only through certain innovations in game theory that we are able to show how the meaning of an utterance may be computed from first principles assuming only the partial rationality of agents. To the best of my knowledge, no other contemporary approach, whether based on Grice or not, comes even close to doing this, and this is because they all lack the constraints on practical reasoning that the situated game-theoretic framework provides. If one wishes to understand linguistic communication and meaning, the appropriate philosophical psychology to adopt is the variant of game theory I call Equilibrium Semantics.

Going further, if one reflects on how human beings act, it becomes clear that in addition to the epistemic and practical levels of agency, there is a third level that might be called \emph{social agency}. While people are, of course, separate entities enclosed in their bodies, there are ways of acting that may involve a direct connection between the minds and bodies of two or more agents through perception rather than inference. Many neuroscientists believe so-called \emph{mirror neurons}\footnote{See \citet{rs:mb}.} may play an important role in  imitation, empathy, and understanding others. This close connection operates in conjunction with the more individualistic modes of acting involved in practical agency and practical reasoning. Partly, as Hegel\ia{Hegel, G. W. F.} argued, individual persons are \emph{molded} through social interactions in ways we do not as yet fully understand, and partly they operate as separate individuals. Game theory grew out of single-person decision theory and so largely shares its methodological individualism making it difficult to study social agency. Likewise, cognitive science is also largely individualistic in its approach. But there are more holistic traditions in the social sciences, especially in \is{sociology}sociology and anthropology,\is{anthropology} and some of their insights could be incorporated into the more finely articulated and more rigorous forms of choice theory. To a certain degree, I argued for something like this when I said that the preferences of agents in games of partial information are \emph{endogenous} and shared, unlike those in traditional game theory where they are exogenously given and individualistic.\footnote{See \emph{The Use of Language} (\citeyear[28]{parikh:ul}).} Further, this endogenous development is in response to the speaker's possible intentions. This implies that the individual preferences of two or more agents are shaped by common social factors as indicated by social agency.

Notions like \emph{shared situation} are holistic in ways that partly elude us. As we saw in \sectref{sec:agents}, common knowledge involves introducing non-well-founded situations where a situation may contain itself as a constituent. It is exactly this sort of change that allows us to bring less individualistic social action into communication. Endogenous shared preferences can also be developed along similar non-well-founded lines.

\is{agency!epistemic}\is{agency!practical}\is{agency!social}

There are thus three dimensions of agency: epistemic and practical -- which are individualistic -- and social, and this last level has barely been mentioned in studies of communication, although it could play a central role. I reproduce \tabref{tab:SummaryofTheories} from \sectref{sec:snapshot} as \tabref{table:table 2} as it is worth contemplating it in light of this discussion.

\begin{table}
\resizebox{\textwidth}{!}{\begin{tabular}{lccccc}
\lsptoprule
	& Logicism\is{logicism} & Wittgenstein\ia{Wittgenstein, Ludwig@Wittgenstein, Ludwig} & Austin\ia{Austin, J. L.@Austin, J. L.} & Grice\ia{Grice, Paul@Grice, Paul} & Lewis\ia{Lewis, David@Lewis, David}\\\midrule
context					& marginal 	& implicit	& implicit	& implicit	& implicit		\\	
action					& partial		& yes	& yes	& yes	& yes			\\	
%force					& no		& no		& yes	& no		& no		& no		& yes			\\	
epistemic		
interaction				& no		& no		& implicit		& yes	& yes	\\	
practical 				
interaction				& no		& implicit		& no		& no 	& partial			\\	
%choice					& no		& no		& no		& no		& partial			\\	
social					
interaction				& no		& no 	& no		& no		& no	 	\\ 					
computable				& no		& no		& no		& partial		& partial			\\ \lspbottomrule
%structure				& no		& no		& no		& no		& no	       \\ \hline\hline
\end{tabular}}
\caption{Summary of Theories of Communication\label{table:table 2}} 
\end{table}

%\begin{table}[htbp]
%\begin{center}
%\renewcommand{\arraystretch}{1.8}
%\begin{tabular}{||c||c|c|c|c|c||}					\hline \hline
%	& Logicism & Wittgenstein & Austin & Grice & Lewis							\\\hline\hline
%context					& marginal 	& implicit	& implicit	& implicit	& implicit		\\	\hline
%action					& partial		& yes	& yes	& yes	& yes			\\	\hline
%%force					& no		& no		& yes	& no		& no		& no		& yes			\\	\hline
%epistemic		
%interaction				& no		& no		& implicit		& yes	& yes	\\	\hline
%practical 				
%interaction				& no		& implicit		& no		& no 	& partial			\\	\hline
%%choice					& no		& no		& no		& no		& partial			\\	\hline
%social					
%interaction				& no		& no 	& no		& no		& no	 	\\ \hline					
%computable				& no		& no		& no		& partial		& partial			\\ \hline\hline
%%structure				& no		& no		& no		& no		& no	       \\ \hline\hline
%\end{tabular}
%\caption{Summary of Theories of Communication} \label{table:table 2}
%\end{center}
%\end{table}

%The table identifies very roughly how different frameworks deal with the \emph{how} of communication, not the what. That is, it deals with the process rather than the content of communication. Two distinct ``yesses'' in the same row (e.g.\ in the row involving action, a ``yes'' under Wittgenstein and a ``yes'' under Grice) do not imply that both accounts do equal justice to the dimension. After all, such a table can capture only so much. I hope I have not been unfair to any approach.\footnote{I have not included more recent attempts such as the Relevance Theory of \citet{sw:r} which is \emph{post}-Gricean. I will address it in detail in Chapter~3. Suffice it to say here that it would score a ``yes'' in the first three rows and a ``no'' in the last three. There is also a great deal of work in artificial intelligence along logicist lines that would have the same scores as Relevance Theory. \citet{hobbs:anlu} provides a survey of the work on abduction as it has been applied to interpretation. Incidentally, as will become clear, Equilibrium Semantics scores a ``yes'' in all rows except the one pertaining to social interaction, where it scores a ``partial'' rating.}

The key missing idea in all of this literature is that a rational decision-maker should make decisions that maximize his expected utility,\footnote{For an introduction to utility theory, see \citet[Chapter~1]{myerson:gt}. The utility scale is like the temperature scale: the numbers are not unique (think of Fahrenheit and Celsius) and can be scaled by a positive number $k$ and translated by any number $k'$, that is, $v' = kv + k'$ where $k > 0$, while still representing the \emph{same} underlying preferences. If action $a$ is preferred to $a'$ we write $a \succeq a'$. If $v$ is the corresponding utility function then, given the assumptions of rationality, $v(a) \geq v(a')$ is equivalent to $a \succeq a'$. Making choices involves actions that are more preferred and this is equivalent to choosing actions with higher utilities. This allows us to deal with numerical utility functions instead of qualitative preferences, a great convenience that allows expected values of utilities to be computed. Incidentally, the payoffs I have shown in earlier chapters are utilities. \label{foot:utility}\is{utility function}} a view that goes back to \citet{bernoulli:entmr} and to the early days of probability theory, though its modern justification is due to the classic work of \citet{vnm:tgeb}. In the context of game theory, which is two-sided, this modus operandi becomes substantially more complex to describe and requires ideas of equilibrium as we have seen. 

Such a procedure can be used prescriptively or descriptively. The seminal work of \citet{allais:risk}, \citet{simon:bmrc, simon:rcse}, and \citet{kt:pt, kst:juu} has demonstrated the descriptive inadequacy of rational decision-making (understood as the maximization of expected utility) in many situations by establishing that people are just partially rational. The consequences of this behavioral revolution for theories of single-person and interactive decision theory have yet to crystallize\footnote{\citet{rubinstein:mbr} is one attempt to formalize bounded rationality.} despite much experimental research (e.g.\ \citealt{camerer:bgt,vcsb:bcv}) and many popular books (e.g.\ \citealt{ariely:pi}) on the subject. 

Such behavioral considerations will be a leitmotif throughout. The Interpretation Game has the kind of structure that does not lead people to deviate from utility theory. The only place where partial rationality may enter is the Content Selection Game. And this is where alternative theories of choice may play a role. There is also a subtler way in which I have already built partial rationality into the structure of communication. By separating it into different and independent levels -- the four components of the Communication Game described in \sectref{sec:micro-semantics} -- rather than trying to build a single large encompassing game, I have implicitly assumed that people are finite agents and cannot process everything at once. At the level of the brain, however, it is likely that there is one seamless parallel processing structure.


\section{The theory of conversation} \label{sec:theory of conversation}

As is well-known, Grice's framework consists of a general principle called the Cooperative Principle and four lower-level sets of conversational maxims that enable addressees to infer meanings beyond the literal content of an utterance. These are as follows \citep[26--27]{grice:landc}:

\begin{description}[wide=0pt]
\item[The Cooperative Principle:] Make your conversational contribution such as is required, at the stage at which it occurs, by the accepted purpose or direction of the talk exchange in which you are engaged.
\begin{itemize}
 \item The maxim of Quality:
 \begin{enumerate}
  \item \emph{Ceteris paribus}, do not say what you believe to be false.
  \item \emph{Ceteris paribus}, do not say that for which you lack
  adequate evidence.
 \end{enumerate}
\end{itemize}

\begin{itemize}
 \item The maxim of Quantity:
 \begin{enumerate}
  \item \emph{Ceteris paribus},  make your contribution as
  informative as required for the current purposes of the exchange.
  \item \emph{Ceteris paribus}, do not make your contribution more
  informative than is required.
 \end{enumerate}
\end{itemize}

\begin{itemize}
 \item The maxim of Relation:
 \begin{itemize}
  \item \emph{Ceteris paribus}, make your contributions relevant.
 \end{itemize}
\end{itemize}

\begin{itemize}
 \item The maxim of Manner:
 \begin{itemize}
  \item \emph{Ceteris paribus}, be perspicuous.
  \begin{enumerate}[label=\alph*.]
   \item \emph{Ceteris paribus}, avoid obscurity.
   \item \emph{Ceteris paribus}, avoid ambiguity.
   \item \emph{Ceteris paribus}, be brief.
   \item \emph{Ceteris paribus}, be orderly.
  \end{enumerate}
 \end{itemize}
\end{itemize}
\end{description}


Grice's basic idea was that the speaker generally observes the Cooperative Principle and the four maxims when they are communicating. The addressee knows this and so, when she detects an infringement of one of the maxims, she supplies a further meaning, an implicature, to preserve the assumption that the higher-level Cooperative Principle has not been violated. This is an extremely powerful mechanism to explain how implied meanings can be derived rationally. When a speaker does not follow the maxims, he may mislead, opt out of cooperating, be faced with a clash of two or more maxims, or \emph{flout} or blatantly fail to fulfill a maxim. Grice then divides implicatures into three groups: where no maxim is violated, where there is a clash of maxims and so a maxim is not obeyed, and where a maxim is flouted. 

For our discussion, it is convenient to consider just situations where a maxim is violated, either literally or by what might be meant. Otherwise, presumably, the maxims play no role in the calculation and it isn't clear how the framework provides a way to obtain such implicatures. The actual example Grice gives to illustrate the case where no maxim is violated is where a stranded motorist beside an immobilized car is told by a passerby that there is a garage around the corner. Grice's own gloss here is that the speaker \emph{would} be infringing the maxim of relevance unless he thinks the garage might be open. This potential violation of the maxim of relevance is not effected by the literal content but by what might be meant by the utterance. This potential violation serves to rule out the possible meaning that the garage may be closed and enables the addressee to opt for the related implicature that the garage may be open. Perhaps one can say that some maxim is violated in all implicatures, either literally or by what might be meant, that is, either actually or potentially, and in certain cases what is meant includes the literal meaning of the utterance and in other cases what is meant excludes the literal meaning.\largerpage[-1]

Consider one of Grice's examples of the flouting of the maxim of Quality:

\begin{quote}

\emph{Irony}: X, with whom A has been on close terms until now, has betrayed a secret of A's to a business rival. A and his audience both know this. A says \emph{X is a fine friend}. (Gloss: It is perfectly obvious to A and his audience that what A has said or has made as if to say is something he does not believe, and the audience knows that A knows that this is obvious to the audience. So, unless A's utterance is entirely pointless, A must be trying to get across some other proposition than the one he purports to be putting forward.) \citep[34]{grice:landc}

\end{quote}\largerpage[-1]

\noindent This example makes clear that certain implicatures arise not because people are generally truthful as required by the maxim of Quality but rather because certain falsehoods are \emph{obvious} to the audience. It is the \emph{manifest} nature of the falsehood that generates the implicature, not its mere presence. That is, its falseness is common knowledge. Why should an overt lie matter in this way? Not because people obey the maxim of Quality which enjoins sticking to the truth but because when they knowingly utter a falsehood they generally try to conceal this fact. In the example, one would expect some stress on the word ``fine'' or an accompanying gesture or facial expression to make the falsehood more obvious. This additional information reveals to the audience that the falsehood is meant to be detected and is therefore not to be taken literally. A and his audience seem to be engaged in a cooperative conversation. If, instead, their own exchange were adversarial and the rest of the setting were identical (and there was no stress or gesture), the audience might merely think A was trying to mislead or had forgiven X; in any case, there would be no implicature and no irony.  

Going back to the immobilized car, the lack of relevance of a possibly closed garage  is common knowledge between the speaker and addressee and that is why the implicature follows. The speaker does not have to do anything special like stressing a word as in the case of irony above; the implicature flows from background knowledge every adult member of contemporary society has. But it is again common knowledge of the lack of relevance of what might be meant -- and not lack of relevance per se -- that leads to an implicature.

The same analysis applies to the other two maxims, the maxim of Quantity and the maxim of Manner. A person may not convey all that he needs to if the speaker and addressee are negotiating or may convey too much if he is trying to be helpful. In either case, he would violate the maxim of Quantity. It is only when such a violation is common knowledge that an implicature may be generated. As mentioned in \sectref{sec:micro-semantics}, if $\cal A$ wants to influence $\cal B$'s choice of a restaurant, he may have to choose between conveying a content like \emph{should we go to Bar Boulud?} or \emph{should we go to Bar Boulud? we haven't been there in a while}. $\cal A$ may pick the latter as it would give a reason to $\cal B$ to accept his suggestion but it may also be unduly prolix and not the least-cost utterance and so violate the maxim of Manner. In such cases, too, it is only when it becomes common knowledge that an utterance is not perspicuous that it triggers an implicature.

Thus, the Gricean maxims will not do. People simply do not speak the way the maxims dictate because they lie or withhold information or do somewhat irrelevant things or fail to be perspicuous more often than not, all without setting off implicatures. A key reason the maxims don't work is the presence of conflict.\footnote{As discussed in \sectref{sec:agents}, all that is required for an interaction to be conflictual is that the interlocutors value the same situations differently. In game-theoretic terms, their payoffs would be misaligned. $\cal A$ and $\cal B$ may both want to eat out but may have somewhat dissimilar preferences about restaurants as shown in Figure~\ref{fig:battle of the sexes in normal form}. This is such a common occurrence that conflict in this sense is more or less ubiquitous. On the other hand, pure conflict is rare and some cooperation is also typically present because the payoffs are seldom completely misaligned. This is why most games are \emph{mixed-motive} games rather than \emph{coordination} games involving pure cooperation or \emph{zero-sum} games involving pure conflict, the two extremes of a continuum.} In addition, utterances affect addressees in positive or negative ways and the desire to produce such effects constrains speakers in ways that lead to routine violations of the maxims that also do not result in implicatures. These observations become especially clear when we consider the Content Selection Game, which is generally partly cooperative and partly conflictual. 

Once one admits the influencing of relationships in order to maintain or alter them as part of the story of communication and, equally importantly, also includes the basics of human psychology and motivation, many large-scale phenomena like indirect speech, politeness, and just plain but ubiquitous \emph{framing} -- something politicians and used car salesmen know a great deal about -- become possible to explain via the Communication Game. Indeed, \emph{all} communication involves framing. $\cal A$ may choose to give $\cal B$ a reason why they should go to Bar Boulud because he knows enough human psychology to know that only some content of that kind would be effective.\footnote{See \citet{cialdini:i}, for example.} And so he might choose a costlier utterance than the cheapest possibility.

The Content Selection Game and the Generation Game are precisely the sites where a speaker chooses the best way to frame his utterance given the situation he is in. This explains a curious paradox. As a rule, we do \emph{not} choose the cheapest utterances overall, as I also observed in my first book.\footnote{See \emph{The Use of Language} (\citeyear[Section~8.3]{parikh:ul}).} But once a \emph{way} to realize the speaker's goal has been identified, that is, once a \emph{framing} has been identified, then the speaker \emph{does} try to choose the cheapest utterance that corresponds to the framing. That is, the contents we select are not usually the cheapest but once they have been identified, we do normally choose the cheapest way to convey them. Of course, cost includes not just the effort of producing a sentence but also includes the kinds of things \cite{pnl:lis} mention (e.g.\ the avoidance of awkwardness in a relationship) based on the Politeness Theory of \citet{bl:p}.

If one followed Grice literally, it would seem that speakers always choose the cheapest sentence (unless they want to implicate something via the maxim of Manner). But the foregoing shows why we often do not utter the cheapest sentences as such; we do so only relative to the contents we have chosen to convey. \emph{This} is why language in ordinary human interactions and in literature is far more complex and colorful than we would expect based on straightforward cost-minimizing of the kind Grice envisaged. It is the Communication Game and its multi-level structure that can give a clear account of such large-scale phenomena, as we will see. It makes possible a kind of \emph{thick description}\footnote{See Gilbert Ryle's 1971 university lectures and especially Clifford \citet{geertz:ic}. \label{foot:geertz}\ia{Ryle, Gilbert}} of communication. 

In fact, the Cooperative Principle itself appears ambiguous. One way to interpret it is to say that people generally cooperate when they communicate. But this would render it completely implausible owing to the pervasive existence of conflict. In clarifying it, \citet[29]{grice:landc} says the following: 

\begin{quote}
The participants have some common immediate aim, like getting a car\linebreak mended; their ultimate aims may, of course, be independent and even in conflict -- each may want to get the car mended in order to drive off, leaving the other stranded. In characteristic talk exchanges, there is a common aim even if, as in an over-the-wall chat, it is a second-order one, namely, that each party should, for the time being, identify himself with the transitory conversational interests of the other.
\end{quote}

There are as many as \emph{three} levels at which the Cooperative Principle may be interpreted. As Grice makes clear above, there are so-called \emph{ultimate} aims that could be cooperative or conflictual. This level is \emph{not} germane for the Cooperative Principle. But there are also \emph{immediate} aims like getting a car mended. As Grice lacked the model of a Communication Game with its distinction between various games, he could not say that the immediate aims or ``conversational interests''  are the goals the agents have in the Setting Game. 

As mentioned in \sectref{sec:micro-semantics}, in the interaction between $\cal A$ and $\cal B$ involving the choice of a restaurant, there is a shared goal of eating out together in the Setting Game. This is the same as the conversational goal. Its subgoals, whether to eat at this or that restaurant, may not be shared. Both the goal and its subgoals are reflected in the payoffs of the Setting Game (e.g.\ possibly a \emph{Battle of the Sexes}). If the Setting Game induces a Content Selection Game, then the latter's payoffs reflect the former's payoffs. It is the Content Selection Game through which cooperation and conflict primarily enter the process of communication, as shown later.

Some conflict also occurs in the choice of utterance the speaker faces in the Generation Game, especially in what to make explicit and what to leave implicit. Such conflict can occur because of the requirements of politeness or appropriateness or to avoid error. The speaker may prefer to be more direct or less explicit whereas the addressee may prefer greater deference or greater explicitness to prevent error. But such conflict is related more to the maxims just discussed and less to the Cooperative Principle. I discuss this in \sectref{sec:solving generation games}.

But once a speaker utters something and the addressee tries to interpret it, the Interpretation Game involved is usually one of cooperation because the interlocutors identify with their mutual interests in communicating even if one is insulting the other. This identification is partly what makes their preferences \emph{endogenous} reactions to the speaker's possible intentions, as I said earlier. But even here, as will be seen in Sections~\ref{sec:solving locutionary global games} and \ref{sec:cl}, conflict can occur because certain interpretations may be more or less preferred by the addressee. This is the \emph{third} level at which the Cooperative Principle could be interpreted.

The correct level at which the Cooperative Principle applies is the intermediate level of ``conversational interests'' or ``immediate aims.'' It does not apply at the level of ultimate aims and it does not apply at the level of the Generation and Interpretation Games. At this intermediate level of the Setting and Content Selection Games, however, there can be cooperation or conflict or both. So the Cooperative Principle does not apply universally to all conversations. However, Grice was right in assuming that it nevertheless plays a crucial role in determining illocutionary meanings such as implicature. This is a subtle point that has been missed by many and so I repeat it: even when there is no common immediate aim or shared conversational goal, the Cooperative Principle still influences implicature. Grice's error lay, as I will show in \chapref{ch:implicature}, in identifying \emph{precisely} what implicatures are generated when the Cooperative Principle does not hold.\footnote{I believe my analysis is different from but related to the proposal by \citet{al:lc}, who distinguish Strong Cooperativity (underlying goals coordinated) from Rhetorical Cooperativity (conversational goals coordinated).}

Thus, the Gricean maxims might be rephrased into a single maxim of Communication as follows:

\begin{description}[wide=0pt]
\item[The maxim of Communication:] If you wish to imply something beyond or in place of what you are literally conveying, ensure that it is common knowledge between you and the addressee that you are either actually or potentially conveying a falsehood or not providing the appropriate amount of information or not being relevant or not being perspicuous and ensure that it is common knowledge that you are not opting out of cooperating at a conversational level.
\end{description}

\noindent Implicatures that themselves do not need to become common knowledge can be secured by weaker forms of shared knowledge than common knowledge, as in cases of hinting or suggesting. If a speaker acts in accordance with such a maxim or a weaker variant, then the desired implicature will be calculable from this maxim together with the Cooperative Principle: the addressee will notice that the speaker has uttered a falsehood or is not providing the expected amount of information or is being irrelevant or is not being perspicuous, either actually or potentially, and that this is common knowledge (or something weaker) between them and, given the Cooperative Principle, will be able to infer the implicature.

The requirement of common knowledge is different from Grice's notion of flouting. Flouting is just one way in which common knowledge is realized but common knowledge can also occur in other ways, as indicated by the garage example discussed above. The key thing that makes implicature possible is common knowledge.

The maxim of Communication is still open to the charge often expressed with regard to the Gricean maxims -- that they are approximate and vague and may also be vulnerable to other counterexamples.\footnote{See \citet{horn:sploe, horn:qr} and \citet{levinson:pragmatics, levinson:pm} for example.} \citet[footnote 32]{thomason:ami} says, in fact: 

\begin{quote}

I have the impression that Grice was tentative in the William James Lectures about his account of conversational implicatures, and may not have been entirely satisfied with its ability to generate detailed explanations of a wide range of cases. On the whole, I believe that linguists and computer scientists have taken the details of Grice's theory more seriously than they perhaps should have. It is important to remember that Grice's William James Lectures were never prepared for publication.

\end{quote}

\noindent This is why the informal expression of maxims, whether Grice's or mine, is itself an inadequate approach to the problem. In \partref{part:IV}, I will show how all such informal principles may be replaced by formal notions and how these formal notions play a role not only in the derivation of implicatures but also in computing other aspects of what I call illocutionary meaning, such as modulation, free enrichment, and figures of speech, which belong to the literal meaning of an utterance. In other words, the Gricean maxims are completely superfluous.\footnote{This was also argued in \emph{The Use of Language} (\citeyear[Chapter~7]{parikh:ul}).} As we will see, however, the Cooperative Principle is required.

Grice's theory of conversation also assumed the pipeline theory of meaning\is{meaning!pipeline view of} where the syntax of the sentence uttered is determined first, then its semantics (i.e.\ its literal meaning), and finally its pragmatics (i.e.\ its implicatures). The implicatures were derived from the literal meaning and other facts of the utterance. As I have shown earlier,\footnote{See \emph{Language and Equilibrium} (\citeyear[Section~4.6]{parikh:le}).} literal meaning and implicature can in fact be interdependent as all meanings tend to be. This two-way influence will be discussed in detail in \partref{part:IV}. I will show, in fact, that not only are the various meanings of an utterance interdependent but so are these meanings and the optimal parse of the utterance. In other words, the implicature can depend in part on the optimal parse and vice versa.

To conclude this section, I point out that the Gricean theory of conversation is really a theory of interpretation and not communication. While it is two-sided, it has little to say about the speaker's calculations. It also offers nothing about how literal meaning is to be derived and one is forced to surmise that Grice probably meant this to be addressed by convention. While conventional meaning does play an important role in the determination of literal meaning, the process is ineluctably contextual as I have shown before and will show in even greater detail in this book.


\section{Speaker meaning and word meaning} \label{sec:speaker meaning and word meaning}

As I said toward the end of \sectref{sec:micro-semantics}, Grice wanted to reduce the intentionality of language, its aboutness or acquisition of meaning, to the intentions and beliefs of speakers, with the idea that these mental states could be further reduced to physical facts. This would explain how meaning could be part of the natural order. In his view, the key notion to effect such a reduction was that of speaker meaning.\footnote{He used the term ``utterer's meaning'' as he had in mind the wider scope of what he called nonnatural meaning of which linguistic meaning was one important part.} If that notion could be defined in terms of the speaker's intentions, then word meaning\footnote{In Equilibrium Semantics, there is no need for the notion of sentence meaning so I omit it from the discussion here but Grice did want to reduce both word meaning and sentence meaning to speaker meaning.}  (and, presumably, literal meaning and implicature) could be reduced to what speakers mean.

\citet{grice:m, grice:umi, grice:sitwow} gave many definitions of speaker meaning, each successively designed to meet various counterexamples and counter-def\-i\-ni\-tions proposed mainly by Strawson\ia{Strawson, P. F.@Strawson, P. F.} and Schiffer.\ia{Schiffer, Stephen@Schiffer, Stephen} Consider the following restatement of an early version by \citet[446]{strawson:intcon}:

\begin{quote}
$S$ meant something by (or in) uttering $x$ iff $S$ uttered $x$ intending
\begin{enumerate}
\item that this utterance of $x$ produce a certain response $r$ in a certain audience $A$;
\item that $A$ recognize $S$'s intention (1);
\item that $A$'s recognition of $S$'s intention (1) shall function as at least part of $A$'s reason for $A$'s response $r$.
\end{enumerate}
\end{quote}

\noindent To summarize the complex intention involved in meaning, the term ``M-in\-ten\-tion'' was introduced. \citet[63]{schiffer:m} argued against both the sufficiency and necessity of these conditions and gave his own more adequate definition:

\begin{quote} 

$S$ meant that $p$ by uttering $x$ iff $S$ uttered $x$
intending thereby to realize a certain state of affairs $E$ which is such
that $E$'s obtaining is sufficient for $S$ and a certain audience
$A$ mutually knowing  that $E$ obtains and that $E$ is conclusive
evidence that $S$ uttered $x$ with the primary intention

\begin{enumerate}

\item that there be some $\rho$ such that $S$'s utterance of $x$
causes in $A$ the activated belief that $p$/$\rho(t)$;\footnote{The symbol ``$p$/$\rho(t)$'' is to be read as the belief $p$ for which $S$ intends $A$ to have the truth-supporting reasons $\rho(t)$.}

and intending

\item satisfaction of (1) to be achieved, at least in part, by virtue of $A$'s belief that $x$ is related in a certain way $R$ to the belief that $p$;

\item to realize $E$.

\end{enumerate}

\end{quote}

While Schiffer's definition may be sufficient, it is not necessary for three reasons. First, his kind of M-intention is too complex for people to entertain in ordinary speech. In fact, even Grice's M-intention is too complex. An argument against both is the inability to form such intricate intentions when we rattle off multiple sentences in quick succession. And yet speakers certainly mean things so such M-intentions cannot be required. Second, there are many utterances such as ``It is raining'' or ``Smith weighs 150 pounds'' involving free enrichments of the content where a location is meant but not always explicitly intended so again the conditions are not necessary. By ``explicitly'' I just mean there is no corresponding mental representation. This objection can be fixed by stipulating that the speaker \emph{implicitly} intends contents that he would assent to if asked. Third, when a language exists with conventionally meaningful words there is no need for M-intentions. Both Grice and Schiffer miss the \emph{circularity} of language where communication explains the emergence of conventional meanings and conventional meanings (partly) explain communication. 

In \citet{parikh:cmi, parikh:ul, parikh:le}, I give game-theoretic definitions of speaker meaning and other related concepts, especially communication and information flow. The main idea is that the kind of state of affairs $E$ that Schiffer envisaged is replaced by an appropriate game and this game is shown to occur naturally  so it does not need to be intended. Moreover, because utterances are ambiguous in many ways as I have remarked above, these ambiguities need to be resolved in some way in order for a speaker to mean the right thing. As a result, the key to speaker meaning is the solution of a certain game.

%The added advantage of this approach is that the \emph{process} of communication (i.e.\ the game and its solution) is included in the specification of the content meant or communicated, that is, the how is part of the what, unlike Grice's and Schiffer's definitions where no disambiguating process is spelled out. So, in my view, a more realistic and more complete definition is as follows:

\begin{definition}

$\cal A$ means $p$ by uttering $\varphi$ in $u$ if and only if there is an addressee $\cal B$ such that $\cal A$ intends to convey $p$ to $\cal B$ in $u$ (a proper part of which intention may be implicit) and the solution to the game $G^{\cal A}_u(\varphi)$ between $\cal A$ and $\cal B$ that $\cal A$ considers in uttering $\varphi$ in $u$ and (nonconsciously) believes to be common knowledge between them is $p$.\footnote{The word ``convey'' in the definition merely signifies a transfer of information as from one computer disk to another.  We could replace ``convey'' with ``bring about a transfer of.''

One approach to the definition would include other attitudes to
$p$ than the having of information by $\cal B$.  For example, $\cal A$
might intend to cause an activated belief that $p$ in $\cal B$.  Or he might
intend to get $\cal B$ to do something.  If we adopt this approach, we can think of this definition as focusing on the case of pure information transfer.  Other possibilities are then easy to define.  

Alternatively, transferring information is a prerequisite for having other attitudes to $p$, or doing something, so that one can circumvent having to specify the other
possibilities by simply using ``convey.''  This seems like a better approach
because the response is left open.}

\label{def:means}
\end{definition}

$G^{\cal A}_u(\varphi)$ is $\cal A$'s mental model of part of the interaction with $\cal B$ and belongs to the Generation Game. It was introduced in \sectref{sec:micro-semantics} and will be described in \partref{part:III} so the definition's evaluation has to be deferred. There is only a simple intention to convey $p$, part of which may be implicit. No complex intentions are required. It should also be noted that $\varphi$ can be a linguistic or nonlinguistic action.\footnote{As I will not be dealing with nonlinguistic cases here, see Chapter~6 of \emph{The Use of Language} for further discussion.}

Notice that an \emph{infinite} amount of ``work'' is being done in both Schiffer's and my definition above. This infinity cannot be avoided. Early attempts by Grice and Strawson led to an unending series of explicit intentions. Schiffer introduced iterated common knowledge which substituted an unending epistemic chain for the unending chain of intentions. Both may be sufficient but are unnecessary as I have argued above. Definition~\ref{def:means} renders this explicit infinity harmless by requiring a single simple intention and a shared situation orientation to common knowledge. The explicit infinity becomes implicit -- mere entailments rather than mental representations -- because of the circular nature of this account of common knowledge. 

%as I said in \sectref{sec:agents}.

Just as there are two ways of approaching the iterated type of definition -- either via infinitely iterated intentions or iterated common knowledge -- so there are also two ways of approaching the circular type of definition, not just via common knowledge but also via  circular intentions. Indeed, \citegen{grice:m} very first attempt was precisely along these lines: ``A meant$_{nn}$ something by $x$'' is (roughly) equivalent to ``A intended the utterance of $x$ to produce some effect in an audience by means of the recognition of this intention.'' Here, the speaker's intention contains a reference to itself and so is circular.

\citet{bh:lcsa} and \citet{bach:oci, bach:mc, bach:smi} but also \citet{searle:sa} and \citet{harman:schiffer, harman:cv} have championed such circular intentions, filling out Grice's somewhat incomplete stab at the definition. In my view, none of their formulations satisfy entirely because they lack a clear language to express the 
circularity. \citet[194--195]{barwise:ssafa}  points out that such circular intentions involve circular (or non-well-founded) situations but does not spell out a definition. My account is as follows.

\begin{definition}

$\cal A$ means $p$ by uttering $\varphi$ in $u$ if and only if there is an addressee $\cal B$ and a situation $s \subset u$ such that (a) $s \vDash {\cal A}$ utters $\varphi$ and (b) $s \vDash {\cal A}$ intends to convey $p$ to $\cal B$ by ${\cal B}$'s recognition of $s$.

\label{def:circular intention}

\end{definition}

Definition~\ref{def:circular intention} shows plainly that the relevant intention is circular and entails an infinite chain of intentions just as happened with common knowledge in \sectref{sec:agents}. ``Recognition of $s$'' is just recognition of all the facts in $s$. One of these is precisely the fact in item (b) in the definition and so we get ${\cal A}$ intends to convey $p$ to $\cal B$ by ${\cal B}$'s recognition that ``${\cal A}$ intends to convey $p$ to $\cal B$ by ${\cal B}$'s recognition of $s$'' and so on ad infinitum. This kind of clear formulation is better because it identifies precisely what is involved in adopting a self-referential intention, something that remains obscure in Grice's original formulation and in subsequent ones by Bach and others as well. Just as with the situational account of common knowledge, such an infinite chain of entailments is harmless because it does not involve an agent actually representing these nested intentions mentally. So while \citet{recanati:odci} was right to say that an infinite number of sub-intentions are implied by a reflexive intention, he was wrong to say this poses a problem.

Despite the infinity of entailed intentions being innocuous, such circular intentions are also too complex, certainly compared with the simple intention in Definition~\ref{def:means}. This is especially true when successive rapid utterances are considered. So while the conditions in Definition~\ref{def:circular intention} may be sufficient, they are also not necessary. However, they are simpler than Strawson's and Schiffer's definitions. In other words, both types of iterated approach are worse than the circular intention approach and the latter is worse than the situated common knowledge approach. Indeed, my claim is that Definition~\ref{def:means} also gets things right.

Some authors (e.g.\ \citealt{devitt:mflp})  argue that such definitions, presumably including Grice's and Schiffer's, confuse the causes of a speaker's meaning $p$ with what \emph{constitutes} a speaker's meaning $p$. For Devitt, speaker meaning is constituted by the speaker's ``thought content'' $p$. In light of the above, this thought content is just a necessary condition that is not sufficient. After all, a speaker will have many thought contents and only those satisfying either Definition~\ref{def:means} or some such definition will count as what the speaker meant. 

%It is the necessary and sufficient conditions for a speaker's meaning $p$ that constitute it.


%\noindent This sounds very much like what I am saying except that Devitt restricts it to implicature whereas I am saying it applies to all of speaker meaning because such inferences are involved even with literal meaning or what is said as I show in detail in Chapters~2 and 3 and as my definition incorporates (just as does Grice's and Schiffer's). The probable reason why Devitt restricts this kind of constitution to implicature is that he believes literal meaning or what is said is largely conventional. But, as will become amply clear in the next chapter, this is just false. Literal meaning is as contextual as implicature, something I have also argued in earlier publications, especially in \citet{parikh:le}. This is because ambiguity is pervasive even in what is said and its contextual disambiguation involves processes similar to those involved in implicature and this has to be anticipated by the speaker in exactly the same way as in the case of implicature. Thus, Definition~1 is as constitutive of speaker meaning as any other similar definition such as Grice's or Schiffer's.

A related issue in \citet{devitt:mflp}  is the use of the term ``metaphysics of meaning'' to describe speaker meaning and distinguish it from the ``epistemology of meaning'' to describe the addressee's (inferred) meaning. Speaker meaning and addressee meaning are certainly kept distinct in my framework. While Contextualists such as \citet{recanati:lm} confuse them by discussing the addressee's interpretation in place of the speaker's production, the terms themselves are misleading and best avoided because the speaker has to take account of how the addressee will interpret his utterance and so the metaphysics and epistemology are intertwined and not separable. This interpenetration of the two will become completely obvious in \partref{part:III}.

Speaker meaning is supposed to help with three things: defining word meaning, literal meaning, and implicature. But, in my view, it is too one-sided a notion as it is $\cal A$'s mental representation of part of the interaction with $\cal B$. This makes it possible for $\cal A$ to mean things even when they are not communicated to $\cal B$. The right notion to use, as suggested by \sectref{sec:macro-semantics} on macro-semantics, is communication which is genuinely two-sided as the interaction with the addressee is not just a representation in $\cal A$'s mind. The cross-conversational constraint of consistency on conventional (i.e.\ word) meanings introduced there applies to \emph{all} users, whether speakers or addressees. That is, conventional meanings arise because both speakers and addressees participate in creating them, not just speakers as Grice supposed. Speakers exploit conventional meanings in speaking and addressees exploit them in interpreting this speech.

Just as there is $\cal A$'s subjective game $G^{\cal A}_u(\varphi)$, it was mentioned in \sectref{sec:micro-semantics} that there is $\cal B$'s subjective game $G^{\cal B}_u(\varphi)$ and there is also the actual objective game $G_u(\varphi)$, the latter two of which emerge after the public utterance of $\varphi$ in $u$. Incidentally, it is these distinct games that keep speaker meaning distinct from addressee meaning. Since all these games will be described later, the reader should just think of them for the moment as the agents' subjective representations of their interaction and their actual interaction.

\begin{definition}

$\cal A$ communicates $p$ to $\cal B$ by uttering $\varphi$ in $u$ if and only if $\cal A$ intends (possibly partly implicitly) to convey $p$ to $\cal B$ in $u$, $\cal B$ intends (possibly partly implicitly) to interpret $\cal A$'s utterance of 
$\varphi$ in $u$, and the games $G^{\cal A}_u(\varphi)$, $G^{\cal B}_u(\varphi)$, $G_u(\varphi)$ induced thereby are equal and common knowledge and their solution is $p$.
\label{def:communicates}
\end{definition}

Communication involves an actual transfer of $p$ from $\cal A$ to $\cal B$ rather than just an intended transfer by $\cal A$ as with speaker meaning. It is such actual transfers that result in conventional and referential meanings emerging from the interlocking Communication Games in society that make up a Language Game. So speaker meaning is not the right concept to be chasing after despite all the ink that has been spilled on it; it is communication that is foundational. \citegen{grice:umsmwm} attempts at defining word meaning from speaker meaning were in any case extremely informal and vague. He also failed to realize that the two notions -- conventional and referential meaning (i.e.\ literal meaning and implicature) -- have to be defined together, as we will see in \partref{part:V}.

Despite the fact that communication underlay meaning for Grice, he never made explicit the connection between his theory of conversation and speaker meaning as indicated in Definition~\ref{def:communicates} and as explained in \partref{part:III}. As a result, the speaker and addressee in his account remain solipsistic, locked within their mental states, and their interaction never gets addressed. It is because of this lack of integration of the two sides of communication that orthodox Griceans focus on speaker meaning and revisionist Griceans focus on addressee interpretation and the heart of the problem -- communication -- eludes the grasp of both.

%\citet{neale:pgpl} says Grice meant to connect speaker meaning with interpretation and his followers have taken up just one or the other strand. But Grice himself led to this because he never made *communication* the central concept - he never tried to define it. Speaker meaning for all its involving the addressee is still a one-sided concept. And so is interpretation. So Grice had two one-sided concepts which he never succeeded in putting together. As I say in ul and le communication implies the one-side concepts but not the other way around if the one-sided concepts are defined as Grice wanted to define them. A means p if A believes the solution to g is p. But it should be defined as A means p if the solution to g is p. Then the one-side concepts together imply communication which makes more intuitive sense.


\section{Semantics and pragmatics} \label{sec:semantics and pragmatics}

When \citet{morris:fts} introduced his trichotomy of syntax, semantics, and pragmatics within semiotics, the general science of signs, content, the subject matter of semantics, was thought of as predominantly conventional. For example, \citet{austin:t} assumed that demonstrative \emph{conventions} correlated statements with historic situations. Because content and convention were believed to coincide, it is arguable that the use of ``semantics'' to describe this aboutness of utterances was ambiguous between the two. Later, when \citet{grice:lc} realized they diverged, he chose to identify semantics with the mostly conventional aspect of content, with what is said, rather than with content per se. This latter possibility dropped out of view.

This also meant that implicature, the part of content left over after what is said is assigned to semantics, was relegated to pragmatics. This created a parallel ambiguity in ``pragmatics'' as applying to communication (as Morris had originally envisaged) and as applying to implicature, an ambiguity \citet[Section~4]{devitt:wmps} mentions, although without tracing it back to Morris and Grice. There were thus two shifts: semantics was narrowed from all content to literal meaning and pragmatics was altered from communication to implicature. 

Although Grice was aware that literal meaning required the disambiguation of lexical and structural ambiguities and the fixing of pronoun references, he seemed to ignore the uncomfortable fact that the context would have to play a role in the determination of literal meaning and not just implicature. As a result, because Grice held literal meaning to be primarily conventional, there was an implicit third shift: semantics was tied to what comes from the linguistic representation alone and pragmatics was tied to what comes from the utterance situation.

There were thus two distinct scopes for Gricean semantics: one was literal meaning, the other was conventional meaning, and the two were thought to coincide -- except for the somewhat unusual category of conventional implicature.  Just as Morris had implicitly identified the full content with convention so Grice implicitly identified  literal meaning with convention. And just as the presence of implicature meant that the full content could not be entirely conventional, so \citet{searle:lm}, \citet{cohen:pa, cohen:ci}, and others made it equally clear that the presence of contextual aspects of literal meaning, such as saturation, modulation, and free enrichment, meant that literal meaning could not be entirely conventional. This was in addition to the already known contextual aspects of literal meaning -- lexical and structural disambiguation and fixing the references of pronouns -- that Grice had chosen to disregard.

%More precisely, the former category was thought to be a subset of the latter.

This divergence between literal and conventional meaning has led to deep schisms in the understanding of semantics and pragmatics. Contemporary writers such as \citet{bach:tr, bach:yds}, \citet{levinson:pm}, \citet{recanati:lm}, \citet{bianchi:spd}, \citet{cl:is}, \citet{szabo:svp}, \citet{carston:lcspd},\ia{Carston, Robyn@Carston, Robyn|(} \citet{devitt:wmps}, and others have tried to redraw the line between semantics and pragmatics in a variety of ways based on different motivations to include either no context in semantics (e.g.\ Carston -- linguistically encoded or conventional meaning, and possibly Levinson -- generalized versus particularized conversational implicatures) or restricted bits of context (e.g.\ Bach -- ``narrow'' and ``wide'' context, Cappelen and Lepore -- shared minimal semantic content, Devitt -- a narrow notion of what is said) or all of context (e.g.\ Recanati -- truth-conditional content or a wide notion of what is said or full literal meaning). However, the Gricean distinction between a largely conventional semantics and a contextual/inferential pragmatics surprisingly continues to underpin all these efforts.

\citet{carston:tcci, carston:lcspd} does an able job of showing the inadequacy of several competing views (Levinson, Bach, Cappelen and Lepore) so I will not discuss their details here. However, her own context-independent linguistically encoded meanings of sentences expressed as ``incomplete, gappy meaning structures (semantic schemata or templates, propositional radicals or matrices)'' have their own problems. First, it isn't clear that sentences have conventional meanings because only words do. At least, such sentential conventional meanings are not required for generating and interpreting utterances, the real problem of communication, though it is possible that the kinds of gappy structures she posits might be definable with some difficulty. In any case, Carston and the Relevance Theorists lack a precise theory of communication as I will show in \chapref{ch:relevance}, so such posits remain otiose. This is a problem for all the writers mentioned above because none of them have a worked-out theory of communication that enables one to derive the meanings of utterances from first principles so the vigorous redrawing of lines between semantics and pragmatics becomes a little scholastic and reminiscent of pre-Copernican astronomy. 

Second, these lexical conventional meanings are \emph{intermediate} steps toward referential meanings, the former being generalizations of Fregean senses and the latter of Fregean references, as I said in \sectref{sec:micro-semantics}. So conventional meanings are not contents that utterances are \emph{about}; they are rather exploited by speakers to convey referential meanings. That is why conventional meanings belong to word \emph{types}; it is only when words are uttered in some situation that they acquire referential meanings based on their conventional meanings. 

Third, as I argued in \sectref{sec:macro-semantics}, the context-independence of conventional meaning, that is, of linguistically encoded meaning, is just a convenient assumption in what I call micro-semantics; the real theater of conversation is macro-semantic and, in this society-wide setting, conventional meanings themselves keep changing and getting reset. While speakers and addressees do generally have access to some initial lexical conventional meanings during communication, the final conventional meaning at the end of the process either remains what it was or gets altered, as will be shown in \partref{part:V}. 

%This does not affect Carston's view directly but it needs to be noted.

On account of the above, especially the first and second points, it is far from clear what purpose is served by calling this complex but relatively small area of lexical conventional meaning within the much larger field of content and communication \emph{semantics}. \citet[Section~4]{carston:lcspd} herself hints at this by saying that semantics could be omitted entirely, leaving just syntax and pragmatics.

Communication involves fundamental indeterminacies of many different\linebreak kinds.\footnote{See \emph{Language and Equilibrium} (\citeyear[Chapter~5]{parikh:le}).} To take just one pertinent example, the basic distinction between literal meaning and implicature is itself indeterminate. In the situation of the stranded motorist referred to earlier where the speaker tells the addressee that there is a garage around the corner, there is no simple way to decide whether there is an implicature that the speaker thinks the garage might be open or there is just a modulated literal meaning that there is a possibly open garage around the corner. Some speakers and addressees might see it one way and others might see it differently. There may not even be a fact of the matter because the speaker may not have had an explicit intention regarding it. Such indeterminacies are pervasive so using, say, the distinction between literal meaning and implicature as the basis for the distinction between semantics and pragmatics seems unwarranted.

The important issue is that this additional information, whether obtained via implicature or via modulation, involves what I call \emph{illocutionary} computations. However, implicature will turn out to be purely illocutionary whereas modulation will have both locutionary and illocutionary dimensions. In general, there are meanings obtained via lexical and structural disambiguation as well as by the fixing of pronoun reference or saturation that involve exclusively locutionary techniques. On the other side, there are meanings such as free enrichment and implicature and direct and indirect illocutionary force that involve exclusively illocutionary techniques. Modulations, the most complex of all, straddle the two. The distinction between locutionary and illocutionary meaning describes \emph{how} the meanings are derived. It crosscuts the other classes of content such as literal meaning which includes everything except for implicature and force. That is, the latter classes are based on the nature of the content, on the \emph{what}, not on how it is computed. These two crosscutting classifications of content will become very clear in \chapref{ch:classifying meaning} by which point I will have shown how all types of content are obtained by the speaker and addressee.

Interestingly, even though it is useful to distinguish between locutionary and illocutionary meaning derivations, these processes are more similar than different. Both involve partial information games and there is an underlying uniformity in communication that Equilibrium Semantics brings out clearly. This common feature is that \emph{all} content computation requires context. Thus, contrary to Grice's distinction between a largely conventional semantics and an inferential pragmatics that undergirds all the more recent approaches, it is better to identify semantics with the problem of inferring the entire content, regardless of what contributes to this content, the linguistic representation or the context. As one reading of Morris's trichotomy suggests, the concern of semantics should be with content per se, a view that has disappeared from sight. 

There are three basic reasons why such an identification of semantics with deriving content is desirable: it allows a uniform view of all representations whether they are linguistic or visual or gestural or mental; it makes the flow of information rather than representation primary; and, as \citet{austin:pu, austin:htdtww} argued, truth conditions parallel felicity conditions, which makes a construal of the semantics-pragmatics distinction that connects the former with semantics and the latter with pragmatics untenable.

%\footnote{From \emph{Language and Equilibrium}~\cite[pages 6--7]{parikh:le}: ``An exclusive focus on language takes hold of the wrong end of the stick and makes grammar primary, and meaning secondary and an afterthought. This leads to a parallel exclusion of context by focusing on `semantic meaning' (meaning derived almost entirely from the linguistic representation) as primary and `pragmatic meaning' (meaning arising from contextual factors) as secondary. Restoring the centrality of information and its flow enables a balanced view of the sources of meaning as such.''}

The reason why it is more appropriate to call everything semantics rather than pragmatics as \citet[Section~4]{carston:lcspd}\ia{Carston, Robyn@Carston, Robyn|)} suggests is that semantics refers to the \emph{content} itself and pragmatics is ambiguous between communicative processes and content. There is, therefore, just phonetics, syntax, and semantics and all three mutually determine optimal phonetic, syntactic, and semantic values in communication in a \emph{circular} way, as I will show. Such an expanded semantics that swallows all of pragmatics, if articulated mathematically as Equilibrium Semantics does, can then be seen as a generalized and contextual model-theoretic account of meaning within some wider notion of a \emph{situated} logic. It also reinforces the broad notion of semantics as the discipline founded by Frege, \ia{Frege, Gottlob@Frege, Gottlob} Russell,\ia{Russell, Bertrand@Russell, Bertrand} Wittgenstein,\ia{Wittgenstein, Ludwig@Wittgenstein, Ludwig} and Austin \ia{Austin, J. L.@Austin, J. L.} that I started with in \sectref{sec:semantics}.

This completes my discussion of Grice and why I believe it is time that his synthesis of communication and meaning is superseded by a fully mathematical framework such as Equilibrium Linguistics that offers a much more comprehensive and correct account of the subject. The full proof of this can only be realized through the rest of the book.


%\subsection{Logicism}

%Logical form: focus on valid inference which is fine as far as it goes; however, it can apply to two things, sentences or propositions and it ought to apply to the latter. That is, propositions have or are logical forms not sentences. Nevertheless, the latter identification was made especially by Russell. See quote in Wikipedia under Analytic philosophy and also under logical form about philosophical logic and logical form.
%
%Emphasis arose on seeing sentences as distorted and misleading forms of propositions or logical forms.
%
%Subject-predicate to function argument but this does not imply Frege's further step in treating quantification as he did. Not intuitive. Infons as function argument as well but quantifiers treated differently preserving subject predicate intuition or Frege's relational intuition.
%
%If you emphasize the use of language in communication then the emphasis should be treated utterances as central as the data rather than some external standard like proposition or logical form. The relationship got inverted.
%
%See Cohen'\citet[234]{cohen:ci} argument against Montague: 
%
%\begin{quote}
%
%``It has to be emphasized, of course, that the details of verbal interaction in natural language are as yet little understood and that very many problems remain as yet unresolved. For example, when one word dominates another, as ``stone'' dominates ``lion'' in (1), or ``geography'' dominates ``drop'' in (2), what ensures that the domination proceeds in one direction rather than the other?\footnote{(1) Four stone lions occupy the corners of Trafalgar Square. (2) Most students here drop geography in their final year (where ``drop'' means ``drop studying'').} Does the less ambiguous dominate over the more ambiguous, or topic over comment, or the relatively abstract over the relatively concrete, or the inanimate over the animate? But whatever be the correct solution of these problems, it seems highly unlikely that we shall obtain any guidance towards finding it from the ideas of Tarski, Montague, etc., about the semantics of artificial languages. Recognition of the difference between insulationist and interactionist conceptions forces us to treat the semantics of natural language as a largely autonomous discipline, rather than as a topic for Davidsonian or Montagueian theory.''

%\end{quote}

%\subsection{Communication and Grammar}
