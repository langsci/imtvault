\chapter{Beyond language} \label{ch:beyond language}\largerpage[-1]

There are broadly two ways to go beyond language. One is toward other symbol systems such as images and gestures. The other is toward actions that also function as utterances. I have dealt with both directions in \emph{The Use of Language} and to a lesser extent in \emph{Language and Equilibrium} but I briefly examine an instance of the second in a little detail to show how Equilibrium Semantics can connect with the wider social sciences.\is{agency|(}

Consider an example from the field of international relations which may be thought of as applied political science and sociology not to mention economics. Imagine a situation where there are two neighboring countries X and Y and X begins massing troops near their common border. This is not at all an uncommon situation and one that can be quite ambiguous.

First, there is the physical action itself which could be an offensive or defensive move in the ongoing relations between X and Y. It could also be a response to internal developments in X (e.g.\ to divert attention from them). Which of these options is actually the case would depend on the context.

But beyond this level, there could also be a second level where the physical action functions as an utterance and conveys a meaning to the government of Y and possibly to other parties such as the citizens of X and Y. This meaning could involve any of the possibly many contentious elements in the relations between X and Y. For example, the massing of the troops could serve as a warning or intimate an intent, depending on the concrete profile of X and Y and their interactions. And it could be highly ambiguous and unclear.

Such a meaning is essentially illocutionary and, indeed, like an implicature because it is calculable, cancelable, indeterminate, nondetachable, and reinforceable. It can be derived by the same methods involving the illocutionary Semantic Constraint consisting of Relevance and Distance and the corresponding Flow Constraint.

There are innumerable such examples where an action's purpose is not restricted to its immediate consequences but also includes a layer of illocutionary meanings. And Equilibrium Semantics applies to all of them. Thus, its relevance for anthropology, sociology, political science, economics and allied fields is not limited to the direct role of language and meaning in the emergence, evolution, and transformation of social institutions but also includes the entire realm of action itself.\is{agency|)}\\\\
\partref{part:V} shows how Language Games and meaningful equilibria may be used to explain the origins of conventional meaning from the ground up and compares my approach with that of others. Once such models are developed in greater detail, they could begin to yield empirical results. An unexpected bonus is a new insight into the nature of convention broadly conceived. A communicative approach to historical linguistics is forged, one that establishes an initial semantic change via modulation that is then propagated step by step to the whole community via a random graph. Lastly, language is just the first rung of a tall ladder that leads to other symbol systems and social action and ultimately to all of civilization. 

















