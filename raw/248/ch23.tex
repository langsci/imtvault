\chapter{Convention} \label{ch:convention}

What justifies calling conventional meanings conventional?

\citet[58]{lewis:c}\ia{Lewis, David@Lewis, David|(} says there are three basic conditions for a regularity in the behavior of members of a population to be a convention: the regularity itself, a system of mutual expectations, and a system of preferences. On top of this, he demands that these three conditions be common knowledge in the population.

There are two undesirable characteristics in this account, one explicit and one implicit, that are best brought out by an example.

First, if a population is reasonably large, no member is likely to know all of it. That is, some people will be strangers to one another. Intuitively, we know that conventions can and do exist in such communities but common knowledge of any regularity will be impossible because the members do not even know of one another's existence in a complete way.

%Imagine that an agent meets a stranger in such a community. How should he greet her? Should he extend his hand for a handshake? Should he do a namaste in traditional Indian style? His choice will depend on the situation and the clues available to him. Based on these facts, he estimates the probability that the stranger is party to a certain practice or convention and proceeds. If he guesses correctly, he will succeed, and if his guess is incorrect, his action will be inappropriate, and he will have to revise it. Likewise, when approaching a stranger on the streets in New York, should one speak in English or in one of the myriad other languages spoken in the city? One just estimates probabilities based on the concrete situation and acts. Similar examples suggest that the requirement of common knowledge may be too strong for a regularity to be a convention. Probabilities suffice and make room for the possibility of error. They also appear to be necessary.

Second, many conventions (e.g.\ a handshake) are instantiated locally (e.g.\ between two agents). Lewis modeled convention with a single large game played by all the members simultaneously. This represents an extreme case and Lewis's implicit use of a single game is a special degenerate case of a network, especially when seen in the context of Language Games that involve a network of games.

%This is similar to conversations being carried out locally and not with the whole population present.

These observations suggest that actual conventions are weaker than Lewis thought. That is, his explicit condition of common knowledge and implicit condition of a single large game are sufficient but not necessary for a regularity to be a convention. His notion thus needs to be generalized. We can amend his definition as follows, keeping his original notation and terminology intact.

\begin{definition}

A regularity $R$ in the behavior of members of a population $P$ when they are agents in a recurrent network of games $S$ is a convention if and only if it is true that, and members' probability estimates of other members belonging to $P$ are sufficiently high that, in any instance of $S$ among members of $P$,

\begin{enumerate}

\item everyone conforms to $R$;

\item everyone expects everyone else to conform to $R$;

\item everyone prefers to conform to $R$ on condition that the others do, since $S$ is a network of games and uniform conformity to $R$ is the preferred equilibrium of $S$ among at least two equilibria.

\end{enumerate}

\end{definition}

There are \emph{two} changes to Lewis's definition: one is the weakening of the requirement of global common knowledge and the other is the replacement of a single large game by a network of games. 

%Something analogous to the Consistency Condition for a meaningful equilibrium would have to be spelled out for the general case.

If we substitute a Language Game $\Gamma$ for $S$ in the definition, we can establish why the meaning that results from a meaningful equilibrium is \emph{conventional} in this generalized sense of convention. In this case, the regularity $R$ is just the pairing of a word with a property. I record this important fact.
\begin{theorem}

The conventional meanings that result from a meaningful equilibrium of a Language Game are in fact conventional. In other words, the name given to this kind of meaning is appropriate.

\end{theorem}

This also implies that other conventions -- like shaking hands to greet someone -- are best analyzed via networks of interconnected games rather than as a single large game as Lewis did. To the extent that norms exist as conventions, this applies to norms as well.

Interestingly, then, our study of the origins of linguistic conventions has led to a broader notion of convention in general and to a revision of Lewis's classic definition.

Sometimes, it appears that we get stuck in suboptimal conventions, that is, in equilibria that are not Pareto-optimal. A notable example is the QWERTY keyboard. In such cases, it often happens that a closer look reveals that the supposed suboptimal convention was, in fact, optimal at a given time in the past, and later, when new technologies arose, other conventions became optimal but inaccessible owing to the high switching costs. In other cases, it may happen that the incremental process of reaching an equilibrium is piecewise optimal but not globally so, resulting in an inferior convention in the end that becomes sticky. This kind of thing is likely to prevail in many historical conventions, such as the overall outcome of an incremental peace process, for example.

The foregoing addresses some of \citegen{burge:kc} criticisms of Lewis's notion. First, I replace Lewis's global common knowledge by probabilities and local (generally nonconscious) common knowledge. Second, I consider the issue of switching via a consideration of costs. Burge is also concerned about an over-zealous reliance on rationality in situations where ``inertia, superstition, and ignorance'' may prevail. This observation may be accommodated by using nonconscious partial rationality and awareness (e.g.\ an agent may know only a fragment of the objective partial information game and so may not realize there are other alternative equilibria).

\citet{gilbert:gtc} makes some interesting points about the details of Lewis's framework but they are not relevant here as my framework is very different.\ia{Lewis, David@Lewis, David|)}








