\chapter{Beyond illocutionary meaning} \label{ch:beyond illocutionary meaning}

An utterance is an extraordinarily complex event. We underestimate and are usually unaware of this complexity because we are able to handle not just one but multiple utterances in quick succession with relative ease. If you tell a layperson about the study of meaning and content, they are generally puzzled by exactly what difficulties the field could possibly pose when a certain level of understanding appears so readily and transparently available. On further thought, they might feel that more intricate texts of the kind found in literature may raise some real questions but then wonder how something as amorphous as content could be explored rigorously, even mathematically.

These are legitimate questions. So far I have been explaining the part of semiosis I have called micro-semantics, something that has been attempted in modern times since the inaugural work of Frege\ia{Frege, Gottlob@Frege, Gottlob} and Russell\ia{Russell, Bertrand@Russell, Bertrand} and in classical times in at least four different traditions.\footnote{See \citet{bhsv:es}.} This puzzle of meaning, its relative ease of partial access despite its immense complexity, should now be at least partly resolved. It is helpful at this stage to step back and survey just what the range of phenomena are that have to be explained. A variety of contents are associated with even a single utterance and what these are often remains in an elusive background with just the particular content of interest becoming the focus of inquiry, whether it is literal meaning or implicature or some other type of meaning.

In this chapter, I try to identify just this scope of the content of an utterance. My \emph{geography} of content is theoretically underpinned by Equilibrium Linguistics. Its key feature is that many types of content can be derived in a more or less uniform manner using the tools of situation theory and situated game theory. This computation typically involves two stages within a larger Communication Game: first generating the \emph{possible} interpretations and then choosing one or more among them. Sometimes, we can avoid carrying out these steps separately and linearly by allowing them to interact in a circular process that is more efficient.

Importantly, there are less acknowledged contents that I address now that require somewhat different methods from the ones I have described so far. They go beyond illocutionary meaning. My goal here is less to describe detailed derivations than to identify as much as possible of the entire field of meaning.

To start with, I have shown that syntactic and phonetic contents are to be treated as legitimate contents of an utterance on par with semantic contents such as locutionary and illocutionary meaning. This insight is often obscured because the processes that determine all three are seldom seen to be analogous and interdependent in the way suggested by Equilibrium Linguistics. So this identification depends very much on the underlying theoretical framework that tackles the three problems in a uniform way.

I have also suggested in the past that higher-order implicatures (and weaker flows of information that do not require common knowledge) may exist. As I say in \citet[Section~7.9]{parikh:ul}:

\begin{quote}
\label{quote:association}
In principle, there is nothing to stop us from considering
higher-order implicatures.  These would arise when a first-order
implicature gives rise to a second-order implicature, the second-order
implicature to a third-order implicature, and so on.  Such higher-order
implicatures are more difficult to derive and so are not commonly found
in ordinary conversation.  They require more explicit calculation and
so are more often found in literary texts like novels and poems, 
where part of the point is not to state things too explicitly, but to
leave them to the imagination of the reader (``show, don't tell'').  This leaves room
for creative readings and misreadings and enriches the text.  A reader has to work harder to derive
higher-order implicatures, and common knowledge at this relatively
rarefied level becomes increasingly more tenuous.  As a result, the reader
has to bring in his own private background of belief and knowledge, and
fill in the text to create a dense world of meanings and associations.  As this
begins to happen, implicature shades off into suggestion, and
we are no longer in the realm of communication and shared understandings, but
rather in the fluid world of imagination and other transformations of
thought.
\end{quote}

When a literary critic says that \emph{Crime and Punishment}\ia{Dostoyevsky, Fyodor@Dostoyevsky, Fyodor} is about the absence of god (or a moral order) in the modern world, that is an example of a higher-order meaning or even possibly a higher-order implicature. While the passage above refers to higher-order implicatures that are communicated, there is no reason to suppose that corresponding higher-order meanings that are not common knowledge cannot arise.\is{meaning!higher-order}

Quite ordinary utterances can also give rise to higher-order meanings. In the example of the mother and child in the previous chapter, it is possible for the child to infer not only that the mother is conveying $\tau$ in the Content Selection Game but also that the situation is $s$ and not $s'$. This tells him that the mother is in a situation where she will not fuss over him. Should this inferred content be seen as a second-order implicature? It is not something the mother intends to convey perhaps but she may be aware that the child can infer it and, in fact, the information becomes common knowledge between them. Alternatively, the content can also be seen as so-called \emph{inverse information}, a point I return to below.

%I describe a few facts about situation theory as they are required for the other types of contents I consider. The world consists of situations linked by constraints. Situations are collections of \emph{infons}, which are tuples of individuals, properties, relations, and possibly other entities.
%
%Recall that a partial order over $\cal I$, the space of infons, is a binary relation $\Rightarrow_{\ell}$ over $\cal I$ which is reflexive, antisymmetric, and transitive; that is, for all $\sigma$, $\tau$, and $\upsilon$ (i.e.\ the Greek letter upsilon) in $\cal I$, we have:
%
%\begin{itemize}
%
%\item Reflexivity: $\sigma \Rightarrow_{\ell} \sigma$
%
%\item Antisymmetry: If $\sigma \Rightarrow_{\ell} \tau$ and $\tau \Rightarrow_{\ell} \sigma$ then $\sigma = \tau$
%
%\item Transitivity: If $\sigma \Rightarrow_{\ell} \tau$ and $\tau \Rightarrow_{\ell} \upsilon$ then $\sigma \Rightarrow_{\ell} \upsilon$
%
%\end{itemize}
%
%A partial order $\Rightarrow_{\ell}$ on $\cal I$ that intuitively captures the relation ``is at least as informative as'' or ``is at least as strong as'' is assumed. Certain infons are naturally more informative or stronger than others. For example, $\soa{P^{\emph{crimson}};\ a} \Rightarrow_{\ell} \soa{P^{\emph{red}};\ a}$ where $a$ is some physical object because anything crimson is also always red. So the first infon is more informative than the second. Likewise, $\soa{P^{\emph{spinster}};\ a} \Rightarrow_{\ell} \soa{P^{\emph{female}};\ a}$ where $a$ now stands for a person. It is also true that $\soa{R;\ a;\ b} \Rightarrow_\ell \soa{R;\ a}$. If $R$ is the relation of eating, then if $a$ is eating $b$, $a$ must be eating. Likewise, $\soa{R;\ a;\ 0} \Rightarrow_\ell \soa{R;\ a;\ b;\ 0}$ because if $a$ is not eating, then $a$ is not eating $b$. In each case, the infon on the left is more informative than the infon on the right. 
%
%Intuitively, it is clear that if we have two items of information, say, that $a$ is red and $b$ is blue, then it is possible to combine these states of affairs in two obvious ways, by conjoining them or by disjoining them. With this in mind, the partially ordered set $({\cal I}, \Rightarrow_\ell)$ is further assumed to be a lattice. 
%
%A lattice is a partially ordered set in which every pair of elements has a unique supremum (the elements' least upper bound, called their join)\footnote{The supremum or least upper bound of a pair of elements, if it exists, is the least element of $\cal I$ that is greater than or equal to each element of the pair.} and a unique infimum (their greatest lower bound, called their meet).\footnote{The infimum or greatest lower bound of a pair of elements, if it exists, is the greatest element of $\cal I$ that is less than or equal to each element of the pair.} Let $\vee$ and $\wedge$ be the induced join and meet operations. If $\tau = \sup\{\sigma,\sigma'\}$, then $\tau = \sigma \vee \sigma'$. Likewise, if $\tau = \inf\{\sigma,\sigma'\}$, then $\tau = \sigma \wedge \sigma'$. 
%
%A lattice is complete if all of its subsets (whether finite or infinite) have both a join and a meet. Intuitively, there does not appear any reason to restrict $\vee$ and $\wedge$ to finite subsets so we also assume the lattice is complete. The supremum of the whole lattice is denoted by $\mathbf{1}$ and the infimum is denoted by $\mathbf{0}$. The former is sometimes expressed as $\soa{}$ to emphasize that it also corresponds to the ``empty'' infon or to ``no information.'' Intuitively, $\mathbf{1}$ will hold in any situation because every situation supports ``no information'' vacuously. $\mathbf{0}$ is the contradictory infon. Intuitively, it will not hold in any situation because no (coherent) situation can support contradictory information.
%
%Since $\cal I$ now has the two binary operations $\vee$ and $\wedge$, we assume each distributes over the other. That is, it is assumed that $\sigma \wedge (\tau \vee \tau') = (\sigma \wedge \tau) \vee (\sigma \wedge \tau')$ and $\sigma \vee (\tau \wedge \tau') = (\sigma \vee \tau) \wedge (\sigma \vee \tau')$.
%
%A valuation on $\cal I$ is a function $v: {\cal I} \functionarrow \R$ such that $v(\sigma) + v(\tau) = v(\sigma \vee \tau) + v(\sigma \wedge \tau)$. A positive valuation is one where $\sigma \Rightarrow_\ell \tau$ implies $v(\sigma) < v(\tau)$. A metric lattice is a lattice with a positive valuation and the corresponding metric is given by:
%
%\[ \delta(\sigma, \tau) = v(\sigma \vee \tau) - v(\sigma \wedge \tau) \]
%
%\noindent Valuations and therefore metrics always exist on distributive lattices and so one can define a metric $\delta$ on $\cal I$. Our interest is in situated metrics on $\cal I$, that is, metrics that depend on the utterance situation $u$. In other words, the valuation $v$ the metric would correspond to would be a situated valuation. I will use this kind of metric in \sectref{sec:translation}.
%
%%If we have two infons $\tau = \soa{R;\ a;\ b}$ and $\tau' = \soa{R';\ a';\ b'}$, we would ideally want to have the distance $d_u(\tau, \tau')$ to equal $d_u(\soa{R}, \soa{R'}) + d_u(\soa{a}, \soa{a'}) + d_u(\soa{b}, \soa{b'})$, where each of the components are so-called \emph{basic} distances. Estimates of basic distances occur in a variety of ways. Some judgments may be primitive. Others may depend on a relation between, say, $\soa{R}$ and $\soa{R'}$, or $\soa{a}$ and $\soa{a'}$, and so on. Or these elements may share some property. And there may be yet other ways. But these are always situated estimates and they lead to judgments of similarity and dissimilarity which are captured via the metric.
%%
%%It is not clear to me at this time whether a valuation that allows distance to be broken down into basic distances in the way described is always available.
%
%A nonempty subset $\cal F$ of $\cal I$ is called a filter if
%
%\begin{enumerate}
%
%\item $\sigma, \tau \in {\cal F}$ implies $\sigma \wedge \tau \in {\cal F}$,
%
%\item $\sigma \in {\cal F}$, $\tau \in {\cal I}$ and $\sigma \Rightarrow_\ell \tau$ imply $\tau \in {\cal F}$. 
%
%\end{enumerate}
%
%A subset $F$ of $\cal I$ is called an \emph{up-set} if, whenever $\sigma \in F$, $\tau \in {\cal I}$ and $\sigma \Rightarrow_\ell \tau$, we have $\tau \in F$. With this in mind, a filter on $\cal I$ can be more compactly described as a nonempty up-set closed under meet. The set of filters on $\cal I$ is ordered by inclusion.
%
%As I've said, situations are collections of infons. The relation between a situation $s$ and an infon $\sigma$ that holds in it is written $s \vDash \sigma$  or $\sigma \in s$, and is described by saying $s$ supports $\sigma$ or $\sigma$ holds in $s$. As mentioned earlier, the information expressed by the relation $\vDash$ is special and $s \vDash \sigma$ is called a proposition. Only propositions can be true or false or indeterminate, the last in borderline cases involving vague terms or when the relevant infon is partial; infons by themselves do not admit of truth values. Utterances typically convey multiple propositions, although these are usually multiple infons relative to a common described situation.
%
%For all situations $s$ and all infons $\sigma$ and $\tau$, the following facts hold:
%
%\begin{enumerate}
%
%\item $s 
%vDash \mathbf{0}$ and $s \vDash \mathbf{1}$.
%
%\item If $s \vDash \sigma$ and $\sigma \Rightarrow_\ell \tau$ then $s \vDash \tau$.
%
%\item $s \vDash \sigma \wedge \tau$ if and only if $s \vDash \sigma$ and $s \vDash \tau$. 
%
%\item $s \vDash \sigma \vee \tau$ if and only if $s \vDash \sigma$ or $s \vDash \tau$.
%
%\end{enumerate}
%

I now introduce certain types of contents that are perhaps less recognized and are seldom admitted into the favored circle of the objects of semantical inquiry. This may partly be because Grice largely defined the terms of the discussion and for reasons that are hard to surmise the field has stuck to what he charted out as the scope of semantics and pragmatics. These Gricean\ia{Grice, Paul@Grice, Paul} terms have involved a more or less exclusive focus on what is \emph{communicated} via an utterance rather than on the information available to be extracted from an utterance. While this restriction makes sense for Grice's own project of naturalizing meaning, it is not the only aspect of meaning that can and should be studied and the goals of the field need to be broadened. To a certain degree, it is continental philosophy that has looked at such meanings, although with no clear understanding of their scope and limits or of how they arise.


\section{Significance} \label{sec:significance}

I have argued that there is already a fair bit conveyed by an utterance as direct and indirect contents. The following question now arises. If $\sigma$ is a locutionary or illocutionary meaning of $\varphi$ uttered in $u$, and if $\sigma \Rightarrow_{\ell} \sigma'$ as defined in \sectref{sec:information}, should $\sigma'$ also count as part of what is conveyed, whether explicitly or implicitly? If, for example, the content \emph{I see the murderer} ``logically'' implies the content \emph{I see a person}, should the latter also be counted as part of what is conveyed by the utterance? To repeat the quote from \citet[p. 349]{thomason:ami} from \chapref{ch:free enrichment}:

\begin{quote}
All this shows, I think, is that `Did you mean to say' is used to query people about the consequences of what they have said, without distinguishing these clearly from what they said. More generally, `Did you mean to do' does not distinguish between intentions and foreseeable side effects, which is exactly the distinction we need to make here. To achieve some progress in these negative cases, it seems that we will need to say more about the intensionality of intentions in general, and of intentions to assert in particular, and also to find tests that extract more reliable conclusions from the evidence. The philosophical background, of course, suggests that it won't be easy to make progress on these matters.
\end{quote}

As Thomason asks, how can we distinguish between what is conveyed and its consequences? My answer is that it may not always be possible to distinguish between intended and unintended consequences and so, depending on our interests, such consequences and others could be included in what is conveyed. 

%I have already looked at an instance of this with an utterance of \Expression{Smith weighs 150 lbs.} conveying \emph{Smith weighs 150 lbs. on Earth}. In this case both the speaker and addressee would give their assent to including \emph{on Earth} in the content in most situations. With some logical implications such assent may require more thought but the speaker would have to accept that they were part of what was conveyed. 

Just as I raised a question about $\Rightarrow_{\ell}$, if $\sigma$ and $\sigma'$ are both part of the content, should $\sigma \wedge \sigma'$ also be part of the content? The response should again be positive even if an agent does not quite realize this and the meet of the two infons remains implicit.

Based on this, we can say how this more capacious content should be characterized. If ${\cal C}_u(\varphi)$ is the combined direct and indirect content of an utterance, then the (logical) \emph{significance} of the utterance is the smallest filter containing ${\cal C}_u(\varphi)$.\footnote{Recall that filters were defined in \sectref{sec:information}.} Unfortunately, things are not so simple because there are probabilities present in ${\cal C}_u(\varphi)$ and I need to first specify how these should be handled.

We can tentatively define $(\sigma,\pi) \wedge (\sigma',\pi')$ to be $(\sigma \wedge \sigma', \min(\pi,\pi'))$ and stipulate that if $\sigma \Rightarrow_{\ell} \sigma'$ and $\pi' \leq \pi$ then it follows that $(\sigma, \pi) \Rightarrow_{\ell} (\sigma', \pi')$. To simplify things, we could just set $\pi' = \pi$ for the latter. With this understanding, the definition of logical significance is complete.

Recall that if $s \vDash \sigma$ and $\sigma \Rightarrow_\ell \tau$ then $s \vDash \tau$ and also that $s \vDash \sigma \wedge \tau$ if and only if $s \vDash \sigma$ and $s \vDash \tau$. These facts tell us that propositions can naturally be broadened to include significances as well.

Intuitively, it makes sense to include all the consequences of the direct and indirect meanings even though agents may not explicitly realize many of them. In an ongoing conversation, such significances can often be taken for granted as part of the understanding among interlocutors. Significance should therefore also be treated as part of what I have called the \emph{referential} meaning of an utterance in addition to direct and indirect contents. If one refers to a murderer, one has also referred to a person and this property is therefore also part of the meaning even if an agent doesn't quite work it out. All the referential meanings dealt with so far can be thought of as \emph{conveyed} by the speaker to the addressee and some subset of this may even be \emph{communicated} in Grice's special sense. In addition, the intentions attaching to such meanings may be either explicit or implicit because it may not be possible for a speaker to foresee all the logical consequences of his utterance. It can happen that a person makes a claim without grasping one or more of its consequences; if some untenable consequence is pointed out to him, he may want to withdraw his claim and this lends some weight to the intuition that such significance be included in the referential meaning of an utterance.

Also, as we have seen in the derivation of illocutionary meaning, situated human reasoning is much broader than this narrower notion $\Rightarrow_{\ell}$ allows. So the idea of significance can also be broadened to encompass such more flexible inferences (relative to $u$) as might be involved in getting to what \emph{Crime and Punishment}\ia{Dostoyevsky, Fyodor@Dostoyevsky, Fyodor} is about, for example. This kind of content becomes more difficult to readily admit into what is conveyed by the speaker or author as remote consequences of utterances are harder to compute. Some interpretations may be foreseen by speakers, others not. The matter is highly indeterminate. But this is precisely why the characterization of what is conveyed by an utterance is not the only goal semantics should pursue. A broader goal is to characterize all the information extractable from an utterance relative to some reference situation such as $u$. In interpreting fiction, moreover, the context $u$ is often highly indeterminate and this is what gives rise to multiple interpretations of the same text. As long as a valid argument exists for such a conclusion it has to be admitted as legitimate and the criterion for assessing alternatives shifts to whether an interpretation is insightful or interesting.

Consider the pithy significance of the simple but profound statement by the poet Wallace Stevens\ia{Stevens, Wallace} that art is an attempt to see the world again with fresh eyes. Its direct and indirect meanings are quite straightforward to determine but its significance is very rich and different recipients will get it to different extents depending on their own depth of experience.


\section{Associations and extended meaning} 

It is possible to go even further if we allow mental processes that are less constrained to operate on the  referential meanings derived from an utterance. I call such meanings \emph{associations} and they are almost entirely subjective and depend on all kinds of connections that may be triggered by a referential meaning. A lot of creative thought is associative in just this way and associations should also be included in the range of meaning. To be sure, there is no way to reproduce exactly the same associations in the same external circumstances and they are not as ``logical'' as referential meanings but so what? The logical bias of the field has prevented the recognition of associations as legitimate meanings even though they form a very important part of our experience especially when we read some more complex text like a novel or poem. Much of the richness of literature and other art relies on evoking associations in people however much these may differ from person to person. A mere mention of a beautiful sunset can conjure all kinds of mental images.

%as I indicated in the quote from my book on page~\pageref{quote:association} above.

Consider the following examples from \citet[244--245]{fsc:fp}.

\ea
\ea He was pounding the nail when \ldots
\ex He was looking for the nail when \ldots
\z
\z

\noindent The first of these is likely to lead an addressee to infer that an instrument like a hammer was being used. This is less likely with the second utterance. Such inferences are stored together with the direct and indirect meaning of the utterance and may not be easily separable from it. In fact, addressees are frequently mistaken about precisely what they have heard because such associations are mixed in with the meaning.

Can associations be inferred from first principles with the same rigor that Equilibrium Linguistics brings to the derivation of direct and indirect meanings? The key thing to realize here is that each association is a connection between one or more referential meanings and a cluster of other infons in an agent's brain generally based on world knowledge. In other words, all associations are a proper subset of an agent's memory which in turn is a proper subset of the infon space $\cal I$. The second thing to notice is that what gets activated depends to a great extent on how the particular agent's memory is \emph{organized}. If $\sigma$ is a referential meaning and it involves items that are stored near $\tau_i$, $i = 1, 2, 3,\ \ldots\ n$, then some or all of the $\tau_i$ could be triggered. But different agents will inevitably have different arrangements of the same infons and that is why their associations will, in general, be different. Part of this structure of memory, as the study of neural networks has shown, is the relative strengths of the linkages between infons. As science progresses, there is no reason why we should not be able to probe the memory of an agent, form a model of it, and then \emph{predict} its associations when subjected to an utterance. For this task, situation theory should have an important role to play in understanding mental representation and its role in evoking associations.

Another type of expanded meaning is the implications of referential meanings and associations when combined with the knowledge and beliefs of an agent. If $\sigma$ is part of the referential meaning (including significance) and if $\sigma'$ is an associated belief of the agent, then the consequences of putting $\sigma$ and $\sigma'$ together is part of what I am calling the \emph{extended} meaning of the utterance. It, too, is largely subjective although, since many beliefs are shared, two or more agents may arrive at some overlap in their extended meanings. We may wish to restrict the set of beliefs taken into account to just the set of activated beliefs or something a little broader such as beliefs activated upon later reflection as well. Extended meanings are roughly the smallest filter containing all referential meanings including significances, all associations, and all considered beliefs \emph{and} all the associations of this filter, and so on. In other words, it is the closure of both operations of association and extension carried out simultaneously.

% % % % % Here is an extended example of extended meaning, a humorous piece by the comedian Dave \citet[Chapter~4]{barry:dbcgg}.
% % % % % 
% % % % % %To bring these diverse meanings together in a single text, consider this somewhat conventional humor piece by the comedian Dave Barry.\footnote{Barry.}
% % % % % 
% % % % % \begin{quote}
% % % % % 
% % % % % The Difference between Men and Women
% % % % % 
% % % % % Let's say a guy named Roger is attracted to a woman named Elaine. He asks her out to a movie; she accepts; they have a pretty good time. A few nights later he asks her out to dinner, and again they enjoy themselves. They continue to see each other regularly, and after a while neither one of them is seeing anybody else.
% % % % % 
% % % % % And then, one evening when they're driving home, a thought occurs to Elaine, and, without really thinking, she says it aloud: ``Do you realize that, as of tonight, we've been seeing each other for exactly six months?''
% % % % % 
% % % % % And then, there is silence in the car.
% % % % % 
% % % % % To Elaine, it seems like a very loud silence. She thinks to herself: I wonder if it bothers him that I said that. Maybe he's been feeling confined by our relationship; maybe he thinks I'm trying to push him into some kind of obligation that he doesn't want, or isn't sure of.
% % % % % 
% % % % % And Roger is thinking: Gosh. \emph{Six months}.
% % % % % 
% % % % % And Elaine is thinking: But, hey, \emph{I'm} not so sure I want this kind of relationship either. Sometimes I wish \emph{I} had a little more space, so I'd have time to think about whether I really want us to keep going the way we are, moving steadily toward \ldots\ I mean, where \emph{are} we going? Are we just going to keep seeing each other at this level of intimacy? Are we heading toward \emph{marriage}? Toward \emph{children}? Toward a \emph{lifetime} together? Am I ready for that level of commitment? Do I really even \emph{know} this person?
% % % % % 
% % % % % And Roger is thinking: \ldots\ so that means it was \ldots\ let's see \ldots\ \emph{February} when we started going out, which was right after I had the car at the dealer's, which means \ldots\ lemme check the odometer \ldots\ \emph{Whoa}! I am \emph{way} overdue for an oil change here.
% % % % % 
% % % % % And Elaine is thinking: He's upset. I can see it on his face. Maybe I'm reading this completely wrong. Maybe he wants \emph{more} from our relationship, \emph{more} intimacy, \emph{more} commitment; maybe he has sensed---even before \emph{I} sensed it---that I was feeling some reservations. Yes, I bet that's it. That's why he's so reluctant to say anything about his own feelings: He's afraid of being rejected.
% % % % % 
% % % % % And Roger is thinking: And I'm gonna have them look at the transmission again. I don't care \emph{what} those morons say, it's still not shifting right. And they better not try to blame it on the cold weather this time. \emph{What} cold weather? It's 87 degrees out, and this thing is shifting like a \emph{garbage truck}, and I paid those incompetent thieving cretin bastards \emph{six hundred dollars}.
% % % % % 
% % % % % And Elaine is thinking: He's angry. And I don't blame him. I'd be angry, too. I feel so \emph{guilty}, putting him through this, but I can't help the way I feel. I'm just not \emph{sure}.
% % % % % 
% % % % % And Roger is thinking: They'll probably say it's only a ninety-day warranty. That's exactly what they're gonna say, the scumballs.
% % % % % 
% % % % % And Elaine is thinking: Maybe I'm just too idealistic, waiting for a knight to come riding up on his white horse, when I'm sitting right next to a perfectly good person, a person I enjoy being with, a person I truly do care about, a person who seems to truly care about me. A person who is in pain because of my self-centered, schoolgirl romantic fantasy.
% % % % % 
% % % % % And Roger is thinking: Warranty? They want a warranty? \emph{I'll} give them a warranty. I'll take their warranty and stick it right up their \ldots
% % % % % 
% % % % % ``Roger,'' Elaine says aloud.
% % % % % 
% % % % % ``What?'' says Roger, startled.
% % % % % 
% % % % % ``Please don't torture yourself like this,'' she says, her eyes beginning to brim with tears. ``Maybe I should never have \ldots\ oh \emph{God}, I feel so \ldots\ '' (\emph{She breaks down, sobbing.})
% % % % % 
% % % % % ``What?'' says Roger.
% % % % % 
% % % % % ``I'm such a fool,'' Elaine sobs. ``I mean, I know there's no knight. I really know that. It's silly. There's no knight, and there's no horse.''
% % % % % 
% % % % % ``There's no horse?'' says Roger.
% % % % % 
% % % % % ``You think I'm a fool, don't you,'' Elaine says.
% % % % % 
% % % % % ``No!'' says Roger, glad to finally know the correct answer.
% % % % % 
% % % % % ``It's just that \ldots\ it's that I \ldots\ I need some time,'' Elaine says.
% % % % % 
% % % % % (\emph{There is a fifteen-second pause while Roger, thinking as fast as he can, tries to come up with a safe response. Finally he comes up with one that he thinks might work.})
% % % % % 
% % % % % ``Yes,'' he says. 
% % % % % 
% % % % % (\emph{Elaine, deeply moved, touches his hand.})
% % % % % 
% % % % % ``Oh, Roger, do you really feel that way?'' she says.
% % % % % 
% % % % % ``What way?'' says Roger.
% % % % % 
% % % % % ``That way about time,'' says Elaine.
% % % % % 
% % % % % ``Oh,'' says Roger. ``Yes.'' 
% % % % % 
% % % % % (\emph{Elaine turns to face him and gazes deeply into his eyes, causing him to become very nervous about what she might say next, especially if it involves a horse. At last she speaks.})
% % % % % 
% % % % % ``Thank you, Roger,'' she says.
% % % % % 
% % % % % ``Thank \emph{you},'' says Roger.
% % % % % 
% % % % % Then he takes her home, and she lies on her bed, a conflicted, tortured soul, and weeps until dawn, whereas when Roger gets back to his place, he opens a bag of Doritos, turns on the TV, and immediately becomes deeply involved in a rerun of a tennis match between two Czechoslovakians he has never heard of. A tiny voice in the far recesses of his mind tells him that something major was going on back there in the car, but he is pretty sure there is no way he would ever understand \emph{what}, and so he figures it's better if he doesn't think about it.
% % % % % 
% % % % % The next day Elaine will call her closest friend, or perhaps two of them, and they will talk about this situation for six straight hours. In painstaking detail, they will analyze everything she said and everything he said, going over it time and time again, exploring every word, expression, and gesture for nuances of meaning, considering every possible ramification.
% % % % % 
% % % % % They will continue to discuss this subject, off and on, for weeks, maybe months, never reaching any definite conclusions, but never getting bored with it, either.
% % % % % 
% % % % % Meanwhile, Roger, while playing racquetball one day with a mutual friend of his and Elaine's, will pause just before serving, frown, and say: ``Norm, did Elaine ever own a horse?''
% % % % % 
% % % % % And that's the difference between men and women.
% % % % % 
% % % % % \end{quote}
% % % % % 
% % % % % It is instructive to try and tease apart the different kinds of meanings in this piece -- from locutionary to illocutionary, from logical significance to a wider significance, and from associations to extended meanings. It also shows how there is almost never an identity of speaker and addressee meaning at a broader level.

Associations and extended meanings are generally not part of what is intended by an utterance even if they may be intended to be evoked abstractly by an author, especially in fiction. But they are nevertheless part of the content of the \emph{utterance} and this is part of why I started this chapter by saying that even a single utterance is an extraordinarily complex event. So far, I have been trying to show that the meanings of an utterance can cover a potentially wide and even infinite territory. But this should not mislead one into thinking that this kind of meaning is \emph{always} vast. It can occasionally be fairly simple and quite small.


\section{Inverse information}
\citet{bp:sa} introduced the idea of inverse information which is the information you can infer from so-to-speak \emph{external} or, better, \emph{inverse} aspects of an utterance. For example, the accent of a speaker may reveal his Indian origins or a statement may reveal a speaker's location (e.g.\ if he says what time it is). This kind of information can be quite subtle depending on the circumstances and on the actual sentence uttered especially if it is combined with public knowledge or private knowledge. Occasionally, such information may qualify as both part of the (intended) indirect content as well as part of inverse information. This can happen if, for example, a speaker emphasizes his accent or other characteristic in some way available to him in order to actively \emph{convey} something about his background. Certain presuppositions of an utterance may also sometimes belong more to inverse information than to the referential meaning. Since it is well known that the United States has a president then saying something about the president would carry a presupposition that such a person exists but since it is taken for granted anyway by the interlocutors, this information may best be seen as being part of the inverse information of the utterance.

I want to now briefly look at a particular kind of inverse information that Barwise and Perry did not bring to light in the context of the mother and child example. In Figure~\ref{fig:content game} in the previous chapter, the mother elects to convey the content $\tau$ to the child. After she utters an appropriate sentence and he figures out the intended interpretation, the child is able to form a model of the Content Selection Game. He solves it and chooses the action $a$ to stop crying. But he also learns that $s$ is the factual situation and not $s'$. By digging into $s$ he infers that the mother does not wish to fuss as she is busy. This is the reason \emph{why} she is conveying $\tau$. This kind of answer could be called inverse information as it is generally not intended and so is not a second-order implicature. Such inverse information is ubiquitous, in fact, because we often try to infer why an agent did something.

This is a more complete description of how inverse information of certain kinds might arise. Inverse information can be treated as part of the content of an utterance even though it may not actually be communicated or conveyed to the addressee.


\section{Latent meaning}
There is the old joke of the three psychoanalysts stepping into an elevator one morning with the first saying ``Good morning'' to the other two, and the second asking the third, ``What do you think he meant?'' Freud\ia{Freud, Sigmund} and his followers revel in such hidden and repressed meanings and literary and art critics (e.g.\ \citealt{barthes:m,bordwell:mm,eagleton:lt}) have made this kind of meaning their special preserve although Bordwell is ``analytic'' in his approach. These are meanings that get disclosed even \emph{despite} their author. For example, the mother's not wanting to fuss could conceivably also be a latent meaning.

Latent meanings are often derived in the context of a set of extraneous beliefs somewhat akin to what I called associations and extended meanings earlier. Indeed, the two categories may overlap. When critics engage with a text they may bring a theoretical framework to bear on it and this framework would then combine with the narrower contents of the text to allow such meanings to emerge. One reason why Continentals and critics focus on these meanings is because the more mundane referential meanings are relatively straightforward for human beings to discern: it is no feat to ascertain that the mother is asking the child to stop crying because the cut is not serious. Of course, \emph{theoretically}, it \emph{is} an accomplishment to show how such a meaning can be derived from first principles and this has occupied researchers for millennia. But critics and Continental philosophers are not interested in this challenge and often also employ frameworks that eschew referential meanings altogether \citep{eagleton:lt}. On the other hand, semanticists have exhibited their own insularities and have failed to appreciate these richer aspects of meaning. What is required is to marry the rigorous methods of semantics with the broader scope of meaning I have been delineating. Otherwise all one gets is the excesses of Continental philosophy and the utter silence of Anglo-American philosophy on dimensions of meaning that are of central importance.



\section{Discourse meaning}

I have been alluding indirectly to discourse meaning when I talked about the meanings of texts but here is a specific example that goes beyond the meaning of a single utterance. In \emph{Getting Even}, Woody Allen\ia{Allen, Woody} says, ``Can we actually `know' the universe? My God, it's hard enough finding your way around in Chinatown.'' The humor in this works partly because of the contrast between the scale of the universe and the scale of Chinatown and partly because of the bathetic fall from the loftiness of epistemological inquiry to merely getting around. But this contrast and bathos are not directly present in either sentence taken by itself. It is their \emph{juxtaposition} as part of a single discourse that gives rise to such discourse meanings. 

It is possible to develop a rigorous model of discourse meaning based on Equilibrium Linguistics by creating a sequence of Communication Games and linking them appropriately. The challenge would be to understand the kinds of relations (e.g.\ contrast, bathos, etc.) that can occur among the meanings of single utterances in the discourse. These relations would form part of the discourse meaning -- so-called \emph{inter-utterance meaning} -- that transcends single-utterance meaning. 

In addition, just as the content of a word depends on the contents of other words in an utterance, so the content of an utterance depends on the contents of other utterances in a discourse. Perhaps the most commonly investigated area of discourse meaning is anaphoric reference,\footnote{See \citet{cp:ga} and the references therein.} and there are many subtleties even in this seemingly simple problem.



\section{Emotive meaning}

All the different types of meaning discussed above have a propositional form. But there is often a non-propositional emotive content conveyed in an utterance. If a speaker says, ``Man! George W. Bush was elected again!'' then either a positive or negative emotion may be conveyed depending on the utterance situation and especially on the identity of the speaker. \citet{mccready:ee} provides an interesting game-theoretic analysis of how such meanings are communicated. However, the particular formulation he chooses based on signaling games appears too complex to be psychologically plausible. In addition, the nonmonotonic logic he uses to determine whether the emotion is positive or negative makes no mention of conversational goals. 

A simpler way to see emotive meanings is via my account of modulation in \chapref{ch:modulation}. For the word \Expression{Man} in the utterance above, there would be an underspecified conventional core which would be modulated by the illocutionary Semantic Constraint consisting of Distance and possibly Relevance either to a positive or negative content depending on the situation and especially the goals and preferences of the interlocutors. Such an analysis would also connect the intended emotive meaning with the other locutionary meanings of the utterance through the prior probabilities in the partial information games of the Flow Constraint. This kind of connection and mutual reinforcement in equilibrium by the contents of other words in the utterance is also missing in McCready's paper. The degree and tone of the emotion may also be inferred via the relevant Communication Game, partly by taking the Phonetic Constraint into account.

Such meanings can be locutionary or illocutionary and may also be extracted from an utterance without being intended. I discuss the related notion of effect in the next chapter.\\\\

\noindent I have now described phonetic and syntactic contents which might be called \emph{linguistic} meanings, and earlier I described conventional meanings and referential meanings, the latter of which include direct and indirect meanings as well as significances, and finally I have described associations, extended meanings, inverse information, and latent meanings. As just mentioned, there can also be emotive meanings. \emph{All} of these together may be described as comprising \emph{utterance meaning}. Discourse meaning goes beyond single utterance meaning. Some of these meanings are communicated, some merely intended, either explicitly or implicitly, and the rest simply occur privately in the mind of each agent depending on what \emph{other} information is applied to the problem.

I may have missed some elements of utterance meaning but it should be possible to insert such items into the framework I have sketched. The key fact about this entire range of meaning is that it is accompanied by a high degree of indeterminacy which implies that each component is often probabilistic and may differ from speaker to addressee, and it may not be possible to draw sharp boundaries between different varieties of meaning. And yet, as I have tried to argue, a fair amount of order and rigor can be brought to the task. Also, just because there can be so much to a single utterance does not imply that meaning is always vast and complex. For many routine utterances it may be relatively simple.

Gricean\ia{Grice, Paul@Grice, Paul|(} communication is too restrictive a focus of inquiry. The flow of information where one studies what may be intended explicitly or implicitly (e.g.\ \citealt{hirsch:vi, hirsch:ai}; some of the essays in \citealt{iseminger:ii}) is more permissive but even this is too conservative in the end. The field of meaning ought to cover \emph{all} the information extractable from an utterance by an agent. When such a program is carried out, semantics begins to connect with the kinds of meaning studied in much literary and art criticism and the hermeneutic sciences. I believe the sketch above shows how such a link may be forged that may help bring greater rigor to the humanities broadly conceived. It also recovers its connection with the history of semantics as can be seen from \citet{bhsv:es}.

Grice's project of naturalizing meaning is nevertheless necessary to preserve, as without it the entire realm of meaning would remain ungrounded. The hermeneutic sciences miss this element in their account. It is only by combining the two sides, intended flows of meaning and extracted meanings that are unintended, that the split in the study of meaning can be healed and the discipline made whole.

However, as I have argued in this book, Grice's own approach to his project is unfortunately flawed. I see Equilibrium Semantics as a more comprehensive and demonstrably superior approach to meaning that remains grounded in communication and therefore in the Gricean project.\ia{Grice, Paul@Grice, Paul|)}


%\section{Salience}
%
%It is well known that salience plays a major role in enabling us to derive meaning. But it is far from clear how to go about identifying what is and is not salient. Part of what is salient certainly comes from the discourse situation in which our utterances occur. This situation typically contains many items that may be perceptually shared like the courtroom scene in our example and the man on trial. But, crucially, it contains elements that \emph{we} make salient through our prior utterances in an ongoing conversation.
%
%One way to set about \emph{defining} the discourse-based component of salience is simply to identify it with one or other variety of meaning, perhaps all of utterance meaning and discourse meaning up to but not including the current utterance. All these meanings are activated by the prior discourse and all of these can be said to be salient. Depending on how much information is set in motion in this way, the field of salience can be very wide or relatively small. It is possible that not all this content is equally salient but that it naturally comes with the probabilistic weights derived from games of partial information as described above.


%My goal in this section has been to broaden the field of meaning beyond its usual confines within communication and even the flow of information. Where immediately possible, I have tried to give some details about each type of meaning identified and have occasionally described how a more or less complete derivation is possible. But a great deal more remains to be done, both at the level of broad understanding and detailed linguistic analysis.

