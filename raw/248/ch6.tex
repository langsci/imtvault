\chapter{Incorporating elements of the Romantic tradition} \label{ch:romantic tradition}

In my previous book, I tried to combine the strengths of two twentieth-century analytic traditions, so-called \emph{ideal language philosophy} and \emph{ordinary language philosophy}. I identified reference as the key notion contributed by the former and use as the key notion contributed by the latter. To these two notions, I added the new ideas of indeterminacy and equilibrium. Thus, Equilibrium Semantics rests on the four fundamental ideas of reference, use, indeterminacy, and equilibrium.\is{reference}\is{use}\is{indeterminacy}\is{equilibrium} Both the analytic traditions are rooted in the rationalist and empiricist doctrines of Descartes,\ia{Descartes, Ren\'{e}} Hobbes,\ia{Hobbes, Thomas} Locke,\ia{Locke, John} and others issuing from the seventeenth-century revolution in science and later Enlightenment ideas. They leave out ideas developed in the Romantic reaction to these doctrines embodied by Herder,\ia{Herder, J. G.} Humboldt,\ia{von Humboldt, Wilhelm} Hegel,\ia{Hegel, G. W. F.} and especially Heidegger.\ia{Heidegger, Martin@Heidegger, Martin} The two sets of ideas, one identified largely with Anglo-American philosophy and the other largely with Continental thought, although these geographic markers no longer hold, have tended to be opposed to each other and consequently neglected by each other. One reason why analytic and science-based approaches have generally ignored the latter is that their expression is often mired in obscure (and sometimes obscurantist) language and another is that their articulation is sometimes genuinely difficult. A third important factor is that most such accounts tend to be nonnaturalistic and hence opposed to science.

Nevertheless, there are some central insights in this Romantic tradition that are worth assimilating. One way to attempt this is by expanding certain aspects of the notion of use that ordinary language philosophy developed in a partial way. My own notion of use was partly based on game theory which, though far ampler in its understanding of action and choice than ordinary language philosophy and much contemporary semantics, is still somewhat narrowly circumscribed. This broadening is obviously a large undertaking and to accommodate it within a single chapter of this book, I restrict myself to the papers of \citet{taylor:halpp},\ia{Taylor, Charles@Taylor, Charles|(} who is unusually clear in his formulations of these Romantic elements.

In \emph{Language and Equilibrium} (\citeyear[Section~1.4.2]{parikh:le}), I identified the following interconnected aspects of use:

\begin{itemize}
\item Belief, desire, intention, and agency
\item Sentence and utterance
\item The situatedness of language
\item The efficiency of language
\item Ambiguity
\item Communication and information flow
\end{itemize}

Of these six, only the first one, the concept of action, needs some direct alteration; the rest change as a result. I divide my comments on Taylor's papers into two parts.


\section{Human agency} \label{sec:human agency}

Compressing greatly, it is possible to pinpoint the following key aspects of action Taylor identifies:

\is{agency}

\begin{enumerate}

\item Human agents are self-interpreting animals and their feelings in particular (e.g.\ pride, shame, sense of worth, love, etc.) essentially involve interpretations or articulations in language that may be more or less adequate to their objects. Further, these articulations partly constitute their objects. That is, these formulations may simultaneously be right or wrong and also partly constitutive. So they are representations but not of extra-linguistic and independent objects. \citep{taylor:sia, taylor:cp}

\item Some actions involve evaluations (e.g.\ emotional, moral, political, aesthetic, etc.) that cannot be reduced to a calculus of preferences as required by decision and game theory. \citep{taylor:wha}

\item Action is not to be explained in terms of the mental (i.e.\ beliefs, desires, intentions) but the mental is to be explained in terms of action. Understanding does not precede action but is achieved by performing the action and this understanding also transforms the action. \citep{taylor:hpm}

\item Collective action is not reducible to individual actions. \citep{taylor:hpm}


\end{enumerate}

\subsection{Persons as self-interpreting animals}

Taylor contrasts two interpretations of persons, one the broadly Enlightenment view and the other the broadly Romantic view, both updated to our current times. Both start with the ordinary concept of persons as beings with certain capacities: ``a person is an agent who has a sense of self, of his/her own life, who can evaluate it, and make choices about it.'' \citep[103]{taylor:cp}

The first view treats all agents as entities that act on the basis of beliefs and desires. There is no essential difference between animals and complex machines. The ends of agents are taken as unproblematic. What is unusual about persons, therefore, is their ability to conceive and make more complex choices: their \emph{strategic} power. The capacities listed above are seen in the context of this ability to plan. The power to represent with clarity plays a crucial role in executing this strategic power. Computation is the key and the difference between persons and other animals and machines is the relative complexity of the calculations involved.

This is the dominant view in current scientific writing about people. However, all modern semantics, including Grice,\ia{Grice, Paul@Grice, Paul} operates with a much thinner version of even this partial picture. It is only game theory that allows us to address the full range of this strategic power underlying communication and language.

The second view conceives persons as beings for whom things matter in certain special ways, as subjects of \emph{significance}. Things matter for all agents but persons have qualitatively different concerns that are \emph{sui generis}. In other words, the peculiar ends of persons come into focus. Thus, the difference between humans and other animals is not just their greater strategic abilities but also their unusual goals. This means consciousness consists not just in the power to represent but also in the power to constitute our concerns.

What makes human ends qualitatively different is that they involve a sensitivity to certain kinds of situated \emph{standards}.\is{standards} To understand what is meant by ``standard,'' first consider a situation where two persons are preparing food and one says to the other, ``The oven is hot.'' Here, the statement is made relative to some situated standard involving temperature and heat issuing from a situated goal, say that of baking pizza.\footnote{I discussed this example from \citet{travis:mrt} in \citet[Section~5.6]{parikh:le}: the key idea is that the standard or goal relative to which the statement is made arises from the utterance situation and is located in the described situation $c$ of the (Austinian) proposition $c \vDash \sigma$ conveyed by the utterance. More broadly, \citegen{austin:pu} felicity conditions\is{felicity conditions} are likewise standards that belong to the described situation part of the proposition conveyed by an utterance. Sadly, mainstream semantics has largely ignored Austin and so has largely ignored the facts about standards as dealt with by Travis and in my previous book. See \citet{barwise:sftp} for a discussion of Austinian propositions.}

But standards involving heat and temperature are immediately physical and other animals and machines could also be sensitive to such standards (and similar ones like experiencing pain or fear). The kind of standard \citet[54]{taylor:sia} has in mind is \emph{subject-referring}. For example, the emotion of shame involves (failing with respect to) standards which are experience-dependent properties related to our desires and to our emotional life and in particular to a sense of dignity, of worth, of how we are seen by others, all things connected with the life of a subject of experience. Such standards are part of the situation in which shame is experienced and are bound up with a whole gamut of other emotions, goals, and experiences. So shame and its accompanying standards refer us back to the subject or agent and so are subject-referring.

Further, such situated standards are partly constituted by accompanying articulations or interpretations and so, consequently, are the corresponding emotions. There cannot be an experience of shame without some awareness of certain distinctions because shame is felt with respect to falling short of standards like worth and this is related to levels of high and low. These distinctions or articulations, while partly constitutive of shame, are not arbitrary: they can be right or wrong in delineating the feeling. As I said above, they can be more or less adequate to their objects which means that such feelings are, to a certain degree, independent of their articulations even though they are partly constituted by them.

Being a subject of significance is then being an agent that experiences subject-referring standards and emotions (such as shame, pride, remorse, moral goodness, admiration, contempt, aspiration, dignity, worth, love, certain joys and anxieties, and countless others) that necessarily involve ongoing articulations or interpretations. And being a subject of significance is what makes us self-interpreting animals. We are partly constituted by such self-interpretations. This self-interpreting nature is the distinctive mark of our humanity, not our strategic capabilities, which we share to a greater or lesser degree with other animals and machines.

For these self-interpretations to partly constitute our emotional life and, therefore, partly constitute us, a symbolic medium is required, that is, language in the widest sense, which includes words, images, and gestures among other things. In this wide sense, language articulates feelings and makes them clearer and more defined and hence transforms them.  So language partly constitutes us and, as a result, the pre-symbolic hominins must have been much impoverished in their mental lives compared with us.

This second view, in highly condensed form, captures Heidegger's concept of a person according to Taylor. Taylor (and Heidegger)\ia{Heidegger, Martin@Heidegger, Martin} claim that persons and, in particular, their subject-referring emotions that are articulated in and partly constituted by language are irreducible, that is, they cannot be naturalized or reduced to physical things. Further, it is this nonnaturalizability of persons that makes them essentially distinct from other animals and machines.

This is a delicate argument and we can leave this question of nonnaturalizability open, although I believe persons are not sharply different from other animals and machines and are naturalizable. One has to concede, however, that any reduction of the interconnected web of subject-referring emotions and goals and their partly constitutive articulations would have to be quite complex and seems beyond our reach today.\footnote{Think, for example, of \emph{Othello}\ia{Shakespeare, W.}\ia{Othello} and of the evil Iago's sharp awareness of the moral and emotional standards that prevail in his community that he exploits so adroitly.} Analytic philosophy has generally not faced up to the difficulties raised by such Romantic intuitions about the nature of persons. It continues to focus on strategic man (and, in the case of mainstream semantics, on an extremely thin version of strategic man) and has, relatively speaking, ignored these subtler dimensions of symbolic man.

If the argument is right about the partial constitution of subject-referring emotions by language, then the insight of real import for our discussion is that the linguistic formulations of feelings are representations but not of wholly independent objects. It has been a basic assumption of referential semantics since Frege\ia{Frege, Gottlob@Frege, Gottlob} (in fact, even earlier -- Bolzano,\ia{Bolzano, Bernard} for example) that the referents of words are wholly independent of them. This is true of objects like tables and chairs but not of subject-referring emotions. In other words, depending on what they are about, utterances can partly constitute the propositions they convey. Despite this partial constitution, propositions remain abstract entities; that is, they are not linguistic.

But this realization does not affect the fundamental fact that language is nevertheless representational, as \citet[101]{taylor:cp} himself concedes. That is, the key concept of reference\is{reference} still applies to all words, as I have argued in \citet{parikh:le}. In some cases, the infons represented will be independent of the uttered words, and in others, they will be partly constituted by them. This is a relatively small adjustment in the framework though it should be acknowledged that it alters the basic nature of a large number of referents.


\subsection{Irreducible evaluations of choices}

\citet{taylor:wha} argues that some choices involve \emph{strong} evaluations (e.g.\ emotional, moral, political, aesthetic, etc.) that cannot be reduced to a calculus of preferences as required by game theory. This is because the evaluations concern considerations of worth that require contrastive articulations like good/bad, high/low and so entail potential conflicts among different self-interpretations. For example, someone may forgo cake because it would adversely affect their cholesterol. This is an instance of weak evaluation and involves utilitarian reasoning. But they might also forgo it because they do not wish to be the kind of person who responds abjectly to bodily appetites: this implies contrasting self-interpretations such as preserving one's dignity versus succumbing to degradation. In other words, desires do not only count in virtue of the positive consummations that result but also in virtue of the kind of life and kind of person considered worthwhile. Taylor maintains that such strong evaluations cannot be reduced to a calculus of preferences and to what he calls instrumental action.

In my view, there is no problem with accepting that, contrary to Taylor, strong evaluations can be so reduced because the preference relations of game theory abstract from the motivations that underlie them, whether they happen to be strong or weak. The contents of the two kinds of evaluation mentioned in forgoing a cake are suppressed in registering that $\hbox{\emph{forgo cake}} \succ \hbox{\emph{eat cake}}$ because preference is a relation among \emph{situations} or \emph{actions}. At this abstract level, the distinction between the two kinds of evaluation -- avoiding cholesterol or avoiding degradation -- can be ignored. And these preferences over situations or actions are all that is needed for the \emph{calculation} of the best course of action. \citet{grice:mr}, for example, does not draw a distinction between these two kinds of evaluation but he does use the words ``value'' and ``optimal'' and, here, I agree with Grice. It is often not realized by critics of choice theory how thin its demands are with respect to the assumptions it needs to make to draw very powerful conclusions.

But this reduction to preferences and calculation does not thereby imply a reduction to \emph{instrumental} action if the latter means nonmoral and non-self-in\-ter\-pre\-ting action. This is because the choice-theoretic description does not fully exhaust our understanding of action, it only provides a means to predict and \emph{partially} explain it. If we also want to understand its underlying motivations, then it will be necessary to return to the contents that we abstracted from in arriving at their preferences. This is true even if we wish to know whether it was cholesterol or fat that drove them to avoid cake, not just higher-level motivations.

All kinds of evaluation occur in communication, especially in the Content Selection Game described in \sectref{sec:micro-semantics} and later. The game-theoretic framework abstracts from the content of the reasons why $\cal A$ wishes to communicate something to $\cal B$ and focuses on explaining why certain choices are made in light of certain abstracted preferences. In carrying this out, it becomes necessary to explain informally how these preferences follow from the relevant motivations but this informal step does not raise any special difficulties. If required, it can be formalized through the notion of higher-order preferences though this would be unduly pedantic.


\subsection{The expressive dimension of action}

Supported by Hegel's\ia{Hegel, G. W. F.} philosophy of mind and action, \citet{taylor:hpm} contrasts two views of action, the causal view and the qualitative view.\footnote{In the paper, Taylor's goal is to explicate Hegel's philosophy in light of the qualitative view, not to infer the latter from the former.} The first view, rooted in the rationalist/empiricist paradigm, takes the performance of an action to be an event (typically involving some physical movement) caused by an agent's beliefs, desires, and intentions that are further reducible to physical things. This can be recognized as the informal underpinning of the notion of action in game theory as well. 

The second view, originating especially in Vico\ia{Vico, G.} but also in Kant\ia{Kant, Immanuel} and Hegel,\ia{Hegel, G. W. F.} takes the performance of an action to be a primitive thing inseparably connected with some direction or purpose that is either semiconscious or fully conscious and that requires articulation to be better understood. This understanding via articulation is an achievement and it transforms the action itself just as it did subject-referring emotions. 

This coming to fuller understanding through articulation in a symbolic medium is the expressive dimension of action as developed in Romanticism which, according to Taylor, is missing from the first view as the latter takes actions as fully determined by their causes. Understanding is itself an activity that does not take beliefs, desires, and intentions as immediately and transparently given but as objects that have to be clarified and partly constituted in their expression. For example, our desires that issued in some prior action may crystallize as we reflect on them. And becoming clear about an action in this way changes the action itself.

Thus, action is not to be explained in terms of the mental  as the causal view does, rather the mental is to be explained in terms of action as the qualitative view recommends. The mental is not a primitive given but something to be achieved via the primitive capacity for action and its subsequent clarification. Further, because utterances are themselves actions, this reversal applies to utterances as well.

There appear to be two irreconcilable views of action here. In one, actions are caused by beliefs, desires, and intentions, and, if we come to appreciate these causes better upon reflection, the action itself is not transformed, only our understanding of it. In the other, these causes are at best partial, and they are, as it were, completed when we express and clarify them, and that changes the action itself. For example, someone who voted a certain way may have done so on the basis of inchoate intuitions. The first view would say the action remains unchanged even if the voter subsequently thinks about it and clarifies their initially unformed intuitions. The second view would say the action is transformed by this clearer insight.

It seems the first view is right because in everyday life and also in the law we hold people responsible for their actions based on their initial beliefs, desires, and intentions, however inarticulate, and not on their later cogitations. In the case of subject-referring emotions, it is possible for the emotions themselves to change from their rudimentary forms to more fully developed forms, but not in the case of actions. This is because emotions are ongoing and evolving, actions are not. As \ia{Khayyam, Omar} Omar Khayyam's memorable verse says, ``The Moving Finger writes; and, having writ, Moves on: nor all thy Piety nor Wit Shall lure it back to cancel half a Line, Nor all thy Tears wash out a Word of it.''

Besides, Taylor overstates his case in contrasting what he takes to be mutually exclusive views of action. Sometimes (e.g.\ when we turn on the lights in a room), we understand our beliefs, desires, and intentions perfectly well prior to our actions and no further clarification is required. At other times, an agent such as a voter may not understand their action until they mull over it. And there is a whole continuum of cases in between. For the entire spectrum, the action itself is what it is independent of its symbolic expression.

So the reversal mentioned above, explaining the mental in terms of the action, is to be understood in a qualified way. It is certainly true that we sometimes come to more fully understand what we did upon further thought. But this understanding is \emph{not} involved in causing the action in any way. The action has its own determinants, however embryonic. 

%In other words, the qualitative view errs in conflating the prior beliefs, desires, and intentions that cause an action with their posterior articulation.

It needs to be noted that the beliefs, desires, and intentions assumed in explaining action are seldom fully developed. They are likely to be partial and partially understood in a variety of ways.\footnote{Incidentally, for Hegel\ia{Hegel, G. W. F.} complete understanding is possible whereas for Heidegger,\ia{Heidegger, Martin@Heidegger, Martin} who shares the qualitative view of action, understanding is always partial. An implicit part always persists. This latter view of partial understanding is similar to the partiality of situations described in \sectref{sec:information}. They can seldom be exhaustively described; there is often something left over.} This affects game theory as much as it does any other causal theory. But the effect is quite limited because a partial intention must nevertheless be sufficiently formed to bring about an action or utterance.


\subsection{Collective action}

According to \citet{taylor:hpm}, it follows from the qualitative view that collective action cannot be reduced to individual actions because action is primary and the mental is secondary. This part of his argument is not clear because if there are irreducible collective actions, there also has to be irreducible collective understanding that partly constitutes them. But unless one is willing to accept collective minds, this is not possible.

%Then actions unmoored from the mental ranging from private to public become possible to contemplate.

Collective actions are ubiquitous: for example, newspapers daily report the collective actions of various composite associations. Indeed, communication itself can be seen as  a joint action of speaker and addressee like two persons together pushing a cart uphill. And language in the large, the subject of macro-semantics, might also be viewed as being a fully public object potentially irreducible to individual actions.

I think it is true that communication and language involve joint or collective action, something that is seldom acknowledged by mainstream semantics as it is rather remote from taking models of actions and agents as seriously as game theory requires one to do. But there is no reason to think that such action cannot be reduced to individual actions. The model of communication and language developed in this book shows concretely how such a reduction might be carried out in these special cases.

For other types of collective action, all that is required is a suitable generalization of Definition~\ref{def:communicates}.\footnote{See \citet[58--59]{parikh:ul}.} This involves shared situations and common knowledge and their non-well-foundedness, which are less individualistic ways of conceiving holistic facts than a directly individualistic account but, in the end, it nevertheless remains a reduction of such facts to the actions and physical states of individuals.


\section{Language} \label{sec:language}

A good bit of the groundwork has already been laid by the discussion of human agency. We have already encountered some key roles of language adumbrated by Herder, Humboldt, and Heidegger.\ia{Herder, J. G.}\ia{von Humboldt, Wilhelm}\ia{Heidegger, Martin@Heidegger, Martin} Part of \citeauthor{taylor:lhn}'s (\citeyear{taylor:lhn, taylor:tm}) goal is to trace the genesis of two contrasting modern approaches to meaning. I will set aside this interesting historical aspect. I also pay less attention to elements where both perspectives are more or less in agreement (e.g.\ the importance of \emph{use} as a key idea for any theory of meaning).

For Taylor, language has two dimensions, the representational and the expressive, the latter being more fundamental. Expression is realized through the activity of speech, through \emph{use}, so communication and the speech community are central to language. Language shapes the community as much as the community shapes language.

Expression has four aspects: the articulation of inchoate and dimly grasped contents; the making public among interlocutors of such contents; the partial constitution of subject-referring emotions and concerns; and the partial constitution of relationships among interlocutors and among members of the wider community. 

The first of these -- making something implicit explicit -- is in essence similar to what happens with emotions and actions but this process of giving shape covers all of language. There are times when speakers know exactly what they want to convey and just have to convert their mental contents into words; at other times, they may struggle and only partially succeed in expressing their thoughts. This wider scope of what happens in speech is a fact of life and so the simple separation of Content Selection Games from Generation Games must be seen as an idealization, as I said in \sectref{sec:micro-semantics}. The full picture must involve a range of cases allowing contents and words to be identified in any order and even together. For now, we can stick with this idealization as it is useful to first clarify how an easier instance works.

Making public is just the securing of common knowledge between speaker and addressee of the content conveyed. Here, Taylor has something irreducibly holistic in mind and implicitly criticizes the Gricean\ia{Grice, Paul@Grice, Paul} tradition's iterated approach to common knowledge. As shown in Sections~\ref{sec:agents} and \ref{sec:communication as rational activity}, the shared situation approach to common knowledge does have a holistic side but it is fully reducible. \citet[Section~6.4]{parikh:ul} and \citet[225]{parikh:le} discuss the publicity of communication and its content in some detail and it is also mentioned in passing in \sectref{sec:agents}.

The partial constitution of subject-referring emotions and concerns has already been addressed. The key here is that language remains representational but the entities represented are not always entirely language-independent. Some contents expressed will be partly formed by the words uttered by the speaker.

The partial constitution of relationships (e.g.\ equality between the interlocutors or  in a wider political sense) has two senses, one immediate as happens via a Communication Game, and another mediate as happens in society via a Language Game. The first was discussed in \sectref{sec:theory of conversation} under \emph{framing} and will be addressed further through an example in \sectref{sec:a complete example}. The second is discussed in \partref{part:V}. 

With regard to both emotions and social relationships, special mention must be made of the significance of \isi{vagueness} in language, a new point that is missing from the Romantic tradition as well. It is because almost all words in a natural language are vague to some degree that a language can constitute both human emotions and concerns and large-scale relationships in society like political equality. If our words were perfectly precise, it might be impossible to have the kinds of conversations we do in fact have that allow language to play a constitutive role. Vagueness \is{vagueness}is essential to language because without it we could not be fully human. This shows how remote such a conception of language is from that of ideal language philosophy. I will say a great deal more about \isi{vagueness} in \chapref{ch:vagueness}.

\citet[273--292]{taylor:tm} argues successfully, I believe, against truth-con\-di\-tio\-nal theories of meaning. A key point seems to be that once an expressive dimension in its four aspects is admitted, understanding must \emph{precede} truth conditions, contrary to truth-con\-di\-tio\-nal meaning theories where understanding consists in grasping truth conditions. A further difficulty with these theories is that they posit a neutral observer who has to apply the theory in order to understand an utterance.\footnote{A third point has to do with the primacy of literal meaning in these theories. Here, I believe Taylor is mistaken because all that analytic theories require is that there be some primary meaning from which secondary, indirect meanings such as implicature can be derived. This primary meaning is what is usually called literal meaning. If what Taylor calls ``invocative'' meanings are primary in other cultures -- though I very much doubt this can ever be the case because the demands of everyday life undergird everything including special invocative occasions -- then they have to make possible the derivation of other secondary meanings (e.g.\ literal meanings); otherwise, the very things that language makes possible will be rendered impossible. But doing justice to this part of Taylor's argument and rebutting it would take us too far afield.} I repeat the middle paragraph from \citet{dummett:oap} quoted in \sectref{sec:macro-semantics}:

\begin{quote}

Or, at least, they have to be explained together so long as Frege's insight continues to be respected, namely that the concept of truth plays a central role in the explanation of sense. On this Fregean view, the concept of truth occupies the mid-point on the line of connection between sense and use. On the one side, the truth-condition of the sentence determines the thought it expresses, in accordance with the theory developed by Frege and adapted by Davidson; on the other, it governs the use to be made of the sentence in converse with other speakers, in accordance with the principles left tacit by both of them. \emph{That leaves open the possibility of describing the use directly, and regarding it as determining meaning, relegating the concept of truth to a minor, non-functional role.} [my italics] This was the course adopted by Wittgenstein in his later work. The concept of truth, no longer required to play a part in a theory explaining what it is for sentences to mean what they do, now really can be characterized on the assumption that their meanings are already given. (page 19)

\end{quote}

As I wrote in \sectref{sec:macro-semantics}, Equilibrium Semantics squarely pursues the possibility of describing use \emph{directly} and deriving meaning from use. So my approach agrees with Taylor\ia{Taylor, Charles@Taylor, Charles|)} (and Dummett)\ia{Dummett, Michael@Dummett, Michael} that use is central and comes first. Moreover, as I have said in \citet[208]{parikh:le}, there is no question of a neutral observer: the ethnographer or theorist may know only part of the Communication Game.

I hope I have made a case for taking the insights of the Romantic tradition seriously. They call for expanding the notion of use that prevails in ordinary language philosophy and incorporating certain elements into Equilibrium Semantics. The adjustment is slight in material terms but the arguments sensitize us to the expressive dimension of language that has been missed by the analytic tradition. It does not conflict with the representational dimension and the four ideas that underlie Equilibrium Semantics -- reference, use, indeterminacy, and equilibrium -- continue with straightforward alterations. Indeterminacy, especially as it occurs via \isi{vagueness}, becomes central rather than something to be shunned. A key lack in Romantic theories, including those of Heidegger,\ia{Heidegger, Martin@Heidegger, Martin} is that they remain inadequately theorized from a scientific standpoint. Part of this has to do with their accepting context fully but, as I said in \sectref{sec:information}, this does not prevent us from going quite far down the scientific road contrary to what many Continental theorists believe. They have also typically adopted an anti-scientific stance and this explains why such attempts have not been forthcoming. For our purposes here, the two traditions have been amalgamated as far as that is possible.

~\\
In closing this chapter, I remind the reader of what I set out to do in \sectref{sec:equilibrium semantics} and reiterate that it is desirable and possible to develop a framework that not only brings together arguments in philosophy, linguistics, and artificial intelligence, but also incorporates a few relevant elements of psycholinguistics, sociolinguistics, and historical linguistics in order to build a \emph{unified} science of language and meaning based on communication. Secondly, it has become evident since the beginning of the last century that language and meaning occupy a central place in the human sciences, and to realize this essential role it is necessary to build a solid theory of communication as their foundation.


%\subsection{Normativity} \label{sec:normativity}
%
%\citet{habermas:wup, habermas:ctm}
%
%1. validity claims not presuppositions but result of rationality; speaker lets himself down if he fails to be comprehensible etc. ALso not universal - in an adversarial context, one may not expect truthfulness.
%strategic / communicative distinction flawed
%2. speech act theory - commands and expression involve not truth but rightness and truthfulness.

%\subsection{A Unified Science of Language and the Sciences of Man}
%\label{sec:unified science}


%syntax merely a wrinkle in this picture like friction in the study of motion
%
%
%
%
%
%
%Chomsky - Explaining Language Use
%
%H2O and water - last sentence of page 22
%
%geology - rocks?
%
%Is it really true that common sense terms have no role to play in science? What about the semantics of theories? Chomsky seems to have an internalist account of this as well.
%
%Does Chomsky draw the boundaries of science too narrowly?
%
%
%chapter 3 on Interpretation
%
%What is language - social practice vs. I-language; what is knowledge of language?
%
%
%Schiffer's point about pipeline view: develop informal picture of communication where even conventional meanings are unstable owing to modulation and language change; general equilibrium view. Draw picture in first chapter of book informally of the circularity of communication at the local and social level.

%One picky point: from the description of your theory in the review, I don't see why your account is inconsistent with the "pipeline" account.  Consider the view that in a literal utterance of s, the meaning of s constrains, but doesn't determine, what the speaker meant; what does determine it are the speaker's communicative intentions (if he's to be speaking literally, he has to mean a proposition that fits the meaning of s).  One can then see your theory as a formal model of how the hearer is to figure out what the speaker meant.
%
%
%
%Tomasello
%
%p.11 turning Chomsky on his head - grammar evolves from communication
%
%chapter 3: common ground and cooperation
%
%CK: how does it emerge? requires awareness of situation; is this awareness purely human? $s \vDash A is aware of s$ is the fundamental fact which explains CK and therefore partially accounts for communication; but what is A aware of - only RELEVANT things in s not all of s. This sort of relevance is based simply on perceptual salience or goal-based inference or shared cultural knowledge - p. 78. Other things that are not immediately relevant shade off into the penumbra - phenomenology of perception for example. See Gourevitch. Non-propositional knowledge. This penumbra allows things not in the core to emerge into the core. Thus, awareness must be understood as a graded kind of knowledge, some more and some less. Crucially, one other entity that becomes salient for A is B when A is interacting with B. ANd thus B's awareness of s also becomes salient. This is the basis of CK. Requires situations - holistics sets/chunks of facts that are smaller than the whole world. CK is not irreducibly social. But if this is the model, how do you explain breakdowns - see p. 96 - of CK? One can see how something went to two levels but not the third or three levels but not the fourth etc. Grice, Strawson, and Schiffer examples. Possibly by revising the assumption that B is aware of s to B is aware of $\sigma$. So the distinction between awareness of s and awareness of $\sigma$ must be carefully explained. Perhaps one way is to say simply that A is PARTIALLY aware of s and so is B. This can account for success and for breakdown. However, one even needs to explain why we are not aware of multiple levels when things go smoothly. We must explain this and the multiple levels we become aware of in breakdown. Partial awareness may be explained by phenomenology - unbra penumbra etc. and Heidegger's being-in-the-world. In phenomenology, there is a field of consciousness. How is this field related to a situation? Field is perceptual/experiential whereas situation is more external. Field may be the extract of a situation by an agent. If the fields sufficiently overlap, we have CK. Mirror neurons. Error of phenomenology: ontologizes experience: distinction between situation and field. Are apes aware of s? If so, then apes might also be able to communicate if they have a model of others' minds - theory of mind - http://en.wikipedia.org/wiki/Theory-of-mind.
%
%cooperation: see E. O. WIlson's latest book on cooperation among insects and other animals. Cooperation is not uniquely human. But cognitive representation of cooperation may be. See Arrow on cooperation in Use of Language. Cooperation is often two tiered. Higher level may be noncooperative. The cooperation in joint activity including communication is deeper and is presupposed even when there is confllict. See Searle on joint acts and also Use of Language. Cooperative reasoning absent in apes. Does this depend on early acquisition of a model of the other person's mind? See studies on models of mind. Premack.
%
%
%shared intentionality: actually, possibly only some of this is shared as the ambient game does the rest of the work and the game is part of what there is partial awareness about. Think also of Halpern and awareness. Awareness of communication game comes from awareness of other games played by children - Bruner.
%
%chapter 4: ontogeny in human child development - shared intentionality and common ground - could it be that the payoffs are actually group payoffs rather than inferred from each other inferentially? THis should be kept open. Beliefs may not be shareable but perhaps goals and payoffs can be through imitation or eye contact or some such mechanism. Or both mechanisms may operate.
%
%
%
%
%
%
%
%\subsection{The Science of Language as Economics}
%
%compare with general equilibrium and also with the theory of the firm and consumer; compare complexity
%
%
%
%Boeckx and \citet{bh:valt}
%
%
%
%
%
%
%
%
%
%\subsection{Grice and Conversation}
%

%1. Devitt's first class: my question: what is the basis for distinguishing between disambiguation and reference fixing on the one hand and other pragmatic factors on the other in what is said? Answer by Devitt: intuition on both the traditionalists side and the contextualists side. That is why there is no need for a distinction between semantics and pragmatics.
%
%2. My answer: there can only be intuition because there are many small contents that are added to create a large content. It does not matter where you draw the line so why draw it at all? But you can say this is literal content and this is implicature but it is fuzzy and indeterminate.







%
%To see where these approaches differ, consider the following simple utterance situation $u$. Suppose there are two Teds, Ted Smith and Ted Jones, and suppose further that the context $u$ is such that if the speaker $\cal A$ were to utter something like ``Ted is asleep,'' the addressee $\cal B$ would interpret the referent as being Ted Smith. Now assume the soeaker wishes to convey something about Ted Jones and contemplates the sentence above.
%
%In Devitt's approach, for example, since the speaker has Ted Jones \emph{in mind} the speaker meaning would be about Ted Jones, even though the context would induce $\cal B$ to interpret the utterance as being about Ted Smith. So there is a disconnect between the speaker's thought which constitutes speaker meaning and the addressee's interpretation, that is, Devitt's notion is cut off from communication. Of course, even in Devitt's approach there is a simple intention to convey Ted Jones and this conflicts with how the utterance would be interpreted and so the logic of intentions would compel the speaker to abandon this sentence and utter a more effective one such as ``Ted Jones is asleep.'' But notice that it is the logic of intentions that has led to the right choice not the speaker meaning and these, like two parallel lines, are not directly tied to each other as in the other accounts.
%
%The Gricean account allows the context and in particular the addressee's intepretation to influence speaker meaning and so in such a situation the speaker meaning would involve two inconsistent pressures, one pushing towards Ted Smith on account of the context and the other pushing towards Ted Jones on account of what the speaker wishes to convey. Since there cannot be a coherent intention on such a basis, the speaker would again seek out an alternative utterance such as ``Ted Jones is asleep.'' Grice and others never quite considered such choices and this dimension of speaker meaning was missed altogether.
%
%In my account, there is just a simple intention to convey Ted Jones but the speaker realizes that an utterance of ``Ted is asleep'' would result in conveying information about Ted Smith to the addressee and so again chooses something like `Ted Jones is asleep.''
%
%So, in none of the approaches would the speaker actually utter the sentence ``Ted is asleep'' because either a simple or complex intention would not be fulfilled and so an alternative utterance such as ``Ted Jones is asleep'' would be selected instead. 
%
%The notion of speaker meaning is meant to play a role in communication and approaches where the speaker's mind fully controls it without taking the context into account will not be able to fulfill this purpose. Grice's definition is ambivalent because it is also located entirely in the speaker's mind but does take into account the external context and the addressee and allows these factors to play a part in the relevant intention and so becomes baroque. My strategy is to retain a simple intention but allow the external context via the speaker game to affect speaker meaning. This has the advantage of accommodating Grice's basic insight into speaker meaning by giving up full control by the speaker of speaker meaning and allowing the external context to affect it as well. Schematically, one could summarize the three approaches as follows:
%
%\begin{quote}
%
%Grice and others: spekaer meaning = complex intention that takes account of external context
%
%Devitt and others: speaker meaning = thought (or meaning of mental sentence)
%
%Parikh: speaker meaning = simple intention $+$ external context
%
%\end{quote}
%
%This will be discussed in more concrete detail in later chapters in this book.
%
%
%
%
%\
%
%
%
%
%
%
%\section{Modeling: Context and Formality}
%
%modeling as abstracting from contexts - more and more. Example of friction. Example of Newton's law of gravitation. Etc. 
%
%relation to situation theory as the dual of category theory.
%
%why the word "formal" isn't a great word, especially in semantics. okay for some parts of logic either but not in general for reasoning. Indian syllogism. the idea of constraint - the relation "involves" - the idea of situation itself. Etc.
%
%formal seems to have two uses: one meaning linguistically precise and the other meaning simply precise in content.
%
%descriptive models - words do not have precisely demarcated meanings - situations that have weak structures
%
%context less relevant in the physical and life sciences and more relevant in the social sciences. Increasing relevance as one moves from inanimate to animate. That is why in physical science it is in a sense easier to get precise results.
%
%
%
%\subsection{Styles of Thinking}
%
%East and West: relational vesus atomistic, contextual vs. context free, particularistic vs. abstract
%
%continental thought: emphasis on context especially in the social sciences and the humanities
%
%situation theory as synthesis of both styles - a way to consider the particular abstractly
%
%\section{Situation Theory}
%
%Four Constraints: Phonetic, Syntactic, Semantic, Flow. Semantic - Locutionary (Conventional + Informational) + Illocutionary (Relevance + Distance). In the end, the first three constraints fall away and all you are left with is the Flow Constraint because grammar and conventional meaning and everything else comes out of the Flow Constraint.
%
%
