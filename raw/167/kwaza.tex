\documentclass[output=paper,
modfonts,nonflat
]{langsci/langscibook}
\author{Hein van der Voort\affiliation{Museu Paraense Emílio Goeldi}%
\and Edileusa Kwaza%
\and Zezinho Kwaza%
\lastand Mario Aikanã% 
}%
\title{Kwaza}
\lehead{H.\ van der Voort, Edileusa Kwaza, Zezinho Kwaza, Mario Aikanã}
\ourchaptersubtitle{Hakai dariya}
\ourchaptersubtitletrans{‘Grandfather Fox’}  
% \abstract{noabstract}
\ChapterDOI{10.5281/zenodo.1008791}

\maketitle

\begin{document}

\section{Introduction} 
 
The story of Grandfather Fox represents a myth as told by the Kwaza people of the southwestern Amazon. Kwaza is a highly endangered isolate language with about 25 speakers. The speakers of Kwaza live in two different indigenous reserves and in a nearby village, in the southeastern corner of the Brazilian state of Rondônia, amidst a sea of deforested lands owned by big cattle ranchers and soy farmers. In one of the reserves, Terra Indígena Tubarão-Latundê, the Kwaza and Latundê (\textsc{northern nambikwara}) form minority populations among the Aikanã (\textsc{isolate}). Several mixed Kwaza and Aikanã families live in another reserve, Terra Indígena Kwazá do Rio São Pedro. Despite having a very fragmented speaker community, Kwaza is still the first language of the youngest generation in two families. 

\begin{figure} 
% \includegraphics[width=\textwidth]{figures/aikana-kwaza-map}

\includegraphics[width=\textwidth]{figures/aikana-kwaza-map-mod}
  \caption{The indigenous reserves where the Kwaza live, shown in yellow.}
\end{figure}

This story was told by Edileusa Kwaza, who had learnt it from her late monolingual Kwaza father, Antonhão. As she told the story, Edileusa was accompanied by her partner Zezinho Kwaza. Most of the time, Edileusa and Zezinho live with their children and grandchildren in the little village outside the first reserve, and have very little contact with other Kwaza speakers. It should be considered admirable that, in spite of their Portuguese-only and often anti-indigenous environment, Kwaza is maintained as the family's home language. The story was recorded in audio and video formats in August 2014, as part of a documentation project funded through the DoBeS programme.The story has been transcribed and analyzed with the help of Mario Aikanã, a bilingual native speaker of Kwaza and Aikanã who, like most Indians of southeastern Rondônia, also has full command of Portuguese.

  The story of Grandfather Fox takes place in mythological times. In those days, animals transformed themselves into humans at will. Grandfather Fox is a very smart animal who has many tricks up his sleeve. In this story, he takes advantage of a young woman after finding out about her plans for the next day. One of the lessons of this and several other traditional stories is that one should avoid speaking about one’s plans for the future, so as not to attract adversity. The Aikanã also tell a mythological Fox story (chapter 11 in this volume) and, although it is quite different from the Kwaza version, it has a similar edifying message on the danger of referring to future plans. In spite of the enormous cultural and ecological changes that the Kwaza and Aikanã peoples have undergone during the 20th century, this taboo is also still very much part of their present way of life.

  Kwaza is a morphologically complex, polysynthetic, suffixing language. Its word order is relatively free. Its word classes are verbs, nouns, adverbs and particles. The only obligatory part of a complete sentence is a verb, which takes inflectional suffixes expressing person and mood, in that order. Person inflections do not seem to be etymologically related to personal pronouns, which are used for disambiguation or emphasis. The person marking system distinguishes inclusive and exclusive first person plural, both inflectionally and pronominally. Third person has zero expression morphologically. There are two sets of mood inflections, distinguishing matrix clause moods such as declarative and interrogative from subordinate clause moods such as conditional, concessional, and cosubordinate. The language has a very productive arsenal of derivational suffixes, which include classifiers, directionals, valency changing morphemes, and nominalizers. Even though the language's morphological complexity is mainly located in its verb forms, nouns can also be morphologically highly complex due to recursive verbalizing  and nominalizing  operations. A comprehensive description of the language is \citet{g:vanderVoort:Kwaza}.

  \largerpage
  From a linguistic point of view, the Fox story largely reflects the typical grammatical structure of traditional narratives in Kwaza. In principle, the entire story is one long sentence, each chained clause being either in a subordinate mood or in a cosubordinate mood. The story then ends with a fixed formulaic expression in a matrix mood, typically the declarative. If matrix moods are encountered inside the story they usually represent quotations, and either these are followed by a verb of saying in a cosubordinative mood or a cosubordinative mood marker is attached directly to the quoted part. Another characteristic of Kwaza text grammar is anticipatory switch-reference, which indicates that the subject — or perhaps rather the topic — to be foregrounded in the following clause is different from that of the current clause. These rules of thumb also hold for the present text, although some exceptions, indicated in footnotes, can be observed. Finally, many cases of morphological ellipsis can be found in the text, where either the verb stem or the verbal inflection is omitted. Such omitted elements can often be understood from the context, but in some cases verb stem ellipsis signals a quotative construction. 

  The story is transcribed in the first line in accordance with a practical orthography that was developed for use and established in 2002 by the speakers of Kwaza. The transcription is then segmented morphologically in the second line and glossed in the third line, while the fourth and fifth lines contain free translations into English and Portuguese respectively. Zezinho’s responses are indicated by [brackets]. The <y> corresponds to IPA [j], and the <'> corresponds to IPA [ʔ]. The only exception to the practical orthography concerns IPA [ɛ], which is spelled here as <ɛ> instead of <ee>.

\section{Hakai dariya}
\translatedtitle{‘Grandfather Fox’}\\
\translatedtitle{‘Vovô Raposa’}\footnote{Recordings of this story are available from \url{https://zenodo.org/record/997445}}


\ea  hakai dariya\\[.3em]
\gll hakai      dariya\\
     grandparent   fox\\
\glt ‘Grandfather Fox’\\
‘Vovô Raposa'
\z 

\ea  etaɨtohoi tsɨwɨdɨte huruyalɛ arakate tya tsɨwɨdɨte tya huruyalɛta ta ata.\\[.3em]
\gll etaɨ-tohoi      tsɨwɨdɨte  huruya-lɛ  arakate    tya    tsɨwɨdɨte  tya    huruya-lɛ-ta    ta    a-ta\\
     woman-\textsc{clf}:child  girl      like-\textsc{recp} young.man  \textsc{cso}  girl      \textsc{cso}  like-\textsc{recp-cso}    \textsc{cso}  exist-\textsc{cso}\\
\glt ‘An adolescent girl and a young man liked one another; they liked each other, that’s how it was.’\\
‘Uma moça e um rapaz novo se gostavam muito; eles se gostavam, assim viviam.'
\z

\ea  hadeya hayanɨ̃tsɨratiwɨ toma'i'ĩta tya anãi tyarahɨ̃ta tyarahɨ̃ta tsɨlehɨ̃\\[.3em]
\gll hadeya  haya-nɨ̃-tsɨ-rati-wɨ    toma=ĩ'ĩta-tya    a-nãi-tyara-hɨ̃-ta{\footnotemark}    tyara-hɨ̃-ta    tsɨ-le-hɨ̃\\
     night  day-\textsc{refl-ger-foc}{}-time  bathe=always-\textsc{cso} exist\textsc{{}-nmlz-proc-nmlz-cso}  \textsc{proc-nmlz-cso}  \textsc{ger-frust-nmlz}\\
\glt ‘Before dawn she would usually take a bath in the river, that’s how she happily lived, but ...’\\
‘Ela costumava tomar banho no rio antes de clarear o dia, assim ela vivia feliz, porém ...'
\footnotetext{The combination -\textit{tyarahɨ̃ta} gives the preceding verbal or zero-verbalized stem the connotation ‘luckily’ or ‘happily’.}
\z

\ea  dariya tswa aretya orita tsɨwɨdɨtewã hɨ̃dɛ ma'ɨ̃tɛ tomaya tyata\\[.3em]
\gll dariya  tswa  are-tya    orita    tsɨwɨdɨte-wã  hɨ̃dɛ    ma'ɨ̃tɛ    tomã-ya{\footnotemark} tyata\\
     fox  man  turn-\textsc{cso}  go.there  girl-\textsc{aobj} let’s.go  cousin    bathe-\textsc{exp}  say\\
\glt ‘Fox turned into a man and went up to the girl, saying “Let’s go! Cousin, let’s take a bath!”’{\footnotemark}\\
‘Raposa se transformou em um homem, e foi falando para a moça, dizendo “Vamos, minha prima, vamos banhar!”'
\addtocounter{footnote}{-1}
\footnotetext{The exhortative paucal is a matrix mood. Since this is a quoted utterance, it can occur embedded in the overall cosubordinated clause chain.}
\stepcounter{footnote}
\footnotetext{Background knowledge required here is that, every morning, the girl's cousin mentioned in line (2) calls her to come out and bathe in the river (as is clear from line (6)). Fox has found out about this and takes advantage by pretending to be the girl’s cousin, while it is still dark.}
\z

\ea ta nãi waxonaryɨ̃ wayata asata\\[.3em]
\gll ta		nãi		wa=xona-ryɨ̃					waya-ta	asa-ta\\
\textsc{cso}	like{\footnotemark}	take=go.pathless-\textsc{cld}:area	bring-\textsc{cso}	leave-\textsc{cso}\\
\glt ‘He took advantage of her and took her deep into the forest and left her there.’\\
‘Ele então se aproveitou dela e levou ela nas profundezas do mato e deixou ela lá.'
\footnotetext{The verb root \textit {nãi}- means ‘to be like, to act thus’. Here it refers to the rape of the girl.}
\z

\ea haya damɨ̃dɨta harɨkɨ tya tomã'etetya'ahɨ̃ hɨkwɛta\\[.3em]
\gll haya-damɨ̃-dɨ-ta		harɨkɨ		tya		tomã-ete-tya			a-hɨ̃			hɨ=kwɛ-ta\\
day-want-\textsc{ds-cso}		now		\textsc{cso}	bathe-\textsc{com-cso}		exist-\textsc{nmlz}		go=enter-\textsc{cso}\\
\glt ‘At dawn he{\footnotemark} would always bathe with her, so he entered the house, saying:’\\
‘Ao amanhecer o rapaz sempre ia tomar banho com ela, aí entrou na casa dizendo:'
\footnotetext{The real young man mentioned in line (2). Note also the different subject marker on the verb stem ‘to dawn’, marking discontinuity of the current topic.}
\z

\newpage 
\ea hɨ̃dɛ tomãya ma'ɨ̃tɛ cwatalehɨ̃\\[.3em]
\gll hɨ̃dɛ		tomã-ya	ma'ɨ̃tɛ		cwata-le-hɨ̃{\footnotemark}\\
let’s.go	bathe-\textsc{exp}	cousin 	say-\textsc{frust-nmlz}\\
\glt ‘“Cousin, let’s take a bath!”, but ...’\\
‘“Prima, vamos banhar!”, mas ...'
\footnotetext{The combination \textit {cwa-ta} (\textsc{isbj-cso}) often occurs in narrations with the lexicalized meaning ‘they say’/‘it is said’, and in some clear cases it has been glossed as such. The etymology of the verb root \textit {tyata-} ‘say to X’ is not clear, but it probably contains the transitivizer (\textit{-ta-}).}
\z

\ea mãtyate o xɨy mãtsasi bui'oyaheredalehɨ̃tsɨ tyatata\\[.3em]
\gll mã-tyate		o	xɨy		mã=tsasi		bui=oya-here-da-le-hɨ̃-tsɨ					tyata-ta\\
mother-\textsc{3poss}	oh!	you	call=follow	leave=arrive-\textsc{intl-1sg-frust-nmlz-resi}	say-\textsc{cso}\\
\glt ‘her mother said to him: “Oh! I thought you had called her already and that’s why she has gone.”'\\
‘a mãe falou: “O! Pensei que você já chamou ela, e que por isso ela já foi.”'
\z

\ea tsicwawata ahatyate tya mãtyate tya ka'awanihɨ̃xotɛ tsɨwɨdɨte\\[.3em]
\gll tsicwa-wa-ta	aha-tyate		tya		mã-tyate		tya		ka'awanihɨ̃-xotɛ	tsɨwɨdɨte\\
begin-\textsc{isbj-cso}	father-\textsc{3poss}	\textsc{cso}	mother-\textsc{3poss}	\textsc{cso}	suffer-\textsc{tr}		girl\\
\glt ‘Then her father and mother started to worry about the girl,’\\
‘Aí, o pai e a mãe dela comecavam ficar preocupado sobre a moça,'
\z

\ea hedutuhɨ̃dwata\\[.3em] 
\gll hedutu-hɨ̃-dwa-ta\\
lose-\textsc{nmlz-ids-cso}\\
\glt ‘they understood that (she) had got lost ...'\\
‘eles entenderam que ela sumiú ...'
\z
 
\ea cwata tsicwawata ka'awanihɨ̃xotɛ tsɨwɨdɨte\\[.3em] 
\gll cwata tsicwa-wa-ta	ka'awanihɨ̃-xotɛ	tsɨwɨdɨte\\
say		begin-\textsc{isbj-cso}	suffer-\textsc{tr}		girl\\
\glt ‘it is said that they started to worry about the girl,’\\
‘e dizem que comecavam ficar realmente preocupado sobre a moça,'
\z

\newpage 
\ea tya tsi'ahɨ̃te a dukɨrinitenãiko a mãtyate lehɨ̃\\[.3em]
\gll tya		tsi'a-hɨ̃-te			a		dukɨri-nite-nãi-ko		a		mã-tyate		le-hɨ̃\\
\textsc{cso}	silent-\textsc{nmlz-nmlz}	exist	long.for-\textsc{instr-nmlz-ins}	exist	mother-\textsc{3poss}	\textsc{frust-nmlz}\\
\glt ‘and they lived in sadness; her mother lived longing for her daughter, but then ...’\\
‘e eles viviam na tristeza, e a mãe vivia sentindo saudades, mas aí ...'
\z

\ea tinãits\\[.3em]
\gll ti-nãi-tsɨ-re\\
what-\textsc{nmlz-pot-int}\\
\glt ‘How is it again?'{\footnotemark}\\
‘Como que é de novo?'
\footnotetext{This is not part of the story. Here Edileusa's memory briefly fails. She turns to Zezinho, but then immediately picks up the thread of the story again.}
\z

\ea kataɨ towɨna cẽrĩ tsãrãnũ'i'ĩta\\[.3em]
\gll kataɨ	towɨ-na		cẽrĩ	tsãrãnũ-ĩ'ĩta\\
agouti	clearing-\textsc{loc}	peanut	dig-always\\
\glt ‘The agouti was in the garden digging for peanuts, as always’\\
‘A cutia estava na roça cavocando atras de amendoim, como sempre,'
\z

\ea mãtyate tsɨwɨdɨtedɨhɨ̃ tsɨlehɨ̃\\[.3em]
\gll mã-tyate		tsɨwɨdɨte-dɨ-hɨ̃	tsɨ-le-hɨ̃\\
mother-\textsc{3poss}	girl-\textsc{poss-nmlz}	\textsc{ger-frust-nmlz}\\
\glt ‘the girl’s mother, however ...’\\
‘mas a mãe da moça ...'
\z

\ea tya xudɛ'oya'i'ĩta xudɛ'oya'i'ĩta\\[.3em]
\gll tya	xudɛ=oya=ĩ'ĩta	xudɛ=oya=ĩ'ĩta\\
\textsc{cso}	offend=run=always	offend=run=always\\
\glt ‘said, while (the agouti was) harassing her and then running away time and again,’{\footnotemark}\\
‘falou enquanto (a cutia estava) xingando e correndo o tempo inteiro'
\footnotetext{When one arrives at a swidden crop field, there may be agoutis scurrying around noisily as if intending to tease the visitor.}
\z

\ea tsuhũ xudɛ'oya'i'ĩta tsɨraire mãrɛ͂ritsaxata utexaraire\\[.3em]
\gll tsũhũ	xudɛ=oya=ĩ'ĩta-tsɨ-rai-re			mãrɛ͂ritsa-xa-ta	ute-xa-rai-re\\
what	offend=run=always-\textsc{ger}-damn-\textsc{int}	person-\textsc{2sg-cso}	tell-\textsc{2sg}-damn-\textsc{int}\\
\glt ‘“Why do you always harass us and run? If you were a person, I’d want you to tell us ...”’\\
‘“Para que você sempre xinga a gente e corre? Se você fosse uma pessoa, eu queria que você contasse ...”'
\z

\ea sidɨhɨ̃ etohoi tiryɨ̃ wayawaredata tsi'ahɨ̃dahɨ̃nãrɛ͂tsɨ tyatawasi\\[.3em]
\gll si-dɨ-hɨ̃		etohoi	ti-ryɨ̃			waya-wa-re-da-ta			tsi'a-hɨ̃-da-hɨ̃-nãrɛ͂-tsɨ			tyata-wa-si{\footnotemark}\\
I-\textsc{poss-nmlz}	child	what-\textsc{cld}:area	bring-\textsc{isbj-int-1sg-cso}	silent-\textsc{nmlz-1sg-nmlz}-real-\textsc{resi}	say-\textsc{isbj-swr}\\
\glt ‘“where they have taken my child; I’m really sad because of this,” it is told ...'\\
‘“para onde eles levaram a minha criança; Estou muito triste por causa disso", se conta ...'
\footnotetext{The switch-reference mood indicates that there will shortly be an important turn of events.}
\z

\ea yãsitya ɛta ɛta ẽryãwãna ɛta mãtɨwã\\[.3em]
\gll yãsi-tya		ɛ-ta		ɛ-ta		ẽryãwã-na		ɛ-ta		mãtɨ-wã\\
hear-\textsc{cso}	go-\textsc{cso}	go-\textsc{cso}	forest-\textsc{loc}		go-\textsc{cso}	daughter-\textsc{aobj}\\
\glt ‘He heard it and he went and went into the forest, going after the daughter.’\\
‘Ele ouviu e ele foi e foi no mato, indo atras da filha.'
\z

\largerpage[2]
\ea awɨ̃ydaki mãxahɨ̃hesi ɛhɛtara wayetanĩ tyatata\\[.3em]
\gll awɨ̃y-da-ki	mã-xa-hɨ̃-he-si				ɛhɛta-ra	waye-ta-ni					tyata-ta\\
see-\textsc{1sg-decl}	mother-\textsc{2sg-nmlz-neg-swr}	tell-\textsc{imp}	bring.back-\textsc{1sgo-exh} 	say-\textsc{cso}\\
\glt ‘[He found her.] “Apparently you saw mother, so,{\footnotemark} tell her to bring me back!” she said to him,’\\
‘[Achou a filha.] Ela falou: “Parece que você viu mamãe. Fala para ela que me traz de volta!”'
\footnotetext{At this stage, Agouti has found the daughter, and she speaks to him, using what is literally a quotative construction (‘Didn't you say: “I saw mother”.’). The combination \textit {hesi} or \textit {hehɨ̃si} also means ‘well, so’ and is used for tail-head linkage.}
\z

\ea watxidɨnãi ta mãrɛ͂tata mãwã\\[.3em]
\gll watxi-dɨnãi	ta	mãrɛ͂-ta-ta			mã-wã\\
true-manner	say	appear-\textsc{tr-cso}{\footnotemark}	mother-\textsc{aobj}\\
\glt ‘“Explain correctly to mother ...’”\\
‘“Explica bem para mamãe ...”'
\footnotetext{The transitivizer is probably lexicalized yielding the meaning ‘to explain’.}
\z

\ea mĩw arwenãdɨxalɛ sidule waidɨnãixalɛ\\[.3em]
\gll mĩu	arwenã-dɨ-xa-lɛ			si-du-le		wai-dɨnãi-xa-lɛ\\
chicha{\footnotemark}	prepare-\textsc{caus-2sg-prec}	I-\textsc{ber}-only	good-manner-\textsc{2sg-prec}\\
\glt ‘“when you prepare it, make proper \textit{chicha} only for me ...”’\\
‘“quando preparar, faz chicha de verdade para mim ...”'
\footnotetext{In the cultures of several indigenous peoples of Rondônia, \textit{chicha} is a lightly alcoholic fermented beverage usually based on yam, manioc, or maize.}
\z

\ea aɨhɨ̃ tswa dariyaxwadule tẽityadɨnãixalɛ tsukuxotɛxahɨ̃\\[.3em]
\gll aɨ-hɨ̃		tswa	dariya-xwa-du-le			tẽitya-dɨnãi-xa-lɛ			tsuku-xotɛ-xa-hɨ̃\\
that-\textsc{nmlz}	man	fox-\textsc{clf}:man-\textsc{ber}-only	alone-manner-\textsc{2sg-prec}	chew-\textsc{tr-2sg-nmlz}\\
\glt ‘“and you mix separately for the fox-man only ...”’\\
‘“e para o homem raposa só,...”'
\z

\ea yerexwanũ kurakuranũ xalɛ tsukuxalɛ\\[.3em]
\gll yerexwa-nũ			kurakura-nũ			xa-lɛ		tsuku-xa-lɛ\\
jaguar-\textsc{clf}:powder	chicken-\textsc{clf}:powder	\textsc{2sg-prec}	chew-\textsc{2sg-prec}\\
\glt ‘“you mix in dog shit and chicken poop ...”’\\
‘“você mistura bosta de cachorro e bosta de galinha ...”'
\z

\ea tẽityadɨnãi kudɨna dokwɛdɨwayanɨ̃xalɛ sidule\\[.3em]
\gll tẽitya-dɨnãi				kudɨ-na		do=kwɛ-dɨ		waya-nɨ̃-xa-lɛ				si-du-le\\
alone-manner-\textsc{2sg-prec}	calabash-\textsc{loc}	pour=enter-\textsc{caus}	bring-\textsc{refl-2sg-prec}	I-\textsc{ber}-only\\
\glt ‘“you put it inside a calabash by itself and bring it to me ...’”\\
‘“você mistura e coloque numa cabaça separada e traz para mim, e separado ...”'
\z

\ea tẽityanãi dokwɛxalɛ kataɨdule emɛ͂ tsẽimɛ͂\\[.3em]
\gll tẽitya-nãi		do=kwɛ-xa-lɛ				kataɨ-du-le		e-mɛ͂			tsẽi-mɛ͂	\\
alone-\textsc{nmlz}	pour=enter-\textsc{2sg-prec}	agouti-\textsc{ber}-only	\textsc{ø-clf}:porridge	one-\textsc{clf}:porridge\\
\glt ‘“and for Agouti you separately put in maize porridge ...”’\\
‘“para Cutia você coloque bagaço de milho,”'
\z

\ea tẽityanãilɛ wayanɨ̃xalɛ wayetatsɨtse tyata daheretsɨlɛ\\[.3em]
\gll tẽitya-nãi-lɛ		waya-nɨ̃-xa-lɛ				waye-ta-tsɨ-tse{\footnotemark}				tyata	da-here-tsɨ-lɛ{\footnotemark}\\
alone-\textsc{nmlz-prec}	bring-\textsc{refl-2sg-prec}	bring.back-\textsc{1sgo-pot-decl}	say		\textsc{1sg-intl-pot-prec}\\
\glt ‘“and you bring it to me,” I think she said.’\\
‘“e você traz isso para mim,” acho que ela falou.’'
\addtocounter{footnote}{-1}
\footnotetext{The declarative mood is a matrix mood. Here it marks the end of a chain of quoted utterances that started with (20) and that is embedded in the general cosubordinate clause chain by \textit{tyata}.}
\stepcounter{footnote}
\footnotetext{This clause in the first person singular represents a metatextual remark by Edileusa.}
\z

\ea watxile hayadɨta tsãtxa atxitxi cwata\\[.3em]
\gll watxile	haya-dɨ-ta		tsãtxa	atxitxi	cwata\\
finally		day-\textsc{ds-cso}	scatter	maize	say\\
\glt ‘And then on the next day mother (went up onto the storage platform and) dropped maize (cobs) on the ground, so ...’\\
‘No dia seguinte a mãe (subiu no paiol e) derrubou (espigas de) milho no chão, aí ...'
\z

\ea arwenãwata waya horomũtya wayata\\[.3em]
\gll arwenã-wa-ta		waya	horo-mũ-tya			waya-ta\\
prepare-\textsc{isbj-cso}	bring	finish-\textsc{clf}:liquid-\textsc{cso}	bring-\textsc{cso}\\
\glt ‘she finished preparing the \textit{chicha} and went to take it there.’\\
‘ela aprontou a chicha e levou lá.'
\z

  
\ea karɛ͂xu katsucwata xalɛ axehɨ̃tse\\[.3em]
\gll karɛ͂xu				katsu-cwa-ta	xa-lɛ		 axe-hɨ̃-tse\\
dry.heartwood	cross-\textsc{isbj-cso}	\textsc{2sg-prec}	find-\textsc{nmlz-decl}\\
\glt ‘“If you go across a dry log” she would meet (her daughter),’\\
‘“Quando você travessa um pau cerno,” ela ia encontrar (a filha),'
\z

\ea tsɨhɨ̃kitya\\[.3em]
\gll tsɨ-hɨ̃-ki-tya\\
\textsc{ger-nmlz-decl-cso}\\
\glt ‘he (Agouti) having told her thus.’\\
‘assim Cutia tinha falado.'
\z

\ea watxile karɛ͂xu katsutyata xareyawata axehɨ̃ko tsadwɛnɛ\\[.3em]
\gll watxile	karɛ͂xu				katsu-tya	ta		xareya-wa-ta		axe-hɨ̃-tya{\footnotemark}	tsadwɛ-nɛ\\
finally		dry.heartwood	cross-\textsc{cso}	\textsc{cso}	search-\textsc{isbj-cso}		find-\textsc{nmlz-cso}	onto.path-\textsc{dir}:hither\\
\glt ‘Later, crossing the dry log, they{\footnotemark} then searched and got back onto the path.’\\
‘Mais tarde, ela com a sua família toda travessaram o cerno, procuraram e sairam na estrada de novo.'
\addtocounter{footnote}{-1}
\footnotetext{In the transcription, Mario replaced the instrumental marker -\textit{ko} by the cosubordinative marker -\textit{tya}.}
\stepcounter{footnote}
\footnotetext{The girl's mother and the rest of her family.}
\z

\ea itso'ɨiri tatsitswa bu'aruryɨ̃ aruxe aɨhɨ̃ tsɨwɨdɨtewã\\[.3em]
\gll itso'ɨi-ri-tatsitswa			bu=aruryɨ̃		aru-xe			aɨ-hɨ̃		tsɨwɨdɨte-wã\\
liana-\textsc{clf}:flat-\textsc{dir}:backside	put=leave		place-\textsc{clf}:leaf	that-\textsc{nmlz}	girl-\textsc{aobj}\\
\glt ‘There, under a huge vine, was that girl, where she had been left sitting (by Fox).’\\
‘Alí, em baixo de um cipozeiro (Raposa) tinha deixado a moça sentada.'
\z
 
\ea etaɨ etsɨratihɨ̃ kudɨnũwata\\[.3em]
\gll etaɨ		e-tsɨ-rati-hɨ̃			ku-dɨ-nũ-wa-ta\\
woman	have-\textsc{ger-foc-nmlz}	insert-\textsc{caus-clf}:powder-\textsc{isbj-cso}\\
\glt ‘The thing that a woman has{\footnotemark} had been stuck in the ground ...’\\
‘A coisa que a mulher tem estava fincado em baixo da terra ...'
\footnotetext{Her clitoris. Edileusa has skipped a part mentioned by Mario where, after Fox raped the girl, her clitoris grew and became long like a penis. The girl was then fixed to the ground with her clitoris, so that she could not get up and run away.}
\z

\newpage 
\ea txe:nũ tsãrãnũ kudɨratsanũcwasi ũryɨ̃ ũxeryɨ̃\\[.3em]
\gll txe-nũ				tsãrãnũ	ku-dɨ-ratsa-nũ-cwa-si								ũryɨ̃	ũxe-ryɨ̃\\
close-\textsc{clf}:powder	soil		insert-\textsc{caus-dir}:underneath-\textsc{clf}:powder-\textsc{isbj-swr}	sit		remain-\textsc{cld}:area\\
\glt ‘it was secured into the ground, underneath the soil where she remained sitting.’\\
‘estava fixado no fundo da terra onde ela permaneceu sentada.'
\z

\ea harɨkɨ orita watxile kuidɨ aɨhɨ̃ dariyaxwawã cwasi isitsɨnãixwadɨta\\[.3em]
\gll harɨkɨ		orita		watxile	kui-dɨ			aɨ-hɨ̃		dariya-xwa-wã	cwa-si	isi-tsɨ-nãixwa-dɨ-ta\\
now		arrive.there	finally		drink-\textsc{caus}	that-\textsc{nmlz}	fox-\textsc{clf}:man-\textsc{aobj}	\textsc{isbj-swr}	die-\textsc{ger-simu-ds-cso}\\
\glt ‘Now, when they arrived, they gave Fox that drink, and he seemed to get drunk.’\\
‘Aí, quando chegaram eles deram aquela bebida para Raposa, e  parecia que ele ficou bêbado.'
\z

\ea watxile duwata yeyedaratuwata ehɨ̃tyatewã cwata\\[.3em]
\gll watxile	du-wa-ta	ye\textasciitilde ye=daratu-wa-ta		e-hɨ̃-tyate-wã			cwa-ta\\
finally		all-\textsc{isbj-cso}	dig\textasciitilde dig=to.after-\textsc{isbj-cso}	have-\textsc{nmlz-3poss-aobj}	\textsc{isbj-cso}\\
\glt ‘Then all (the family) got together and started digging after her thing.’\\
‘Aí, eles todos se juntaram e começavam cavocando atrás a coisa da menina.'
\z
 
 
\ea yewata yewata lonã yecwata oyahaɨryɨ̃lɛ dɨcwasi\\[.3em]
\gll ye-wa-ta		ye-wa-ta	lonã	ye-cwa-ta		oya=haɨ-ryɨ̃-le				dɨ-cwa-si\\
dig-\textsc{isbj-cso}		dig-\textsc{isbj-cso}	hole	dig-\textsc{isbj-cso}		arrive=final-\textsc{cld}:area-only	cut-\textsc{isbj-swr}\\
\glt ‘They dug and dug a hole and dug and (by accident) cut it off at the very end ...’{\footnotemark}\\
\footnotetext{As Mario explains, the part of the girl's  clitoris that stayed in the ground turned later into a caecilian, known in Portuguese as “cobra-cega" (\textit{Gymnophiona}), or an amphisbaenian worm lizard, known in Portuguese as “cobra-de-duas-cabeças" (\textit{Amphisbaenidae}).}
\newpage 
‘Cavocaram e cavocaram um buraco e no fundo eles cortaram (acidentalmente) o finalzinho da coisa ...'
\z

  
\ea cwata waye etaɨ cwata wɛ'ɨhɨ̃ cwasi\\[.3em]
\gll cwa-ta		waye			etaɨ		cwa-ta		wɛ-ɨ̃-hɨ̃			cwa-si\\
\textsc{isbj-cso}		bring.back		woman	\textsc{isbj-cso}		carry-\textsc{att-nmlz}	\textsc{isbj-swr}\\
\glt ‘then they took her and brought her back ...’\\
‘aí levaram ela e trouxeram de volta ...'
\z

\ea etaɨ dɨcwahɨ̃ txuhũitsɨhɨ̃le karitsutɨ areki cwarati\\[.3em]
\gll etaɨ		dɨ-cwa-hɨ̃		txuhũi-tsɨ-hɨ̃-le			karitsutɨ		are-ki		cwa-rati{\footnotemark}\\
woman	cut-\textsc{isbj-nmlz}	small-\textsc{ger-nmlz}-only		worm.lizard	turn-\textsc{decl}	\textsc{isbj-foc}\\
\glt ‘only the woman’s small cut-off part (which stayed behind) turned into a worm lizard, that is what they say.’\\
‘somente aquele pequeno pedaço cortado ficou para trás e virou numa cobra-cega, é que dizem.'
\footnotetext{The story ends provisionally with the declarative matrix mood and a fixed formulaic quotative expression (‘they (say)’). However, nudged by Zezinho's minimal responses, several concluding parts are added hereafter.}
\z

\ea {hm}\\[.3em]
\glt [response from Zezinho:] ‘Hm.'\\
\glt [resposta do Zezinho:] ‘Hm.'
\z

\ea cwata etaɨle wɛ axɨna cwata\\[.3em]
\gll cwa-ta		etaɨ-le			wɛ		axɨ-na			cwa-ta\\
\textsc{isbj-cso}		woman-only	carry	house-\textsc{loc}	\textsc{isbj-cso}\\
\glt ‘Then they carried the woman home ...’\\
‘Aí carregaram a menina para casa ...'
\z
 
\ea dariyaxwale isi mĩw kuita ta\\[.3em]
\gll dariya-xwa-le		isi	mĩu	kui-ta		ta\\
fox-\textsc{clf}:man-only	die	chicha	drink-\textsc{cso}	\textsc{cso}\\
\glt ‘Fox got completely drunk drinking \textit{chicha} ...’\\
‘Raposa ficou completamente bêbado da chicha ...'
\z

\ea wɛwata wasi aɨryɨ̃\\[.3em]
\gll wɛ-wa-ta		wa-si		aɨ-ryɨ̃\\
carry-\textsc{isbj-cso}	\textsc{isbj-swr}		that-\textsc{cld}:area\\
\glt ‘while they were taking her there ...’\\
‘enquanto estavam levando a moça ...'
\z

\ea harɨkɨ dariyaxwale turwetya isi mĩw kuita isihɨ̃ tya\\[.3em]
\gll harɨkɨ		dariya-xwa-le		turwe-tya	isi	mĩu	kui-ta		isi-hɨ̃		tya\\
now		fox-\textsc{clf}:man-only	heal-\textsc{cso}	die	chicha	drink-\textsc{cso}	die-\textsc{nmlz}	\textsc{cso}\\
\glt ‘Then Fox recovered from getting drunk on the \textit{chicha}.’\\
‘Aí, Raposa sarou da bebedeira da chicha.'
\z

\ea kataɨwãle tiryɨ̃ oyare etaɨ awɨ̃yxare etaɨ wayeware etaɨ tyata ta\\[.3em]
\gll kataɨ-wã-le		ti-ryɨ̃			oya-re		etaɨ		awɨ̃i-xa-re	etaɨ		waye-wa-re		etaɨ		tya-ta	ta\\
agouti-\textsc{aobj}-only	what-\textsc{cld}:area	arrive-\textsc{int}	woman	see-\textsc{2sg-int}	woman	bring.back-\textsc{isbj-int}	woman	\textsc{isbj-cso}	\textsc{cso}\\
\glt ‘Then (he asked) Agouti: “Where has the woman gone? Did you see the woman? Did they take her back?” he said.’\\
‘Aí, perguntou para Cutia: “Onde foi a mulher? Viu a mulher? Eles levaram ela de volta?”'
\z

\ea kataɨwãle tinãi mã'are tinãi tsotsotsi tĩcwahɨ̃\\[.3em]
\gll kataɨ-wã-le		ti-nãi		mã-a-re		ti-nãi		tso\textasciitilde tsotsi		tĩ-cwa-hɨ̃\\
agouti-\textsc{aobj}-only	what-\textsc{nmlz}	call-\textsc{1pl.incl-int}	what-\textsc{nmlz}	\textsc{red}\textasciitilde cotton	spin-\textsc{isbj-nmlz}\\
\glt ‘On Agouti ... What did we use to call it? What is the thing for spinning cotton?’\\
‘Na Cutia ... Como a gente costumava chamar? Qual é a coisa para fazer linha de algodão?'
\z

\ea towɛyasi towɛyahɨ̃\\[.3em]
\gll towɛ-ya-si	towɛ-ya-hɨ̃\\
break-\textsc{iobj-swr}	break-\textsc{iobj-nmlz}\\
\glt [response from Zezinho:] ‘The thing to pierce and to break off?’\\
\glt [resposta do Zezinho:] ‘Aquele de furar e quebrar?'
\z

\ea tsotsotsi'ɨyniteko tsotsotsi tĩcwaratihɨ̃kole toweyase cwasi\\[.3em]
\gll tso\textasciitilde tsotsi-ɨi-nite-ko					tso\textasciitilde tsotsi		tĩ-cwa-rati-hɨ̃-ko-le			towe-ya-se			cwa-si\\
\textsc{red}\textasciitilde cotton-\textsc{clf}:thread-\textsc{instr-ins}	\textsc{red}\textasciitilde cotton	spin-\textsc{isbj-foc-nmlz-ins}-only	break-\textsc{iobj-clf}:anus	\textsc{isbj-swr}\\
\glt ‘With the thing for making yarn, with that very thing for spinning yarn, Fox poked Agouti in his ass and it broke off.’\\
‘Com a coisa para tecer linha, com aquela coisa mesma, Raposa enfiou na bunda do Cutia e quebrou.'
\z

\ea cwahɨ̃ta kataɨ esinyũtorihɨ̃ eki	tsɨnãi	ewahɨ̃tsɨ\\[.3em]
\gll cwa-hɨ̃-ta		kataɨ	esinyũ-tori-hɨ̃		e-ki=tsɨ-nãi			e-wa-hɨ̃-tsɨ\\
\textsc{isbj-nmlz-cso}	agouti	tail-\textsc{clf}:tip-\textsc{nmlz}	have-\textsc{decl=ger-nmlz}	have-\textsc{isbj-nmlz-resi}\\
\glt ‘That is why people say Agouti has a short tail like that.’\\
‘Pessoal conta que é por isso que a cutia tem um rabinho curto assim.'
\z

\ea cwata xui wadɨcwahɨ̃ki cwahɨ̃ \\[.3em]
\gll cwa-ta		xui		wadɨ-cwa-hɨ̃-ki		cwa-hɨ̃	\\
\textsc{isbj-cso}		bag	give-\textsc{isbj-nmlz-decl}		\textsc{isbj-nmlz}\\
\glt ‘Then it is said he was given a crochet bag,’{\footnotemark}\\
‘Aí a família da moça deu uma bolsa de tucum para Cutia;'
\footnotetext{By way of thanks, the girl gave Agouti a traditional type of crochet bag (known in local Portuguese as \textit{marico}), the yarn of which is handmade out of the fibers of the young leaves of the \textit{wade} ‘tucuma’ (\textit{Astrocaryum tucuma}) or \textit{hakare} ‘burití’ (\textit{Mauritia flexuosa}) palm trees.}
\z

\ea xuinyɛ͂ etuna exɨitswa unãhɨ̃'e nãi e kataɨ cwasi\\[.3em]
\gll xui-nyɛ͂			etu-na		exɨitswa	unã-hɨ̃-e		nãi		e		kataɨ	cwa-si\\
bag-\textsc{clf}:leaf	back-\textsc{loc}	fur			long-\textsc{nmlz}-too	like	have	agouti	\textsc{isbj-swr}\\
\glt ‘giving him a bag (made out of palm leaves) on his back, that is why Agouti also has long fur.’\\
‘deram a ele uma bolsa (de fibras de tucum) e assim é que a cutia tem pêlo comprido nas costas.'
\z

\newpage 
\ea etaɨletsɨhɨ̃ tsi'ahɨ̃te ata a ta tswa aure'ɨhɨ̃\\[.3em] 
\gll etaɨ-le-tsɨ-hɨ̃				tsi'a-hɨ̃-te			a-ta		a		ta		tswa	aure-ɨ̃-hɨ̃\\
woman-only-\textsc{ger-nmlz}	silent-\textsc{nmlz-nmlz}	exist-\textsc{cso}	exist	\textsc{cso}	man	marry-\textsc{att-nmlz}\\
\glt ‘Only the girl remained sad (having lost a piece of herself), married to a man,’\\
‘Aí, somente a menina ficava na tristeza (porque perdeu um pedaço), agora casou com um homem,'
\z

\ea harɨkɨ tswatete mãrɛ͂ritsaxwa tse cwaratihɨ̃tsɨ\\[.3em]
\gll harɨkɨ	tswa-tete		mãrɛ͂ritsa-xwa	tse		cwa-rati-hɨ̃-tsɨ{\footnotemark}\\
now	man-\textsc{intens}	person-\textsc{clf}:man	yes		\textsc{isbj-foc-nmlz-resi}\\
\glt ‘now a real human man, they say, that’s the way it is told.’\\
‘agora um homem de verdade, dizem, assim que é contado.'
\footnotetext{The last word of this sentence represents a  formulaic expression that terminates the traditional story. Hereafter a few other clauses are added, in (55) and (56), explaining the norms of olden times, which also end in a formulaic expression. Then in (57) and (59) these norms are contrasted with those of the present.}
\z

\ea a'ayawɨ cwata unɨ̃tetawata txarwa hakahɨ̃ awɨ\\[.3em]
\gll a\textasciitilde a-ya-wɨ				cwa-ta		unɨ̃teta-wa-ta		txarwa		haka-hɨ̃	a-wɨ\\
exist\textasciitilde exist-\textsc{iobj}-time	\textsc{isbj-cso}		converse-\textsc{isbj-cso}	first		old-\textsc{nmlz}	exist-time\\
\glt ‘Speaking today about our olden times,’\\
‘Agora, falando sobre o tempo que os velhos viviam,'
\z

\ea tẽitya hɨhɨrwatsi tsɨwɨdɨte tyatawa mãrɛ͂tawata cwaki cwarati\\[.3em]
\gll tẽitya	hɨhɨrwa-tsi	tsɨwɨdɨte	tyata-wa	mãrɛ͂-ta-wa-ta		cwa-ki		cwa-rati\\
alone	walk-\textsc{mon}		girl			say-\textsc{isbj}		explain-\textsc{tr-isbj-cso}	\textsc{isbj-decl}		\textsc{isbj-foc}\\
\glt ‘they didn’t let girls walk alone; they gave them advice, this is the way it is told.’\\
‘eles não deixaram as meninas andar sozinho; eles deram conselho, assim que fizeram.'
\z

\ea txarwa txana heyahɨ̃\\[.3em]
\gll txarwa		txana		he-ya-hɨ̃\\
first		we.\textsc{incl}	\textsc{neg-iobj-nmlz}\\
\glt ‘Nowadays we don’t.’{\footnotemark}\\
‘Hoje, a gente não faz isso.'
\footnotetext{Lit. ‘Nowadays we aren't (given advice).'}
\z

\ea {hm}\\[.3em]
\glt [response from Zezinho:] ‘Hm.'\\
\glt [resposta do Zezinho:] ‘Hm.'
\z

\ea tsɨwɨdɨte xareredɨnãiko adɨ'ata\\[.3em]
\gll tsɨwɨdɨte	xarere-dɨnãi-ko	a-dɨ-a-ta\\
girl			crazy-manner-\textsc{ins}	exist-\textsc{caus-1pl.incl-cso}\\
\glt ‘We let girls act crazy like that, in present day life.’\\
‘Hoje deixamos as moças fazer bagunça, vivendo assim.'
\z

\ea só isso\\[.3em]
\gll a-a-hɨ̃		aɨ-hɨ̃-le-tse{\footnotemark}\\
exist-\textsc{1pl.incl-nmlz}	that-\textsc{nmlz}-only-\textsc{decl}\\
\glt ‘That’s it, just that.’\\
‘Só isso.'
\footnotetext{On the original recording Edileusa ended her narrative in Portuguese by saying: \textit{só isso} ‘just that/that’s it’. In the transcription Mario replaced those words by their Kwaza equivalent \textit{a'ahɨ̃ aɨhɨ̃letse}, in order to complete the sentence — as well as the entire story — in a declarative mood.}
\z

\section*{Acknowledgements}
Generous funding by the VolkswagenStiftung of DoBeS (Dokumentation Bedrohter Sprachen) project nr. 85.611 is hereby gratefully acknowledged.

\section*{Non-standard abbreviations}

\begin{tabularx}{.45\textwidth}{lQ}
\textsc{aobj } & animate object \\
\textsc{att } & attributive \\
\textsc{ber } & beneficiary (nominal case marker)\\
\textsc{cld } & classifier/directional\\
\textsc{cso } & cosubordinative\\
\textsc{dir } & directional \\
\textsc{ds } & different subject\\
\textsc{emph } & emphatic \\
\textsc{exh } & exhortative \\
\textsc{exp } & exhortative paucal \\
\textsc{frust } & frustrative \\
\textsc{ger } & gerundi(v)al\\
\textsc{ids } & indefinite different subject \\
\end{tabularx}
\begin{tabularx}{.5\textwidth}{lQ}
\textsc{instr } & instrument nominalizer \\
\textsc{int } & interrogative \\
\textsc{intens } & intensivizer \\
\textsc{intl } & intentional \\
\textsc{iobj } & indefinite object \\
\textsc{isbj } & indefinite subject \\
\textsc{mon } & monitory \\
\textsc{nei } & negative imperative \\
\textsc{pot } & potential \\
\textsc{prec } & preconditional \\
\textsc{red } & reduplication \\
\textsc{resi } & resignation \\
\textsc{sgo } & singular object \\
\textsc{simu } & simulative \\
\textsc{swr } & switch reference mood\\
\end{tabularx}

{\sloppy
\printbibliography[heading=subbibliography,notkeyword=this]
}
\end{document}
