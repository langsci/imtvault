\documentclass[output=paper]{langsci/langscibook} 
\ChapterDOI{10.5281/zenodo.1048188} %will be filled in at production

\author{Fernando Prieto Ramos\affiliation{University of Geneva}}
\title{The evolving role of institutional translation service managers in quality assurance: Profiles and challenges}
\shorttitlerunninghead{The evolving role of institutional translation service managers in QA}
\abstract{The diversification of translation services, tools and quality expectations in an increasingly globalized translation industry has accentuated the significance of translation quality assurance (TQA) processes and their management. This paper focuses on the profile of institutional translation service managers from a holistic TQA perspective. After a short account of their most common duties as described in sample vacancy notices from various international organizations, including senior service managers and mid-level language unit heads, the role of the latter as competence and process managers responsible for translation quality in specific languages is analyzed in more detail. The data compiled in 24 interviews with mid-level managers serve to outline an inventory of their main TQA-related functions and challenges, and lead to conclusions on their growing relevance and expected expertise in this evolving field.
}
\maketitle

\begin{document}


\section{Introduction: managing translation quality in a changing landscape}\label{sec:prietoramos:1}

Translation management functions have been the subject of little research in Translation Studies. Yet, they play a key role in monitoring \isi{translation quality assurance} (\isi{TQA}) processes in a context of increasing diversification of translation services and greater automatization of production workflows. The growing differentiation of \isi{translation quality} and pricing levels according to market segment and client needs has further reinforced the significance of translation management. These trends have been shaped by a number of interrelated factors, in particular, heightened global interconnectivity, an exponential surge in information flows and multilingual content needs in multiple formats, and the development of technological tools, including online translation applications and crowdsourcing platforms (see e.g. \citealt{García2015} and \citealt{Jiménez-Crespo2017}). \citet[388]{Görög2014} summarizes the industry perspective in the following terms: “The only way to offer large amounts of information and goods in multiple languages fast while staying within reasonable budgets is by making a compromise and provide content with different levels of quality using new translation channels and translation technology.”


For \isi{translation service} providers (TSPs) in the private sector, which has been at the frontline of this diversification, providing less-than-maximal quality is not a taboo but may prove a fit solution depending on job specifications and market conditions (see e.g. \citealt{Wright2006}; \citealt{Gouadec2010}; \citealt{O’Brien2012}; \citealt{Drugan2013}; \citealt{FieldsEtAl2014}). It is not surprising that the last few years have witnessed the emergence of \isi{translation quality} standards specifically aimed at certifying TSPs in this more diversified and globalized market: the European EN 15038:2006 \citep{EN2006} and its successor, the international \citeauthor{ISO2015}. These standards do not provide any variables to define or measure quality, but focus on two key elements that contribute to ensuring \isi{quality translation}: workflow specifications (including systematic revision), and definition of roles and competence requirements for the different actors involved in the service provision process. 



These industry developments (and their conceptual underpinnings) are finding their way into translator training programmes and are gradually permeating the translation services of public institutions, especially in situations where budgetary constraints push them towards cost-benefit considerations typically found in the private sector. The translation services of international organizations are no exception. In light of increasing productivity demands and strain on resources, traditional models based on ideals of absolute quality are nuanced by \isi{efficiency} considerations of content prioritization and \isi{quality control} modulation according to potential risk or impact of translations (on risk management in translation projects, see e.g. \citealt{Dunne2013,CanforaOttmann2015}). Although this approach is not new, it is becoming more explicit and widespread, as illustrated by \citet{Prioux2007} in their “economy of revision” at the Organization for Economic Cooperation and Development. The translation services of the \isi{European Union} (\isi{EU}) institutions provide the most clear example of this trend: the addition of nine \isi{official languages} in 2004 and three more in 2007 compounded the need to streamline practices and triggered “a move towards a more conscious, structured and systematic approach to \isi{quality assurance}” \citep[52]{Strandvik2017}.
 


All the above services and processes ultimately rely on \isi{translation expertise} for effective implementation. In any \isi{holistic approach} to \isi{translation quality}, competence is indeed at the core of \isi{quality assurance}, together with process and product, as no standard, guideline or assessment grid can be effective without the necessary skills to apply them (see e.g. \citealt[20]{PrietoRamos2015}). From this perspective, we will delve into the profiles and challenges of those who are generally responsible for the \isi{recruitment} and \isi{coordination} of translation professionals in the pursuit of quality at international organizations: institutional \isi{translation service} managers (ITSMs), including directors of entire translation services (senior ITSMs) and, in particular, heads or chiefs of language units or sections (mid-level ITSMs or language-bound ITSMs).\footnote{In this chapter, denominations are used in a non-exhaustive way to include all possible titles of comparable structures or profiles. For instance, “translation services” might be part of a “division” or a “department”, and be composed of various language “sections”, “units” or “departments”, depending on the institution. Likewise, “heads” might be called “chiefs” or “directors” at different management levels, while “quality advisers” might be “managers”, “coordinators”, “controllers” or “focal points” depending on rank and nature of the job in each institution. These arbitrary denominations are secondary to the rationale behind the structures and duties presented in this paper.} The aim of this study is to contribute to our understanding of what it takes to be an \isi{ITSM} today by asking two key questions: What is the role of \isi{institutional translation} team managers in \isi{TQA}? What skills are required of them considering global trends in the field? 



To this end, the following sections will draw an overview of institutional \isi{translation service} management based on the combined analysis of: (1) the management structures of 12 organizations (eight intergovernmental and four \isi{EU} institutions); (2) the job descriptions contained in 14 \isi{ITSM} vacancy notices (seven for language section chiefs and seven for service directors, randomly selected among vacancies announced at international organizations between 2010 and 2016, including four from \isi{EU} institutions, two for each level of management); and (3) 24 interviews with mid-level ITSMs (three per language service) conducted in three representative institutional settings between April and July 2017: the \isi{EU} (more precisely, the \isi{European Commission}’s \isi{Directorate-General for Translation} (\isi{DGT}), the Council of the \isi{EU}, the European Parliament and the Court of Justice of the \isi{EU}), with 24 \isi{official languages}; the United Nations (UN), including three of its duty stations, with six \isi{official languages}; and the World Trade Organization (WTO), with three \isi{official languages}, an example of medium-size specialized intergovernmental institution. In order to focus on commonalities and preserve anonymity, reference to specific organizations is avoided to the extent possible and no individual \isi{ITSM} is quoted in the presentation of results.



The next section will provide a brief account of the most common responsibilities of ITSMs at international organizations (\sectref{sec:prietoramos:2}). This account will prepare the ground for a closer exploration of TQA-related practices and challenges of translation unit managers responsible for quality in their respective languages in the three selected institutional settings (\sectref{sec:prietoramos:3}). These data will inform conclusions regarding ITSMs’ and \isi{TQA} (\sectref{sec:prietoramos:4}).  



\section{Institutional translation management duties: the common ground}\label{sec:prietoramos:2}


\largerpage
The responsibilities of ITSMs vary depending on the structures and features of their translation services, which, in turn, reflect institutional approaches to \isi{multilingualism}. All the services surveyed (12 in total) were established to support the production of legal, policy and administrative documents in a diversity of \isi{official languages}. They epitomize a hierarchical paradigm in which ITSMs stand at the top and tend to specialize more or less in managerial or translation and revision tasks depending on the size of the service. While the prototypical \textit{management structure} includes at least one director of service and several heads of language units or sections, the largest services (in terms of staff, translation volumes and number of languages) have more complex organigrams and a higher degree of division of management and transversal tasks. The service structure often justifies the delegation of functions such as coordinating thematic projects or monitoring certain aspects of \isi{TQA} implementation (e.g. in the case of quality managers or advisers at \isi{EU} institutions). This is explained by the more significant need to coordinate or harmonize policies and practices between, and even within, service units, at times located in different duty stations. 



At the other extreme, in very small services, ITSMs may actually be the only in-house staff. This is the case of one small organization included in our sample. The service is composed of one manager per target language (two in total) in charge of outsourcing translations depending on workload fluctuations, as well as translating, revising and managing all aspects of \isi{quality assurance}. ITSMs deal with a significant concentration of translation and project management functions but have a more modest administrative workload compared to larger services, particularly with regard to \isi{human resources}.


\largerpage
The most common \textit{duties} of mid-level ITSMs, as found in the structures and vacancy notices examined, can be summarized under four categories, from more strategic to more practical translation work:


\begin{itemize}
\item 
Strategic, administrative and financial matters
\item 
Staffing matters, including \isi{recruitment} and performance appraisal
\item 
Translation workflow \isi{coordination}  
\item 
Contribution to translation, technical and \isi{quality control} tasks
\end{itemize}

All categories are interrelated, particularly categories 1 to 3, as strategic and financial matters are then reflected in implementation aspects supervised under categories 2 and 3, while category 4 tends to vary enormously depending on the size of the service. In larger services, managers are not always expected to revise on a regular basis, and they rarely translate, as opposed to managers of smaller services. 



Among the sample job descriptors of mid-level ITSMs, the most detailed one was published by the UN. The responsibilities listed in the relevant vacancy notice are reproduced in \tabref{tab:prietoramos:1} below and aligned to the four functional categories identified above. This list can be considered representative of management duties at language unit level in large translation services.


\begin{table}
\footnotesize
\begin{tabularx}{\textwidth}{lX}
\lsptoprule
\multicolumn{2}{l}{1. Strategic, administrative and financial matters}\\
& 
\parbox{12cm}{
  \begin{itemize}
  \item participating in the senior management group of the Division and assuming the leadership role on ad hoc task forces or projects as required
  \item making recommendations to the Director of the Documentation Division on policy, administrative and operational matters of the Service, including monitoring and highlighting technological advances that could facilitate the work of the Service/Division 
  \item serving as Officer-in-Charge of the Division when required\item coordinating long-term meeting coverage with other précis-writing Services   
  \item preparing reports on all aspects of the Service
  \end{itemize}
}\\
\multicolumn{2}{l}{2. Staffing matters, including \isi{recruitment} and performance appraisal}\\
& 
\parbox{12cm}{
  \begin{itemize}
  \item  making long and short-term projections of the work of the Service and its staffing requirements
  \item supervising and monitoring the performance of all staff in the Service and preparing e-Performance reports as First Reporting Officer for the staff directly reporting to the Chiefs and acts as Second Reporting Officer accordingly 
  \item screening applications from and evaluating potential freelance staff and contractors, and preparing requests for contractual translation or the \isi{recruitment} of temporary assistance as needed
  \item making recommendations on such personnel actions as \isi{recruitment}, renewal of contracts, transfers, assignments and promotions
  \item selecting papers for examinations, marking scripts, establishing pass lists, participating in examination boards and interviews to ensure appropriate \isi{recruitment}
  \item organizing training
  \end{itemize}
}\\
\multicolumn{2}{l}{3. Translation workflow coordination}\\
& 
\parbox{12cm}{
  \begin{itemize}
  \item  managing the staff and work programme of the Service to ensure the timely issuance in [language] of documents 
  \item organizing all activities of the Service to ensure maximum \isi{efficiency} and \isi{cost-effectiveness}
  \item formulating and developing guidelines, instructions and priorities governing translation, revision, précis writing and \isi{terminology} work of the Service \item preparing internal information notes on work procedures
  \end{itemize}
}\\
\multicolumn{2}{l}{4. Contribution to translation, technical and \isi{quality control} tasks}\\
& 
\parbox{12cm}{
  \begin{itemize}
  \item  carrying out \isi{quality control} checks for work done in-house and by outside contractors
  \item serving as the final arbiter on all technical problems connected with the language of the Service
  \item translating and/or revising particularly important, sensitive or confidential texts, as required
  \end{itemize}
}\\
\lspbottomrule
\end{tabularx}
\caption{Duties of mid-level ITSM in representative vacancy notice (large service)}
\label{tab:prietoramos:1}
\end{table}


The same exercise can be applied to other descriptors. The list of duties in each category will be similar or shorter depending on profile variations and the level of detail provided. For instance, strategic and \isi{human resources} matters in the above vacancy notice contrast with the short reference to planning, training and guidance in the equivalent vacancy notice at a much smaller agency, the International Atomic Energy Agency, where ITSMs at unit level devote more time to translation and revision tasks (see \tabref{tab:prieto:2} below).


\begin{table}
\footnotesize
\begin{tabularx}{\textwidth}{Q}
 \lsptoprule
 \parbox{.9\textwidth}{

\begin{itemize} 
  \item 
  Plan, supervise and monitor the work of the Section and provide training and guidance.
  \item 
  Take authoritative decisions regarding \isi{terminology}, style and usage in [language].
  \item 
  Revise the translations done by other members of the Section.
  \item 
  Translate and self-revise texts mainly from English into [language] covering a range of scientific, technical, administrative and legal subjects ensuring that translations are equivalent in meaning and style to the original texts.
  \end{itemize} 
 }\\
 \lspbottomrule
\end{tabularx}
\caption{Duties of mid-level ITSM in representative vacancy notice (medium-size service)}
\label{tab:prieto:2}
\end{table}


In the same vacancy notice, the role of the section head is previously presented as “a translator and a reviser who is responsible for the overall quality and timeliness” of translations, and “a manager planning and monitoring the work and supervising the staff of the Section”; and the “main purpose” of the position is: “To plan, coordinate and supervise the translation/revision activities of the Section, ensuring the high quality and timely delivery of texts translated into [language] for distribution to Member States and/or members of the Secretariat.”



As to directors of entire translation services (or senior ITSMs), the most common functions listed generally fall under category 1 above (strategic, administrative and financial matters) but at a higher level of responsibility. The more managerial profile of these positions is also reflected in the discourse used in the job descriptors. They all include the following core duties: \isi{coordination} of the various component sections or units, strategic planning and leadership, liaison with other departments and external representation. Most organizations surveyed (except for \isi{EU} institutions) group together all language service departments, including interpreting and other documentation services, under the same management line. The example below (\tabref{tab:prieto:3}), a notice published by the WTO in 2016, illustrates the duties of senior ITSMs who manage translation services and other sections within the same division. 

\begin{table}[t]	
\footnotesize
\begin{tabularx}{\textwidth}{Q}
 \lsptoprule
 \parbox{.9\textwidth}{

\begin{enumerate}
\item  Manage the operations of the Division to provide language, documentation and information management services in alignment with the needs of the Secretariat and Members. To this end refine and implement a Divisional strategy and relevant policies as necessary.\\ 
\item  Achieve annual service targets and ensure the Division performs within budget (ca. 28 million CHF per annum) obtaining \isi{efficiency} gains and bringing costs into line with international standards and market considerations.\\ 
\item  Lead, motivate and provide guidance to line managers. Establish their responsibilities and performance objectives for his/her direct reports, provide feedback on their performance and implement any changes required. Build the divisional team, developing their potential ensuring they are trained as required and providing leadership by setting standards.\\ 
\item  Establish a culture of continuous improvement in the Division and work to refine policies, systems and processes so as to maximise quality, \isi{efficiency} and value for money. \\ 
\item  Ensure collaboration with other Divisions and provide contributions to their work as necessary. \\ 
\item  Represent the WTO Secretariat in various events in Geneva and abroad that address matters related to the work of the Division, including speaking engagements.\\ 
\item  Support the Director-General and Senior Management by providing reports, briefings and other information and advice as required.
\end{enumerate}
}\\
\lspbottomrule
\end{tabularx} 
\caption{Duties of senior ITSM in representative vacancy notice}
\label{tab:prieto:3}
\end{table}


A comparison between this notice and the wording used on the same position six years earlier is indicative of recent management trends in the field, with more explicit references to notions of \isi{cost-effectiveness} and streamlining of practices. For instance, “work to \textit{refine policies, systems and processes} so as to \textit{maximise quality, \isi{efficiency} and value for money}” (our emphasis) in point 4 replaces the following wording in the 2010 notice: “Manage and ensure \textit{continuous improvement / modernization of the operations} in all areas of the Division, \textit{ensuring a high level of \isi{efficiency}, service orientation and quality}” (our emphasis).



With regard to quality, as in the above examples, the other job descriptions of mid-level and senior ITSMs (except for two in the second group) refer to \isi{translation quality} in formulations such as “ensure quality / \isi{quality control}”, “meet required quality standards” and “set standards for \isi{translation quality assurance}”. Overall, these formulations tend to refer to overarching policy aspects in the case of senior ITSMs, as opposed to more technical implementation and monitoring aspects in the case of mid-level ITSMs. As managers responsible for everyday decisions on \isi{quality assurance} in their respective languages, the latter will be the focus of the next section. 



\section{Translation unit heads and TQA: practices and challenges}\label{sec:prietoramos:3}



What lies behind the descriptors of ITSMs’ duties in terms of practical \isi{TQA}? To what extent may they have a real impact on \isi{translation quality}? In order to complete the overview of \isi{TQA} operations among ITSMs, we will examine the reported practices of those in charge of monitoring \isi{TQA} implementation by language, i.e. language unit heads or mid-level ITSMs, at three representative institutional settings: the \isi{EU}, the UN and the WTO. For the sake of comparability, heads of unit of three \isi{official languages} common to the three settings (English, French and Spanish) were interviewed,\footnote{For logistical reasons, one of them was replaced \textit{ad interim}.} up to a total of 24 language unit heads. The primary aim was to provide a snapshot of common practices and challenges on the basis of data compiled through open-ended questions in structured interviews on working procedures, with a focus on \isi{quality assurance}.


Quality managers and advisers (who support the \isi{coordination} of quality matters at the \isi{EU} institutions) and service directors were also interviewed in the context of a larger project on \isi{institutional translation} quality.\footnote{“Legal Translation in International Institutional Settings: Scope, Strategies and Quality Markers (LETRINT)”.} The replies of these professionals were very useful: (1) to triangulate and better understand the information on institutional structures and management duties presented in the previous section; (2) to confirm that strategic and policy aspects of \isi{TQA} are generally perceived by service directors as the collective responsibility of translation units in each language, with the peculiarity of more visible strategic support and \isi{coordination} by quality advisers in the case of the \isi{EU} institutions; and (3) to corroborate the relevance of focusing on language unit heads as the most comparable and comprehensive decision-making profile among the institutions in terms of everyday \isi{TQA} implementation. In this respect, it must also be noted that this section reports on their practices and perceptions, but does not purport to assess \isi{TQA} approaches, which falls outside the scope of the present study.  

In analyzing TQA-related activities of mid-level ITSMs, the first important commonality is that they all bear the \textit{responsibility} and are accountable for the quality of translation in their respective target languages, and they are all conditioned by institutional goals and working procedures. They are not bound by any shared international standard, as \citeauthor{ISO2015} would require a policy of systematic full revision that does not apply to their services. Many managers were not acquainted with this new standard, which does not actually describe the tasks and qualification requirements of service managers themselves. However, \citeauthor{ISO2015} on “Translation services — Requirements for translation services” establishes the “actions necessary for the delivery of a quality \isi{translation service}” by TSPs, including provisions “concerning the management of core processes, minimum qualification requirements, the availability and management of resources, and other actions” \citep[vi]{ISO2015}. It is therefore implied that the role of managers of TSPs is to make sure that the specified measures are taken with a view to delivering a quality product. It also follows that \citeauthor{ISO2015} provides a useful yardstick to compare \isi{TQA} tasks in institutional settings and to employ standardized \isi{terminology} in the field. In fact, this international standard is explicitly quoted as a key source in the quality \textit{guidelines} of one \isi{EU} institution in particular.\footnote{In connection with general quality criteria, \textit{\isi{DGT} Translation Quality Guidelines} establish that: “All translated texts should comply with the general principles and \isi{quality requirements} for professional translation laid down in the international standard \isi{ISO} \isi{17100}” \citep[3]{DGT2015a}. The criteria contained in \isi{ISO} provision 5.3.1 \citep[10]{ISO2015} are then listed in a footnote: 
\begin{quote}
a) compliance with specific domain and client \isi{terminology} and/or any other reference material provided and ensuring terminological \isi{consistency} during translation; b) semantic \isi{accuracy} of the target language content; c) appropriate syntax, spelling, punctuation, diacritical marks, and other orthographical conventions of the target language; d) lexical cohesion and phraseology; e) compliance with any proprietary and/or client \isi{style guide} (including domain, language register, and language variants); f) locale and any applicable standards; g) formatting; h) target audience and purpose of the target language content.
\end{quote}
}


The shift towards more explicit principles and frameworks was also confirmed at other \isi{EU} institutions, while it was not felt as a compelling necessity in intergovernmental organizations with fewer languages. In these cases, the core \isi{TQA} principles were primarily based on well-established practices as gradually inherited and adapted to changing needs. Whether driven by custom or formal guidelines, the \textit{discretionary margin} of all language unit heads was conditioned by shared core principles. Within this margin, \textit{implementation variations per language} were noticeable in all institutions, although they were almost negligible in the service with the smallest number of languages. These variations reflected not only different unit backgrounds and working approaches, but also, at times, more pro-active managerial attitudes towards \isi{TQA} procedures and supporting tools.



\begin{table}[p]

\caption{\label{tab:prietoramos:4} TQA-related tasks of mid-level ITSMs}
\footnotesize
\begin{tabularx}{\textwidth}{Q}
\lsptoprule 

 \parbox{.9\textwidth}{
\textbf{I. PRODUCTION WORKFLOW SUPERVISION} 



\textbf{a) Overall planning and coordination} 


\begin{itemize}
\item 
Planning and team \isi{coordination} according to budget targets, deadlines, priorities and available \isi{human resources}; contingency plans and problem-solving in case of unexpected changes in programme or project implementation.
\item 
Communication with requesting units, other translation units, support units (such as technology, \isi{terminology} or documentation) or any other relevant actors where appropriate to meet targets.
\end{itemize}

\textbf{b) Job categorization and assignment} 


\begin{itemize}
\item 
Translation brief processing and, where appropriate, clarification of specifications or discussion of conditions with requesting unit.
\item 
Content profiling and risk assessment: text categorization according to \isi{text type} within institutional hierarchy, subject matters, sensitivity and confidentiality; definition of expected level of quality; potential impact of lower-than-expected quality.
\item 
Assignment of job to translator: best possible match between text and translator profile (specialization, in-house / external, speed, etc.) depending on availability of \isi{human resources}, risk assessment and time constraints.
\item 
Assignment of job to \isi{quality controller}\footnote{This term is used in a broad sense to refer to the person who conducts any \isi{quality control} task (see e.g. \citealt{Mossop2014}: 116), regardless of the institutional category held by the person. In \isi{EU} institutions, for example, revision practices are not as hierarchically organized as in the other two selected settings, and in-house translators are usually expected to translate and revise from their very entry into the position.} (except for cases of self-revised translation): type and level of \isi{quality control} (full \isi{bilingual revision}, monolingual target text checks or review, etc.) considering the two previous points; best possible match with \isi{quality controller} profile depending on \isi{quality control} needs, availability of \isi{human resources} and time constraints.
\end{itemize}

\textbf{c) Monitoring of \isi{quality assessment} and handling of feedback}


\begin{itemize}
\item 
Monitoring of compliance with \isi{quality assessment} procedures: mandatory for external translators (criteria generally harmonized for all units); more or less systematic and detailed depending on language unit and translator rank or seniority in the case of in-house translators.
\item 
Supervision of ex-post quality checks at unit level.
\item 
Processing of feedback on production and satisfaction survey results.
\end{itemize}
}\\
\midrule
\end{tabularx}
\end{table}

\begin{table}[t]
\footnotesize
\begin{tabularx}{\textwidth}{Q}
\midrule

 \parbox{.9\textwidth}{
\textbf{II. CONTRIBUTION TO TRANSLATION, QUALITY CONTROL OR ASSESSMENT TASKS} 


\begin{itemize}
\item 
Contribution to \isi{quality control} or, less often, translation tasks in specific projects (e.g. texts of highest importance).
\item 
Regular or random assessment of translation or \isi{quality control} output.
\item 
Advice and arbitration on linguistic issues where appropriate (e.g. cases of internal disagreement or sensitive institutional \isi{terminology}).
\end{itemize}

\textbf{III. COMPETENCE MANAGEMENT}


\begin{itemize}
\item 
Participation in \isi{recruitment} tests and selection of applicants. 
\item 
Individual goal-setting and periodic performance appraisal exercises with members of staff.
\item 
Training initiatives in line with team and individual competence development needs.
\item 
Mentoring \isi{coordination} for junior staff members.
\end{itemize}

\textbf{IV. RESOURCE DEVELOPMENT AND CONTRIBUTION TO INSTITUTIONAL \isi{TQA} POLICY}


\begin{itemize}
\item 
Resource needs analysis and monitoring of language-specific resource development, including cooperation with \isi{terminology} units or focal points.
\item 
Transmission of feedback on tools and resources for potential improvements at a broader level. 
\item 
Monitoring of impact of management measures and reporting on unit level quality developments.
\item 
Contribution to \isi{TQA} policy formulation or implementation innovations that may be exported to other units.
\end{itemize} 
}\\
\lspbottomrule
\end{tabularx}
\end{table}


\largerpage

Mid-level ITSMs were involved in the following \textit{TQA-related tasks} to varying degrees, depending on structures and workload fluctuations. \tabref{tab:prietoramos:4} below classifies the data compiled during the interviews in an attempt to draw an up-to-date structured inventory.
While some of the tasks can be facilitated by computer applications (e.g. translation management tools), partially delegated within a language unit (e.g. follow-up of quality checks, job assignments or specific project management) or centralized by a separate dedicated unit (e.g. contracting operations), language unit managers are ultimately the orchestra conductors with a full overview of, and the capacity to influence, the different components of \isi{TQA} in each language in which quality is measured. As evidenced by a comparison between the inventory below (\tabref{tab:prietoramos:4}) and \tabref{tab:prietoramos:1}, most of the responsibilities listed in their job descriptors actually have a \isi{TQA} dimension. From this perspective, mid-level ITSMs play a central role in promoting quality, especially as \textit{process and competence managers}. In the case of regular contributions to \isi{quality control} as revisers (category II tasks), particularly common in smaller services, the short-term impact on the \isi{translation product} can be more direct. However, the \textit{long-term impact} of ITSMs on \isi{translation product} quality as \isi{TQA} managers  crucially depends on the cumulative effect of the other three elements of a virtuous circle: (1) \isi{competence management} (selection of talent and professional development actions under category III); (2) workflow supervision (category I tasks) with a view to maximizing the benefits of competence assets in the production process; and (3)~input into procedural and material conditions that may improve workflow and performance (category IV actions).



\newpage 
The degree of risk in job assignment is directly related to the degree of suitability and reliability of team members, so administering talent and keeping motivation high, despite productivity pressures, emerges as a key requirement for effective \isi{TQA} management in hierarchical institutional structures. By the same token, adapting selection processes to competence needs can only be a sound quality enhancement investment. 



Unsurprisingly, any measures that introduce uncertainty in job \isi{allocation} equations or other links of the production chain feature as the most common concern among mid-level ITSMs with regard to \isi{TQA}. The main \textbf{challenges} identified revolve around three interrelated sources of uncertainty:


\begin{itemize}
\item 
\textit{Resource availability and productivity pressures} as a result of streamlining or downsizing: translation services are regularly in the spotlight given the proportion of institutional budgets they represent; as in any other public or private sector, it is generally felt that insufficient resources could lead to lower-than-expected quality outcomes if the limits of \isi{cost-effectiveness} are exceeded. 
\end{itemize}
\begin{itemize}
\item 
\textit{External contracting conditions}: in light of the increasing level of \isi{outsourced translation}, it has become paramount to build professional relations of trust and promote quality among external translators; in institutions where outsourcing procedures prevent language units from matching job specifications to individual translator profiles, and jobs are assigned by \isi{external contractors}, the translator’s reliability is generally unknown by ITSMs, so the risk assessment and production chain can be affected as a result (including greater unpredictability of \isi{quality control} needs, impact on motivation and human \isi{resource allocation} in cases of lower-than-expected quality of \isi{outsourced translation}, and final cost/quality ratio). 
\end{itemize}
\begin{itemize}
\item 
\textit{Workflow changes and expectations deriving from technological developments}: the integration of new tools is perceived as globally positive in supporting and speeding up certain tasks, but has also brought new variables and dependencies into the workflow, as well as new \isi{error} patterns in the production process; in this connection, heightened expectations of the benefits of \isi{machine translation} and over-simplification of badly-needed human \isi{translation expertise} represent an added challenge in the context of cost-saving initiatives. 
\end{itemize}

Other challenges fall outside the decision-making scope of language units (in particular, the persistent issue of poor quality of originals) or were specific to certain units, for example, the complexity of managing a high number of language pairs and finding available translators in some of them, or insufficient adaptation of initial screening exercises to professional translation profiles in some \isi{recruitment} processes. 



\section{Concluding remarks: expertise for evolving TQA management}\label{sec:prietoramos:4}


The insights gathered in previous sections point to the skills expected of ITSMs at language unit level from a holistic \isi{TQA} perspective. \citeauthor{ISO2015} does not refer to the profile of service managers but to \isi{TSP} project managers, who are required to have “appropriate translation project \isi{management competence}” \citep[7]{ISO2015}. As outlined above, mid-level ITSMs are crucial decision-makers in sustained \isi{TQA} endeavours beyond translation project level. Their services can be equated to those of \isi{TSP} branches in the private sector, but conditioned by specific institutional goals and conventions. 

In their capacity as competence and process managers, an optimal combination of \isi{translation expertise} and managerial skills would be expected of these mid-level ITSMs, particularly in connection with the textual and extra-textual parameters considered in \isi{resource allocation}. As noted by \citet[275]{Gouadec2010}, \isi{quality management} systems require “a very clear view of “product” quality grades and levels, overall, per domain, per parameter, and per sub-parameters”. In international institutional settings, this relates to established text categories, legal hierarchies and priority policy matters that constitute essential knowledge for \isi{TQA} managers. As team leaders in the provision of and advocacy for \isi{translation quality} in the relevant language, they would also be expected to have advanced translation and revision skills, as well as sufficient expertise in \isi{TQA} processes (including approaches to \isi{quality control} and \isi{evaluation}), \isi{translation competence} development, and translation tools and resources. The ideal \isi{ITSM} profile also demands general management skills such as planning, \isi{coordination}, risk management and networking abilities.

The appropriate balance between translation and management skills will depend on the degree to which translation and revision work or other routine operations are delegated from ITSMs in each organization. What seems clear in all the settings analyzed is that the shift from one-fits-all \isi{quality control} to a more modulated \isi{approach to quality} variables has made ITSMs’ role more critical and influential in \isi{TQA}. In light of the growing prominence of external and machine assisted translation, it is also clear that \isi{TQA} management functions and their impact deserve further attention. As the translation landscape continues to evolve rapidly, the sophistication and added value of this profile will certainly evolve as well. To the extent that international organizations renew their commitment to quality multilingual communication, \isi{TQA} managers are called to play a central role in promoting effective solutions at the intersection between top-down processes of policy implementation and bottom-up input for quality enhancement.  


\section*{Acknowledgements} 
I would like to thank all institutional partners for their valuable cooperation, as well as the Swiss National Science Foundation for its support through a Consolidator Grant.
  
  
\sloppy
\printbibliography[heading=subbibliography,notkeyword=this] 
\end{document}