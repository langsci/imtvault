\documentclass[output=paper]{langsci/langscibook} 
\ChapterDOI{10.5281/zenodo.1048192} %will be filled in at production

\author{Karolina Stefaniak\affiliation{Directorate-General for Translation, European Commission}}
\title{Terminology work in the European Commission: Ensuring high-quality translation in a multilingual environment}
\shorttitlerunninghead{Terminology work in the European Commission}
\abstract{Terminology is an integral part of every translation process, necessary to achieve high-quality translation. In the case of EU law, terminology is additionally a matter of legal certainty and clarity. Terminological errors may lead to citizens and companies misunderstanding their rights and obligations, make the harmonization of laws between Member States more difficult and often result in legal disputes at national or EU level, thus tarnishing the image of the European Union and its institutions. This is why EU language services place great emphasis on terminology work and on integrating terminology in their translation process. The aim of terminology work is, firstly, to give translators timely terminological support: to find a correct equivalent, to clear the meaning of a concept, to coin a brand new term or to help them choose the right equivalent in a given context, out of many equally correct terms, based on the criteria of consistency, accuracy and clarity. Secondly, the aim of terminology work is to manage the existing terminology resources. This work is both of a descriptive and prescriptive nature and the central hub for EU terminology is the multilingual termbase IATE, jointly managed by several EU institutions and accessible also to the general public. This chapter describes how the terminology work is done in the Directorate-General for Translation of the European Commission on the example of the practices in the Polish Language Department.
}
\maketitle

\begin{document}

\section{Introduction}\label{sec:stefaniak:1} 
  

Almost every \isi{EU} institution has its own \isi{translation service} and almost all of these services also do \isi{terminology} work,\footnote{Terminology work is “work concerned with the systematic collection, description, processing and presentation of concepts and their designations”. A concept is a “unit of knowledge created by a unique combination” of properties common to a set of objects. A designation is a “representation of a concept by a sign which denotes it”; designations can be symbols, appellations or terms (cf. \citeauthor{ISO2000}). Part of \isi{terminology} work is \isi{terminology} management, which is concerned with the recording and presentation of data.} alongside translation, organized both on the central level, in \isi{terminology} \isi{coordination} units, and on the local level, in language departments. Terminology \isi{coordination} units cooperate with each other on the interinstitutional level and deal mostly with multilingual projects. Language departments, on the other hand, work on bilingual projects. Everybody involved in \isi{terminology} work follows the same general good practices, however, actual \isi{terminology} work may look slightly different in each institution and in each department, depending on the needs of the \isi{translation service} in question and its resources.

Terminology work is carried out in tight connection with the texts being translated. On the one hand, it consists in solving ongoing \isi{terminology} problems, which are lodged in by translators when they are working on a text. On the other hand, systematic \isi{terminology} work is conducted, which consists in collecting and processing terms from the text to be translated, if possible, even before the translator starts working on it. The extraction of terms from the source text and the clarification of the concepts are usually done by the central \isi{terminology} unit. When their part is ready, language departments start working on equivalents of extracted terms in their languages. Less often, \isi{terminology} work consists of developing a conceptual system for a given domain; such projects usually aim at deleting duplicates from the termbases and consolidating the existing entries, and they are conducted in cooperation with other interested \isi{EU} institutions.

The main source of \isi{terminology} for all \isi{EU} institutions is IATE\footnote{\url{iate.europa.eu}} – InterActive Terminology for Europe – a multilingual \isi{terminology database} launched in 2004. Before that \isi{EU} \isi{terminology} was created, collected, stored and managed separately by various institutions in a few collections and termbases. These resources were later imported to IATE. There are at present about 8.6 million terms in IATE, distributed through approximately 1.4 million entries, in 24 official \isi{EU} languages (and in some other languages like Russian, Chinese, Arabic and also Latin).\footnote{Data from May 2016.} The database is concept-oriented, i.e. one entry corresponds to one concept. It is jointly administered by the \isi{European Commission}, the European Parliament, the Council of the \isi{EU}, the Court of Justice, the European Court of Auditors, the Committee of the Regions, the European Economic and Social Committee, the European Central Bank, the European Investment Bank and the Translation Centre for the Bodies of the \isi{European Union}. IATE was made available to the public in 2007.

The main reason for the existence of IATE is to facilitate multilingual \isi{drafting} and translating of \isi{EU} legal texts. This is why the database contains not only terms in the narrow sense, but also quasi-terms, proper names, abbreviations, titles and phrases that repeatedly occur or could occur in \isi{EU} texts and which should be uniformly used and translated. The amount of information entered in IATE should be sufficient for an unequivocal identification of the concept in question and should have added value in respect to information commonly available in other sources. Owing to the way IATE was created, the quality of the entries still varies substantially, from very well edited entries in all languages, to entries completed only by few language departments and containing very little information. Entries containing at least one term, a reliable source for that term and a definition are considered high-quality entries (when there is no definition, at least the context for the usage of the term should be given for the entry to be considered high-quality). Thus, \isi{terminology} work also consists in constant monitoring of the use of terms and the subsequent updating, correcting or completing of the relevant IATE entries. Terms, like all other components of the specialized language, evolve, get accepted or rejected, change their conceptual scope, go out of usage or become marked. All these processes need to be properly registered.

\section{Terminology work in the Directorate-General for Translation (DGT) of the European Commission}\label{sec:stefaniak:2}

The aim of \isi{DGT} it to supply the Commission services with high-quality translations, whereby quality is understood as the degree to which translation corresponds to the expressed or implied expectations of the recipients (cf. \citeauthor{ISO2015b}). Terminology is considered to be a key element of translation, without which high quality cannot be achieved. The process of translation in \isi{DGT} is a team effort with translation and terminological decisions often being taken collectively. The simplest scenario involves two translators — the first translating the text and the second revising it. However, if any of them has questions or doubts concerning \isi{terminology}, he or she contacts the department \isi{terminologist}, who assists them in finding answers to their questions and suggests solutions.

Requests from translators usually concern assistance in finding an equivalent for a challenging term or in coining a new term. The \isi{terminologist} begins by defining the concept to which the term refers in the source language. Firstly, he or she analyzes the source text, then broadens the analysis to other \isi{EU} texts and then to other specialized texts in a given domain. If necessary, the \isi{terminologist} may contact the persons responsible for the text and ask them for clarification. If the term is part of a bigger, central project, it is the central \isi{terminology} unit that works out the concepts and supplements IATE with terms and definitions in the source language. On the basis of these definitions, the terminologists in the language departments can look for or create terms in their respective languages. When looking for equivalents of a term, the \isi{terminologist} consults numerous bilingual sources, paper and electronic ones, chooses a possible equivalent in view of the reliability of such sources and checks the occurrence of this equivalent in various monolingual sources, over and over again, until he or she is finally happy with the result, which is then entered into IATE.

\largerpage
The majority of translators’ questions\footnote{Approximately 90 per cent according to the internal statistics of \isi{DGT}’s Polish Language Department.} concern scientific terms from the domains under regulation. For example, the translator needs a Polish equivalent of the term \textit{melting furnace}. The term comes from a Commission proposal establishing the Best Available Techniques (BAT) for non-ferrous metals. The definition for the English term is already available in IATE. The English-Polish Dictionary of Science and Technology \citep[692]{Berger2004} suggests the equivalent \textit{piec do topienia} [literally: furnance for melting]. Generally, dictionaries may be regarded as reliable sources; however, each source has to be treated critically and also dictionary terms are verified for their \isi{accuracy}, adequacy and usage. Therefore, a simple internet query for the term \textit{piec do topienia} is performed and it does not yield desirable results, i.e. no reliable texts where the term would be used in context are found. Some results are obtained only when the query is restricted to sites in the Polish language only and to the domain of metallurgy. One of the results of such a modified search is an article in a scientific journal, the title of which contains the term in question. The article is in Polish, but it contains an English version of the title with a slightly different term \textit{piec topielny} [melting-ADJ furnance]. An internet query is performed again for the new term. It confirms the use of this term in specialized texts on reliable sites. Finally, the term \textit{piec topielny} is entered into IATE.

Such queries do not always yield expected results. The \isi{European Commission} often regulates domains that are very specialized or novel. As a result, it happens that the \isi{terminology} needed does not yet exist in the national languages or is not commonly spread or used by experts, who might prefer to communicate in English. Moreover, the available sources may be scarce or not reliable. This characteristics of \isi{EU} terms may be illustrated with financial terms, such as \textit{front running}, \textit{capped notional value} or \textit{LTROs}. In such a case the translator or the \isi{terminologist} suggest possible equivalents according to their best knowledge and these suggestions are then consulted with national experts. Most language departments maintain contacts either with national administration of their Member State, or with experts in the Permanent Representation in Brussels, or even have direct contacts with specialists from various fields. When there is no time or possibility for such consultations, a \isi{descriptive equivalent} is often used or a word-for-word (literal) equivalent, with the English term provided in brackets for extra \isi{clarity} (the so-called translation couplet (cf. \citealt{Newmark1981}: 32)). Such solutions are also often suggested by experts.

\textbf{Literal translation} is often criticized by text recipients in Member States. However, literal equivalents are a conscious technique used on purpose to minimize the risk of misinterpretation and to ensure \isi{consistency} between all language versions. This concerns especially the so-called category A texts, where the lack of \isi{consistency} may have legal consequences, i.e. \isi{EU} legal documents; documents used in administrative or legal proceedings and inquiries, such as infringements or anti-dumping cases; documents for procurement or funding programmes, tenders, grants applications or contracts; as well as \isi{recruitment} notices, EPSO (European Personnel Selection Office) competition notices and EPSO test documents \citep{DGT2017}. For example, the term \textit{vehicular language}, which might have been translated as \textit{język roboczy} [working language], \textit{język wspólny} [joint language] or \textit{język uniwersalny} [universal language] in other contexts, was translated $-$ upon the advice of the lawyer-linguists $-$ literally, as \textit{język wehikularny} in an EPSO competition notice, to avoid doubts as to which concept does this term refer to and to ensure \isi{consistency} between all the language versions of the notice. 

In particular \textbf{ambiguous terms} are most often translated literally. Word-for-word equivalents minimize the risk of future translation difficulties if a similar term were to appear to denote a different concept or if two terms that were synonymous at first had to be differentiated. For instance, the term \textit{cross-zonal capacity} was first\footnote{Commission Regulation (\isi{EU}) 2015/1222 of 24 July 2015 establishing a guideline on \isi{capacity allocation} and congestion management (OJ L 197, 25.7.2015, p. 24–72).} translated as \textit{transgraniczne} \textit{zdolności przesyłowe} [literally: transborder transfer capacity], because most of the bidding zones in Europe corresponded to the Members States’ borders and the term was used interchangeably with the term \textit{cross-border capacity}. It seemed a reasonable solution at that time; however, shortly afterwards, as a result of another act regulating the electricity market,\footnote{Commission Regulation (\isi{EU}) 2016/1719 of 26 September 2016 establishing a guideline on forward \isi{capacity allocation} (OJ L 259, 27.9.2016, p. 42–68).} the responsible Directorate-General requested to keep the two terms apart. In consequence, the equivalent \textit{transgraniczne zdolności przesyłowe} had to be changed by way of a \isi{corrigendum}.\footnote{A \isi{corrigendum} is a legal act, the purpose of which is to realign the published legislative text with the legislative body’s original intent by removing obvious mistakes that occurred in the \isi{drafting} and publication process \citep[950]{Bobek2009}.} This could have been avoided if a more direct equivalent (\textit{międzyobszarowe zdolności przesyłowe}) had been used from the very beginning.

Difficulties of another kind result from situations where a \isi{scientific term} in the legal act to be translated is given a different meaning than it has in scientific discourse. This was the case with the terms \textit{clinical study} and \textit{clinical trial}. These two English terms refer to the same concept in medical texts and the equivalents \textit{badanie kliniczne} or \textit{próba kliniczna} can be used in Polish. However, Regulation 536/2014\footnote{Regulation (\isi{EU}) No 536/2014 of the European Parliament and of the Council of 16 April 2014 on clinical trials on medicinal products for human use, and repealing Directive 2001/20/EC (OJ L 158, 27.5.2014, p. 1$-$76).} started to use them as two distinct concepts, with \textit{clinical trial} being defined as a category of \textit{clinical study}. This made it necessary to distinguish these concepts in Polish by using terms that would take this difference into account to the greatest extent possible, while also taking into consideration the established and recognized Polish \isi{terminology} in the field of clinical trials. After internal consultations the term \textit{badanie kliniczne} [literally: clinical study] was kept as the equivalent of \textit{clinical trial}, to maintain \isi{consistency} with the \isi{terminology} already established in Directive 2001/20/EC.\footnote{Directive 2001/20/EC of the European Parliament and of the Council of 4 April 2001 on the approximation of the laws, regulations and administrative provisions of the Member States relating to the implementation of good clinical practice in the conduct of clinical trials on medicinal products for human use (OJ L 121, 1.5.2001, p. 34$-$44, special edition in Polish: Chapter 13 Volume 026 P. 299$-$309).} The term \textit{badanie biomedyczne} [biomedical study] was used as the equivalent of \textit{clinical study}. This term is not used in Polish law; hence, it was “empty” and it was possible to use it to denote the new concept.

\textbf{Strictly legal terms} do not occur in Commission texts that often, but when they do, they pose much greater difficulties than technical or scientific terms. The interpretation of legal terms, which are expressed in natural language, always has to take into account the legal system in which they are used. This means that identically sounding terms belonging to different legal systems might have different meaning (e.g. terms in English in the English and American legal systems; terms in French in the French and Canadian legal systems; terms in German in the German and Austrian legal systems, etc.). The \isi{European Union} also has its own specific legal system, although it does not have its own language and hence it has to “borrow” its legal \isi{terminology} from the legal languages of the Member States (cf. \citealt[79, 80]{Kjær2007}; \citealt{Robertson2010}: 154). Still, it needs to be stressed that the concept system of \isi{EU} law is distinct from that of the Member States because the \isi{EU} legal system is distinct from the legal systems of \isi{EU} Member States (cf. Case 282/81 \textit{Srl CLIFIT and Lanificio di Gavardo SpA v Ministry of Health}). This distinction is the basis for the uniform application of \isi{EU} law in all Member States and has to be accounted for in translation appropriately. 

When translating from English (or more rarely French or other \isi{EU} languages), the translator cannot be blinded by the meaning of the term in question in the English (or French) legal system, and during the search for equivalents, he or she has to be particularly cautious when borrowing terms from the national law.\footnote{Cf. e.g. point 5.3.2 of the \textit{Joint practical guide of the European Parliament, the Council and the Commission}: “As regards legal \isi{terminology}, terms which are too closely linked to a particular national legal system should be avoided.” \citep[18]{EU2015}.} This does not mean that using functional equivalents is not practiced; otherwise all translated terms would have to be neologisms. In particular, when the term is defined or when the context clearly points out to its “European” character, the \isi{functional equivalent} may be good enough. For instance, the term \textit{corruption} is translated simply as \textit{korupcja}, although there is no single understanding of this concept that is common in all Members States \citep{Szulik2012}. On the other hand, the translation of the term \textit{identity card} with \textit{dowód osobisty}, coming from the Polish Act on ID Cards,\footnote{Ustawa z dnia 6 sierpnia 2010 r. o dowodach osobistych (Dz.U. 2010 nr 167 poz. 1131).} would probably associate this term too much with the Polish legal system. In consequence, a less marked term \textit{dowód tożsamości} is used.

\largerpage%long distance
Legal terms may be very specific and thus easily distinguishable, or they may be homonyms of everyday words that also have a specific meaning in the legal language (e.g. \textit{goods}). Translators, who are for the most part not lawyers, may not be able to identify such terms properly; besides, not being experts, they have a tendency to use words in translation that are everyday equivalents of legal terms (e.g. \textit{adopcja} instead of \textit{przysposobienie} [adoption], \citealt[273]{Biel2014}). A \isi{legal term} may also be politically sensitive; in such a case, its equivalent may change whenever there is a change in the \isi{EU} policy. Therefore, when the term \textit{illegal immigrant} was changed to \textit{irregular immigrant}, as the former started to be considered as stigmatizing, its translation in the Polish language versions of \isi{EU} legislation also had to change from \textit{nielegalny imigrant} to {\textit{imigrant o nieuregulowanym statusie}}.

\section{Criteria of translation choices}\label{sec:stefaniak:3}
\largerpage
Translation is a decision-making process (cf. e.g.: \citealt{Levý1967}). In the \isi{EU} context, translators and terminologists make their choices based on three basic criteria: \isi{consistency}, \isi{accuracy} and \isi{clarity}. Consistency refers to the lack of terminological discrepancies, \isi{accuracy} means using correct and precise terms in a given context, and \isi{clarity} is the degree to which the translation is understandable and fluent.

Above all, the target text has to be internally \textbf{consistent}. Consistency applies not only to \isi{terminology}, but also to recurrent sentences and phrases; however, the consequence of the lack of terminological \isi{consistency} tends to be much more serious. Various translations of the same term, especially in legal acts, may mislead the reader to think that these terms denote different concepts and make it difficult to interpret legislation. For the same reason translation has to be consistent with other \isi{EU} legal acts, so that there is \isi{consistency} within the \isi{EU} legal order. Thus, the \isi{terminology} in delegated or implementing acts has to be consistent with the \isi{terminology} in the basic act while the \isi{terminology} in the basic act has in turn to be consistent with the \isi{terminology} in the primary legislation. 

This means that when translating a regulation implementing a directive \isi{consistency} has to be kept with the respective language version of that directive and not with the national legislation transposing it, even though it is the regulation that will be directly applicable in a given Member State. The Solvency II Directive is a good example.\footnote{Directive 2009/138/EC of the European Parliament and of the Council of 25 November 2009 on the taking-up and pursuit of the business of Insurance and Reinsurance (Solvency II) (recast) (OJ L 335, 17.12.2009, p. 1$-$55).} The Polish Act on Insurance and Reinsurance activity,\footnote{Ustawa z dnia 11 września 2015 r. o działalności ubezpieczeniowej i reasekuracyjnej (Dz.U. 2015 poz. 1844).} which transposed the directive to the Polish law, changed or corrected many terms, e.g. \textit{non-life insurance} was changed from \textit{ubezpieczenia inne niż na życie} [literally: insurance other than life insurance] to \textit{pozostałe ubezpieczenia osobowe i ubezpieczenia majątkowe} [literally: other personal insurance and property insurance]. The delegated regulation to this directive\footnote{Commission Delegated Regulation (\isi{EU}) 2015/35 of 10 October 2014 supplementing Directive 2009/138/EC of the European Parliament and of the Council on the taking-up and pursuit of the business of Insurance and Reinsurance (Solvency II) (OJ L 12, 17.1.2015, p. 1–797).} is consistent with the directive; however, it creates discrepancies between the Polish version of the regulation and the Polish transposing act, which are not easy to rectify as not every discrepancy is necessarily an \isi{error}. It demonstrates that it is of utmost importance to ensure that the translation, especially the \isi{terminology}, is right from the very beginning, and to maintain good contacts with national experts at each stage of the \isi{translation process}.

\textbf{Consistency} is often more important than any other criterion. For example, during the translation of the proposal for a Directive on certain aspects concerning contracts for the online and other distance sales of goods\footnote{COM(2015)0635 final.} the translator had doubts about translating the key term – \textit{sales of goods} – as \textit{sprzedaż towarów} [literally: sale of goods]. When consulted, a Polish expert in the requesting Directorate-General suggested to use the term \textit{sprzedaż rzeczy} [literally: sale of things]. The \isi{terminologist} agreed with the expert on the \isi{accuracy} of this equivalent; however, since the explanatory memorandum to the proposed Directive contained an explicit recommendation to keep the \isi{terminology} consistent with the existing \isi{EU} legislation, it was decided in the end to keep the equivalent \textit{sprzedaż towarów}, to maintain \isi{consistency} with Directive 2011/83/\isi{EU},\footnote{Directive 2011/83/\isi{EU} of the European Parliament and of the Council of 25 October 2011 on consumer rights, amending Council Directive 93/13/EEC and Directive 1999/44/EC of the European Parliament and of the Council and repealing Council Directive 85/577/EEC and Directive 97/7/EC of the European Parliament and of the Council (OJ L 304, 22.11.2011, p. 64$-$88).} which the proposed Directive complemented, and where the term \textit{goods} was defined and translated as \textit{towary}.

The \isi{European Union} does not produce 24 legal acts, but just one legal act in 24 language versions \citep[119--120]{Doczekalska2009production}. Therefore all language versions of an \isi{EU} legal act must be consistent also between each other (which is referred to as \textbf{multilingual concordance}). In order to ensure this type of \isi{consistency} to the greatest extent possible, translators cannot interfere with the structure of the source text, e.g. by splitting long sentences or rearranging paragraphs, nor can they correct any factual errors they spot in the source text, such as errors in numbers, even if they are obvious. They are asked to restrict their interpretation of the text to the actual wording of the source text. In the case of ambiguities in the source text, translators try to obtain clarifications from the person responsible for the text; such clarifications are then shared with translators in all other language departments (and other institutions, if necessary). However, ambiguities are very often used on purpose and translators are asked to keep them (cf. \citealt[92–93]{Šarčević1997}). This is yet another reason for resorting to the literal translation technique (see above).

The second key criterion of translation choices is \textbf{accuracy}, both of specialized terms and of the EU-specific \isi{terminology}. In the case of conflict between \isi{consistency} and \isi{accuracy}, it is usually \isi{consistency} that prevails; however, each such case is examined separately. Quotations and clear references to a particular place in a legal act have to be cited verbatim, even if they contain outdated or incorrect terms, but it is still possible to correct spelling mistakes or apply the current spelling conventions in such quotations. In other cases, when the risk of misleading the reader as to which concept is meant is minimal, the use of correct \isi{terminology} may be considered. For example, Regulation 1831/2003\footnote{Regulation (EC) No 1831/2003 of the European Parliament and of the Council of 22 September 2003 on additives for use in animal nutrition (OJ L 268, 18.10.2003, p. 29--43, Special edition in Polish: Chapter 03 Volume 040 P. 238--252).} contains the term \textit{compounds of trace elements}, which was translated as \textit{mieszanki pierwiastków śladowych} [literally: mixtures of trace elements], despite the fact that Polish law commonly uses the term \textit{związki pierwiastków śladowych} [literally: compounds of trace elements]. Fortunately, it is not a legally defined term and it is only mentioned once as the name of one of the functional groups. Commission Implementing Regulations concerning authorizations to use certain substances as feed additives are regularly adopted on the basis of Regulation 1831/2003. According to the \isi{consistency} criterion, the term from the basic regulation should be used in the corresponding implementing acts. However, implementing acts do not refer explicitly to the term in the basic act, and so in this particular context the correct term is used, even though it leads to inconsistencies with the basic regulation. In such situations, where no solution is perfect, solutions like this are considered “lesser evil” \citep[61]{Stefaniak2013}.

Translation errors and discrepancies between language versions caused by them can be rectified by means of a \textbf{corrigendum}. However, in the case of corrigenda, too, one has to take into account the rule of \isi{consistency} with the previous legislation and consider potential consequences that a \isi{corrigendum} might have for legal acts already in force. The validity of each proposal to change \isi{terminology} in an already published act is thoroughly investigated. For example, errors consisting in using a common word instead of a specialized term for the same concept do not qualify for a \isi{corrigendum}. The use of the phrase \textit{woreczek żółciowy} [literally: gall sack] as an equivalent of \textit{gallbladder} is a \isi{mistake}, because the right medical term in Polish is \textit{pęcherzyk żółciowy} [literally: gall bladder]. However, the phrase \textit{woreczek żółciowy} is widely understood and its use should not mislead the reader or have legal consequences. On the other hand, the terms \textit{dokładność} [\isi{accuracy}] and \textit{precyzja} [precision], which seem to be synonymous, have very different meaning in analytical chemistry and cannot be used interchangeably in this context. Therefore, the use of the term \textit{dokładność} as an equivalent of \textit{precision} would be a serious \isi{mistake} that requires a \isi{corrigendum}.

The third criterion taken into account when making terminological decisions is \textbf{clarity}. When striving for \isi{clarity}, the quality of the source text is of key importance. Unfortunately, the majority of \isi{EU} texts intended for translation are not written by native speakers; moreover, they are a result of negotiations on various political levels and hence a compromise. The lack of a single author and the necessity to satisfy the needs and wishes of many parties engaged in the \isi{drafting} process increase, inter alia, the tendency for generalizations and stylistic neutrality \citep{Koskinen2008}, making the text less clear. Moreover, the \isi{drafting} phase and the translation phase often overlap, and in consequence the translator receives a text that is not a final version of the legal act and is still being drafted (cf. \citealt{Doczekalska2009production}; \citealt{Stefaniak2013}). In other words, translators have to deal with many versions of the same text: some elements are deleted, others are added, concepts are redefined and \isi{terminology} is changed. Many changes to the original version of the source text also result from translators’ comments, who notice mistakes or make suggestions for improvements, but for the sake of multilingual concordance are not allowed to correct them by themselves without a new version.

Because of the above mentioned factors, which are independent of the translator, and because of the necessity to maintain above all the \isi{consistency} and \isi{accuracy} of translation, translators have very limited possibility to influence the \isi{clarity} of their texts. This also means that the \isi{textual fit} of national language versions of \isi{EU} legal acts, i.e. a degree to which these legal acts depart from the conventions of legal acts originally written in a given language (neutrality of translation), is considered to be divergent for the Polish language \citep[289--292]{Biel2014}. It is, however, hardly surprising. \isi{EU} translators are expected to create texts which are comprehensible, linguistically correct and terminologically accurate, and at the same time consistent with \isi{EU} legislation and with other language versions, and on top of that able to fit in the national legislation. Creating a translation that fulfils these contradictory expectations is practically impossible.

  
\section{Conclusions}\label{sec:stefaniak:4}

Terminology errors have particularly serious consequences for citizens and business entities, who usually rely only on one language version and thus can misunderstand their rights and obligations. Discrepancies between language versions\footnote{Of course, not all translation errors are errors in \isi{terminology}, however, in as much as 84 per cent of cases dealt with by the Court of Justice that involved discrepancy between language versions the discrepancy concerned \isi{terminology} \citep[28]{DGT2012}.} make the harmonization, interpretation and application of \isi{EU} legislation more difficult, may lead to court proceedings on the national or Union level, lower the trust of citizens towards the \isi{EU} and undermine the image of \isi{EU} institutions. This attracts the interest of all \isi{EU} institutions’ translation services in the \isi{terminology} work and its integration in the \isi{translation process}. Of course, the right \isi{terminology} alone does not guarantee high-\isi{quality translation}, but it is difficult to imagine good translation without the right \isi{terminology}.

% \section*{Abbreviations}
% \section*{Acknowledgements}

\sloppy
\printbibliography[heading=subbibliography,notkeyword=this] 
\end{document}