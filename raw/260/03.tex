\documentclass[output=paper,newtxmath,colorlinks,citecolor=brown]{langsci/langscibook}
\ChapterDOI{10.5281/zenodo.3764847}
%\bibliography{localbibliography}
%\input{localpackages.tex}
%%Copy this to localcommands.tex

\usepackage[english]{babel}
\usepackage{amsmath}
\usepackage{amssymb,amsfonts,textcomp}
\usepackage{array}
\usepackage{hhline}
\usepackage{hyperref}

\newenvironment{styleStandard}{}{}
\newenvironment{stylelsAbstract}{}{}
\newenvironment{stylelsSectioni}{}{}
\newenvironment{stylelsSectionii}{}{}
\newenvironment{stylelsBulletList}{}{}
\newenvironment{styleBibliographyi}{}{}
\newenvironment{listWWNumxxvleveli}{}{}
\newenvironment{listWWNumxxvlevelii}{}{}
\newenvironment{listWWNumxxvleveliii}{}{}
\newenvironment{listWWNumxxvleveliv}{}{}
\newenvironment{listWWNumixleveli}{}{}
\newenvironment{listWWNumixlevelii}{}{}
\newenvironment{listWWNumixleveliii}{}{}
\newenvironment{listWWNumixleveliv}{}{}

\newcommand\textstyleListLabelxvi[1]{#1}
\newcommand\labellistWWNumxxvleveli{\thelistWWNumxxvleveli.}
\newcommand\labellistWWNumxxvlevelii{\thelistWWNumxxvlevelii.}
\newcommand\labellistWWNumxxvleveliii{\thelistWWNumxxvleveliii.}
\newcommand\labellistWWNumxxvleveliv{\thelistWWNumxxvleveliv.}
\newcommand\labellistWWNumixleveli{[F0B7?]}
\newcommand\labellistWWNumixlevelii{\textstyleListLabelxvi{o}}
\newcommand\labellistWWNumixleveliii{[F0A7?]}
\newcommand\labellistWWNumixleveliv{[F0B7?]}

\newcounter{listWWNumxxvleveli}
\newcounter{listWWNumxxvlevelii}[listWWNumxxvleveli]
\newcounter{listWWNumxxvleveliii}[listWWNumxxvlevelii]
\newcounter{listWWNumxxvleveliv}[listWWNumxxvleveliii]
\newcounter{itemize} 

\title{Whom to oblige?}

\author{Julie Goncharov\affiliation{Hebrew University of Jerusalem / University of Göttingen}}


\abstract{In this paper, I argue that  deontic modals can be relativized on a par with epistemic modals, contrary to what is generally believed (see, for instance, \citealt{ret16}). The evidence comes from  the behaviour of Russian deontic modals under negation. In Russian, these constructions have an aspectual restriction: they are well-formed with the imperfective, but not with the perfective aspect. This restriction, however,  can be circumvented when the modal is relativized to the addressee rather than the subject. This obviation effect  shows that the relativization of the deontic modal base (i.e. whose obligations are relevant) is not a function of the context, but is rather encoded in the grammar. The analysis is implemented within the grammaticalized speech act system (\citealt{speten03}, \citealt{wil17}, a.o.). The account proposed here is extended to imperatives providing support for the presence of a deontic component in imperatives (\citealt{han99,nin05,kau12} a.o.). I also discuss cross-Slavic variation when it comes to the aspectual restriction with deontic modals and imperatives.

\keywords{necessity modals, deontic modals, negation, aspect, Russian}
}

\IfFileExists{../localcommands.tex}{
  \input{../localpackages}
  %Copy this to localcommands.tex

\usepackage[english]{babel}
\usepackage{amsmath}
\usepackage{amssymb,amsfonts,textcomp}
\usepackage{array}
\usepackage{hhline}
\usepackage{hyperref}

\newenvironment{styleStandard}{}{}
\newenvironment{stylelsAbstract}{}{}
\newenvironment{stylelsSectioni}{}{}
\newenvironment{stylelsSectionii}{}{}
\newenvironment{stylelsBulletList}{}{}
\newenvironment{styleBibliographyi}{}{}
\newenvironment{listWWNumxxvleveli}{}{}
\newenvironment{listWWNumxxvlevelii}{}{}
\newenvironment{listWWNumxxvleveliii}{}{}
\newenvironment{listWWNumxxvleveliv}{}{}
\newenvironment{listWWNumixleveli}{}{}
\newenvironment{listWWNumixlevelii}{}{}
\newenvironment{listWWNumixleveliii}{}{}
\newenvironment{listWWNumixleveliv}{}{}

\newcommand\textstyleListLabelxvi[1]{#1}
\newcommand\labellistWWNumxxvleveli{\thelistWWNumxxvleveli.}
\newcommand\labellistWWNumxxvlevelii{\thelistWWNumxxvlevelii.}
\newcommand\labellistWWNumxxvleveliii{\thelistWWNumxxvleveliii.}
\newcommand\labellistWWNumxxvleveliv{\thelistWWNumxxvleveliv.}
\newcommand\labellistWWNumixleveli{[F0B7?]}
\newcommand\labellistWWNumixlevelii{\textstyleListLabelxvi{o}}
\newcommand\labellistWWNumixleveliii{[F0A7?]}
\newcommand\labellistWWNumixleveliv{[F0B7?]}

\newcounter{listWWNumxxvleveli}
\newcounter{listWWNumxxvlevelii}[listWWNumxxvleveli]
\newcounter{listWWNumxxvleveliii}[listWWNumxxvlevelii]
\newcounter{listWWNumxxvleveliv}[listWWNumxxvleveliii]
\newcounter{itemize} 
  \input{../localhyphenation}
  \togglepaper[3]%%chapternumber
}{}
%\togglepaper[3]

\begin{document}
\maketitle

\il{Russian|(}
\section{Two core observations}\label{sect:coreobs}
We begin straightaway with examining the two observations that are the \isi{focus} of this paper. The first observation is that negated strong \isi{deontic} modals in \ili{Russian} cannot be used with the \isi{perfective} aspect  (\isi{aspectual restriction}). The second observation is that the \isi{aspectual restriction} can be circumvented when the \isi{modal} is relativized to the \isi{addressee} rather than the subject. I use these observations to argue that: (i) \isi{deontic} modals can be relativized on a par with \isi{epistemic} modals (contra \citealt{ret16}) and (ii) the \isi{relativization} takes place in the grammar rather than being a function of the context.


\subsection{The aspectual restriction}
\ili{Russian} strong \isi{deontic} modals, like \textit{dolžen} 	`must', can be used with a \isi{verb} in either \textsc{\isi{imperfective} (\isi{ipfv})} or \textsc{\isi{perfective} (\isi{pfv})} in positive sentences, \REF{arpos}. There are minimal interpretative differences between \im and \p in \REF{arpos} due to aspect, which will not concern us here.  The important observation is that  under \isi{negation} \p is unavailable, \REF{arneg}.\footnote{Note that \REF{arneg} with \p can have an \isi{epistemic} reading irrelevant here. Unless marked otherwise, all \isi{modal} bases in this paper are interpreted as \isi{deontic}. I gloss \textit{dolžen} as `must' because like \ili{English} \textit{must} it can have both \isi{deontic} and \isi{epistemic} interpretations. However, in negative sentences, I translate \textit{dolžen} as `have to' because this translation better represents the fact that \textit{dolžen} scopes below \isi{negation}, see the discussion around  example \REF{arigen}.
}
This \isi{aspectual restriction} on \ili{Russian} strong \isi{deontic} modals under \isi{negation} has been widely discussed in the literature, e.g. \cite{for70,rap85,haa97,zal06,pad13}.\footnote{The \isi{aspectual restriction} applies only to strong \isi{deontic} modals, i.e.\ modals that have a universal force. Existential (weak) \isi{deontic} modals do not show the \isi{aspectual restriction}. This difference is predicted by the analysis proposed in this paper, see the discussion around example \REF{existdeon}.
}

\ea	\label{ar} \ea \gll Ivan dolžen \minsp{\{} uxodit' / ujti\}.\\
	Ivan must.\textsc{ptcp} {} {leave.\im} {} leave.\p {}\\
	\glt `Ivan must leave.' \label{arpos}
	\ex  \gll Ivan ne dolžen \minsp{\{} uxodit' / \minsp{*} ujti\}.\\
    Ivan  not must.\textsc{ptcp} {} {leave.\im} {} {} leave.\p {}\\
	\glt `Ivan doesn't have to leave.' \\ \label{arneg}
	\z \z

\noindent The observation in \REF{ar} is also true for other strong \isi{deontic} modals and \isi{modal} expressions in \ili{Russian}, \REF{nuzhno}--\REF{objazan}. However, for reasons of space, I limit the presentation of  data and discussion to \textit{dolžen}.

\ea \label{nuzhno}	\ea \gll Ivanu nu\v zno \minsp{\{} uxodit' / ujti\}.\\
	Ivan.\textsc{dat} need.\textsc{ptcp} {} leave.\textsc{ipfv} {} leave.\textsc{pfv} {} \\
    \glt `Ivan needs to leave.'
	\ex \gll Ivanu ne nu\v zno \minsp{\{} uxodit' / \minsp{*} ujti\}. \\
	Ivan.\textsc{dat} not need.\textsc{ptcp} {} leave.\textsc{ipfv} {} {}  leave.\textsc{pfv} {}  \\
    \glt `Ivan doesn't need to leave.'
	\z \z

\ea \label{objazan}	\ea \gll Ivan objazan \minsp{\{?} uxodit' / ujti.\}\\
	Ivan obliged.\textsc{ptcp} {} leave.\textsc{ipfv} {} leave.\textsc{pfv} {} \\
	\glt `Ivan is obliged to leave.'
	\ex \gll Ivan ne objazan \minsp{\{} uxodit' / \minsp{*} ujti\}. \\
	Ivan not obliged.\textsc{ptcp} {} leave.\textsc{ipfv} {} {} leave.\textsc{pfv} {} \\
	\glt `Ivan is not obliged to leave.'
	\z \z

\noindent It is important to note that in \REF{arneg} with \im repeated in \REF{ari} the \isi{deontic} \isi{modal} is interpreted below \isi{negation} (i.e.\ there is no obligation for Ivan to leave), \REF{arinegm}. The interpretation where the \isi{deontic} is interpreted above \isi{negation} (i.e.\ Ivan has an obligation to stay/not to leave) is not available, \REF{arimneg}. When the \isi{verb} is \p in \REF{arneg}, the \isi{deontic} cannot be interpreted either below or above \isi{negation}.

\ea \label{arigen} \ea  \gll Ivan ne dolžen uxodit'.\\
    Ivan  not must.\textsc{ptcp} {leave.\im} \\
	\glt `Ivan doesn't have to leave.' \label{ari}
	\ex Available scope: `There is no obligation for Ivan to leave.'\\
    Abbreviated as \ding{51} $\neg\ \Box$ \im \label{arinegm}
    \ex Unavailable scope: `Ivan has an obligation to stay/not to leave.'\\
     Abbreviated as \ding{55} $\,\Box\ \neg$ \im \label{arimneg}
  	\z \z

\noindent \REF{p11} summarizes the first pattern that we need to account for. As can be seen in \REF{p11}, there is only one available configuration where the \isi{verb} is in \im and the \isi{modal} is interpreted below \isi{negation}, \REF{p11a}. Three other configurations are not well-formed, \REF{p11b}--\REF{p11d}.

	\ea  {Pattern 1 to explain (the \isi{aspectual restriction})}\label{p11}
	\ea[\ding{51}]{$\neg \ \Box$ \im \label{p11a}}
    \ex[\ding{55}]{$\,\Box\ \neg$ \im \label{p11b}}
    \ex[\ding{55}]{$\,\neg \ \Box$ \p \label{p11c}}
    \ex[\ding{55}]{$\,\Box\ \neg$ \p \label{p11d}}
    \z\z


 \subsection{Relativization and obviation}
 Our second core observation is that the \isi{aspectual restriction} is lifted when obligations are set on the \isi{addressee} rather than the subject. In an out-of-the-blue context in examples \REF{ar}--\REF{arigen}, the obligations are set on the subject (= Ivan). Consider now the context in \REF{relat}, in which police officers are ordered to stop Ivan from leaving.\footnote{The interpretation of deontics in contexts like \REF{relat} is similar to that of imperatives. The connection between deontics and imperatives is discussed in \sectref{sect:conclusion}.
 }
 In \REF{relat}, the subject (and the agent) of leaving (= Ivan) does not hold any obligations. The obligations to stop Ivan from leaving are on the \isi{addressee} (= police officers). In this context, \p under negated \isi{deontic} modals becomes available.\footnote{To the best of my knowledge, this observation has not been discussed in published work, although it seems to be common place for \ili{Russian} speakers.
 }



\ea \label{relat} \textit{Situation:} Police arrive at a crime scene and see Ivan fleeing with the stolen \emph{Mona Lisa}. A police chief to police officers:\\[5pt]
	{\gll  Ivan ne dolžen ujti!\\
	Ivan  not must.\textsc{ptcp} leave.\p \\
	\glt `Ivan must not leave/escape.'}
	\z

\noindent The \isi{relativization} of the \isi{deontic} \isi{modal base} can also be illustrated using conjunctive sentences in which the second conjunct denies the prejacent. As discussed in \citet{nin05}, strong \isi{deontic} modals, \REF{conjstr-a}, unlike weak ones, \REF{conjstr-b}, are infelicitous when the prejacent of the \isi{modal} is negated in the second conjunct.

\ea \label{conjstr}  \ea[\#]{Sam must go to confession, but he's not going to.}\label{conjstr-a}
	\ex[]{Sam should/ought to go to confession, but he's not going to.}\label{conjstr-b}
     \hfill (\citealt[2]{nin05})
        \z \z

\noindent The deviant \REF{conjstr-a} has the following form: \textit{Sam must p and $\neg$Sam is going to p}. Suppose that in \REF{relat} the \isi{modal} can only be interpreted above \isi{negation} (see below). If the \isi{modal base} is relativized to Ivan, the configuration \textit{Ivan must $\neg$p and Ivan is going to p} should be as infelicitous as \REF{conjstr-a}. But this is not the case:\footnote{I thank an anonymous reviewer for pointing out problems with this example in the earlier version of the paper.
}

\ea In the situation described in \REF{relat}: \\
	{\gll Ivan ne dolžen ujti, xotja on i sobiraetsja. \\
    Ivan not must.\textsc{ptcp} {leave.\p}  although he \textsc{foc}  going.to \\
    \glt `Ivan must not escape, although he is going to.'\label{conjivan}}
	\z

\noindent However, the counterpart of \REF{conjivan}, in which the second conjunct denies that the police are going to retain Ivan (stop him from leaving) is deviant, \REF{conjpolice}. This suggests that \textit{dolžen} in cases like    \REF{relat}, \REF{conjivan}, and \REF{conjpolice} is relativized to the \isi{addressee} (= police officers)  rather than the subject (= Ivan).\footnote{The interpretation of \REF{relat}, \REF{conjivan}, and \REF{conjpolice} is more involved. It resembles that of coercion constructions such as \textit{The \isi{addressee} must bring it about that Ivan doesn't leave.} This, however, does not mean that they are structurally different from non-relativized constructions as in \REF{ar}. \cite{gra17} has an informative discussion of coercion-free semantics for similar constructions. For reasons of space, I do not elaborate on this point here. However, I believe that something along the lines proposed in \cite{gra17} can be adopted for \ili{Russian} facts with \textit{dolžen}.
}


\ea[\#]{\gll Ivan ne dolžen ujti, no my ne sobiraemsja ego zader\v zivat'. \\
    	Ivan not must.\textsc{ptcp} {leave.\p} but we not going him retain.\textsc{inf} \\
    	\glt `Ivan must not escape, but we are not going to retain him.' \label{conjpolice}}
       \z

\noindent With respect to scopal possibilities, it is important to note that cases like \REF{relat}, \REF{conjivan}, and \REF{conjpolice} have only one reading, in which the \isi{modal} is interpreted above \isi{negation}, \REF{relatsca}. The surface scope reading, in which the \isi{modal} is below \isi{negation}, is not available, \REF{relatscu}.\footnote{Here and below, I use superscripts \cnst{adr} = \isi{addressee} and \cnst{sbj} = subject to mark whose obligations are relevant and who is the agent of the action described by the \isi{verb}.
}

\ea \label{relatscope} \ea In the situation described in \REF{relat}:\\
	{\gll  Ivan ne dolžen ujti!\\
	Ivan  not must.\textsc{ptcp} {leave.\p} \\
	\glt `Ivan must not leave/escape.'}
	\ex Available scope: `You, officers, must make Ivan  not leave/stay.'\\
    Abbreviated as \ding{51} $\Box^{\cnst{adr}}\ \neg\ \textsc{pfv}^{\cnst{sbj}}$ \label{relatsca}
    \ex Unavailable scope: `You, officers, don't have to make Ivan leave.'\\
    Abbreviated as \ding{55} $\,\neg\ \Box^{\cnst{adr}}\  \textsc{pfv}^{\cnst{sbj}}$ \label{relatscu}
	\z \z

\noindent The second pattern to be accounted for in this paper is summarized in \REF{p22}. This pattern concerns only configurations with the \isi{verb} in \isi{perfective}. The summary shows that the only possible construal is \REF{p22d}, in which the \isi{deontic} is relativized to the \isi{addressee} and scopes above \isi{negation}.


\ea {Pattern 2 to explain (\isi{relativization} and \isi{obviation})}\label{p22}
    \ea[\ding{55}]{$\,\neg \ \Box^{\cnst{sbj}}\ \textsc{pfv}^{\cnst{sbj}}$ \label{p22a}}
    \ex[\ding{55}]{$\,\Box^{\cnst{sbj}} \ \neg\ \textsc{pfv}^{\cnst{sbj}}$ \label{p22b}}
    \ex[\ding{55}]{$\,\neg \ \Box^{\cnst{adr}}\ \textsc{pfv}^{\cnst{sbj}}$ \label{p22c}}
    \ex[\ding{51}]{$\Box^{\cnst{adr}} \ \neg\ \textsc{pfv}^{\cnst{sbj}}$ \label{p22d}}
\z \z


\section{What we already know}\label{sect:preliminaries}
That \isi{deontic} modality interacts with \isi{negation} and aspect has been reported in the literature and these interactions have received some accounts (e.g. \citealt{han99,nin05,iatzei13}). In this section, I briefly show that the \ili{Russian} facts discussed in \sectref{sect:coreobs} are not reducible to previously reported observations. The \ili{Russian} facts constitute a superset; thus, previous  analyses under-generate and their extension to the \ili{Russian} data is hopeless.

\subsection{Deontics and negation}\label{subsect:deonneg}
It has been claimed that some \isi{deontic} modals in \ili{Dutch}, \ili{English}, \ili{German}, \ili{Greek}, and \ili{Spanish} exhibit the behaviour characteristic of \textsc{positive polarity items} (PPIs), e.g. \cite{hom11,iatzei10,iatzei13}. The examples in \REF{modalppis} illustrate  PPI-like properties of \ili{English} \textit{must} under its \isi{deontic reading}. \REF{modalppisa} shows that \textit{must} can only be interpreted above \isi{negation}. \REF{modalppisb} shows the rescuing effect when \textit{must} appears under two (Strawson) downward entailing operators. \REF{modalppisc} is an intervention effect: \textit{must} can be interpreted under \isi{negation} when a universal \isi{quantifier} intervenes.

\ea \label{modalppis} \ea John mustn't leave. \hfill \ding{55} $\,\neg\  \Box$ / \ding{51} $\Box\ \neg$ \label{modalppisa}
        \ex Only John must not work tonight. \hfill only $\neg\ \Box$ \label{modalppisb}
        \ex A student's mistake mustn't necessarily be hurled on the shoulders of his teachers. \hfill $\neg\ \forall\ \Box$ \label{modalppisc}\\
        \hfill (\citealt[543, 539]{iatzei13})
        \z \z

\noindent \ili{Russian} \textit{dolžen}, however, cannot be assimilated to \isi{PPI} \isi{deontic} modals such as \ili{English} \textit{must}. As we saw above, \textit{dolžen} can (in fact, must) scope below \isi{negation} when the \isi{verb} is \isi{imperfective}. The deviance under \isi{negation} arises only when the prejacent has a \isi{perfective} \isi{verb}. The relevant examples are repeated in \REF{rusmust}:

\ea \label{rusmust} \ea[]{\gll Ivan ne dolžen uxodit'.\\
		 	Ivan  not must.\textsc{ptcp} {leave.\im} \\
        	\glt `Ivan doesn't have to leave.' \hfill \ding{51} $\neg\ \Box$ / \ding{55} $\,\Box\ \neg$}
			\ex[*]{\gll Ivan ne dolžen ujti.\\
		 	Ivan  not must.\textsc{ptcp} {leave.\p} \\
        	\glt Intended: `Ivan doesn't have to leave.' \hfill \ding{55} $\,\neg\ \Box$ / \ding{55} $\,\Box\ \neg$}
			\z \z

\subsection{Deontics and aspect}\label{subsect:deonaspect}
It has been also discussed in the literature (\citealt{han99,nin05,hel16}, a.o.) that the \isi{deontic reading} of \textit{must} is unavailable when the \isi{verb} is \isi{perfective}, \REF{deonasp}. This is attributed to the future-oriented nature of the \isi{deontic} \textit{must}, which is corroborated by examples like \REF{deonfut}.

	\ea John must\textsubscript{\isi{epist}/*deon} have left.\label{deonasp}
    \z
    \ea \label{deonfut} \ea[]{You must finish your homework tonight/now.}
    \ex[\#]{You must have finished your homework yesterday. \hfill (\citealt[485]{han99})}
	\z \z

\noindent Be as it may, this line of reasoning cannot be extended to the \ili{Russian} data. As we already saw, in positive sentences, \textit{dolžen} can have the \isi{deontic} interpretation when the \isi{verb} is \isi{perfective}. The relevant example is repeated in \REF{deonasprus}:

\ea \gll Ivan dolžen ujti.\\
		Ivan must.\textsc{ptcp} {leave.\p} \\
        \glt `Ivan must\textsubscript{\isi{epist}/deon} leave.' \label{deonasprus}
        \z

\noindent To sum up, in this section we saw that the interaction between \isi{deontic} modals and \isi{negation} as well as the interaction between \isi{deontic} modals and the \isi{perfective} aspect, has already been discussed in the literature. The \ili{Russian} data, however,  are not reducible to either of these interactions. We should, therefore, seek an answer elsewhere.

\section{Towards an account}\label{sect:proposal}
\subsection{Preliminaries}\label{subsect:prelim}
We saw in the previous section that accounts that explain the PPI-like behaviour of \isi{deontic} modals and the unavailability of the \isi{deontic reading} with the \isi{perfective} aspect cannot be extended to the \ili{Russian} facts. However, these studies are instructive providing us with the assumptions that we can use to develop our analysis of the \ili{Russian} data.

In particular, we  need two sets of assumptions to get off the ground. The first set of assumptions concerns the syntax of (sentential) \isi{negation} and \isi{deontic} modals, as well as their scopal relations. The second set of assumptions pertains to the interpretation of \ili{Russian} aspect.

We start with the first set. Following \cite{iatzei10,iatzei13}, we make the three assumptions in \REF{assume}. For justification of these assumptions, I refer the interested reader to \cite{iatzei10,iatzei13}.

	\ea \label{assume} \ea Negation never lowers at LF
        \ex Deontic modals are base-generated lower than the \isi{inflectional} head (I$^0$).
        \ex Semantic scope has a corresponding configuration in the (overt or covert) syntax
        \z \z

\noindent Using these assumptions, the behaviour of ``neutral'' \isi{deontic} modals (i.e. modals that do not show polarity sensitivity), such as \ili{English} \textit{have to}, is straightforwardly explained. The \isi{modal} is base-generated below I$^0$ and therefore,  below \isi{sentential negation}. The scopal relation between the \isi{modal} and \isi{negation} is determined by the surface structure in  overt syntax. The reverse order is ruled out by the ban on \isi{negation} lowering. This is illustrated in \REF{neut}.\footnote{\cite{iatzei10,iatzei13} do not commit themselves to particular syntactic projections; so, we will stay away from filling in these particulars as well.
}
Note that \cite{iatzei13} assume that modals are raising verbs and the subject reconstructs.

\ea	Neutral \isi{deontic} modals \label{neut}
    \ea John doesn't have to leave. \hfill \ding{51} $\neg \ \Box$ / \ding{55} $\,\Box\ \neg$
	\ex Syntax: [ John$_j$ [ not [ have-to  [\textsubscript{\textit{v}P} t$_j$ leave ]]]]
	\ex LF: [not [ have-to [ John leave ]]]
   \z \z

\noindent \isi{PPI} modals, such as \ili{English} \textit{must}\textsubscript{deon}, involve head movement that does not reconstruct, as reconstruction results in ungrammaticality (due to the \isi{PPI} nature of the \isi{modal}), see \cite[549]{iatzei13}.  This is illustrated in \REF{overtmv}. Configurations in which \isi{PPI} modals surface below \isi{negation}, as in the \ili{Spanish} example in \REF{spaa}, are derived by a QR-like covert movement of the \isi{modal} to the position above \isi{negation}, see \REF{spab} and \REF{spac}.

	\ea \isi{PPI} \isi{deontic} modals (overt movement) \label{overtmv}
		\ea John must not leave. \hfill \ding{55} $\,\neg \ \Box$ / \ding{51} $\Box\ \neg$
		\ex Syntax: [ John$_j$ [ must$_i$ [ not [ t$_i$ [\textsubscript{\textit{v}P} t$_j$ leave ]]]]]
       	\ex LF: [ must$_i$ [ not [ t$_i$ [ John leave ]]]]
		\z \z

	\ea \isi{PPI} \isi{deontic} modals (covert movement) \label{debe}
    	\ea \gll Juan no debe ir. \\
    	Juan not must go \\ \hfill (\ili{Spanish})
       	\glt `Juan must not go.' \hfill \ding{55} $\,\neg \ \Box$ / \ding{51} $\Box\ \neg$ \label{spaa}
    	\ex Syntax: [ Juan$_j$ [ not [ must [\textsubscript{\textit{v}P} t$_j$ leave ]]]] \label{spab}
      	\ex LF: [ must$_i$ [ not [ t$_i$ [ Juan leave ]]]] \label{spac}
        \z \z

\noindent We now turn to the second set of assumptions, which concern the interpretation of \ili{Russian} aspect. Following \citet{zinfil14}, I assume that the \isi{perfective} aspect in \ili{Russian} asserts that the action has achieved the end-point and has an inference that the action has started. Moreover, this inference is generated as a \isi{scalar implicature} (\isi{SI}), \REF{pfv}.\footnote{Much previous work on \ili{Slavic} aspect erroneously claimed that the inference found with \p is a \isi{presupposition} (e.g.\ \citealt{bog85}, \citealt{rap85}). For evidence that the presuppositional account cannot be on the right track see the text below and \citet{zinfil14}. I also thank an anonymous reviewer for pointing out to me some additional data.
}
 The \isi{imperfective} aspect asserts that the action has started and generates no \isi{SI}, \REF{ipfv}. For expository purposes, I abbreviate \isi{aspectual} inferences as in \REF{abbrasp} where EP = end-point and S = start.

\ea \label{pfv} \ea \gll Ivan ne pročital ėtu knigu. \\
    	Ivan not {read.\p} this book \\
        \glt `Ivan  didn't read this book completely through.'
        \ex  Assertion: `Ivan did not finish reading this book.'
   		\ex  \isi{SI}: `Ivan started reading/read a part of this book.' \label{pfvsi} \\
        \hfill  (\citealt[383]{zinfil14})
	\z \z

 \ea \label{ipfv} \ea \gll Ivan ne čital ėtu knigu. \\
    	Ivan not {read.\im} this book \\
        \glt `Ivan  didn't read (any part of) this book.'
 		\ex  Assertion: `Ivan didn't start reading/read any part of this book.'
        \ex no \isi{SI}
        \z \z

	\ea \label{abbrasp} \ea Perfective, \REF{pfv}: \\
    	Assertion: $\neg$ Ivan-read-book EP \\
        \isi{SI}: Ivan-read-book S
   \ex Imperfective, \REF{ipfv}: \\
        Assertion: $\neg$ Ivan-read-book S \\
        (no \isi{SI})
	\z \z


\noindent \cite{zinfil14} argue that evidence for treating the start-inference of the \isi{perfective}, see \REF{pfvsi}, as an \isi{SI} rather than a \isi{presupposition} comes from two observations. First, this inference is cancelable, \REF{cancel}:

\ea \label{cancel} \gll Ivan ne pročital ėtu knigu. On da\v ze ne otkryl e\"e.  \\
    	Ivan not {read.\p} this book he even not opened it\\
       \glt  `Ivan  didn't read this book. He even didn't open it.' \\
        \hfill (\citealt[391]{zinfil14})
	\z

\noindent Second, the start-inference, \REF{pfvsi}, shows the projective behaviour characteristic of SIs rather than presuppositions. \cite{che09} shows that SIs project existentially under negated universal quantifiers, whereas presuppositions project universally in the same configuration. \cite{zinfil14} conducted an informal survey that showed that most \ili{Russian} speakers prefer  the existential inference of the \isi{perfective}, \REF{exist}, to the universal one, \REF{univ}. The numbers in square brackets show mean acceptability judgments. These results strongly suggest that the start-inference of the \isi{perfective} is an \isi{SI}.

\ea \label{project} \ea \gll Nikto iz nas ne pročital učebnik. \\
        	nobody of us not {read.\p} textbook \\ 			\glt `None of us read the textbook.'
        \ex Some of us started reading at least a part of the textbook. \hfill [$3.11/4$] \label{exist}
    \ex All of us started reading at least a part of the textbook \hfill [$1.65/4$] \label{univ}\\
    \hfill (\citealt[396--398]{zinfil14})
	\z \z

\noindent With these assumptions at hand, we are now ready to  account for the two core observations of this paper.

 \subsection{Proposal}\label{subsect:proposalnutshell}
 To the best of my knowledge, there is no formal analysis of the \isi{aspectual restriction} in \ili{Russian}, nor is there one for \isi{relativization} and \isi{obviation}. Below, I briefly discuss some intuitions in \cite{rap85}, which are repeated (with minor elaborations) in more recent accounts. Then, I present my own proposal.

According to \cite{rap85}, the use of \isi{perfective} in negated strong \isi{deontic} statements is pragmatically unjustified because it is weaker than a corresponding structure with the \isi{imperfective}. In his own words:

	\begin{quote}
	if the \isi{imperfective} \isi{verb} form in [...] \textit{on ne dolžen opravdyvat'sja} `he need not justify himself' is replaced by a \isi{perfective} form [...], the result would be a statement saying that while there is no need for him to succeed in justifying himself, there may be a need for him to attempt to do so. There is nothing logically incoherent about such a state of affairs, but it makes little pragmatic sense [...]   \hfill (\citealt[218--219]{rap85})
	\end{quote}

\noindent I believe \citeauthor{rap85}'s intuitions to be on the right track, although he does not formalize them and assumes that the start-inference of the \isi{perfective} is presup\-position-like. \cite{rap85} also does not discuss the \isi{relativization} and \isi{obviation} facts.

The account I propose in this paper capitalizes on \citeauthor{rap85}'s intuitions, but uses recent developments in formal semantics to formalize them. To start, let's see how an \isi{SI} is generated in a simple \isi{perfective} sentence like \REF{simplepfv}. The assertion of \REF{simplepfv} using the abbreviations introduced above is shown in   \REF{assertpfv}. Suppose that \REF{simplepfv} competes (for informativity) with a corresponding \isi{imperfective} statement, whose meaning is given in \REF{altpfv} as an alternative to \REF{simplepfv}. This \isi{imperfective} alternative, \REF{altpfv}, is stronger than the original \isi{perfective} statement, \REF{assertpfv}, as shown by the asymmetric \isi{entailment} relation in \REF{ent}. Therefore, the use of \REF{simplepfv} is justified if the speaker supposes that the stronger alternative is not true. This derives the desired \isi{SI} that Ivan started to leave, as shown in \REF{derivesipfv}.\footnote{The description of \isi{SI} generation is deliberately vague. As far as I can tell, both Neo-Gricean and grammatical approaches to SIs are compatible with the proposal in this paper, with some adjustments.
}

\ea \ea \gll Ivan ne u\v sel. \\
	Ivan not {leave.\p} \\
    \glt `Ivan didn't leave.' \label{simplepfv}
    \ex Assertion: $\neg$ Ivan-leave EP \label{assertpfv}
    \ex Alternative: $\neg$ Ivan-leave S \hfill (= \isi{imperfective}) \label{altpfv}
    \ex \gll {Asymmetric \isi{entailment}:}  $\neg$ Ivan-leave S $\Rightarrow\ \neg$ Ivan-leave EP \\
    {} $\neg$ Ivan-leave S $\not\Leftarrow\ \neg$ Ivan-leave EP  \label{ent} \\
    \ex \isi{SI}: $\neg \neg$ Ivan-leave S $\equiv$ Ivan-leave S \label{derivesipfv}
    \z \z

 \noindent In a negated \isi{deontic} sentence with \textsc{pfv}, as in \REF{dsmpl}, with the meaning schematized in \REF{assertdsmpl}, the generated \isi{SI} is as shown in \REF{sidsmpl}.\footnote{The reverse scope interpretation is discussed in \sectref{dar}.
 }


\ea \ea[*]{\gll Ivan ne dolžen ujti.\\
	Ivan not must.\textsc{ptcp} {leave.\p}\\ \label{dsmpl}
	\glt Intended: `Ivan doesn't have to leave.'}
    \ex[]{Assertion: $\neg\ \Box$ Ivan-leave EP \label{assertdsmpl}}
	\ex[]{Alternative:  $\neg\ \Box$ Ivan-leave S \hfill (= \isi{imperfective})}
	\ex[]{\gll {Asymmetric \isi{entailment}:}  {$\neg\ \Box$ Ivan-leave S $\Rightarrow\ \neg\ \Box$ Ivan-leave EP} \\
    {} {$\neg\ \Box$ Ivan-leave S $\not\Leftarrow\ \neg\ \Box$ Ivan-leave EP}  \label{entdsmpl} \\}
    \ex[]{\isi{SI}:  $\neg \neg\ \Box$ Ivan-leave S $\equiv\ \Box$ Ivan-leave S \label{sidsmpl}}
	\z \z

\noindent The asymmetric \isi{entailment} in \REF{entdsmpl} captures \citeauthor{rap85}'s intuition that the use of the \isi{perfective} in the negated \isi{deontic} sentences is pragmatically unjustified. Note, however, that the derivation of the \isi{implicature} in \REF{sidsmpl} by itself does not explain the \isi{aspectual restriction}. Combined together, the assertion in \REF{assertdsmpl} and the \isi{SI} in \REF{sidsmpl} give rise to the following interpretation: Ivan doesn't have an obligation to finish leaving, but he has an obligation to start leaving. \REF{dsmpl} does not have this reading; rather, the sentence is ungrammatical. Therefore, we need to strengthen the account in order to derive the ungrammaticality of \REF{dsmpl}.

To achieve this, I propose a more elaborate semantics of \isi{deontic} modals. I capitalize on the intuition that \isi{deontic} worlds are \textit{idealized} worlds (\citealt{kra12}; Maribel Romero, p.c.). That is to say, in such worlds, if an action (with a defined telos) starts, it must achieve its end point.\footnote{This is a simplification. The idealized nature of \isi{deontic} worlds is connected to the fact that the \isi{deontic} \isi{modal base} is compatible only with intentional actions, as argued in \citet{gon18a}.}
To capture this intuition, I add the following conditional to the \isi{deontic} \isi{modal base}: `if $x$-action S, then $x$-action EP', where $x$ is an individual whose obligations are relevant (usually the subject). \REF{sxdolzhen} shows a preliminary denotation of \ili{Russian} \textit{dolžen}.\footnote{I assume the standard interpretation of modals in terms of a \isi{modal base} and ordering source (\citealt{kra91}). The conditional `if $x$-action S, then $x$-action EP' restricts the \isi{modal base} to worlds where the conditional is true of any action.
}

\ea Denotation of \textit{dolžen} (to be revised) \\
	\sib{dolžen$_x$}$^{w_c}(p)$ = 1 iff \\
    $\forall w$[$w$ is compatible with $x$'s obligations in $w_c$ and $w$ is such that \\
    if $x$-action S, then $x$-action EP][$p$ is true in $w$] \\
    (where $x$ is an individual whose obligations are relevant, usually the subject) \label{sxdolzhen}
        \z

\noindent To see how the denotation in \REF{sxdolzhen} helps accounting for the \isi{aspectual restriction}, consider again the assertion and \isi{SI} of \REF{dsmpl}, now with the contribution of the \isi{modal} spelled out:

\ea \label{mb} \ea[*]{\gll Ivan ne dolžen ujti.\\
	Ivan not must.\textsc{ptcp} {leave.\p}\\
	\glt Intended: `Ivan doesn't have to leave.'}
    \ex[]{Assertion: \\
    $\neg \forall w$[$w$ is compatible with Ivan's obligations in $w_c$ and $w$ is such that if Ivan-action S, then Ivan-action EP][Ivan-leave EP in $w$] \label{assertmb}}
    \ex[]{\isi{SI}: \\
    $\forall w$[$w$ is compatible with Ivan's obligations in $w_c$ and $w$ is such that \\
    if Ivan-action S, then Ivan-action EP][Ivan-leave S in $w$] \label{simb}}
 	\z \z

\noindent It is not difficult to see that combining \REF{assertmb} with \REF{simb} results in a contradiction. The \isi{modal base} consists of worlds in which every action that Ivan starts is completed by him. \REF{simb} states that in all wolds in the \isi{modal base} Ivan starts leaving. Therefore, by Modus Ponens, all worlds in the \isi{modal base} must be such that Ivan's leaving is completed. But \REF{assertmb} requires there to be at least one world where Ivan's leaving is not completed. Following \citet{gaj02}, we assume that contradiction results in ungrammaticality.\footnote{This account assumes that there are situations in which implicatures are not easily cancelable,  see, for example, \cite{mag09,mag11}.
}

This account makes an immediate prediction, namely, that weak/existential \isi{deontic}/root modals are allowed with both \im and \p in \ili{Russian}. This prediction is borne out:

\ea \textit{Context:} `According to the prison regulations...' \\
    \gll Ivan ne mo\v zet ottuda \minsp{\{} zvonit' / pozvonit'\}. \\
	Ivan not can from.there {} {call.\im} {} {call.\p} {} \\
	\glt `Ivan is not allowed to call from there.' \label{existdeon}
	\z

\noindent To account for \isi{relativization} and \isi{obviation}, I extend \citeauthor{ste07a}'s (\citeyear{ste07a}) account of \isi{epistemic} modals to \isi{deontic} modals. More precisely, I propose that \isi{deontic} modals, just like epistemics, take an individual/PRO argument, which determines whose obligations are relevant, see \REF{sxdolzhenfinal}:

\ea Denotation of \textit{dolžen} (final) \\
	\sib{dolžen}$^{w_c}(x,p)$ = 1 iff \\
    $\forall w$[$w$ is compatible with $x$'s obligations in $w_c$ and $w$ is such that \\
    if $x$-action S, then $x$-action EP][$p$ is true in $w$]  \label{sxdolzhenfinal}
        \z

\noindent PRO is syntactically present and is co-indexed with the closest referential nominal at LF. The ungrammatical \REF{dsmpl}, repeated in \REF{dsmpl1}, has a simplified syntactic representation in \REF{dsmpl1syn} and a corresponding LE in \REF{dsmpl1lf}. PRO is co-indexed with the subject (which reconstructs); thus the \isi{modal base}  is relativized to Ivan and the ungrammaticality arises, as discussed above.

\ea \ea[*]{\gll Ivan ne dolžen ujti.\\
	Ivan not must.\textsc{ptcp} {leave.\p}\\ \label{dsmpl1}
	\glt Intended: `Ivan doesn't have to leave.'}
	\ex[]{Syntax: [ Ivan$_j$  [ not [ [must PRO]  [\textsubscript{\textit{v}P} t$_j$ {leave.\p} ]]]] \label{dsmpl1syn}}
    \ex[]{LF: [ not [ [must PRO$_j$] [ Ivan$_j$ leave ]]] \label{dsmpl1lf}}
    \z \z

\noindent To account for cases where \textit{dolžen} is relativized to the \isi{addressee}, such as \REF{relat}, repeated in \REF{obv}, we assume following \citet{speten03}, among others that a Speech Act (\isi{SA}) projection is syntactically present. Among other projections, it contains the Addressee Phrase (AdrP), \REF{obvsyn}. The \isi{modal} covertly moves to AdrP and PRO is co-indexed with the \isi{addressee} rather than the subject, \REF{obvlf}. Relativization of the \isi{modal base} to the \isi{addressee} does not derive a contradiction as the reader can verify by conjoining \REF{assertobv} with \REF{siobv}.

\ea \label{obvgen} \ea \label{obv} \textit{Situation:} Police arrive at a crime scene and see Ivan fleeing with the stolen \emph{Mona Lisa}. A police chief to police officers:\\[5pt]
	{\gll  Ivan ne dolžen ujti!\\
	Ivan  not must.\textsc{ptcp} leave.\p \\
	\glt `Ivan must not leave/escape.'}
	\ex Syntax:  [\textsubscript{AdrP} Adr ... [ Ivan$_j$  [ not [ [must PRO]  [\textsubscript{\textit{v}P} t$_j$ {leave.\p} ]]]]] \label{obvsyn}
	\ex LF:  [ [must PRO$_j$]$_k$ [ Adr$_j$ [ not [ t$_k$ [ Ivan leave  ]]]]] \label{obvlf}
         \ex Assertion: \\
    $\forall w$[$w$ is compatible with police' obligations in $w_c$ and $w$ is such that \\
    if police-action S, then police-action EP][$\neg$ Ivan-leave EP in $w$] \label{assertobv}
    \ex \isi{SI}: \\
    $\forall w$[$w$ is compatible with police' obligations in $w_c$ and $w$ is such that \\
    if police-action S, then police-action EP][Ivan-leave S in $w$] \label{siobv}
    \z \z


\noindent Thus, making the \isi{modal base} (including the conditional) relativizable to the \isi{addressee} straightforwardly accounts for the absence of the \isi{aspectual restriction} in cases like above. It also explains the absence of the parse where the \isi{modal} is interpreted below \isi{negation}, see \sectref{dar}.

 In this section, we saw how the \isi{aspectual restriction} and \isi{relativization} and \isi{obviation} are derived. More precisely, the \isi{aspectual restriction} is due to a contradiction between the assertion and \isi{SI} of the \isi{perfective}. This account requires that the \isi{deontic} worlds are idealized such that every action that starts achieves its end-point. Relativization and \isi{obviation} are explained by two assumptions: (i) deontics, like epistemics, take an individual PRO argument  in syntax and (ii) deontics can covertly move to the \isi{SA} projection where PRO is co-indexed with the \isi{addressee}. This section, however, did not explore all possible parses. This is the task for the next section.

 \section{Explaining the patterns}\label{dar}
 The first pattern we need to account for is in \REF{p11}, repeated in \REF{arpattern}. As can be seen in \REF{arpattern}, the only configuration in which \ili{Russian} \isi{deontic} necessity modals can appear is when they are followed by the \isi{verb} in the \isi{imperfective} and are interpreted below \isi{negation}, \REF{p1a}. All other parses are ill-formed.
 \largerpage[-1]

 	\ea  {Pattern 1 to explain (the \isi{aspectual restriction})}\label{arpattern}
	\ea[\ding{51}]{$\neg \ \Box$ \im \label{p1a}}
    \ex[\ding{55}]{$\,\Box\ \neg$ \im \label{p1b}}
    \ex[\ding{55}]{$\,\neg \ \Box$ \p \label{p1c}}
    \ex[\ding{55}]{$\,\Box\ \neg$ \p \label{p1d}}
    \z\z


\noindent  The contrast between \REF{p1a} and \REF{p1b} can be straightforwardly accounted for if we analyze \ili{Russian} \textit{dolžen} as a `neutral' \isi{modal} (similar to \textit{have to}) in the system proposed in \citet{iatzei13}. Recall from the discussion in  \sectref{subsect:deonneg} that neutral modals (i.e.\ modals that are not polarity sensitive) have surface scope. That is to say, if they appear below \isi{negation}, they scope below \isi{negation}. \REF{detailsp1a} shows the representation of the well-formed parse in \REF{p1a} for \ili{Russian} \textit{dolžen}:

	\ea Parse in \REF{p1a}: \ding{51} $\neg\ \Box\ \im$ \label{detailsp1a}
    		\ea \gll Ivan ne dolžen uxodit'. \\
            	Ivan not must.\textsc{ptcp} leave.\textsc{ipfv}\\
                \glt `Ivan doesn't have to leave.'
			\ex Syntax: [ Ivan$_j$ [ not [ [must PRO] [\textsubscript{\textit{v}P} t$_j$ leave.\textsc{ipfv} ]]]]
            \ex LF: [ not [ [must PRO$_j$]  [ Ivan$_j$ leave ]]] \label{detailsp1ac}
		\z \z


\noindent Note that in \REF{detailsp1a}, we continue to assume that modals are raising verbs and the subject reconstructs at LF (\citealt{iatzei13}). We also represent the individual argument of \textit{dolžen} as PRO co-indexed with the subject.\footnote{It is important to mention that PRO here is not indexical, bound, or controlled, see  \citet[500]{ste07} for discussion. I leave it open how its interpretation is determined. What is important for the account proposed here is that PRO is co-indexed with the closest nominal at LF.
}
The \isi{imperfective} does not generate any relevant \isi{implicature}; thus, no contradiction arises.

The parse in \REF{p1b}, in which \textit{dolžen} is interpreted above \isi{negation}, is ruled out by the ban on \isi{negation} lowering:

\ea Parse in \REF{p1b}: \ding{55} $\,\Box\ \neg\ \im$ \label{detailsp1b}
	\ea \gll Ivan ne dolžen uxodit'. \\
  	Ivan not must.\textsc{ptcp}  leave.\textsc{ipfv}\\
  	\glt Available: `Ivan doesn't have to leave.'\\
 	Unavailable: `Ivan has to stay/not leave.'
	\ex Syntax: [ Ivan$_j$ [ not [ [must PRO]  [\textsubscript{\textit{v}P} t$_j$ leave.\textsc{ipfv} ]]]]
    \ex LF: *[ t$_k$ [ [must PRO$_j$] [ not$_k$  [ Ivan$_j$ leave  ]]]] \hfill (impossible)
            \z \z

\noindent The parse in \REF{p1c} was discussed in the previous section when I showed the derivation of the \isi{aspectual restriction}, \REF{mb}. For completeness, I repeat its syntactic and LF representations in \REF{dtailsp1c}:

\newpage
\ea  Parse in \REF{p1c}: \ding{55} $\,\neg \ \Box\ \p$ \label{dtailsp1c}
	\ea[*]{\gll Ivan ne dolžen ujti. \\
	Ivan not must.\textsc{ptcp} leave.\textsc{pfv}\\
	\glt Intended: `Ivan doesn't have to leave.' \label{p1ca}}
	\ex[]{Syntax: [ Ivan$_j$ [ not [ [must PRO] [\textsubscript{\textit{v}P} t$_j$ leave.\textsc{pfv} ]]]] }
	\ex[]{LF: [ not [ [must PRO$_j$]  [ Ivan$_j$ leave ]]]}
	\z \z

\noindent The configuration in \REF{p1d}, in which \isi{negation} is interpreted below the \isi{modal}, can be ruled out by the ban on \isi{negation} lowering, like in  \REF{detailsp1b}.  Alternatively, it can be shown that \REF{p1d} results in a contradiction, like in \REF{mb}. In the latter case, it can be supposed that the \isi{modal} QRs at LF as in the \ili{Spanish} example in \REF{debe}. The second alternative is illustrated in \REF{detailsp1d}. I leave it to the reader to verify that \REF{detailsp1d} gives rise to a contradiction.

\ea  Parse in \REF{p1d}: \ding{55} $\,\Box\ \neg\ \p$ \label{detailsp1d}
    \ea[*]{\gll Ivan ne dolžen ujti. \\
	Ivan not must.\textsc{ptcp} leave.\textsc{pfv}\\
	\glt Intended: `Ivan has to stay/not leave.'}
	\ex[]{ Syntax: [ Ivan$_j$ [ not [ [must PRO] [\textsubscript{\textit{v}P} t$_j$ leave.\textsc{pfv} ]]]]}
	\ex[]{ LF:  [ [must PRO$_j$]$_k$ [ not [ t$_k$ [ Ivan$_j$ leave ]]]]}
	\z \z

\noindent The second pattern to be accounted for is repeated in \REF{pattern2}. The crucial point in deriving the \isi{aspectual restriction}
is that obligations (i.e.\ the \isi{modal base}) are relativized to the subject. Thus, \REF{p1c} and \REF{p1d}, with \isi{relativization} information, can be represented as \REF{p2a} and \REF{p2b} respectively. The well-formedness of \REF{p2d} (\isi{obviation}) is explained in \REF{obvgen}. What remains to be account for is  the unavailability of \REF{p2c}.

\ea  {Pattern 2 to explain (\isi{relativization} and \isi{obviation})}\label{pattern2}
    \ea[\ding{55}]{$\,\neg \ \Box^{\cnst{sbj}}\ \textsc{pfv}^{\cnst{sbj}}$ \label{p2a}}
    \ex[\ding{55}]{$\,\Box^{\cnst{sbj}} \ \neg\ \textsc{pfv}^{\cnst{sbj}}$ \label{p2b}}
    \ex[\ding{55}]{$\,\neg \ \Box^{\cnst{adr}}\ \textsc{pfv}^{\cnst{sbj}}$ \label{p2c}}
    \ex[\ding{51}]{$\Box^{\cnst{adr}} \ \neg\ \textsc{pfv}^{\cnst{sbj}}$ \label{p2d}}
\z \z

\noindent For completeness, I repeat the syntactic and LF representations of \REF{p2d}:

\newpage

\ea Parse in \REF{p2d}: \ding{51} $\Box^{\cnst{adr}}\ \neg\ \textsc{pfv}^{\cnst{sbj}}$ \label{detailsp2d}
	\ea \gll  Ivan ne dolžen ujti! \\
	Ivan  not must.\textsc{ptcp} leave.\textsc{pfv} \\
	\glt `Ivan must not leave/escape!' \hfill (uttered in the situation described in \REF{relat})
	\ex Syntax:  [\textsubscript{AdrP} Adr ...  [ Ivan$_j$  [ not [ [ must PRO ] [\textsubscript{\textit{v}P} t$_j$ leave.\textsc{pfv} ]]]]]
	\ex LF:  [ [ must PRO$_j$ ]$_k$ [ Adr$_j$ [ not [ t$_k$ [ Ivan leave  ]]]]]
        \z \z

\noindent The configuration in \REF{p2c}, in which the \isi{modal} is relativized to the \isi{addressee}, but scopes below \isi{negation}, is ruled out by the standard claim that (sentential) \isi{negation} cannot scope above  \isi{SA}. This is schematically shown in \REF{detailsp2c}:

\ea Parse in \REF{p2c}: \ding{55} $\,\neg\ \Box^{\cnst{adr}}\ \textsc{pfv}^{\cnst{sbj}}$ \label{detailsp2c}
	\ea \gll Ivan ne dolžen ujti! \\
	Ivan  not must.\textsc{ptcp} leave.\textsc{pfv} \\
	\glt Available: `Ivan must not leave/escape!' $\approx$ `You, officers, must make Ivan stay.' \\
	Unavailable: `It's not the case that you, officers, must make Ivan leave.'
	\ex Syntax:  [\textsubscript{AdrP} Adr ... [ Ivan$_j$  [ not [ [must PRO] [\textsubscript{\textit{v}P} t$_j$ leave.\textsc{pfv} ]]]]]
	\ex LF: *[ not [ [must PRO$_j$]$_k$ [ Adr$_j$ [ t$_k$ [ Ivan leave  ]]]]]	\hfill (impossible)
        \z \z


\section{In lieu of conclusion}\label{sect:conclusion}
This paper discussed the \isi{aspectual restriction} of \ili{Russian} strong \isi{deontic}  modals under \isi{negation}. Capitalizing on previous work on \ili{Russian} aspect and interaction between \isi{deontic} modals and \isi{negation}, I proposed a formalization of the \isi{aspectual restriction}. I also showed that the \isi{aspectual restriction} can be circumvented when the \isi{deontic} \isi{modal} is relativized to the \isi{addressee} rather than the subject.

From a cross-\ili{Slavic} perspective, it is important to mention that not many \ili{Slavic} languages show the \isi{aspectual restriction} discussed for \ili{Russian} in this paper.\footnote{I thank an anonymous reviewer for asking about cross-\ili{Slavic} variation with respect to the \isi{aspectual restriction} in negated \isi{deontic} constructions. Although I agree with the reviewer that this variation deserves thorough investigation, the claims made in this paper are independent from the cross-\ili{Slavic} observations. Apart from cursory remarks in this section, I leave the question for future research.
}
\REF{crossslavdeon} shows that in \ili{Polish} and \ili{Serbian} negated strong \isi{deontic} modals can be used with either \isi{imperfective} or \isi{perfective}.

\ea \label{crossslavdeon} \ea \gll Ivan ne dolžen \minsp{\{} uxodit' / \minsp{*} ujti\}. \\
	Ivan  not must.\textsc{ptcp} {} {leave.\im} {} {} {leave.\p} {} \\ \hfill (\ili{Russian})
	\glt `Ivan doesn't have to leave.'
	\ex \gll Ivan nie musi \minsp{\{} iść / wyjść\}. \\
	Ivan not must {} {leave.\im} {} {leave.\p} {} \\ \hfill (\ili{Polish})
	\glt `Ivan doesn't have to leave.'
	\ex \gll Ivan nije du\v zan \minsp{\{} odlaziti / otići\} kući kasno. \\
	Ivan not.be obliged {} {go.\im} {} {go.\p} home late \\ \hfill (\ili{Serbian})
	\glt `Ivan doesn't have to go home late.'
	\z \z

\noindent Interestingly, most \ili{Slavic} languages do show the \isi{aspectual restriction} with negative imperatives. Compare \REF{crossslavimp} with \REF{crossslavimpneg}:

\ea \label{crossslavimp} \ea \gll \minsp{\{} Otkryvaj / otkroj\} okno! \\
  	{} open.\textsc{\isi{ipfv}.imp} {} open.\textsc{\isi{pfv}.imp}  window \\ \hfill (\ili{Russian})
 	\glt `Open the window!'
    \ex \gll \minsp{\{} Jedz / zjedz\} tego jabłka! \\
	{} eat.\textsc{\isi{ipfv}.imp} {} eat.\textsc{\isi{pfv}.imp} that apple\\ \hfill (\ili{Polish})
	\glt `Eat that apple!'
    \ex \gll \minsp{\{} Jedi / pojedi\} tu jabuku! \\
	{} eat.\textsc{\isi{ipfv}.imp} {} eat.\textsc{\isi{pfv}.imp} that apple\\ \hfill (\ili{Serbian})
	\glt `Eat that apple!' \hfill (\citealt[2]{des16})
    \z \z

   \ea  \label{crossslavimpneg} \ea \gll Ne \minsp{\{} otkryvaj / \minsp{*}  otkroj\} okno! \\
  	not {} open.\textsc{\isi{ipfv}.imp} {} {} open.\textsc{\isi{pfv}.imp} window \\ \hfill (\ili{Russian})
 	\glt `Don't open the window!'
    \ex \gll Nie \minsp{\{} jedz / \minsp{*}  zjedz\} tego jabłka! \\
	not {} eat.\textsc{\isi{ipfv}.imp} {} {} eat.\textsc{\isi{pfv}.imp} that apple\\ \hfill (\ili{Polish})
	\glt `Don't eat that apple!'
    \ex \gll Ne \minsp{\{} jedi / \minsp{*}  pojedi\} tu jabuku! \\
	not {} eat.\textsc{\isi{ipfv}.imp} {} {} eat.\textsc{\isi{pfv}.imp} that apple\\ \hfill (\ili{Serbian})
	\glt `Don't eat that apple!' \hfill (\citealt[2]{des16})
    \z \z

\noindent The parallel behaviour of strong deontics and imperatives, as we see in \ili{Russian}, is not unexpected.  In many accounts, deontics and imperatives receive similar treatment (e.g.\ \citealt{han99,nin05,kau12}). The challenge is to explain why some \ili{Slavic} languages (like \ili{Russian}) show the \isi{aspectual restriction} with both strong \isi{deontic} modals and imperatives, whereas other \ili{Slavic} languages (like \ili{Polish} and \ili{Serbian}) show the \isi{aspectual restriction}  only with imperatives. This challenge is summarized in \tabref{tab:1:impdeon}.


\begin{table}
\caption{Deontics and imperatives across Slavic}
\label{tab:1:impdeon}
		\begin{tabular}{lll}
 		\lsptoprule
        &{imperatives} &{deontics} \\
        \midrule
       \ili{Russian}		& imp $\neg$ \{{\im} / *{\p}\} & $\neg \ \Box$\textsubscript{deon} \{{\im} / *{\p}\} \\
        \ili{Serbian}, \ili{Polish} & imp $\neg$ \{{\im} / *{\p}\} &$\neg \ \Box$\textsubscript{deon} \{{\im} / {\p}\} \\
	\lspbottomrule
    \end{tabular}
\end{table}

In the remainder of this concluding section, I briefly outline how the challenge presented by cross-\ili{Slavic} variation can be addressed. In particular, I would like to suggest that the observed difference between \ili{Slavic} languages is due to the differences in their \isi{aspectual} systems.  \ili{Slavic} aspect is a complex topic and I will not be able to do justice to the vast literature on this subject. However, I would like to point out that there are accounts that try to systematize \isi{aspectual} phenomena across \ili{Slavic} languages. One such account is the so-called East-West Theory of \ili{Slavic} aspect. According to this Theory, there is a systematic difference between \ili{Eastern Slavic} languages (\ili{Russian}, \ili{Ukrainian}, \ili{Belarus}) and \ili{Western Slavic} languages (\ili{Serbian}, \ili{Czech}, \ili{Slovenian}, etc.), with some mixed cases (\ili{Polish}, \ili{Bulgarian}, \ili{Macedonian}), see \cite{forkam15} for a recent review. The difference can be summarized as follows:


	\begin{quote}
	[In the Eastern group] the meaning of the [{\textsc{pfv}}] is made up of three ``layers'':

\begin{itemize}
    \item[(a)] the \isi{event} expressed by the predicate is terminative;
    \item[(b)] the \isi{event} is seen as a totality [...] such that there is a change of situation;
    \item[(c)] the \isi{event} expressed by the [\textsc{pfv}] \isi{verb} is sequentially connected to a following and/or preceding situation.
\end{itemize}

    \noindent [In the Western group, \isi{perfective} only needs to satisfy (a) and (b).]

    \raggedleft (\citealt[165]{forkam15})
	\end{quote}

The difference in use of \isi{imperfective}/\isi{perfective} between Western and Eastern groups can be seen in \tabref{tab:2:aspect}. \tabref{tab:2:aspect} shows that the use of \isi{perfective} in \ili{Eastern Slavic} languages is more restricted. Arguably, this is due to the fact that \isi{perfective} in the Eastern group has an additional condition: it must be sequentially connected (condition (c) above).

\begin{table}
\caption{The distribution of aspect in Western and Eastern Slavic languages (from \citealt[173, 182]{forkam15})}
\label{tab:2:aspect}
\begin{tabular}{llcc}
\lsptoprule
&&Western group& Eastern group\\ \midrule
Habitual &Non-past contexts&  {\im} / {\p} & {\im} ( {\p}) \\
& Past  contexts&  {\im} / {\p} & \im \\
Narration &Present \isi{tense} narration &
{\im} / {\p}  & \im \\
&Past \isi{tense} narration& {\im} / {\p} & \p \\
\lspbottomrule
\end{tabular}
 \end{table}


 I would like to suggest that the difference between \ili{Russian}, on the one hand, and \ili{Polish} and \ili{Serbian}, on the other hand, with respect to the \isi{aspectual restriction} in \isi{deontic} and imperative constructions is due to the same factor.  In imperatives (by their nature) the sequential connection to a following situation is present in both Eastern and \ili{Western Slavic} languages (\citealt{bog85,han99}). This makes Western \ili{Slavic} languages superficially look like \ili{Eastern Slavic} languages with regard to imperatives. Deontics, on the other hand, do not require sequential connection, which creates the difference between Eastern and \ili{Western Slavic} languages in negated \isi{deontic} construction. This idea is illustrated in \tabref{tab:3:asptab}.

\begin{table}
\caption{Aspect in deontics and imperatives}
\label{tab:3:asptab}
\begin{tabular}{cc|cc}
\lsptoprule
\multicolumn{2}{c}{{imperatives} with \p} & \multicolumn{2}{c}{{ deontics} with \p} \\
Western group &Eastern group&Western group &Eastern group\\
\midrule
(a) & (a) & (a) & (a)  \\
(b) & (b) & (b) & (b) \\
seq. connected &(c) seq. connected&&(c) seq. connected \\
\lspbottomrule
\end{tabular}
\end{table}


\noindent Suppose that the sequential connection to a preceding or following situation goes hand-in-hand with \isi{SI} generation in the \isi{aspectual} system. Recall that in this paper I argued that \isi{SI} of the \isi{perfective} is responsible for the \isi{aspectual restriction}. This line of reasoning will correctly account for the fact that the \isi{aspectual restriction} with imperatives exists in both Eastern and \ili{Western Slavic} groups, whereas the \isi{aspectual restriction} with strong deontics is only active in the Eastern group. I leave further investigation of this line of reasoning for future research.


\section*{Abbreviations}

\begin{tabularx}{.45\textwidth}{lX}
\cnst{adr}&addressee\\
\textsc{dat}&{dative}\\
deon&{deontic}\\
EP&end-point\\
{epist}&{epistemic}\\
\textsc{foc}&{focus}\\
\textsc{imp}&imperative\\
\textsc{inf}&{infinitive}\\
\textsc{ipfv}&{imperfective}\\
\end{tabularx}
\begin{tabularx}{.45\textwidth}{lX}
LF&logical form\\
\textsc{pfv}&{perfective}\\
{PPI}&positive polarity item\\
\textsc{ptcp}&{participle}\\
{QR}&{quantifier} raising\\
S&start\\
{SA}&{speech act} \\
\cnst{sbj}&subject\\
{SI}&{scalar implicature}\\
\end{tabularx}


\section*{Acknowledgements}
I would like to thank Vedran Dronjic and  Bogna Wiench for discussing Serbian and Polish data with me. I am also grateful to the audience of FDSL 12.5 and two anonymous reviewers for their helpful comments and questions. I am also grateful to the editors of this volume whose thoughtfulness and attentiveness to details greatly improved this paper. This research project was financially supported by the State of Lower-Saxony, Hannover, Germany (VWZN3181). All errors are my own.

\sloppy
\printbibliography[heading=subbibliography,notkeyword=this]

\il{Russian|)}
\end{document}
