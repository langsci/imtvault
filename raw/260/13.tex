\documentclass[output=paper,nonflat,colorlinks,citecolor=brown,newtxmath]{langsci/langscibook}
\ChapterDOI{10.5281/zenodo.3764867}
% \tolerance=1
% \emergencystretch=\maxdimen
% \hyphenpenalty=10000
% \hbadness=10000

%\bibliography{localbibliography}
%\input{localpackages.tex}
%%Copy this to localcommands.tex

\usepackage[english]{babel}
\usepackage{amsmath}
\usepackage{amssymb,amsfonts,textcomp}
\usepackage{array}
\usepackage{hhline}
\usepackage{hyperref}

\newenvironment{styleStandard}{}{}
\newenvironment{stylelsAbstract}{}{}
\newenvironment{stylelsSectioni}{}{}
\newenvironment{stylelsSectionii}{}{}
\newenvironment{stylelsBulletList}{}{}
\newenvironment{styleBibliographyi}{}{}
\newenvironment{listWWNumxxvleveli}{}{}
\newenvironment{listWWNumxxvlevelii}{}{}
\newenvironment{listWWNumxxvleveliii}{}{}
\newenvironment{listWWNumxxvleveliv}{}{}
\newenvironment{listWWNumixleveli}{}{}
\newenvironment{listWWNumixlevelii}{}{}
\newenvironment{listWWNumixleveliii}{}{}
\newenvironment{listWWNumixleveliv}{}{}

\newcommand\textstyleListLabelxvi[1]{#1}
\newcommand\labellistWWNumxxvleveli{\thelistWWNumxxvleveli.}
\newcommand\labellistWWNumxxvlevelii{\thelistWWNumxxvlevelii.}
\newcommand\labellistWWNumxxvleveliii{\thelistWWNumxxvleveliii.}
\newcommand\labellistWWNumxxvleveliv{\thelistWWNumxxvleveliv.}
\newcommand\labellistWWNumixleveli{[F0B7?]}
\newcommand\labellistWWNumixlevelii{\textstyleListLabelxvi{o}}
\newcommand\labellistWWNumixleveliii{[F0A7?]}
\newcommand\labellistWWNumixleveliv{[F0B7?]}

\newcounter{listWWNumxxvleveli}
\newcounter{listWWNumxxvlevelii}[listWWNumxxvleveli]
\newcounter{listWWNumxxvleveliii}[listWWNumxxvlevelii]
\newcounter{listWWNumxxvleveliv}[listWWNumxxvleveliii]
\newcounter{itemize} 
\IfFileExists{../localcommands.tex}{
  \input{../localpackages}
  %Copy this to localcommands.tex

\usepackage[english]{babel}
\usepackage{amsmath}
\usepackage{amssymb,amsfonts,textcomp}
\usepackage{array}
\usepackage{hhline}
\usepackage{hyperref}

\newenvironment{styleStandard}{}{}
\newenvironment{stylelsAbstract}{}{}
\newenvironment{stylelsSectioni}{}{}
\newenvironment{stylelsSectionii}{}{}
\newenvironment{stylelsBulletList}{}{}
\newenvironment{styleBibliographyi}{}{}
\newenvironment{listWWNumxxvleveli}{}{}
\newenvironment{listWWNumxxvlevelii}{}{}
\newenvironment{listWWNumxxvleveliii}{}{}
\newenvironment{listWWNumxxvleveliv}{}{}
\newenvironment{listWWNumixleveli}{}{}
\newenvironment{listWWNumixlevelii}{}{}
\newenvironment{listWWNumixleveliii}{}{}
\newenvironment{listWWNumixleveliv}{}{}

\newcommand\textstyleListLabelxvi[1]{#1}
\newcommand\labellistWWNumxxvleveli{\thelistWWNumxxvleveli.}
\newcommand\labellistWWNumxxvlevelii{\thelistWWNumxxvlevelii.}
\newcommand\labellistWWNumxxvleveliii{\thelistWWNumxxvleveliii.}
\newcommand\labellistWWNumxxvleveliv{\thelistWWNumxxvleveliv.}
\newcommand\labellistWWNumixleveli{[F0B7?]}
\newcommand\labellistWWNumixlevelii{\textstyleListLabelxvi{o}}
\newcommand\labellistWWNumixleveliii{[F0A7?]}
\newcommand\labellistWWNumixleveliv{[F0B7?]}

\newcounter{listWWNumxxvleveli}
\newcounter{listWWNumxxvlevelii}[listWWNumxxvleveli]
\newcounter{listWWNumxxvleveliii}[listWWNumxxvlevelii]
\newcounter{listWWNumxxvleveliv}[listWWNumxxvleveliii]
\newcounter{itemize} 
  \input{../localhyphenation}
  \togglepaper[13]%%chapternumber
}{}
%\togglepaper[13]


\title{Serbo-Croatian is developing stem-based prosody. Why so?}

\author{Marko Simonović\affiliation{University of Nova Gorica}\orcid{0000-0002-9651-6399}\lastand René Kager\affiliation{Utrecht University}\orcid{0000-0002-5811-839X}}

\abstract{Evidence is presented that Serbo-Croatian, in contrast to other Slavic languages with lexical prosody, is developing a prosodic system in which stressed non-stem material is avoided and surface stress is becoming a property of the word stem. Five case studies are shown in which stress is moving from non-stem material to stems. We analyse the general move towards stem-stressed prosody as the final step in a chain of language changes initiated by the Neo-Štokavian retractions, which were automatic and contrast-preserving, but led to a massive removal of stress from inflectional endings. We discuss the general reasons behind this language change in terms of markedness and, more specifically, the constraints proposed within Optimality Theory. We propose an analysis of the change under consideration in terms of a promotion of the markedness constraint \textsc{StemStress}, which requires stems to be stressed independently of lexical prominence.

    \keywords{lexical prosody, stem-controlled stress, inflection, paradigm uniformity, \textsc{StemStress}, Serbo-Croatian}
}



\begin{document}
\maketitle
\il{Serbo-Croatian|(}
\section{Introduction} \label{sec:kager:1}
\sloppy Standard \ili{Serbo-Croatian}, in contrast to other \ili{Slavic} languages with lexical \isi{prosody} (e.g. \ili{Slovenian}, \ili{Bulgarian}, and \ili{Russian}), is moving towards a system in which stressed \isi{inflectional} material is avoided and stress is becoming restricted to the word stem. In this paper, we present evidence for this change from nominal and verbal domains and consider different possible formalisations of the driving force behind this change.

This paper deals with \isi{prosodic prominence} in morphologically complex words. Prosodic prominence can be a consequence of morpheme-specific \isi{lexical prominence}, of general prosodic restrictions of a language or of an interaction between the two. For instance, when different morphemes within a word have conflicting lexical specifications or when a morpheme has a specification which clashes with a general prosodic requirement of the language, the \isi{prosodic pattern} of the word is a result of an interaction of different prosodic preferences. \citet{Revithiadou1999} presents a survey of typologically different systems with lexical \isi{prosody}, showing that morphemes are not all equal in their influence on the resultant \isi{prosody} of the word. Two observed tendencies are important for our purposes:

\begin{itemize}
\item Inflectional affixes lose when competing with \isi{derivational} affixes or roots, and
\item In cases where there is no preference (e.g. because no part of the complex word carries \isi{lexical prominence}), prominence will go to the stem.
\end{itemize}

\noindent Working in \textsc{Optimality Theory} (\isi{OT}) \citep[]{Prince1993}, \citet{Revithiadou1999} uses constraints to formalise these two observations. \textsc{HeadFaith} is a \isi{faithfulness constraint} which protects \isi{lexical prominence} of syntactic heads (\isi{derivational} affixes and roots are argued to be syntactic heads, unlike \isi{inflectional} affixes). \textsc{HeadStress} is a \isi{markedness constraint} that militates against stress on non-heads. This constraint is violated whenever \isi{inflectional} affixes are stressed.

The asymmetries described above make a prediction concerning \isi{diachronic change}. Since prosodically prominent \isi{inflectional} affixes are a marked option, they are expected to get lost in the course of language history. We argue that \ili{Serbo-Croatian} is a case at hand, as will be illustrated by a number of case studies in which stress moved to the stem.

\isi{OT} is a convenient tool for formalising this type of change. In \isi{OT}, \isi{diachronic change} is analysed as a promotion of a universal constraint, which therefore gets to play a more important role in the evaluations. The new/old pairs can then serve as ranking arguments. When looking for the constraint that got promoted, we are looking for a constraint which used to be violated by old forms but is not violated by the new ones. Often more than one constraint will fit this description, since differences between candidate forms in terms of constraint violations are rarely restricted to a single constraint. This is why we need to consider all candidate constraints, striving to find the best match for the change under consideration. As will be discussed below, such a quest will leave us with three well-established constraints which correspond to three analyses of the driving force behind the change discussed here.

Before moving on, several remarks are in order concerning aspects in which we depart from the terminology used by \citet{Revithiadou1999}. First, \isi{prosodic prominence} is often termed ``\isi{accent}'' and morphemes carrying \isi{lexical prominence} are called ``accented''. We avoid this term here and stick to ``\isi{lexical prominence}'' because we reserve the term ``\isi{accent}'' for the tonal contours which surface in \ili{Serbo-Croatian} (discussed in \sectref{sec:kager:2}). Second, since we are not concerned with headedness in general but only with the distinction between stems and non-stems, in order to improve the general readability of the paper, we will not refer to heads and non-heads but to stems and non-stem material. This also means that Revithiadou's \textsc{HeadFaith} will continue its life in this paper under the name of \textsc{StemFaith}, whereas  \textsc{HeadStress} will be called \textsc{StemStress}. \textsc{StemFaith} is violated whenever in the domain of the stem there are differences between the input and the output form, but it is insensitive to any such differences outside of the stem. \textsc{StemStress} is violated whenever the stem is not stressed.



Now we can turn to the three constraints on which \ili{Serbo-Croatian} seems to improve by moving stress from non-stem material to stems.

\textit{Option 1:} \textsc{StemFaith}. Prosody becomes (more) stem-controlled in the sense that the \isi{lexical stem} prominence wins (more often) when it is in conflict with other constraints. The promoted constraint is then \textsc{StemFaith}. Being a \isi{faithfulness constraint}, this constraint is not a priori violated by any surface pattern. It tolerates stressed \isi{inflectional} affixes whenever there is no \isi{lexical stem} prominence (or, in models which allow this, when \isi{lexical stem} \isi{prosody} imposes stress away from the stem). This constraint family has a long history within \isi{OT} \citep[see also][]{Alderete2001}.

\textit{Option 2:} \textsc{StemStress}. Stress on the stem becomes an overall requirement. The relevant constraint is \textsc{StemStress}, a \isi{markedness constraint} that requires \isi{prosodic prominence} on the stem, regardless of whether it is lexically sponsored or not. Stressed non-stem material always incurs a violation of \textsc{StemStress}. While this constraint is readily derivable from \citeposst{Revithiadou1999} \textsc{HeadStress}, we are the first to use it under this name and limited to stems.

\textit{Option 3:} \textsc{ParadigmUniformity}. Prosody becomes more uniform in morphologically related forms (within paradigms). The relevant constraint is  \textsc{Para\-digm\-Uni\-formity}, which compares all the paradigm members and penalizes any differences between them \citep{Burzio1996constraints, Kenstowicz1996}. \textsc{Para\-digm\-Uni\-form\-ity} can favour stressed stems, especially in languages which have null \isi{inflectional} affixes. Under the dominance of this constraint, stressed non-stem material is only tolerated if it is stressed in the whole paradigm.

The first option is relatively easily distinguishable from the other two: if \textsc{StemFaith} gets promoted, the change should be restricted to words in which both stems and non-stem material are lexically prominent. The result of a promotion of \textsc{StemFaith} would be that in such words, non-stem material loses its influence on the \isi{prosody} of the complex form. As will be clear from our discussion in \sectref{sec:kager:2} and  \sectref{sec:kager:3}, there are no compelling cases of such a constellation in the data illustrating the \isi{diachronic change} discussed in this paper.

An issue that arises with respect to the two remaining constraints is that, in many cases, they have the same effect. For instance, with monosyllabic stems, any promotion of \textsc{StemStress} will automatically mean that the winning candidates also incur fewer violations of \textsc{ParadigmUniformity}. We can illustrate this using a minimal toy language Tatitotu. Tatitotu only has one stem \textit{ta} and three \isi{inflectional} suffixes: \textit{ti}, \textit{to} and \textit{tu}. The paradigm of the only Tatitotu word is \textit{táti}\textasciitilde{}\textit{táto}\textasciitilde{}\textit{tatú}. If \textsc{StemStress} becomes undominated in Tatitotu, the new paradigm will be \textit{táti}\textasciitilde{}\textit{táto}\textasciitilde{}\textit{tátu}. This \isi{language change} would also be compatible with an analysis in which \textsc{ParadigmUniformity} got promoted, since the number of violations of \textsc{ParadigmUniformity} incurred by the paradigm went from 1 to 0.

However, the opposite is not true if \textsc{ParadigmUniformity} gets promoted. This does not necessarily mean that the new paradigms incur fewer violations of \textsc{StemStress}. Returning to our Tatitotu example, a promotion of \textsc{ParadigmUniformity} could also have the effect of \textit{táti}\textasciitilde{}\textit{táto}\textasciitilde{}\textit{tatú} becoming \textit{tatí}\textasciitilde{}\textit{tató}\textasciitilde{}\textit{tatú}, and the number of violations of \textsc{StemStress} would grow from 1 to 3.

In sum, \textsc{StemStress} offers a more restricted formalisation of a \isi{language change} than \textsc{ParadigmUniformity} but faced with data compatible with both a \textsc{StemStress} and a \textsc{ParadigmUniformity} analysis, we cannot exclude either, at least as long as we are in the domain of monosyllabic stems. However, polysyllabic stems can help us decide. Consider Tatitotu's closest \isi{relative} Tatetitotu, which has the disyllabic stem \textit{tate} and the same three suffixes: \textit{ti}, \textit{to} and \textit{tu}. Tatetitotu has the following paradigm:  \textit{táteti}\textasciitilde{}\textit{tatéto}\textasciitilde{}\textit{tatetú}. In such a paradigm, the only form that violates \textsc{StemStress}  is the last one and therefore this is the only form that should change if \textsc{StemStress} gets promoted: it should become either \textit{tatétu} or \textit{tátetu}. So the paradigms which are compatible with the promotion of \textsc{StemStress} to an undominated position are \textit{táteti}\textasciitilde{}\textit{tatéto}\textasciitilde{}\textit{tatétu} and \textit{táteti}\textasciitilde{}\textit{tatéto}\textasciitilde{}\textit{tátetu}. As always, these are improvements on \textsc{ParadigmUni\-formity} as well, but \textsc{ParadigmUniformity} is also violated by the difference between the first two forms, and, if promoted to an undominated position, \textsc{ParadigmUniformity} would crucially also level the \isi{stress pattern} in the first two forms, yielding either \textit{táteti}\textasciitilde{}\textit{táteto}\textasciitilde{}\textit{tátetu}, or \textit{tatéti}\textasciitilde{}\textit{tatéto}\textasciitilde{}\textit{tatétu} or \textit{tatetí}\textasciitilde{}\textit{tatetó}\textasciitilde{}\textit{tatetú}. Due to the described asymmetries between the predictions of the two accounts, we will not only \isi{focus} on the changes we encounter but also on those that are predicted to happen by the less restrictive account (in this case: \textsc{ParadigmUniformity}) but fail to happen. This line of argumentation determines the way in which the paper is structured, as discussed in the next section.

We present five case studies of \isi{language change} within standard \ili{Serbo-Croatian}. In each case the stem is becoming more prosodically prominent at the expense of non-stem material. We consider several possible formal accounts along the lines hinted at above, adducing new data where necessary.

The rest of this contribution is organised as follows. \sectref{sec:kager:2} presents case studies from classical (pitch-\isi{accent}) standard \ili{Serbo-Croatian}. In  \sectref{sec:kager:3} we consider the three analyses briefly presented above. We first establish that the data are compatible with a promotion of \textsc{StemStress}, which also means that a \textsc{ParadigmUniformity} analysis cannot be excluded, as discussed above. For this reason, we introduce some additional data from both verbal and nominal domains in order to pit the two analyses against each other. Finally we come to the conclusion that \textsc{ParadigmUniformity}, which does not make any distinction between stems and non-stem material, cannot account for the \isi{diachronic change} under consideration. In \sectref{sec:kager:4} we discuss an additional data set from relatively new and uncharted stress-only standard varieties, which seem to have reached the endpoint of the described change: in these varieties, non-stem material receives virtually no \isi{prosodic prominence}. \sectref{sec:kager:5} places the change discussed here in the broader context of prosodic changes in \ili{Serbo-Croatian}. In \sectref{sec:kager:6} the conclusions are presented.

\section{Case studies} \label{sec:kager:2}
\subsection{Standard Serbo-Croatian prosody} \label{sec:kager:subsec21}

Standard \ili{Serbo-Croatian} is a \isi{pitch-accent system} with distinctive \isi{vowel length}. Four tonal accents are traditionally distinguished: two falling and two rising. Each \isi{accent} is traditionally called long or short, depending of the length of its leftmost (or only) syllable. Falling accents (in which stress and \textsc{high tone} (H) co-occur) are monosyllabic and restricted to word-initial syllables (e.g. long-\isi{falling accent} in \textit{grȃda} `town.\textsc{gen}', and short-falling in \textit{grȁda} `hail.\textsc{gen}' cf. \textit{*gradȁ}, \textit{*gradȃ}). The rising accents are traditionally analysed as spans of two adjacent syllables which both have a H, whereas only the first syllable also carries stress (so \textit{bèžati} `flee' stands for \textit{bé\textsubscript{H}ža\textsubscript{H}ti}). More recently \citet{Zsiga-Zec2013} have argued that the first syllable of the rising accents only carries stress (so \textit{bèžati} `flee' stands for \textit{béža\textsubscript{H}ti}). As for the underlying contrast, all analyses assume that every word has one or zero underlying H's. Rising accents then originate from underlying H's on non-initial syllables, which spread onto the preceding syllable (so \textit{bèžati} `flee' is underlying /beža\textsubscript{H}ti/). If an underlying H is on the first syllable, it gets realized as a \isi{falling accent} because there is no room for it to spread. This explains why falling accents can only occur on the first syllable whereas the rising accents can occur on any syllable (but cannot start on a final syllable). Finally, if no H is available, a H gets assigned to the first syllable, so underlyingly toneless words surface with a \isi{falling accent} as well.

The traditional notation of \ili{Serbo-Croatian} tonal accents (used in all standard works in \ili{Serbo-Croatian}, e.g. \citealt{Stevanovic1979} and \citealt{Klaic2013}) puts diacritics on all stressed syllables (and a macron on all long unstressed syllables). This feature makes them convenient for our purposes: the reader can immediately tell the location of the stress in the word. This is why we will use this traditional notation in this article. Note, however, that the function of the diacritics in this notation bears no relation to their function in IPA.

\begin{table}
\caption{Diacritics used for Serbo-Croatian accents and vowel length}
\label{tab:kager:1}
 \begin{tabular}{ c  c  c c  }
\lsptoprule & \isi{rising accent} & \isi{falling accent} & unstressed \\ %\hline
  \midrule
  long  &   á  &    ȃ  &    ā\\
  short  &  à &  ȁ &   a\\
 \lspbottomrule
 \end{tabular}
\end{table}

The \isi{pitch-accent system} described above has been the only official standard ever since the beginning of the standardisation in the 19th century. It was based on the \isi{prosody} of the largest and most central dialect group called \ili{Neo-Štokavian}. The implementation of the pitch-\isi{accent} standard has been quite different in the \ili{Neo-Štokavian} dialect areas, where speakers already spoke a four-\isi{accent} variety natively, and in the remaining areas, where in most cases nowadays a stress-only version of the standard is spoken.

In this section, we \isi{focus} on the dynamics within the \ili{Neo-Štokavian} \isi{pitch-accent system} (and we turn to the stress varieties in \sectref{sec:kager:4}). We mostly rely on the normative manual by \citet[]{Klaic2013} based on a manuscript that was ``over 50 years old'' at the moment this edition came out. Of special value for our purposes are the editor's notes, which often compare Klaić's description to modern usage, enabling the construction of new vs. old pairs that we use for documenting the \isi{language change}. We furthermore presented the new vs. old pairs to five native speakers and report on their recognition and evaluation of the forms.

It should be pointed out that the distinction between \ili{Neo-Štokavian} and Non-\ili{Neo-Štokavian} areas that we are making in this article is not the same as the one that can be found in traditional dialect descriptions. This is because we are interested in the implementation of the standard \ili{Serbo-Croatian} \isi{prosody}, whereas traditional dialect descriptions usually reflect the non-standard language use of the oldest and least mobile inhabitants of an area. When it comes to the standard usage, rural areas are characterised by massive exposure to the vernaculars and the standard usage of the neighbouring cities (e.g. \citealt{Kapović2004} discusses the modern \ili{Croatian} situation in this respect). As a consequence, certain rural areas where traditionally Non-\ili{Neo-Štokavian} dialects are spoken can be and have to be included in the \ili{Neo-Štokavian} area when it comes to standardisation. A case in point is Slavonia, the area in which Bratoljub Klaić was born. The dialect of Klaić's native village, Bizovac, has an \ili{Old-Štokavian} prosodic system, without the \ili{Neo-Štokavian} rising accents \citep[19]{Klaic2007} and is as such part of the traditional \ili{Old-Štokavian} \ili{Slavonian} dialect. However, Bizovac is at only 18 km from the \ili{Neo-Štokavian} city of Osijek and the speakers from Bizovac adopt the \isi{prosody} of Osijek when they speak standard \ili{Serbo-Croatian}. Since there are no big cities where the \ili{Slavonian} dialect is spoken, this dialect has not led to any specific features in the implementation of the standard \isi{prosody}. This is why we include this dialect area, as well as the other dialect areas which, speaking in terms of \citet{Kapović2004}, ``gravitate toward'' big cities in the area to which the big city itself belongs.

\subsection{From \textit{bežíte} to \textit{bèžīte}} \label{sec:kager:subsec22}
This change targets stressed theme vowels. Stressed theme vowels used to be possible in verbs with a stem-final short-\isi{rising accent} in the \textsc{sg}  present \isi{tense}. The type is illustrated by the form \textit{bèž-ī-m} `I flee', where \textit{bež-} is the stem, \textit{-i-} the theme vowel and \textit{-m} is the 1\textsc{sg} suffix. In such verbs, 1\textsc{pl} and 2\textsc{pl} present-\isi{tense} forms traditionally had the stress on the theme vowel: \textit{bež-í-mo} `we flee' and \textit{bež-í-te} `you(\textsc{pl}) flee'. Nowadays this pattern is virtually absent from standard \ili{Serbo-Croatian} and only \textit{bèž-ī-mo} and \textit{bèž-ī-te} is possible (editor's remark in \citealt[379]{Klaic2013}). The \isi{stress pattern} is now the same in the whole paradigm and the stress always falls on the stem. The old and the new paradigm are illustrated in \tabref{tab:kager:2} and \tabref{tab:kager:3}.  The forms which underwent change are in bold.

\begin{table}
\caption{The old present-tense paradigm of \textit{bežati} `flee'}
\label{tab:kager:2}
 \begin{tabular}{  l l l }
  \lsptoprule
& \textsc{sg} & \textsc{pl}\\
  \midrule
  1  &   bèž-ī-m  &   \textbf{bež-í-mo}\\
  2  &  bèž-ī-š & \textbf{bež-í-te}\\
3  &  bèž-ī & bèž-ē\\
  \lspbottomrule
 \end{tabular}
\end{table}


\begin{table}
\caption{The new present-tense paradigm of \textit{bežati} `flee'}
\label{tab:kager:3}
 \begin{tabular}{  l l l }
  \lsptoprule
& \textsc{sg} & \textsc{pl}\\
  \midrule
  1  &   bèž-ī-m  &   \textbf{bèž-ī-mo}\\
  2  &  bèž-ī-š & \textbf{bèž-ī-te}\\
3  &  bèž-ī & bèž-ē\\
  \lspbottomrule
 \end{tabular}
\end{table}


Additional evidence for the unacceptability of the old paradigm is provided by a recent social media hype after a \ili{Serbian} politician uttered \textit{Gdȅ bežíte?} `Where are you running to?'. In the ensuing parodies of his statement, modern standard speakers applied the pattern to verbs in which it was never possible (e.g. \textit{zviždíte} `you(\textsc{pl}) whistle', for actual \textit{zvíždīte}). This points at the conclusion that the old grammar (which produced stressed theme vowels only in 1\textsc{pl} and 2\textsc{pl} present-\isi{tense} forms of those verbs which have a stem-final short-\isi{rising accent} in the \textsc{sg}  present \isi{tense}) is no longer available to the speakers and that the modern \ili{Serbo-Croatian} grammar only derives \textit{bèžīmo}, \textit{bèžīte}. Forms like \textit{zviždíte} show that  speakers are also able to form a generalised usage pattern based on their experience with another variety of the same language, similar to patterns which \citet{Janda1994} discuss as types of hypercorrection. This pattern is now applicable to all verbs.
\subsection{From \textit{gradóvā} to \textit{grȁdōvā}, from \textit{gradòvima} to \textit{grȁdovima}} \label{sec:kager:subsec23}

This change targeted stressed plural augments, which were possible in a large class of regular long-vowelled monosyllabic masculine nouns (type \textit{grȃd} `town’) and a small class of disyllabic masculine nouns that have a \isi{falling accent} and a stem-final unstressed syllable (type \textit{gȍlūb} `pigeon’). Such nouns typically take the augment \textit{-ov-} in the plural forms. While most of the forms had and still have the stress on the stem (e.g. in \textit{grȁd-ov-i} `towns', where \textit{grad} is the stem, \textit{-ov-} is the augment and \textit{-i} is the \textsc{nom}.\textsc{pl} suffix), the augment was traditionally stressed in \textsc{gen}.\textsc{pl} (e.g. \textit{grad-óv-ā}) and in \textsc{dat}.\textsc{ins}.\textsc{loc}.\textsc{pl} (e.g. \textit{grad-òv-ima}). \citet[19--20]{Klaic2013} acknowledges the forms like \textit{grad-óv-ā} and \textit{grad-òv-ima} are losing ground to stem-stressed \textit{grȁd-ōv-ā} and \textit{grȁd-ov-ima}, which leads to prosodically perfectly regular paradigms.  Klaić lists the old and new paradigms as in  \tabref{tab:kager:4}.

\begin{table}
\caption{The plural paradigm of \textit{grad} `town'}
\label{tab:kager:4}
 \begin{tabular}{  l  l  l }
 \lsptoprule
& Old & New\\
 \midrule
  Nominative/Vocative  &  grȁd-ov-i  &   grȁd-ov-i\\
  Accusative  &  grȁd-ov-e &grȁd-ov-e\\
Genitive  &  \textbf{grad-óv-ā}& \textbf{grȁd-ōv-ā}\\
Dative/Instrumental/Locative  & \textbf{grad-òv-ima}& \textbf{grȁd-ov-ima}\\
  \lspbottomrule
 \end{tabular}
\end{table}

For modern-day speakers, the pattern with the stressed augment \textit{-ov-} has a different status for the two case endings. Forms like \textit{grad-óv-ā} are judged familiar but `extremely archaic', whereas forms like \textit{grad-òv-ima} are judged unfamiliar. The pattern in \textit{-óvā} seems to have lexicalised to index archaic language use, but also generalised: it is now also possible (in parodies) in short-vowelled stems, where it was not possible before, e.g. for \textit{rȍb} `slave' speakers accept the parody \textit{robóvā} `slave.\textsc{gen}.\textsc{pl}', for actual \textit{ròbōvā}. As such, \textit{-óvā} seems to have become what \citet{Janda1994} termed ``hyper-archaism", comparable to the use of once 2\textsc{sg} \textit{-st} for all persons in \ili{English}.

\subsection{From \textit{kamènu} to \textit{kȁmenu}, from \textit{poglédu} to \textit{pȍglēdu}} \label{sec:kager:subsec24}
This change affects some dozens of polysyllables with an initial \isi{falling accent} and a long final syllable in \textsc{nom}.\textsc{sg} that refer to something inanimate, e.g.\textit{ kȁmēn} `stone' and \textit{pȍglēd} `gaze'. Such nouns generally kept and keep the \isi{falling accent} throughout the paradigm (e.g. in the  \textsc{gen}.\textsc{sg} and  \textsc{dat}.\textsc{sg} forms \textit{pȍglēd-a} and \textit{pȍglēd-u}), but traditionally the \textsc{loc}.\textsc{sg} forms displayed a stem-final \isi{rising accent} (e.g. \textit{pogléd-u}). \citet[30]{Klaic2013} mentions the general tendency for the locative to take up `the \isi{dative} forms' (which means that the new \textsc{loc}.\textsc{sg} form is \textit{pȍglēd-u}) and the editor adds that forms like \textit{poglédu} only survived in some fixed expressions. Both paradigms are illustrated in \tabref{tab:kager:5}.

\begin{table}
\caption{The singular paradigms of \textit{kamen} `stone' and \textit{pogled} `gaze'}
\label{tab:kager:5}
\fittable{
 \begin{tabular}{  l l l l l }
  \lsptoprule
& Old & New & Old & New\\
 \midrule
 Nominative/Accusative  &  kȁmēn  &  kȁmēn&  pȍglēd &  pȍglēd \\
Genitive  &  kȁmen-a  &  kȁmen-a &  pȍglēd-a &  pȍglēd-a \\
Instrumental  &  kȁmen-om  &  kȁmen-om &  pȍglēd-om &  pȍglēd-om \\
Dative  &  kȁmen-u  &  kȁmen-u &  pȍglēd-u &  pȍglēd-u \\
Locative  & \textbf{kamèn-u} &  \textbf{kȁmen-u} &  \textbf{pogléd-u} &  \textbf{pȍglēd-u} \\
  \lspbottomrule
 \end{tabular}
 }
\end{table}

As for the status of the old \textsc{loc}.\textsc{sg} forms for modern-day speakers, the forms with a short-\isi{rising accent} have a different status from those with a long-\isi{rising accent}. The short-rising pattern (\textit{kamènu}) is unfamiliar to most speakers, whereas the long-rising pattern (\textit{poglédu}) is impossible outside fixed expressions (as pointed out by Klaić's editor as well). For instance, speakers accept the form \textit{poglédu} in the phrase \textit{u tom poglédu} `in that regard', but they reject it in \textit{u tvojem pȍglēdu} *\textit{poglédu} `in your gaze'. \citet{SimonovicArsenijevic2015b} discuss the reasons for the survival of the long-rising pattern in fixed expressions, showing that the long-rising penultimate \isi{accent} pattern appears in a range of similar contexts and claiming that this pattern should be considered the default in \ili{Serbo-Croatian} in forms and paradigms with overt endings.

Before we move on to the next data set, a caveat is in order regarding the productivity of the rising \textsc{loc}.\textsc{sg} pattern in the reconstructed older stage of \ili{Serbo-Croatian}. There is a long tradition of representing this pattern as once fully productive (and \citeauthor{Klaic2013} seems to follow this tradition). Assuming a stage of full productivity does simplify the analysis of the \isi{stress shift}, this analysis can then go along the following lines. All underlyingly toneless nouns (which generally surface with falling accents) had a \isi{stress shift} in combination with the \textsc{loc}.\textsc{sg} \textit{-u\textsubscript{H}}, which, as its representation shows, had a H. This used to produce \textsc{nom}.\textsc{sg}\textasciitilde{}\textsc{loc}.\textsc{sg} alternations \textit{kȁmēn}\textasciitilde{}\textit{kamènu} and \textit{pȍglēdu}\textasciitilde{}\textit{poglédu}, but also \textit{grȃd}\textasciitilde{}\textit{grád}\textit{u} `town'. The latter alternation is still active in modern-day \ili{Serbo-Croatian}, where \textit{-u\textsubscript{H}} now marks the unified \isi{dative}/locative inanimate case \citep{SimonovicArsenijevic2015a}. This idyllic picture of the older stage is unfortunately an oversimplification. As shown by \citet[219--220]{Stevanovic1979}, there is no real evidence that polysyllables ever productively had a stress-shift in interaction with \textsc{loc}.\textsc{sg} \textit{-u\textsubscript{H}}. \citeauthor{Stevanovic1979} lists 35 nouns in which the \isi{stress shift} is attested, pointing out that many are optional and not all of the forms are attested in the same dialect. \citeauthor{Stevanovic1979} then goes on to point out that there are more nouns which have the same \isi{prosodic pattern} but never display any \isi{stress shift} (e.g. \textit{mȉrīs} `smell' and \textit{jȁblān} `black poplar’). On the other hand, monosyllabic stems have the fully productive tonal \isi{accent} shift of the type \textit{grȃd}\textasciitilde{}\textit{grádu} (but note that there is no \isi{stress shift} in \textit{grȃd}\textasciitilde{}\textit{grádu}). In sum, it seems that the shift \textit{kȁmēn}\textasciitilde{}\textit{kamènu} and \textit{pȍglēdu}\textasciitilde{}\textit{poglédu} actually was a feature of a lexical class. An analysis would then be possible along the lines of \citet{Kager2008} in which the words in this class used to have two stored allomorphs (e.g. /po\textsubscript{H}gle:d\textasciitilde{}pogle:\textsubscript{H}d/) and the first allomorph used to win in almost all cases because it carries the default \isi{prosody} generally preferred by the \isi{markedness} constraints (which is proven by the fact that it gets assigned to toneless nouns as discussed in \sectref{sec:kager:subsec21}). However, the second allomorph used to win in one case: when it also helped realise the H of the ending \textit{-u\textsubscript{H}}, which could only get realized as the second part of a \isi{rising accent}. If this analysis is on the right track, then \isi{inflectional} affixes with a H never had the ability to cause a \isi{stress shift} except in the rare cases where two allomorphs were stored. If this is the case, the fact that the old locative forms disappeared can be accounted for as a result of a lexicon-internal simplification (/po\textsubscript{H}gle:d\textasciitilde{}pogle:\textsubscript{H}d/ became /po\textsubscript{H}gle:d/, or even /pogle:d/), rather than as a consequence of a grammar change.

\subsection{From \textit{putòvati} to \textit{pȕtovati}} \label{sec:kager:subsec25}
\sloppy This change seems to be restricted to standard \ili{Serbo-Croatian} in Serbia. It targets approximately a dozen \isi{denominal} verbs derived using the suffix \textit{-ovati} from nouns which themselves display allomorphy in their paradigm. Traditionally, \isi{denominal} verbs ending in \textit{-ovati} displayed two possible prosodic patterns. One is extremely uniform and reserved for verbs derived from nouns which display no stem allomorphy: \textit{rȁt}\textasciitilde{}\textit{rȁta} `war.\textsc{nom}\textasciitilde{}\textsc{gen}' derives \textit{rȁtovati}\textasciitilde{}\textit{rȁtujēm}\textasciitilde{}\textit{rȁtovao} `wage war.\textsc{inf}\textasciitilde{}\textsc{prs.1}\textsc{sg}\textasciitilde{}\textsc{pst}'. The other \isi{prosodic pattern} displays prosodic allomorphy and surfaces in verbs derived from nouns which themselves display prosodic allomorphy: \textit{pȗt}\textasciitilde{}\textit{púta} `travel.\textsc{nom}\textasciitilde{}\textsc{gen}' derives \textit{putòvati}\textasciitilde{}\textit{pùtujēm}\textasciitilde{}\textit{pȕtovao} `travel.\textsc{inf}\textasciitilde{}\textsc{prs.1}\textsc{sg}\textasciitilde{}\textsc{pst}' \citep[for a detailed analysis see][]{Simonovic2015}.
In most standard varieties in Serbia, the \isi{denominal} verbs of the type \textit{putòvati}\textasciitilde{}\textit{pùtujēm}\textasciitilde{}\textit{pȕtovao} are moving to the new type \textit{pȕtovati}\textasciitilde{}\textit{pùtujēm}\textasciitilde{}\textit{pȕtovao}, as illustrated in \tabref{tab:kager:6}.

\begin{table}
\caption{ The paradigm of \textit{putovati} `travel'}
\label{tab:kager:6}
 \begin{tabular}{  l l l }
  \lsptoprule
& Old & New\\
 \midrule
Infinitive  &  \textbf{putòvati}  &   \textbf{pȕtovati}\\
Present \isi{tense} (1\textsc{sg})  &  pùtujēm & pùtujēm\\
Past \isi{participle}  &  pȕtovao & pȕtovao \\
 \lspbottomrule
 \end{tabular}
\end{table}

\newpage
\citet[]{Simonovic2015} found that approximately one third of modern speakers from Serbia reject the old forms entirely, whereas the remaining two thirds accept both old and new forms to different extents.
The new \isi{infinitive} \isi{prosody} is clearly copied from the \isi{past participle}. However, this copying only targeted the \isi{infinitive} form, which had the stress away from the stem. In other words, what did not happen is the creation of a perfectly regular paradigm *\textit{pȕtovati}\textasciitilde{}\textit{pȕtujēm}\textasciitilde{}\textit{pȕtovao}, which would basically mean that the type \textit{putòvati}\textasciitilde{}\textit{pùtujēm}\textasciitilde{}\textit{pȕtovao} moved to the existing type \textit{rȁtovati}\textasciitilde{}\textit{rȁtujēm}\textasciitilde{}\textit{rȁtovao}. Foreshadowing our analysis from the following section, it seems like this did not happen because introducing the form \textit{pȕtujēm} would not improve anything from the perspective of \textsc{StemStress}, because the stem is already stressed in the existing form \textit{pùtujēm}.

\section{Choosing the right analysis: \textsc{StemStress} vs \textsc{ParadigmUniformity}} \label{sec:kager:3}
Out of the three options for a formal analysis presented in \sectref{sec:kager:1}, \textsc{StemFaith} is clearly the worst match for the data presented. The case studies in \sectref{sec:kager:subsec22},  \sectref{sec:kager:subsec23} and \sectref{sec:kager:subsec25} show no compelling cases of lexically prominent endings. The data described in  \sectref{sec:kager:subsec24} do appear like a case of an accented ending (\textit{{-u}\textsubscript{H}}) that loses ground. However, all relevant data come from words with toneless stems, in which there is no \isi{lexical stem} prominence to be protected by \textsc{StemFaith}. Moreover, as discussed in \sectref{sec:kager:subsec24}, upon closer scrutiny, there seems to be no convincing evidence that there ever was a system in which the case ending \textit{-u\textsubscript{H}} could cause a stress-shift in the stem unless supported by a stored allomorph of the stem (so essentially not violating \textsc{StemFaith}). In sum, \textit{-u\textsubscript{H}} never caused and never causes any violations of \textsc{StemFaith}. If it did, there would exist paradigms of the type illustrated below. Note that \textit{Jùpiter} `Jupiter' is a noun with a lexical H: /jupi\textsubscript{H}ter/.

\begin{table}
\caption{The non-existent and existent paradigms of \textit{Jupiter} `Jupiter'}
\label{tab:kager:7}
\begin{tabular}{  l l l }
  \lsptoprule
& Non-existent  & Existent\\
 \midrule
Nominative/Accusative  &  Jùpiter  &   Jùpiter\\
Genitive  &  Jùpiter-a  & Jùpiter-a\\
Instrumental  &  Jùpiter-om & Jùpiter-om\\
Dative  & Jùpiter-u & Jùpiter-u\\
Locative  & \textbf{Jupitèr-u} & \textbf{Jùpiter-u}\\
  \lspbottomrule
 \end{tabular}
\end{table}

\newpage
As hinted in \sectref{sec:kager:1} the difficult part is distinguishing between the remaining two options: \textsc{ParadigmUniformity} and \textsc{StemStress}. All four changes we have discussed so far can in principle be seen as improving the uniformity of the paradigms. This is evident from the uniform paradigms that have resulted from the changes, illustrated by \tabref{tab:kager:3}, \tabref{tab:kager:4}, and \tabref{tab:kager:5}, in which the number of prosodic allomorphs of the stem is now 1. Even the new paradigm in  \tabref{tab:kager:6} improves on \textsc{ParadigmUniformity} to a certain extent, since the new \isi{infinitive}'s \isi{prosody} is shared with the \isi{past participle} (see \textit{putòvati} → \textit{pȕtovati}) which makes the number of prosodic allomorphs of the stem drop from 3 to 2. However, the survival of the present \isi{tense} \textit{pùtujēm} shows that tonal paradigm uniformity is not enforced when stress is already on the stem. Indeed, based on what we have seen so far, the \textsc{ParadigmUniformity} constraint in charge would have to be one only considering the location of stress, while remaining blind to tonal information: \textsc{\isi{ParadigmUniformity}(Stress Location)}.

The question is then: would \textsc{\isi{ParadigmUniformity}(Stress Location)} do the same job as \textsc{StemStress}? The answer is that in all the cases we have seen \textsc{\isi{ParadigmUniformity}(Stress Location)} does indeed predict the right result and the fact that stress ends up on the stem simply follows from the fact that most paradigm cells had stressed stems already. However, \textsc{\isi{ParadigmUniformity}(Stress Location)} seems to have a different problem: it predicts several changes that did not happen. Below we show two extremely frequent paradigms, which are also extremely stable throughout the history of standard \ili{Serbo-Croatian}.

\begin{table}
\caption{The present-tense paradigm of \textit{kopirati} `copy'}
\label{tab:kager:8}
\begin{tabular}{ l l l }
  \lsptoprule
& \textsc{sg}  & \textsc{pl}\\
 \midrule
 1  &  kòpīr-ā-m  &   kòpīr-ā-mo \\
2  &  kòpīr-ā-š  & kòpīr-ā-te \\
3  &  kòpīr-ā & kopír-a-jū \\
  \lspbottomrule
 \end{tabular}
\end{table}

\begin{table}
\caption{The singular paradigm of \textit{delfin} `dolphin'}
\label{tab:kager:9}
 \begin{tabular}{   l l }
    \lsptoprule
 Nominative  & dèlfīn  \\
Genitive/Accusative  & delfín-a \\
Instrumental   &  delfín-om \\
Dative/Locative   &  delfín-u \\
\lspbottomrule
 \end{tabular}
\end{table}

An analysis based on a promotion of \textsc{\isi{ParadigmUniformity}(Stress Location)} would predict such paradigms to regularise. One of the allomorphs should then spread to the whole paradigm, resulting either in forms like *\textit{kopír-ā-m} `I copy' (if the 3\textsc{pl} allomorph was to spread to other forms) or in forms like *\textit{kòpīr-a-jū} `they copy' (if the allomorph attested in all the other persons spread to 3\textsc{pl}). We have not found any attestation for such forms.

In sum, changes seem to only have happened if they rendered unstressed stems stressed. Hence the driving force behind this change is a promotion of \textsc{StemStress}. Additional evidence comes from stress varieties of standard \ili{Serbo-Croatian} presented in the next section.

\section{Additional evidence: Stress varieties} \label{sec:kager:4}
Stress-only standard varieties emerged in cities outside the \ili{Neo-Štokavian} area (Zagreb, Rijeka, Pula, Bor, Niš, etc.). They are virtually undescribed: all the data presented here come from our data collection in the five cities listed above.
Stress-only standard varieties are generally seen as a hybrid between \ili{Neo-Štokavian} and Non-\ili{Neo-Štokavian} varieties. However, as shown by \citet{SimonovicKager2017}, while indeed often combining some stress patterns from \ili{Neo-Štokavian} and some from Non-\ili{Neo-Štokavian}, these varieties do not simply allow both in free variation, but rather impose their own restrictions and generalisations. One such generalisation, absent from both \ili{Neo-Štokavian} and Non-\ili{Neo-Štokavian} dialects, is that stress always falls on the stem. This is quite surprising given the fact that most Non-\ili{Neo-Štokavian} dialects have paradigms in which the stem is consistently unstressed. One such paradigm is illustrated in  \tabref{tab:kager:10}. (We only mark the stress because some of these dialects only have stress, and the tonal dialects may have different tonal patterns).

\begin{table}
\caption{The present-tense paradigm of \textit{bežati} `flee' in most Non-Neo-Štokavian dialects}
\label{tab:kager:10}
 \begin{tabular}{ l  l  l }
  \lsptoprule
& \textsc{sg}  & \textsc{pl} \\
 \midrule
 1  &  beˈž-i-m  &   beˈž-i-mo \\
2  &  beˈž-i-š  & beˈž-i-te \\
3  &  beˈž-i & beˈž-e \\
  \lspbottomrule
 \end{tabular}
\end{table}

\newpage
Paradigms of the type illustrated above never made it into the stress-only standard varieties. Preserving such a paradigm in a stress variety should not present any problem in terms of prosodic constraints, because the stress varieties, unlike \ili{Neo-Štokavian}, allow final stress (e.g. in \textit{kriˈstal} cf. \ili{Neo-Štokavian} \textit{krìstāl}). \textsc{ParadigmUniformity} cannot offer an account for the systematic selection of the paradigms with stem stress because the Non-\ili{Neo-Štokavian} paradigm above is perfectly uniform. Again, the correct result is predicted by promoting \textsc{StemStress}. Stress-only standard \ili{Serbo-Croatian} varieties are unique in \ili{Serbo-Croatian} (and possibly also entire \ili{Slavic}) for having lost any \isi{prosodic prominence} on non-stem material while still having lexical \isi{prosody}. In that sense, they seem to be spearheading the change to come.
In the following section, we summarise what the change is and how it can be accounted for.

\section{Historical context: From automatic retraction to promoting \textsc{StemStress}} \label{sec:kager:5}

The general move towards a system with stem stress is the final step in a chain of language changes initiated by the \ili{Neo-Štokavian} retractions in the 15th century \citep[162--168]{Bethin2006} which created the rising accents. This first step amounted to an automatic leftward tone spreading from all non-initial syllables, thereby removing (a) all cases of final stress and (b) a huge portion of stress from \isi{inflectional} endings.
\begin{description}
    \item[Step 1:] \textit{bež-ī-tȅ} → \textit{bež-í-te} (tone/\isi{stress shift} + removal of final stress)
\end{description}

\noindent The remaining portion of stress on non-stem material was targeted by the changes discussed here.

\begin{description}
    \item[Step 2:] \textit{bež-í-te} → \textit{bèž-ī-te}  (complete removal of non-stem stress)
\end{description}

\noindent There are several possible accelerators of step 2. Dialect contact played a role: \ili{Neo-Štokavian} has been the standard since mid-19th century, which led to extensive dialect contact in the whole language area. More important than external factors is the issue of the driving force – the linguistic factor that set the change in motion and determined in which contexts it took place. As argued above, there are general reasons behind this \isi{language change} in terms of \isi{markedness}. In \isi{OT} literature, this translates into constraint interactions favouring stem stress \citep[e.g.][]{Revithiadou1999, Alderete2001, Kager2000}. The most direct result is achieved by promoting a single \isi{markedness constraint}: \textsc{StemStress}.
Step 2 in the considered \isi{language change} seems to show a rather advanced stage in promoting \textsc{StemStress}. After step 1, stress had been removed from most unambiguous \isi{inflectional} endings (\textit{-te} in the example above only expresses phi-features). Next, in step 2, stress is removed from the remaining non-stem material (theme vowels, augments etc.).


\section{Conclusions} \label{sec:kager:6}

In this contribution, we have presented an account of several diachronic changes within standard \ili{Serbo-Croatian}, showing that there is the same underlying force behind all of them: strengthening stem \isi{prosody}. We assumed \isi{OT} constraints as a tool for formalising the force behind \isi{language change} and using this tool proved to be useful, enabling us to distinguish between different processes (e.g. singling out the process discussed in  \sectref{sec:kager:subsec24}). This inquiry resulted in identifying \textsc{StemStress} as the constraint that got promoted, leading to a number of seemingly unrelated changes in different morphological contexts. Crucially, considering any of the changes in isolation would not have revealed the presented generalisation.

\section*{Abbreviations}

\begin{tabularx}{.45\textwidth}{lX}
\textsc{1}&1st person\\
\textsc{2}&2nd person\\
\textsc{3}&3rd person\\
\textsc{dat}&{dative}\\
\textsc{gen}&genitive\\
H&{high tone}\\
\textsc{inf}&{infinitive}\\
\end{tabularx}
\begin{tabularx}{.45\textwidth}{lX}
\textsc{ins}&instrumental\\
\textsc{loc}&locative\\
\textsc{nom}&nominative\\
{OT}&Optimality Theory\\
\textsc{\textsc{pst}}&past\\
\textsc{pl}&plural\\
\textsc{sg}&singular\\
\end{tabularx}

\section*{Acknowledgements}

We are thankful to the audience of FDSL 12.5 and the two anonymous reviewers for the useful comments and suggestions. We acknowledge financial support from the Slovenian Research Agency (program No. P6-0382).

%\section*{Appendix: }

\sloppy
\printbibliography[heading=subbibliography,notkeyword=this]

\il{Serbo-Croatian|)}
\end{document}
