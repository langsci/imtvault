\documentclass[output=paper,nonflat,newtxmath]{langsci/langscibook}
\ChapterDOI{10.5281/zenodo.3764855}
%\bibliography{localbibliography}
%\input{localpackages.tex}
%%Copy this to localcommands.tex

\usepackage[english]{babel}
\usepackage{amsmath}
\usepackage{amssymb,amsfonts,textcomp}
\usepackage{array}
\usepackage{hhline}
\usepackage{hyperref}

\newenvironment{styleStandard}{}{}
\newenvironment{stylelsAbstract}{}{}
\newenvironment{stylelsSectioni}{}{}
\newenvironment{stylelsSectionii}{}{}
\newenvironment{stylelsBulletList}{}{}
\newenvironment{styleBibliographyi}{}{}
\newenvironment{listWWNumxxvleveli}{}{}
\newenvironment{listWWNumxxvlevelii}{}{}
\newenvironment{listWWNumxxvleveliii}{}{}
\newenvironment{listWWNumxxvleveliv}{}{}
\newenvironment{listWWNumixleveli}{}{}
\newenvironment{listWWNumixlevelii}{}{}
\newenvironment{listWWNumixleveliii}{}{}
\newenvironment{listWWNumixleveliv}{}{}

\newcommand\textstyleListLabelxvi[1]{#1}
\newcommand\labellistWWNumxxvleveli{\thelistWWNumxxvleveli.}
\newcommand\labellistWWNumxxvlevelii{\thelistWWNumxxvlevelii.}
\newcommand\labellistWWNumxxvleveliii{\thelistWWNumxxvleveliii.}
\newcommand\labellistWWNumxxvleveliv{\thelistWWNumxxvleveliv.}
\newcommand\labellistWWNumixleveli{[F0B7?]}
\newcommand\labellistWWNumixlevelii{\textstyleListLabelxvi{o}}
\newcommand\labellistWWNumixleveliii{[F0A7?]}
\newcommand\labellistWWNumixleveliv{[F0B7?]}

\newcounter{listWWNumxxvleveli}
\newcounter{listWWNumxxvlevelii}[listWWNumxxvleveli]
\newcounter{listWWNumxxvleveliii}[listWWNumxxvlevelii]
\newcounter{listWWNumxxvleveliv}[listWWNumxxvleveliii]
\newcounter{itemize} 
\IfFileExists{../localcommands.tex}{
  \input{../localpackages}
  %Copy this to localcommands.tex

\usepackage[english]{babel}
\usepackage{amsmath}
\usepackage{amssymb,amsfonts,textcomp}
\usepackage{array}
\usepackage{hhline}
\usepackage{hyperref}

\newenvironment{styleStandard}{}{}
\newenvironment{stylelsAbstract}{}{}
\newenvironment{stylelsSectioni}{}{}
\newenvironment{stylelsSectionii}{}{}
\newenvironment{stylelsBulletList}{}{}
\newenvironment{styleBibliographyi}{}{}
\newenvironment{listWWNumxxvleveli}{}{}
\newenvironment{listWWNumxxvlevelii}{}{}
\newenvironment{listWWNumxxvleveliii}{}{}
\newenvironment{listWWNumxxvleveliv}{}{}
\newenvironment{listWWNumixleveli}{}{}
\newenvironment{listWWNumixlevelii}{}{}
\newenvironment{listWWNumixleveliii}{}{}
\newenvironment{listWWNumixleveliv}{}{}

\newcommand\textstyleListLabelxvi[1]{#1}
\newcommand\labellistWWNumxxvleveli{\thelistWWNumxxvleveli.}
\newcommand\labellistWWNumxxvlevelii{\thelistWWNumxxvlevelii.}
\newcommand\labellistWWNumxxvleveliii{\thelistWWNumxxvleveliii.}
\newcommand\labellistWWNumxxvleveliv{\thelistWWNumxxvleveliv.}
\newcommand\labellistWWNumixleveli{[F0B7?]}
\newcommand\labellistWWNumixlevelii{\textstyleListLabelxvi{o}}
\newcommand\labellistWWNumixleveliii{[F0A7?]}
\newcommand\labellistWWNumixleveliv{[F0B7?]}

\newcounter{listWWNumxxvleveli}
\newcounter{listWWNumxxvlevelii}[listWWNumxxvleveli]
\newcounter{listWWNumxxvleveliii}[listWWNumxxvlevelii]
\newcounter{listWWNumxxvleveliv}[listWWNumxxvleveliii]
\newcounter{itemize} 
  \input{../localhyphenation}
  \togglepaper[7]%%chapternumber
}{}
%\togglepaper[7]



% \title{A syntactic re-analysis of the Slovenian impersonal \textit{se}-construction}
% \title{A syntactic re-analysis of the Slovenian impersonal se-construction}

\markuptitle{A syntactic re-analysis of the Slovenian impersonal \textit{se}-construction}{A syntactic re-analysis of the Slovenian impersonal se-construction}
\renewcommand{\lsCollectionPaperFooterTitle}{A syntactic re-analysis of the Slovenian impersonal \noexpand\textit{se}-construction}

\author{Jakob Lenardič\affiliation{University of Ljubljana}}


\abstract{The following paper discusses the syntactic derivation of the Slovenian impersonal \textit{se}-construction (e.g. \textit{Gradilo se je hišo.\textsc{acc}} `People were building a house'). On the one hand, the paper argues for an analysis of the construction in which all the thematic arguments selected by the predicate normally enter the syntactic derivation (e.g. \citealt{riverosheppard2003}; contra \citealt{marelj2004,grahek2008}). To this end, the paper discusses the construction in relation to the causative alternation (e.g. \citealt{alexiadouet2015}), which shows that agentive participation is invariably entailed in contrast to the personal variant (\textit{Gradila se je hiša.\textsc{nom}}). On the other hand, the paper shows that \citeposst{riverosheppard2003} syntactic derivation is set up in such a way that it cannot offer an explanation as to why the construction only allows unergative verbs. In its stead, the paper offers a new analysis of the construction within the theory of the Voice domain (e.g. \citealt{legate2014,schaffer2017}). The new analysis posits that \textit{se} is the head of an argument-introducing thematic VoiceP and is a separate element from the phonologically-null thematic argument that the construction introduces. Ungrammaticality arises in the case of unaccusative verbs because such predicates merge the null thematic argument within VP yet still combine with \textit{se}, which introduces a participant variable into the derivation that cannot be saturated.

\keywords{argument structure, impersonal construction, reflexive clitics, Slovenian, unaccusative verbs}
}



\begin{document}
\maketitle
\il{Slovenian|(}
\section{Introduction}
The paper discusses the \textsc{\isi{impersonal} \textit{se}-construction} in \ili{Slovenian}, as exemplified by \REF{ex:lenardic:1}, in terms of the syntactic representation of its thematic arguments. The main morphosyntactic features of this construction, which will be discussed in the paper, include (i) the \isi{clitic} \textit{se}, (ii) the \isi{accusative case} on the \isi{internal argument} assigned by a \isi{transitive predicate} in absence of an overt \isi{external argument} bearing \isi{nominative case}, and (iii) default (singular neuter) agreement which follows from the superficial omission of the \isi{external argument}.

\ea  \label{ex:lenardic:1}
\gll Celo leto se je gradilo hišo.\\
   whole year \textsc{se} \textsc{aux}.\textsc{3sg} build.\textsc{ptcp.sg.n} house.\textsc{acc}\\
\glt `People were building the house for a whole year.'
\z

\noindent In the formal literature, there are generally two opposing approaches to the number of arguments that are present in the \isi{syntactic derivation} of \REF{ex:lenardic:1}. The 1-argument analysis (\citealt{marelj2004}, \citealt{grahek2008}) proposes that the derivation mimics the surface structure, so that the \isi{external argument} is excluded from the syntactic component and is only implicitly present in the semantics. The 2-argument analysis (\citealt{riverosheppard2003}) regards the structure as containing – in addition to the overt \isi{internal argument} \textit{hišo} `house' – a phonologically null \isi{pronoun} that fills the \isi{external argument} slot in the syntactic representation and semantically corresponds to the implicit \isi{agentive} participants paraphrased as `people' in the \ili{English} translation.

In this paper, I first present the empirical problems of the syntactic set-ups of such existing accounts. On the one hand, I defend the 2-argument analysis by discussing transitive \isi{impersonal} \textit{se}-sentences headed by causative verbs, which show that the construction entails the participation of a \isi{volitional initiator} (i.e., an agent). This fact is surprising on the 1-argument analysis given that causative verbs are in contemporary morphosyntactic frameworks analysed as predicates that are not lexically specified for an external causer or agent and consequently give rise to a non-\isi{agentive} anticausative interpretation when they do not combine with an external \isi{thematic argument} during derivation (e.g. \citealt{alexiadouet2015}). On the other hand, I show that the \isi{syntactic derivation} that underlies \citeposst{riverosheppard2003} analysis is set up in such a way that it cannot offer a principled explanation as to why the \isi{impersonal construction} does not allow unaccusative verbs in the case of intransitive predicates.

\largerpage[-1]
To account for such empirical gaps, I present a new analysis of the \isi{impersonal} \textit{se}-construction that is cast within the theory of the \isi{Voice} domain (\citealt{schaffer2017}). The core idea is that the \isi{clitic} \textit{se} is re-analysed as the head of a thematic \isi{VoiceP} which introduces an \isi{initiator variable} into the semantic derivation, and that the null argument is a pro\textsubscript{\textsc{imp}} without $\phi$-features (\citealt{Fenger2017}, \citealt{ackema2013, ackema2018})  and with unmarked \isi{nominative case} (\citealt{kornfilt2015}). In this sense, the analysis follows the 2-argument account in that all thematic arguments of a \isi{transitive predicate} normally enter the \isi{syntactic derivation}. In the case of unaccusative verbs, ungrammaticality arises because such predicates combine with pro\textsubscript{\textsc{imp}} within VP to saturate the theme variable, yet still merge with \textit{se}, which introduces an \isi{open initiator variable} that cannot be saturated.

The paper is structured as follows. \sectref{sec2} discusses \isi{impersonal} \textit{se}-sentences with transitive predicates in relation to the \isi{causative alternation}, which shows that a 2-argument analysis fares better empirically than the 1-argument analysis. \sectref{sec3} argues that the inadmissibility of \isi{unaccusative predicates} is problematic for the core syntactic assumptions of \citeposst{riverosheppard2003} account. \sectref{sec4} lays out the new analysis. \sectref{sec5} is the conclusion.

\section{The number of arguments in transitive impersonal \textit{se}-sentences – evidence from causative predicates} \label{sec2}

\subsection{The clitic \textit{se} as a case absorber}

In \ili{Slovenian}, the \isi{impersonal} \textit{se}-construction \REF{ex:lenardic: 2} alternates with an agreeing personal variant, which generally has the same interpretation if it is headed by a \isi{transitive predicate} such as \textit{spoštovati} `respect' or \textit{graditi} `build' \REF{ex:lenardic: 3}.

\begin{exe}
\ex  {Impersonal} \textit{se}-{sentences}  \label{ex:lenardic: 2}
\begin{xlist}
\ex \gll Starše se spoštuje.\\
parents.\textsc{acc} \textsc{se} respect.\textsc{3sg}\\
\trans `Parents are respected.' $/$ `People respect their parents.'
\ex \gll Hišo se gradi.\\
house.\textsc{acc} \textsc{se} build.\textsc{3sg}\\
\trans `The house is being built.' $/$ `People were building a house.'
\end{xlist}
\ex {Personal} \textit{se}-{sentences} \label{ex:lenardic: 3}
\begin{xlist}
\ex \gll Starši se spoštujejo.\\
parents.\textsc{nom} \textsc{se} respect.\textsc{3pl}\\
\trans `Parents are respected.' $/$ `People respect their parents.'
\ex \gll Hiša se gradi.\\
house.\textsc{nom} \textsc{se} build.\textsc{3sg}\\
\trans `The house is being built.' $/$ `People were building a house.'
\end{xlist}
\end{exe}


\newpage
\noindent \citeauthor{marelj2004} (\citeyear{marelj2004}: 268) has proposed that the interpretative equivalence points to a 1-argument analysis of both constructions.\footnote{To be clear, \citet{marelj2004} does not discuss \ili{Slovenian} data specifically, but rather focuses on other \ili{Slavic} languages, such as \ili{Polish} and \ili{Serbo-Croatian}, in which a \textit{se}-sentence can also surface either as a non-agreeing \isi{impersonal} or as an agreeing personal construction with the same structural and interpretative characteristics as the \ili{Slovenian} sentences under \REF{ex:lenardic: 2} and \REF{ex:lenardic: 3}. For cross-linguistic examples, the reader is referred to \citeauthor{marelj2004} (\citeyear{marelj2004}: 267--270).} Concretely, she claims that \textit{se} indirectly reduces the \isi{argument structure} of a predicate because it is first and foremost a \isi{case absorber} (see also \citealt{givon1998,franks1995}). In this respect, \textit{se} can either absorb \isi{nominative case}, resulting in the \isi{impersonal} sentences in \REF{ex:lenardic: 2}, or \isi{accusative case}, resulting in the corresponding personal variants in \REF{ex:lenardic: 3}. Crucially, since \textit{se} is a \isi{case absorber}, it ensures that a DP corresponding to an external \isi{thematic argument} is excluded from a \isi{syntactic argument} position in both personal and \isi{impersonal} variants, so the \isi{external argument} of the predicates \textit{respect} and \textit{build}  is only present in the semantic component and is invariably interpreted as an indefinite group of people in all \textit{se}-sentences.

For \ili{Slovenian}, this approach to the \isi{argument structure} of \textit{se}-sentences is also the most recent one in the formal literature, proposed by \citet{grahek2008}. Although \citeauthor{grahek2008} does not offer an explicit formal account, she makes a similar claim that “both personal and \isi{impersonal} [\textit{se}-sentences] contain the same type of \textit{se} which reduces the human subject role during their derivation'' and that “the choice between nominative and \isi{accusative}/\isi{genitive} is not associated with any interpretative difference'' (\citealt{grahek2008}: 44--48).

However, the interpretative equivalence does not hold for all predicates. With causative predicates like \textit{odpreti} `open' and \textit{potopiti} `sink', it is only the \isi{impersonal} variants in \REF{ex:lenardic: 4} that entail that the reported events involved a \isi{volitional initiator}; i.e., an agent.

\begin{exe}
\ex \label{ex:lenardic: 4}
\begin{xlist}
\ex \gll Vrata se je odprlo (*\hspace{-2pt} same od sebe).\\
door.\textsc{acc.pl} \textsc{se} \textsc{aux}.\textsc{3sg} open.\textsc{ptcp.sg.n} {} all by itself\\
\trans `The door was opened by someone.'\\ Impossible: `The door opened (by itself).'
\ex \label{ex:lenardic: 4b} \gll Ladjo se je potopilo (*\hspace{-2pt} samo od sebe).\\
ship.\textsc{acc.sg.f} \textsc{se} \textsc{aux}.\textsc{3sg} sink.\textsc{ptcp.sg.n} {} all by itself\\
\trans `The ship was sunk by someone.'\\ Impossible: `The ship sank (by itself).'
\end{xlist}
\ex \label{ex:lenardic: 5} \begin{xlist}
\ex \gll Vrata so se odprla (\hspace{-2pt} sama od sebe).\\
door.\textsc{nom.pl} \textsc{aux}.\textsc{3pl} \textsc{se} open.\textsc{ptcp.pl} {} all by itself\\
\trans `The door opened (by itself.)'
\ex \gll Ladja se je potopila (\hspace{-2pt} sama od sebe).\\
ship.\textsc{nom.sg.f} \textsc{se} \textsc{aux}.\textsc{3sg} sink.\textsc{ptcp.sg.f} {} all by itself\\
\trans `The ship sank (by itself).'
\end{xlist}
\end{exe}

\noindent In the remainder of this section, I will argue that this interpretative non-equiv\-a\-lence speaks in favour of a 2-argument analysis of the transitive \isi{impersonal} \textit{se}-construction.

\subsection{The (anti)causative alternation and the impersonal \textit{se}-construction}

We begin with a brief review of the recent literature on the cross-linguistic properties of the \isi{causative alternation} (e.g. \citealt{hovav1995}; \citealt{alexiadou2010}; \citealt{alexiadouet2015}; \citealt{schaffer2017}, amongst others). These authors take causative verbs like \textit{open} to be   unique among predicates because they are the only ones that are not specified for a certain type of initiator at the level of the lexicon. As an apparent consequence in \ili{English}, they can either appear in the transitive causative construction, as exemplified by sentence \REF{soop}, or in the intransitive anticausative construction, as exemplified by sentence \REF{thdo}.

\begin{exe}
\ex
\begin{xlist}
\ex \label{soop} Someone opened the door.
\ex \label{thdo} The door opened.
\end{xlist}
\end{exe}

\noindent Sentence \REF{thdo} is structurally poorer than \REF{soop} -- crucially, it lacks in its syntactic representation a \isi{functional projection} like \textit{v}P or \isi{VoiceP}, which in the transitive variant \REF{soop} introduces the \isi{external argument} DP (\citealt{kratzer1996}).  This alternation in \isi{argument structure} is possible because the causative predicate \textit{open} is not lexically specified for an agent/causer (\citealt{alexiadouet2015}). In this sense, causative predicates contrast with predicates like \textit{read}, which are lexically specified for an \isi{external participant} and which consequently have to combine with a \isi{functional head} like little \textit{v} or \isi{Voice} so as to introduce the \isi{external argument}. For this reason, a sentence like \REF{thbo} -- contrary to \REF{thdo} -- is ungrammatical.

\begin{exe}
\ex[*]{The book read.} \label{thbo}
\end{exe}

\noindent In contrast with \ili{English} \REF{thdo}, \ili{Slovenian} intransitive sentences interpreted as anticausatives are additionally marked with the \isi{clitic} \textit{se}, as seen in the examples under \REF{ex:lenardic: 5}. In spite of the \isi{clitic}’s presence in overt syntax, such \textit{se}-sentences do not differ in interpretation from the unmarked variants in \ili{English}, as shown by the corresponding translations. Since the \isi{clitic} does not make an interpretative contribution to the construction, \citet{schaffer2017}, who discusses \textit{se}-anticausatives in \ili{Romance} languages, analyses it as a semantically-vacuous specifier of a \isi{VoiceP} (the equivalent of \citepossalt{chomsky2001} \textit{v}P) whose head is also semantically vacuous and takes the VP as its complement.

\begin{exe}
\ex \label{ex:lenardic: 6} {[\textsubscript{VoiceP} \textsc{se} [\textsubscript{\isi{Voice}$'$} \isi{Voice} [\textsubscript{VP} open door.\textsc{nom}]]]}
\end{exe}

\noindent In the \isi{semantic representation}, the denotation of the VP in \REF{ex:lenardic: 6} is a function of the set of all change-of-state events in which the door becomes open \REF{ex:lenardic: 7}. By contrast, both \isi{Voice} and \textit{se} denote identity functions that map a semantic argument corresponding to a set of events onto itself \REF{ex:lenardic: 8}, so neither makes any semantic contribution  (\citealt{schaffer2017}).

\begin{exe}
\ex \label{ex:lenardic: 7} $\llbracket$VP$\rrbracket = \lambda e[\textsc{open}(e) \wedge \cnst{theme}(e,\textsc{door})]$
\ex \label{ex:lenardic: 8} $\llbracket\textsc{se}\rrbracket = \llbracket$\isi{Voice}$\rrbracket = \lambda P_{\langle s,t\rangle}[P]$
\end{exe}

\noindent Consequently, when VP combines with the semantically vacuous \isi{Voice} head and \textit{se}, the resultant interpretation only entails the existence of a change-of-state \isi{event} in which the door becomes open. That this is indeed how the \ili{Slovenian} \textit{se}-anticausatives under \REF{ex:lenardic: 5} are interpreted is further emphasized by the admissibility of the \textit{sama od sebe} phrase, which is the \ili{Slovenian} equivalent of the \ili{English} \textit{by itself} phrase. This phrase is used to highlight the fact that such sentences denote situations that can be conceptualized as though occurring “without outside help'' (\citealt{hovav1995}: 88); that is, without the presence of an agent or causer.\footnote{Note that, if the \textit{by itself} phrase is not merged, then the \isi{event} denoted by the anticausative construction is still conceptually compatible with an agent opening the door, as shown by the fact that the transitive sentence in \REF{soop} asymmetrically entails its anticausative variant in \REF{thdo} (\citealt{schafervivanco2016}). The point is that the possible agent in an anticausative construction is missing from its syntactic-\isi{semantic representation} due to the fact that \isi{Voice} and its specifier make no semantic contribution \REF{ex:lenardic: 8}.}

By contrast, the corresponding \isi{impersonal} variants in \REF{ex:lenardic: 4} are not interpreted as anticausatives, but rather as events externally caused by agents. Crucially, however, such an interpretation does not follow straightforwardly from the semantics of a causative predicate like \textit{odpreti} `open', considering the fact that such a predicate is taken to be cause unspecified in its lexical semantics and the VP that it projects only denotes the characteristic function of a set of events in which the door becomes open \REF{ex:lenardic: 7}. On top of that, \citeposst{marelj2004} 1-argument analysis for \isi{impersonal} sentences also predicts that a thematic \isi{external argument} has to be excluded because \textit{se} is analysed as a \isi{case absorber}. Note that, because of such case absorption on part of \textit{se}, the \isi{syntactic derivation} that \citeauthor{marelj2004} (\citeyear{marelj2004}: 268) proposes only allows a null \isi{expletive} in the grammatical \isi{subject position} of an \isi{impersonal} \textit{se}-sentence:
\begin{exe}
\ex \label{ex:lenardic: 9} {[\textsubscript{CP} door.\textsc{acc} [\textsubscript{TP} \textsc{expl} [\textsubscript{T$'$} \textsc{se} [\textsubscript{VP} open \st{door}]]]]}
\end{exe}

 \noindent However, if the \isi{grammatical subject} is semantically null in \isi{impersonal} \textit{se}-sen\-ten\-ces (\textsc{expl} in \REF{ex:lenardic: 9}), then it stands to reason that combining the denotation of the VP in \REF{ex:lenardic: 9}, which should be equivalent to \REF{ex:lenardic: 7}, with those of its c-commanding categories (which make no semantic contribution to expanding the \isi{argument structure} of the VP) would lead to an interpretation like `There was an \isi{event} in which the door became open'. This, however, would be the same interpretation as that of the anticausatives in \REF{ex:lenardic: 5}, contrary to the actual meaning.

To get the desired meaning, it rather seems that the VP of the \isi{impersonal construction} must be predicated of a thematic \isi{external argument}, just like in the case of the ordinary transitive sentence \REF{soop}. Here, the VP \textit{opened the door} combines via \isi{event} identification (\citealt{kratzer1996}) with the \isi{external argument} DP \textit{someone} located in the specifier of \textit{v}P/\isi{VoiceP} so as to bring about an external \isi{volitional initiator} into the semantics of the construction.

This is exactly what the 2-argument analysis of the \isi{impersonal} \textit{se}-construction predicts. In this approach, \citeauthor{riverosheppard2003} (\citeyear{riverosheppard2003}: 120) propose that transitive \isi{impersonal} \textit{se}-sentences have the following syntactic representation:

\begin{exe}
\ex \label{ex:lenardic: 10} {[\textsubscript{ClP} [\textsubscript{\isi{Cl}$'$} \textsc{se} [\textsubscript{TP} [\textsubscript{\textit{v}P} NP\textsubscript{1} [\textsubscript{VP} V DP\textsubscript{2}]]]]]}
\end{exe}

\noindent The representation in \REF{ex:lenardic: 10} involves two syntactically-projected thematic arguments. While DP\textsubscript{2} corresponds to the overt \isi{accusative} DP \textit{ladjo} in sentence \REF{ex:lenardic: 4b}, NP\textsubscript{1} is taken to be the \isi{external argument}, which is analysed as a phonologically-null \isi{pronoun} that semantically corresponds to an existential \isi{quantifier} over a group of humans (\citealt{riverosheppard2003}: 135).

\begin{exe}
\ex \label{ex:lenardic: 11} $\exists x$\textsubscript{[hum]}$[\textsc{open}(x$\textsubscript{[hum]}$, \textsc{door})]$
\end{exe}

\noindent The LF in \REF{ex:lenardic: 11} perfectly corresponds to the intuitive meaning of the \isi{impersonal} \textit{se}-sentences under \REF{ex:lenardic: 4}; that is, `There is a human (or a group of humans) $x$ such that $x$ opened the door', which is the desired result. What is crucial is that the \isi{syntactic derivation} of an \isi{impersonal} \textit{se}-sentence on such an account fundamentally differs from that of the personal variant, which cannot introduce the \isi{external argument} into the syntax due to the fact that its \isi{VoiceP} \REF{ex:lenardic: 6} is already filled with semantically vacuous material \REF{ex:lenardic: 8}. Consequently, the interpretative difference between the causative impersonals, which entail an initiator, and the corresponding anticausative personals, which do not, becomes obvious.

In sum, what we have shown in this section is that examples of the transitive \isi{impersonal construction} with causative predicates support a 2-argument analysis (\citealt{riverosheppard2003}) over a 1-argument analysis (e.g. \citealt{marelj2004,grahek2008}), as the latter does not account for the discussed interpretative differences between the \isi{impersonal} and personal variants; that is, it is unclear from the latter account why only the \isi{impersonal construction} should entail \isi{agentive participation}.

\section{Intransitive predicates and the impersonal \textit{se}-construction} \label{sec3}

\subsection{The data}

It has been cross-linguistically observed (e.g. \citealt{cinque1995} and \citealt{dalessandro2008} for \ili{Italian}; \citealt{ilcmarvin2016} for \ili{Slovenian}) that the \isi{impersonal} \textit{se}-construction is licit with \isi{unergative} predicates, but unaccusative predicates are out. Consider the following contrast between the grammatical sentences with unergatives \REF{ex:lenardic: 12} and ungrammatical sentences with unaccusatives \REF{ex:lenardic: 13} in \ili{Slovenian}, adapted from \citeauthor{ilcmarvin2016} (\citeyear{ilcmarvin2016}: 152–154):

\begin{exe}
\ex \begin{xlist} \label{ex:lenardic: 12}
\ex \label{ex:lenardic: 12a} \gll Včeraj se je plesalo.\\
yesterday \textsc{se} \textsc{aux}.\textsc{3sg} dance.\textsc{ptcp.sg.n}\\
\trans `Some people danced yesterday.'
\ex \gll Včeraj se je cel dan spalo.\\
yesterday \textsc{se} \textsc{aux}.\textsc{3sg} whole day sleep.\textsc{ptcp.sg.n}\\
\trans `Some people were sleeping yesterday.'
\end {xlist}
\ex \label{ex:lenardic: 13}
\begin{xlist}
\ex[*]{\gll Včeraj se je umrlo v tej bolnici.\\
yesterday \textsc{se} \textsc{aux}.\textsc{3sg} die.\textsc{ptcp.sg.n} in this hospital\\
\trans Intended: `Some people died yesterday in this hospital.'} \label{VcUm}
\ex[*]{\gll Včeraj se je padlo na teh stopnicah.\\
yesterday \textsc{se} \textsc{aux}.\textsc{3sg} fall.\textsc{ptcp.sg.n} on these stairs.\\
\trans Intended: `Some people fell on these stairs yesterday.'}
\end{xlist}
\end{exe}

\noindent In this section, I will show that \citeposst{riverosheppard2003} analysis for the \ili{Slovenian} \isi{impersonal} \textit{se}-construction does not predict this contrast in acceptability.

\subsection{The syntax of impersonal \textit{se}-sentences in \citet{riverosheppard2003} and the Unaccusative Hypothesis} \label{sec3.2}
On \citeposst{riverosheppard2003} account, there are two key structural assumptions that underlie the \isi{impersonal} \textit{se}-construction. Both are tied to the feature configuration of the null argument NP in the  bracketed representation \REF{ex:lenardic: 14} of sentence \REF{ex:lenardic: 12a}, which is \isi{unergative} and thus acceptable.

\begin{exe}
\ex \label{ex:lenardic: 14} {[\textsubscript{ClP} [\textsubscript{\isi{Cl}$'$} \textsc{se}\textsubscript{\textbf{$\exists$}} [\textsubscript{TP} T\textsubscript{[$-\phi$]} [\textsubscript{VoiceP} \textbf{NP}\textsubscript{\textbf{[uCASE][iHUM]}} [\textsubscript{\isi{Voice}$'$} \isi{Voice} [\textsubscript{VP} dance]]]]]]}
\end{exe}

\noindent On the one hand, \citet{riverosheppard2003} assume that the \isi{external argument} NP has an uninterpretable \isi{case feature} that needs to be checked prior to spell-out. Since T, which is otherwise the canonical case-checker in ordinary finite sentences, lacks $\phi$-features because the construction is non-agreeing, \citet{riverosheppard2003} furthermore assume that NP bypasses TP and checks its \isi{case feature} against the \isi{clitic} by remerging in its specifier, as in \REF{ex:lenardic: 15}:

\begin{exe}
\ex \label{ex:lenardic: 15} {[\textsubscript{ClP} \textbf{NP}\textsubscript{\textbf{\st{[uCASE}][iHUM]}} [\textsubscript{\isi{Cl}$'$} \textsc{se}\textsubscript{\textbf{$\exists$}} [\textsubscript{TP} T\textsubscript{[$-\phi$]} [\textsubscript{VoiceP} \textbf{\st{NP}} [\textsubscript{\isi{Voice}$'$} \isi{Voice} [\textsubscript{VP} dance]]]]]]}
\end{exe}

\noindent On the other hand, \citet{riverosheppard2003} propose that  NP is a \isi{pronoun} with a human feature that is otherwise referentially deficient (i.e., NP is a “simplex-expression anaphor” in the sense of \citealt{reinhartreuland1993}). Because of this assumption, they claim that NP moves to SpecClP, where it repairs its deficiency by forming a chain with the \isi{clitic}, which is taken to correspond semantically to an existential operator (\citealt{riverosheppard2003}: 129), labelled here with the subscripted {$\exists$}. Thus in the LF in \REF{ex:lenardic: 11}, the \isi{quantifier} represents \textit{se}, while the bound variable corresponds to the copy/trace in SpecVoiceP.

On such a syntactic configuration, the derivation of the unaccusative sentences in \REF{ex:lenardic: 13} should be like \REF{ex:lenardic: 15}, with the only difference being the locus of the initial merge of the phonologically null NP. In this respect, the standard assumption of the \isi{Unaccusative Hypothesis} (beginning with \citealt{perlmutter1978}) is that an \isi{unaccusative predicate} differs from an \isi{unergative} one only in that it merges its sole argument within the VP (for instance, \citealt{burzio1986}, \citealt{alexiadouet2004}), and thus dispenses with a projection external to VP (like \textit{v}P in \citepossalt{chomsky2001} system or \isi{VoiceP} in \citepossalt{kratzer1996} system) that would introduce the argument in the case of an \isi{unergative} predicate. Translating this idea into the configuration of \citet{riverosheppard2003}, the \isi{impersonal construction} headed by an \isi{unaccusative predicate} should be as follows in syntax before movement:

\begin{exe}
\ex \label{ex:lenardic: 16} {[\textsubscript{ClP} [\textsubscript{\isi{Cl}$'$} \textsc{se}\textsubscript{\textbf{$\exists$}} [\textsubscript{TP} T\textsubscript{[$-\phi$]}  [\textsubscript{VP} die \textbf{NP}\textsubscript{\textbf{[uCASE][iHUM]}}]]]]}
\end{exe}

\noindent However, since \citeauthor{riverosheppard2003}’s analysis assumes the only relevant syntactic operation to be the remerger of the null NP in the specifier of ClP driven by feature checking/repairing referential deficiency, there is no obvious reason as to why \REF{ex:lenardic: 16} should not result in a grammatical sentence. Note that using this approach the derivation of the \isi{argument structure} of the \isi{impersonal construction} with an \isi{unaccusative predicate} should run in parallel to that of the perfectly licit \ili{English} structure in \REF{sope}, where the DP \textit{some people} also starts off VP internally and is then remerged in a higher position.

\begin{exe}
\ex \label{sope} {[\textsubscript{TP} some people [\textsubscript{T$'$} T [\textsubscript{VP} died \st{some people}]]]}
\end{exe}
\noindent Crucially, the null NP in \REF{ex:lenardic: 16} should also trivially remerge in SpecClP to check its \isi{case feature} and to form a chain with the \isi{clitic}, as in  the \isi{unergative} derivation in \REF{ex:lenardic: 15}. It seems, then, that such a configuration predicts that the unaccusative sentences in \REF{ex:lenardic: 13} should actually be just as grammatical as the \isi{unergative} ones in \REF{ex:lenardic: 12}.

\section{The proposal -- \textit{se} is the head of VoiceP} \label{sec4}
\largerpage[-1]
\subsection{\citeposst{schaffer2017} Voice typology} \label{sec4.1}

To account for the inadmissibility of unaccusative verbs, I will now propose a re-analysis of the \ili{Slovenian} \isi{impersonal} \textit{se}-construction that is rooted in the typology of \isi{Voice} heads presented in \citet{schaffer2017}. Following \citet{kratzer1996}, \isi{VoiceP} is the \isi{functional projection} atop VP which introduces the \isi{external argument} DP into the \isi{syntactic derivation}. The key idea behind \citeauthor{schaffer2017}’s proposal, which is an update of \citeauthor{kratzer1996}’s, is that the syntactic and semantic requirements of a \isi{Voice} head vary from one type of construction to another, as follows: (i) a \isi{Voice} head can introduce the \isi{external argument} variable into the \isi{semantic representation} in several different ways and (ii) a \isi{Voice} head may or may not have a syntactic need for a DP in its specifier slot.
For \ili{Slovenian}, this approach seems especially fruitful because it provides a straightforward explanation of the licensing properties of the so-called \textit{se}-passives (personal \textit{se}-structures in our terms), which differ from canonical passives in that they do not allow their entailed \isi{volitional initiator} (i.e., agent) to be expressed via the \ili{Slovenian} equivalent of the \textit{by}-phrase. To see this approach in action, consider the differences between the feature sets of the \isi{Voice} heads for the following three sentences, where the first feature in the brackets corresponds to the specifier requirement of \isi{Voice} and the second to its semantics (\citealt{schaffer2017}: 14).\footnote{Note that \citet{schaffer2017} does not discuss \ili{Slovenian} examples, but rather uses the following \textit{se}-sentence from \ili{French} to exemplify  the \isi{Voice} head in \REF{ex:lenardic: 19b} (\citealt{schaffer2017}: 16):
\begin{exe}
\ex \gll Trois maisons se sont louées (*\hspace{-2pt} par des touristes) hier.\\
three houses \textsc{se} aux rented {} by some tourists yesterday\\
\trans `Three houses were rented (by some tourists) yesterday.'
\end{exe}}

\begin{exe}
\ex \begin{xlist} \label{ex:lenardic: 17}
\ex \label{ex:lenardic: 17a} John read a book.
\ex \label{ex:lenardic: 17b} \isi{Voice} = \{D, $\lambda x\lambda e[\cnst{initiator}(x, e)]$\}
\end{xlist}
\ex \begin{xlist} \label{ex:lenardic: 18}
\ex \label{ex:lenardic: 18a} The book was read (by John).
\ex \isi{Voice} = \{/, $\lambda x\lambda e[\cnst{initiator}(x, e)]$\}
\end{xlist}
\ex \begin{xlist} \label{ex:lenardic: 19}
\ex \label{ex:lenardic: 19a} \gll Knjiga se je brala (*\hspace{-2pt} s strani Janeza).
\\
book.\textsc{nom}.\textsc{fem} \textsc{se} \textsc{aux}.\textsc{3sg} read.\textsc{ptcp.sg.f} {} on part Janez\\
\trans `The book was being read.'
\ex \label{ex:lenardic: 19b} \isi{Voice} = \{D, $\lambda e\exists x[\cnst{initiator}(x, e)]$\}
\end{xlist}
\end{exe}

\noindent On this approach, a canonical passive sentence \REF{ex:lenardic: 18} differs from its active variant \REF{ex:lenardic: 17} only in that its \isi{Voice} head lacks the D-feature, which accounts for the fact that the \isi{external argument} DP cannot directly enter the derivation. In semantics, however, this self-same \isi{Voice} head introduces an \isi{initiator variable}. Since this variable is open, it can be saturated  by a DP introduced  via the \textit{by}-phrase, which enters the derivation as a \isi{VoiceP} adjunct. By contrast, the \isi{Voice} head in the personal \textit{se}-structure in \REF{ex:lenardic: 19a}, which does entail \isi{agentive participation} much like \REF{ex:lenardic: 18a}, introduces an \isi{initiator variable} that is inherently closed off by the existential \isi{quantifier}. Consequently, merging a \textit{s-strani} `on-part-of' phrase is correctly predicted to be illicit since the \isi{external argument} DP that the phrase introduces does not have an open variable to saturate.  Finally, since \REF{ex:lenardic: 19b} has the D-feature, the \textit{se} \isi{clitic} is assumed to be a semantically-null DP that is merged in SpecVoiceP to satisfy this c-selectional requirement of \isi{Voice} (\citealt{schaffer2017}).

As to how the \isi{impersonal} \textit{se}-construction is positioned within this kind of \isi{Voice} theory, only a brief speculative account is provided in \citeauthor{schaffer2017} (\citeyear{schaffer2017}: Footnote 14). Following \citet{dalessandro2008}, he claims that it is \textit{se} itself that is a thematic \isi{syntactic argument} that expresses the initiator role in a transitive sentence, in contrast to \textit{se} being semantically-vacuous in the personal variant in \REF{ex:lenardic: 19}. However, it seems that this kind of analysis brings us back to the same problems that were discussed in the previous section – that is, unaccusatives should not be disallowed, since if \textit{se} behaves like an ordinary referring DP, there is no principled reason as to why it cannot also correspond to the \isi{internal argument} of an \isi{unaccusative predicate}, especially given that \isi{VoiceP} is not built in this case.

\subsection{The syntactic and semantic features of the new proposal} \label{SyntacticSemanticProposal}

To solve this issue, I will now propose a partial re-analysis of the construction, where the core idea is that \textit{se} itself does not correspond to a \isi{syntactic argument}, but is rather the overtly realized head of a \isi{VoiceP} that introduces an open \isi{participant variable} into the \isi{semantic representation}.\footnote{This assumption falls in line with the idea that clitics like \textit{se} are categorically ambiguous in that they can either function as full-fledged XP’s or X$^0$’s (\citealt{boskovic2001}: 31). In this sense, \textit{se} can have the characteristics of an XP in that it can be merged as the specifier of \isi{VoiceP} in personal \textit{se}-sentences \REF{ex:lenardic: 19a} or the characteristics of an X$^0$ in that it is the head of a \isi{VoiceP} in our proposed re-analysis for \isi{impersonal} \textit{se}-sentences. For a similar proposal for the \ili{Slovenian} negative marker (though with different syntactic repercussions), see \citet{ilc2011}.}  Since I maintain that the \isi{impersonal} \textit{se}-construction projects all of its arguments into the syntax, I follow \citeposst{riverosheppard2003} analysis in assuming that there is a phonologically-null \isi{pronoun} (henceforth, pro\textsubscript{\textsc{imp}}) corresponding to the \isi{external argument}  within the structure, which is distinct from \textit{se}.

The main grammatical characteristic of pro\textsubscript{\textsc{imp}} is that it is structurally impoverished -- crucially, it lacks in its morphosyntactic make-up a layer which encodes $\phi$-features. Following recent work on the morphosyntactic properties of person features (\citealt{ackema2013,ackema2018}), I assume that it is the lack of $\phi$-features that gives pro\textsubscript{\textsc{imp}} its idiosyncratically ambiguous interpretation with respect to its person reference. Concretely, this means that pro\textsubscript{\textsc{imp}} allows for at least  three readings, disambiguated by other sentential constituents or by the context: in existential sentences, (i) it can refer to a group of people that includes the speaker and/or \isi{addressee} \REF{ex:lenardic: 20}; (ii) it can refer to a group of people that excludes the speaker and/or \isi{addressee} \REF{ex:lenardic: 21}; and (iii) in the case of a generic context, it can refer to an arbitrary participant similar in interpretation to the \ili{English} generic \isi{pronoun} \textit{one} \REF{ex:lenardic: 22} (see also the discussion in \citealt{Fenger2017}, where \ili{Germanic} \textit{man}-type pronouns are also analysed as completely lacking $\phi$-features and thereby allowing for the same kinds of interpretation).
\largerpage[2]


\begin{exe}
\ex \label{ex:lenardic: 20} \gll Včeraj se je na zabavi ves čas plesalo. Imeli smo se prav super.\\
yesterday \textsc{se} \textsc{aux}.\textsc{3sg} on party all time dance.\textsc{ptcp.sg.n} have.\textsc{ptcp.pl.m} \textsc{aux}.\textsc{1pl} \textsc{se} just great\\
\trans `Yesterday, we were dancing all the time at the party and we had such fun.'
\ex \label{ex:lenardic: 21} \gll Včeraj se nas je obvestilo o novi knjigi o skladenjski teoriji.\\
yesterday \textsc{se} us.\textsc{acc} \textsc{aux}.\textsc{3sg} inform.\textsc{ptcp.sg.n} about new book on syntactic theory\\
\trans `Yesterday, someone/people told us about a new book on syntactic theory.'
\ex \label{ex:lenardic: 22} \gll Ko se gre v Italijo, se navadno je pašto.\\
when \textsc{se} go.\textsc{3sg} in Italy, \textsc{se} usually eat.\textsc{3sg} pasta.\textsc{acc}\\
\trans `When one goes to Italy, one usually eats pasta.'
\end{exe}

 \noindent The claim that pro\textsubscript{\textsc{imp}} is morphosyntactically impoverished is further motivated by the fact that the verbal element which shows $\phi$-feature contrasts is invariantly spelled out with third person singular features even though pro\textsubscript{\textsc{imp}} typically has a plural interpretation glossed as `people' (with a contextually determined person reference) in the translations.\footnote{More precisely, the number interpretation of pro\textsubscript{\textsc{imp}} actually seems to be ambiguous between a plural and a singular reading; e.g. it is unclear how many people are denoted by pro\textsubscript{\textsc{imp}} in \REF{ex:lenardic: 21}. However, this ambiguity  seems to be expected. In  the contemporary semantic/pragmatic approach to grammatical number (e.g. \citealt{sauerland2005}), a plural NP makes the same semantic contribution as the alternative singular NP (i.e., both logically mean `one or more referents'), while the actual `more than one meaning' of the former obtains through a process of pragmatic strengthening, by means of which the  intended pragmatic meaning of the singular NP is negated. Crucially, such strengthening occurs only because a plural NP and the alternative singular NP display a $\phi$-feature contrast in number and are seen as competitors for the intended number interpretation.  By contrast, the null \isi{pronoun} in \ili{Slovenian} \isi{impersonal} \textit{se}-sentences is invariant (i.e., there is no other null \isi{impersonal} \isi{pronoun} in \ili{Slovenian} with which it would contrast in number), so it does not have a competitor for the number interpretation. As a consequence, the process of pragmatic strengthening does not take place in this case.} Compare, for instance,  the first sentence in \REF{ex:lenardic: 20}, which is \isi{impersonal} and thus spells out the auxiliary with non-agreeing third person singular features (i.e., \textit{je}), with the second one, which is personal and thus spells out the auxiliary with first person plural features (i.e., \textit{smo}).\footnote{Consequently, the second sentence in \REF{ex:lenardic: 20} contains a typical agreeing pro with matching first person plural features as the \isi{grammatical subject}.} According to \citet{ackema2013}, such an apparent mismatch between number marking and meaning in \isi{impersonal} \textit{se}-structures is possible because third person singular represents the default spell-out of $\phi$-features,  and is therefore compatible with the absence of matching interpretable features in the \isi{pronoun}.

 In addition, the lack of $\phi$-features in the \isi{pronoun} explains a crucial distributional fact of the \ili{Slovenian} \isi{impersonal} \textit{se}-construction that has gone -- to the best of my knowledge -- unnoticed in the formal literature. Namely, \ili{Slovenian} \isi{impersonal} \textit{se}-sentences do not allow pro\textsubscript{\textsc{imp}} to be used in the structural \isi{object position} \REF{ex:lenardic: 23}. In this respect, \ili{Slovenian} pro\textsubscript{\textsc{imp}} patterns with the overt \isi{impersonal} \isi{pronoun} \textit{men} in \ili{Dutch} \REF{ex:lenardic: 24a}, and  contrasts with  generic \textit{one} in \ili{English}, which is allowed in such a position \REF{ex:lenardic: 24b}.

\begin{exe}
\ex[*]{\gll Ta ideja se spominja na vojno.\\
this idea.\textsc{nom} \textsc{se} reminds.\textsc{3sg} on war\\
\trans Intended: `This idea reminds one of the war.'} \label{ex:lenardic: 23}
\ex \label{ex:lenardic: 24} \begin{xlist}
\ex[*]{\gll Dit herinnert men aan de oorlong.\\
this reminds \textsc{imp} of the war\\\label{ex:lenardic: 24a}\\
\trans Intended: `This reminds one of the war.'}
\ex[]{This idea reminds one of the war. \label{ex:lenardic: 24b}}
\hfill (Examples under \REF{ex:lenardic: 24}  taken from \citealt{Fenger2017}: 298)
\end{xlist}
\end{exe}


 \noindent According to \citet{Fenger2017}, the \ili{Dutch} sentence in \REF{ex:lenardic: 24a} is ungrammatical because  \textit{men} cannot be assigned \isi{accusative case}, as the latter is  only compatible with those nominal constituents that are able to project a K(ase) Phrase (\citealt{bayer2001}). Crucially, only nominals that contain a rich enough morphosyntactic structure can project a \isi{KP}. In this respect, \citet{Fenger2017} claims that it is the lack of $\phi$-features in the \ili{Dutch} \isi{pronoun} \textit{men}  that precludes the projection of the \isi{KP}, whereas \ili{English} \textit{one} is richer in structure in the sense that it does contain a $\phi$-layer encoding person features, and can thus occupy the  \isi{object position} where it gets \isi{accusative case}.\footnote{According to \citet{Fenger2017}, \ili{English} \textit{one} has a set of $\phi$-features that are underspecified for person (see also \citealt{ackema2018}). Such underspecification means that the \isi{pronoun} ``must be compatible with any arbitrary choice of person, including the speaker/hearer" (\citealt{Fenger2017}: 307), which limits its distribution to generic contexts.}

Note, now, that the proposed lack of $\phi$-features on pro\textsubscript{\textsc{imp}} has repercussions for how case is assigned in \isi{impersonal} \textit{se}-sentences.   Minimalist theory has generally relied on what \citet{kornfilt2015} call “a positively-specified account of \isi{case assignment}'', which means that \isi{nominative case} is assigned to the \isi{grammatical subject} only under $\phi$-feature agreement with a specific \isi{functional head}, which is canonically taken to be T (\citealt{chomsky2001}). However, the thing is that if pro\textsubscript{\textsc{imp}} completely lacks $\phi$-features, then \isi{nominative case} cannot be assigned to it under the standard agreement relation, in which T, bearing uninterpretable $\phi$-features, would find its goal with matching interpretable $\phi$-features in the null \isi{pronoun}.

Recall from \sectref{sec3.2}, example \REF{ex:lenardic: 15}, that \citet{riverosheppard2003} circumvented this problem by assuming that \isi{nominative case} is assigned to the null \isi{pronoun} under a special variant of A-movement, whereby the null \isi{pronoun}  bypasses TP and checks its \isi{case feature} against a higher \isi{functional projection} headed by the \isi{clitic}. However, I would like to propose a solution that does not require recourse to such a stipulated modification of A-movement: that is,  \isi{case assignment} in \isi{impersonal} \textit{se}-sentences occurs configurationally (i.e., not under agreement), in the sense of recent approaches to \isi{dependent case} (e.g. \citealt{levin2015,kornfilt2015}).

For transitive \textit{se}-structures such as sentence \REF{ex:lenardic: 21} and the second clause in \REF{ex:lenardic: 22}, this entails that the \isi{accusative case} of the \isi{internal argument} DP  (e.g. \textit{pašto} `pasta' in \REF{ex:lenardic: 22}) is the \isi{dependent case}, which means that it is assigned “in opposition to another argument position'' (\citealt{marantz1991}: 24); that is, the \isi{internal argument} DP gets dependent \isi{accusative case} because it is in a position in which it is asymmetrically c-commanded by another DP (i.e., pro\textsubscript{\textsc{imp}}).\footnote{There is an additional requirement: in order to obtain dependent \isi{accusative case}, the DP must also not be independently assigned idiosyncratic case by a head which imposes its own lexical case requirements, as is typical of prepositions (\citepossalt{marantz1991} Disjunctive Case Hierarchy).} By contrast, \isi{nominative case} is not assigned to the c-commanding DP, but ``represent[s] the absence of any otherwise assigned case" (\citealt{kornfilt2015}: 298) and is thus possible when no agreement in $\phi$-features takes place.\footnote{For \ili{English}, \citeauthor{Fenger2017} (\citeyear{Fenger2017}: 313), following previous work (e.g. \citealt{schutze}), claims that unmarked case should be treated as different from default case, as the latter is spelled out only when the DP is part of a clauseless sentence, as in the exclamation \textit{Me, intelligent?!}} On this account, the syntactic structure of a transitive  \isi{impersonal} \textit{se}-sentence is as in \figref{ex:lenardic: tree}, with the dotted line representing \isi{dependent case} assignment between pro\textsubscript{\textsc{imp}} and the overt \isi{internal argument} DP.

\begin{figure}
\begin{forest}
[\isi{VoiceP}
[pro\textsubscript{\textsc{imp}}, name = ext] [\isi{Voice}$'$
[\isi{Voice}\\
\textsc{se}] [VP
[V] [DP, name = int]
]
]
]
\draw[<-, dotted] (int) .. controls +(south west:4em) and +(south west:4em) .. node[midway, sloped, below] {\small\textsc{accusative}} (ext);
\end{forest}
\caption{Case assignment in impersonal \textit{se}-sentences} \label{ex:lenardic: tree}
\end{figure}

 Having \isi{dependent case} assignment in mind, let's now return to the \ili{Slovenian} example in \REF{ex:lenardic: 23}, where pro\textsubscript{\textsc{imp}} functions as the grammatical object in the same manner as \textit{men} in the \ili{Dutch} example \REF{ex:lenardic: 24a}. In this case, and in contrast with \figref{ex:lenardic: tree}, it is pro\textsubscript{\textsc{imp}} that is assigned dependent \isi{accusative case} because it is c-commanded by the \isi{grammatical subject} \textit{ta ideja} `this idea', which occupies the SpecVoiceP position in a structure like \figref{ex:lenardic: tree}. This kind of structural configuration, however, crashes the derivation, since the morphosyntactically impoverished pro\textsubscript{\textsc{imp}} is incompatible with dependent \isi{accusative case}, which in contrast with nominative imposes structural requirements that the null \isi{pronoun}, lacking $\phi$-features, simply cannot satisfy (\citealt{Fenger2017}). In other words, it is the lack of $\phi$-features that restricts pro\textsubscript{\textsc{imp}} to a position where it can only get unmarked \isi{nominative case}.\footnote{An anonymous reviewer suggests that the assignment of \isi{nominative case} to pro\textsubscript{\textsc{imp}} may be compatible with an approach which does not assume \isi{dependent case}, but simply disassociates \isi{case assignment}, which is still feature-driven, from the checking of $\phi$-features (e.g. \citealt{boskovic2007local}). On such an approach, pro\textsubscript{\textsc{imp}} would get \isi{nominative case} by checking its [uCASE] feature against T  after moving to the SpecTP position, but in contrast with the traditional minimalist approach (e.g. \citealt{chomsky2001}), such case-checking would not hinge on the parallel checking of $\phi$-features. However, I am not sure how this approach explains the fact that pro\textsubscript{\textsc{imp}} cannot occur in a position where it gets \isi{accusative case}, given that the valuation of its \isi{case feature} as either nominative or \isi{accusative} would simply depend on the type of case-checking head that is available or closest to it (\textit{v} or \isi{Voice} for \isi{accusative} and T for nominative). By contrast, pro\textsubscript{\textsc{imp}}'s limited distribution follows on the \isi{dependent case} approach from the fact that nominative is the unmarked case, and is thus not really assigned to a DP (\citealt{kornfilt2015}) in contrast with \isi{accusative}.}

Finally, an account of \isi{impersonal} \textit{se}-sentences has to explain why the \isi{external argument} is restricted to pro\textsubscript{\textsc{imp}} \REF{ex:lenardic: 26a}, and why the derivation crashes if a full-fledged referring DP is merged in its stead \REF{ex:lenardic: 26b}.

\begin{exe}
\ex \begin{xlist}
\ex[] {\gll Svoje starše se spoštuje.\\
self parents.\textsc{acc} \textsc{se} respect.\textsc{3sg}\\
\trans `People respect their parents.'} \label{ex:lenardic: 26a}
\ex[*] {\gll  Janez se spoštuje svoje starše.\\
Janez.\textsc{nom} \textsc{se} respect.\textsc{3sg} self parents.\textsc{acc}\\
\trans Intended: `Janez respects his parents.'} \label{ex:lenardic: 26b}
\end{xlist}
\end{exe}

\noindent At the beginning of this section, I have posited that \textit{se} is the overt realisation of a special type of \isi{impersonal} \isi{Voice} head. This assumption is not without precedent, as there are in fact languages in which a \isi{Voice} head is overtly realized. According to \citet{legate2014}, one such language is \ili{Balinese}, in which the \isi{Voice} head in a low-register \isi{passive construction} is realized as the morpheme \textit{-a} (p. 44). Crucially, \citet{legate2014} claims that this morpheme does not display agreement with the \isi{grammatical subject} of the \isi{passive construction}, but rather restricts the $\phi$-features of the demoted \isi{external argument}, which is introduced via the preposition \textit{teken} `by', to 3rd person. Consequently, a 1st person \isi{external argument} PP is inadmissible in the construction \REF{ex:lenardic: 27b}.

\begin{exe}
\ex \begin{xlist}
\ex[]  {\gll Bli Man nyidaang masih tepuk-a teken Made Arini.\\
brother man can still see-\textsc{pass}.\textsc{3} by Made Arini\\
\trans `Brother Man can still be seen by Made Arini.'}\label{ex:lenardic: 27a}
\ex[*]  {\gll Bli Man nyidaang masih tepuk-a teken tiang.\\
brother man can still see-\textsc{pass}.\textsc{3} by me\\
\trans Intended: `Brother Man can still be seen by me.'}
\hfill (\ili{Balinese}; \citealt{legate2014}: 44)\label{ex:lenardic: 27b}
\end{xlist}
\end{exe}

\noindent Formally, \citeauthor{legate2014} (\citeyear{legate2014}: 39) captures these facts by assuming that the denotation of \isi{Voice}, which is a function that relates an \isi{external participant} to an \isi{eventuality} (\citealt{kratzer1996}), combines via a predicate modification rule called Restrict with the denotation of the morpheme, which is semantically a function that imposes a restriction on the kinds of $\phi$-features that an initiator DP can have. In this sense, the \isi{VoiceP} of sentence \REF{ex:lenardic: 27a} is as in \figref{BalineseFigure}, where \isi{Voice} not only introduces an \isi{open initiator variable}, but specifies that the DP realising the initiator has to have the 3rd person feature.

\begin{figure}
\begin{forest}
[\isi{VoiceP}
    [$\lbrack$Spec$\rbrack$]
    [\isi{Voice}$'$\\$\lambda x\lambda e\lbrack\cnst{initiator}({x,e}) \wedge \cnst{3person}(x) \wedge{}$\\$\textsc{see}(e)\wedge\cnst{patient}({\textsc{brother man},e})\rbrack$
        [\isi{Voice}\\$\lambda x\lambda e\lbrack{\cnst{initiator}(x,e})\wedge \cnst{3person}(x)\rbrack$
        [\isi{Voice}\\$\lambda x\lambda e\lbrack{\cnst{initiator}(x,e})\rbrack$]
        [\textsc{-pass.3}\\$\lambda x\lbrack\cnst{3person}(x)\rbrack$
]
]
[VP\\
$\lambda e\lbrack\textsc{see}(e) \wedge{}$\\
$\cnst{patient}({\textsc{brother man}, e})\rbrack$
]
]
]
\end{forest} \caption{The VoiceP of sentence \REF{ex:lenardic: 27a}} \label{BalineseFigure}
\end{figure}

For \ili{Slovenian} \isi{impersonal} \textit{se}-sentences, I would tentatively like to propose that \textit{se} operates in a similar manner to the \ili{Balinese} morpheme, in that it is a \isi{Voice} head that restricts the \isi{external argument} position to an \isi{impersonal} \isi{pronoun}, possibly banning all nominals that are specified for $\phi$-features.   The only difference is that an \isi{impersonal} \textit{se}-construction is not a \isi{passive construction}; on the present account, this means that a \isi{Voice} head realized as \textit{se} has a syntactic D-feature in the sense of Schäfer’s \isi{Voice} typology discussed in \sectref{sec4.1}, and thus needs a nominal (i.e., pro\textsubscript{\textsc{imp}})  to be merged in the specifier position. By contrast, the \ili{Balinese} construction lacks this feature, which means that its \isi{external argument} must be realized as a `demoted argument'; that is, via the \textit{teken} `by' phrase, as is standardly assumed to be the case of passives.

\subsection{The solution to the unaccusative problem}

On this syntactic proposal, the ungrammaticality of \isi{impersonal} \textit{se}-sentences with unaccusative verbs follows from the fact that – as posited at the beginning of the previous subsection – \textit{se} is the overt head of a \isi{VoiceP}, so it is an element separate from the syntactically-projected thematic arguments. With this in mind, recall from the Unaccusative Hypothesis discussed in \sectref{sec3.2} that an \isi{unaccusative predicate} like \textit{die} initially merges its sole argument within VP (in simplified terms, the \isi{syntactic derivation} of a sentence like \textit{John died} is [$_{\textsc{TP}}$ John [$_{\textsc{VP}}$ died \st{John}]], cf. \REF{sope}). It stands to reason, then, that in the \ili{Slovenian} \isi{impersonal} \textit{se}-sentence headed by the \isi{unaccusative predicate} \textit{umreti} `die' \REF{VcUm2}, our proposed pro\textsubscript{\textsc{imp}} is also merged VP-internally, as in \REF{VPUm}.
\begin{exe}
\ex
\begin{xlist}
\ex[*]{\gll Včeraj se je umrlo.\\
 yesterday \textsc{se}  \textsc{aux}.\textsc{3sg} die.\textsc{ptcp.sg.n}\\
\trans Intended: `Some people died yesterday.'} \label{VcUm2}
\ex[] {$[$\textsubscript{VP} umrlo pro\textsubscript{\textsc{imp}}$]$ \label{VPUm}}
\end{xlist}
\end{exe}


\noindent However, the VP in \REF{VPUm} then combines with \textit{se} heading a \isi{VoiceP}, and such a \textit{se} has semantic import in that it introduces an \isi{open initiator variable} into the \isi{semantic representation}. Crucially, this \isi{initiator variable} introduced by \textit{se} is distinct from the theme variable of \textit{umreti} `die', so the key prediction is that the derivation crashes because there is now an additional open variable that a one-place predicate like \textit{umreti} `die' cannot saturate. To better see the problem, compare \figref{ex:lenardic: 29}, which provides the semantic-syntactic representation of the \isi{VoiceP} of the licit transitive sentence \REF{ex:lenardic:1}, repeated here as \REF{repeated}, with \figref{ex:lenardic: 30}, which illustrates the \isi{VoiceP} of the illicit unaccusative sentence in \REF{VcUm2}.
\begin{exe}
\ex \label{repeated} \gll Celo leto se je gradilo hišo.\\
   whole year \textsc{se} \textsc{aux}.\textsc{3sg} build.\textsc{ptcp.sg.n} house.\textsc{acc}\\
\trans `People were building the house for a whole year.'
\end{exe}

\begin{figure}

\begin{forest}
[\isi{VoiceP}\\
$\lambda e\lbrack\cnst{initiator}(${\mbox{pro}\textsubscript{\textsc{imp}}$, e$})$ \land \textsc{build}(e) \land \cnst{patient}({\textsc{house}, e})\rbrack$
[pro\textsubscript{\textsc{imp}}] [\isi{Voice}$'$\\
$\lambda x\lambda e\lbrack\cnst{initiator}({x,e}) \land{}$\\$\textsc{build}(e) \land \cnst{patient}({\textsc{house},e})\rbrack$
[\isi{Voice}\\
$\lambda x\lambda e\lbrack\cnst{initiator}({x,e})\rbrack$\\
\textsc{se}]
[VP\\
$\lambda e\lbrack\textsc{build}(e) \land{}$\\ $\cnst{patient}({\textsc{house},e})\rbrack$
[V\\build] [DP\\house] ]
]
]
\end{forest}
\caption{The VoiceP of sentence \REF{repeated}}  \label{ex:lenardic: 29}
\end{figure}
\begin{figure}
\begin{forest}
[\isi{VoiceP}\\
$\lambda e\lbrack\cnst{initiator}({???, e}) \land \textsc{die}(e) \land \cnst{patient}(${\mbox{pro}\textsubscript{\textsc{imp}}$, e$})$\rbrack$
[???] [\isi{Voice}$'$\\
$\lambda x\lambda e\lbrack\cnst{initiator}({x,e}) \land{}$ \\ $\textsc{die}(e) \land \cnst{patient}(${\mbox{pro}\textsubscript{\textsc{imp}}$, e$})$\rbrack$
[\isi{Voice}\\
$\lambda x\lambda e\lbrack\cnst{initiator}({x,e})\rbrack$\\
\textsc{se}]
[VP\\
$\lambda e\lbrack\textsc{die}(e) \land \cnst{patient}(${\mbox{pro}\textsubscript{\textsc{imp}}$, e$})$\rbrack$
[V\\die] [pro\textsubscript{\textsc{imp}}] ]
]
]
\end{forest}
\caption{The VoiceP of sentence \REF{VcUm2}} \label{ex:lenardic: 30}
\end{figure}
\noindent In \figref{ex:lenardic: 29}, \textit{se} introduces an \isi{initiator variable} into the semantics of VP, which by itself denotes the characteristic function of the set of all events in which the house is built. The variable is then saturated by pro\textsubscript{\textsc{imp}}, which enters the derivation as the \isi{external argument} in SpecVoiceP.\footnote{The semantic translations of the nodes in \figref{ex:lenardic: 29} and \figref{ex:lenardic: 30} are -- for ease of exposition -- somewhat simplified with respect to pro\textsubscript{\textsc{imp}}, which remains untranslated. Following \citeposst{ackema2018} account of \isi{impersonal} pronouns with existential readings, I take it that pro\textsubscript{\textsc{imp}} semantically contributes an existential \isi{quantifier} that makes an arbitrary selection from the domain of individuals present in the discourse, which explains the ambiguous person reference discussed in \sectref{SyntacticSemanticProposal}.}  As a consequence, the propositional meaning that \figref{ex:lenardic: 29} results in after its higher structure is built is `People, who acted as volitional initiators (i.e., agents), were building the house', which is indeed how we interpret the \ili{Slovenian} transitive \isi{impersonal} \textit{se}-construction. In the illicit structure in \figref{ex:lenardic: 30}, however, pro\textsubscript{\textsc{imp}} enters the derivation as the \isi{internal argument} of \textit{umreti} `die' so as to saturate its patient variable. The resultant interpretation of the unaccusative VP headed by \textit{umreti} `die' is one that denotes the characteristic function of the set of all events in which someone (i.e., the denotation of pro\textsubscript{\textsc{imp}}) dies. The problem, then, is that the $\theta$-requirements of \textit{umreti} `die' – being a one-argument predicate – are already satisfied at the VP level, so \textit{se}, which on our account invariably introduces an external \isi{participant variable}, incorrectly turns \textit{umreti} `die' into a two-place predicate, violating the $\theta$-Criterion.
Lastly, one may wonder, as did an anonymous reviewer,  why only an \isi{external argument} is disallowed in an unaccusative sentence, while an applicative argument, such as the \isi{dative} \isi{clitic} \textit{mu} in sentence \REF{ummu}, is allowed, observing that a general  $\theta$-Criterion-based constraint on \isi{argument structure} augmentation would disallow the inclusion of both.

\begin{exe}
\ex \label{ummu} \gll Umrl mu je pes.\\
die.\textsc{ptcp.sg.m} him.\textsc{dat} \textsc{aux}.\textsc{3sg} dog.\textsc{nom}\\
\trans `His dog died on him.'
\end{exe}

\largerpage[2]
\noindent Apart from claiming that ethical datives aren't `real arguments' and thus not subjected to the $\theta$-Criterion, I acknowledge that it needs to be stipulated that unaccusative predicates are somehow able to lexically specify how their default arity can be modified (i.e., by datives, but not by external arguments). In any case, note that, in languages like \ili{English} and \ili{Slovenian}, the VP of an \isi{unaccusative predicate} also cannot combine with just about any \isi{Voice} head. For instance, if an unaccusative VP merges with an ordinary active \isi{Voice} head \REF{ex:lenardic: 17b}, the result is an ungrammatical sentence:

\begin{exe}
\ex[*]  {\gll Janez je umrl Marijo.\\
Janez.\textsc{nom} \textsc{aux}.\textsc{3sg} die.\textsc{ptcp.sg.m} Marija.\textsc{acc}\\
\trans Intended: `Janez caused Mary to die.'} \label{janezje}
\end{exe}

\noindent It seems to me that a system of \isi{argument structure} that introduces the \isi{external argument} outside of the VP always requires recourse to a stipulation preventing an \isi{unaccusative predicate} to merge its  VP with an active \isi{Voice} head, especially since the only formal requirement of active \isi{Voice} is that its semantic argument (i.e., the denotation of VP) is of type $\langle$s,t$\rangle$, given \citeposst{kratzer1996} rule of Event Identification.

\subsection{Additional evidence – the generic impersonal \textit{se}-construction}

What the proposal outlined in the previous subsections entails is that, in simplified terms, an unaccusative \isi{impersonal} \textit{se}-sentence like \REF{VcUm2} is ungrammatical because an argument is missing in the syntactic structure. That this is so is, I believe, further corroborated precisely by the fact that unaccusatives can also be licit in the \isi{impersonal} \textit{se}-construction, but only in case their interpretation is generic. Compare the ungrammatical non-generic sentences \REF{ex:lenardic: 31} with the grammatical generic variants in \REF{ex:lenardic: 32}, taken from \citet{ilcmarvin2016}.

\begin{exe}
\ex \label{ex:lenardic: 31} \begin{xlist}
\ex[*]  {\gll Včeraj se je umrlo.\\
yesterday \textsc{se} \textsc{aux}.\textsc{3sg} die.\textsc{ptcp.sg.n}\\
\trans Intended: `Someone/People died yesterday.'} \label{ex:lenardic: 31a}
\ex[*]  {\gll Včeraj se je padlo.\\
yesterday \textsc{se} \textsc{aux}.\textsc{3sg} fall.\textsc{ptcp.sg.n}\\
\trans Intended: `Someone/People fell yesterday.'} \label{ex:lenardic: 31b}
\end{xlist}
\ex \label{ex:lenardic: 32} \begin{xlist}
\ex \label{ex:lenardic: 32a}\gll Še dandanes se umre od kuge.\\
still nowadays \textsc{se} die.\textsc{3sg} from plague\\
\trans `People still die from the plague nowadays.'
\ex \label{ex:lenardic: 32b} \gll V tem hodniku se zlahka pade.\\
in this corridor \textsc{se} easily fall.\textsc{3sg}\\
\trans `Anyone can easily fall in this corridor.'
\end{xlist}
\end{exe}

\noindent According to \citeauthor{hartl2012} (\citeyear{hartl2012}: 95--97), a sentence that is interpreted generically often allows a lexically \isi{transitive predicate}, such as \textit{kill}, to be used intransitively \REF{ex:lenardic: 33a}. If the interpretation were existential/episodic, such use would generally be ungrammatical \REF{ex:lenardic: 33b}.
\largerpage[2]

\begin{exe}
\ex \label{ex:lenardic: 33}
\begin{xlist}
\ex[]  {The tiger kills to survive.}\label{ex:lenardic: 33a}
\ex[*]  {Yesterday, the tiger killed.\\
Intended: `Yesterday, the tiger killed something/someone.'} \label{ex:lenardic: 33b}
\end{xlist}
\end{exe}

\noindent A similar improvement in acceptability can be observed in the so-called \isi{middle construction}, where the predicate \textit{read} in both \REF{ex:lenardic: 34a} and \REF{ex:lenardic: 34b} is used intransitively, in the sense that the \isi{external argument} denoting the reader of the book is missing from the overt syntactic representation.

\begin{exe}
\ex \label{ex:lenardic: 34}
\begin{xlist}
\ex[] {The book reads easily.\\
`Anyone can read this book easily.'}\label{ex:lenardic: 34a}
\ex[*] {Yesterday, the book read.\\
Intended: `Someone was reading the book yesterday.'} \label{ex:lenardic: 34b}
\end{xlist}
\end{exe}

\noindent The general idea behind this improvement is tied to the fact that the generic \isi{middle construction} \REF{ex:lenardic: 34a} interpretatively involves a dispositional `property' interpretation, which is ascribed to the \isi{grammatical subject} (\citealt{lekakou2004}). In the case of sentence \REF{ex:lenardic: 34a}, such a \isi{property interpretation} can be observed from the fact that its interpretation corresponds to the paraphrase `This book has properties such that it is easy to read' (\citealt{fagan1992}). According to \citeauthor{hartl2012}, it is this \isi{property interpretation} tied to genericity that licenses the intransitive use of \textit{read} in \REF{ex:lenardic: 34a}, in the sense that “the interpretation of a generic property can assist the [inference] of [the implicit] evaluative entity'' (\citealt{hartl2012}: 114).\footnote{To be clear, I do not wish to assume -- and, if I understand him correctly, neither does \citeauthor{hartl2012} -- that  these generic sentences are licit in contrast with their eventive variants because  some kind of post-syntactic operation is at work,  accessing the syntactic component after derivation and circumventing the $\theta$-Criterion by suspending the predicate's need for a syntactically-projected \isi{external argument}. \citet{hartl2012} himself remains quite neutral with respect to how and at which stage of derivation the argument can be dropped.  At the end of this section, I propose a tentative solution, following \citet{bruening2012}, in the sense that such sentences contain a null operator that blocks the merger of an overt DP by closing off the \isi{open initiator variable} introduced by \isi{Voice} at the stage when the \isi{syntactic derivation} is still taking place.} This means that \REF{ex:lenardic: 34a} is licit because the generic interpretation allows the hearer to easily infer the overtly missing initiator argument, who is an evaluative entity insofar as he/she can read the book easily due to some properties that it possesses.\footnote{Note that the generic context is not the only environment that can license the intransitive use of a lexically \isi{transitive predicate}. An anonymous reviewer cites the following two examples with the predicate \textit{kill}, which omit an overt \isi{internal argument} even though the interpretation is existential: \begin{exe} \ex \label{fn 1} Jack the Ripper killed (someone) again last night. \ex \label{fn 2} Donald Glover killed (the audience) at SNL last night.\end{exe}
It seems to me that sentence \REF{fn 1}, albeit non-generic, is licit because it also encourages a property-reading of the \isi{grammatical subject}, in the sense that its interpretation is like that of a characterizing sentence; i.e., \textit{Jack the Ripper killed again last  night, as he is wont to do}. Note, also, that the sentence becomes degraded if the adverb is omitted -- ?\textit{Jack the Ripper killed last night}. In sentence \REF{fn 2}, \textit{kill} is used in the unrelated sense `to make a favourable impression on (someone)'. Such a sense of \textit{kill} seems to be very context-specific (i.e., some kind of performance in front of an audience), so a property-meaning can be more easily inferred from the \isi{grammatical subject}, i.e. \textit{Donald Glover}, which might explain its generic-like behaviour in that it allows the omission of an overt \isi{internal argument}.}

The generically interpreted \isi{impersonal} \textit{se}-sentences in \REF{ex:lenardic: 32}, which are licit in contrast with their eventive variants \REF{ex:lenardic: 31}, also involve a \isi{property interpretation}. However, in contrast with the \isi{middle construction} in \REF{ex:lenardic: 34a}, the \isi{property interpretation} is not associated with the logical object but rather with an external source, which causes the logical object (i.e., people denoted by pro\textsubscript{\textsc{imp}}) to either fall or die. In this sense, \REF{ex:lenardic: 32a} is intuitively interpreted with respect to the properties of the plague, which is such that it still causes people to die nowadays. Along the same lines, \REF{ex:lenardic: 32b} has the interpretation that the corridor has certain properties such that it causes people to fall. In other words, the interpretation of generic \isi{impersonal} \textit{se}-sentences roughly corresponds to `Something causes people to fall/die', and what is crucial is that the syntax we have proposed in the previous section does in fact predict this kind of quasi-two-argument interpretation. This is so because we have posited \textit{se} to be the overt head of a \isi{Voice} head which invariably introduces its own argument corresponding to an \isi{external participant}/initiator.

The question, then, is why the generic sentences in \REF{ex:lenardic: 32} are licit if -- on the present account -- \textit{se}’s formal requirement for an additional argument renders their eventive variants in \REF{ex:lenardic: 31} ungrammatical.  To try and answer this, I would tentatively like to adopt a proposal by \citet{bruening2012}, which pertains to the \isi{argument structure} of the \ili{English} \isi{middle construction}.  Bruening claims that an \ili{English} middle sentence such as \REF{ex:lenardic: 34a}, lacking a syntactic \isi{external argument}, is derived via an operator that is defined as follows (\citealt{bruening2012}: 30):

\begin{exe}
\ex \label{ex:lenardic: 35} $\llbracket$MiddleOp$\rrbracket = \lambda f_{\langle e,\langle s,t\rangle\rangle}\lambda g_{\langle s,t\rangle}\textsf{GN}e\exists x[f(x, e) \Rightarrow g(e)]$
\end{exe}


\noindent This operator, which is merged during the \isi{syntactic derivation} of \isi{VoiceP} (\citealt{bruening2012}), targets the denotation of an unsaturated \isi{Voice} projection and maps it into a semantic scope where the \isi{initiator variable}  gets existentially bound. This, in turn, also means that a referring DP cannot be merged as an \isi{external argument} in the \ili{English} construction, as its merger would saturate the \isi{open initiator variable} and thus violate the semantic-type requirement of the first argument of the middle operator in \REF{ex:lenardic: 35}, which has to be a function of type $\langle e,\langle s,t\rangle\rangle$.

I suggest that the generic \textit{se}-sentences headed by unaccusative predicates undergo a similar operation tied to their valency. That is, an operator such as \REF{ex:lenardic: 35} targets an unsaturated \isi{Voice} projection headed by \textit{se} and thereby closes off the \isi{open initiator variable}. As a result, a sentence like \REF{ex:lenardic: 32b} is interpreted as follows.

\begin{exe}
\ex \label{ex:lenardic: 36} $\textsf{GN}e\exists x[\cnst{initiator}(x, e) \land \textsc{fall}(e) \land \cnst{patient}(\mbox{pro}$\textsubscript{\textsc{imp}}$, e) \Rightarrow \textsc{easy}(e)]$
\end{exe}

\noindent The LF in \REF{ex:lenardic: 36} corresponds to the above-mentioned paraphrase  – `generally, there is an $x$ such that it causes people (denoted by pro\textsubscript{\textsc{imp}}) to fall easily'. What is key here is that the merger of the  operator in \REF{ex:lenardic: 35} precludes the merger of an argument in the SpecVoiceP position, just as an \isi{external argument} DP is blocked in the \ili{English} \isi{middle construction} by the operator. This semantic operation, however, is not available with the eventive versions. This means that the only way the unaccusatives in \REF{ex:lenardic: 31} can satisfy \textit{se}’s requirement for an \isi{external participant} is by introducing a referential argument in the SpecVoiceP position, which is -- as discussed in the previous section -- inadmissible as it flies in the face of the lexical selectional properties of unaccusatives.\footnote{There is a conceptual issue here: even if there is a generic operator that saturates the open variable in the generic \isi{impersonal} \textit{se}-construction, then the construction must still somehow circumvent the syntactic requirement that an overt DP be merged in SpecVoiceP, given that it is an active construction in terms of \isi{Voice} and thus has a D-feature that needs to be checked by a nominal argument landing in the SpecVoiceP position (cf. \sectref{sec4.1}). For \ili{English} middles, \citet{schaffer2007} claims that the \isi{internal argument} satisfies this feature by stopping off in the SpecVoiceP position on its way to its final position in SpecTP. It may be the case that the pro\textsubscript{\textsc{imp}} in the \isi{internal argument} position of generic \isi{impersonal} \textit{se}-sentences moves in a similar manner; however, this aspect  of the derivation merits further research.}

\section{Conclusion} \label{sec5}
On the one hand, the paper has argued that all of the thematic arguments selected by a predicate enter the \isi{syntactic derivation} of the \isi{impersonal} \textit{se}-construction. Evidence was given in the form of the \isi{causative alternation}, since the fact that the \isi{impersonal} \textit{se}-construction with a causative predicate like \textit{odpreti} always entails \isi{agentive participation} does not trivially follow from an analysis in which the \isi{external argument} does not enter the derivation. On the other hand, the paper has shown that the syntactic set-up of \citeposst{riverosheppard2003} account does not provide an explanation as to why the construction is incompatible with unaccusative verbs. In its stead, the paper has proposed an analysis of the construction in which the \isi{clitic} heads a \isi{VoiceP} and introduces an agent variable to the \isi{semantic representation} of the VP. This variable is saturated by transitive and \isi{unergative} predicates because they merge a null \isi{pronoun} in the SpecVoiceP position. Unaccusative predicates, however, merge the \isi{pronoun} within the VP but do not lexically select for another argument, so the \isi{participant variable} introduced by \textit{se} remains unsaturated. Finally, I have argued that the licit generic interpretation of the \isi{impersonal} \textit{se}-construction with unaccusative verbs diagnoses the fact that an unaccusative \textit{se}-sentence is ruled out because of a missing argument.


\section*{Abbreviations}
\begin{tabularx}{.45\textwidth}{lX}
\textsc{1}&1st person\\
\textsc{3}&3rd person\\
\textsc{acc}&{accusative}\\
\textsc{aux}&auxiliary\\
{Cl}&{clitic}\\
\textsc{expl}&{expletive}\\
\textsc{f}&feminine\\
\textsc{hum}&human\\
\textsc{imp}&{impersonal}\\
\end{tabularx}
\begin{tabularx}{.45\textwidth}{lX}
\textsc{m}&masculine\\
\textsc{n}&neuter\\
\textsc{nom}&nominative\\
\textsc{pass}&passive\\
\textsc{pl}&plural\\
\textsc{ptcp}&{participle}\\
\textsc{sg}&singular\\
{KP}&{kase phrase}\\
LF& logical form\\
\end{tabularx}

\section*{Acknowledgements}
I would like to thank Adrian Stegovec, Željko Bošković, Paula Fenger, Gašper Ilc, and Tatjana Marvin for valuable discussions and help with the data. I would also like to thank the two anonymous reviewers for their suggestions and comments, which have improved the content and the presentation of the paper immensely.

\sloppy
\printbibliography[heading=subbibliography,notkeyword=this]
\il{Slovenian|(}
\end{document}
