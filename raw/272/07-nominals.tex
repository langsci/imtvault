\chapter{Nominals}\label{chap:7}

The nominals are divided into the subsets, i.e. common nouns, address nouns, reflexive pronouns, numerals, and indefinite pronouns. They are all free forms and are distinguished primarily by semantic criteria. Additionally, there is the deverbal nominal, i.e. the nominal derived from the verbal stems. These nominals will be discussed in \sectref{sec:key:7.1} to \sectref{sec:key:7.6.} The formal nouns are also nominals, but they are clitics, which was already discussed in \sectref{sec:key:6.2.2.} We discussed that personal pronominals, demonstratives and interrogatives may be categorized not only as nominals, but also as other word classes, so they are called “cross-over categories” (see Chapter 5 for more details). These various kinds of nominals in Yuwan have strong relationships with the animacy hierarchy, and the details were discussed in \sectref{sec:key:6.4.}

The affixes that attach only to the nominal stems are called the nominal affixes. Yuwan has only two nomnal affixes: \textit{{}-taa} (\textsc{pl}) and \textit{{}-kkwa} (\textsc{dim}). The plural affix \textit{{}-taa} was discussed in \sectref{sec:key:6.4.1} compared with other morphemes that can express plural meaning. The diminutive affix \textit{{}-kkwa} will be discussed in the last section in this chapter (see \sectref{sec:key:7.7}). It should be noted that \textit{{}-kja} (PL) in \sectref{sec:key:5.1} is not categorized in the nominal affix, since it attaches to the personal pronominal stems (not nominal stems). In fact, \textit{{}-kja} (PL) is a kind of nominalizer that can also express number, and the same point can be made about the other number affixes, i.e. \textit{{}-n} (\textsc{sg}) and \textit{{}-ttəə} (\textsc{du}).

\section{Common nouns}

In \sectref{sec:key:7.1.1}, I will discuss the morphosyntax of common nouns.

In \sectref{sec:key:7.1.2}, I will discuss the semantic remarks on number of common nouns.

\subsection{Morphosyntax of common nouns}

A common noun can function as an NP of any kind (an argument, a predicate or an NP modifer). Nominals other than address nouns, reflexive pronouns, numerals, and indefinite pronouns are regarded as “common nouns.”

\ea \label{ex:7:1}  Common nouns (animate)
\ea \label{ex:7:a}Argument

{\TM}
\gllll  muccjɨ  ikjoojəəcjɨ  maganu  jˀicjun   joosɨ.\\
\textit{mut-tɨ}  \textit{ik-oo=jəə=ccjɨ}  \textit{\Highlight{maga=nu}}  \textit{jˀ-tur-n}   \textit{joosɨ}\\
have-\textsc{seq}  go-\textsc{int}=\textsc{cfm}2=\textsc{qt}  [grandchild=\textsc{nom}]  say-\textsc{prog}-\textsc{ptcp} atomosphere\\
{}[Subject]  \\
\glt ‘The grandchild seems to say that, “(I) will take (the pears).”’ [\textsc{pf}: 090827\_02.txt]

\ex \label{ex:7:b}Predicate

{\TM}
\gllll  kun  cˀjoo,  ido..,  taa ..  maga   jataru?\\
\textit{ku-n}  \textit{cˀju=ja}  \textit{ido}  \textit{ta-a}  \textit{\Highlight{maga}} \textit{jar-tar-u}\\
this-\textsc{adnz}  person=\textsc{top}  oh  [who-\textsc{adnz}  grandchild \textsc{cop}-\textsc{pst}-\textsc{pfc}]\\
{}[Nominal predicate]\\
\glt ‘Whose grandchild was this person?’ [Co: 120415\_00.txt]

\ex \label{ex:7:c}NP modifier

    [Context: Complaining about the decline of her memory]

{\TM}
\gllll  maganu  cˀjuigadəə  sicjussɨga,                                tˀaimekaroo  sijandoojaa.\\
\textit{\Highlight{maga=nu}}  \textit{cˀjui=gadɨ=ja}  \textit{sij-tur-sɨga}         \textit{tˀai-me=kara=ja}  \textit{sij-an=doo=jaa}\\
\{[grandchild=\textsc{gen}]  [one.\textsc{clf}]\}=\textsc{lmt}=\textsc{top}  know-\textsc{prog}-\textsc{pol}                  two.\textsc{clf}-time=\textsc{abl}=\textsc{top}  know-\textsc{neg}=\textsc{ass}=\textsc{sol}\\
\{[Modifier]  [Head]\}\textsubscript{NP}  \\
\glt ‘(I) know (the name of) one grandchild, but don’t know (that of) the second one (and more).’ [Co: 110328\_00.txt]

\z
\z

In (\ref{ex:7-1}a), the animate common noun \textit{maga} ‘grandchild’ fill the argument slot, which is the subject of the clause. In (\ref{ex:7-1}b), \textit{maga} ‘grandchild’ fill the predicate slot of the clause, and it becomes nominal predicate with the copula verb \textit{jar-} (\textsc{cop}). In (\ref{ex:7-1}c), \textit{maga} ‘grandchild’ fills the modifier slot of an NP, whose head is a numeral \textit{cˀjui} ‘one person.’ The plurality of common nouns can be expressed by \textit{nkja} (\textsc{appr}).

\ea \label{ex:7:2}  Common noun (animate) in the plural

  [Context: Remembering that \textsc{ms}’s grandmother used to make kimono for grandchildren]

{\TM}
\glll uraa  baasanna  jazin  magankjanu urakjaa  taməə,\\
\textit{ura-a}  \textit{baasan=ja}  \textit{jazin}  \textit{\Highlight{maga=nkja}=nu}   \textit{urakja-a}  \textit{taməə}\\
    2.\textsc{nhon}.\textsc{sg}-\textsc{adnz}  grandmother=\textsc{top}  necessarily  grandchild=\textsc{appr}=\textsc{gen}  2.\textsc{nhon}.\textsc{pl}-\textsc{adnz}  sake\\
\glt    ‘Your grandmother necessarily for grandchildren, for you all, ...’ [Co: 120415\_01.txt]

\z

In \REF{ex:7:2}, \textit{maga=nkja} (grandchild=\textsc{appr}) ‘grandchildren’ has a plural meaning.

  The above examples are all animate, but the same thing can be said to inanimate common nouns.

\ea \label{ex:7:3}  Common nouns (inanimate)
\ea \label{ex:7:3a}Argument
    [Context: Old people chanted an incantation when they felt the earthquakes.]

{\TM}
\gllll  jaanu  jurɨppoo,  kjon  cɨkɨ  kjon   cɨkɨccjɨ  jutassɨgana.\\
\textit{\Highlight{jaa}=nu}  \textit{jurɨr-boo}  \textit{kjoo=n}  \textit{cɨk-ɨ}  \textit{kjoo=n}   \textit{cɨk-ɨ}  \textit{jˀ-jur-tar-sɨga=na}\\
{}[house=\textsc{nom}]  shake-\textsc{cnd}  Kyoto=\textsc{dat}1  attach-\textsc{imp}  Kyoto=DAT1  attach-IMP  say-\textsc{umrk}-\textsc{pst}-\textsc{pol}=\textsc{plq}\\
{}[Subject]        \\
\glt ‘If the house shakes, (old people said) that, “Send (it) to Kyoto! Send (it) to Kyoto!” [lit. “Attach to Kyoto! Attach to Kyoto!”]’ [Co: 110328\_00.txt]

\ex \label{ex:7:3b}Predicate
{\TM}
\gllll  arəə  attaa  məəra  muratən  jaa  jappa.\\
\textit{a-rɨ=ja}  \textit{a-rɨ-taa}  \textit{məə=kara}  \textit{muraw-təər-n}  \textit{\Highlight{jaa}}  \textit{jar-ba}\\
\textsc{dist}-\textsc{nlz}=\textsc{top}  [DI\textsc{st}-NLZ-\textsc{pl}  front=\textsc{abl}  receive-\textsc{rsl}-\textsc{ptcp}  house                               \textsc{cop}-\textsc{csl}]\\
{}[Nominal predicate]\\
\glt ‘Since that is the house (he) has received from them.’ [Co: 111113\_01.txt]

\ex \label{ex:7:3c}NP modifier
    [Context: Seeing a picture, where bundles of rice were hung out in the sun]

{\TM}
\gllll jaanu  məəninkjadu  gan  sjɨ   sagɨjutanwake  zjajaa.\\
\textit{\Highlight{jaa=nu}}  \textit{məə=nan=nkja=du}  \textit{ga-n}  \textit{sɨr-tɨ} \textit{sagɨr-jur-tar-n=wake}  \textit{zjar=jaa}\\
      \{[house=\textsc{gen}]  [front]\}=\textsc{loc}1=\textsc{appr}=\textsc{foc}  \textsc{mes}-\textsc{advz}  do-\textsc{seq}                  hang-\textsc{umrk}-\textsc{pst}-\textsc{ptcp}=\textsc{fn}  \textsc{cop}=\textsc{sol}\\
      \{[Modifier]  [Head]\}\textsubscript{NP}    \\
\glt    ‘(They) would hang (bundles of rice) in front of (their) houses like this.’       [Co: 111113\_02.txt]

\ex \label{ex:7:3d}In the plural
{\TM}
\glll  kan  sjɨ  jankjanu  dɨkɨɨjukkjaija {\textbar}nan+nengoro{\textbar}karakai?\\
\textit{ka-n}  \textit{sɨr-tɨ}  \textit{\Highlight{jaa=nkja}=nu}  \textit{dɨkɨr-∅+jukkjaar-i=ja}  \textit{nan+nen-goro=kara=kai}\\
\textsc{prox}-\textsc{advz}  do-\textsc{seq}  house=\textsc{appr}=\textsc{nom}  be.made-\textsc{inf}+\textsc{ingr}-INF=\textsc{top}  what+year-about=\textsc{abl}=\textsc{dub}\\
\glt ‘When did the houses begin to be made like this?’ [Co: 110328\_00.txt]

\z
\z

In (\ref{ex:7-3}a), the inanimate common noun \textit{jaa} ‘house’ fill the argument slot, which is the subject of the clause. In (\ref{ex:7-3}b), \textit{jaa} ‘house’ fill the predicate slot of the clause, and it becomes nominal predicate with the copula verb \textit{jar-} (\textsc{cop}). In (\ref{ex:7-3}c), \textit{jaa} ‘house’ fills the modifier slot of an NP, whose head is also a common noun \textit{məə} ‘front.’ In (\ref{ex:7-3}d), \textit{jaa=nkja} (house=\textsc{appr}) ‘houses’ has a plural meaning.

\subsection{Semantic remarks on number of common nouns}

We have seen that the plurality of common nouns is expressed by \textit{nkja} (\textsc{appr}) in the previous section. There is, however, a case, where the bare form of common nouns can imply plurality in itself. In the following discussion, the “bare form” indicates the form which is not followed by the plural markers in Yuwan.

\ea \label{ex:7:4}  Common noun (indefinite and unspecific)

  [Context: Speaking of a woman]

{\TM}
\glll  kˀwoo  ippaidoojaa.\\
\textit{kˀwa=ja}  \textit{ippai=doo=jaa}\\
child=\textsc{top}  many=\textsc{ass}=\textsc{sol}\\
\glt ‘(She has) many children, you know.’ [Co: 120415\_01.txt]

\z

In \REF{ex:7:4}, \textit{kˀwa} ‘child’ indicates plural referents in effect, since the predicate (i.e. \textit{ippai} ‘many’) means plurality, but it does not need \textit{nkja} (\textsc{appr}). However, such an implication of plurality is only allowed for indefinite (and unspecific) referents as in \REF{ex:7:4}. If the referent is definite, specific, and also human, the bare form must indicate only one referent. See \REF{ex:7:5}.

\ea \label{ex:7:5}  Common noun (definite, specific, and human)

  [Context: Three boys noticed that another boy fell his hat, so they called the boy.]

{\TM}
\glll saki  izjan  micjaija ..  xxx  mata  isjoobiki  hucjɨ,  un  kˀwaba  abɨtɨ,\\
\textit{saki}  \textit{ik-tar-n}  \textit{micjai=ja}    \textit{mata}  \textit{isjoobiki}  \textit{huk-tɨ}  \textit{\Highlight{u-n}}  \textit{\Highlight{kˀwa}=ba}  \textit{abɨr-tɨ}\\
    first  go-\textsc{pst}-\textsc{ptcp}  three.\textsc{clf}=\textsc{top}    again  whistle  blow-\textsc{seq}  \textsc{mes}-\textsc{adnz}  child=\textsc{acc}  call-\textsc{seq}\\
\glt    ‘The three (boys) who went first again whistled, and called the boy, and ...’ [\textsc{pf}: 090222\_00.txt]

\z

In the above context, the referent called by three boys is only one. In other words, the expression \textit{u-n} \textit{kˀwa} ‘the boy [lit. that child],’ which is definite, specific, and human, must have only a singular meaning. As mentioned in \sectref{sec:key:6.4.1.1}, the plural markers in Yuwan, including \textit{nkja} (\textsc{appr}), can indicate a single specific referent alone. Such an ambiguous characteristic of plural markers make it a little complicated to code or decode the meaning of number in Yuwan. The above contrast between \REF{ex:7:4} and \REF{ex:7:5} is summarized in the follwing tables (see \tabref{tab:key:45} and \tabref{tab:key:46}).

\begin{table}
\caption{\label{tab:key:45}. Common nouns (indefinite and unspecific)}

Form  <<< Encoding <<<  Meaning on number

Bare form  %%[Warning: Draw object ignored]

%%[Warning: Draw object ignored]

a. One referent

Bare from + \textit{nkja} (\textsc{appr})  %%[Warning: Draw object ignored]

%%[Warning: Draw object ignored]

b. One referent as an example of the member of an unspecific group

c. More than one referent

  >>> Decoding >>>
\end{table}

The meaning “b” in the right-most column in \tabref{tab:key:45} is characteristic of the plural markers in Yuwan (see \sectref{sec:key:6.4.1.1} for more details). \tabref{tab:key:45} shows that the common nouns that are indefinite and unspecific are ambiguous about their number in both encoding and decoding. The coding relation in \REF{ex:7:4} corresponds to that of “bare form” and “more than one referent.” In another context, the bare form, which indicates an indefinite and unspecific referent, can also be decoded into simply “one referent.” However, if the common nouns indicate definite, specific, and human referents, the bare form cannot be used to indicate more than one referent, which is presented below.

\begin{table}
\caption{\label{tab:key:46} Common nouns (definite, specific, and human)}

Form  <<< Encoding <<<  Meaning on number

\ea
Bare form  %%[Warning: Draw object ignored]

\ea \label{ex:7:a}One referent

Bare from + \textit{nkja} (\textsc{appr})  %%[Warning: Draw object ignored]

%%[Warning: Draw object ignored]

\ex \label{ex:7:b}One referent as an example of the member of an unspecific group

  \ex \label{ex:7:c}More than one referent

  >>> Decoding >>>
  \z
\z
\todo[inline]{no example numbers in tables}
\end{table}
In \tabref{tab:key:46}, a line that existed in \tabref{tab:key:45}, i.e. the connection between “bare form” and “more than one referent,” was omitted. Thus, the coding relation between “bare form” and “one referent” is straightforward. Therefore we can know that the bare form in \REF{ex:7:5} indicates only one referent.

\section{Address nouns}

Address nouns can be used to call the opponent, which include a part of elder kinship terms and personal names. Additionally, certain profession, e.g. \textit{soncjoo-san} (village.mayor-\textsc{hon}) ‘village mayor’ or \textit{sinsjei} ‘teacher’ can be used as address nouns.

The elder kinship terms that can be used to address the opponent are as follows: \textit{zjuu} ‘father,’ \textit{cˀjan} ‘father,’ \textit{anmaa} ‘mother,’ \textit{okkan} ‘mother,’ \textit{kaacjan} ‘mother,’ \textit{uzii} ‘grandfather,’ \textit{hannjəə} ‘grandmothr,’ \textit{ubaa} ‘grandmother,’ \textit{nii} ‘older brother,’ \textit{nəə} ‘older sister,’ which all appeared in my texts. In those kinship terms, \textit{zjuu} ‘father,’ \textit{anmaa} ‘mother,’ \textit{hannjəə} ‘grandmother,’ and \textit{anjoo} ‘old brother’ are relatively old expression, and the others are relatively new (borrowed) ones. These elder kinship terms, especially the relatively new ones, can be used even if the speaker does not have an actual relative relation with the opponent, e.g., \textit{uzii} ‘grandfathr’ in \REF{ex:6-136} in \sectref{sec:key:6.4.3.6}, where \textit{uzii} is glossed and translated into ‘old man’ to fit in the context. The personal names that can be used to address people are all the first names, not the family names.

It should be mentioned that several kinship terms cannot be used to address the opponents, e.g., \textit{uja} ‘parents,’ \textit{jinga-nəə} (man-parent) ‘father [lit. male parent],’ \textit{wunagu-nəə} (woman-parent) ‘mother [lit. female parent],’ \textit{kjoodəə} ‘brother,’ \textit{wunai} ‘younger sister,’ \textit{jɨɨi} ‘younger brother,’ and \textit{maga} ‘grandchild.’ These kinship terms that cannot be used to address the opponent are included in the common nouns in Yuwan (see \sectref{sec:key:7.1}).

The address nouns can function as an NP of any kind (an argument, a predicate or an NP modifer). In Yuwan, personal names are frequently compounded with elder kinship terms, e.g. \textit{zjennjuki+anjoo} (Zenyuki+older.brother) ‘Zenyuki,’ where the elder kinship terms function like the honorific titles ‘Mr.’ or ‘Ms.’ in English, although they are used in a more friendly way. The honorific meaning is not translated in English in this grammar.

\ea \label{ex:7:6}  Address nouns (elder kinship)
\ea \label{ex:7:6a}Argument

{\TM}
\gllll  zjennjukianjooga  {\textbar}heitai{\textbar}kaci  izjɨ,\\
\textit{\Highlight{zjennjuki+anjoo=ga}}  \textit{heitai=kaci}  \textit{ik-tɨ}\\
{}[Zenyuki+older.brother=\textsc{nom}]  soldier=\textsc{all}  go-\textsc{seq}\\
{}[Subject]    \\
\glt ‘Zenyuki went to (be) a soldier, and ...’ [Co: 120415\_00.txt]

\ex \label{ex:7:6b}Predicate

{\TM}
\gllll  kurɨ  sigemasaanjoo  jappa.\\
\textit{ku-rɨ}  \textit{\Highlight{sigemasa+anjoo}}  \textit{jar-ba}\\
\textsc{prox}-\textsc{nlz}  [Shigemasa+older.brother  \textsc{cop}-\textsc{csl}]\\
{}[Nominal predicate]  \\
\glt ‘This (person on the picture) is Shigemasa.’ [Co: 120415\_00.txt]

\ex \label{ex:7:6c}NP modifier

  {\TM}
\glll kun  cˀjoo  kisasianjoo  zjuuja  arannən,\\
\textit{ku-n}  \textit{cˀju=ja}  \textit{\Highlight{kisasi+anjoo}}  \textit{zjuu=ja} \textit{ar-annən}\\
      \textsc{prox}-\textsc{adnz}  person=\textsc{top}  \{[Kisashi+older.brother]  [father]\}=\textsc{top}         \textsc{cop}-\textsc{neg}.\textsc{seq}\\
          \{[Modifier]  [Head]\}\textsubscript{NP}\\
\glt    ‘This person is not Kisashi’s father, and ...’       [Co: 120415\_00.txt]

\ex \label{ex:7:6d}In the plural

{\TM}
\glll an  junɨzooanjootaaga  simautaba hozonsiicjɨ  jˀicjɨ,\\
\textit{a-n}  \textit{\Highlight{junɨzoo+anjoo-taa}=ga}  \textit{sima+uta=ba}  \textit{hozon+sɨr-i=ccjɨ}  \textit{jˀ-tɨ}\\
      \textsc{dist}-\textsc{adnz}  Yonezo+older.brother-\textsc{pl}=\textsc{nom}  community+song=\textsc{acc}         preservation+do-\textsc{inf}=\textsc{qt}  say-\textsc{seq}\\
\glt    ‘Those (people,) Yonezo and his family said that (they would) do the preservation of the (traditional) songs (of) the community.’      [Co: 111113\_01.txt]

  \z
\z

In (\ref{ex:7-6}a), the (compounded) personal name \textit{zjennjuki+anjoo} ‘Zenyuki’ fill the argument slot, which is the subject of the clause. In (\ref{ex:7-6}b), \textit{sigemasa+anjoo} ‘Shigemasa’ fill the predicate slot of the clause, and it becomes nominal predicate with the copula verb \textit{jar-} (\textsc{cop}). In (\ref{ex:7-6}c), \textit{kisasi+anjoo} ‘Kisashi’ directly fills the modifier slot of an NP, whose head is also an address noun \textit{zjuu} ‘father’. In (\ref{ex:7-6}d), \textit{junɨzoo+anjoo-taa} (Yonezo+older.brother-\textsc{pl}) ‘Yonezo and his family’ has a plural meaning.

  As mentioned in \sectref{sec:key:6.4.1.1}, the plural forms in Yuwan may indicate not only plural specific referents, but also a single specific referent. Therefore, the plural forms are ambiguous about the semantic plurality in a narrow sense. The bare forms (i.e. the forms without the plural affix \textit{{}-taa}) of address nouns, however, are different, since the bare forms of address nouns must indicate only one specific referent (with no other referents). Therefore, it may be appropriate to admit that the bare forms of address nouns have a zero affix that only incicates the singular meaning, e.g., \textit{zjennjuki+anjoo-\Highlight{∅}} (Zenyuki+older.brother-\textsc{\Highlight{sg}}). Here, it should be remembered that a similar problem has happened in common nouns, where certain common nouns must have correspondence between bare forms and (genuine) singular meanings (see \sectref{sec:key:7.1.2}). Those common nouns must indicate definte, specific, and human referents, which are the usual characteristics of address nouns (with the exceptoin of elder kinship terms used to indicate non-relatives). Considering these facts, it is more appropriate to think that the obligatory “singularness” of the address nouns is not attributed to the alleged affix \textit{{}-∅} (\textsc{sg}), but on the meaning of the NP (with which the plural affixes co-occur). Thus, I propose that the address nouns in bare forms do not have any singular affix such as \textit{{}-∅} (SG).

\section{Reflexive pronouns}

Yuwan has two reflexive pronouns, \textit{nusi} and \textit{duu}, and the choice of them seems to depend on the difference among idiolects. For example, \textsc{tm} only uses \textit{nusi}, \textsc{my} basically uses \textit{nusi} but sometimes uses \textit{duu}, which is always compounded like \textit{duu+duu}, and \textsc{ms} uses only \textit{duu}; the other people have not used reflexive pronouns in my texts. In many cases, the antecedent of the reflexive pronoun is the subject of the clause. In the following examples, the reflexive and its antecedent is marked by the small italic “\textit{i}” in the underlying level. In addition, the reflexive pronouns in the underlying level and their correspondents in the free translation are underlined.

\ea \label{ex:7:7}  
\ea\label{ex:7:7a}  [Context: Talking about a riverboat of the \textsc{ms}’s family] = (\ref{ex:6-99}c)

  {\TM}
\glll urakjoo,  nusinkjanu  atattudu,   siccjuro.\\
\textit{urakja\textsubscript{i}}\textit{=ja}  [\textit{\Highlight{nusi=nkja}\textsubscript{i}}\textit{=nu}  \textit{ar-tar-tu}]\textsubscript{Adverbial clause}\textit{=du}  \textit{sij-tur-oo}\\
      2.\textsc{nhon}.\textsc{pl}=\textsc{top}  \textsc{rfl}=\textsc{appr}=\textsc{nom}  exist-\textsc{pst}-\textsc{csl}=\textsc{foc}   know-\textsc{prog}-\textsc{supp}\\
\glt    ‘You probably know (it), because you have a riverboat of \Highlight{your own}.’      [Co: 111113\_01.txt]

\ex\label{ex:7:7b}  [Context: Speking about an acquaintance] = \REF{ex:6-68}

{\TM}
\glll  wanga  kucisjɨ  nusiboo  jamacjuncjɨ,\\
{}[\textit{wan=ga}  \textit{kuci=sjɨ}  \textit{\Highlight{nusi}=ba=ja}  \textit{jam-as-tur-n=ccjɨ}]\textsubscript{Complement clause}\\
1\textsc{sg}=\textsc{nom}  mouth=\textsc{inst}  \textsc{rfl}=\textsc{acc}=\textsc{top}  have.a.pain-\textsc{caus}-\textsc{prog}-\textsc{ptcp}=\textsc{qt}\\
\glt ‘(The person said) that I was making \Highlight{the person}\textit{\textsubscript{} }ill using (my) mouth, and ...’ [Co: 120415\_01.txt]

\z
\z

In (\ref{ex:7-7}a), the antecedent of \textit{nusi} (\textsc{rfl}) is \textit{urakja} ‘you,’ and it overtly appears in the sentence. On the contrary, in (\ref{ex:7-7}b), the antecedent of \textit{nusi} (RFL), i.e. ‘the person,’ does not overtly appear in the sentence, but it can be traced by the context. In both of the above examples, \textit{nusi} (RFL) is in the subordinate clauses, but it can correspond with the antecedents in the main clauses.

  Additionally, there are examples where \textit{nusi} (\textsc{rfl}) does not seem to correspond with any specific antecedent, but seems to correspond with unspecific referents.

\ea \label{ex:7:8}  [Context: The husband of a couple did not bring back a pot filled with gold coins, since happiness comes naturally to honest people.]

{\TM}
\glll nusarɨja  nusinu  jaakaci,  nusarɨja  sizinnidu   həncjɨ  kjuncjɨ.\\
\textit{nusarɨ=ja}  \textit{\Highlight{nusi}=nu}  \textit{jaa=kaci}  \textit{nusarɨ=ja}  \textit{sizin=n=du} \textit{hənk-tɨ}  \textit{k-jur-n=ccjɨ}\\
    happiness=\textsc{top}  \textsc{rfl}=\textsc{gen}  house=\textsc{all}  happiness=\textsc{top}  nature=\textsc{dat}1=\textsc{foc}  enter-\textsc{seq}  come-\textsc{umrk}-\textsc{ptcp}=\textsc{qt}\\
\glt    ‘(He said to his wife) that the happiness comes into \Highlight{one}’s house, (i.e.) the happiness (comes home) naturally.’ [Fo: 090307\_00.txt]

\z

In \REF{ex:7:8}, it may be possible to think that \textit{nusi} (\textsc{rfl}) corresponds to the man, i.e. the husband of the couple, but it is more natural to think that it corresponds to unspecific people. In other words, it is more appropriate to think that the utterance said by the man in \REF{ex:7:8} is a kind of conventional wisdom.

  The above examples show that \textit{nusi} (\textsc{rfl}) behaves in the same way with common nouns, since it takes \textit{nkja} (\textsc{appr}) as in (\ref{ex:7-7}a), and takes \textit{nu} (\textsc{gen}) in the modifier slot of an NP as in \REF{ex:7:8}. Additionally, it usually takes \textit{nu} (\textsc{nom}) as the subject of the clause as follows.

\ea \label{ex:7:9}  [Context: Asking \textsc{tm} if she made the pickles.]

{\MY}
\glll  kurəə  nusinu  cukutɨ?\\
    \textit{ku-rɨ=ja}  \textit{\Highlight{nusi}=nu}  \textit{cukur-tɨ}\\
    \textsc{prox}-\textsc{nlz}=\textsc{top}  \textsc{rfl}=\textsc{nom}  make-\textsc{seq}\\
    \glt [lit.] ‘Did \Highlight{yourself} make this?’ [Co: 101023\_01.txt]

\z

In \REF{ex:7:9}, the antecedent of \textit{nusi} (\textsc{rfl}), i.e. ‘you,’ is not overtly expressed, but it can be inferred from the context. Considering this example, it may be appropriate to say that the antecedents of \textit{nusi} (RFL) is the agent (or possibly experiencer) of the event expressed by the clause, rather than the subject of the clause.

\textit{nusi} (\textsc{rfl}) can be reduplicated as follows, where the following root is lengthened.

\ea \label{ex:7:10}  [Context: Remembering the day the outdoor lamps were set in the shopping street of the village]

{\TM}
\glll  nusinusiinu  jaanu  kadukadunan  tatɨtancjɨjo.\\
\textit{\Highlight{nusi+nusi}=nu}  \textit{jaa=nu}  \textit{kadu+kadu=nan}  \textit{tatɨr-tar-n=ccjɨ=joo}\\
\textsc{red}+\textsc{rfl}=\textsc{gen}  house=GEN  RED+corner=\textsc{loc}1  stand-\textsc{pst}-\textsc{ptcp}=\textsc{qt}=\textsc{cfm}1\\
\glt ‘(They) stood (the outdoor lamps) at each corner of \Highlight{each one}.’ [Co: 120415\_00.txt]

\z

  In the examples discussed above, \textit{nusi} (\textsc{rfl}) indicates only a human referent. Additionally, \textit{nusi} (RFL) can indicate non-human referents, e.g., \textit{mjaa} ‘cat’ as in \REF{ex:7:11}.

\ea \label{ex:7:11}{\TM}
\glll  mjaanu  nusinu  maiba  kada  sjuttoo.\\
\textit{mjaa\textsubscript{i}}\textit{=nu}  \textit{\Highlight{nusi}\textsubscript{i}}\textit{=nu}  \textit{mai=ba}  \textit{kada}  \textit{sɨr-jur=doo}\\
cat=\textsc{nom}  \textsc{rfl}=\textsc{gen}  buttock=\textsc{acc}  smell  do-\textsc{umrk}=\textsc{ass}\\
\glt ‘A cat smells the buttock of \Highlight{itself}. [El: 130820]

\z

\section{Numerals}

A numeral is constituted of a numeral root plus a classifier affix. So far, the following classifier affixes are found in Yuwan: \textit{{}-cɨ} (\textsc{clf}.thing), \textit{{}-kəəi} (\textsc{clf}.time), and \textit{{}-}(\textit{ta})\textit{i} (\textsc{clf}.human). However, these numerals are not very productive, and people usually borrow numerals from Standard Japanese. The numeral in Yuwan usually fills the head slot of an NP and does not fill the modifier slot. If it should fill the modifier slot of an NP, it takes \textit{nu} (\textsc{gen}). Numerals, if they are the subjects of the clauses, take \textit{ga} (\textsc{nom}) or nothing except for the cases where they take limitter particles. There are no examples where numerals take any plural marker in my texts so far.

  In \sectref{sec:key:7.4.1}, I will discuss the syntax of numerals. In \sectref{sec:key:7.4.2}, I will discuss the morphology of numerals.

\subsection{Syntax of numerals}

First, we will examine the examples of \textit{{}-cɨ} (\textsc{clf}.thing). The combinations of numeral roots and \textit{{}-cɨ} (\textsc{clf}.thing) are summarized in \tabref{tab:key:47}. The morphological analysis of the numerals in \tabref{tab:key:47} is shown in \sectref{sec:key:7.4.2.}

\begin{table}
\caption{\label{tab:key:47} Numerals made with \textit{-cɨ} (\textsc{clf}.thing) (surface forms)}

\begin{tabular}{lll}
\lsptoprule
Numbers  & Word forms  & Meaning\\
\midrule
1 & tˀɨɨ  &a thing\\
2 & tˀaacɨ&  two things     \\
3 & miicɨ & three things    \\
4 & juucɨ & four things     \\
5 & ɨcɨcɨ & five things     \\
6 & muucɨ & six things      \\
7 & nanacɨ&  seven things   \\
8 & jaacɨ & eight things    \\
9 & kˀuunucɨ & nine things  \\
10&  tuu    & ten things    \\
\lspbottomrule
\end{tabular}
\end{table}

For the numbers more than ten in \tabref{tab:key:47}, there are no native terms, so we have to use borrowings from standard Japanese. I wil present examples of \textit{{}-cɨ} (\textsc{clf}.thing), where the numerals head the NPs.

\ea \label{ex:7:12}
\ea \label{ex:7:12a} [Context: A man had put two baskets under a big pear tree.]

{\TM}
\gllll  un  kagonu  tˀɨɨ  cɨdɨ   ikjunwake.\\
\textit{u-n}  \textit{kago=nu}  \textit{\Highlight{tˀɨɨ}}  \textit{cɨm-tɨ} \textit{ik-jur-n=wake}\\
\{[\textsc{mes}-\textsc{advz}  basket=\textsc{gen}]  [one.\textsc{clf}.thing]\}  load-\textsc{seq}          go-\textsc{umrk}-\textsc{ptcp}=\textsc{cfp}\\
\{[Modifier]    [Head]\}\textsubscript{NP}  \\
\glt ‘(The boy) puts the one of the baskets on (the front of his bicycle) and goes.’ [\textsc{pf}: 090222\_00.txt]

\ex \label{ex:7:12b}[Context: There is a big pear tree, from which a man is picking up pears.] = (\ref{ex:6-62}a)

  {\TM}
\glll kɨɨnu  sjanannja  kagonu  tˀaacɨ  ucjutɨ,\\
\textit{kɨɨ=nu}  \textit{sja=nan=ja}  \textit{kago=nu}  \textit{\Highlight{tˀaacɨ}}  \textit{uk-tur-tɨ}\\
      tree=\textsc{gen}  under=\textsc{loc}1=\textsc{top}  \{[basket=GEN]  [two.\textsc{clf}.thing]\}   put-\textsc{prog}-\textsc{seq}\\
          \{[Modifier]  [Head]\}\textsubscript{NP}\\
\glt    ‘Under the tree, (the man) put two baskets, and ...’      [\textsc{pf}: 090222\_00.txt]

\ex \label{ex:7:12c}[Context: A boy tumbled off his bicycle and the pears in the basket in front of the bicycle scattered. Three other boys helped him to gather the pears. After that, the one of the three boys found the boy’s hat, so he called him and handed the hat to him.]

{\TM}
\glll gan  sjan  tuki  mata  joonasinu  miicɨ,\\
\textit{ga-n}  \textit{sɨr-tar-n}  \textit{tuki}  \textit{mata}  \textit{joonasi=nu}  \textit{\Highlight{miicɨ}} {\textbar}hora{\textbar},  muratɨ  cˀjaroo.\\
      \textsc{mes}-\textsc{adnz}  do-\textsc{pst}-\textsc{ptcp}  time  again  \{[pear=\textsc{gen}]  [three.\textsc{clf}.thing]\}    \textit{hora}  \textit{muraw-tɨ}  \textit{k-tar-oo}\\
              \{[Modifier]  [Head]\}\textsubscript{NP}    hey  receive-\textsc{seq}  come-\textsc{pst}-\textsc{supp}\\
\glt    ‘(At) that time, probably (the boys) received three pears again, and came (back).’      [\textsc{pf}: 090222\_00.txt]

    \z
\z

The numerals tend to fill the head slot of an NP (except for the case of “quantifier-float” below). However, there is an example where the numeral fills the modifier slot of an NP as in \REF{ex:7:15}. After you have read the descritption about quantifier-float below, it should be noted that all of the numerals as in \REF{ex:7:12} are not the examples of quantifier-float. This was shown by the case particles which the NP modifiers take in \REF{ex:7:12}, where the NP modifiers take a genitive case \textit{nu}, not \textit{ba} (\textsc{acc}), despite the NP’s being the objects of the clauses. This fact shows that the numerals are not apart from the preceding NPs, i.e, not floated quontifiers, but that they fills the head slots of the NPs with the preceding NP modifiers.

Second, the combinations of numeral roots and \textit{{}-kəəi} (\textsc{clf}.time) are summarized in \tabref{tab:key:48}. The morphological analysis of the numerals in \tabref{tab:key:48} is shown in \sectref{sec:key:7.4.2.}

\begin{table}
\caption{\label{tab:key:48}. Numerals made with \textit{-kəəi} (\textsc{clf}.time) (surface forms)}
\begin{tabular}{lll}
\lsptoprule
Numbers & Word forms  & Meaning\\
\midrule
1 & cˀjukəi&  once        \\
2 & tˀakəi & twice        \\
3 & mikəi  &three times   \\
4 & jukəi  &four times    \\
5 & ɨcɨkəi & five times   \\
6 & mukəi  &six times     \\
7 & nanakəi&  seven times \\
8 & jakəi  &eight times   \\
9 & kunkəi & nine times   \\
10&  tukəi & ten times    \\
\lspbottomrule
\end{tabular}
\end{table}

For the numbers above ten in \tabref{tab:key:48}, there are no native terms, so we have to use borrowings from standard Japanese. I will present examples of \textit{{}-kəəi} (\textsc{clf}.time), where the numeral behaves as an adverb.

\ea \label{ex:7:13}
\ea \label{ex:7:13a}{\TM}
\glll  an  tacɨgəə  cˀjukəəin  toorɨtɨn  njan.\\
\textit{a-n}  \textit{tacɨgɨ=ja}  \textit{\Highlight{cˀjukəəi}=n}  \textit{toorɨr-tɨ=n}  \textit{\Highlight{nj}-an}\\
\textsc{dist}-\textsc{adnz}  prop=\textsc{top}  one.\textsc{clf}.time=even  fall-\textsc{seq}=ever  \textsc{exp}-\textsc{neg}\\
\glt ‘That prop has never fallen even once.’ [El: 130816]

\ex \label{ex:7:13b}{\TM}
\glll  mata..  uma  tˀakəi  izjai,  cˀjai,  sjattu.\\
\textit{mata}  \textit{u-ma}  \textit{\Highlight{tˀakəəi}}  \textit{ik-tai}  \textit{k-tai}  \textit{sɨr-tar-tu}\\
again  \textsc{mes}-place  two.\textsc{clf}.time  go-\textsc{lst}  come-L\textsc{st}  do-\textsc{pst}-\textsc{csl}\\
\glt ‘(The three boys) went there and came back two times.’ [\textsc{pf}: 090225\_00.txt]

\z
\z

\textit{{}-kəəi} (\textsc{clf}.time) goes through the phonological rule in \sectref{sec:key:2.4.5.} Therefore, one of the vowels is deleted as in (\ref{ex:7-13}b) or \tabref{tab:key:48}. However, if \textit{n} ‘even’ follows \textit{{}-kəəi} (\textsc{clf}.time), the environment is out of the application of the rule, and the underlying form appears in the surface form without any modification as /cˀju-kəəi=n/ (one-\textsc{clf}.time=even) ‘even once’ in (\ref{ex:7-13}a).

Third, the combinations of numeral roots and \textit{{}-tai} (\textsc{clf}.person) are summarized in \tabref{tab:key:49}. The morphological analysis of the numerals in \tabref{tab:key:49} is shown in \sectref{sec:key:7.4.2.}

\begin{table}
\caption{\label{tab:key:49}. Numerals made with \textit{-tai} (\textsc{clf}.person) (surface forms)}
\begin{tabular}{lll}
\lsptoprule
Numbers  & Word forms  & Meanings\\
\midrule
1&  cˀjui & a person      \\
2&  tˀai  &two people     \\
3&  micjai&  thee people  \\
4&  jutai & four people   \\
\lspbottomrule
\end{tabular}
\end{table}

For the numbers above four in \tabref{tab:key:49}, there are no native terms, so we have to use borrowings from standard Japanese. The following examples show the numerals containing \textit{{}-(ta)i} (\textsc{clf}.person).

\ea \label{ex:7:14}
\ea \label{ex:7:14a}
{\TM}
\gllll  hunto,  an  tˀaiga  wuppoo,  muru   jiccja  atanmundoo.\\
\textit{hunto}  \textit{a-n}  \textit{\Highlight{tˀai}=ga}  \textit{wur-boo}  \textit{muru}  \textit{jiccj-sa}  \textit{ar-tar-n=mun=doo}\\
really  \{[\textsc{dist}-\textsc{adnz}]  [two.\textsc{clf}.person=\textsc{nom}]\}  exist-\textsc{cnd}  very   good-\textsc{adj}  \textsc{st}V-\textsc{pst}-\textsc{ptcp}=\textsc{advrs}=\textsc{ass}\\
\{[Modifier]  [Head]\}\textsubscript{NP}    \\
\glt ‘Really, if there were the two [i.e. if the two were alive], it would be very good.’ [\textsc{pf}: 090305\_01.txt]

\ex \label{ex:7:14b}
{\TM}
\glll un  micjaiga  {\textbar}cjanto{\textbar}  hijatɨ  ɨrɨtɨ,\\
      \textit{u-n}  \textit{micjai=ga}  \textit{cjanto}  \textit{hijaw-tɨ}      \textit{ɨrɨr-tɨ}
      \{[\textsc{mes}-\textsc{adnz}]  [three.\textsc{clf}.person]\}=\textsc{nom}  correctly  pick.up-\textsc{seq}            put.in-\textsc{seq}\\
      \{[Modifier]  [Head]\}\textsubscript{NP}    \\
\glt    ‘The three correctly picked up (the pears) and put (them) in (the basket), and ...’     [\textsc{pf}: 090827\_02.txt]

    \z
\z

As mentioned above, numerals in Yuwan rarely fill the modifier slot of an NP. However, there is an example of the case.

\ea \label{ex:7:15}  Numeral filling the modifier slot of an NP

  [Context: Three children were walking a way.] = (\ref{ex:6-135}b)

{\TM}
\gllll  un  kˀwanu,  cˀjuinu  kˀwanu  isjoobiki   hucjɨ,\\
\textit{u-n}  \textit{kˀwa=nu}  \textit{\Highlight{cˀjui}=nu}  \textit{kˀwa=nu}  \textit{isjoobiki} \textit{huk-tɨ}\\
\textsc{mes}-\textsc{adnz}  child=\textsc{nom}  \{[one.\textsc{clf}=\textsc{gen}]  [child]\}=NOM  whistle   blow-\textsc{seq}\\
\{[Modifier]  [Head]\}\textsubscript{NP}      \\
\glt ‘That child, the child (who is) one (of them) whistled, and ...’ [\textsc{pf}: 090305\_01.txt]

\z

So far, the reason for the above use of numerals in the modifier slot of an NP is not clear for me.

Furthermore, the numerals sometimes immediately follow the heads of the core arguments. In \REF{ex:7:16}, the address noun \textit{uzii}, which usually means ‘grandfather’ but means ‘an old man’ here, takes the nominative case \textit{ga}. The \textit{ga} (\textsc{nom}) must not be a genitive case, since address nouns do not take any case particle in the modifier slot of an NP (see \sectref{sec:key:7.2}). Thus, it is clear that the numeral \textit{cˀjui} (one.\textsc{clf}.person) in \REF{ex:7:16} is neither the modifier nor head of the NP.

\ea \label{ex:7:16}  Quantifier-float (After subject NP) [= \REF{ex:6-136}]

[Context: The very beginning of the monologue. \textsc{tm}: ‘(I will) start from the scene (where a man) picks up the pears. There is a pear-tree, (i.e.) a big tree, ...’]

{\TM}
\glll unnəntɨ  uziiga  cˀjui  joonasi   mutunwake.\\
\textit{u-n=nəntɨ}  \textit{uzii=ga}  \textit{\Highlight{cˀjui}}  \textit{joonasi} \textit{mur-tur-n=wake}\\
    \textsc{mes}-\textsc{adnz}=\textsc{loc}2  old.man=\textsc{nom}  one.\textsc{clf}.person  pear  pick.up-\textsc{prog}-\textsc{ptcp}=\textsc{cfp}\\
\glt    ‘There, an old man is picking up pears.’ [\textsc{pf}: 090225\_00.txt]

\z

Semantically, the numeral \textit{cˀjui} (one.\textsc{clf}.person) modifies \textit{uzii} ‘old man’ meaning that the man indicated by \textit{uzii} ‘old man’ is alone. Syntactically, however, the numeral \textit{cˀjui} (one.\textsc{clf}.person) is separated from the NP where \textit{uzii} ‘old man’ exists. This kind of phenomenon is called “quantifier float” in Japanese linguistics \citep[286]{Shibatani1990}. The example in \REF{ex:7:17} below may be an example of quantifier float, but it may also be analyzed as a single NP.

\ea \label{ex:7:17}  [Context: A boy tumbled in riding bicycle, and was injured.]

{\TM}
\glll gan  jinganu  micjai,  warabɨnu  micjai,  tuutɨ,\\
\textit{ga-n}  \textit{jinga=nu}  \textit{\Highlight{micjai}}  \textit{waraɨ=nu} \textit{\Highlight{micjai}}  \textit{tuur-tɨ}\\
    \textsc{mes}-\textsc{advz}  man=\textsc{nom}/\textsc{gen}  three.\textsc{clf}.person  child=NOM/GEN   three.\textsc{clf}.person  pass-\textsc{seq}\\
\glt    ‘There three men, (i.e.) three child passed, and ...’ [\textsc{pf}: 090827\_02.txt]

\z

In \REF{ex:7:17}, the expression \textit{jinga=nu} \textit{micjai} can be analyzed as either (man=\Highlight{\textsc{nom}} three.\textsc{clf}.person), i.e. quantifier float, or (man=\Highlight{\textsc{gen}} three.\textsc{clf}.person), i.e. a single NP, because the common noun \textit{jinga} ‘man’ can take both \textit{nu} (\textsc{nom}) and \textit{nu} (\textsc{gen}) (see \sectref{sec:key:7.1.1}). In the former analysis, the numeral \textit{micjai} (three.\textsc{clf}.person) is a floated quantifier apart from the preceding NP. In the latter analysis, the numeral fills the head slot of the NP, where the preceding nominal \textit{jinga} ‘man’ fills the modifier slot. The same argument can be applied to another NP in \REF{ex:7:17}, i.e. \textit{warabɨ=nu} \textit{micjai}. There is no answer to determine which analysis is really correct.

  All of the numerals in the above examples expressed cardinal numbers. If you want to express ordinary numbers, you may have the affix \textit{{}-me} (\textsc{odn}) follow the numerals introduced above. Considering the phoneme /e/, the affix \textit{{}-me} (ODN) is thought to be borrowed from the standard Japanese relatively recently.

\ea \label{ex:7:18}  [Context: Complaining about the decline of her memory]

{\TM}
\glll  maganu  cˀjuigadəə  sicjussɨga,   tˀaimekaroo  sijandoojaa.\\
\textit{maga=nu}  \textit{cˀjui=gadɨ=ja}  \textit{sij-tur-sɨga}  \textit{\Highlight{tˀai-me}=kara=ja}  \textit{sij-an=doo=jaa}\\
grandchild=\textsc{gen}  one.\textsc{clf}.person=\textsc{lmt}=\textsc{top}  know-\textsc{prog}-\textsc{pol}            two.\textsc{clf}-\textsc{odn}=\textsc{abl}=\textsc{top}  know-\textsc{neg}=\textsc{ass}=\textsc{sol}\\
\glt ‘(I) know (the name of) one grandchild, but don’t know (that of) the second one (and more).’ [Co: 110328\_00.txt]

\z

Before concluding this section, I will present some combinations of the numerals with a few morphemes. First, the numerals can be compounded with the adverb \textit{naa}. The combination means there are other referents whose number is indicated by the numerals. I will present examples in (\ref{ex:7-19}a-b).

\ea \label{ex:7:19}  Numerals compounded with \textit{naa} ‘other’
\ea \label{ex:7:19a}[Context: Seeing some acquaintances of \textsc{tm} in a picture]

{\TM}
\glll  naacˀjuinu  cˀjoo  koogi  jappa.\\
\textit{\Highlight{naa+cˀjui}=nu}  \textit{cˀju=ja}  \textit{koogi}  \textit{jar-ba}\\
other+one.\textsc{clf}.person=\textsc{gen}  person=\textsc{top}  Kogi  \textsc{cop}-\textsc{csl}\\
\glt ‘Since another person is Kogi.’ [Co: 120415\_00.txt]

\ex \label{ex:7:19b}
{\MY}
\glll cɨkɨmunukkwaja  naatˀɨɨ  {\textbar}itadak{\textbar}oojəə\\
      \textit{cɨkɨ+mun-kkwa=ja}  \textit{\Highlight{naa+tˀɨɨ}}  \textit{itadak-oo=jəə}\\
      pickle.\textsc{inf}+thing-\textsc{dim}=\textsc{top}  other+one.\textsc{clf}.thing  eat.modesty-\textsc{int}=\textsc{cfm}2\\
\glt    ‘(I) will eat another (piece of) pickles.’       [Co: 101023\_01.txt]

  \z
\z

  Additionally, the numerals may be followed by a particle \textit{naa} ‘each.’

\ea \label{ex:7:20}  [Context: Remembering the way of traditional funerals]

{\TM}
\glll  aahata,  miicɨnaa,  tˀaacɨnaa\\
\textit{aa+hata}  \textit{\Highlight{miicɨ=naa}}  \textit{\Highlight{tˀaacɨ=naa}}\\
red+flag  three.\textsc{clf}.thing=each  two.\textsc{clf}.thing=each\\
\glt ‘(They stood) red flags, three (of which in front of) each (line of the funeral), two (of which in front of) each (line of the funeral).’ [Co: 111113\_01.txt]

\z

  Furtheremore, the numerals can be followed by \textit{{}-gɨna} ‘together.’

\ea \label{ex:7:21}  [Context: Talking about two acquaintaces, who lived outside the community.]

{\TM}
\glll  tˀaigɨna  kaaranba,\\
\textit{tˀai-gɨna}  \textit{kaar-an-ba}\\
two.\textsc{clf}.person-together  relate-\textsc{neg}-\textsc{csl}\\
\glt ‘Both of the two did not contact (with the people in our community), so ...’ [Co: 120415\_01.txt]

\z

The combinations of numeral roots and classifier affixes are far from productive. Therefore, the morphological analyses of numerals in the underlying forms are not expressed in the above discussion. The tentative morphological analyses of numerals in Yuwan will be discussed in the following subsection.

\subsection{Morphology of numerals}

It is possible to divide the numerals in Yuwan into the following morphemes, shown in \tabref{tab:key:50}.

\begin{table}
\caption{\label{tab:key:50}. Morphological analyses of the numeral (surface forms)}
\begin{tabularx}{\textwidth}{Xl ll ll}
\lsptoprule
Numbers  \textit{{}-cɨ} (\textsc{clf}.thing)    Numbers  \textit{{}-kəəi} (\textsc{clf}.time)    Numbers  \textit{{}-}(\textit{ta})\textit{i} (\textsc{clf}.person)\\
\midrule
1  &tˀɨɨ          &  1 & cˀju  {}-kəi& 1  &  cˀju  {}-i\\
2  &tˀaa    {}-cɨ &  2 & tˀa  {}-kəi & 2  & tˀa  {}-i  \\
3  &mii    {}-cɨ  &  3 & mi  {}-kəi  & 3  & mi  {}-cjai \\
4  &juu    {}-cɨ  &  4 & ju  {}-kəi  & 4  & ju  {}-tai  \\
5  &ɨcɨ    {}-cɨ  &  5 & ɨcɨ  {}-kəi &    &            \\
6  &muu    {}-cɨ  &  6 & mu  {}-kəi  &    &            \\
7  &nana    {}-cɨ &  7 & nana  {}-kəi&    &            \\
8  &jaa    {}-cɨ  &  8 & ja  {}-kəi  &    &            \\
9  &kˀuunu {}-cɨ  &  9 & kun  {}-kəi &    &            \\
10 & tuu          & 10 & tu  {}-kəi  &    &            \\
\lspbottomrule
\end{tabularx}
\end{table}

 The above table shows that the numerals indicating 1, 9, and 10 behave irregularly.

 The numeral that means ‘one thing,’ i.e, \textit{tˀɨɨ} at the upper-most and left-most position in \tabref{tab:key:50}, appears that it is not followed by the classifier \textit{{}-cɨ} (\textsc{clf}.thing) and that it indicates the notion by itself. Additionally, the form \textit{tˀɨɨ} (one.\textsc{clf}.thing) is very different from the tentative root form \textit{cˀju-} ‘one,’ which is used to indicate a single referent with \textit{-kəəi} (\textsc{clf}.time) and \textit{{}-i} (\textsc{clf}.person).

 The numeral root that indicates nine referents is \textit{kˀuunu-} ‘nine’ when it is followed by \textit{{}-cɨ} (\textsc{clf}.thing), but is \textit{kun-} ‘nine’ when it is followed by \textit{{}-kəəi} (\textsc{clf}.time).

The numeral that means ‘ten things,’ i.e. \textit{tuu} at the lower-most and left-most position in \tabref{tab:key:50}, appears that it is not followed by the classifier \textit{{}-cɨ} (\textsc{clf}.thing) and that it indicates the notion by itself. The same form appears to be followed by \textit{{}-kəəi} (\textsc{clf}.time) with vowel deletion, i.e. /tu-kəi/ (ten-\textsc{clf}.time) ‘ten times.’

  The classifiers to count human is \textit{{}-i} (\textsc{clf}.person) if the preceding numeral roots indicate one or two person(s) such as /cˀju-i/ (one-\textsc{clf}.person) ‘a person’ or /tˀa-i/ (two-\textsc{clf}.person) ‘two people,’ and it is \textit{{}-tai} (\textsc{clf}.person) if the preceding numeral roots indicate three or four people such as /mi-cjai/ (three-\textsc{clf}.person) ‘three people’ (with the palatalization of //tai// to /cjai/) or /ju-tai/ (four-\textsc{clf}.person) ‘four people.’

  It is difficult to determine the underlying forms of the numeral root. In surface forms, they have more than one mora before \textit{{}-cɨ} (\textsc{clf}.thing), but do not necessarily have more than one mora before \textit{{}-kəəi} (\textsc{clf}.time) or \textit{{}-}(\textit{ta})\textit{i} (\textsc{clf}.person). While there may be some other analyses, I propose the following analysis as the best.

\begin{table}
\caption{\label{tab:key:51} Numeral roots in Yuwan (underlying forms)}
\begin{tabular}{ll}
\lsptoprule
Numbers & Numeral roots\\
\midrule
1  &\textit{tˀɨɨ} / \textit{cˀju-}      \\
2  &\textit{tˀaa-}                      \\
3  &\textit{mii-}                       \\
4  &\textit{juu-}                       \\
5  &\textit{ɨcɨ-}                       \\
6  &\textit{muu-}                       \\
7  &\textit{nana-}                      \\
8  &\textit{jaa-}                       \\
9  &\textit{kˀuunu-} / \textit{kun-}    \\
10 & \textit{tuu}                         \\
\lspbottomrule
\end{tabular}
\end{table}

{In \tabref{tab:key:51}, only} {\textit{tˀɨɨ} and} {\textit{tuu} are free morphemes, and the others are bound morphemes. If numeral roots that have the same-vowel sequences at their root-final positions are followed by} {\textit{{}-kəəi} (\textsc{clf}.time) or} {\textit{{}-}}{(}{\textit{ta}}{)}{\textit{i} }{(\textsc{clf}.person), the vowel sequences become a single vowel. For example,} {\textit{jaa-} ‘eight’ plus} {\textit{{}-kəəi} (\textsc{clf}.time) becomes /ja-kəi/, where //jaa// ‘eight’ becomes /ja/ because of the root-final vowel deletion. This analysis} {can avoid assuming a putative underlying form} {\textit{tˀɨ} ‘one thing,’ which does not appear in any surface form. In other words, I propose that all of the morphemes that have long vowel at their root-final position in the numerals to count things are originally long. Other examples that are relevant to vowel deletion are shown below.}

\begin{table}
\caption{\label{tab:key:52} Morphophonological alternation with \textit{{}-kəəi} (\textsc{clf}.time)}
\begin{tabular}{lll}
\lsptoprule
Numbers & Underlying forms    & Surface forms\\
Numeral  roots  &  Classifiers    & Numerals\\
\midrule
2 & \textit{tˀaa-} +  \textit{{}-kəəi} (\textsc{clf}.time) & >  tˀa-kəi\\
3 & \textit{mii-}  +                                       & >  mi-kəi \\
4 & \textit{juu-}  +                                       & >  ju-kəi \\
6 & \textit{muu-}  +                                       & >  mu-kəi \\
8 & \textit{jaa-}  +                                       & >  ja-kəi \\
10&  \textit{tuu}  +                                       & >  tu-kəi \\
\lspbottomrule
\end{tabular}

\end{table}

\begin{table}
\caption{\label{tab:key:53} Morphophonological alternation with \textit{{}-}(\textit{ta})\textit{i} (\textsc{clf}.person)}
\begin{tabular}{lll}
\lsptoprule
Numbers  & Underlying forms    & Surface forms\\
  Numeral roots    & Classifiers    & Numerals\\
\midrule
2  & \textit{tˀaa-}  +  \textit{{}-i} (\textsc{clf}.person)  &>  tˀa-i    \\
3  & \textit{mii-}  +  \textit{{}-tai} (\textsc{clf}.person) & >  mi-cjai \\
4  & \textit{juu-}  +  \textit{{}-tai} (\textsc{clf}.person) & >  ju-tai  \\
\lspbottomrule
\end{tabular}
\end{table}

The above tables show that the root-final long vowels become short before \textit{{}-kəəi} (\textsc{clf}.time) or \textit{{}-}(\textit{ta})\textit{i} (\textsc{clf}.person). In \tabref{tab:key:53}, the initial morphophoneme //t// in \textit{{}-tai} (\textsc{clf}.person) undergoes palatalization (plus affrication) and becomes /cj/, which is thought to be caused by the preceding morphophoneme //i// in \textit{mii-} ‘three.’

In this grammar, the morphemic boundaries of numeral words are not expressed (even if they are present at the underlying level) unless they need to be clearly distinguished.

\section{Indefinite pronouns}

Yuwan has affixes that turns interrogative nominal stems into indefinite pronouns: \textit{{}-nkuin}, which is labeled as the “indefinitizer” (\textsc{indfz}) in this grammar. The combinations of the interrogative nominal stems and \textit{-nkuin} (\textsc{indf}Z) are shown in the following table.

\begin{table}
\caption{\label{tab:key:54} Indefinite pronouns in Yuwan}
\begin{tabular}{lll}
\lsptoprule
Interrogative nominals  &   Indefinitizer    & Indefinite pronouns\\
\midrule                                                                                                \\
\textit{nuu}  ‘what’   & +  \textit{{}-nkuin} (\textsc{indfz})&  >  /nunkuin/  ‘anything’               \\
\textit{daa}  ‘where’  &                                      &  >  /dankuin/  ‘anywhere’               \\
\textit{ɨcɨɨ}  ‘when’  &                                      &  >  /ɨcɨnkuin/  ‘always’                \\
\textit{taru}  ‘who’   &                                      & >  /tarunkuin/  ‘anybody’               \\
\textit{dɨru}  ‘which’ &                                      &   >  /dɨrunkuin/  ‘anyone (of them)’    \\
\lspbottomrule
\end{tabular}
\end{table}

Interrogative nominals that have the same-vowel sequence at stem-final positions undergo the vowel deletion discussed in \sectref{sec:key:2.4.5}, e.g. //\Highlight{nuu}// ‘what’ + \textit{{}-nkuin} (\textsc{indfz}) > /\Highlight{nu}-nkuin/.

  I will present examples of \tabref{tab:key:54}. The indefinite pronouns in the underlying level and their correspondents in the free translation are underlined below.

\ea \label{ex:7:22}  Interrogative nominals + \textit{{}-nkuin} (\textsc{indfz})
\ea \label{ex:7:22a}\textit{nuu} ‘what’ + \textit{{}-nkuin} (\textsc{indfz})

    [Context: \textsc{tm} tells the present author that US always does not sit still, but that she always tries to serve something to eat for the guest.]

{\TM}
\glll nunkuin  izjasicjɨjo.  hanasinkjoo  sɨrancjɨjo.\\
\textit{\Highlight{nuu-nkuin}}  \textit{izjas-i=ccjɨ=joo}  \textit{hanasi=nkja=ja}  \textit{sɨr-an=ccjɨ=joo}\\
      what-\textsc{indfz}  put.out-\textsc{inf}=\textsc{qt}=\textsc{cfm}1  conversation=\textsc{appr}=\textsc{top}                  do-\textsc{neg}=\textsc{qt}=\textsc{cfm1}\\
\glt    ‘(She) puts out [i.e. serves] \Highlight{anything}. (She) does not (begin) the conversation.’      [Co: 110328\_00.txt]

\ex \label{ex:7:22b}\textit{daa} ‘where’ + \textit{{}-nkuin} (\textsc{indfz})

{\TM}
\glll  naa,  dankuinkaci  abɨrattɨ,\\
\textit{naa}  \textit{\Highlight{daa-nkuin}=kaci}  \textit{abɨr-ar-tɨ}\\
\textsc{fil}  where-\textsc{indfz}=\textsc{all}  call-P\textsc{ass}-\textsc{seq}\\
\glt ‘(My mother) was called (for the recording of the traditional songs) \Highlight{anywhere}, and ...’ [Co: 111113\_01.txt]

\ex \label{ex:7:22c}\textit{ɨcɨɨ} ‘when’ + \textit{{}-nkuin} (\textsc{indfz})

{\TM}
\glll  waakjoo  ɨcɨnkuin  waratuncjɨjo.\\
\textit{waakja=ja}  \textit{\Highlight{ɨcɨɨ-nkuin}}  \textit{waraw-tur-n=ccjɨ=joo}\\
1\textsc{pl}=\textsc{top}  when-\textsc{indfz}  laugh-\textsc{prog}-\textsc{ptcp}=\textsc{qt}=\textsc{cfm}1\\
\glt ‘I am \Highlight{always} laughing (remembering the old days).’ [Co: 120415\_00.txt]

\ex \label{ex:7:22d}\textit{taru} ‘who’ + \textit{{}-nkuin} (\textsc{indfz})

{\TM}
\glll  tarunkuin,  ta ..  jiccjan  munnu  appoo,\\
\textit{\Highlight{ta-ru-nkuin}}  \textit{ta}  \textit{jiccj-sa+ar-n}  \textit{mun=nu}  \textit{ar-boo}\\
who-\textsc{nlz}-\textsc{indfz}  who  good-\textsc{adj}+\textsc{st}V-\textsc{ptcp}  thing=\textsc{nom}  exist-\textsc{cnd}\\
\glt ‘If (my grandfather) had something good, he would give it to \Highlight{anybody}.’ [Co: 120415\_01.txt]

\ex \label{ex:7:22e} \textit{dɨru} ‘which’ + \textit{{}-nkuin} (\textsc{indfz})

{\TM}
\glll  dɨrunkuin  kamɨjoo.\\
\textit{dɨ-ru-nkuin}  \textit{kam-ɨ=joo}\\
which-\textsc{nlz}-\textsc{indfz}  eat-\textsc{imp}=\textsc{cfm}1\\
\glt ‘Eat \Highlight{anything} (there).’ [El: 130820]

\z
\z

The above examples show that \textit{{}-nkuin} (\textsc{indfz}) changes the questional meanings of the interrogative stems to the indefinite ones. As mentioned in \sectref{sec:key:4.3.6}, there are other affixes that can also turn interrogative stems into indefinite words, i.e. \textit{{}-ninkuinin} (\textsc{indf}Z) and \textit{{}-sjɨnkaasjɨn} (INDFZ). The difference among them is that \textit{{}-nkuin} (INDFZ) forms a nominal, but that \textit{{}-ninkuinin} (INDFZ) and \textit{{}-sjɨnkaasjɨn} (INDFZ) form adverbs. In fact, \textit{{}-nkuin} (INDFZ) is very similar to \textit{{}-ninkuinin} (INDFZ). One might think that the former could be divided into several morphemes such as /nkuin/ =\textit{n=kui=n} (any=INDFZ=any). However, we do not accept this analysis. The indefinite pronoun \textit{{}-nkuin} can be followed by \textit{kaci} as in (\ref{ex:7-22}b). If we analyzed it as /nkuinkaci/ =\textit{n=kui=n=kaci} (any=INDFZ=any=\textsc{all}), we would have to admit the order of \textit{=n=kaci} (any=ALL), but \textit{kaci} (ALL) usually precedes (not follows) \textit{n} ‘any’ when it follows interrogative nominals, e.g. \textit{daa=kaci=n} (where=ALL=any) ‘anywhere’ in (\ref{ex:5-42}a) in \sectref{sec:key:5.3.2.} Thus, we do not divide \textit{{}-nkuin} (INDFZ) into multiple morphemes.

\section{Deverbal nominals}

There is an affix that can change verbal stems to nominal stems, i.e. \textit{{}-jaa} ‘person.’ Additionally, verbal stems can become nominal stems by compounding, which was discussed in \sectref{sec:key:4.2.3.1} and \sectref{sec:key:4.2.3.2.}

Semantically, \textit{{}-jaa} means ‘a person who does the action frequently and/or deliberately,’ which is abbreviated to ‘person’ or simply “\textsc{nlz}” (i.e. nominalizer) in the gloss. Morphologically, \textit{-jaa} ‘person’ can directly follow the verbal root as in (\ref{ex:7-23}a-b). Morphophonologically, it belongs to Type C verbal affixes (see \sectref{sec:key:8.2.1.3}). For example, the final //r// of \textit{tur-} ‘take’ is lost before \textit{{}-jaa} ‘person’ as in (\ref{ex:7-23}b).

\ea \label{ex:7:23}\ea \label{ex:7:a}\textit{hasij-} ‘run’ + \textit{{}-jaa} ‘person’ [Context: Talking about students who participate in the training camp held in the village]

{\TM}
\glll  hasijaankjanu  {\textbar}gassjuku{\textbar}sjɨ  kjuuroogai?\\
\textit{\Highlight{hasij-jaa}=nkja=nu}  \textit{gassjuku=sjɨ}  \textit{k-jur-oo=ga=i}\\
run-person=\textsc{appr}=\textsc{nom}  training.camp=\textsc{inst}  come-\textsc{umrk}-\textsc{supp}=\textsc{cfm}3=\textsc{plq}\\
\glt ‘Runners would come for training camp, you know.’ [Co: 110328\_00.txt]

\ex \label{ex:7:b}\textit{tur-} ‘take’ + \textit{{}-jaa} ‘person’ [Context: Talking about the relationship between a person and some people]

{\TM}
\glll  attaa   sisitujaa.\\
\textit{a-rɨ-taa}  \textit{sisi+\Highlight{tur-jaa}}\\
\textsc{dist}-\textsc{nlz}-\textsc{pl}  boar+take-person\\
\glt ‘(He is) their boar-taker [i.e. a person who always takes boars, and he is their relative].’ [Co: 120415\_00.txt]

\z
\z

Interestingly, the nominalized verbal stem in (\ref{ex:7-23}b), i.e, \textit{tur-jaa} (take-person), can form a compound with a preceding nominal, i.e. \textit{sisi} ‘boar.’

As mentioned above, the meaning of \textit{{}-jaa} is not so simple that it is not very productive. However, if we restrict the context, it can follow a few derivational affixes, i.e. \textit{{}-as} (\textsc{caus}) and \textit{{}-arɨr} (P\textsc{ass}). The contexts of the following examples are suggested by the present author, and the speaker uttered the appropriate sentences according to the context.

\ea \label{ex:7:24}
\ea \label{ex:7:24a}\textit{{}-as} (\textsc{caus}) + \textit{{}-jaa} ‘person’ [Context: Talking about a naughty boy who always makes other children cry]

{\TM}
\glll  agaraa  munna  nakasjaadoo.\\
\textit{aga-raa}  \textit{mun=ja}  \textit{nak-\Highlight{as-jaa}=doo}\\
\textsc{dist}-\textsc{drg}.\textsc{adnz}  substance=\textsc{top}  cry-\textsc{caus}-person=\textsc{ass}\\
\glt ‘That bad boy always makes someone cry.’ [lit. ‘That bad boy is a person who always makes (someone) cry.’] [El: 121010]

\ex \label{ex:7:24b}\textit{{}-arɨr} (P\textsc{ass}) + \textit{{}-jaa} ‘person’ [Context: Some children are talking about their mischief and trying to determine the person who apologize on their behalf.]

{\TM}
\glll  uroo  oosarɨjaa  naiccjɨdaroogai?\\
\textit{ura=ja}  \textit{oos-\Highlight{arɨr-jaa}}  \textit{nar-i=ccjɨ=daroo=ga=i}\\
2.\textsc{nhon}.\textsc{sg}=\textsc{top}  scold-P\textsc{ass}-person  become-\textsc{inf}=\textsc{qt}=\textsc{supp}=\textsc{cfm}3=\textsc{plq}\\
\glt ‘Probably, you will undertake the role of a person who is scolded, right?’ [lit. ‘Probably, you intend to become the person who is scolded, right?’] [El: 121010]

\z
\z

The above examples show that \textit{{}-jaa} ‘person’ does not necessarily indicates the “agent” of the action that the verbal root indicates. In (\ref{ex:7-24}b), the referetnt indicated by \textit{oos-arɨr-jaa} (scold-P\textsc{ass}-person) ‘a person who is scolded (of the person’s own free will)’ is the patient of \textit{oos-} ‘scold’ (not the agent).

\section{Diminutive affix \textit{{}-kkwa}}

There is an affix \textit{{}-kkwa}, which tends to attach to nominal stems that indicate small (or short) referents as in (\ref{ex:7-25}a-e), but it also attaches to the words that do not necessarily indicate small (or short) referents by themselves as in (\ref{ex:7-25}f-j). It never attaches to the personal pronouns or address nouns.

\ea \label{ex:7:25}  \textit{{}-kkwa} (\textsc{dim})

\ea \label{ex:7:25a}{\TM}
\glll  waakjaga  warabɨkkwa  sjuin,\\
\textit{waakja=ga}  \textit{warabɨ-\Highlight{kkwa}}  \textit{sɨr-tur-i=n}\\
1\textsc{pl}=\textsc{nom}  child-\textsc{dim}  do-\textsc{prog}-\textsc{inf}=\textsc{dat}1\\
\glt ‘When I was a child [lit. was doing a child], ...’ [Co: 111113\_01.txt]

\ex \label{ex:7:25b}{\TM}
\glll  {\textbar}cjoodo  mikan{\textbar}nu  (kun)  kun  huukkwanu  tˀɨɨ  kamboo,  xxx  jiccjai.\\
\textit{cjoodo}  \textit{mikan=nu}  \textit{ku-n}  \textit{ku-n}  \textit{huu-\Highlight{kkwa}=nu} \textit{tˀɨɨ}  \textit{kam-boo}    \textit{jiccj-sa+ar-i}\\
just  mikan=\textsc{gen}  \textsc{prox}-\textsc{adnz}  PROX-\textsc{adnz}  piece-\textsc{dim}=GEN  one.thing  eat-\textsc{cnd}    good-\textsc{adj}+\textsc{st}V-\textsc{npst}\\
\glt ‘If (I) eat just a piece of this \textit{mikan}, (it) is good [i.e. sufficient] (for me).’ [Co: 101023\_01.txt]

\ex \label{ex:7:25c}{\TM}
\glll  injahunɨkkwakacigadɨ  {\textbar}bonbon  bakudan  utusi{\textbar}tattu.\\
\textit{inja+hunɨ-\Highlight{kkwa}=kaci=gadɨ}  \textit{bon+bon}  \textit{bakudan}  \textit{utus-tar-tu}\\
small+ship-\textsc{dim}=\textsc{all}=\textsc{lmt}  \textsc{red}+bong  bomb  fall-\textsc{pst}-\textsc{csl}\\
\glt ‘(The American soldiers) dropped bombs even on the small ships.’ [Co: 110328\_00.txt]

\ex \label{ex:7:25d}{\TM}
\glll  magakkwanu  cˀjɨ,\\
\textit{maga{}-\Highlight{kkwa}=nu  k-tɨ}\\
grandchild-\textsc{dim}=\textsc{nom}  come-\textsc{seq}\\
\glt ‘The grandchild came, and ...’ [\textsc{pf}: 090305\_01.txt]

\ex \label{ex:7:25e}{\TM}
\glll  {\textbar}ittoki{\textbar}kkwa  umanan  ucjutɨ,\\
\textit{ittoki-\Highlight{kkwa}}  \textit{u-ma=nan}  \textit{uk-tur-tɨ}\\
for.a.while-\textsc{dim}  \textsc{mes}-place=\textsc{loc}1  put-\textsc{prog}-\textsc{seq}\\
\glt ‘Putting (the pickles) there for a while, ...’ [Co: 101023\_01.txt]

\ex \label{ex:7:25f}
 {\TM}
\glll  haruesanga  wuinnja  dusikkwa  jatanmun,\\
\textit{harue-san=ga}  \textit{wur-i=n=ja}  \textit{dusi-\Highlight{kkwa}}  \textit{jar-tar-n=mun}\\
Harue-\textsc{hon}=\textsc{nom}  exist{}-\textsc{inf}=\textsc{dat}1=\textsc{top}  friend-\textsc{dim}  \textsc{cop}-\textsc{pst}-\textsc{ptcp}=\textsc{advrs}\\
\glt ‘When Ms. Harue was here, (she and I) were friends, but ...’ [Co: 120415\_01.txt]

\ex \label{ex:7:25g}{\TM}
\glll  usikkwa  kawuroojaacjɨ  jˀicjɨ,\\
\textit{usi-\Highlight{kkwa}}  \textit{kawur-oo=jaa=ccjɨ}  \textit{jˀ-tɨ}\\
cow-\textsc{dim}  raise-\textsc{int}=\textsc{sol}=\textsc{qt}  say-\textsc{seq}\\
\glt ‘(The couple) said that, “Let’s raise a cow,” and ...’ [Fo: 090307\_00.txt]

\ex \label{ex:7:25h} {\TM}
\glll  utakkwadu  utajutattu,  waakjaa  anmaaja.\\
\textit{uta-\Highlight{kkwa}=du}  \textit{utaw-jur-tar-tu}  \textit{waakja-a}  \textit{anmaa=ja}\\
song-\textsc{dim}=\textsc{foc}  sing-\textsc{umrk}-\textsc{pst}-\textsc{csl}  1\textsc{pl}-\textsc{adnz}  mother=\textsc{top}\\
\glt ‘My mother used to sing a song.’ [Co: 111113\_01.txt]

\ex \label{ex:7:25i}  [= \REF{ex:6-29}]\\

{\TM}
\glll  sijan  huikkwa  sjɨ,\\
\textit{sij-an}  \textit{hui-\Highlight{kkwa}}  \textit{sɨr-tɨ}\\
know-\textsc{neg}  pretend-\textsc{dim}  do-\textsc{seq}\\
\glt ‘Pretending not to know (about the thrown snacks), ...’ [Co: 120415\_01.txt]

\ex \label{ex:7:25j}
 {\TM}
\glll  kaman  minzjɨn  njanban,  nukkwan,\footnotemark\\
\textit{kama=n}  \textit{ming-tɨ=n}  \textit{nj-an=ban}  \textit{nuu-\Highlight{kkwa}=n}\\
sickle=even  grasp-\textsc{seq}=even  \textsc{exp}-\textsc{neg}=\textsc{advrs}  what-\textsc{dim}=even\\
\glt ‘(The person said that) despite never having grasped a sickle (or) anything, ...’ [Co: 120415\_01.txt]

\footnotemark{A nominal root composed of only a syllable with a long vowel usually retains its vowel length before \textit{-kkwa} (\textsc{dim}), but \textit{nuu} ‘what’ in this example became /nu/, which conformed to the phonological rule in \sectref{sec:key:2.4.5}.}
\z
\z

In (\ref{ex:7-25}a-e), the nominal stems preceding \textit{{}-kkwa} (\textsc{dim}) indicate small (or short) things, e.g., \textit{warabɨ} ‘child’ in (\ref{ex:7-25}a). In (\ref{ex:7-25}f-j), the nominal stems preceding \textit{{}-kkwa} (DIM) do not necessarily indicate small (or short) things. The nominal stems in (\ref{ex:7-25}f-h) seem to indicate referents that are familiar to the speaker, e.g., \textit{dusi} ‘friend’ in (\ref{ex:7-25}f). The \textit{{}-kkwa} (DIM) in (\ref{ex:7-25}i-j) seem to express some insulting meaning towards the referents of the nominal stems, e.g., \textit{sij-an} \textit{hui} (know-\textsc{neg} pretend) ‘pretending not to know’ in (\ref{ex:7-25}i).

  Morphophonologically, \textit{{}-kkwa} (\textsc{dim}) needs the insertion of /u/ after a nominal stem that ends with //n//.

\ea \label{ex:7:26}  Vowel insertion between //n// and \textit{{}-kkwa} (\textsc{dim})
  \ea  \textit{mun}  ‘thing’  +  \textit{{}-kkwa} (\textsc{dim})  >  /mu.n\Highlight{u}k.kwa/
  \ex  \textit{ɨn}  ‘dog’      >  /ɨ.n\Highlight{u}k.kwa/
  \ex  \textit{gazjan}  ‘mosquito’      >  /ga.zja.n\Highlight{u}k.kwa/
    \z
\z

The vowel insertion in (\ref{ex:7-26}a-c) conforms to the phonological rule in \sectref{sec:key:2.4.3.} Additionally, \textit{{}-kkwa} (\textsc{dim}) sometimes deletes a vowel in the same vowel sequence, e.g., \textit{mizjuu} ‘ditch’ + \textit{{}-kkwa} (DIM) > /mizjukkwa/, which conforms to the phonological rule in \sectref{sec:key:2.4.5.} However, if the nominal stem is composed of only a syllable with a long vowel, the vowel deletion is not likely to occur, e.g., \textit{koo} ‘river’ + \textit{{}-kkwa} (DIM) > /kookkwa/. There is an adverb that seems to include \textit{{}-kkwa} (DIM), i.e. /joikkwa/ [jo̞ikkʷɑ̞] ‘silently,’ which is frequently pronounced as /joikwa/ [jo̞ik(ˀ)ʷɑ̞]. However, /joi/ cannot form a free form by itself, which means /k(k)wa/ in this adverb is not the diminutive affix in modern Yuwan.

  Morphologically, \textit{{}-kkwa} (\textsc{dim}) can attach not only to common nouns as in (\ref{ex:7-25}a-h), but also to formal nouns as in (\ref{ex:7-25}i), interrogative nominals as in (\ref{ex:7-25}j), and demonstrative nominals as in (\ref{ex:7-27}a-b).

\ea \label{ex:7:27}  \textit{{}-kkwa} (\textsc{dim}) attaching to demonstrative nominals

\ea \label{ex:7:a}{\TM}
\glll  kurɨkkwakaci  sɨmɨtɨ,  (e, e,)  naracjui.\\
\textit{\Highlight{ku-rɨ-kkwa}=kaci}  \textit{sɨmɨr-tɨ}    \textit{naras-tur-i}\\
\textsc{prox}-\textsc{nlz}-\textsc{dim}=\textsc{all}  do.\textsc{caus}-\textsc{seq}    make.sound-\textsc{prog}-\textsc{npst}\\
\glt ‘(I) made (him dub the song) to this [i.e. cassette tape], and am (always) making (it) sound [i.e. listening to it].’ [Co: 120415\_00.txt]

\ex \label{ex:7:b}{\TM}
\glll  {\textbar}oiwai{\textbar}nu  umakkwanan  motodacunekocjɨbəi kacjɨ,    \\
\textit{oiwai=nu}  \textit{\Highlight{u-ma-kkwa}=nan}  \textit{motoda+cuneko=ccjɨ=bəi}  \textit{ka-tɨ}\\
monetary.gift=\textsc{gen}  \textsc{mes}-place-\textsc{dim}=\textsc{loc}1  Motoda+Tsuneko=\textsc{qt}=only       write-\textsc{seq}    \\
\glt ‘Writing (my name) Tsuneko Motoda on that place on (the envelope to put in a) monetary gift, ...’ [Co: 110328\_00.txt]

\z
\z

In (\ref{ex:7-27}a-b), the demonstrative nominals indicate small things, i.e. \textit{ku-rɨ} (\textsc{prox}-\textsc{nlz}) ‘this’ indicates a cassette tape, and \textit{u-ma} (\textsc{mes}-place) ‘there’ indicates the small part on the envelop.

  It is probable that the diminutive affix \textit{{}-kkwa} disscussed above is a cognate with the common noun \textit{kˀwa} ‘child,’ since \textit{kˀwa} ‘child’ is sometimes realized as /kkwa/ as in (\ref{ex:7-28}b).\footnote{\citet[39]{Niinaga2010} argued that the nominal \textit{kˀwa} ‘child’ is always realized with glottalization, i.e. [ʔ͡kʷɑ̟]. However, it is merely a tendency, since there is an example like /ujakkwa/ \textit{uja+kˀwa} (parent+child) as in (\ref{ex:7-28}b).}

\ea \label{ex:7:28}  \textit{kˀwa} ‘child’

\ea \label{ex:7:a}{\TM}
\glll  kˀwamaganu  acɨmatɨ,\\
\textit{\Highlight{kˀwa}+maga=nu}  \textit{acɨmar-tɨ}\\
child+grandchild=\textsc{nom}  gather-\textsc{seq}\\
\glt ‘Children and grandchildren gather, and ...’ [Co: 111113\_01.txt]

\ex \label{ex:7:b}{\TM}
\glll  ujakkwa  jappoojoo,\\
\textit{uja+\Highlight{kˀwa}}  \textit{jar-boo=joo}\\
parent+child  \textsc{cop}-\textsc{cnd}=\textsc{cfm}1\\
\glt ‘If (we) are parent and child, ...’ [Co: 120415\_01.txt]

\ex \label{ex:7:c}{\TM}
\glll  daibankˀwadoo.\\
\textit{daiban+\Highlight{kˀwa}=doo}\\
big+child=\textsc{ass}\\
\glt ‘(He is) a big child.’ [El: 110327]

\ex \label{ex:7:d}{\TM}
\glll  kun  mjan  kˀwakkwanu  sjugɨsajaa.\\
\textit{ku-n}  \textit{mjaa=nu}  \textit{\Highlight{kˀwa-kkwa}=nu}  \textit{sjugɨ-sa=jaa}\\
\textsc{prox}-\textsc{adnz}  cat=\textsc{gen}  child-\textsc{dim}=\textsc{nom}  small-\textsc{adj}=\textsc{sol}\\
\glt ‘This kitten [lit. cat’s child] (is) small.’ [El: 110327]

\z
\z

The above examples show that \textit{kˀwa} ‘child’ is realized as /kˀwa/ with the exception of (\ref{ex:7-28}b). I propose that \textit{kˀwa} ‘child’ is different from \textit{{}-kkwa} (\textsc{dim}) in the modern Yuwan. First, \textit{kˀwa} ‘child’ does not induce the vowel insertion when it attaches to //n// as in (\ref{ex:7-28}c).\footnote{\textit{daiban} ‘big’ can form a compound with another nominal root, e.g., \textit{daiban} ‘big’ + \textit{kɨɨ} ‘tree’ > /daibangɨɨ/ ‘big tree,’ where “rendaku” (or sequential voicing) (see \sectref{sec:key:4.2.3.4}) also happens, i.e. //k// > /g/.} On the contrary, \textit{{}-kkwa} (DIM) always induce the vowel insertion when it attaches to //n// as in (\ref{ex:7-26}a-c). Secondly, \textit{{}-kkwa} (DIM) can co-occur with \textit{kˀwa} ‘child,’ and each morpheme expresses a meaning different from each other as in (\ref{ex:7-28}d). Thus, I propose that the affix \textit{{}-kkwa} (DIM) is different from (the compounding of) \textit{kˀwa} ‘child’ in the modern Yuwan.

  Before concluding this section, it should be mentioned that \textit{{}-kkwa} (\textsc{dim}) can follow two kinds of adjectival roots, i.e. \textit{inja-} ‘small’ and \textit{sjugɨ-} ‘small’ as in (\ref{ex:7-29}a-b).

\ea \label{ex:7:29}  Adjectival roots + \textit{kˀwa} ‘child’

\ea \label{ex:7:a}{\TM}
\glll  kan  sjan  injakkwa  muccjutɨ,\\
\textit{ka-n}  \textit{sɨr-tar-n}  \textit{\Highlight{inja-kkwa}}  \textit{mukk-tur-tɨ}\\
\textsc{prox}-\textsc{adnz}  do-\textsc{pst}-\textsc{ptcp}  small-\textsc{dim}  bring-\textsc{prog}-\textsc{seq}\\
\glt ‘(The person) was bringing a small thing like this, and ...’ [Co: 120415\_00.txt]

\ex \label{ex:7:b}{\TM}
\glll  sjugɨkkwabəi.\\
\textit{\Highlight{sjugɨ-kkwa}=bəi}\\
small-\textsc{dim}=only\\
glt ‘(There are) only small things.’ [El: 110327]

\z
\z

The above examples show that \textit{{}-kkwa} (\textsc{dim}) can also follow adjectival roots (not only nominal roots). Therefore, one may think that \textit{{}-kkwa} (DIM) is a clitic (not an affix) according to the criteria in \sectref{sec:key:4.2.1.} However, we do not accept this analysis, since there are only two adjectival roots that can precede \textit{-kkwa} (DIM). It is probable that this irregularity can be explicable considering the diminutive affix’s preference for small referents as its preceding stems as in (\ref{ex:7-25}a-e). Additionally, there is another environment where the adjectival root behaves like the nominal root. For example, the adjectival root and the nominal root can fill the preceding slot in compounds without any affix; on the contrary, the verbal root needs an infinitival affix, which makes the verbal stem like nominal, in order to fill the preceding slot in compounds (see \sectref{sec:key:4.2.3.1} for more details). Thus, I propose that \textit{{}-kkwa} (DIM) is still an affix (not a clitic).

Furtheremore, there is a case where \textit{{}-kkwa} (\textsc{dim}) seems to follow an adjectival “word” (not an adjectival “root”), i.e. /injaasakkwa/ ‘small.’

\ea \label{ex:7:30}{\TM}
\glll  nobujataa  amakkwakaci  injaasakkwa  kan  sj ..\\
\textit{nobuja-taa}  \textit{a-ma-kkwa=kaci}  \textit{\Highlight{injaasakkwa}}  \textit{ka-n}  \textit{sɨr-tɨ}\\
Nobuja-\textsc{pl}  \textsc{dist}-place-\textsc{dim}=\textsc{all}  small  \textsc{prox}-\textsc{advz}  do-\textsc{seq}\\
\glt ‘(The ditch extends) small like this to that place (that belongs to) Nobuja and his friends ...’ [Co: 120415\_00.txt]

\z

At first sight, one may think the word /injaasakkwa/ can be divided into \textit{inja-sa-kkwa} (small-\textsc{adj}-\textsc{dim}). However, we do not accept this analysis because of the two reasons. First, the word /injaasakkwa/ is always used adverbially as in \REF{ex:7:30}. Secondly, the vowel in its middle position is always long, i.e. /inj\Highlight{aa}sa/ (not /inj\Highlight{a}sa/). Thus, I will propose that /injaasakkwa/ is an adverb composed of only one root (at least) in the modern Yuwan.
