\chapter{Introduction}\label{chap:1}
\hypertarget{RefHeadingToc395696949}{}
\section{Typological overview}
\hypertarget{RefHeadingToc395696950}{}
Yuwan has six vowels /i, ɨ, u, ə, o, a/ and twenty-two consonants /p, t, k, tˀ, kˀ, b, d, g, c, cˀ, s, h, z, m, n, mˀ, nˀ, w, j, wˀ, jˀ, r/, and its syllable structure is \textsc{cgvv} or \textsc{cgvc} (G: glide slot). Additionally, it has an agglutinative morphology, and its basic word order is \textsc{sv} or \textsc{aov}. S and O are marked by the nominative case \textit{ga} (or \textit{nu}), and O is marked by the accusative case \textit{ba}, although there are some examples where O does not take any case.

\section{Geography}
\hypertarget{RefHeadingToc395696951}{}
Yuwan is spoken in the Yuwan district, in the western district of Amami Ōshima, an island situated just south of mainland Japan. The size of Amami Ōshima is about 710 km\textsuperscript{2}, and it is the biggest island of the Amami Islands, which includes seven other major islands. Amami Ōshima is situated in the northern part of the Ryūkyū archipelago but belongs to the Kagoshima prefecture, while most of the other Ryūkyū islands belong to the Okinawa prefecture. Amami Ryukyuan is a Northern Ryukyuan language. (The map in \figref{fig:key:1} was made in the following web site: \url{http://www.craftmap.box-i.net/japan/line.php}).

  
%%please move the includegraphics inside the {figure} environment
%%\includegraphics[width=\textwidth]{figures/grammarofyuwanfinal9-img001.png}
 

\begin{figure}
\textmd{\figref{fig:key:1}. Japan in the Far East}
\end{figure}

%%[Warning: Draw object ignored]
%%[Warning: Draw object ignored]
%%[Warning: Draw object ignored]
%%[Warning: Draw object ignored]
%%[Warning: Draw object ignored]
%%[Warning: Draw object ignored]
%%[Warning: Draw object ignored]
  
%%please move the includegraphics inside the {figure} environment
%%\includegraphics[width=\textwidth]{figures/grammarofyuwanfinal9-img002.png}
              
%%please move the includegraphics inside the {figure} environment
%%\includegraphics[width=\textwidth]{figures/grammarofyuwanfinal9-img003.png}
 

\begin{figure}
\textmd{\figref{fig:key:2}. Japan}
\end{figure}

\begin{figure}
\textmd{\figref{fig:key:5}. Uken village}
\end{figure}

Yuwan

%%[Warning: Draw object ignored]
%%[Warning: Draw object ignored]
%%[Warning: Draw object ignored]
%%[Warning: Draw object ignored]
  
%%please move the includegraphics inside the {figure} environment
%%\includegraphics[width=\textwidth]{figures/grammarofyuwanfinal9-img004.png}
              
%%please move the includegraphics inside the {figure} environment
%%\includegraphics[width=\textwidth]{figures/grammarofyuwanfinal9-img005.png}
 

\begin{figure}
\textmd{\figref{fig:key:4}. Amami islands}
\end{figure}

\begin{figure}
\textmd{\figref{fig:key:3}. Ryukyu islands}
\end{figure}

Amami Ōshima

The above maps in Figures 2-5 were made by the following free softwares:

a. “白地図 MapMap” (http://www5b.biglobe.ne.jp/t-kamada/\textsc{cb}uilder/mapmap.htm);

b. “白地図 KenMap” (http://www5b.biglobe.ne.jp/t-kamada/\textsc{cb}uilder/kenmap.htm).

\section{Affiliation}
\hypertarget{RefHeadingToc395696952}{}
According to \citet[771--774, 779--783]{Uemura1992}, Ryukyuan is in a sister relationship to Japanese, and Ryukyuan can be divided into two primary subgroups, Nothern group and Southern group. The Nothern group can be divided into Amami and Okinawa. According to \citet[263]{Pellard2009}, the accurate order of branching off of the three language groups, i.e. Amami, Okinawa, and Southern goup (“Sud” under “Ryukyu” in the following figure), is not clear. However, the subgrouping of Amami can be shown as in \figref{fig:key:6}. Yuwan belongs to “Ōshima” in this figure.

  
%%please move the includegraphics inside the {figure} environment
%%\includegraphics[width=\textwidth]{figures/grammarofyuwanfinal9-img006.emf}
 

\begin{figure}
\begin{forest} for tree={forked edge', grow=south}
[Ryukyu
 [?
  [Sud
  [Macro-yaeyama [Yonaguni] [Yaeyama]]
  [Miyako [Tarama] [Miyako\\commun,align=center]]
  ]
  [Okinawa [Sud] [Nord]]
  [Amami [Yoron] [,tikz+={\draw(.parent anchor)--(.child anchor);} [,tikz+={\draw(.parent anchor)--(.child anchor);} [Okinoerabu] [Tokunoshima]] [Oshima]]]
 ]
]
\end{forest}
\caption{\label{fig:key:6}Affiliation of Ryukyuan \citep[following][263]{Pellard2009}}
\end{figure}

\section{Sociolinguistic overview}
\hypertarget{RefHeadingToc395696953}{}\subsection{The number of speakers}
\hypertarget{RefHeadingToc395696954}{}
The population of Yuwan is 521 (valid as of January 1, 2010); however, a fewer number of people can speak the traditional dialect. The inhabitants are typically monolingual Japanese speakers or speak Japanese as a second language. In fact, the varieties of Japanese spoken here have been influenced by the traditional dialects of each location, especially in terms of the intonation and lexicon.

\subsection{Dialects}
\label{bkm:Ref367370857}\hypertarget{RefHeadingToc395696955}{}
In Amami Ōshima, there are many dialcects including Yuwan. There are some researches of linguistic geography about the dialects in Amami Ōshima: especially, \citet{HirayamaEtAl1966} and \citet{Shibata1984} among others. The detailed comparison among the lexemes in the dialects in Amami Ōshima is beyond the scope of this grammar. I present only one characteristic regarded as a major difference between Yuwan and the other dialects in Amami Ōshima. The phonetic sequence [ɾi] in the other dialects (and some [ɾ] in Koniya dialect) correspond to [i] in Yuwan (but not vice versa) (\citealt{HirayamaEtAl1966}: 71). \tabref{tab:key:1} illustrates this point with data from Yuwan, Suko, spoken in a village located about 800 meters from Yuwan, Ura, a Northern Amami dialect spoken in a village located about 32 km from Yuwan (the Ura data are provided courtesy of Dr. \name{Hiromi}{Shigeno} (p.c., 2009)), and Koniya, a Southern Amami dialect spoken in a village located about 15km from Yuwan (the Koniya data is taken from \citealt{HirayamaEtAl1966}: 70, which uses a phonetic symbol [r], but this phone is explained as “tap” (ibid.: 33). Thus, I transcribed it as [ɾ] in this example).

\begin{table}
% \hypertarget{RefHeadingToc395697266} 
\caption{Dialectal variation in Amami\label{tab:key:1}}
\begin{tabular}{lllll} 
\lsptoprule
       & Yuwan & Suko & Ura & Koniya\\\midrule
‘bird’ & [tui] & [tuɾi] &  [tuɾi] &  [tuɾ]\\
\lspbottomrule
\end{tabular}
\end{table}

Yuwan is spoken in a small district, so there do not appear to be regional variations; however, there seems to be a generational variation concerning honorific (and polite) expressions. Yuwan has an auxiliary verb \textit{moor-} (\textsc{hon}), which expresses the speaker’s respect for the subject of the clause (see \chapref{chap:3}). For example, in the case of /a-i/ \textit{ar-i} (exist-\textsc{npst}) ‘exist’ vs. /a-tɨ moo-ju-i/ \textit{ar-tɨ} \textit{moor-jur-i} (exist-\textsc{seq} HON-\textsc{umrk}- N\textsc{pst}) ‘would exist,’ the former is formed with the lexical verbal root \textit{ar-} ‘exist’ and it does not show the speaker’s respect to the subject, but the latter is created with both of the lexical verbal root \textit{ar-} ‘exist’ and the auxiliary verbal root \textit{moor-}, which expresses the speaker’s respect to the subject (see also §\ref{bkm:Ref367132477}). This honorific strategy is frequently used by older people, but not by younger people. Instead, younger people use a verbal affix \textit{{}-jawur} or \textit{{}-joor} to express respect for the hearer (not for the subject of the clause), e.g., /a-jawu-i/ \textit{ar-jawur-i} (exist-\textsc{pol}-NP\textsc{st}) ‘exist.’ Older speakers of Yuwan, however, are not likely to use this politeness affix.

Furthermore, there is another generational variation concerning morphophonological alternation. Yuwan has a topic marker \textit{ja}, and older speakers use the alternative form /na/ if its preceding word ends with a nasal consonant such as \textit{san} ‘three.’ However, younger speakers use /ja/ as the topic marker in any morphophonological environment. This variation is illustrated in the following example. Example (1-1) shows that the older speaker uses /na/ (\textsc{top}) after \textit{san} ‘three’ but the younger speaker does not.

\ea\label{ex:key:1}\relax[Context: The following examples are taken from a conversation between \textsc{ms} and \textsc{tm}, who are talking about the old educational system in Japan.]\\
\textsc{ms}: \glll {\textbar}roku, roku, san{\textbar}.ja arannən.\\
 roku roku san=ja ar-an-nən\\
 six six three=\textsc{top} \textsc{cop}-\textsc{neg}-\textsc{seq}\\
\glt\hspaceThis{\textsc{ms}:} ‘(It) is not (divided into) six, six, three (years like now).’\\
\textsc{tm}: \glll {\textbar}roku, roku, san{\textbar}na arannən.\\
 roku roku san=ja ar-an-nən\\
 six six three=\textsc{top} \textsc{cop}-\textsc{neg}-\textsc{seq}\\
\glt\hspaceThis{\textsc{tm}:} ‘(It) is not (divided into) six, six, three (years like now).’\\\hspaceThis{TM:} [Co: 120415\_00.txt]
\z

\subsection{Viability}
\hypertarget{RefHeadingToc395696956}{}
The number of speakers of traditional Yuwan is decreasing. Typically, people over seventy years old can speak traditional Yuwan, and people who are fifty to sixty years old can speak a more or less traditional Yuwan, but people under fifty years old are only passively bilingual. The younger generations cannot speak or understand the traditional dialect; however, some of them use a few traditional expressions such as \textit{wan} ‘I’ or \textit{ccjɨ} (\textsc{qt}).

\subsection{Previous work}
\hypertarget{RefHeadingToc395696957}{}
In addition to the present study, there are two previous works on Yuwan: \citet{HirayamaEtAl1966} and \citet{UchimaEtAl1976}. The former compared the accent patterns and the lexicons among a number of Ryukyuan dialects, and only a small amount of information was presented about Yuwan. In fact, this study contained only thirty or so nominal lexical entries with their prosodic information. The latter, \citet{UchimaEtAl1976}, included a list of several hundred lexical items and several verb paradigms. However, the phonology of Yuwan has not yet been fully investigated, and its morphology has been only partially researched. The syntax of Yuwan has not been investigated at all, with the exception of \citet{Niinaga2008}, which describes the case system of Yuwan, and \citet{Niinaga2010}, which sketches a grammar of Yuwan.

A broader review of the literature brings to light a number of articles about Amami, of which Yuwan is a dialect. Here, only books or special issues of journals are mentioned. A brief comparison of several dialects of Amami can be found in \citet{HirayamaEtAl1966}. Lexical and phonological differences between some dialects in Amami Ōshima are discussed in \citet{Shibata1984}. Naze, which is spoken in the largest city in Amami Ōshima, is examined by \citet{Terashi1985}, and \citet{UemuraSuyama1997} describe its phonology, verbal morphology, and case markers. \citet{Shigeno2010} provides a sketch grammar of Ura, spoken in the northern part of Amami Ōshima. Yamatohama (or Yamatoma in the local pronunciation), spoken in the western part of Amami Ōshima, is the subject of study in Nagata et al. (1977–1980), which includes a detailed study of the lexicon but also some information on its grammar. \citet{UchimaEtAl1976} also describe the verbal morphology of Koniya, spoken in the southern part of Amami Ōshima. \citet{NakamotoUchima1978} provides a description of the lexicon and verbal morphology of Shitooke, spoken in the northern part of Kikai. \citet{ShirataEtAl2011} is a sketch grammar and a text of Kamikatetsu, spoken in the southern region of Kikai. \citet{OkamuraEtAl2009} describe the verbal morphology and list two thousand sentences in Asama, spoken in the northern region of Tokunoshima. \citet{KikuTakahashi2005} describe the lexicon of Yoron, and \citet{Yamada1981} focuses on the use of nominals in Yoron.

\section{Database for this study}
\label{bkm:Ref347173399}\hypertarget{RefHeadingToc395696958}{}
This grammar is based on a corpus of twelve texts (total duration is 4 hours) in addition to other elicited information that complements these texts. The data set was collected during the author’s field work in the region, which began in October 2006. The total length of time for the field work was 595 days. The details of the texts are shown in \tabref{tab:key:2}, and brief information about the speakers is shown in \tabref{tab:key:3}.

\begin{table}
\caption{\label{tab:key:2}Data of texts}
\resizebox{\textwidth}{!}{\begin{tabular}{l>{\ttfamily}lS[table-format=2.1] c c}
\lsptoprule
{Genre} & {\normalfont File \textsc{id}} & {Duration} & {Main} & {Sub-speaker\footnote{(or hearer)}}\\
        &                       & {(min.)}   & {speaker}\\\midrule
{P(ear) F(ilm)} & 090222\_00.txt & 3.5 & {\textsc{tm}} & {(\textsc{mm})}\\
                & 090225\_00.txt & 2.5 & {\textsc{tm}} & {(\textsc{mm})}\\
                & 090305\_01.txt & 3 & {\textsc{tm}} & {(\textsc{sm})}\\
                & 090827\_02.txt & 4 & {\textsc{tm}} & {(\textsc{my})}\\
{Fo(lktale)}    & 090307\_00.txt & 4 & {\textsc{tm}} & {(\textsc{mm})}\\
{Co(nversations)} & 101020\_01.txt & 1 & {\textsc{tm}} & {\textsc{my}}\\
                & 101023\_01.txt & 15 & {\textsc{tm}} & {\textsc{my}}\\
                & 110328\_00.txt & 28 & {\textsc{tm}} & {US, \textsc{my}, (\textsc{mm})}\\
                & 111113\_01.txt & 28 & {\textsc{tm}} & {\textsc{ms}}\\
                & 111113\_02.txt & 22 & {\textsc{tm}} & {\textsc{ms}}\\
                & 120415\_00.txt & 63 & {\textsc{tm}} & {\textsc{ms}}\\
                & 120415\_01.wav & 66 & {\textsc{tm}} & {\textsc{ms}}\\
{El(icited)}   & N/A & {N/A} & {\textsc{tm}, \textsc{mt}} & {(the present author)}\\
\lspbottomrule
\end{tabular}}
\end{table}

The Pear Film is a silent six-minute film made at the University of California at Berkeley in 1975. It is helpful to collect the monologue data from the speaker.\footnote{A brief explanation of the Pear Film can be seen at \url{http://www.linguistics.ucsb.edu/faculty/chafe/pearfilm.htm}.} About the data classified in Pear Film, the speaker told the story to the hearer remembering the film (as soon as the speaker had watched it). About the folktale, the speakr heared it from her acquaintance who had told the story in a speech contest of the Amami dialects.

\begin{table}
\caption{\label{tab:key:3}Information about the Yuwan speakers}
\resizebox{\textwidth}{!}{\begin{tabular}{lllcc}
\lsptoprule
 \textsc{id} & {First (Second) name} & {Family name} & {age in 2012} & {period of absence}\\
          &                         &              &               &  {from Yuwan}\\\midrule
\textsc{tm} & Sachi (Tsuneko) & Motoda & {89} & {14--21 years old}\\
\textsc{us} & Mine (Umine) & Shinozaki & {95} & {15--52}\\
\textsc{my} & Sumie (Mutsu) & Yamaki & {88} & {28--49}\\
\textsc{mt} & Mitsuko & Toshioka & {78} & {24--26}\\
\textsc{mm} & Masako & Motoda & {73} & {15--38}\\
\textsc{nm} & Nobuari & Motoda & {62} & {20--29}\\
\textsc{sm} & Sawako & Motoda & {61} & {15--26}\\
\textsc{ms} & Mioya & Sunao & {59} & {16--53}\\
\lspbottomrule
\end{tabular}}
\end{table}


The recordings were transcribed by the present author with the help of some Yuwan speakers. In particular, Masako Motoda (\textsc{mm}), Nobuari Motoda (\textsc{nm}), and Mioya Sunao (\textsc{ms}) generously donated their time in order to help the present author’s transcription. During the recordings, I tried, when possible, to not be present in order to avoid promoting the speaker’s use of Standard Japanese, which was a lingua franca I shared with the Yuwan speakers. As for the elicitation data, the expressions in Yuwan that were produced by the present author and not by the speaker are not regarded as grammatical even if the speaker’s judgment was “grammatical.” In other words, I regard the elicitation data as grammatical only when the speaker pronounced the expression by herself.

Many of the examples in this grammar do not end at sentence-final positions -- i.e., they end with commas, not with periods. The verbs in Yuwan are rich with affixes that can mark subordinate clauses (see “the converb” in §\ref{bkm:Ref306802119}). This language is a “broadly characterizable as ‘chaining’” \citep[399]{Longacre2007} as well as Japanese. Just as the languages regarded as ‘chaining’ type in \citet{Longacre2007}, the finite verb occurs after ‘a sizeable strech of discourse which can on occasion be as long as two or three pages’ (ibid.: 400). Therefore, I have to omit the irrelevant parts from the clausal sequences.

Most of the data on the grammar of Yuwan comes from Sachi Motoda (\textsc{tm}), and the subsidiary information is taken from the other participants. All of these participants and their parents were born in Yuwan.

All of the examples in this grammar represent actual utterances of Yuwan speakers, and the sources of these utterances are clarified as much as possible. For example, the code “Co: 120415\_00.txt” means the example was transcribed in the text file 120415\_00.txt (the first six numbers indicate the recoding date, i.e. April 15, 2012), and its genre is “Co(nversation).” In the case of the elicited data, only the date of research is indicated after the abbreviation, e.g., “El: 120415.” In addition, the speaker \textsc{id} is shown at the beginning of each transcription to represent who produced the utterance. For example, “\textsc{tm}: cjaa.” means the speaker TM said /cjaa/ (see also “Transcription methods” in the beginning of this book).

\section{Organization of this grammar}
\hypertarget{RefHeadingToc395696959}{}
In \chapref{chap:2}, the phonology of Yuwan is explained in detail. A brief explanation about the grammatical relations in Yuwan is given in \chapref{chap:3}. The descriptive preliminaries are presented in \chapref{chap:4} through a discussion of the basic construction and constituents of sentences of Yuwan. In \chapref{chap:5}, categories that can cross over several word classes, e.g., demonstratives or personal pronominals, are discussed. \chapref{chap:6} deals with nominal phrases, and \chapref{chap:7} investigates the detail of nominals. Verbal morphology is explained in detail in \chapref{chap:8}. \chapref{chap:9} explains three types of predicate phrases, i.e. verbal predicate, adjectival predicate, and nominal predicate. The details of particles are examined in \chapref{chap:10}. Finally, the inter-clausal phenomena is presented in \chapref{chap:11}. The appendix shows the detailed lists of morphophonological alternations of verbs.
