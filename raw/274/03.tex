\chapter{Externalism}
As noted above, the externalist approach is currently somewhat of an orthodoxy in the philosophy of mind and in the philosophy of language. Moreover, this approach is not limited to philosophy, for much work in linguistics (for example, in formal semantics and formal pragmatics) takes the truth-conditional approach to meaning. There are a number of positions that go by the name of externalism, but I will focus here on Twin Earth externalism. That is, the accounts that rest on a Twin Earth-style thought experiment, the most famous version of which is Putnam (1975). This is the standard understanding of externalism. Indeed, as \citet{Farkas2003} notes, some philosophers use Twin Earth thought experiments as part of the very definition of what externalism is (see also \citealt{McLaughlinTye1998}). Moreover, standard linguistics textbooks in formal semantics also approvingly discuss these thought experiments (see, for example, \citealt[7ff.]{Portner2005}). The Twin Earth thought experiment claims to show that two subjects can have identical internal mental states but that the content of these states can be different due to particular variations in the environment. In other words, the claim is that the content of mental states can vary with variations in the environmental or socio-environmental conditions of two subjects while their respective internal mental states remain identical. The conclusion of externalist semantics, then, is that meanings are individuated by reference to environmental features or social contexts, and that therefore in order for a person’s utterance to have a particular meaning, it must be related to the environment in the right way.

Note that even though much of the discussion in the externalist literature is couched in terms of content, it is clear that the conclusions in regard to content are meant to apply to linguistic meaning as well. Burge, for example, is explicit about this: “The arguments for anti-individualistic individuation of mental kinds can be extended in relatively obvious ways to show that much of semantics is not purely individualistic [i.e., it is externalist]” \citep[279]{Burge1989}. This can be generalised, so that externalist arguments couched in terms of mental content can be applied to linguistic meaning, as indeed they often are. Content is perhaps a broader term that can apply to non-linguistic mental states such as visual perceptions, or perhaps content is entirely distinct from linguistic meaning. But whatever content turns out to be, externalists argue that its nature will have direct bearing on the nature of meaning. In addition, the fact that Burge says that his philosophy of language has a direct bearing on much of semantics shows that he construes his externalist theoretical aims to overlap with those of linguistics and semantics in particular.

There are two traditional assumptions that lurk in the background of the debate in the externalist literature. These are (i) the claim that the meanings of words are fixed by the psychological states of those who use them, and (ii) the claim that the meanings of words determine their extension or reference. Externalism is supposed to entail that (i) and (ii) are incompatible. Thus, the Twin Earth thought experiments purport to show that since meaning determines reference – and so terms with the same meaning will have the same reference – the psychological states of the twins cannot determine their meaning because the reference of their utterance of, for example, \textit{water} is different. In other words, if one agrees that a difference in meaning implies a difference in reference, then one cannot hold that meaning is determined by psychological states; this is because, \textit{ex hypothesi}, the psychological states of the twins are identical but the reference is different. Putnam argued that it is possible for two speakers to be in exactly the same psychological state, even though the extension of a term in one speaker is different to the extension of the \textit{same term} in the second speaker. If this is correct, he argued, one must give up one of the traditional assumptions (i.e., give up either (i) or (ii)).

I want to argue, however, that it is possible to hold both (i) and (ii), but not for the reasons one might expect. That is, (i) is a psychological explanatory project in regard to meaning and mind (as illustrated by internalist semantics in the previous chapter), and (ii) is a hermeneutic explanatory project. As a result, there is no tension between holding that a person’s psychological states fix their meanings and holding that meaning determines reference. In other words, (i) is part of the explanation of the psychological mechanisms in virtue of which meaning is made possible, whereas (ii) plays an interpretive function for the theorist by linking internal psychological states and the world.

As noted above, \citet{Jackendoff1991} makes a similar point when he compares the tension that exists in regard to fundamental questions for a theory of mind. \citet{Smith1992} draws a parallel distinction between what he calls the interpretive, descriptive, and explanatory stances to a theory of meaning. At one end of the spectrum “we have the language-dissolving view of interpretation, which at the limit slides into hermeneutics and literary theory where there are no standards of correctness at all. (‘Anyone can do the philosophy of language’)”. At the other end of the spectrum there is “an extreme philosophy of language” which “gives way to empirical research. (‘Best left to the scientists’)”. Smith advocates a theory of meaning that “must come somewhere in between”, and he takes Chomsky’s framework as setting “a constraint on any satisfactory solution [to the problem of meaning]” \citep[138--139]{Smith1992}. Smith’s position is more conciliatory and closer to the position I argue for here, for it leaves some room for the philosophical and hermeneutic approach to semantics. In contrast, Jackendoff’s claim that the philosophical and psychological approaches to a theory of mind are disparate in the sense of Kuhn is too strong, for it neglects the interest and value that lies in the hermeneutic explanatory project. Nevertheless, there \textit{is} a systematic difference between the questions that externalists attempt to answer and the questions that internalists attempt to answer. Thus, if the externalist mode of explanation is a hermeneutic one, then despite the claims of many externalists it will be unable to provide an explanatory framework for a science of semantics.

If we assume that meaning is externalist as defined by Lewis and others and that we cannot individuate the meanings of utterances without reference to mind-external factors, what follows in regard to the science of meaning? I agree that one can only discern what a person’s utterance refers to by consulting the external environment, and that the referents or extensions or denotations of thoughts cannot be exclusively determined by mind-internal matters – the question, say, of whether a referent is a sheep or a bush that looks like a sheep cannot be determined without consulting the external environment. But these are questions of meaning ascription or individuation. I argue in what follows that scientific explanations do not attempt to answer such questions, at least not in the way that is claimed in the externalist literature. That is, I argue that the externalist claim that there exists a deep-rooted link between inner psychological states and their extension is not problematic – indeed, it is essential and potentially fruitful – for the hermeneutic explanatory project of thought contents. However, in regard to the psychological explanatory project of the latter, questions of attribution or individuation do not play a key role.


\section{The subject matter of externalism}
Leading externalists explicitly and repeatedly state that their theories are part of the scientific project. What are we to make of such claims? Do criteria for ascription and description of meaning belong in a scientific theory? We should question the assumption that externalism is directly relevant to a scientific theory of meaning that attempts to unearth the mechanisms in virtue of which meaning works in the mind. Putnam argues that a better philosophy and a better science of language is externalist. But what does Putnam understand to be the details of his science of language? One of his other thought experiments supposes that “[i]f Twin Earth organisms have a silicon chemistry, for example, then their ‘tigers’ aren’t really tigers, even if they look like tigers, although the linguistic habits of the lay Twin Earth speaker exactly correspond to those of Earth speakers” \citep[167]{Putnam1975}. This stems from Putnam’s (and the externalists’) belief that “\textit{extension is tied to the notion of truth}” and that the “extension of a term is just what the term is true of” \citep[154, emphasis in original]{Putnam1975}. Thus, according to this reasoning, we can only determine the meaning of a person’s utterance – whether the meaning of the person’s utterance \textit{really} \textit{is} \textit{tiger} – by consulting the external environment and checking whether the utterance is true of that environment. A person on Earth and his doppelgänger on Twin Earth can have the same internal psychological states or concepts (tigers are striped, they’re quadrupeds, they have paws, they have whiskers, etc.) and yet mean different things when they utter \textit{tiger} because the tigers in their environment are different – one is carbon-based and the other is silicon-based. Putnam concludes that the same utterance spoken by him and by his doppelgänger can have different meanings, “but this will not be an assertion about our psychological states” \citep[165]{Putnam1975}. 

This is a curious statement. If we take a science of language to encompass externalist relations in the way that Putnam urges, and if we agree that language processing is somehow instantiated in the brain, what are we to make of the claim that externalist theories of meaning do not make assertions about psychological states? Fodor similarly argues that “[i]t is, to put the point starkly, the heart of externalism that \textit{semantics isn’t part of psychology}. The content of your thoughts (/utterances), unlike for example, the syntax of your thoughts (/utterances), does not supervene on your mental processes” \citep[38, emphasis in original]{Fodor1994}. It is far from unusual to find such statements claiming that semantics (or philosophy of language, for that matter) is not about underlying psychological states. \citet{Soames1984}, for example, argues that “linguistic theories are conceptually distinct and empirically divergent from psychological theories of language acquisition and linguistic competence”. Soames denies that linguistic theories are theories of “[c]omplex, unconscious, computational states and processes [that] underlie language acquisition and mastery” \citep[155]{Soames1984}. There is much more to say on the matter, but for the moment let us note that such claims are in direct opposition to the way in which internalist semantics and biolinguistics practice their research programs. Internalist semantics, as we saw above, explicitly takes its theories to be about the underlying computational mechanisms in virtue of which language production and comprehension are made possible. Soames, however, claims that generative linguistics and psychology “are concerned with different domains, make different claims, and are established by different means” and thus “linguistics does not yield computational and representational theories in cognitive psychology” \citep[157]{Soames1984}. Soames insists on separating the conceptual and empirical foundations of generative linguistics from those of psychology. But contrary to his claims, his construal of linguistics does not apply to generative linguistics and in fact he has misunderstood its aims and methodology. 

At the risk of gratuitously discussing yet another misconstrual of internalism, I would like to briefly discuss Soames’s misconstrual. It is important to understand how widespread this confusion is. People like Fodor, Soames, and others discussed here, are leading figures who know the literature of generative linguistics quite well. It is thus of great interest to unearth the source of their misconstrual of internalism, for it both sheds light on the underlying (and I would argue mostly implicit) theoretical assumptions of hermeneutic projects such as externalism and helps to clarify the research programme of internalism. The claim of generative linguistics (and, later, biolinguistics) to be part of psychology (and ultimately biology) has been debated for decades with little consensus. This is also true in regard to the internalist semantics claim that meanings are internal psychological phenomena. Unpacking the reason for this seemingly recalcitrant debate shows the Kuhnian split between the two camps is due to the often unnoticed fact that one is a hermeneutic project and the other is a scientific project.
	
Soames proposes three “Leading Questions” that linguistics should attempt to answer; they are: 
\begin{quote}
(i) In what ways are English and Italian/the Romance languages/all Indo-European languages/all natural languages alike and in what ways do they differ?

(ii) What (if anything) distinguishes natural languages from some set of artificial languages (such as finite state languages) or from animal communication systems?

(iii) In what ways has a particular language (a description (or list) of one or more natural languages) changed and in what ways has it remained the same? \citep[158]{Soames1984}
\end{quote}
Later in the article he remarks that “by \textit{linguistics} I mean the discipline defined by the Leading Questions discussed earlier and practiced today by generative grammarians” \citep[178, fn. 24, emphasis in original]{Soames1984}. However, whatever virtue and interest the answers to the Leading Questions may yield, it is not at the core of what generative linguists do. The separation that Soames wants to draw between linguistics and psychology  is that between a linguistics that “aims at providing theories of natural languages” and  that of a “cognitive psychology [that] aims at providing theories of natural language users” \citep[157]{Soames1984}. But this distinction is not applicable to generative linguistics, for generative linguistics explicitly deals with providing a theory of what is in the head of language users in virtue of which their language production and comprehension are made possible.

Soames wishes, then, to distinguish between “non-mentalistic” linguistics and “non-linguistic” psychology; he remarks that “[s]ince the job of a linguistic theory is to specify the similarities and differences among (possible) languages, such a theory must be sensitive to truth conditions (or elements that determine them)” \citep[163]{Soames1984}. The appeal to truth conditions (a hallmark of externalism) is the real reason for Soames’s insistence that linguistics is not psychology: “languages may differ not only with respect to syntactic and phonological properties, but also with respect to semantic properties involving truth conditions”, and thus a linguistic theory “that failed to account for truth conditions would miss these differences” \citep[162]{Soames1984}. In other words, if “linguists’ grammars were simply psychological theories, then claims about truth conditions would themselves be psychological”, but since “these claims are not (purely) psychological in nature, it follows that grammars are not wholly psychological in nature and that linguistics is not merely a branch of psychology” \citep[163]{Soames1984}. The argument is that since semantics (as construed by Soames and other externalists) makes use of “extra-psychological notions” (that is, “to give the truth conditions of sentences is to specify the non-linguistic conditions that would make them true”), “it follows that linguistic semantics is \textit{conceptually distinct} from psychological models of semantic competence” \citep[163, emphasis in original]{Soames1984}. More recently, \citet{Soames2009} has been somewhat more charitable to psychological models of semantic competence, but he still argues in favour of a “nonpsychologistic perspective” according to which “[s]entences, and other expressions, have grammatical structures and representational contents that can be studied in abstraction from questions about how they initially came to have those structures and contents, what psychological states and processes are responsible for their retaining them, or how speakers come to know whatever they do know about them” \citep[1--2]{Soames2009}.

One is of course free to choose the perspective within which to study meaning that they feel will yield the most fecund explanation. One can agree with \citet[183]{Soames2009} that the “job of semantics is to specify the principles by which sentences represent the world”. That is, since it “is impossible to represent the world as being a certain way without implicitly imposing conditions that must be satisfied if the world is to conform to the representation”, “whatever else a semantic theory must do, it must at least characterize truth conditions”. Soames sees this construal of the job of semantics to be part of the “basics” of the field, implying that if one were not to do semantics in this way then one would not be doing semantics. But this at best reduces to the \textit{a priori} assertion that “real semantics” \citep{Lewis1970} is concerned with truth conditions. It of course follows from this assertion that linguistics isn’t part of psychology, for in order to specify truth conditions one must make use of “extra-psychological notions”. But this assertion begs the question: the issue in generative linguistics (which Soames claims is characterised by the Leading Questions) is what kind of semantic theory fits within a scientific framework of the study of the mind. One cannot legislate what “real semantics” is and then proceed to argue that therefore semantics is not part of psychology. Moreover, by asserting that a linguistic theory must be sensitive to truth conditions, Soames has indeed distinguished linguistics from psychology, but this characterisation of linguistics, \textit{pace} \citet[178, fn. 24]{Soames1984}, is not that of generative linguistics.

It could be objected here that truth-theoretic semanticists do not see their semantic theory as contributing to a scientific theory of the mind that seeks to unearth the underlying mechanisms in virtue of which meaning is made possible, and thus it would be moot to compare the two. But there are many leading externalists and formal semanticists who explicitly claim that their project is a scientific one with similar aims to those of generative linguistics. It is thus both warranted and illuminating to look at the externalist research programme from the perspective of scientific explanatory strategies and to ask whether it is a promising avenue in regard to constructing an explanatory scientific theory.

That said, however, there are plenty of philosophers of language who wish to separate an externalist semantics from a semantics construed psychologically. Dummett, for example, argues that philosophy “is not concerned with what \textit{enables} us to speak as we do, but what it is for utterances to have the meanings that they have” \citep[187--188, emphasis in original]{Dummett1994}. That is,
\begin{quote}
[…] a theory of meaning is required to make the workings of language open to our view. To know a language is to be able to employ a language; hence, once we have an explicit account of that in which the knowledge of a language consists, we thereby have an account of the workings of that language; and nothing short of that can give us what we are after. \citep[4]{Dummett1993}
\end{quote}
Dummett insists that “once we can say what it is for someone to know a language, in the sense of knowing the meanings of all expressions of the language, then we have essentially solved every problem that can arise concerning meaning” \citep[4]{Dummett1993}. In other words, “[a]ny theory of meaning which was not, or did not immediately yield, a theory of understanding, would not satisfy the purpose for which, philosophically, we require a theory of meaning” \citep[4]{Dummett1993}. The claim here is that semantics is not part of psychology, nor is it part of a linguistics construed as part of psychology (as biolinguistics and internalist semantics take themselves to be). Indeed, \citet[187]{Dummett1994} complains that Chomsky’s theory of meaning “is really a theory of something very complicated that goes on in the brain” and “that is a completely unphilosophical way of looking at the matter". 

So what is the subject matter of an externalist semantics if not psychology or the mental states of language users? Externalist semanticists see their task as providing an account of the relation between linguistic expressions and things in the world. This relation can be fleshed out in several ways, but what underlies them all is the concern with the conditions or rules of ascription: when is one justified in ascribing a particular content to a particular utterance, and what is the correct content that should be ascribed. Given such aims, then, what does the research project take itself to be explaining? Once this question is answered one can address the further question of whether externalist explanations are scientific as some leading externalists claim to be the case. In other words, does externalism employ a form of understanding that is appropriate to a cognitive psychological explanation? Or, in contrast, is the way in which externalism understands the mind orthogonal to that of cognitive psychology? This highlights an important distinction, one between different research projects that are often conflated. That is, the distinction is that between, on the one hand, the notion of understanding that refers to interpreting a sentence as a speaker of the language (\textit{verstehen}) and, on the other hand, the notion of understanding that refers to explaining as in science (\textit{erklären}). As \citet{Slezak2004} shows, the question of semantics or the content of mental representations is often confused between whether representations are \textit{intelligible} to the \textit{theorist} and whether they are explainable by the \textit{theory} (see also \citealt{Slezak1990,Slezak2018}). 

\citet{Peacocke1993}, which details how explanation by externalist states is supposed to work, is a case in point of such a conflation. It is, he says, “partially constitutive” of the identity of any externalist state that “it can explain, or be explained by, relational properties of external objects or events” \citep[206]{Peacocke1993}. Relational properties, such as the “highly relational property of saying something which has a certain meaning”, are argued to be explained psychologically \citep[204]{Peacocke1993}. Psychology can be done in an externalist fashion on Peacocke’s conception (and thus aid in the study of semantics), for a psychological explanation of an event explains a particular set of its relational properties. Peackocke thus explicitly merges the externalist approach with the psychological approach, taking the externalist framework as a means to achieve a psychological explanation of the mind.

But what is the nature of these relational properties? These are properties that exist only in virtue of relations to other objects; the properties depend on the relations and there is no sense in which they could exist independently of them. \citet[3]{Nuccetelli2003} gives the example of the property of being west of Central Park, where “whether one has it depends on how one is geographically situated with respect to Central Park”. She then argues for the externalist claim that such relations are analogous to thoughts having a certain content. That is, “[g]iven externalism, having either the belief that water is wet or other propositional attitudes with certain contents would be in some sense analogous to being west of Central Park, simply because the content \textit{type} of some such attitudes would supervene on the relations of those who entertain them with their physical and/or social environments” \citep[3, emphasis in original]{Nuccetelli2003}. This analogy is of course only illustrative, but it outlines the foundations of externalism in that it is claimed that even though thoughts are inside a person’s head, their content supervenes on external factors in the environment of the person who has those thoughts. Thus, as \citet{Ben-Menahem2005} notes in regard to one of Putnam’s examples, “to speak of coffee tables it does not suffice for us merely to have the concept of a coffee table, but we must be in contact with \textit{actual} coffee tables” \citep[10, emphasis in original]{Ben-Menahem2005}.

Another analogy is that of sunburn: \citet{Davidson1987} remarks (in the context of a discussion of meanings and their relations to objects outside of the head) that “[m]y sunburned skin may be indistinguishable from someone else’s skin that achieved its burn by other means (our skins may be identical […]); yet one of us is really sunburned and the other not” \citep[451--452]{Davidson1987}. That is, in order for one to have the property of \textit{being sunburned} (to \textit{really} be sunburned), one’s skin must have had the proper relationship to the outside environment, namely, to the sun. Such analogies illustrate the kind of reasoning that lies behind the externalists’ claim that a particular thought or utterance cannot have a particular meaning unless it has had the proper relation to the outside environment. In other words, according to externalism, content or meaning (or their individuation), in this case the meaning of \textit{sunburn}, is in some way \textit{essentially} tied to the environment (see \citealt{Egan1999} for discussion). Thus, as \citet[9--10]{McGinn1989} remarks, “mind and world are not, according to externalism, metaphysically independent categories, sliding smoothly past each other. To regard them as so is to commit oneself to an ‘untenable dualism’, to marking a metaphysical boundary that does not exist”.

Talk of such relations is systematic and their justification and validity is rarely explicitly defended. \citet[466]{Burge2003}, for example, remarks that “[t]aking account of language–world relations is part of the way semantics is actually practiced” and thus there is “no reason to think that there is anything scientifically wrong or fruitless in studying language–world relations, or with taking them to be part of the formal structures elaborated in semantical theory”. This has led to the situation in which relational properties are read into research programs such as internalism that explicitly deny them. \citet[556]{Higginbotham1991}, for example, argues that Chomsky’s notion of competence involves “an epistemic relation between a person and the principles that determine her language”. He argues against a view he calls representationalism that generative linguists supposedly hold according to which “having a language just amounts to having a system of mental representations” and “that one stands in epistemic relations to the principles of one’s own language” \citep[557]{Higginbotham1991}. Higginbotham complains that according to this doctrine that he pings on generative linguistics “there is only a pedantic distinction between representation and represented” \citep[557]{Higginbotham1991}. That is, “it conflates questions about what is apprehended, language, with questions about the means of apprehension”. Analogously,  Higginbotham remarks, “[n]o one confuses the mental representation of a tree with a tree” and so “[w]hy should it be so common among linguists to write as if intending to confuse the mental representation of a sentence with a sentence?” \citep[555]{Higginbotham1991}. In contrast to the representationalism that Higginbotham claims conflates the two, he wishes to retain the “philosophical distinction between language and its representation” \citep[558]{Higginbotham1991}. The latter distinction encapsulates the E-language conception of language. 

There are two questions that, according to Higginbotham, generative linguistics investigates: (i) what is the nature of language? and (ii) what is the relation between speakers of languages and the languages that they speak? These questions arise, he says, “if we are interested in a systematic metaphysical view of the conception of linguistics as a chapter of the cognitive sciences”: “in short, they are philosophical problems”. Referring to generative linguistics, Higginbotham claims that a “popular but confused answer” to the aforementioned questions “is that languages are systems of mental representation, and that the growth of competence is the growth of such systems under appropriate environmental contingencies” \citep[555]{Higginbotham1991}. This answer is confused, according to Higginbotham, because of the conflation between the mental representation of language and language itself that rejects the philosophical distinction between language and its representation.

Despite persistent efforts by \citet{Chomsky1975,Chomsky1995,Chomsky2000,Chomsky2016} and others to clarify their position, misunderstandings of this sort persist. \citet{Devitt2003,Devitt2006,Devitt2006a} has for years maintained a version of the \textit{linguistics is not psychology} argument, arguing forcefully for an epistemological reading of generative linguistics. He claims that “there is a natural interpretation which takes Chomsky pretty much at his word” in which the answer to the  question of what constitutes knowledge of language “urges that competent speakers of a language have propositional knowledge of its rules” and that this “knowledge underlies the speakers’ intuitive judgements about the syntax of expressions” \citep[107--108]{Devitt2003}. This reading of Chomsky, which Devitt curiously takes to be the natural interpretation, leads to the conclusion that “there is something theoretically interesting for a grammar to be true about other than the internal reality of speakers” and thus the “grammar might be true about a symbolic system, a \textit{linguistic} reality”. In other words, Devitt claims that “we can take the grammar realistically without taking it to be true of \textit{psychological} reality” and that “given the weight of evidence adduced for a grammar, it is plausible that it \textit{is} (more or less) true of linguistic reality” \citep[131, emphasis in original]{Devitt2003}.

The claim that a grammar can be “true of” something other than the internal structure of the mind originates with \citet{Stich1972}, but since then there has been no clear elucidation of the implication that, as Devitt puts it, evidence from psychology, psycholinguistics, and the like, “bears on the grammar even without the assumption [that the grammar is psychologically real]” \citep[128, fn. 28]{Devitt2003}. What does it mean to say that the grammar is \textit{linguistically} real but is not in the head? Devitt claims that the grammar is true of a symbolic system, “a linguistic reality made up of the spoken, written, etc., symbols that speakers produce” \citep[483, fn. 5]{Devitt2006a}; but what is the nature of this system? Devitt takes a grammar to be about a non-psychological realm of expressions, a realm where physical entities form representational systems that are somehow distinct from the creatures that use them. Moreover, he claims that “the truth of a grammar for a language leaves the question of the psychological reality of the language open” \citep[136]{Devitt2003}. He argues that we should begin with studying the grammar’s linguistic reality and that only “in the end we will need to study the psychological [reality] in order to explain the linguistic [reality]. But in the beginning we do not” \citep[135]{Devitt2003}. Such an approach is perhaps correct in regard to well-formed formulae (wffs) in formal logic, where we can study, to use Devitt’s terminology, the structure rules (the rules governing outputs of a person’s competence) without consideration, at least initially, of the processing rules (the rules governing the psychological production of the aforementioned outputs). Devitt correctly argues that these two rules are very different in regard to formal logic (the structure rules of formal logic are very different to the processing rules of the mind or a computer that govern these outputs), but \textit{natural language} does not fit this mould. 

Formal languages are invented, natural languages are not. We stipulate the properties of formal languages to suit particular purposes in logic and mathematics, but in natural languages we do not stipulate but rather \textit{discover} their properties. \citet{Chomsky2002} remarks that there is no right answer to questions such as \textit{What are the true rules of formation for well-formed formulas of arithmetic?} or \textit{What are the axioms of arithmetic?} because, at least in principle, any set of axioms can generate the theorems in question. Particular axioms are a particular way of describing the theorems, but not the only way. The same goes for computer languages: the rules that are chosen to characterise their expressions can be almost anything because they can be implemented on a wide range of distinct platforms. That is, the expressions themselves are the language, not the specific computational system that characterises them. In natural language, however, the reverse is the case. That is:
\begin{quote}
In natural language there is something in the head, which is the computational system. The generative system is something real, as real as the liver; the utterances generated are like an epiphenomenon. This is the opposite point of view [to that of a computer language or to formal logic]. \citep[110]{Chomsky2002}
\end{quote}

There is thus no analogy between formal languages and natural languages because in the former one can choose any set of axioms to generate the same theorems and it is thus these theorems that are the language. In \textit{natural} languages, however, one cannot choose any set of axioms because there is a computational system in the head with a specific set of axioms or principles, and it is these principles that are the language. This is because expressions are generated by a computational system that is the same for all language users (the output of this generative system varies depending on various factors such as what natural language is spoken by the community of individual language users, but the underlying system that makes this possible is universal). Thus, an explanatory scientific theory of human language cannot, to use Devitt’s terminology, separate psychological reality from linguistic reality, nor can it postulate grammars that are merely “true of” the speaker but are not internally represented in the speaker. This is of course an empirical question that could turn out to be wrong, but we should not misunderstand the claim of internalist semantics, which explicitly postulates a computational procedure that is instantiated in the mind. Devitt claims that “the grammar is describing the syntactic properties of (idealized) linguistic expressions, certain sounds in the air, inscriptions on paper, and the like” but that even though these items “are produced by minds” and “presumably get many of their properties somehow from minds”, “they are not themselves mental” \citep[v]{Devitt2006}. If one assumes an E-language perspective, then one can agree with Devitt that generative linguists conflate a theory of language with a theory of linguistic competence and that “a person could be competent in a language without representing it or knowing anything about it: she could be totally \textit{ignorant} of it” \citep[5, emphasis in original]{Devitt2006}. But internalist semantics takes the I-language perspective and so there is no conflation. Notice that this is not a matter of terminology: the biolinguistic proposal is an empirical proposal that could be right or wrong and that will stand or fall on the merits of its explanatory fecundity. We should be wary of reading into it epistemological relations that are explicitly denied.

Why do Soames, Higginbotham, Devitt, and others insist on retaining the distinction between the mental representation of the language and the language itself and thus misinterpret internalist semantics as also holding the same distinction? The answer has to do with their conception of knowledge. From the I-language perspective the distinction between language itself and its mental representation is superfluous: speakers of a particular language do not represent their language like they would represent some aspects of the external world; they just have their language (\textit{qua} I-language). The mental representations proposed by generative linguistics to account for, say, a particular language, are not \textit{about} that language; \textit{they are that language}. This is language understood intensionally in terms of the generative procedure that produces the set of structural descriptions as opposed to the extensional understanding that sees language as the set itself that is the output of the generative procedure. Higginbotham is not satisfied with this because he wants a linguistic theory to explain not only the properties of language but also how one \textit{knows} (\textit{qua} justified true belief) one’s language. He remarks that the representationalist doctrine, which he reads generative linguistics as holding, fails when it comes to semantics because “[w]e seem to use words with their meaning, when we have only a partial or even a mistaken conception of what that meaning is” \citep[563]{Higginbotham1991}. 

Knowledge of meaning (and language itself), says Higginbotham, “becomes social when we acknowledge others as knowing more than we about meaning, or as correcting us about meaning, and where our grounds for doing so are cognitively based”. But if language is social, Higginbotham continues, then the distinctions “between mental representations and what they represent” and “between what we think the properties of our language are and what they are in fact” are significant \citep[563]{Higginbotham1991}. In other words, if you want a linguistic theory to explain not only the mechanisms in virtue of which language production and comprehension are possible but also how it is that speakers can misrepresent or be mistaken about a particular meaning or grammatical feature, then the “language with its properties must be distinguished from what the speaker knows about it” \citep[563]{Higginbotham1991}. This is a clear expression of externalism, which is “principally a view about the conditions for truth and reference, and invokes the same considerations whether it is the condition for the truth of a sentence, or for the truth of a belief is in question” \citep[328]{Farkas2006}.

Notice again that the externalist position conflates two notions of understanding: the notion of understanding that refers to interpreting a sentence as a speaker of the language (\textit{verstehen}) and, on the other hand, the notion of understanding that refers to explaining as in science (\textit{erklären}) (see \citealt{Slezak2004,Slezak2018}). This is clear in Peacocke's discussion of the “how-questions” in science. He says that when scientists know that something has a given property, they then look to find out how it is able to have that property. Peacocke’s how-questions include: “How is the human body able to avoid waste products building up in the blood?” and “How is a person able to understand a sentence he has never previously encountered?” \citep[315]{Peacocke1994}. However, the former question is of a different type to that of the latter – at least on the usual externalist reading of the latter question. But herein lies the problem, for there is an ambiguity in the question “How is a person able to understand a sentence he has never previously encountered?”: an ambiguity between understanding \textit{qua} speaker and understanding \textit{qua} scientist. It is all too easy to slip in and out of these two very different projects. In other words, the question “How is a person able to understand a sentence he has never previously encountered?” can be answered hermeneutically in terms of interpreting the sentence as a speaker of the language; this is the externalist approach. But the question can also be answered by taking the internalist approach and trying to unearth the psychological mechanisms in virtue of which interpretation is made possible.

Externalist relational properties and their corresponding how-questions are of a different type to the \textit{scientific} how-questions of, say, how certain functions of the blood operate or, crucially, how a person can comprehend a novel sentence understood scientifically. Peacocke’s discussion is indicative of externalism’s conflation of the two types of question. He states that to have “one of the properties identified in Mendelian genetics is to have a highly relational property” and that for a person “to have a recessive gene for red hair” is “something that involves \textit{his relations} to hair colour, to other genes (or factors), and to parents and empirically possible descendants” \citep[315, emphasis mine]{Peacocke1994}. Peacocke assumes that these relational properties are in the same category as properties in biology in virtue of which genetic explanations are possible. He claims that Mendelian theory spells out exactly what these relational properties are. But this assumption is far from obvious, and it’s unclear how a scientific explanation is at all improved by postulating that a \textit{person} has certain relations to \textit{their own} hair colour or to their genes. One can of course postulate such relations, but not without explaining them fully and unpacking the underlying assumption that such relations are relevant to fruitful scientific explanations – for it is this assumption that is doing all the work (as I argue in the next chapter, this assumption is problematic, for fruitful scientific explanations are mechanistic). This assumption is exemplary of the underlying conflation in the externalist literature in which externalists erroneously assume that their questions and theoretical aims are identical with the questions and theoretical aims of scientists studying similar phenomena.


\section{Externalism as a hermeneutic explanatory project}
As we have seen, in the founding article of externalism Putnam claimed that “a better philosophy and \textit{a better science of language}” must encompass the “social dimension of cognition” and the “contribution of the environment, other people, and the world” to semantics \citep[193, emphasis mine]{Putnam1975}. Other externalists make similar claims: \citet{Horwich1998,Horwich2005} argues that his externalist use-based semantics is compatible with a linguistics construed as an empirical science. Moreover, he says of Davidson’s externalist truth-theoretic programme that it “became widely accepted, instigating several decades of ‘normal science’ in semantics” \citep[371]{Horwich2001}. Davidson himself is somewhat ambivalent, but still holds that “my own approach to the description, analysis (in a rough sense), and explanation of thought, language, and action has […] what I take to be some of the characteristics of a science” \citep[123]{Davidson1995}. And Burge says that he sees no reason why formal semantics, which postulates “reference, or a technical analogue, as a relation between linguistic representations and real aspects of the world, should not be an area of fruitful systematic scientific investigation” \citep[465]{Burge2003}. Moreover, \citet[607]{Lassiter2008} claims to have responded to “Chomsky’s challenge to articulate an externalist theory of meaning that can be used in the scientific investigation of language”.
	
In order to assess such claims, let us look at a particular externalist theory of meaning in detail: Davidson’s truth-conditional semantics. Davidson is one of the most influential philosophers of the second half of the twentieth century, and his work has had a significant impact not only on philosophy, but also on linguistics and cognitive science. \citet[viii]{LeporeLudwig2005} remark that “Davidson’s proposal to use a Tarski-style truth theory as the core of a theory of meaning for natural languages […] sparked a revolution in philosophical semantics". Davidson argues that the best way to construct a compositional meaning theory for natural language is to construct a truth theory (based on the work of the logician Alfred Tarski) that assigns, from a finite set of axioms, truth conditions to each sentence of the language. The assignment of truth conditions to the sentences of a natural language is supposed to allow a person to be able to interpret those sentences. The notion of truth, which for Davidson is “the most obvious semantic property” and “one of the clearest and most basic concepts we have” \citep[2, 55]{Davidson2005}, thus acquires a central place in a theory of semantics. Davidson argues that the meaning of a sentence \textit{is} its truth conditions. This conception of semantics is so widespread that, as \citet[310]{LeporeLudwig2004} point out, “any approach to the semantics of natural languages is now likely to begin by stating whether it is based on adopting or rejecting a truth-conditional approach inspired by Davidson’s work”. 

Is a theory of meaning construed in this way compatible with a scientific semantics? Do they have the same aims and explanatory goals? After discussing Davidsonian truth-conditional semantics I will argue that the answer to these two questions is negative. My analysis is applicable not only to the Davidsonian programme, for what I highlight is a symptom of all externalist theories of meaning. Such attempts fail, I argue, not so much because there is no connection between the outside world and what is in the head; rather, an externalist theory of meaning fails because the sort of connection claimed by externalists is either nonexistent or so amorphous that its attempted systematisation puts into question a coherent, fruitful and scientific externalist theory of semantics.

In “Truth and meaning”, the classic paper that convinced many philosophers and linguists of the indispensability of truth conditions in semantics, Davidson attempts to construct a compositional theory of meaning that can “give the meaning of all expressions in a certain infinite set on the basis of the meaning of the parts” \citep[305]{Davidson1967}. He stresses the primacy that sentences have in his theory of meaning and argues (with Frege) that only in the context of a sentence do words have a meaning. Davidson considers the fruitfulness of using the locutions \textit{means that} or \textit{meanings} and concludes that “the one thing that meanings do not seem to do is oil the wheels of a theory of meaning”. That is, his “objection to meanings in the theory of meaning is not that they are abstract or that their identity conditions are obscure, but that they have no demonstrated use” \citep[307]{Davidson1967}. We saw above that internalist semantics posits that syntax (broadly conceived) plus a lexicon are sufficient to explain semantics in natural language. Davidson rejects such a proposal by appealing to “the fact” that “recursive syntax with dictionary added is not necessarily recursive semantics” \citep[308]{Davidson1967}. The locutions \textit{s means m} or \textit{meanings} cannot form the basis of a semantic theory because they cannot be used to provide “for every sentence \textit{s} in the language under study, a matching sentence [… that] ‘gives the meaning’ of \textit{s}” \citep[309]{Davidson1967}. Such locutions, says Davidson, lead us nowhere and present problems that are as hard as or identical to the problems that a theory of meaning attempts to solve. Davidson then concludes that the only way to construct a theory of meaning is to “sweep away the obscure ‘means that’” and replace it with truth conditions. That is, for a semantic theory to “have done its work” it must provide for each sentence in the language under study, a matching sentence that gives the meaning of the former sentence.

Davidson argues that instead of constructing a theory of meaning using such locutions as \textit{s means that p}, one must replace them with the following T-sentence: “\textit{s} is \textit{T} if and only if \textit{p}”. This schema allows one to translate the sentence \textit{s} by replacing the sentence \textit{p} with a sentence in a metalanguage. Davidson then argues, augmenting Tarski’s Convention T \citep{Tarski1956}, that “it is clear that the sentences to which the predicate ‘is \textit{T}’ applies will be just the true sentences of [a language] \textit{L}” \citep[309]{Davidson1967}. In other words, this proposal amounts to replacing the locution \textit{s means that p} with \textit{s is true if and only if p}. That is, “a theory of meaning for a language L shows ‘how the meanings of sentences depend upon the meanings of words' if it contains a (recursive) definition of truth-in-L”, and thus “to give truth conditions is a way of giving the meaning of a sentence” \citep[310]{Davidson1967}.

The classic example of what truth definitions look like in a Davidsonian theory of meaning is the following T-sentence: “\textit{Der Schnee ist weiss}" \textit{is true iff} “\textit{snow is white}”. Thus, an object language sentence replaces \textit{s}, and a metalanguage that provides conditions under which \textit{s} is true replaces \textit{p}. As \citet[4]{LeporeLudwig2007} put it, “the key idea of truth-theoretic semantics is that placing certain constraints on an axiomatic truth theory will […] put us in a position, knowing that the theory meets the constraints, to use it to interpret object language sentences and to see how understanding of them depends on an understanding of their parts and mode of combination”. Truth-theoretic semanticists at times say that truth is not supposed to be understood as meaning and that stating conditions under which the object language is true is not to state what that sentence means. However, Davidson early on argued that “‘[s]ince a truth definition determines the truth value of every sentence in the object language (relative to a sentence in the metalanguage), it determines the meaning of every word and sentence” \citep[322, fn. 8]{Davidson1967}. But this is just a terminological matter, for the theory is not conceived as a meaning theory per se; but rather as an interpretive theory that provides “all the knowledge that a compositional meaning theory is intended to” \citep[317]{LeporeLudwig2004}. Davidson thus aims to “sweep away the obscure ‘means that’” while still retaining the explanatory insights a compositional meaning theory offers \citep[309]{Davidson1967}.

To summarise, Davidson’s theory includes the following three interrelated claims: (i) that a theory of meaning for \textit{L} is a truth-conditional semantics for \textit{L}, (ii) that to know the meaning of an expression in \textit{L} is to know a satisfaction condition for that expression, and (iii) that meanings are satisfaction conditions. \citet[309]{Horwich2008} remarks that a version of a truth-theoretic approach to semantics is “widely endorsed amongst both linguists and philosophers”. This conception of semantics is so ingrained that, as Cummins puts it, “it is something of a challenge to get philosophers of language to realize that The Conjecture [of Davidson’s] is not obviously \textit{true}. Generations of philosophers have been trained to regard The Conjecture as a truism. What else could semantics be? Surely, to understand an expression, one must know the conditions under which it is satisfied!” \citep[153, emphasis in original]{Cummins2002}. Like Horwich but for different reasons, Cummins is skeptical of truth-theoretic semantics and remarks that “we are now in a position to see that it is probably false, but I do not expect many to agree with me about this” \citep[153]{Cummins2002}. He argues that the effect of taking truth-conditional semantics as a truism meant that much of philosophy of mind took it upon itself to explain how mental representations could have the satisfaction conditions that truth-conditional semantics required. In other words, once you assume “a Davidsonian story about the semantics of natural language, it is nearly irresistible to conclude that intentional states or mental representations (or both) must have a truth-conditional semantics as well” \citep[153]{Cummins2002}.

But this inference from semantics to the nature of mental representation is problematic, says Cummins, for we have good independent reasons to think that mental representations do not have a truth-conditional semantics. Cummins distinguishes two notions of meaning: the first is the \textit{communicative meaning} of a term, which is whatever must be in the mind to allow the understanding of the term. The second is the \textit{truth-conditional meaning} of a term, which is the term’s satisfaction condition or its role in generating one in context. Cummins argues that if one accepts that a constraint on a theory of meaning is that it needs to explain whatever it is that has to be grasped or possessed for linguistic communication to be successful, then the mental representations required for linguistic understanding do not have a truth-conditional semantics. Thus, “a theory of language understanding will make no use of truth-conditional semantics”, for “there is no good reason to think that a truth-conditional semantics for natural language will have any place in a mature psycholinguistics” \citep[155]{Cummins2002}.

The independent reason Cummins refers to that shows that mental representations do not have truth conditions is as follows. On the externalist understanding of meaning, the concept of, say, \textit{horse} is the mental representation the reference of which is either horses or the property of being a horse. Cummins rejects this conception of meaning in favour of the claim that the concept of a horse is “a body of knowledge loosely identified by its topic. Just as a book about horses has horses as its topic, but not its referent, so a concept of horses has horses or the property of being a horse as its topic rather than its referent” \citep[158]{Cummins2002}. A loose analogy to this conception of meaning would be to say that “a concept (of horses, say) is a \textit{theory} (of horses), the idea being that theories are organized bodies of knowledge that we identify in much the way we identify concepts—viz., by specifying a topic” \citep[158, emphasis in original]{Cummins2002}. Theories, of course, are identified by their topics not by their referents or satisfaction conditions, and the same is true in the case of concepts. If we conceive of concepts as tacit theories, and agree that they are what you need to have in the mind in order to understand particular terms, then concepts do not semantically combine in the way that is required by truth-conditional semantics. In other words, the sort of semantics invoked by “Tarskian combinatorics [is] hopeless in connection with the sorts of psychological structures concepts must be to do their jobs” \citep[159]{Cummins2002}.

Cummins argues that truth-conditional semantics is psychologically implausible, for we have independent reasons to think that the content of concepts is not truth-conditional. But there is a deeper problem with the sort of externalist semantics exemplified by Davidson. As \citet{Slezak2014,Slezak2018} discusses, the semantic intuitions of the theorist are in this case relied upon to identify the objects to be explained by the theory. But this reliance on the intuitions of meaningfulness is illegitimate for it relies upon the very mental ability to be explained. The T-sentence is explicitly invoked as a sentence that is expressed in a theoretical metalanguage understood by the theorist. Davidson is clear about this when he  remarks that the “inevitable goal of semantic theory is a theory of a natural language couched in a natural language (the same or another)” \citep[71]{Davidson1973a}, and that “it is one condition on the correctness of a theory of meaning that it be such that if an interpreter knew it to be true of a speaker, the interpreter could understand what the speaker said” \citep[131]{Davidson1995}.\footnote{See \citet{Slezak2018} for discussion of these quotes and others.} But understanding the language as speaker is implicitly relied upon here in order to make the explanation work.

The two volume study of truth-theoretic semantics by \citet{LeporeLudwig2005,LeporeLudwig2007} is very clear about the conception of semantics being an interpretive enterprise, and thus as being in direct opposition to the internalist conception of semantics. As they put it:
\begin{quote}
An interpretive truth theory shows how we understand complex expressions on the basis of understanding their significant components. But, as we have said, it does not state how we do it. For the illumination for a particular language presupposes grasp of another language, the metalanguage, in which the theory is given. It is through our already grasping a language which is at least equal in expressive power to the object language, and in some respects greater (the object language need not have the resources to give its own truth theory), that we are able to see in detail how the semantic combinatorics of the object language work. \citep[9]{LeporeLudwig2007}
\end{quote}
This should come as no surprise, they say, for “there is no question of a standpoint for understanding meaning that is outside of language altogether”. That is, “the most fundamental and powerful devices for representation can obviously not be explicated without the use of just those devices. We can then at best show how they work by showing how they systematically contribute to how we understand sentences in which they appear. And there will be no way to do this that does not mirror the structure of the sentences whose structure we seek to illuminate” \citep[9]{LeporeLudwig2007}.

\citet{Riemer2005, Riemer2010, Riemer2015} also argues that semantics is interpretive. He says that semantics is a project that is "essentially hermeneutic” and that it is “a humanistic discipline closely linked with literary studies” \citep[42-43]{Riemer2019}. It follows from this that the “subjective character of semantic analysis is irreducible, and that real empirical progress in all varieties of linguistics is dependent on an acceptance of this fact” \citep[4]{Riemer2005}. On this view, the irreducibly interpretive character of the study of meaning is due to “the fact that central theoretical features of the explanation of semantic phenomena have no other justification than the subjective judgement of the investigators” \citep[3]{Riemer2005}. One might worry that the inherent subjective character of semantic analysis clashes with the goals of empiricism, psychological realism, and scientificity in linguistics, but Riemer argues that analyses of meaning such as those of cognitive semantics \citep{LakoffJohnson1980, Lakoff1987, Langacker1987} “remain genuinely useful and explanatory – despite, or rather because of, the acknowledged subjectivity at their core” \citep[4]{Riemer2005}. In other words, if their “explanatory power turns out not to be a ‘scientific’ one, then so be it”, for such semantic models preserve “their explanatory value regardless of their ultimate status as ‘science’” \citep[4]{Riemer2005}.

The description of meaning is, according to Riemer, “infinitely less constrained and more open to varying characterizations than is the description of morphology or syntax”, and as a result “semantics has much more to lose by a tolerance towards alternative descriptions, and runs the risk that any analytical specificity about the nature of a single meaning/conceptualization will be lost in a scatter of divergent but equally endorsed analyses” \citep[8]{Riemer2005}. The comparison here with syntax is revealing, for of course we cannot study syntax hermeneutically in the same way that we study semantics hermeneutically (there is no meaning to understand \textit{qua} speaker in syntax in the same way that there is in semantics). However, we \textit{can} study semantics \textit{non}-hermeneutically. That is, we can study semantics on the model of syntax, phonology, and morphology. One can agree with \citet[8]{Riemer2005} that “any meaning is open to a variety of different, often incompatible, descriptions” and that “the choice of the optimal description is a prerequisite if the analysis is to attain a minimal degree of empirical specificity", but such an argument applies only to semantics construed as a project aimed at producing meta-linguistic \textit{descriptions} of meaning. The internalist project discussed in this book has a different aim, one that is not descriptive but rather generative in the sense that it explicitly aims to unearth the mechanisms that are responsible for the generation of meaning. That is, the aim of internalist semantics is not to provide an optimal description of a particular meaning but rather to discover the underlying mechanisms that generated that meaning in the first place. Riemer concludes that semantics is not science because of its irreducibly interpretive character, and he claims that real empirical progress in \textit{all varieties of linguistics} is dependent on an acceptance of this fact. But this is only the case if semantics is understood hermeneutically, and it fails to acknowledge that semantics in the internalist and biolinguistic tradition is not understood nor practiced in this way.

Unlike leading externalists, Riemer recognises that semantics \textit{qua} hermeneutics is not scientific. Cognitive semantics is inherently interpretive in the same sense that truth-theoretic semantics is. However, as \citet{Slezak2018} shows, if conceived as a \textit{scientific} project, truth-theoretic semantics has a fatal explanatory error, one that is apparent in other domains and that renders such theories explanatorily vacuous from a scientific perspective. The recent study of Quine and Davidson by \citet{Kemp2012} comes to the same conclusion. He remarks that the “bottom line is that the intuition or semantical judgement of the interpreter cannot be removed from the loop, and thus the theory fails to measure up to the standards of impersonal science” \citep[12]{Kemp2012}. Davidson’s is a “non-naturalist standpoint” that is “an unscientific if intuitive standpoint” \citep[12]{Kemp2012}; it is a pragmatic account that “relies ineliminably on an inarticulate human skill or art” \citep[143]{Kemp2012}.

Given the above, one would perhaps be content to conceive of truth-theoretic semantics as a hermeneutic project, but what is one to make of Davidson’s proclamations to the contrary? As noted above, he remarks that “my own approach to the description, analysis (in a rough sense), and explanation of thought, language, and action has […] what I take to be some of the characteristics of a science” \citep[123]{Davidson1995}. Moreover, his unified theory of speech and action (of which his theory of meaning is part) “presents a clear and precise formal structure with demonstrable merits”, and it does so “like any scientific theory” \citep[126]{Davidson1995}. Davidson takes his theory to be a psychological theory, but he hedges his bets by saying that whether “the features of a psychological theory I have been rehearsing [...] show that a psychological theory is so different from a theory in the natural sciences as not to deserve to be called a science I do not know, nor much care”. What he is “sure of is that such a theory, though it may be as genuine a theory as any, is not in competition with any natural science” \citep[134]{Davidson1995}. Here and elsewhere Davidson is clearly ambivalent about the scientific aims of his theory, but as we’ll see in the next chapter there are plenty of externalists (both Davidsonians and not) who unambiguously claim their project to have the same explanatory aims as scientific pursuits like cognitive psychology. 

What is going on here? On the one hand, we have claims that “[e]xternalism sets limits to how complete psychological explanation can be, since it introduces into the heart of the subject elements that no psychological theory can pretend to explain”. But on the other hand, “this feature in itself makes psychological theory no less scientific than volcanology, biology, meteorology, or the theory of evolution” \citep[129--130]{Davidson1995}. Davidson at times claims that there is a fundamental difference in aims and interests between his theory and Chomsky’s in that he is not trying to understand and explain the same phenomena. Yet at other times he explicitly conceives of his externalist theory as psychological and scientific. If \citet{Slezak2018} is right, then Davidson’s truth-theoretic semantics is a specific and revealing example of an explanatory theory that relies on the inner abilities it purports to explain. But conceiving of Davidsonian semantics as a hermeneutic project shields it from such criticism, for we can understand each other in a hermeneutic manner as speakers of a language and not as theorists of language in the scientific sense. As noted above, there are two senses of understanding here that are often conflated. One sense of understanding language is the project that takes semantics to be essentially hermeneutic, as a largely humanistic discipline. The other sense is the scientific or psychological sense \textit{qua} internalist semantics.\footnote{See \cite{Riemer2019} for evidence that cognitive linguistics makes the same conflation.}

It is noteworthy that late in their careers, both Quine and Davidson acknowledged the affinities between their Analytical philosophical methods, attitudes and traditions and that of the hermeneutic and Continental project (see \citealt{Slezak2018} for quotes and discussion). This affinity is explicit in \citet{Malpas2011}, which is a collection of leading scholars reflecting on Davidson’s work. In the book’s foreword, Dagfinn Føllesdal writes that it is “easy to connect him [Davidson] with the hermeneutic tradition, particularly with the new hermeneutics, Heidegger and Gadamer and their followers” (xii).
