\chapter{Clarifications and methodological preliminaries}

The problem of intentionality is the problem of how some entities can be “about” something. That is, words and sentences, among others, display intentionality in that they are about something else; they are said to be a representation of something. The notion of intentionality can be traced back at least as far as Aristotle, though the German philosopher Franz Brentano is generally credited with introducing the notion to contemporary philosophy in the late nineteenth century. Brentano’s oft-quoted remark is that “[e]very mental phenomenon is characterized by […] the intentional (or mental) inexistence of an object” and “reference to a content, direction toward an object”. In other words, “[e]very mental phenomenon includes something as object within itself, although they do not all do so in the same way. In presentation something is presented, in judgement something is affirmed or denied, in love loved, in hate hated, in desire desired and so on” \citep[68]{Brentano1874}. The usual way to frame the problem of intentionality is in terms of meaning or content. What is the status of the semantics of a sentence over and above its syntactic aspects? What makes it the case that a particular proposition has the content that it does? Is meaning only dependent upon mind-internal properties? Or must we make use of mind-external factors such as the context of the utterance or the speaker’s social history in order to determine the meaning?

Those who argue that the relevant and scientifically interesting properties that are involved in meaning are overwhelmingly, though not entirely, within the mind are referred to as internalists. On the other hand, externalists argue that there is something more to meaning than purely mind-internal events and their happenstance connection to the world: externalists insist that the meanings of our words (or sentences, or the contents of our thoughts, etc.) depend on some deep metaphysical (perhaps causal) connection between the mind and objects in the world that are independent of the mind. Externalists argue that a semantic theory needs to provide an account of the relation between linguistic expressions and things in the world \citep{Cann1993}. In other words, the claim is that in order to explain meaning we must provide an account of the purported formal/causal/metaphysical relation between linguistic expressions and the things that they can be used to talk about. 

The externalist position has become a widely held position in the philosophy of language. The classic arguments for externalism are found in \citet{Putnam1975}, \citet{Burge1979,Burge1986}, and \citet{Kripke1980}. Broadly speaking, externalist theories take a model-theoretic approach to semantics. They model the interpretation of natural language sentences by making use of set-theoretic structures and truth conditions. According to such theories, understanding a sentence involves at least in part the grasping of its truth conditions. This approach is not limited to philosophy of language, for there is a great deal of work in linguistics, for example in formal semantics \citep{HeimKratzer1998,Portner2005} and formal pragmatics \citep{Kadmon2001}, that takes the truth-conditional approach. For example, in his formal semantics textbook, \citet[11, 13]{Portner2005} argues that “meanings are not internal to language, are not in the mind, and are not merely social practices. Rather, they are based in language- and mind-external reality”. Moreover, “knowledge of meaning involves (at least) the knowledge of the conditions under which a sentence is true”. In other words, “all there is to the meaning of a sentence is its truth-conditions”. It should thus be clear at the outset that philosophers \textit{and }linguists take the externalist understanding of meaning seriously both in theory and in practice. Moreover, just like the externalist philosophers discussed in this book, some linguists also aim “to approach meaning as scientists” \citep[4]{Portner2005} and not, presumably, as members of the project that construes meaning in a hermeneutic or interpretive fashion. 

Putnam argues that “a better philosophy and a better science of language” must encompass the “social dimension of cognition” and the “contribution of the environment, other people, and the world” to semantics \citep[193]{Putnam1975}. His Twin Earth thought experiment is the most famous argument in favour of externalism; it claims to show that two subjects can have identical internal psychological mental states but that the content of these states can be different due to particular variations in the environment. Putnam asks us to imagine a world (Twin Earth) in which water is not composed of H\textsubscript{2}O like it is on our world but is rather composed of XYZ. When a person (call him Oscar) says \textit{water} on Earth the word refers to H\textsubscript{2}O, but when a different person (call him Twin Oscar) says \textit{water} in a different place (on Twin Earth) the word refers to XYZ. This seems intuitively clear; the word \textit{water} refers to what the word is about in that particular environment (so when Oscar utters \textit{water} that word is about H\textsubscript{2}O in his environment). Putnam asks what would happen if Oscar is transported to Twin Earth. Would the word \textit{water} uttered by Oscar on Twin Earth now refer to H\textsubscript{2}O or XYZ? Notice that the thought experiment legislates that the only change that takes place when Oscar is transported from Earth to Twin Earth is the change in his environment (i.e., all of his psychological states remain unchanged). Now, Putnam reasons that if knowing the meaning of a term is just a matter of being in a certain psychological state, then \textit{water} on Twin Earth when uttered by Oscar should refer to H\textsubscript{2}O and not to XYZ as we might expect. This is because Oscar’s psychological state was fixed on Earth, and if the psychological state fixes the reference then \textit{water} refers to H\textsubscript{2}O regardless of the environment the subject is in. 

Another way to put the matter is as follows: when Twin Oscar on Twin Earth says \textit{water} whilst pointing to a lake that is entirely composed of XYZ, as all watery things are composed of on Twin Earth, \textit{water} refers to XYZ and not to H\textsubscript{2}O. But, Putnam’s argument claims, if knowing the meaning of a term is just a matter of being in a certain psychological state then \textit{water} uttered on Twin Earth by Oscar transported from Earth cannot mean XYZ and must mean H\textsubscript{2}O. Something seems to be wrong here. If two people utter the same word in the same environment we expect that word to refer to the same thing. Thus, if we want to hold on to the claim that the meaning of a term determines its reference or extension then, the argument claims, we must concede that, as Putnam famously put it, “[c]ut the pie any way you like, ‘meanings’ just ain’t in the head!” \citep[144]{Putnam1975}. That is, the claim here is that mind-internal properties on their own cannot fix the meanings of words or what their reference is. Note again that for Putnam and others such claims are made within the realm of a science of language. They are claims about the nature of meaning in the mind that they see as having direct bearing on the psychology of meaning.

One might conclude that externalism has to be right, for how could meaning not depend on the outside world? Surely the meaning of the word \textit{elephant} cannot be due to only mind-internal properties. The word is about elephants, it could be argued, which are in the mind-external world, not inside the mind. As we will see, however, internalists argue that there are good reasons to question the externalist claim that meanings are connected to the world in the way in which externalists claim they are. In other words, internalism does not deny the link to the outside world but rather has a different explanation of how our mind generates and interprets semantic content. Internalism argues that, for the purposes of scientific inquiry into language and mind, the internal properties of the human mind are the most relevant and fruitful subject matter. Thus construed, internalism is not so much a solution to the issues that externalists grapple with. Rather, as I detail in the forthcoming chapters, internalism is a different research programme. There is a difference in the sorts of questions that externalists and internalists attempt to answer. This is important to stress at the outset, for there has been a great deal of misunderstanding due to terminological choices. This is because the label \textit{internalism} has been used to refer to several different and opposing research programs.

In the remainder of this chapter I show that the research project that surrounds the debate between externalism and individualism is orthogonal to the research project of internalism. Too often individualism and internalism are used interchangeably, but the way in which internalists in linguistics and the philosophy of language (such as Noam Chomsky and Paul Pietroski) understand and practice their research project is very different from that of individualism. I don’t want to engage in a terminological dispute about what \textit{internalism} means or should mean, rather I want to describe and thus help in preserving a particular understanding of internalism (and of meaning) that has been used in the literature for a long time and has provided interesting and valuable insights into the nature of language and mind. Individualism, anti-individualism and externalism are each concerned with the criteria for the ascription of meaning, whereas internalism is concerned with the underlying generative mechanisms of meaning.

\citet[3-4]{Burge1986} defines individualism as “a view about how kinds are correctly individuated, how their natures are fixed”. According to individualism about the mind, then, “the mental natures of all a person’s or animal’s mental states (and events) are such that there is no necessary or deep individuative relationship between the individual’s being in states of those kinds and the nature of the individual’s physical or social environments”. Individualism is concerned with attribution, with the proper labels that should be assigned to particular mental states. In the case of language, the individualist (and the anti-individualist or externalist) project is part of the approach that aims to provide meta-linguistic semantic descriptions of the linguistic usage of speakers in particular contexts. Internalism, on the other hand, is concerned with the underlying mechanisms in virtue of which language use is made possible. Anti-individualism or externalism argues that there is a deep individuative relationship between mental states and the environment, but this is not a claim about the underlying psychological mechanisms of language but rather about how to interpret particular utterances given their context of use.

\citet{Wikforss2008} argues that externalism is “the thesis that meaning (and content) fails to supervene on internal facts”. In other words, “[f]or all natural kind terms T, and all meanings M, the totality of facts that determine that T expresses M include external facts” \citep[161]{Wikforss2008}. Wikforss then remarks that “internalism, by contrast, is the thesis that the determination basis includes only internal facts”. However, the main force and substance of the internalist position discussed in this book is not a mirror image or a negation of the externalist or anti-individualist position. That is, internalism is not concerned with how we determine (\textit{qua} speakers) what the meaning of a particular utterance is. A fortiori, it does not claim that the determination basis of utterances includes only internal facts. Internalists are skeptical that there is a deep metaphysical relation between the “things in the world” and linguistic expressions, and they dispute the externalist contention that the relations between linguistic expressions and the “things in the world” are desirable or even tractable in a scientific theory of language. The literature in the philosophy of language that discusses individualism is immersed in debates about the correct attribution of semantic content to utterances given a particular context of language use; indeed, this question is the focal point of the individualism/anti-individualism debate. But whatever the merit and explanatory force of the search for the correct attribution conditions, it is clearly separate to internalism. Note again the parallel with linguistics: semanticists also aim to uncover the correct criteria for the attribution of semantic content. But this is not the only way in which to do semantics. Indeed, as detailed in this book, the internalist semantics of biolinguistics has a different understanding of the aims of a semantic theory.

\citet{Burge2003} remarks that even though he at times uses the term \textit{internalism}, he prefers to use the term \textit{individualism.} He admits that some of Chomsky’s “arguments for ‘internalism’ do not directly connect with my objections to the view I designate with the term ‘individualism.’ So some apparent disagreement may not be real” \citep[453]{Burge2003}. Indeed, for Chomsky’s internalism is not the same as individualism. Burge is explicit about this: “Internalism, in \textit{my }sense, concerns not the locus of the psychological states, or the best ways to study them, but whether being in them presupposes individual-environmental relations. It concerns whether the existence and nature of certain psychological kinds depends necessarily on the existence and nature of certain relations to specific kinds or situations in the environment” \citep[454, emphasis in original]{Burge2003}. As we will see below, however, despite Burge’s explicit distinction between his sense of internalism (individualism) and Chomsky’s internalism, the conflation of the two continues.

Let us now briefly rehearse a classic argument about the individuation of meaning in order to detail and clarify the thesis of individualism and to separate it from internalism as understood in this book. This is important to sort out because too often internalism is dismissed as being another species of individualism, but it is far from that and in fact offers genuine insights into the nature of meaning in natural language. Moreover, this discussion shows that the concerns of philosophers of language often overlap with the concerns of linguists. The two camps may use different terminology and they may have little contact with each other, but it is clear that they are often involved in the same research programme with similar theoretical working assumptions.

\citet{Burge1979} argues against individualism by stressing the necessity of the inclusion of mind-external factors in the descriptions of an individual’s mental states. Burge’s anti-individualist argument revolves around a Twin Earth thought experiment in which a person is said to have a large number of propositional attitudes with the content of \textit{arthritis}. So, for example, this person correctly thinks that he has had arthritis for years, or that stiffening joints is a symptom of arthritis. In addition to these attitudes, Burge’s thought experiment continues, the person falsely believes that he has developed arthritis in his thigh: this is impossible by definition, as the person is informed by his doctor, since the speech community does not use the term \textit{arthritis} to apply to ailments outside of the joints. Next is the counterfactual supposition of the thought experiment in which we are asked to imagine a second person whose life has proceeded from birth through an identical course of physical events, right to and including the time at which the first person initially reports his fear that he has arthritis in his thigh to his doctor. The only difference that is postulated to exist between the two people is that the latter’s community of physicians and informed laymen apply the term \textit{arthritis} not only to arthritis but to various other rheumatoid ailments. 

So in the first case, the person falsely believes that he has arthritis in his thigh. Whereas in the second case, the person correctly believes that he has arthritis in his thigh. Burge concludes that the “upshot of these reflections is that the patient’s mental contents differ while his entire physical and non-intentional mental histories, considered in isolation from their social context, remain the same”, and the “differences seem to stem from differences ‘outside’ the patient considered as an isolated physical organism”. That is, the “difference in his mental contents is attributable to difference in his social environment” \citep[79]{Burge1979}. So the contents of one’s thoughts, according to Burge, are individuated by and depend on the meaning of the terms as used in one’s linguistic community. That is, “social factors may enter in complex ways into individual psychology and the semantics of idiolects” \citep[275]{Burge1989}. Burge argues that the correct attribution of meaning is impossible without reference to the social context in which the individual uses that meaning. Burge is here concerned with the criteria for semantic attribution, which is related to but independent of the study of the underlying mechanisms that make the production and comprehension of meaning possible. Michael Devitt makes a similar point when he argues that “thoughts are one thing, their ascription another”. He believes that it is a mistake for philosophers to “start with the theory of thought ascription, leaving the theory of thought pretty much to look after itself” \citep[385]{Devitt1984}. The upshot of the difference between individualism and internalism is that one can search for the correct criteria for the individuation of mental states (or for the correct way in which to produce meta-linguistic semantic descriptions) without being committed to the nature of the mechanisms that underlie these mental states.

Incidentally, the amalgamation of internalism with a form of individualism (thus excluding the semantic internalism discussed in this book) is not limited to the externalist literature. Consider the internalist (individualist) accounts of \citet{Segal2000}, \citet{Farkas2008}, \citet{Mendola2008}, and \citet{Georgalis2015}. These books are perhaps the most notable of the so-called internalist accounts of meaning of the last two decades, but revealingly none deal with internalism in the sense discussed here. The work of Chomsky and others is barely mentioned let alone discussed in sufficient depth (or at all). This is not meant as a criticism. They do not do so because they deal with individualism, with matters of ascription, description, and truth-conditional semantics, and argue against anti-individualism (externalism). Indeed, as \citet[63]{Yli-VakkuriHawthorne2018} remark in a recent critical book on narrow content, “the most natural of our structural conditions” is “nearly universally accepted by internalists”. This condition is “that narrow content should be truth-conditional”. \citet{Chalmers2003} also argues for this sort of internalist content. And \citet[184]{Farkas2008} concludes her book by remarking that “[t]here is no need for the internalist to give up the idea that contents are truth conditional”. 

One might wonder whether internalists \textit{qua} individualists also claim their account to be scientific like some leading externalists do. The answer is that some do so, and so as far as their account is a variation of truth-theoretic semantics, what I will have to say in regard to externalist theories of meaning will apply to individualists too. \citet{Mendola2008} argues that science can settle the debate between internalism and externalism. He understands internalism to be a claim about the content of, say beliefs and desires, and argues that the neuroscience of vision and other sciences support his internalist position. But there is a problem with this strategy that \citet[10]{Mendola2008} himself notices, though he draws the wrong conclusion from it. He says that this strategy is even more popular on the other side, with externalists, and that for “every internalist who claims to be deferring to cognitive science, there are two externalists who do the same”. The problem is that, “even if we take our current cognitive science and psychology as gospel, the deference-to-science strategy doesn’t work right now, for internalists or externalists” because “it doesn’t now clearly cut one way or the other, or at the very least there is no consensus on how it cuts”. But the reason for the lack of consensus is not that science has not yet shown which side is the clear-cut winner. Rather, the reason there is no consensus about whether science supports internalism (individualism) or externalism (anti-individualism) is that science is a different project altogether. The two projects (science and externalism) can and should inform one another, but they are distinct. To repeat, the notion of internalism understood as the mirror image of externalism (that is, understood as rejecting anti-individualism but still clinging to reference and content understood truth-conditionally) is very different to the internalist position described in this book.

As noted above, externalism has become a widely held position that is especially popular within the philosophies of mind and language. Indeed, some feel that “externalism has been so successful that the primary focus of today’s debate is not so much on whether externalism is right or wrong, but rather on what its implications are” \citep[158]{Wikforss2008}, and that “[o]ver the past 30 years much of the philosophical community has become persuaded of the truth of content externalism“ \citep[257]{MajorsSawyer2005}. Externalism has thus become “almost an orthodoxy in the philosophy of mind” \citep[187]{Farkas2003}. Since the internalist position is very much in the minority, it is necessary to begin by outlining its conception of a semantic theory before it can be compared with the received view of externalism. A clear understanding of this strand of internalism is essential, for not only is it a minority view but it is also widely misunderstood. After comparing the two approaches to semantics, I will argue that internalism is significantly more promising in regard to constructing an explanatory scientific theory of meaning. Note again that what follows is that the externalist (hermeneutic and interpretive) approach is a different research project to that of a semantics construed scientifically, and so nothing follows about the validity or fecundity of externalism construed hermeneutically.