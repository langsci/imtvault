\addchap{Acknowledgments}
This grammar is a revised version of my doctoral dissertation at the University of Zürich, which I have submitted in January 2014 and successfully defended in February 2014. It would not exist in its present form without the support of various people and institutions. First of all, I am very grateful to Prof. Novel Kishor Rai for suggesting Yakkha as a language to work on for my doctoral dissertation and for establishing the contact to the Yakkha community in 2009.

None of this work would have been possible without the generous support and the overwhelming hospitality of so many people from the Yakkha community. I would like to thank from all my heart Kamala  Jimi (Linkha), who opened her home to me and my husband Lennart. She became our friend and also my most important Yakkha teacher. This grammar owes much to her enthusiasm. My deepest gratitude also goes to Magman Linkha and Man Maya Jimi, who took the time to work with me and share their native speaker intuitions with me. Kamala Linkha, Magman Linkha and Mohan Khamyahang also painstakingly went through each record of my lexical database and offered corrections and additions where appropriate.

Many people were so kind to let me record and archive their speech, thus creating the basis for my linguistic analyses. {\footnotesize\Deva धेरै धन्यबाद्} to Prem Kumari Jimi, Kamala Jimi (Koyongwa), Kamala Jimi (Linkha), Ram Kul Bahadur Jimi, Dhan Kumari Jimi, Ganga Ram Jimi, Sita Linkha, Magman Linkha, Lanka Maya Jimi, Om Bahadur Jimi (Koyongwa), Desh Kumari Jimi, Padam Kumari Jimi, Chom Bahadur Jimi, Kaushila Jimi, Man Bahadur Khamyahang, Hasta Bahadur Khamyahang, Man Maya  Jimi, Bhim Maya  Jimi, Mohan Khamyahang and his mother. Many thanks also go to Magman Linkha, Ajaya Yakkha and Shantila Jimi for letting me incorporate their written stories into my database.

I would also like to thank everyone in the Kirant Yakkha Chumma (Indigenous Peoples Yakkha Organization) for their trust and their interest in my work and also for practical and administrative support, especially in the early phase of the project, in particular Kamala Jimi (Koyongwa), Indira Jimi, Ramji Kongren and his family in Dandagaun, and Durgamani Dewan and his family in Madi Mulkharka. Heartfelt thanks also go to Dhan Kumari Jimi, Dil Maya Jimi, Nandu Jimi and their families for their hospitality. I am very grateful to Kaushila Jimi, Sonam Jimi and Vishvakaji Kongren in Kathmandu for their spontaneous help, and to the teachers at Shree Chamunde Higher Secondary School and Ram Kumar Linkha for taking an interest in my work.

I wish to express sincere appreciation to Balthasar Bickel for sharing his insights and expert knowledge on Himalayan languages and on Kiranti languages in particular. This thesis has also greatly  benefited from numerous discussions with Martin Haspelmath, whose comments gave me new perspectives on various topics throughout this work.

I would like to thank my colleagues and friends at the MPI EVA and the University of Leipzig for linguistic discussions and shared enthusiasm: Iren Hartmann, Katherine Bolaños, Eugenie Stapert, Kristin Börjesson, Lena Terhart, Swintha Da\-niel\-sen, Falko Berthold, Sven Grawunder, Alena Witzlack, Zarina Molochieva, John Peterson, Netra Paudyal and Robert Schikowski. I have also benefited greatly from the  ELDP language documentation workshop held at SOAS in March 2012. Conversations with colleagues at conferences and other occasions have also been valuable, especially with Mark Donohue, Martin Gaenszle, Kristine Hildebrandt, Gwendolyn Hyslop,  Eva van Lier,  Tom Owen-Smith and Volker Gast.


Special thanks go to An Van linden,  Mara Green, Alena Witzlack, Iren Hartmann, Katherine Bolaños, Lennart Bierkandt, Falko Berthold, Tom Owen-Smith and Tyko Dirksmeyer for their comments on individual chapters, to Hans-Jörg Bibiko for auto\-mating the dictionary clean-up to the greatest extent possible, and to Lennart Bierkandt, additionally,  for elegantly formatting the \isi{kinship} charts  and numerous diagrams, and for levelling the  LaTeX learning curve for me. Of course, I take responsibility for any mistakes or omissions in this work.

My work on Yakkha has been  funded by a graduate scholarship from the State of Saxony (2009–2012) and by an Individual Graduate Studentship from the Endangered Languages Programme ELDP (2012–2013, Grant No. IGS 154). The field trip in 2011 was financed by a travel grant from the German Academic Exchange Service DAAD. I would also like to thank Bernard Comrie, director of the Linguistics Department at the Max Planck Institute for Evolutionary Anthropology (MPI EVA)  for hosting my ELDP project and also for financing my field trips in 2009 and 2010. The MPI EVA provides ideal conditions for such work, and my thesis has benefited greatly from the resources at this institution and from discussions with colleagues and guests of the department. I thank Claudia Büchel and Julia Cissewski in Leipzig as well as Sascha Völlmin in Zürich for being so incredibly helpful in all administrative matters.

I am very thankful to Martin Haspelmath for offering me the opportunity to publish my work at Language Science Press, to three anonymous reviewers for their valuable comments and suggestions, to Sebastian Nordhoff for managing the publication process and, last but not least, to the numerous proofreaders, who did a wonderful job.

My heartfelt gratitude goes to my family and my friends in Germany, Nepal and elsewhere, especially to my mother for her support during all these years, and to Laxminath, Rita and the whole Shrestha family in Kathmandu. Without Laxminath's efforts, my spoken Yakkha skills would probably be better now, because my Nepali skills would have been much worse. This work is dedicated to the memory of Belayati Shrestha.

Finally, I thank Lennart (again): for making those Nepal journeys  “our” journeys.

