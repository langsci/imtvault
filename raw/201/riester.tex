\documentclass[output=paper
,modfonts
,nonflat]{langsci/langscibook} 
%\usepackage{glossaries}

\ChapterDOI{10.5281/zenodo.1402551}

\title{Information structure in Sumbawa: A QUD analysis}

\author{Arndt Riester\affiliation{Institute for Natural Language Processing (IMS), University of Stuttgart}\lastand Asako Shiohara\affiliation{Research Institute for Languages and Cultures of Asia and Africa (ILCAA),\newline Tokyo University of Foreign Studies}}

% \chapterDOI{} %will be filled in at production
% \epigram{}

\abstract{This paper describes the constituent ordering and other basic morphosyntactic properties of Sumbawa and their relation to information structure. Our study is based on conversational corpus data and makes use of a novel method of in\-for\-ma\-tion-struc\-tu\-ral discourse analysis, which is based on the reconstruction of implicit questions under discussion (QUDs).}

\begin{document}
	
\maketitle
	
\section{Introduction}\label{sect:intro}

\ili{Sumbawa} (indigenous designation: Samawa; ISO-639-3 code: SMW) is a language spoken in the western part of \ili{Sumbawa} Island, Indonesia. \ili{Sumbawa} belongs to the \ili{Bali}-Sasak-\ili{Sumbawa} subgroup of the \ili{Malayo-Polynesian} branch of the \ili{Austronesian} language family \citep[][19]{adeal05,mbeaa90}. In this paper, we investigate the variation of \isi{constituent order} in \ili{Sumbawa} verbal clauses, using transcripts of a spoken conversation. In particular, we are interested in the question of how morphosyntactic variation (in particular, pre- and postverbal argument realization as well as the occurrence of clitics) is correlated with information structure, an aspect of \ili{Sumbawa} about which so far relatively little is known.


In order to understand how information-structural variation is expressed, we use a novel method of textual analysis, developed in \citet{reyuw16,riear17b}, whose goal it is to identify for each elementary \isi{assertion} the implicit \textit{question under discussion (\isi{QUD})} \citep{kupja95,bueda03,robcr12} to which the \isi{assertion} provides an answer. Based on these QUDs the information structure of each \isi{assertion} can be straightforwardly determined. The structure of this article is as follows: \sectref{sect:grammarOutline} provides an outline of the verbal \isi{clause} structure in \ili{Sumbawa}, with a special focus on syntactically possible constituent-order variation and its correlation with the presence or absence of a \isi{clitic pronoun} on the predicate, whose dependence on information structure we will explore in the subsequent sections. In \sectref{sect:theory}, we will introduce the annotation method we will apply to the conversational data in order to determine the information structure of each utterance. \sectref{sect:corpusAnalysis} and \sectref{sect:clitic} provide the result of the application: \sectref{sect:corpusAnalysis} gives a rough picture how the three categories \textit{focus,} \textit{background}, and \textit{\isi{contrastive} topic} shape the general \isi{constituent order} of \ili{Sumbawa}, while \sectref{sect:clitic} focuses on the order of argument and predicate in relation to the presence or absence of the \isi{clitic pronoun}. In \sectref{sect:summary}, we will give a summary of the sections and evaluate the effect of the method.

\section{Constituent order and clitics in Sumbawa}\label{sect:grammarOutline}
In this section, we discuss the morphosyntax of \ili{Sumbawa} verbal clauses, with a special focus on syntactically possible \isi{constituent order} variation and its correlation with the presence or absence of a \isi{clitic pronoun} on the predicate, largely based on \citet{shias13a,shias13b}, drawing on elicited data.

In \ili{Sumbawa}, as reported in \cite[174]{shias13b}, sentences can be formed using only a predicate which may carry clitics indicating, for instance, tense and subject/agent. One or several participants of the situation expressed by the predicate can be omitted
when their \isi{identity} is clear from the previous utterance or the utterance situation; see examples (\ref{fell1})-(\ref{drank3}).

\begin{multicols}{3}
	\ea\label{fell1}
	\gll ka=\textbf{ku}=teri'\\  
	\textsc{pst=1sg}=fall\\ 
	\glt `I fell.'
	\z
	
	\ea\label{fell2}
	\gll ka=\textbf{mu}=teri'\\
	\textsc{pst=2sg}=fall\\
	\glt `You fell.'
	\z
	
	\ea\label{fell3}
	\gll ka=teri'\\
	\textsc{pst}=fall\\
	\glt `He/She/They fell.'
	\z
\end{multicols}

\begin{multicols}{3}
	\ea\label{drank1}
	\gll ka=\textbf{ku}=inum\\
	\textsc{pst=1sg}=drink\\
	\glt `I drank it.\footnote{Something the \isi{referent} of which is clear from the context.}'
	\z
	
	\ea\label{drank2}
	\gll ka=\textbf{mu}=inum\\
	\textsc{pst=2sg}=drink\\
	\glt `You drank it.'
	\z
	
	\ea\label{drank3}
	\gll ka=\textbf{ya}=inum\\
	\textsc{pst}=3=drink\\
	\glt `He/She/They drank it.'
	\z
\end{multicols}

\noindent The occurrence of the subject/agent \isi{clitic} is determined by three factors: (i) (in-)\-transi\-ti\-vi\-ty of the main verb, (ii) the person of the single core intransitive par\-ti\-ci\-pant (S) or transitive agent (A), and (iii) the overt realization or absence of the argument and its position. The information structure of a \isi{clause} has an influence on (iii) and, therefore, on the occurrence of the \isi{clitic}.

\begin{table}
	\caption{Sumbawa independent and clitic pronouns}
	\label{tab:Sumbawa_pronouns}
	\begin{tabularx}{.8\textwidth}{>{\scshape}Xlcl} 
		\lsptoprule
		{\upshape Person \& Number} &Free \isi{pronoun}& & Clitic \isi{pronoun}\\  
		\midrule
		1sg       & aku  & & ku= \\
		1pl.incl  & kita & \multirow{2}{*}{\bigg\}} & \multirow{2}{*}{tu=} \\ 
		1pl.excl  & kami & & \\ 
		2sg       & kau & & mu= \\ 
		2pl       & n\`en\`e & & n\`en\`e= \\ 
		3         & nya & & ya= \\
		\lspbottomrule
	\end{tabularx}
\end{table}

\noindent Table~\ref{tab:Sumbawa_pronouns} shows a list of the \isi{clitic} and independent pronouns. There is no distinction between singular and plural in the third person. As can be seen in examples (\ref{fell1})-(\ref{drank3}), the first and second person \isi{clitic pronoun} may (but need not) occur both on intransitive and on transitive verbs. The third person \isi{clitic} \textit{ya} exhibits an exceptional behavior in that it can only occur on transitive, e.g.\ (\ref{drank3}), but not on intransitive verbs, e.g.\ (\ref{fell3}). Sentence (\ref{fell3alt}), in which \textit{ya} co-occurs with an intransitive verb, is not accepted by the speakers.\footnote{Unlike some other \ili{Austronesian} languages, such as \ili{Acehnese} \citep{durma85} and \ili{Tukang Besi} \citep{donma96}, which exhibit so-called split-intransitivity, \ili{Sumbawa} does not make a syntactic distinction between agentive intransitive verbs, which typically denote a volitional action, and non-agentive intransitive verbs, which typically denote a non-volitional situation. Thus, intransitive verbs such as \textit{barari} `run' or \textit{barnang} `swim' that denote a volitional action behave in the same way as the non-volitional verb \textit{teri} `fall', which we use as a representative of intransitive verbs throughout this section.}

\ea[*]{\label{fell3alt}
	\gll ka=ya=teri’\\
	\textsc{pst}=3=fell\\
	\glt Intended meaning: `He/She/They fell.’}
\z

\noindent The person of the transitive patient (P) is never coded on the predicate, but can be expressed by use of an independent \isi{pronoun} (or, of course, a lexical NP). Pronouns and lexical NPs behave syntactically in the same way; see (\ref{tari1}).

\ea\label{tari1}
\gll ya=tari	aku {\USSlash} kau {\USSlash} nya {\USSlash} tau	 nan\\
3=wait \textsc{1sg} {} \textsc{2sg} {} 3 {} person that\\
\glt `She is waiting for me/you/him/that person.'
\z

\noindent The constituents for S, A, or P may principally occur before or after the predicate. However, first and second person pronominal S arguments cannot occur after the predicate. Sentences (\ref{fell1S}) and (\ref{fell2S}) are not permitted by the speakers.

\begin{multicols}{2}
	\ea[*]{\label{fell1S}
		\gll ka=ku=teri’	aku\\
		\textsc{pst=1sg}=fall	\textsc{1sg}\\
		\glt Intended meaning: `I fell.’}
	\z
	
	\ea[*]{\label{fell2S}
		\gll ka=mu=teri’	kau\\
		\textsc{pst=2sg}=fall	\textsc{2sg}\\
		\glt Intended meaning: `You fell.’}
	\z
\end{multicols}

\noindent By contrast, the S constituent for the third person can occur after the predicate, for instance, as the NP \textit{t\'od\'e nan} `that child' in sentence (\ref{fell3np}) or \textit{nya} `he/she/they' in (\ref{fell3pro}).

\begin{multicols}{2}
	\ea\label{fell3np}
	\gll ka=teri’	t\'od\'e	nan\\
	\textsc{pst}=fall	child	that\\
	\glt `That child fell’.
	\z
	
	\ea\label{fell3pro}
	\gll ka=teri'	nya\\
	\textsc{pst}=fall	3\\
	\glt `He/She/They fell’. 
	\z
\end{multicols}

\noindent \ili{Sumbawa} does not exhibit inflectional case marking. Yet, as for the post-predicate constituents, the case frame exhibits an ergative pattern, in that A occurs in a PP form with the preposition \textit{ling}, as shown in examples (\ref{drank1Ppp})-(\ref{drank3Ppp}), while an S constituent, as shown in examples (\ref{fell3np}) and (\ref{fell3pro}) above, and P, as given in the NP \textit{kawa nan} `that coffee' in sentences (\ref{drank1Ppp})-(\ref{drank3Ppp}), occurs in the form of an NP.\footnote{Strictly speaking, it is only for the third person \isi{referent} that the ergative pattern is fully observed since, as we already showed in examples (\ref{fell1S}) and (\ref{fell2S}), a first or second person S may not occur in post-predicate position. Another split in alignment, which we already mentioned in connection with (\ref{fell3}) and (\ref{drank3}), is observed with regard to \isi{clitic} pronouns. The distribution of the first and second person \isi{clitic} exhibits the accusative pattern in that it may code the person of S and A, but not that of P, while that of the the third person \isi{clitic} exhibits the ergative pattern in that it may only code the person of A, not that of S and P. A split in alignment determined by person is commonly observed cross-linguistically, cf.\  \citet{siean13}.}

\ea\label{drank1Ppp}
\gll ka=ku=inum kawa nan ling aku\\
\textsc{pst=1sg}=drink coffee that by \textsc{1sg}\\
\glt `I drank that coffee.'
\z

\ea\label{drank2Ppp}
\gll ka=mu=inum kawa nan ling kau\\
\textsc{pst=2sg}=drink coffee that by \textsc{2sg}\\
\glt `You drank that coffee.’
\z 

\ea\label{drank3Ppp}
\gll ka=ya=inum 	kawa 	nan	ling	nya Amin\\
\textsc{pst}=3=drink	coffee	that	by	Mr. Amin\\
\glt `Amin drank that coffee.’
\z

\noindent S, A and P may all occur before the predicate, as  (\ref{Sfell1})–(\ref{Pdrank3pp}) show. All arguments are expressed by an NP (i.e.\ without preposition) in this position. (We omit the second person, since it is analogous to the first.)

\ea\label{Sfell1}
\gll aku ka=teri'\\
\textsc{1sg} \textsc{pst}=fall\\
\glt `I fell.'
\z

\ea\label{Sfell3}
\gll t\'od\'e nan ka=teri’\\
child that \textsc{pst}=fall\\
\glt `That child fell.’
\z

\ea\label{Adrank1P}
\gll aku	ka=inum		kawa	nan\\
\textsc{1sg}	\textsc{pst}=drink	coffee	that\\
\glt `I drank that coffee.’
\z

\ea\label{npdrank3P}
\gll nya Amin	ka=inum		kawa	nan\\
Mr. Amin	\textsc{pst}=drink	coffee	that\\
\glt `Amin drank that coffee.’
\z

\ea\label{prodrank3P}
\gll nya	ka=inum		kawa	nan\\
3		\textsc{pst}=drink	coffee	that\\
\glt `He/She/They drank that coffee.’
\z

\ea\label{Pdrank3pp}
\gll kawa	nan	ka=ya=inum	ling	nya Amin\\
coffee	that	\textsc{pst}=3=drink	by	Mr. Amin\\
\glt `Amin drank that coffee.’
\z

\noindent Whenever the S or A argument occur in pre-predicate position, there is no \isi{clitic pronoun} on the predicate. In other words, the \isi{clitic} and the pre-predicate argument are in complementary distribution. Examples (\ref{Sclfell1})-(\ref{Acldrank3P}) are all ungrammatical. Note that the post-predicate PP indicating A obligatorily occurs with the \isi{clitic pronoun}, as shown in examples (\ref{drank1Ppp})-(\ref{drank3Ppp}) and (\ref{Pdrank3pp}) above.

\ea[*]{\label{Sclfell1}
	\gll aku ka=ku=teri’\\
	\textsc{1sg} \textsc{pst=1sg}=fall\\
	\glt Intended meaning: `I fell.’}
\z

\ea[*]{\label{Acldrank1P}
	\gll aku ka=ku=inum kawa\\
	\textsc{1sg} \textsc{pst=1sg}=drink coffee\\}
\z

\ea[*]{\label{Acldrank3P}
	\gll nya Amin ka=ya=inum kawa\\
	Mr. Amin \textsc{pst}=3=drink coffee\\}
\z

\noindent Only one NP may be fronted at a time; examples (\ref{APdrank3}) and (\ref{PAdrank3}), with two fronted NPs, are not permitted by speakers. This is the only purely syntactic constraint on the relative order of verb and arguments. 

\ea[*]{\label{APdrank3}
	\gll nya	Amin	kawa	nan		inum {\hspace{3em}{\USOParen}A P V{\USCParen}}\\
	Mr.	Amin	coffee	that	drink {}\\
	\glt Intended meaning: `Amin is drinking that coffee.’}
\z

\ea[*]{\label{PAdrank3}
	\gll kawa	nan 	nya	Amin	inum {\hspace{3em}{\USOParen}P A V{\USCParen}}\\
	coffee	that	Mr.	Amin	drink\\
	\glt Intended meaning: `Amin is drinking that coffee.’}
\z

\noindent The following tables summarize the complex correlation between the type and location of the S/A argument on the one hand and the occurrence of a \isi{clitic pronoun} on the verb on the other hand, for intransitive (\tabref{tab:intransitive}) and transitive (\tabref{tab:transitive}) predicates. The symbols have the following meaning: - ``no \isi{clitic pronoun}", + ``a \isi{clitic pronoun} occurs", *~``ungrammatical construction". The examples from above are indicated in brackets.

\begin{table}
	\caption{Intransitive predicates: occurrence of clitics, subject type and subject position (examples indicated in brackets)}
	\label{tab:intransitive}
	\begin{tabularx}{\textwidth}{Xlll} 
		\lsptoprule
		& 1\textsuperscript{st} person & 2\textsuperscript{nd} person & 3\textsuperscript{rd} person \\ 
		\midrule
		No overt subject    & + (\ref{fell1}) &+ (\ref{fell2}) & - (\ref{fell3}) \\
		Post-predicate subject NP   & * (\ref{fell1S}) & * (\ref{fell2S}) & - (\ref{fell3np}, \ref{fell3pro})\\
		Post-predicate ``subject PP" & * & * & * \\
		Pre-predicate subject NP    & - (\ref{Sfell1}) & - & - (\ref{Sfell3})\\
		\lspbottomrule
	\end{tabularx}
\end{table}

\begin{table}
	\caption{Transitive predicates: occurrence of clitics, agent type and agent position}
	\label{tab:transitive}
	\begin{tabularx}{\textwidth}{Xlll} 
		\lsptoprule
		& 1\textsuperscript{st} person & 2\textsuperscript{nd} person & 3\textsuperscript{rd} person \\ 
		\midrule
		No overt agent   & + (\ref{drank1}) & + (\ref{drank2}) & + (\ref{drank3}) \\
		Post-predicate agent NP   & * & *  & * \\
		Post-predicate agent PP \textit{(ling)} & + (\ref{drank1Ppp}) & + (\ref{drank2Ppp})& + (\ref{drank3Ppp}, \ref{Pdrank3pp}) \\
		Pre-predicate agent NP    & - (\ref{Adrank1P}) & - & - (\ref{npdrank3P}, \ref{prodrank3P})\\
		\lspbottomrule
	\end{tabularx}
\end{table}

\section{Information structure theory and questions under discussion}\label{sect:theory}

In this section, we change from the grammatical description of \ili{Sumbawa} to information structure theory, which, as we will show, will later help us account for the patterns described in the previous section. The central problem when studying the morphosyntactic realization of information structure in any language is the avoidance of circularity. Although variation in the \isi{constituent order} of sentences often goes hand in hand with a variation of the \isi{discourse} context in which these sentences occur, there is, at the surface, no syntactic focus-marking strategy that would universally apply in all situations to all languages. First of all, many languages have an \textit{in-situ} focus as a default option, but some also have the possibility to explicitly realize focus clause-finally, by sorting constituents according to their \isi{information status} or by applying extraposition. The opposite strategy, available in many languages, is the fronting or clefting of the focal constituent. Yet other languages exhibit no or very few morphosyntactic reflexes of information structure but instead mainly rely on \isi{intonation} or \isi{prosodic phrasing} to mark a focus that syntactically remains \textit{in-situ}. For overviews and comparisons see e.g.\ \citet{zubma98,bueda09,skost10,gueto15,ferca16}. In general, it is methodologically questionable to use morphosyntactic (or prosodic) indicators for the identification of information structure if, at the outset, little is known about information-structure marking in a particular language.

\newpage 
Information structure is a pragmatic phenomenon, i.e.\ it is usually described in terms of meaning categories which relate to context. This is what accounts for both its universality and its somewhat enigmatic status. Throughout the literature, \textit{focus} has -- very inconsistently -- been described as the answer to a question \citep{pauhe80, halmi67, robcr12}, as alternative-evoking \citep{rooma92}, asserted \citep{Lambrecht1994}, new \citep{halmi67, schro99}, identificational \citep{ekika98},  exhaustive \citep{rooro08} or \isi{contrastive} \citep{katjo11}. A \textit{topic} is usually defined -- in a rather noncommittal way -- as that which a sentence is ``about" \citep{hocch58, strpe64, reita81, mcnlo98, jacjo01, Krifka2008}, and the notion of \textit{\isi{contrastive} topic} has received a sophisticated characterization in terms of a speaker strategy \citep{bueda03} to answer a complex question by working through a list of subquestions. Neither of these concepts easily connects to naturally occurring text or speech data. In order to study the information structure of language data gathered in fieldwork, people have, therefore, typically reverted to question-answer scenarios or other semi-spontaneous methods like the use of pictures or stories \citep[cf.][]{skost06} to elicit controlled, information-structurally relevant material.

In the current study, we will apply a new annotation method based on implicit\linebreak \textit{questions under discussion} (QUDs), cf.\ \citet{stuch89, kupja95, robcr12, beada17}, which enables a pragmatic information-structure analysis of textual fieldwork data. The method, so far, has mainly been applied to \ili{French} and \ili{German} \citep[e.g.][]{riear18}.\footnote{See also \citetv{latan18} on the use of QUDs for the description of \isi{voice selection} in \ili{Tagalog}.}
Among the aforementioned definitions of focus, we adopt the one that takes focus to be the answer to the current question under discussion.

When investigating dialogues, as we do in this study on \ili{Sumbawa}, explicit questions, of course, allow us to study the morphosyntactic realization of the background-focus divide. Examples of overt (Q)uestion-(A)nswer pairs are provided in the following \ili{Sumbawa} examples (\ref{verum1}).\footnote{In this example, the question is signalled by a rising final \isi{intonation}.}

\begin{exe}
	\ex\label{verum1}
	\begin{xlist}[Q\textsubscript{{7}}:]
		\exi{Q\textsubscript{{7}}:}
		\gll ada ke cabe nana\\
		exist \textsc{q} chili over.there\\
		\glt `Was there chili over there (in Japan)?'
		\exi{A\textsubscript{7}:}
		\gll {{\ob}ada}	{{\ob}si{\cb}\focus} {{\ob}cabe,{\cb}\topic{\cb}\sq}\\
		\hphantom{[}exist \textsc{ptc} \hphantom{[}chili\\
		\glt `There was chili.'
	\end{xlist}
\end{exe}

\noindent (\ref{verum1}) is an example of a polarity question, which triggers a \textit{yes-} or \textit{no-}answer. In this case the addressee answers by means of a full clause, containing a verum focus, which is realized on the particle \textit{si} within the predicate.

As for the information-structural categories and markup we assume the following definitions: the focus (F) of a sentence is that part which answers the question, whereas the remainder (the information already contained within the question itself) is the background. Following \citet{reita81,jacjo01} or \citet{Krifka2008} we, furthermore, assume that aboutness topics (T) are referential entities (terms) that are properly contained in (and, therefore, potentially smaller than) the background.\footnote{Usually, the ``comment" is taken to be the complement of a topic. Since we see no use for such a category in our current work, we will ignore it.} Following standard assumptions in Alternative Semantics \citep{rooma92,bueda08,bueda16}, answers (consisting of an obligatory focus and an optional background) are so-called \textit{focus domains}, which are marked by the \sq\ symbol.\footnote{In \citet[85ff.]{rooma92} \sq\ operators are used, among other purposes, to establish question-answer coherence: both questions and focused answers represent sets of alternatives, and the \sq\ operator identifies the question set as a proper subset of the focus alternatives.}
Choice questions (also known as alternative questions) presented as disjunctions may trigger a constituent focus, which is why the answer in (\ref{choiceq})\footnote{The alternation \textit{satoan-katoan} seems due to dialectal variation.}is not a polarity focus \isi{clause}.

\begin{exe}\label{choiceq}
	\ex
	\begin{xlist}[Q\textsubscript{{27}}:]
		\exi{Q\textsubscript{{27}}:}
		\gll no	roa	tu=satoan	tris	ke, atau	ada	waya	tu=satoan\\
		\textsc{neg}	possible	\textsc{1pl}=ask	always	\textsc{q}	or	exist	time	\textsc{1pl}=ask\\
		\glt `Can we never ask, or is there a (proper) time to ask?'
		\exi{A\textsubscript{{27}}:}
		\gll {{\ob}{\ob}Ada}	{waya{\cb}\focus}	{tu=katoan{\cb}\sq}\\
		\hphantom{[[}exist	time	\textsc{1pl}=ask\\
		\glt `There is a time for asking.'
	\end{xlist}
\end{exe}

\noindent Since both alternatives contain the same verbal element \textit{tu=satoan} `\textsc{1pl}=ask', this element is assumed to figure as the background of the answer. (We will say more on this below, e.g.\ example (\ref{parallel1}), when discussing parallel statements.) 

Not all assertions, even in dialogues, are made in response to explicit questions, and not all explicit questions in dialogues receive a direct answer. In both cases we need to develop an idea of how to reconstruct the \textit{intended} question, lest a large part of the assertions of the \isi{discourse} will remain unanalyzed.
Following \citet{robcr12} and earlier work, e.g.\ by \citet{stuch89} or \citet{kupja95}, we assume that every \isi{assertion} in a text is actually the answer to a (typically implicit) \textit{question under discussion (\isi{QUD})}. Thus, if we manage to determine the \isi{QUD} of an \isi{assertion}, its information-structural analysis (focus, background, aboutness topics) will follow, as in the case of overt questions. 

The non-trivial part, of course, consists in the identification of the QUDs of predominantly monological passages of text. A solution to the problem is described at length in \citet{riear17b}, and we will only shortly sketch it here. First, we segment the text into separate speech acts (which are predominantly assertions). Besides orthographic sentence boundaries we also split coordinated phrases into separate semantic assertions (under the assumption that a coordination is an effective way of communicating a series of statements at one stroke). However, argument clauses will not be separated from their matrix clauses, since this would result in ungrammatical sentence fragments. For instance, (\ref{segment1})-(\ref{segment4}) is the result of the segmentation of a short paragraph, where a letter A is assigned to each separate \isi{assertion}.

\begin{exe}
	\ex\label{segment1}
	\begin{xlist}
		\exi{A:}
		\gll seandai	parak	ne	ketakit	nya\\
		if	close	\textsc{itj}	scared	3\\
		\glt `If someone approaches, they get scared.'
	\end{xlist}
\end{exe}

\begin{exe}
	\ex\label{segment2}
	\begin{xlist}
		\exi{A:}
		\gll min	ada	tau	asing	ne	ketakit	nya	sate	yang de	sate\\
		if	exist	person	foreign	\textsc{itj}	scared	3	want	like \textsc{rel}	want\\
		\glt `If there is a foreigner, they are scared that he is after them like a	kidnapper (lit.\ \textit{a wanter})'
	\end{xlist}
\end{exe}

\begin{exe}
	\ex\label{segment3}
	\begin{xlist}
		\exi{A:}
		\gll barari\\
		run\\
		\glt `They run away.'
	\end{xlist}
\end{exe}

\begin{exe}
	\ex\label{segment4}
	\begin{xlist}
		\exi{A:}
		\gll nan	luk	model	nan\\
		that	way	character	that\\
		\glt `That's their character.'
	\end{xlist}
\end{exe}

\noindent As we can see, the conditional clauses in (\ref{segment1}) and (\ref{segment2}) are not split into parts because this would lead to ungrammaticality: neither of the clauses with the conjunctions \textit{seandai} `if' in (\ref{segment1}) and \textit{min} `if' in (\ref{segment2}) is grammatical in isolation. 

The next step consists in the identification of the QUDs. The determination of QUDs in non-parallel text passages follows three basic principles (for a justification see \citealt{riear17b} and references therein):
\begin{itemize}
	\item[(i)] \textsc{Q-A-Congruence} requires that the \isi{QUD} for an \isi{assertion} targets an actual constituent of the \isi{assertion}. (It is not permitted to choose a question which does not ask for at least one constituent of the answer.)
	\item[(ii)] \textsc{Q-Givenness} says that implicit QUDs can only contain linguistic material that has been mentioned or is salient in the current \isi{discourse} context.
	\item[(iii)] Finally, \textsc{Maximize-Q-Anaphoricity} determines that all given material that occurs in the \isi{assertion} is in fact mentioned in the question, thereby producing a maximally cohesive \isi{discourse} (and, at the same time, a focus that is as narrow as possible).
\end{itemize}

\noindent For instance, in a small toy \isi{discourse} A\textsubscript{0}-A\textsubscript{1}, the implicit \isi{QUD} Q\textsubscript{1} is the one shown in example (\ref{toy1})\footnote{The indentation symbol (>) before A\textsubscript{1} indicates that A\textsubscript{1} is subordinate to Q\textsubscript{1} in the corresponding \isi{discourse} tree, whereas A\textsubscript{0} and Q\textsubscript{1} are at the same level. On different tree representation formats see \citet{riear18}.} and its tree representation in \figref{fig:toy1}.

\begin{exe}
	\ex\label{toy1}
	\begin{xlist}
		\exi{A\textsubscript{0}:} Last Sunday, we had a picnic in the park.
		\exi{Q\textsubscript{1}:} \textbf{\{What about the picnic?\}}
		\exi{> A\textsubscript{1}:} {[[The picnic]\topic\ [consisted of sandwiches]\focus.]\sq}
	\end{xlist}
\end{exe}

\begin{figure}
	\begin{tikzpicture}
	\Tree [.{\dots}
	{A\textsubscript{0}}
	[.{Q\textsubscript{1}}
	{A\textsubscript{1}}
	]
	]
	\end{tikzpicture}
	\caption{Tree representation of example (\ref{toy1})}
	\label{fig:toy1}
\end{figure}

\noindent In contrast to Q\textsubscript{1}, the questions in (\ref{invalidQ}) are all invalid because each of them violates at least one of the \isi{QUD} constraints.

\begin{exe}
	\ex\label{invalidQ}
	\begin{xlist}
		\ex\label{invalida} Q: \{Which park was it?\}\hfill\textsc{\#Q-A-Congruence}
		\ex\label{invalidb} Q: \{What about the sandwiches?\}\hfill\textsc{\#Q-Givenness}
		\ex\label{invalidc} Q: \{What happened in the park?\}\hfill\textsc{\#Maximize-Q-Anaphoricity}
	\end{xlist}
\end{exe}

\noindent The question in (\ref{invalida}) violates \textsc{Q-A-Congruence} because it cannot have A\textsubscript{1} as its answer. (\ref{invalidb}) contains the discourse-new expression \textit{sandwiches}, thereby violating \textsc{Q-Givenness} and, finally, (\ref{invalidc}) and in fact all questions in (\ref{invalidQ}) violate \textsc{Maximize-Q-Anaphoricity}, because neither of them contains the expression \textit{picnic}, which is given in A\textsubscript{1}, since it already occured in A\textsubscript{0}.

However, a violation of the principle of \textsc{Q-Givenness} is acceptable in two cases. The first one is the beginning of a text. Here, the implicit \isi{QUD} sometimes needs to contain linguistic material that is discourse-new (for details on this process of \textit{accommodation}, see citet{riear17b}).

The other acceptable violation is found in connection with parallelisms. When a \isi{QUD} is answered by a series of structurally analogous assertions, the assertions are defined to be parallel. In that case, the corresponding \isi{QUD} is allowed to contain the shared content, even though it may not be given in the preceding \isi{discourse}. For example in (\ref{parallel1}) the verb \textit{ate} is not given in the context, and \textsc{Q-Givenness} would predict question Q\textsubscript{2}. However, the double occurrence of the verb within two parallel statements (marked as A\textsubscript{{2.1'}} and A\textsubscript{{2.1''}}) licenses the formulation of the more specific \isi{QUD} Q\textsubscript{{2.1}}. Note that it is generally the case that a sub-\isi{QUD} like this is always entailed by the more general one, i.e.\ every answer to Q\textsubscript{{2.1}} is at the same time an answer to Q\textsubscript{2}, even though Q\textsubscript{{2.1}} determines a narrower (here: object) focus.

\begin{exe}
	\ex\label{parallel1}
	\begin{xlist}
		\exi{A\textsubscript{1}:} We had a lot of fun at our picnic.
		\exi{Q\textsubscript{2}:} \{What did we do at the picnic?\}\hfill (\isi{QUD} licensed by \textsc{Q-Givenness})
		\exi{> Q\textsubscript{{2.1}}:} \{What did we \textbf{eat} at the picnic?\}\hfill (sub-\isi{QUD}, def.\ by parallelism)
		\exi{≫ A\textsubscript{{2.1'}}:} {[[We]\topic\ \textbf{ate} [delicious sandwiches]\focus]\sq}
		\exi{≫ A\textsubscript{{2.1''}}:} {and [[we]\topic\ even \textbf{ate} [sashimi]\focus]\sq}.
	\end{xlist}
\end{exe}

\begin{figure}
	\begin{tikzpicture}
	\Tree [.{\dots}
	{A\textsubscript{1}}
	[.{Q\textsubscript{2}}
	[.{Q\textsubscript{{2.1}}}
	{A\textsubscript{{2.1'}}}
	{A\textsubscript{{2.1''}}}
	]
	]
	]
	\end{tikzpicture}
	\caption{Tree representation of example (\ref{parallel1})}
	\label{fig:toy2}
\end{figure}

\noindent \citet{riear17b}, following \citet{bueda03}, also postulate a more complex case of parallelism that includes \isi{contrastive} topics (indexed as \textsc{ct} in the annotation). This type involves two (or more) assertions, which are contrasted against each other at two different positions. An example is given in (\ref{ctf1}).

\begin{exe}
	\ex\label{ctf1}
	\begin{xlist}
		\exi{Q\textsubscript{3}:} \{Who ate what at the picnic?\}
		\exi{> Q\textsubscript{{3.1}}:} \{What did my girlfriend eat?\}
		\exi{≫ A\textsubscript{{3.1}}:} {[[My girlfriend]\ctopic\ ate [delicious sandwiches]\focus]\sq}
		\exi{> Q\textsubscript{{3.2}}:} \{What did I eat?\}
		\exi{≫ A\textsubscript{{3.2}}:} {and [[I]\ctopic\ even ate [sashimi]\focus]\sq.}
	\end{xlist}
\end{exe}

\begin{figure}
	\begin{tikzpicture}
	\Tree [.{Q\textsubscript{3}}
	[.{Q\textsubscript{{3.1}}}
	{A\textsubscript{{3.1}}}
	]
	[.{Q\textsubscript{{3.2}}}
	{A\textsubscript{{3.2}}}
	]
	]
	\end{tikzpicture}
	\caption{Tree representation of example (\ref{ctf1})}
	\label{fig:toy3}
\end{figure}

\noindent Analogous to example (\ref{parallel1}) above, the two statements Q\textsubscript{{3.1}} and Q\textsubscript{{3.2}} in (\ref{ctf1}) are parallel, because they both describe events of eating, and they answer the same \isi{QUD} Q\textsubscript{3}. However, the difference to (\ref{parallel1}) is that the assertions in (\ref{ctf1}) vary in two positions (here: subject and object), and that Q\textsubscript{3} contains two \isi{interrogative} pronouns instead of one. This is typically a sign that Q\textsubscript{3} is not answered directly but broken down into subquestions about a list of salient individuals (here: my girlfriend and I). We shall assume that the elements of the answers which answer Q\textsubscript{3} but which are \isi{backgrounded} with regard to one of the subquestions are so-called \textit{\isi{contrastive} topics}, whereas those elements which answer both the superquestion Q\textsubscript{3} and one of the subquestions are \textit{foci}.

Example (\ref{undeliberately}), cited from the \ili{Sumbawa} conversation, involves \textsc{ct}s. The conversation topic is about the difficulties   Muslims face in connection with \ili{Japanese} food. In this example, conditional clauses represent the topical options \citep{haijo78,ebech14} that the speakers are contrasting against each other.\footnote{Note that speaker I.\ is repeating the statement made by speaker S., which is indicated by a `+' on the index. More on this in example (\ref{repetition1}) below.} 

\begin{exe}
	\ex \label{undeliberately}
	\begin{xlist}
		\exi{Speaker S:}
		\exi{Q\textsubscript{{3}}:} \{What if we ate pork unwittingly or knowingly?\}
		\exi{> Q\textsubscript{{3.1}}:} \{What if we did it unwittingly?\}
	\end{xlist}
	\begin{xlist}[≫ A\textsubscript{{3.1+}}:]
		\exi{≫ A\textsubscript{{3.1}}:}
		\gll {{\ob}o ba} {terang,{\cb}\nai} {{\ob}{\ob}lamin} nongka {tu=sangaja{\cb}\ctopic} {{\ob}terang{\cb}\nai} {{\ob}no} {sikuda{\cb}\focus{\cb}\sq}\\
		{\hphantom{[}\textsc{itj}} clear \hphantom{[[}if \textsc{neg}.\textsc{pst} \textsc{1pl}=act.deliberately \hphantom{[}clear \hphantom{[}\textsc{neg} problem\\
		\glt `But it is clear, if we didn’t act deliberately, it is clear that it is not a problem.’
		\exi{Speaker I:\hfill}
		\exi{≫ A\textsubscript{{3.1+}}:}
		\gll ya {{\ob}{\ob}lamin} nongka {tu=sangaja{\cb}\ctopic} {{\ob}no} {sikuda{\cb}\focus{\cb}\sq}\\
		yes \hphantom{[[}if \textsc{neg}.\textsc{pst} \textsc{1pl}=act.deliberately \hphantom{[}\textsc{neg} problem\\
		\glt `Yes, if we didn’t act deliberately, it’s no problem.’
	\end{xlist}
	\begin{xlist}
		\exi{> Q\textsubscript{{3.2}}:} \{What if we did it knowingly?\}
	\end{xlist}
	\begin{xlist}[≫ A\textsubscript{{3.1+}}:]
		\exi{≫ A\textsubscript{{3.2}}:}
		\gll tapi {{\ob}{\ob}lamin} {ka=tu=to{\cb}\ctopic} {{\ob}kan{\cb}\nai} {{\ob}no} {roa{\cb}\focus} {{\ob}dean} {nan{\cb}\topic{\cb}\sq}\\
		but \hphantom{[[}if \textsc{pst}=\textsc{1pl}=know \hphantom{[}you.know \hphantom{[}\textsc{neg} comfortable \hphantom{[}that that\\
		\glt `but if we knew, we are not comfortable with that.’
	\end{xlist}
\end{exe}

\noindent Finally, an \isi{assertion} may contain phrases which neither form part of the \isi{QUD} nor contribute to answering it, i.e.\ they are neither \isi{backgrounded} nor focused. We classify such phrases as \textit{non-at-issue \textsc{(nai)} material}, cf.\ \citet{tonju13,andsc15}. More specifically, we are talking about triggers of conventional implicatures \citep{potch05}, which include appositions, non-restrictive relative clauses, parentheticals, speaker-oriented adverbs, evidentials as well as adjunct phrases/clauses, cf.\ \citet{riear17b} for more details and precise definitions. A few of these, namely the speaker- or addressee-oriented phrases \textit{it is clear that} or \textit{you know} are contained in (\ref{undeliberately}). Another example is shown in (\ref{nai1}). 

\begin{exe}
	\ex\label{nai1}
	\begin{xlist}
		\exi{Q\textsubscript{{19.1}}:} \{As for work, what attitude do people in Japan have?\}
	\end{xlist}
	\begin{xlist}[> A\textsubscript{{19.1}}:]
		\exi{> A\textsubscript{{19.1}}:}
		\gll {{\ob}\textbf{maklum}{\cb}\nai} {{\ob}tau} {{\ob}ana{\cb}\ctopic} {{\ob}mega} {disiplin{\cb}\focus{\cb}\sq} {{\ob}\textbf{sia}{\cb}\nai}\\
		you.know people over.there very disciplined \textsc{2pl.hon}\\
		\glt `\textbf{You know, my friend,} people over there are very disciplined.'
	\end{xlist}
\end{exe}


\section{Corpus analysis}\label{sect:corpusAnalysis}

\subsection{The nature of conversational data}

The data we are examining is the transcript of a \ili{Sumbawa} conversation titled \textit{Memory of Japan}, which consists of approximately 1500 words. The conversation took place between two \ili{Sumbawa} speakers and Shiohara, one of the authors. The main speaker, I., had been working in Japan as a trainee, and is talking about his experiences to Speaker S., one of his friends, and to Shiohara, in reply to their questions.\footnote{A transcription and translation will be made available in \citet{shiasta}. The video recording of the conversation, available on YouTube, was created in collaboration with the \ili{Sumbawa} Literature Association (Yayasan Bungaku \ili{Sumbawa}), of which the two speakers and Shiohara are members, cf.\ \url{https://www.youtube.com/watch?v=D8gOyhJi1VI}} 

In the conversation, Speaker I.\ talks about \ili{Japanese} people, culture and society. The nature of the conversation has a crucial influence on the syntactic and pragmatic properties of the \isi{discourse}; many clauses have a first person A or S. Since the speaker talks about a number of situations as if they were general facts rather than his personal experiences, he often uses the generic first person plural form for the A and S arguments, especially the \isi{clitic} \textit{tu}; many 
verbal clauses carry the \isi{clitic pronoun} \textit{tu} or the independent \isi{pronoun} \textit{kita}. (Compare \tabref{tab:Sumbawa_pronouns} in \sectref{sect:grammarOutline}.) There are only few
examples of a first person singular A or S, in which the speaker talks about a specific episode he was part of.

\subsection{Focus identification} 

The conversation contains about 100 assertions (or \isi{discourse} units), which were analyzed according to the guidelines in \citet{riear17b}. The general tendency observed in the conversation is that a \isi{referent} or a situation that has been mentioned in the directly preceding \isi{discourse} tends not to be expressed in the current \isi{clause}. Thus, entities in the background are often left unexpressed, and a substantial amount of assertions are only formed by the focal expression itself. Nevertheless, because of the implicit referents, these are strictly speaking not all-focus assertions but rather elliptical clauses with narrow \isi{verb focus}. A\textsubscript{{38}} in example (\ref{ellipsis1}) is an example of an \isi{assertion} that only consists of a focused expression. The silent A argument for this \isi{clause} is co-referential with the S argument \textit{tau} in A\textsubscript{{35}} and \textit{nya} in A\textsubscript{{36}}.

\begin{exe}
	\ex\label{ellipsis1}
	\begin{xlist}[A\textsubscript{{35}}:]
		\exi{A\textsubscript{{35}}:}
		\gll {Toyama} {ta} mega tertutup {tau}\\
		Toyama this very unsocial person\\
		\glt `In Toyama people are very unsocial.'
	\end{xlist}
	\begin{xlist}
		\exi{Q\textsubscript{{36}}:}	\{What about the people in Toyama?\}
	\end{xlist}
	\begin{xlist}[> A\textsubscript{{36'}}:]
		\exi{> A\textsubscript{{36'}}:}
		\gll {{\ob}seandai} {parak{\cb}\nai} ne {{\ob}{\ob}ketakit{\cb}\focus} {{\ob}nya{\cb}\topic{\cb}\sq}\\
		\hphantom{[}for.example close \textsc{itj} \hphantom{[[}scared 3\\
		\glt `For instance, if someone approaches, they get scared.' 
		\exi{} {[\dots]}
	\end{xlist}
	\begin{xlist}
		\exi{> Q\textsubscript{{38}}:}	\{What do they do in such a case?\}
	\end{xlist}
	\begin{xlist}[≫ A\textsubscript{{38}}:]
		\exi{≫ A\textsubscript{{38}}:}
		\gll {{\ob}{\ob}\textbf{barari}{\cb}\focus{\cb}\sq}\\
		\hphantom{[[}run\\
		\glt `They run away.'
	\end{xlist}
\end{exe}

\noindent A better example of a \isi{sentence focus}, which shows that the default \isi{constituent order} of intransitives is V S, is shown in A\textsubscript{{14}} of (\ref{allfocus1}).

\begin{exe}
	\ex\label{allfocus1}
	\begin{xlist}
		\exi{Q\textsubscript{{14}}:}	\{What is the situation in winter?\} 
	\end{xlist}
	\begin{xlist}[> A\textsubscript{{14}}:]
		\exi{> A\textsubscript{{14}}:}
		\gll {{\ob}{\ob}kukir}	sarea	apa	{godong{\cb}\focus{\cb}\sq}\\
		\hphantom{[[}withered	all	what	leaf\\
		\glt `All leaves are withered.'
	\end{xlist}
\end{exe}

\noindent The assertions can be classified according to the syntactic range of their focus constituents. In what follows, we will show examples of different types of focus, and we will examine the relative order of focus, background and \isi{contrastive topic} in clauses with a narrow (or mid-size) focus. Our main interest in this research is the relative order of a predicate and its argument(s), and the effect this has on the presence of a \isi{clitic}. 

\subsection{Information structure and constituent order}\label{sect:isqud}

In an \isi{argument focus} \isi{clause}, the argument always precedes the background.\footnote{A fronted NP may be followed by the \isi{discourse} particle \textit{si}, as observed in elicited sentences like (\ref{Ssifell3}). The typical usage of \textit{si} is the indication of a contrast or a situation that is counter to an expectation. This usage was not detected in the current conversation. 
	
	\ea\label{Ssifell3}
	\gll t\'od\'e 	nan	si	ka=teri’\\
	child	that	\textsc{ptc}	\textsc{pst}=fell\\
	\glt `That child fell.’
	\z
	
	\noindent The precise conditions under which \textit{si} occurs other than as a polarity particle need to be addressed in future research.} Two examples of preverbal (patient) \isi{argument focus} can be observed in A\textsubscript{{1.1''}} in (\ref{argfoc1}) and A\textsubscript{{4.1'}} in (\ref{argfoc2}).

\begin{exe}
	\ex\label{argfoc1}
	\begin{xlist}
		\exi{Q\textsubscript{{1.1}}:} \{What (else) can we not eat?\}
	\end{xlist}
	\begin{xlist}[> A\textsubscript{A.1''}:]
		\exi{> A\textsubscript{A.1''}:}
		\gll atau {{\ob}{\ob}apa} de terkait ke bawi {nan{\cb}\focus} no tu=bau {kakan{\cb}\sq}\\
		or \hphantom{[[}what \textsc{rel} related with pig that \textsc{neg} \textsc{1pl}=can eat\\
		\glt `also we cannot eat whatever is related to the pig.'
	\end{xlist}
	\ex\label{argfoc2}
	\begin{xlist}
		\exi{Q\textsubscript{{4.1}}:} \{How much salary do we get?\}
	\end{xlist}
	\begin{xlist}[> A\textsubscript{{4.1'}}:]
		\exi{> A\textsubscript{{4.1'}}:}
		\gll {{\ob}{\ob}balu-pulu-ribu} yen {perbulan{\cb}\focus} {tu=terima{\cb}\sq}\\
		{\hphantom{[[}eighty thousand} yen per.month \textsc{1pl}=receive\\
		\glt `we are getting eighty thousand yen per month'
	\end{xlist}
\end{exe}

\noindent This generalization holds throughout our data. 
The only apparent exception occurs in the form of a repetition. The three assertions in (\ref{repetition1}) below convey the same \isi{assertion} and are, therefore, labelled as A\textsubscript{{25.1}}, A\textsubscript{{25.1+}} and A\textsubscript{{25.1++}}. 

\begin{exe}
	\ex\label{repetition1}
	\begin{xlist}
		\exi{Q\textsubscript{{25.1}}:\hfill} \{Which privacy-related issues -- said the friend -- can you not ask a \ili{Japanese} person about?\}
	\end{xlist}
	\begin{xlist}[> A\textsubscript{{25.1++}}:]
		\exi{> A\textsubscript{{25.1}}:}
		\gll {{\ob}umpama{\cb}\nai} {{\ob}{\ob}umir,{\cb}\focus{\cb}\sq}\\
		{\hphantom{[}for.example} \hphantom{[[}age\\
		\glt `For example, about the age,'
		\exi{> A\textsubscript{{25.1+}}:}
		\gll {{\ob}{\ob}umir} {nan{\cb}\focus} na coba-coba {katoan{\cb}\sq} {{\ob}ling{\cb}\nai}\\
		\hphantom{[[}age that \textsc{neg.imp} try ask \hphantom{[}\textsc{hearsay}\\
		\glt `don't try to ask about the age, he said.'
		\exi{}{[\dots]}
		\exi{> A\textsubscript{{25.1++}}:}
		\gll {No} roa \textbf{tu=katoan} \textbf{umir} tau jepang {ta,}\\
		\textsc{neg} possible \textsc{1pl}=ask age person Japan this\\
		\glt `It is not possible that we ask about the age of a \ili{Japanese} person'
	\end{xlist}
\end{exe}

\noindent Repetitions, in some sense, defy the rules of proper information transfer because, from a logical point of view, a speaker should not assert something which is already implied by the common ground -- which would amount to making an all-given (or focus-less) \isi{assertion}. Intuitively however, speakers repeat themselves precisely because they are not confident that their interlocutor has already accepted their previous statement. It is, therefore, reasonable to assume that, under normal circumstances, a structurally identical repetition has the same information structure as its previous mention. However, in example (\ref{repetition1}), the focus argument \textit{umir} `age' first occurs in an elliptical environment (A\textsubscript{25.1}), it then precedes the background in the second statement A\textsubscript{{25.1+}} (an imperative), while in the third, assertive, statement A\textsubscript{{25.1++}} it suddenly follows the verb. We believe, since this is the only case of a potential focus argument following a \isi{backgrounded} verbal predicate, that, by means of the repetition, the speaker frees himself from the pragmatic requirements of the \isi{discourse} context; thereby performing a kind of ``context reset" with regard to the contents of his statement.  It is, therefore, possible that the actual information structure of A\textsubscript{{25.1++}} is that of an all-new \isi{assertion}, in which the patient argument follows the verb, as shown below. 

\begin{exe}
	\ex\label{repetition2}
	\begin{xlist}
		\exi{Q:} \{What is the way things are?\}
	\end{xlist}
	\begin{xlist}[> A:]
		\exi{> A:}
		\gll {{\ob}{\ob}No} roa tu=katoan umir tau jepang {ta,{\cb}\focus{\cb}\sq}\\
		\hphantom{[[}\textsc{neg} possible \textsc{1pl}=ask age person Japan this\\
		\glt `It is not possible that we ask the age of a \ili{Japanese} person'
	\end{xlist}
\end{exe}

\noindent
When an argument is \isi{backgrounded} it follows the predicate in most  cases, as shown in in A\textsubscript{2.1’’’’+} (a \isi{backgrounded} subject argument following the focused stative verb \textit{mogang} `to be light’). Another example of the same kind is A\textsubscript{{36'}} / (\ref{ellipsis1}), discussed above.

\begin{exe}
	\ex\label{easyWork}
	\begin{xlist}
		\exi{Q\textsubscript{{2.1}}:} \{How was the work like?\}
	\end{xlist}
	\begin{xlist}[> A\textsubscript{{2.1''''+}}:]
		\exi{}{[\dots]}
		\exi{> A\textsubscript{{2.1''''+}}:}
		\gll {{\ob}{\ob}mogang} {mogang{\cb}\focus} {{\ob}boat{\cb}\topic{\cb}\sq}\\
		\hphantom{[[}light light work\\
		\glt `The work is easy, so easy.'
	\end{xlist}
\end{exe}

\subsection{Contrastive topics}

Apart from the conditional clauses previously discussed in (\ref{undeliberately}), \sectref{sect:theory}, we do not find in our data any paramount examples of overtly realized pairs of \isi{contrastive} topics, i.e.\ those marked on referring argument expressions. What we do find is that, in a few cases, an argument that was given somewhere in the earlier \isi{discourse} but didn't play a role in the preceding sentence, occurs in pre-predicate position. The question is now whether these entities should be assigned to the background (therefore representing aboutness topics) or not. For instance, in A\textsubscript{{19.1}} of (\ref{nai1}), repeated in (\ref{nai1b}), the phrase \textit{people over there} refers to a group given in a distant part of the \isi{discourse} context. Obviously, the speaker is not using this phrase as a simple \isi{aboutness topic}. Instead, what he does is implicitly contrast the \isi{referent} with a different group of people (presumably, those at home). Since we need to account for the topic change anyway, we make use of the structure introduced in \sectref{sect:theory} in connection with \isi{contrastive} topics, i.e. a question-subquestion tree of which, however, only the first half is overtly realized, compare \cite[Sect.\ 5.2]{riear17b}. Since there is no essential discourse-structural difference between such a ``new" topic and the \isi{contrastive} topics introduced above, we also tentatively assign (the \isi{contrastive} element of) the referring expression the label \textsc{ct}, in the style of \cite[526]{bueda03}. 

\begin{exe}
	\ex\label{nai1b}
	\begin{xlist}
		\exi{Q\textsubscript{{19}}:} \{As for work, what attitude do different people have?\}
		\exi{> Q\textsubscript{{19.1}}:} \{As for work, what attitude do people in Japan have?\}
	\end{xlist}
	\begin{xlist}[≫ A\textsubscript{{19.1}}:]
		\exi{≫ A\textsubscript{{19.1}}:}
		\gll {{\ob}\dots{\cb}} {{\ob}tau} {{\ob}ana{\cb}\ctopic} {{\ob}mega} {disiplin{\cb}\focus{\cb}\sq} {{\ob}\dots{\cb}}\\
		{} people over.there very disciplined\\
		\glt `People over there are very disciplined.'
	\end{xlist}
\end{exe}

\noindent In the very similar case of A\textsubscript{{17.1}} / (\ref{singlect1}), the \isi{demonstrative} \textit{denan} `that' is introduced by the form \textit{lamin}, tentatively glossed `as for' here, and refers to `the time to get up', which is the P argument of the predicate \textit{determine}. What the speaker wants to express is that this particular issue was not regulated, although other things were, like e.g.\ the working hours. Again, we represent this in the form of a \isi{contrastive} / non-continuous topic, which is merely implicitly \isi{contrastive}. 

\begin{exe}
	\ex\label{singlect1}Speaker S.:
	\begin{xlist}
		\exi{Q\textsubscript{{17}}:} \{Which things were regulated?\}
	\end{xlist}
	\begin{xlist}[> Q\textsubscript{{17.1}}:]
		\exi{> Q\textsubscript{{17.1}}:}
		\gll {{\ob}\dots{\cb}} waya mleng tunung apa waya, ada ke ya=tentukan kalis-kalis ana e\\
		{} time get.up sleep what time exist \textsc{q} 3=determine from over.there \textsc{itj}\\
		\glt `Is there a regulation when to get up from sleep over there?'
	\end{xlist}
	\begin{xlist}[≫ A\textsubscript{{17.1}}:]
		\exi{Speaker I.:}
		\exi{≫ A\textsubscript{{17.1}}:}
		\gll {{\ob}Lamin} {{\ob}denan{\cb}\ctopic} {{\ob}no=soka{\cb}\focus} {ya=tentukan,{\cb}\sq}\\
		\hphantom{[}as.for that \textsc{neg.pst} 3=determine\\
		\glt `As for that, they didn’t prescribe it.'
	\end{xlist}
\end{exe}

\noindent Note that examples like these represent a challenge to the rules currently formulated in the guidelines of \citet{riear17b}, since the lack of an overt alternative makes the identification of such ``implicitly \isi{contrastive} topics" very difficult. Obviously, the QUD-tree framework  needs to be enhanced with clearer rules concerning the identification of such non-continuous, implicitly \isi{contrastive} topics. \cite[][136]{kropa04} provides a brief and clear discussion on a distinction of topic types similar to those observed here. He discusses that ``(T)he topic of a sentence, when it is the same as the topic of the preceding sentence, needs no special marking. It can often be referred to with an unstressed \isi{pronoun}, an agreement marker (as in the `pro-drop' languages), or even by being omitted entirely (`zero anaphora').  But in certain contexts the topic may require more elaborate marking. This may happen when there is a change in topic, a contrast between one topic and another, or a choice among several available topics". See also \cite[][117ff.]{Lambrecht1994} for a detailed discussion on the distinctions among various types of topics, and \textit{Centering Theory} \citep{walma98} for a closely related \isi{discourse} approach to topics (or ``backward-looking centers").


\subsection{Other focus types found in the data}

In this section, we mention a few other focus types found in our data, before returning to our main issues defined in \sectref{sect:grammarOutline}, \isi{constituent order} and clitics. As observed in example (\ref{verum1}), \sectref{sect:isqud},  repeated here as (\ref{verum1b}), a verum focus is realized on the particle \textit{si}. 

\begin{exe}
	\ex\label{verum1b}
	\begin{xlist}[Q\textsubscript{{7}}:]
		\exi{Q\textsubscript{{7}}:}
		\gll ada ke cabe nana\\
		exist \textsc{q} chili over.there\\
		\glt `Was there chili over there (in Japan)?'
	\end{xlist}
	\begin{xlist}[> A\textsubscript{7}:]
		\exi{> A\textsubscript{7}:}
		\gll {{\ob}ada}	{{\ob}si{\cb}\focus} {{\ob}cabe,{\cb}\topic{\cb}\sq}\\
		\hphantom{[}exist \textsc{ptc} \hphantom{[}chili\\
		\glt `There was chili.'
	\end{xlist}
\end{exe}

\noindent The negative counterpart, a falsum focus, can be seen in A\textsubscript{{17.1}} of example (\ref{singlect1}) in the previous section or, using a slightly different negative element, in A\textsubscript{{12}} of (\ref{falsum1}). In both cases focus is realized on the negative element, which is  followed by the predicate.

\begin{exe}
	\ex\label{falsum1}
	\begin{xlist}[Q\textsubscript{{12}}:]
		\exi{Q\textsubscript{{12}}:}
		\gll nka kadu gita cabe pang Jepang ana{\USQMark}\\
		\textsc{neg.pst} ever see chili in Japan over.there\\
		\glt `Have you never seen chili over there in Japan?'
	\end{xlist}
	\begin{xlist}[> A\textsubscript{{12}}:]
		\exi{> A\textsubscript{{12}}:}
		\gll {{\ob}{\ob}nka{\cb}\focus} kadu gita {{\ob}puin} {cabe{\cb}\topic{\cb}\sq}\\
		\textsc{\hphantom{[[}neg.pst} \hphantom{[}ever see tree chili\\
		\glt `I have never seen a chili tree.'
	\end{xlist}
\end{exe}

  
\noindent In the case of focus on an NP modifier, the modifier retains its canonical (post-nominal) position.\footnote{The situation is different with numerals. In A\textsubscript{{12}} of example (\ref{modfocus1}), the focal numeral \textit{lima} `five' precedes the head noun. 
	
	\begin{exe}
		\ex (Speaker talking to himself:) 
		\begin{xlist}[Q\textsubscript{{12}}:]
			\exi{Q\textsubscript{{12}}:} 
			\gll jadi ada pida kamar\\
			then exist how.many room\\
			\glt `Then, how many rooms were there?'
		\end{xlist}
		\begin{xlist}[> A\textsubscript{{12}}:]  
			\exi{> A\textsubscript{{12}}:}
			\gll {{\ob}{\ob}lima{\cb}\focus} {kamar{\cb}\sq} {{\ob}amen} no {salah{\cb}\nai}\\
			\hphantom{[[}five room \hphantom{[}if \textsc{neg} wrong\\
			\glt `There were five rooms, if I am not wrong.'
		\end{xlist}
	\end{exe}
	
	\noindent As \citet{shias14} suggests, the position of a numeral relative to its head noun varies according to its focal status. If a numeral is not in focus, it follows the head noun, like a modifier does, though the data we examined in this article do not include any such example.
} 

\begin{exe}
	\ex\label{modfocus1}
	\begin{xlist}
		\exi{Q\textsubscript{8}} \{What kind of chili was there?\}
		\exi{> Q\textsubscript{{8.1}}:} \{What about a lot of the chili?\}
	\end{xlist}
	\begin{xlist}[≫ A\textsubscript{{8.1}}:]
		\exi{≫ A\textsubscript{{8.1}}:}
		\gll cuma {{\ob}de} ka {{\ob}peno{\cb}\ctopic} cabe ne, \textbf{cabe} {{\ob}\textbf{instan}{\cb}\focus{\cb}\sq}\\
		only \hphantom{[}\textsc{rel} \textsc{pst} many chili \textsc{itj} chili \hphantom{[}instant\\
		\glt `It is just that a lot of chili was instant chili.'
	\end{xlist}
\end{exe}

\noindent Compare this to \isi{assertion} A\textsubscript{{33}} in example (\ref{nomodfocus1}), in which a modified phrase with the same \isi{word order} occurs in an adjunct \isi{clause} (classified as non-at-issue) that is all-new and, hence, not narrowly focused.

\begin{exe}
	\ex\label{nomodfocus1}
	\begin{xlist}
		\exi{Q\textsubscript{{33}}:}	\{What do we have to do?\}
	\end{xlist}
	\begin{xlist}[> A\textsubscript{{33'}}:]
		\exi{> A\textsubscript{{33'}}:}
		\gll {{\ob}min} sate parak ke \textbf{tau} \textbf{sowai} {\textbf{gera}{\cb}\nai} {{\ob}\dots{\cb}} {{\ob}harus} {{\ob}tu=tahan} {rasa-ate{\cb}\focus{\cb}\sq}\\
		\hphantom{[}if want approach with person woman beautiful {} \hphantom{[}have.to \hphantom{[}\textsc{1pl}=control emotion\\
		\glt `If we want to approach a beautiful woman, we need to control our emotions.'
	\end{xlist}
\end{exe}

\section{Grammatical relations, person, agreement and constituent order}\label{sect:clitic}
\subsection{Intransitives}

In \sectref{sect:grammarOutline}, we introduced the basic morphosyntactic rules of \ili{Sumbawa} -- based on elicited data -- in which A, S and P are realized. In this section, we will look at how information structure, observed in our annotated data, can explain some of the variation found.

According to the rules postulated in \tabref{tab:intransitive} in \sectref{sect:grammarOutline}, a first (or second) person S is expected to be realized either as a pronominal \isi{clitic} or as a pre-predicate argument. The first case is exemplified by (\ref{cl1pl}). The tendency that an entity that was mentioned in the directly preceding utterance or that is generally salient is not overtly expressed applies to all types of arguments. Thus, when the \isi{referent} of 
S or A is topical, it is expected to be realized (only) as a \isi{clitic pronoun} within the predicate in first and second person. This is the case with the first person plural subject in (\ref{cl1pl}).

\begin{exe}
	\ex\label{cl1pl}
	\begin{xlist}
		\exi{Q\textsubscript{{33}}:}{\{What do we have to do?\}}
	\end{xlist}
	\begin{xlist}[> A\textsubscript{{33''}}:]
		\exi{> A\textsubscript{{33''}}:}
		\gll {{\ob}harus} {{\ob}\textbf{tu}=bersabar} {benar{\cb}\focus{\cb}\sq} {{\ob}ampa{\cb}\nai}\\
		\hphantom{[}have.to \hphantom{[}\textsc{1pl}=patient \hphantom{[}really \textsc{itj}\\
		\glt `We need to be really patient.'\\
	\end{xlist}
\end{exe}

\noindent An unexpected \textsc{ct}-case, which runs counter to the predictions, is A\textsubscript{{23.1}} in example (\ref{doublemarked1}), in which S is doubly marked. 

\begin{exe}
	\ex\label{doublemarked1}
	\begin{xlist}
		\exi{Q\textsubscript{{23}}:} \{Who does what in the morning?\}
	\end{xlist}
	\begin{xlist}
		\exi{> Q\textsubscript{{23.1}}:} \{What do we (Muslims) do in the morning?\}
	\end{xlist}
	\begin{xlist}[≫ A\textsubscript{{23.1}}:]
		\exi{≫ A\textsubscript{{23.1}}:}
		\gll {{\ob}kan{\cb}\nai} {{\ob}{\ob}\textbf{kita}{\cb}\ctopic} {{\ob}\textbf{tu}=sembayang-subu{\cb}\focus} {{\ob}dunung,{\cb}\topic{\cb}\sq}\\
		\hphantom{[}you.know \hphantom{[[}\textsc{1pl.incl} \hphantom{[}\textsc{1pl}=do.early-morning-prayer \hphantom{[}before\\
		\glt `As you know, we first do the early morning prayer.'
	\end{xlist}
\end{exe}

\noindent As mentioned in \sectref{sect:grammarOutline}, this construction was not accepted by participants in an elicitation task when asked for their grammatical judgment, and was therefore classified as non-canonical, if not ungrammatical. In many languages, the complementary distribution of \isi{clitic} and argument has made way for a more canonical agreement system, in which the two forms co-occur. We may see the phenomenon in \ili{Sumbawa} in an ongoing process of a grammatical change.\footnote{In some \ili{Austronesian} languages spoken in eastern Indonesia, such as \ili{Kambera} \citep[][69--70]{klama98} and \ili{Bima} \citep[][92]{satma09}, this type of co-occurrence of \isi{clitic pronoun} and argument is permitted.}
At the present stage of our research, though, we do not have enough data to say more about this. 

The third person S is expected to be realized as either a pre-predicate argument, post-predicate argument, or left unexpressed, and all three types are observed in the conversation. Examples of post-predicate realization already discussed are the broad-focus example (\ref{allfocus1}) as well as the narrow-verb-focus clauses A\textsubscript{{36'}} / (\ref{ellipsis1}) and (\ref{easyWork}). A further example is (\ref{postPredicate}). Zero marking was shown in A\textsubscript{{38}} / (\ref{ellipsis1}) and can also be witnessed in (\ref{zeroMarked}). Finally, pre-predicate subject realization was demonstrated in connection with contrastive-topichood in (\ref{nai1b}). As expected, there is never a \isi{clitic} on any of these predicates.

\begin{exe}
	\ex\label{postPredicate}
	\begin{xlist}
		\exi{Q\textsubscript{{41.1}}:}{\{How did the person who came look like?\}}
	\end{xlist}
	\begin{xlist}[> A\textsubscript{{41.1}}:]
		\exi{> A\textsubscript{{41.1}}:}
		\gll {{\ob}{\ob}barangkali,} {lenge} {benar{\cb}\focus} {{\ob}ade} datang {ta{\cb}\topic{\cb}\sq}\\
		\hphantom{[}maybe \hphantom{[}ugly really \hphantom{[}\textsc{rel} come this\\
		\glt `Maybe the person who came was really ugly.'
	\end{xlist}
	\ex\label{zeroMarked}
	\begin{xlist}
		\exi{Q\textsubscript{{21}}:}{\{How was it to start working at 7:15 a.m.?\}}
	\end{xlist}
	\begin{xlist}[> A\textsubscript{{21+}}:]
		\exi{> A\textsubscript{{21+}}:}
		\gll {{\ob}{\ob}disiplin} {benar{\cb}\focus{\cb}\sq}\\
		\hphantom{[[}disciplined really\\
		\glt `It was really disciplined.'
	\end{xlist}
\end{exe}

\subsection{Transitives}

According to \tabref{tab:transitive}, \sectref{sect:grammarOutline}, the A argument can be realized as either a pre-predicate argument, a post-predicate PP, or simply as a pronominal \isi{clitic}. The post-predicate PP may co-occur with a \isi{clitic}, while a pre-predicate argument may not. However, post-predicate PPs were not found in our data.\footnote{It can be observed that the post-predicate PP argument typically occurs in narrative sentences, indicating a series of events in which several people take turns at being the  agent. In such cases, an agent, despite being salient in the preceding utterance, needs to be expressed for the sake of disambiguation. (\ref{dogPoop}) is a typical example cited from a folktale in \cite[][214]{jonjc34}.
	
	\begin{exe}
		\ex\label{dogPoop}
		\gll ya=beang mo \textbf{ling} \textbf{Salam} lako guru; meng ka=ya=kakan mo \textbf{ling} \textbf{guru}: \hspace{3em} {``e,} nyaman tai asu {ode"}\\
		3=give \textsc{ptc} by Salam to teacher; when \textsc{pst}=3=eat \textsc{ptc} by teacher {} \textsc{itj} delicious shit dog small\\
		\glt `Salam gave (it) to the teacher. When the teacher ate (it), (he said): ``Ah, the shit of the small dog is delicious."'
	\end{exe}
	
	\noindent In our present conversation data the speaker is mainly talking about his own experiences. There are no comparable cases that would require agent disambiguation. Therefore, a broader ana\-ly\-sis of narrative data, like folktales, is required to verify this particular function of the post-predicate PP.}

The two remaining options, pre-predicate argument and \isi{clitic}, are expected to occur in complementary distribution. In our data, clitics are particularly frequent in combination with a first person A, as shown, for instance, in (\ref{nomodfocus1}) above. Another example, which shows a \isi{clitic} on the predicate of the main \isi{clause} and a preverbal \isi{pronoun} (no \isi{clitic}) in the embedded \isi{clause}, is given in (\ref{kitaEmbedded}).

\begin{exe}
	\ex\label{kitaEmbedded}
	\begin{xlist}
		\exi{Q\textsubscript{{3}}:}{\{What should we do towards the \ili{Japanese} people concerning the consumption of pork?\}}
	\end{xlist}
	\begin{xlist}[> A\textsubscript{{3+}}:]
		\exi{> A\textsubscript{{3+}}:}
		\gll {{\ob}harus} {{\ob}tu=bada{\cb}\focus} {{\ob}nya{\cb}\topic} luk \textbf{kita} no bau kakan {nan{\cb}\sq}\\
		\hphantom{[[}have.to \textsc{1pl}=tell 3 that \textsc{1pl.incl} \textsc{neg} can eat that\\
		\glt `We must tell them that we can't eat that.'
	\end{xlist}
\end{exe}

\noindent Apparently, the speaker felt the need to realize the -- given -- agent of the embedded \isi{clause} by use of the overt \isi{pronoun} \textit{kita}. Note that in this construction the entire \isi{clause} is \isi{backgrounded}. It is very likely that the occurrence of the preverbal \isi{pronoun} is meant to convey a so-called secondary contrast,\footnote{The current \isi{QUD} framework does not consider the phenomenon of \textit{second-occurrence focus} \citep{beada11,bueda15,baust16} and will, therefore, not mark any secondary foci within the background, since there is no nesting of focus domains.} i.e.\ ``We must \textit{tell} them that\textit{we} can't eat that (although \textit{you} can)." So, once more, the entity is not just topical but implicitly \isi{contrastive}.

Furthermore, according to the rules spelled out in \sectref{sect:grammarOutline}, clitics (in all persons) are expected to appear instead of ``zero-marked" transitive clauses, but this is clearly not always the case in our conversation. There are several instances of transitive clauses in which neither a \isi{clitic} nor an overt agent occurs, like (\ref{transNoClitic}).

\begin{exe}
	\ex\label{transNoClitic}
	\begin{xlist}
		\exi{Q\textsubscript{9}:}{\{What kind of chili did the speaker see how often?\}}
		\exi{> Q\textsubscript{{9.1}}:}{\{What kind of chili did the speaker see rarely?\}}
	\end{xlist}
	\begin{xlist}[≫ A\textsubscript{{9.1}}:]
		\exi{≫ A\textsubscript{{9.1}}:}
		\gll {{\ob}{\ob}jarang{\cb}\ctopic} \textbf{gita} cabe {{\ob}mira{\cb}\focus{\cb}\sq}\\
		\hphantom{[[}rarely see chili \hphantom{[}red\\
		\glt `I rarely saw red chili.'
	\end{xlist}
\end{exe}

\noindent There are also a few preverbal third person A arguments, like (\ref{thirdPre}), which all seem to mark a \isi{contrastive} (or at least a non-continuous) topic. 

\begin{exe}
	\ex\label{thirdPre}
	\begin{xlist}
		\exi{Q\textsubscript{1}:} {\{What did the people in various places make?\}}
		\exi{> Q\textsubscript{{1.1}}:} {\{What did the people at the factory make?\}}
	\end{xlist}
	\begin{xlist}[≫ A\textsubscript{{1.1}}:]
		\exi{≫ A\textsubscript{{1.1}}:} 
		\gll {{\ob}Jadi} tau {{\ob}ana{\cb}\ctopic} pina, {{\ob}mara} leng tau {nana,{\cb}\nai} pina {{\ob}peralatan} untuk {elepator{\cb}\focus{\cb}\sq}\\
		\hphantom{[}then person \hphantom{[}over.there make \hphantom{[}like word person over.there make \hphantom{[}tool parts elevator\\
		\glt `At the time, people there, as they said, were making parts for an elevator.'
	\end{xlist}
\end{exe}

\noindent Finally, the P argument may occur as either pre-predicate argument or post-predicate argument. In all cases found, pre-predicate P arguments either correspond to the focused constituent (\isi{argument focus}), as in examples (\ref{argfoc1}) and (\ref{argfoc2}), or to the \isi{contrastive topic}, A\textsubscript{{17.1}} / (\ref{singlect1}), while a post-predicate P argument either belongs to the background, (\ref{falsum1}) or (\ref{kitaEmbedded}), is included in a (wide or mid-size) focus spanning the entire VP constituent, (\ref{nomodfocus1}), or represents the focus in a \textsc{ct-f} combination, as shown in (\ref{transNoClitic}) and (\ref{thirdPre}). A zero marked P argument is, as one would expect, one that is always clear from the context, see A\textsubscript{{11}} in example (\ref{implicitP}).

\begin{exe}
	\ex\label{implicitP}
	\begin{xlist}
		\exi{Q\textsubscript{{11}}:} \{How did we get there?\}
	\end{xlist}
	\begin{xlist}[> A\textsubscript{{11}}:]
		\exi{> A\textsubscript{{11}}:}
		\gll {{\ob}{\ob}olo{\cb}\focus} {{\ob}pang} {nan{\cb}\topic{\cb}\sq}\\
		put.in place that\\
		\glt `(They) put  \textbf{(us)} in that place.'
	\end{xlist}
\end{exe}


\section{Summary and conclusions}\label{sect:summary}
\largerpage
\begin{table}[b]
	\caption{Intransitive predicates: constituent order, subject realization and information structure}
	\label{tab:isintransitive}
	\begin{tabularx}{\textwidth}{XCC} 
		\lsptoprule
		& 1\textsuperscript{st}/2\textsuperscript{nd} person & 3\textsuperscript{rd} person \\ \midrule
		no overt       & \multicolumn{2}{c}{[[V]\focus]\sq} \\ 
		subject        &  cf.\ (\ref{cl1pl}) & (\ref{ellipsis1}, \ref{zeroMarked}) \\ \midrule
		& \multirow{2}{*}{*} & [[V~S]\focus]\sq\\
		post-predicate &                    & (\ref{allfocus1}, )\\ \cmidrule{2-3}
		S-NP           & \multirow{2}{*}{*} & [[V]\focus\ [S]\topic]\sq\\
		&                    & (\ref{ellipsis1}, \ref{easyWork}, \ref{postPredicate})\\ \midrule
		& \multicolumn{2}{c}{[[S]\focus\ V]\sq}\\
		pre-predicate  & \multicolumn{2}{c}{\textit{(not attested)}} \\ \cmidrule{2-3}
		S-NP           & \multicolumn{2}{c}{[[S]\ctopic\ [V]\focus]\sq} \\
		& (\ref{doublemarked1}) & (\ref{nai1b})\\
		\lspbottomrule
	\end{tabularx}
\end{table}

\begin{table}[t]
	\caption{Transitive predicates: constituent order, agent realization and information structure}
	\label{tab:istransitive}
	\begin{tabularx}{\textwidth}{XCC} 
		\lsptoprule
		& 1\textsuperscript{st}/2\textsuperscript{nd} person & 3\textsuperscript{rd} person \\ \midrule
		& \multicolumn{2}{c}{[[V]\focus]\sq}   \\
		& \multicolumn{2}{c}{(\textit{not attested)}} \\ \cmidrule{2-3}
		& \multicolumn{2}{c}{[[V P]\focus]\sq}  \\
		no                   & (\ref{nomodfocus1}) & \\ \cmidrule{2-3}
		overt                & \multicolumn{2}{c}{[[P]\focus\ V]\sq} \\
		agent                & (\ref{argfoc1}, \ref{argfoc2}) &  \\ \cmidrule{2-3}
		& \multicolumn{2}{c}{[[P]\ctopic\ [V]\focus]\sq} \\
		& & cf.\ (\ref{singlect1}) \\ \cmidrule{2-3}
		& \multicolumn{2}{c}{[[V]\focus\ [P]\topic]\sq} \\
		& (\ref{kitaEmbedded}) & \\ \midrule
		post-predicate       & \multicolumn{2}{c}{\textit{(not enough data)}}\\
		A-PP \textit{(ling)} & \multicolumn{2}{c}{} \\  \midrule
		pre-predicate        & \multicolumn{2}{c}{[[A]\ctopic\ V [P]\focus]\sq}  \\
		A-NP                 & \multicolumn{2}{c}{(\ref{thirdPre})}  \\
		& \multicolumn{2}{c}{\textit{(no other patterns attested)}} \\
		\lspbottomrule
	\end{tabularx}
\end{table}

\noindent We have investigated conversational corpus data from \ili{Sumbawa} and demonstrated how information structure, determined on the basis of con\-textu\-al-prag\-ma\-tic constraints and the concept of questions under discussion, is related to con\-sti\-tu\-ent-or\-de\-ring patterns known from earlier morphosyntactic descriptions of the language. In order to present the findings in a more systematic way, we revisit \tabref{tab:intransitive} (now \tabref{tab:isintransitive}) and \tabref{tab:transitive} (now \tabref{tab:istransitive}) from \sectref{sect:grammarOutline} and include our new findings about information structure.

Summarizing, we can say that, by default (i.e.\ in the case of \isi{broad focus}), the arguments follow the verbal predicate. Arguments occur pre-predicatively whenever any of them is narrowly focused and there is no \isi{contrastive topic} present. In addition to that, a \isi{contrastive topic} may also occur pre-predicatively. We found several candidates for potential \isi{contrastive} topics. Since in all the cases the contrast can only be construed implicitly, we acknowledge that some readers may prefer to use the terminology `non-continuous topic' instead, although we see no essential difference between a \isi{contrastive topic} (with or without an overt alternative) and the shift to a \isi{new topic}, since from a pragmatic point of view, both require the formulation of a new \isi{QUD} \textit{about} this topical entity. In either case there is no topical continuity. As we said, both focus and \isi{contrastive}/non-continuous topic may occur in the pre-predicate argument slot. This correspondence is very common cross-linguistically, since ``the beginning of a sentence is a highly prominent position" \citep[][139]{kropa04} and both \textsc{ct} and \textsc{f} share the property of evoking alternatives \citep{bueda03}. However, the two pragmatic roles are usually distinguished phonologically in \ili{Sumbawa}: a pre-predicate NP in focus occurs with a falling \isi{intonation}, while the \isi{contrastive topic} expression occurs with a rising \isi{intonation}. 

We have demonstrated how the use of the QUD-tree framework can be successfully applied to conversational data from a lesser-studied language like \ili{Sumbawa}. This is the first comprehensive application of this method to a non-Eu\-ro\-pe\-an language.\footnote{Compare also \citetv{latan18}.} Not only were we able to provide an in-depth analysis of the in\-for\-mat\-ion-struc\-ture system of the language but we could also show that the method is generally a valuable instrument to explain some of the con\-sti\-tu\-ent-or\-der variation and the occurrence of clitics. With this method, we gain access to the information structure phenomena and corresponding syntactic properties contained in various kinds of text and dialogue data. This means that we are now able to investigate more potential variation on more natural data than when studying information structure only in elicitation experiments. Some of the variety may be explained by semantic-pragmatic factors, while others may reflect an ongoing grammatical change in \ili{Sumbawa}. 

\largerpage 
We are confident that the phenomena and variety discovered in this study are the beginning of a series of new discoveries in the field of pragmatic data analysis on larger and more varied natural corpora. 

\section*{Acknowledgements}
The authors would like to thank two anonymous reviewers for their comments. We are very grateful to Sonja Riesberg for her encouragement and support. We also would like to express our gratitude for the members of the \ili{Sumbawa} Literature Association (Yayasan Bungaku \ili{Sumbawa}); Syamsul Bahri (SMU3), Syamsul Bahri (SMK1), Ade Erma Lestari (SMU3), Iwan Irwansyah (SMA3), Sambahadi Kurniawan (SMU3), Nyoman Purayasa (SMU3), who worked with Asako Shiohara in recording  and annotating the conversation we used as our data in our research. The financial support by the \ili{German} Science Foundation (DFG), Collaborative Research Center (SFB) 732 Stuttgart, Project A6,  the Linguistic Dynamics Science Project 3 (LingDy3) at ILCAA, Tokyo University of Foreign Studies (2016--2022AY), and JSPS Grants-in-Aid for Scientific Research: JP15K02472 are kindly acknowledged.

\section*{Abbreviations}

\begin{multicols}{2}
	\begin{tabbing}
		glossgloss \= \kill
		\textsc{\sq} \> \isi{focus domain}\\
		\textsc{ct} \> \isi{contrastive topic}\\
		\textsc{excl} \> exclusive\\
		\textsc{f}  \> focus\\
		\textsc{hon} \> honorific\\
		\textsc{imp} \> imperative\\
		\textsc{incl} \> inclusive\\
		\textsc{itj} \> interjection\\
		\textsc{nai} \> non-at-issue\\
		\textsc{neg} \> negation\\
		\textsc{pl} \> plural\\
		\textsc{pst} \> past tense\\
		\textsc{q} \> question particle\\
		\textsc{qud} \> question under discussion\\
		\textsc{rel} \> relativizer\\
		\textsc{sg} \> singular\\
		\textsc{t}  \> \isi{aboutness topic}\\
	\end{tabbing}
\end{multicols}

\sloppy
\printbibliography[heading=subbibliography,notkeyword=this]

\end{document}
