\documentclass[output=paper
,modfonts
,nonflat]{langsci/langscibook}  
\title{Constituent order and information structure in Indonesian discourse} 
\author{Dwi Noverini Djenar\affiliation{The University of Sydney}}
% \chapterDOI{} %will be filled in at production

\ChapterDOI{10.5281/zenodo.1402545}
% \epigram{}

\abstract{This study draws on theories of information structure to examine the findings from Cumming’s (1991) study showing changing preferences in constituent order in Indonesian. Cumming found that predicate-initial clauses, an important grammatical resource for encoding events in Classical Malay, are diminishing in Indonesian, and agentive clauses are now preferred. Based on data from fictional discourse and television reports, three clause structures are examined: [\textit{meN}-V], [\textit{di}-V-\textit{nya}] and [\textit{ia} V]. I show that the use of [\textit{meN}-V] to denote successive events and to mark the climactic portion of a story episode illustrate most dramatically the preference for agentive clauses in Indonesian. I also argue that [\textit{di}-V-\textit{nya}] remains an important resource for encoding events but its pragmatic function seems to have weakened. This clause type is now mainly used to mark a new focus, drawing the addressee’s attention to a particular event or series of events. Meanwhile, objective voice [\textit{ia} V] is also used to encode events; however, unlike [\textit{di}-V-\textit{nya}] which is structurally different from [\textit{meN}-V], the use of [\textit{ia} V] alongside agentive clause structure [\textit{ia meN}-V] creates an impression of structural symmetry and can serve two goals simultaneously: marking event and signalling a new focus. It could be that, as the pragmatic force of [\textit{di}-V-\textit{nya}] is weakening, [\textit{ia} V] is increasingly preferred for marking focal events.}

\begin{document}

\maketitle

\section{\label{s:djenar:1}Introduction}

In her work on \isi{constituent order} in \ili{Malay}, \citet{Cumming1991} makes an important observation regarding syntactic change in this language, arguing that the predominant VSO order in \ili{Classical Malay} has gradually given way to SVO order in \ili{Indonesian} and \ili{Malaysian}.\footnote{Cumming uses the term “Modern \ili{Indonesian}” to refer to contemporary Standard \ili{Indonesian}, and ‘\ili{Malay}’ to the language of which \ili{Indonesian} and \ili{Malaysian} national languages are varieties (\citeyear[2]{Cumming1991}).}  She points out that predicate-initial clauses, an important grammatical resource for encoding “\isi{eventiveness}” in \ili{Classical Malay}, are diminishing in \ili{Indonesian}, and agentive clauses are now preferred \citep[199]{Cumming1991}. Eventiveness, in her term, refers to “a related class of phenomena having to do more with characteristics of an event than with characteristics of its participants”; while some phenomena “are primarily related to the inherent semantics of the event”, others “have more to do with the way the event is presented” \citep[123]{Cumming1991}. Eventiveness is also associated with high \isi{discourse} transitivity and event sequentiality \citep[161--162]{Cumming1991}. \citet[176]{Cumming1991} notes that in \ili{Indonesian}, predicate initial clauses with passive morphology such as [\textit{di}-V-\textit{nya}] are still used to encode \isi{eventiveness} but \isi{eventiveness} is not the only motivation for this passive morphology. Such clauses tend to be “especially eventive” \citep[176]{Cumming1991}. Clauses with pre-predicate patient indicate that the patient is treated as “topical”, that is, it has either been mentioned in the immediately preceding \isi{clause} or a few clauses back and needs recalling \citep[176]{Cumming1991}.\footnote{“Topical” in \posscitet{Cumming1991} sense is slightly different from “topic” in theories of information structure adopted here. Cumming’s use of the term considers the tracking of a \isi{referent} across clauses and narrative episodes, and the position of the participant relative to other participants in narrative, whereas topic in the information structure theories is defined at \isi{clause} or sentence level. Nevertheless, in both cases, a topic \isi{referent} is understood as one that has been previously mentioned and hence can be presupposed.}  

In this study, I examine Cumming’s findings in light of more recent data from standard \ili{Indonesian}.\footnote{\citet{ColeEtAl2008} distinguish between standard \ili{Indonesian} that is the formal variety of the language and the “prescriptive standard” variety, i.e., the highly institutionally prescribed variety. No such distinction is made in this study.}  My goal is to demonstrate that, although the data support Cumming’s observation, the preference in \ili{Indonesian} is for events to be encoded with [\textit{meN}-V] clauses, and that predicate initial clauses such as those encoded with [\textit{di}-V-\textit{nya}] are marked, there seems to be a further development since her study. I focus my discussion on the interaction between three types of \isi{clause} structure – agentive clauses with [\textit{meN-}V], [\textit{di}-V-\textit{nya}] ‘passive’ clauses, and objective clauses with [\textit{ia} V] – to show the following. First, [\textit{meN}-V] clauses are indeed preferred for encoding various events, including successive events, irrespective of whether they are performed by the same agent or different agents. \citet[175]{Cumming1991} states that in \ili{Indonesian}, [\textit{di}-V-\textit{nya}] passive morphology is reserved for “especially eventive” clauses that mark the “\isi{climactic portion}” of a story episode. The data in this study show that [\textit{meN}-V] clauses are also used for this purpose. These provide strong support for Cumming’s finding that the preference in \ili{Indonesian} is for agentive clauses.

Second, the function of [\textit{di}-V-\textit{nya}] appears to be weaker now than Cumming found in her data. Events encoded by [\textit{di}-V-\textit{nya}] in the present data do not need to be “especially eventive” or to occur in the \isi{climactic portion} of a story episode; rather, [\textit{di}-V-\textit{nya}] is mainly used to signal a \isi{new focus}. A switch from [\textit{meN}-V] to [\textit{di}-V-\textit{nya}] invites the addressee/reader to turn their attention to a different event or series of events. The switch may indicate a shift from a series of descriptive clauses to an eventive one, or from a series of eventive clauses to a different event or series of events. 

Third, in addition to [\textit{di}-V-\textit{nya}], another type of \isi{clause} is also employed to mark a \isi{new focus}, namely objective \isi{clause} with free third person \isi{pronoun} \textit{ia} ‘3\textsc{sg}’. The objective \isi{clause} structure [\textit{ia} V] is often preceded and followed by [\textit{ia} \textit{meN}-V] agentive clauses. A switch from [\textit{ia} \textit{meN}-V] to [\textit{ia} V] signals a \isi{new focus}, similar to a switch from [\textit{meN}-V] to [\textit{di}-V-\textit{nya}], but whereas [\textit{di}-V-\textit{nya}] marks a structural difference from [\textit{meN}-V], the use of [\textit{ia} V] objective \isi{clause} alongside [\textit{ia} \textit{meN}-V] agentive \isi{clause} creates an impression of structural sameness. In both cases, the third person \isi{pronoun} \textit{ia} ‘3\textsc{sg}’ is preverbal, thus highlighting the continuity of the third person as agent (or actor in \isi{intransitive clause}) through explicit mention of the agent. In events encoded by [\textit{meN}-V] clauses, agent continuity tends to be indicated with zero.  

I draw on theories of information structure advanced by \citet{Lambrecht1994}, \citet{Gundel1999}, \citet{Gundel1993} and \citet{Gundel2004} to demonstrate how a switch from [\textit{meN}-V] clauses to [\textit{di}-V-\textit{nya}] or [\textit{ia} V] can be explained as a shift to a \isi{new focus}. As will be shown, a switch to a different grammatical construction functions to draw the addressee’s attention to a particular action or series of actions the agent is performing which cannot presupposed from the previous clauses. Drawing on the notion of “topic” in these theories is also useful for explaining why one participant, and not another, is referred to with zero in a series of [\textit{meN}-V] clauses as well as what the relation is between the topic participant and the focal elements in the \isi{clause}. More generally, examining the functions of the different \isi{clause} structures in terms of information structure is helpful for explaining why clauses occur in the structures that they do. 

Most of the data for this study are taken from six \ili{Indonesian} novels. One of the novels was published in 1977, and the others between 2002 and 2012. I chose fiction rather than other types of text based on the following considerations. First, fictional texts contain numerous instances of \isi{clause} chains and different \isi{clause} structures, including [\textit{di}-V-\textit{nya}] clauses which may be rare in other text types, thus enabling us to examine the interaction between the different \isi{clause} structures. Second, Cumming’s findings on \ili{Indonesian} were based on fictional texts, hence using data from the same text type but a different period is helpful for showing in what respects her findings are current and how they can be extended. I include examples from television news broadcast to show that the arguments I offer are also relevant for non-fictional texts. 

This chapter is structured as follows. \sectref{s:djenar:2} provides a description of the different \isi{clause} structures to be analysed and introduces \posscitet{Cumming1991} “trigger” system. \sectref{s:djenar:3} provides an overview of “topic” and “focus”, key concepts in information structure, to facilitate the analysis of the three \ili{Indonesian} \isi{clause} types in \sectref{s:djenar:4}, \sectref{s:djenar:5}, and \sectref{s:djenar:6} respectively. The study concludes in \sectref{s:djenar:7} with a summary of findings and suggestions for further research.

\section{\label{s:djenar:2}Agent trigger and patient trigger clauses}

In her study, \citet[29]{Cumming1991} discusses different \isi{clause} structures in terms of “trigger”. She distinguishes three types of \isi{clause}: the \isi{intransitive clause}, the “agent trigger” (AT) \isi{clause} and “patient trigger” (PT) \isi{clause}, the latter two being types of transitive clauses. “Trigger” is a syntactic role label for “the participant which is a) obligatory, and b) functions as the shared argument or ‘\isi{pivot}’ in \isi{clause} combining” (\citeyear{Cumming1991}: 31), and whose semantic role “triggers” the verb morphology. In the AT \isi{clause}, the trigger has the semantic role of agent, while in the PT \isi{clause}, it has the semantic role of patient. I adopt her terminology here. 

In Indonesian AT clauses, the agent typically precedes the verb and the verb is generally prefixed with \textit{meN}-, as shown in (\ref{e:djenar:1}). 

\begin{exe}
	\ex\label{e:djenar:1}
	\gll \textit{Mel} \textbf{\textit{meng-ambil}} \textit{kimono}.\\
	Mel  \textsc{meN}-fetch  kimono\\	
	\glt ‘Mel \textbf{fetched} (her) kimono.’ \hfill\citep[14]{Nuranindya2009}
\end{exe}

\noindent
PT clauses are of two types. The first has [\textit{di-}V] verbal morphology. In this \isi{clause} type, the agent, when expressed, occurs in an adjunct phrase typically marked by \textit{oleh} ‘by’, as shown in (\ref{e:djenar:2}). This type of PT \isi{clause} is also referred to in the literature as “passive” or “passive type 1” \citep{SneddonEtAl2010}. 

\begin{exe}
	\ex\label{e:djenar:2}
	\gll \textit{Sepasang}   \textit{suami}     \textit{istri}  \textit{yang}   \textit{sedang}   \textit{pergi}   \textit{ke} \textit{tahlilan}   \textit{keluarga}   \textit{di}   \textit{Cirebon} \textbf{\textit{di-serang}} \textbf{\textit{oleh}} \textit{geng}   \textit{motor}.\\
	one.couple  husband  wife  \textsc{rel}  in.process  go  to \textit{tahlilah} family    \textsc{loc}   Cirebon  \textsc{di}-attack  by gang  motor\\	
	\glt ‘A husband and wife, who went to a family \textit{tahlilan} event in Cirebon, \textbf{were attacked} \textbf{by} a motorcycle gang.’\footnote{\textit{Tahlilan} is a Muslim social gathering to commemorate the dead.}  \hfill(\citetitle{FokusPagi}, \textit{Indosiar}, 9/11/2015)
\end{exe}

\noindent
A related \isi{clause} type is one where [\textit{di}-V] is followed by the cliticised third person agent –\textit{nya} ‘\textsc{3sg}’. The enclitic –\textit{nya} is attached to the verb rather than to the preposition \textit{oleh} ‘by’ in an adjunct phrase (\textit{olehnya} ‘by her/him/it’). In example (\ref{e:djenar:3}), [\textit{di-}V-\textit{nya}] occurs with pre-predicate patient. 

\begin{exe}
	\ex\label{e:djenar:3}
	\gll \textit{Dan}  \textit{setelah}    \textit{pamili}  \textit{ini}  \textit{agak}  \textit{berada}    \textit{sedikit}, \textit{kamar}  \textit{itu} \textbf{\textit{di-pakai-nya}} \textit{sebagai}  \textit{kandang}  \textit{anjing}  \textit{yang} \textit{baru}    \textit{di}-\textit{beli}-\textit{nya}.\\
	and  after    family  this  rather  wealthy  a.little room  that  \textsc{di}-use-3\textsc{sg} as    kennel    dog  \textsc{rel} recently  \textsc{di}-buy-3\textsc{sg}\\
	Lit. ‘And after the family became a little wealthy, the room \textbf{was used by him/her} as a kennel for the dog they had bought.’\\
	\glt ‘After the family became a little wealthy, they used the room as a kennel for the dog s/he had just bought.’ \hfill(Idrus, cited in \citealt[204]{Kaswanti1988}; my translation)
\end{exe}

\noindent
[\textit{di}-V-\textit{nya}] clauses also occur with post-predicate patient, as in (\ref{e:djenar:4}). 

\begin{exe}
	\ex\label{e:djenar:4}
	\gll \textit{Babu}  \textit{itu}  \textit{me}-\textit{lihat}  \textit{air}  \textit{yang}  \textit{berkilauan}. {\ob}…{\cb}.  \textit{Kemudian} \textbf{\textit{di-ambil-nya}} \textit{abu},  \textit{di}-\textit{siram}-\textit{kan}  \textit{pada}     \textit{air}  \textit{kencing} \textit{dan} {\ob}…{\cb}.\\
	maid  that  \textsc{meN}-see  water  \textsc{rel}  \textsc{ber}-shine  {}  then \textsc{di}-get-\textsc{3sg}  ash  \textsc{di}-pour-\textsc{appl}  \textsc{loc}    water  urine and\\
	\glt ‘The maid looked at the shining liquid. […]. Then \textbf{she took} some ashes and \textbf{sprinkled} (them) on the urine and […].’ (Pramoedya Ananta \citealt[25--26]{Toer1963}; cited in \citealt[219]{Kaswanti1988}; translation from \citealt[219]{Kaswanti1988})\footnote{For consistency, I use my own glossing in this example rather than the original.} 
\end{exe}

\noindent
As \posscitet{Cumming1991} and \posscitet{Hopper1979} studies show, [\textit{di}-V-\textit{nya}] is a common device for marking foregrounding in \ili{Classical Malay}. Foregrounding clauses, in Hopper’s definition, are clauses denoting “events belonging to the skeletal structure of the \isi{discourse}” \citep[213]{Hopper1979}. These are clauses that move a story along temporally, in contrast to backgrounding material that do not. \citet[123]{Cumming1991} uses the term “\isi{eventiveness}” rather than “foregrounding” to place less emphasis on the participants and more on the characteristics of the event. The frequent use of this structure in \ili{Malay} texts has led to its conventionalisation as a structure denoting an action performed by some third person agent. \citet[205]{Kaswanti1988} suggests that [\textit{di}-V-\textit{nya}] is “the type of \textit{di}- used to describe a series of chronologically ordered punctiliar actions by a single actor”. This is certainly true of example (\ref{e:djenar:5}) below. 

\begin{exe}
	\ex\label{e:djenar:5}
	\gll \textbf{\textit{Di-cuci-nya}} \textit{muka}-\textit{nya},   \textit{kemudian} \textbf{\textit{di-lap-nya}} \textit{kering}-\textit{kering},   \textit{lalu}   \textit{kembali}   \textit{ke}   \textit{kamar}-\textit{nya}.\\
	\textsc{di}-wash-3\textsc{sg}  face-\textsc{3sg}  then    \textsc{di}{}-wipe-3\textsc{sg}  dryly then  return    to  room-\textsc{3sg}\\
	\glt ‘\textbf{She}\textsubscript{i} \textbf{washed} her face\textsubscript{j}, \textbf{she}\textsubscript{i} \textbf{wiped} [ ]\textsubscript{j} dry, then [ ]\textsubscript{i} returned to her room.’\\ \hfill \citep[119]{Krisna1977}
\end{exe}

\noindent
However, [\textit{di}-V-\textit{nya}] is not the only construction that can be used to encode successive actions by a single actor, as will be shown later in this chapter. In \ili{Indonesian}, [\textit{di}-V-\textit{nya}] is found mainly in written \isi{discourse}, particularly in older fictional narrative (the example in (\ref{e:djenar:5}) is taken from a novel published in 1977). As we saw in (\ref{e:djenar:4}), [\textit{di}-V] is also used to encode such actions. This example is from a collection of short stories published in 1963 by acclaimed writer Pramoedya Ananta Toer. In more contemporary texts, this is more likely to be rendered in either in [\textit{meN}-] AT clauses or a combination of AT and PT clauses, as will be discussed in \sectref{s:djenar:4}, \sectref{s:djenar:5} and \sectref{s:djenar:6}. 

The second type of PT \isi{clause} contains an unprefixed verb with the agent in pre-predicate position, either in the form of a \isi{clitic}, such as in \textit{ku}- ‘\textsc{1sg}’ in (\ref{e:djenar:6}), or a \isi{free pronoun}, such as \textit{ia} ‘3\textsc{sg}’ in (\ref{e:djenar:7}).\footnote{The agent may also be proper name or kin term.}  This type of PT \isi{clause} is better known in the literature as “objective \isi{voice}” – henceforth \isi{OV} (\citealt{ArkaEtAl1998}; \citealt{ColeEtAl2008}; \citealt{Kroeger2014}). This study is mainly concerned with \isi{OV} clauses containing the third person \isi{pronoun} \textit{ia}.\footnote{Beside \textit{ia}, the third person \isi{pronoun} \textit{dia} is also used in \ili{Indonesian}. The latter is not discussed in this study.}

\begin{exe}
	\ex\label{e:djenar:6}
	\gll \textbf{\textit{Ku-guyur}} \textit{seluruh}   \textit{tubuh}-\textit{ku}     \textit{dengan} \textit{air}   \textit{dari}  \textit{keran}   \textit{wudu}.\\
	\textsc{1sg}-pour.water  entire    body-\textsc{1sg}    with water  from  tap  \textit{wudu}\\
	\glt ‘\textbf{I poured water on} my body with water from the \textit{wudu} tap.’\footnote{\textit{Kran wudu} is the tap Muslims use to get water from to wash their face, hands, and feet in preparation for prayer.}  \hfill \citep[12]{Kumalasari2006}
\end{exe}

\begin{exe}
	\ex\label{e:djenar:7}
	\gll \textit{Buku-buku}   \textit{yang}   \textit{semula}   \textit{berserakan} \textit{di}  \textit{atas}  \textit{tempat}   \textit{tidur},   \textit{meja}   \textit{dan}   \textit{lantai} \textbf{\textit{ia}} {\textbf{\textit{masuk-kan}}\footnotemark} \textit{ke}   \textit{dalam}     \textit{kotak}     \textit{kardus}.\\
	book-\textsc{rdp}  \textsc{rel}  previously  scattered  \textsc{loc}  top  place  sleep  table  and  floor  \textsc{3sg}  put.in-\textsc{appl} \textsc{dir}  inside    box    cardboard \\
	Lit. ‘The books which were previously scattered on the bed, table and floor, \textbf{he put} in cardboard boxes.’\\
	\textbf{‘He put} \textbf{the books} which were previously scattered on the bed, table and floor in cardboard boxes.’ \hfill \citep[195--196]{Kurniawan2002}
\end{exe}
\footnotetext{\ili{Indonesian} has two applicative suffixes, -\textit{kan} and –\textit{i}. In this study I treat [\textit{di}-V-\textit{kan}] and [\textit{di}-V-\textit{i}] as a subset of [\textit{di}-V].}

\noindent
\citet[34]{Cumming1991} uses the following shorthand to refer to the different types of \isi{constituent order}: V>T for clauses in which the predicate occurs before the patient, and T>V for those  the patient precedes the predicate (in her terms, “predicate before trigger” and “trigger before predicate” respectively). She categories \isi{OV} clauses with pre-predicate agent such as in (\ref{e:djenar:6}) as PT clauses and not AT clauses even though the agent is in pre-predicate position (and the order of the constituents is therefore A V P). Her argument for doing so is that in this type of construction, the preverbal agent occupies the same slot as the prefix \textit{di-} in passive. Nevertheless, as I argue later in this study, the choice of the \isi{free pronoun} \textit{ia} instead of the enclitic –\textit{nya} is pragmatically and stylistically meaningful. 

The foregoing examples show that events can be encoded in different ways, such with [\textit{meN-}V] AT \isi{clause}, [\textit{di}-V] or [\textit{di}-V-\textit{nya}] PT clauses with either pre-predicate or post-predicate patient and \isi{OV} \isi{clause} with proclitic or a \isi{free pronoun} as agent. Given these different possibilities, what motivates the use of PT clauses and what is the functional difference between PT clauses with T>V order and that with V>T order? As Cumming argues, PT clauses with post-predicate patient (V>T order) are an important grammatical resource for encoding \isi{eventiveness} in \ili{Classical Malay} while pre-predicate patient (T>V order) in PT clauses indicates the topical status of the patient. I argue that in contexts where both AT and PT clauses are used, a switch from AT clauses to a PT \isi{clause} marks a shift in focus. By switching to a PT \isi{clause}, the writer signals that there is new information and the reader is invited to direct their attention to it. In the next section, the notions of “topic” and “focus” are explained.   

\section{\label{s:djenar:3}Topic and focus}

“Topic” and “focus” in the sense of \citet{Lambrecht1994}, \citet{Gundel1977}, \citet{Gundel1999}, \citet{Gundel2004}, and \citet{Gundel1993} are relational pragmatic categories. Topic is defined in terms of aboutness: “the thing which the proposition expressed by the sentence is \textsc{about}” (\citealt[118]{Lambrecht1994}; emphasis in original). The term “topic” designates the “entity” or “the \isi{discourse} \isi{referent}” about which the proposition is construed \citep[127]{Lambrecht1994}. A topic must be referential, individuated \citep[187]{Gundel2004}, and cognitively active, i.e., it is mentioned in the sentence or \isi{clause}, and not merely inferred. In the example below (from \citealt[127]{Lambrecht1994}; capitalisation in original), the expressions ‘Pat’, ‘she’, and ‘her’ all refer to the same entity, and that entity is construed as being what the propositions are about. The \isi{referent} of these topic expressions is identifiable through the mention of the proper name ‘Pat’ and the third person pronominal forms ‘she’ and ‘her’. 

\begin{exe}
	\ex\label{e:djenar:8}
	\begin{xlist}
		\ex Pat said they called her \textsc{twice}.
		\ex Pat said she was called \textsc{twice}.
		\ex Pat said \textsc{she} was called.
		\ex Pat said they called \textsc{her}. 
	\end{xlist}
\end{exe}

\noindent
\citet[204]{VanValin1997} explain that topics “either name a topic \isi{referent} in the \isi{discourse}, or they are simply involved in the expression of a semantic relation between a topic \isi{referent} and a predication.” Topics that have the former function are generally coded as lexical NPs, while those in the latter function, are “most often” coded as zero or unstressed pronouns. In terms of syntactic coding for topic and focus, \citet[205]{VanValin1997} provide a useful scale of acceptability, shown below in \figref{fig:1}. 

\begin{figure}
	\begin{tabular}[t]{lccccc}
		&&&\multicolumn{3}{r}{Markedness of occurrence as focus}\\
		&&&\multicolumn{1}{l}{\tikzmark{b}}&&\multicolumn{1}{r}{\tikzmark{a}}\\
		& Clitic / &&& Definite & Indefinite \\
		Zero&bound \isi{pronoun}&Pronoun  & Pronoun&NP&NP\\
		&&{[-stress]} & {[+stress]} \\
		\tikzmark{c}&&\multicolumn{1}{r}{\tikzmark{d}}&&&\\
		\multicolumn{6}{l}{Markedness of occurrence as topic}
	\end{tabular}
	\begin{tikzpicture}[overlay, remember picture, line width=2pt]
	\draw [->,>=latex] ({pic cs:a}) to ({pic cs:b});
	\draw [->,>=latex] ({pic cs:c}) to ({pic cs:d});
	\end{tikzpicture}
\caption{Coding of referents in terms of possible functions \citep[205]{VanValin1997}}
\label{fig:djenar:1}
\end{figure}

%[overlay, remember picture, yshift=.25\baselineskip, shorten >=.5pt, shorten <=.5pt]

\noindent
This figure shows that zero coding is the least marked coding for a topic, while indefinite NP is the least marked coding for a focus. According to \citet[205]{VanValin1997}, “while indefinite NPs can be topics under special contextual circumstances, it is impossible for a focal element to be zero”. Zero correlates with the \isi{referent}’s cognitive status as active \citep[379]{Givón1975}.

Focus, like topic, is defined relationally; focus is a relational notion that “determines the main predication in the sentence, that predication being assessed relative to topic” \citep[190]{Gundel2004}. \citet[213]{Lambrecht1994} defines it as “the semantic component of a pragmatically structured proposition whereby the \isi{assertion} differs from the \isi{presupposition}”. The difference between a \isi{pragmatic presupposition} and a \isi{pragmatic assertion} has to do with the difference between what the hearer is assumed to know at the time of hearing the sentence, and what s/he is expected to know as a result of hearing it, as spelt out below (\citealt[52]{Lambrecht1994}; capitalisation in original). 

\begin{description}
	\item\textsc{Pragmatic presupposition}: the set of propositions lexicogrammatically evoked in a sentence which the speaker assumes the hearer already knows or is ready to take for granted at the time the sentence is uttered. 
	
	\item\textsc{Pragmatic assertion}: the proposition expressed by a sentence which the hearer is expected to know or take for granted as a result of hearing the sentence uttered. 
\end{description}

\noindent
The relation between topic and focus is one between “pragmatically non-recoverable to the recoverable component of a proposition”. Focus creates a new state of information in the mind of the addressee \citep[218]{Lambrecht1994}, thus is often conflated with \posscitet{Chafe1994} notion of “new information” in his discussion on \isi{referent} accessibility; however, as \citet[see chapter3]{Lambrecht1994} stresses, the two are not the same. 

\begin{exe}
	\ex\label{e:djenar:9}
	\begin{xlist}
		\exi{Q:} When did you move to Switzerland?
		\exi{A:} When I was \textsc{seventeen}. \\
		\hfill \citep[48, 217]{Lambrecht1994}
	\end{xlist}
\end{exe}

\noindent
The stressed element in the answer is identified as the focus. What makes the proposition informative is \textit{not} the fact that the person was seventeen years of age at some point in her life (which is the \isi{pragmatic presupposition} in this example), but that s/he moved to Switzerland at the time she was seventeen (the \isi{pragmatic assertion}). The \isi{assertion} (focus) thus stands in relation to the topic (the person moving) and the action of the topic \isi{referent} (i.e., moving to Switzerland). Similarly, in (\ref{e:djenar:10}), the \isi{referent} of ‘Bill’ is cognitively active (it is mentioned in the question Q). However, Bill is the focus here because in the \isi{presupposition} ‘speaker saw x’ where ‘x = Bill’, it is “new” information. This is an example of what \citet[182]{Gundel2004} call “\isi{contrastive focus}”. 

\begin{exe}
	\ex\label{e:djenar:10}
	\begin{xlist}
		\exi{Q:} Did you see John or Bill?
		\exi{A:} Bill.\\
		\hfill \citep[205]{VanValin1997}
	\end{xlist}
\end{exe}

\noindent
\citet[221--235]{Lambrecht1994} distinguishes between argument-focus, predicate-focus, and sentence-focus structures. Argument-\isi{focus structure} is a sentence in which the focus is the missing argument in an open proposition (similar to “\isi{narrow focus}” in \citealt{Givón1975}). This is illustrated in (\ref{e:djenar:11}). Here the focal element ‘car’ is the argument of the proposition. 

\begin{exe}
	\ex\label{e:djenar:11}
	\begin{xlist}
		\exi{Q:} I heard your motorcycle broke down?
		\exi{A:} My \textsc{car} broke down.\\
		\hfill \citep[48, 223]{Lambrecht1994}
	\end{xlist}
\end{exe}

\noindent
In predicate-\isi{focus structure}, the predicate is the focus, and the subject with all its elements are in the \isi{presupposition}.  

\begin{exe}
	\ex\label{e:djenar:12}
	\begin{xlist}
		\exi{Q:} What happened to your car?
		\exi{A:} My car/it broke \textsc{down}.\\
		\hfill \citep[48, 223]{Lambrecht1994}
	\end{xlist}
\end{exe}

\noindent
In sentence-\isi{focus structure}, it is both the subject and predicate in the sentence (minus any topical, non-subject elements) that constitute the focus (e.g., in presentative constructions).

\begin{exe}
	\ex\label{e:djenar:13}
	\begin{xlist}
		\exi{Q:} What happened?
		\exi{A:} My \textsc{car} broke down.\\
		\hfill \citep[48, 223]{Lambrecht1994}
	\end{xlist}
\end{exe}

\noindent
All languages have some grammatical means for marking topic and focus, including morphosyntax, \isi{prosody}, \isi{word order}, or any combination of these \citep[201]{VanValin1997}. With regard to \ili{Indonesian} PT clauses, \citet[175]{Cumming1991} notes that the position of patient relative to the predicate provides an indication of its topical status (or “thematic” in \citeauthor{Kaswanti1988}'s \citeyear[226]{Kaswanti1988} term). Pre-predicate position suggests \isi{topicality}, while post-predicate position indicates the patient is “not made prominent” and “not highlighted”. As will be shown in \sectref{s:djenar:5}, PT clauses are marked and they are not always used to indicate a climactic point in a story. Agentive [\textit{meN}-V] clauses can also be used to mark a \isi{climactic portion}, as will be shown in \sectref{s:djenar:4}. Drawing on the notions of focus and topic helps us explain the significance of pre-predicate patient by taking into account its relation to the predicate, as well as the significance of the \isi{referent} of the patient argument in the \isi{discourse}. 

\section{\label{s:djenar:4}\textit{MeN}- clauses}

 One of the most important findings in \posscitet{Cumming1991} study is that modern \ili{Indonesian} strongly shows a preference for AT clauses. A recent study by \citet{Shiohara2015} supports this finding. Using short videos showing a young person performing a series of actions (e.g., breaking an egg into a bowl, whisking the egg, adding some milk, then whisking it again) for data elicitation, Shiohara’s study shows that of all the transitive clauses found, 88.5\% were AT, while the remaining clauses were rendered in PT (8.6\% in \textit{di}- passive and 2.9\% in objective \isi{voice}). Shiohara’s study tells us that the preference for [\textit{meN-}V] clauses is most dramatic when successive actions by the same agent are involved. It also shows that [\textit{meN}-V] primarily occurs in eventive clauses, contrary to  \posscitet[226]{Kaswanti1988} claim that they tend to be used for backgrounding. In what follows I show that the preference for [\textit{meN}-V] persists even when there are multiple referents that are accorded the status of agent and a switch in the semantic function of the participant is involved (cf. \citealt{FoleyEtAL1984}). Moreover, [\textit{meN}-] AT clauses are also used to mark the \isi{climactic portion} of a story episode, unlike in older texts where PT is preferred for this purpose. 
 
 Two examples from the novel \textit{Lelaki Harimau} ‘Man Tiger’ \citep{Kurniawan2004} show how a series of AT clauses can be used to encode a series of actions performed by two different participants, with a switch in the semantic function of one participant – from patient to agent – but without the accompanying change in the verbal morphology. The excerpt in (\ref{e:djenar:14}) is included here to provide a context for the AT examples in (\ref{e:djenar:15}) and to show the switch from PT to AT. The protagonist, Margio, has just killed the man who had courted his mother and made her pregnant but refused to marry her. Margio is subsequently captured for his deed and detained at the local military office. The information about his arrest is rendered in PT with T>V order, signalling to the reader to pay attention to the event that describes the capture. The \isi{referent} ‘Margio’ is shown as the “topical” patient, as indicated by its pre-predicate position \citep[175]{Cumming1991}. As will be further discussed in the next section, PT clauses are commonly used to attract the reader’s attention to a \isi{new focus}.

\begin{exe}
	\ex\label{e:djenar:14}
	\textit{Kini, ketika malam telah runtuh ke bumi mengapungkan bintang-bintang dan bulan sepotong tergantung enggan, lampu-lampu dinyalakan di pelataran rumah dan pinggir jalan, dan codot tak lagi tampak beterbangan disebabkan hitam yang menghapus hitam tubuhnya,} \textbf{\textit{Margio diseret Joni Simbolon ke rayon militer}}. \textit{Selalu begitu sebelum seseorang dibawa ke kantor polisi, sebab tanpa itu para prajurit tak punya lagi keriangan di dunia republik yang tak ada perang.} \\
	\glt ‘Now, when night has fallen on the earth making the stars appear floating and a piece of moon is hanging reluctantly, the lights lit on the front yards and along the streets, and the bats no longer flying around as the darkness hides the blackness of their bodies, \textbf{Margio is being dragged by Joni Simbolon} \textbf{to the local military office}. That always happens before someone is taken to the police station, because otherwise, the soldiers would no longer be able to have fun in a republic where there is no war happening.’ \hfill (\citealt[37]{Kurniawan2004}; my translation)\footnote{In this translation I provide as close a translation as possible to the original to facilitate the discussion. This is slightly different from the published \ili{English} translation (see \citealt{Kurniawan2015}; translated by Labodalih Sembiring).}
\end{exe}

\noindent
In the immediately following \isi{discourse}, shown in (\ref{e:djenar:15}), the author switches to AT to describe two successive events depicting what Margio’s captors did to him. In the first \isi{clause}, the agent (Joni Simbolon and his friends) is referred to with the third person plural \isi{pronoun} \textit{mereka} ‘they’, while the patient (Margio), with the enclitic \textit–{nya} ‘\textsc{3sg}’. These pronominal forms indicate the presupposed status of the two participants. In the second [\textit{meN}-V] \isi{clause}, the \isi{referent} of ‘Joni Simbolon (and his friends)’ is coded as zero (reflecting its \isi{topic status}), while ‘Margio’ is mentioned with the enclitic -\textit{nya} ‘\textsc{3sg}’ again. These two [\textit{meN}-V] clauses are followed by a switch in the semantic function of the patient participant. Margio, coded as zero, is now actor of the \isi{intransitive clause} \textit{meringkuk di satu dipan} ‘curled up on a wooden bed’ and agent of the subsequent AT transitive clauses. 

\begin{exe}
	\ex\label{e:djenar:15}
	\gll \textit{Mereka} \textbf{\textit{me-ngurung}}-\textbf{\textit{nya}} \textit{di} \textit{dalam} \textit{sel}, \textbf{\textit{meng-ganti}}   \textbf{\textit{pakaian}}-\textbf{\textit{nya}} \textit{dengan} \textit{seragam} \textit{hitam} \textit{bau} \textit{kapur} \textit{barus} \textit{dan}   \textit{lemari} \textit{kayu}, \textbf{\textit{me-ringkuk}} \textit{di} \textit{satu} \textit{dipan} \textbf{\textit{meng-hadap-i}} \textit{susu} \textit{hangat} \textit{yang} \textit{tak} \textit{di-cecap-}\textbf{\textit{nya}}, \textbf{\textit{meng-hadap-i}} \textit{sepiring} \textit{nasi} \textit{ikan} \textit{tongkol} \textit{yang} \textit{tak} \textit{di-lumat-}\textbf{\textit{nya}}.\\
	\textsc{3pl} \textsc{meN}-cage-\textsc{3sg} \textsc{loc} inside cell \textsc{meN}-change clothes-\textsc{3sg} with uniform black smell chalk  barus.tree and wardrobe wood  \textsc{meN}-curl.up \textsc{loc} one wooden.bed \textsc{meN}-face-\textsc{appl} milk warm \textsc{rel} \textsc{neg} \textsc{di}{}-taste-\textsc{3sg} \textsc{meN}{}-face-\textsc{appl} one.plate rice fish tuna \textsc{rel} \textsc{neg}  \textsc{di}-devour-\textsc{3sg}\\
	\glt ‘They\textsubscript{i} \textbf{put him}{\textsubscript{j}} \textbf{in a cell}, [ ]\textsubscript{i} \textbf{changed his}{\textsubscript{j}} \textbf{clothes} into a black uniform smelling of mothballs and wooden wardrobe, [ ]\textsubscript{j} \textbf{curled up} on a wooden bed [ ]\textsubscript{j} \textbf{facing} (a glass of) warm milk he{\textsubscript{j}} didn’t drink, [ ]\textsubscript{j} \textbf{facing} a plate of tuna rice he{\textsubscript{j}} didn’t devour.’ \hfill (\citealt[37]{Kurniawan2004}; my translation)
\end{exe}

\noindent
In this excerpt, the \isi{referent} of ‘Margio’ is treated as highly continuous; he is mentioned with the enclitic –\textit{nya} ‘\textsc{3sg}’ in the first two clauses, followed by zero in the subsequent three (main) clauses. In the final two clauses, the reader is told that Margio is not touching the food given to him. The author uses the same predicate in these clauses and also relative clauses, creating parallel structures of the kind [V NP \textit{yang tak di}-V-\textit{nya}]. As discussed later in this chapter, parallelism is an important stylistic resource fiction authors often draw on to produce a \isi{rhythmic effect}. 

\begin{exe}
	\ex\label{e:djenar:16}
	\begin{xlist}
		\ex \gll \textit{meng-hadap-i} \textit{susu} \textit{hangat} \textit{yang} \textit{tak} \textit{di-cecap-}\textbf{\textit{nya}}\\
		\textsc{meN}-face-\textsc{appl}  milk  warm  \textsc{rel}  \textsc{neg}  \textsc{di}-taste-\textsc{3sg}\\
		\glt ‘face (a glass of) warm milk \textbf{he} didn’t drink’\label{e:djenar:42a}
		\ex \gll \textit{meng-hadap-i} \textit{sepiring} \textit{nasi} \textit{ikan} \textit{tongkol} \textit{yang} \textit{tak} \textit{di-lumat-}\textbf{\textit{nya}}\\
		\textsc{meN}-face-\textsc{appl} one.plate rice fish tuna \textsc{rel} \textsc{neg} \textsc{di}-devour-\textsc{3sg}\\
		\glt ‘face a plate of tuna rice that \textbf{he} didn’t devour’\label{e:djenar:42b}
	\end{xlist}
\end{exe}

\noindent
Below is a summary of the different clauses and arguments (expressed arguments are italicised, while the zero is rendered in square brackets). 

\begin{enumerate}
	\item \textit{Margio}\textsubscript{i} is captured by \textit{Joni Simbolon and his friends}\textsubscript{j}.
	\item \textit{They}\textsubscript{j} put \textit{him}\textsubscript{i} in a prison cell.
	\item {[ ]}\textsubscript{j} changed \textit{his clothes} to a prison uniform.
	\item {[ ]}\textsubscript{i} curls up on the wooden bed.
	\item {[ ]}\textsubscript{i} faces \textit{a glass of warm milk} he didn’t drink.
	\item {[ ]}\textsubscript{i} faces \textit{a plate of tuna rice} he didn’t devour. 
\end{enumerate}

\noindent
We can see in this example that [\textit{meN}-V] AT clauses are the preferred construction for encoding agency even when there are multiple agents and the agency is switched between different human referents. In addition, stylistic considerations are also important in the presentation of events. The string of [\textit{meN}-V] clauses and the repeated [NP \textit{yang tak di}-V-\textit{nya}] relative clauses create a \isi{rhythmic effect} and an impression of quick, successive actions. Though the last three AT clauses are not foregrounding clauses, the \isi{rhythmic effect} remains and this is due to the repetition of \textit{meN}- verbs (\textit{meringkuk} ‘curl up’ and \textit{menghadapi} ‘face something’, the latter being used twice in succession) and the relative clauses in (\ref{e:djenar:16}).  

Another example showing [\textit{meN}-V] used in succession is given in (\ref{e:djenar:17}). This example describes a physical clash between Margio’s sister, Maesa Dewi, and a luminous shadow. These two participants are overtly mentioned in the initial AT \isi{clause}. 

\begin{exe}
	\ex\label{e:djenar:17}
	\gll \textit{Maesa} \textit{Dewi}\textsubscript{i} \textit{me-lihat} \textit{bayangan} \textit{cemerlang}\textsubscript{j} \textit{yang} \textit{tak} \textit{di-kenal-i-}\textbf{\textit{nya}}\textsubscript{i}, \textit{me-mancar} \textit{mem-beri} \textit{silau} \textit{kepada} \textit{mata-nya}, \textbf{\textit{men-desak-nya}}\textsubscript{i} \textit{ke} \textit{belakang} \textit{dan} \textit{suatu} \textit{benturan} \textbf{\textit{meng-hantam}} \textit{kepala-}\textbf{\textit{nya}}\textsubscript{i} \textit{saat} \textit{itu} {\ob \cb}\textsubscript{i}   \textbf{\textit{me-nentang}} \textit{dinding} \textit{pintu} \textit{di} \textit{belakang}, {\ob \cb}\textsubscript{j}   \textbf{\textit{meng-henti-kan-nya}}\textsubscript{i} \textit{sejenak} \textit{sebelum} {\ob \cb}\textsubscript{i} \textbf{\textit{me-rosot}} \textit{dan} \textit{tumbang} \textit{ke} \textit{lantai}.\\
	Maesa Dewi \textsc{meN}{}-see shadow brilliant \textsc{rel} \textsc{neg} \textsc{di}-recognise-\textsc{appl}-\textsc{3sg}  \textsc{meN}-shine \textsc{meN}-give blinding toward eye-\textsc{3sg} \textsc{meN}-force-\textsc{3sg} to back and \textsc{indef} knock    \textsc{meN}-whack head-\textsc{3sg} moment that {} \textsc{meN}-hit wall door \textsc{loc} back {} \textsc{meN}-stop-\textsc{appl}-\textsc{3sg} for.a.moment before {} \textsc{meN}-slide.down and collapse to floor\\
	\glt ‘Maesa Dewi\textsubscript{i} saw a luminous shadow\textsubscript{j}, which \textbf{she\textsubscript{i}} did not recognise, [ ]\textsubscript{j} shone brightly, giving her eyes a blinding light, [ ]\textsubscript{j} \textbf{pushed her}\textsubscript{i} backwards and a knock \textbf{hit} her head at that moment (until) [ ]\textsubscript{i} \textbf{knocked} the back door, [ ]\textsubscript{j} \textbf{stopped} \textbf{her} momentarily before [ ]\textsubscript{i} \textbf{slid down} and [ ]\textsubscript{i} fell to the floor.’ \hfill \citep[34]{Kurniawan2004}
\end{exe}

\noindent
The chain of events with the different participants and their semantic roles is summarised below (as previously, expressed participants are shown in italics, and the zero is indicated as square brackets).

\begin{enumerate}
	\item \textit{Maesa Dewi}\textsubscript{i} (Agent) saw \textit{a luminous shadow}\textsubscript{j} (Patient).
	\item {[ ]}\textsubscript{j} (Agent) shone brightly, giving \textit{her}\textsubscript{i} \textit{eyes} (Patient) a blinding light. 
	\item {[ ]}\textsubscript{j} (Agent) pushed \textit{her}\textsubscript{i} (Patient) backwards.
	\item {[ ]}\textsubscript{i} (Agent) knocked the back door (Patient).
	\item {[ ]}\textsubscript{j} (Agent) stopped \textit{her}\textsubscript{i} (Patient) momentarily.
	\item {[ ]}\textsubscript{i} (\isi{Undergoer}) slid down.
	\item {[ ]}\textsubscript{i} (\isi{Undergoer}) fell to the floor. (note: this event is encoded with a bare verb.)
\end{enumerate}

\noindent
With the exception of the first two clauses, the clauses in this example are eventive, as indicated by the punctual verbs. The \isi{referent} of ‘Maesa Dewi’ is presented as agent in the initial \isi{clause}, then patient and \isi{undergoer} in the following clauses. The use of zero in subject position throughout this long example potentially creates referential confusion and slows down processing: is it the shadow or Maesa Dewi that slid down and collapsed? However, this potential confusion is resolved by the \isi{discourse} context. In the subsequent clauses, Maesa Dewi is described as lying down as if frozen, unable to speak and her memory hazy. All the clauses in this story episode are rendered in [\textit{meN}-V] AT clauses and most of them denote quick, successive actions by different agents/actors. In this example, as with example (\ref{e:djenar:15}), the repetition of [\textit{meN}-V] produces vividness and a sense of immediacy as well as a \isi{rhythmic effect}. It seems that for the author, creating this kind of effect is more important in this long sentence than avoiding potential referential confusion. These examples provide evidence that in contemporary \ili{Indonesian}, [\textit{meN}-V] serves the kind of pragmatic function previously encoded with [\textit{di}-V-\textit{nya}]. 

The next two examples are from television news reports aired in 2015 and 2017, showing [\textit{meN}-V] AT clauses marking the \isi{climactic portion} of a story episode. In these examples, a PT \isi{clause} with [\textit{di}-V] is used to mark an event that leads to the climax and one subsequent to it. The first example of [\textit{meN}-V] occurs in (\ref{e:djenar:19}). This example follows directly from the sentence in example (\ref{e:djenar:2}), repeated below as (\ref{e:djenar:18}) for convenience. When this news item was broadcast, the sentence was read like a headline about a husband and wife couple being attacked by people in a motorbike. This headline is rendered in PT.

\begin{exe}
	\ex\label{e:djenar:18}
	\gll \textit{Sepasang} \textit{suami} \textit{istri} \textit{yang} \textit{sedang} \textit{pergi} \textit{ke} \textit{tahlilan} \textit{keluarga}   \textit{di} \textit{Cirebon} \textbf{\textit{di-serang}} \textbf{\textit{oleh}} \textbf{\textit{geng}} \textbf{\textit{motor}}.\\
	\textsc{se}-couple husband wife \textsc{rel} in.process go to \textit{tahlilah} family \textsc{loc} Cirebon \textsc{di}-attack by gang  motor\\
	\glt ‘A husband and wife, who went to a family \textit{tahlilan} event in Cirebon, \textbf{were attacked} \textbf{by a motorcycle gang}.’\footnote{\textit{Tahlilan} is a Muslim social gathering to commemorate the dead. \textit{Pak} is a short form of \textit{bapak}, a kin term used to address or refer to one’s father and adult males in general.}  (\citetitle{FokusPagi}, \textit{Indosiar}, 9/11/2015)
\end{exe}

\noindent
Following the headline, a series of chronological events were reported, as shown in (\ref{e:djenar:19}). The first \isi{clause} contains a [\textit{di}-V] PT \isi{clause} with pre-predicate patient, the second has a [\textit{meN}-V] \isi{clause} with zero agent, and the third, another [\textit{di}-V] \isi{clause} also with pre-predicate patient. The agentive [\textit{meN}-V] \isi{clause} marks the climax of the story, while the first and third PT clauses inform the addressee of the event that leads to and follows this climax respectively. The pre-predicate patient (\textit{pasangan suami istri itu} ‘the husband and wife couple’) in the PT clauses indicates its “topical” status. 

\begin{exe}
	\ex\label{e:djenar:19}
	\gll \textit{Naas} \textit{pasangan} \textit{suami} \textit{istri} \textit{itu} \textbf{\textit{di-pepet}} \textit{oleh} \textit{kendaraan} \textit{bermotor} \textit{lalu} \textbf{\textit{me-nyerang}} \textit{dengan} \textit{senjata} \textit{tajam}, \textit{kemudian} \textit{sang}  \textit{istri} \textbf{\textit{di-tusuk}}.\\
	bad.luck couple husband wife that \textsc{di}-press.against by vehicle have.engine then \textsc{meN}-attack with arm sharp then \textsc{def}  wife \textsc{di}-stab\\
	(Lit. ‘Bad luck, the husband and wife couple was pressed against by a motorbike then attack with a sharp object, after which the wife was stabbed.’)\\
	\glt ‘Bad luck for the husband and wife couple\textsubscript{i}, a motorbike\textsubscript{j} approached them\textsubscript{i} and [~]\textsubscript{j} attacked [~]\textsubscript{i} with a sharp object, then stabbed the wife.’ \\\hfill{(\citetitle{FokusPagi}, \textit{Indosiar}, 9/11/2015).}
\end{exe}

\noindent
A similar example is shown in (\ref{e:djenar:20}) but with the post-climactic event rendered in an \isi{intransitive clause} marked by the prefix \textit{ber}-. This example is taken from a news item about a crime committed against Novel Baswedan, a senior investigator for Indonesia’s \textit{Komisi Pemberantasan Korupsi} ‘Commission for the Eradication of Corruption’. Novel was reported as walking home from the nearby mosque at five am when two people in a motorbike rode towards him and threw acid on his face. The initial [\textit{di}-V] PT \isi{clause} alerts the addressee to new information, while the [\textit{meN}-V] marks the eventive \isi{clause} and the climax of the story.\footnote{In this example, the verb contains the applicative suffix –\textit{i}. As mentioned, this is treated as [\textit{di}-V].}  The pre-predicate patient in the PT \isi{clause} indicates the patient’s “topical” status. The following [\textit{ber}-V] \isi{clause} describes what happens after the climax.

\begin{exe}
	\ex\label{e:djenar:20}
	\gll \textit{Tiba-tiba} \textit{Novel} \textbf{\textit{di-datang-i}} \textit{dua} \textit{orang} \textit{bersepeda} \textit{motor} \textit{dan} \textit{langsung} \textbf{\textit{me-nyiram-nya}} \textit{dengan} \textit{air} \textit{keras}. \textit{Novel} \textit{lalu} \textbf{\textit{berteriak}} \textit{dengan} \textit{sekencang-kencangnya}.\\
	suddenly-\textsc{rdp} Novel \textsc{di}-come-\textsc{appl} two person on.bicyle motor and immediately \textsc{meN}-pour-3\textsc{sg} with water hard Novel then scream with as.loud.as.possible-\textsc{rdp}\\
	\glt ‘Suddenly Novel\textsubscript{i} \textbf{was approached} by two people\textsubscript{j} on a motorbike and \\{[ ]}\textsubscript{j} \textbf{threw} acid \textbf{on him}\textsubscript{i}. Novel \textbf{screamed} as loudly as he could.’ (\citetitle{Eksklusif}, \textit{Aiman}, \textit{Kompas TV}, 3 July 2017)
\end{exe}

\noindent
To summarise, this section has discussed, first, the use of [\textit{meN}-V] clauses to encode a series of events involving different agents. In the examples shown, vividness and a sense of immediacy are created through successive use of [\textit{meN}-V] in both transitive and intransitive clauses involving different referents in subject position (in the role of either agent, actor or \isi{undergoer}). I discussed two examples from the same novel, so one might argue that the preference for [\textit{meN}-V] clauses is particularised to the author of that novel (i.e., Eka Kurniawan). However, as \citet{Cumming1991} and \citet{Kaswanti1988} have amply shown through their examples, the use of [\textit{meN}-V] or [\textit{di}-V-(\textit{nya})] clauses to denote successive actions is common in \ili{Indonesian} fiction. What seems to be a new development is the strong preference for AT even when agency is switched among participants. This section has also shown that [\textit{meN}-V] clauses can mark the \isi{climactic portion} of a story episode, with non-eventive clauses marked by [\textit{di}-V] and [\textit{ber}-V] preceding and following them. This is the opposite of the kind of pattern in older texts studied by Cumming and Kaswanti Purwo, where PT is the preferred \isi{clause} type for marking climax and AT for backgrounding.

\section{\label{s:djenar:5}\textit{Di}-V-\textit{nya} and topic-focus distinction}

Some have argued that [\textit{di}-V-\textit{nya}] clauses with post-predicate patient are ergative (see \citealt[232--233]{Hopper1979}; \citealt[14]{ArkaEtAl1998}). However, others disagree (e.g., \citealt{Cumming1987}; \citealt[33--34]{Cumming1991}; \citealt{Kroeger2014}). The debate will not be entered into here; in this study I follow \citet{Cumming1991} in considering [\textit{di}-V-\textit{nya}] as a type of PT \isi{clause}. One of the common environments in which [\textit{di}-V-\textit{nya}] appears in \ili{Indonesian} is where it is preceded and followed by AT clauses. A switch from AT clauses to a [\textit{di}-V-\textit{nya}] \isi{clause} signals a shift to a \isi{new focus} and serves as an invitation to the addressee/reader to take notice. This is exemplified in (\ref{e:djenar:21}).

\begin{exe}
	\ex\label{e:djenar:21}
	\gll \textit{Almashira}\textsubscript{i} \textbf{\textit{mem-buka}} \textit{gorden} \textit{biru} \textit{muda} \textit{bermotif} \textit{vineyard},   \textbf{\textit{me-natap-i}} \textit{tetes} \textit{hujan} \textit{deras} \textit{yang} \textit{mem-basah-i} \textit{jendela} \textit{kamar}-\textit{nya} \textit{yang} \textit{ter}-\textit{letak} \textit{di} \textit{tempat} \textit{yang} \textit{paling} \textit{private} \textit{di} \textit{rumah}-\textit{nya}, \textit{alias} \textit{di} \textit{loteng}. \textbf{\textit{Di-ambil-nya}}\textsubscript{i} \textit{sweter-nya}\textsubscript{j} \textit{yang} \textit{gombrong} \textit{dan} \textit{nyaman}, \textit{lalu} \textbf{\textit{me-makai-kan-nya}}\textsubscript{j} \textit{di} \textit{tubuh}-\textit{nya} \textit{yang} \textit{langsing}.\\
	Almashira \textsc{meN}-open curtain blue light have.pattern vineyard \textsc{meN}-gaze-\textsc{appl} drop rain hard \textsc{rel} \textsc{meN}-wet-\textsc{appl} window bedroom-3\textsc{sg} \textsc{rel} \textsc{ter}-position \textsc{loc} place \textsc{rel} most private \textsc{loc} house-3\textsc{sg} none.other.than \textsc{loc} upstairs \textsc{di}-get-3\textsc{sg} sweater-3\textsc{sg} \textsc{rel} loose    and comfortable then \textsc{meN}-put.on-\textsc{appl}-3\textsc{sg} \textsc{loc} body-3\textsc{sg} \textsc{rel} slim\\
	\glt ‘Almashira\textsubscript{i} \textbf{opened} the light blue curtain with a vineyard motif, \textbf{stared at} the raindrops falling on the window of her bedroom which is located in the most private section of her house, namely upstairs. \uline{\textbf{She}\textsubscript{i} \textbf{grabbed}} her loose and comfortable sweater\textsubscript{j}, and \textbf{put it}\textsubscript{j} \textbf{on} her slim body.’ \citep[7]{Karina2008}
\end{exe}

\noindent
Here, the initial \isi{clause} describes the protagonist \textit{Almashira} opening the curtain in her bedroom and subsequently staring at the raindrops falling on her window. \textit{Almashira} is the agent in the AT clauses in this sentence. The first \isi{clause} (referring to Almashira opening the curtain) is eventive, while the second (Almashira staring at the raindrops) is a process (descriptive). In the second sentence, a switch to PT with \textit{di}-V-\textit{nya} occurs. This switch marks a shift from a non-eventive \isi{clause} in the previous sentence (Almashira staring at the raindrops) to an eventive one (Almashira fetching her sweater). The switch to [\textit{di}-V-\textit{nya}] also coincides with a new sentence. The [\textit{di}-V-\textit{nya}] \isi{clause} is then followed by a return to AT. In this AT \isi{clause}, the agent, which is mentioned in the PT \isi{clause} with the \isi{clitic} \textit{–nya} ‘\textsc{3sg}’, is now rendered with zero, indicating the \isi{referent}’s status as topic. Meanwhile, the patient argument (the \isi{referent} of \textit{sweternya} ‘her sweater’) is now rendered also with the \isi{clitic} –\textit{nya} ‘\textsc{3sg}’. 

\citet[175]{Cumming1991} suggests that a switch to PT is usually associated with the “\isi{climactic portion}” of a story episode and creates a feeling of vividness or immediacy. However, this argument does not apply to (\ref{e:djenar:21}). The excerpt is taken from the opening paragraph of the novel \textit{Circa} \citep{Karina2008} and this is the first time that the reader is introduced to the protagonist and her actions. The switch from a descriptive \isi{clause} (Almashira staring at the rain drops) to the [\textit{di}-V-\textit{nya}] \isi{clause} (Almashira fetching her sweater) draws our attention to new information but does not indicate a climactic point in the episode. Moreover, the patient cannot be considered topical either as it has not been mentioned previously. The PT \isi{clause} merely introduces a new event that contrasts from the descriptive \isi{clause} in the previous sentence. Once this event has been introduced, a switch back to AT with zero agent follows. 

\largerpage
Compare (\ref{e:djenar:21}) with the example below from the novel \textit{Senja di Jakarta} by Mochtar Lubis published in 1970 (quoted without gloss in \citealt[174]{Cumming1991}) where [\textit{di}-V-\textit{nya}] marks the \isi{climactic portion} of the story episode. Here Dahlia’s actions of arranging Hasnah’s hair, taking a lipstick from her bag and applying it on Hasnah’s lips form successive actions and all are rendered in PT. The beginning and end of the episode are rendered in AT.

\begin{exe}
	\ex\label{e:djenar:22}
	\textit{Dahlia memegang rambut Hasnah, mengambil sisir, dan asyik mengerjakan rambut Hasnah. Mula-mula Hasnah membantah, akan tetapi tidak diperdulikan Dahlia. Setelah \textbf{rambut Hasnah} \textbf{disusunnya}, \textbf{diambilnya} \textbf{cat bibir} dari tasnya, dan \textbf{digincunya} \textbf{bibir Hasnah}. Dia mengambil kaca dari dinding, memegang kaca di depan Hasnah, dan berkata, “Nah, lihat, kan cantik?”}\\
	\glt ‘Dahlia grasped Hasnah’s hair, took a comb, and zealously did Hasnah’s hair. At first Hasnah rebelled, but (she) was ignored by Dahlia. After \textbf{she had arranged Hasnah’s hair}, \textbf{she took a lipstick} from her bag and \textbf{painted Hasnah’s lips}. She took a mirror from the wall, held the mirror before Hasnah, and said, “Now, look, aren’t you pretty?”’ 
\end{exe}

\noindent
Although the beginning and end of example (\ref{e:djenar:21}) are also marked by AT, the sense of immediacy generated by the series of quick, successive actions in (\ref{e:djenar:22}) is largely absent. In (\ref{e:djenar:21}), the use of [\textit{di}-V-\textit{nya}] merely signals to the reader that the protagonist is now performing an action that contrasts with the one before. I would argue that [\textit{di}-V-\textit{nya}] with post-predicate patient in both examples marks a \isi{new focus}. By using this \isi{clause} structure, the author is making a \isi{pragmatic assertion} (about what action the agent is performing and in relation to what kind of patient \isi{referent}), expecting the reader to know what the proposition is only after reading the \isi{clause}. This \isi{assertion} cannot be presupposed; so, for example, in (\ref{e:djenar:21}) Almashira fetching her sweater cannot be presupposed after reading about her opening the window and staring at the raindrops. Using [\textit{di}-V-\textit{nya}] with V>T order is a grammatically effective way of alerting the reader to the new information.    

In a similar example, shown earlier in (\ref{e:djenar:5}) and repeated below as (\ref{e:djenar:23}) with a fuller context, a switch from AT to PT with V>T order is found. Here a series of AT clauses is followed by a series of PT clauses. As with the previous examples, the switch from the AT to PT (the PT clauses are underlined in the \ili{English} translation) does not mark the climactic part of a story episode but rather, a shift in focus. 

\begin{exe}
	\ex\label{e:djenar:23}
	\gll \textit{Rosna} \textbf{\textit{me-langkah}} \textit{ke} \textit{dalam}, \textbf{\textit{mem-buka}} \textit{pintu} \textit{tengah}, \textit{lalu} \textbf{\textit{pergi}} \textit{menuju} \textit{kamar} \textit{mandi} \textit{setelah} \textit{ia} \textbf{\textit{meng-ambil}} \textit{handuk} \textit{yang} \textit{ter}-\textit{gantung} \textit{di} \textit{sampiran}. \textbf{\textit{Di-cuci-nya}} \textit{muka}-\textit{nya}, \textit{kemudian} \textbf{\textit{di-lap-nya}} \textit{kering}-\textit{kering}, \textit{lalu} \textit{kembali} \textit{ke} \textit{kamar}-\textit{nya}.\\
	Rosna \textsc{meN}-step to inside \textsc{meN}-open door middle then go toward room bath after 3\textsc{sg} \textsc{meN}-fetch towel \textsc{rel} \textsc{ter}-hang \textsc{loc} clothes.line \textsc{di}-wash-3\textsc{sg} face-\textsc{3sg} then \textsc{di}-wipe-3\textsc{sg}  dryly then return to room-\textsc{3sg}\\
	\glt ‘Rosna \textbf{stepped} inside, \textbf{opened} the door in the middle (of the room), and then \textbf{headed} towards the bathroom after she \textbf{fetched} a towel that was hanging on the rail. \uline{\textbf{She washed}} her face\textsubscript{i}, then \uline{\textbf{she wiped}} [ ]\textsubscript{j} dry, after which [ ]\textsubscript{i} returned to her room.’ \hfill \citep[119]{Krisna1977}
\end{exe}


\noindent\largerpage
Here, Rosna is described as performing a series of actions (stepping inside, opening the middle door, fetching a towel and going towards the bathroom), and these actions are rendered in AT clauses with \textit{meN}- verbs and an \isi{intransitive clause} with a bare verb (\textbf{\textit{pergi}} \textit{menuju kamar mandi} ‘headed towards the bathroom’). The switch to the [\textit{di}-V-\textit{nya}] PT \isi{clause} marks the beginning of a different series of actions (i.e., washing and drying the face). These are actions for the purpose of which the preceding series of actions are taken (i.e., Rosna walking to the bathroom and fetching a towel on the way are performed for the purpose of washing and drying her face). It is in this way that we can understand the PT clauses as being narratively significant. In the first PT \isi{clause}, the patient \textit{mukanya} ‘her face’ cannot be presupposed (Rosna could have washed her hands rather than her face, for instance). Furthermore, the actions denoted by the PT clauses, although significant, cannot be considered climactic. The excerpt describes the protagonist Rosna performing the series of actions in preparation to speak to Benda, the man she has been wishing to date. Although the actions of washing and drying the face are significant within the context of the episode where Rosna is concerned to making herself presentable to Benda, they are only part of the series of events that lead to the climax. Later in the episode, the reader is told that Benda, who is waiting for her in the living room while Rosna freshens herself, decides to leave suddenly and not engage romantically with her, leaving Rosna deeply disappointed. It is that part of the episode in my view that marks the \isi{climactic portion}.  

It is possible to render the focal action(s) rendered in PT in (\ref{e:djenar:21}) and (\ref{e:djenar:23}) as AT clauses, as shown in the reconstructed examples in (\ref{e:djenar:24}) and (\ref{e:djenar:25}), but these clauses merely denote a series of chronological events rather than signal a shift to a different event (or a new series of events) the author wants the reader to pay particular attention to.

\begin{exe}
	\ex\label{e:djenar:24}
	\gll \textbf{\textit{Ia}} \textbf{\textit{mengambil}} \textit{sweter-nya}\textsubscript{j} \textit{yang} \textit{gombrong} \textit{dan} \textit{nyaman}, \textit{lalu} \textbf{\textit{me-makai-kan-nya}}\textsubscript{j} \textit{di} \textit{tubuh}-\textit{nya} \textit{yang} \textit{langsing}.\\
	3\textsc{sg} \textsc{meN}-take sweater-3\textsc{sg} \textsc{rel} loose and comfortable then \textsc{meN}-put.on-\textsc{appl}-3\textsc{sg}  \textsc{loc} body-3\textsc{sg} \textsc{rel} slim\\
	\glt ‘\textbf{She} \textbf{grabbed} her loose and comfortable sweater, and \textbf{put it on} her slim body.’ \hfill (adapted from \citealt[7]{Karina2008})
\end{exe}

\begin{exe}
	\ex\label{e:djenar:25}
	\gll \textbf{\textit{Ia}} \textbf{\textit{men-cuci}} \textit{muka}-\textit{nya}, \textit{kemudian} \textbf{\textit{me}}-\textbf{\textit{lap}} \textit{kering}-\textit{kering}, \textit{lalu} \textit{kembali} \textit{ke} \textit{kamar}-\textit{nya}.\\
	3\textsc{sg} \textsc{meN}-wash face-\textsc{3sg} then \textsc{meN}-wipe-3\textsc{sg} dryly-\textsc{rdp} then return to room-\textsc{3sg}\\
	\glt ‘\textbf{She}\textsubscript{i} \textbf{washed} her face\textsubscript{j}, then \textbf{wiped} [ ]\textsubscript{j} dry, after which [ ]\textsubscript{i} returned to her room.’ \hfill (adapted from \citealt[119]{Krisna1977})
\end{exe}

\noindent
In the original examples we saw earlier in (\ref{e:djenar:23}), it is the entire PT clauses, and not individual elements within them, that are treated as focus. This exemplifies what \citet[233]{Lambrecht1994} calls “\isi{sentence focus}”, the type of focus where the scope extends to the entire \isi{clause}, and not only the predicate or an argument. To explain how the focus is determined, let us turn to example (\ref{e:djenar:2}) again, repeated below without the gloss.

\begin{exe}
	\ex\label{e:djenar:26}
	\textit{Almashira}\textsubscript{i} \textbf{\textit{membuka}} \textit{gorden biru muda bermotif vineyard,} [ ]\textsubscript{i} \textbf{\textit{menatapi}} \textit{tetes hujan deras yang membasahi jendela kamarnya yang terletak di tempat yang paling private di rumahnya, alias di loteng.} \textbf{\textit{Diambilnya}}\textsubscript{i} \textit{sweternya yang gombrong dan nyaman, lalu} [ ]\textsubscript{i} \textbf{\textit{memakaikannya}}\textsubscript{j} \textit{di tubuhnya}\textsubscript{i} \textit{yang langsing}.
	\glt ‘Almashira\textsubscript{i} \textbf{opened} the light blue curtain which a vineyard motif, [ ]\textsubscript{i} \textbf{gazed at} the drops of water coming from the pouring rain that is wetting her bedroom window which is located in the most secluded part of her house, namely upstairs. \uline{\textbf{She grabbed}} her loose and comfortable sweater, then [ ]\textsubscript{i} \textbf{put it}\textsubscript{j} \textbf{on} her\textsubscript{i} slim body’ \hfill \citep[7]{Karina2008}
\end{exe}

\noindent
Here we have the protagonist, \textit{Almashira}, mentioned in the initial \isi{clause} as opening the curtain in her bedroom. This \isi{clause} is rendered in a [\textit{meN}-V] AT \isi{clause}. The \isi{referent} of the name \textit{Almashira} is rendered as zero in the following [\textit{meN}-V] \isi{clause}, indicating that it is now treated as topic. In the [\textit{di}-V-\textit{nya}] PT \isi{clause} that follows, the \isi{referent} is referred to with enclitic –\textit{nya}, while in the subsequent “resumptive” [\textit{meN}-V] AT \isi{clause}, it is rendered as zero, once again indicating her status as topic. In this example, the entire PT \isi{clause} is the focus because the event denoted by it adds new information. The patient \textit{sweternya yang gombrong dan nyaman} ‘her loose and comfortable sweater’ cannot be the only focal element although it adds new information to the \isi{presupposition} ‘the person fetched x’, where ‘x =  \textit{sweternya yang gombrong dan nyaman}’, because saying so does not explain why [\textit{di}-V-\textit{nya}] occurs, and why the patient is in post-predicate position (V>T order). It is grammatically and semantically acceptable to have the patient placed in pre-predicate position: \textit{Sweternya yang gombrong dan nyaman dipakainya} ‘She put on her loose and comfortable sweater’ (T>V order). However, the switch to the PT \isi{clause} puts the highlight not only on the sweater but also on the fact that Almashira is grabbing it and putting it on, thus it is the event that is focal, not only the patient argument. The PT \isi{clause} is a grammatical device that informs the reader to direct their attention to a \isi{new focus}. In this sense, it is similar in function to presentatives marked by particle \textit{lah} in \ili{Classical Malay} (see \citealt[90]{Cumming1991} for examples). Cumming points out that the use of \textit{lah} to mark presentativeness is greatly diminished in \ili{Indonesian}. It could be that switching from AT clauses to a PT \isi{clause} is one mechanism that \ili{Indonesian} writers use for this purpose.

The focal status of the PT \isi{clause} can be shown by using an adapted version of the tests in \citet[223]{Lambrecht1994}. A focus element can occur on its own as an open proposition. Thus below, the entire PT \isi{clause} can occur as the open proposition in (\ref{e:djenar:27a}), but the predicate alone cannot do so, as shown in (\ref{e:djenar:27b}).

\begin{exe}
	\ex\label{e:djenar:27}
	\begin{xlist}
		\ex{Sentence-focus}\label{e:djenar:27a}\\
		\begin{tabular}{lll}
		Q: & \textit{Apa} \textit{yang} \textit{terjadi}? & ‘What happened?’\\
		A: & \textit{Diambilnya} \textit{sweternya}. & ‘She grabbed her sweater.’\\
		\end{tabular}
		\ex{Predicate-focus}\label{e:djenar:27b}\\
		\begin{tabular}{lll}
		Q: & \textit{Apa} \textit{yang} \textit{terjadi} \textit{pada} \textit{sweternya}? & ‘What happened to her sweater?’\\
		A: & \textsc{\textit{Diambilnya}} \textit{sweternya}. & ‘She grabbed.’\\
		\end{tabular}
	\end{xlist}
\end{exe}

\noindent
Applying the same tests to the first PT \isi{clause} in example (\ref{e:djenar:5}) and (\ref{e:djenar:23}), repeated below without the gloss, the open proposition in (\ref{e:djenar:29a}) makes sense, while (\ref{e:djenar:29b}) and (\ref{e:djenar:29c}) which put the stress on \textit{dicucinya} ‘she washed’ and \textit{mukanya} ‘her face’ respectively, are awkward. The proposition in (\ref{e:djenar:29b}) would be more appropriately responded to with an AT \isi{clause} (\textit{Dia} \textsc{mencuci} \textit{mukanya} ‘She \textsc{washed} her face’); similarly, a more suitable response to (\ref{e:djenar:29c}) would be to front the patient (\textsc{Mukanya} \textit{yang dicucinya} ‘It is \textsc{her face} that she washed’). This suggests that the focus of the first PT \isi{clause} extends to the entire \isi{clause} rather than being limited to the predicate or the patient argument.
   
\begin{exe}
	\ex\label{e:djenar:28}
	\textbf{\textit{Dicucinya}}\textsubscript{i} \textit{mukanya}\textsubscript{j}, \textit{kemudian} \textbf{\textit{dilapnya}}\textsubscript{i} \textit{kering}-\textit{kering}, \textit{lalu} \textit{kembali} \textit{ke} \textit{kamar\-nya}.
	\glt ‘\textbf{She}\textsubscript{i} \textbf{washed} her face, \textbf{she}\textsubscript{i} \textbf{wiped} {[ ]}\textsubscript{j} dry, then {[ ]}\textsubscript{i} returned to her room.’ \hfill \citep[119]{Krisna1977}
\end{exe}

\begin{exe}
	\ex\label{e:djenar:29}
	\begin{xlist}
		\ex{Sentence-focus}\label{e:djenar:29a}\\
		\begin{tabular}{lll}
			Q: & \textit{Apa} \textit{yang} \textit{terjadi}? & ‘What happened?’\\
			A: & \textit{Dicucinya} \textit{mukanya}. & ‘She washed her face.’\\
		\end{tabular}
		\ex{Predicate-focus}\label{e:djenar:29b}\\
		\begin{tabular}{lll}
			Q: & \textit{Apa} \textit{yang} \textit{dilakukannya}? & ‘What did she do?’\\
			A: & \textsc{\textit{Dicucinya}} \textit{mukanya}. & ‘\textsc{She} \textsc{washed} her face.’\\
		\end{tabular}
		\ex{Argument-focus}\label{e:djenar:29c}\\
		\begin{tabular}{lll}
			Q: & \textit{Apa} \textit{yang} \textit{dilakukannya}? & ‘What did she do?’\\
			A: & \textit{Dicucinya} \textsc{\textit{mukanya}}. & ‘She washed \textsc{her face}.’\\
		\end{tabular}
	\end{xlist}
\end{exe}

\noindent
The question that remains now is: what is the status of the second \textit{di}-V-\textit{nya} \isi{clause} (\textit{dilapnya} \textit{kering-kering} ‘she wiped it dry’)? I propose that in this case too, the focus extends to the entire \isi{clause}. Here, the patient in the preceding \isi{clause} (\textit{mukanya} ‘her face’) is now presupposed and occurs as zero, indicating its \isi{topic status}. If it were expressed, the \isi{clause} would appear as \textit{dilapnya mukanya kering-kering} or \textit{dilapnya} \textit{kering-kering mukanya}, both meaning ‘she wiped her face dry’. 

The \isi{topic status} of the patient \textit{mukanya} ‘her face’ can also be shown, for example, by applying a combination of left- and right-dislocation test (cf. \citealt[186]{Gundel2004}). \citet[186]{Gundel2004} state that “the structure most widely and consistently associated with topic marking is one in which the constituent referring to the topic of the sentence is adjoined to the left or right of a full sentence comment/focus”. I apply a combination of left- and right-dislocation to test the \isi{topic status} of the \isi{referent} of \textit{mukanya} ‘her face’. The resulting sentence is well-formed though, I submit, it would be more likely to occur in spoken language rather than written language (as \citet[188]{Gundel2004} also note with regard to right dislocation construction in \ili{Norwegian}). 

\begin{exe}
	\ex\label{e:djenar:30}
	\begin{xlist}
		\ex \textit{Dilapnya kering-kering}.
		\glt ‘She wiped [it] dry.’
		\ex \textit{Kalau mukanya, dilapnya kering-kering, mukanya itu}.
		\glt ‘As for her face, she wiped it dry, that face of hers.’
	\end{xlist}
\end{exe}

\noindent
One might argue that the topic in the \isi{clause} \textit{dilapnya kering-kering} is the \isi{referent} of the enclitic -\textit{nya} ‘\textsc{3sg}’ (i.e., the \isi{referent} of ‘Rosna’), and not the \isi{referent} of \textit{mukanya} ‘her face’. However, as mentioned, \textit{di}-V-\textit{nya} is conventional in the sense that this structure presupposes an action performed by some third person agent, therefore –\textit{nya} ‘\textsc{3sg}’ is presupposed as its structural element, not as the topic of the PT \isi{clause}. The same left-/right-dislocation topic test therefore does not apply. The resulting sentence below is nonsensical (note: in the test below, the enclitic must be rendered as a \isi{free pronoun} in the ‘as for’ phrase for well-formedness).

\begin{exe}
	\ex\label{e:djenar:31}
	\begin{xlist}
		\ex \textit{Dilapnya kering-kering}. 
		\glt ‘She wiped [it] dry.’
		\ex \textit{Kalau} \textbf{\textit{dia}}\textit{, dilapnya kering-kering,} \textbf{\textit{dia}} \textit{itu}. 
		\glt ‘As for \textbf{her}, she wiped [ ] dry, \textbf{she} is.’
	\end{xlist}
\end{exe}

\noindent
Treating [\textit{di}-V-\textit{nya}] clauses with V>T order (i.e., PT clauses with post-predicate patient) in terms of \isi{sentence focus} sits well with \posscitet{Hopper1979} argument that this structure marks foregrounding in \ili{Classical Malay}. The series of [\textit{di}-V-\textit{nya}] clauses with V>T order in (\ref{e:djenar:28}) and also those in (\ref{e:djenar:22}) denote focal events that move the stories along the temporal axis. By using a series of [\textit{di}-V-\textit{nya}] clauses the authors invite the reader to pay particular attention not to a single new event in the narrative, as is the case with a switch from AT clauses to a single PT \isi{clause}, but rather to sustain their attention throughout a series of events. To strengthen this argument we can compare with a switch from AT clauses to a single PT \isi{clause} in (\ref{e:djenar:32}). This example describes a young man feeling shocked at seeing a torn picture of an ex-girlfriend who had left him. The initial main \isi{clause} contains the serial verbs \textit{tertegun melihat} ‘stunned in seeing’. The [\textit{di}-V-\textit{nya}] PT \isi{clause} marks a shift from these process verbs to an event (picking up the picture). The return to an AT \isi{clause} after the PT \isi{clause} marks a return to a process (the agent staring at the picture). (Note: as indicated in the subscript, the \isi{referent} of –\textit{nya} in \textit{dipungut}\textbf{\textit{nya}} ‘\textbf{he} picked up’ is not coreferential with that in \textit{menatapi}\textbf{\textit{nya}} ‘he stared at \textbf{it}’ in the following \isi{clause}.) The PT \isi{clause} marks an event, while the AT clauses occur as descriptive clauses.

\begin{exe}
	\ex\label{e:djenar:32}
	\gll \textit{Cowok} \textit{berparas} \textit{kutu} \textit{buku} \textit{ini} \textbf{\textit{tertegun}} \textbf{\textit{me-lihat}} \textit{salah.satu} \textit{potongan} \textit{gambar} \textit{yang} \textit{terasa} \textit{begitu} \textit{familiar} \textit{di} \textit{ingatan}-\textit{nya}. \textit{Perlahan} \textbf{\textit{di-pungut-nya}}\textbf{\textsubscript{i}} \textit{ujung} \textit{kertas} \textit{itu}, \textbf{\textit{me-natap-i-nya}}\textbf{\textsubscript{j}} \textit{lama} \textit{dengan} \textit{berbagai}  \textit{luapan} \textit{emosi} \textit{bermain} \textit{di} \textit{refleksi} \textit{mata}-\textit{nya} \textit{yang} \textit{nanar}.\\
	guy have.the.look lice book this stunned \textsc{meN}-see one.of cut picture \textsc{rel} feel so familiar \textsc{loc} memory.3\textsc{sg} slowly \textsc{di}-pick.up-\textsc{3sg} corner paper that \textsc{meN}-stare.at-\textsc{appl}-\textsc{3sg} long with various explosion emotion play \textsc{loc} reflection eye-\textsc{3sg} \textsc{rel} wild\\
	\glt ‘This guy with the bookish face was stunned (at) \textbf{seeing} a piece of a picture which felt so familiar in his memory. Slowly \uline{\textbf{he picked up}} the corner of the paper, [ ] \textbf{stared at it} for a long time with all sorts of emotions reflecting in his wild eyes.’ \hfill \citep[9]{Karina2008}
\end{exe}

\noindent
Here, as in (\ref{e:djenar:28}), the PT \isi{clause} \textit{dipungutnya ujung kertas itu} ‘he picked up the corner of the paper’ is focal and its domain spans the entire \isi{clause}. In the “resumptive” AT \isi{clause}, the agent occurs as zero while the patient is expressed as –\textit{nya} ‘\textsc{3sg}’. However, this AT \isi{clause} with zero agent sounds rather awkward, but this is possibly due to the fact that the situation is narrated in the third person but incorporates the point of view of the character. The \isi{relative clause} \textit{yang terasa begitu familiar di ingatannya} ‘which feels so familiar in his mind’ is told from the point of the view of the narrator but the situation is viewed from the perspective of the character (as indicated by the scalar/intensifier adjective \textit{begitu familiar} ‘so familiar’). This technique of presenting point of view, known as ‘free indirect \isi{discourse}’ (see e.g., \citealt{Cohn1978}; \citealt{Fludernik1993}; \citealt{Maier2014}), is widely used in fictional texts.

To summarise, I have argued in this section that the switch from an AT \isi{clause} to a [\textit{di}-V-\textit{nya}] PT \isi{clause} marks a shift in focus, signalling a new event (or a series of new events) that contrasts with what occurs before. The next section shows that [\textit{di}-V-\textit{nya}] is not the only PT form used for this purpose. The \isi{OV} structure [\textit{ia} V] serves a similar function but with an added pragmatic function of creating an impression of structural symmetry. 

\section{\label{s:djenar:6}PT clauses with \textit{ia}, repetition, and parallelism}

In this section, I consider the interaction between \isi{OV} clauses with the free third person \isi{pronoun} \textit{ia} ‘\textsc{3sg}’, [\textit{ia} V], and AT clauses containing the same \isi{pronoun} [\textit{ia} \textit{meN}-V] to show that, similar to the switch from [\textit{meN}-V] AT clauses to a [\textit{di}-V-\textit{nya}] \isi{clause} (or a series of clauses), the switch from [\textit{ia meN}-V] to [\textit{ia} V] marks a shift of focus. The use of [\textit{ia} V] also serves an additional function, namely producing a “synchronising” effect between the \isi{OV} and AT clauses through an appearance of structural similarity. In both of these \isi{clause} types, the agent \textit{ia} ‘\textsc{3sg}’ is in pre-predicate position. When \textit{ia} ‘\textsc{3sg}’, either in [\textit{ia} V] or [\textit{ia} \textit{meN}-V], is repeated across clauses, the repetition creates an impression of parallel structures.\largerpage

Consider example (\ref{e:djenar:33}). Here Alma, the sister of the guy with the bookish face mentioned in (\ref{e:djenar:32}), is described as sympathetic to her brother’s situation and doing her best to console him. An AT \isi{clause} with  [\textit{ia meN}-V] occurs in the first sentence, followed by [\textit{ia} V] \isi{OV} \isi{clause} in the first \isi{clause} of the second sentence. Subsequent clauses in this sentence are rendered in [\textit{meN}-V] once again but with zero agent. In the previous examples we saw a PT \isi{clause} being preceded and followed by AT clauses; here we see a similar pattern of AT clauses preceding and following an \isi{OV} \isi{clause}, marking the \isi{OV} \isi{clause} as focal.

\begin{exe}
	\ex\label{e:djenar:33}
	\textit{Sesaat} \textbf{\textit{ia}} \textit{juga} \textbf{\textit{merasakan}} \textit{emosi yang sama. Dengan perasaan tak menentu}, \uline{\textbf{\textit{ia}}} \uline{\textbf{\textit{palingkan wajahnya}}}, \textbf{\textit{merenung}}, \textit{lalu kembali} \textbf{\textit{menepuk}} \textit{punggung abangnya dengan ceria}.
	\glt ‘For a moment \textbf{she}\textsubscript{i} also \textbf{felt} the same emotion. Feeling uncertain, \uline{\textbf{she}\textsubscript{i}} \uline{\textbf{turned} \textbf{away}}, [ ]\textsubscript{i} \textbf{thought for a while}, and then [ ]\textsubscript{i} \textbf{tapped} her brother on the back cheerfully.’ \hfill 
	\mbox{\citep[9]{Karina2008}}
\end{exe}

\noindent
The \isi{referent} of \textit{ia} in the \isi{OV} \isi{clause} \textit{ia palingkan wajahnya} ‘lit. she turned her face away’ is coreferential with \textit{ia} in the preceding sentence \textit{ia juga merasakan emosi yang sama} ‘she also felt the same emotion’; in both cases the \isi{referent} is presupposed and treated as topic. In this \isi{clause}, it is the entire \isi{clause} \textit{ia} \textit{palingkan wajahnya} that constitutes the focus. The NP patient \textit{wajahnya} ‘her face’ cannot, on its own, be treated as the focus because \textit{palingkan} ‘turn away’ and \textit{wajahnya} ‘her face’ are in relations of entailment (\textit{memalingkan} ‘turn away’ entails \textit{memalingkan} \textit{wajah} ‘turning (one’s) face away’). Thus in the test for focus shown in (\ref{e:djenar:34}), interpretation (\ref{e:djenar:34a}) makes sense, while (\ref{e:djenar:34b}) is odd. 

\begin{exe}
	\ex\label{e:djenar:34}
	\gll \textit{Ia} \textit{paling}-\textit{kan} \textit{wajah}-\textit{nya}.\\
	\textsc{3sg} turn.away-\textsc{appl} face-\textsc{3sg}\\
	\glt ‘She turned away.’
	\begin{xlist}
		\ex\label{e:djenar:34a}
		\begin{description}
			\item\textsc{Pragmatic presupposition}: ‘the person did x’
			\item\textsc{Pragmatic assertion}: ‘x = turn away’
		\end{description}
		\ex\label{e:djenar:34b}
		\begin{description}
			\item\textsc{Pragmatic presupposition}: ‘the person turned away x’
			\item\textsc{Pragmatic assertion}: ‘x = her face’
		\end{description}
	\end{xlist}
\end{exe}

\noindent
The occurrence of the \isi{free pronoun} \textit{ia} ‘\textsc{3sg}’ in the AT and \isi{OV} clauses is significant not only because the \isi{pronoun} is the controller for the zero in the following [\textit{meN}-V] clauses, but importantly, because the repetition of this \isi{pronoun} in the different \isi{clause} structures creates “stylistic alignment” through structural parallelism. Repetition, as \citeauthor{Tannen2007} argues, is a common device for promoting a heightened sense of involvement in spoken and written \isi{discourse}, and “rhythmic synchrony” is an important and widespread feature in both conversational and written fictional \isi{discourse} \citep[32]{Tannen2007}. Repetition and parallel structures in texts also aid processing \citep{Carlson2002}. By using the [\textit{ia} V] \isi{OV} \isi{clause} rather than [\textit{di}-V-\textit{nya}] PT \isi{clause}, the author thus achieves several goals simultaneously: signalling a \isi{new focus}, creating a \isi{rhythmic effect} and thus promoting heightened involvement by stylistically aligning the \isi{OV} \isi{clause} with the preceding and following AT clauses, and highlighting the continuity of the third person agent. The series of clauses in this example are reproduced below.

\begin{exe}
	\ex\label{e:djenar:35}
		\begin{tabular}[t]{ll}
			\textbf{\textit{ia}}\textsubscript{i} \textbf{\textit{me}}\textit{rasakan emosi yang sama}  & ‘\textbf{she}\textsubscript{i} felt the same emotion’\\
			\textbf{\textit{ia}}\textsubscript{i} \textit{palingkan wajahnya} & ‘\textbf{she}\textsubscript{i} turned her face away’\\
			{[ ]}\textsubscript{i} \textbf{\textit{me}}\textit{renung} & ‘{[ ]}\textsubscript{i} thought for a while’\\
			{[ ]}\textsubscript{i} \textbf{\textit{me}}\textit{nepuk punggung abangnya} & ‘{[ ]}\textsubscript{i} tapped her brother’s back’
		\end{tabular}
\end{exe}

\noindent
It is important to mention here that repetition and parallel structures can be achieved not only by aligning [\textit{ia meN}-V] with [\textit{ia} V]. In the earlier examples, we saw that the repetition of [\textit{meN}-V] AT clauses and [\textit{di}-V-\textit{nya}] PT clauses to denote a series of events by the same agent can also produce a similar effect. What is interesting about [\textit{ia} V] \isi{clause} in (\ref{e:djenar:33}) is that, the effect is achieved through an “impression of sameness” between this \isi{OV} \isi{clause} and the [\textit{ia meN}-V] AT \isi{clause}. This sameness is created by the repetition of the \isi{free pronoun} \textit{ia} ‘\textsc{3sg}’ in pre-predicate position. While multiple occurrences of the same \isi{clause} structure denote a series of actions performed in quick succession by the same agent (\citealt[174]{Cumming1991}; also see \citealt[225]{Kaswanti1988}), the pragmatic and stylistic effects that result from the use of different \isi{clause} structures, to my knowledge, have not been discussed. Taking these effects into account helps us understand why different grammatical structures are chosen, not only what types of events they encode. 

Both \citet[175--178]{Cumming1991} and \citet[222--228]{Kaswanti1988} have noted that the position of the patient in PT clauses tells us something about its \isi{information status}. The patient in [\textit{ia} V] \isi{OV} clauses can occur in post-predicate position, as seen earlier in (\ref{e:djenar:33}), or in pre-predicate position, as in (\ref{e:djenar:7}). According to \citet[222]{Kaswanti1988}, one of the environments in which pre-predicate patient can occur is where there is a gap between the last mention of the NP and the current mention.\footnote{According to \citet[222]{Kaswanti1988}, the other two environments in which the patient occurs preverbally are a) when two NPs are contrasted and b) when an NP is promoted from oblique position to argument position.}  \citet[177]{Cumming1991} also notes that the patient in this position is treated as “topical”, i.e., the \isi{referent} has been mentioned “in the immediately previous \isi{clause} or several clauses back, and is of some importance” (\citeyear{Cumming1991}: 179). Both of these observations are borne out in an example from the acclaimed novel \textit{Cantik itu Luka} ‘Beauty is a Wound’ by Eka Kurniawan, given below. I include the excerpt in (\ref{e:djenar:36}) to give a context for the \isi{OV} \isi{clause} V>T order in (\ref{e:djenar:37}). 

The excerpt begins with the description of Dewi Ayu, the protagonist, waking early to go to the toilet. A wealthy woman who made her money from working as a prostitute, Dewi Ayu was taken to a prison camp along with others from her village when the \ili{Japanese} came to Indonesia. Not wanting to lose her precious jewellery, she decided that the best way to safeguard it was to swallow some of the pieces (six gold rings) and she buried the rest in the broken toilet at the back of her house. She did this before leaving for the camp. At the camp, she safeguards the rings by retrieving them from her faeces when she defecates and swallowing them again afterwards. The [\textit{ia meN}-V] AT transitive clauses occur twice.

\begin{exe}
	\ex\label{e:djenar:36}
	\textbf{\textit{Dewi Ayu}} \textit{yang terbiasa bangun pagi sekali untuk buang air segera bergegas ke toilet, namun antrian panjang telah menunggu. [….] Cara terbaik adalah mengambil air dengan kaleng margarin Blue Band-nya, dan pergi ke halaman belakang sel. Di sana, di antara pohon ketela yang entah ditanam siapa}, \textbf{\textit{ia}} \textbf{\textit{menggali}} \textit{tanah seperti seekor kucing, dan berak di lubangnya. Setelah cebok dengan menyisakan sedikit air,} \textbf{\textit{ia}} \textbf{\textit{mengorek}} \textit{tainya untuk menemukan keenam cincinnya. Beberapa perempuan lain melihat cara beraknya yang buruk, dan menirunya dalam jarak yang cukup berjauhan; mereka tak tahu ia punya harta karun.}\\
	\glt ‘Dewi Ayu, who is used to getting up very early to pass stool, quickly rushed to the toilet, but a long queue was already waiting [….] The best alternative is (for her to) get water using the Blue Band margarine tin, and go to the yard behind the cell. Over there, between cassava trees planted by some unknown person, \textbf{she} \textbf{dug} the ground like a cat, and defecated in the hole. After rinsing herself and leaving some water aside, \textbf{she} \textbf{poked through} her faeces to find her six rings. Some women saw how bad her manner of passing stool was, and [ ] imitated her from a distance; they didn’t know she had some treasure.’ (\citealt[67]{Kurniawan2002}; my translation)
\end{exe}

\noindent
Immediately following (\ref{e:djenar:36}) is the sentence with an \isi{OV} \isi{clause} and pre-predicate agent, shown in (\ref{e:djenar:37}). This \isi{clause} is followed by another a [\textit{meN}-V] AT \isi{clause} but with zero agent. 

\begin{exe}
	\ex\label{e:djenar:37}
	\gll \textit{Cincin-cincin} \textit{tersebut} \textbf{\textit{ia}}\textsubscript{i} \textbf{\textit{cuci}} \textit{dengan} \textit{sisa} \textit{air}, \textit{dan} \textbf{\textit{me-nelan-nya}}\textsubscript{j} \textit{kembali}.\\
	ring-\textsc{rdp} aforementioned 3\textsc{sg} wash with leftover  water  and  \textsc{meN}-swallow-\textsc{3sg} again\\
	\glt ‘\uline{\textbf{She washed}} the rings with the remaining water, and [ ]\textsubscript{i} \textbf{swallowed them}\textsubscript{j} again.’ \hfill \citep[67]{Kurniawan2002}
\end{exe}

\noindent
In the \isi{OV} \isi{clause}, the pre-predicate patient \textit{cincin-cincin tersebut} ‘the rings’ is the topic. It occurs as a definite NP and the \isi{referent} is presupposed, having been mentioned in the previous \isi{discourse}. Its \isi{topic status} can be shown for example, by applying the same left- and right-dislocation test from \citet{Gundel2004} that we applied to the previous examples.

\begin{exe}
	\ex\label{e:djenar:38}
	\textit{Kalau cincin-cincin tersebut, ia cuci dengan sisa air, cincin-cincin itu}.\\
	\glt ‘As for the rings, she rinsed (them) with the remaining water, those rings.’
\end{exe}

\noindent
The \isi{OV} predicate \textit{ia cuci dengan sisa air} is focal in relation to the topic expression \textit{cincin-cincin tersebut} ‘the rings’. The test below shows that the predicate can stand alone as an open proposition in answer A, whereas the [\textit{ia} V] phrase in answer B is slightly odd when it occurs on its own as a response to the question Q. 

\begin{exe}
	\ex\label{e:djenar:39}
	\begin{tabular}[t]{ll}
		Q: & \textit{Apa yang terjadi pada cincin-cincin tersebut}?\\
		& ‘What happened to the rings?’\\
		A: & \textit{Ia cuci dengan sisa air}.\\
		& ‘She washed (them) with the remaining water.’\\
		B: & \textit{Ia cuci}.\\
		& ‘She washed (them).’
	\end{tabular}
\end{exe}

\noindent
This analysis supports the idea that the relation between topic and focus is in some respects like that between “topic” and “comment” \citep[175]{Gundel2004} in topic-comment constructions. The [\textit{ia} V] \isi{OV} \isi{clause} is similar to topic-comment in that the pre-predicate patient (\textit{cincin-cincin tersebut} ‘the rings’) is the topic, and the elements to its right (\textit{ia cuci dengan sisa air} ‘she washed with the remaining water’) are a comment about the topic. Also the pre-predicate patient is “topical” in Cumming’s sense in that the \isi{referent} is accorded some importance in the episode, and this is reflected in its syntactic position. As indicated in (\ref{e:djenar:36}), the six rings are significant in relation to the protagonist; they are valuable belongings the protagonist wants to protect. They are also important in the subsequent episode where the reader is told that the rings saved Dewi Ayu financially. After being released from the camp, she was able to survive by pawning them and the rest of the jewellery she could retrieve.

The “topical” status of the patient \isi{referent} (in Cumming’s sense) can also be inferred from the previous \isi{discourse} and does not need to have been explicitly mentioned in prior \isi{discourse}. The excerpt in (\ref{e:djenar:40}) describes the character Kliwon tidying up his room in preparation for leaving home. The example is taken from an episode where Kliwon is depicted as a young, budding left-wing activist about to leave his hometown to take up university study in Jakarta. Among his belongings are books that are scattered around his room. The pre-predicate patient \textit{buku-buku} ‘books’ is the topic of the second sentence. The \isi{referent} of this NP is presupposed by virtue of its association with the patient NP in the preceding sentence (\textit{semua barang-barangnya} ‘all his belongings’) and inferable from a mention in prior \isi{discourse} where the reader learns that Kliwon has been given a book by his socialist mentor.

\begin{exe}
	\ex\label{e:djenar:40}
	\gll \uline{\textbf{\textit{Ia}}} \textit{telah} \textbf{\textit{me-ngemas}} \textit{semua} \textit{barang}-\textit{barang}-\textit{nya}, \textit{dengan} \textit{harapan} \textit{jika} \textbf{\textit{ia}} \textit{pergi} \textit{me}-\textit{ninggalkan}-\textbf{\textit{nya}}, \textit{semua} \textit{dalam} \textit{keadaan} \textit{rapi}. \textbf{\textit{Buku-buku}} \textit{yang} \textit{semula} \textit{berserakan} \textit{di} \textit{atas} \textit{tempat} \textit{tidur}, \textit{meja} \textit{dan} \textit{lantai} \uline{\textbf{\textit{ia}}} \textbf{\textit{masukkan}} \textit{ke} \textit{dalam} \textit{kotak} \textit{kardus} \textit{dan} \textbf{\textit{me-numpuk-nya}} \textit{rapi} \textit{di} \textit{sudut} \textit{kamar}.\\
	\textsc{3sg} \textsc{perf} \textsc{meN}-pack all belonging-\textsc{rdp}-\textsc{3sg} with hope if \textsc{3sg} go \textsc{meN}-leave-\textsc{3sg/pl} all \textsc{loc} condition neat book-\textsc{rdp} \textsc{rel} previously scattered \textsc{loc} top place sleep table and floor \textsc{3sg} put.in to inside box cardboard and \textsc{meN}-stack.up-\textsc{3sg/pl} neat \textsc{loc} corner room\\
	\glt ‘\uline{\textbf{He}} has \textbf{packed up} all his belongings, with the hope that if he has to go away leaving \textbf{them} behind, all would be in a neat condition. \uline{\textbf{He put}} \textbf{the books} which were lying on the bed, table and floor in cardboard boxes and [ ] \textbf{stacked them up} neatly in the corner of the room.’ \citep[195--196]{Kurniawan2002}
\end{exe}

\noindent
The example begins with an eventive AT \isi{clause} containing completive marker \textit{telah}. Two successive events follow, encoded by eventive \isi{OV} \isi{clause} with pre-predicate patient and eventive AT \isi{clause} respectively. The pre-predicate position of the patient \textit{buku}-\textit{buku} ‘books’ suggests that the switch to \isi{OV} is made mainly to direct the reader’s attention to a \isi{new focus}. After this, the author resumes his use of AT. As with (\ref{e:djenar:37}), the multiple occurrences of \textit{ia} create an impression of parallel structures, and together with the zero in the final \isi{clause}, highlight the continuity of the human agent. 

To summarise this section, the foregoing discussion has shown [\textit{ia} V] \isi{OV} clauses can be used to introduce a \isi{new focus}, similar to [\textit{di}-V-\textit{nya}] PT clauses discussed in the previous section. In both cases, the \isi{clause} highlights the agentivity of the human topic \isi{referent}.\footnote{Agentivity is understood here in the sense of \citet[21]{Cruse1973} as referring to “an action performed by an object which is regarded as using its own energy in carrying out the action”, where “object” includes “living things, certain types of machine, and natural agents”.}  The difference is that, the repetition of the \isi{free pronoun} \textit{ia} ‘\textsc{3sg}’ across \isi{OV} and AT clauses gives a “synchronising” effect; that is, it makes both [\textit{ia} V] \isi{OV} clauses and [\textit{ia} \textit{meN}-V] AT clauses appear structurally similar. In contrast, the occurrence of a [\textit{di}-V-\textit{nya}] PT \isi{clause} among [\textit{meN}-V] AT clauses marks the event encoded by it as structurally different. One could argue, then, that [\textit{di}-V-\textit{nya}] is more marked than [\textit{ia} V]. Further research could support this argument with quantitative evidence.

\section{\label{s:djenar:7}Conclusion}

In this paper I have discussed different \isi{clause} structures – [\textit{meN}-V] AT clauses, PT clauses with [\textit{di}-V-\textit{nya}] and [\textit{di}-V], and [\textit{ia} V] \isi{OV} clauses (considered here as a type of PT) – to show that in \ili{Indonesian}, \isi{eventiveness} can be encoded by means of any of these structures. Thus the preference for AT clauses in \ili{Indonesian} cannot be understood narrowly as a preference for eventive clauses, or that \isi{eventiveness} is coded only by AT clauses. The foregoing analysis shows that each \isi{clause} type has similar as well as different functions, summarised as follows. First, [\textit{meN}-V] clauses used to denote successive events, either involving the same agent or different agents in transitive clauses and actor or \isi{undergoer} in intransitive clauses, and to mark the \isi{climactic portion} of a story episode, illustrate most dramatically the preference in \ili{Indonesian} for agentive clauses in SVO order. \citet[226]{Kaswanti1988} points out that \textit{meN}- verbs tend to occur in subordinate, backgrounding clauses. Meanwhile, \citet[203]{Cumming1991} found that the “basic” \isi{clause} type in \ili{Indonesian} novels is AT with T>V order; PT clauses with V>T order, though still used to encode \isi{eventiveness}, are marked. As we saw, [\textit{meN}-V] occurs in both eventive/foregrounding and backgrounding clauses. Precisely whether and how [\textit{meN}-V] AT clauses are used to encode \isi{eventiveness} in \ili{Indonesian} is a question that needs to be explored further. For now, it will have to be sufficient to say that these clauses can be used to encode both.

Second, PT clauses with [\textit{di}-V-\textit{nya}] remain an important resource for encoding \isi{eventiveness} in \ili{Indonesian} but their function is now mainly to signal a \isi{new focus}, drawing the addressee/reader’s attention to a particular event (or series of events) that contrasts with the event or state that comes before. In \sectref{s:djenar:4} I showed that [\textit{di}-V] clauses occur with pre-predicate patient, indicating the patient’s “topical” status. It may be that these clauses are mainly used to highlight patient continuity, while [\textit{di}-V-\textit{nya}] is primarily used to emphasise a particular event the agent is performing. Further research would be able to confirm this early observation. 

Third, \isi{OV} clauses with [\textit{ia} V] structure are also used to encode \isi{eventiveness}. Its co-occurrence with [\textit{ia meN}-V] AT clauses emphasises the continuity of the third person agent and creates an impression of structural symmetry. The repetition of \textit{ia} ‘3\textsc{sg}’ across two different \isi{clause} types produces a \isi{rhythmic effect} much desired in fictional texts. To what extent rhythm and structural symmetry are also important considerations in other types of texts (e.g., news reporting) is an interesting topic that can be explored in a larger study. Such a study would also be able to inform us whether, with the weakening in pragmatic force of [\textit{di}-V-\textit{nya}], [\textit{ia} V] is now preferred for marking focal events.

The majority of the data on which this study is based are taken from fictional narrative. Whereas \citet{Shiohara2015} tested Cumming’s findings using experimental data, I have drawn on fictional narrative to approximate the kind of data that Cumming used in drawing her generalisations. The results presented here might invite questions regarding genre; that is, the arguments I have raised might be objected to as valid only for narrative \isi{discourse}. However, I hope to have demonstrated that they are also relevant for television news \isi{discourse}. Finally, stylistic preferences can vary greatly between language users, as \citet[174--175]{Cumming1991} herself has also noted. By including news \isi{discourse} and fictional texts by different authors I hope to have shown that the choice of \isi{constituent order} is not just a matter of individual style.

\section*{Acknowledgements}

I am grateful to Sander Adelaar, Michael Ewing, Nerida Jarkey, Umar Muslim, and two anonymous reviewers for taking the time to patiently read the earlier versions of this paper and offering insightful comments. I have tried my best to incorporate their suggestions but no doubt inconsistencies remain, and for these I alone claim responsibility.

\section*{Abbreviations}

\begin{multicols}{2}
	\begin{tabbing}
		glossgloss \= \kill
		1 \> first person\\
		2 \> first person \\
		3 \> third person\\
		\textsc{appl} \> applicative\\
		\textsc{def} \> definite article\\
		\textsc{dir} \> directional\\
		\textsc{indef} \> indefinite\\ 
		\textsc{loc} \> locative\\
		\textsc{neg} \> negation\\
		\textsc{perf} \> perfective\\
		\textsc{rdp} \> reduplication\\
		\textsc{rel} \> \isi{relative clause}\\
		\textsc{sg} \> singular
	\end{tabbing}
\end{multicols}

\sloppy
\printbibliography[heading=subbibliography,notkeyword=this]

\end{document}
