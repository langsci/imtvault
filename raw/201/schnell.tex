\documentclass[output=paper
,modfonts
,nonflat]{langsci/langscibook} 

\ChapterDOI{10.5281/zenodo.1402539}

\title{Attention focus and information packaging in Vera{\textquotesingle}a demonstratives} 
\author{Stefan Schnell\affiliation{Centre of Excellence for the Dynamics of Language \& University of Melbourne}}
% \chapterDOI{} %will be filled in at production

% \epigram{}

\abstract{I investigate endophoric uses of demonstratives in the Oceanic language Vera{\textquotesingle}a (North Vanuatu) in relation to their exophoric use, drawing on different types of linguistic data, in particular conversations during the building of a house and narrative texts. I find that Vera{\textquotesingle}a demonstratives mark a three-way distinction of speaker-oriented (excluding the addressee) versus addressee-oriented (including the speaker) versus distal (excluding both speaker and addressee). Seeking to determine the core meaning of the three demonstratives that would square with both their exophoric and their endophoric use, I develop the hypothesis that considerations of joint attention focus rather than localisation are its central aspect: speaker-oriented forms are used to \emph{draw} the addressee's attention to an entity, addressee-oriented forms to \emph{maintain} attention to an entity, and distal forms to \emph{delay} attention to an entity. The joint attention meaning of demonstratives in endophoric use yields effects in the domain of information packaging: speaker-oriented forms are used to introduce new information; addressee-oriented forms in specific syntactic contexts yield effects akin to topic and focus; distal forms are used to keep track of background information. Hence, attention focus is the invariable core meaning of Vera{\textquotesingle}a demonstratives that accounts for all senses in different contexts of use.}

\begin{document}

\maketitle
\section{Introduction} 
The role of demonstratives in \isi{discourse} has been the focus of a growing body of corpus-based research, starting with \cite{Himmelmann1996}. Recent work has been undertaken for the \ili{Australian} language \ili{Dalabon} by \cite{Cutfield2012} and the \ili{Oceanic} language Logea by \cite{Dawuda2009}. The central question concerning the endophoric (inner-textual) usage of demonstratives is its relationship to their \isi{exophoric} (\isi{deictic}) usage. One line of thought sees \isi{discourse} essentially as a metaphorical extension of space (\cite{Diessel1999} and \cite[121]{Buehler1934}), with spatial distance being equivalent to \isi{anaphoric} (antecedent) distance, and \isi{referent} activation. An example of this is \cite{Terrill2001}, who examines the function of two demonstratives, \textit{foia} and \textit{oia}, in \isi{discourse} from \ili{Lavukaleve}: she finds that the use of these two forms in endophoric use mirrors their function to mark distance distinctions (proximal versus distal) in situational use, so that spatial distance in the physical environment corresponds to antecedent distance within \isi{discourse}, for instance how many clauses lie between an anaphor and its antecedent within a given text. Other authors stress the speaker/addressee orientation of demonstratives, but still see them merely as anchor points for marking spatial relations between speaker, \isi{referent}, and addressee. This view is reflected in \cite{Margetts2004}. Yet other authors also focus on the speaker/addressee orientation of demonstratives, but see the use of demonstratives as a reflection of communicative interaction that can override `objectively' determined spatial circumstances (see \citealt{Enfield2003} on \ili{Lao}). In this line of thought, \cite{Mosel2004} shows that \ili{Samoan} speaker-oriented demonstratives are often used to introduce new information into \isi{discourse}, whereas speaker-plus-addressee-oriented demonstratives are used for given referents, reflecting the knowledge state of speaker and addressee during \isi{discourse} production/comprehension rather than their location in space relative to each other or to the entity referred to. This paper lends further support to this latter view, and it links the significance of knowledge states of interlocutors to that of \isi{joint attention}. On this view, demonstratives can be characterised as linguistic devices that serve to ``coordinate the interlocutors' joint focus of attention'' \citep[481]{Diessel2006}.


In this paper I present findings from a study of the use of demonstratives in the \ili{Oceanic} language Vera{\textquotesingle}a. I argue that Vera{\textquotesingle}a demonstratives do not have spa\-tial-lo\-cal\-ising but attention-focussing meaning. Typical for an \ili{Oceanic} language from Melanesia, Vera{\textquotesingle}a has a rich inventory of interrelated \isi{demonstrative} forms that can be divided into three functionally defined sets of forms: 1. speaker-oriented (\textsc{dem.spkr}), 2. addressee-oriented (\textsc{dem.addr}), and 3. neither speaker- nor addressee-oriented (distal) (\textsc{dem.dist}). In instances of \isi{exophoric} reference, these forms do not merely serve the localisation of an entity in terms of their distance relative to either speaker, addressee, or both. Rather, it is the awareness state of participants in the communicative situation that is crucial for their \isi{exophoric} use, serving to coordinate the attention focus of speaker and addressee \citep{Diessel2006}. Thus, addressee-oriented forms are used where speaker and hearer share awareness of a \isi{referent} and attend to it, even when the \isi{referent} is located in the speaker's \textit{here-space}. Conversely, speaker-oriented forms can be used where a \isi{referent} is at a large distance from both speaker and addressee if the speaker believes the addressee to not be aware of or not attending to a \isi{referent} they want them to attend to. Distal forms are used to refer to something that speaker is not focussing their attention on. 



Crucially, the same considerations of attention focus account for the endophoric use of demonstratives in narrative texts \citep[476]{Diessel2006}: here, speaker-oriented forms are restricted to two cases of \textit{Deixis am Phantasma} \citep[121--140]{Buehler1934}, a subtype of situational use in \cite[222]{Himmelmann1996}  (see \sectref{demvraexouse} for a more detailed characterisation): 1. announcements of a narrative that is about to be produced by the speaker, and 2. cases of direct speech, where the narrated situation is rendered in analogy to the physical context of the actual speech situation. These uses are classified as \isi{exophoric} here. In endophoric use, the choice between addressee-oriented and distal forms is sensitive to referential distance; however, the use of addressee-oriented forms can yield information-packaging effects in specific \isi{discourse} contexts, often corroborated with other structural properties, for instance left-dislocation and emphatic particles. I argue here that these information-packaging effects do not constitute an extended meaning of demonstratives, but follow straightforwardly from their attention-focussing meaning: speaker-oriented forms are used cataphorically to draw the addressee's attention to the narrative that is about to ensue, whereas addressee-oriented forms are used anaphorically, but serve to reinforce \isi{joint attention} to already shared and activated information. This latter function then contributes to information-packaging effects, which are essentially a matter of pragmatic inference and the semantics of given syntactic and \isi{discourse} contexts. Addressee-oriented forms are thus not simply used to mark the activation of a \isi{referent}, and other forms of reference -- typically pronouns -- are used for activated referents. Distal forms, on the other hand, are used to re-activate a \isi{referent}.

The paper unfolds as follows: \sectref{section2} provides a brief overview of the Vera{\textquotesingle}a language and its speakers, and the corpus data underlying this study. In \sectref{section3}, I first define the class of demonstratives in Vera{\textquotesingle}a, and related \isi{demonstrative} adverbs derived from them, and describe their syntactic distribution. I then turn to the description of \isi{exophoric} uses, followed by endophoric uses. In \sectref{section4}, I discuss the findings with regard to spatial and attention-focus accounts of demonstratives. In \sectref{section5}, I conclude the paper, and provide an outlook on future research.

\section{Language data and methodological notes} \label{section2}
Vera{\textquotesingle}a (ISO 639-3: vra) is an \ili{Austronesian} language of the \ili{Oceanic} subgroup spoken by 500 people on the island of Vanua Lava, the largest of the Banks group of islands in North Vanuatu. It belongs to what \cite{KalyanFrancoisfc} call the `Vanuatu linkage'. Vera{\textquotesingle}a is spoken in close vicinity to Vur\"es \citep{Malau2016} and \ili{Mwotlap} \citep{Francois2001}, and adult speakers are usually fluent in the former, and often at least understand the latter. All speakers of Vera{\textquotesingle}a above the age of approx. 3 years speak \ili{Bislama}, the \ili{English}-lexified contact language that has the status of Vanuatu's `national language'. Nonetheless, Vera{\textquotesingle}a is not severely endangered and it is very vital: most children grow up (until the age of about 3) monolingually with Vera{\textquotesingle}a, which is used virtually in all domains of everyday communication, except for some church services. Immigrants into the Vera{\textquotesingle}a community usually acquire Vera{\textquotesingle}a, but may continue to use their first language with their children in their homes (this concerns mainly women from neighbouring communities who marry into the Vera{\textquotesingle}a community) \citep[cf.][]{Francois2012}. The language is endangered only due to the small size of the community, so that a language shift, for instance to Vur\"es, could potentially occur within a short time frame if the positive attitude towards the language should falter in the future.

This study draws on different types of text data in the Vera{\textquotesingle}a language documentation that I have been compiling since December 2006 in close collaboration with speakers of the language (\url{http://dobes.mpi.nl/projects/vures\_veraa/}). The documentation encompasses recordings of a wide range of communicative events, ranging from narratives -- produced mainly for the sake of  being documented -- to public speeches and (very few) casual conversations, which represent the most naturally occurring communicative events. Particularly relevant for this study are a set of video recordings documenting the construction of a larger house (approx. 4 hours), audio recordings of plant (150 descriptions, 30sec average recording length) and fish (254 descriptions, 1.5min average length) descriptions, and a collection of narrative texts (65 narratives of different lengths), most of which were only audio-recorded. Video recordings exist for a handful of narratives, and I will point to specific visual clues where relevant in the discussion to follow. Note, however, that for the purposes of this study, neither gestures nor prosodic structures were taken into account systematically, and a systematic and sound analysis of these two aspects of language use will have to be left for future work on the language. The time-aligned annotation of the audio signal in all recordings was segmented into pause units.

A smaller set of  10 narratives from different male and female speakers of different ages -- some traditional legends and myths, some more similar to (moral) fables in the \ili{European} tradition -- contain rich morphosyntactic annotations that serve as the basis for some quantitative examination into the use of demonstratives in Vera{\textquotesingle}a. The texts show some variation in length, ranging from 178 to 929 \isi{clause} units. \tabref{table1} gives an overview.

\begin{table}
\caption{GRAID-annotated corpus of Vera{\textquotesingle}a narrative texts}
\label{table1}
 \begin{tabular}{llllr} 
  \lsptoprule
    {Text ID} & {Speaker} & {age} 	& {text} 		& {no. of} \\
    & {initials} & {group} & {type}	&	{clause units}		\\
  \midrule
    
\textsc{anv} 		&	AN	& 18--25		& narrative &	208 \\
\textsc{as.1}			&	AS		& 35--50	&	narrative	&	224 \\
\textsc{gabg} & GA		& 35--50	&	narrative	& 178 \\
\textsc{gaqg} &	GA	&	35--50	&	narrative	& 232 \\
\textsc{hhak} &	HH	&	18--25		&	narrative	& 446 \\
\textsc{isam}	& 	IS			&	\phantom{25--}50+				&	narrative	& 248 \\
\textsc{iswm}	& IS			& \phantom{25--}50+				&	narrative	& 608 \\
\textsc{jjq} 			&	JJ				& \phantom{25--}50+				&	narrative	& 929 \\
\textsc{mvbw} &	MV	&	25--35		&	narrative	& 314 \\
\textsc{pala} 		&	PH		&	35--50	&	narrative	& 402 \\
		\midrule
{Totals}	& &&&			{3789} \\
  \lspbottomrule
 \end{tabular}
\end{table}

\noindent
All of these ten narratives are part of the Vera{\textquotesingle}a corpus \cite{Schnell2016a} within the Multi-CAST collection \sloppy{(\citealt{HaigSchnell2016a}; \url{https://lac.uni-koeln.de/de/multicast/})}. Like all other text corpora in Multi-CAST, the Vera{\textquotesingle}a texts have been annotated according to the GRAID conventions \citep{HaigSchnell2014}, reading `Grammatical Relations and Animacy in Discourse'. This annotation scheme has been designed specifically for purposes of research into \isi{referentiality}, argument structure and \isi{discourse structure}, as pioneered by Wallace Chafe and Talmy Giv\'on, and associates. Essentially, GRAID annotations couple glosses for form (zero, \isi{pronoun}, full NP) with those for syntactic functions (s, a, p, etc.), as well as information on person and animacy. The Vera{\textquotesingle}a corpus also contains annotation of phrasal sub-constituents, merely indicated by symbols for ``word to the right of NP head'' (\textlangle rn\textrangle) and ``word to the left of NP head'' (\textlangle ln\textrangle), and likewise for constituent words in complex verbal predicates. Also, adjuncts and other clause-level constituents, like adverbs and particles, have been annotated as \textlangle other\textrangle.

In the study of demonstratives reported here, GRAID annotations were used systematically together with morphemic glossing: since both layers of annotation are \textit{symbolically associated} (and the respective type has been used for the GRAID tier relative to the morphemic glossing tier in ELAN), all \isi{demonstrative} forms can be searched with respect to their syntactic function, either as an argument, a clause-level adjunct/particle, or an NP-level satellite constituent. Thus, using ELAN's layered search function, one can easily determine figures for and all instances of say \textsc{dem.addr} forms occurring on NP level by searching for \textlangle dem.addr\textrangle\  on the morphemic glossing tier and \textlangle rn\textrangle\ on the GRAID tier. It was thus possible to determine a total of 759 instances of \isi{demonstrative} forms in use in this corpus, comprising 528 basic demonstratives and 231 \isi{demonstrative} adverbs. Since GRAID triggers basic features of referring expressions, we can also compare quantitatively the use of demonstratives in relation to NPs without demonstratives and other forms, in particular zero and \isi{pronoun}. This latter investigation is, however, beyond the scope of this paper, and is planned to be undertaken in the future.

In this paper, I generally draw on the wider Vera{\textquotesingle}a documentation, discussing different types of data in turn. It will thus be clear during the discussion from which type of data a specific example has been drawn. The Vera{\textquotesingle}a Multi-CAST corpus serves merely to establish a quantitative picture of syntactic distributions of different \isi{demonstrative} forms in narrative texts.


\section{Demonstratives in Vera{\textquotesingle}a} \label{section3}
In this section, I first define demonstratives in Vera{\textquotesingle}a following \cite{Himmelmann1996} and describe the inventory of basic demonstratives and derived \isi{demonstrative} adverbs in \sectref{demvradefinv}, and their syntactic distribution in \sectref{sec:demvrasyntdistr}. I then turn to their \isi{exophoric} and endophoric uses in \sectref{demvraexouse} and \sectref{demvraendouse} respectively. My use of these two terms is adopted from \cite[33]{HalidayHasan1976} in the following way: exophora is a direct relationship between a linguistic expression and an extra-linguistic entity; this is essentially the same as deixis. Endophoric relations can be defined in analogy to the definition of anaphora, as stated by \cite[1]{Huang2000}: 

\begin{quote} 
[An anaphora; added by Stefan Schnell] is a relation between two linguistic elements, wherein the interpretation of one (called an anaphor) is in some way determined by the interpretation of the other (called an antecedent) [...].
\end{quote}

\noindent
I use endophoric here as a hyperonym to both anaphora and cataphora, where the latter defines a relationship of ``carrying forward'' rather than backward. Like \cite{Huang2000} for anaphora, I understand endophoric relations as essentially quite vague and general, relying to a large degree on pragmatic inference. Thus, in determining whether a referring expression is \isi{exophoric} or endophoric (for instance a definite full noun phrase), interlocutors will seek different clues as to whether they require to consider the \isi{discourse} context of the same text, or whether they should resort directly to the physical surroundings of the speech situation in order to establish its reference. Although some form types are more common than others in one of the two types of relation, referential form classes are generally not specified for their type of reference. For instance, a personal \isi{pronoun} in Vera{\textquotesingle}a can be endophoric or \isi{exophoric}, regardless of the fact that the former use is possibly more common than the latter. Likewise, demonstratives are not \emph{per se} \isi{exophoric}. 


Exophoric uses of demonstratives are attested in two different communicative contexts and related recordings, namely in non-narrative contexts (house building videos, observations) and in or immediately connected to narratives. In the latter context, they come in two different types: 1. the introduction of a narrative, and 2. instances of \textit{Deixis am Phantasma}, in the sense of \citealp[131]{Buehler1934}:

\begin{quote}
\textit{[...] wenn ein Erz\"ahler den H\"orer [...] ins Reich der konstruktiven Phantasie f\"uhrt und ihn dort mit denselben Zeigw\"ortern traktiert, damit er sehe und h\"ore, was es dort zu sehen und zu h\"oren [...] gibt. Nicht mit dem \"ausseren Auge, Ohr usw., sondern mit dem, was man [...] das "`innere"' oder "`geistige"' Auge und Ohr zu nennen pflegt.} 
[[...] when the narrator leads the hearer [...] into the realm of constructive imagination and treats him to the same \isi{deictic} words as before so that he may see and hear what can be seen and heard there [...]. Not with the external eye, ear, and so on, but with what is [...] called the ``mind's'' eye or ear [...] (\ili{English} translation by Donald Goodwin and Achim Fraser Eschbach)]
\end{quote}

\noindent
Although we may expect this special kind of \isi{deictic} reference within direct speech to work somewhat differently from deixis within the actual speech act frame, I take the similarities eluded to by B\"uhler sufficient to treat them basically alike here (but I will point out to the reader what type is represented by any given example). Before turning to the \isi{exophoric} and endophoric uses of demonstratives, I shall provide a definition of these forms together with their inventory in Vera{\textquotesingle}a, and outline their syntactic distribution.



\subsection{Definition and inventory of demonstrative forms} \label{demvradefinv}
Vera{\textquotesingle}a distinguishes three basic forms of \isi{demonstrative}, marking a three-way opposition between speaker-oriented (\textsc{dem.spkr}), addressee-oriented (\textsc{dem.addr}), and distal (\textsc{dem.dist}). What is labelled here `addressee-oriented' is understood as essentially encompassing the sphere of both speaker and addressee, as will become clear below. The three forms fall under the definition put forward by \cite[210]{Himmelmann1996}, albeit with one qualification. According to Himmelmann, demonstratives have the following necessary criteria:

\newpage 
\begin{exe}
\ex Demonstratives are linguistic expressions that 
\begin{enumerate} 
\item	are in a paradigmatic relationship to elements that serve to locate a \isi{referent} on a distance scale, i.e. proximal, distal, etc.
\item	cannot be used with reference to entities that are uniquely identifiable in a culture (or the world), or in so-called associative anaphora, e.g. where an anaphor is a part of the whole expressed by the antecedent
\end{enumerate}
\end{exe}

\noindent\largerpage[-1]
As for criterion 1, in situational use \textsc{dem.spkr} forms can be used to identify a \isi{referent} that is located close to the speaker, whereas \textsc{dem.addr} forms can be used where it is further away from the speaker, possibly between addressee and speaker. The form \textsc{dem.dist} can be used for referents even further away from the speaker, possibly at a large distance from both speaker and addressee. Examples in \sectref{demvraexouse} will illustrate these points. While these uses concern the extension of demonstratives, their intentional definition in Vera{\textquotesingle}a does not draw precisely on the distance parameter, as will become clear in the sections to follow. Moreover, it seems that the Vera{\textquotesingle}a facts provide some illustrative evidence that the exophoric-situational use is not necessarily basic to the meaning of demonstratives, as assumed in earlier work by \citealp[93]{Diessel1999}. Instead, my findings support the view that the meaning of demonstratives is general with respect to \isi{exophoric} and endophoric uses \citep[cf.][242]{Himmelmann1996}, and is indeed grounded in considerations of interlocutor interaction and `attention focus', as found by \cite{Enfield2003} for \ili{Lao} \citep[cf. also][]{Diessel2006}. \tabref{tabdemvrainv} summarises the inventory of basic and adverbial \isi{demonstrative} forms. Forms of \textsc{dem.spkr} and \textsc{dem.addr} alter between a `plain' form and a form prefixed with \textit{a-}. The latter is quite likely to represent a crystallised combination of the specific locative article \textit{a} and a \isi{demonstrative}. The formal distinction is relevant insofar as they correlate with constraints on syntactic distribution: for instance only the prefixed but not the plain forms of basic demonstratives can occur adnominally. Where a form is prefixed, this is indicated by \textsc{.a} in the gloss, for instance \textsc{dem.spkr.a} for the prefixed speaker-oriented base form.

\begin{table}
\caption{Inventory of Vera{\textquotesingle}a demonstrative forms}
\label{tabdemvrainv}
 \begin{tabularx}{\textwidth}{Xlllll}
  \lsptoprule
																	& \multicolumn{2}{c}{\textsc{dem.addr}} 				& \multicolumn{2}{c}{\textsc{dem.spkr}} 		& \multicolumn{1}{c}{\textsc{dem.dist}} \\ 

																	& \multicolumn{1}{c}{\textsc{plain}}			&	\multicolumn{1}{c}{\textsc{a-form}}		&	\multicolumn{1}{c}{\textsc{plain}}				&	\multicolumn{1}{c}{\textsc{a-form}}				&	\\
	\midrule
\textsc{basic}    				&   n\={e} 							& an\={e}     				&   g\={e}n(\={e})    	& ag\={e}n(\={e}) 		& \={e} \\
 \textsc{manner}   		& sen\={e} 							&	asen\={e}	 				& seg\={e}n(\={e}) 		& aseg\={e}n(\={e})  & 	\\
\textsc{temporal}		&  va{\textquotesingle}an\={e}  				&	      										&  va{\textquotesingle}ag\={e}n       			& 																& \\
\textsc{locative}    	& ({\textquotesingle}e)k\={e}n\={e} &		 											& ({\textquotesingle}e)k\={e}g\={e}n 	&   															& 	\\
  \lspbottomrule
 \end{tabularx}
\end{table}

\noindent
\tabref{tabdemvrainv} also lists \isi{demonstrative} adverbs in addition to basic demonstratives. These are derived from the base forms as follows: manner adverbs are derived by adding a prefix \textit{se-} whose exact meaning and origin are unclear at this stage. Temporal adverbs essentially consist of an adverb \textit{va{\textquotesingle}a} `already' plus the base form. Locational adverbs are combinations of the noun \textit{{\textquotesingle}ek\={e}} `place' and a basic \isi{demonstrative}. All three types of adverb can be used adnominally and occupy the same phrase-final slot as base forms (compare example (\ref{ex2})):

\ea	\label{ex1}
\gll n=	gie		si	=n	gengen	si	\textbf{=n}	\textbf{sava}	\textbf{sen\={e}} di		=m		w\=olw\=ol	ma		m\=e	no.		{\USOParen}0.6{\USCParen} \\
     \textsc{art}=	kava	or	 \textsc{=art} food or \textsc{=art} what	\textsc{man.dem.addr} \textsc{3sg}	\textsc{=tam1}	\textsc{red:}buy	hither		\textsc{dat}	\textsc{1sg}  \\
\glt `[He gave me money for the work.] Also kava, food, or such things, he bought for me.' \hfill{MVRP.080}
\z

\noindent
I will be dealing mainly with the use of base form demonstratives in this chapter. However, I will occasionally outline particular usage properties of \textsc{dem.spkr} and \textsc{dem.addr} by reference to adverbs where particularly illustrative examples show that the findings presented here apply to the wider \isi{demonstrative} system.

As for Himmelmann's second criterion, the forms in \tabref{tabdemvrainv} are indeed incompatible with the first usage context, but a single example is attested that is similar to the binding (associative) anaphora example mentioned by \cite[210]{Himmelmann1996}:

\ea \label{ex2}
\gll		q\=o\=n 		v\=owal 	e 		ruwa		m\=e	=n 	gunu-ru\=o	duru	=m		{\textquotesingle}\=ogo	{\textquotesingle}\=ogo	vaa-van	\textbf{=n}	\textbf{re\=ne}		\textbf{an\=e}		ne			wotoqtoqo.  \\
night		one		\textsc{pers}	\textsc{hum:dl}	\textsc{dat}	\textsc{=art}	spouse-\textsc{3dl} \textsc{3dl}	\textsc{=tam1}	stay	stay	\textsc{red}-go \textsc{=art}	woman	\textsc{\textsc{dem.addr.a}}	\textsc{aor:3sg} pregnant	\\
\glt	`One day, (there was) a couple. They stayed and stayed (and after some time) that woman got pregnant.'			\hfill{ANV.001--003}
\z

\noindent
In this example, the speaker first introduces the couple, and two clauses later refers to one in the couple, the woman,  using an adnominal \textsc{dem.addr}. This usage would disqualify the forms in question as demonstratives according to Himmelmann's second criterion. Thus, if we take this particular use of a \textsc{dem.addr} form as evidence against its status as a \isi{demonstrative} we will probably have to conclude that the language does not possess demonstratives at all, given the clear paradigmatic relation of the form with \textsc{dem.spkr} and \textsc{dem.dist} forms. This would in my view not be a plausible analysis, given the universality of demonstratives (and hence their expectedness in Vera{\textquotesingle}a) and the otherwise clear indications that the forms in question function like demonstratives in other languages. 

There are two possibilities to account for this `abnormal' use of a \isi{demonstrative}: firstly, we might be dealing simply with a production mistake, given that it is the only example in the entire corpus. Though this is somewhat speculative, it is not impossible that speakers occasionally `miscalculate' the identifiability of referents, believing that there is an eligible antecedent expression in the \isi{discourse} context, although there isn't \citep[72]{Ariel1990}. The second possibility is that Himmelmann's criterion is too narrow, and that Vera{\textquotesingle}a treats associative anaphora differently from other languages. Which of these two possibilities is more appropriate remains an open question at this point, and is left for future investigations of more, and possibly more varied, \isi{discourse} data to determine whether such uses are more common with some speakers or in some contexts than our current data suggests.

\subsection{Syntactic distribution of demonstrative forms} \label{sec:demvrasyntdistr}
For base form demonstratives, three syntactic positions and respective functions are attested: NP-internal modifier (adnominal use), argument on \isi{clause} level (pronominal use), and clause-level adverbs and particles. \tabref{tab:demvrasyntdistr} provides figures for the frequencies of uses of \textsc{dem.addr}, \textsc{dem.spkr}, and \textsc{dem.dist} in all three syntactic contexts, with the restriction that \textsc{dem.dist} cannot be used pronominally.


\begin{table}
\caption{Frequencies of basic demonstratives in 10 Vera{\textquotesingle}a narratives}
\label{tab:demvrasyntdistr}
 \begin{tabular}{ccccccccccc} 
  \lsptoprule
   \multicolumn{3}{c}{\textsc{dem.addr}}   &  \multicolumn{3}{c}{\textsc{dem.spkr}}		&	\multicolumn{2}{c}{\textsc{dem.dist}}		&	\multicolumn{3}{c}{{Totals}}	\\\cmidrule(lr){1-3} \cmidrule(lr){4-6}\cmidrule(lr){7-8}\cmidrule(lr){9-11}
	 \textsc{adn}	&	 \textsc{pro}	&	 other	&	 \textsc{adn}	&	 \textsc{pro}	&	 other	&	 \textsc{adn}	&	 other	&	 \textsc{adn}	&	 \textsc{pro}	&	 other \\
 \midrule
  255 &	20	&	127	&	21	&	4	&	10	&	74	&	17	&	350	&	24	&	154 	\\
  \lspbottomrule
 \end{tabular}
\end{table}



All three forms can be used adnominally, as modifiers in the noun phrase, as shown by the following examples (\ref{ex3}) and (\ref{ex4}):

\ea \label{ex3}
\gll		ne				{\textquotesingle}\=en	\textbf{e}		\textbf{ruwa}	\textbf{\=mal\=mala}	\textbf{\=e}		=s 		sag	{\textquotesingle}i		\textbf{e}		\textbf{ruwa} 	\textbf{\=mal\=mala}	\textbf{an\=e}		so	\\
\textsc{aor:3sg}	see		\textsc{pers}	\textsc{hum:dl}	girl		\textsc{dem.dist}	\textsc{=sim}	sit	\textsc{del}	\textsc{pers} 	\textsc{hum:dl}	girl		\textsc{\textsc{dem.spkr.a}}		\textsc{quot}	\\
\glt	`[\dots] and then (he) spotted those two girls (who were) sitting (there), and the two girls said: \dots'		\hfill{1.PALA.059--060}
\z

\ea \label{ex4}
\gll no		=k			kaka	\textbf{=n}	\textbf{nel\=no}	\textbf{vu{\textquotesingle}}	\textbf{ag\=en\=e} \\
		\textsc{1sg}	\textsc{aor:1sg}		story	\textsc{=art}	\isi{voice}		spirit	\textsc{dem.spkr.a} \\
\glt		`I am going to tell this (the following) story: \dots'						\hfill{ISWM.001}
\z

\noindent
The functions of these different forms in adnominal use will be discussed in \sectref{demvraexouse}	and \sectref{demvraendouse}. Pronominal use is restricted to \textsc{dem.addr} and \textsc{dem.spkr} forms. While \textsc{dem.addr} demonstratives occur as subjects (ex \ref{ex5}) and predicates (ex \ref{ex6}) in non-verbal clauses, all pronominal uses of \textsc{dem.spkr} are instances of subjects in non-verbal clauses (ex \ref{ex7}):
\ea \label{ex5}
\gll \textbf{n\=e}		=n	t\=ekt\=ek		mu-n			e		Qo{\textquotesingle}.	\\
		\textsc{dem.addr}	\textsc{=art}	\textsc{red:}speech	\textsc{poss.gen-cs}	\textsc{pers}	Q.	\\
\glt		`This was Qo{\textquotesingle}’s speech [what Qo{\textquotesingle} said].'				\hfill{JJQ.306}
\z
\ea	\label{ex6}
\gll		bul		wova{\textquotesingle}al	\=e 		di 		sa		\textbf{n\=e}		\\
		bunch		pawpaw	\textsc{dem.dist}	\textsc{3sg}	\textsc{emph}	\textsc{dem.addr}	\\
\glt	`That bunch of pawpaw (over there), it is exactly that [which we had in mind]'		\hfill{GABG.075}
\z
\ea	\label{ex7}
\gll	kamadu\=o \textbf{g\=en}.	\\
		\textsc{1dl.ex}	\textsc{dem.spkr}			\\
\glt	`This (is) us.'				\hfill{2.PALA.125}
\z

\noindent
Pronominal uses of demonstratives are very rare, in particular where \textsc{dem.spkr} forms are involved. These cases are in fact restricted to contexts of \textit{Deixis am Phantasma} in narrative texts, to be discussed in \sectref{demvraexouse}. This corroborates the findings from the sample presented in \cite[215]{Himmelmann1996} for texts from different languages.

Similar to other \ili{Oceanic} languages (\cite{Mosel2004} on \ili{Samoan}; \cite{Margetts2004} on Saliba; \cite{Francois2001} on \ili{Mwotlap}), basic demonstratives in Vera{\textquotesingle}a can occur on \isi{clause} level in clause-final position, for example:
\ea \label{ex8}
\gll		e 		raga 		an\=e		=k 		tek 	so				=n 	t\=oo 	me 	ma' 	\textbf{n\=e}		\\
		\textsc{pers}	\textsc{hum:pl}	\textsc{\textsc{dem.spkr.a}}	\textsc{=aor:nsg}	say	\textsc{quot}	\textsc{=art}	fowl	\textsc{fut}	dead	\textsc{dem.addr}	\\
\glt	`And they said: ``These chickens will die!'''			\hfill{ANV.037}
\z
\ea	\label{ex9}
\gll	dir{\textquotesingle}\=ol			=k 		\=on\=on 	{\textquotesingle}i 		\textbf{an\=e}		\\
		\textsc{3tl}	\textsc{=aor}	\textsc{red:}lie	\textsc{del}	\textsc{\textsc{dem.spkr.a}}	\\
\glt	`They lay down [and shortly after that \dots]'	\hfill{MVBW.109}
\z
\ea	\label{ex10}
\gll	e 		Qo{\textquotesingle} 	ne 			sag 	r\=en 		sag 	k\=ek\=e 		\textbf{\=e}	\\
		\textsc{pers}	Q.	\textsc{aor:3sg}	sit 		stuck		up		above		\textsc{dem.dist}			\\
\glt	`But Qo{\textquotesingle} is trapped up (in the tree), have you forgotten?'		\hfill{JJQ.151}
\z
\ea	\label{ex11}
\gll	kamam 			mi{\textquotesingle}ir 		sa 		\textbf{g\=en\=e} 		l\=e		=n 	wio.		\\
		\textsc{1pl.ex.tam1}		sleep		\textsc{emph}	\textsc{dem.spkr}	\textsc{loc}	\textsc{=art}	bamboo	\\
\glt	`Let’s sleep right here, in the bamboo.'			\hfill{JJQ.354}
\z

\noindent
In examples (\ref{ex8}) and (\ref{ex9}), \textsc{dem.addr} forms occur in clause-final position. In (\ref{ex8}), it functions as an assertive particle that serves to underscore the speaker's conviction about the truth of a proposition. In (\ref{ex9}), it fulfils a clause-combining function, details of which will be discussed in \sectref{demvraendouse}. The same function is carried out by \textsc{dem.dist} in this position (ex \ref{ex10}). Basic \textsc{dem.spkr} forms always function as locative adverbs when occurring on \isi{clause} level (ex \ref{ex11}).

Demonstrative adverbs are used mainly adverbially on \isi{clause} level (category `other' in \tabref{vrademadvdistrfreq}). However, they do occur -- in rare instances -- as modifiers in the noun phrase. \tabref{vrademadvdistrfreq} lists the relevant figures for frequencies of use.


\begin{table}
\caption{Distribution and frequency of demonstrative adverbs in 10 narratives}
\label{vrademadvdistrfreq}
 \begin{tabularx}{.7\textwidth}{Xrrrrrr} 
  \lsptoprule
            & \multicolumn{2}{c}{\textsc{dem.addr}}  & \multicolumn{2}{c}{\textsc{dem.spkr}} 	&	\multicolumn{2}{c}{\textsc{Totals}}		\\\cmidrule(lr){2-3}\cmidrule(lr){4-5}\cmidrule(lr){6-7} 
						&	\textsc{adn}	&	other	&	\textsc{adn}	&	other	&	\textsc{adn}	&	other	\\
  \midrule
\textsc{manner}		& 2 & 57	&	3		&	24	&	5	& 81	\\
\textsc{temporal}& 1	&	61	&	0	&	5		&	1	&	66	\\
\textsc{locative}	&	4	&	57	&	1	&	16	&	5	&	73	\\
		\midrule
\textsc{totals}			&	7	&	175	&	4	&	45	&	11	&	220	\\
  \lspbottomrule
 \end{tabularx}
\end{table}

I confine myself here to presentation of a single example of a locative adverb in adnominal function (see also example (\ref{ex1}) above):
\ea \label{ex12}
\gll		\textbf{{\textquotesingle}er\={e}}	\textbf{{\textquotesingle}ama{\textquotesingle}}	\textbf{{\textquotesingle}ek\=en\=e}	dir	=\=em		van 	se-serge			ma.	\\
		\textsc{pl}		devil	\textsc{loc.dem.addr}	\textsc{3pl}	\textsc{=tam1}	go	\textsc{red-}together		hither	\\
\glt		`The spirits (of) here, they came together here.'		\hfill{ISWM.281--282}
\z

\noindent
In the following subsections, I will focus on the use of basic demonstratives, and only occasionally turn to the use of \isi{demonstrative} adverbs.

Before we turn our attention to the different uses of demonstratives, it should be noted that adnominal demonstratives are the only NP-level element that directly influences the referential interpretations of NPs. Pre-head articles are merely phrase markers whose only function is the marking of the type of phrase, and the subtype of NP (common versus personal). NPs with adnominal demonstratives are definite, and have an identifiable \isi{referent}. We will see below that this is merely an entailment of demonstratives, and not their core meaning. NPs without demonstratives are likewise regularly used in contexts where their \isi{referent} is identifiable, but they are also compatible with discourse-new reference. As for pronominal uses of demonstratives, these may be relatively rare in the corpus because third person pronouns are regularly used for \isi{anaphoric} and -- to a lesser degree -- \isi{deictic} reference to identifiable entities. As indicated above, this paper will be confined to the development of a first hypothesis concerning the use and meaning of Vera{\textquotesingle}a demonstratives. A systematic corpus study of the use of demonstratives in competition with other types of referring expressions will be left for future work on the language.


\subsection{Exophoric use of Vera{\textquotesingle}a demonstratives}	\label{demvraexouse}
The \isi{exophoric} functions of demonstratives are often seen as basic, for instance in \cite{Diessel1999}. While it is the \isi{exophoric} use of demonstratives that defines the class as a whole in \posscitet{Himmelmann1996} definition, it is questionable whether spatial considerations are part of the core meaning of demonstratives. Himmelmann himself doubts this, as do do studies like \cite{Enfield2003}, as well as later work by \cite{Diessel2006}. My findings concerning the use of demonstratives in Vera{\textquotesingle}a lend further support to the view that the basic function of demonstratives is in interlocutor interaction and attention focussing.\largerpage[-1]

The purpose of this section is to provide an outline of \isi{exophoric} uses of the three demonstratives in Vera{\textquotesingle}a. I will here draw on two types of data, namely the house building video and observation where speakers point to entities and locations in the physical environment of the recording. Secondly, I will illustrate the same patterns of \isi{demonstrative} use in instances of \textit{Deixis am Phantasma} in the direct speech of characters pointing to entities within the narrated reality. The last type of \isi{exophoric} use to be discussed here is in a narrative's frame, in particular where a narrative is announced just prior to its narration, and where it is wrapped up after the last events have been reported.

It should be noted here that at this stage, no more systematic investigation of situational uses of demonstratives -- like the studies by \cite{Enfield2003} on \ili{Lao} and \cite{Margetts2004} on Saliba -- has been carried out for Vera{\textquotesingle}a as yet. This is, however, not strictly necessary for understanding their use in narrative \isi{discourse}, as will become clear below. The basic distinctions of speaker orientation, addressee orientation, and distal interpretation is relatively clear from the few examples in the video recordings (the house-building recording and a subset of narratives), and is the only one relevant for their use in narrative \isi{discourse}. We will see in some examples here that even in \isi{exophoric} use, information-related categories are more relevant than spatial considerations.

Vera{\textquotesingle}a demonstratives are participant-oriented, which does not so much draw on distance relative to the speaker, but on considerations as to whether the addressee (or the speaker) can be regarded as sharing some \textit{here-space} with the \isi{referent}. In this system, \textsc{dem.spkr} forms are strictly associated with the speaker, as can be seen from the following two examples where the \textsc{dem.spkr} form contributes the location of an object near the speaker, but outside of the immediate perception of the addressee:  
\ea \label{ex13}
\gll	\textbf{woqe{\textquotesingle}enge} 	\textbf{ag\=en\=e} 		{\textquotesingle}er\=e 	mom 	kal 	{\textquotesingle}i 			{\ob}\dots{\cb}	\\
		wood			\textsc{dem.spkr.a}	\textsc{pl}		put	up	\textsc{del}					\\
\glt	`This wood here [in speaker's hand], put (it) up, I have just cut (it).'\\							\hfill{HouseBuilding10\_FQ.005}
\z
\ea \label{ex14}
\gll 	n		gasel	luwo	mu-m			\textbf{g\=en}	\\
		\textsc{art}	knife	big		\textsc{poss.gen-2sg}	\textsc{dem.spkr}		\\
\glt		`Your bush knife is here [where I am]!'			\hfill{observed}
\z

\noindent
Example (\ref{ex13}) is taken from a house building recording. In this situation the speaker is holding a piece of wood in his hand and is about to pass it up to someone standing on the scaffold, so that it becomes a strut in the house's rafter. In uttering (\ref{ex13}), he brings this new piece of roof structure to the other person's attention. Examples like (\ref{ex14}) can be observed daily in the Vera{\textquotesingle}a community. Here, the speaker is sitting in a sleeping house and shouts out to the addressee, who is searching for his knife outside and inside the kitchen house. One crucial aspect concerning the use of \textsc{dem.spkr} is that it seems to entail the speaker's assumption that the addressee is not attending to the entity in question (and is possibly not familiar with it). This is illustrated by the following examples of direct speech involving situational use within narratives. Here, the speaker is drawing the addressee's attention to something that is not only clearly not in their \textit{here-space}, but is also presented in such a way that the addressee is not aware of it:
\ea \label{ex15}
\gll 	ruwa		\textbf{=n}	\textbf{n\=o-k}			\textbf{\=en}		\textbf{b\=e}		\textbf{ag\=en\=e}	n=	b\=e 	n\=o-k 				an\=e		ga		mana	n\=e.	\\
\textsc{hum:dl}	\textsc{=art}		\textsc{poss.dom-1sg}	\textsc{art}	water	\textsc{dem.spkr.a}	\textsc{art=}	water	\textsc{poss.dom-1sg}		\textsc{\textsc{dem.spkr.a}}	\textsc{stat}	power	\textsc{dem.addr} \\
\glt	`Hey you two, I have some water here. This water of mine has magic powers.'		\hfill{AS.1.100--101}
\z
\ea \label{ex16}
\gll \textbf{n} 		\textbf{q\=o} 	\textbf{g\=en\=e} wo	\textbf{=n}	\textbf{ma\=nra}	\textbf{g\=en\=e} 	no		me	gis	gid\=e		=k		van.	\\
\textsc{art}	pig	\textsc{dem.spkr}	and	\textsc{=art}		money	\textsc{dem.spkr}	\textsc{1sg}	\textsc{fut}	hold	\textsc{1pl.in}	\textsc{=aor}		go	\\
\glt	`I have a pig here, and some money here. I'll take it and then let's go.'\\													\hfill{ISWM.219}
\z

\noindent
In both examples, the \isi{referent} is presented as new to the addressee. Once attention has been established in (\ref{ex15}), the speaker uses the \textsc{dem.addr} form in the following \isi{clause} which further elaborates on the \isi{referent}. The data examined thus far lend themselves to the following hypothesis: \textsc{dem.spkr} forms are used where an entity is only in the \textit{here-space} of the speaker, excluding that of the addressee, and where the addressee is not yet attending to that entity.

By contrast, \textsc{dem.addr} forms are used where the \isi{referent} is not exclusively in the speaker's \textit{here-space}, for instance where addressee also has perceptual access to it or where it is in the addressee's \textit{here-space}. The following two examples illustrate this:
 
\ea	\label{ex17}
\gll 	o			\textbf{k\=en\=e} 			\textbf{n\=e} 	ga 	t\=el\=egl\=eg.		\\
\textsc{interj}	\textsc{loc.dem.addr}	\textsc{dem.addr}	\textsc{stat}	askew			\\
\glt	`No, there, that one is (still) uneven [not horizontal].'\\ \hfill{HouseBuilding04\_PB.008}
\z
\ea	\label{ex18}
\gll		{\ob}\dots{\cb}	le		rieg	\textbf{\=en}		\textbf{\=nara}		\textbf{gasel}		\textbf{an\=e}		{\ob}\dots{\cb}		\\
				{}	take	out	\textsc{art}	mouth	knife	\textsc{\textsc{dem.spkr.a}}	\\
\glt	`Remove that (blade of that) knife!.'							\hfill{HouseBuilding09\_JJ.009}
\z

\noindent
Again, both examples stem from the house building video recordings. In (\ref{ex17}), the speaker is first looking at the measuring string between two posts on the other side of the house's plane. The string is used to check whether both posts have the same height, and here the speaker states that it is not horizontal yet. The adjustment of the string is the main thing that is happening at this stage, and the speaker can assume everybody to be attending to it. In (\ref{ex18}), the speaker -- standing outside the scaffold -- is referring to the knife lying in the middle of the house`s plane, surrounded by other builders. One of these in fact then approaches the knife, apparently following the request. Everybody in this scene can be regarded as being aware of the knife lying there. We can assume that the use of a \textsc{dem.spkr} form would have triggered the addressees' turning their heads towards the speaker, the \textit{here-space} exclusive to the speaker.

Equivalent examples can be found in direct speech within narrative texts where both characters, the speaker and the addressee, are rendered as being aware of an entity and attending to it:
\ea	\label{ex19}
\gll		\textbf{n\=e}		=n		\=en		men	n\=e		\\
\textsc{dem.addr}	\textsc{=art}	\textsc{art}	bird	\textsc{dem.addr}		\\
\glt	`[discussion among characters after hearing some noise out in the bush] ``That [what we are hearing here] is a bird, believe me!'' (There is a bird singing.)'				\hfill{MVBW.061}
\z
\ea \label{ex20}
\gll		o 		bul	w\=ew\=e	wova{\textquotesingle}al	ne		v\=owal 	sar\=en\=e 	ga 	mine	\textbf{ne{\textquotesingle}\=e} 	sa		gidu		me	gen	n\=e	\\
\textsc{interj}	friend	fruit	pawpaw	\textsc{num}	one		up.there	\textsc{stat}	ripe		\textsc{dem.addr}	\textsc{emph}	\textsc{1dl.in}	\textsc{fut}	eat		\textsc{dem.addr}		\\
\glt	`Oh friend, that one ripe pawpaw fruit up there, that's the one we will eat, I tell you.'	\hfill{GABG.051--052}
\z

\noindent
The second example also illustrates the omnipresent potential ambiguity between exophoric-situational and endophoric-tracking use: as will become clear below, \textsc{dem.addr} forms are also the ones used for given-activated referents -- i.e. those referents already attended to by both speaker and addressee -- and \textsc{dem.addr} forms could here be understood as referring to the entity both characters attend to in the depicted physical environment, or the \isi{referent} in the depicted \isi{discourse} of direct speech. Example (\ref{ex15}) illustrates a similar ambiguity in the second \isi{clause}, where the magic water just presented to the two is taken up again in the second \isi{clause}.

The following examples of direct speech illustrate the same functional aspect of \textsc{dem.addr} forms, this time used adnominally:
\ea	\label{ex21}
\gll		dir		=m		gal	no		ma	no		=m		rem	a			\textbf{=n}		\textbf{woqe'enge}	\textbf{an\=e}.	\\
\textsc{3pl}	\textsc{=tam1}	lie	\textsc{1sg}	hither		\textsc{1sg}	\textsc{=tam1}	climb	\textsc{loc.sp}	\textsc{=art}	tree			\textsc{dem.spkr.a}		\\
\glt	`They tricked me into coming here, and I climbed up this tree.'			\hfill{JJQ.182--183}
\z
\ea	\label{ex22}
\gll			{\textquotesingle}ei			bulsal	nik	\=e			dada		s\=ev\=ee	l\=e		\textbf{=n}		\textbf{b\=e}		\textbf{nanara}	\textbf{an\=e}	\\
\textsc{interj}	friend		\textsc{2sg}	\textsc{aor:2sg}	\textsc{red:}do	how	\textsc{loc}	\textsc{=art}	trunk	tree.sp		\textsc{dem.spkr.a}	\\
\glt	`Hey, friend, what are you doing on this Nanara trunk here?'	\hfill{GAQG.034}
\z

\noindent
Summarising the observations concerning the use of \textsc{dem.addr}, the following can be stated: 1. the entity in question is not clearly located in the speaker's \textit{here-space} and is not clearly excluded from that of the addressee, and 2. both speaker and addressee attend to it, or are at least aware of it.\largerpage

Comparing the use of \textsc{dem.spkr} and \textsc{dem.addr} forms, we can conclude that the former are used where the entity is excluded from the addressee's \textit{here-space} and is being brought to the addressee's attention just at the time of utterance, and the latter is used where the converse situation holds, i.e. where an entity is not excluded from the addressee's \textit{here-space} and both speaker and addressee are already attending to it at the beginning of the utterance. This hypothesis would be falsified, for instance, by examples where a \textsc{dem.addr} form is used for an entity clearly in the \textit{here-space} of the speaker but not the addressee -- for instance where both are separated by a wall or so, as in example (\ref{ex14}) above -- or where a \textsc{dem.spkr} form is used when clearly both speaker and addressee were already attending to the entity at the beginning of the utterance. Such examples have not been identified in the data.

The decisive factor in the use of these two forms is the inclusion versus exclusion of the addressee, rather than the speaker's \textit{here-space}. This is illustrated by the following example where the \isi{referent} of the NP `head post' is in the speaker's \textit{here-space}, but not excluded from that of the addressee. The speaker here appeals to the fact that the addressee -- in this case myself behind/next to the camera -- is attending to the post, pointing the camera at it and him:
\ea	\label{ex23}
\gll		di 		ne 		{\textquotesingle}i-{\textquotesingle}iris 	e 		raga 		a 		sag 	k\=ek\=e 		a		=n 	w\=olw\=olo \=e	n\=e		=n 	w\=oqa{\textquotesingle}ag	qi{\textquotesingle}i 	\textbf{n\=e} 	no		=s 		{\textquotesingle}ar{\textquotesingle}ar 	an\=e.	\\
		\textsc{3sg}	\textsc{aor:3sg}	\textsc{red-}even	\textsc{pers}	\textsc{hum:pl}	\textsc{loc.sp}		up		above		\textsc{loc.sp}	\textsc{=art}	cross		\textsc{dem.dist}	\textsc{dem.addr}	\textsc{=art}	post			head		\textsc{dem.addr}	\textsc{1sg}		\textsc{=sim}		\textsc{red}:carve	\textsc{dem.addr.a}	\\
\glt	`So that it is becoming even with the others up there on that cross (beam). This is the head post, you know, that one I am cutting.'\hfill{HouseBuilding08\_JJ.002--003}
\z

\noindent
This shows that \textsc{dem.addr} forms are used where the addressee is assumed to be aware of or attending to the entity in question, for instance looking straight at it as in this example. The \textsc{dem.spkr} forms are, on the other hand, restricted to uses where only the speaker has perceptual access to or is aware of the entity in question, as was illustrated by example (\ref{ex14}) above. The relevant functional aspect of \textsc{dem.spkr} forms, distinguishing them from \textsc{dem.addr} forms, is that they are exclusive to the speaker's access.

The hypothesis that \textsc{dem.spkr} forms are associated with a strictly speaker-delimited \textit{here-space} receives further support from the observation that \textsc{dem.spkr} forms cannot be used as expansions of the speaker's \textit{here-space} to include that of the addressee, or a wider spatial dimension. For such contexts -- where \ili{English} would use \textit{here} -- \textsc{dem.addr} forms are used. In the following example, taken from an account of local history recounting the resettlement of Vera{\textquotesingle}a speakers at the location of the contemporary village, the speaker refers to the village of Vera{\textquotesingle}a, where the recording is taking place:
\ea	\label{ex24}
\gll	l\=e		=n 	mas\=ogi 	dir	=\=em 	mul 	kal 	ma 	\textbf{k\=en\=e}			\\
	\textsc{loc}	\textsc{=art}	time	\textsc{3pl}	\textsc{=tam1}	move	up		hither		\textsc{loc.dem.addr}		\\
\glt	`When they came up here [to today`s Vera{\textquotesingle}a village, where the recording takes place]...'		\hfill{BSVH.066}
\z

\noindent
The same is illustrated by the following examples from direct speech within narratives: in (\ref{ex25}) and (\ref{ex26}), the speaker refers to the house where he, his associates and the addressee are. 
\ea	\label{ex25}
\gll		kamam 	\=e 			mi-mi{\textquotesingle}ir 	ros 		a 			vie				kamam 		mi{\textquotesingle}ir 	wal 	sa 		\textbf{k\=en\=e}		\\
\textsc{1pl.ex}	\textsc{neg.gen1}	\textsc{red-}sleep	\textsc{neg.gen2}	\textsc{loc.sp}	where		\textsc{1pl.ex:tam1}		sleep	\textsc{intens}	\textsc{emph}	\textsc{loc.dem.addr} \\
\glt	`We didn't sleep anywhere (else), we slept right here!'				\hfill{JJQ.396}
\z
\ea	\label{ex26}
\gll		kamam	ne 	gitag 	\=en 	\=mo-mam 					{\ob}\dots{\cb}				kamam ga		mi{\textquotesingle}ir	wal		sa		l\=e		\textbf{=n}	\textbf{ni\=m\=e}		\textbf{an\=e}	\\
\textsc{1pl.ex}	\textsc{??}		not.exist		\textsc{art}	\textsc{poss.house-1pl.ex}		{}	\textsc{1pl.ex}	\textsc{stat}	sleep	\textsc{intens}	\textsc{emph}	\textsc{loc}	\textsc{=art}	house		\textsc{dem.spkr.a}		\\
\glt	`We don't have another house that we could move to. We slept in this house right here!'		\hfill{JJQ.425}
\z

\noindent
The respective \textsc{dem.spkr} forms of locative adverbs are used only where the addressee is clearly not included in the speaker's \textit{here-space}, as in the following example of direct speech where a character is inviting a group of people from far down the coast to stay at his place:
\ea	\label{ex27}
\gll		van 	ma 		g\=ed\=e		me	{\textquotesingle}og \textbf{k\=eg\=en}.		\\
		go		hither		\textsc{1pl.in}	\textsc{fut}	stay	\textsc{loc.dem.spkr}		\\
\glt	`(You guys) come here, and we will stay here together.'		\hfill{JJQ.273}
\z

\noindent
That attention focus is the driving force behind the choice between \textsc{dem.addr} and \textsc{dem.spkr} forms is further supported by examples where \textsc{dem.spkr}	forms are used with referents that are clearly situated at an enormous distance from the position of the speaker. The crucial point is that it is \emph{not in the addressee's here-space}. Moreover, it seems that the speaker in this example intends to draw the addressee's attention to something new, or something the speaker would assume the addressee not to be aware of at a given point. The most blatant example illustrating this point comes from another instance of direct speech in a narrative: in (\ref{ex28}), a boy asks his father about a light that is far away in the bush (and the video shows the narrator pointing into the distance when depicting the boy's gesture, while looking to the other side to address his father), using the \textsc{dem.spkr} form:
\ea	\label{ex28}
\gll	l\=e		=n 	q\=o\=n 	ne 	v\=owal 	Wow\=ot 		ne 		vesir 	e 		{\textquotesingle}ama-gi 		so			mam 	ba 	\textbf{=n} 		\textbf{bur} 	\textbf{ag\=en\=e}		=n 		sava		\\
\textsc{loc}	\textsc{=art}	night	\textsc{num}		one		\textit{pers.name}	\textsc{aor:3sg}	ask		\textsc{pers}	father-\textsc{3sg}	\textsc{quot}		dad		but		\textsc{=art}		light		\textsc{dem.spkr.a}	\textsc{=art}		what	\\
\glt	`Then one day, Wowot would ask his father: ``Dad, but this light here [pointing gesture], what is that?'''		\hfill{ISWM.075--076}
\z

\noindent
The explanation for the use of a \textsc{dem.spkr} form in line with our hypothesis is that here the speaker has the light in mind and now wants to direct his father's attention to it so he can ask him about it. Similarly in the following example, the group of people -- already featured in examples (\ref{ex25}) -- (\ref{ex27}) -- is looking for a place to sleep to hide from the evil spirit who had invited them to stay with him. The group has been sleeping in different parts of the house, hiding from the spirit who wants to kill them. In example (\ref{ex29}), one member of the group gives them away to the spirit by telling him the actual place they spent the night in:
\ea	\label{ex29}
\gll 	kamam 	mi{\textquotesingle}ir 	sa 		\textbf{g\=en\=e} 		l\=e		=n 		wio		\\
		\textsc{1pl.ex}	sleep	\textsc{emph}	\textsc{dem.spkr}	\textsc{loc}	\textsc{=art}	bamboo \\
\glt	`[``Where did you sleep?'' And then Tagarqonqon said:] ``We slept right here, in the bamboos [We slept in the (bamboo) struts.]'''			\hfill{JJQ.354}
\z

\noindent
This example contrasts with examples (\ref{ex25}) and (\ref{ex26}) -- taken from the same context. Here the speaker draws the addressee's attention to something new, rather than the house, that the addressee is familiar with, aware of and shares his \textit{here-space} with. Specifically, the speaker here points to a more specific place within the house that -- given the context -- the addressee was not aware of. Finally, an example from the house building video illustrates the same idea. In (\ref{ex30}), the speaker -- standing next to the camera and commenting on what is happening in the frame -- points to different spots where the side posts of the house are going to be located, thereby drawing the attention of the addressee -- me in this case, behind the camera -- to these different locations:
\ea\label{ex30}
\gll 	r\=ov{\textquotesingle}\=e	{\textquotesingle}i		so	=n	{\textquotesingle}aval-gi	a			k\=el	sar			\textbf{g\=en\=e}			dir	=\=ek		mom		q\=e		=n		{\textquotesingle}er\=e	qa{\textquotesingle}aga \hspace{2em}{\USOParen}0.4{\USCParen}\hspace{2em}			wo 	dir	=\=ek		q\=e{\textquotesingle}\=eg		k\=el	\=en		{\textquotesingle}aval-gi 	\textbf{ag\=en\=e}. {\USOParen}0.3{\USCParen}	\\
close	\textsc{del}	\textsc{cpl}	\textsc{=art}	side-\textsc{3sg}	\textsc{loc.sp}	back	bushwards	\textsc{dem.spkr}	\textsc{3pl}	\textsc{=aor}	put		finish	\textsc{=art}	\textsc{pl}	post {}	and	\textsc{3pl}	\textsc{=aor}	begin		back	\textsc{art}	side-\textsc{3sg}	\textsc{dem.spkr.a}\\
\glt `It’s almost like that side further bushwards, they have put all the posts, and then they will start on this side over here.'\hfill{HouseBuilding05\_IS.008--009}
\z

\noindent
Here, the speaker first points away from the two of us towards the bushwards side of the plane of the house, using a \textsc{dem.spkr} form, and then turn to the other side, again using a \textsc{dem.spkr} form. This consecutive pointing to new locations and referents seems similar to the listing use of proximal demonstratives in \ili{Samoan} \citep{Mosel2004}, and is compatible with the idea that such forms are used to draw the addressee's attention to the next, new point in the series.

Finally, \textsc{dem.dist} forms are used where an entity is not associated with the speaker or addressee, and is not to become the focus of attention. Example (\ref{ex24}) above illustrates this: the speaker uses a \textsc{dem.addr} form when referring to the post he is cutting, assuming that the addressee is also attending to it. But in the immediately preceding \isi{clause}, he refers to the cross beam of the house's scaffold that is outside of what speaker and addressee will then be attending to, and relevant only as background. The \textsc{dem.dist} form is used here. \textsc{dem.dist} forms are also used where the speaker cannot fully attend to the \isi{referent} in question because it is outside their perception. This is the case in (\ref{ex31}), where Cat is asking Rat about the pawpaw that only Rat has access to:
\ea	\label{ex31}
\gll		o 			bul 		r\=ov-r\=ov{\textquotesingle}\=e 	nik	e 			kur 		kir\=m\=o \textbf{=n}		\textbf{gako} 	\textbf{wova{\textquotesingle}al} \textbf{\=e}	\\
\textsc{interj}	friend		\textsc{red-}close	\textsc{2sg}	\textsc{2sg:aor}	gnaw		break		\textsc{=art}	stalk		pawpaw		\textsc{dem.dist}		\\
\glt	`Hey friend, will you almost have gnawed through that pawpaw stalk?'\\		\hfill{GABG.083}
\z

\noindent
I conclude that the \textsc{dem.addr} and the \textsc{dem.spkr} forms are both used where the speaker seeks \isi{joint attention} focus with the addressee; \textsc{dem.dist} forms, on the other hand, are used where the \isi{referent} is outside of the interlocutor's attention sphere, and relevant only generally but not right here, or simply impossible to attend to conjointly. Obviously, this can be the case where the \isi{referent} is at a greater distance from both speaker and hearer, hence its characterisation as distal.

Lastly, \textsc{dem.spkr} forms are often used in two further contexts that belong to narrative texts, but are to be classified as exophoric-situational since they involve consideration of the communicative situation of the narration itself. The first of these are cases of \textit{Deixis am Phantasma} \citep{Buehler1934} where a narrator uses the physical environment and/or their own body to illustrate aspects of the depicted reality of the narrative (see also \citealt[224]{Himmelmann1996}). Thus, in (\ref{ex32}) the speaker shows where a character is holding the different kinds of leaves (to signal his peaceful mindset). In (\ref{ex33}), the narrator demonstrates how a piece of breadfruit is broken off. For this latter example, two interpretations seem possible: one in terms of the manner of breaking it off, and one in terms of the dimensions (size, shape) of the piece. In (\ref{ex34}), the speaker illustrates the dimensions of a rock wall.
\ea	\label{ex32}
\gll 	di 		ga		gis	\=en		du		\=m\=el		\textbf{val-gi}			\textbf{ag\=en\=e} 	wo	=n	seg		\textbf{val-gi}			\textbf{ag\=en\=e}.	\\
		\textsc{3sg}	\textsc{stat}	hold	\textsc{art}	leaf	tree.sp		opposite-\textsc{3sg}	\textsc{dem.spkr.a}	and	\textsc{=art}	tree.sp		opposite-\textsc{3sg}	\textsc{dem.spkr.a}	\\
\glt	`He was holding a \textit{\=m\=el} leaf in this hand here, and a \textit{seg} leaf on the other side here.'	\hfill{ISWM.258}
\z
\ea	\label{ex33}
\gll 	{\textquotesingle}ubu-gi						=m		{\textquotesingle}uq			rak	den 			\=en 	menre-gi	ne		v\=owal	\textbf{seg\=en\=e} 		{\ob}...{\cb}	\\
		grandparent/child-\textsc{3sg}	\textsc{=tam1}	break.off		out	\textsc{abl}	\textsc{art}	piece-\textsc{3sg}	\textsc{num}		one		\textsc{man.dem.spkr}	\\
\glt	`His grandma broke off a piece (of breadfruit) like this.'	\hfill{MVB.149}
\z
\ea	\label{ex34}
\gll 	kel 	qa-qa\=n 		luwo \textbf{seg\=en}.			\\
		big		\textsc{red}-walling	big	\textsc{man.dem.spkr} \\
\glt	`a wall big like this.'		\hfill{ISWM.175}
\z

\noindent
These uses of \textsc{dem.spkr} forms are compatible with the hypothesis outlined above that these forms function to establish \isi{joint attention}. The respective narrator is pointing out aspects of his body and corresponding aspects of the narrated reality that he believes the audience is not aware of. Similarly, and occasionally hard to distinguish from \textit{Deixis am Phantasma}, narrative frames and meta-comments are essentially \isi{exophoric} in nature. Narrative frames are those passages where a narrative is introduced or ended, or where the narrator makes meta-narrative comments, for example stating that the narrative will soon come to an end. As for the former case, speakers often start a narrative by referring to the act of narrating or the narrative (as a textual whole) that is going to follow. In these cases, \textsc{dem.spkr} forms are used, as can be seen from the following two examples:

\ea	\label{ex35}
\gll		no		=k			kaka	\textbf{=n}		\textbf{nel\=no}	\textbf{vu{\textquotesingle}}	\textbf{ag\=en\=e}	\\
\textsc{1sg}	\textsc{=aor:1sg}	story	\textsc{=art}		voice-		spirit	\textsc{dem.spkr.a}	\\
\glt	`I am going to tell the following story:'		\hfill{ISWM.001}
\z
\ea	\label{ex36}
\gll no		ga		moros	no		=k 			dedicatem		\textbf{\=en} \textbf{kaka}	\textbf{ag\=en\=e}	m\=e	=n	e		raga		\=m\=er\=m\=ere 	\\
\textsc{1sg}	\textsc{stat}	want	\textsc{1sg}	\textsc{=aor:1sg}	dedicate		\textsc{art}	story	\textsc{dem.spkr.a}	\textsc{dat}	\textsc{=art}	\textsc{pers}	\textsc{hum:pl}	kid		\\
\glt	`I want to dedicate this story to the children [...]'		\hfill{MVBW.004}
\z

A similar context featuring \textsc{dem.spkr} use is the following, where the narrator announces a song to follow, sung in this case by another person, and meant to be sung by a character in the story:
\ea	\label{ex37}
\gll 	l\=e		=n 	nes 	{\textquotesingle}a			b\=ene 	di		=m 		van 	\textbf{eseg\=en} \\
	\textsc{loc}	\textsc{=art}	song	\textsc{ass.sp}	\textsc{obl.pro}	\textsc{3sg}	\textsc{=tam1}	go		\textsc{dem.spkr.a}	\\
\glt	`The song belonging to it [i.e. the story] goes like this:'			\hfill{HHAK.091}
\z

\noindent
Hence, we are dealing here with cataphoric uses of the speaker-oriented forms, which are notionally similar to \textit{this-new} \citep[93]{Wald1983} uses in \ili{English}, and which square with the attendance-establishing hypothesis: the function of \textsc{dem.spkr} forms here is to get the audience to attend to the story to come, which of course they are not familiar with, and which metaphorically rests with the speaker up to that point where they start the narration. As observed by \cite{Himmelmann1996} for a range of languages, these instances of cataphoric reference with speaker-based forms to introduce new information are often restricted to discourse-\isi{deictic} use, and this is true for Vera{\textquotesingle}a as well. Hence, \textsc{dem.spkr} forms are used cataphorically in this way only where an event or content noun is involved, as in examples (\ref{ex35}--\ref{ex37}).

Conversely, at the end of a story, narrators conclude the narration with reference back to the story, stating that it has come to an end. Interestingly, in these instances no \isi{demonstrative} is ever used to modify the noun `story'. However, a locative adverb of the \textsc{dem.addr} series is used to refer to that end point in the story, for example:
\ea	\label{ex38}
\gll	kaka	wunva	di		=m		di\=n	wuva	\textbf{ek\=en\=e}.		\\
story	probably	\textsc{3sg}	\textsc{=tam1}	reach	only		\textsc{loc.dem.addr}		\\
\glt	`The story, it is probably over at this point.'		\hfill{ISWM.361}
\z
\ea	\label{ex39}
\gll		ba		=n	kaka	=m		q\=e{\textquotesingle}	sa		\textbf{k\=en\=e}.		\\
but		\textsc{=art}		story	\textsc{=tam1}	finish	\textsc{emph}	\textsc{loc.dem.addr}		\\
\glt		`[\dots and what happened to them I don’t know], but the story is over right here.'\\\hfill{MVBW.137}
\z

\noindent
This context is not clearly \isi{exophoric}, but can be seen as an instance of endophoric and discourse-\isi{deictic} use. Two observations seem relevant here. For one thing, these examples highlight that \textsc{dem.addr} forms are not markers of definiteness in endophoric use, since the NPs `story' are clearly to be understood as definite in both examples. For another thing, the use of \textsc{dem.addr} locative adverbs suggests a spatial metaphor for \isi{discourse}, where narrators frequently make meta-comments by referring to the point of the narration where `we are' at a specific point \citep[cf.][475]{Diessel2006}. That \textsc{dem.addr} is used is motivated by the speaker’s assumption that the addressee is conjointly attending to the narration, and to that moment in it.

Finally, there are examples where a story is referred back to by a plain \textsc{dem.addr} form, which seems to be most similar to endophoric discourse-\isi{deictic} use:
\ea	\label{ex40}
\gll so		n\=e{\textquotesingle}\=e		=n		ni{\textquotesingle}i	kaka		wo	=n		nes	{\ob}\dots{\cb}		\\
		??		\textsc{dem.addr}		\textsc{=art}	small	story		and	\textsc{=art}		song			\\
\glt	`So this is a little story and song [about how the incubator bird builds its nest].'\\\hfill{GATG.065}
\z

\noindent
The \textsc{dem.addr} form is used in these instances because the speaker assumes that the addressee is familiar with the \isi{referent} and attending to it.

In sum then, Vera{\textquotesingle}a has a three-way system of demonstratives. The choice of forms is driven primarily by considerations of attention focus and information management: \textsc{dem.dist} forms are used where an entity is outside the \textit{here-space} of both speaker and addressee, and \isi{joint attention} to it is not relevant or cannot be obtained. \textsc{dem.spkr} forms are used where an entity is in the \textit{here-space} of the speaker, excluding the addressee. It is used where the speaker seeks to draw attention to an entity that only they are currently familiar with or aware of. \textsc{dem.addr} forms are used where speaker and addressee share a \textit{here-space} with regard to the entity in question, and the speaker assumes the addressee to already be attending to the entity in question. The crucial aspect of this system is that it has a marked speaker-oriented \textit{here-space} and attention focus, and that wherever the addressee is seen as sharing \textit{here-space} and attention, the \textsc{dem.addr} form is used.  The functions of \textsc{dem.spkr} and \textsc{dem.addr} forms are essentially different in that only the former alters the focus of attention. The function of \textsc{dem.addr} is to maintain \isi{joint attention} focus on an entity for what follows. This can then yield different pragmatic effects, as can be seen from the following examples (\ref{ex41}) and (\ref{ex42}): 
\ea	\label{ex41}
\gll		ei			\textbf{nike}	\textbf{an\=e}		sa		=s	dada		kel			\=en		nak		mu-k\\
		\textsc{interj}	\textsc{2sg}	\textsc{dem.addr.a}	\textsc{emph}	\textsc{=sim}	\textsc{red:}do	back		\textsc{art}		canoe	\textsc{poss.gen-1sg}\\
\glt	`Hey, so is it you who has been putting my canoe back up all the time?'\hfill{JJQ.064}
\z
\ea	\label{ex42}
\gll	{\textquotesingle}ei			\textbf{kamadu}	\textbf{an\=e}		=m		van	ma 				sir 	nik 	an\=e.		\\
		\textsc{interj}	\textsc{1dl.ex}			\textsc{dem.addr.a}	\textsc{=tam1}	go		hither		for	\textsc{2sg}		\textsc{dem.addr.a}	\\
\glt	`Hey, we came here because of you, [because we want to be together with you].'\\\hfill{1.PALA.059--061}
\z

\noindent
In (\ref{ex41}), the \textsc{dem.addr}-marked \isi{pronoun} occurs in left-dislocated position, which is in other cases also associated with \isi{topicalisation}. Here, however, it is clear that the \isi{referent} has a role more akin to what has been called focus (\citealt{Krifka2008}, \citealt{Lambrecht1994}), picking out a \isi{referent} from a putative open set (someone who can do things). Likewise, the use of a \textsc{dem.addr} form with the pronouns in (\ref{ex42}) would be hard to motivate by considerations of identification, or topic marking. If anything, it seems, the subject \isi{pronoun} would have to be construed as a `topic', whereas the PP complement would be in `focus' -- `we are not here just for the dancing or other people, but for \emph{you}'. These examples thus underscore the point that demonstratives do not serve as markers of definiteness, since the personal pronouns already entail definiteness. Instead of definiteness, demonstratives seem to bear functions in the domain of information structure. However, it is also clear from these examples that they do not mark specific information-structural roles like `topic' or `focus'. Rather, in my view, the use of \textsc{dem.addr} forms is motivated here by the same considerations of maintained attention focus as in other uses discussed thus far, and this has the effect of an inference that what follows is in some way relevant for the entities thus expressed. Given other structural properties, like left-dislocation and emphatic marking, this may produce a `focus effect' \citep{MaticWedgwood2013}. Moreover, it does seem to have the effect of singling out entities as `information packages' \citep{Ozerov2015}, the interpretation of which depends on the \isi{discourse} context. I will take up this point in \sectref{section4} and \sectref{section5}.

\subsection{Endophoric use of Vera{\textquotesingle}a demonstratives}	\label{demvraendouse}
I now turn to endophoric uses of demonstratives within narrative texts, excluding its frame, meta-comments, and direct speech. I call this instance of \isi{discourse} `narration text'. The functions of demonstratives in this context are typically associated with anaphora and \isi{referent} tracking (called `tracking use' in \citealt{Himmelmann1996}), but also with considerations of \isi{information packaging} \citep{Diessel1999}. In what follows I will go through the uses of demonstratives in different syntactic contexts. It should be noted at this point that I will be dealing almost exclusively with the use of \textsc{dem.addr} and \textsc{dem.dist} forms since \textsc{dem.spkr} forms are very restricted in this context, occurring practically only in one interesting case which will be discussed first. It will be made clear in \sectref{section4} below that the virtual absence of \textsc{dem.spkr} forms in narration text follows straightforwardly from their attention-establishing cataphoric function discussed in \sectref{demvraexouse}.

\subsubsection{Speaker-oriented forms in narrative discourse}
The only example of a \textsc{dem.spkr} form in narration text is the following, where the narrator seems to slip into the report of a character's thought:
\ea	\label{ex43}
\gll		di		ne			van	ma		ne			r\=er\=e				\textbf{e}		\textbf{ruwa}		\textbf{ag\=en}		\=o			di		r\=os.		\\
		\textsc{3sg}	\textsc{aor:3sg}		go		hither		\textsc{aor:3sg}		head		\textsc{pers}	\textsc{hum:dl}		\textsc{dem.spkr.a}	\textsc{interj}	\textsc{3sg}	\textsc{neg2}		\\
\glt	`Then she would come over and take a closer look at ``these two here, no, not him''.'		\hfill{ISWM.326}
\z

\noindent
This appears to be one of many examples from Vera{\textquotesingle}a narratives where a narrator seems to slide seamlessly from the narration text into direct speech or thought. As in this example, such cases often feature interjections. Thus, in this example direct thought and narration text are not clearly marked off. The use of a \textsc{dem.spkr} form here in fact already represents an instance of \textit{Deixis am Phantasma}, and it is motivated by the same attention-directing principles discussed above. Hence, for the remainder of this section, I will be dealing only with \textsc{dem.addr} and \textsc{dem.dist} forms. 

\subsubsection{Plain addressee-oriented forms on clause level in narration text}
Plain, i.e. unprefixed, \textsc{dem.addr} forms are very rare in narration text. One recurrent occurrence is their pronominal use in subject function. Their function here is discourse-\isi{deictic}, as in the following example (\ref{ex44}):
\ea	\label{ex44}
\gll		\textbf{n\=e}		=n 		t\=ekt\=ek 	mu-n 			e 			Qo{\textquotesingle}.	\\
\textsc{dem.addr}	\textsc{=art}		speech	\textsc{poss.gen-cs}	\textsc{pers}	\textsc{pers.name}	\\
\glt	`That was Qo{\textquotesingle} talking there.'			\hfill{JJQ.308}
\z

\noindent
The \textsc{dem.addr} form here refers back to the last few \isi{intonation} units in which the narrator clarifies that it was the hero of the story, Qo{\textquotesingle}, who said these things. As in \isi{exophoric} use, a \textsc{dem.addr} form can occur in clause-final position in narration text. Here, however, they seem to have a somewhat different function in maintaining the addressee's attention to a specific point in a narrative:
\ea	\label{ex45}
\gll		di		=m 		{\textquotesingle}\=og 	kelkel 		{\textquotesingle}\=og 	kelkel 		n= 	me{\textquotesingle} 	ne 		onon 	r\=ow 		\textbf{n\=e{\textquotesingle}}	di	=m {\textquotesingle}\=og kelkel {\ob}...{\cb} vavavavan.	\\
	\textsc{3sg}	\textsc{=tam1}	stay	\textsc{red:}back		stay	\textsc{red:}back		\textsc{art=}	reef	\textsc{aor:3sg}	\textsc{red:}lie	at.sea		\textsc{dem.addr}	\textsc{3sg}	\textsc{=tam1}	stay	\textsc{red:}back	{}	\textsc{red-}go		\\
\glt	`He was staying up (on the island), and the reef was lying down at the sea. So now he [the man] was spending his time (there) [and one night he was dreaming, the reef came to him in his dream]. '			\hfill{ISAM.048}
\z

\noindent
Such instances of \textsc{dem.addr} forms are relatively rare in the corpus, and more data needs to be analysed before more substantial conclusions can be drawn. Nonetheless, I would like to argue that the use of the \textsc{dem.addr} form here has the same function of maintaining \isi{joint attention} to a specific point in a narration. This then has two related effects in this context, I believe. For one thing, it yields a simultaneity effect: while the reef is sitting down at the shore the man is up at the top of the shore. It also seems to have the effect of keeping the reef relevant for the following propositions, leading over to its appearance in the man's dream. Although these interpretations would have to be further substantiated, they do seem to square with the information-packaging effects to be discussed in \sectref{section4}.


\subsubsection{Adnominal use of addressee-oriented versus distal demonstratives in narration text}
\textsc{dem.addr} forms occur adnominally in cases where the \isi{referent} in question has been mentioned in the immediate context. A typical usage context is where the \isi{referent} was introduced in the previous \isi{clause} and is then taken up immediately again, for example:
\ea	\label{ex46}
\gll		duru	=k 	{\textquotesingle}\=en 	ma	=n 	lumgav 		ne 	v\=owal			\textbf{lumgav} 		\textbf{ne}		\textbf{v\=owal} \textbf{an\=e}			di		ne 		laa-laka 	sen\=e 	wo 			ne 		virig 	ma~~~~~{\ob}\dots{\cb}		\\
		\textsc{3dl}	\textsc{=aor}	see		hither	\textsc{=art}	young.man	\textsc{num}		one		young.man 	\textsc{num}	one		\textsc{dem.addr.a}	\textsc{3sg}	\textsc{aor:3sg}	\textsc{red-}dance		\textsc{man.dem.addr}			and			\textsc{aor:3sg}	rush	hither \\
\glt	`Then the two spotted a young man. And the young man, he danced, and then (he) came over, [and sat down under the wild kava plant].'\hfill{1.PALA.044--045}
\z
\ea	\label{ex47}
\gll		van 	ma 		di\=n 		\=en 	b\=e		\textbf{b\=e} 		\textbf{ne} 	\textbf{v\=owal} \textbf{an\=e}		=n 		so-gi	=n		B\=e{\textquotesingle}elmamgin.		\\
go		hither		reach		\textsc{art}		water		water		\textsc{num}	one		\textsc{dem.addr.a}	\textsc{=art}		name-\textsc{3sg}	\textsc{=art}	\textsc{place.name}		\\
\glt	`Went here and reached a river. This (some) river, its name (is) B\=e{\textquotesingle}elmamgin [lit. `Cold to dive water'].'	\hfill{AS.1.071--073}
\z
\ea	\label{ex48}
\gll		ote 	mu-n 		e 		ni{\textquotesingle}i- 	re\=ne 		an\=e 		ne 		ma{\textquotesingle}		e 		duru	=k 	{\textquotesingle}\=en \dots eh		=n 	{\textquotesingle}ama-gi	ne 		{\textquotesingle}\=en 	g\=or 		\=en 	ni{\textquotesingle}i- 	re\=ne 		an\=e 		vaavan			{\textquotesingle}e		\textbf{=n} \textbf{	{\textquotesingle}ama-gi} 		\textbf{an\=e} 		ne 		ma{\textquotesingle}.			\\
mum	\textsc{poss.gen-cs}	\textsc{pers}	small-	female	\textsc{dem.addr.a}	\textsc{aor:3sg}	dead		\textsc{disc}	\textsc{3dl}	\textsc{=aor}	see	\textsc{cor}	\textsc{=art}	father-\textsc{3sg}	\textsc{aor:3sg}	see		secure	\textsc{art}	small-	female	\textsc{dem.addr.a}	\textsc{red-}go		\textsc{disc}	\textsc{=art}	father-\textsc{3sg}	\textsc{dem.addr.a}	\textsc{aor:3sg}		dead		\\
\glt	`Then the mother of the little girl died. So her father looked after the little girl, on and on, and then her father died.'	\hfill{ANV.005--007}
\z

\noindent
These examples show again that \textsc{dem.addr} forms do not merely mark an NP as definite, and this is also clear from its compatibility with personal pronouns, see \sectref{demvraexouse} above. Moreover, the total of 361 adnominal uses of \isi{demonstrative} forms (including all adverbs) accounts for only a small fraction of full (lexical) NPs with given referents in our corpus, hence \isi{givenness}/identifiability is likewise not \emph{marked} by demonstratives. This was also made clear with respect to reference back to the story in narrative frames. The point is further supported by examples where \textsc{dem.addr} is used with personal names or, as in the following example (\ref{ex49}). Typically, these are nouns meaning `older same-sex sibling, firstborn' or `younger same-sex sibling, lastborn', with couples of siblings typically featuring as heroes in Vera{\textquotesingle}a literature.
\ea	\label{ex49}
\gll 	wo 	\textbf{{\textquotesingle}isi\=m\=ere} \textbf{an\=e} 		ne 		{\textquotesingle}a\=n	v\=e{\textquotesingle}\=e				=n 	dudu 	vada			ne		v\=owal.		\\
		and	firstborn	\textsc{dem.addr.a}	\textsc{aor:3sg}		hand	release		\textsc{=art}		leaf		pandanus		\textsc{num}	one		\\
\glt	`(The devil was then sucking in the sea again, and the sea current [that arose from that] brought the canoe [the two siblings are sitting in] ever closer to him.) So then the older sister let go of a pandanus leaf.'\hfill{2.PALA.091}
\z

\noindent
This example also underscores that \textsc{dem.addr} forms are not merely used to differentiate between referents, since this function can be clearly ascribed to the use of the noun itself. In contrast to \textsc{dem.addr} forms, \textsc{dem.dist} forms are used adnominally where the \isi{referent} of the NP has been mentioned a while ago in the previous \isi{discourse}, for example:
\ea	\label{ex50}
\gll		duru	=k 	wos 		\=en 	\=nar	wos 		\=en 	\=nar 				an\=e 		dir{\textquotesingle}\=ol 	da		=n 	wig	me 	big 	{\textquotesingle}\=o 		sa		\textbf{=n} 	\textbf{r\=or\=o{\textquotesingle}} 		\textbf{\=e} 			al\=e 		duru	=k 		wos 			\=en 	\=nar.	\\
		\textsc{3dl}	\textsc{=aor}	hammer	\textsc{art}		canarium.nut		hammer	\textsc{art}		canarium.nut	\textsc{dem.addr.a}	\textsc{3tl}		do		\textsc{=art}	\textit{wig}	\textsc{fut}	eat		carry	\textsc{emph}	\textsc{=art}		k.o.cabbage		\textsc{dem.dist}	\textsc{interj}	\textsc{3dl}	\textsc{=aor}		hammer		\textsc{art}		canarium.nut			\\
\glt	`So they would smash up the canarium nuts. Smash up the canarium nuts to make wig (a kind of nalot (\ili{Bislama}), i.e. K.o. pudding), which was to be had with that cabbage (that they had gotten earlier on). All right, so they smashed the canarium nuts (and then started making nalot).'		\hfill{AS.1.028--030}
\z
\ea		\label{ex51}
\gll		wede		di		=m 		luwo				di		ne 		le		\textbf{=n}		\textbf{nak} 		\textbf{susu} 		\textbf{\=e}.\\
		rain		\textsc{3sg}	\textsc{=tam1}	big		\textsc{3sg}	\textsc{aor:3sg}	take	\textsc{=art}		canoe		\textsc{red-}paddle		\textsc{dem.dist}\\
\glt	`Then the rain became big, and it took out that canoe.'	\hfill{JJQ.104--105}
\z

\noindent
A \textsc{dem.dist} form can also be used where the \isi{referent} is merely evoked by or inferable from earlier happenings, as is the case in the following example:

\ea	\label{ex52}
\gll		ba 	di		=m 		gis 	\textbf{\=en} 	\textbf{menre-} 	\textbf{k\=ol\=ov-} 		\textbf{{\textquotesingle}ar-{\textquotesingle}ara} 		\textbf{\=e}	ne 		mul 	{\textquotesingle}\=o{\textquotesingle} 		l\=e		=n 	l\=ol\=o- 		i\=m\=e.	\\
		but		\textsc{3sg}	\textsc{=tam1}	hold		\textsc{art}		piece		chipping	\textsc{red-}chop	\textsc{dem.dist}		\textsc{aor:3sg}	go		carry	\textsc{loc}	\textsc{=art}	iniside-	house\\
\glt	`But he took a piece of that chipping, and took it home with himself.'\hfill{JJQ.052}
\z

\noindent
Here, the actual chopping of the tree was mentioned seven \isi{intonation} units away, and the \textsc{dem.dist} form is used because it requires the addressee to re-activate this event to relate the `piece of chipping' to it. But apparently its inferability via frame semantics is sufficient to warrant the use of a \textsc{dem.dist} form that is otherwise used for individually given referents. Though there is good reason to assume that the \textsc{dem.dist} form in the previous examples is an NP-internal constituent, its position is nonetheless ambiguous between NP- and clause-final, the latter being another possible slot for both \textsc{dem.addr} and \textsc{dem.dist} forms (see below). For the sake of completeness, the following example shows a \textsc{dem.dist} form in a subject NP. This is likewise an instance of inference, since the dancing was mentioned ten \isi{intonation} units before inside a character's direct speech who states that they want to dance:
\ea	\label{ex53}
\gll		\textbf{laklaka} \textbf{	\=e}			=m 		van.		\\
		dance	\textsc{dem.dist}		\textsc{=tam1}		go		\\
\glt	`And then that dance happened.'	\hfill{ISWM.319}
\z

\noindent
In this example, the narrator is referring back to the idea of holding a dance for the hero of the story, who is temporarily dead. After this was mentioned first by some of the characters ten \isi{intonation} units earlier, it was reported how people were walking to the place where the dead would have to be found and the dances would take place.

\subsubsection{Addressee-oriented and distal demonstratives in clause combining constructions}
The prefixed basic \textsc{dem.addr} form can occur in clause-final position where it is used in \isi{clause} combining constructions. In the following examples, it occurs in clause-final position and the event described therein comes to an end once the second one sets in:
\ea	\label{ex54}
\gll		dir	=m 		su\=o 		kal 	wal 	di\=n sar 		ma 		\textbf{an\=e}		e 		Qo{\textquotesingle} 	ne 		la\=n 	w\=or 	wal 	\=en 	nak.		\\
		\textsc{3pl}	\textsc{=tam1}	paddle	up		once	reach	inland		hither		\textsc{dem.addr.a}	\textsc{pers}	\textsc{pers.name}		\textsc{aor:3sg}	slap	split	once	\textsc{art}	canoe	\\
\glt	`And as they had just come up on the shore, Qo{\textquotesingle} smashed the canoe to pieces.'\\\hfill{JJQ.257}
\z
\ea	\label{ex55}
\gll		al\=e		duru	=m		inin			va-van	\textbf{an\=e}				e		D\=ol	so		o			no		man	q\=e{\textquotesingle}.		\\
		\textsc{interj}	\textsc{3dl}	\textsc{=tam1}	\textsc{red:}drink	\textsc{red-}go	\textsc{dem.addr.a}	\textsc{pers}	\textsc{pers.name}	\textsc{quot}	\textsc{interj}	\textsc{1sg}	\textsc{pfv}	finish		\\
\glt	`All right, so the two of them drank and drank until [up to this point, and then…] D\=ol said: ``I am done. (The kava has kicked in.)'''	\hfill{1.AS.039--040}
\z

\noindent
In other instances, the two events are depicted as taking place simultaneously:
\ea	\label{ex56}
\gll 	l\=e		=n	mas\=ogi	di		=m		sagsag	\textbf{an\=e}		ba		di		ne			susur		lik		\=en		akl\=e	dudu woqe {\textquotesingle}enge.	\\		
	\textsc{loc}	\textsc{=art}	time		\textsc{3sg}	\textsc{=tam1}	\textsc{red:}sit	\textsc{dem.addr.a}	but		\textsc{3sg}	\textsc{aor:3sg}	\textsc{red:}tie	more	\textsc{art}	some-	leaf	tree	\\
\glt	`And as he was sitting (up at the top of the shore), he also assembled some leaves (to make some decoration for himself).'	\hfill{ISAM.076}
\z
\ea	\label{ex57}
\gll		ba 	dir{\textquotesingle}\=ol	=s 		vanvan		\textbf{an\=e}		=n	{\textquotesingle}isi\=m\=ere 	mal 		lama{\textquotesingle}i 	so		=n		{\textquotesingle}ama{\textquotesingle}.\\
		but		\textsc{3tl}	\textsc{=sim}	\textsc{red:}go		\textsc{dem.addr.a}	\textsc{=art}	lastborn		\textsc{rem.pfv}		know		\textsc{cpl}	\textsc{=art}		devil\\
\glt	`But as the three were walking along, the younger sister had already realised that is was a devil.'	\hfill{2.PALA.018--019}
\z

\noindent
These examples of \textsc{dem.addr} uses are similar to the use of the plain form in (\ref{ex45}). In yet other cases, the first \isi{clause} marked by a clause-final \textsc{dem.addr} form expresses a state of affairs that is a \isi{presupposition} for the proposition of the following \isi{clause}, expressing for example the reason for that proposition holding, as in the following example:
\ea	\label{ex58}
\gll		so Qo{\textquotesingle}	di		ne			sag	{\textquotesingle}i		\textbf{an\=e}			{\ob}\dots{\cb}		di		e			lama{\textquotesingle}i		ros		so di		me		{\textquotesingle}isiw				sivi\=e.				\\
		??	\textsc{pers.name}	\textsc{3sg}	\textsc{aor:3sg}	sit	\textsc{del}	\textsc{dem.addr.a}		{}	\textsc{3sg}	\textsc{neg1}	know		\textsc{neg2}	\textsc{cpl}	\textsc{3sg}	\textsc{fut}	climb.down	how		\\
\glt	`(And then they turned towards Gaua). But Qo{\textquotesingle}, he was sitting up there, [that darag towla it is so big so] he didn’t know how to climb down.'\hfill{JJQ.171--172}
\z

\noindent
Generally speaking, clause-final \textsc{dem.addr} forms seem to have the effect of interpreting the preceding \isi{clause} as immediately relevant for the interpretation of the subsequent \isi{clause}. By contrast, clause-final \textsc{dem.dist} forms have the effect of steering away from the proposition expressed in its \isi{clause}, while keeping it on hold for future reference. The following example (\ref{ex59}) is taken from the same story as (\ref{ex58}), and occurs long before that latter. The \textsc{dem.dist}-marked \isi{clause} expresses the same state as the \textsc{dem.addr}-marked one in (\ref{ex58}); but here, the subsequent \isi{clause} is not immediately related to that proposition, but instead shows a shift in topic and theme, now dealing with Qo{\textquotesingle}'s brothers and their moving away from him. The story then goes on to recall how they steal his wife and canoe and travel to Gaua. Qo{\textquotesingle}'s situation is then taken up again in the \isi{clause} preceding the one in (\ref{ex58}).
\ea	\label{ex59}
\gll		e 		Qo{\textquotesingle} 	ne 		sag 	r\=en 		sag 	k\=ek\=e 		\textbf{\=e}		sag 	wo 	\textbf{e}		\textbf{raga} 		\textbf{{\textquotesingle}i-{\textquotesingle}isi-gi} 	\textbf{an\=e} 		van 	r\=ow 	l\=e		=n 		vunu\=o.		\\
\textsc{pers}	\textsc{pers.name}	\textsc{aor:3sg}	sit		stuck		up		above		\textsc{dem.dist}		sit		and	\textsc{pers}	\textsc{hum:pl}	\textsc{red-}younger.same.sex.sibling-\textsc{3sg}	\textsc{dem.addr.a}	go		down		\textsc{loc}	\textsc{=art}	village		\\
\glt	`So Qo{\textquotesingle} was now stuck up there (in the tree). Sat (there) while his brothers went down to the village. (They stole his wife and his canoe, and paddled off with it.)'	\hfill{JJQ.151--152}
\z

\noindent
Noteworthy in this example is also the switch of topic in the second \isi{clause} and the concomitant use of an adnominal \textsc{dem.addr} form. Another illustrative example involving a clause-final \textsc{dem.dist} form shows its theme-switching effect: in (\ref{ex60}), the context is that two men find a turtle at the beach and tie it up to pick it up later. After these recountings, the two men walk away from the beach and up the shore to their home village. The following \isi{discourse} is about the men's reporting back to their chief and the villager's planning to go down, cook the turtle, and hold a feast. The clause-final \textsc{dem.dist} forms in this instance have the effect of signalling this momentary switch in scene before the narration turns back to this \isi{referent} later:
\ea	\label{ex60}
\gll		ge{\textquotesingle}ew{\textquotesingle}i 	ne 		wil 			{\textquotesingle}ala\=n 	\textbf{\=en} \textbf{nu\=o} 		\textbf{\=e}			q\=e{\textquotesingle}	duru	=k	vrig	kal	k\=el	l\=e		=n	l\=ol\=o		vunu		\\
other		\textsc{aor:3sg}	turn.over		upside.down	\textsc{art}	turtle		\textsc{dem.dist}		finish		\textsc{3dl}	\textsc{=aor}	rush	up		back	\textsc{loc}	\textsc{=art}	inside		village		\\
\glt	`And then the other one turned the turtle over onto its back, and after that the two ran back up into the village.'		\hfill{GAQG.057--058}
\z

\noindent
Regarding this particular example, it should be noted again that the position and function of the \textsc{dem.dist} form is ambiguous here. The discussion  of other examples of adnominal \textsc{dem.dist} forms above would suggest that its use in (\ref{ex60}) is not to be classified as adnominal, since the turtle had just been the theme of the preceding \isi{discourse}, and it seems more plausible to interpret it as clause-final, bringing about the effect outlined here. This conclusion should, however, be taken as preliminary, and further examples are required to substantiate -- or falsify -- it. 


\section{Discussion: Attention and information}	\label{section4}
I first summarise the findings from the discussion of \isi{exophoric} and endophoric uses, and then present a preliminary analysis of these. In \isi{exophoric} use, \textsc{dem.spkr} forms are used where only the speaker is familiar with or aware of an entity, and attends to it. The form is used in order to draw the addressee's attention to the same entity, thereby establishing a \isi{joint attention} focus. Relevant contexts exemplified above are those where only the speaker can see (or otherwise perceive) the entity; where the speaker assumes that the addressee is not familiar with an entity, for instance the story in introductions to narratives; where the speaker assumes the addressee to be unaware of the entity to be talked about, for instance the piece of wood brought to a house builder, the enacting of bodily aspects in \textit{Deixis am Phantasma}, or the light seen at night in the bush. The last example makes it clear that spatial considerations are not the relevant ones: what counts instead is the awareness-state of the speaker, and their assumptions concerning that of the addressee. Obviously, in communication the speaker will always be aware of the entities they verbalise and elaborate upon, thus speaker-oriented forms are naturally used when the addressee is assumed to lack that awareness or knowledge (see \citealt{Mosel2004} for the same point on \ili{Samoan} proximal demonstratives). The \textsc{dem.spkr} forms are, however, never used cataphorically for new referents in a narrative, and it seems possible that this kind of use is typologically very restricted (\ili{English} \textit{new-this}), as already remarked by \cite{Himmelmann1996}, a tendency corroborated by the case of Vera{\textquotesingle}a. As for locative adverbs, \textsc{dem.spkr} forms are used where the speaker perceives their location as not overlapping with that of the addressee.

The addressee-oriented forms, \textsc{dem.addr}, are used where the speaker assumes the addressee to already be aware of and attending to an entity. With the plain forms in pronominal function within \isi{exophoric} use, this concerns mostly immediately perceived sensations, like sounds. In endophoric use, these pronominal instances of \textsc{dem.addr} forms have discourse-\isi{deictic} functions, referring back to the immediately preceding \isi{discourse}. In adnominal use, the forms are used where the entity in question is already being attended to by the addressee. One effect of its use is that the \isi{referent} has to be understood as individuated, whereas an unmarked NP can always have class reference; and given that Vera{\textquotesingle}a does not distinguish generic from specific contexts in other parts of the grammar, for instance tense, aspect, mood-marking, this is an omnipresent potential ambiguity. In narrative texts, a \textsc{dem.addr} form is used adnominally in NPs whose referents are identifiable from the immediately preceding \isi{discourse} context. This can be seen as equivalent to established attention to an entity in \isi{exophoric} contexts. Their use does not \emph{mark} the NP as having an identifiable \isi{referent}, since most other NPs with such identifiable referents are unmarked, and \textsc{dem.addr} forms also occur with pronouns and proper names. My hypothesis is that their use is connected to considerations of \isi{information packaging} (see below). Finally, \textsc{dem.addr} forms are used in clause-combining constructions where the \isi{clause} they mark is immediately relevant for the interpretation of the subsequent \isi{clause}.\largerpage[-1]

Distal forms are used in \isi{exophoric} contexts where an entity is outside of the speaker's and the addressee's interaction space, and also not within speaker's \textit{here-space}. This was illustrated with an example from the house building video where the speaker refers to the cross beam of the scaffold, but obviously then wants the addressee to focus their attention on the wood in his hands. In endophoric use, a \textsc{dem.dist} form is used adnominally where a \isi{referent} is being re-activated, and is used clause-finally at thematic or episodic shifts.

Comparison of \textsc{dem.addr} and \textsc{dem.dist} forms reveals that the latter are used merely to activate information, but not to focus on it, whereas the former have the effect of demanding focus of attention. This was demonstrated in example (\ref{ex3}), repeated here as (\ref{ex61}):
\ea	\label{ex61}
\gll		ne				{\textquotesingle}\=en	e		ruwa	\=mal\=mala	\=e		=s 		sag	{\textquotesingle}i		e		ruwa 	\=mal\=mala	an\=e		so		\\
\textsc{aor:3sg}		see		\textsc{pers}	\textsc{hum:dl}		girl		\textsc{dem.dist}	\textsc{=sim}		sit		\textsc{del}	\textsc{pers}	\textsc{hum:dl}		girl			\textsc{dem.addr.a}	\textsc{quot}			\\
\glt	`[\dots] and then (he) spotted those two girls (who were) sitting (there), and the two girls said: \dots'	\hfill{1.PALA.059--060}
\z

\noindent
The use of a \textsc{dem.dist} form in the first mention of `the girls' functions to re-activate the \isi{referent} who had been mentioned before, but the few \isi{intonation} units immediately before this dealt exclusively with the young man's dancing. In the second \isi{clause}, a \textsc{dem.addr} form is used, and here seems to have a topicalising effect, signalling that the \isi{clause} is about them. This suggests furthermore that \isi{information packaging} and \isi{referent} activation are two different aspects of \isi{discourse structure}, as illustrated in \cite[Chapter~2]{Vallduvi1993} , although this is often blurred in a Giv\'{o}n'ian sense of `topic' that seems to conflate notions dealt with under the headings `\isi{discourse} \isi{referent}', `\isi{discourse topic}', and `(sentence) topic' in other work.

But does a \textsc{dem.addr} form as such have a topicalising function, or even `topic-marking' function, as is sometimes claimed for \ili{Oceanic} languages? At least in Vera{\textquotesingle}a, the answer should be no. We have seen in examples above that \textsc{dem.addr} forms are in fact found to mark pronouns that are clearly in a relation one would traditionally call `focus', namely where the \isi{pronoun} is left-dislocated and marked with the emphatic marker \textit{sa}. As outlined above, I suggest here that \textsc{dem.addr} forms are not `polyfunctional' in any way, sometimes `marking or expressing topics', and in other contexts `marking or expressing focus'. Instead, I analyse these uses in the spirit of \cite{MaticWedgwood2013}, \cite{Ozerov2014} and \cite{Ozerov2015} as \emph{bringing about an effect} that is compatible with both `topic' and `focus', as discussed in the literature (\citealt{Lambrecht1994}, \citealt{Krifka2008}). In this way, a topicalising effect is merely brought about by the \textsc{dem.addr} form meaning `addressee keep attention focused on this', which can trigger the inference that something relevant will be said about the entity in question. Likewise, keeping attention focus can have the effect of stressing someone's involvement in a state-of-affairs, as in the examples where someone is asked whether he was the one who put a tree back up.

The clause-combining uses of \textsc{dem.addr} forms can be accounted for in a similar manner: namely, they have the effect of holding a proposition in attention focus. Given the maintained attention to the proposition, it is interpreted as immediately relevant for the proposition to follow, so that the latter is interpreted as being \emph{about} the former. The use of \textsc{dem.dist} forms, on the other hand, has the general function of activating a \isi{referent}, but not making it the attention focus. In clause-combining constructions, this bears the effect of topic and/or theme/episodic shift, together with the expectation that the previous episode may be relevant in some way at a later point in the \isi{discourse}.

Therefore, the use of demonstratives in \isi{exophoric} and endophoric contexts follows straightforwardly from their meaning related to attention focus, as shown in \tabref{demvrameaning}. This meaning of demonstratives can be stated as relatively slim, much in the spirit of \cite{Enfield2003}. All readings summarised above are the result of these stable meanings interpreted by interlocutors in relevant contexts. 

\begin{table}
\caption{Meaning of demonstratives in Vera{\textquotesingle}a}
\label{demvrameaning}
 \begin{tabularx}{\textwidth}{Xl} 
  \lsptoprule
 speaker-oriented		&		you do not attend to this	\\
addressee-oriented	&		I and you attend to this	\\
distal												&		I	do not attend to this	\\
  \lspbottomrule
 \end{tabularx}
\end{table}


\section{Conclusions}	\label{section5}
In conclusion, demonstratives in Vera{\textquotesingle}a bear attention-focusing meaning in the sense of \cite{Diessel2006}, and their interpretative effects result from interaction of speaker and addressee in relevant communicative contexts, much in the spirit of \cite{Enfield2003}. Spatial considerations are not primary, that is to say that orientation with regard to speech-act participants is not spatial, but communicative-interactional, relating to dimensions of knowledge and attention coordination. The interactional interpretations triggered by the different meanings of demonstratives produce different effects in different contexts, which means that we do  not need to assume polysemy for demonstratives. I conclude that their meaning is absolutely stable across \isi{exophoric} and endophoric uses, and sub-uses therein.

Finally, it should be noted again that more extensive examination of more video data -- which has been collected in large amounts over the past ten years -- is necessary to substantiate the extensional aspects of demonstratives in \isi{exophoric} use, and this should be done in future research. Thus, for instance it would be interesting to see whether Vera{\textquotesingle}a speakers can use a speaker-oriented form to point to something unnoticed by the addressee on the latter's body or clothing, and the like. Possibly also the use of certain stimuli (like the MPI space games) may help clarify some usages. I do believe, however, that even these initial observations from the currently available data are sufficient to draw up the basic outline of the system as done here. Moreover, it should be noted that spatial considerations as observable in video data are not an objective heuristic to explain the use of demonstratives, as illustrated convincingly by \posscitet{Enfield2003} examination of such data in \ili{Lao}. It seems that Vera{\textquotesingle}a presents another illustrative example of the salience of interactional pragmatics considerations in accounting of multi-contextual uses of demonstratives.

\section*{Acknowledgements}
The research reported in this contribution was made possible through the following research grants: \ili{Australian} Research Council DECRA, awarded to the author (grant no. DE120102017), and two VolkswagenStiftung's DoBeS grants (grant no. II/81 898 and II/84 316) awarded to Catriona Hyslop-Malau (PI) for the documentation of the Vur\"es and the Vera{\textquotesingle}a language. I also wish to acknowledge the ongoing support for my research within the ARC Centre of Excellence for the Dynamics of Language and the School of Languages and Linguistics at the University of Melbourne. I warmly thank everyone in the Vera{\textquotesingle}a community who have been so kind and generous to host me for more than ten years of our joint effort to document the Vera{\textquotesingle}a language.

I would like to thank Birgit Hellwig for her insightful comments on a very early version of this paper. Furthermore, I thank two anonymous reviewers for their comments on the submitted manuscripts, and suggestions for improvement. All remaining errors are of course my own responsibility.

\newpage 
\section*{Abbreviations}

\begin{multicols}{2}
	\begin{tabbing}
		glossglossglo \= \kill
		{1} \> 1st person\\
		{2} \> 2nd person\\
		{3} \> 3rd person\\
		\textsc{a} \> \textit{a} prefix \\
		\textsc{abl} \> ablative\\
		\textsc{addr} \> addressee-oriented\\
		\textsc{adn} \> adnominal\\
		\textsc{aor} \> aorist\\
		\textsc{art} \> common article\\
		\textsc{ass} \> associative\\
		\textsc{cor} \> correction\\
		\textsc{cpl} \> complementizer\\
		\textsc{cs} \> construct suffix\\
		\textsc{dat} \> dative\\
		\textsc{del} \> delimitative aktionsart\\
		\textsc{dem} \> \isi{demonstrative}\\
		\textsc{disc} \> \isi{discourse} marker\\
		\textsc{dist} \> distal\\
		\textsc{dl} \> dual\\
		\textsc{emph} \> emphatic\\
		\textsc{ex} \> exclusive\\
		\textsc{fut} \> future\\
		GRAID \> Grammatical Relations\\ \> and Animacy in\\ \> Discourse\\
		\textsc{hum} \> human\\
		\textsc{in} \> inclusive\\
		\textsc{intens} \> intensifier\\
		\textsc{interj} \> interjection\\
		\textsc{loc} \> locative\\
		\textsc{man} \> manner\\
		\textsc{neg.gen} \> general negation\\
		\textsc{nsg} \> non-singular\\
		\textsc{num} \> numeral article\\
		\textsc{obl} \> oblique\\
		\textsc{pers} \> personal article\\
		\textsc{pfv} \> perfective\\
		\textsc{pl} \> plural\\
		\textsc{poss.dom} \> domestic possession\\
		\textsc{poss.gen} \> general possession\\
		\textsc{poss.house} \> house possession\\
		\textsc{pro} \> \isi{pronoun}\\
		\textsc{quot} \> quotative\\
		\textsc{red} \> reduplication\\
		\textsc{rem.pfv} \> remote perfective\\
		\textsc{sg} \> singular\\
		\textsc{sim} \> simultaneous\\
		\textsc{sp} \> specific\\
		\textsc{spkr} \> speaker-oriented\\
		\textsc{stat} \> stative\\
		\textsc{tam} \> tense-aspect-mood\\
		\textsc{tl} \> trial
	\end{tabbing}
\end{multicols}

\printbibliography[heading=subbibliography,notkeyword=this]

\end{document}



