%tblpronomenchr\ili{LiJi1} 
 \begin{table}[!h]
		\begin{tabularx}{\textwidth}{lccccc}
		\midrule 

\textbf{Quelle} &\textbf{1.\,Sg.}  & \textbf{2.\,Sg.} &  \textbf{3.\,Sg.\,m.} &\textbf{1.\,Pl.\,Nom.}  & \textbf{Höfl.}\\ \midrule 

\hai{PP} & \cellcolor{gray!30!white}%Dat. $\gets$ Akk. 
& – & –& –& –\\ 
\hai{BW} & \multicolumn{1}{@{}Y@{}}{\tikzmark[a]{\raisebox{-1ex}{–}\raisebox{1ex}{\hspace{1ex}\textsubscript{\hai{PP}}}}}%\cellcolor{gray!30!white}/\textsubscript{\hai{PP}}%  Dat. $\gets$ Akk.
& \cellcolor{gray!30!white}% Dat. $\gets$ Akk.
&– & – & –\\
% &  &  \cellcolor{gray!30!white}\textsubscript{\hai{PP}}%[Dat. $\gets$ Akk.]\textsubscript{\hai{PP}}
% & &  & \\
\hai{LS} &\cellcolor{gray!30!white}%  Dat. $\gets$ Akk. 
& – & –& –& –\\
\hai{BP} &\cellcolor{gray!30!white}%  Dat. $\gets$ Akk.
& & & & \cellcolor{gray!30!white}%Dat. $\gets$ Akk.
\\
\hai{FE} & \cellcolor{gray!30!white}%Dat. $\gets$ Akk.  
& – & –& –& \multicolumn{1}{@{}Y@{}}{\tikzmark[d]{\raisebox{-1ex}{–}\raisebox{1ex}{\hspace{1ex}\textsubscript{\hai{PP}}}}}%\cellcolor{gray}/\textsubscript{\hai{PP}}%Nom./Akk. $\gets$ Dat.
\\ 
% & &  & & &\cellcolor{gray}\textsubscript{\hai{PP}}% [Nom./Akk. $\gets$ Dat.]\textsubscript{\hai{PP}} 
 %\\ 
\hai{AO} & \cellcolor{gray!70!white}%Akk. $\gets$ Dat. 
& – & –& –& \multicolumn{1}{@{}Y@{}}{\tikzmark[e]{\raisebox{-1ex}{–}\raisebox{1ex}{\hspace{1ex}\textsubscript{\hai{PP}}}}}%\cellcolor{gray}/\textsubscript{\hai{PP}}%Nom./Akk. $\gets$ Dat.
\\ 
% & &  & & & \cellcolor{gray}\textsubscript{\hai{PP}}%[Nom./Akk. $\gets$ Dat.]\textsubscript{\hai{PP}} 
 %\\ 
\hai{GP} & – &– &  \cellcolor{gray}\textsubscript{\hai{PP}}%[Akk. $\gets$ Dat.]\textsubscript{\hai{PP}}
& –& –\\ 
\hai{PG} & \cellcolor{gray}\textsubscript{\hai{PP}}%[Akk. $\gets$ Dat.]\textsubscript{\hai{PP}}  
& – &\cellcolor{gray!30!white}\textsubscript{\hai{PP}}% [Dat. $\gets$ Akk.]\textsubscript{\hai{PP}}
 & –& –\\ 
\hai{JK} & – & \cellcolor{gray!30!white}\textsubscript{\hai{PP}}%[Dat. $\gets$ Akk.]\textsubscript{\hai{PP}} 
& –& –&\cellcolor{gray}% Nom./Akk. $\gets$ Dat.
\\ 
\hai{TH} & \cellcolor{gray}%Akk. $\gets$ Dat. 
& – & –& –& –\\ 
\hai{AJ} & \cellcolor{gray!30!white}%Dat. $\gets$ Akk. 
& \multicolumn{1}{@{}Y@{}}{\tikzmark[b]{\raisebox{-1ex}{–}\raisebox{1ex}{\hspace{1ex}\textsubscript{\hai{PP}}}}}%\cellcolor{gray!30!white}/\textsubscript{\hai{PP}}%Dat. $\gets$ Akk. 
& –& –& \cellcolor{gray}\textsubscript{\hai{PP}}%[Nom./Akk. $\gets$ Dat.]\textsubscript{\hai{PP}}
\\ 
%& & \cellcolor{gray!30!white}\textsubscript{\hai{PP}}%[Dat. $\gets$ Akk.]\textsubscript{\hai{PP}} 
%& & & \\ 
\hai{DP}*& \cellcolor{gray!30!white}%Dat. $\gets$ Akk.  
& – & –& –& –\\ 
\hai{UT}* & \cellcolor{gray}%Akk. $\gets$ Dat.  
&\cellcolor{gray!30!white}% Dat. $\gets$ Akk.  
& –& –& –\\ 
\hai{PA} & \multicolumn{1}{@{}Y@{}}{\tikzmark[c]{\raisebox{-1ex}{–}\raisebox{1ex}{\hspace{1ex}\textsubscript{\hai{PP}}}}}%\cellcolor{gray!30!white}/\textsubscript{\hai{PP}}%Dat. $\gets$ Akk. 
&\cellcolor{gray!30!white}% Dat. $\gets$ Akk.  
& –& –& \cellcolor{gray!30!white}%Dat. $\gets$ Akk.
\\ 
%& \cellcolor{gray!30!white}\textsubscript{\hai{PP}}%[Dat. $\gets$ Akk.]\textsubscript{\hai{PP}}
%& & & & \cellcolor{gray}\textsubscript{\hai{PP}}%[Nom./Akk. $\gets$ Dat.]\textsubscript{\hai{PP}} %Ricarda hier PP anders als ohne
%\\ 
\hai{IA} &\cellcolor{gray}\textsubscript{\hai{PP}}% [Akk. $\gets$ Dat.]\textsubscript{\hai{PP}}  
&\cellcolor{gray}\textsubscript{\hai{PP}}% [Akk. $\gets$ Dat.]\textsubscript{\hai{PP}}
 & –& –& –\\ 
\hai{FM} & \cellcolor{gray!30!white}%Dat. $\gets$ Akk. 
& – & –& –& \cellcolor{gray}%Nom./Akk. $\gets$ Dat.
\\ 
\hai{AB} & \cellcolor{gray!30!white}%Dat. $\gets$ Akk. 
& – & –& –& \cellcolor{gray}%Nom./Akk. $\gets$ Dat.
\\ 
\hai{JP} &\cellcolor{gray!30!white}% Dat. $\gets$ Akk. 
&\cellcolor{gray!30!white}% Dat. $\gets$ Akk. 
& –&  \cellcolor{black}%1. Sg. Dat. 
 &\cellcolor{gray!30!white}% Dat. $\gets$ Akk.
 \\ 
\hai{SS} & \cellcolor{gray!30!white}%Dat. $\gets$ Akk. 
& – & –& –& –\\ 
\hai{FL} & – & – & –&  \cellcolor{black}%1. Sg. Dat.  
& –\\ 
\hai{LP} & – & – & –& –& \cellcolor{gray!30!white}%Dat. $\gets$ Akk.
\\ 
\hai{VD} &\cellcolor{gray!30!white}% Dat. $\gets$ Akk. 
 & \cellcolor{gray!30!white}%Dat. $\gets$ Akk.  
 & –& –& –\\ 
\hai{AD} & \cellcolor{gray!30!white}% Dat. $\gets$ Akk.
 & \cellcolor{gray!30!white}%Dat. $\gets$ Akk.  
 & –& –&\multicolumn{1}{@{}Y@{}}{\tikzmark[e2]{\raisebox{-1ex}{–}\raisebox{1ex}{\hspace{1ex}\textsubscript{\hai{PP}}}}}%\cellcolor{gray!30!white}%Dat. $\gets$ Akk.;
 \\
%& & & & & \cellcolor{gray}\textsubscript{\hai{PP}}%[Nom./Akk. $\gets$ Dat.]\textsubscript{\hai{PP}}
%\\  %Ricarda hier PP anders als ohne
\hai{MV} & \cellcolor{gray!30!white}% Dat. $\gets$ Akk. 
& – & –& –&\cellcolor{gray}\textsubscript{\hai{PP}}% [Nom./Akk. $\gets$ Dat.]\textsubscript{\hai{PP}}
\\ 
\hai{DG} &\cellcolor{gray!30!white}\textsubscript{\hai{PP}}% [Dat. $\gets$ Akk.]\textsubscript{\hai{PP}}
 & – & –& –& –\\ 
\hai{GW} & – & – & –& \cellcolor{black}% 1. Sg. Dat.  
& \cellcolor{gray!30!white}%Dat. $\gets$ Akk.
\\ 
\hai{SV} & \cellcolor{gray!30!white}%Dat. $\gets$ Akk.
 & – & \cellcolor{gray}%Akk. $\gets$ Dat. 
 & \cellcolor{black}% 1. Sg. Dat.  
 & –\\ 
\hai{AK} & – & – & –& \cellcolor{black}% 1. Sg. Dat. 
& –\\\midrule 
\cellcolor{gray!30!white} & \footnotesize{= Dat. statt Akk.} & \cellcolor{gray} & \footnotesize{= Akk. statt Dat.} & \cellcolor{black} & \footnotesize{= 1. Sg. Dat.}  \\ %\midrule 
   
 \multicolumn{3}{c}{\scriptsize{\hai{/PP} = neben Belegen mit \isi{Pronomen} nach Präposition}}&\multicolumn{3}{c}{\scriptsize{\hai{PP} = ausschließlich Belege mit \isi{Pronomen} nach Präposition}}\\

  \end{tabularx}
		 \caption{Kasussynkretismen bei Personalpronomen im \hai{chrLiJi1}.} 		 \label{tblpronomenchrLiJi1}
		 \end{table}

\begin{tikzpicture}[remember picture,overlay]
\path[fill=white,opacity=0.2](a.north west)--(a.south west) -- (a.south east) -- cycle;
\path[fill=gray,opacity=0.3](a.north east)--(a.south east) -- (a.north west) -- cycle;

\path[fill=white,opacity=0.2](b.north west)--(b.south west) -- (b.south east) -- cycle;
\path[fill=gray,opacity=0.3](b.north east)--(b.south east) -- (b.north west) -- cycle;

\path[fill=white,opacity=0.2](c.north west)--(c.south west) -- (c.south east) -- cycle;
\path[fill=gray,opacity=0.3](c.north east)--(c.south east) -- (c.north west) -- cycle;


\path[fill=white,opacity=0.2](d.north west)--(d.south west) -- (d.south east) -- cycle;
\path[fill=gray,opacity=0.95](d.north east)--(d.south east) -- (d.north west) -- cycle;

\path[fill=white,opacity=0.2](e.north west)--(e.south west) -- (e.south east) -- cycle;
\path[fill=gray,opacity=0.95](e.north east)--(e.south east) -- (e.north west) -- cycle;

\path[fill=gray,opacity=0.3](e2.north west)--(e2.south west) -- (e2.south east) -- cycle;
\path[fill=gray,opacity=0.95](e2.north east)--(e2.south east) -- (e2.north west) -- cycle;
\end{tikzpicture}

\FloatBarrier


%%% tblpronomenjüd\ili{LiJi1}
	 \begin{table}[h!] 

%\begin{longtable}{cccccc}

		\begin{tabularx}{\textwidth}{lccccc}
		\midrule 

\textbf{Quelle} &\textbf{1. Sg.}  & \textbf{2. Sg.} &  \textbf{3. Sg. m.} &\textbf{1. Pl. Nom.}  & \textbf{Höfl.}\\ \midrule 

\hai{GuS1} & \multicolumn{1}{@{}Y@{}}{\tikzmark[f]{\raisebox{-1ex}{–}\raisebox{1ex}{\hspace{1ex}\textsubscript{\hai{PP}}}}}%\cellcolor{gray!30!white}%Dat. $\gets$ Akk. 
& – & –& 
\cellcolor{black}%1. Sg. Dat.  
& –\\  
% & \cellcolor{gray!30!white}\textsubscript{\hai{PP}}%[Dat. $\gets$ Akk.]\textsubscript{\hai{PP}} 
% & && & \\
\hai{GuS5} & \cellcolor{gray!30!white}%Dat. $\gets$ Akk. 
& – & –& \cellcolor{black}%1. Sg. Dat.
& –\\
\hai{GuS10} & \multicolumn{1}{@{}Y@{}}{\tikzmark[g]{\raisebox{-1ex}{–}\raisebox{1ex}{\hspace{1ex}\textsubscript{\hai{PP}}}}}%\cellcolor{gray!30!white}%Dat. $\gets$ Akk.  
& \cellcolor{gray!30!white}%Dat. $\gets$ Akk.  
& –& –&  \cellcolor{gray}%Nom./Akk. $\gets$ Dat.
\\
% & \cellcolor{gray!30!white}\textsubscript{\hai{PP}}%[Dat. $\gets$ Akk.]\textsubscript{\hai{PP}} 
% & && & \\
\hai{GuS15} & \cellcolor{gray!30!white}%Dat. $\gets$ Akk. 
& – & –& –&  \cellcolor{gray}%Nom./Akk. $\gets$ Dat.
\\
\hai{GuS23} & \cellcolor{gray!30!white}%Dat. $\gets$ Akk. 
& – & \cellcolor{gray!30!white}%Dat. $\gets$ Akk.
& –& \cellcolor{gray!30!white}%Dat. $\gets$ Akk.
\\
\hai{PAlsleben} & \cellcolor{gray!30!white}%Dat. $\gets$ Akk. 
& – & –& –& –\\
\hai{PBreslau} & \cellcolor{gray!30!white}%Dat. $\gets$ Akk.  
& \cellcolor{gray!30!white}%Dat. $\gets$ Akk.  
& –& –& –\\
\hai{PBerlin1} & – & – & –& –&  \cellcolor{gray}%Nom./Akk. $\gets$ Dat.
\\
\hai{PBerlin2} & – & – & –& \cellcolor{black}%1. Sg. Dat. 
& –\\\midrule 
\cellcolor{gray!30!white} & \footnotesize{= Dat. statt Akk.} & \cellcolor{gray} & \footnotesize{= Akk. statt Dat.} & \cellcolor{black} & \footnotesize{= 1. Sg. Dat.}  \\ %\midrule 


 \end{tabularx}
		 \caption{Kasussynkretismen bei Personalpronomen im \hai{jüdLiJi1}}
		 \label{tblpronomenjüdLiJi1}
		 \end{table}
\begin{tikzpicture}[remember picture,overlay]
\path[fill=white,opacity=0.2](f.north west)--(f.south west) -- (f.south east) -- cycle;
\path[fill=gray,opacity=0.3](f.north east)--(f.south east) -- (f.north west) -- cycle;

\path[fill=white,opacity=0.2](g.north west)--(g.south west) -- (g.south east) -- cycle;
\path[fill=gray,opacity=0.3](g.north east)--(g.south east) -- (g.north west) -- cycle;


\end{tikzpicture} 

	 \FloatBarrier