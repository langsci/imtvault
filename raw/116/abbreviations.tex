\addchap{Abkürzungsverzeichnis}
%\section*{Abkürzungsverzeichnis} 
\subsection*{Sprachen}
Jiddische Varietäten
\begin{tabbing}
daaaaaaaaaas\qquad\ \= \kill

\hai{aj.} \> \ili{altjiddisch}\\
\hai{FiJi} \> \ili{Filmjiddisch}\\
\hai{LiJi} \> \ili{Literaturjiddisch}\\
\hai{LiJi1} \> \ili{Literaturjiddisch} des 18. und 19. Jahrhunderts\\
\hai{LiJi2} \> \ili{Literaturjiddisch} des späten 20. und 21. Jahrhunderts\\
\hai{LiHe} \> \ili{Literaturhebräisch}\\
\hai{mj.} \> \ili{mitteljiddisch}\\
\hai{NOJ} \> \ili{Nordostjiddisch}\\
\hai{NÜJ} \> \ili{nördliches Übergangsjiddisch}\\
\hai{NWJ} \> \ili{Nordwestjiddisch}\\
\hai{oj.} \> \ili{ostjiddisch}\\
\hai{OJ} \> \ili{Ostjiddisch}\\
\hai{SOJ} \> Süd\ili{ostjiddisch}\\
\hai{SÜJ} \> \ili{südliches Übergangsjiddisch}\\
\hai{SWJ} \> \ili{Südwestjiddisch}\\
\hai{urj.} \> \ili{urjiddisch, protojiddisch}\\ %rs kleinschreiben
\hai{wj.} \> \ili{westjiddisch}\\
\hai{WJ} \> \ili{Westjiddisch}\\
\hai{ZOJ} \> Zentral\ili{ostjiddisch}\\
\hai{ZWJ} \> Zentral\ili{westjiddisch}\\
\end{tabbing}
%  %\noindent
Weitere germanische Varietäten
\begin{tabbing}
daaaaaaaaaas\qquad\ \= \kill
\hai{afr.} \> \ili{afrikaans}\\
\hai{ahd.} \> \ili{althochdeutsch}\\
\hai{aleman.} \> \ili{alemannisch}\\
\hai{bair.} \> \ili{bairisch}\\
\hai{dän.} \> \ili{dänisch}\\
\hai{elsäss.}\> elsässisches Nieder\ili{alemannisch}\\
\hai{engl.} \> \ili{englisch}\\
\hai{frnhd.} \> \ili{frühneuhochdeutsch}\\
\hai{germ.} \> \ili{germanisch}\\
\hai{isl.} \> \ili{isländisch}\\
\hai{mhd.} \> \ili{mittelhochdeutsch}\\
\hai{moselfränk.} \> \ili{moselfränkisch}\\
\hai{ndt.} \> \ili{niederdeutsch}\\
\hai{ndl.} \> \ili{niederländisch}\\
\hai{rheinfränk.} \> \ili{rheinfränkisch}\\%rs n fehlt
\hai{schwäb.} \> \ili{schwäbisch}\\
%\hai{schwed.} \> \ili{schwedisch}\\
\hai{westfl.} \> \ili{westflämisch}\\
\hai{westfr.} \> \ili{westfriesisch}\\
%\clearpage
\end{tabbing}
%  %\noindent
Nicht-germanische Sprachen
\begin{tabbing}
daaaaaaaaaas\qquad\ \= \kill
\hai{fr.} \> \ili{französisch}\\
\hai{hebr.} \> \ili{hebräisch}\\
\hai{it.} \> \ili{italienisch}\\
\hai{poln.} \> \ili{polnisch}\\
%\hai{russ.} \> \ili{russisch}\\
\hai{tsch.} \> \ili{tschechisch}\\

\end{tabbing}


%\thispagestyle{empty}
\subsection*{Fachtermini}
\begin{tabbing}
daaaaaaaaaas\qquad\ \= \kill
\hai{ACI} \>  Accusativus cum infinitivo\\
\hai{Adv.} \> Adverb\\
%\hai{Adj.} \> \isi{Adjektiv}\\
\hai{AdvP} \> Adverbialphrase\\%rs kein Spatium
\hai{Akk.} \> \isi{Akkusativ}\\
\hai{AP} \> Adjektivphrase\\
\hai{Art.} \> \isi{Artikel}\\
\hai{CxG} \>  Construction grammar (Konstruktionsgrammatik)\\
\hai{Dat.} \> \isi{Dativ}\\
\hai{Dim.} \> Diminutiv\\
\hai{f.} \> feminin\\
\hai{FK} \> Frequenzklasse\\
\hai{Gen.} \> \isi{Genitiv}\\
%\hai{intrans.} \> intransitiv\\
\hai{IPP} \> Infinitivus pro participio (\isi{Ersatzinfinitiv})\\
%\hai{Konj.} \> \isi{Konjunktion}\\
\hai{LSK} \> linke Satzklammer\\
\hai{m.} \> maskulin\\
\hai{MF} \> \isi{Mittelfeld}\\
\hai{ND} \> Negative Doubling\\
\hai{NF} \> \isi{Nachfeld}\\
\hai{NS} \> Negative Spread\\
\hai{n.} \> neutrum\\
\hai{Nom.} \> Nominativ\\
\hai{NP} \> Nominalphrase\\
\hai{OV} \> Objekt-Verb Grundwortstellung\\
\hai{Pl.} \> Plural\\
\hai{PP} \> Pr\"apositionalphrase\\
\hai{PPI} \> Participium pro infinitivo\\
%\hai{Präp.} \> \isi{Präposition}\\
\hai{Rel.} \> Relativ\\
Partkl. \> \isi{Partikel}\\
\hai{Pron.} \> \isi{Pronomen}\\
\hai{RSK} \> rechte Satzklammer\\
\hai{Sg.} \> Singular\\
\hai{Stabw.} \> Standardabweichung\\
%\hai{trans.} \> transitiv\\
%\hai{UG} \> Universalgrammatik\\
\hai{VO} \> Verb-Objekt Grundwortstellung\\
\hai{VR} \> Verb raising\\
\hai{VPR} \> Verb projection raising\\
Zsf. \> Zusammenfassung\\
∅ \> Nullendung\\
\end{tabbing}
%\thispagestyle{empty}

%\hai{chr.} \> christlich}
%\hai{jüd.} \> jüdisch}
% \newpage

\subsection*{Atlanten}
 
\begin{tabularx}{\textwidth}{lX}
\hai{ADA} &Atlas der deutschen Alltagssprache; online publiziert über\\
&  \url{www.atlas-alltagssprache.de/} (Stand: September 2014)\\

%\hai{DSA} \> Deutscher Sprachatlas; online zugänglich über\\
 %\>  \url{http://www.regionalsprache.de/} (Stand: September 2014)\\

\hai{KDSA} & Kleiner Deutscher Sprachatlas. Im Auftrag des Forschungsinstituts \\
& für deutsche Sprache – Deutscher Sprachatlas – Marburg (Lahn).\\ 
& Dialektologisch bearbeitet von Werner H. Veith, computativ \\
& bearbeitet von Wolfgang Putschke. Band 1: \isi{Konsonantismus},\\
& Teil 1: \isi{Plosive}. Band 2: \isi{Vokalismus}, Teil 1: Kurzvokale. Unter Mitarbeit von Lutz Hummel. Band 2: \isi{Vokalismus}, Teil 2: Langvokale, Diphthonge, Kombinationskarten. Niemeyer, Tübingen 1983–1999.\\
\end{tabularx}

\noindent
\begin{tabularx}{\textwidth}{lX}
\hai{LCAAJ}&Language and Culture Atlas of Ashkenazic Jewry. Hg. von Marvin Herzog, Ulrike Kiefer et al., Tübingen, Vol. 1. Historical and theoretical foundations (1992), Vol. 2. Research tools (1995), Vol. 3. The eastern Yiddish – western Yiddish continuum (2000)\\

\hai{WA} &Sprachatlas des Deutschen Reichs. Georg Wenkers handgezeichnetes Original (kurz: Wenkeratlas); online zugänglich über \url{www.regionalsprache.de/} (Stand: September 2014)\\

\hai{WApron} & Wenker, Georg (1886): Pronomina in Nordwestdeutschland. Straßburg: Trübner; online zugänglich über \url{www.regionalsprache.de/} (Stand: September 2014)\\

\hai{WEK}&Ergänzungskarten zum Deutschen Sprachatlas.  Nacherhebungen in Süd- und Osteuropa von Peter Wiesinger; online zugänglich über \url{www.regionalsprache.de/} (Stand: September 2014)\\

\hai{SDS} & Sprachatlas der Deutschen Schweiz; online zugänglich über \url{www.regionalsprache.de/} (Stand: September 2014)\\
\end{tabularx}
 



\subsection*{Korpora}

 \begin{table}[H]
\begin{tabularx}{\textwidth}{lX}
\hai{DCY} & \qu{Diachronic Corpus of Yiddish} (Eigenbezeichnung)  =  \isi{Korpus} aus 103 Texten für den Zeitraum 1462–1993;  erstellt, annotiert und zur Verfügung gestellt von Beatrice Santorini.  Datengrundlage für Santorini (1989; 1992; 1993a; 1993b; 1994; 1995)  und Wallenberg (2012b; 2012b; 2013) \\
\hai{DeReKo} &Deutsches Referenzkorpus / Archiv der Korpora geschriebener \\
& Gegenwartssprache 2013-I; Mannheim: Institut für Deutsche Sprache; online zugänglich über \url{www.ids-mannheim.de/DeReKo} (Stand: September 2014)\\
\hai{CMY} & Corpus of Modern Yiddish; Birzer et al. (Universität Regensburg); online  zugänglich über \url{http://web-corpora.net/YNC}  (Stand: September 2014)\\
\end{tabularx}

\end{table}%

% \vspace{2cm}
% \begin{tabbing}
% daaaaaaaaaas\qquad\ \= \kill
\noindent\ding{43} Die Kürzel der untersuchten Quellen werden im Anhang (S. \pageref{appendixchrliji1}, \pageref{appendixjuedliji1}) %RS Klammer fehlt
 aufgeschlüsselt. %\pageref{appendixliji2}) 
% \end{tabbing}


% \clearpage

%\thispagestyle{empty}
% \subsection*{Transliterationssystem} %\phantomsection \addcontentsline{toc}{section}{Transliterationssystem} %\renewcommand\refname{Transliterationssysteme} 


\newpage
\subsection*{Transliterationssysteme}\enlargethispage{2\baselineskip}

{
\small
\begin{tabular}{lll}
		%\begin{tabular}{lll}
		\lsptoprule
Graphem  & Transliteration \hai{OJ} (\hai{YIVO}) 	&  Transliteration spätes \hai{WJ}\\
\midrule 
א& &
\\  \RL{א\makebox(-1.25,-1.25)[r]{\libertineGlyph{uni05B7}}}&a&a
\\  \RL{א\makebox(-1.25,-1.25)[r]{\libertineGlyph{uni05B8}}}&o&o
\\ב&b&b
\\ \RL{{ב\makebox(-0.8,9)[r]{\libertineGlyph{uni207B}}}}&v&w
\\ג&g&g
\\ד&d&d
\\ה&h&h (am Satzende z.\,T. e)
\\ו&u&u
\\ז&z&s
\\ח&kh&ch
\\ט&t&t
\\י&y, i&j/i
\\כּ&k&k
\\ ך כ &kh&ch
\\ל&l&l
\\ ם מ &m&m
\\ן נ &n&n
\\ס&s&s
\\ע&e&e
\\פּ&p&p
\\ ף  {פ\makebox(-0.8,9)[r]{\libertineGlyph{uni05B7}}} &f&f
\\ץ צ &ts&z
\\ק&k&k
\\ר&r&r
\\ש&sh&sch
\\    {ש\makebox(-4,9)[r]{\libertineGlyph{uni0307}}}&s&s
\\תּ&t&t
\\ת&s&s
\\װ&v&w
\\זש&zh&–
\\טש&tsh&tsch
\\ױ&oy&au/ou
\\ײ&ey&ei
\\  \RL{{יי}\makebox(-1.5,-3.5)[r]{\libertineGlyph{uni207B}}}&ay&ai			 	  % \end{tabular}%\caption{Transliteration}%\label{tranksliteration}
		\\\lspbottomrule\end{tabular}
}
         
