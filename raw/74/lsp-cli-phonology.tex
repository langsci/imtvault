%talk about CVNNV, generally homorganic CC sequence

\chapter{Phonology outline}
\label{sec:chap-phono}


\section{Introduction}
\label{sec:intro-phono}


This section presents a brief outline of Chakali phonology. An inventory of phonetic and phonemic vowels and consonants, the syllable structures, the phonotactics and the suprasegmentals are introduced.   The description makes use of the International Phonetic Alphabet (IPA)  symbols to represent the sounds of the language. These should not be confused with the same IPA symbols used to represent  sets of phonological features, i.e.  distinctive feature bundles. This domain representation mismatch is usually resolved by containing phonemes and underlying representations within slash brackets  and  speech sounds and surface forms within square brackets, e.g.  /kæt/ vs.  [kʰæʔ]  `cat'. The former is an abstraction, while the latter represents an utterance.  For the rest of this exposition, if a Chakali expression is presented without the slash or  square brackets, it should be interpreted as a broad phonetic transcription. The parts of speech of  Chakali expressions are provided in many instances: on the one hand,  having the  information on the part of speech avoids ambiguity since the English gloss is often inadequate. On the other hand,  it assists the search for phonological behaviour conditioned by lexical category.  All the examples used as evidence are candidates for look-up in the dictionary of Part \ref{part:part1}. The abbreviation list starts on page \pageref{sec-ABB}.

\newpage 
\section{Segmental phonemes inventory}
\label{sec:seg-phon-invent}

This section introduces the segmental phonemes of Chakali and their contrasts 
by determining the phonetic properties in minimal contexts of speech sound 
patterns, when possible.  Near-minimal pairs appear, yet the majority of the 
evidence provided is based on  minimal pairs. The vowels are examined first, 
followed by the consonants.

 %I assumed standard features?
 %I assumed the phonetic feature IPA, but uses ...?
% The set of (articulatory) features proposed in 
% \cite{Lade07} is assumed.

\subsection{Vowels}
\label{sec:vowels}

Chakali is treated as a language with nine underlying vowels and eleven
surface vowels. They
are presented in Figure \ref{fig:Phon-phon-srf} in  vowel diagrams.  The 
surface vowels [ɑ] and  [ə] are discussed  at the end of this section. 


\begin{figure}


\subfigure[Vowel Phonemes]{
\begin{tabular}{ccc}
 \begin{vowel}[plain]
\putcvowel{i}{1}
\putcvowel{e}{2}
\putcvowel{ɛ}{3}

\putcvowel{ɔ}{6}
\putcvowel{o}{7}
\putcvowel{u}{8}

\putcvowel{ɪ}{13}
\putcvowel{ʊ}{14}
\putcvowel{a}{4}
\end{vowel}
\end{tabular}
}
\qquad
\subfigure[Surface Vowels]{

 \begin{tabular}{ccc}

\begin{vowel}[plain]
\putcvowel{i}{1}
\putcvowel{e}{2}
\putcvowel{ɛ}{3}
\putcvowel{ɑ}{5}
\putcvowel{ɔ}{6}
\putcvowel{o}{7}
\putcvowel{u}{8}
\putcvowel{ə}{11}
\putcvowel{ɪ}{13}
\putcvowel{ʊ}{14}
\putcvowel{a}{4}
\end{vowel}


 \end{tabular}

}

\caption{Vowel phonemes and surface vowels in Chakali \label{fig:Phon-phon-srf}}
\end{figure}


Each vowel is presented below with minimal contrasts to motivate their phonemic
status.   Two sounds are contrastive if interchanging the two can change the
meaning of the word. The vowels are presented in opposition for their height,
roundness,  and tongue root properties. Since Chakali does not show any
contrast of roundness and backness in the non-low vowels, roundness, and 
backness   are put together in the description under a {\sc ro}(und) feature. The tongue root distinction is gathered under the feature {\sc atr} (i.e. advanced
tongue root). Low and high are treated under {\sc height} in the
subsequent tables, but are captured in the summary Table \ref{tab:featspec}
with the features {\sc hi}   and {\sc lo},  and the feature values +
and --. 


\newpage
\largerpage[2]
\subsubsection{Close front unrounded {i}.}
\label{sec:i-phon-vowel}
The vowel [{i}] is  front, unrounded, high, and tense. 

\begin{center}
\begin{Qtabular}{llll}
\lsptoprule\small
Contrast &   cli. example & Gloss & PoS\\[1ex] \midrule
{\sc height}	& zíŋ 	&	tail	& n\\	
	& záŋ &	rest area	& n\\
    
	&	pítí			&	survive	& v\\
	&	pétí	&	finish	& v\\[0.5ex] \midrule


{\sc ro}    &	gbíŋ	&	bracelet	& n\\
	&	gbóŋ	&	type of tree	& n\\
    
	&	kísì &  pray & v\\
	&	kùsì & unable &  v\\[0.5ex] \midrule
	

{\sc atr}  	& ɲìŋ́ &  sore  & n\\
    &  ɲɪ́ŋ &   tooth & n\\
	& dì	&	eat	&	v\\
	& dɪ̀    &	if	&		conn\\
\lspbottomrule
\end{Qtabular}
\end{center}


\subsubsection{Near-close near-front unrounded {ɪ}.}
\label{sec:I-phon-vowel}
The vowel [{ɪ}] is  front, unrounded, high, and lax. 
\begin{center}
 \begin{Qtabular}{llll}
\lsptoprule\small
Contrast &   cli. example & Gloss & PoS\\[1ex] \midrule

{\sc height}  & pɪ̀sɪ̀	&	scatter		&	v  \\
	&	pɛ́sɪ́	& slap	&	v\\
	&	hɪ́lː	&	witch	&	n\\
	& 	hál	&	egg	& n\\[0.5ex] \midrule


{\sc ro}  	& tɪ̀sɪ̀	&	shallow (be) & v  \\
	&	tʊ́sɪ́	&	move over & v \\
	&	tʃɪ́ŋá	&	stand  & 	v \\
	&	tʃʊ́ŋá	&	carry load & 	v  \\[0.5ex] \midrule
	
{\sc atr}  	& fɪ̀	&	would	& pv\\
	&	fí	&	ten  &	num  \\
	&	zɪ̀ŋ́	&	bat	& n  \\
	&	zíŋ	&	tail	&  n \\
\lspbottomrule
\end{Qtabular}
\end{center}



\pagebreak
\subsubsection{Close-mid front unrounded {e}.}
\label{sec:e-phon-vowel}


The vowel [{e}] is  front, unrounded, mid, and tense. 

\begin{center}

\begin{Qtabular}{llll}
\lsptoprule\small
Contrast &   cli. example & Gloss & PoS\\[1ex] \midrule


{\sc height} 	&	bèlè	&	type of bush dog	&	n\\
	&	bìlè	&	put down	&	v\\
	&	péŋ	&	penis &	n\\
	&	páŋ	&	molar & 	n\\[0.5ex] \midrule



{\sc ro} & zèŋ́ & big & n	\\
	&zóŋ & insult & n	\\
	&	pél̀	&	roofing beam&		n\\
	&	pól	&	vein	&	n\\[0.5ex] \midrule
	
{\sc atr} 	& 	bèŋ́&	law	& n\\
	&	bɛ́ŋ	&	type of tree	& n \\
\lspbottomrule
\end{Qtabular}

\end{center}




\subsubsection{Open-mid front unrounded {ɛ}.}
\label{sec:E-phon-vowel}
The vowel [{ɛ}] is  front, unrounded, mid, and lax. 



\begin{center}

\begin{Qtabular}{llll}
\lsptoprule\small
Contrast &   cli. example & Gloss & PoS\\[1ex] \midrule


{\sc height} 	&	tʃɛ̀rà	&	barter	&v\\
	&	tʃàrà	&	straddle	&v\\
	&	pɛ́lá	 & lean on &   v\\
	&	pɪ̀là	& hit down repeatedly  & v\\[0.5ex] \midrule

{\sc ro}  	&	mɛ̀ŋ́	&	dew	& n\\
	&	mɔ́ŋ	&	vagina	& n\\
	&	pɛ́	&	add	& v\\
	& 	pɔ̀	&	protect	& v\\[0.5ex] \midrule

	
{\sc atr} 	& 	sélː	& animal & n\\
	&	sɛ́l &	wood shaving & n \\
\lspbottomrule

\end{Qtabular}

\end{center}



\pagebreak

\subsubsection{Close-mid back rounded {o}.}
\label{sec:o-phon-vowel}
The vowel [{o}] is  back, rounded, mid, and tense. 

\begin{center}
\begin{Qtabular}{llll}
\lsptoprule\small
Contrast &   cli. example & Gloss & PoS\\[1ex] \midrule


{\sc height} 	&	ʔól	&	type of mouse	&  n  \\
	&	ʔúl	&	navel	& n\\
	&	hól	&	type of tree	& n  \\
	&	hál	&	egg	& n\\[0.5ex] \midrule



{\sc ro}  	&	bóŋ	& big water pot	& n  \\
	&	bèŋ́	& law &  	n \\		  
	&	pól	&	pond	& n  \\
	&	pél	&	roofing support	& n\\[0.5ex] \midrule


	
{\sc atr}   	&	kóŋ	&	Kapok tree	& n  \\
	&	kɔ́ŋ	&	cobra	& n\\		  
	&	hól	&	type of tree	& n  \\
	&	hɔ́l	&	charcoal	& n \\
\lspbottomrule

\end{Qtabular}

\end{center}



\largerpage[3]
\subsubsection{Open-mid back rounded {ɔ}.}
\label{sec:O-phon-vowel}
The vowel [{ɔ}] is  back, rounded, mid, and lax.

\begin{center}

\begin{Qtabular}{llll}
\lsptoprule\small
Contrast &   cli. example & Gloss & PoS\\[1ex] \midrule


{\sc height} 	&	pɔ̀	&	protect  & 	v  \\
	&	pʊ́	&	spit	& v\\		  
	&	kɔ́lá	& sharpen	& v  \\
	&	kàlà	 & rope & v\\[0.5ex] \midrule


{\sc ro}  	&	mɔ́ŋ	&  vagina &	n \\ 
	&	mɛ̀ŋ́	& mist	& n\\	  
	&	pɔ̀là	 &  fat &	v  \\
	&	pɛ́lá	 & lean on &	v\\[0.5ex] \midrule


{\sc atr}	&	pɔ̀	&	protect	& v  \\
	&	pó	&	collect	& v\\		  
	&	kɔ́ŋ	& cobra  & n\\
	&	kóŋ	 & type of tree &	n \\
\lspbottomrule

\end{Qtabular}

\end{center}


 

\subsubsection{Close back rounded {u}.}
\label{sec:u-phon-vowel}
The vowel [{u}] is   back, rounded, high, and tense.


\begin{center}

\begin{Qtabular}{llll}
\lsptoprule\small
Contrast &   cli. example & Gloss & PoS\\[1ex] \midrule


{\sc height} 	&	pú	&	lie  on  stomach	& v  \\
	&	pó	&	collect	&  v  \\
	&	súl	&	mud  fish	& n  \\
	&	sàĺ	&	flat  roof &  n		\\[0.5ex] \midrule	
 

{\sc ro} 	&	bùú & silo & n\\
& bíí & seed &   n\\

	&	kùsì	&	unable	& v \\
	&	kísì&	pray	&v \\[0.5ex] \midrule	  


{\sc atr}	&	zúl&	millet 	& n  \\
	&	zʊ́ʊ́l	&	tuber & n\\
	&	pú	&	cover	& v \\ 
	&	pʊ́	&	spit	& v \\
\lspbottomrule
\end{Qtabular}

\end{center}


\largerpage[2]
\subsubsection{Near-close near back rounded {ʊ}.}
\label{sec:-phon-vowel}
The vowel [{ʊ}] is   back, rounded, high, and lax.


\begin{center}

\begin{Qtabular}{llll}
\lsptoprule\small
Contrast &   cli. example & Gloss & PoS\\[1ex] \midrule


{\sc height} 	&	vʊ́g	&	shrine	& n  \\
	&	vɔ̀ǵ	&	south	&  n  \\
	&	lʊ́lá	&	give birth&	v \\
	&	lálá	&	open	& v\\[0.5ex] \midrule	 
				  

{\sc ro}	&	mʊ́sɪ́ &rain& v\\
&mɪ́sɪ́& sprinkle& v\\
	&	bʊ̀là	 & tasteless	 & v   \\
	&	bɪ̀là & try to solve &	v  \\[0.5ex] \midrule
 

{\sc atr}	&	tʃʊ́ʊ́rɪ́ &	torn	& v  \\
	&	tʃùùrì	&	pour & v\\	  
	&  zʊ́ʊ́l	&		tuber 	& n  \\
	&	zúl	&	millet	& n \\
\lspbottomrule
\end{Qtabular}

\end{center}

\pagebreak

\subsubsection{Open front unrounded {a}.}
\label{sec:LOW-phon-vowel}
The vowel [{a}] is unrounded and low.



\begin{center}

\begin{Qtabular}{llll}
\lsptoprule\small
Contrast &   cli. example & Gloss & PoS\\[1ex] \midrule

e	& 	gàŕ	&	cloth	&	n\\  
	&	gèŕ	&	lizard	& 	n\\[0.5ex] \midrule	  


ɛ   	& 	pàrà	&	farm	&		v  \\
	&	pɛ̀rà	&	weave  & v\\[1ex]\midrule	


i	&	záŋ &	rest area	&  n\\  
	&	zíŋ 	&	tail	& n\\	[1ex]\midrule	

ɪ	&	tàtɪ̀ & stretch&	v\\ 
	&	tɪ̀tɪ̀	& rub &	v\\[1ex]\midrule	

o	&	hál&	egg 	&n  \\
	& hól&	type of tree 	& 	n\\[1ex]\midrule

ɔ 	&	pàlà & 	flow	& v \\ 
	&	pɔ̀là	 & be  fat &	v\\[1ex]\midrule
			

u	&	páŋ	&		molar &	n  \\
	&	púŋ	&	feather	&	n\\[1ex]\midrule

ʊ	&	bár	&	chance	& n  \\
	&	bʊ́r	&	dust	& n \\
\lspbottomrule

\end{Qtabular}

\end{center}

When considering  \cite{Rowl65, Crou66, Gray69,   Toup95, Crou03}, the Chakali vowel phoneme 
inventory appears to match one of the two posited types of  phonemic inventories found in other  
Southwestern \ili{Grusi} (SWG) languages.\footnote{`Phonemic' is used in its broad sense. Since phonology 
has diverse theoretical orientations,  an inventory of phonemes does not mean much unless the 
features making those phonemes are expressed in the model.  Thus in the phonological descriptions 
of the five SWG languages cited (i.e. \ili{Sisaala}, \ili{Vagla}, \ili{Tampulma},  \ili{Pasaale}, and  \ili{Dɛg}), it is assumed 
that the phonemic inventory in each monograph is built upon the classification proposed in their 
tables and charts, which use features like \textsc{atr}, \textsc{round}, \textsc{back}, etc.} In 
\citet[15]{Rowl65} the chart of \ili{Sisaala} phonemes gives one [\textsc{low}, \textsc{central}] vowel 
/a/ and one [\textsc{mid}, \textsc{open}, \textsc{central}] vowel /ʌ/. \citet[17]{Crou66} provides 
the same symbols /a/ and /ʌ/ for \ili{Vagla}, the former for a [\textsc{low}, \textsc{open}, 
\textsc{central}] vowel and the latter for a [\textsc{low}, \textsc{close}, \textsc{central}] one. 
In \citet[3]{Crou03}, the same symbols /a/ and /ʌ/ are found for \ili{Dɛg}. For them /a/ represents a 
[\textsc{low}, \textsc{--atr}, \textsc{central}] vowel and /ʌ/ a [\textsc{low}, \textsc{+atr}, 
\textsc{central}] vowel.\footnote{\label{fn:info-deg}Modesta Kanjiti, a  \ili{Dɛg} speaker,  and I 
reviewed  in April 2009 the words given as evidence for the contrast /a/ and /ʌ/ in 
\citet[20--21]{Crou03}. Despite  \citeauthor{Crou03}'s assertion,   Mme. Kanjiti could not confirm that /a/ 
and /ʌ/ were different sounds based on the word list provided. This contrast needs to be verified, 
although dialect difference could account for this.}   The phoneme inventories of \citet[16]{Toup95} 
and  \citet[21]{Gray69} do not report the distinction. The former identifies the contrast 
phonetically and claims that [a] and [ʌ] occur in free  \is{variant}variation. In fact, Toupin provides the 
reader with [a] and [ʌ] in exactly the same environment: the word for `hoe' and `back' are both 
transcribed with [a] and [ʌ]  \citep[26]{Toup95}. He postulates one [\textsc{low}] phoneme (i.e.  
/a/) in the inventory \citep[16]{Toup95}.

\begin{figure}

\begin{tabular}{ccc}
\begin{vowel}[plain]
\putcvowel{i}{1}
\putcvowel{e}{2}
\putcvowel{ɛ}{3}
%\putcvowel[l]{\textscripta}{5}
\putcvowel{ɔ}{6}
\putcvowel{o}{7}
\putcvowel{u}{8}
%\putcvowel{\textschwa}{11}
\putcvowel{ɪ}{13}
\putcvowel{ʊ}{14}
\putcvowel{a}{4}
\end{vowel}
\hspace*{3ex}
\begin{vowel}[plain]
\putcvowel{i}{1}
\putcvowel{e}{2}
\putcvowel{ɛ}{3}
\putcvowel{ɑ/ʌ}{5}
\putcvowel{ɔ}{6}
\putcvowel{o}{7}
\putcvowel{u}{8}
%\putcvowel{\textschwa}{11}
\putcvowel{ɪ}{13}
\putcvowel{ʊ}{14}
\putcvowel{a}{4}
\end{vowel}
%{3\vowelhunit}{3\vowelvunit}
\end{tabular}

\caption[Two vowel inventories in SWG]{9-  vs. 10-vowel inventory in some 
Southwestern \ili{Grusi} languages}
\label{tab:9vs10inventory}
\end{figure}


Even though \citet{Mane79} reconstructs a 7-vowel inventory for Proto-Cen\-tral 
\ili{Gur}, the  phonological inventories  appearing in Figure \ref{tab:9vs10inventory} 
are common to many Volta-Congo languages (\citealt[81]{Daku97}, 
\citealt[18]{Casa03a}). Further, they usually encode a phenomenon known as 
Cross-Height Vowel Harmony (CHVH) \citep{Stew67, Casa03, Casa08}, in which 
harmony is operative at more than one height. In Chakali, the two  \textsc{atr} 
harmony sets  \{{i, e, u, o}\} and  \{{ɪ, ɛ, ʊ, ɔ}\} contain high 
and non-high vowels,  and as a rule,  vowels agree  in \textsc{atr} value 
within the  stem domain. Typically the vowel /{a}/ co-occurs with 
\textsc{--atr} vowels within monomorphemic words.\footnote{This is common among 
9-vowel  inventory according to \citet[528]{Casa08}. However, some English loans 
violate that statement, e.g.  {\sls sìgáárì} `cigarette',  {\sls ʔéékà} 
`acre',  {\sls sódʒà} `soldier',  and  {\sls mítà} `meter'.}  The topic is 
discussed in detail in Section \ref{sec:vowel-harmony}, but for now let us say 
that a  monomorphemic word cannot carry two vowels of different \textsc{atr} 
sets, that is, [{\sls lʊpɛ}] is possible (it means `seven') but 
*[{\sls lɪpe}], *[{\sls lɛpe}]  *[{\sls lʊpe}] and *[{\sls lɔpe}] are 
ungrammatical strings.

Apart from the nine vowels presented above, the surface vowels  [{ɑ}] and 
[{ə}] can be heard;  [{ɑ}] is perceived as 
if it was produced with the tongue further back in the mouth compared to [{a}]. In 
addition, the vowel [{ɑ}]  is often found  following the \textsc{--atr} vowels (i.e. {ɪ, ɛ, ɔ, ʊ}). Despite the fact that vowel harmony predicts a `lax version'  of /{a}/ in 
some environments (Section \ref{sec:vowel-harmony}),  a 
distinction between [{ɑ}] and  [{a}] is not established. Yet, 
there is  evidence which shows that Chakali should be considered to have only 
one phonemic low vowel, which would make its vowel inventory equivalent the one 
described for  \ili{Pasaale} by \citet{Toup95}.  And, as  written in the 
description of the  noun class system (Section \ref{sec:GRM-noun-classes}),  Chakali behaves similarly to other 9-vowel  languages 
\citep[see][41]{Casa03a}.


The vowel [{ə}] is either an epenthetic vowel or a reduction of a full 
vowel. It surfaces only in specific environments and is never a part of the 
underlying form (see  Section \ref{sec:phonotac}). While both   
[{ɑ}] and   [{ə}] are treated as phonetic vowels, only  
[{ə}]  appears in the dictionary in the phonetic form of an entry. Table 
\ref{tab:featspec} displays the set of features which determines the nine vowel 
phonemes. 


\begin{table}
 \caption{Vowel inventory and distinctive features bundles}
  \label{tab:featspec}
%\begin{footnotesize}
  \begin{tabular}{ll}
\lsptoprule
IPA  & features   \\[1ex] \midrule 

{i} & $[$ {\sc +atr}, {\sc +hi}, {\sc --lo},  {\sc --ro} $]$\\
{ɪ} & $[$ {\sc --atr},  {\sc +hi}, {\sc --lo}, {\sc --ro}$]$\\
{e} & $[$  {\sc +atr},  {\sc --hi}, {\sc --lo},  {\sc --ro}$]$\\
{ɛ} & $[$  {\sc --atr},  {\sc --hi},  {\sc --lo},  {\sc --ro}$]$\\
{o} & $[$  {\sc +atr},  {\sc --hi},  {\sc --lo},  {\sc +ro}$]$\\
{ɔ} & $[$  {\sc --atr},  {\sc --hi}, {\sc --lo}, {\sc +ro}$]$\\
{u} & $[$  {\sc +atr},  {\sc +hi},  {\sc --lo},  {\sc +ro}$]$ \\
{ʊ} & $[$  {\sc --atr},  {\sc +hi},  {\sc --lo},  {\sc +ro}$]$\\
{a} &  $[$  {\sc --atr}, {\sc --hi},  {\sc +lo}, {\sc --ro}$]$\\

\lspbottomrule 


  \end{tabular}
%\end{footnotesize}
 
\end{table}

\newpage %longdistance

\subsubsection{Nasal vowels.}
\label{sec:nasal-v}

Except for  [{ə}],  all vowels have a nasalized counterpart. As 
expected, nasal vowels are less
frequent than their oral counterparts. Nasalized  low vowels   are the most
frequent, whereas  close-mid back rounded vowels  are the least
frequent. Consider the examples in Table
\ref{tab:PHO-nasal-v}.


\begin{table} 

  

  \caption{Nasal vowels
  \label{tab:PHO-nasal-v}}


\begin{Qtabular}{llll}
\lsptoprule
Contrast &   cli. example & Gloss & PoS\\[1ex] \midrule

ẽ	& 	hẽ́hẽ́sè	& 	announcer & 	n\\
	&	sàpúhĩ́ẽ̀	&	pouched rat& 	n\\
        & kálɛ́ŋ-bɪ́lèŋẽ́ẽ̀	&  adjuster	& n\\[0.5ex] \midrule	  



ɛ̃   	& hɛ̃́ŋ	&	arrow&	n\\
& tʃɛ̃̀ɪ̃̀	&	attractiveness&	n\\
& ɲɛ̃́sà	&	malnourished child	&n \\[1ex]\midrule	


ĩ	& hĩ́ĩ́	&	hind leg	&n\\
& mĩ̀ĩ́	&	gun front sight& 	n\\
& záɣắfĩ̀ĩ̀	&	yellow fever& 	n\\[1ex]\midrule	

ɪ̃	 & fɪ̃́ɪ̃́	&	type of fish	&n\\
&  fɪ̃̀ɪ̃̀		&urinate	&v\\
& pɪ̃́ & be fed up&	v\\[1ex]\midrule	

õ &mṍŋgò	&	mango  (ultm. Eng.)	&n\\
 & kpõ̀ŋkpṍŋ	&	cassava	&n \\[1ex]\midrule

ɔ̃ 	& nã̀ɔ̃́	&	cow&	n  \\
&àɲɔ̃́	&	five	&num\\
&hɔ̃́ʊ̃̀	&	 type of grasshopper& 	n \\[1ex]\midrule
			

ũ	& dũ̀ũ̀	&	sow		&	v\\
	&sũ̀ṹ	&	guinea fowl	&	n\\
	& fũ̀ũ̀	&	burn		&	v \\[1ex]\midrule

ʊ̃	& bʊ̃́ʊ̃́ŋ 	&	goat			&	n\\
	&dʊ̃́ʊ̃̀	&	type of snake	&	n \\
	&kʊ̃̀ʊ̃̀	&	to be tired			& 
v\\[1ex]\midrule


ã &		ʔã́ã́	&	bushbuck		& 	n\\
  & 		bã́ã́	&	type of monitor lizard		&	n\\
  & 		sã̀ã̀	&	carve	 		& 	
v\\\lspbottomrule

\end{Qtabular}


\end{table}




At first glance the treatment of nasal vowels may be  reduced to the influence 
 of a nasal speech sound. Overall, nasal vowels are mainly found
adjacent to a nasal consonant (or sometimes preceded by
a glottal fricative).  So it may be more accurate to specify them as oral and
explain the perception of nasality as a coarticulation phenomenon. 
Nonetheless,
nasal vowels are attested where adjacent nasal features are absent. The
(near-)minimal pairs  {\sls fáà}
`ancient' / {\sls fã̀ã̀} `do by force',  {\sls fɪ̀} `preverb particle' / {\sls 
fɪ̃́ɪ̃́}
`type of fish', {\sls zʊ̀ʊ̀} `enter' /  {\sls zʊ̃̀ʊ̃̀} `laziness'  and  {\sls 
tùù}
`go down' /  {\sls tṹṹ} `honey'  show that nasal and oral vowels do 
contrast. 




\subsubsection{Vowel sequences}
\label{sec:vowels-seq}
This section is concerned with the duration of vowel sounds and their segmental
content.  It is shown that Chakali contrasts word meanings based on vowel
length. Section \ref{sec:syllable-types} will present the syllables types in 
which various vowel sequences can occur.

\paragraph{Vowel length.}
\label{sec:short-long-vowels}

A phonetic contrast exists between short and long vowels. The fourth column of Table \ref{tab:Lenght-Phon} gives an hypothesised CV-form of  selected  words spoken isolation by six speakers.
Judging from this data, which consists of (near-)minimal pairs,  a
difference in vowel length can change the meaning of a word. Further, as we will see in Section \ref{sec:GRM-precerv}, there are in addition  slight differences in meaning when some preverb particles are longer. 

\begin{table} 
 
 \caption[Vowel duration]{Vowel duration.
Abbreviation:
cli = Chakali, Gloss = English gloss,  $\sigma$ = syllable type,  
PoS = part of speech,  and  V-duration = mean of
vowel duration for six speakers in milliseconds.}
 \label{tab:Lenght-Phon}
\begin{Qtabular}{lllll}
\lsptoprule
 cli. & Gloss & PoS & $\sigma$ & V-duration\\[1ex]
\midrule

tá	 & abandon	&v&	CV	& 142\\
tàá	 & language	&n&	CVV		& 	227\\

kpà	& take 		&v&	CV	& 139\\
kpáá	 & type of dance	&n&	CVV	& 255\\

mà 	& 2.pl.w 	&pro& 	CV 	& 170\\
mã̀ã́	&  mother	&n&	CVV	& 202\\

ná	& see		&v&	 CV	& 102\\
nã̀ã̀	& leg 		&n& 	CVV 	& 233 \\

\lspbottomrule
\end{Qtabular}
             \end{table}
             
While these are no conclusive experimental evidence, in Section \ref{sec:syllable-types}, it is shown that nouns in the language cannot have a CV surface form, whereas  verbs 
can. Still, many noun roots  are  of the type CV.  The lexical database contains   a few pairs of words with exactly  the same consonant and vowel quality but differing in length, i.e. {\sls ɲã̀ã̀} `lack' and {\sls ɲã́} `defecate', {\sls záŋ} `rest area' and {\sls zàáŋ̀} `today', and {\sls wàsɪ̀}	`reproduce' and {\sls wààsɪ̀} `pour libation'.	The following sections present evidence for two types of vowel-vowel sequence 
in the language.

\newpage %longdistance
\paragraph{V$_{i}$V$_{i}$ vowel sequences.}
\label{sec:V1V1vowelseq}

A V$_{i}$V$_{i}$ vowel sequence identifies a sequence of two vowels of the same 
quality without intervening consonants or vowels.  Table \ref{tab:V1V1sequence} 
provides some attested cases of V$_{i}$V$_{i}$ sequence.


\begin{table} 
 
\small
\caption[Vowel sequences V1V1]{$V_{i}V_{i}$ sequence \label{tab:V1V1sequence}}
\begin{Qtabular}{llllll}
\lsptoprule
$V_{i}V_{i}$ & Gloss &  PoS & $V_{i}V_{i}$  & Gloss &  PoS\\ 
\midrule
\multicolumn{3}{l}{ aa}  &  \multicolumn{3}{l}{  ãã} \\[0.5pt] 

váà	&	dog &	n & fã̀ã̀	&	draw milk from 	& v\\
táál	&	cloud &	n & ɲã̀ã́	&	poverty	& n\\
tàá	&	language &	n & sã̀ã́	&	axe	& n\\
bááŋ	&	temper  &	n & tʃã́ã́	&	broom	 & n\\
\midrule
\multicolumn{3}{l}{ ɪɪ}  &  \multicolumn{3}{l}{  ɪ̃ɪ̃} \\[0.5pt] 

wɪ̀ɪ̀	&	sick (be)	& v	&  fɪ̃́nɪ̃́ɪ̃́	&	harassment	
& n\\
ʔàrɪ́ɪ̀&	grasscutter	& n	&  mɪ̃́ɪ̃́	&	guinea corn	
& n\\
nɪ́ɪ́	&	water	& n	&  fɪ̃̀ɪ̃̀	&	urinate	
& v\\
bɪ́ɪ́	&	stone	& n & tʃɪ̃́ɪ̃́ŋ	&	 ankle-rattles & 	
n \\
\midrule
\multicolumn{3}{l}{ ɛɛ}  &  \multicolumn{3}{l}{ ɔɔ} \\[0.5pt] 

lɛ̀hɛ́ɛ́		& cheek	&  n & bɔ̀ɔ̀bɪ́	&	undergarment &	n \\
sɔ́mpɔ̀rɛ́ɛ̀	&	type of frog	& n &lɔ́ɔ́lɪ̀ & car & n\\
wátʃɛ̀hɛ́ɛ̀	&	type  of  mongoose &	n &bɔ́ɔ́l & type of shape & n\\
ʔálɛ́ɛ̀fʊ́		& type  of  leaf	& n &&&\\
\midrule
\multicolumn{3}{l}{ ʊʊ}  &  \multicolumn{3}{l}{  ʊ̃ʊ̃} \\[0.5pt] 

fʊ̀ʊ̀sɪ̀	&	inflate	& v &bʊ̃́ʊ̃́ŋ	&	goat	& n\\
jʊ̀ʊ́	&	rainy  season	&  n &dʊ̃́ʊ̃̀	&	African rock python	& n\\
jʊ̀ʊ̀	&	marry	& v & fʊ̃̀ʊ̃́	&	lower back& n\\
tʃʊ́ʊ́rɪ́	&	torn	& v & nʊ̃́ʊ̃́&	shea butter	& n\\
\midrule

\multicolumn{3}{l}{ ii}  &  \multicolumn{3}{l}{   ĩĩ} \\[0.5pt] 



bàmbíí	&	chest &	n &  ʔĩ́ĩ̀	&	push
& v\\
pìèsíí	&	sheep	& n &hĩ̀ĩ́	&	bad	& interj\\
píí	&	yam mound	& n &mĩ̀ĩ́ &	gun front sight	& n\\
tíísí&	grind roughly	& v &záɣắfĩ́ĩ̀	&	yellow fever
& n\\
\midrule

\multicolumn{3}{l}{ ee}  &  \multicolumn{3}{l}{  oo} \\[0.5pt] 



dèmbélèè	&	fowl house	& n & tʃòòrì	& strain &	v\\
zànzàpúrèè & type of bat & n &lòòtó	& 	intestine	& n\\
zóŋgòréè	&	mosquito	&  n &mũ̀sóóró & clove & n\\
téébùl & table (ultm. Eng.) & n &kpógúlóò & soya bean dish & n\\
\midrule 


\multicolumn{3}{l}{ uu}  &  \multicolumn{3}{l}{  ũũ} \\[0.5pt] 

bùú	&	silo	& n &sũ̀ṹ	& guinea fowl & n\\
púúrí	& reduce	& v & tṹṹ	& honey	& n\\
ɲúù	&	head	& n & ʔṹũ̄	& bury	& v\\
tùù & go down& v & dũ̀ũ̀ & sow & v\\


\lspbottomrule
\end{Qtabular}
 
\end{table}

The V$_{i}$V$_{i}$ sequences can also surface nasalized, except for the front mid vowels: only one sequence [{ẽẽ}] (i.e. {\sls 
kálɛ́ŋbɪ́lèŋẽ́ẽ̀} `adjuster')  and one  [{ɛ̃ɛ̃}]  (i.e.  
\isi{interjection} {\sls ɛ̃̀ɛ̃́ɛ̃̀} `yes') are recorded.  The vowel sequences in Table 
\ref{tab:V1V1sequence} can either be treated as cases of long vowels or as a 
sequence of two short vowels: the two underlying structures  
assumed are presented in (\ref{ex:V1V1vowel-seq}).

\begin{exe}
\ex\label{ex:V1V1vowel-seq}
\begin{xlist}

\ex  {\rm V$_{i}$]-V$_{i}$: a morpheme boundary intervenes} \\
 mɪ̃]ɪ̃ $\rightarrow$ mɪ̃́ɪ̃́  {\rm `guinea corn'}, {\sc pl.} mɪ̃́ã́  {\rm 
 ({\sc class 4}, Section \ref{sec:class4})} \\
  lɛhɛ]ɛ  $\rightarrow$  lɛ̀hɛ́ɛ́  {\rm `cheek'},  {\sc pl.}   lɛ̀hɛ̀sá  
{\rm ({\sc class 1},  Section \ref{sec:class1})}


\ex {\rm V$_{i}$V$_{i}$ : no morpheme boundary intervenes} \\
ɲúù {\rm  `head'}, {\sc pl.} ɲúúnò {\rm  ({\sc class 5}, Section 
\ref{sec:class5})}\\	
bʊ̃́ʊ̃́ŋ	 {\rm  `goat'}, {\sc pl.} bʊ̃́ʊ̃́ná  
{\rm  ({\sc class 3}, Section \ref{sec:class3})}
\end{xlist}
\end{exe}





\newpage
\paragraph{V$_{i}$V$_{j}$ vowel sequences.}
\label{sec:V1V2vowel-seq}

A V$_{i}$V$_{j}$ vowel sequence identifies a sequence of two vowels of different 
quality without intervening consonants or vowels.
Most of the sequences in the 
data involve  the set of high vowels \{{i, u, ɪ, ʊ}\}  as first 
vowel.\footnote{\label{fn:deg-labial} An alternative would be to treat them as 
the set of glide consonants \{{j, w}\}. As a matter of fact, the notion of  `suspect 
sequences' 
was coined by GILLBT/GIL fieldworkers when faced with  transcription 
  involving the segments  \{i, u, ɪ, ʊ\} (\citealt[4]{Gray69}, 
\citealt[8]{Toup95}, among others). ```Suspect' is an old SIL heuristic term for phonetic sounds 
which may have different phonemic function in different languages'' (T. Naden, p.c.).
Some  tokens of V$_{i}$V$_{j}$ vowel 
sequences  would then be treated as suspect sequences under their analyses. For 
instance, {\sls bie} `child', a monosyllabic word,  would be represented as 
{\sls bije}, a disyllabic word \citep[see also][100]{Kedr97}. Correspondingly,  `arrow' could be 
transcribed as {\sls tuo}, {\sls tʷo} or {\sls tuwo}.  My decision is purely 
based 
on the impression of  consultants who do not favour a syllable break.  Further, 
unlike  \ili{Dɛg}, Chakali consonants do not have corresponding  labialized  phonemes. 
In \citet[2]{Crou03},  13 of the 22 phonemes have a labialized 
counterpart. I also perceive the labialized consonants of \ili{Dɛg} (see footnote 
\ref{fn:info-deg}).}

\begin{table} 
 \small
\caption[]{$V_{i}V_{j}$ sequence \label{tab:V1V2sequence}}
\begin{Qtabular}{llllll}
\lsptoprule
$V_{i}V_{j}$ & Gloss &  PoS &$V_{i}V_{j}$    & Gloss &  PoS\\ 
\midrule
\multicolumn{3}{l}{ ʊɪ}  &  \multicolumn{3}{l}{  ui} \\[0.5pt] 

bʊ́ɪ̀ & stone & n & múfúí  & exclamation & ideo\\
pʊ́ɪ̄ &   spitting  & n  &   súī  & being full &    n  \\ 

\midrule
\multicolumn{3}{l}{ ʊɔ}  &  \multicolumn{3}{l}{  uo} \\[0.5pt] 

sʊ̀ɔ̀rá	&	odor	         & n  & bùól	&	song	& n\\ 
lʊ̀ɔ́ŋ	&	animal chest hair & n &   túò& bow   & n      \\ 

\midrule
\multicolumn{3}{l}{ʊa}  &  \multicolumn{3}{l}{  } \\[0.5pt] 

tʃʊ̀à & lie & v &&& \\ 
dʊ̀à & be in/at/on & v&&& \\ 

\midrule
\multicolumn{3}{l}{ ɪɛ}  &  \multicolumn{3}{l}{  ie} \\[0.5pt] 

sɪ̀ɛ̀	& poor quality meat       & n & bìé 	&	child & n \\
kɪ̀ɛ̀ &    collect contribution  & v & fíél	&	type  of  grass	& n\\

\midrule
\multicolumn{3}{l}{ ɪʊ}  &  \multicolumn{3}{l}{  iu} \\[0.5pt] 

wɪ́lɪ́ʊ́ & kob & n  & kásìù & cashew (ultm. Eng.) & n\\

\midrule
\multicolumn{3}{l}{ ɪa}  &  \multicolumn{3}{l}{  io} \\[0.5pt] 

dɪ̀á	&	house	& n & fíó & totally not&  interj\\
tɪ́ásɪ́	&	vomit	& v &&&\\

\midrule
\multicolumn{3}{l}{ ɛʊ}  &  \multicolumn{3}{l}{  eu} \\[0.5pt] 

lɛ́ʊ́rá  & door hinge &	n & pèú	&	wind	&	n \\
sɛ̀ʊ́   & death &n &  tèú	&	warthog 	&	n \\

\midrule
\multicolumn{3}{l}{ ɛɪ}  &  \multicolumn{3}{l}{ eo} \\[0.5pt] 

lɛ̀ɪ́	&	not& neg & màtʃéó  & twenty & num\\
bìvɪ́ɛ́ɪ̀  & stubborn child & n & bàléò & calamity& n\\

\midrule
\multicolumn{3}{l}{ ɔɪ}  &  \multicolumn{3}{l}{  oi} \\[0.5pt] 

pɔ́ɪ̄	&	planting	& n &   ʔóí & surprise & interj   \\
tɔ́ɪ́	&	covering	& n \\

\midrule
\multicolumn{3}{l}{ ɔʊ}  &  \multicolumn{3}{l}{  ou} \\[0.5pt] 

lɔ́ʊ̀	&	hartebeest	& n &  tóù & 	o.k.	(ultm. \ili{Hausa}) & interj\\
tɔ́ʊ̀	&	settlement	& n &  wóù & yam harvest & n \\

\midrule
\multicolumn{3}{l}{ aʊ}  &  \multicolumn{3}{l}{  aɪ} \\[0.5pt] 
láʊ́  &	hut	 & n  & ʔàɪ́	&	no	 & interj\\
tʃàʊ́ & type of termite & n  &ɲã́ɪ̃̀& rusty&n \\


\lspbottomrule
\end{Qtabular}
 
\end{table}

Similar to the  V$_{i}$V$_{i}$ vowel sequences,  the V$_{i}$V$_{j}$ sequences in 
Table \ref{tab:V1V2sequence} may be the result of  two underlying structures; 
one with a morpheme boundary intervening and the other without such a boundary.  
They are shown in (\ref{ex:V1V2vowel-seq}). It includes both underlying 
structures, and among them, examples of words formed with the nominaliser suffix 
{\sc -[+hi, --ro]}, e.g. {\sls tɔ́} {\it v.} `cover' $\rightarrow$ {\sls tɔ́ɪ́} 
{\it n.}  `covering', and the verbal assertive suffix  {\sc -[+hi, +ro]}, e.g. 
{\sls jélé}  {\it v.} `bloom' $\rightarrow$ {\sls jéléó}  {\it v.} 
`bloom.{\sc  pfv.foc}' (Sections \ref{sec:GRM-verb-act-stem} and \ref{sec:GRM-focus}).  These 
two productive morphological mechanisms are responsible for the prevalence of 
V$_{i}$V$_{j}$ sequences, of which V$_{j}$ is a high front vowel or a high 
rounded one. Their surface forms depend on phonotactics, which is the topic of 
Section \ref{sec:phonotac}.

\begin{exe}
\ex\label{ex:V1V2vowel-seq}
\begin{xlist}

\ex   {\rm  V$_{i}$]-V$_{j}$ : a morpheme boundary intervenes}\\
tɔ]ɪ   $\rightarrow$ tɔ́ɪ́   {\rm `covering'} {\rm (see {\sc class 4},  
Section \ref{sec:class4})}\\
jele]u $\rightarrow$  jéléó  {\rm `bloom.{\sc pfv.foc}}', 
{\rm (see Section  \ref{sec:GRM-verb-word})} \\  
bi]e	 $\rightarrow$ bìé     {\rm `child'},   bìsé  {\sc pl.},     
 {\rm  (see {\sc class 1},  
Section \ref{sec:class1})}	

\ex  {\rm  V$_{i}$V$_{j}$ : no morpheme boundary intervenes}\\
 dʊ̀à]    {\rm `be in/at/on'}\\
tʃàʊ́]    {\rm  `type of termite'}

\end{xlist}
\end{exe}

The V$_{i}$V$_{j}$ vowel sequences are  summarized in Figure 
\ref{fig:Phon-vowel-transit}.  Each vowel diagram displays possible 
vowel-to-vowel transitions. For the first two diagrams, i.e. (a) and (b), the 
transitions are arranged according to the first vowel on the basis of their {\sc 
atr} value. The third diagram displays the transitions in which the vowel /a/  is the first vowel. 

\begin{figure}


\subfigure[Transition from {\sc --atr} vowels]{
\begin{tabular}{ccc}

 \begin{vowel}[plain]
%\putcvowel{i}{1}
%\putcvowel{e}{2}
\putcvowel{ɛ}{3}

\putcvowel{ɔ}{6}
%\putcvowel{o}{7}
%\putcvowel{u}{8}

\putcvowel{ɪ}{13}
\putcvowel{ʊ}{14}
\putcvowel{a}{4}

\end{vowel}
\psset{arrowsize=.75ex, nodesep=.25ex}
%\ncline{->}{v4}{v8}
\ncline{->}{v3}{v4}
%\ncline{->}{v4}{v14}
%\ncline{->}{v4}{v13}
%\ncline{->}{v2}{v8}
%\ncline{->}{v2}{v7}
\ncline{->}{v3}{v13}
\ncline{->}{v3}{v14}
%\ncline{->}{v1}{v4}
%\ncline{->}{v1}{v8}
%\ncline{->}{v1}{v2}
\ncline{->}{v13}{v4}
\ncline{->}{v13}{v3}
\nccurve[angleA=40,angleB=20,ncurv=0.5]{->}{v13}{v14}
\ncline{->}{v14}{v13}
\ncline{->}{v6}{v13}
\ncline{->}{v6}{v14}
%\ncline{->}{v8}{v4}
%\ncline{->}{v8}{v7}
%\ncline{->}{v8}{v1}
\ncline{->}{v14}{v4}
\ncline{->}{v14}{v6}

\end{tabular}
}
\hspace{74pt}
\subfigure[Transition from {\sc +atr} vowels]{
\begin{tabular}{ccc}

 \begin{vowel}[plain]
\putcvowel{i}{1}
\putcvowel{e}{2}
%\putcvowel{ɛ}{3}

%\putcvowel{ɔ}{6}
\putcvowel{o}{7}
\putcvowel{u}{8}

%\putcvowel{ɪ}{13}
%\putcvowel{ʊ}{14}
%\putcvowel{a}{4}



\end{vowel}
\psset{arrowsize=.75ex, nodesep=.25ex}
%\ncline{->}{v4}{v8}
%\ncline{->}{v4}{v1}
%\ncline{->}{v4}{v14}
%\ncline{->}{v2}{v1}
\ncline{->}{v2}{v8}
\ncline{->}{v2}{v7}
%\ncline{->}{v3}{v13}
%\ncline{->}{v3}{v14}
%\ncarc{->}{v1}{v4}
\nccurve[angleA=20,angleB=20,ncurv=0.5]{->}{v1}{v8}
\ncline{->}{v1}{v2}
%\ncline{->}{v13}{v4}
%\ncline{->}{v13}{v3}
%\ncline{->}{v13}{v14}
%\ncline{->}{v6}{v13}
%\ncline{->}{v6}{v14}
%\ncline{->}{v8}{v4}
\ncline{->}{v8}{v7}
\ncline{->}{v7}{v8}
\ncline{->}{v7}{v1}
\ncline{->}{v1}{v7}
\nccurve[angleA=40,angleB=40,ncurv=0.8]{->}{v8}{v1}
%\ncline{->}{v14}{v4}
%\ncline{->}{v14}{v6}
\end{tabular}
}
\quad
\subfigure[Transition from /a/]{
\begin{tabular}{ccc}

 \begin{vowel}[plain]
%\putcvowel{i}{1}
%\putcvowel{e}{2}
%\putcvowel{ɛ}{3}

\putcvowel{ɔ}{6}
%\putcvowel{o}{7}
%\putcvowel{u}{8}

\putcvowel{ɪ}{13}
\putcvowel{ʊ}{14}
\putcvowel{a}{4}



\end{vowel}
\psset{arrowsize=.75ex, nodesep=.25ex}
%\ncline{->}{v4}{v8}
%\ncline{->}{v4}{v1}
\ncline{->}{v4}{v6}
\ncline{->}{v4}{v14}
\ncline{->}{v4}{v13}
%\ncline{->}{v2}{v8}
%\ncline{->}{v2}{v7}
%\ncline{->}{v3}{v13}
%\ncline{->}{v3}{v14}
%\ncline{->}{v1}{v4}
%\ncline{->}{v1}{v8}
%\ncline{->}{v1}{v2}

%\ncline{->}{v13}{v3}
%\ncline{->}{v13}{v14}
%\ncline{->}{v6}{v13}
%\ncline{->}{v6}{v14}
%\ncline{->}{v8}{v4}
%\ncline{->}{v8}{v7}
%\ncline{->}{v8}{v1}
%\ncline{->}{v14}{v4}
%\ncline{->}{v14}{v6}
\end{tabular}
}

\caption{Attested vowel transitions \label{fig:Phon-vowel-transit}}
\end{figure}

%Non-attested  transitions are {ao, ae, aɔ, aɛ,   

The direction of the arrow reproduces the transitions. A step in the analysis of 
vowel sequences would be to identify them as either unit diphthongs or two 
independent vowels. On the one hand there are relatively few languages with unit 
diphthongs  \citep[133]{Madd84}, and on the other hand it is necessary to 
understand better syllable structures, phonotactics, and the effect of 
coarticulation when vowel features are suffixed to vowel-ending stems in 
Chakali.  In theory, true restrictions  are due to obligatory harmonies, 
specifically with regard to the {\sc atr} and {\sc ro} features:  more sequences 
should be attestable than those presented in Figure 
\ref{fig:Phon-vowel-transit}.  The most common sequences are \{ʊa, ʊɔ, ɪɛ, ɪa, ɔɪ, uo, ie, eu, aʊ\}, the remaining ones being very rare or unattested.  For 
instance, 
the [{ei}] and   [{aɛ}] sequences never occur,  the [{ɛa}] sequence 
occurs 
only once  (and {\sls  ʔàtànɛ́à} `Monday' is ultimately of \ili{Hausa} origin),  
and 
 the sequence [aɔ], which occurs in  {\sls máɲã̀ɔ̃̀} `type of mongoose', is  found twice. 
In the latter case, both tokens are nasalized so it  affects the vowel quality  and how I perceived it. 

\newpage 
\subsection{Consonants}
\label{sec:conso}  


The consonantal phonemes amount to twenty-five, a number close to the average number of consonants in the consonant inventories of languages catalogued in  \citet{Madd09}. In this section, the phonemic status of the consonants is
identified using distributional criteria. When possible the segments are aligned
in three word positions:  initial, medial,  and final. Although it is crucial to identify a stem boundary in a word in order to differentiate between the onset of a non-initial stem (e.g. in a compound word) and the medial position of a monomorphemic word,  this is often not possible given our knowledge of the language. The feature {\sc voice} represents voicing (i.e. voiced vs. voiceless) and is reflected in the way  the description is organized below.
Table \ref{tab:ConsChart-1} provides an overview of the segments introduced in this section.


\begin{table} 
 \caption{Phonetic and phonemic consonants in Chakali}
 \small
\label{tab:ConsChart-1} 
 \fittable{
  \begin{tabular}{llp{1cm}lllllp{1cm}}
\lsptoprule
    & Bilabial & Labial-dental & Alveolar & Postalv. & Palatal & Velar & Glottal
 & Labial-velar\\ 
 \midrule 
Plosives &     p      b  & &     t      d  & & &     k      g  &    ʔ  & 
   kp     gb  \\ 
Fricatives &&    f     v  &    s     z  & && \ \ \   (ɣ) &     h  &\\

Affricates &&& &   tʃ     dʒ  &&&& \\ 
Nasals&    m  &&    n  & &    ɲ  &    ŋ &  &    ŋm \\ 
Liquid &&&    l     r  &&&& &\\ 
Semi-vowels &    &&&&   j &&& w (ɥ) \\ \lspbottomrule
  \end{tabular}
  }
% \end{small}

\end{table}


   
\subsubsection{Plosives and affricates}

All plosives and affricates contrast pairwise for the glottal stricture feature 
{\sc voice} (except  the glottal plosive /{ʔ}/). They are moderately 
aspirated word-initially. They all involve a single primary place of articulation, except  the 
doubly articulated [{d͡ʒ}], [{t͡ʃ}], [{k͡p}] and [{g͡b}].  
The affricates  [{d͡ʒ}]  and [{t͡ʃ}] have two sequential parts, while 
labiovelars  [{k͡p}] and [{g͡b}] have two parts which overlap 
temporally.\footnote{For the remainder,  the linking diacritic over the 
labial-velars is not used,  since there are just a few ambiguous contexts and 
these are accounted for by the syllabification procedures presented in  Section \ref{sec:syllable-types}.}

% The voicing distinction is clearly visible on the voice bar, consequence of
%the vocal folds' abduction/adduction.

\paragraph{Bilabial plosives.}

The bilabial plosives can occur in word-initial and -medial positions, although, 
in many  cases, when they are found in  word-medial positions, they are
 onsets of a non-initial stem. This position can be problematic, since one cannot 
always treat words as compounds in the synchronic sense. For instance, {\sls 
álʊ̀pɛ̀} `seven' is treated in Section \ref{sec:NUM-bas-comp} as   
monomorphemic, however,  it is obvious that taken from a Proto-SWG perspective 
it is not. Bilabial plosives  can also be  found in borrowed words' medial 
positions, e.g. {\sls kàpɛ̀ntà} (ultm. Eng.) `carpenter' and {\sls kàpálà} 
(\ili{Waali}) `type of staple food'.  Neither the 
voiceless nor the voiced bilabial plosive are attested word-finally. Table 
\ref{tab:bilabial-plosives} provides examples of contrast between /{p}/ 
and /{b}/ for the {\sc voice} opposition.


\begin{table} 
\caption{Bilabial plosives \label{tab:bilabial-plosives}}

\subfigure[Voiceless bilabial plosive]{
\begin{Qtabular}{lll}

páŋ	&	molar &  	n \\
pɛ̀rà	&	weave	 & v  \\
pílè	&	cover  with	 & v  \\
púl̀	&	type  of  river  grass	 & n\\
kúmpíí	&	thorny spear grass &	n\\
àlʊ̀pɛ̀	  & seven & 	num\\
kàpɛ̀ntà	&	carpenter  (ultm. Eng.)	&  n\\
kàpálà & staple food,  Gh. Eng. {\it fufu}  & n\\

\end{Qtabular} 
}
\quad
\subfigure[Voiced bilabial plosive]{
\begin{Qtabular}{lll}
 
bàŋ	&	here	 & adv  \\
bɛ̀rà	&	dry	 & qual  \\
bìlè	&	put	 & v  \\
bùĺ	&	type  of  tree &	n  \\
ʔàbɛ́ &  palm tree (\ili{Akan}) & n\\
 fɪ̀ɛ̀bɪ̀ &  whip & v\\
hámbák	&	type of tree	&		n\\
\end{Qtabular} 
}

\end{table}

\newpage 
\paragraph{Alveolar plosives.}
\label{sec:PHO-alveo-plos}  

The alveolar plosives can occur in word-initial and  -medial positions. Similar
to the bilabial plosives, the voiceless and the voiced alveolar plosives  are
not attested word-finally.\footnote{On one of the field trips, I was given a dog
and  called it [{\sls táát}]. People in \isi{Ducie} would repeat its name and call 
the 
dog
[{\sls táátə̀}]. The way they pronounced the name suggests that alveolar 
plosives are disallowed in word-final position. \label{fn:taat-epenthesis}}  
When it occurs in word-medial position,  [{d}] is found only at the onset of a non-initial stem of polymorphemic words or in loans, whereas [{t}] does 
not 
have such a restriction. Examples of such loans are  {\sls síídì} `cedi', 
{\sls kùòdú} `banana', and {\sls bɔ̀rdɪ́á} `plaintain'  for words of \ili{Akan} 
origin, and  {\sls gáádìn} `garden', {\sls bìléédì} `blade',  and 
{\sls pʊ́ɔ́dà} `powder' for words of English origin. An example of occurences 
in onset of non-initial stem of polymorphemic words is {\sls fi-dɪ-anaasɛ} 
[{\sls fídànáásɛ̀}] `fourteen' (Section \ref{sec:NUM-bas-comp}),   
{\sls ɲɪ́n-dáá}	`horn', and {\sls nɪ̀-dʊ̀má}  `spirit'.  Examples 
{\sls kàndɪ́à} `Kandia'  and {\sls kódì} `or' appear to be lexicalized 
polymorphemic words or loans. The rhotic [{r}] may be argued to be an 
allophone of /{d}/ as   [{r}] occurs mostly where [{d}] is 
never 
found, e.g. intervocalically in monomorphemic words (Section \ref{sec:flap}). 
Table \ref{tab:alveolar-plosives} provides examples of contrast between the two 
alveolar plosives for the {\sc voice} opposition in word-initial and  -medial 
positions.


 
\begin{table} 
\caption{Alveolar plosives \label{tab:alveolar-plosives}}

\subfigure[Voiceless alveolar plosive]{
\begin{tabular}{lp{2cm}l}
%bàtɔ́ŋ	&	skin	&	n  \\
té	&	early	&	adv  \\
tíŋ	&	spearhead	&	n  \\
tɔ́ŋ	&	book	&	n  	\\ 
túò	&	bow	&	n  \\
tʊ́má & work & n\\
kàɲɪ̀tɪ̀	&	patience\newline (\ili{Hausa})		&	n\\
kètì		&	break		&	v\\
sɔ̀tá		&	thorn			&	n	\\
\end{tabular} 
}
\quad
\subfigure[Voiced alveolar plosive]{
\begin{tabular}{lp{1.5cm}l}
%badɔŋ	&	male  fellow	&	n  \\
dé	&	there	&	adv  \\
díŋ	&	fire	&	n  \\
dɔ́ŋ	&	enemy	&	n  	\\	
dùò	&	sleep	&	v  \\
dʊ̀má & soul & n\\
síídì   &  cedi\newline (\ili{Akan}) & n\\
lɛ̀-dáá		&	 lower~jaw	&	n\\
kàndɪ́à & Kandia  &	propn\\
\end{tabular} 
}
\end{table}

The segment [{r}]  can surface when [{t}]  is expected.  For instance, 
the 
\isi{plural} form of the word {\sls gèŕ} `lizard' is {\sls gété} `lizards' and 
the 
\isi{plural} form of the word {\sls sɔ̀tá} `thorn'  is {\sls sɔ̀ràsá}. The 
underlying 
segmental representation /{get}/ may be given for the lexeme `lizard'. 




\newpage 
Rule \ref{PHO-rule-t-r} is postulated, which turns a /{t}/ into  [{r}] 
in word-final position and in weak syllables (see Section 
\ref{sec:PHO-weak-syll}).\footnote{Since the voiced  alveolar plosive never 
occurs in word-medial position, there may be another rule involved which devoice 
the /{d}/ in   {\sls gété} `lizards'. In fact, by omitting [{\sc 
-voiced}],  Rule \ref{PHO-rule-t-r} captures /{d}/  as well. Notice that  Rule  \ref{PHO-rule-t-r} undergenerates in some instances, e.g. {\sls bùtér} `turtle', {\sls bùtété} `turtles'  *{\sls burete}.}


\begin{Rule}\label{PHO-rule-t-r}{Lenition}\\
An  alveolar stop changes into a  trill in word-final position or in word-medial onset.\\
 {\sc [alveolar, obstruent]} $\rightarrow$  r   /   \_ \#  or  CV.\_ V.CV
\end{Rule}





Rule \ref{PHO-rule-t-r} operates  only on a few nouns, probably due to the fact 
that an underlying coda /t/ is rare.  Further,  all the examples involve {\sc 
[+atr, --ro]} vowels,  e.g. {\sls bùtérː} - {\sls bùtété} `turtle(s)' and 
{\sls tʃìíŕ} - {\sls tʃìíté} `taboo(s)'.  Examples of minimal pairs 
involving 
a [r]-[t] contrast are  {\sls pàrà}  `farm' -  {\sls pátá}  `trousers', 
{\sls lúró} `scrotum' - {\sls lùtó} `root', and {\sls tʃárɪ̀} 
`diarrhoea' - {\sls tʃátɪ̀} `type of guinea corn'.


\paragraph{Velar plosives.}
\label{sec:PHO-vel-plos}  

The velar plosives are found in word-initial and -medial positions. In addition, 
among the plosives,  the velar plosive is the only one which is allowed word 
finally. This is shown is Tables \ref{tab:velar-plosives}(a)  and 
\ref{tab:velar-plosives}(b). 



\begin{table} 

\caption{Velar plosives and fricative \label{tab:velar-plosives}}

\subfigure[Voiceless velar plosive]{
\begin{Qtabular}{lll}
 

%bɔ̀k	& 	type  of  tree	& n\\
kààsɪ̀	&	clear  throat	& v  \\
kɔ́ŋ	&	cobra	& n  \\
kʊ̀tɪ̀	&	fine  grinding	& v\\

hákɪ̀lá	& 	cognition	&  n\\
kàkà		& toothache	& n\\
%nòkúǹ	&  type of tree 		& n\\
tùḱ   & type of nest & n\\
pààtʃák	& leaf	& n\\



\end{Qtabular} 
}
\quad
\subfigure[Voiced velar plosive]{
\begin{Qtabular}{lll}
 

%́bɔ̀g	&	fiber	& n  \\
gáásɪ́	&	pass	& v  \\
gɔ́ŋ	&	type  of  plant	& n  \\
gʊ̄tɪ́	&	roll	& v  \\

%bélégè	&	bathing area	&	n\\
%hóɣúl  &  cockroach & n\\
bégíí	& 	heart &	n\\
kùgsó		& rib cage	& n\\

hóg & bone & n\\
vʊ́g & small god & n\\

\end{Qtabular} 
}

\subfigure[Velar fricative]{
\begin{Qtabular}{llll}
 /kpaga/ & [kpàɣà] &	have & v\\
  /dɔga/  & [d͏ɔ̀ɣà]	& Doga&  propn\\
/tʃaktʃak/ &	[tʃáɣətʃák] &		tattoo	& ono\\	
/tig-si/	& [tíɣĭ́sī]& gather & v\\
/hogul/	& [hóɣúl] &  cockroach & n\\
%/patʃɪgɪɪ/-/wɪɪla/ & [pàtʃɪ́gwɪ̀ɪ̀là]  & 	stomachache&  n\\
% /vʊga/-/tɪɪna/ & [vʊ́ɣtɪ́ɪ́ná]	& priest &  n\\
\end{Qtabular} 
}
\end{table}








Further the segment [ɣ], which appears 
between vowels in a weak syllable (see Section \ref{sec:PHO-weak-syll}),  is 
underlyingly a /k/ or a  /g/.\footnote{For simplicity, I use [g] throughout instead of the IPA symbol for the voiced velar plosive  [ɡ].} Since the notion of weak syllable 
has 
not been justified, Rule \ref{PHO-rule-k-g} partially accounts for the 
spirantization  of velar plosives.


\begin{Rule}\label{PHO-rule-k-g}{Spirantization}\\
The velar obstruents  /k/ and /g/  change into  [ɣ]  
when they occur
between vowels in a weak syllable.\\
{\sc [velar, obstruent]}  $\rightarrow$  ɣ  /  V. \_ V or  \_ . C
\end{Rule}


\largerpage[2]
As shown in Table \ref{tab:velar-plosives}(c), the segment [ɣ] appears 
in 
word-medial position, but never in word-initial or  -final position.  A  voicing 
distinction between [ɣ] and a potential voiceless velar fricative [x] is not perceived, which,  if identified, would create two corresponding 
pairs with /g/ and /k/ respectively. However, it seems that 
/g/ and /k/ are spirantised medially except when adjacent to a [{\sc 
+atr}, {\sc +hi}, {\sc --ro}] vowel. Nevertheless a few counterexamples, such as 
{\sls kpégíí} `hard' and {\sls sígìì} `misery',  must be taken into 
account.\footnote{In \ili{Mòoré} and \ili{Koromfe} /{g}/ is spirantised medially 
except when adjacent to a [{\sc +atr}, {\sc +hi}] vowel  (John Rennison, p.c.). 
Chakali  {\sls hóɣúl} `cockroach' and {\sls nàŋjóɣúl} `butcher'  are clear 
spirantization cases.}






\paragraph{Glottal plosive.}


The glottal plosive, or ``glottal stop'', occurs only at the beginning of vowel-initial word stems. 
Word-initially it is optional, but it is obligatory at the beginning of a 
vowel-initial stem contained within polymorphemic words such as  {\sls 
nɔ́ʔɔ́rɔ́ŋ} `type  of tree' and {\sls fáláʔúl}  `calabash node'. Table 
\ref{tab:glottal-plosive} provides examples of word-initial and (stem-initial) 
word-medial positions.

 \begin{table} 

\caption{Glottal plosive \label{tab:glottal-plosive}}
\begin{Qtabular}{lll} 
ʔàbɛ́		&	palm  tree (\ili{Akan}) &	n  \\
ʔã́ã́		&	bushbuck	&	n  \\
ʔɪ́l		&	breast		&	n  \\
ʔìlèʔìlè	&	type of colour	&	ideo  \\
bàʔɔ̀rɪ́ɪ̀		&	swelling	&	n  \\
nɔ́ʔɔ́rɔ́ŋ		&	type  of  tree	&	n  \\
\end{Qtabular} 
\vspace*{2ex}


\end{table}

\newpage
\paragraph{Labial-velar plosives.}

Among the twenty-five consonants,  five are complex segments. These include the 
plosives  /{kp}/ and /{gb}/. The term ``complex'' in this context means 
that two primary places of articulation are involved in the production of the sounds, that is, the  velum and the lips. Nonetheless, they behave as single phonemes. The labial-velar plosives  can occur in initial and medial positions, but as the bilabial plosives, when they are found in  a word-medial position, the position  is typically  the onset of a non-initial stem.  Table \ref{tab:labial-velar-plosives} gives examples of  labial-velar plosives in word-initial positions and shows that they contrast with both the labial and the velar plosives.


\begin{table} 

\caption{Labial-velar  plosives \label{tab:labial-velar-plosives}}
\subfigure[Voiceless labial-velar plosive]{
\begin{Qtabular}{lp{2.2cm}l}
  
kpà & take &v \\
kpáá	&	type~of  dance	& n  \\
kpòŋ́	&	location	& propn  \\

\end{Qtabular} 
}
\quad
\subfigure[Voiced labial-velar  plosive]{
\begin{Qtabular}{lp{2.5cm}l}
 
    gbà & also & quant\\
gbáà	&	control  animal	& v  \\
gbóŋ	&	type~of  tree	& n  \\

\end{Qtabular} 
}

\subfigure[Contrast with /k/ and  /p/]{
\begin{Qtabular}{lll}


kpòŋ́	&	location	& propn\\
kóŋ	&	Kapok	& n\\
kpɪ́sɪ́	&	sneeze	& v\\
pɪ̀sɪ̀	&	scatter	& n\\
kpʊ̀& kill & v\\
pʊ́ & spit & v\\
 
\end{Qtabular} 
}
\subfigure[Contrast with  /g/ and /b/]{
\begin{Qtabular}{lll}
gbár	&	watcher	&	n\\
gár	&	stable	&	n\\
gbɛ́nɪ́ɪ́	&	pink	&	qual\\
gɛ́nɪ́ɪ́	&	fool	&	n\\

gbʊ̀ŋà & dense &v\\
bʊ̀ŋà & bend & v\\
\end{Qtabular} 
}
\end{table}

 
\paragraph{Affricates.}

The affricates /{tʃ}/ and  /{dʒ}/ are  treated as single phonemes. 
They can occur in word-initial and word-medial positions, although the voiced 
affricate is comparatively less used. Notice that while  /{kp}/ and /{gb}/ do contrast with /{p}/, /{b}/,  /{k}/,  and /{g}/,   
/{ʃ}/ and  /{ʒ}/ do not exist in the language (except for the \isi{interjection} 
{\sls ʃɪ̃́ã̀ã̀}  `insult').  Table \ref{tab:affricates} provides (near-)minimal  
pairs, when available. 


\begin{table} 

\caption{Affricates\label{tab:affricates}}

\subfigure[Voiceless affricate]{
\begin{Qtabular}{lll}

tʃʊ̀ɔ̀ŋ & type of fish & n\\
tʃáásá   &  	comb & n\\
tʃã̀ã̀nɪ̀	&	shine & v\\
kátʃál & type of tree & n\\
pààtʃák & leaf & n\\

\end{Qtabular} 
}
\quad
\subfigure[Voiced affricate]{
\begin{Qtabular}{lll}
dʒʊ̀ɔ́ŋ & hammock & n\\
dʒàá		&	unexpectedly & adv\\
dʒáŋã́ã́	&	bearing tray & n\\
təráádʒà & trousers (ultm. Eng.)  & n\\
bádʒɔ̀gʊ́ & type of  lizard & n\\
\end{Qtabular} 
}
\end{table}

Also, the sound [{tʃ}] is pronounced [{k}]  by some members of the 
oldest generation, e.g. {\sls tʃìíŕ} $\sim$ {\sls kìíŕ} `taboo',   {\sls 
tʃímmã̀ã́} $\sim$ {\sls kímmã̀ã́} `pepper',  {\sls tʃɪ́ɛ́ŋɛ̃́} $\sim$  {\sls kɪ́ɛ́ŋɛ̃́} `break', etc.  This could be evidence that, in the  recent past, the affricates originated as  stops in an environment conditioned by a high front vowel.  However, examples of minimal pairs [{tʃ}]-[{k}] exist:  {\sls tʃògò}  `ignite' vs. {\sls kògò} `hold', {\sls tʃʊ́l̀} `clay' vs.  {\sls kʊ́l} `type of 
staple food',  {\sls tʃàɣà} `to face'  vs.  {\sls kàɣà} `to choke',  among 
others.\footnote{It could be that the lexemes involved in these minimal pairs 
underwent semantic change and phonological change, but originated from a single 
source.  \ili{Vagla} data suggest that a conditioning of front vowel is 
not unique to Chakali (see footnote \ref{ft:GRM-naden-donate}). Looking at the form/meaning of cognates in other related languages would be revealing.}

\subsubsection{Fricatives}
\label{sec:fricative}

The four fricatives /{f}/, /{v}/, /{s}/, and /{z}/  are distinguished by their place of articulation and by their voicing.

\paragraph{Labio-dental fricatives.}

In general, the segments /{f}/ and /{v}/ have the same distribution: 
they can occur in  word-initial and -medial positions, but never in  a final 
position, and they both can precede any vowel. They  contrast exclusively on 
the feature {\sc voice}. This is shown in Table  
\ref{tab:labio-dental-fricatives}.  Contrasts 
with alveolar fricatives are given  in  Table 
\ref{tab:alveolar-fricatives} of Section \ref{sec:alv-frica}.

\begin{table} 

\caption{Labio-dental fricatives\label{tab:labio-dental-fricatives}}

\begin{Qtabular}{lll}
fàà	&	ancient  time	& n\\
váà	&	dog	& n\\
fã̀ã̀	&	do by force	& v\\
vã̀ã̀	&	be beyond	& v \\
fáárɪ́ &	be between &		v\\
vààrɪ̀ &	do abruptly&	v\\
\end{Qtabular}

\end{table}


\paragraph{Alveolar fricatives.}
\label{sec:alv-frica}
The alveolar fricatives /{s}/ and  /{z}/ can occur in word-initial 
and 
-medial positions, but never word-finally. The glottal stricture is the only
property which differentiates the alveolar and labio-dental fricatives.
Overall, 
the voiceless alveolar fricative is more frequent than the voiced one. In
word-medial positions,  the voiceless alveolar fricative acts mainly as the
onset
of a non-initial stem. Table
\ref{tab:alveolar-fricatives}(a) presents the alveolar fricatives in 
opposition for the feature {\sc voice},  and Table 
\ref{tab:alveolar-fricatives}(b)  presents  the alveolar fricatives contrasting 
with the 
labio-dental fricatives in  word-initial positions.

%\footnote{Note that Chakali s is often z in \ili{Vagla} }

\begin{table} 

\caption{Alveolar  fricatives\label{tab:alveolar-fricatives}}

\subfigure[Alveolar  fricatives]{
\begin{Qtabular}{lp{2.4cm}l}
sɪ̀ɛ́	&	imitating	& n\\
zɪ̀ɛ́	&	wall	& n\\
sɔ́ŋ	&	name	& n\\
zɔ̀ŋ́	&	weakling	& n\\
sʊ́ʊ́	&	front	& n\\
zʊ̀ʊ̀	&	enter	& v\\


pɪ̀sá & grass mat & n\\
kʊ́zàà & basket  &n \\ 
tʃàsɪ́ɛ̀ &cough disease  & n \\ 
zɪ́ɛ́zɪ́ɛ́ &  light~weight & ideo \\  



\end{Qtabular} 
}
\quad
\subfigure[Contrast with /f/ and /v/]{
\begin{Qtabular}{lp{2cm}l}
  sã̀ã́	&	axe	& n\\
fã̀ã̀	&	do by force	& v\\
zɪ̀ɛ́	&	wall	& n\\
vɪ̀ɛ̀	& refuse & v\\
sìì	& bambara~bean &  n\\
víí	& cooking  pot & n\\

\end{Qtabular} 
}


\end{table}


\subsubsection{Nasals}
\label{sec:PHON-nasal}

There are  five distinct nasal consonants in the language: a bilabial, an 
alveolar, a 
palatal, a velar, and a labial-velar. Phonological processes involving the 
nasal feature are frequent in the language. One is discussed in Section 
\ref{sec:nasalisation-verb-suffix}.  In word-initial position,  only  [{
ŋ}] 
is not attested. The distribution of nasals in word-final position is as 
follows:  rare cases with the bilabial [{m}], a few words with the 
alveolar [{n}], and the large majority with the velar [{ŋ}]. Chakali appears to have one velarization alternation, as stated in Rule \ref{Gen-nas-vel}.


\begin{Rule}\label{Gen-nas-vel}{Velarization}\\
Nasals surface as [{ŋ}] word-finally.\\
$[${\sc +nasal}$]$  $\rightarrow$   {ŋ}  / \_ \#
\end{Rule}

% Given this pattern of data, Rule  
% \ref{PHO-nas-vel} suggests that the source of a final 
% velar nasal could ever be an alveolar nasal
% 
% 
% \begin{Rule}\label{PHO-nas-vel}{Velarization}\\
% Bilabial nasal surface as {\sls ŋ} word finally.\\
% $[${\sls m}$]$  $\rightarrow$   {\sls ŋ}  / \_ \#
% \end{Rule}



\paragraph{Bilabial nasal.}
\label{sec:PHON-bil-nas}

The bilabial nasal /{m}/ occurs in word-initial and -medial positions. 
This is shown in Table \ref{tab:bilabial-nasal}. It is rarely  found in  word-final 
positions: the \isi{onomatopoeia} {\sls ʔángùm} `monkey's scream', the \isi{adverbial} {\sls  
tʃérím} `quietly',   the noun {\sls súrúm} `silence' (ultm. \ili{Hausa}), and  
{\sls géèm} `game reserve' (ultm. Eng.) are the only examples.  However, the languages 
\ili{Vagla} and \ili{Kasem}, surely among others,  allow final [{m}]. Both languages 
are genealogically related,  but only the former is in contact with Chakali. It is 
assumed that Chakali speakers  are accustomed to hearing   a bilabial nasal  in 
final position. However, an underlying final /{m}/ is possible, e.g. /dɔm/ 
$\rightarrow$ {\sls dɔ́ŋ́} {\sc sg.} {\sls dɔ́má}  {\sc pl.} `enemy' and /dɔŋ/ 
$\rightarrow$ {\sls dɔ́ŋ̀} {\sc sg.} {\sls dɔ́ŋà}  {\sc pl.}   `comrade'  (see 
Section \ref{sec:GRM-noun-classes} and Rule \ref{Gen-nas-vel}). Table \ref{tab:bilabial-nasal}(b) displays two 
minimal pairs involving the bilabial nasal in opposition with a bilabial plosive 
and  a labial-velar.
 
\begin{table} 

\caption{Bilabial nasal\label{tab:bilabial-nasal}}

\subfigure[Bilabial nasal]{
\begin{Qtabular}{lll}
mã̀ã́	&	mother & n\\
mɔ́	& 	work clay &  v\\
múrː	&	story  & n\\
dʊ̀má	&	soul & n\\
ɲʊ̀mɛ̀	&	blind & n\\
kɪ̀m-bɔ́ŋ	&	bad thing & n\\

\end{Qtabular} 
}
\quad
\subfigure[Contrast with a /b/ and /ŋm/]{
\begin{Qtabular}{llp{2cm}}
mɛ̀ŋ́	& mist & n\\
bɛ́ŋ	& type of tree&	n\\
ŋmɛ́ŋ	& okro &	n\\
\end{Qtabular} 
}
\end{table}


\paragraph{Alveolar nasal.}

\begin{table}[b]
\caption{Alveolar nasal\label{tab:alveolar-nasal}}
\subfigure[Alveolar nasal]{
\begin{Qtabular}{lll}
náàl	& 	grand-father & n\\
ná	&	see	& v\\
kànà	&	arm ring &	 n\\
zùpʊ̀ná	&	millet crazy top disease & n\\
nòkúǹ  &	type of tree & n\\
sàbáán	& 	roof top & n\\

\end{Qtabular} 
}
\quad
\subfigure[Contrast with a /l/ and /r/]{
\begin{Qtabular}{lll}

bɪ̀là &	try to solve	&	v\\
bɪ̀nà	&	old	&	v\\
náhɪ̃́ɛ̃̀ 	&	sense	&	n\\
lɛ̀hɛ́ɛ̀&	wooden spoon &	n\\
pɛ̀ná	&	moon	&	n\\
pɛ̀rà	&	weave & v\\

\end{Qtabular} 
}

\end{table}

The alveolar nasal  /{n}/ can occur in all three positions: word-initial, 
word-medial and word-final. Table \ref{tab:alveolar-nasal}(a) presents the 
alveolar nasal in those positions.  However,  as mentioned in Section 
\ref{sec:PHON-bil-nas}, Rule \ref{Gen-nas-vel} turns 
word-final 
nasals into a 
velar nasal. The number of words which allow a word-final alveolar nasal is very 
limited, and the majority are ultimately `non-native': {\sls 
dàmbàfúlánáán} `fifth month' (\ili{Waali}),  {\sls lɪ̀máàn} `imam, prayer-leader' (\ili{Arabic}),  {\sls 
méésìn} 
`mason' (Eng.),  {\sls ʔólŭ̀pléǹ} `airplane' (Eng.),  {\sls pɛ̀n} `pen'  
(Eng.), 
and  {\sls gáádìn} `garden'  (Eng.). In Table \ref{tab:alveolar-nasal}, the 
alveolar nasal is found in word-final positions in  {\sls nòkúǹ} and {\sls  
sábáán}. If these words were  uttered at the end of a phrase in normal 
speech, they would be  velarized. Nonetheless, when elicited in isolation, the 
alveolar nasals  do not always surface velarized, so a certain number of lexical exceptions  may exist (cf. Rule \ref{Gen-nas-vel}). Table \ref{tab:alveolar-nasal}(b) 
provides 
evidence that the alveolar nasal, the lateral, and the trill  are indeed 
distinct phonemes.

 


\paragraph{Palatal nasal.}

The palatal nasal /{ɲ}/ is found in word-initial  and word-medial 
positions, but never in  a word-final position. It never precedes another 
consonant and only one word where a consonant precedes the palatal nasal is 
identified, i.e. {\sls sámbálɲàŋá} `type of grass'.  Table  
\ref{tab:palatal-nasal}(a) provides examples where the palatal nasal occurs word-initially and -medially. The examples in Table \ref{tab:palatal-nasal}(b)  show 
that [n] and [ɲ] contrast in word-initial position.  


 
\begin{table} 

\caption{Palatal nasal\label{tab:palatal-nasal}}

\subfigure[Palatal nasal]{
\begin{Qtabular}{lll}
ɲã̀ã́	&	poverty &	n\\
ɲínè	&	look  &		v\\
ɲɪ́nà	&	father  &	n\\
ɲʊ̃̀ã̀	&	smoke &		v\\
ɲéɲáŋ̀	&	worm &		n\\
ʔàɲã̀ã́	&	type of snake &	n\\
bʊ̀ɲɛ́	&	respect with (\ili{Waali}) &	n\\

\end{Qtabular} 
}
\quad
\subfigure[Contrast with a /n/]{
\begin{Qtabular}{lll}

ɲã̀ã́ & poverty & n\\ 
nã̀ã̀ & leg & n\\

ɲɪ́ŋ	&	tooth	&	n\\
nɪ́ŋ̀	& this	&	adv\\

ɲʊ̃̀ʊ̃̀ & crowd & v\\
nʊ̃̀ʊ̃̀ & hear & v\\


\end{Qtabular} 
}
\end{table}

\paragraph{Velar nasal.}
\label{sec:velar-nasal}

\begin{table}[b]
\caption{Velar nasal\label{tab:velar-nasal}}
\subfigure[Velar nasal]{
\begin{Qtabular}{lll}

bʊ̀ŋà & bend & v\\
dɔ́ŋá		&	people & n\\
pɪ̀ŋà & be satisfied & v\\
kónsɪ́áŋ		&	red dove &  n\\
ŋmɛ́ŋ		&	okro &  n\\
kùŋkùŋ		&	brain &  n\\
\end{Qtabular} 
}
\quad
\subfigure[Contrast with a /n/]{
\begin{Qtabular}{lll}

kàŋá	&	back	&	n\\
kànà	&	arm ring	&	n\\
tɔ̀ŋà	&	type of sickness	&	n\\
tɔ̀ná	&	profit	&	n\\
tɪ̀ŋà & follow & v\\
tɪ̀nà &  cloud gather & v \\

\end{Qtabular} 
}

\end{table}

The segment [{ŋ}] is by far 
the most frequent nasal sound found in word-final position.  When it precedes a 
consonant, the velar nasal is  the last segment of  a preceding syllable.  
Unlike the other nasals it never appears in word-initial position. Table 
\ref{tab:palatal-nasal}(a)  provides  examples of the velar nasal in word-medial 
and -final positions. In Table \ref{tab:palatal-nasal}(b),   [{n}] and 
[{ŋ}] show contrast in word-medial  positions. 
 



 

\paragraph{Labial-velar nasal.}


The labial-velar nasal /{ŋm}/ is one of the four doubly-articulated 
segments in 
the language. It occurs in both word-initial and word-medial positions, as shown 
in Table \ref{tab:labialvelar-nasal}(a),  but  never in a word-final position. 
Table \ref{tab:labialvelar-nasal}(b) displays  minimal pairs involving the 
labial-velar nasal  in opposition with the other nasals. A single near-minimal 
pair with a palatal nasal is identified, but no minimal pair involving the 
labial-velar and the velar  nasal is found. The labial-velar nasal   mainly 
occurs in word-initial position, whereas the velar nasal occurs  in word-final 
position.  All SWG languages of Ghana have been reported with a phonemic contrast  between a labial-velar and a velar nasal \citep{Crou66, Gray69,  Toup95, Crou03}. Even though the labial-velar nasal is sometimes perceived as slightly palatalized when followed by a non-high front vowel, e.g. {\sls ŋmʲɛ̀ná} `chisel', it is not rendered in the transcription.
 
\begin{table}


\caption{Labial-velar nasal\label{tab:labialvelar-nasal}}

\subfigure[Labial-velar nasal]{
\begin{Qtabular}{lll}
ŋmá	&	tell 		&	v \\
ŋmɛ́dàà	&	thread holder 	&	n \\
ŋmɛ́ŋtɛ́l	&	eight 		&	num\\
ŋmɪ̃́ɛ̃́r	&	thief 	&	n\\
dʊ̀ŋmɛ́ŋ	&	type of snake	&	n\\
ŋmʊ̀nàŋmʊ̀nà 	& type of colour  & ideo\\
\end{Qtabular} 
}
\quad
\subfigure[Contrast with /{m}/,  /{ɲ}/,  and  /{n}/]{
\begin{Qtabular}{lll}
 ŋmá & say & v \\
má &you &2.pl.wk  \\
ɲã́ & defecate & v\\
ná & see& v \\
ŋmɛ́ŋ	&	okro	& n\\
mɛ̀ŋ́	&	dew	& n\\

\end{Qtabular} 
}
\end{table}


%Final word on homorganic sequence; there are nn sequences no?


\subsubsection{Lateral and trill}
\label{sec:approx-lat-trill}


\paragraph{Alveolar lateral approximant.}

The alveolar lateral approximant /{l}/ is found in word-initial positions, 
as well 
as word-medial and word-final positions. This is shown in Table 
\ref{tab:alveolar-lateral-approximant}(a).  There is only one token where the 
alveolar lateral precedes a nasal vowel, e.g. {\sls kɔ̀lʊ̃̀ŋ́} 
`well'  (but see Section \ref{sec:nasal-v} on nasal vowels).     In Table \ref{tab:alveolar-lateral-approximant}(b)  [r] and  [l] are 
shown to contrast in word-medial and word-final positions.
 
\begin{table}
\caption{Alveolar lateral approximant\label{tab:alveolar-lateral-approximant}}
\subfigure[Alveolar lateral approximant]{
\begin{Qtabular}{lll}

làà	&	take		& v\\
lɪ̀ɪ̀	&	go out		& v\\
jálá	&	burst 		&	v\\
pàtɪ̀lá	&	small hoe 	&	n\\
gántál	&	outside 	&	n\\
ʔɪ́l	&	breast  	&	n\\
%púàl	&	liver 		&		n\\
\end{Qtabular} 
}
\quad
\subfigure[Contrast with /r/]{
\begin{Qtabular}{lll}
 pàlà	&	flow	&	v  \\
pàrà	&	farm	&	v  \\
sʊ̀ɔ̀lá	&	type of cloth	&	n\\
sʊ̀ɔ́rá	&	odor &	n\\
púl̀	&	type of river grass	&	n  \\
púrː	&	skin bag	&	n  \\

\end{Qtabular} 
}
\end{table}

\newpage 
\paragraph{Alveolar trill or flap.}
\label{sec:flap}
\largerpage
 In careful speech,  the rhotic consonant is often produced with the blade 
of the tongue
vibrating against the alveolar ridge. However,
it would be wrong to treat  the production of /{r}/ in Chakali  and, for 
instance,
the /{r}/ in
Spanish, as similar. In normal speech,  the rhotic consonant is
usually perceived  as a flap-like sound. For instance,  the 
rhotic in {\sls pàrà} `to farm' sounds as if the tongue strikes its point of
articulation once, instead of repetitively. There is only one rhotic consonant,
but  even though it is not perceived as an alveolar flap in most cases, it is
transcribed as {\sls r},  instead of (the standard and more precise but less
practical) {\sls ɾ}.  Nonetheless,   /{r}/ in coda position is especially 
subject to
tongue vibration, e.g. {\sls gàŕ} `cloth'. 

Rhotic /{r}/ is found both word-medially and word-finally. In coda 
position, 
it is often emphasized; in such cases a diacritic is used to represent a lengthy 
trill, i.e. [{rː}]. It is also  the only consonant which occurs in the 
second position of a CC sequence  (Section \ref{sec:syllable-types} example 
\ref{ex:PHO-attested-CConsets}). It never occurs word-initially, except for the 
\isi{focus} marker {\sls ra}, which is nevertheless treated as a word unit (see 
Section 
\ref{sec:focus-forms} for the different forms the \isi{focus} marker can take), and  
the English loan {\sls rɔ́bà} `rubber' in {\sls rɔ́bàkàtásà} `plastic 
bowl'. 
Given that [{r}]  can be found in coda position but never in word-initial 
onset,  and [{d}] is mainly found in word-initial onset but never in the 
word-medial position of a monomorphemic word, the rhotic consonant could be 
treated as an allophone of /{d}/ (see  \citealt[30--31]{Awed02} and 
\citealt[62--64]{Daku02a}). Provisionally, though,  this solution is not favoured since it creates two issues which cannot be accommodated at this stage: (i) the  
CC sequence in onset becomes /Cd/,  e.g. /{pd}/ in {\sls prɪ́ŋ} `type of 
tree' and /{dd}/ in {\sls dráábà} `driver',  and (ii)  [{r}] and 
[{t}] 
are sounds distinguished by several minimal pairs, as opposed to [{d}], 
e.g. 
 {\sls tʃárɪ̀} `diarrhoea' and  {\sls tʃátɪ̀} `type of guinea corn', {\sls 
pàrà} 
`farm' and  {\sls pátá} `trousers',  {\sls lúró} `scrotum' and  {\sls lùtó} 
`root'.\footnote{Another piece of evidence would be the alveolar flap as the 
realization of a {\sls/t/} in a weak syllable, e.g.  ({\sc sg/pl}) {\sls 
sɔ̀tá}/{\sls 
sɔ̀ɾàsá}.}

  
Minimal pairs involving the alveolar rhotic and alveolar lateral approximant are 
given in Table \ref{tab:alveolar-trill}(b).\footnote{In 
\ref{tab:alveolar-trill}(b), the word {\sls kùòdú} `banana' is part of a 
minimal pair used as evidence for a  nonallophonic alternation between [{
r}]/[{d}]. However, the word {\sls kùòdú} is ultimately borrowed as it 
``exists all over West Africa in some form or other''  (M. E. Kropp-Dakubu, p. 
c.). It is the only minimal pair  [{r}]/[{d}] in the lexicon.}


 
\begin{table} 


\caption{Alveolar trill\label{tab:alveolar-trill}}
\subfigure[Alveolar trill]{
\begin{Qtabular}{lll}
pàrà	&	farm &	n\\
kʊ̀ɔ̀rɪ̀	&	built &	v\\
ʔàrɪ́ɪ̀	&	grass cutter &	n\\
grɪ́ɪ́ & cheek & n \\
gáŕː	&	stable &	n\\
gèŕː	&	lizard &	n\\
kórː	&	bench  &	 n\\
kpʊ́rː	&	palm tree	 &	n\\


\end{Qtabular} 
}
\quad
\subfigure[Contrast with /l/ and /d/]{
\begin{Qtabular}{lll}
fòrò & blanch & v\\
fòlò & make loose & v\\

hàrà	&lock&	v\\
hàlà &	fry &	v
\\
bílígí	&rub &	v
\\
bìrĭ̀gì	&delay	&v\\
kùórù & chief & n\\
kùòdú & banana & n\\

\end{Qtabular} 
}


\end{table}


\subsubsection{Glides}
\label{sec:approx-glide}

\paragraph{Voiced labio-velar approximant.}
\label{par:labio-velar-approximant}

The voiced labio-velar approximant /{w}/ appears both in word-initial and 
word-medial positions, but never in a word-final position.\footnote{Whether 
/{w}/ and /{j}/ occur word-finally results from one’s decision about 
syllable structure. Is [{aʊ}] phonologically /{aʊ}/ or /{aw}/? 
This question will not be resolved without a finer phonological model.}  There are a few words which are transcribed with superscript [ʷ] (e.g. {\sls  bʷɔ́ŋ} `difficult' and {\sls  zàkʷʊ́ʊ́l} `beetle'),  representing a labialized consonant, but there are  no definite regularities. When it  occurs, it is in front of a round vowel.\footnote{As mentioned in 
footnote \ref{fn:deg-labial}, \ili{Dɛg} is claimed to have an inventory of 13 phonemic labialized consonants \citep[2]{Crou03}. } In Table \ref{tab:labio-velar-approximant}(b)  examples are offered which set in opposition the voiced labio-velar approximant and the palatal 
approximant.\footnote{In field notes I transcribed [{ɥ}]  a highly 
aspirated and palatalized version of /{w}/ found before high front 
vowels, e.g. {\sls ɥìì} `weep' and {\sls ɥɪ́ɪ́}  `matter'. This sound needs  further 
investigation because I did not perceive it consistently in that environment. It is transcribed throughout with {\sls w}.}


 
\begin{table}


\caption{Voiced labio-velar approximant\label{tab:labio-velar-approximant}}

\subfigure[Voiced labio-velar approximant]{
\begin{Qtabular}{lll}
wáá	&	he, she, it &	3.sg.st.\\
wɪ́ɪ́ &  matter & n\\
wóŋ &  deaf person & n\\
fɔ̀wà & wrap & v\\
jɔ̀wá	&	market &	n\\
pèwò & blow & v\\


\end{Qtabular} 
}
\quad
\subfigure[Contrast with /j/]{
\begin{Qtabular}{lll}
wàá &	Wa town	& propn	\\
jàà	&	fetch	&	v\\
wàà	&	come	&	v\\
jà	&	we, our	&	1.pl.wk\\
tàwà & inject & v\\
tájà &catapult (ultm. Eng.) & n\\

\end{Qtabular} 
}


\end{table}

\newpage 
\paragraph{Palatal approximant.}
The palatal approximant /{j}/  appears both in word-initial and word-medial
positions, as shown in Table  \ref{tab:palatal-approximant}(a),   but never in a
word-final position.   Table \ref{tab:palatal-approximant}(b) provides additional minimal pairs in which the
palatal approximant and the voiced labio-velar approximant contrast.

 
\begin{table} 

\caption{Palatal approximant\label{tab:palatal-approximant}}
\subfigure[Palatal approximant]{
\begin{Qtabular}{lll}

júò	&	fight, quarrel &	n\\
tájà &catapult (ultm. Eng.) & n\\ 
bàjúòrà	&	lazy &	qual\\
ɪ̀jɛ̀là & clan name &  propn\\ 

\end{Qtabular} 
}
\quad
\subfigure[Contrast with /w/]{
\begin{Qtabular}{lll}
jàà	&	fetch	&	v\\
wáá	&	he, she, it &	3.sg.st.\\
jóŋ̀	&	slave	&	n\\
wóŋ	&	deaf	&	n\\

		
\end{Qtabular} 
}


\end{table}



\paragraph{Glottal approximant.}
\label{sec:glot-approx}

The glottal approximant   /{h}/  
occurs only in word-initial and -medial positions.
Table \ref{tab:glot-approx}(b) shows examples in which [{h}]
contrast with the fricatives and the glottal plosive.

 
\begin{table} 

\caption{Glottal  approximant\label{tab:glot-approx}}

\subfigure[Glottal  approximant]{
\begin{Qtabular}{lll}


há	&	hire&	v\\
hɔ́l	&	piece of charcoal	&n\\
hìrè	&	dig&	v\\
nàhã́		& ego's grand-mother	& n\\
lúhò	& 	funeral	& n\\
lɛ̀hɛ́ɛ̀		& wooden spoon	& n\\
\end{Qtabular} 
}
\quad
\subfigure[Contrasts]{
\begin{Qtabular}{lll}

hàlà &	fry	& v \\
vàlà  & walk & v\\
fàlá &	calabash	& n  \\
hɪ́ɛ́ŋ	&	relative	& n\\
zɪ́ɛ́ŋ	&	snake venum	&  n\\
hól	&	type of tree & n\\
sólː	&	clearly	& adv\\
ʔól	&	type of mouse	&n\\
%hɪ́lː	&	witch	&  n\\
%ʔɪ́l	&	breast	& n\\


\end{Qtabular} 
}
\end{table}




\newpage 
\subsubsection{Summary} 
\label{sec:sum-consonant}

The consonants of Chakali were introduced and the majority were
presented in a pairwise fashion to highlight specific contrasts. In Table
\ref{tab:ConsChart},  the  consonantal phonemes are arranged according to their
place and manner of articulation. Among them,  the surface consonant [ɣ] is 
derived from underlying phonemes, i.e. {/g/} or  { /k/}.   Due to the
limited scope of the present section,  the phonological
features making up the consonant phonemes were not introduced. They will be
presented along the way
when necessary.\footnote{In order to maintain neatness, the label `Liquid' was given to laterals, approximants and trills.}


\begin{table} 


 \caption{Phonetic and phonemic consonants in Chakali}
\label{tab:ConsChart}
 % \begin{small}
 \fittable{
  \begin{tabular}{llp{1cm}lllllp{1cm}}
\lsptoprule
    & Bilabial & Labial-dental & Alveolar & Postalv. & Palatal & Velar & Glottal
 & Labial-velar\\ \midrule 

Plosives &     p      b  & &     t      d  & & &     k      g  &    ʔ  & 
   kp     gb  \\ 
Fricatives &&    f     v  &    s     z  & && \ \ \   (ɣ) &     h  &\\

Affricates &&& &   tʃ     dʒ  &&&& \\ 
Nasals&    m  &&    n  & &    ɲ  &    ŋ &  &    ŋm \\ 
Liquid &&&    l     r  &&&& &\\ 
Semi-vowels &    &&&&   j &&& w (ɥ) \\ \lspbottomrule

  \end{tabular}
  }
% \end{small}

\end{table}



\newpage 
\section{Phonotactics}
\label{sec:phonotac}


\subsection{Syllable types}
\label{sec:syllable-types}

This section deals with the restrictions on possible syllable types. The
necessary generalizations responsible for (im)possible segment sequences are
introduced. Again, the  syllabification procedure  used to 
extract the syllable types is implemented in {\it Dekereke} and uses the database's
pronunciation field.\footnote{Software 
written and maintained by Rod Casali (version 
1\_0\_0\_180 \href{http://casali.canil.ca/}{http://casali.canil.ca/}).}  
First, 
syllabic nasals are marked with a diacritic and are treated as one
syllable.  Secondly, all word-initial consonant clusters are assigned to
the onset of the first syllable, and all word-final consonant clusters to the
coda of the last syllable. Then,  intervocalic consonant clusters are
syllabified by maximizing onsets, as long as the resulting onsets match an
attested word-initial consonant sequence or segment, and the resulting coda
matches an attested word-final consonant sequence or segment. An onset cluster 
respects a sonority slope similar to the one given in
(\ref{ex:sonority-slope}). 



\begin{exe}
\ex\label{ex:sonority-slope}{{\rm Phonetically grounded sonority scale for
consonants   \citep[236]{Park02}}}\\
 laterals $>$
trills $>$nasals $>$ /h/ $>$ voiced fricatives $>$ voiced stops $>$ voiceless
fricatives $>$ voiceless stops $>$ affricates
\end{exe}

This means that  (i) as one
proceeds towards the nucleus the sonority must increase, and (ii) as one 
proceeds away from the nucleus the sonority must  decrease.   This 
sonority-based implementation  generates the ill-formed onset clusters
given in (\ref{ex:PHO-illf-onsets}).  


\begin{exe}
\ex\label{ex:PHO-illf-onsets}
\begin{xlist}
\ex *mb \\
 .ʔɛ.mbɛ.lɪ.		{\rm `shoulder'}    (.ʔɛm.bɛ.lɪ.)
\ex *ɣl\\
 .ha.ɣlɪ.bie.		{\rm`type of ants'}   (.hag.lɪ.bie.)	
\ex *ɣj\\
 .pa.tʃɪ.ɣja.ra.	{\rm`healer'}  (.pa.tʃɪg.ja.ra.)


\end{xlist}
\end{exe}

The forms in parentheses following the glosses in  (\ref{ex:PHO-illf-onsets}) 
are correctly 
syllabified. The forms preceding the glosses are clusters that either satisfy 
(i.e. ɣl, ɣj) or do not satisfy  (i.e. mb)  the sonority requirement, but are 
nonetheless not correctly syllabified. To remedy  this problem,   *mb, *ɣl,  and 
*ɣj  become {\it ad hoc} constraints on onset clusters. This leaves us with a few
attested C$_{1}$C$_{2}$ sequences in (\ref{ex:PHO-attested-CConsets}), which 
will be discussed below.  


\begin{exe}
\ex\label{ex:PHO-attested-CConsets}{C$_{1}$= {\sc sonorant} C$_{2}$ = {\sc
trill}}\\ 
.prɪŋ.	  	{\rm `type of Mahogany'} \\
.bri.ge.	 		{\rm   `type of snake'} \\
	.draa.ba.		{\rm   `driver'  (Eng.)}\\
\end{exe}


The first column of Table   \ref{tab:syll-type} displays the ten syllable types
attested.  The other columns display the number of instances of a given
syllable in three positions, i.e.  word-initial, word-medial, and word-final, 
regardless of  grammatical category distinctions. The table shows that Chakali 
words  mainly comprise CV, CVC, and CVV syllables. Table 
\ref{tab:syll-type-examples} provides examples of words which contain each of 
the ten syllable types. They are given in the same order as in Table 
\ref{tab:syll-type}.  


\begin{table} 
 
\caption[Syllable Types]{Attested syllable types (version 10/09/15)
\label{tab:syll-type}}
\begin{tabular}{lrrr}
\lsptoprule
Syllable type & Word-initial & Word-medial & Word-final \\[1ex]
\midrule
CV & 1528 & 1184 & 1483  \\ 
CVV & 717 & 242 & 903  \\ 
CVC & 572 & 222& 388  \\
CVVC & 79 & 22 & 122  \\ 
V & 25 & 0 & 5 \\ 
N & 5 & 0 & 3  \\ 
CVVV & 5&0&12 \\
CCVC & 2 & 0 & 2  \\
CCVV & 2 & 0 & 1 \\ 
CCV & 1 & 0 & 1 \\ 


 \lspbottomrule
\end{tabular}
\end{table} 



\begin{table} 

 
\caption[Tokens for each syllable
type]{Tokens for each syllable
type\label{tab:syll-type-examples}}
\begin{Qtabular}{llll}
\lsptoprule
Syllable type & Instantiation & Gloss & PoS\\[1ex]
\midrule 

CV 	&  .{\sls pa}.tʃɪ.gɪɪ.		&	abdomen	&n\\
	&.gbɛ.{\sls ta}.ra.		&	pond	&n\\
	& .ʔɔ.{\sls ma}.	 &	fear	&v\\ 

CV$_{\alpha}$V$_{\alpha}$ & .bãã.               &   type of lizard         & 
n\\
			  &.ʔa.{\sls  lɛɛ}.fʊ.	        &	type of leaf  &
n\\	
			  & .sɪɪ.{\sls maa}.	         &	food	     & n
 \\ 
CV$_{\alpha}$V$_{\beta}$  & .{\sls dɪa}.tɪɪ.na.		&	landlord  &
n\\
	& .ba.{\sls  juo}.ra.		&	lazy	& n\\
	& .tʊɔ.{\sls nɪ̃ã}.	        &	type of genet &	n\\
		
CVC 	&  	.{\sls ʔɛm}.bɛ.lɪ.	&	wing	&	n\\
	&	.ga.{\sls lan}.zʊr. &	mad person &	n\\	
	&	.nãã.{\sls pol}.	&	Achilles tendon	& n\\
	
CVVC	&  	.baal.		&	man  &	n\\
	        &	.bʊ̃ʊ̃ŋ.		&	goat	&      n\\
	       
	& 	.tʃiir.		& 	taboo			& n	\\	


V 		& 	.ɪ.	&	you, your   &	2.sg.wk.\\
  		& .a.	 	&	the  	    &	art\\

N	 &  .n̩.		&	I, my	&	1.sg.wk\\
	&.{\sls m̩}.buo.ɲõ.	&	hunter's rank (\ili{Gonja})	&	n \\ 
%VC &   \\ 
CCVC         	&.prɪŋ.			&	type of Mahogany	&
n\\


CCV		&	.{\sls bri}.ge.	 & 	  type of snake & n \\


CCVV 	&  	.draa.ba	&	driver  (ultm. English)	&n\\
	
CVVV &   .bʊ̃ɛ̃ɪ̃.bʊ̃ɛ̃ɪ̃.	&	carefully &	ideo\\
	&.ŋmɪ̃ɛ̃ɪ̃. & stealing & n\\
	&.paaʊ.	 &	collect.{\sc foc}	& v\\
		&.paaʊ.	 &	collect.{\sc 3.sg}	& v\\
	& .ʃɪ̃ãã.  &	insult &		interj\\

\lspbottomrule
\end{Qtabular}


\end{table} 


The low-frequency  syllable types  of Table \ref{tab:syll-type} need explanation. The syllabic nasal has a few tokens, e.g.  the various surface forms of the first person singular \isi{pronoun},  the word {\sls .m̩.bu.o.ɲõ.} `hunter's rank' (borrowed from \ili{Gonja}), and  the name of one of my consultants, Fuseini Mba Zien, whose second name  originally means  `my father' (in several \ili{Oti-Volta} languages and beyond) and is syllabified [{\sls .m̩.ba.}].  Adding to these examples,  there are contexts in which  a nasal makes the syllable peak following an  onset consonant. For instance, when involved in some compounds, the stem  /{\sls bagɛna}/  `neck'  yields  [{\sls .ba.gn̩.}], as in {\sls .ba.gn̩.pʊɔ.gɪɪ.} `lateral goiter',  {\sls .ba.gn̩.bʊa.} `hollow behind the collarbone',   and {\sls .ba.gn̩.tʃu.gul.}  `dowager's hump'. 

There are restrictions on the type of segments which can act as coda. All velars are permitted in coda position, i.e. \{{k, g, ɣ, ŋ}\} . The alveolar nasal [{n}],   lateral [{l}],  trill [{r}], plus rare instances of [{m}],  are also permitted. 

For the CC sequences, it was mentioned in Section \ref{par:labio-velar-approximant} that labialized consonants are rarely perceived. Still, a few words are transcribed as   [{Cʷ}], a sequence that could be read as  [{Cw}] by the syllabification procedure, i.e.  {\sls bʷɔ́ŋ̀} `bad' and {\sls zákʷʊ́ʊ́l} `beetle grub'.  That leaves us with one  instance of the syllable type CCVC, i.e. [{\sls .prɪŋ.}],  a sequence mentioned in (\ref{ex:PHO-attested-CConsets}) above. Syllable types CCVV and CVVV are scarce, but for different reasons. The former involves a CC onset cluster which  is infrequent, as mentioned in Section \ref{sec:flap}.  The latter is also rarely attested in the lexical database,  but could become very frequent if some cases of suffixation  were consistently included in the lexicon. That is, given a verbal lexeme with a CVV final syllable,  a CVVV sequence is produced by adding the nominalization or the assertion suffix  (i.e. CVV-{i/ɪ}   and   CVV-{u/ʊ},  respectively).  These are described in Sections \ref{sec:GRM-verb-act-stem} and \ref{sec:GRM-verb-suffix}.  

%\newpage
\subsubsection{Syllable representation}
\label{sec:syllable-rep}

In this section,  a unified representation of the syllable is provided. The
notion
of  {\it weight unit} captures aspects of the internal structure of a syllable. Weight distinctions are encoded in mora count, which has been proposed as an intermediate
level of structure between the segments and the syllable \citep{Hyma85}.   The mora is of particular importance since it determines vowel length and  tone assignment, among other things.   In (\ref{exe:light-heavy}) the top node symbol $\sigma$ represents the syllable. At a level under the syllable, the symbol $\mu$ represents the mora.  The main opposition is between monomoraic (light) and bimoraic (heavy) syllables, but trimoraic (superheavy)  syllables are also possible. The light syllables are composed of  a single consonant and a single vowel (CV),  a single vowel (V), or  a syllabic nasal (N). The heavy and superheavy syllables are  CVV,  CVVC, CCVC, CCV, CVVV, and  CCVV. The type CVC can be both light and heavy.   

 

% \begin{minipage}{12cm}
\begin{exe}
\ex\label{exe:light-heavy}
\begin{multicols}{3}
\begin{xlist}

\ex\label{ex:light1}{\it light}\\{\Treek[-1]{1}{&\Kq{$\sigma$}\Bq{ddl}\Bq{d}\\
                     & \Kq{$\mu$} \Bq{d}  \\ 
                      \Kq{{n}} & \Kq{{ɪ}} } }
                 

\ex\label{ex:light2}{\it light}\\{\Treek[-1]{1}{\Kq{$\sigma$}\Bq{d}\\
                      \Kq{$\mu$} \Bq{d}  \\ 
                        \Kq{{a}} } }


\ex\label{ex:light3}{\it light}\\{\Treek[-1]{1}{\Kq{$\sigma$}\Bq{d}\\
                      \Kq{$\mu$} \Bq{d}  \\ 
                        \Kq{N} } }

                 

\ex\label{ex:light4}{\it light}\\{\Treek[-1]{1}{&
\Kq{$\sigma$}\Bq{ddl}\Bq{ddr}\Bq{d} &\\
                     & \Kq{$\mu$}\Bq{d}&   \\ 
                      \Kq{{b}} &\Kq{{ɔ}} &\Kq{{k}}}}



\ex\label{ex:heavy1}{\it heavy}\\{\Treek[-1]{1}{&
\Kq{$\sigma$}\Bq{ddl}\Bq{d}\Bq{dr} &\\
                     & \Kq{$\mu$}\Bq{d}& \Kq{$\mu$}\Bq{d}   \\ 
                      \Kq{{k}} &\Kq{{u}} &\Kq{{o}}}}

\ex\label{ex:heavy2}{\it heavy}\\{\Treek[-1]{1}{&
\Kq{$\sigma$}\Bq{ddl}\Bq{dr}\Bq{d} &\\
                     & \Kq{$\mu$}\Bq{d}& \Kq{$\mu$}\Bq{d}   \\ 
                      \Kq{{s}} &\Kq{{a}} &\Kq{{l}}}}


\end{xlist}
\end{multicols}
\end{exe}
% \end{minipage}
%\vspace*{15pt}








% given in (\ref{ex:PHO-attested-CConsets}), e.g. {\sls prɪŋ} `type of
% Mahogany'. 
%how are the monosyllable word counted, as word-initial or final or medial

 The syllable structure in (\ref{ex:light1}) is  found in many verbs and
function words (e.g. \isi{postposition} {\sls nɪ}, \isi{focus} marker {\sls ra},
preverbal particles  {\sls ka},  {\sls bɪ}, and {\sls ha}, verbs {\sls na} 
`see', 
{\sls pɛ} `add',  and  {\sls tɔ} `cover', etc.) The light syllable in
(\ref{ex:light2}) is exemplified by the definite \isi{article} {\sls a} `the' and
the
second and third person singular weak pronouns {\sls ɪ} `you, yours'  and  
{\sls 
ʊ}
 `he, she, it, his, her, its'. Vowel
coalescence (i.e. when two consecutive vowels fuse into a long one) suggests 
that these pronouns are not CV-syllables with glottal plosives in onset 
positions  (see Section \ref{sec:internal-sandhi}).  A syllabic nasal constitutes
a light syllable (\ref{ex:light3}): apart from their segmental content, 
structure (\ref{ex:light2}) and (\ref{ex:light3}) are identical, that is, they are also 
both syllable structures of singular pronouns.  Another light syllable is the 
one in (\ref{ex:light4}). The choice of treating a CVC sequence as light comes 
from a certain division in the consonants, that is,  those which are perceived 
with a tone and those which are not. Thus both (\ref{ex:light4}) and 
(\ref{ex:heavy2}) can represent the structure of a CVC sequence, but only the 
latter contains a moraic coda.\footnote{A reviewer pointed out that tonological generalizations are much better evidence concerning the moraic status of coda consonants. There are many  suggestions for further research, but studies of tone and \isi{intonation} are urgently needed for the languages of the area. For instance, questions relevant to moraic coda consonants are how to  properly account for consonants which are found to bear tones and how to  treat contour tones on CVC words.  This distinction between (\ref{ex:light4}) and (\ref{ex:heavy2}) would need to be spelt out carefully in a phonological study.}



The heavy syllables are those with two moras. The structure  in 
(\ref{ex:heavy1}) represents any vowel sequence, e.g. {\sls sã̀ã́} `axe' or 
{\sls  kùó}  `farm', and the one in (\ref{ex:heavy2}) a sequence in which the 
final consonant projects a mora, e.g.  {\sls sàĺ} `flat  roof'. Thus, the set 
of consonants which are found to bear tones are those which project moras; namely /{l}/, /{r}/,  and the nasals. This suggests  that at least a feature 
{\sc sonorant}  must be involved for a segment to bear tone. However,  a tone on a {\sc sonorant} segment in syllable final position is not always transcribed. 


The superheavy syllables are commonly described as consisting of CVCC or CVVC.
The former syllable is not attested; a coda consisting of two or more consonants
does not exist. The latter type is instantiated in (\ref{ex:sup-heavy1}) with
 the word {\sls báàl} `male':  other examples are {\sls hùór} `raw',  
{\sls vàáŋ} `front leg', among others.  Although not attested in a single
morpheme (except perhaps in the \isi{ideophone} {\sls  bʊ̃̀ɛ̃̀ɪ̃̀bʊ̃̀ɛ̃̀ɪ̃̀} `slowly' 
and the \isi{interjection} {\sls ʃɪ̃́ã̀ã̀} `insult'),    the CVVV syllable types are 
treated as trimoraic. The words in
(\ref{ex:sup-heavy2}) `collect.{\sc nmlz}'  and  (\ref{ex:sup-heavy3}) 
`collect.{\foc}'
are made from the verbal CVV stem {\sls laa} `collect'. In these examples,  
CVVV 
syllables
arise  from the suffixation of nominal 
and assertive  morphology, 
(\ref{ex:sup-heavy2}) and (\ref{ex:sup-heavy3}) respectively. As presented in 
Sections
\ref{sec:GRM-personal-pronouns} and \ref{sec:GRM-morph-opro}, cliticized
pronouns in object positions also
create CVVV syllables.


\eabox[-.85\baselineskip]{\label{exe:superheavy}
% \begin{multicols}{3}
\begin{xlist}

\parbox{.29\textwidth}{
\ex\label{ex:sup-heavy1}{\it superheavy}\\
{\Treek[-1]{1}{&
 \Kq{$\sigma$}\Bq{ddl}\Bq{dr}\Bq{d}\Bq{drr} & & \\
                     & \Kq{$\mu$}\Bq{d}& \Kq{$\mu$}\Bq{d}  & \Kq{$\mu$}\Bq{d} \\ 
                      \Kq{{b}} &\Kq{{a}} &\Kq{{a}} & \Kq{{
l}} }}
}
\parbox{.29\textwidth}{
\ex\label{ex:sup-heavy2}{\it superheavy}\\
{\Treek[-1]{1}{&
 \Kq{$\sigma$}\Bq{ddl}\Bq{dr}\Bq{d}\Bq{drr} & & \\
                     & \Kq{$\mu$}\Bq{d}& \Kq{$\mu$}\Bq{d}  & \Kq{$\mu$}\Bq{d} \\ 
                      \Kq{{l}} &\Kq{{a}} &\Kq{{a}} & \Kq{{
ɪ}} }}
}
\parbox{.22\textwidth}{
\ex\label{ex:sup-heavy3}{\it superheavy}\\
{\Treek[-1]{1}{&
 \Kq{$\sigma$}\Bq{ddl}\Bq{dr}\Bq{d}\Bq{drr} & & \\
                     & \Kq{$\mu$}\Bq{d}& \Kq{$\mu$}\Bq{d}  & \Kq{$\mu$}\Bq{d} \\ 
                      \Kq{{l}} &\Kq{{a}} &\Kq{{a}} & \Kq{{
ʊ}} }}
}
\end{xlist}
% \end{multicols}
}

Likewise,  some of the representations in (\ref{exe:light-heavy}) can
either be
projected by a single lexeme or by the combination of  one lexeme and a vowel
suffix.  For example,  the word {\sls bìé} `child' is analysed as being 
composed of the
stem {\sls bi}    and a singular suffix vowel, but the word {\sls tàá}  
`language' is  formed by the stem {\sls taa}  and a  zero-suffix for
singular.   Noun class morphology is discussed in Section 
\ref{sec:GRM-noun-classes}. 

 


\subsubsection{Weak syllable}
\label{sec:PHO-weak-syll}

It has already been noted in Section \ref{sec:PHO-vel-plos} that a segment may 
change into another in a phonological domain called a weak syllable.  This is  
defined  as the state resulting from a reduction or augmentation of a syllable 
in a specific environment.  For instance, in noun formation, the 
generalizations in 
(\ref{ru:epen-V-weak}) are observed when a CV number suffix attaches to a CVC 
stem, i.e.\ CVC ] -CV, or a CVCV stem, i.e.\ CVCV ] -CV. 



\ea\label{ru:epen-V-weak}
\ea  {\rm Vowel epenthesis}\par\nobreak\smallskip
 {\it Insert a \textsc{[+syll]} segment between medial adjacent 
consonants}

\ex {\rm Vowel weakening}\par\nobreak\smallskip
 {\it Reduce the duration and loudness of a vowel between medial consonants}

\ex {\rm  Intervocalic lenition}\par\nobreak\smallskip
 {\it Velar stops become fricatives between vowels}

\z
\z

In the case of a CVC stem, vowel \isi{epenthesis} creates a vowel between the stem's coda consonant and the suffix's onset consonant (more on epenthesis in Section \ref{sec:epenthesis}).   In a resulting CVCVCV environment the quality of the second interconsonantal vowel  is not  as full as other vowel(s) in the same word: possible outcomes are the reduction of any vowel to [{ə}],  shortening  (marked as extra-short, e.g.  [{ă}]),  or its  deletion. Also in the same CVCVCV environment, intervocalic spirantization operates on the onset consonant of the second syllable,  turning the velar obstruents /{k}/ and /{g}/ into [{ɣ}] (see Sections \ref{sec:PHO-alveo-plos} and \ref{sec:PHO-vel-plos}). 



\subsubsection{Consonant cluster}
\label{sec:PHO-gemination}

A sequence of consonants is not phonologically distinctive and many tokens are the results of place assimilation. It is treated as a repetition of adjacent and identical segments within a word, closing one and opening the next  syllable. Only the set of consonants \{{n, l, m, ŋ}\} is attested. 


\begin{exe}
	
	\ex\label{ex:PHO-gemin-trans}{\rm Transparent polymorphemic}
	\begin{xlist}
	\ex {\sls kpã̀ã̀n-nɪ̃́ɪ̃́}   \quad {\rm [yam-water]}  \quad    {\rm `water yam'} 
	\ex {\sls gɔ́n-nã́ã́}  \quad {\rm [river-leg]} \quad  {\rm `branch of a river'}
	\ex {\sls bà-lál-là}  \quad {\rm [body-open-{\sc nmlz}]}  \quad  {\rm `happiness'}
			\end{xlist}	
		
	\ex\label{ex:PHO-gemin-opaq}{\rm Opaque}
		\begin{xlist}
	\ex {\sls kúmmì} \quad  {\rm `fist'}
	\ex 	{\sls ɲáŋŋɪ́}  \quad {\rm `be worse'}
	\ex 		{\sls tʃímmã̀ã́} \quad {\rm  `pepper'}
		\end{xlist}
\end{exe}


Example (\ref{ex:PHO-gemin-trans})  shows  a consonant cluster in fully 
transparent  polymorphemic lexical items, while (\ref{ex:PHO-gemin-opaq})  in 
morphologically opaque ones.\footnote{Despite being infrequent in Chakali ({\sls n} = 19, {\sls l} = 6, {\sls m} = 54, {\sls ŋ} 8), ``the verb {\sls ɲáŋŋɪ́} `be worse’ is a \ili{Vagla} verb with normal-for-\ili{Vagla} form'' (T. Naden, p.c.).}

\subsection{Sandhi}
\label{sec:sandhi}

In this section,  some morphophonological processes are introduced. First,
the  processes occurring within the word are presented, then the processes
occurring at word boundaries. 

\subsubsection{Internal sandhi}
\label{sec:internal-sandhi}
Internal sandhi  refers to insertions, deletions,  or modifications of sounds
at morpheme boundaries within the word. 


\paragraph{Nasal place assimilation.}
\label{sec:internal-sandhi-nasal-place}

In words composed of more than one stem, a nasal ending the first stem  assimilates the  place feature of the following consonantal segment. In this manner,  the bilabial [{m}] surfaces when  the first consonant of the second stem is {\sc labial}, the velar  [{ŋ}] when it is {\sc velar}  and  the alveolar [{n}] elsewhere. Yet, in front of [{h}], the underlying velar nasal stays unchanged. The same process takes place when a stem and a noun class suffix are combined, e.g /{\sls gʊm}/  ({\sc cl.3}),  {\sls gʊ̀má} {\sc sg}  and  {\sls gʊ̀nsá} {\sc pl} `hump(s)'. Table \ref{tab:w-inter-nasal-place} provides some examples (see Section
\ref{sec:ext-nasal-place} for similar processes at word boundaries).


% within compound words and first person singular \isi{pronoun} with grammatical
% function possessive in noun phrases and subject in sentences).


\begin{table}[!htp]

 
 \caption{Word-internal nasal place assimilation 
\label{tab:w-inter-nasal-place}}
% \begin{center}
\fittable{
\begin{Qtabular}{lllll}
\lsptoprule
Stems & Literal meaning  &Word & Gloss & PoS\\ \midrule
kɪn-bɔŋ	& {thing-bad} &	kɪ̀mbɔ́ŋ		& 	bad & n
\\
%kɪn-biriŋ	&{thing-whole}&	kɪ̀mbɪ́rɪ́ŋ	&	whole & n
%\\
loŋ-bɔla	& {calabash-oval} &	lómbɔ̄l		&
 calabash type & n\\
nɔŋ-buluŋ      & {stone-black} &	nɔ̀mbúlúŋ̀		&	
grinding stone type & n\\
sɪŋ-tʃaʊ		& {drink-termite} &sɪ́ntʃáʊ́	 &	
type of tree& n\\
sɪŋ-pʊmma 	& {drink-white} & 	sɪ̀mpʊ̀mmá	&	palm
wine & n \\
sɪŋ-sɪama 	& {drink-red} & 	sɪ́nsɪ̀àmá 	&	fermented pito &
n \\
galaŋa-zʊʊ-r & {madness-enter-agent} & gàlànzʊ́ʊ́r  & mad person & n\\

 \lspbottomrule
 \end{Qtabular}
 }
% \end{center}
\end{table}


Rule \ref{PHO-n-assimln} captures the phenomenon.
% 
\begin{Rule}\label{PHO-n-assimln}{N-regressive assimilation}\\
A nasal consonant assimilates the place feature of the following consonant
 (conditions: internal and external sandhi).\\
C[{\sc +nasal}] $\rightarrow$   [$\alpha${\sc place}] /  \_  C [$\alpha${\sc place}]
\end{Rule}



\paragraph{Nasalisation of verbal suffixes.}
\label{sec:nasalisation-verb-suffix}

The two suffixes under consideration are discussed in Section
\ref{sec:GRM-verb-perf-intran}
and \ref{sec:GRM-focus}.  The first is  the perfective suffix.  It takes
either the form {\sls -je/jɛ} or {\sls -wa}.
The quality of the surface vowel depends on (i) whether the verb takes the
assertive suffix (glossed {\sc foc}, standing for `in \isi{focus}'),  and (ii) the
vowel quality of the verbal stem. To isolate each effect, negating a proposition
makes sure that the assertive suffix does not appear on the verb. The second is
the assertive suffix, which can appear on a verb stem both in the imperfective
and perfective aspects.  To portray the two suffixes in a non-nasal environment,
 the verb {\sls kpé} `crack and remove' in Table \ref{tab:crack-remove}  is 
placed in two paradigms (reproduced from Section \ref{sec:GRM-verb-suffix}). 



\begin{table}


\caption{{\sls kpé} {\rm `crack
and remove'} (c\&r)  \label{tab:crack-remove}}

\subfigure[Positive]{
\begin{Qtabular}{lll}
 {\sc fut} &  ʊ̀ kàá kpē 	& `She will c\&r'\\
{\sc ipfv}& ʊ̀ʊ̀ kpéū  	 & `She  is c-\&r-ing'\\
{\sc pfv} &  ʊ̀ kpéjòō 	&  `She   c-\&r-ed'\\
{\sc imp} & kpé  		& `C\&r!'  
\end{Qtabular} 
}\quad
\subfigure[Negative]{
\begin{Qtabular}{ll}
ʊ̀ wàá kpè  	& `She will not c\&r'\\
	ʊ̀ wàà kpé  	& `She  is not c-\&r-ing'\\
	ʊ̀ wà kpéjè  	&  `She  did not c-\&r-ed'\\
	té kpéì 		&  `Don't c\&r!'
\end{Qtabular} 
}
\end{table}

Since this section is concerned with nasalisation, the meaning and function of each form is ignored. As seen from the examples, and leaving tones aside, the verbal stem {\sls kpé} has two forms in the negative and three in the positive. The positive is seen as a paradigm in which the event is in \isi{focus}, as opposed to the argument {\sls ʊ} `she' of the predicate {\sls kpé}. Because of {\sc atr}-harmony (Section \ref{sec:vowel-harmony}), the perfective suffix {\sls -je/-jɛ} agrees in {\sc atr} with the stem vowel and is rendered {\sls -je} (perfective negative form {\sls kpéjè}). In the affirmative,  when assertive suffix  {\sls -u/-ʊ}  follows {\sls  -je}, the two vowels coalesce, the assertive suffix is lowered and the two  surface as [{oo}]. A process similar to (\ref{ex:PHO-suf-perf-ass-prcss})  accounts for the negative and positive perfective forms.

\begin{exe}
\ex\label{ex:PHO-suf-perf-ass-prcss}
kpe-j{\ob}{\sc  --hi, --ro}{\cb}  $\rightarrow$  $\alpha${\sc atr} $\rightarrow$ kpe-je 
 $\rightarrow$
kpeje-{\ob}{\sc +hi,+ro}{\cb}  $\rightarrow$  kpejoo
\end{exe}

The explanation for the form {\sls kpéū} is equivalent, except that the
perfective suffix is not involved. Thus,  the verbal stem triggering the {\sc
atr} agreement on the assertive suffix is the only step  accounted
for. The process in shown in (\ref{ex:PHO-suf-ass-prcss}).


\begin{exe}
\ex\label{ex:PHO-suf-ass-prcss}
kpe-{\ob}{\sc +hi,+ro}{\cb}  $\rightarrow$  $\alpha${\sc atr} $\rightarrow$ kpeu
\end{exe}


Nasalisation takes place within these two processes. For instance, when the verb stem {\sls sáŋá} `sit' is placed in the same environment as {\sls kpé} in Table \ref{tab:crack-remove}, all vowels following the velar nasal are nasalized.\footnote{The interplay of vowel coalescence and  length is not yet fully understood. This is reflected in the current state of the orthography.}  The process is shown in (\ref{ex:PHO--sit-suf-prcss}).

\begin{exe}
\ex\label{ex:PHO--sit-suf-prcss}
\begin{xlist}

\ex\label{ex:PHO-sit-suf-perf-ass-prcss}
 saŋa-j{\ob}{\sc  --hi,--ro}{\cb}$\rightarrow\!\alpha${\sc atr}$\rightarrow\!\alpha${\sc nasal}
$\rightarrow$saŋ(ə)jɛ̃(ɛ̃)-{\ob}{\sc +hi,+ro}{\cb}$\rightarrow$saŋ(ə)jʊ̃ʊ̃

\ex\label{ex:PHO--sit-suf-ass-prcss}
saŋa-{\ob}{\sc +hi,+ro}{\cb}   $\rightarrow$  $\alpha${\sc atr}  $\rightarrow$  
$\alpha${\sc nasal} $\rightarrow$ saŋʊ̃ʊ̃
\end{xlist}
\end{exe}

In this environment, the vowels are  automatically nasalized, even when the
approximant of the perfective suffix intervenes. Rule \ref{PHO-n-harmony}
attempts to capture the process.

\begin{Rule}\label{PHO-n-harmony}{N-harmony}\\
A non-nasal vowel assimilates the nasal feature of a nasal segment, with or
without an intervening consonant. \\
 V $\rightarrow$  {\ob}{\sc +nasal}{\cb} /  {\ob}{\sc +nasal}{\cb} C$_0$  \_  
\end{Rule}




\paragraph{Vowel epenthesis and vowel reduction.}
\label{sec:epenthesis}


Vowel \isi{epenthesis} refers to the insertion of a vowel in specific phonological contexts. First,  the pronunciation of loan words is treated.\footnote{On  loan nouns in particular, see Section \ref{sec:GRM-borr-noun}. Section \ref{sec:PHO-weak-syll} touches upon a similar topic.}  Second,  the occurrences of the surface vowel [{ə}] are regarded as  either cases of vowel epenthesis or the reduction of  underlying vowels in specific environments. 

One should be careful in assuming that the insertion of  [{ə}] is phonologically-driven.  Take the case of loan words, particularly those ultimately coming  from  English. It is not clear whether the presence of [{ə}] in the Chakali word form [{bə̆̀lùù}] `blue'  is an example of vowel epenthesis, i.e.  ($<$ {\sls bluu}),  or perhaps a case of vowel reduction, i.e.  ($<$ {\sls buluu}).  On the one hand, the  consonant sequence  /bl/ is not attested, therefore  vowel epenthesis in an impossible consonant sequence  could  provide an explanation for the presence of  the vowel [{ə}].  On the other hand, given our knowledge of the sociolinguistic situation,  the majority of the English words used by Chakali speakers   were introduced by speakers of neighbouring languages. Thus it is more likely that a speaker borrows the form {\sls bəluu} -- with the schwa --  than without it. The latter scenario suggests that  [{ə}] in {\sls bəluu} does not come from vowel epenthesis produced by the phonology of Chakali, but perhaps from other phonologies.  Other examples of loan words from English are {\sls tə̆́rádʒà} `trouser' and  {\sls báátə̀rbɪ́ɪ́} `battery', to mention a couple. However if   [{ə}]  in {\sls bəluu} is rejected as a case of vowel epenthesis, `live' examples of borrowing  which are or have been nativized are needed.  

On a field trip, I was given a dog  and named it  `Taat', but the community members called him {\sls táátə̀} (see footnote \ref{fn:taat-epenthesis}). In this case the  vowel  [{ə}] could be treated as a true case of vowel epenthesis: alveolar plosives are prohibited in word-final position and the   vowel  [{ə}] is  inserted, which allows for the syllabification of the expression as CVV.CV, i.e. {\sls .taa.tə.}.  In general, it seems that vowel epenthesis in loan words should be treated case by case. Nonetheless there are good reasons to believe that Chakali uses vowel \isi{epenthesis} as a common strategy to allow the syllabification of  some phonological sequences (see Section \ref{sec:PHO-weak-syll}). 

\largerpage
\begin{Rule}\label{PHO-v-reduction}{Vowel reduction}\\
A vowel changes into a schwa in a weak syllable.\\
 V  $\rightarrow$ ə /  CV.C \_ .CV
\end{Rule}

\begin{Rule}\label{PHO-v-epenthesis}{Vowel epenthesis}\\
A schwa is inserted between a coda consonant and an onset consonant. \\
$\emptyset$  $\rightarrow$  ə   /  VC. \_  .CV    
\end{Rule}

In addition to its presence in loan words, the   vowel  [{ə}] is found  in cases of vowel reduction  and vowel \isi{epenthesis}  conditioned by the position of certain segments and syllabification procedures. A vowel reduction takes place when a vowel occurs in a weak syllable (Section \ref{sec:PHO-weak-syll}).  Also, as  mentioned above, vowel epenthesis can create proper sequences for syllabification.  In Table \ref{tab:vowel-reduction},  the first three examples are cases of vowel reduction, whereas the four at the bottom of the table are cases of vowel epenthesis. Rules \ref{PHO-v-reduction}  and \ref{PHO-v-epenthesis} account for the observed phenomena.\footnote{Rule \ref{PHO-v-epenthesis} overgenerates: an improvement would say that the less sonorant the flanking consonants are, the more likely the schwa is perceived.} 


\begin{table}[ht]

 
 \caption{Vowel reduction and epenthesis}
 \label{tab:vowel-reduction}

\begin{Qtabular}{llll}
\lsptoprule

 &  Underlying form &  Phonetic form & Gloss \\ \midrule
Vowel reduction &&&\\

&  bugulie		&	.bù.ɣə̀.líè.		&	big water pot
\\
& bifʊla &	bìfə̀lá &	baby\\
& mankir		&	.mán.kə̀rː. 	&   type of yam \\

\midrule
Epenthesis &&&\\
& maŋsa &  .má.ŋə́.sá.  & groundnuts\\
 & tʃɛrbʊa		&	.tʃɛ́.rə̀.bʊ̀á.	&	hip	\\
 & tʃagtʃag	&	.tʃá.ɣə́.tʃák.	&	tattoo	\\

 \lspbottomrule
 \end{Qtabular}
\end{table}




The words in Table \ref{tab:vowel-reduction} show  that it is either in the weak syllables, or in order to create a weak syllable (due to the adjacency of two consonants in the underlying form) that a vowel [{ə}] surfaces. The position of the vowel [{ə}] in the word {\sls mánkə̀rː} `type of yam'  is not consistent with the three others and its realization can only be explained by the presence of the trill in coda position, which may cause a vowel to lose the exclusive control of the nucleus of the syllable. However, in Chakali most of the yam names are borrowed.\footnote{The tone melody HL on disyllabic words is rare and typical of English loan words,  but, obviously, no yam appellations come from English.} 


This section gave an overview of why and how the surface vowel [{ə}] appears, and further established that whenever two stems meet to form a word, if the first ends with a consonant and the second begins with a consonant, i.e. VC$_{i}$][C$_{j}$V, the vowel [{ə}] is inserted between the two consonants. After syllabification the last consonant of the first morpheme becomes onset of a syllable and the vowel [{ə}] functions as the nucleus of that syllable, i.e. V]$_{\sigma}$[C$_{i}$ə]$_{\sigma}$[C$_{j}$V. 


\subsubsection{External sandhi}
External sandhi refers to processes found at word boundaries.  Two cases
of  assimilation are presented.

\paragraph{Nasal place assimilation.}
\label{sec:ext-nasal-place}

Nasal place assimilation at word boundaries occurs in the environment where the subject \isi{pronoun}
 {\sc 1.sg.wk} `I'  immediately precedes a verbal lexeme.  The  {\sc 1.sg.wk} 
\isi{pronoun} is represented by  /N/ in (\ref{ex:EXT-SAND-nasal-pl-ass-VERB}).

\begin{exe}
\ex\label{ex:EXT-SAND-nasal-pl-ass-VERB} 
\begin{xlist}
 \ex\label{ex:EXT-SAND-N} 
/N$]_{wb}$ kaalɪ sukuu {\sc foc}/ $\rightarrow$ {\rm [{\sls ŋ̩̀ káálɪ̀ sùkúù rō}] `I go to 
school'}
 \ex\label{ex:EXT-SAND-m} 
/N$]_{wb}$ buure-{\sc 3.sg} {\sc foc}/   $\rightarrow$  {\rm [{\sls m̩̀ búúrúú rō}]  `I love 
it'}
 \ex\label{ex:EXT-SAND-n} 
/N$]_{wb}$ sɔ nɪɪ {\sc foc}/   $\rightarrow$   {\rm [{\sls ǹ sɔ́  nɪ́ɪ́ rā}] `I'm 
bathing'}
\end{xlist}
\end{exe}

Moreover,  the same nasal place assimilation occurs in an environment where the
possessive \isi{pronoun}  immediately precedes a nominal lexeme. As in 
(\ref{ex:EXT-SAND-nasal-pl-ass-VERB})  /N/  stands for the  first person
singular
possessive \isi{pronoun}  in
(\ref{ex:EXT-SAND-nasal-pl-ass-NOM}). Rule \ref{PHO-n-assimln} of Section
\ref{sec:internal-sandhi-nasal-place}  describes both word-internal and
 -external nasal place assimilation.\footnote{The possessive pronouns 
are sometimes  lengthened (Section \ref{secːGRM-poss-pro}).}

\begin{exe}
\ex\label{ex:EXT-SAND-nasal-pl-ass-NOM} 
\begin{xlist}
 \ex\label{ex:EXT-SAND-N-nom} 
/N$]_{wb}$ gar/ $\rightarrow$ {\rm  [{\sls ŋ̩̀ gàŕ}]  `My cloth'}
 \ex\label{ex:EXT-SAND-m-nom} 
/N$]_{wb}$ par/ $\rightarrow$  {\rm [{\sls m̩̀ pár}]   `My hoe'}
 \ex\label{ex:EXT-SAND-n-nom} 
/N$]_{wb}$ ʔul/ $\rightarrow$  {\rm [{\sls ǹ̩ ʔúl}]  `My navel'}
\end{xlist}
\end{exe}



\paragraph{Focus particle's place assimilation and vowel harmony.}
\label{sec:focus-forms}

Focus encodes assertive information and has different forms in the language 
(Section \ref{sec:GRM-focus}). One of the forms is a  \isi{focus} particle which 
always follows a noun phrase. This particle is glossed as {\sc foc} and  represented as /{\it RV}/, in which R is an abstract consonant (the surface default is [{r}]) and V a vowel. The possible patterns responsible for the form of the \isi{focus} particle are listed in (\ref{ex:PHON-focu}).\footnote{Note that this 
is not a case of syntactic gemination since no underlying segments are doubled.}

\begin{exe}
\ex\label{ex:PHON-focu} 
%\begin{multicols}{2}
\begin{xlist}
\ex\label{ex:focus-ra} 
{\rm V[{\sc --atr}]  C[{\sc -lat}, {\sc -nas}]  $]_{wb}$ /RV/   $\rightarrow$ [ra] }\\ 
{\sls par ra} {\rm  `hoe {\sc foc}'}

\ex\label{ex:1} 
{\rm  V[{\sc --atr}]  C[{\sc +lat}]   $]_{wb}$ /RV/   $\rightarrow$ [la]} \\
{\sls tɪl la} {\rm  `gum {\sc foc}'}

\ex\label{ex:2}
{\rm V[{\sc --atr}] C[{\sc +nas}]   $]_{wb}$  /RV/   $\rightarrow$ [na]}\\
 {\sls tɔn na}   {\rm `skin {\sc foc}'}

\ex\label{ex:3} 
{\rm V[{\sc +atr}{\sc +ro}]  C[{\sc -lat}, {\sc -nas}] $]_{wb}$  /RV/  $\rightarrow$ [ro]} \\
 {\sls hog ro}   {\rm `bone {\sc foc}'}
\ex\label{ex:4} 
{\rm  V[{\sc +atr}{\sc +ro}]  C[{\sc +lat}] $]_{wb}$  /RV/   $\rightarrow$ [lo]} \\
 {\sls pul lo}   {\rm `river {\sc foc}'}
\ex\label{ex:5}
 {\rm V[{\sc +atr}{\sc +ro}]  C[{\sc +nas}] $]_{wb}$ /RV/   $\rightarrow$ [no]}\\
 {\sls lon no}   {\rm `calabash {\sc foc}'}

\ex\label{ex:6} 
{\rm  V[{\sc +atr}{\sc --ro}]  C[{\sc --lat}, {\sc -nas}] $]_{wb}$  /RV/$\rightarrow$ [re]} \\
 {\sls ger re}   {\rm `lizard {\sc foc}'}
\ex\label{ex:7} 
{\rm  V[{\sc +atr}{\sc --o}] C[{\sc +lat}]  $]_{wb}$  /RV/   $\rightarrow$ [le] }\\
 {\sls bil le}   {\rm `grave {\sc foc}'}
\ex\label{ex:8}
 {\rm V[{\sc +atr}{\sc --ro}] C[{\sc +nas}]   $]_{wb}$  /RV/   $\rightarrow$ [ne]} \\
 {\sls nen ne}   {\rm `arm {\sc foc}'}


\end{xlist}

\end{exe}


The patterns presented in (\ref{ex:PHON-focu}) are exhaustive. Taking (\ref{ex:focus-ra}) as an example, it should be read as follows: [ra] is the surface form of the \isi{focus} particle if the preceding vowel is   {\sc --atr} and the immediately preceding consonant is \{{\sc -lat(eral), -nas(al)}\}. The quality of the vowel is predicted by the \is{harmony rule} harmony rules of Section \ref{sec:vowel-harmony}.  When there is no immediately preceding consonant,  the surface consonant is [r], e.g. {\sls à tàà rá} `the language {\sc foc}', {\sls  à píí ré} `the yam mound {\sc foc}', and {\sls à kpólúŋkpōō rò} `the type of bird {\sc foc}'.   The surface consonant  [{w}] is sometimes found in  environments where  [{r}] is expected. An alternation[{w}] - [{r}] as onset of the \isi{focus} marker is presented in Section \ref{sec:GRM-morph-opro}.


\section{Suprasegmentals}
\label{sec:suprasegmentals}


At a word level,  nasalisation, tone patterns, and vowel harmony are phenomena which are treated as suprasegmentals. Nasalisation phenomena were discussed under sandhi processes.  In this section,  two suprasegmental aspects of language are treated: \is{tone}tone and \is{intonation}intonation, and vowel harmony.



\subsection{Tone and intonation}
\label{sec:tone-intonation}


Chakali is a tone language with both lexical and grammatical tone. Tones are distinctive pitch \is{variant}variations and are contrastive in the language since they can affect the meaning of  words/phrases, where the words/phrases consist of exactly the same segmental sequences.

Distinct tonal melodies at the lexical level provide evidence that a pitch distinction affects the meaning of words comprising identical sequences of segments.  An example of three different tonal melodies, using the minimal triplet, is   {\sls ŋmɛ́ná} `okro', {\sls ŋmɛ́nà} `to cut' and {\sls ŋmɛ̀ná}   `chisel'. The same can be said about tonal melodies at the phrasal level. Thus, the sentences {\sls ǹ̩ǹ̩ dí kʊ́ʊ́ rá} `I am eating t.z.' and {\sls ǹ̩ dí kʊ̄ʊ̄ rā} `I ate t.z.' are composed of the same sequence of segments (except the length of the \isi{pronoun} in subject function), but it is mainly  the tonal melody which distinguishes the former utterance from the latter.   Minimal examples involving \isi{intonation} are shown  in Section \ref{sec:GRM-trans-intran}.


Table \ref{tab:tone-sing-noun} displays  the tonal melodies of the singular noun category.  These are words uttered in isolation, so the tones are cut off from contextual influences. The subtables are divided according to the moraic content of the syllable. The  logical possibilities are accommodated with an example.



Based on the evidence of nominal paradigms,  two \is{tone} tones are suggested, i.e. high (H) and low (L). They are transcribed on segments with an acute and a grave  accent, respectively. Since tones are assigned to moras, light syllables can get a single tone, i.e. H or L. The heavy syllables may get high (H) or low (L), or either one of the contour tones, i.e. falling (HL) or rising (LH). A mid tone is often perceived, but no contrast is found  at the lexical level. Provisionally,  the mid tone is said to be a derived tone, that is, a raised low tone  or a lowered high tone. On rare occasions I  perceived a falling tone on the last vowel of a word, e.g. {\sls  bùgùnsô} `stupidity'.


\ili{Vagla}, \ili{Dɛg},  \ili{Tampulma},  \ili{Sisaala},  and  \ili{Pasaale} are all described with two tones (\citealt{Rowl65, Crou66, Gray69,  Toup95, Crou03})  One finds in this literature descriptions of two-tone systems and a considerable number of \isi{tone} rules. I am not going to delve in that area in detail, but among them,  a downstep rule lowers a high tone (i.e. {\T ꜜ}H)  when a low tone intervenes between two high tones, e.g. {\sls dʊ̃́ʊ̃̀} ({\sc sg.} HL), {\sls dʊ̃́{\T ꜜ}sá} ({\sc pl.} HLH).  This is however not consistently identified in the dictionary.



\largerpage[2]
Falling \isi{intonation} is a phrasal property by which a sequence of tones is cumulatively lowered; underlyingly though, the tones are either high or low. This gradual pitch fall may result in a low tone at the beginning of a phrase being as high as a high tone at the end of the phrase. Example (\ref{ex:PHO-downdrift}) illustrates the phenomenon. While the first line shows how the tones are perceived, the second line provides the lexical  tones normally associated with each of the words.\footnote{There is an important level of analysis lacking in this description in that there are no tone rules to account for phrasal and lexical intonations, so example (\ref{ex:PHO-downdrift}) must be interpreted with vigilance.}

\newpage 

\begin{table} 
\small
 \caption{Tonal patterns of singular nouns}
  \label{tab:tone-sing-noun}
\subfigure[One light syllable CVC: non-moraic coda]{
\begin{Qtabular}{p{.5cm}p{1.2cm}p{2.7cm}}
H& hóg	&	bone		\\
H& vʊ́g	&	small god\\	
L& bɔ̀g	&	  type of tree	\\	
\end{Qtabular}
}
\qquad
\subfigure[One heavy syllable CVC: moraic coda]{
\begin{Qtabular}{p{.5cm}p{1.4cm}p{2.5cm}}
H& 		kórː		&seat\\	
L &		sʊ̀lː		& dawadawa\\		
HL&		fʊ́l̀	& 	type of  climber \\
LH& 	pòĺ		 & pond	 	\\
\end{Qtabular}
}
\qquad
 \subfigure[One heavy syllable CVVC]{
\begin{Qtabular}{p{.5cm}p{1.2cm}p{2.7cm}}
H & fíél	&	type of grass\\
L & tʃʊ̀àr		& line	\\
HL & báàl	&	male	\\
LH & vàáŋ		& front leg	\\
\end{Qtabular}
}
\qquad
\subfigure[One heavy syllable CVV]{
\begin{Qtabular}{p{.5cm}p{1.4cm}p{2.5cm}}
H&	bíí		& seed  \\
L&	 zùù	&	type of  weather 	\\	
HL&	lɔ́ʊ̀	& 	hartebeest	\\
LH&	 bìé	&  	child		\\
\end{Qtabular}
}
\qquad
\subfigure[Two light syllables CVCV]{
\begin{Qtabular}{p{.5cm}p{1.2cm}p{2.4cm}}
 
H& bɪ́ná	& 	excrement	\\
L  & bɔ̀là		&elephant	\\
HL&	góŋò	 & 	type of  tree	\\
LH & bɪ̀ná & 		year\\	
\end{Qtabular}
}
\qquad
\subfigure[One heavy CVC: non-moraic coda, one light]{
\begin{Qtabular}{p{3mm}@{}p{.5cm}p{1.4cm}p{16mm}p{5mm}}
~&H& tʃéllé		& outlaw &~	\\		
&L & kpã̀nnà	&	lead\\
&HL& dántà	&	clan title\\		
&LH& kùksó	&	ribs	\\		
\end{Qtabular}
}
\qquad
\subfigure[One light CV, one heavy CVC]{
\begin{Qtabular}{p{.5cm}p{1.2cm}p{2.7cm}}
H & búzóŋ	&	bachelor	\\
HL&  bʊ́zál̀ː	&	type of bird\\
LH & kàtʃíg	&	type of  bird\\
\end{Qtabular}
}
\qquad
\subfigure[One heavy CVV, one light CV]{
\begin{Qtabular}{p{.5cm}p{1.4cm}p{2.5cm}}
 HHH & díésé & 		dream\\
HHL &  kpáásà	&	whip	\\
LHL & kùórù	&	chief	\\
LHH &	tùósó	&	added amount	\\
LLH &	fùòló	&	whistle	\\
LLL &	bʊ̀ɔ̀gà	&	moon	\\
\end{Qtabular}
}
 
\subfigure[Three light syllables CVCVCV]{
\begin{Qtabular}{lll}
HHH &  kásɪ́má	&	corpse uniform	\\ 
HHL &  bélégè	& drain	\\
 LHL &  dùlúgù  &	type of bird \\
LLH &  gɛ̀rɛ̀gá	&	sickness\\
 LLL &  dɪ̀gɪ̀nà	&	ear	\\
 LLH &  tʃɪ̀rɪ̀bɔ́		&gun firing pin	\\
LHH &  ʔàmʊ́nʊ́	& 	type of bush cat\\ 
HLL & dʊ́kpènì &  Royal python\\
\end{Qtabular}
}
\vspace*{-3mm}
\end{table}



\newpage 

 
\begin{exe}
\ex\label{ex:PHO-downdrift}

\glll {\Tten \Teight} {\Tseven \Tnine} {\Tfive \Tsix} {\Ttwo \Tfour} {\Tthree}\\
váà tʃʊ̀á dɪ̀á nʊ̃̀ã́ nɪ́ \\
dog lie house mouth {\sc postp} \\
\glt  `A dog lies at the entrance of a house.'
\end{exe}




Generally seen as a discourse function, Chakali has a falling final pitch at the end of polar question (see Section \ref{sec:GRM-interr-polar}). Final vowel lengthening is also perceived, but not consistently.  Falling final pitch is marked with a bottom tone diacritic on a vowel [v̏]. Rule \ref{PHO-polar-drop} describes the \isi{intonation}  of  polar questions (drop of pitch) by the addition of an extra-low tone.

\begin{Rule}\label{PHO-polar-drop}{Polar question drop}\\
An extra-low tone is added at the utterance-final boundary in polar question\\

\end{Rule}


\subsection{Vowel harmony}
\label{sec:vowel-harmony}

Vowel harmony is a process in  which all the vowels in a particular domain come to share one or more phonological feature(s).   This agreement  is triggered in specific phonological domains and  has a particular direction which is often treated as the spreading of one or more vowel feature(s).  In Section \ref{sec:vowels},  evidence was provided for the establishment of nine underlying vowels with five {\sc --atr} and four  {\sc +atr} vowels. This type of  vowel inventory has been referred to as  a five-height (5Ht) system \citep[308]{Casa03},  in which the feature {\sc atr} is contrastive within both the {\sc +hi} and {\sc [--hi, --lo]} vowels (see Table \ref{tab:featspec}).  \citet[81--82]{Daku97} and \citet[312]{Casa03} maintain that it is the most common inventory among \ili{Gur} and \ili{Kwa} languages. 

In Section \ref{sec:LOW-phon-vowel},  the
{\sc --atr} specification of the low vowel at the phonemic level was assumed
  on the basis of its behavior with the set of {\sc --atr} vowels. In
fact, the  realization of the low vowel in vowel harmony suggests that the set
of vowels specified as {\sc --atr}  contains the low vowel. To illustrate
the properties of vowel harmony, let us consider
how they function in  monosyllabic noun roots. Consider the data in
Table
\ref{tab:examples-harmony}.



\begin{table}


\caption{Vowel harmony in nouns\label{tab:examples-harmony}}
 \begin{tabular}{lllll}
\lsptoprule
Root vowel feature & Root &  Singular & Plural & Gloss\\ \midrule

{\sc [+atr, --hi, --lo, --ro]} &sel& sélː&sélé & animal\\
{\sc [+atr, +hi, --lo, --ro]} &bi &bíí &bíé& seed \\
{\sc [+atr, --lo, --ro]} &kie&  kìé 	&kìété	&half of a bird\\
{\sc [+atr, +hi, --lo, +ro]} &ʔul & ʔúl 	& ʔúló 	 
 & 	navel\\
{\sc [+atr, --hi, --lo, +ro]} &hol& hól & hóló & type of tree   \\
{\sc [+atr, --lo, +ro]} &buo& bùó 	& bùósó  &	funeral item\\
{\sc [--atr, +hi, --ro]} &bɪ& bɪ́ɪ́	&	bɪ́á 		&	stone\\
{\sc [--atr,  --hi, --lo, --ro]} &bɛl & bɛ̀ĺ &bɛ́llá & type of tree \\
{\sc [--atr, +hi, --lo, +ro]} & ɲʊg& ɲʊ́g & ɲʊ́gá & crocodile \\
{\sc [--atr,  --hi, --lo, +ro]} & hɔl& hɔ́l & hɔ́lá & piece of charcoal  \\
{\sc [--atr, --lo, +ro]} & bʊɔ& bʊ̀ɔ́	& bʊ̀ɔ̀sá	  &	hole\\
{\sc [--atr, +lo]} &vaa& váá  & vásá & dog \\
{\sc  [--atr, +lo]} &baal& báàl& báàlá& male \\

  \lspbottomrule
 \end{tabular}

\end{table}  

 Chakali is a language with noun classes (see Section \ref{sec:GRM-noun-classes}). A class is defined as a pair of singular and \isi{plural} suffixes associated with
a particular root. Table \ref{tab:examples-harmony} shows that only three vowels can occur in the \isi{plural} suffix position, i.e. [a],  [e],  and [o]. The distribution is such that  when the suffixes occur after a stem containing any member of the set \{ɪ, ɛ, ɔ, ʊ, a\},  they are realized as {\sls -a}.  The \isi{plural} suffix vowel {\sls -e} is realized when the root 
features are {\sc [+atr, --ro]}, whereas the \isi{plural} suffix vowel {\sls -o} is realized when 
the root features are {\sc [+atr, +ro]}.  Notice that the height feature(s) of a
vowel is irrelevant in all cases (see \cite{Stew79} for cross-height vowel harmony).  Rules 
\ref{RULE-nc-rule-1} and \ref{RULE-nc-rule-2} accommodate the surface forms of Table
\ref{tab:examples-harmony}.


\begin{Rule}\label{RULE-nc-rule-1}{\rm Noun classes realization (1)}\\
A noun class suffix vowel becomes {\sc +atr} if preceded by a {\sc +atr}
stem vowel, and shares the same value for the
feature {\sc ro}  as the one specified on the preceding stem vowel. \\
-V$_{nc}$  $\rightarrow$ [ $\beta${\sc ro},  {\sc +atr},  {\sc --hi}] 
 / [ 
$\beta${\sc ro},
{\sc +atr}] C$_0$ \_

\end{Rule}


\begin{Rule}\label{RULE-nc-rule-2}{\rm Noun classes realization (2)}\\
A noun class suffix vowel becomes {\sls -a} if the preceding stem vowel is 
{ɪ},
{ɛ}, {ɔ}, {ʊ} or {a}.\\
-V$_{nc}$ $\rightarrow$ {\sc +lo}  / {\sc --atr} C$_0$ \_ 
\end{Rule}


The same rules may be used to account for the vowel
quality of the \isi{focus} marker (Section \ref{sec:focus-forms}) and  the verbal
suffixes (Section \ref{sec:nasalisation-verb-suffix}). Yet, the rules need to be rewritten in order to be  applicable to wider domains and elements than those defined in their definition. Rules \ref{RULE-atr} and \ref{RULE-ro} break down Rules \ref{RULE-nc-rule-1} and \ref{RULE-nc-rule-2} into components able to be applied to other relevant domains.



\begin{Rule}\label{RULE-atr}{{\sc atr} harmony}\\
A vowel suffix agrees with the {\sc atr} value of   the preceding stem/word 
vowel (domains: noun classes, verbal suffixes, \isi{focus} marker).\\
V $\rightarrow$ $[\alpha${\sc atr}$]$  / $[\alpha${\sc atr}$]$ C$_0$ \_
\end{Rule}


\begin{Rule}\label{RULE-ro}{{\sc ro} harmony}\\
A vowel suffix  agree with the {\sc ro} value of  the   preceding stem/word
 vowel (domains: noun classes, verbal suffixes, \isi{focus} marker).\\
V $\rightarrow$ [$\alpha${\sc ro}]  / [$\alpha${\sc ro}] C$_0$ \_
\end{Rule}

Up to the present, the data suggest that the low vowel is excluded from 
co-occurring with {\sc +atr}  vowels.  So the prediction seems to be that if a 
word contains a {\sc +atr} vowel,  either the low vowel {/{a}/} cannot be 
realized and is thus changed by (one of) the above rules, or  the  low vowel is 
banned  altogether from the underlying form. Caution is necessary, however, since complex stem nouns (Section \ref{sec:GRM-com-stem-noun}) are attested containing both  low vowels and {\sc +atr} vowels, e.g. {\sls pàzèŋ́} ({\sls par-zeŋ}, {\sc hoe-big})  `big hoe'. Moreover, some multisyllabic words which cannot be treated as morphologically complex  due to their lack of morphological transparency do appear with both  a {\sc +atr} vowel and  the low vowel, e.g. {\sls dáárɪ́} `dig' vs.  {\sls dààrì}  `be half asleep'. When they do co-occur the general tendency is for a low 
vowel to precede any {\sc +atr} vowels in a word. 



Across phrase boundaries, when the \isi{postposition}  {\sls nɪ}  occurs between the \isi{focus} particle and the preceding nominal (see Section \ref{sec:SPA-postp} on the \isi{postposition} and Sections \ref{sec:GRM-foc-neg} and 
\ref{sec:GRM-focus} on the \isi{focus} particle),  the \isi{focus} particle's
vowel agrees with the vowel features of  the  preceding word despite the fact that the 
required adjacency is no longer satisfied (Section  \ref{sec:focus-forms}). This can be noticed  
especially in normal-speech rate and context.


\ea\label{ex:GRM-focus-form}

 
\ea\label{ex:GRM-foc-form-1}
\gll  à máŋkɪ́sɪ̀ ɲúú nī rò/rè.\\
    {\sc art} {match} {\reln} {\postp} {\foc}\\
\glt `on the top of the matchbox'

\ex\label{ex:GRM-foc-form-2}
\gll  à  pùl  ní rō/rē.\\
    {\sc art} {river} {\sc postp}  {\sc foc}\\
\glt `on/at the river'

\z 
 \z

In (\ref{ex:GRM-focus-form}), there is a retention of harmony across phrase boundaries, either because the \isi{postposition} becomes `transparent' and vowel-harmony can still operate (i.e. though not the place assimilation of  consonant) or because the high vowel of the \isi{postposition} itself acquires the relevant  vowel features of  the  preceding word. The fact that either {\sls ro} or {\sls re} can surface as \isi{focus} marker shows 
that i) the  {\sc ro} feature may be controlled by a non-adjacent word, and/or ii)   {\sc +atr} may be a vowel feature of the \isi{postposition}.\footnote{A more extreme case is found in example (\ref{ex:GRM-obl-obj-no-spa-foc}).}  Because it is more reduced, the quality of the high front vowel is difficult to hear at normal speech rate in that position, thus the distinction between the  \textsc{--atr} and   \textsc{+atr} versions is not always reflected in the 
transcription of the \isi{postposition}.




