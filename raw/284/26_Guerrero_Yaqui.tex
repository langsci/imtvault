\documentclass[output=paper]{langscibook}

\author{Lilian Guerrero\affiliation{Universidad Nacional Autónoma de México}}
\title{Reflexive constructions in Yaqui} 
\abstract{In Yaqui (Uto-Aztecan, Mexico), coreferential participants within the same clause can be expressed by reflexive pronouns or nonreflexive personal pronouns. Reflexive pronouns express agent-patient and agent-beneficiary coreference; when non-coreferential, the patient and the beneficiary take accusative case. Nonreflexive personal pronouns express the coreference between the agent and several other semantic roles (e.g., theme, interlocutor, recipient, source, location); when non-coreferential, these participants take oblique case. The agent-possessor coreference alternates: it is usually expressed by nonreflexive pronouns but, under certain circumstances, it is reflexive-marked. These patterns suggest that the use of reflexive pronouns in Yaqui is syntactically conditioned, i.e., reflexive pronouns cannot be combined with postpositions and cannot serve as adnominal modifiers.}

\IfFileExists{../localcommands.tex}{
  \addbibresource{localbibliography.bib}
  \input{../localpackages}
  %Copy this to localcommands.tex

\usepackage[english]{babel}
\usepackage{amsmath}
\usepackage{amssymb,amsfonts,textcomp}
\usepackage{array}
\usepackage{hhline}
\usepackage{hyperref}

\newenvironment{styleStandard}{}{}
\newenvironment{stylelsAbstract}{}{}
\newenvironment{stylelsSectioni}{}{}
\newenvironment{stylelsSectionii}{}{}
\newenvironment{stylelsBulletList}{}{}
\newenvironment{styleBibliographyi}{}{}
\newenvironment{listWWNumxxvleveli}{}{}
\newenvironment{listWWNumxxvlevelii}{}{}
\newenvironment{listWWNumxxvleveliii}{}{}
\newenvironment{listWWNumxxvleveliv}{}{}
\newenvironment{listWWNumixleveli}{}{}
\newenvironment{listWWNumixlevelii}{}{}
\newenvironment{listWWNumixleveliii}{}{}
\newenvironment{listWWNumixleveliv}{}{}

\newcommand\textstyleListLabelxvi[1]{#1}
\newcommand\labellistWWNumxxvleveli{\thelistWWNumxxvleveli.}
\newcommand\labellistWWNumxxvlevelii{\thelistWWNumxxvlevelii.}
\newcommand\labellistWWNumxxvleveliii{\thelistWWNumxxvleveliii.}
\newcommand\labellistWWNumxxvleveliv{\thelistWWNumxxvleveliv.}
\newcommand\labellistWWNumixleveli{[F0B7?]}
\newcommand\labellistWWNumixlevelii{\textstyleListLabelxvi{o}}
\newcommand\labellistWWNumixleveliii{[F0A7?]}
\newcommand\labellistWWNumixleveliv{[F0B7?]}

\newcounter{listWWNumxxvleveli}
\newcounter{listWWNumxxvlevelii}[listWWNumxxvleveli]
\newcounter{listWWNumxxvleveliii}[listWWNumxxvlevelii]
\newcounter{listWWNumxxvleveliv}[listWWNumxxvleveliii]
\newcounter{itemize}  
  \input{../localhyphenation} 
  \togglepaper[26]%%chapternumber
}{}

\begin{document}
\maketitle
\section{Introduction}\label{sec:guerrero:1}

%In Yaqui (Uto-Aztecan, Mexico), coreferential participants within the same clause can be expressed by reflexive pronouns or nonreflexive personal pronouns. Reflexive pronouns express agent-patient and agent-beneficiary coreference; when non-coreferential, the patient and the beneficiary take accusative case. Nonreflexive personal pronouns express the coreference between the agent and several other semantic roles (e.g., theme, interlocutor, recipient, source, location); when non-coreferential, these participants take oblique case. The agent-possessor coreference alternates: it is usually expressed by nonreflexive pronouns but, under certain circumstances, it is reflexive-marked. These patterns suggest that the use of reflexive pronouns in Yaqui is syntactically conditioned, i.e., reflexive pronouns cannot be combined with postpositions and cannot serve as adnominal modifiers.

It is a universal tendency that languages avoid using two or more coreferential full NPs within the same clause. As a result of this tendency, coreferential NPs can be marked in two different ways: one of the coreferential NPs may be replaced by a (reflexive) pronoun, or it may be deleted; in the latter case the verb may receive a special reflexive marking (\citealt{Kemmer1993, Kazenin2001, KoenigGast2008}; Haspelmath, this volume).\todo{KoenigGast2008 or 2006?} There are two ways to express coreferential participants in Yaqui: use of reflexive pronouns and use of nonreflexive personal pronouns. Reflexive pronouns are used when the agent is coreferential with the patient \REF{ex:guerrero:1a} or the beneficiary \REF{ex:guerrero:1b}. Nonreflexive pronouns are used when the agent is coreferential with the recipient \REF{ex:guerrero:1c} and other semantic roles. When taking non-coreferential NPs, \textit{a’ana} ‘dress’ takes an accusative patient, \textit{maka} ‘give’ takes an accusative beneficiary, and \textit{bittua} ‘send’ takes an oblique recipient. In Yaqui, oblique core arguments are marked by postpositions.



\ea%1
    \label{ex:guerrero:1}
\ea
\label{ex:guerrero:1a}
\gll Ino=ne     a’ana-n.\\
\textsc{1sg.refl}=\textsc{1sg.nom}   dress\textsc{{}-pastc}\\
\glt ‘I dressed myself.’

\ex
\label{ex:guerrero:1b}
\gll Joan-Ø   u-ka  toto’i-ta   emo   maka-k.\\
John-\textsc{nom} \textsc{det-acc}    hen\textsc{{}-acc} \textsc{refl}  give-\textsc{pfv}\\
\glt ‘John gave the hen to himself.’

\ex
\label{ex:guerrero:1c}
\gll Lupe-Ø   supem   a-u  bittua-k.\\
Lupe-\textsc{nom}  cloth\textsc{.pl}     \textsc{3sg.obl-dir}   send-\textsc{pfv}\\
\glt ‘Lupe sent clothes to him/her/it, to herself.’
\z
\z


Accordingly, coreferential participants in direct \REF{ex:guerrero:1a} and indirect \REF{ex:guerrero:1b} reflexive constructions are marked by reflexive pronouns, whereas coreferential participants in oblique reflexives \REF{ex:guerrero:1c} are expressed by nonreflexive personal pronouns. Strictly speaking, oblique reflexives are not reflexive constructions because there is no special form that signals the coreference (Haspelmath, this volume). Note that nonreflexive pronouns in \REF{ex:guerrero:1c} allow a disjoint reference interpretation. In this chapter, coreferential constructions without a special form are called nonreflexive constructions. The agent-possessor coreference is slightly more complex: it is usually expressed by nonreflexive pronouns unless the anaphoric pronoun occupies the object position, i.e., direct and indirect reflexive constructions. Based on these patterns, I propose that the use of reflexive pronouns in Yaqui is syntactically conditioned, i.e., reflexive pronouns cannot be combined with postpositions and cannot function as adnominal modifiers.



I begin this chapter by presenting some basic information about the Yaqui language. In \sectref{sec:guerrero:2}, I give a summary of the pronominal system, and briefly touch on reflexive coding in other Uto-Aztecan languages. In \sectref{sec:guerrero:3}, I present the analysis of direct, indirect, oblique and adpossessive reflexive domains. Then, I discuss some issues on middle voice (\sectref{sec:guerrero:4}), and coreferential NPs outside simple clauses (\sectref{sec:guerrero:5}). In \sectref{sec:guerrero:6}, I offer some conclusions.



\subsection{Yaqui and the Uto-Aztecan family}\label{sec:guerrero:1.1.}



Yaqui belongs to the Uto-Aztecan language family, one of the largest and most widespread language families in the Americas, with representative languages spoken from the western United States all the way to southern Mexico. Uto-Aztecan languages are classified into a southern branch and a northern branch. The southern branch includes Tepiman, Corachol, Nahuatl, and Taracahita languages; the last group includes Yaqui, Mayo, Guarijio and Tarahumara. Historically, Yaqui was spoken by the Yoeme people living along the Rio Yaqui, in Sonora, Mexico and, following the Mexican Revolution of 1920, a large group of Yaqui speakers settled in Arizona, United States. Today, there are fewer than 1,000 speakers in Arizona (\citealt{SimonsFennig2017}) and approximately 16,500 speakers in Sonora \citep{INEGI2010}, where Yaqui is spoken in several communities spread across eight towns (\figref{fig:guerrero:1}). The data analyzed in this chapter come from one of these Sonoran communities, Vicam, where Yaqui is spoken in daily life and taught in several bilingual elementary schools. By age six, most community members are bilingual speakers of Yaqui and Spanish.



\begin{figure}
\includegraphics[width=\textwidth]{figures/GuerreroYaqui-img001.png}
\caption{Yaqui communities (adapted from \citealt{Estrada2009}: 18)}
\label{fig:guerrero:1}
\end{figure}


\subsection{Basic morphosyntactic features of Yaqui}\label{sec:guerrero:1.2}



Yaqui is an agglutinating, dependent-marking, head-final, primary object language (\citealt{Lindenfeld1973}; \citealt{Escalante1990}; \citealt{DedrickCasad1999}; \citealt{Felix2000}; \citealt{Guerrero2006}). It is the only southern Uto-Aztecan language still spoken where case marking on nominals is preserved. Yaqui distinguishes between direct core arguments (marked by nominative and accusative case) and oblique core arguments (marked by postpositions). In nominals, the nominative is morphologically unmarked, and the accusative is marked by \textit{-ta} (\ref{ex:guerrero:2a}--\ref{ex:guerrero:2b}). There are some issues related to direct case marking that I would like to elaborate upon. First, the nominative and accusative affixes and the plural suffix \textit{{}-(i)m} are mutually exclusive, meaning plural arguments only take the plural suffix, as does \textit{ume o’ouim} ‘the men’ in \REF{ex:guerrero:2b}. Second, the accusative -\textit{ta} covers several grammatical functions, including the possessed noun inside genitive phrases in \REF{ex:guerrero:2c} and the nominal complement of some postpositions in \REF{ex:guerrero:2d} and \REF{ex:guerrero:3a} below.\footnote{ \textrm{See Guerrero (2019a,b) for a detailed discussion of direct and oblique core arguments, the syntactic functions of the suffix -}\textrm{\textit{ta}}\textrm{, and the use of postpositions as oblique case markers. There is also a set of nouns that are always plural, e.g.,} \textrm{\textit{supem}} \textrm{‘clothes’ \REF{ex:guerrero:1c}. In these cases, the plural suffix is not morphologically segmented.}}



\ea%2
    \label{ex:guerrero:2}
 \ea
 \label{ex:guerrero:2a}
\gll U-Ø    chu’u-Ø  batwe-u  bwite-k.\\
      \textsc{det-nom}  dog\textsc{{}-nom}   river-\textsc{dir}  run.\textsc{sg-pfv}\\
\glt ‘The dog ran to the river.’

\ex
\label{ex:guerrero:2b}
\gll U-me  o’ou-im  u-ka   chu’u-ta  bicha-k.\\
\textsc{det-pl}  man-\textsc{pl}  \textsc{det-acc}  dog-\textsc{acc}  see\textsc{{}-pfv}\\
\glt ‘The men saw the dog.’

\ex
\label{ex:guerrero:2c}
\gll Joan-ta  juubi-Ø  ne=bicha-k.\\
  John-\textsc{acc}  wife=\textsc{nom}  1\textsc{sg.acc=}see-\textsc{pfv}\\
\glt ‘John’s wife saw me.’

\ex
\label{ex:guerrero:2d}
\gll Lupe-Ø  bwa’a-m-ta   mabeta-k  u-e  kobanao-ta-betana.\\
    Lupe\textsc{{}-nom}   eat-\textsc{nmlz-acc}  receive-\textsc{pfv}  \textsc{det-obl}   government-\textsc{acc}{}-from\\
\glt ‘Lupe received food from the government.’
\z
\z



Postpositions such as the directional -\textit{u} ‘to’, the locatives -\textit{po} ‘in, on’ and -\textit{t} ‘at, on the top of, about’, and a few others mark oblique core arguments. In \REF{ex:guerrero:2d}, the third participant of a three-place predicate is marked by -\textit{betana} ‘from’. In ({ex:guerrero:3a}--{ex:guerrero:3b}), the second argument of two-place predicates take -\textit{u} ‘to’ and -\textit{t} ‘about’.  When present, determiners reflect the case marking of the head noun. Thus, they are unmarked when modifying a nominative NP \REF{ex:guerrero:2a}, take -\textit{ka} when modifying an accusative NP \REF{ex:guerrero:2b}, -\textit{me} when the NP is plural \REF{ex:guerrero:2b} and \REF{ex:guerrero:3a}, and -\textit{e} if the NP is marked by a postposition \REF{ex:guerrero:2d} and \REF{ex:guerrero:3a}. The absence of a determiner favors an indefinite reading of the NP, as \textit{bwa’am} ‘food’ in \REF{ex:guerrero:2d}. Clause-level word order is rigidly SOV, but other orders are possible, e.g., postverbal phrases.



\ea%3
    \label{ex:guerrero:3}

\ea
\label{ex:guerrero:3a}
\gll U-me   yoeme-m  u-e  jamut-ta-u    jina’ateo-Ø.\\
  \textsc{det-nom}  man\textsc{{}-nom}  \textsc{det-obl}  woman-\textsc{acc-dir}    complain-\textsc{prs}\\
\glt ‘The men are complaining with the woman.’

\ex
\label{ex:guerrero:3b}
\gll Jaibu=ne    ae-t    ju’unea-Ø.\\
    already=\textsc{1sg.nom}  \textsc{3sg.obl-locc}  know\textsc{{}-prs}\\
\glt ‘I already know about it.’
\z
\z



Verbs in Yaqui do not inflect for person or number, though a number of verbs have suppletive stems that show number agreement, as in \REF{ex:guerrero:2a}. There are few intransitive/transitive verb pairs coded by suppletion, e.g. \textit{uba/ubba} ‘take a bath/bathe someone’, and many verb pairs that morphologically distinguish between an intransitive form ending in -\textit{e, -te,} or \textit{{}-ke} and a transitive form ending in -\textit{a, -ta,} or \textit{{}-cha} (\citealt{DedrickCasad1999,Guerrero2004}). When the basic stem describes a change of state, the intransitive/transitive endings encode spontaneous/causative change of state distinction; these verbs have a stative counterpart ending in -\textit{i, -ti,} or \textit{{}-ia} that encodes a result state. The examples in \REF{ex:guerrero:4} show the three aspectual classes of the verb ‘break’. When the stem denotes an active predicate, the endings merely indicate syntactic transitivity, as in \textit{tubukte/tubukta} ‘jump/jump something’. It is not the case, however, that all verbs ending in -\textit{e} are intransitive and/or have a transitive counterpart, and vice versa, not all verbs ending in -\textit{a} must be transitive and/or have an intransitive counterpart.



\ea%4
    \label{ex:guerrero:4}
 \ea
 \label{ex:guerrero:4a}
\gll Empo    mesa-ta  kok-ta-k.\\
  2\textsc{sg}.\textsc{nom}  table-\textsc{acc}  break-\textsc{tr-pfv}\\
\glt ‘You broke the table.’

\ex
\label{ex:guerrero:4b}
\gll U-Ø  mesa-Ø  kok-te-k.\\
  \textsc{det-nom}  table-\textsc{nom}    break-\textsc{intr-pfv}\\
\glt ‘The table broke.’

\ex
\label{ex:guerrero:4c}
\gll U-Ø    mesa-Ø  kok-ti-Ø.\\
  \textsc{det-nom}  table-\textsc{nom}    break-\textsc{sta-prs}\\
\glt ‘The table is broken.’
\z
\z



Previous studies on Yaqui verbs have focused on valency-changing functions, e.g. valency and transitivity (Álvarez 2007; \citealt{EstradaEtAl2015}; \citealt{Tubino2017}), causative (\citealt{Guerrero2008}; \citealt{Tubino2011}), applicative \citep{Guerrero2007}, and passive \citep{Escalante1990}. These mechanisms are marked by verbal suffixes. For instance, the causative suffix -\textit{tua} adds a new (agent) argument to the verb; the example in \REF{ex:guerrero:5a} corresponds to the causative version of \REF{ex:guerrero:4a}. The suffix -\textit{wa} marks passive and impersonal clauses. Compare the active clause in \REF{ex:guerrero:4a} and the -\textit{wa} clauses below. In the passive version, the accusative object serves as the nominative subject \REF{ex:guerrero:5b}, whereas in the impersonal version, the object remains the same, i.e., an accusative object \REF{ex:guerrero:5c}. In \textit{{}-wa} clauses, the agent cannot be expressed syntactically.



\ea%5
    \label{ex:guerrero:5}
 \ea
 \label{ex:guerrero:5a}
\gll Inepo    mesa-ta  enchi    kok-ta-tua-k.\\
  \textsc{1sg.nom}    table-\textsc{acc}  2\textsc{sg.acc}    break-\textsc{tr-}cause\textsc{{}-pfv}\\
\glt ‘I made you break the table.’

\ex
\label{ex:guerrero:5b}
\gll Mesa-Ø  kok-ta-wa-k.\\
  table-\textsc{nom}  break-\textsc{tr-pass-pfv}\\
\glt ‘The table was broken.’

\ex
\label{ex:guerrero:5c}
\gll Mesa-ta  kok-ta-wa-k.\\
  table-\textsc{acc}  break-\textsc{tr-pass-pfv}\\
\glt ‘(Someone) broke the table.’
\z
\z



The expression of reflexives, reciprocals, and middles has been largely ignored in Yaqui grammar. Unlike applicative, causative, and passive constructions, they do not use verbal affixes, but instead use pronominal forms. Before I begin the discussion of these often overlooked constructions, a few words on the Yaqui pronominal system are needed.

\section{The pronominal system}\label{sec:guerrero:2}



\subsection{Personal pronouns}\label{sec:guerrero:2.1}



The Yaqui pronominal system formally distinguishes between nominative, accusative, oblique, possessive, and self-intensifier functions (\tabref{tab:guerrero:1}). Pronominal elements range in status from fully independent forms to clitics and affixes. Additionally, there are full and reduced pronouns. Full pronouns, such as \textit{inepo} ‘I’ in \REF{ex:guerrero:5a}, behave like lexical elements in terms of their distribution, while reduced nominative pronouns can behave like “second position” clitics, as in \REF{ex:guerrero:6a}. Occasionally, the two forms co-occur, especially for the first and second person \REF{ex:guerrero:6b}. Nominative third-person pronouns are commonly omitted, and reduced accusatives (available only for third person) tend to cliticize to the verb, as seen in \REF{ex:guerrero:2c} above. There is also a set of oblique pronouns used as complements of postpositions.


\begin{table}

\begin{tabularx}{0.9\textwidth}{p{1cm}p{1.7cm}p{1.7cm}p{1.3cm}p{1.5cm}p{1.5cm}}
\lsptoprule
& {Nominative}  & {Accusative}  & {Oblique} & {Possessive} & {Emphatic}\\
\hline
\textsc{1}\textsc{sg} & {inepo, ne} & {nee, ne} & {ne-} & {in, nim} & {inepola, inepela}\\
\textsc{2}\textsc{sg} & {empo, ’e} & {enchi} & {e-} & {em} & {empola, empela}\\
\textsc{3}\textsc{sg} & {aapo, Ø}  & {aapo’ik, a’a, a} & {a(e)-} & {aapo’ik, a, -wa} & {aapola, aapela}\\
\textsc{1}\textsc{pl} & {itepo, te} & {itom} & {ito-} & {itom} & {itepola, itopela}\\
\textsc{2}\textsc{pl} & {eme’e, ’em} & {enchim} & {emo-, eme-} & {em, enchim} & {emepola, emepela}\\
\textsc{3}\textsc{pl} & {bempo, Ø}  & {aapo’im, am}

{bempo’im} & {ame-} & {bem, bempo’im} & {bempola, bempela}\\
\lspbottomrule
\end{tabularx}
\caption{Yaqui pronominal system}
\label{tab:guerrero:1}
\end{table}


\ea%6
    \label{ex:guerrero:6}

\ea
\label{ex:guerrero:6a}
\gll Kuta-m   ili=ne  yeu=to-toja-n   kaa   bu-bu’u-m   juni’i.\\
  wood-\textsc{pl}  little=\textsc{1sg.nom}   out=\textsc{red}{}-take\textsc{-pstc} \textsc{neg}  \textsc{red-}a.lot\textsc{{}-pl} although \\
\glt ‘I took out wood, even if it was just little by little.’ (\citealt{Guerrero2019}; HVF: 93)

\ex
\label{ex:guerrero:6b}
\gll Empo=’e  kaa   ’aman   wee-’ean.\\
    \textsc{2sg.nom=2sg.nom}  \textsc{neg}   there   go-ought\\
\glt ‘You ought not go there.’ (\citealt{DedrickCasad1999}: 243)
\z
\z



In Yaqui, personal pronouns are necessarily referential, i.e., they cannot have a non-specific or generic interpretation. For instance, the direct object of \textit{bwa’e} ‘eat’ in \REF{ex:guerrero:7a} is \textit{tajkaim} ‘tortillas’; this NP can be substituted by the accusative pronoun \textit{am} (e.g. ‘they eat them’). In \REF{ex:guerrero:7b} the verb takes a non-specific object marked by the prefix \textit{ji’i}{}- ‘thing’, but \textit{ji’i-} cannot be replaced by an accusative pronoun \textit{a} (e.g. ‘they eat it’). Accusative and oblique pronouns are also obligatory when a core argument is extraposed to the right, as illustrated in \REF{ex:guerrero:7c}. In this context, the extraposed NP needs to be topical, as it encodes referents previously introduced in discourse, and it must also be a definite NP (\citealt{BelloroGuerrero2010}).



\ea%7
    \label{ex:guerrero:7}

\ea
\label{ex:guerrero:7a}
\gll Bempo   tajkaim   bwa’e-Ø.\\
  \textsc{3pl.nom}  tortilla.\textsc{pl}  eat-\textsc{prs}\\
\glt ‘They eat tortillas.’

\ex
\label{ex:guerrero:7b}
\gll Bempo   ji’i-bwa’e-Ø.\\
  \textsc{3pl.nom}  thing-eat-\textsc{prs}\\
\glt ‘They eat something.’

\ex
\label{ex:guerrero:7c}
\gll Aapo  jiba    a\textsubscript{=}bitchu-k,  u-ka  jamut-ta.\\
    \textsc{3sg.nom}  always   \textsc{3sg.acc}=watch-\textsc{pfv}  \textsc{det-acc}  woman-\textsc{acc}\\
   \glt   ‘He watched her all the time, the woman.’  (\citealt{SilvaEtAl1998}; 2:26)
\z
\z



\subsection{Reflexive pronouns}\label{sec:guerrero:2.2}


As shown in \tabref{tab:guerrero:2}, the paradigm of reflexive pronouns in Yaqui varies according to different descriptions of the language. The first column shows the paradigm proposed by \citet[246]{DedrickCasad1999}. Note that all persons have their own reflexive form except the second and third person plural, which are both coded by \textit{’emo}. The second column presents the reflexive pronouns listed by \citet[32]{Estrada2009}. In her paradigm, \textit{emo} also expresses the second person singular and serves as an alternative coding for the third person singular. As shown in \REF{ex:guerrero:8}, reflexive pronouns behave like full pronouns, e.g., they are free forms and occupy the object position (pre-verbally). Yaqui does not allow reflexive pronouns in subject function.


\begin{table}
\begin{tabularx}
{0.9\textwidth}{p{1cm}p{3cm}p{1.5cm}p{2.5cm}p{1.3cm}}
\lsptoprule
& \citet{DedrickCasad1999} & \citet{Estrada2009} & {Field Notes} (since 1997) & \citet{Buelna1890} \\
\hline
\textsc{1}\textsc{sg} & {’ino} & {ino} & {ino, emo, omo} & {inone}\\
\textsc{2}\textsc{sg} & {’emp} & {emo} & {emo, omo} & {emore}\\
\textsc{3}\textsc{sg} & {’au} & {au, emo} & {au, emo, omo} & {auo}\\
\textsc{1}\textsc{pl} & {’ito} & {ito} & {ito, emo, omo} & {itote}\\
\textsc{2}\textsc{pl} & {’emo} & {emo} & {emo, omo} & {emorem}\\
\textsc{3}\textsc{pl} & {’emo} & {emo} & {emo, omo} & {emorim}\\
\lspbottomrule
\end{tabularx}
%%please move \begin{table} just above \begin{tabular
\caption{Yaqui reflexive pronouns}
\label{tab:guerrero:2}
\end{table}


\ea%8
    \label{ex:guerrero:8}

\ea
\label{ex:guerrero:8a}
\gll Hunama   beha   ’au  ko’okoi-su-ka   ’au   ine’e-te-k.\\
  there   well  \textsc{refl}  get.sick-\textsc{cmpl-ptcp}   \textsc{refl}   feel-\textsc{intr-pfv}\\
\glt ‘Well, after having fallen sick, she recovered.’ (\citealt{DedrickCasad1999}: 246)

\ex
\label{ex:guerrero:8b}
\gll Juan-Ø  batwe-u  emo  himaa-k.\\
  John-\textsc{nom}  river-\textsc{dir}  \textsc{refl}  throw-\textsc{pfv}\\
\glt ‘John threw himself into the river.’ \citep[129]{Estrada2009}
\z
\z



The third column shows the reflexive pronouns I have found in the field. From the examples in (9a-b), it is clear that the reflexive pronoun \textit{emo} has extended to all grammatical persons. I also found that, for some young speakers, \textit{emo} alternates with \textit{omo}, as illustrated in \REF{ex:guerrero:9c}.



\ea%9
    \label{ex:guerrero:9}

\ea
\label{ex:guerrero:9a}
\gll Kuta-e=ne emo beeba-k.\\
stick-with=\textsc{1sg.nom}  \textsc{refl}  hit-\textsc{pfv}\\
\glt ‘I hit myself with the stick.’

\ex
\label{ex:guerrero:9b}
\gll Empo lautia emo supe-tua-Ø.\\
  \textsc{2sg.nom}  quick  \textsc{refl}  dress-cause-\textsc{prs}\\
\glt ‘You get dress yourself very quickly’

\ex
\label{ex:guerrero:9c}
\gll Wa’a-Ø  ili  jamut-Ø  si  yolisia  omo  chichike-Ø.\\
  \textsc{dem-nom}  little  woman-\textsc{nom}  \textsc{int}  pretty  \textsc{refl}      brush-\textsc{prs}\\
\glt ‘That girl brushes herself very prettily.’
\z
\z



Therefore, the reflexive pronouns \textit{ino}, \textit{au}, and \textit{ito} can be called personal reflexive pronouns since they vary according to the person of the subject. Since \textit{emo {\textasciitilde} omo} can co-refer with any person, it can be considered a general reflexive pronoun ‘self’. Apparently, there are no differences in use between personal reflexive pronouns and the ‘self’ form. It is important to distinguish the reflexive pronoun \textit{au} ‘himself/herself/itself’ in \REF{ex:guerrero:8a} from the homophonous oblique \textit{a-u} ‘to him/her/it’ in \REF{ex:guerrero:10a}. First, the reflexive \textit{au} cannot be split morphologically, and thus cannot take a plural form to indicate a plural referent, though the oblique pronoun can, \REF{ex:guerrero:10b}. Second, reflexive \textit{au} cannot combine with case markers and postpositions, while the oblique pronoun is the base for all postpositions. And third, several Yaqui verbs take oblique arguments marked by the directional postposition -\textit{u} (\citealt{Guerrero2019a, Guerrero2019b}). However, most of these verbs do not accept reflexive readings. In \REF{ex:guerrero:10} the participants are non-coreferential; the intended reflexive reading for \REF{ex:guerrero:10a} is ungrammatical because, according to my consultants, ‘it does not make any sense to talk to oneself’.



\ea%10
    \label{ex:guerrero:10}

\ea
\label{ex:guerrero:10a}
\gll Peo-Ø   a-u  nooka-k.\\
  Peter-\textsc{nom}  \textsc{3sg.obl-dir}  talk\textsc{{}-pfv}\\
\glt ‘Peter talked to him/her/it, *to himself.’

\ex
\label{ex:guerrero:10b}
\gll Inepo  ame-u  wat-te-k.\\
  \textsc{1sg.nom}  \textsc{3pl.obl-dir}  miss-\textsc{intr-pfv}\\
\glt ‘I missed them.’
\z
\z



\subsection{Historical notes on reflexive pronouns}\label{sec:guerrero:2.3}


\citet[47]{Langacker1977} claims that “innovation, loss, and modifications of reflexive pronouns is an exceedingly complex subject in the Uto-Aztecan grammar”. For Proto-Uto-Aztecan, Langacker reconstructs the reciprocal verbal prefix *na-, and the reflexive verbal prefixes *ni- ‘myself’, *ta- ‘ourselves’, *i{}- ‘yourselves’, and *mo- for all other persons. The reflexive prefixes have been lost in all northern languages; hence the reciprocal prefix indicates both senses. In some southern languages, reflexive pronouns may cover both functions.



There are no known historical documents on Yaqui that permit us to trace the evolution of its reflexive forms, though there is a grammatical sketch of Cahita \citep{Buelna1890}, a linguistic ancestor of Yaqui and two related languages, Mayo and Tehueco (now extinct). In Buelna’s sketch of Cahita, reflexive pronouns (\tabref{tab:guerrero:2}, last column) include \textit{inone} ‘myself’\textit{, emore} ‘yourself’\textit{, auo} ‘him/herself/itself’\textit{, itote} ‘ourselves’\textit{, emorem} ‘yourselves’, \textit{emorim} ‘themselves’; see the example in \REF{ex:guerrero:11}.



\ea%11
    \label{ex:guerrero:11}
   \gll Emore  mahau-tua.\\
  \textsc{2sg.refl}  scare-cause\\
\glt ‘You make yourself scare.’ \citep[53]{Buelna1890}
\z



Except for their endings, Cahita and Yaqui reflexive pronouns look remarkably similar. In fact, one can see the diachronic evolution of the reflexive verb prefix *mo- in Proto-Uto-Aztecan (used for second and third person singular and third person plural) to the reflexive pronoun \textit{emo {\textasciitilde} omo} in Yaqui (now used for all persons). It is also worth noting that, within the Taracahita group, Yaqui is the only language that has both personal reflexive pronouns (\textit{ino, au, ito}) and a general reflexive form (\textit{emo {\textasciitilde} omo}). The Tarahumara languages only make use of two general reflexive pronouns, e.g. \textit{binóipi} for singular and \textit{abóipi} for plural \citep{Caballero2002}. Guarijio has no distinct reflexive pronouns, but coreferential NPs are coded by anaphoric non-nominative personal pronouns \citep{Felix2005}.



\subsection{Self-intensifier pronouns}\label{sec:guerrero:2.4}



\citet{Buelna1890} also lists two sets of emphatic pronouns in Cahita. The first group ends in -\textit{riua} or \textit{{}-e}, as in \textit{empe} for the second person singular \REF{ex:guerrero:12a}. The second group ends in -\textit{(e)la,} as in \textit{empola} ‘you alone, by yourself’. The second pronominal set is preserved in Yaqui (fifth column, \tabref{tab:guerrero:1}) and in \REF{ex:guerrero:12b}. Whereas \citet[53-54]{Buelna1890} calls these forms ‘semi-pronouns’, \citet[243-244]{DedrickCasad1999} call them “emphatic reflexive subject pronouns”.



\ea%12
    \label{ex:guerrero:12}

\ea
\label{ex:guerrero:12a}
\gll Empe   aman   sim-naque.\\
  \textsc{2sg.emph}  there   go.\textsc{sg}{}-want\\
\glt ‘You (by yourself) will go there.’  \citep[53]{Buelna1890}

\ex
\label{ex:guerrero:12b}
\gll ’aapela  ’am   kooba-k.\\
    \textsc{3sg.emph}     \textsc{3sg.acc}   win-\textsc{pfv}\\
\glt ‘He beat them all by himself.’ (\citealt{DedrickCasad1999}: 244)
\z
\z



These pronominal forms do not trigger a reflexive meaning, but they function as self-intensifiers (\citealt{Koenig2001}; Haspelmath, this volume). They can occur by themselves \REF{ex:guerrero:13a}, be adjacent to the coreferential NP \REF{ex:guerrero:13b}, or co-occur with the general reflexive ‘self’ \REF{ex:guerrero:13c}. When translated into Spanish, these structures generally correspond to the adverbial \textit{solo} ‘alone’.



\ea%13
\label{ex:guerrero:13}

\ea
\label{ex:guerrero:13a}
\gll Inepola Potam-meu-bicha bwite-k.\\
  \textsc{1sg.emph}  Potam.\textsc{pl-dir.pl-}towards  run.\textsc{sg-pfv}\\
\glt ‘I ran towards Potam by myself.’

\ex
\label{ex:guerrero:13b}
\gll U-Ø   kora-Ø   aapela   weche-k.\\
  \textsc{det-nom}  corral-\textsc{nom}  \textsc{3sg.emph}  fall.\textsc{sg-pfv}\\
\glt ‘The corral fell down by itself.’

\ex
\label{ex:guerrero:13c}
\gll Inepo=ne  kaa  enchi  beba-k, empola emo   beba-k.\\
  \textsc{1sg.nom=1sg.nom}   \textsc{neg}   \textsc{2sg.acc}   hit-\textsc{pfv}   \textsc{2sg.emph}  \textsc{refl}  hit-\textsc{pfv}\\
\glt ‘I didn’t hit you, you hit yourself.’
\z
\z

\section{Yaqui reflexive constructions}\label{sec:guerrero:3}


“Reflexive” is a cover term that has, at least, two senses: it may refer to the coreference between two participants in a minimal clause, and/or it may refer to the forms that signal coreference (\citealt{Kemmer1993, FrajzyngierCurl1999, KoenigGast2008, Creissels2016}). In \REF{ex:guerrero:14a}, the accusative clitic signals a disjoint-reference between the agent and the patient; in \REF{ex:guerrero:14b} the agent and the patient are the same person, hence there must be a reflexive pronoun in object position. In the present description, semantic roles like agent, patient, and recipient are used in a broad sense.\footnote{The use of semantic roles instead of terms like subject, object, and indirect object in this chapter is purposeful. While the terms subject and object may be unproblematic, the term ‘indirect object’ is inadequate in Yaqui grammar for two three main reasons \citep{Guerrero2019a}. (i) Even though some authors have considered -\textit{u} to be a dative, indirect marker \citep{Estrada2009}, \textit{-u} is one among several postpositions marking oblique arguments (recall the examples in \REF{ex:guerrero:3}); (ii) -\textit{u} can introduces several semantic roles not necessarily related to dative arguments (e.g., source); (iii) the coding of the third participant in three-place predicates varies: it can take accusative, and it can be marked by -\textit{u} or by other postpositions \citep{GuerreroVanValin2004}. The use of semantic roles avoids one having to use multiple syntactic terms for this function (e.g., indirect object, primary object, directional object, locative object).}



\ea%14
  \label{ex:guerrero:14}

\ea
\label{ex:guerrero:14a}
\gll U-Ø maejto-Ø si Peo-ta uttia-Ø\\
  \textsc{det-nom} teacher-\textsc{nom}  \textsc{int} Peter-\textsc{acc}  admire-\textsc{prs}\\
\glt ‘The teacher admires Peter a lot.’

\ex
\label{ex:guerrero:14b}
\gll U-Ø maejto-Ø si omo uttia-Ø\\
  \textsc{det-nom}  teacher-\textsc{nom}  \textsc{int}  \textsc{refl}   admire-\textsc{prs}\\
\glt ‘The teacher admires himself a lot.’
\z
\z

In what follows, reflexive constructions with reciprocal meaning (\sectref{sec:guerrero:3.1}), direct (\sectref{sec:guerrero:3.2}), indirect (\sectref{sec:guerrero:3.3}) oblique (\sectref{sec:guerrero:3.4}) and adpossessive reflexive domains (\sectref{sec:guerrero:3.5}) are first discussed, followed by middle voice (\sectref{sec:guerrero:4}), and coreferential NPs in complex constructions (\sectref{sec:guerrero:5}).

\subsection{Reflexive constructions with reciprocal meanings}\label{sec:guerrero:3.1}

Yaqui reflexive pronouns allow a reciprocal reading when the antecedent (coreferential agent) is plural. The construction in \REF{ex:guerrero:15a} is ambiguous: it can mean ‘they lick themselves’ or ‘they lick each other’. In \REF{ex:guerrero:15b}, the combination of the reflexive and the adverbial \textit{nau} ‘together’ highlights the reciprocal interpretation.\footnote{Most likely,\textit{nau} is related to the reciprocal verbal prefix *na- reconstructed for Proto-Uto-Aztecan \citep{Langacker1977}. However, the adverbial \textit{nau}is not limited to reciprocal meanings in Yaqui.} The reciprocal meaning is not limited to the form \textit{emo}, as confirmed by \REF{ex:guerrero:15c} with the 1PL reflexive pronoun.




\ea%15
\label{ex:guerrero:15}

\ea
\label{ex:guerrero:15a}
\gll U-me  ili  miisi-m   emo  te’ebwa-Ø\\
    \textsc{det-pl}  little  cat-\textsc{pl}    \textsc{refl}    lick-\textsc{prs}\\
\glt ‘The kittens are licking themselves/each other.’

\ex
\label{ex:guerrero:15b}
\gll U-me    ili   miisi-m   nau  emo  te’ebwa-Ø\\
    \textsc{det-pl}   little  cat-\textsc{pl}      together  \textsc{refl}  lick-\textsc{prs}\\
\glt ‘The kittens are licking each other.’

\ex
\label{ex:guerrero:15c}
\gll Pues   nanancha  te  ito   ania-taite-k\\
    well    equally  \textsc{1pl.nom}  \textsc{1pl.refl}  help-start-\textsc{pfv}\\
\glt ‘So, both of us started to help ourselves/each other.’ (\citealt{Guerrero2019c}; HVF: 371)
\z
\z

\subsection{Direct reflexive constructions}\label{sec:guerrero:3.2}



Cross-linguistically, the most common pattern of coreferential participants involves two-place predicates, with the agent as the antecedent and the patient as the anaphoric form. This coreferential pattern exemplifies the ‘autopathic domain’ (Haspelmath, this volume) or, more simply put, direct reflexives (\citealt{Kemmer1993}: 41; \citealt{Kazenin2001}: 918). In \REF{ex:guerrero:16a}, \textit{bicha} ‘see’ takes a non-coreferential agent and patient, hence there is an accusative NP; in \REF{ex:guerrero:16b} the two participants are coreferential and there is reflexive pronoun in object position.



\ea%16
    \label{ex:guerrero:16}

\ea
\label{ex:guerrero:16a}
\gll U-Ø   ili   jamut-Ø   Peo-ta   bicha-k.\\
    \textsc{det-nom}  little  woman-\textsc{nom}  Peter-\textsc{acc}  see\textsc{{}-pfv}\\
\glt ‘The girl saw Peter.’

\ex
\label{ex:guerrero:16b}
\gll U-Ø   ili   jamut-Ø  ejpeeko-po  emo   bichu-k.\\
    \textsc{det-nom}  little  woman-\textsc{nom}  mirror-\textsc{loc}  \textsc{refl}  see.\textsc{cmpl-pfv}\\
\glt ‘The girl saw herself in the mirror.’
\z
\z

Reflexive pronouns satisfy the syntactic valency of transitive verbs. Compare the intransitive-transitive verb pairs in \REF{ex:guerrero:17}. The transitive form \textit{omta} ‘hate’ takes a non-coreferential NP in \REF{ex:guerrero:17a} and a reflexive pronoun when the agent is coreferential with the patient in \REF{ex:guerrero:17b}; the intransitive counterpart \textit{omte} disallows the occurrence of the reflexive pronoun \REF{ex:guerrero:17c}.



\ea%17
   \label{ex:guerrero:17}

\ea[]{
\label{ex:guerrero:17a}
\gll Joan-Ø  Peo-ta   om-ta-Ø.\\
  John-\textsc{nom}  Peter-\textsc{acc}  hate-\textsc{tr}{}-\textsc{prs}   \\
\glt ‘John hates Peter.’
}

\ex[]{
\label{ex:guerrero:17b}
\gll Joan-Ø   au   om-ta-Ø.\\
  John-\textsc{nom}  \textsc{3sg.refl}  hate-\textsc{tr}{}-\textsc{prs}     \\
\glt ‘John hates himself.’
}

\ex[*]{
\label{ex:guerrero:17c}
\gll Joan-Ø   au   om-te-Ø.\\
  John-\textsc{nom}  \textsc{3sg.refl}  hate-\textsc{intr}{}-\textsc{prs}   \\
\glt ‘John hates himself.’
}
\z
\z



The suppletive transitive verb \textit{me’a} ‘kill’ takes a non-coreferential anaphoric pronoun in \REF{ex:guerrero:18a}, and a reflexive pronoun in \REF{ex:guerrero:18b}. Again, the intransitive form \textit{muuke} ‘die’ in \REF{ex:guerrero:18c} disallows reflexive pronouns. It means that, within the autopathic domain, reflexive pronouns combine with the morphologically marked transitive verb form.



\ea%18
    \label{ex:guerrero:18}

\ea[]{
\label{ex:guerrero:18a}
\gll Joan-Ø   a=me’a-k.\\
  John-\textsc{nom}  \textsc{3sg.acc}=kill.\textsc{sg-pfv}\\
\glt ‘John killed him/her/it.’
}

\ex[]{
\label{ex:guerrero:18b}
\gll Juan-Ø   omo  me’a-k.\\
  John-\textsc{nom}  \textsc{refl} kill.\textsc{sg-pfv}\\
\glt ‘John killed himself.’
}

\ex[*]{
\label{ex:guerrero:18c}
\gll Juan-Ø   omo  muuke-k.\\
  John-\textsc{nom}  \textsc{refl}  die.\textsc{sg-pfv}\\
\glt ‘John killed himself.’
}
\z
\z

\subsection{Indirect reflexive constructions}\label{sec:guerrero:3.3}



The expression of indirect reflexives, that is, the coreference of the agent with a participant other than the patient (recipient, goal, beneficiary) has received little focus in the literature (\citealt{Kemmer1993}; \citealt{Kazenin2001}: 918). There are two types of indirect reflexives in Yaqui and both involve the beneficiary. The first type includes a few three-place predicates. For example, the verb \textit{maka} ‘give’ takes an accusative theme and an accusative beneficiary in \REF{ex:guerrero:19a}. When the agent is coreferential with the beneficiary as in \REF{ex:guerrero:19b}, there is a reflexive pronoun. In addition to \textit{emo}, one of my consultants also made use of the nominative personal pronoun as a reinforcement element. In \REF{ex:guerrero:19c}, the agent and the beneficiary of \textit{majta} ‘teach’ are the same person.



\ea%19
    \label{ex:guerrero:19}

\ea
\label{ex:guerrero:19a}
\gll Juana-Ø  mo’obei-ta   Lupe-ta   maka-k.\\
    Juana-\textsc{nom}  hat\textsc{-acc} Lupe\textsc{{}-acc}  give-\textsc{pfv}\\
\glt ‘Juana gave Lupe a hat.’
 
 \ex
 \label{ex:guerrero:19b}
\gll Juana-Ø  (aapo)   mo’obei-ta  omo   maka-k.\\
    Juana-\textsc{nom}  \textsc{3sg.nom}  hat\textsc{{}-acc}    \textsc{refl}  give-\textsc{pfv}\\
\glt ‘Juana gave a hat to herself.’

\ex
\label{ex:guerrero:19c}
\gll Aapo     jiak-nok-ta   emo   majta-siime-Ø  in   pamiiliam-mak\\
    \textsc{3sg.nom}  yori-talk-\textsc{acc}  \textsc{refl}   teach-go.\textsc{sg-prs}  \textsc{1sg.poss}  family.\textsc{pl}{}-with\\
\glt ‘She tries to teach herself Yaqui with my family.’ (\citealt{Buitimea2007}; pueplou: 106)
\z
\z



The second and most common type of indirect reflexive construction involves applicative constructions. In Yaqui, the applicative suffix -\textit{ria} combines with stative, intransitive, and transitive verbs; when associated with transitive verbs, it adds a new (applied) argument with the role of beneficiary. Compare (\ref{ex:guerrero:20a}--\ref{ex:guerrero:20b}). In the non-derived clause, the beneficiary is coded as an adjunct marked by the postposition \textit{betchi’ibo} ‘for’; in the applicative counterpart, the same participant is coded as an accusative NP. In (\ref{ex:guerrero:20c}--\ref{ex:guerrero:20d}) the agent and the beneficiary are coreferential; in the non-derived version, the coreferential NP is coded as an oblique pronoun, while in the applicative version, the reflexive pronoun serves as the applied argument. An additional example is presented in \REF{ex:guerrero:20e}.



\ea%20
    \label{ex:guerrero:20}

\ea
\label{ex:guerrero:20a}
\gll Kari-ta=ne   jinu-k Maria-ta-betchi’ibo.\\
  house\textsc{{}-acc}=\textsc{1sg.nom} buy-\textsc{pfv}   Mary-\textsc{acc}{}-for\\
  \glt   ‘I bought a house for Mary.’

\ex
\label{ex:guerrero:20b}
\gll Kari-ta=ne    Maria-ta  jinu-ria-k.\\
  house\textsc{{}-acc}=\textsc{1sg.nom}  Mary-\textsc{acc}   buy-\textsc{appl-pfv}\\
\glt ‘I bought Mary a house.’

\ex
\label{ex:guerrero:20c}
\gll Empo kari-ta jinu-k  e-betchi’ibo.\\
    \textsc{2sg.nom} house\textsc{{}-acc}  buy-\textsc{pfv} \textsc{2sg.obl}{}-for        \\
\glt ‘You bought a house for yourself.’

\ex
\label{ex:guerrero:20d}
\gll Empo kari-ta emo jinu-ria-k.\\
    \textsc{2sg.nom} house\textsc{{}-acc}  \textsc{refl} buy-\textsc{appl-pfv}    \\
\glt ‘You bought yourself a house.’

\ex
\label{ex:guerrero:20e}
\gll Komo=ne   jaibu  ju’unea   ISSSTE-po  bea=ne  ino   nok-ria-ne.\\
  like=\textsc{1sg.nom}  already   know   ISSSTE-\textsc{loc} \textsc{dm=1sg.nom}   \textsc{1sg.refl}   talk-\textsc{appl-pot}\\
\glt ‘Since I was already familiar with ISSSTE, I could defend myself.’
(Guerrero ms; HVL: 201)
\z
\z



As pointed out by \citet[4]{ZunigaKittilä2010}, while some languages ban agents from being beneficiaries in the same clause, others may use a special construction in these cases, i.e., self-benefactives. Yaqui is a good example of a language that makes use of applicative self-benefactive constructions.


\subsection{Oblique nonreflexive constructions}\label{sec:guerrero:3.4}



As mentioned previously, adjuncts and oblique core arguments are marked by postpositions. When the complement of a postposition is pronominal, it must take the form of an oblique pronoun. However, reflexive pronouns do not combine with postpositions. In \REF{ex:guerrero:20c} above, the pronominal complement of \textit{betchi’ibo} ‘for’ is \textit{e-} ‘for you’, instead of the reflexive form \textit{emo}. In the examples below, the agent is coreferential with the theme \REF{ex:guerrero:21a} and interlocutor \REF{ex:guerrero:21b} of speech act verbs, the recipient \REF{ex:guerrero:21c}, as well as the location \REF{ex:guerrero:21d}. In all these cases, there is an anaphoric personal pronoun. When the participant refers to the third person, the construction is ambiguous; both coreferential and non-coreferential readings are possible. In \REF{ex:guerrero:21b}, the nonreferential oblique pronoun \textit{ae} can refer to Mary, Lupe, or someone else.



\ea%21
    \label{ex:guerrero:21}

\ea
\label{ex:guerrero:21a}
\gll Fermin-Ø     ae-t       nooka-k.\\
  Fermin-\textsc{nom}  \textsc{3sg.obl-loc}    talk-\textsc{pfv}\\
\glt ‘Fermin talked about him/her/it, about himself.’

\ex
\label{ex:guerrero:21b}
\gll Maria-Ø   Lupe-ta-mak   ae-betana   etejo-k.\\
  Mary\textsc{{}-nom}  Lupe-\textsc{acc-}with    \textsc{3sg.obl-}from  tell-\textsc{pfv}\\
\glt ‘Mary talked with Lupe about her/him/it, about Mary, about Lupe.’

\ex
\label{ex:guerrero:21c}
\gll Inepo  ne-u     ji’i-jioste-bae-Ø\\
    \textsc{1sg.nom}  \textsc{1sg.obl-dir}  thing-write-want-\textsc{prs} \\
\glt ‘I want to write something to myself.’

\ex
\label{ex:guerrero:21d}
\gll U-Ø   amureo-Ø  maso-ta  ae-bicha-po   bicha-k.\\
    \textsc{det-Ø}  hunter-\textsc{nom}  deer-\textsc{acc}   \textsc{3sg.obl}{}-toward\textsc{{}-loc}  see-\textsc{pfv}\\
\glt ‘The hunter saw a deer in front of him/her/it, in front of himself.’
\z
\z



The examples below illustrate agent-goal \REF{ex:guerrero:22a} and agent-source (\ref{ex:guerrero:20b}--\ref{ex:guerrero:20c}) coreference in three-place predicates. Note that the nonreflexive personal pronoun can be implicit \REF{ex:guerrero:22c}. According to my consultants, an implicit goal or source favors a coreferential reading.



\ea%22
    \label{ex:guerrero:22}

\ea
\label{ex:guerrero:22a}
\gll U-Ø  jamut-Ø  mo’obei-ta  ea-t  yecha-k.\\
      \textsc{det-nom}  woman-\textsc{nom}  hat-\textsc{acc} \textsc{3sg.obl-loc}  put\textsc{.sg}{}-\textsc{pfv}\\
\glt ‘The woman put a hat on her/him/it, on herself.’

\ex
\label{ex:guerrero:22b}
\gll U-Ø   jamut-Ø  relo-ta    a-u  u’ura-k.\\
    \textsc{det-nom}  woman-\textsc{nom}  watch-\textsc{acc}    \textsc{3sg.obl-dir}   take\textsc{{}-pfv}\\
\glt ‘The woman took the watch off him/her/it [the arm], off herself.’

\ex
\label{ex:guerrero:22c}
\gll U-Ø   jamut-Ø  lentem  u’ura-k.\\
    \textsc{det-nom}  woman-\textsc{nom}  glasses    take\textsc{{}-pfv}\\
\glt ‘The woman took off the glasses.’
\z
\z



\subsection{Adpossessive nonreflexive constructions}\label{sec:guerrero:3.5}



In some languages, reflexive pronouns can combine with possessive pronouns to show agent-possessor coreference (Haspelmath, this volume)\todo{this volume}. As shown in \tabref{tab:guerrero:1} above, Yaqui has a set of possessive pronouns. When the agent refers to the first or second person, the corresponding first or second person possessive forms are used; see the example in \REF{ex:guerrero:23a}. When the agent refers to the third person, there are three coding options: the possessive suffix -\textit{wa} \REF{ex:guerrero:23b}, the possessive pronoun \textit{a} and -\textit{wa} \REF{ex:guerrero:23c}, and a genitive phrase \REF{ex:guerrero:23d}. Even though the most likely reading of (\ref{ex:guerrero:23b}--\ref{ex:guerrero:23c}) is coreference, a disjoint-reference interpretation is also possible. The explicit use of a genitive phrase leads to a disjoint-reference reading. The same referential ambiguity prevails with an alienable possessee as in \REF{ex:guerrero:23e}. Note that possessive NPs in object position optionally take the accusative suffix -\textit{ta}; genitive phrases disallow a second suffix -\textit{ta}.



\ea%23
    \label{ex:guerrero:23}

\ea
\label{ex:guerrero:23a}
\gll Inepo    nim     soa(-ta)  ubba-k.\\
    \textsc{1sg.nom}  \textsc{1sg.poss}  son-\textsc{acc}  bath.\textsc{tr}{}-\textsc{pfv}\\
\glt ‘I bathed my son.’

\ex
\label{ex:guerrero:23b}
\gll Lupe-Ø    asoa-wa(-ta)  ubba-k.\\
    Lupe-\textsc{nom}  son-\textsc{poss-acc}  bath.\textsc{tr-pfv}\\
\glt ‘Lupe bathed her/his son.’

\ex
\label{ex:guerrero:23c}
\gll Lupe-Ø    a     asoa-wa(-ta)   ubba-k.\\
    Lupe-\textsc{nom}  \textsc{3sg.poss}  son-\textsc{poss-acc}  bath.\textsc{tr-pfv}\\
\glt ‘Lupe bathed her/his son.’

\ex
\label{ex:guerrero:23d}
\gll Lupe-Ø    Maria-ta   a   soa  ubba-k.\\
    Lupe-\textsc{nom}  Mary\textsc{{}-acc}  \textsc{3sg.poss}  son  bath.\textsc{tr-pfv}\\
\glt ‘Lupe bathed Mary’s son.’ (lit. bathed Mary’s her son)

\ex
\label{ex:guerrero:23e}
\gll Joan-Ø tekile-u a karro-wa-po  siika.\\
    John-\textsc{nom}  work\textsc{{}-dir}  \textsc{3sg.poss} car-\textsc{poss-loc}  go.\textsc{sg.pfv}\\
\glt ‘John went to work on his own car.’
\z
\z



When the possessee is a body part, the use of possessive pronouns is complex, and this is true of both coreferential and non-coreferential participants \citep{Guerrero2020} \todo{missing reference - 2020?}. The clause in \REF{ex:guerrero:24a} was rejected by two of my consultants and was considered odd by a third one. In this context, there are two coding options: the body part is unpossessed and keeps the accusative case \REF{ex:guerrero:24b}, or it is unpossessed and is marked by locative postpositions \REF{ex:guerrero:24c}. The former results in referential ambiguity, while the latter bears a coreferential sense. With disjoint-reference, an external possessive construction is also possible \REF{ex:guerrero:24d}.



\ea%24
    \label{ex:guerrero:24}

\ea[\#]{
\gll  Joan-Ø a koba-(ta)  beba-k.\\
  John-\textsc{nom}  \textsc{3sg.poss}   head-\textsc{acc}  hit-\textsc{pfv}\\
\glt ‘John hit his head.’ (=John’s head or someone’s else) \label{ex:guerrero:24a}
}

\ex[]{
\label{ex:guerrero:24b}
\gll Joan-Ø    koba-ta    beba-k.\\
  John-\textsc{nom}  head-\textsc{acc}  hit-\textsc{pfv}\\
\glt ‘John hit his head.’ (=John’s head or someone’s else)
}

\ex[]{
\label{ex:guerrero:24c}
\gll Joan-Ø    koba-po    beba-k.\\
John-\textsc{nom}  head-\textsc{loc}  hit-\textsc{pfv}\\
\glt ‘John hit his head.’ (lit. hit on head) (=John’s head)
}

\ex[]{
\label{ex:guerrero:24d}
\gll Joan-Ø  koba-t    enchi  beba-k.\\
 John-\textsc{nom}  head-\textsc{loc}  \textsc{2sg.acc}  hit-\textsc{pfv}\\
\glt ‘John hit you on the head.’
}
\z
\z



The examples in \REF{ex:guerrero:23} and \REF{ex:guerrero:24} confirm that agent-possessor coreference does not use reflexive pronouns in Yaqui. The clause in \REF{ex:guerrero:25a} is ruled out because there is a reflexive pronoun serving as a possessive pronoun; \REF{ex:guerrero:25b} is also ruled out because there is an accusative NP and a reflexive pronoun in the same clause. The presence of an overt possessive pronoun with a reflexive form would also be ruled out, e.g., \textit{a omo.}



\ea%25
    \label{ex:guerrero:25}

\ea[*]{
\label{ex:guerrero:25a}
\gll Joan-Ø    omo  koba-ta  beba-k.\\
  John-\textsc{nom}  \textsc{refl}  head-\textsc{acc}  hit-\textsc{pfv}\\
\glt ‘John hit his (own) head.’
}

\ex[*]{
\label{ex:guerrero:25b}
\gll  Joan-Ø koba-ta omo beba-k. \\
John-\textsc{nom}  head-\textsc{loc}  \textsc{refl}  hit-\textsc{pfv}\\
\glt ‘John hit himself on the head.’
}
\z
\z



Nevertheless, there are two contexts in which adpossessive coreference might be expressed by reflexive pronouns. In the first context, the possessee is coded as an oblique (locative) argument and the anaphoric reflexive pronoun occupies the object position; the reflexive counterpart of \REF{ex:guerrero:24c} is illustrated in \REF{ex:guerrero:26a}. In the second context, the possessor is introduced as an applied argument within an applicative construction; compare (\ref{ex:guerrero:26a}--\ref{ex:guerrero:26a}). The first option corresponds to direct reflexives, and the second to indirect reflexives.



\ea%26
    \label{ex:guerrero:26}
\ea
\label{ex:guerrero:26a}
\gll Joan-Ø    koba-po    omo  beba-k.\\
John-\textsc{nom}  head-\textsc{loc}  \textsc{refl}  hit-\textsc{pfv}\\
\glt ‘John hit his head.’ (lit. hit himself on head)

\ex
\label{ex:guerrero:26b}
\gll U-Ø  ili   jamut-Ø  pujba-ta     baksia-k.\\
  \textsc{det-nom}    little  woman-\textsc{nom}  face-\textsc{acc}  wash-\textsc{pfv}\\
\glt ‘The girl is washing her face.’

\ex
\label{ex:guerrero:26c}
\gll U-Ø  ili   jamut-Ø  pujba-ta   au  baksia-ria-k.\\
  \textsc{det-nom}    little  woman-\textsc{nom}  face-\textsc{acc}  \textsc{3sg.refl}  wash-\textsc{appl-pfv}\\
\glt ‘The girl is washing her face.’ (lit. washing herself the face)
\z
\z


The discussion on coreferential oblique and possessive participants suggests that it is not the semantic role but its syntactic function that determines whether or not a reflexive pronoun is used in Yaqui, i.e., reflexive pronouns cannot be complements of postpositions and cannot be associated with adnominal possession. The use of nonreflexive personal pronouns in these domains oscillates between coreference readings and disjoint interpretations. The actual interpretation depends on the linguistic context and/or discourse-pragmatic information.



\section{Reflexive pronouns and middle situations}\label{sec:guerrero:4}



In middle situations, the agent participant is viewed as the doer of the action as well as the place on which this action is performed; the doer and the place are construed as one and the same entity (\citealt{Kemmer1993}; \citealt{Creissels2006}). In Yaqui, several middle situations are expressed by a reflexive + transitive verb combination, but many others are expressed by non-reflexive-marked intransitive clauses. Grooming verbs that can combine with reflexive pronouns include \textit{baksia} ‘wash’, \textit{bekta} ‘shave’, \textit{a’ana} ‘dress (formal ceremonies)’, \textit{supetua} ‘put on clothes’, and \textit{chichike} ‘comb’. These verbs can take a non-coreferential NP as well as a reflexive pronoun in object position; compare the uses of \textit{baksia} as ‘wash something’ in \REF{ex:guerrero:26a} and ‘wash something on oneself’ in \REF{ex:guerrero:26b} above, and ‘wash oneself’ in \REF{ex:guerrero:27a} below. The examples in (\ref{ex:guerrero:27b}--\ref{ex:guerrero:27c}) show \textit{bekta} ‘shave’, and (\ref{ex:guerrero:27d}--\ref{ex:guerrero:27e}) illustrate \textit{a’ana} ‘dress’.



\ea%27
    \label{ex:guerrero:27}

\ea
\label{ex:guerrero:27a}
\gll Joan-Ø  emo  baksia-Ø.\\
  John-\textsc{nom}  \textsc{refl}  wash\textsc{{}-prs}\\
\glt ‘John washes himself.’

\ex
\label{ex:guerrero:27b}
\gll Joan-ta=ne    bekta-k.\\
  John-\textsc{acc=1sg.nom}  shave\textsc{{}-prs}\\
\glt ‘I shave John.’

\ex
\label{ex:guerrero:27c}
\gll Joan-Ø    chau-t   omo   bekta-k.\\
  John-\textsc{nom}  beard-\textsc{loc}  \textsc{refl}   shave\textsc{{}-prs}\\
\glt ‘John shaved his beard.’ (lit. shave himself on the beard)

\ex
\label{ex:guerrero:27d}
\gll Lupe-Ø    ne   a’ana-n.\\
  Lupe-\textsc{nom}  \textsc{1sg.acc}  dress\textsc{{}-pastc}\\
\glt ‘Lupe dressed me.’

\ex
\label{ex:guerrero:27e}
\gll Ino=ne a’ana-n.\\
  \textsc{1sg.refl=1sg.nom}  dress\textsc{{}-pastc}\\
\glt ‘I dressed.’
\z
\z



In opposition, grooming verbs like \textit{baima} ‘wash hands’, \textit{baju’urina} ‘wash face’, \textit{uba} ‘bathe’, and \textit{tajo’ote} ‘dress (everyday clothing)’ are not reflexive-marked. See the use of \textit{baima} in \REF{ex:guerrero:28a}. The last two verbs have a transitive counterpart, but reflexive pronouns are banned in this context. Contrast \textit{ubba} ‘bathe someone’ in \REF{ex:guerrero:23} above, with the intransitive version \textit{uba} ‘bathe oneself’ in \REF{ex:guerrero:28b}. The verb pair \textit{tajo’ota/tajo’ote} ‘dress someone/oneself’ is illustrated in (\ref{ex:guerrero:28c}--\ref{ex:guerrero:28d}). The clause in \REF{ex:guerrero:28e} is ruled out because \textit{tajo’ota} combines with a reflexive pronoun.



\ea%28
    \label{ex:guerrero:28}

\ea[]{
\label{ex:guerrero:28a}
\gll U-Ø  ili   yoeme-Ø  baima-Ø.\\
  \textsc{det-nom}  little  man-\textsc{nom}  wash\_hands-\textsc{prs}\\
\glt ‘The little boy washes hands.’
}

\ex[]{
\label{ex:guerrero:28b}
\gll Joan-Ø     batwe-po   uba-Ø.\\
  John-\textsc{nom}  river-\textsc{loc}  bath\textsc{.intr-prs}\\
\glt ‘John bathes in the river.’
}

\ex[]{
\label{ex:guerrero:28c}
\gll Maria-Ø   enchi  tajo’o-ta-Ø.\\
  Mary-\textsc{nom}  2\textsc{sg.acc}  dress\textsc{{}-tr-prs}\\
\glt ‘Mary dresses you.’
}

\ex[]{
\label{ex:guerrero:28d}
\gll Empo     chumti   tajo’o-te-Ø.\\
  \textsc{2sg.nom}    quickly  dress\textsc{{}-intr-prs}\\
\glt ‘You dress quickly.’
}

\ex[*]{
\label{ex:guerrero:28e}
\gll Empo  chumti   emo  tajo’o-ta-Ø.\\
  \textsc{2sg.nom}  quickly  \textsc{refl}  dress\textsc{-tr-prs}\\
\glt ‘You dress yourself quickly.’
}
\z
\z



In Yaqui, the expression of body-part actions does not necessarily differ from whole-part actions. The reflexive-marked \textit{baksia} ‘wash’ can target a body-part action in (\ref{ex:guerrero:26b}--\ref{ex:guerrero:26c}) and \REF{ex:guerrero:27a}, but \textit{baima} ‘wash hands’ and \textit{baju’urina} ‘wash face’ are not reflexive-marked. Dressing verbs can be understood as whole-body activities, but \textit{a’ana} combines with reflexive forms and \textit{tajo’ote} does not. In addition, a few body-function action verbs can be used with a reflexive pronoun or not, depending on the degree of affectedness (\citealt{FrajzyngierCurl1999}). This is the case of \textit{siise} ‘urinate’, \textit{bwita} ‘defecate’, and \textit{pocho’okunte} ‘defecate (outside, in the woods)’. With the reflexive form \REF{ex:guerrero:29a}, the action is assumed to be an accident; without the reflexive \REF{ex:guerrero:29b}, a regular activity is implied. Verbs like \textit{ko’okoi} ‘be/get sick’ and \textit{ine’ete} ‘recover’ in \REF{ex:guerrero:8a}, \textit{elpeiya/peiya} ‘feel/get better’ in \REF{ex:guerrero:29c}, \textit{i’a} ‘be/get spoiled’, and \textit{mammatte} ‘understand’ are also reflexive-marked.



\ea%29
    \label{ex:guerrero:29}

\ea
\label{ex:guerrero:29a}
\gll U-Ø   ili  uusi-Ø  emo  siise-k.\\
  \textsc{det-nom}    little   child-\textsc{nom}   \textsc{refl}   urinate-\textsc{pfv}\\
\glt ‘The child urinated on himself.’

\ex
\label{ex:guerrero:29b}
\gll U-Ø   ili  uusi-Ø  siise-k.\\
  \textsc{det-nom}  little   child-\textsc{nom}   urinate-\textsc{pfv}\\
\glt ‘The child urinated.’

\ex
\label{ex:guerrero:29c}
\gll Into=bea  a   waiwa-Ø    jaibu  ili   emo  pa-p-peiya-n.\\
  \textsc{dm=dm}  \textsc{3sg.poss}   sister-\textsc{nom}   already  little  \textsc{refl}   \textsc{red-red}{}-get.better-\textsc{pastc}\\
\glt ‘And then her sister was getting a little better already.’ (\citealt{Buitimea2007}; ili baro: 70)
\z
\z



Non-translational motion and body-posture verbs are mostly unmarked, e.g., \textit{yehte} ‘stand’ in \REF{ex:guerrero:30a} and \textit{bwalsapte} ‘stretch’ in \REF{ex:guerrero:30b}. The exceptions I have found so far include \textit{cha’a} ‘hang’ in \REF{ex:guerrero:30a} and \textit{yooa} ‘tremble’ in \REF{ex:guerrero:30c} which are reflexive-marked.



\ea%30
    \label{ex:guerrero:30}

\ea
\label{ex:guerrero:30a}
\gll Au  kom=cha’a-tu-k   u-Ø     buuru-Ø   ’aman   jika-t   yehte-k.\\
  \textsc{3sg.refl}  down=hang-\textsc{vblz-pfv}   \textsc{det-nom}   donkey-\textsc{nom}  there   up-\textsc{loc}   stand.\textsc{sg-pfv}\\
\glt ‘The donkey bent down and stood up.’ (\citealt{Johnson1962}; burro\& coyote: 34)

\ex
\label{ex:guerrero:30b}
\gll Aapo   bwalsap-te-Ø.\\
    \textsc{3sg.nom}  stretch-\textsc{intr}{}-\textsc{pfv}     \\
\glt ‘He is stretching.’

\ex
\label{ex:guerrero:30c}
\gll Mejiko-po   u-Ø   bwia-Ø   jiba   au   yooa-Ø.\\
  Mexico-\textsc{loc}  \textsc{det-nom}  earth-\textsc{nom}  always   \textsc{3sg.refl}   tremble-\textsc{prs}\\
\glt ‘In Mexico, the earth always trembles.’
\z
\z



Spontaneous change of state verbs are not reflexive-marked. Compare the intransitive-transitive verb pair in (31a-b). However, there are a few verbs that combine with reflexive forms: \textit{eta} ‘close’, \textit{etapo} ‘open’, \textit{esso} ‘hide’, \textit{ta’aru} ‘lose’, \textit{jiima} ‘throw’, \textit{piarora} ‘borrow’. Compare (31c-d). In this context, \textit{emo} functions as a kind of anti-causative marker, i.e. it does not imply any potential agent.



\ea%31
    \label{ex:guerrero:31}

\ea[]{
\label{ex:guerrero:31a}
\gll U-Ø  ba’am    poj-te-k.\\
    \textsc{det-nom}  water.\textsc{pl}  boil-\textsc{intr}{}-\textsc{pfv}\\
\glt ‘The water boiled.’
}

\ex[*]{
\label{ex:guerrero:31b}
\gll U-Ø  ba’am    omo  poj-ta-k.\\
    \textsc{det-nom}  water.\textsc{pl}  \textsc{refl}  boil-\textsc{tr}{}-\textsc{pfv}\\
\glt ‘The water boiled.’
}

\ex[]{
\label{ex:guerrero:31c}
\gll U-Ø  jeeka-Ø  u-ka   pueta-ta   etapo-k.\\
  \textsc{det-nom}  wind\textsc{{}-nom}  \textsc{det-acc}  door\textsc{{}-acc} open\textsc{{}-pfv}\\
\glt ‘The wind opened the door.’
}

\ex[]{
\label{ex:guerrero:31d}
\gll U-Ø     pueta-Ø  emo   etapo-k.\\
  \textsc{det-nom}  door\textsc{{}-nom}  \textsc{refl}  open\textsc{{}-pfv}\\
\glt ‘The door opened.’
}

\ex[]{
\label{ex:guerrero:31e}
\gll U-Ø   tomi-Ø   boosa-po   kateka-me   emo   ta’aru-k.\\
  \textsc{det-nom}   money-\textsc{nom}   purse\textsc{{}-loc} sit.\textsc{sg.pfv-nmlz}   \textsc{refl}   lose\textsc{{}-pfv}\\
\glt ‘The money that was in the purse got lost.’
}
\z
\z



Two things appear to be clear at this point: (i) not all morphologically-marked transitive verbs combine with reflexive pronouns, and (ii) the use of reflexive pronouns as middle markers is unpredictable (i.e., lexically determined). The lack of productivity of Yaqui reflexive forms in middle situations contrasts not only with Romance, Germanic, and Slavic languages (see Janic, this volume, for Polish), but also with other Southern Uto-Aztecan languages. In Southern Tepehuan \citep{Garcia2005}, for example, the use of reflexive pronouns as middle markers is very productive; Pima Bajo \citep{Estrada2005} uses the third-person non-subject as a middle marker; Wixárika \citep{Ramos2017} uses the reflexive third-person prefix \textit{yu}{}- as signs middle functions.



\section{Coreferential participants within complex clauses}\label{sec:guerrero:5}



Thus far, I have focused on examples of two coreferential NPs within the same clause. However, two participants can also be coreferential within complex constructions. When the main subject and the dependent subject are coreferential, the coding of the anaphoric pronoun depends on the clause linkage type \citep{Guerrero2006}: some linkage types demand an implicit participant \REF{ex:guerrero:32a}, others require a possessive \REF{ex:guerrero:32b} or accusative \REF{ex:guerrero:32c} anaphoric pronoun, and a few allow anaphoric nominative pronouns \REF{ex:guerrero:32d}. The accusative and the nominative pronouns lead to referential ambiguity.



\ea%32
    \label{ex:guerrero:32}

\ea
\label{ex:guerrero:32a}
\gll Nim achai-Ø ju’une’ea-k [loteria-ta       yo’o-kai].\\
  \textsc{1sg.poss}  father-\textsc{nom}  know-\textsc{pfv} lottery-\textsc{acc}   win-\textsc{clm}\\
\glt ‘My father\textsubscript{i} knew he\textsubscript{i} had won the lottery.’ (=my father won the lottery)

\ex
\label{ex:guerrero:32b}
\gll Ne\textsubscript{i} a-u\textsubscript{j} wawate-n  [nim\textsubscript{i} enchi  ji’i-beje-tua-ne-’u]\textsubscript{j}.\\
  1\textsc{sg.nom}  \textsc{3sg.obl-dir}  remember-\textsc{pastc}  1\textsc{sg.poss}  2\textsc{sg.acc}  thing\textsc{{}-}cost-cause\textsc{{}-pot-clm}\\
 \glt  ‘I didn’t remember (it) to pay you.’



\ex
\label{ex:guerrero:32c}
\gll Jorge-Ø\textsubscript{i} a-beas\textsubscript{j} kopte-k           [taream    a\textsubscript{i} ya’a-ne-po]\textsubscript{j}.\\
Jorge-\textsc{nom} 3\textsc{sg.obl-}about  forget-\textsc{pfv} homework.\textsc{pl}  3\textsc{sg.acc} make-\textsc{pot-clm}\\
\glt ‘Jorge forgot about doing the homework.’

\ex
\label{ex:guerrero:32d}
\gll Peo-Ø\textsubscript{i} Vicam-meu  siika    [bweituk  aapo\textsubscript{i/j} kaba’i-ta  jinu-n].\\
    Peter-\textsc{nom}    Vicam.\textsc{pl}{}-\textsc{dir.pl}  go\textsc{.sg.pfv}   \textsc{clm}  3\textsc{sg.nom}    horse-\textsc{acc}  buy-\textsc{pastc}\\
    \glt  ‘Peter\textsubscript{i} went to Vicam because he\textsubscript{i/j} bought a horse.’
\z
\z



There are two mental verbs that seem to allow a reflexive pronoun when the two subjects are the same person: \textit{{}-machia} ‘believe’ and \textit{’ea} ‘think’. The examples in \REF{ex:guerrero:33} resemble long-distance reflexives; in \REF{ex:guerrero:33a}, the presence of the reflexive seems optional, but not in (\ref{ex:guerrero:33b}--\ref{ex:guerrero:33c}).



\ea%33
    \label{ex:guerrero:33}

\ea
\label{ex:guerrero:33a}
\gll Ne  (ino)  tui  kaba’i-ta    jinu-maachia-Ø.\\
  \textsc{1sg.nom}  \textsc{1sg.refl}  good  horse-\textsc{acc}    buy-believe\textsc{{}-prs}\\
\glt ‘I believe I would buy a good horse.’

\ex
\label{ex:guerrero:33b}
\gll Aapa  [lautia  emo  siim-bae-benasia]  ’ea-Ø.\\
  3\textsc{sg.nom}  early      \textsc{refl}  go-want\textsc{{}-clm} think-\textsc{prs}\\
\glt ‘She\textsubscript{i} has the feeling that she\textsubscript{i} wants to go early.’

\ex
\label{ex:guerrero:33c}
\gll Nim  ae\textsubscript{i}  tuisi  omo\textsubscript{i} ye’e-t-‘ea-Ø  bweta  ka  luturia.\\
  \textsc{1sg.poss}  mother  good  \textsc{refl}  dance-\textsc{clm-}think-\textsc{prs}  but  \textsc{neg}  true\\
\glt ‘My mother\textsubscript{i} thinks she\textsubscript{i} dances pretty well, but it is not true!’
\z
\z



\section{Conclusion}\label{sec:guerrero:6}



As evidenced in this chapter, Yaqui reflexive pronouns signal agent-patient and agent-beneficiary coreferential participants, but they cannot express the coreference between the agent and the recipient, source, goal, theme, location, or possessor. There is a syntactic explanation for these patterns: reflexive pronouns must occupy the object position (autophatic domain) and are thus banned as complements of postpositions (oblique domain) or as adnominal modifiers (adpossessive domain). In this context, a nonreflexive personal pronoun must be used. The use of nonreflexive pronouns in the oblique and adpossessive domains alternates between coreference readings and disjoint interpretations. Personal pronouns are also preferred in clause combining. Additionally, the use of reflexive pronouns as middle markers is allowed with some but not all middle situations in Yaqui.



\section*{Acknowledgments}



I would like to thank the anonymous reviewers and editors for their helpful comments and suggestions on earlier versions of this paper. All remaining errors are exclusively my responsibility. This study was partially supported by the UNAM-DGAPA-PAPIIT grant (IN400919) and Conacyt-Ciencia Básica (A1-S-24378).



\section*{Abbreviations}

I use the Leipzig glossing rules, with the following additions:\\


\begin{tabularx}{.45\textwidth}{>{\scshape}lQ}
clm& clause linkage marker\\
cmpl& completive\\
dir& directional\\
int& intensifier\\
\end{tabularx}
\begin{tabularx}{.45\textwidth}{>{\scshape}lQ}
pastc& past continuative\\
pot& potential\\
red& reduplication\\
sta& stative
\end{tabularx}\medskip

All uncited data are taken from my field notes; examples from oral texts include the story title and page number of the digital manuscript. The examples are presented using a practical orthography accepted by the Yaqui community except for data quoted from grammatical studies, in which case the original orthography has been preserved (except accents) but the morphological glossing has been amplified or adjusted.













{\sloppy\printbibliography[heading=subbibliography,notkeyword=this]}
\end{document}
