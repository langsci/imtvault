\documentclass[output=paper]{langscibook}

\author{Katarzyna Janic\affiliation{Leipzig University}}
\title{Reflexive constructions in Polish}
\abstract{Polish is an Indo-European language of the West Slavic sub-branch which has three types of reflexive constructions. The coreference between agent and patient participant roles can be expressed by one of the following reflexivizers: \textit{siebie}, \textit{swój} or \textit{się}. The reflexive nominal form siebie shares the inflectional pattern with the personal pronoun, which is rather uncommon from a crosslinguistic perspective. The second reflexive nominal \textit{swój} is used in the context of the third person to make a formal distinction between third-person reflexive possessive pronouns and their nonreflexive counterparts. Finally, the reflexive clitic \textit{się} is verbal, modifying the syntactic and semantic value of the verbal valency. Even if \textit{się} is particularly frequent in impersonal constructions, its omnipresence in middle or reflexive domains is non-negligible either. Like many Slavic languages, \textit{się} may also be used to encode the antipassive function. }
\IfFileExists{../localcommands.tex}{
 \addbibresource{localbibliography.bib}
 \input{../localpackages}
 %Copy this to localcommands.tex

\usepackage[english]{babel}
\usepackage{amsmath}
\usepackage{amssymb,amsfonts,textcomp}
\usepackage{array}
\usepackage{hhline}
\usepackage{hyperref}

\newenvironment{styleStandard}{}{}
\newenvironment{stylelsAbstract}{}{}
\newenvironment{stylelsSectioni}{}{}
\newenvironment{stylelsSectionii}{}{}
\newenvironment{stylelsBulletList}{}{}
\newenvironment{styleBibliographyi}{}{}
\newenvironment{listWWNumxxvleveli}{}{}
\newenvironment{listWWNumxxvlevelii}{}{}
\newenvironment{listWWNumxxvleveliii}{}{}
\newenvironment{listWWNumxxvleveliv}{}{}
\newenvironment{listWWNumixleveli}{}{}
\newenvironment{listWWNumixlevelii}{}{}
\newenvironment{listWWNumixleveliii}{}{}
\newenvironment{listWWNumixleveliv}{}{}

\newcommand\textstyleListLabelxvi[1]{#1}
\newcommand\labellistWWNumxxvleveli{\thelistWWNumxxvleveli.}
\newcommand\labellistWWNumxxvlevelii{\thelistWWNumxxvlevelii.}
\newcommand\labellistWWNumxxvleveliii{\thelistWWNumxxvleveliii.}
\newcommand\labellistWWNumxxvleveliv{\thelistWWNumxxvleveliv.}
\newcommand\labellistWWNumixleveli{[F0B7?]}
\newcommand\labellistWWNumixlevelii{\textstyleListLabelxvi{o}}
\newcommand\labellistWWNumixleveliii{[F0A7?]}
\newcommand\labellistWWNumixleveliv{[F0B7?]}

\newcounter{listWWNumxxvleveli}
\newcounter{listWWNumxxvlevelii}[listWWNumxxvleveli]
\newcounter{listWWNumxxvleveliii}[listWWNumxxvlevelii]
\newcounter{listWWNumxxvleveliv}[listWWNumxxvleveliii]
\newcounter{itemize}  
 \input{../localhyphenation} 
 \togglepaper[1]%%chapternumber
}{}

\begin{document}
\maketitle

\section{Introduction} \label{sec:Janic:1}

\subsection{Classification, distribution and dialects of Polish} \label{sec:Janic:1.1}

Polish belongs to the Indo-European language family, which together with Atlantic-Congo, Austronesian and Sino-Tibetan is one of the most populous language families of the world. Within Indo-European, Polish belongs to the Slavic group that falls into three major sub-groups: East, West and South. Together with Czech, Slovak and Sorbian, Polish belongs to the West Slavic group. When compared to other members of West Slavic, Polish has the largest number of speakers. It is also the second most widely spoken Slavic language.

It is a well-documented and well-studied language, spoken mainly in Poland where it is an official language (see \figref{fig:Janic:1}). Today’s calculations indicate that there are 38.5 million people who speak Polish as the first language. In the Czech Republic, Slovakia, Hungary, west Belarus and Ukraine, and central-west Lithuania, Polish is spoken by many people as a second language. 

\begin{figure}
\includegraphics[width=\textwidth]{figures/Polish.pdf}
  \caption{\label{fig:Janic:1}Map of Poland}
\end{figure}


Polish does not exhibit a strong regional diversification. This refers to both grammar and lexicon. It attests four or five dialects, depending on whether Kashubian is included or not. The latter is spoken in the north-west of Poland around the city of Gdańsk and presents characteristics shared by both languages and dialects. Another dialectal area includes Great Poland in the west, centred around the cities of Poznań and Gniezno. The dialect spoken in this area served as the basis for the formation of literary Polish. Another area is Little Poland in the south-east, centred on Kraków. This region greatly influenced the modern standard language. The third area is Mazovia. It encompasses the region around the capital city of Warsaw, extended to east and north-east Poland. The last area is Silesia in the south-west, with the major city of Katowice.


\subsection{General remarks on Polish morphosyntax} \label{sec:Janic:1.2}

\subsubsection{Clause structure} \label{sec:Janic:1.2.1}

Polish clause structure has a flexible word order. The dominant pattern is SVO, which is the second most common word order type in the world (cf. \citealt{Dryer2013}). Case coding and gender-number agreement between a verb and its core arguments shape the language in the direction of accusative alignment. It is a pro-drop language where the omitted pronoun can always be pragmatically or grammatically inferred from the context. Reflexive, middle, impersonal and antipassive are valency-changing operations expressed by verbal morphology.


\subsubsection{Noun Phrase} \label{sec:Janic:1.2.2}

Polish has a well-developed gender system. Among various categories, nouns systematically recognize grammatical gender distinction. The latter is based on a three-fold division, including masculine, feminine and neuter. All singular nouns are either masculine, feminine or neuter. Within a class of singular masculine nouns, Polish offers a more fine-grained differentiation based on masculine animate and masculine inanimate recognition. By contrast, plural nouns recognize only masculine (or ‘virile’) and non-masculine (or ‘non-virile’) gender distinction. Gender plays a prominent role in agreement. Specifically, noun gender is of relevance for noun-adjective agreement pattern, and for past tense agreement. Even if the noun gender is inherent in Polish, one cannot deduce its specific value, building on the noun form alone. It is only possible after the recognition of the class declension to which a given noun belongs. 

Polish has a relatively rich case system including nominative, accusative, genitive, dative, locative and instrumental. Contrary to other Slavic languages (e.g. Bulgarian, Macedonian), it did not develop articles corresponding to the English definite \textit{the} and indefinite \textit{a/an} distinction. In this respect, Polish does not differ much from many languages of the world. It belongs to 198 languages out of 620 that lack definite and indefinite articles (cf. \citealt{Dryer2005}). The noun phrases are thus vague in terms of definiteness and whether a particular noun receives a definite or indefinite interpretation is either deduced from a context or resolved by demonstratives. 


\section{Pronouns} \label{sec:Janic:2}

Polish has a rich set of pronouns, including personal, reflexive, possessive, demonstrative, interrogative, distributive, relative and indefinite. In the present section, I will focus only on those that are relevant for reflexive constructions, namely reflexive \sectref{sec:Janic:2.2} and possessive \sectref{sec:Janic:2.3} pronouns. A special attention will also be given to personal pronouns \sectref{sec:Janic:2.1}. 



\subsection{Personal pronouns} \label{sec:Janic:2.1}

The paradigm of the Polish independent personal pronouns with their clitic counterparts is illustrated in Table~\ref{tab:Janic:1} and in Table~\ref{tab:Janic:2}. While Table~\ref{tab:Janic:1} is dedicated to the first- and second-person personal pronouns, Table~\ref{tab:Janic:2} deals with the third-person status of personal pronouns.

%%TABLES 1 - 2[ht?]
\begin{table}%[ht]
 \centering
 \begin{tabularx}{\textwidth}{p{2cm}p{2cm}p{2cm}p{2cm}p{2cm}}
 \lsptoprule
 & \textsc{1sg} & \textsc{2sg} & \textsc{1pl} & \textsc{2pl} \\
 \midrule
 \textsc{nom} & ja & ty & my & wy\\
 \textsc{gen} & mnie/mię & ciebie/cię & nas & was\\
 \textsc{dat} & mnie/mi	& tobie/ci	& nam & wam\\
 \textsc{acc} & mnie/mię	& ciebie/cię &	nas	& was\\
 \textsc{loc} & mnie	 & tobie	 & nas & was\\
 \textsc{ins} & mną	 & tobą	 & nami	& wami\\
 \lspbottomrule
 \end{tabularx}
 \caption{1st and 2nd person pronouns in Polish}
 \label{tab:Janic:1}
\end{table}

\begin{table}%[ht]
 %\centering
 \begin{tabularx}{\textwidth}{p{1cm}p{1.7cm}p{1.7cm}p{1.7cm}p{1.7cm}p{1.7cm}}
 \lsptoprule
 & \textsc{1sg(m)} & \textsc{1sg(f)} & \textsc{1sg(n)} & \textsc{1pl(vir)} & \textsc{1pl(nvir)}\\
 \midrule
 \textsc{nom}	& on & ona	& ono & oni	 & one\\
 \textsc{gen} & jego/go &	jej	& jego/go	& ich & ich \\
 \textsc{dat}	& jemu/mu &	jej	& jemu/mu	& im	 & im\\
 \textsc{acc}	& jego/go	 & ją	& je	 & ich	 & je\\
 \textsc{loc}	& nim	& niej &	nim & nich & nich\\
 \textsc{ins}	& nim	 & nią	& nim	 & nimi &	nimi\\
 \lspbottomrule
 \end{tabularx}
 \caption{3rd person pronouns in Polish}
 \label{tab:Janic:2}
\end{table}
 

As it can be seen from above Tables, Polish personal pronouns clearly distinguish between the first, second and the third person. They are also sensitive to number and case of a noun or a noun phrase they substitute. The third-person pronouns additionally recognizes gender. Specifically, the singular form of a third-person pronoun is based on masculine, feminine and neuter distinction, whereas its plural counterpart offers only masculine vs. non-masculine opposition. Like in English, the declension of personal pronouns in Polish is quite peculiar as the nominative form differs from other cases i.e. \textit{ja} ‘I’ vs. \textit{mnie} ‘me’, etc.

The nominative first-person and second-person personal pronouns are typically omitted. Their realization is, however, necessary when one emphasizes the importance of the subject, as in (\ref{ex:Janic:1})\footnotetext{Unless specified otherwise, I am the author of all examples.} or seeks for clarification of meaning, as in (\ref{ex:Janic:2}), or contrasts the pronominal subjects, as in (\ref{ex:Janic:3}).

\ea
 \label{ex:Janic:1}
 \gll{Tylko} {ja} {pracuję}	{w} {weekendy.}{\footnotemark}\\
	 only \textsc{1sg.nom} work.\textsc{prs.3sg}	in weekend.\textsc{acc.pl(nvir)}\\
	\glt ‘Only I work on the weekends.’
\z
	
\ea
 \label{ex:Janic:2}
 \gll {Czy}	{my} {się}	{znamy?}\\
	 Q	\textsc{1pl.nom} \textsc{self} know.\textsc{prs.1pl}\\
	\glt ‘Do we know each other?’ \citep[267]{Sadowska2012} 
\z


\ea
 \label{ex:Janic:3}
 \gll{Jeśli} {ty} {się}	{teraz}	{zabawiasz,} {to}	{ja} {sobie}		{idę.} \\
	 if		\textsc{2sg.nom}	\textsc{self}	now	have.fun.\textsc{prs.2sg}	then	\textsc{1sg.nom}	\textsc{self.dat}	go.\textsc{prs.1sg}\\
 	\glt‘If you’re having fun now, then I’m on my way.’
 \z
 

The deletion of the nominative third person pronouns is possible when their referent is easily inferred from the context. They are, however, expressed when used for the first time in a paragraph. In the subsequent ones, they can be omitted as long as their referent remains clear.

Another peculiarity of the Polish personal pronouns is that some of them recognize long vs. short opposition. In Table~\ref{tab:Janic:1} and Table~\ref{tab:Janic:2}, the short forms appear after the slashes. Represented by the six forms \textit{mię}, \textit{mi}, \textit{cię}, \textit{ci}, \textit{go} and \textit{mu}, they behave like clitics. Neither can the short forms carry their own stress nor appear sentence initially, (\ref{ex:Janic:4}). They also manifest distributional restrictions: unlike their long counterparts, the short forms cannot appear after a preposition, (\ref{ex:Janic:5}).

\ea \label{ex:Janic:4}
\ea \label{ex:Janic:4a}
 \gll {Szukam} {cię.} {Kogo} {szukasz?}\\
 look.for.\textsc{prs.1sg}	\textsc{2sg.acc} who	search.for.\textsc{prs.2sg}\\
 \glt ‘I’m looking for you. Who are you looking for? \citep[155]{Swan2002}
 
\ex\label{ex:Janic:4b}
 \gll {Ciebie} {/} {*Cię} {szukam}\\	
	 \textsc{2sg.acc} / \textsc{2sg.acc}	look.for.\textsc{prs.1sg}\\
	\glt 'I am looking for you.’ 
\z
\z
 
\ea \label{ex:Janic:5}
\ea \label{ex:Janic:5a}
 \gll Patrzy		 na	mnie / *mię.\\
		look.\textsc{prs.3sg}	on	\textsc{1sg.accC} / \textsc{1sg.acc}\\
 \glt‘He is looking at me.’

\ex \label{ex:Janic:5b}
 \gll Myślę		 o	 tobie / *ci.\\
		 think.\textsc{prs.1sg}	about	\textsc{2sg.dat} / \textsc{2sg.dat}\\	
	\glt ‘I am thinking of you.’
\z
\z

The longer forms: \textit{mnie} (vs. \textit{mi}, \textit{mię}), \textit{ciebie} (vs. \textit{cię}), \textit{tobie} (vs. \textit{ci}), \textit{jego} (vs. \textit{go}), \textit{jemu} (vs. \textit{mu}) are called emphatic pronouns and are used when the emphasis is required \citep{Bielec1998}. They obligatorily carry the stress. Like the remaining independent pronouns, \textit{mnie}, \textit{ciebie}, \textit{tobie}, \textit{jego} and \textit{jemu} can also occur at the beginning of the clause and after a preposition. The contrast between first-person short clitic form mi and its longer equivalence \textit{mnie} is illustrated in (\ref{ex:Janic:6}).

\ea \label{ex:Janic:6}
\ea \label{ex:Janic:6a}
\gll Nauczyciel	  dał  mi	 książkę. \\
		teacher.\textsc{nom.sg(vir)}	give.\textsc{pst.3sg}	\textsc{1sg.dat}	 book.\textsc{acc.sg(f)}\\
\glt ‘The teacher gave me the book.’

\ex \label{ex:Janic:6b}
\gll Nauczyciel	 dał  mnie książkę.\\
		teacher.\textsc{nom.sg(vir)}	give.\textsc{pst.3sg}	\textsc{1sg.dat}	 book.\textsc{acc.sg(f)}\\
\glt	‘I was the one the teacher gave the book to.’ \citep[65]{Feldstein2001}
\z
\z

The independent personal pronouns that do not recognize short vs. long opposition can be stressed, depending on whether they are emphasized or not. 

Regardless of their name, the referent of the personal pronouns in Polish may also denote animals or inanimate objects. This is illustrated in \ref{ex:Janic:7}, where the inflected third person masculine pronoun \textit{nim.} \textsc{3sg(m).loc} corefers with the masculine noun \textit{stół} ‘table’.

\ea \label{ex:Janic:7}
\gll Książki	 leżą 	 na 	stole 	 i 	 pod 	nim.\\
 book.\textsc{nom.pl(nvir)}	lie.\textsc{pst.3pl(nvir)}	on	table.\textsc{loc.sg(m)}	and 	under	\textsc{loc.sg(m)} \\	
\glt ‘The books are on the table and under it.’ \citep[265]{Sadowska2012}
\z

Personal pronouns in Polish share many properties with their English equivalences. For instance, they form a paradigm, are not morphologically transparent and exhibit a restricted possibility in terms of the modification. However, they are necessarily referential, in particular definite. Thus, they cannot have a non-specific or generic interpretation. Neither can the Polish personal pronouns be used as bound variables. Example (\ref{ex:Janic:8}) illustrates the last point.

\ea \label{ex:Janic:8}
\ea \label{ex:Janic:8a}
\gll Każda	 kobieta\textsubscript{i}	uważa, 	 że	 ona\textsubscript{j}	jest		 mądra.\\
 every.\textsc{nom.sg(f)}	woman.\textsc{nom.sg(f)} consider.\textsc{prs.3sg}	that	\textsc{3sg(f)} be.\textsc{PRS.3SG}	clever.\textsc{nom.sg(f)}\\
\glt ‘Every woman\textsubscript{i} thinks that she\textsubscript{j} is clever.’

\ex \label{ex:Janic:8b}
\gll Każda	 kobieta\textsubscript{i} uważa, 	 że	 jest\textsubscript{i}		mądra.\\
 every.\textsc{nom.sg(f)} woman.\textsc{nom.sg(f)}	consider.\textsc{prs.3sg}	that	be.\textsc{prs.3sg} clever\textsc{nom.sg(f)}\\
\glt ‘Every woman\textsubscript{i} thinks that she\textsubscript{i} is clever.’ \citep[11]{Siewierska2004}
\z
\z

In \ref{ex:Janic:8}, the anaphoric pronoun \textit{ona} can be construed as coreferential only with some entity situated outside of the clause. Since personal pronouns in Polish are necessarily referential, they cannot be interpreted as bound variables. As pointed out by \citet{Siewierska2004}, a bound variable interpretation is still possible, however, only of the person inflection on the verb, (\ref{ex:Janic:8}). 

The contrast in the interpretation between (\ref{ex:Janic:8}) corresponds to two kinds of coreference recognized in the literature under discourse-referential interpretation vs. co-varying interpretation. (\ref{ex:Janic:8a}) exemplifies the discourse-referential reading because the anaphoric pronoun \textit{ona} denotes a particular woman the referent of which can only be identified in the discourse. In contrast,(\ref{ex:Janic:8b}) triggers a co-varying interpretation. The person inflected on the verb can be construed as bound by the quantified subject \textit{każda kobieta} ‘every woman’ of the main clause. 

Polish personal pronouns share one nominal feature, based on sociolinguistic implications (cf. \citealt{Siewierska2004}). Among adults, the use of the third-person pronouns is highly informal. Thus, the system of the language developed special third-person pronouns with a formal flavour: \textit{Pan, Pani} and \textit{Państwo}, which function as honorific second-person pronouns. They serve as alternatives to the informal second-person singular \textit{ty} and plural \textit{wy} forms. The third-person pronouns of polite, formal address is still visible in the agreement of the verb, as shown in (\ref{ex:Janic:9}).

\ea \label{ex:Janic:9}
\gll Pani	 powinna	 przeprosić	 za	swoje		 zachowanie.\\
	 Mrs. 	should.\textsc{prs.3sg(f)}	apologize.\textsc{inf}	for	\textsc{ref.poss.3sg(n)}	behaviour.\textsc{acc.sg(n)} \\
\glt ‘Mrs. (you) should apologize for your behaviour.’
\z
	
Polish speakers use the honorific pronouns when they address a stranger, a person they do not know well or who is of authority to express respect and distance. In the system, the honorific pronouns \textit{Pan, Pani} and \textit{Państwo} function in parallel with their corresponding grammaticalized nouns, meaning ‘gentleman’, ‘lady’, and ‘ladies and gentlemen, Mrs. and Mr.’ accordingly. 


\subsection{Reflexive pronouns} \label{sec:Janic:2.2}

Polish has two reflexive pronouns, \textit{siebie} and \textit{się}, which display different formal and functional characteristics. In the current section, I will briefly summarize their similarities and differences. I will discuss these two reflexive forms in detail later on in \sectref{sec:Janic:3.1} and \sectref{sec:Janic:3.2} respectively. 

In terms of the morphosyntactic characteristics, neither \textit{się} nor \textit{siebie} makes a gender distinction. They are also indifferent to number. Both, however, inflect for case. While \textit{siebie} recognizes all of the cases except for nominative, \textit{się} is sensitive to genitive, dative and accusative alone. Both forms thus constitute an incomplete (‘defective’) pronominal paradigm, given in Table~\ref{tab:Janic:3}.\footnote{Table~\ref{tab:Janic:3} has been slightly modified.}


% CLITIC FORM	INDEPENDENT FORM
% tables needs improvements
			

\begin{table}[ht]
 \centering
 \begin{tabularx}{\textwidth}{p{0.8cm}p{1.8cm}p{2.7cm}p{2.7cm}p{2.7cm}}
 
 \lsptoprule
 {} & {clitic form} & {independent form} & {independent form} & {independent form}\\
 {free} & {reflexive pronoun} & {reflexive pronoun} & {personal pronoun} & {personal pronoun}\\
 
 \hline
 \textsc{nom}	& - & -	 & ty.\textsc{2sg}	 & ja.\textsc{1sg} \\
 \textsc{gen}	& się &	siebie	 & ciebie 	 & mnie \\
 \textsc{dat}	& (se)* &	sobie & tobie	 & mnie \\
 \textsc{acc}	& się	 & siebie	 & ciebie &	mnie \\
 \textsc{loc}	& -	 & (prep.+) sobie	 & tobie	 & mnie \\
 \textsc{ins}	& -	 & sobą	 & tobą	 & mną \\
 \lspbottomrule
 \end{tabularx}
 \caption{The reflexive pronouns, adopted from \citet[517]{Wiemer2007}}
 \label{tab:Janic:3}
\end{table}

Since the reflexive pronoun \textit{siebie} has the same inflectional pattern as the first-person and second-person personal pronoun (cf. Table~\ref{tab:Janic:3}), undoubtedly they belong to the same paradigm. In addition to the similar inflectional paradigm, \textit{siebie} exhibits other pronoun-like features. For instance, it lacks article or possessive modifications. The coalition of the reflexive pronoun with the (personal) pronouns rather than with nouns is rare crosslinguistically. It is a particularity of western Indo-European languages in particular of Slavic and Germanic groups.

The difference between \textit{siebie} and \textit{się} also involves morpho-phonological variation. \textit{Siebie} is defined as an independent pronoun. Hence, it is realized as separate word and what is more important it takes a primary word stress. It also manifests syntactic independence, as it may occur in isolation as an elliptical answer. Example (\ref{ex:Janic:10}) illustrates this possibility.

\ea \label{ex:Janic:10}
\ea \label{ex:Janic:10a}
\gll Komu	kupiłaś		 lody? \\
		whom	buy.\textsc{pst.2sg}	ice.cream.\textsc{acc.pl(nvir)} \\
\glt ‘For whom did you buy ice cream?’

\ex \label{ex:Janic:10b}
\gll Sobie.\\
		\textsc{self.dat}	\\
\glt ‘Myself.’ \citep[278]{Sadowska2012}
\z
\z

By contrast, the morpho-phonological characteristics of the reflexive form \textit{się} clearly shows the properties of clitics. Even if \textit{się} occurs as a separate word, it is clearly phonologically and morphologically dependent on the host. For instance, it lacks prosodic independence i.e. it cannot be stressed. The presence of \textit{się} does not affect the place of stress of words to which it is adjacent either. Finally, \textit{się} shows little, if at all, syntactic independence. It has a restricted distribution relative to the independent form. For instance, it cannot appear in isolation or after a preposition. Thus, the reflexive forms \textit{siebie} and to two different paradigms.

The morpho-phonological variation between \textit{siebie} and \textit{się} corresponds to what \citet{Kemmer1993}  calls the heavy vs. light distinction. The form of \textit{siebie} is thus defined as heavy because it contains more phonological ‘body’ that can be measured in terms of number of segments. \textit{Się}, by contrast, is considered to have a light form. This means that like many other languages with heavy vs. light opposition in the reflexive domain (e.g. Djola, Old Nurse, Surselvan, Slavic), this dependent form demonstrates a reduced phonological material relative to the heavy form. 

The difference between \textit{siebie} and \textit{się} reflects a nominal/verbal distinction. The possibility to occur in non-object position or in isolation can be taken as indication of the (pro)nominal features associated with \textit{siebie}. \textit{Się} is more of the verbal type. It cannot appear in the non-object position. Its combination with a transitive verb may result in the modification of the syntactic structure of the predicate. Moreover, both reflexive forms demonstrate functional differences. In contrast to \textit{siebie} the use of which is primarily limited to express the coreference meaning, \textit{się} is highly polyfunctional with a wide scope beyond the coreference domain. Finally, both the reflexive forms are diachronically related, where \textit{się} is claimed to originate from the pronoun \textit{siebie}. In the reminder of this chapter, I will address these issues respectively. 


\subsection{Possessive pronouns} \label{sec:Janic:2.3}
%slightly modified the content of the text (differences with a submitted word version)

Polish has independent possessive pronouns that agree in gender, number and case with the noun they refer to. Table~\ref{tab:Janic:4} \footnote{Table~\ref{tab:Janic:4} has been slightly modified.} and Table~\ref{tab:Janic:5} offer their forms with differential and coreferential interpretation respectively.

\begin{table}[ht]
 \centering
 \begin{tabularx}{0.7\textwidth}{XXXXX}
 \lsptoprule
 \textsc{} & \textsc{1sg.m} & \textsc{1pl} & \textsc{2sg} & \textsc{2pl} \\
 \midrule
 \textsc{M} & mój	& moi	 & twój	& twoi\\
 \textsc{F} & moja	& moje & twoja	& twoje\\
 \textsc{N} & moje	& moje	& twoje	& swoje\\
 \lspbottomrule
 \end{tabularx}
 \caption{Differential possessive pronouns in nominative case; adopted from \citet[519]{Wiemer2007}}
 \label{tab:Janic:4}
\end{table}

\begin{table}[ht]
 \centering
 \begin{tabular}{p{2cm}p{2cm}p{2cm}p{2cm}p{2cm}}
 \lsptoprule
 \textsc{} & \textsc{3sg} & \textsc{3pl} & \textsc{3sg} & \textsc{3pl} \\
 \midrule
 \textsc{M} & swój	& swoi	& jego	& ich \\
 \textsc{F} & swoja	& swoje	& jej	& ich \\
 \textsc{N} & swoje	& swoje	& jego	& ich \\
 \lspbottomrule
 \end{tabular}
 \caption{Coreferential possessive pronouns in nominative case; adopted from \citet[519]{Wiemer2007}}
 \label{tab:Janic:5}
\end{table}
%footnote table 5 has been slightly modified
%correct below small caps and related glosses

The peculiarity of the Polish possessive pronouns is a formal split at the level of the third-person pronoun, leading to the distinction between coreference vs. disjoint interpretation. Since the third-person possessive pronouns: \textit{swój} (\textsc{3sg(m)}), \textit{swoja} (\textsc{3sg(f)}), \textit{swoje} (\textsc{3sg(n)}) (together with their plural equivalents) co-refer with the subject participant of the clause, they are labelled reflexive possessive pronouns. They contrast with their possessive nonreflexive counterexamples: \textit{jego} (\textsc{SG.M/N}) and \textit{jej} (\textsc{sg(f)}) (also with their plural equivalents). These pronouns signal that a possessor referent is different than subject. The formal split based reflexive vs. non-reflexive possessive opposition is rare crosslinguistically. Many languages lack this distinction, leading \textit{ipso facto} to referential ambiguity. A textbook example comes from English where in the clause \textit{She went to her room}, the possessive pronoun \textit{her} oscillates between coreference reading (\textit{She went to her own room}) and a disjoint one (\textit{She went to somebody else’s room}).


\section{Reflexive constructions} \label{sec:Janic:3}

Polish distinguishes three kinds of reflexive constructions: (i) reflexive construction with the independent reflexive pronoun \textit{siebie}, (ii) reflexive construction with the clitic form \textit{się}, and (iii) reflexive construction with the possessive reflexive pronoun \textit{swój}. Since the general morphosyntactic characteristics of these three reflexive forms have already been introduced in \sectref{sec:Janic:2}, in what follows, I will discuss their functional aspects and the idiosyncratic properties. 


\subsection{Reflexive constructions with the reflexive independent pronoun \textit{siebie}} \label{sec:Janic:3.1}
\subsubsection{Functions} \label{sec:Janic:3.1.1}

Depending on the subject, \textit{siebie} is translated as ‘myself’, ‘yourself’, ‘herself’, ‘himself’, ‘itself’, ‘ourselves’, ‘yourselves’, ‘themselves’. It primarily performs two functions. In the first place, the pronoun co-refers with a singular subject, leading to the reflexive interpretation, (\ref{ex:Janic:11}).

\ea \label{ex:Janic:11}
\ea \label{ex:Janic:11a}
\gll Oskarżony	 	bronił			 siebie		 zaciekle.\\
		accused.\textsc{nom.sg(m)}	defend.\textsc{pst.3sg(m)}	\textsc{self.acc}	fiercely \\
\glt ‘The accused defended himself fiercely.’
	
\ex \label{ex:Janic:11b}
\gll Matka	 chroniła	 siebie 	 i 	swoje			 dzieci.\\
		mother.\textsc{nom.sg(f)}	protect.\textsc{pst.3sg(f)}	\textsc{self.acc}	and	\textsc{refl.poss.acc.pl(nvir)}	child.\textsc{acc.pl(nvir)} \\
\glt	‘The mother protected herself and her children.’
\z 
\z
	
	
The independent reflexive pronoun \textit{siebie} can also be coreferential with the subject participant in the plural form. Here, it performs a reciprocal function, where it carries the meaning of ‘each other’ and/or ‘one another’. In fact, many Polish clauses with a plural subject and the reflexive pronoun \textit{siebie} are ambiguous, situated at the interface of reflexive and reciprocal interpretations. Thus, in (\ref{ex:Janic:12a}) and (\ref{ex:Janic:12b}) the both reflexive and reciprocal readings are equally acceptable, and to resolve an interpretative conflict, a broader context is required.

\ea \label{ex:Janic:12}
\ea \label{ex:Janic:12a}
\gll Przyjaciele bronili  siebie 	długo.\\
		friend.\textsc{nom.pl(vir)} defend.\textsc{pst.3pl(vir)}	\textsc{self.acc}	for.a.long.time\\
\glt i. ‘The friends were defending themselves for a long time.’
\glt ii. ‘The friends were defending each other for a long time.’

\ex \label{ex:Janic:12b}
\gll Magda		 i	Marta	 	lubiły			 siebie. \\			
 Magda\textsc{nom}		and	Marta\textsc{nom}		like.\textsc{pst.3pl(nvir)}	\textsc{self.acc}\\	
\glt i. ‘Magda and Marta liked themselves.’
\glt ii. ‘Magda and Marta liked each other.’ \citep[263--264]{Nedjalkov2007}
\z 
\z 


Not all clauses with a plural subject and pronoun \textit{siebie} in object function are ambiguous. The pragmatic context may occasionally help to provide the disambiguation, as illustrated in (\ref{ex:Janic:13}). 

\ea \label{ex:Janic:13}
\gll Przyjaciele	 obudzili		 siebie.\\	
	 friend.\textsc{nom.pl(vir)}	 wake.\textsc{pst.3pl(vir)}	 \textsc{self.acc}\\					
\glt i. *‘The friends woke themselves up.’
\glt ii. ‘The friends woke each other up (e.g. by snoring).’	 \citep[264]{Nedjalkov2007}
\z 

\subsubsection{Domain of coreference} \label{sec:Janic:3.1.2}

Example (\ref{ex:Janic:14}) illustrates the distribution of the reflexive \textit{siebie} in various syntactic contexts.

\ea \label{ex:Janic:14}
\ea \label{ex:Janic:14a}
\gll Marek	 szanuje		 tylko		siebie.\\
		Marek.\textsc{nom}	respect.\textsc{prs.3sg}	only		\textsc{self.acc} \\
\glt ‘Marek respects only himself.’ 		

\ex \label{ex:Janic:14b}
\gll Dziewczyny	 lubią			 tylko		siebie.\\
		girl.\textsc{nom.pl(nvir)}	like.\textsc{prs.3pl(nvir)}	only		\textsc{self.acc}\\
\glt ‘The girls like only themselves.’

\ex \label{ex:Janic:14c}
\gll Często 	mówicie		 do 	siebie 		 na 	głos.\\	
		often	speak.\textsc{prs.2pl}	to	\textsc{self.gen}	on	voice.\textsc{acc.sg(m)}\\
\glt ‘You often speak to themselves.’

\ex \label{ex:Janic:14d}
\gll Zawsze	noszę		 na	sobie		 czyste		ubrania.\\
		always	wear.\textsc{prs.1sg}	on	\textsc{self.loc}	clean		cloth.\textsc{acc.pl(nvir)}\\
\glt ‘I always wear clean clothes.’

\ex \label{ex:Janic:14e}
\gll Zamknij		 drzwi		 za		sobą.	\\ 			
		close.\textsc{imp.2sg}	door.\textsc{acc.pl(nvir)}	behind	\textsc{self.ins}\\
\glt ‘Close the door behind you.’

\ex \label{ex:Janic:14f}
\gll Szybko	znalazł 		 sobie 		 nową 		 dziewczynę.\\
		quickly	find.\textsc{pst.3sg(m)}	\textsc{self.dat}	new.\textsc{acc.sg(f)}	girl.\textsc{acc.sg(f)}\\
\glt ‘He quickly found a new girl.’
\z 
\z 

Example (\ref{ex:Janic:14}) shows that \textit{siebie} accepts two kinds of antecedent domain. The first is ‘autopathic domain’ (cf. Haspelmath, this volume §8) and refers to the coreference relation between subject and object in a monotransitive clause, (\ref{ex:Janic:14}). The second kind, recognized in the literature as oblique domain indicates the coreference relation holding between subject and an oblique participant of the same minimal clause. This can be observed in (\ref{ex:Janic:14}).

Unlike Turkish, Kashmiri and some other languages, Polish disallows the coreference of the independent reflexive pronoun \textit{siebie} with the grammatically less salient antecedent i.e. object, (\ref{ex:Janic:15}). In order to express the coreference between the complement of the PP and the object, the language makes use of the pronominal non-reflexive anaphoric pronoun, as shown in (\ref{ex:Janic:16}).

\ea \label{ex:Janic:15}
\gll Jan\textsubscript{i} opowiedział		 Piotrowi\textsubscript{j} o	 sobie.\\
	 Jan.\textsc{nom}	 tell.\textsc{pst.3sg(m)}	Peter.\textsc{dat}	 about	\textsc{self.loc}\\
\glt ‘John\textsubscript{i} told Peter\textsubscript{j} about self\textsubscript{i,*j}.'
\z 
	
\ea \label{ex:Janic:16}
\gll Jan\textsubscript{i}		opowiedział		 Piotrowi\textsubscript{j}	 o	 nim\textsubscript{*i/j/k}.\\
	 Jan.\textsc{nom}	 tell.\textsc{pst.3sg(m)}	Peter.\textsc{dat}	 about	\textsc{3sg(m).loc}\\
\glt ‘John\textsubscript{i} told Peter\textsubscript{j} about him\textsubscript{*i/j/k}.’ \citep[193]{Siewierska2004}
\z 

\subsubsection{Coreference of subject with the beneficiary} \label{sec:Janic:3.1.3}


Another type of context relevant for the present discussion involves a coreference between the subject and the non-patient participant such as beneficiary. 

In general, benefactive events refer to the type of event wherein the subject participant performs an action that is of benefit either for himself or for a distinct participant. This leads to ‘self-benefactive’ and ‘other-benefactive’ distinction. In self-benefactive events, the subject argument assumes thus two semantic roles: the one of the agent and the one of the beneficiary and the coreference between these two participants can be signalled in languages in multiple ways. For instance, in English, the agent-beneficiary coreference can be expressed either through the reflexive pronoun alone, as in \textit{Paula bought herself a book} or by a reflexive pronoun coupled with the preposition for: \textit{Paula bought a book for herself}. By contrast, the clause \textit{Paula bought a book for John} exemplifies an benefactive event where \textit{John}, construed as the beneficiary, takes advantage of the action performed for him by the agent participant. Like English, Polish also expresses the coreference of subject with a beneficiary by means of the reflexive pronoun that may occur in two syntactic configurations. While the first context involves the reflexive pronoun in dative form,(\ref{ex:Janic:17a}), in the second context, the reflexive pronoun is in the accusative form and accompanied by the preposition \textit{dla} ‘for’,(\ref{ex:Janic:17b}). 

\ea \label{ex:Janic:17}
\ea \label{ex:Janic:17a}
\gll Dziecko	 kupiło		 sobie		 lizaka.\\
		child.\textsc{nom.sg(n)}	buy.\textsc{pst.3sg(n)}	\textsc{self.dat}	lollipop.\textsc{acc.sg(m)}\\
\glt ‘The child bought herself a lollipop.’

\ex \label{ex:Janic:17b}
\gll Dziecko	 kupiło		 lizaka			 dla	 siebie.\\
	 child.\textsc{nom.sg(n)}	buy.\textsc{pst.3sg(n)}	lollipop.\textsc{acc.sg(m)}	for	 \textsc{self.acc}\\	
\glt ‘The child bought a lollipop for herself.’	
\z 
\z
	 
	 
The alternation in coding the beneficiary coreference is common for all Slavonic languages. Even if it is subject to free variation in Polish, there is a tendency to favour a morphologically less complex beneficiary expressed by dative form rather than a prepositional phrase. This goes hand in hand with Swan's \citeyear{Swan2002} observation and corpus-based study. A survey of the National Corpus of Polish shows, for instance, that the verbal form \textit{kupił} ‘buy.\textsc{pst.3sg(m)}’ occurs with the dative reflexive beneficiary form 287 times against 4 occurrences wherein the same participant is expressed through a prepositional phrase \textit{dla siebie} ‘for oneself’. Building on the text-frequency criterion for markedness, \citet{Kemmer1993}  argues that self-benefactive constructions of the type (\ref{ex:Janic:17a}) are expected to happen far more frequent than their prepositional equivalents (\ref{ex:Janic:17b}). The author reports two further pieces of empirical evidence in support of this prediction. The first comes from Surselvan (Indo-European). \citet[50]{Stimm1973}, who carried out a corpus-based study for this language found only two examples of self-benefactive construction of the type (\ref{ex:Janic:17b}), against several hundred constructions of the type (\ref{ex:Janic:17b}). The second evidence comes from English. \citet{Kemmer1993}  extracted from the corpus of written British texts benefactive self-forms, where the constructions of the type (\ref{ex:Janic:17a}) vastly outnumbered those corresponding to example (\ref{ex:Janic:17b}).


\subsubsection{Dative and accusative form of \textit{siebie}} \label{sec:Janic:3.1.4}

The dative form of \textit{siebie} is used in many contexts optionally with a colloquial flavour. The pronoun gains an extra meaning, adding a nuance of casualness, volitionally, subjectivity, aimlessness, perverseness or even disregard. The omission of \textit{siebie} shifts a register of the language into formal. Compare (\ref{ex:Janic:18a}) with (\ref{ex:Janic:18b}).

\ea \label{ex:Janic:18}
\ea \label{ex:Janic:18a}
\gll Jak	sobie	 chcesz.\\
	 as	\textsc{self.dat}	want.\textsc{prs.2sg}\\
\glt	‘As you want.’

\ex \label{ex:Janic:18b}
\gll Jak	chcesz.\\
 	as	want.\textsc{prs.2sg}\\
\glt ‘As you want.’	
\z 
\z 

Concerning the accusative form, \textit{siebie} appears to compete in a formal speech with its light equivalent \textit{się}. The contrastive analysis of this heavy-light opposition in the accusative context reveals some differences. \citet{Sadowska2012} specifically points out the emphatic (\ref{ex:Janic:19a}) and contrastive (\ref{ex:Janic:19b}) function performed by the heavy reflexive form alone. In other accusative contexts, the light form \textit{się} is particularly favoured, as shown in (\ref{ex:Janic:19c}).

\ea \label{ex:Janic:19}
\ea \label{ex:Janic:19a}
\gll Tylko	\textsc{SIEBIE}	/	*się 	widzę		 w	lustrze.\\
		only	\textsc{self.acc} { } \textsc{self}	see.\textsc{prs.1sg}	in	mirror.\textsc{loc.sg(n)}\\
\glt ‘I see only myself in the mirror.’ 
	
\ex \label{ex:Janic:19b}
\gll Widzę	 siebie	/	 *się,	 ale	 ciebie		 nie widzę. \\
	 see.\textsc{prs.1sg}	 \textsc{self.acc} { }	\textsc{self}	but	 \textsc{2sg.acc}	\textsc{neg}	see.\textsc{prs.1sg}	\\
\glt ‘I see myself, but I don’t see you.’ 
	
\ex \label{ex:Janic:19c}
\gll Widzę się	 w	lustrze.\\
		see.\textsc{prs.1sg}	\textsc{self}	in	mirror.\textsc{loc.sg(n)}\\
\glt ‘I see myself in the mirror.’

\z 
\z 


\subsection{Reflexive constructions with the reflexive clitic form \textit{się}} \label{sec:Janic:3.2}
\subsubsection{Functions} \label{sec:Janic:3.2.1}

\textit{Się} literally means ‘self’. One of the functions it performs is to signal in the clause the coreference between two participants. This situation is illustrated in (\ref{ex:Janic:20}), where the agent \textit{oskażony} ‘the accused’ in subject function instead of defending a distinct participant, performs the act of defence on himself. The coreference is signalled by means of the reflexive clitic \textit{się}.

\ea \label{ex:Janic:20}
\gll Oskarżony	 bronił	 się	 w	sądzie.\\
	 accused.\textsc{nom.sg(m)}	defend.\textsc{pst.3sg(m)}	\textsc{self}	in	court.\textsc{loc.sg(m)}\\	
\glt ‘The accused defended himself in the court.’
\z 
	
Like the corresponding independent reflexive form, the clitic form \textit{się} can also signal the reciprocal meaning in a clause. This observation holds particularly for the \textit{się}-constructions that admit the presence of the plural subject. In Polish, such constructions are frequently ambiguous, oscillating between reflexive and reciprocal interpretations, as shown in (\ref{ex:Janic:21}). 

\ea \label{ex:Janic:21}
\gll Asia	 i	Janek	 czesali	 się	 codzienne. \\
	 Asia.\textsc{nom}	and	Janek.\textsc{nom} 	 comb.\textsc{pst.3pl(vir)}	\textsc{self}	every.day\\
\glt i. ‘Every day Asia and Janek combed each other.’
\glt ii. ‘Every day Asia and Janek combed themselves.’ \citep[515]{Wiemer2007}
\z 
	
To disambiguate such clauses, either an extended context or the use of specific adverb is required. For instance, in (\ref{ex:Janic:21}), the reciprocal interpretation becomes explicit if one of the two synonymous adverbs \textit{nawzajem} ‘one another’ or \textit{wzajemnie} ‘each other’ is added to.


\subsubsection{Dative form} \label{sec:Janic:3.2.2}

We have already mentioned in \sectref{sec:Janic:2.2} that in formal register \textit{się} displays an accusative-genitive syncretism alone. However, linguistic descriptions occasionally mention the existence of dative form \textit{se} limited to the colloquial use. The dative status of \textit{się} is intriguing. Since it has gone out of use in standard Polish, many grammars do not discuss it explicitly (\citealt{Feldstein2001, Swan2003, Sadowska2012}). Nevertheless, \textit{se} appears nowadays to be very productive in colloquial usage. At this occasion, \citet{Swan2002} mentions that in informal communication, the dative form \textit{se} is highly marked stylistically. It adds to a communication a flavour of peasants’ talks. This is because the use of the non-standard \textit{se} is a distinctive feature of rural dialects. We observe that nowadays this form is widely accepted even by well-educated people, who employ it to colour their utterances. The colloquial use of dative \textit{se} in Polish contrasts with other Slavic languages like Czech or Bulgarian, in which this form does not carry any stylistic and sociolinguistic implications and is perfectly acceptable in formal register. 

\subsubsection{Positioning} \label{sec:Janic:3.2.3}

The position of the reflexive form \textit{się} within a clause is not stable. As a clitic form, it may have different hosts, preceding or following them. Even if different syntactic and stylistic factors may condition this variable position, some clear tendencies can be distinguished. For instance, \textit{się} favours the second position in a clause, as shown in (\ref{ex:Janic:12}) and (\ref{ex:Janic:23}).

\ea \label{ex:ch:11:22}
\gll Dzieci			 się	 źle		czują.\\
	 child.\textsc{nom.pl(nvir)}	\textsc{self}	bad 	feel.\textsc{prs.3pl}\\		
\glt ‘Children feel bad.’ (Bielec 1998: 59)
\z 

\ea \label{ex:Janic:23}
\gll Janek 	się	 chce	 popatrzyć	 na	ogród.\\
 	Janek\textsc{nom}	\textsc{self}	want.\textsc{prs.3sg} 	look.\textsc{inf} 	at	garden.\textsc{acc.sg(m)}\\
\glt ‘Janek wants to have a look at the garden.’ \citep[391]{SussexCubberley2006}
\z 	
Even if in \ref{ex:Janic:23} \textit{się} is hosted by the infinitive \textit{popatrzyć} ‘to look’ of a subordinate clause, it still occupies the second position of the main clause. This possibility results from the fact that in Polish, a subordinate clause may shift the reflexive clitic to the left to meet a preference of this form for the second position. However, the configuration in which \textit{się} is immediately adjacent to its host \textit{popatrzeć} is also acceptable, as shown in \textit{Janek chce się popatrzyć na ogród}.

According to \citet{Swan2003}, the positioning of \textit{się} with regard to its verbal host is subject to language register. While in formal context, the reflexive clitic favours the post-verbal position, \ref{ex:Janic:24a}, in colloquial speech, it tends to precede the verb, as shown in \ref{ex:Janic:24b}.

\ea \label{ex:Janic:24}
\ea \label{ex:Janic:24a}
\gll Bardzo		spieszę	 się.\\
		very	 hurry.up.\textsc{prs.1sg}	\textsc{self}\\
\glt ‘I am in a big hurry.’
		
\ex \label{ex:Janic:24b}
\gll Bardzo	się	 spieszę.\\
		very	\textsc{self} 	hurry.up.\textsc{prs.1sg}\\	
\glt ‘I am in a big hurry.’		
\z 
\z 

\citet{Swan2002} goes one step further and formulates the correlation between language register, the positioning of \textit{się} and the length of the verbal host: the less formal the style, and the shorter the verb, the more likely it is that \textit{się} will take the position before the verb.

Another context in which \textit{się} demonstrates to some extent more or less stable position involves clauses in which it co-occurs with an enclitic (i.e. unstressed) personal pronoun. In this environment, the reflexive form tends to follow the pronoun rather than to precede it, \ref{ex:Janic:25}.

\ea \label{ex:Janic:25}
\gll On	 mi	 się	 nie	 podoba.\\
	 \textsc{3sg(m).nom}	 \textsc{1sg.dat}	\textsc{self}	\textsc{neg}	like.\textsc{prs.3sg}	\\
\glt ‘I don’t like him.’ (Swan 2002: 318)
\z 
	
Finally, \textit{się} shows a strong regularity in the context of verb-initial-clauses. The clitic systematically occupies the position after the verb. The imperative clause illustrated in \ref{ex:Janic:26} may serve as an illustration of this type of structural configuration.

\ea \label{ex:Janic:26}
\gll Śpiesz	 	 się	 powoli!\\
	 hurry.up.\textsc{imp.2sg}	\textsc{self}	slowly\\
\glt ‘Hurry up slowly.’					
\z 

In some contexts, the employment of \textit{się} may be optional. This is particularly noticeable with a multiple use of the reflexive verbs within a single clause, wherein there is a strong tendency not to repeat the final \textit{się}, \REF{ex:Janic:27}.

\ea \label{ex:Janic:27}
\ea \label{ex:Janic:27a}
\gll Chłopcy	 	 myją		 się	 i	 ubierają		 (się).\\
		boy.\textsc{nom.pl(vir)}	 wash.\textsc{prs.3pl}	\textsc{self}	and	 dress.up.\textsc{prs.3pl(vir)}	\textsc{self}\\
\glt ‘The boys are washing and dressing (themselves).’

\ex \label{ex:Janic:27b}
\gll Kasia		 uczy		 się	 i	bawi		 (się).	\\
		Kasia.\textsc{nom}	learn.\textsc{prs.3sg}	\textsc{self}	and	play.\textsc{prs.3sg}	\textsc{self}\\
\glt ‘Kasia learns and plays’.	\citep[60]{Bielec1998}
\z 
\z 
	
Another context worth mentioning involves the preposition phrase, in which the occurrence of \textit{się} is prohibited. Thus, in \REF{ex:Janic:28}, the only possible way to express the coreference of the subject is to make use of the corresponding accented reflexive form \textit{siebie}.

\ea \label{ex:Janic:28}
\gll Patrzę		 na	siebie	/	 *się	 w	lustrze.\\
	 look.\textsc{prs.1sg}	on	\textsc{self.acc} { }	\textsc{self}	in	mirror.\textsc{loc.sg(n)}\\
\glt ‘I look at myself in the mirror.’			
\z 

\subsubsection{Reflexive verbs} \label{sec:Janic:3.2.4}

The term ‘reflexive verb’ refers to any type of verb equipped with the form \textit{się}, without necessarily implying the coreference meaning (e.g. \textit{spieszyć się} ‘to hurry up’). Reflexive verbs recognize a three-fold partition. The first group involves reflexive verbs that have active counterparts and where the presence of \textit{się} does not affect the lexical meaning of the verb (\textit{myć} ‘to wash sb.’ vs. \textsc{myć się} ‘to wash oneself’, \textit{zginać} ‘to bend sth.’ vs. \textit{zginać się} ‘to bend oneself’). 

The second class encompasses reflexive verbs called deponents i.e. the verbs that do not have non-reflexive counterparts \citep[251]{Kemmer1993}, such as \textit{bać się} ‘to fear’, \textit{bawić się} ‘to have a good time’, \textit{śmiać się} ‘to laugh’, \textit{opiekować się} ‘to look after’, \textit{kłócić się} ‘to argue’, \textit{uśmiechać się} ‘to smile’. Another characteristic of this group is that even if they couple with the reflexive clitic \textit{się}, it is difficult to assign any particular meaning to this form. Finally, in Polish, deponent verbs often demonstrate a complex morphological form, being derived either from verbs, adjectives or nouns. When derived from verbs, they carry one of the following prefixes: \textit{do-, na-, o-(ob-), od-, po-, prze-, przy-, roz-, u-, w-, wy-, z-, za-} \citep[256]{Brooks1975}.

The last group encompasses lexicalized reflexive forms i.e. the verbs that have active counterparts, but wherein the presence of \textit{się} shifts the lexical meaning of the base verb. The meaning of the resulting lexicalized verbs is related in one way or another to the original meaning of the base verb, as in \textit{uczyć} ‘to teach’ vs. \textit{uczyć się} ‘to learn’, \textit{czuć} ‘to detect a smell’ vs. \textit{czuć się} ‘to feel’, \textit{chwalić} ‘to praise’ vs. \textit{ chwalić się }‘to boast’.

Reflexive verbs occur in all conjugation forms and follow the same tense rules as active counterparts. Both syntactically intransitive and transitive verbs accept the presence of the reflexive form \textit{się}. Regarding the former, the language imposes some restrictions on possible combination that are difficult to grasp by a general rule (but see \sectref{sec:Janic:4.2} on impersonal use of \textit{się}). Hence, this constellation must be learnt individually, on a case-by-case basis. Regarding the combination of transitive verbs with \textit{się}, many of such verbs occur with the reflexive form without any constraints. In such a case, the clitic \textit{się} functions as valency-changing operator that reduces the syntactic transitivity of an input verb (cf. \sectref{sec:Janic:4}). The fact that the \textit{się}-verbs do not retain the syntactic structure of the core verb may serve as indication of this reduction,(\ref{ex:Janic:29}).

\ea \label{ex:Janic:29}
\ea \label{ex:Janic:29a}
\gll Chłopiec			 chwycił	 gałąź.\\
	 boy.\textsc{nom.sg(m)} 	grap.\textsc{pst.3sg(m)}	branch.\textsc{acc.sg(f)}\\
\glt ‘The boy grasped the branch.’

\ex \label{ex:Janic:29b}
\gll Chłopiec		 chwycił	 się	 gałęzi.\\
	 boy.\textsc{nom.sg(m)} 	grap.\textsc{pst.3sg(m)}	\text{self}	branch.\textsc{gen.sg(f)}\\
\glt ‘The boy grasped the branch.’
\z 
\z 
	
In (\ref{ex:Janic:29b}), the argument coding of the reflexive verb \textit{chwycić się} ‘to grasp oneself’ differs from the one associated with the corresponding transitive verb \textit{chwycić} ‘to grasp’, (\ref{ex:Janic:29a}). The object is no longer coded like a core argument, since it carries the oblique i.e. genitive case.

In Polish, not only verbs can host \textit{się}. Deverbal nouns can also perform this function. Hence, the expressions such as \textit{mycie się zimną wodą} ‘washing oneself with cold water’, where the reflexive noun \textit{mycie się} relates to the verb \textit{myć się} ‘to wash oneself’, is perfectly acceptable. A similar observation holds for the non-clitic form \textit{siebie}. The ability to combine deverbal nouns with the reflexive forms seems to be rare in the languages of the world. Among Slavic languages, only Polish seems to attest this possibility \citep{SussexCubberley2006}).


\subsection{Reflexive constructions with the reflexive possessive pronoun \textit{swój}} \label{sec:Janic:3.3}

As indicated in \sectref{sec:Janic:2.3}, Polish makes a formal distinction between third-person reflexive possessive pronouns and their nonreflexive counterparts. This split leads to coreference vs. disjoint-reference opposition as illustrated in (\ref{ex:Janic:30}) and (\ref{ex:Janic:31}). 

\ea \label{ex:Janic:30}
\gll Marek		 odwiedza	 swojego		 brata, a	nie	 jego			 brata.\\
	 Marek.\textit{nom}	visit.\textsc{prs.3sg}	\textsc{ref.poss.acc.sg(m)}	borther.\textsc{acc.sg(m)} but \textsc{neg}	\textsc{poss.gen.sg(m)}	borther.\textsc{gen.sg(m)}\\
\glt ‘Mark is visiting his (own) brother and not his (someone else’s) brother.’
\z 

\ea \label{ex:Janic:31}
\gll Dzieci		 nie	 mają		 swoich			 paszportów, my	 mamy	 ich		 paszporty.\\
	 child.\textsc{nom.pl(nvir)}	\textsc{neg}	have.\textsc{prs.3pl}	\textsc{refl.poss.gen.pl(nvir)}	passport.\textsc{gen.pl(nvir)}	\textsc{1pl.nom}	have.\textsc{prs.3pl}	\textsc{poss.acc.3pl(nvir)}	 passport.\textsc{acc.pl(nvir)}\\
\glt ‘The children do not have their (own) passports, we have their passports.’ \citep[162]{Bielec1998} 
\z 		
		

In (\ref{ex:Janic:30}), the accusative form \textit{swój} ‘his own’ co-refers with the subject, which is not a case of its non-reflexive anaphoric counterpart \textit{jego}. The same contrast holds in (\ref{ex:Janic:31}) between \textit{swoich} and \textit{ich}, meaning ‘their’. In the context of the first and second-person possessive pronouns, the referential ambiguity does not hold any more and the choice between reflexive and non-reflexive forms is in general stylistically determined, (\ref{ex:Janic:32})-(\ref{ex:Janic:33}), \citep[73]{Feldstein2001}. 

\ea \label{ex:Janic:32}
\gll Mam		 moją		 /	 swoją	 książkę.\\
 	have.\textsc{prs.1sg}	\textsc{poss.acc.1sg(f)}	/	 \textsc{refl.poss.acc.sg(f)}	book.\textsc{acc.sg(f)}	\\	
\glt ‘I have my / my own book.’
\z 

\ea \label{ex:Janic:33}
\gll Masz		 twoją 	 / 	 swoją	 książkę.\\
	 have.\textsc{prs.2sg}	\textsc{poss.acc.2sg(f)}	/ 	 \textsc{refl.poss.acc.sg(f)}	book.\textsc{acc.sg(f)} \\
\glt ‘You have your / your own book.’					 (Feldstein 2001: 73)
\z 
 	
Unlike English and many other languages, Polish is not very prone to code the possessive relation overtly. This includes both inalienable and alienable possession. When the context is transparent, there is a tendency to omit a possessive pronoun. This is clear in the following two examples: in (\ref{ex:Janic:34})it is self-evident that the addressee can only close his/her own eyes; and that in (\ref{ex:Janic:35}) the agent could only defend the dissertation that she is the author of. 

\ea \label{ex:Janic:34}
\gll Zamknij	oczy.\\
	 close.\textsc{imp.2sg}	eye.\textsc{acc.pl(nvir)}\\
\glt ‘Close your eyes.
\z

\ea \label{ex:Janic:35}	
\gll Obroniłam	 doktorat		 pod	 koniec		 2013	roku.\\
	 defend.\textsc{pst.1sg}	thesis.\textsc{acc.sg(m)}	under	end.\textsc{acc.sg(m)}	2013	year.\textsc{gen.sg(m)}\\
\glt ‘I defended my dissertation at the end of 2013.’
\z
	
However, some contexts ask for explicit coding of the possessive relation. When the possessor is not subject, a possessive pronoun serves to clarify the meaning, as shown in (\ref{ex:Janic:36}). 

\ea \label{ex:Janic:36}
\gll Jadę odwiedzić jego babcię.\\
	 go.\textsc{fit.1sg}		visit.\textsc{inf}		\textsc{poss.acc.3sg}	grandmother.\textsc{acc.sg.(f)}\\
\glt ‘I am going to visit his (not mine) grandma.’
\z 	
	
The possessive relationship is also explicitly coded in the context of contrastive emphasis. The comparison of (\ref{ex:Janic:37a}) with (\ref{ex:Janic:37}) manifests this contrast.

\ea \label{ex:Janic:37}
\ea \label{ex:Janic:37a}
\gll Weź			 ubrania	 i	 daj	 mi	 święty	spokój.\\
		take.\textsc{imp.2sg}	clothes.\textsc{acc.pl(nvir)}	and	 give.\textsc{imp.2sg}	\textsc{1sg.dat}	saint\textsc{ac.sg(m)} 	peace.\textsc{ac.sg(m)}\\
\glt ‘Take (your) clothes and leave me in peace.’
	
\ex\label{ex:Janic:37b}
\gll Weź 	 \textsc{SWOJE}	 ubrania	 a	\textsc{MOJE} zostaw w	spokoju.\\
		take.\textsc{imp.2sg}	\textsc{refl.poss.acc.pl(nvir)}	 clothes.\textsc{acc.pl(nvir)}	and	\textsc{poss.acc.pl(nvir)} leave.\textsc{imp.2sg} in	peace.\textsc{loc.sg(m)}\\
\glt ‘Take your clothes and mine leave in peace.’			
\z 
\z 

Finally, the reflexive possessive pronoun \textit{swój} ‘one’s own’ is also used when a speaker intends to highlight the greater specificity of the possessed item. Contrast (\ref{ex:Janic:38a}) with (\ref{ex:Janic:38b}). 

\ea \label{ex:Janic:38}
\ea \label{ex:Janic:38a}
\gll Ewa	 jeździ			 do	pracy		 samochodem.\\
		Ewa.\textsc{nom}	go.\textsc{prs.3sg}	 to	work.\textsc{loc.sg(f)}	car.\textsc{ins.sg(m)}\\
\glt ‘Ewa drives to work by car.’

\ex \label{ex:Janic:38b}
\gll Ewa		 jeździ	 do	pracy	 swoim			 samochodem.\\
		Ewa.\textsc{nom} go.\textsc{prs.3sg}	to	work.\textsc{loc.sg(f)}	\textsc{refl.poss.ins.sg(m)}	car.\textsc{ins.sg(m)}\\
\glt ‘Ewa drives to work in her own car.’ \citep[180]{Sadowska2012}

\z 
\z 

\section{Related functions performed by the reflexive form \textit{się}}
\label{sec:Janic:4}

The functional scope of the reflexive clitic form \textit{się} goes far beyond the coreference meaning. This fully grammaticalized pronoun is nowadays highly polysemous, performing a range of valency-reducing operations, including middle \sectref{sec:Janic:4.1}, impersonal \sectref{sec:Janic:4.2} and antipassive \sectref{sec:Janic:4.3}. 


\subsection{Middle function} \label{sec:Janic:4.1}

Middle formations denote events in which the subject participant is viewed not only as the doer of the action but also as the place on which this action is performed \citep{Benveniste1966, Kemmer1993, Creissels2006}. Both the doer and the place of the event are construed as one single inseparable entity. This contrasts with the reflexive type of events, in which the subject assumes two semantic roles, agent and patient, the referents of which are conceived as distinct entities. 

In Polish, the reflexive clitic \textit{się} often participates in middle derivations. \citet[20]{Swan2003} argues that the use of \textit{się} to code the coreference meaning (i.e. reflexive and reciprocal) is definitely not as frequent as to express middle types of events. The author reports the particularly frequent presence of się in grooming actions e.g. \textit{czesać się} ‘to comb oneself’, \textit{myć się} ‘to wash oneself’, \textit{kąpać się} ‘to bathe oneself’,(\ref{ex:Janic:39}), or \textit{golić się} ‘to shave oneself’, (\ref{ex:Janic:40}).

\ea \label{ex:Janic:39}
\gll Codziennie	się	 kąpię.\\
	 every.day	\textsc{self}	bathe.\textsc{prs.1sg}\\
\glt ‘I take a bath every day.’
\z 
	
\ea \label{ex:Janic:40}
\gll Golę		 się		 przed	śniadaniem. \\
	 shave.\textsc{prs.1sg}	\textsc{self}		before	breakfast.\textsc{ins.sg(n)}\\
\glt ‘I shave before breakfast.’						 \citep[584]{Swan2003}
\z 	
	
Grooming verbs may denote actions performed either on the whole body or only on its part. In Polish, the coding of whole-body actions may differ from body-part actions. For instance, when the action targets a particular body part, the language calls for a transitive construction with a body-part referent expressed as object. Compare (\ref{ex:Janic:41}) with (\ref{ex:Janic:42}).

\ea \label{ex:Janic:41}
\gll Muszę		 się	 umyć.\\
	 have.to.\textsc{prs.1sg}	\textsc{self}	wash.\textsc{inf}\\
\glt ‘I have to wash up.’
\z
	
\ea \label{ex:Janic:42}
\gll Muszę		 umyć		 ręce.\\
	 have.to.\textsc{prs.1sg}	wash.\textsc{inf}	hand.\textsc{acc.pl(nvir)}\\
\glt ‘I have to wash my hands.’						 \citep[584]{Swan2003}
\z 
	
Within a middle domain, the clitic form is also productively engaged in coding the change of body posture such as kłaść ‘to lie down’ vs. \textit{kłaść się} ‘to lie down oneself’, \textit{podnieść} ‘to uplift’ vs. \textit{podnieść się} ‘to get up’, \textit{opierać} ‘to lean’ vs. \textit{opierać się} ‘to lean against’. Another middle type of events with \textit{się} involves non-translational motions like \textit{obrócić} ‘to turn’ vs. \textit{obrócić się} ‘to turn oneself’. The reflexive form się is also highly productive in expressing emotional reactions or mental agitations: \textit{złościć} ‘to make sb. angry’ vs. \textit{złościć się} ‘to get angry’, \textit{rumienić} ‘to brown sth’ vs. \textit{rumienić się } ‘to blush’, \textit{martwić } ‘to make sb. worry’ vs. \textit{martwić się} ‘to worry oneself’, \textit{denerwować} ‘to make sb. angry’ vs. \textit{denerwować się} ‘to get angry’. Finally, derivations in \textit{się} are also open to decausative reading. The latter refers to verbs that express a change of state or physical process with no clearly implied agent, as shown in (\ref{ex:Janic:43}).

\ea \label{ex:Janic:43}
\gll W	tym		 czajniku	 woda	 gotuje	 się 	 bardzo	szybko.\\
 	in	\textsc{dem.loc.sg(m)}	kettle.\textsc{loc.sg(m)}	water.\textsc{nom.sg(f)}	boil.\textsc{prs.3sg}	\textsc{self}	very	quickly\\
\glt ‘In this kettle, the water boils very quickly.’	
\z 
	
In Polish, decausative formations alternate with impersonal reflexive derivations (cf. \sectref{sec:Janic:4.2}). Both remain in a close semantic affinity, revealing, however, a slight semantic difference. Unlike impersonal reflexive verbs, (\ref{ex:Janic:44a}) decausative ones, (\ref{ex:Janic:44b}), do not imply any potential agent, which would be necessarily involved in the development of an action denoted by a verb.

\ea \label{ex:Janic:44}
\ea \label{ex:Janic:44a}
\gll Kawę				 się		 gotuje.\\
		coffee.\textsc{acc.sg(f)}	\textsc{self}	boil.\textsc{prs.3sg} \\	
\glt ‘The coffee is being boiled.’

\ex \label{ex:Janic:44b}
\gll Kawa		 się	 gotuje.\\
		coffee.\textsc{nom.sg(f)}		\textsc{self}	boil.\textsc{prs.3sg}\\
\glt ‘The coffee is boiling.' \citep[320]{Swan2002} 
\z 
\z 


\subsection{Impersonal function} \label{sec:Janic:4.2}

Polish has a well-developed impersonal system. It recognizes three kinds of impersonal constructions, including impersonal reflexive, impersonal passive and impersonal with dedicated verbal \textit{-no/-to} forms. Impersonal reflexive constructions, (\ref{ex:Janic:45}), select a verb in an invariable third person singular form. The sentence lacks the grammatical subject, which leads to the impersonal interpretation. Polish employs reflexive impersonal constructions extensively, which constitutes one of the peculiarities of the grammatical system of this language.

\ea \label{ex:Janic:45}
\gll Rozumie	 się.	\\
	 understand.\textsc{prs.3sg}	\textsc{self}\\
\glt ‘It is understandable.’	
\z 
	
When referring to the past events, impersonal reflexive verbs occur invariably in the third person neuter singular past-tense indicative form, equipped with the suffix \textit{-ło}, (\ref{ex:Janic:46}). When denoting the present events, they are in the third person singular present-tense form, (\ref{ex:Janic:47}). Finally, in the context of future events, predicates are complex, consisting of an auxiliary in the third person singular future-tense form and the third person neuter singular past \textit{-ło} form, (\ref{ex:Janic:48}).

\ea \label{ex:Janic:46}
\gll Kiedyś		wyłącznie	 pisało			 się	 listy. \\
 once		exclusively	 write.\textsc{pst.3sg(n)}	\textsc{self}	letter.\textsc{acc.pl(nvir)}	 \\
\glt ‘In the past only letters were written.’
\z 
	
\ea \label{ex:Janic:47}
\gll Teraz	 pisze	 się	 listy			 i	 e-maile. \\
	 now	 write.\textit{prs.3sg}	\textsc{self}	letter.\textsc{acc.pl(nvir)}	and	 email.\textsc{acc.pl(nvir)}\\
\glt ‘Now letters and emails are [being] written.’
\z 

\ea \label{ex:Janic:48}	
\gll W	przyszłości		 będzie		 się	 pisało	 tylko 	e-maile. \\
	 in	future.\textsc{loc.sg(f)}	be.\textsc{fut.3sg}	\textsc{self}	write.\textsc{pst.3sg(n)}	only	email.\textsc{acc.pl(nvir)}\\
\glt ‘In the future only emails and SMSs will be written.’	\citep[428]{Sadowska2012}
\z 
	
Another distinctive feature of Polish impersonal reflexive constructions is that their verbs accept a direct object much in the same way as corresponding active verbs. However, what is atypical for them and what distinguishes these constructions from their equivalents in other languages e.g. Serbo-Croatian is that this noun phrase occurs in accusative rather in nominative, and that a verb invariably remains in the third person singular form. This type of construction is an approximate equivalent of the English clauses translated by ‘one’, ‘you’ or ‘they’. Example (\ref{ex:Janic:49}) illustrates this point.

\ea \label{ex:Janic:49}
\ea \label{ex:Janic:49a}
\gll Owe	 przesądy		 dzisiaj		inaczej		się	 interpretuje.\\
	 such	prejudice.\textsc{acc.pl(nvir)}	today		differently	\textsc{self}	interpret.\textsc{prs.3sg}\\
\glt ‘One interprets such prejudices differently nowadays.’

\ex \label{ex:Janic:49b}
\gll Sprawę		 załatwi	 się	 od	 ręki.\\
	 matter.\textsc{acc.sg(f)}	fix.\textsc{prs.3sg} 	\textsc{self}	from	hand.\textsc{gen.sg(f)}\\
\glt ‘One will fix the matter without any problems.’	 \citep[262, 246]{Siewierska1988}
\z 
\z 
	
Impersonal reflexive constructions may also occur with dative arguments. The latter can be either represented by a personal pronoun e.g. \textit{ci}, (\ref{ex:Janic:50}), or by a noun phrase e.g. \textit{ludziom}, (\ref{ex:Janic:51}). 

\ea \label{ex:Janic:50}
\gll Jak	 ci		 się	 spało?\\
	 how		\textsc{2sg.dat}	\textsc{self}	sleep.\textsc{pst.3sg(n)}\\
\glt ‘How did you sleep?’ (lit. How was sleeping to you?)
\z 
	
\ea \label{ex:Janic:51}
\gll Czy	 ludziom	 się	 tu	 dobrze	mieszka?\\
 	Q		people.\textsc{dat.pl(nvir)}	\textsc{self}	here	well	live.\textsc{prs.3sg}	\\
\glt ‘Do people live happily here?’ (lit. Is living happy to people here?)	 \citep[60]{Bielec1998} 
\z 
		
When compared to the corresponding active constructions, impersonal reflexives occurring with dative may imply a nuance of involuntary act, as in (\ref{ex:Janic:52b}) or disclaim responsibility, as in (\ref{ex:Janic:53b}). The semantic difference is, however, very subtle and difficult to grasp by English translations. 

\ea 
\label{ex:Janic:52}

\ea 
\label{ex:Janic:52a}
\gll Dobrze 	śpię.\\
		well	sleep.\textsc{prs.1sg}\\
\glt ‘I sleep well.’

\ex \label{ex:Janic:52b}
\gll Dobrze	mi	 się		 sypia.\\
		well	\textsc{1sg.dat}	\textsc{self}	sleep.\textsc{prs.3sg}\\	
\glt ‘I sleep well.’
\z 
\z 

\ea \label{ex:Janic:53}
\ea \label{ex:Janic:53a}
\gll Tak	 tylko	powiedziałem.\\
		so		only	say.\textsc{pst.1sg(m)}	\\	
\glt ‘I only said that (i.e. I did not mean it).’

\ex \label{ex:Janic:53b}	
\gll Tak	mi	 się		 tylko	powiedziało.\\
		so	\textsc{1sg.dat}	\textsc{self}	only	say.\textsc{pst.3sg(n)}	\\
\glt ‘I only said that (i.e. I did not mean it).’				\citep[312]{Swan2002}
\z 
\z
	
In the past tense, impersonal reflexive clauses, (\ref{ex:Janic:54a}), may alternate with dedicated \textit{-no/-to} impersonals i.e. constructions with the neutral singular past indicative verbal form, (\ref{ex:Janic:54b}). Both types of impersonal clauses remain in strong semantic affinity and are subject to free variation.

\ea \label{ex:Janic:54}
\ea \label{ex:Janic:54a}
\gll Wymieniło			 się		 kilka	 nazwisk.\\
		mention.\textsc{pst.3sg(n)} \textsc{self}	few.\textsc{acc}	name.\textsc{acc.pl(nvir)}\\	
\glt ‘Several names were mentioned.’

\ex \label{ex:Janic:54b}
\gll Wymieniono	 	kilka 	nazwisk.\\
		mention.\textsc{pst.3sg(n)}		few\textsc{acc} name.\textsc{acc.pl(nvir)}\\
\glt ‘Several names were mentioned.’	\citep[316]{Swan2002}
\z 
\z 
		
The occurrence of \textit{się} in the impersonal context is very high. This may result from the fact that active verbs that normally do not combine with the reflexive clitic realise this restriction in the impersonal context. Practically, any non-\textit{się}-verbs can admit the reflexive clitic to express the impersonal meaning as \textit{być} ‘to be’ and \textit{mieć} ‘to have’ in (\ref{ex:Janic:55}), or \textit{spać} ‘to sleep’ in (\ref{ex:Janic:56}). 

\ea \label{ex:Janic:55}
\gll Jak	się	 było		 młodym,	 to	 się	 miało	 więcej	czasu.\\
	 how	\textsc{self}	be.\textsc{pst.3sg(n)}	young.\textsc{ins.sg(m)}	then	\textsc{self}	have.\textsc{pst.3sg(n)}	more	time.\textsc{acc.sg(m)}\\
\glt ‘As you were young, you had more time.’
\z 

\ea \label{ex:Janic:56}
\gll Tutaj	się	 dobrze	śpi.\\
	 here	\textsc{self}	well	sleep.\textsc{prs.3sg}\\
\glt ‘One sleeps well here.’							 \citep[60]{Bielec1998}
\z 
	
Impersonal reflexive clauses are particularly frequent in the interrogative context, (\ref{ex:Janic:57}). 

\ea \label{ex:Janic:57}
\ea \label{ex:Janic:57a}

\gll Jak	tam	 się			 jedzie?\\
		how	there	\textsc{self}	go.\textsc{prs.3sg}	\\
\glt ‘How does one get there?’

\ex \label{ex:Janic:57b}
\gll Co	 się	 mówi		 w	takiej	 sytuacji?\\
		what	\textsc{self}	say.\textsc{prs.3sg}	in	such	 situation.\textsc{loc.sg(f)}\\
\glt ‘What does one say in such a situation?’ 				 \citep[320]{Swan2002}
\z 
\z 
	
In impersonal reflexives, the implicit subject receives a human, indefinite interpretation. Thus, it may be unknown, generic and/or of a low degree of specificity. Logically such clauses cannot occur with overtly expressed subject and can only refer to the situations based on human activities, leading to a three-fold distinction: requests, as in (\ref{ex:Janic:58a}), commands, as in (\ref{ex:Janic:58b}), and statements, as in (\ref{ex:Janic:58c}).

\ea \label{ex:Janic:58}
\ea \label{ex:Janic:58a}

\gll Jak	się 	 jedzie		 do	Łodzi?\\
		how	\textsc{self}	go.\textsc{prs.3sg}		to	Łódź.\textsc{gen} \\
\glt ‘How do you get to Łódź? 	\citep[583]{Swan2002}

\ex \label{ex:Janic:58b}
\gll Tak	 się	 mówi.\\
		so		\textsc{self}	say.\textsc{prs.3sg}\\			
\glt ‘That’s how it is said.’	\citep[60]{Bielec1998}
	
\ex \label{ex:Janic:58c}
\gll Tutaj	się	 tańczyło.\\
	 here	\textsc{self}	dance.\textsc{pst.3sg(n)}\\
\glt ‘There was dancing here.’ 
\z 
\z 
	
Whether impersonal reflexive constructions are indeed subjectless is in fact a matter of controversy in linguistic discussions. For instance, \citet{Comrie1985} approaches this type of constructions as impersonal passive clauses with no overt subject and where the implied human agent is represented as a demoted underlying subject. On the other hand, \citep{Siewierska1988} mentions that in the tradition of the Polish linguistics, impersonal reflexives are often viewed as fully active clauses where the implied human agent is both the underlying and surface subject. The description by Swan \citep[538]{Swan2003} aligns with this observation. The author argues that \textit{się} occupies a quasi-nominal position, functioning \textit{de facto} as subject.



\subsection{Antipassive function} \label{sec:Janic:4.3}

The reflexive clitic \textit{się} may also perform the antipassive type of valency-changing operation. This means that it operates on a transitive verb without affecting the semantic roles of the associated arguments. The resulting construction is syntactically intransitive, wherein the P argument loses the properties of a core argument. The syntactically downgraded P argument can either be realized as oblique, or as in (\ref{ex:Janic:59b}) eliminated from the surface structure of a verb.

\ea \label{ex:Janic:59}
\ea \label{ex:Janic:59a}
\gll Wasz  syn bije dzieci.\\
	 \textsc{poss.nom.sg(m)} son.\textsc{nom.sg(m)}	beat.\textsc{prs.3sg}	child.\textsc{acc.pl(nvir)}\\
\glt ‘Your son beats up the children.’

\ex \label{ex:Janic:59b}
\gll Wasz  syn  bije  się.\\
 	\textsc{poss.nom.sg(m)}	 son.\textsc{nom.sg(m)}	beat.up.\textsc{prs.3sg} \textsc{self}\\
\glt ‘Your son has a tendency to beat up [others].’

\z
\z
	
In general, Polish antipassive constructions with omitted P-argument are characterised by the fact that this argument is in fact suppressed (or syntactically ‘blocked’). Hence, it cannot be overtly realized. This type of an antipassive construction is known in the literature under the label ‘absolutive antipassive’. In Polish, the suppressed argument of absolutive antipassive clauses systematically receives a human interpretation. Unless explicitly specified by a context, it tends to display a low degree of specificity, triggering a generic, indefinite and/or non-referential reading. The verb denotes, irrealis, generic type of event, whereas the agent participant is viewed as having special inclination or tendency to perform a denoted action. 

Polish reveals a strong correlation between lexical meaning of a verb and the type of antipassive structure in which it occurs. Specifically, only verbs expressing the antagonistic action such as \textit{kopać} ‘to kick’, \textit{szczypać} ‘to pinch’, \textit{pchać} ‘to push’ \textit{przezywać} ‘to nickname’, \textit{bić} ‘to beat up’, \textit{drażnić} ‘to annoy’, \textit{drapać} ‘to scratch’ \textit{chlapać} ‘to splash’, \textit{gryźć} ‘to bite’ and \textit{pluć} ‘to spit’ can occur in absolutive antipassive constructions (Janic 2016: 157).


\section{Diachronic development} \label{sec:Janic:5}

In her discussion on middle voice, \citet{Kemmer1993} classifies languages according to whether they express reflexive and middle functions through the same form or not. In case of the positive results, the author raises the question of whether these forms are related diachronically. Subsequently, she typologies languages into three types: i) those with one-form middle system, ii) those with two-form cognate system, and iii) those with two-form non-cognate system. Polish belongs to the second type, which is considered to be rare crosslinguistically. Among other languages with a two-cognate system, one can also mention Jola (Atlantic-Congo) with -ɔrɔ and ɔ distinctive though diachronically related forms and other Slavic languages. 

\citet{Kemmer1993}  argues that a two-form cognate system results from a diachronic process of repartition \citep{Bréal1897}. The outcome of such an evolution is a division of a single form into two distinct, heavy and light forms. The heavy form usually displays (pro)nominal features, whereas the latter due to grammaticalization shares the characteristic of clitics. The occurrence of the light form results from renewing or reinforcing of the heavy form. This form is reintroduced to a language system as relatively independent element. Then, due to coalescence or erosion, it undergoes phonological reduction. Thus, at the synchronic level, the light form is viewed as a reduced form of the heavy form. The formal split of a single form converges with the semantic division of labour. The light form is typically assigned to the middle domain, in contrast to its heavy counterpart, which maintains its initial coreference meaning.

The analogous development took place in Polish, where the light form \textit{się} that demonstrates the properties of clitics originated in the heavy form \textit{siebie}. The formal split aligned with the semantic extension. The grammaticalized form \textit{się} extended the functional scope to the middle domain, preserving, however, the initial reflexive function. The next step of grammaticalization involves desemanticization (or ‘semantic bleaching’) where in some contexts, the clitic \textit{się} loses the semantic content and starts to operate on a structural basis alone (e.g. impersonal or antipassive). In Polish, the encroachment of \textit{się} into more structural-based field did not, however, lead to its total desemanticization \citep{Heine1999}. Even if \textit{się} is particularly frequent in impersonal contexts, its omnipresence in middle or reflexive domains is non-negligible either.

 
 
\section*{Abbreviations}

{\sloppy\printbibliography[heading=subbibliography,notkeyword=this]}
\end{document}
