\documentclass[output=paper,colorlinks,citecolor=brown,modfonts,nonflat]{langscibook}
\author{Roberto Zariquiey\affiliation{Pontificia Universidad Católica del Perú}}
\title{Reflexive constructions in Kakataibo (Pano, Peru)}
\abstract{The present paper presents a discussion of reflexive constructions in Kakataibo, a Pano language spoken in Peruvian Amazonia. The language exhibits a productive verbal reflexive, which is mainly used on transitive verbs, as well as a middle marker which is also used to express reflexive meanings. Kakataibo emphatic pronouns and the noun \textit{nami} ‘body’ can also participate in reflexive constructions, but require additional indicators of co-referentiality: emphatic pronouns require the presence of the verbal reflexive marker, whereas the noun \textit{nami} ‘body’ needs a possessive marker and an emphatic clitic. Their need for extra markers of co-reference suggests that neither emphatic pronouns nor the noun \textit{nami} ‘body’ are fully grammatical reflexive nominals.}
\IfFileExists{../localcommands.tex}{
  \input{../localpackages}
  %Copy this to localcommands.tex

\usepackage[english]{babel}
\usepackage{amsmath}
\usepackage{amssymb,amsfonts,textcomp}
\usepackage{array}
\usepackage{hhline}
\usepackage{hyperref}

\newenvironment{styleStandard}{}{}
\newenvironment{stylelsAbstract}{}{}
\newenvironment{stylelsSectioni}{}{}
\newenvironment{stylelsSectionii}{}{}
\newenvironment{stylelsBulletList}{}{}
\newenvironment{styleBibliographyi}{}{}
\newenvironment{listWWNumxxvleveli}{}{}
\newenvironment{listWWNumxxvlevelii}{}{}
\newenvironment{listWWNumxxvleveliii}{}{}
\newenvironment{listWWNumxxvleveliv}{}{}
\newenvironment{listWWNumixleveli}{}{}
\newenvironment{listWWNumixlevelii}{}{}
\newenvironment{listWWNumixleveliii}{}{}
\newenvironment{listWWNumixleveliv}{}{}

\newcommand\textstyleListLabelxvi[1]{#1}
\newcommand\labellistWWNumxxvleveli{\thelistWWNumxxvleveli.}
\newcommand\labellistWWNumxxvlevelii{\thelistWWNumxxvlevelii.}
\newcommand\labellistWWNumxxvleveliii{\thelistWWNumxxvleveliii.}
\newcommand\labellistWWNumxxvleveliv{\thelistWWNumxxvleveliv.}
\newcommand\labellistWWNumixleveli{[F0B7?]}
\newcommand\labellistWWNumixlevelii{\textstyleListLabelxvi{o}}
\newcommand\labellistWWNumixleveliii{[F0A7?]}
\newcommand\labellistWWNumixleveliv{[F0B7?]}

\newcounter{listWWNumxxvleveli}
\newcounter{listWWNumxxvlevelii}[listWWNumxxvleveli]
\newcounter{listWWNumxxvleveliii}[listWWNumxxvlevelii]
\newcounter{listWWNumxxvleveliv}[listWWNumxxvleveliii]
\newcounter{itemize}  
  \input{../localhyphenation} 
  \bibliography{../localbibliography}
  \togglepaper[28]%%chapternumber
}{}

\begin{document}

\maketitle 

\section{Introduction}\label{sec:zariquiey:1}

The present paper presents a discussion of reflexive constructions in Kakataibo, a Pano language spoken in Peruvian Amazonia. Following Haspelmath’s (this volume) definition of reflexive constructions, I discuss here all the Kakataibo constructions that satisfy the two main criteria listed in \REF{ex:zariquiey:1} (taken from Haspelmath, this volume: 1).\todo{Link to be added}

\ea%1
    \label{ex:zariquiey:1}
    A reflexive construction is a grammatical construction\\
    \begin{itemize}
        \item[(i)] that can only be used when two participants of a clause are coreferential; and 
        \item[(ii)] that contains a special form (a reflexivizer) that signals this coreference.
    \end{itemize}
\z

The criteria stipulated in \REF{ex:zariquiey:1} are satisfied in Kakataibo by constructions with a reflexive verbal marker. Some constructions featuring emphatic pronouns or the noun \textit{nami} ‘body’ also fit the definition in \REF{ex:zariquiey:1}, under the condition that they also carry a verbal reflexive marker. Reflexive voice markers include the reflexive marker and the middle marker (see \citealt[306ff]{Zariquiey2018}). The reflexive voice marker is a truly derivational verbal suffix (with a rich morphophonology), whereas the middle marker is used in some cases as a derivative marker, while in other instances it is part of what \citet[416ss]{Valenzuela2017} calls “alternative derivation” (pairs of transitive-intransitive verbs obtained from bound roots which lack a transitive category and can never be used in discourse by themselves; see \sectref{sec:zariquiey:3.2} for some illustration of this). 

This chapter is structured as follows. In \sectref{sec:zariquiey:2}, I present some background information on the Kakataibo language and its speakers. The verbal markers that express reflexive meanings are discussed in \sectref{sec:zariquiey:3} (\sectref{sec:zariquiey:3.1} illustrates the reflexive marker and \sectref{sec:zariquiey:3.2} discusses the middle marker). The reflexive constructions with emphatic pronouns and the noun \textit{nami} ‘body’ are presented in \sectref{sec:zariquiey:4.1} and in \sectref{sec:zariquiey:4.2}, respectively. Finally, some conclusions are listed in \sectref{sec:zariquiey:5}. 

\section{The Kakataibo language and its speakers}\label{sec:zariquiey:2}

Kakataibo is a Pano language spoken by approximately 3000--3500 people in the Peruvian departments of Huánuco and Ucayali. The Kakataibo people live along the Aguaytía, San Alejandro, Shamboyacu, Sungaroyacu and, more recently, Pisqui Rivers (see \figref{fig:zariquiey:1}), where the language remains vital and is learnt by children despite intense contact with Spanish.

\begin{figure}
    \caption{Location of major Kakataibo settlements}
    %\includegraphics[width=\textwidth]{figures/zariquiey-fig1.jpg}
    \label{fig:zariquiey:1}
\end{figure}

Kakataibo is the westernmost Pano language and, within the language family, Kakataibo is the only member of its branch (\citealt{Shell1965, dAns1973, Loos1999}; and \citealt{Fleck2013}). As described in \citet{Zariquiey2011}, there are four extant Kakataibo dialects, spoken in the Lower Aguaytía, Upper Aguaytía, Sungaroyacu and San Alejandro Rivers, respectively. “Nokamán”, a variety named and minimally documented by \citet{Tessman1930}, was a fifth dialect of the language, now extinct \citep{Zariquiey2013}. The Lower Aguaytía variety is the one studied in this paper (for a full grammar of this dialect, see \citealt{Zariquiey2018}).\footnote{ \textrm{The examples in this paper are given in the Kakataibo practical orthography: (IPA symbols are enclosed in square brackets where these differ from the practical representation): a, e, ë [ɨ], i, o, u, p, t, k, ku [k}\textrm{\textsuperscript{w}}\textrm{], b [$\beta ̞$], r [ɾ], m, n, ñ [ɲ], s, sh [ʃ], x [ʂ], ts, ch [tʃ] and ‘ [ʔ].} } This dialect exhibits the phonological inventory given in Tables \ref{tab:zariquiey:1} and \ref{tab:zariquiey:2} (the orthographic conventions followed in this paper are given in angle brackets if different from IPA).

\begin{table}
    \caption{Kakataibo consonant inventory}
    \label{tab:zariquiey:1}
    \fittable{
    \begin{tabularx}{\textwidth}{p{1.7cm}XXXXXXX}
    \lsptoprule
    & {Labial} & {Alveolar} & {Post-alveolar} & {Retroflex} & {Palatal} & {Velar} & {Glottal}\\
        \hline
        {Stop} & p & t & & & & k kʷ <ku> & ʔ <‘> \\
        {Affricate}  &  & ts & tʃ <ch> &  &  &  &   \\
        {Fricative}  &  & s & ʃ <sh> & ʂ <x> &  &  &   \\
        {Nasal} & m & n &  &  & ɲ <ñ> &  &   \\
        {Liquid} &  & ɾ <r> &  &  &  &  &   \\
        {Approximant} & β <b> &  &  &  &  &  &  \\
        \lspbottomrule
    \end{tabularx}
    }
\end{table}

\begin{table}
    \caption{Kakataibo vowel inventory}
    \label{tab:zariquiey:2}
    \begin{tabularx}{\textwidth}{XXXX} 
     \lsptoprule
    & {Front} & {Central} & {Back}\\
    \hline
        {High} & i & ɨ <ë> & u\\
        {Mid} & e &  & o\\
        {Low} &  & a & \\
       \lspbottomrule
    \end{tabularx}
\end{table}

\todo[inline]{/e/ is misplaced.}

Kakataibo is an agglutinative language with scarce instances of fusion. Verbal morphology is far more complex than nominal morphology and verbal forms may include a long number of affixes (see example \ref{ex:zariquiey:2}), although the average number of suffixes per verbal root in natural speech is only 1.64 (see \citealt[150]{Zariquiey2018}).\footnote{The examples included in this paper come both from elicitation and texts. Some of the examples have been adapted from \citet[308]{Zariquiey2018}.}

\ea%2
    \label{ex:zariquiey:2}
    \glll   Pimibëtsintëkënkankëxa\\
            pi-mi-bëtsin-tëkën-kan-akë-x-a\\
            eat-\textsc{caus-}coming:\textsc{tran-}again\textsc{-pl-rem.pst-3-non.prox}\\
    \glt    ‘while coming, they made (someone) eat again a long time ago’
\z    

The language exhibits a complex alignment system that combines ergative and tripartite case marking with accusative subject cross-referencing both on verbs and second position enclitics. Clausal constituent order is pragmatically determined, but there is a tendency towards verb-final sentences. Word order in the noun phrase is not fixed and most nominal modifiers can appear either before or after the nominal head. The language also exhibits a rich switch-reference system and a pervasive use of nominalizations in discourse.

In this chapter, I assume a very basic distinction between transitivity and valence in Kakataibo. In Kakataibo, transitivity is a lexical property of verbs, and two lexical transitivity classes can be distinguished in the language: intransitive and transitive. The transitivity class of the verb is encoded in various parts of the sentence, by means of various transitivity agreement and transitivity harmony phenomena. Therefore, it is always obvious if a verb is lexically transitive or intransitive. This is illustrated in the following examples. In \REF{ex:zariquiey:3}, which features the predicate \textit{pi} ‘eat’, the pronominal subject bears the A-enclitic \textit{=n}, the switch-reference marker takes the form \textit{{}-xun} ‘subject > A, simultaneous event’ and the associated motion suffix is \textit{{}-bëtsin} ‘coming’, which exclusively appear with transitive verbs. In turn, in \REF{ex:zariquiey:4}, which features the verb \textit{tan} ‘rest’, we find the S-enclitic \textit{=x} on the pronominal subject, as well as the switch-reference marker \textit{{}-ax} ‘subject > S, simultaneous event’ and the associated motion suffix \textit{{}-kuantsin} ‘coming’, which exclusively appear with intransitive verbs. All Kakataibo transitive verbs behave like \textit{pi} ‘eat’ and all Kakataibo intransitive verbs behave like \textit{tan} ‘rest’.

\ea%3
    \label{ex:zariquiey:3}
    \gll kuan–\textbf{xun} kana ‘ë=\textbf{n} ‘atsa    pi–\textbf{bëtsin}–i–n\\
            go\textsc{–subj>A:sim} \textsc{nar:1sg} \textsc{1sg=A} manioc eat–coming\textsc{:tran–ipfv}{–1/2}\\
    \glt    {‘Having gone, I am eating manioc while coming.’}
\z

\ea%4
    \label{ex:zariquiey:4}
    \gll    kuan–\textbf{ax} kana ‘ë=\textbf{x} tan–\textbf{kuantsin}–i–n\\
            go–\textsc{subj>S:sim} \textsc{nar:1sg} \textsc{1sg=S} rest–coming:\textsc{intr–ipfv}–1/2\\
    \glt    ‘Having gone, I am resting while coming.’
\z

I understand valence, in turn, as the number of arguments with which a verb is used in a specific construction (following, for example, \citealt[3]{DixonAikhenvald2000}). Although there are strong cross-linguistic associations between transitivity and valence (e.g., transitive verbs have a valence of at least two, whereas verbs with one argument are intransitive), in Kakataibo, the transitivity category of a verb cannot be predicted 100\% of the time based on its valence (see \citealt{Zariquiey2017} and \citealt[217ff]{Zariquiey2018}). This mainly relates to the existence in Kakataibo of a small set of bivalent intransitive predicates, whose non-subject arguments (which are called \textbf{quasi-objects} in \citealt{Zariquiey2017}) are reminiscent of objects due to their lack of marking, but exhibit critical behavioral differences in relation to them. Let us compare the examples in \REF{ex:zariquiey:5} and \REF{ex:zariquiey:6}. The fact that the predicate in \REF{ex:zariquiey:5} is transitive is revealed by the form of the pronominal subject, which bears the A-enclitic \textit{=n}. Therefore, \textit{atsa} ‘manioc’ in \REF{ex:zariquiey:5} is a grammatical object, which remains unmarked as is the case of absolutive arguments. In \REF{ex:zariquiey:6}, we find the bare noun phrase ‘\textit{atsa} ‘manioc’ as the quasi-object of the intransitive verb \textit{pishin} ‘lack’ (note that the subject of \textit{pishin} ‘lack’ carries the S-marker \textit{=x}).

\ea%5
    \label{ex:zariquiey:5}
    \glll   ‘Ën kana ‘atsa pin.\\
            ‘Ë=n  kana  \textbf{‘atsa}  pi–i–n\\
            1\textsc{sg}=A  \textsc{nar:1sg}  manioc  eat–\textsc{ipfv–1/2}\\
    \glt    ‘I eat manioc.’
\z

\ea%6
    \label{ex:zariquiey:6}
    \glll   ‘Ëx kana ‘atsa pishinin.\\
            ‘Ë=x  kana  \textbf{‘atsa}  pishin–i–n\\
            1\textsc{sg}=S  \textsc{nar:1sg}  manioc  lack–\textsc{ipfv–1/2}\\
    \glt    ‘I lack manioc.’
\z

\section{Verbal reflexive markers}\label{sec:zariquiey:3}

The languages of the world may show different mechanisms for reducing valence, including (i) passives and anticausatives; (ii) antipassives; and (iii) reflexives and reciprocals (see the discussion in \citealt{DixonAikhenvald2000}). In Kakataibo, there is special verbal morphology for reflexives and reciprocals (and a non-productive middle marker), but there are no passive, antipassive or anticausative markers. 

Following Haspelmath (this volume), I define a reflexive voice marker as a verbal affix that indicates the coreference of two participants of a verb (the object participant is coreferential with the subject participant). In this section, I briefly discuss the uses and functions of the reflexive marker in Kakataibo (\sectref{sec:zariquiey:3.1}), and introduce the middle marker of the language, which has reflexive functions (\sectref{sec:zariquiey:3.2}).

\subsection{The reflexive marker}\label{sec:zariquiey:3.1}

The Kakataibo reflexive verbal marker \textit{{}-akat} (and its allomorph) is a derivative suffix (see \citealt[307]{Zariquiey2018}). In terms of its usage, the Kakataibo reflexive marker fits the definition provided by Haspelmath (this volume): it is a marker that appears on the verb stem and indicates that two participants of the event expressed by the verb are coreferential. In many languages, the reflexive voice marker reduces the verbal valency. One important piece of information, however, is that, as indicated in \sectref{sec:zariquiey:1}, Kakataibo makes a rigid distinction between transitive and intransitive verbs (see examples \ref{ex:zariquiey:3} and \ref{ex:zariquiey:4}). Therefore, in morphological terms, the reflexive marker can only be used on transitive and ditransitive stems. Reflexive transitive and ditransitive stems become grammatically intransitive, in terms of how transitivity is encoded and defined in the language (see \citealt[217ss]{Zariquiey2018} for more on transitivity in Kakataibo, and \citealt{Zariquiey2017}, for a discussion of more intricate cases related to bivalent intransitives). 

As reported by \citet[307]{Zariquiey2018}, the Kakataibo reflexive marker exhibits one of the most complex allomorphic alternations in the language, as it may surface as one of the following allomorphs: \textit{{}-akat}, \textit{{}-(ë)kët}, \textit{{}-(u)kut}, \textit{{}-(i)kit}, \textit{-mët} and \textit{{}-mëkët}. The first allomorph appears in the majority of contexts, while \textit{{}-(ë)kët}, \textit{{}-(u)kut}, and \textit{(i)kit} surface when following a stem that ends in a syllable containing \textit{ë}, \textit{u}, and \textit{i}, respectively.\footnote{I have no examples of \textit{{}-(e)ket}, simply because there are no transitive predicates attested in my database that end in the vowel \textit{e}. In addition, there is no \textit{{}-(o)kot} allomorph: when a transitive predicate ends in \textit{o}, as is the case of forms carrying the factitive \textit{{}-o}, it takes the reflexive form \textit{akat}.} Thus, \textit{{}-(ë)kët}, \textit{{}-(u)kut}, and \textit{(i)kit} are the result of a vowel harmony process. If the preceding syllable does not have an overt coda (only fricatives and the nasal \textit{n} can act as syllable codas), we obtain \textit{{}-kët}, \textit{{}-kut}, and \textit{{}-kit.} If the preceding syllable ends in a fricative, we obtain \textit{{}-ëkët}, \textit{{}-ukut}, and \textit{{}-ikit}. Finally, if the stem ends in \textit{n}, the allomorphs \textit{mët} and \textit{mëkët} appear in apparently free variation. In the examples in \REF{ex:zariquiey:7} and \REF{ex:zariquiey:8}, we find two instances of the reflexive marker. In \REF{ex:zariquiey:7}, it surfaces as \textit{{}-kut} and attaches to the transitive verb \textit{churu} ‘untie’, and in \REF{ex:zariquiey:8}, it surfaces as \textit{{}-mët} and attaches to the transitive predicate \textit{bëman} ‘touch in the eyes’. The list of abbreviations is included at the end of this chapter.

\ea%7
    \label{ex:zariquiey:7}
    \gll    matsut-ia=bi kaisa chaxu a=n churu\textbf{-kut}-kwain-kin kaisa xanu xëni-rá chaxu=n makwëx-akë-x-ín\\
            sweep-\textsc{s/a>p:se=emph} \textsc{nar:rep:3} deer \textsc{3sg=a} untie-\textbf{\textsc{refl}}-passing:\textsc{intr-s/a>a:se} \textsc{nar:rep:3} woman    old:\textsc{abs-dim} deer=\textsc{erg} beat.up-\textsc{rem.pst-3-prox}\\
    \glt    ‘It is said that, while (the woman) was sweeping, the deer beat her up, untying himself.’
\z

\ea%8
    \label{ex:zariquiey:8}
    \gll    kaisa    uni    ëëëëëë ki-i        kaisa    bë-man-\textbf{akat}-akë-x-ín\\
            \textsc{nar:rep:3} person:\textsc{abs} ëëëëëë say:\textsc{intr-s/a>s:se} \textsc{nar:rep:3} eyes-touch-\textbf{\textsc{refl}}\textsc{{}-rem.pst-3-prox}\\
    \glt    ‘It is said that the man touched himself in his eyes saying “ëëëëëë”.’
\z

In \REF{ex:zariquiey:9} and \REF{ex:zariquiey:10}, we illustrate the verbs \textit{churu} ‘untie’ (also in \REF{ex:zariquiey:7}) and \textit{bë-man} ‘eyes-touch’ (also in \REF{ex:zariquiey:8}) in their non reflexive usage, proving that these two verbs are lexically transitive.

\ea%9
    \label{ex:zariquiey:9}
    \gll    Juan\textbf{=nën} ka ain kamon churu-\textbf{bëtsin}-a-x-a\\
            Juan\textbf{=\textsc{erg}} \textsc{nar:3} \textsc{3:pos} dog:\textsc{abs} untie-\textbf{coming:}\textbf{\textsc{tran}}\textsc{-ipfv-3-non.prox}\\
    \glt    ‘Juan untied his dog, while coming’
\z

\ea%10
    \label{ex:zariquiey:10}
    \gll    Juan=nën ka ain kamon bë-man-\textbf{bëtsin}-a-x-a\\
            Juan=\textsc{erg} \textsc{nar:3} \textsc{3:pos} dog:\textsc{abs} eyes-touch-\textbf{coming:\textsc{tran}}\textsc{-ipfv-3-non.prox}\\
    \glt    ‘Juan touched his dog in the eyes, while coming’
\z

With a few transitive verbs, the reflexive marker is used to build a construction, where the S argument is linked to the patient of the event, and the agent is not overtly expressed. See the example in \REF{ex:zariquiey:11}. Although there is not a proper passive construction in Kakataibo, constructions like the one in \REF{ex:zariquiey:11} can be interpreted as passive-like, which are primarily attested with the verb \textit{më} ‘beat up’ and, for some speakers, also with \textit{bits} ‘pick up’ and \textit{mëra} ‘find’ (this passive interpretation is more widely found in Shipibo-Konibo reflexive constructions; see \citealt[775--800]{Valenzuela2003}). For many Kakataibo speakers, the passive-like use of the reflexive implies that there is some sort of kinship relationship between the two participants. Thus, there is no coreference relation between the two participants, but there is some sort of inalienable relation, which make them in some way related. See the example in \REF{ex:zariquiey:11}, where the only possible interpretation of the passive-like reading is that the agent was the father or the uncle of the patient. In this kind of construction, the Agent cannot be overtly expressed and is always inferred (see \citealt[253--254]{Keenan1985}, for a brief discussion of reflexives as a cross-linguistically common source for passives). Note that in \REF{ex:zariquiey:11} a reflexive interpretation is also possible.

\ea%11
    \label{ex:zariquiey:11}
    \gll    mi=x kamina më-akat-a-n\\
            \textsc{2sg=s} \textsc{nar:2} beat.up-\textsc{refl-pfv-1/2}\\
    \glt    ‘You were beaten beaten up (by your father/your uncle). / You beaten yourself up.’
\z

\subsection{The middle marker}\label{sec:zariquiey:3.2}

The Kakataibo marker \textit{{}-t}, glossed as ‘middle’ in \citet[308]{Zariquiey2018}, is not very productive, but it appears in combination with a few transitive verb roots, as a valence decreasing and transitivity reducing strategy. Crucially, as is often the case with so-called middle markers, \textit{{}-t} in Kakataibo gets both stative (non-reflexive) and reflexive interpretations. For some verbs, one of these functions is not available, as indicted in the examples in \tabref{tab:zariquiey:3}. As can be seen in \sectref{sec:zariquiey:3.1}, all the allomorphs of the Kakataibo reflexive marker (\textit{{}-akat}, \textit{{}-(ë)kët}, \textit{{}-(u)kut}, \textit{{}-(i)kit}, \textit{-mët} and \textit{{}-mëkët}) exhibit a final \textit{t}, which is likely to be diachronically related to the marked glossed here as middle. Note that the reflexive marker discussed in \sectref{sec:zariquiey:3.1} does not have a stative interpretation. A further difference between the reflexive and the middle marker is that the former is much more widespread.

\begin{table}
    \caption{The Kakataibo middle marker}
    \label{tab:zariquiey:3}
    \begin{tabularx}{\textwidth}{XXXXX}
        \lsptoprule
        Transitive form & Meaning & Intransitive form & Stative meaning & Reflexive meaning\\
        \midrule
        \textit{pëxku} & ‘cure somebody’ & \textit{pexku-t} & ‘become cured & cure oneself’\\
        \textit{unë} & ‘hide’ & \textit{unë-t} & be hidden & hide oneself’\\
        \textit{xui} & grill & \textit{xui-t} & be grilled’ & {}-{}-\\
        \textit{këñu} & ‘finish’ & \textit{këñu-t} & finish up & {}-{}-\\
        \textit{chuka} & ‘wash’ & \textit{chuka-t} & {}-{}- & ‘wash oneself’\\
        \lspbottomrule
    \end{tabularx}
\end{table}

As indicated in \sectref{sec:zariquiey:2}, Kakataibo verbs are lexically either transitive or intransitive. A few roots like *\textit{tsó-} ‘seat, sit down’ or *\textit{ërë-} ‘light, burn’ are not subcategorized for transitivity and are obligatorily combined with one of the suffixes \textit{{}-n} ‘transitive’ or \textit{{}-t} ‘intransitive’, thus producing pairs of verbs that are distinguished by transitivity. This constitutes an instantiation of what \citet[416ss]{Valenzuela2017} calls “alternative derivation”. The marker in the intransitive form in these transitivity-based pairs, \textit{{}-t}, is the semi-productive middle described in this subsection, and the meaning of the intransitive member of the verb pairs often gets reflexive-like interpretations, as illustrated in \tabref{tab:zariquiey:4}. The difference between the examples in \tabref{tab:zariquiey:3} and \tabref{tab:zariquiey:4} has to do with the fact that in \tabref{tab:zariquiey:3} the unmarked form of the predicate is lexically transitive, whereas in \tabref{tab:zariquiey:4}, both the transitive and the intransitive predicates are equally marked.

\begin{table}
    \caption{The Kakataibo middle marker}
    \label{tab:zariquiey:4}
    \begin{tabularx}{\textwidth}{XXXXX}
        \lsptoprule
        Etymological root & Transitive form & Meaning & Intransitive form & Meaning\\
        \midrule
        *tsó & \textit{tsón} & ‘seat’ & \textit{tsót}  & sit down, live’\\
        *ërë & \textit{ërën} & ‘light’ & \textit{ërët} & ‘burn’\\
        *niri & \textit{nirin} & ‘drag’ & \textit{nirit} & crawl’\\
        *nanë & \textit{nanën} & ‘submerge (something)’ & \textit{nanët} & ‘submerge oneself’\\
        *chiki & \textit{chikin} & ‘take out’ & \textit{chikit} & ‘go out’\\
        \lspbottomrule
    \end{tabularx}
\end{table}

\section{Reflexive nominals}\label{sec:zariquiey:4}

According to Haspelmath (this volume), reflexive nominals are the most prominent type of reflexivizer in the world’s languages. Kakataibo exhibits a full paradigm of emphatic pronouns which are etymologically made up of anaphoric (non-reflexive) pronouns combined with the self-intensifier adverbial enclitic \textit{=bi}. According to Haspelmath (this volume), the use of emphatic pronouns as reflexive pronouns is well attested cross-linguistically, and \citet{KönigEtAl2005} reported 94 languages (of a total of 168) with identity of reflexive pronouns and emphatic pronouns of the Kakataibo sort. Kakataibo also has another instance of a reflexive nominal: the noun \textit{nami} ‘body’ may also be used to express agent-patient coreference, or coreference in a minimal clause. Both empathic pronouns and the noun \textit{nami} ‘body’ require the verbal reflexive marker in order to be part of reflexive constructions (but see the example in \ref{ex:zariquiey:16a}). Emphatic pronouns are discussed in \sectref{sec:zariquiey:4.1}, whereas the instances of the noun \textit{nami} ‘body’ in reflexive constructions is presented in \sectref{sec:zariquiey:4.2}.

\subsection{Emphatic pronouns}\label{sec:zariquiey:4.1}

Kakataibo personal pronouns make up a quite complex paradigm. Kakataibo pronouns are associated with two non-singular forms. One can be rendered as archaic (see \citealt{Zariquiey2006}), whereas the other, which is based on the general plural marker of the language \textit{=kama}, can be considered as innovative. The archaic forms are often interpreted as dual (in the case of first and second person) or paucal (in the case of third person) by some speakers, but this interpretation is not systematic. Kakataibo also exhibits a distinction between first person plural inclusive and first person plural exclusive, which is falling into disuse, but can be reconstructed for the proto-language \citep{Zariquiey2006}. Pronouns in Kakataibo exhibit a tripartite case marking system, where the A, S and P functions are expressed differently (the first two functions are marked by two different enclitics and the last one is unmarked). Kakataibo pronominal forms are presented in \tabref{tab:zariquiey:5} (adapted from \citealt[177]{Zariquiey2018}). 

\begin{table}
    \caption{Personal pronouns in Kakataibo}
    \label{tab:zariquiey:5}
    \fittable{
    \begin{tabularx}{1.2\textwidth}{XXXX}
        \lsptoprule
        Person & A & S & P\\
        \midrule
        1p singular & \textit{‘ë=n} & \textit{‘ë=x} & \textit{‘ë}\\
        2p singular & \textit{mi=n} & \textit{mi=x} & \textit{mi}\\
        3p singular & \textit{a=n} & \textit{a=x} & \textit{a}\\
        1p dual (inclusive)
        
        1p plural (inclusive)
        
        1p plural (exclusive) & \textit{nu=n}
        
        \textit{nukama=n}
        
        \textit{‘ëkama=n} & \textit{nu=x}
        
        \textit{nukama=x}
        
        \textit{‘ëkama=x} & \textit{nu}
        
        \textit{nukama}
        
        \textit{‘ëkama}\\
        2p (dual)
        
        2p (plural) & \textit{mitsu=n}
        
        \textit{mikama=n} & \textit{mitsu=x}
        
        \textit{mikama=x} & \textit{mitsu}
        
        \textit{mikama}\\
        3p (dual/paucal)
        
        3p (plural) & \textit{atu=n}
        
        \textit{akama=n} & \textit{atu=x}
        
        \textit{akama=x} & \textit{atu}
        
        \textit{akama}\\
        \lspbottomrule
    \end{tabularx}
    }
\end{table}

Kakataibo’s personal pronouns can be combined with the adverbial enclitic \textit{=bi} ‘emphatic’ to produce emphatic pronominal forms, which can be translated as \textit{self}{}-pronouns into English. Emphatic pronouns with =\textit{bi} seem to some extent lexicalized. \citet[188--191]{Valenzuela2003} reports for the sister language Shipibo-Konibo that emphatic pronouns can be modified (again) by the enclitic \textit{=bi} producing forms like \textit{ëbi=bi} ‘\textsc{1sg:emph=emph}’, but so far I have not found equivalent examples in Kakataibo. 

Differently from non-empathic pronouns, emphatic ones exhibit a neutral case alignment, according to which they remain unmarked, regardless of their grammatical function. However, the first person singular emphatic pronoun can also take a dedicated ‘S’ marker \textit{=x}, thus producing an example of a horizontal alignment type. All this is summarized in \tabref{tab:zariquiey:6} (taken from \citealt[177]{Zariquiey2018}).\todo{Reference to Table 2 changed to Table 6.}

\begin{table}
    \caption{Emphatic personal pronouns in Kakataibo}
    \label{tab:zariquiey:6}
    \fittable{
    \begin{tabularx}{1.1\textwidth}{XXXX}
        \lsptoprule
        Person & A & S & P\\
        \midrule
        1p singular & ‘ëbi & ‘ëbi(=x) & ‘ëbi\\
        2p singular/plural & mibi & mibi & mibi\\
        3p singular/plural & abi & abi & abi\\
        1p plural & nubi & nubi & nubi\\
        \lspbottomrule
    \end{tabularx}
    }
\end{table}

In \REF{ex:zariquiey:12}, I illustrate the paradigm of the first person emphatic pronoun. In \REF{ex:zariquiey:12a}, it appears as the S argument of the intransitive verb \textit{ux} ‘sleep’; in \REF{ex:zariquiey:12b} it appears as the A argument of the transitive verb \textit{mëë} ‘hit’; and in \REF{ex:zariquiey:12c} the first person emphatic pronoun appears as the P argument of the same transitive verb. None of the examples in \REF{ex:zariquiey:12} features a reflexive use of an emphatic pronoun, since the criteria proposed by Haspelmath (this volume) are not satisfied: there is no coreference relation with an antecedent with subject function. 

\ea%12
    \label{ex:zariquiey:12}
    \ea%12a
    \label{ex:zariquiey:12a}
    \gll    ‘ëbi=x kana ‘ux-a-x-a\\
            \textsc{1sg:emph=s}    \textsc{nar:1}    sleep-\textsc{pfv-3-non.prox}\\
    \glt    ‘I myself slept.’
    \ex%12b
    \label{ex:zariquiey:12b}
    \gll    ‘ëbi    kana  a  mëë-a-x-a\\
            \textsc{1sg:emph}    \textsc{nar:1}  \textsc{3sg:p}  hit-\textsc{pfv-3-non.prox}\\
    \glt    ‘I myself hit him.’
    \ex%12c
    \label{ex:zariquiey:12c}
    \gll    ‘ëbi    ka  a=n  mëë-a-x-a\\
            \textsc{1sg:emph}    \textsc{nar:3}  3p=\textsc{a}  hit\textsc{{}-pfv-3-non.prox}\\
    \glt    ‘He hit ME (of all people).’
    \z
\z

Emphatic pronouns can also be used in combination with the verbal reflexive marker introduced in \sectref{sec:zariquiey:3.1}, as illustrated in \REF{ex:zariquiey:13a}, which is basically synonymous with \REF{ex:zariquiey:13b}, although \REF{ex:zariquiey:13a} is often interpreted with a more volitional and controller subject. In any case, it is clear that the reflexive meaning comes from the verbal marker and not from the emphatic pronoun. Note that due the fact that emphatic pronouns exhibit a neutral alignment type, it is not obvious whether the pronoun is the subject or the object of the construction in \REF{ex:zariquiey:13a}, but since the predicate carries the reflexive marker and thus is detransitivized in Kakataibo, we may assume that \textit{abi} in \REF{ex:zariquiey:13a} is a subject pronoun.

\ea%13
    \label{ex:zariquiey:13}
    \ea%13a
    \label{ex:zariquiey:13a}
    \gll    \textbf{abi} ka is-\textbf{akat-}a-x-a\\
            \textbf{\textsc{3sg:emph}}      \textsc{nar:3}  see-\textbf{\textsc{refl}}\textsc{{}-pfv-3-non.prox}\\
    \glt    ‘She saw herself (voluntarily) / he saw himself (voluntarily)’. 
    \ex%13b
    \label{ex:zariquiey:13b}
    \gll    a=x  ka  is\textbf{-akat}-a-x-a\\
            \textsc{3sg=s}  \textsc{nar:3}  see-\textsc{refl-pfv-3-non.prox}\\
    \glt    ‘She saw herself / he saw himself.’
    \z
\z

A slightly different situation is found with reflexive ditransitive constructions, in which the use of an emphatic pronoun does trigger a significant difference in interpretation. Ditransitive predicates use two different reflexive strategies depending on the argument that is coreferential with the subject. If the T (theme) is coreferential with the subject, only the reflexive marker is required. If the R (recipient) is coreferential with the subject, in addition to the reflexive marker, an emphatic pronoun is required. Therefore, non-emphatic pronouns in reflexivized ditransitives always trigger a co-reference relation between the subject and the T argument (a T-reflexive), whereas emphatic ones systematically trigger a co-reference relation between the subject and the R argument (a R-reflexive). This is illustrated in \REF{ex:zariquiey:14}. In \REF{ex:zariquiey:14a}, the non-emphatic pronoun triggers a T-reflexive, whereas in \REF{ex:zariquiey:14b}, the emphatic pronoun triggers a R-reflexive.

\ea%14
    \label{ex:zariquiey:14}
    \ea%14a
    \label{ex:zariquiey:14a}
    \gll    ‘ë=x  kana  ‘inan-mët-i-n\\
            \textsc{1sg=s}    \textsc{nar:1sg}    give-\textsc{refl-ipf-1/2}\\
    \glt    ‘I will give myself (to someone else)’\\
            (*’I will give (something) to myself’)
    \ex%14b
    \label{ex:zariquiey:14b}
    \gll    ‘ëbi=x    kana    ‘inan-mët-i-n\\
            \textsc{1sg:emph=s}    \textsc{nar:1sg}  give\textsc{{}-refl-ipfv-1/2}\\
    \glt    ‘I will give (something) to myself.’\\
            (*‘I will give myself (to someone else)’)
    \z
\z

The first person emphatic pronominal form \textit{‘ëbi} in \REF{ex:zariquiey:14b} lacks an identifiable antecedent and both the pronominal form in \REF{ex:zariquiey:14a} and the one in \REF{ex:zariquiey:14b} are subjects. The reflexive meaning in both examples comes from the verbal reflexivizer. The use of a non-emphatic pronoun in \REF{ex:zariquiey:14a} and an emphatic one in \REF{ex:zariquiey:14b} only triggers a different type coreference relation: in \REF{ex:zariquiey:14a} the Subject is coreferential with T, whereas in \REF{ex:zariquiey:14b} the Subject is coreferential with R.

A more prototypical instance of an emphatic pronoun used as a reflexive pronoun follow in \REF{ex:zariquiey:15a}. In \REF{ex:zariquiey:15a}, we find an example of an emphatic pronoun occurring as a reflexive object and it is clear that there is an antecedent which is crucially the subject of the clause. For this function to be accomplished by an emphatic pronoun, two requirements are in order: (i) the subject cannot be itself an emphatic pronoun (see \ref{ex:zariquiey:15b}); and (ii) the verb must carry itself a reflexive marker (see \ref{ex:zariquiey:15c}). The fact that the \REF{ex:zariquiey:15c} is ungrammatical reveals that emphatic pronouns are not a truly reflexive pronouns, but just emphatic pronouns used in reflexive constructions.

\ea%15
    \label{ex:zariquiey:15}
    \ea%15a
    \label{ex:zariquiey:15a}
    \gll    ‘ë=x  kana \textbf{‘ëbi} {is}\textbf{-akat}-i-n\\
            \textsc{1sg=s}  \textsc{nar:1sg}  \textbf{\textsc{1sg:emph}}    see\textbf{{}-}\textbf{\textsc{refl}}\textsc{{}-ipfv}{}-1/2\\
    \glt    ‘I look at myself.’
    \ex[*]{%15b
    \label{ex:zariquiey:15b}
    \gll    ‘\textbf{ëbi=x} kana \textbf{‘ëbi} is\textbf{-akat}-i-n\\
            \textbf{\textsc{1sg:emph=s}}  \textsc{nar:1sg}  \textbf{\textsc{1sg:emph}}  see\textbf{-\textsc{refl}}\textsc{-ipfv-1/2}\\}
    \glt    (‘I look at myself.’)
    \ex[*]{%15c
    \label{ex:zariquiey:15c}
    \gll    ‘ë=x  kana \textbf{‘ëbi} is-i-n\\
            \textsc{1sg=s}  \textsc{nar:1sg}  \textbf{\textsc{1sg:emph}}    see-\textsc{ipfv-1/2}\\}
    \glt    (‘I look at myself.’)
    \z
\z

A more prototypical use of an emphatic pronoun as a reflexive pronoun can be found with the intransitive verb \textit{bana} ‘speak’. In this case, the requirement of the presence of a verbal reflexivizer does not apply since the verbal reflexive marker can only be combined with transitives and ditransitives \REF{ex:zariquiey:16a}. Note, however, that the restriction regarding the use of an emphatic pronoun in the subject position holds \REF{ex:zariquiey:16b}. It would be interesting to explore if such construction is applicable with other intransitives.

\ea%16
    \label{ex:zariquiey:16}
    \ea[]{%16a
    \label{ex:zariquiey:16a}
    \gll    ‘ë=x  kana  ‘ëbi    bana-i-n\\
            \textsc{1sg=s}  \textsc{nar:1sg}  \textsc{1sg:emph}    speak-\textsc{ipfv-1/2}\\
    \glt    ‘I speak to myself.’
    }
    \ex[*]{%16b
    \label{ex:zariquiey:16b}
    \gll    ‘ëbi=x    kana  ‘ëbi    bana-i-n\\
            \textsc{1sg:emph=s}    \textsc{nar:1sg}  \textsc{1sg:emph}  speak-\textsc{ipfv-1/2}\\
    \glt    (‘I speak to myself.’)
    }
    \z
\z

The examples in \REF{ex:zariquiey:15a} and \REF{ex:zariquiey:16a} feature instances of emphatic pronouns being used in reflexive constructions. It is important to note, however, that in both examples the predicate needs to be intransitive, either because of the presence of a reflexive marker in the verb \REF{ex:zariquiey:15a}, or because the verb is already intransitive \REF{ex:zariquiey:16a}. Only the example in \REF{ex:zariquiey:16a} might be seen as a true instance of an emphatic pronoun being used as a proper reflexive pronoun, since in \REF{ex:zariquiey:15a}, although we do find co-reference with the subject, there is a reflexive marker in the verb. The example in \REF{ex:zariquiey:16a} is highly idiosyncratic and it might be a very special use exclusively associated with the verb \textit{bana} ‘say’.

With the exception of \REF{ex:zariquiey:16a}, all the cases of emphatic pronouns in reflexive constructions, even those ones with a co-referential subject in the same clause require the use of an external reflexivizer: a verbal reflexive. This suggests that emphatic pronouns are not reflexive pronouns. One may hypothesize a future stage in which the verbal reflexive marker is not required anymore and thus truly reflexive pronouns are developed in Kakataibo. This stage, however, has not occurred in the language (with the exception of the highly idiosyncratic example in \REF{ex:zariquiey:16a}.

\subsection{Nami ‘body’}\label{sec:zariquiey:4.2}

With some transitive verbs, Kakataibo exhibits a construction in which the word \textit{nami} ‘body’ is used to express a co-referentiality relation equivalent to the one associated with reflexive constructions. Languages of world often exhibit reflexive pronouns etymologically related to nouns with meanings like ‘body’ or ‘head’ \todo{crossref}(Haspelmath, this volume; \citealt{Schladt1999}). In the case of Kakataibo, the use of \textit{nami} ‘body’ as part of reflexive constructions requires it to be explicitly accompanied by a possessive pronoun (the possessive pronoun has to be coreferential with the subject of the verb) and it also needs to carry the emphatic marker =\textit{bi}. This is illustrated in \REF{ex:zariquiey:17}. Note that a non-reflexive interpretation (‘his own body’) was also given in elicitation.

\ea%17
    \label{ex:zariquiey:17}
    \gll    Juan  ka  ain  nami=bi  is-i-a\\
            Juan  \textsc{nar:3}  \textsc{3sg:gen}    body\textsc{:abs=emph}  see-\textsc{ipfv-non.prox}\\
    \glt    ‘Juan sees himself / Juan sees his own body.’
\z

As in the case of the emphatic pronouns discussed in \sectref{sec:zariquiey:3.1}, it is clear that the coreference interpretation associated with the reflexive meaning in \REF{ex:zariquiey:16} comes from an external element, in this case the possessive pronoun and the emphatic marker =\textit{bi}. Therefore, the use of \textit{nami} ‘body’ as a reflexive nominal has not been fully grammaticalized.

\section{Conclusions}\label{sec:zariquiey:5}

The present chapter has discussed the main strategies for expressing reflexive meanings in Kakataibo. The language exhibits a productive verbal reflexive voice marker, which is used on transitive verbs in order to indicate coreferentiality between their two participants (which also detransitivizes the verb). The middle marker of the language, \textit{{}-t}, also gets reflexive overtones with some predicates, but is not fully productive. Kakataibo also has constructions that may be seen as reflexive nominals. These include the use of emphatic pronouns and the use of the noun \textit{nami} ‘body’ with transitive predicates to express reflexive meanings. The reflexive interpretation of emphatic pronouns requires the presence of the reflexive marker, whereas the reflexive use of the noun \textit{nami} ‘body’ requires the possessive marker and the emphatic clitic =\textit{bi}. This fact suggests that they are not proper reflexive nominals  but elements that contribute to the interpretation of reflexive constructions, in which the co-referentiality component of the meaning comes from a different element. Their need for some extra markers of co-reference (a reflexive marker or a possessive pronoun) suggests that in Kakataibo there are not fully grammatical nominal reflexives. How should we call the reflexive uses of the emphatic pronouns and the noun \textit{nami} ‘body’ in Kakataibo? What does it imply for such nominal elements to participate in reflexive constructions under the condition that the verbal reflexive is also there? Answering such questions may have consequences for our understanding of reflexive nominals as a descriptive category and the nature of nominal reflexives in languages which already have a reflexive verbal marker. 

\section*{Abbreviations}

\begin{tabularx}{.45\textwidth}[t]{lQ}
\textsc{1} & first person\\
\textsc{2} & second person\\
\textsc{3} & third person\\
\textsc{a} & most agentive participant of a transitive predicate\\
\textsc{abs} & absolutive\\
\textsc{dim} & diminutive\\
\textsc{emph} & emphatic\\
\textsc{erg} & ergative\\
\textsc{gen} & genitive\\
\textsc{intr} & intransitive\\
\textsc{ipfv} & imperfective\\
\textsc{nar} & narrative\\
\textsc{non.prox} & non-proximal to the addressee\\
\textsc{p} & most patientive participant of a transitive predicate\\
\end{tabularx}
\begin{tabularx}{.45\textwidth}[t]{lQ}
\textsc{pfv} & perfective\\
\textsc{prox} & proximal\\
\textsc{r} & recipient-like argument of ditransitive construction\\
\textsc{refl} & reflexive\\
\textsc{rem.pst} & remote past\\
\textsc{rep} & reportative\\
\textsc{s} & single participant of an intransitive predicate\\
\textsc{se} & simultaneous event\\
\textsc{sg} & singular\\
\textsc{t} & theme argument of ditransitive constructions\\
\end{tabularx}

\sloppy\printbibliography[heading=subbibliography,notkeyword=this]\end{document} 