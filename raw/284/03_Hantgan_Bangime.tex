%\documentclass[output=paper,modfonts,nonflat,citecolor=brown,showindex]{langsci/langscibook}
%\setlength{\parindent}{1em}
%%\tolerance=1
%emergencystretch=\maxdimen
%\hyphenpenalty=10000
%\hbadness=10000

\documentclass[output=paper]{langscibook}

\author{Abbie Hantgan\affiliation{CNRS-LLACAN}}
\title{Reflexive constructions in Bangime}  
\abstract{Bangime, a language isolate spoken in Central Eastern Mali, has two ways to expresses coreference between clause participants. One strategy is through coordinated markers from one of the language's pronominal series. These markers can be considered to be the language's reflexive pronouns, though it is of typological interest to note that, in object position, an anaphoric pronoun of this series can be coreferential with the main clause's subject. Furthermore, Bangime displays the unusual property of aligning second persons singular and plural to the exclusion of all other persons. This chapter also discusses an additional coreference strategy, namely that of a possessed form of the noun `head', an areally robust feature of West Africa.}

%\usepackage{tabularx}
%\usepackage{langsci-gb4e}
%\usepackage{langsci-optional}
%\usepackage{float}
%\usepackage{multicol}
%\bibliography{Bangime}

%\usepackage{tikz}
%\newcommand*{\tkzmk}[1]{\tikz[remember picture,overlay] \node (#1) {};}
%\newcommand*{\tkzuline}[2]{\tikz[overlay,remember picture]{ \draw (#1.south) -- (#2.south);}}
%\newcommand*{\UL}{\tkzmk{1}}
%\newcommand*{\LU}{\tkzmk{2}}
%\newcommand*{\gluline}{\tkzuline{1}{2}}

\IfFileExists{../localcommands.tex}{%hack to check whether this is being compiled as part of a collection or standalone
  \input{../localpackages}
  %Copy this to localcommands.tex

\usepackage[english]{babel}
\usepackage{amsmath}
\usepackage{amssymb,amsfonts,textcomp}
\usepackage{array}
\usepackage{hhline}
\usepackage{hyperref}

\newenvironment{styleStandard}{}{}
\newenvironment{stylelsAbstract}{}{}
\newenvironment{stylelsSectioni}{}{}
\newenvironment{stylelsSectionii}{}{}
\newenvironment{stylelsBulletList}{}{}
\newenvironment{styleBibliographyi}{}{}
\newenvironment{listWWNumxxvleveli}{}{}
\newenvironment{listWWNumxxvlevelii}{}{}
\newenvironment{listWWNumxxvleveliii}{}{}
\newenvironment{listWWNumxxvleveliv}{}{}
\newenvironment{listWWNumixleveli}{}{}
\newenvironment{listWWNumixlevelii}{}{}
\newenvironment{listWWNumixleveliii}{}{}
\newenvironment{listWWNumixleveliv}{}{}

\newcommand\textstyleListLabelxvi[1]{#1}
\newcommand\labellistWWNumxxvleveli{\thelistWWNumxxvleveli.}
\newcommand\labellistWWNumxxvlevelii{\thelistWWNumxxvlevelii.}
\newcommand\labellistWWNumxxvleveliii{\thelistWWNumxxvleveliii.}
\newcommand\labellistWWNumxxvleveliv{\thelistWWNumxxvleveliv.}
\newcommand\labellistWWNumixleveli{[F0B7?]}
\newcommand\labellistWWNumixlevelii{\textstyleListLabelxvi{o}}
\newcommand\labellistWWNumixleveliii{[F0A7?]}
\newcommand\labellistWWNumixleveliv{[F0B7?]}

\newcounter{listWWNumxxvleveli}
\newcounter{listWWNumxxvlevelii}[listWWNumxxvleveli]
\newcounter{listWWNumxxvleveliii}[listWWNumxxvlevelii]
\newcounter{listWWNumxxvleveliv}[listWWNumxxvleveliii]
\newcounter{itemize} 
  \input{../localhyphenation}
  \bibliography{localbibliography}
  %\togglepaper[5]
}{}


\begin{document}
\maketitle

\section{Introduction}
\label{sec:intro}
In Bangime, coreference between clause participants can be expressed in two ways. The first, as with many other West African languages \citep{Heine2011}, including those of the Atlantic branch as well as surrounding Dogon languages, but excluding Mande, involves the noun `head'. However, Bangime is different from neighboring languages in that, in Bangime, the person and number of the clause's subject are indexed (cf. \citealt{Haspelmath2013}) on verbs and tense-aspect auxiliaries, and, in the case of the reflexive construction, on nouns. This is shown in (\ref{ex:oldman}), an excerpt from a narrative, where these indexes are glossed as lettered sets that are explained in (\S \ref{sec:pf}).
\ea \label{ex:oldman}
\gll 
mɛ̀ɛ̀ à ɡò m̀=bòɡò-ɛ̄ɛ̄ ŋ̄=kārā \textbf{ŋ̄}=\textbf{dēɡē} ŋ̄=kɛ̀ \\
but \textsc{def} man 3\textsc{sg}.B=big-\textsc{dim} 3\textsc{sg}.B=find.3\textsc{sg}.\textsc{pfv} 3\textsc{sg}.B=head 3\textsc{sg}.B=\textsc{prf} \\
\glt `\dots but the old man [lit. `little old man'] had found \textbf{himself} again.' \\
 \citep[10]{HantganHeath2018} \\
\z
In the surrounding Dogon languages, reflexive constructions are formed with a possessed form of the noun `head' so that `my head' can be interpreted to mean `myself'. In Bangime, as will be discussed in detail in (\S \ref{subsec:poshead}), the portion of (\ref{ex:oldman}) highlighted in bold differs from the language's typical possessive construction.

The second method of expressing coreference between clause participants, as illustrated in (\ref{ex:hit1})--(\ref{ex:hit4}) drawn from \cite[438]{HantganHeath2018}, involves a pronoun (series) which is described as reflexive pronoun here, but also occurs with other middle functions (\S \ref{subsec:refpro}). It is also of interest to note that the language opposes second persons to first and third persons plural, as well as third person singular, shown below in (\ref{ex:hit6}).
\begin{multicols}{2}
	\ea \label{ex:hit1}
	\gll 
	à dɛ̀ɡù \textbf{á} \\
	2\textsc{sg}.A hit.2\textsc{sg}.\textsc{pfv} 2\textsc{sg}.D \\
	\glt `You hit \textbf{yourself}.'
	\z
	\ea \label{ex:hit2}
	\gll àà dɛ̀ɡū \textbf{āà} \\
	2\textsc{pl}.A hit.2\textsc{pl}.\textsc{pfv} 2\textsc{pl}.D \\
	\glt `You (plural) hit \textbf{yourselves}.'
	\z
\end{multicols}
\begin{multicols}{2}
	\ea \label{ex:hit3}
	\gll 
	nɛ̀ ǹ=dɛ̀ɡù \textbf{mīì} \\
	1\textsc{pl}.A 1\textsc{pl}.B=hit.1\textsc{pl}.\textsc{pfv} 1\textsc{pl}.D \\
	\glt `We hit \textbf{ourselves}.'
	\z
	\ea \label{ex:hit4}
	\gll 
	nì ǹ=dɛ̀ɡù \textbf{mīì} \\
	3\textsc{pl}.A 3\textsc{pl}.B=hit.3\textsc{pl}.\textsc{pfv} 3\textsc{pl}.D \\
	\glt `They hit \textbf{themselves}.'
	\z
\end{multicols}
Furthermore, as shown in (\ref{ex:hit5})--(\ref{ex:hit6}) drawn from \cite[401]{HantganHeath2018}, pronouns for coreference and disjoint reference for third singular antecedents are identical and therefore potentially ambiguous in meaning.
\begin{multicols}{2}
	\ea \label{ex:hit5}
	\gll 
	$\emptyset$ dɛ̀ɡū \textbf{mīì} \\
	3\textsc{sg}.A hit.3\textsc{sg}.\textsc{pfv} 3\textsc{sg}.C \\
	\glt `He/She\textsubscript{x} hit \textbf{him/her\textsubscript{y}}.'
	\z
	\ea \label{ex:hit6}
	\gll 
	$\emptyset$ dɛ̀ɡū \textbf{mīì} \\
	3\textsc{sg}.A hit.3\textsc{sg}.\textsc{pfv} 3\textsc{sg}.D \\
	\glt `He/She\textsubscript{x} hit \textbf{himself/herself\textsubscript{x}}.' 
	\z
\end{multicols}
The fact that, in Bangime, an anaphoric pronoun in object position can be coreferential with the subject of its clause may be interesting from a typological perspective as discussed by \cite[4]{Haspelmath2019}; this is explored further in (\S \ref{sec:ref}). Furthermore, that the pronoun \textit{mii} serves to mark coreference and disjoint reference for the third persons singular and plural, as well as the first person plural to the exclusion of the second persons singular and plural and the first person singular is somewhat surprising; the corresponding first person singular reflexive pronoun is given in (\ref{ex:hit1s}).
\ea \label{ex:hit1s}
\gll 
ń dɛ́ɡù \textbf{mí} \\
1\textsc{sg}.A hit.1\textsc{sg}.\textsc{pfv} 1\textsc{sg}.D \\
\glt `I hit \textbf{myself}.' \citep[438]{HantganHeath2018}
\z
This chapter seeks to explore means of coreference in Bangime through an exposition of the pronominal system in general. The next section, (\S \ref{sec:back}), provides a brief background on the speakers of Bangime and the language's status as an isolate. An overview of the language's pronominal system is given in (\S \ref{sec:pf}). In (\S \ref{sec:ref}), reflexive constructions in Bangime are presented, followed by a discussion of intensifier uses involving reflexive constructions in (\S \ref{sec:foc}). A conclusion is given in (\S \ref{sec:conc}).

Data are drawn from both a published grammatical description and a doctoral thesis as well as unpublished textual sources and newly elicited examples. Transcriptions are phonetic, following the IPA, and glossing follows Leipzig conventions with additions noted in the Appendix.

\section{Background}
\label{sec:back}
Bangime is a language isolate spoken by around 1,500 people in seven small villages in a remote area of central-eastern Mali. The Bangime-speaking community, the Bangande, together with the Dogon ethno-linguistic group, are the sole inhabitants of the Bandiagara Escarpment, an arduous cliff range located east of the Niger River and south of the Sahara Desert. The Dogon languages were not well described until recently; it is only now apparent that there exist at least 21 different Dogon languages. Nevertheless, and despite the fact that the Bangande say that they and their language are Dogon, the linguistic divergences between Bangime and the Dogon languages separate them completely. Moreover, Bangime is not related at all to the other neighboring language, Jenaama, of the Bozo-Mande grouping.

Grammatical structures found in Bangime pertinent to this study include its almost complete lack of affixal morphology, a tripartite tonal system, and subject-initial clausal word order in non-focus constructions. Possessive pronouns and the definite article precede a noun in a noun phrase (e.g. \textit{ā kùwò} `the house', \textit{màā kùwò} `his/her house'), but adjectives follow the noun. With certain kinship relations, possession is expressed in a manner which differs from other possessed nouns, as discussed in (\S \ref{subsec:possp}) below. Verbs are divided into classes based on their morpho-phonological properties and thus follow different patterns of inflection therein. A verb phrase either consists of simply a verb stem (with inflectional marking on the verb itself), or it also contains a auxiliary specifying the aspect of the clause which either precedes or follows the verb stem depending on the aspect in question. It is relevant to note that a pronoun forms precede both the verb stem and the auxiliary when present.

Major word classes in Bangime consist of nouns, verbs, numerals, adjectives, adverbs, and ideophones. Particles in the language include the determiner, postpositions, and a question marker. Word classes can be distinguished morphologically as well as syntactically. The small number of bound morphemes in the languages are all suffixes or clitics including an agentive, diminutive, and plural markers on nouns and a causative and aspectual markers on verbs. Tone is both lexical and grammatical and the interaction between the two is intricate. The elaborate argument-indexing system is presented in the following, section (\ref{sec:pf}). Constituent word order depends on the tense or aspect of the clause and is elaborated upon in (\S \ref{subsec:vp}). 

\section{Person forms}
\label{sec:pf}
Bangime has a complex system of person forms, including both bound and free forms, which is essential to understand in order to evaluate the evidence put forth for the presence or absence of a special form that signals coreference, a reflexive pronoun, in the language. The following, subsection (\ref{subsec:pp}) presents Bangime personal pronouns in terms of four sets, and (\S \ref{subsec:vp}) provides the slots into which these pronouns fit in a clause.

\subsection{Person form sets}
\label{subsec:pp}
Person and number marking in Bangime consists of both bound person indexes and free personal pronouns. The Bangime person forms consist of four sets depending on the phonetic (surface) realization and position in a clause, as listed in Table \ref{tab:bpp} and exemplified in (\S \ref{subsec:vp}) below. As already seen in (2)-(5) above, the members of set D are used as reflexive pronouns.
\begin{table}[H]
	\caption{Bangime person forms}
	\label{tab:bpp}
	\begin{tabular}{l rrrr}
		\lsptoprule
		Person/Number & A & B & C & D \\ 
		\midrule
		1\textsc{sg} & ǹ {\textasciitilde} $\emptyset$ & n & mí & mí \\
		3\textsc{sg} & $\emptyset$  & n & mì(ì) {\textasciitilde} mīì & mīì \\	
		\midrule
		1\textsc{pl} & nɛ̀ & n & nɛ̄ɛ̀ & mīì \\
		3\textsc{pl} & nì & n & nīì &   mīì \\
		\midrule 		
		2\textsc{sg} & à & á & á & á \\
		2\textsc{pl} & àà & á & āā(-rú) & āà \\
		\lspbottomrule
	\end{tabular}
\end{table}
As represented in Set A, singular first and third persons are either zero or, as in Set B, reduced to a nasal that assimilates in place to the initial consonant of the following constituent in the clause (there are few vowel-initial words in Bangime) and receives its tonal specification from the final tone of the preceding word. As was illustrated in (\S \ref{sec:intro}), for sets C and D, forms of the first and third persons plural are identical to that of the third person singular. Second person singular is the same for all sets (second person singular indexes are distinguished by tones alone) but second person plural is sometimes followed by one of the few suffixes in the language: a suppletive plural suffix (\textit{-rú}) that is also used with kin terms in the language. The following subsection demonstrates where each person form set occurs in the three main clausal paradigms presented in this chapter.

\subsection{Person form slots}
\label{subsec:vp}
In Bangime, person and number marking occurs multiple times throughout a sentence by the person forms presented in Table \ref{tab:bpp}, and by tonal melodies on the verb stem. The simple perfective paradigm is shown first as it represents the language's most basic clausal construction. The linear order of the verb and the object is reversed between a perfective (Tables \ref{tab:tp}--\ref{tab:tpp}), and an imperfective (Table \ref{tab:ti}) verb phrase.
\begin{table}[H]
	\caption{Simple transitive perfective clausal paradigm}
	\label{tab:tp}
	\begin{tabular}{|c|c|c|c|c|}
		\lsptoprule
		(S) & S & S & V & O \\ 
		\midrule 		
		(NP) & Pronoun Set A & Pronoun Set B & \textsc{Stem} & NP $\land$ Pronoun Set C $\land$ D  \\
		\lspbottomrule
	\end{tabular}
\end{table}
\begin{table}[H]
	\caption{Transitive perfective clausal paradigm}
	\label{tab:tpp}
	\begin{tabular}{|c|c|c|c|c|c|}
		\lsptoprule
		(S) & S & AUX & S & V & O \\ 
		\midrule 		
		(NP) & Pronoun Set A & \textsc{pfv} & Pronoun Set B & \textsc{Stem} & NP $\land$ Pronoun Set C $\land$ D  \\
		\lspbottomrule
	\end{tabular}
\end{table}
\begin{table}[H]
	\caption{Transitive imperfective clausal paradigm}
	\label{tab:ti}
	\begin{tabular}{|c|c|c|c|c|c|}
		\lsptoprule
		(S) & S & AUX & O & S & V \\ 
		\midrule 		
		(NP) & Pronoun Set A & \textsc{ipfv} & NP $\land$ Pronoun Set C $\land$ D & Pronoun Set B & \textsc{Stem} \\
		\lspbottomrule
	\end{tabular}
\end{table}
As mentioned above and can be seen from Tables \ref{tab:tp}--\ref{tab:ti}, a personal pronoun or person index consistently precedes the verb stem, and an auxiliary, if present. The next subsection gives concrete examples of the sets and slots presented here.

\subsection{Personal pronoun examples}
\label{subsec:pe}
In general, intransitive verb stems are not preceded by person indexing; the verb `go' is an exception. Examples in the perfective aspect featuring the verb \textit{wōrè} `go' in the form of elicited examples in (\ref{ex:go1s})--(\ref{ex:buy3s}) are drawn from \citet[~273]{HantganHeath2018}.
\begin{multicols}{2}
	\ea \label{ex:go1s}
	\gll 
	ŋ̀ kóó ŋ́=wóré \\
	1\textsc{sg}.A 1\textsc{sg}.\textsc{pfv} 1\textsc{sg}.B=go.1\textsc{sg}.\textsc{pfv} \\
	\glt `I had gone.'
	\z
	\ea \label{ex:go2s}
	\gll à kwá á wóré \\
	2\textsc{sg}.A 2\textsc{sg}.\textsc{pfv} 2\textsc{sg}.B go.2\textsc{sg}.\textsc{pfv} \\
	\glt `You had gone.'
	\z
\end{multicols}
\begin{multicols}{2}
	\ea \label{ex:buy3s}
	\gll $\emptyset$ kóó ŋ́=wōrè \\
	3\textsc{sg}.A 3\textsc{sg}.\textsc{pfv} 3\textsc{sg}.B=go.3S.\textsc{pfv}  \\
	\glt `He had gone.'
	\z
	\ea \label{ex:buy2p}
	\gll àà kwá á wōrè \\
	2\textsc{pl}.A 2\textsc{pl}.\textsc{pfv} 2\textsc{pl}.B go.2P.\textsc{pfv} \\
	\glt `You (PL) had gone.'
	\z	 
\end{multicols}
\begin{multicols}{2}
	\ea \label{ex:buy1p}
	\gll nɛ̀ kóó ŋ́=wōrè \\
	1\textsc{pl}.A 1\textsc{pl}.\textsc{ipfv} \textsc{pl}.B=go.1\textsc{pl}.\textsc{pfv}  \\
	\glt `We had gone.'
	\z
	\ea \label{ex:buy3p}
	\gll nì kóó ŋ́=wōrè \\
	3\textsc{pl}.A 3\textsc{pl}.\textsc{pfv} 3\textsc{pl}.B=go.3\textsc{pl}.\textsc{pfv} \\
	\glt `They had gone.'
	\z	 
\end{multicols}
These examples illustrate person form sets A--B. Set A occurs in subject position. Based on its tonal behavior (the nasal of Set B has no phonemic tone), I consider the nasal person index in set B to be a proclitic. The person forms of set C and D are free personal pronouns. The perfective clausal paradigm further illustrates that which was mentioned above in (\S \ref{sec:intro}): in Bangime, second persons singular and plural are marked almost identically, and in opposition to the other persons in the language.

Examples using the verb `bathe' in the simple perfective which are drawn from \cite[~325]{HantganHeath2018} are given in (\ref{ex:bathe})--(\ref{ex:bathe2}) to illustrate person form sets C--D.
\begin{multicols}{2}
	\ea \label{ex:bathe}
	\gll $\emptyset$ tùù \textbf{mí} \\
	3\textsc{sg}.A bathe.3\textsc{sg}.\textsc{pfv} 1\textsc{sg}.C \\
	\glt `He/She bathed me.' 
	\z
	\ea \label{ex:bathe2}
	\gll $\emptyset$ tùū \textbf{mīì} \\
	3\textsc{sg}.A bathe.3\textsc{sg}.\textsc{pfv} 3\textsc{sg}.D \\
	\glt `He/She bathed (him/herself).' 
	\z
\end{multicols}
Although some person forms from Set C are homophonous with those from Set D, the former cannot be used together with those from Set A to express coreference between subject and an object. That is, for those persons which differ in form, such as first and third persons plural, Set C cannot be interchanged with Set D; the former strictly marks disjoint-reference between participants while the latter marks coreference.

Further adding to the ambiguity, syntactically, both person form sets C--D occupy the same position, save for when an object pronoun is focalized as illustrated in (\ref{ex:like})--(\ref{ex:help}).
	\begin{multicols}{2}
	\ea \label{ex:like}
	\gll $\emptyset$ mūw̃ɔ̄ \textbf{mí} \\
	3\textsc{sg}.A like.\textsc{neg}.3\textsc{sg}.\textsc{ipfv} 1\textsc{sg}.C \\
	\glt `\dots he does not like \textbf{\textit{me}}.' Narrative [TB2008-07-12, Line 1]
	\z			
	\ea \label{ex:help}
	\gll \textbf{mí} ń=dèɡè \\
	1\textsc{sg}.C 3\textsc{sg}.B=hit.3\textsc{sg}.\textsc{pfv} \\
	\glt `It hit \textbf{\textit{me}}'. \\ 
	Survey [AD2012-08-06, Line 14]
	\z
\end{multicols}
In focused-object position, the expected word order for Bangime constituents is reversed yet again: a focused object appears after the verb in the imperfective aspect and before it in the perfective.

As shown in (\ref{ex:kill1p})--(\ref{ex:killchief}), the language does not require that an object is overtly expressed. Otherwise, as can be seen in (\ref{ex:killchief}) and Table \ref{tab:tpp}, non-focused object noun phrases, like free pronouns, occur post-verbally in the perfective aspect.
\ea \label{ex:kill1p}
\gll nì kóó ɲ́=ɥùrù \\
3\textsc{pl}.A 3\textsc{pl}.\textsc{pfv} 3\textsc{pl}.B=kill.3\textsc{pl}.\textsc{pfv} \\
\glt `They killed (him).' Narrative \citep[~394]{HantganDiss}
\z
\ea \label{ex:killchief}
\gll nì kóó ɲ́=ɥùrù à dèɡè ǹ=ɕɛ̰́ɛ̰́\\
3\textsc{pl}.A 3\textsc{pl}.\textsc{pfv} 3\textsc{pl}.B=kill.3\textsc{pl}.\textsc{pfv} \textsc{def} head.\textsc{def} 3\textsc{sg}.B=owner\\
\glt `They killed the chief [lit. `head owner'].' Narrative \citep[~477]{HantganDiss}
\z 
Persons other than third person singular may also be omitted in object position but with lower frequency. On the other hand, pronouns of Set D are obligatory in reflexive constructions.

%Thus, pronouns of Set C are primarily used as anaphoric objects and syntacticly may appear in subject position when focused. Uses of clause-initial \textit{mi} are examples of focused objects. In a similar fashion, the morpheme \textit{mi} may be combined with pronouns of Set A to express focus in the perfective aspect as shown in (\ref{ex:focpfv1})--(\ref{ex:focpfv2}).
%\ea \label{ex:focpfv1}
%Elicit \citep[339]{HantganHeath2018} \\
%\gll á mì ŋ̀=wóré \\
%2\textsc{sg}.A 3\textsc{sg}.C(?) 3\textsc{sg}.B=go.2\textsc{sg}.\textsc{pfv} \\
%\glt `It was \textit{you} (who) went.'
%\z
%The labeling of \textit{mi} as the third person singular may be incorrect as it does not align with the sentence's translation and thus its being homophonous with this form may be a coincidence. The verb `say' is always preceded by the morpheme \textit{mi} when introducing a discourse report.
%\ea \label{ex:focpfv3}
%Elicit \citep[304]{HantganHeath2018} \\
%\gll nɛ̀ mì kéndé \\
%1\textsc{pl}.A 3\textsc{sg}.C(?) say.1\textsc{pl}.\textsc{pfv} \\
%\glt `We said\dots'
%\z		
%\ea \label{ex:focpfv2}	
%Narrative (SD-2010-08-4) \\
%\gll mì kéndé báŋá=ndɛ̀ pà̰à̰ wàj \\
%3\textsc{sg}.C(?) say.1\textsc{sg}.\textsc{pfv} Bangime=\textsc{pl} all \textsc{dat} \\
%\glt `I said, ``All the Bangande people\dots'''
%\z
%This use is included here for the sake of descriptive completeness and the determining of its precise role is left for future research.

The following subsection outlines two possessive strategies in the language as these will be crucial to the comprehension of the reflexive constructions presented in (\S \ref{sec:ref}).

\subsection{Possessive pronouns}
\label{subsec:possp}
There are two ways of marking possession in Bangime: most possessed nouns are preceded by a pronoun from Set A plus the possessive morpheme \textit{maa}(a kind of genitive postpostion). However, the second person singular possessive pronoun is simply [àà]. Examples provided in (\ref{ex:poss1})--(\ref{ex:poss4}) are drawn from \cite[57]{HantganHeath2018}.
\begin{multicols}{2}
\ea \label{ex:poss1}
\gll $\emptyset$ máá kùwò \\
1\textsc{sg}.A 1\textsc{sg}.\textsc{poss} house.\textsc{poss}\\
\glt `my house'
\z
\ea \label{ex:poss2}
\gll àā kùwò \\
2\textsc{sg}.A.\textsc{poss} house.\textsc{poss}\\
\glt `your (sg) house'
\z
\ea \label{ex:poss3}
\gll $\emptyset$ màā kùwò \\
3\textsc{sg}.A 3\textsc{sg}.\textsc{poss} house.\textsc{poss}\\
\glt `his/her house'
\z
\ea \label{ex:poss4}
\gll séédù màā kùwò \\
Seydou 3\textsc{sg}.\textsc{poss} house.\textsc{poss}\\
\glt `Seydou's house'
\z
\end{multicols}
Another means of expressing possession in Bangime is with the use of person forms alone. With certain kin terms, for example `father' as shown in the following examples drawn from \cite[58--59]{HantganHeath2018}, person forms of Set A may be used with the possessive morpheme (\ref{ex:possfather3}), or alone (\ref{ex:possfather1})--(\ref{ex:possfather2}), (\ref{ex:possfather4}).    
\begin{multicols}{2}
	\ea \label{ex:possfather1}
	\gll $\emptyset$ bɔ́w \\
	1\textsc{sg}.A father.1\textsc{sg}.\textsc{poss} \\
	\glt `my father'
	\z
	\ea \label{ex:possfather2}
	\gll à bɔ̀w \\
	2\textsc{sg}.A father.2\textsc{sg}.\textsc{poss}\\
	\glt `your father'
	\z
	\ea \label{ex:possfather3}
	\gll séédù màà bɔ́w \\
	Seydou 3.\textsc{poss} father.\textsc{poss} \\
	\glt `Seydou's father'
	\z
	\ea \label{ex:possfather4}
	\gll séédù bɔ̀w \\
	Seydou father.3\textsc{sg}.\textsc{poss} \\
	\glt `Seydou's father'
	\z
\end{multicols}
Note that each type of possessive marking influences the tone of the possessed noun differently; the possessive morpheme bears its own tone depending on the person and number of the possessee which triggers a kind of default tonal marking on the possessed noun. When the possessive morpheme is not present, the possessed noun represents the tone of the person and number of the possessee.

In addition to the constructions with the possessive morpheme \textit{maa} (in \ref{ex:poss1}--\ref{ex:poss4}) and with kin terms (in (\ref{ex:possfather1}--\ref{ex:possfather4}), there is a third possessive construction: similar to the kinship-type of possession, a possessive, often compound-like, construction in Bangime may be formed using the person indexes from Set B; compare (\ref{ex:beehead}) with (\ref{ex:beecontainer}).
	\ea \label{ex:beehead}
	\gll míró ń=déɡé \\
	bee.\textsc{indf} 3\textsc{sg}.B=head.3\textsc{sg}.\textsc{poss} \\
	\glt `bee's head'
	\z
	\ea \label{ex:beecontainer}
	\gll míró ḿ=páɡà \\
	bee.\textsc{indf} 3\textsc{sg}.B=container.3\textsc{sg}.\textsc{poss} \\
	\glt `apiary [lit. `bee's container']'
	\z
Somewhat curiously, body parts belonging to animals, particularly insects are usually expressed using this construction while humans use the possessive morpheme. More about this and how it relates to reflexive constructions in Bangime will be said below in (\S \ref{subsec:poshead}).

Now that an overview of person forms has been presented, the following, (\S \ref{sec:ref}) depicts the strategies found in the language to express coreference between clause participants.

\section{Reflexive constructions}
\label{sec:ref}
As stated in Section \ref{sec:intro}, there are two ways of expressing coreference between clause participants in Bangime.\footnote{I follow \cite[2--3]{Haspelmath2019} in using the semantically-based term `clause participant' rather than the syntactic `clause argument' as the subject pronoun is not necessarily overtly expressed in Bangime.} Henceforth, these two constructions will be discussed as the ``reflexive noun'' and ``reflexive pronoun'', presented in (\S \ref{subsec:poshead}) and (\ref{subsec:refpro}) respectively.

\subsection{Reflexive noun}
\label{subsec:poshead}
The reflexive noun construction consists of the genitive construction with the noun \textit{deɡe} `head', of the possessed type presented above in (\S \ref{subsec:possp}), in non-focused object position of a transitive clause. That is, in the reflexive noun construction, the possessive is formed from Set B of the person forms provided in Table \ref{tab:bpp} above. The reflexive noun paradigm is illustrated with the following examples which are drawn from \cite[~442-443]{HantganHeath2018}.
\ea \label{ex:head1}
\gll ɲ̀ jáɡù ǹ=déɡé \\
1\textsc{sg}.A cut.1\textsc{sg}.\textsc{pfv} 1\textsc{sg}.B=head.1\textsc{sg}.\textsc{poss} \\
\glt `I cut myself.'
\z
\ea \label{ex:head5}
\gll $\emptyset$ jàɡū n̄=dēɡè \\
3\textsc{sg}.A cut.3\textsc{sg}.\textsc{pfv} 3\textsc{sg}.B=head.3\textsc{sg}.\textsc{poss} \\
\glt `He cut himself.'
\z
\ea \label{ex:head3}
\gll à jáɡù à déɡé \\
2\textsc{sg}.A cut.2\textsc{sg}.\textsc{pfv} 2\textsc{sg}.B head.2\textsc{sg}.\textsc{poss} \\
\glt `You (SG) cut yourself.'
\z	
\ea \label{ex:head4}
\gll àà jàɡū à dēɡè \\
2\textsc{pl}.A cut.2\textsc{pl}.\textsc{pfv} 2\textsc{pl}.B head.2\textsc{pl}.\textsc{poss} \\
\glt `You (PL) cut yourselves.'
\z	
\ea \label{ex:head2}
\gll nɛ̄ ɲ̀ jàɡū n̄=dēɡè \\
1\textsc{pl}.A 1\textsc{pl}.B cut.1\textsc{pl}.\textsc{pfv} 1\textsc{pl}.B=head.1\textsc{pl}.\textsc{poss} \\
\glt `We cut ourselves.'
\z	
\ea \label{ex:head6}
\gll nì ɲ̀ jàɡū n̄=dēɡè \\
3\textsc{pl}.A 3\textsc{pl}.B cut.3\textsc{pl}.\textsc{pfv} 3\textsc{pl}.B=head.3\textsc{pl}.\textsc{poss} \\
\glt `They cut themselves.'
\z	
However, besides the possessive constructions as listed above in (\S \ref{subsec:possp}), possessed body parts are usually expressed with the possessive morpheme; compare (\ref{ex:throat}) with (\ref{ex:head5}) above. 
\ea \label{ex:throat}

\gll 
ɲ̀ jàɡù màā kwāà \\
3\textsc{sg}.A cut.3\textsc{sg}.\textsc{pfv} 3\textsc{sg}.\textsc{poss} throat.\textsc{poss} \\
\glt `He slaughtered it (the sheep) [lit. cut its throat].'\\
Narrative [NB2010-07-16, Line 41]
\z
Therefore, as stated by \cite[442]{HantganHeath2018}, \textit{maa deɡe} translates to `possessor's head' while \textit{n deɡe} is the equivalent of `X's self', and therefore the reflexive noun as depicted here.

Recall from Tables \ref{tab:tp}--\ref{tab:ti} that in perfective clauses, an object follows the verb whereas in imperfective clauses, an object precedes the verb. As illustrated by  (\ref{ex:head1})--(\ref{ex:head6}) above and (\ref{ex:headsee}) here, the reflexive noun follows the expected word order and person marking pattern for genitive constructions.
\ea \label{ex:headsee}

\gll ǹ dá ń=dēɡē n̄=jìjɛ̀ \\
1\textsc{sg}.A 1\textsc{sg}.\textsc{ipfv} 1\textsc{sg}.B=head.1\textsc{sg}.\textsc{poss} 1\textsc{sg}.B=see.1\textsc{sg}.\textsc{ipfv}  \\
\glt `I see myself.' Survey [AD2020-01-15, Line 1]
\z
Note that there is no specific reflexive possessive construction in Bangime. That is, `possessor's (own) possessed' is expressed the same as the regular possessive construction, unless a potentially semantic ambiguity may arise with the noun `head'. Compare examples using the verb \textit{kara} `shave' in (\ref{ex:beardshave})--(\ref{ex:headshave}).
\ea \label{ex:beardshave}

\gll ǹ dá máá sémbō ŋ̄=kàà \\
1\textsc{sg}.A 1\textsc{sg}.\textsc{ipfv} 1\textsc{sg}.\textsc{poss} beard.\textsc{poss} 1\textsc{sg}.B=shave.1\textsc{sg}.\textsc{ipfv} \\
\glt `I am shaving my beard.' Survey [AD2020-01-15, Line 5]
\z
\ea \label{ex:headshave}
\gll ǹ dá máá \textbf{ń}=dēɡè ŋ̀=kàà \\
	1\textsc{sg}.A  1\textsc{sg}.\textsc{ipfv}  1\textsc{sg}.\textsc{poss} 1\textsc{sg}.B=head.1\textsc{sg}.\textsc{poss} 1\textsc{sg}.B=shave.1\textsc{sg}.\textsc{ipfv} \\
	\glt `I am shaving my \textbf{(own)} head.' Survey [AD2020-01-15, Line 6]
	\z
While the possessive morpheme is sufficient to indicate that the subject is shaving his or her own beard in (\ref{ex:beardshave}), in (\ref{ex:headshave}) the addition of the person index from Set B disambiguates disjoint-reference with the subject. 

One other method of distinguishing coreference from disjoint-reference is with the reflexive pronoun that is presented below in (\S \ref{subsec:refpro}). While the person index preceding the noun `place' in (\ref{ex:chez1}) could indicate either coreference or disjoint-reference with the subject, the reflexive pronoun in (\ref{ex:chez2}) can only mean coreference with the subject.
\ea \label{ex:chez1}
\gll dòò $\emptyset$ dà ɲàw̃ɔ̄ ŋ̄=jàw \\  
	sleep.\textsc{indf} 3\textsc{sg}.A 3\textsc{sg}.\textsc{ipfv}  sleep.3\textsc{sg}.\textsc{ipfv} 3\textsc{sg}.B=place \\
	\glt `S/He\textsubscript{x} will sleep at his/her\textsubscript{x,y} place.' \citep[441]{HantganHeath2018} \\
	\z
\ea \label{ex:chez2}
\gll dòò $\emptyset$ dà ɲàw̃ɔ̄ mīì jàw \\  
	sleep.\textsc{indf} 3\textsc{sg}.A 3\textsc{sg}.\textsc{ipfv} sleep.3\textsc{sg}.\textsc{ipfv} 3\textsc{sg}.D place \\
	\glt `S/He\textsubscript{x} will sleep at his/her\textsubscript{x} place.' \citep[441]{HantganHeath2018}
	\z
This is the only use that I am aware of in the language of the reflexive pronoun in a possessive function. Note that locative positions, next to, besides, in front of, etc. use the possessive morpheme followed by a postpostion and not the reflexive noun (or pronoun).

For some reflexive constructions such as those given in  (\ref{ex:killself})--(\ref{ex:hateself}), speakers preferred the use of the reflexive noun to that of the reflexive pronoun described in (\S \ref{subsec:refpro}).	
\ea \label{ex:talkself}

\gll ǹ dá dìɡā n̄=dīɡà n̄=dēɡē wáj \\
1\textsc{sg}.A 1\textsc{sg}.\textsc{ipfv} talk.\textsc{indf} 1\textsc{sg}.B=talk.1\textsc{sg}.\textsc{ipfv} 1\textsc{sg}.B=head.1\textsc{sg}.\textsc{poss} \textsc{rslt} \\
\glt `I am talking to myself.' Survey [AD2020-01-15, Line 11]
\z
\ea \label{ex:killself}
\gll ń jùrà ǹ=dēɡè \\
3\textsc{sg}.A kill.3\textsc{sg}.\textsc{pfv} 1\textsc{sg}.B=head.3\textsc{sg}.\textsc{poss} \\	
\glt `S/he killed her/himself.' Survey [AD2020-01-15, Line 12]
\z
\ea \label{ex:hateself}
\gll ǹ pɔ́ndɔ̄ n̄=déɡé \\
1\textsc{sg}.A hate.1\textsc{sg}.\textsc{pfv} 1\textsc{sg}.B=head.1\textsc{sg}.\textsc{poss} \\	
\glt `I hate myself.' Survey [AD2020-01-15, Line 13]
\z
Thus, the reflexive noun is the favored coreference strategy when used with verbs that can be considered to be otherwise hetero-directed (cf. \citealt[7]{Puddu2019}), or the autopathic domain as defined by \cite[17]{Haspelmath2019}. In the following subsection, (\S \ref{subsec:refpro}), the reflexive pronoun is shown to be used with a middle type of meaning as depicted in \cite{Kemmer1994}.

\subsection{Reflexive pronoun}
\label{subsec:refpro}
The second strategy for indicating coreference between a participant in object role and its antecedent in subject role is in the form of the person form from Set D above in Table \ref{tab:bpp}, which, in all persons except second, is \textit{mi(i)}. Such reflexive pronouns can also be used in middle functions. As expected on the basis of \cite{Kemmer1994}, Bangime uses the reflexive pronoun with self-directed verbs such as bodily care, verbs of posture (or change of posture), motion, and emotion. Some examples of this type of verb are given in the third person singular form in Table \ref{tab:rv}.
\begin{table}[H]
	\caption{Middle-like verb phrases}
	\label{tab:rv}
	\begin{tabular}{l rrr}
		\lsptoprule
		Gloss & \textsc{ipfv} & \textsc{pfv} & \textsc{prf}/\textsc{rslt} \\ 
		\midrule
		hide  &  mīì ǹ=dààndà  &  dààndà mīì  & n dāāndī mīì ŋ̀=kɛ̀ \\
		stretch  &  mīì m̀=bòrndà  & bòrndà mīì  & m bōrndì mīì ŋ̀=kɛ̀   \\
		scratch & mīì ŋ̀=kɔ̀ɡɔ̀jɔ̀ & kɔ̀ɡɔ̀jɔ̀ mīì & kɔ̄ɡɔ̄jì ŋ̀=kɛ̀ \\
		lie straight & mīì m=bàràɡà & bàràɡà mīì & m=bārɡì wáj \\
		\midrule
		lean & m=pɛ̀ɡɛ̀ mīì & pɛ̄ɡɛ̄rɛ̀ & pɛ̄ɡɛ̄rɛ̀ \\
		\lsptoprule
		Gloss & \textsc{VblN} & \textsc{ipfv} & \textsc{pfv} \\ 
		\midrule
		bathe & mīì ǹ=tùrà & mīì ǹ=tùrà & tùū mīì \\
		\lspbottomrule
	\end{tabular}
\end{table}
In Bangime, these verbs obligatorily take the reflexive pronoun in object position. This type of reflexive or middle marking is discussed in \cite[7]{Haspelmath2019} as belonging to the category of reflexive voice markers. The verbal noun `bathing' also necessitates the presence of the reflexive pronoun and is identical to its use in the imperfective aspect; compare (\ref{ex:bathe3}) with (\ref{ex:bathe4}). 
	\ea \label{ex:bathe3}

	\gll à bɔ̀w dà \textbf{mīì} ǹ=tūrā \\
		 2\textsc{sg}.A father.2\textsc{sg}.\textsc{poss} 3\textsc{sg}.\textsc{ipfv} 3\textsc{sg}.D 3\textsc{sg}.B=bathe.3\textsc{sg}.\textsc{ipfv} \\
		\glt `Your father is bathing.' Survey [AD2020-01-15, Line 3]
		\z
	\ea \label{ex:bathe4}
	\gll nɛ̀ tùū \textbf{mīì} \\ 
	1\textsc{pl}.A 1\textsc{pl}.B=bathe.1\textsc{pl}.\textsc{pfv} 1\textsc{pl}.D \\
	\glt `We bathed.' Survey [AD2020-01-15, Line 4]
	\z
%Additional examples of this type of verb are given in (\ref{ex:anger1})--(\ref{ex:anger2}) which are drawn from \cite[~325--326]{HantganHeath2018}.
%\begin{multicols}{2}
%	\ea \label{ex:anger1}
%	\gll $\emptyset$ dʒìì-ndì \textbf{mí} \\
%	3\textsc{sg}.A eat-\textsc{caus} 1\textsc{sg}.C  \\
%	\glt `He/She angered [lit. fed] me.' \\
%	\z
%	\ea \label{ex:anger2}
%	\gll $\emptyset$ dʒīī-ndī \textbf{mīì} \\
%	3\textsc{sg}.A eat-\textsc{caus} 3\textsc{sg}.C  \\
%	\glt `He/She angered him/her.'
%	\z
%\end{multicols}
Although most of the verbs that are formed with the reflexive pronoun, or voice marker, are of the type described by \cite{Kemmer1994}, some idiomatic uses do involve more typically active verbs such as those depicted in (\ref{ex:slip})--(\ref{ex:dym}).
\ea \label{ex:slip}
\gll màà nɔ́w̃ ŋ̄=kóó ń=tāɣā \textbf{mīì} \\
3\textsc{sg}.\textsc{poss} mouth.\textsc{poss} 3\textsc{sg}.B=\textsc{pfv} 3\textsc{sg}.B=take.3\textsc{sg}.\textsc{pfv} 3\textsc{sg}.D \\
\glt `His mouth slipped [lit. took himself, fig. spoke inappropriately].'\\
Narrative [NB2010-07-16, Line 25]
\z
\ea \label{ex:jackal}

\gll ā ɡòndì-ɛ̀ɛ̀ ŋ̄=kóó ń=tāɣā \textbf{mīì} ŋ̄=kɛ̀ \\
\textsc{def} jackal.\textsc{def}-\textsc{dim} 3\textsc{sg}.B=\textsc{pfv} 3\textsc{sg}.B=take.3\textsc{sg}.\textsc{pfv} 3\textsc{sg}.D 3\textsc{sg}.B=\textsc{prf} \\
\glt `The jackal has left [lit. has taken himself].' Narrative \citep[~401]{HantganDiss}
\z
The pronoun \textit{mii} appears throughout my corpus of texts in which it is translated as either a reflexive, as exemplified in (\ref{ex:dym}), or a middle, (\ref{ex:rain}), meaning.
\ea \label{ex:dym}

\gll ŋ kóó á bṵ̀ṵ̀-mí=ndɛ̀ tíɡé nī ŋ=kòò jērò nùùw̃à \textbf{mīì} \\
3\textsc{pl}.B \textsc{pfv} \textsc{def} Bounou.\textsc{def}-\textsc{dym}=\textsc{pl} also 3\textsc{pl}.A 3\textsc{pl}.B=\textsc{pfv} become.3\textsc{pl}.\textsc{pfv} prepare.3\textsc{pl}.\textsc{pfv} 3\textsc{pl}.D \\
\glt `The people of Bounou, they also prepared themselves.' \\
Narrative [SD2010-10-01, Line 10]
\z
\ea \label{ex:rain}

\gll à ɥɔ̄w̃ máá \textbf{mīì} \\
\textsc{def} rain.\textsc{def} like.3\textsc{pl}.\textsc{pfv} 3\textsc{pl}.D  \\
\glt `The rain (gods) were pleased.' Narrative [SD2013-03-29, Line 11]
\z
To my knowledge, this example does not imply a reflexive reading such as `they, themselves, were pleased'. However, in the next section (\S \ref{sec:foc}), focus and intensifying constructions using the reflexive noun are discussed.%It may be of interest that one other set of verbs in Bangime is formed with an obligatory object `thing' \textit{ki}, preceded by pronouns of set B, and thus interpreted as a genitive construction, similar to that of the reflexive noun construction. 

\section{Focus and intensifying constructions}
\label{sec:foc}
The reflexive pronoun can be combined with the reflexive noun to convey an intensified meaning in the sense of \cite{KoenigSiemund2000}. Two textual examples are provided in  (\ref{ex:marry})--(\ref{ex:slave}).
\ea \label{ex:marry}

\gll \textbf{mí} \textbf{déɡē} jáátì mí ɲànà \\
	1\textsc{sg}.C head.1\textsc{sg}.\textsc{poss} definitely 1\textsc{sg}.C take.1\textsc{sg}.\textsc{pfv} \\
	\glt  `\textbf{Me, myself} definitely, I [lit. `it is me (who)'] married [lit. `took'] (her).'\\
	Narrative [TB2010-10-20, Line 185]
	\z
\ea \label{ex:slave}
\gll mì kéndé mā nìŋà \textbf{ā} \textbf{déɡé} kɔ̀mɛ̀ \\
	3\textsc{sg}.C say \textsc{proh} say.2\textsc{sg}.\textsc{pfv} 2\textsc{sg}.B head.2\textsc{sg}.\textsc{poss} slave.\textsc{indf} \\
	\glt `He said, ``do not say that you, \textbf{yourself}, are a slave''.' \\
	 Narrative [TB2010-10-20, Line 201]
	\z
As in English, another use of the reflexive noun is that of a doing something for (\ref{ex:built4self}), or by (\ref{ex:liveself})--(\ref{ex:builtself}), oneself.
\ea \label{ex:built4self}
\gll m máárà à kùwò \textbf{ǹ=déɡé} wāj \\
1\textsc{sg}.A build.1\textsc{sg}.\textsc{pfv} \textsc{def} house.\textsc{def} 1\textsc{sg}.B=head.1\textsc{sg}.\textsc{poss} \textsc{dat} \\
\glt `I built the house \textbf{for myself}.' Survey [AD2020-01-15, Line 10]
\z 
\ea \label{ex:liveself}
\gll $\emptyset$ tɔ̀pɔ̀w \textbf{ǹ=dēɡè} ǹ=tērō màà kúwò ŋ̀=kò \\
3\textsc{sg}.A alone 3\textsc{sg}.B=head.3\textsc{sg}.\textsc{poss} 3\textsc{sg}.B=sit.3\textsc{sg}.\textsc{pfv} 3\textsc{sg}.\textsc{poss} house.3\textsc{sg}.\textsc{poss} 3\textsc{sg}.B=\textsc{loc} \\
\glt `He lived \textbf{by himself} in his house.' Survey [AD2010-10-30, Line 2]
\z
\ea \label{ex:builtself}

\gll mí \textbf{ǹ=déɡé} máárà à kùwò \\
1\textsc{sg}.C 1\textsc{sg}.B=head.1\textsc{sg}.\textsc{poss} build.1\textsc{sg}.\textsc{pfv} \textsc{def} house.\textsc{def}\\
\glt `\textit{I} built the house \textbf{by myself}.' Elicit \citep[443]{HantganHeath2018}
\z 
Described by \cite[8--9]{GastKoenig2002} as adverbial uses of ``self-forms'', this additional use of the reflexive noun is defined as the reflexive adverb in \cite[10]{Haspelmath2019}.

\section{Conclusion}
\label{sec:conc}
According to the criteria provided by \cite{Haspelmath2019}, Bangime utilizes two productive strategies for expressing coreference between clause participants: in all the persons except second singular and plural, a special morpheme \textit{mii} is used which may be diachronically related to the third person singular object personal pronoun \textit{m(i)i}. The other option in the language is to use a possessed form of the word for `head', but this construction does not use the possessive morpheme that is usually used in the language but rather a pronominal index that is otherwise only found with kin terms for the purposes of possession. Frequency counts have not yet been obtained from the corpus; it appears that each option is robustly used, but, based on comments from speakers and observations put forth here, the two options seem to be semantically differentiated.

Bangime has a striking feature of multiple markers of subject throughout a phrase: subject marking occurs as the initial constituent of a clause, pre-verbally, and also, when present, before an auxiliary. Even if these markers are represented by null-morphemes, tones serve to signify the subject of the clause. Additionally, Bangime may be considered a pro-drop language in that object pronouns may be pushed to the end of a phrase or even omitted, but the reflexive pronoun remains intact to the clause in question.

%One question remains as pertains to reflexive constructions in Bangime: what is the strategy for long-distance coreference in the language? There is only one such example listed in the corpus which appears to provide evidence for the necessity of a clausemate antecedent. This usage is given in (\ref{ex:longdis}).
%\ea \label{ex:longdis}
%Narrative (NB2010-08-10, Line 13) \\
%\gll $\emptyset$ kóó ŋ́=wòrè à jìjɛ̀ ā ŋījɛ̄rɛ̀ $\emptyset$ māā mɛ̀ \textbf{màà} \textbf{mīì} jāà ŋ̀=kòmbò ā ŋījɛ̄rɛ̀ \\
%3\textsc{sg}.A 3\textsc{sg}.\textsc{pfv} 3\textsc{sg}.B=go.3\textsc{sg}.\textsc{pfv} \textsc{inf} see.\textsc{irr} \textsc{def} woman.\textsc{def} 3\textsc{sg}.A want.3\textsc{sg}.\textsc{pfv} \textsc{rel} 3\textsc{sg}.\textsc{poss} 3\textsc{sg}.D see.3\textsc{sg}.\textsc{pfv} 3\textsc{sg}.B=look.for.3\textsc{sg}.\textsc{pfv} \textsc{def} woman.\textsc{def} \\
%\glt `He\textsubscript{x} went and saw the woman whom he\textsubscript{x} wanted (before) he\textsubscript{x} searched for his\textsubscript{x} wife [lit. the woman]'.
%\z

\section*{Acknowledgments}

This study has received funding from the European Research Council (ERC) under the European Union's Horizon 2020 research and innovation programme (grant agreement No 758232), of which Tatiana Nikitina is the Principal investigator. This research was also funded by US National Science Foundation grant BCS-0537435, ``Dogon languages of Mali'', PI Jeffrey Heath. I am indebted to the villagers of Bounou with whom I conducted my fieldwork and to Adama Soh Dicko who provided additional data specific to this study on reflexives in Bangime. I also wish to express my gratitude to the editors, an anonymous reviewer and Yvonne Treis for their immensely helpful comments on an earlier version of this paper.

\section*{Abbreviations (outside of Leipzig glossing conventions)}
\begin{tabularx}{.45\textwidth}{lQ}
	\textsc{agent} & Agentive \\
	\textsc{dym} & Demonym \\
	\textsc{dim} & Diminutive \\
	\textsc{rslt} & Resultative \\
	\textsc{VblN} & Verbal noun \\
\end{tabularx}

\pagebreak
\printbibliography[heading=subbibliography,notkeyword=this]

\end{document}
