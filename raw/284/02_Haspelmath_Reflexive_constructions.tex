\documentclass[output=paper]{langscibook}

\author{Martin Haspelmath\affiliation{MPI-EVA Leipzig \& Leipzig University}\orcid{0000-0003-2100-8493}}

\title{Comparing reflexive constructions 
in the world’s languages}\label{chapter 2} 

\abstract{}

\IfFileExists{../localcommands.tex}{
  \input{../localpackages}
  %Copy this to localcommands.tex

\usepackage[english]{babel}
\usepackage{amsmath}
\usepackage{amssymb,amsfonts,textcomp}
\usepackage{array}
\usepackage{hhline}
\usepackage{hyperref}

\newenvironment{styleStandard}{}{}
\newenvironment{stylelsAbstract}{}{}
\newenvironment{stylelsSectioni}{}{}
\newenvironment{stylelsSectionii}{}{}
\newenvironment{stylelsBulletList}{}{}
\newenvironment{styleBibliographyi}{}{}
\newenvironment{listWWNumxxvleveli}{}{}
\newenvironment{listWWNumxxvlevelii}{}{}
\newenvironment{listWWNumxxvleveliii}{}{}
\newenvironment{listWWNumxxvleveliv}{}{}
\newenvironment{listWWNumixleveli}{}{}
\newenvironment{listWWNumixlevelii}{}{}
\newenvironment{listWWNumixleveliii}{}{}
\newenvironment{listWWNumixleveliv}{}{}

\newcommand\textstyleListLabelxvi[1]{#1}
\newcommand\labellistWWNumxxvleveli{\thelistWWNumxxvleveli.}
\newcommand\labellistWWNumxxvlevelii{\thelistWWNumxxvlevelii.}
\newcommand\labellistWWNumxxvleveliii{\thelistWWNumxxvleveliii.}
\newcommand\labellistWWNumxxvleveliv{\thelistWWNumxxvleveliv.}
\newcommand\labellistWWNumixleveli{[F0B7?]}
\newcommand\labellistWWNumixlevelii{\textstyleListLabelxvi{o}}
\newcommand\labellistWWNumixleveliii{[F0A7?]}
\newcommand\labellistWWNumixleveliv{[F0B7?]}

\newcounter{listWWNumxxvleveli}
\newcounter{listWWNumxxvlevelii}[listWWNumxxvleveli]
\newcounter{listWWNumxxvleveliii}[listWWNumxxvlevelii]
\newcounter{listWWNumxxvleveliv}[listWWNumxxvleveliii]
\newcounter{itemize}  
  \input{../localhyphenation} 
  \togglepaper[1]%%chapternumber
}{}

\begin{document}
\maketitle

 
\section{Reflexive constructions}\label{sec:Haspelmath:1}

This paper starts out from the presupposition that the comparison of reflexive constructions in the world’s languages must be based on a clear definition of the term \textsc{reflexive} \textsc{construction} as a comparative concept, as well as a range of additional technical terms (summarized in Appendix B~\ref{sec:Haspelmath:b}). I begin with the definition in (\ref{ex:Haspelmath:1}), which I think is largely in line with current usage and is at the same time sufficiently clear for rigorous cross-linguistic comparison.

\ea  \label{ex:Haspelmath:1}  reflexive construction\\
  A reflexive construction is a grammatical construction \\
  (i) that can only be used when two participants of a clause are coreferential \\
	(ii) and that contains a special form (a reflexivizer) that signals this coreference.
\z

Some examples of reflexive constructions are given in (2a-c).

\ea \label{ex:Haspelmath:2}
  \ea \label{ex:Haspelmath:2a}
   Lithuanian\\
  \gll aš	prausi-uo-s\\
     I  wash-\textsc{1sg-refl}   \\
    \glt ‘I wash (myself).’
    
    
  \ex \label{ex:Haspelmath:2b}
   French\\
  \gll Asma\textsubscript{1} 	parle 	d’	elle-même\textsubscript{1}\\
      Asma  talks  of  her-\textsc{refl}  \\
    \glt ‘Asma talks about herself.’
  \ex \label{ex:Haspelmath:2c}
   Malay\\
  \gll Ahmat\textsubscript{1}	tahu	[Salmah\textsubscript{2}	akan	membeli	baju	untuk	dirinya\textsubscript{1/2}].\\
      Ahmat  know  Salmah  \textsc{fut}  buy  clothes  for  \textsc{refl}.\textsc{3sg} \\
    \glt   ‘Ahmat (\textsc{m}) knows that Salmah (\textsc{f}) will buy clothes for him/herself.’  (\citealt[25]{Cole2005})
   \z
\z 
    
    
In (\ref{ex:Haspelmath:2a}) from Lithuanian, the washer and the washed must be the same person, and the verb shows a \textsc{reflexive} \textsc{voice} \textsc{marker}. In (\ref{ex:Haspelmath:2b}) from French, the subject \textit{Asma} and the \textsc{reflexive} \textsc{pronoun} \textit{elle-même} must likewise be coreferential. By contrast, a \textsc{nonreflexive} \textsc{personal} \textsc{pronoun} like \textit{elle} ‘she’ would give rise to a \textsc{disjoint-reference} interpretation here, indicated in the examples by a different subscript number (\textit{Asma\textsubscript{1}} \textit{parle} \textit{d’elle\textsubscript{2}} ‘Asma talks about her’). Disjoint reference means that the pronoun is not coreferential with the subject, and does not even have overlapping reference. In (\ref{ex:Haspelmath:2c}) from Malay, the form \textit{dirinya} must be coreferential either with the subject of its minimal clause (\textit{Salmah}) or with the subject of the matrix clause (\textit{Ahmat}). The participant with which the anaphoric pronoun is coreferential is called its \textsc{antecedent}.

  In the remainder of this paper, I will first discuss the two conditions of the definition in (\ref{ex:Haspelmath:1}) further (\sectref{sec:Haspelmath:2}-\ref{sec:Haspelmath:3}), before introducing a number of additional comparative concepts that are important for comparing subtypes of reflexive constructions (\sectref{sec:Haspelmath:4}-\ref{sec:Haspelmath:13}). Along the way, I will illustrate the most important types of reflexive constructions from a wide range of languages, and I will mention a few generalizations. The wider research programme in which this paper is embedded is the study of Human Language through the identification of common structural traits in the world’s languages (\citealt{Greenberg1963}, and much subsequent work). Importantly, this line of research does not aim to contribute to elegant language-particular analyses, let alone to descriptions of the speakers’ mental grammars. When other linguists adopt very different perspectives in studying reflexive constructions, this is often motivated by additional goals (such as elegant description, mental description, or even the study of innate grammatical knowledge). Appendix A (\ref{sec:Haspelmath:a}) lists a number of proposed universals (primarily to illustrate the need for the technical terms developed throughout the paper), while Appendixes B (\ref{sec:Haspelmath:b}) and C (\ref{sec:Haspelmath:c}) contain lists of terms with definitions and some further discussion. The definitions are important in order to allow us to identify the common structural traits of the world’s languages independently of innatist claims, and ideally, we would have standard definitions of many commonly used terms \citep{Haspelmath2020}.

  For other surveys of reflexive constructions in the world’s languages, see \citet{Faltz1977} (this old dissertation is still very readable), \citet{Geniusiene1987}, \citet{Huang2000}, \citet[Chapter 22]{Dixon2012}, and \citet{Everaert2013}.

\section{First condition: Two coreferential clause participants}\label{sec:Haspelmath:2}

Reflexive constructions express coreference between two clause participants (semantically present entities). These need not be expressed as overt arguments. In verb-marked constructions like Lithuanian \textit{prausiuo-s} (‘I wash’, \ref{ex:Haspelmath:2a} above), there is only a single expressed argument which can be said to bear both semantic roles (agent and patient), and thus to represent both participants.

  More generally, the antecedent of an anaphoric pronoun need not be overtly present but can be inferred from the context. This happens in languages where the subject participant need not be overt, as illustrated in (\ref{ex:Haspelmath:3}).

\ea Polish
    \label{ex:Haspelmath:3}\\
    \gll  {Widziała}  {siebie}    {w}  {lustrze.}\\
      she.saw  self.\textsc{acc}  in  mirror  \\
    \glt‘She saw herself in the mirror.’
 \z

There is no reason to assume that the subject is present in the syntax of languages like Polish, so the condition on coreference is best formulated in semantic terms, with respect to semantic participants rather than syntactic arguments (see also \citealt{Jackendoff1992}).\footnote{{Note also that the coreference may be} {\textsc{partial}} {(e.g. ‘I exploit us’, \citealt{HampeLehmann2013}), or the antecedent may be} {\textsc{split}} {(see \citet{Volkova2017} on situations like ‘Petja}{\textsubscript{1}} {showed Ivan}{\textsubscript{2} }{themselves}{\textsubscript{1+2}} {on the photo’, which is possible with one type of reflexive pronoun in Meadow Mari). The opposite of coreference is disjoint reference, which excludes partial or split coreference.}}


\section{Second condition:  A special form that signals coreference}\label{sec:Haspelmath:3}


The second condition mentioned in (\ref{ex:Haspelmath:1}) is that reflexive constructions must contain a special form signaling coreference. Thus, the constructions illustrated in (\ref{ex:Haspelmath:4}) and (\ref{ex:Haspelmath:5}) are not regarded as reflexive constructions, even though they can only be used when there is coreference of two participants 

\ea   \label{ex:Haspelmath:4}
    {He} {undressed.}
\z

\ea \label{ex:Haspelmath:5}
    {She} {wants} {to} {sing.}
\z 

In (\ref{ex:Haspelmath:4}), it is clear from the meaning of the verb and from the construction that the two participants of \textit{undress} (the undresser and the undressed) are coreferential, and in (\ref{ex:Haspelmath:5}), the animate participant of \textit{want} (the wanter) and the participant of \textit{sing} (the singer) are coreferential. But there is no special form that signals the coreference, so these are not reflexive constructions.\footnote{{Linguists have often found it useful to have different terms for grammatical meanings and corresponding grammatical markers or constructions, e.g.} {\textit{recipient}} {vs.} {\textit{dative}}, {\textit{question}} {vs.} {\textit{interrogative}}, {\textit{sex}} {vs.} {\textit{gender}}, {\textit{time}} {vs.} {\textit{tense}}, {\textit{speech-act} \textit{role}} {vs.} {\textit{person}}, {\textit{property} \textit{concept}} {vs.} {\textit{adjective}}, {\textit{causal}} {vs.} {\textit{causative}} {\citep{Haspelmath2016}, and} {\textit{mutual}} {vs.} {\textit{reciprocal}} {\citep{Haspelmath2007}. There are of course some authors who call cases like (\ref{ex:Haspelmath:4}) “reflexive” (e.g. \citealt{Reinhart1993}), but I find it clearer to reserve the term} {\textit{reflexive}} {to (constructions with) special forms that signal coreference. (Cases like (\ref{ex:Haspelmath:4}) may be called “unmarked autopathic verbs”; see \S~\ref{sec:Janic:2.2}{sec:Haspelmath:8}).}} (Below in \sectref{sec:Haspelmath:12} I say more about coreference constructions that are not reflexive constructions.)


\section{Coreference within the clause can be expressed in other ways}\label{sec:Haspelmath:4}

Reflexive pronouns like French \textit{elle-même} and English \textit{herself} have often been discussed in the general context of \textsc{anaphora}, i.e. the use of linguistic forms or constructions to signal coreference within the discourse or within a clause. But reflexive pronouns are not the only way in which anaphoric reference can be expressed.  All languages also have nonreflexive anaphoric pronouns like English \textit{he/she/they}, whose use is also often syntactically conditioned. 

Nonreflexive anaphoric pronouns may often refer to participants in the nonlinguistic context (as in 6a), and they may be coreferential with participants in the discourse (as in \ref{ex:Haspelmath:6b}).\footnote{{There are interesting pragmatic conditions on such} {\textsc{exophoric}} (\ref{ex:Haspelmath:6a}) and \textsc{endophoric} ({\ref{ex:Haspelmath:6b}) uses of personal pronouns (cf. \citealt{Ariel1990,Ariel2001}), but for reflexive constructions, they play no role, and only grammatical conditions on anaphora are considered in the present paper.}}

\ea \label{ex:Haspelmath:6}
    \ea \label{ex:Haspelmath:6a}
   (watching a politician\textsubscript{1} talk:) {I} {disagree} {with} {her\textsubscript{1}}.
    \ex \label{ex:Haspelmath:6b}
   {Angela} {Merkel\textsubscript{1}} {has} {been} {chancellor} {for} {too} {long.} {Many} {people} {think} {that} {she\textsubscript{1}} {should} {go.}
    \z
\z 

But in addition, we often find syntactic conditions on anaphoric pronouns that have interested many syntacticians since the 1960s (e.g. \citealt{Langacker1969}). In many or most languages, a nonreflexive anaphoric pronoun in object or oblique position cannot be coreferential with the subject of its clause, as can be illustrated from English:\footnote{ {Such anaphoric pronouns are also called} {\textsc{obviative}} {(\sectref{sec:Haspelmath:10}).}}

\ea%7
    \label{ex:Haspelmath:7}
   \ea  \label{ex:Haspelmath:7a}
   {*Pedro\textsubscript{1}} {admires} {him\textsubscript{1}}.
   \ex  \label{ex:Haspelmath:7b}
   {*Angela} {Merkel\textsubscript{1}} {was} {astonished} {by} {her\textsubscript{1}}.
\z
\z 

Instead, English must use a special set of \textsc{reflexive} \textsc{pronouns,} i.e. anaphoric pronouns that are specialized for coreferential use within a clause. But other languages can use their nonreflexive pronouns also for coreference with the subject.\footnote{ {Note that in this paper, the term} {\textit{subject}} {is used in the sense} {\textit{‘}}{S- or A-argument’, and} {\textit{object}} {in the sense ‘P-argument or R-/T-argument’ (cf. \citealt{Haspelmath2020}).} } This is well-known for Old English, and the same has been reported for several creole languages (e.g. Haitian Creole) and for several Austronesian languages (e.g. Jambi Malay) (\citealt[222]{Huang2000} gives a longer list of such languages).

\ea Old English\label{ex:Haspelmath:8}\\
  \gll {þa}  {behydde}  {Adam\textsubscript{1}}  {hine\textsubscript{1/2}}     \\
    then  hid  Adam  him\\
   \glt   ‘then Adam hid himself’ OR: ‘then Adam hid him’  (\citealt{KoenigVezzosi2004}:232)  
\z 

\ea Haitian Creole \label{ex:Haspelmath:9}\\
    \gll {yo}  {wè}  {yo}
\\
    they  see  they \\
    \glt ‘they saw them’ OR: ‘they saw themselves’ (OR: ‘they saw each other’) (\citealt[203]{Dechaine1994})
\z

\ea Jambi Malay\label{ex:Haspelmath:10}\\
    \gll  {dio\textsubscript{1}}  {cinto}  {dio\textsubscript{1/2}} \\
      he    love  he  \\
    \glt‘he loves him’ OR: ‘he loves himself’ (\citealt{Cole2015}: 147)
\z

Such anaphoric forms are not considered reflexive pronouns (and the constructions are not reflexive constructions) because they can also be used when there is no coreference witin the clause.

Additionally, ordinary first and second person pronouns can often be used subject-coreferentially, as in German in (\ref{ex:Haspelmath:11}). And in some languages, the same nominal can be repeated with identical reference in the same sentence, as has been reported for Zapotec of San Lucas Quiaviní (see \ref{ex:Haspelmath:12}).

\ea German\label{ex:Haspelmath:11} \\
  \gll {Gestern}   {habe}  {ich\textsubscript{1}}   {mich\textsubscript{1}}   {im}   {Fernsehen}   {gesehen.}\\
  yesterday  have  I  me  on  television  seen\\
   \glt ‘Yesterday I saw myself on television.’ (Lit. ‘I saw me’)
\z 

\ea San Lucas Quiaviní Zapotec (Mexico)\label{ex:Haspelmath:12}     \\ 
    \gll{R-yu’lààa’z}  {Gye’eihlly}  {Gye’eihlly.} \\
      \textsc{hab}{}-like   Mike   Mike   \\
    \glt‘Mike likes himself.’ (Lit. ‘Mike likes Mike.’) \citep[84]{Lee2003} 
\z


Unlike (\ref{ex:Haspelmath:8}-\ref{ex:Haspelmath:10}), these sentences are unambiguously subject-coreferential, but they are not reflexive constructions either, because they do not involve any special forms.

 In the literature, following the tradition of \citet{Reinhart1976,Reinhart1983a} and \citet{Chomsky1981}, the syntactic conditions on clause-internal coreference are often treated under the heading of “binding” (using a term borrowed from mathematical logic), and there is a substantial and highly complex literature in this tradition (e.g. \citealt{Everaert2003}; \citealt{Büring2005}; \citealt{Truswell2014}).\footnote{ {\textit{Binding}} {is typically defined as syntactic coindexing of two elements X and Y when X c-commands Y. Note that “bound“ elements in this sense may or may not be reflexive pronouns, and may or may not have a co-varying interpretation (involving (semantic) bound variable anaphora, \sectref{sec:Haspelmath:13}). Coreferential forms may or may not involve syntactic binding, and co-varying interpretations may or may not involve syntactic binding. The relationship of syntactic binding to coreference, to reflexive and nonreflexive pronouns, and to pronoun interpretation is thus quite indirect (and often unclear, given the problems with determining c-command that are mentioned in \sectref{sec:Haspelmath:7}). All these problems are avoided when one avoids the term} {\textit{binding}}.} For the purposes of cross-linguistic comparison, it seems best to avoid the term “binding” and to talk about \textsc{coreference} (for anaphoric relations in the broadest sense) and \textsc{subject-coreferential} uses of anaphoric forms (for anaphoric relations between the subject and an anaphoric pronoun).\footnote{ {It should be noted that the term} {\textit{coreference}} {has also been used more narrowly, for discourse coreference excluding co-varying interpretations (as in} {\textit{Every} \textit{woman}}{\textit{\textsubscript{1}}}{ \textit{loves} \textit{her}}{\textit{\textsubscript{1}}}{ \textit{dog}}{); for more on the two subtypes of coreference, see \sectref{sec:Haspelmath:13}}.}


\section{Types of reflexivizers}\label{sec:Haspelmath:5}
Reflexive constructions always include some special form that signals the impossibility of the disjoint-reference interpretation. Such forms are called \textsc{reflexivizers} here, and three main types are distinguished: \textsc{reflexive} \textsc{nominals}, \textsc{reflexive} \textsc{voice} \textsc{markers}, and \textsc{reflexive} \textsc{argument} \textsc{markers}. These are defined and exemplified in this section. In the final subsection (\sectref{sec:Haspelmath:5.4}), I briefly mention other kinds of reflexive constructions which do not fall into the three main types.


\subsection{Reflexive nominals (or pronouns)}\label{sec:Haspelmath:5.1}

The most prominent type of reflexivizer is what would ideally be called \textsc{reflexive} \textsc{nominal}, illustrated in (\ref{ex:Haspelmath:13}). Such forms are often called \textit{reflexive} \textit{pronouns}, and some of them are sometimes called “reflexive nouns”.

\ea \label{ex:Haspelmath:13}
    \ea English\label{ex:Haspelmath:13a}\\ 
   {They} {criticized} \textbf{{themselves}}.
    \ex {Basque \label{ex:Haspelmath:13b}\\
    \gll  {Geu-re}  \textbf{{buru-a}}  {engaina-tzen}  {d-u-gu.}\\
     we-\textsc{gen}  head-\textsc{def}  deceive-\textsc{ipfv}  \textsc{3.abs-tr}{}-\textsc{1pl.erg}\\
     \glt ‘We deceive ourselves.’ (\citealt{Evseeva2018}: 400)}
    \ex {Egyptian Arabic\label{ex:Haspelmath:13c}\\
    \gll  {Šaaf-it}  \textbf{{nafsa-ha.}}\\
      saw-\textsc{3sg.f}  self-\textsc{3sg.f.poss}\\
    \glt  ‘She saw herself.’  }
    \ex {Modern Greek\label{ex:Haspelmath:13d}\\
    \gll  {O}  {Pétros}  {aghapái}  \textbf{{ton}}  \textbf{{eaftó}}   \textbf{{tu}}.\\
    the  Petros  loves  \textsc{def}  self   \textsc{3sg.poss}\\
    \glt ‘Petros loves himself.’ (Lit. ‘Petros loves his self’) \citep[202]{Everaert2013}  }
    \ex {Tukang Besi (Austronesian)\label{ex:Haspelmath:13e}\\
    \gll  {O-pepe-’e}  \textbf{{na}}  \textbf{{karama-no}}  {te}  {ana.}\\
    3.\textsc{real}{}-hit-3.\textsc{obj}  \textsc{nom}  self-3.\textsc{poss}  \textsc{core}  child\\
    \glt  ‘The child hit himself.’ \citep[418]{Donohue1999}}
      \z
\z 
   


The term \textit{reflexive} \textit{nominal} emphasizes that in many languages, these forms behave like full nominals, e.g. in that they can take a definite article (as in Basque, \ref{ex:Haspelmath:13b}), an adpossessive person index (as in Egyptian Arabic, \ref{ex:Haspelmath:13c}), or both (as in Modern Greek,  \ref{ex:Haspelmath:13d}).

  The term \textsc{reflexive} \textsc{nominal} would be ideal for these forms because what they share is that they can occur in the regular object position (as P-argument, as in \ref{ex:Haspelmath:13a}--\ref{ex:Haspelmath:13e}) and as adpositional complements, as in (\ref{ex:Haspelmath:14a}--\ref{ex:Haspelmath:14b}). But since the term \textit{reflexive} \textit{pronoun} is also very widespread and unambiguous, I use the two terms interchangeably.

\ea%14
    \label{ex:Haspelmath:14}
    \ea English\label{ex:Haspelmath:14a}\\
    {They} {talked} {about} \textbf{{themselves}}.
    \ex  Basque \label{ex:Haspelmath:14b}\\
    \gll {Bere}  \textbf{{buruari}}  {buruz}  {hitz}  {egin}  {zuten.}\\
         their  heads  about  talk  do  \textsc{aux}.\textsc{3pl.pst}\\
    \glt ‘They talked about themselves.’
    \z
\z 
  
  

Moreover, these forms can normally occur in isolation, e.g. in elliptical answers (\textit{Who} \textit{did} \textit{they} \textit{talk} \textit{about?} \textit{Themselves}). In this regard, reflexive nominals are like full nominals, and crucially distinct from person indexes \citep{Haspelmath2013}, which are bound (i.e. do not occur in isolation) and usually cannot occur equally as objects and as adpositional complements. More on subtypes of reflexive nominals and their properties will be said below in \sectref{sec:Haspelmath:6}.


\subsection{Reflexive voice markers}\label{sec:Haspelmath:5.2}

A reflexive voice marker is a verbal affix that indicates the coreference of two participants of a verb. While this is not logically necessary, it is in fact always an object participant that is coreferential with the subject participant. Most often, the reflexive voice marker occurs on the verb stem, as in (\ref{ex:Haspelmath:15a}--\ref{ex:Haspelmath:15d}). 

\ea%15
    \label{ex:Haspelmath:15}
    \ea Turkish  (suffix {{}-n})\label{ex:Haspelmath:15a}\\
    \gll{kurula-}\textbf{{n}}{{}-dı-m}\\
     dry-\textsc{refl-pst-1sg}   \\
    \glt ‘I dried myself.’
    \ex Thulung (Trans-Himalayan; suffix {{}-si})\label{ex:Haspelmath:15b}\\
     \gll {Memma}  {thʌ{}-\textbf{si-m}}  {sintha} {koŋŋa}   {je}.\\
        then  hide-\textsc{refl-suff}  night  only  come.out \\
    \glt ‘Then he hides (himself) and only comes out at night.’  \citep[54]{Lahaussois2016}
     \ex Hebrew (prefix {hit-})\label{ex:Haspelmath:15c}\\
     \gll  {Dan}  \textbf{{hit}}{{}-raxec.}\\
       Dan  \textsc{refl}{}-washed  \\
    \glt ‘Dan washed (himself).’ (\citealt{Reinhart2005}: 390)
     \ex Kolyma Yukaghir (Siberia; prefix {met-})\label{ex:Haspelmath:15d}\\
     \gll {Tudel}  \textbf{{met}}{{}-juø-j.}\\
      he  \textsc{refl}{}-see-\textsc{3sg.intr}  \\
    \glt‘He is looking at himself.’ \citep[227]{Maslova2003}
    \z
\z   
      

But occasionally, the reflexive voice marker is cumulated with (= expressed as the same marker as) a person marker, as in Modern Greek.

\ea Modern Greek \label{ex:Haspelmath:16}\\
    \ea  \label{ex:Haspelmath:16a}
    \gll  {xteníz-}\textbf{{ome}}\\
       comb-\textsc{1sg.refl} \\
    \glt ‘I am combing (myself, my hair).’
    \ex  \label{ex:Haspelmath:16b}
    \gll {xteníz-}\textbf{{ese}}\\
       comb-\textsc{2sg.refl}  \\
    \glt ‘You are combing (yourself, your hair).’
    \z
\z 
          
Finally, the reflexive voice marker may occur in a peripheral position, outside of a tense affix, as in Panyjima, and additionally outside a subject number affix, as in Russian (and in Lithuanian, as in (\ref{ex:Haspelmath:2a}) above).\footnote{ {A pattern of this type also exists marginally in English, with the prefix} {\textit{self-}} {(e.g.} {\textit{she} \textit{self-medicates}}).}

\ea  \label{ex:Haspelmath:17}
    \ea Panyjima (Pama-Nyungan; suffix -{pula}) \label{ex:Haspelmath:17a}\\
    \gll {Ngatha}  {wirnta-rna-}\textbf{{pula}}  {jina.}\\
     \textsc{1sg.nom}  cut-\textsc{pst-refl}  foot \\
    \glt ‘I cut myself in the foot.’ \citep[160]{Dench1991}
    \ex Russian (suffix {{}-s’/-sja}) \label{ex:Haspelmath:17b}\\
    \gll  {My}  {my-l-i-}\textbf{{s’}}.\\
      we  wash-\textsc{pst-pl-refl}\\
      \glt ‘We washed (ourselves).’
    \z
\z   

    

In many languages with reflexive voice markers, these are not as general as typical inflectional markers, and they are often regarded as derivational as they my be restricted and unproductive. Verbs with reflexive voice markers are therefore often called \textsc{reflexive} \textsc{verbs}.

Reflexive voice markers are not always easy to distinguish from reflexive argument markers, which are discussed next.


\subsection{Reflexive argument markers}\label{sec:Haspelmath:5.3}

In some languages, a reflexive form is very similar to object person indexes in that it occurs in the same paradigmatic slot as the person index and cannot cooccur with a person index of the same role. Some examples are given below, where a nonreflexive third-person index is contrasted with a reflexive person index.

\ea French 
    \label{ex:Haspelmath:18}\\
    \gll  {il}  \textbf{{la}}  {voyait}      {il}  \textbf{{se}}  {voyait}\\
      he  \textsc{3sg.f}  saw        he  \textsc{refl}  saw   \\
    \glt‘he saw her’        ‘he saw himself’
\z


\ea Swahili 
    \label{ex:Haspelmath:19}\\
    \gll {a-li-}\textbf{{m}-kata}        {a-li-}\textbf{{ji}}{{}-kata}\\
       \textsc{3sg-pst-3sg.obj}{}-cut    \textsc{3sg-pst-refl}{}-cut  \\
    \glt    ‘she cut him’        ‘she cut herself’
\z


\ea Abkhaz \footnote{\citet[77,105]{Hewitt1979}}  
    \label{ex:Haspelmath:20}\\
    \gll \textbf{{bə}}{{}-z-bò-yt’} \textbf{{lçə-}}{l-š-we-yt’}\\
       \textsc{2sg.obj}{}-\textsc{1sg.sbj}{}-see-\textsc{fin}    \textsc{refl.f-3sg.f.sbj}{}-kill-\textsc{dyn-fin}  \\
    \glt ‘I see you’        ‘she kills herself’
\z
    

       

Person indexes like the French proclitic (or prefix) \textit{la=}, the Swahili prefix \textit{m-,} and the Abkhaz prefix \textit{bə-} are crucially different from independent personal pronouns in that they cannot occur in isolation, but are bound to the verb (or occur in a special slot for second-position clitics) (see \citealt{Haspelmath2013}). They are thus not nominals (= reference-performing expressions that can occur in isolation), contrasting with full nominals and independent personal pronouns. The forms \textit{se=}, \textit{ji-} and \textit{lçə-} in the examples above are different from the voice markers in \sectref{sec:Haspelmath:5.2} in that they occur in the same slot and in complementary distribution with person indexes, so they can be treated as argument indexes, even though they do not (necessarily) vary for person. The Abkhaz reflexive argument index does vary for person (\textit{sçə-s-š-we-yt’} ‘I kill myself’), but the Swahili prefix \textit{ji-} does not (\textit{ni-li-ji-kata} ‘I cut myself’),\footnote{ {See also \citet[\S. 4]{DechaineWiltschko2017a} on} {\textit{zvi-}} {in Shona (another Bantu language), which works very similarly.}} and the French \textit{se} occurs only in the third person.\footnote{ {French allows first and second person object indexes to be used subject-coreferentially (e.g.} {\textit{je} \textit{me} \textit{vois}} {’I see myself’). This seems to be rare in the world’s languages: Paradigms with subject and object indexes typically have gaps in all the coreferential paradigm slots (cf. \citealt{HampeLehmann2013}).}}

Some authors have claimed, especially for French and other Romance languages, that constructions such as \textit{il} \textit{se} \textit{voit} should be treated as intransitive (e.g. \citealt[\S2.1]{Reinhart2005}; \citealt[27--28]{Creissels2006}), and that French \textit{se} should not be regarded as an object clitic, but as a voice marker. This is based on a number of additional characteristics of the construction that go beyond the simple form paradigm (e.g. their behaviour in verb-subject and causative constructions) and can thus hardly be used in cross-linguistic comparison.\footnote{ {\citet{Doron2009} provide a rich set of arguments against \citeauthor{Reinhart2005}'s claims. Their view, that French} {\textit{se}} {should be analyzed as an “anaphor”, is more in line with the classification chosen here. But it should be kept in mind that I do not treat typological classifications as “analyses”, and that “arguments” which go beyond the definitional properties are not relevant for the classification.}} But it needs to be admitted that the criterion of “occurrence in the same slot” may not always be clearly applicable (e.g. when different object indexes occur in different slots).


\subsection{Other types of reflexive constructions}\label{sec:Haspelmath:5.4}

The great majority of reflexive constructions that have been reported in the literature and that have been called “reflexive” belong to one of the three types seen so far, and the great majority of languages have been reported to have either reflexive nominals or reflexive voice markers or both. But there are other construction types which are attested occasionally. 

The first case is a construction in which it is not the lower-ranked anaphoric form that indicates the coreference, but the subject antecedent. According to \citet[166]{Bowden2001}, Taba has “an invariant reflexive particle \textit{do} which occurs as an attribute of the Actor nominal, and which indicates that the Actor of the verb is coreferential with the Undergoer of the same verb”. This is illustrated in (\ref{ex:Haspelmath:21}).

\ea  Taba (Austronesian; Indonesia)\label{ex:Haspelmath:21}\\
    \ea  
   \gll  {I}  \textbf{{do}}  {n=wet}  {i.}\\
      \textsc{3sg}  \textsc{refl}  \textsc{3sg}=hit  \textsc{3sg}\\
    \glt        ‘He hit himself.’
    \ex 
    \gll {Yak}  \textbf{{do}}  {k=alcoma-k}    {yak}  {surat.}\\
      \textsc{1sg}  \textsc{refl}  {\textsc{1sg}=send=\textsc{appl}}  \textsc{1sg}  {letter}\\
     \glt   ‘I am sending myself a letter.’
     \z 
\z 

If this construction were restricted to personal pronoun subjects, it would be a reflexive pronoun that violates the rank scale generalization (discussed below in \sectref{sec:Haspelmath:7}), but Bowden’s description does not report such a restriction.

  A related construction uses a kind of bipartite reflexive pronoun which bears the flagging of both the antecedent and the position in which the anaphoric pronoun occurs. This has been documented for a number of Dagestanian languages, e.g. Avar.

\ea 
    \label{ex:Haspelmath:22}
    \gll   {ʕali-ca}  \textbf{{žin-ca-go}}  \textbf{{ži-w-go}}  {l”uk”-ana.}\\
      Ali-\textsc{erg}  self-\textsc{erg-emph}  self-\textsc{g1}{}-\textsc{emph}  hurt-\textsc{aor}  \\
    \glt  ‘Ali hurt himself.’ (\citealt{Testelec1998}: 45)
\z


Here the first part of the bipartite reflexive pronoun (\textit{žin-ca-go} \textit{ži-w-go}) bears the ergative case of the antecedent nominal (the subject), and the second part is in the absolutive case, as is appropriate for the role of the pronoun. This can probably be regarded as a bipartite reflexive pronoun, though the case-form of the first part links it closely to the antecedent, and thus makes it look somewhat like the case of Taba mentioned in the preceding paragraph.

In some languages, an adverbial expression (meaning ‘alone’, or ‘again’) that is not closely associated with an argument expression can indicate coreference of the object with the subject. This might be called a \textsc{reflexive} \textsc{adverb}. An example comes from an Austronesian language:

\ea Fagauvea (Polynesian)
    \label{ex:Haspelmath:23}\\
    \gll {E}  \textbf{{hage}}  {matea}  {ie}  {ia}  {a}  {cica.}\\
     \textsc{ipfv}  alone  admire  \textsc{abs}  \textsc{3sg}  \textsc{art}  dad     \\
    \glt‘Dad admires himself.’ (\citealt{Moyse-Faurie2008}: 138)
\z

 Finally, I should briefly mention \textsc{logophoric} \textsc{pronouns}, which indicate coreference between a participant of an embedded clause and the subject (or another prominent participant) of the matrix clause. Consider the contrast in (\ref{ex:Haspelmath:24a}-\ref{ex:Haspelmath:24b}), where coreference is indicated by \textit{inyemeñ}, and disjoint reference by the nonlogophoric pronoun \textit{woñ} \citep[1056]{Culy1994}.

\ea  Donno So (Dogon)
    \label{ex:Haspelmath:24}\\
    \ea \label{ex:Haspelmath:24a}
    \gll {Oumar}  {[Anta}  {inyemeñ}  {waa}  {be]}  {gi.}\\
        Oumar  Anta  \textsc{logophor.acc}  seen  \textsc{aux}  said \\
    \glt ‘Oumar\textsubscript{1} said that Anta\textsubscript{2} had seen him\textsubscript{1}.’
     \ex \label{ex:Haspelmath:24b}
    \gll {Oumar}  {[Anta}  {woñ}  {waa}  {be]}  {gi.}\\
       Oumar  Anta  him.\textsc{acc}  seen  \textsc{aux}  said  \\
    \glt  ‘Oumar\textsubscript{1} said that Anta\textsubscript{2} had seen him\textsubscript{3}.’
\z
\z 
     
Such pronouns are not normally treated as reflexive pronouns, though by the definition that I have given so far, they should be regarded as reflexive pronouns. Perhaps their special treatment in the literature is entirely due to the fact that the research tradition has been focused on West African languages.

\section{Types of reflexive nominals}\label{sec:Haspelmath:6}


The first of the three main types of reflexivizers, reflexive nominals (\sectref{sec:Haspelmath:5.1}), shows a lot of internal diversity, so we can distinguish a number of salient subtypes here.
\subsection{Nouns with adpossessive person forms (= possessive-indexed reflexive nouns)}\label{sec:Haspelmath:6.1}


In many languages, the reflexive nominal looks like a noun that takes adpossessive person forms, so that the literal translation is ‘my self’, ‘your self’, ‘his self’, and so on. These nouns sometimes have plural forms when the antecedent is plural, i.e. ‘our selves’, ‘your selves’, ‘their selves’. Some examples are given in \tabref{tab:Haspelmath:1} (for Hausa, see \citet{Newman2000}; for Chalcatongo Mixtec, see \citet[144--145]{Macaulay1996}; for Finnish, see \citet[137]{Karlsson1999}; for Hebrew, see \citet[67]{Glinert1989}).

\begin{table}
    \centering
    \begin{tabular}{cccccc}
    \lsptoprule
       & Modern Greek  & Hausa  & C. Mixtec & Finnish  & Hebrew          \\
    \hline 
1\textsc{sg} & \textit{ton eaftó mu}  & \textit{kâin-ā} & \textit{máá=rí}    & \textit{itse-ni}  & \textit{ʕacm-i}          \\
2\textsc{sg} & \textit{ton eaftó su}  & \textit{kân-kà} & \textit{máá=ro}    & \textit{itse-si}  & \textit{ʕacm-exa/-ex}    \\
3\textsc{sg} & \textit{ton eaftó tis} & \textit{kân-sà} & \textit{máá=ñá}    & \textit{itse-nsä} & \textit{ʕacm-o/-a}       \\
1\textsc{pl} & \textit{ton eaftó mas} & \textit{kân-mù} &           &\textit{itse-mme} & \textit{ʕacm-enu}        \\
2\textsc{pl} & \textit{ton eaftó sas} & \textit{kân-kù} &           & \textit{itse-nne} & \textit{ʕacm-exem/-exen} \\
3\textsc{pl} & \textit{ton eaftó tus} & \textit{kân-sù} &           & \textit{itse-nsä} & \textit{ʕacm-am/-an}    \\
\lspbottomrule
    \end{tabular}
    \caption{Examples of possessive-indexed reflexive nouns}\label{tab:Haspelmath:1}
\end{table}

In Georgian, the possessive person form is not a bound form (\textit{čemi} \textit{tavi} ‘myself’, \textit{šeni} \textit{tavi} ‘yourself’), and it is not obligatory (\citealt{Amiridze2002}). Perhaps one can say in general that when the possessive person form is a bound form as in \tabref{tab:Haspelmath:1}, it is obligatory, but when it is a free form, it may or may not occur.

  \citet{Faltz1977} calls such noun-like reflexive forms “head reflexives”, because they can be the “head” of a reflexive nominal.\footnote{ {This term is not ideal, for two reasons: (i) reflexive nouns often come from body-part nouns meaning ‘head’ (see \sectref{sec:Haspelmath:11.2} below), so it may be misinterpreted, and (ii) the syntactic notion of “head” is not well-defined (it may often be unclear whether a reflexive-marking form is a “head” or not).}}

\subsection{Noun-like forms without adpossessive indexes}\label{sec:Haspelmath:6.2}

In languages lacking adpossessive person indexes, reflexive nouns are not person-marked. They are noun-like primarily in that they can occur with adpositions and/or case-markers. Examples come from Japanese (\textit{jibun}) and Hindi-Urdu (\textit{apne}) in (25a-b). For the Ute form \textit{nanɵs,} Givón’s description only gives examples of object use, so it is less clearly noun-like (and could be said to resemble the voice prefixes in (15b-c) above). 

\ea%25
    \label{ex:Haspelmath:25}
    \ea Japanese \label{ex:Haspelmath:25a}\\
    \gll {Ken}  {wa}  \textbf{{jibun}}  {o}  {hihanshi-ta.}\\
      Ken  \textsc{top}  self  \textsc{acc}  criticize-\textsc{pst}  \\
    \glt‘Ken criticized himself.’ \citep[380]{Hirose2018}
    \ex Hindi-Urdu \label{ex:Haspelmath:25b}\\
    \gll  {Siitaa\textsubscript{1}}{{}-ne  Raam}{\textsubscript{2}}{{}-ko} [\textbf{{apne}}{\textsubscript{1/2}}{{}-ko  dekh-ne-ke}]  {liye}  {majbuur}  {kiyaa.}\\
     Sita-\textsc{erg}  Ram-\textsc{dat}  self-\textsc{dat}  look-\textsc{inf-gen}  for  force  did\\
    \glt  ‘Sita (f) forced Ram (m) to look at her/himself.’ \citep[47]{Davison2001}
     \ex  Ute (Uto-Aztecan) \label{ex:Haspelmath:25c}\\
    \gll \textbf{{Nan}}{ɵ}\textbf{{s}}  {pʉnikya-qhay-ˈu.}\\
     self  see-\textsc{ant}{}-\textsc{3sg}\\
    \glt ‘She saw herself.’ \citep[237]{Givón2011}
    \z
\z 
         


\subsection{Self-intensified anaphoric pronouns}\label{sec:Haspelmath:6.3}

In some languages, reflexive nominals are etymologically made up of anaphoric pronouns combined with self-intensifiers (i.e. forms that are used like English \textit{himself/herself/themselves}, as in \textit{Is} \textit{the} \textit{queen} \textit{coming} \textit{herself}?).

\ea%26
    \label{ex:Haspelmath:26}
    \ea  Irish\label{ex:Haspelmath:26a}\\
    \gll  {Chonaic}  {na}  {cailíni}  \textbf{{iad}  {féin}}.  \\
      see.\textsc{pst}  the  girls  them  self  \\
    \glt ‘The girls saw themselves.’ \citep[36]{Nolan2000}
    \ex Mandarin Chinese  \label{ex:Haspelmath:26b}\\
    \gll  {Zhangsan}   {ai}  \textbf{{ta-ziji}}. \\
        Zhangsan  love  him-self \\
    \glt‘Zhangsan loves himself.’ \citep[98]{Tang1989}
    \z
\z 

An example from French (\textit{Asma} \textit{parle} \textit{d’elle-même}) was seen earlier in (\ref{ex:Haspelmath:2b}), and an example from Malayalam is seen below in (\ref{ex:Haspelmath:40a}). Self-intensifiers are often closely related to reflexive nominals (\citealt{KoenigSiemund1999}; \citealt{KoenigEtAl2005}), and I will say a little more about them in \sectref{sec:Haspelmath:11.2}.


\subsection{Anaphoric pronouns with other reinforcements}\label{sec:Haspelmath:6.4}

Reflexive nominals may also be made up from anaphoric pronouns combined with other reinforcing elements, e.g.

\ea Tok Pisin  
    \label{ex:Haspelmath:27}\\
    \gll  {Em}  {go}  {na}  {em}  {kilim}  \textbf{{em}  {yet}}.\\
      he  go  and  he  kill  him  \textsc{emph}  \\
    \glt  ‘He went and killed himself.’ (\citealt{SmithSiegel2013})
\z

\ea Kikongo-Kituba  
    \label{ex:Haspelmath:28}\\
    \gll {Bo}      {bula}  \textbf{{bo}  {mosi}}.\\
       they  hit  them  one \\
    \glt ‘They hit themselves.’ \citep{Mufwene2013}
\z

    
\ea%29
    Fijian\label{ex:Haspelmath:29}\\
    \gll {O}  {Josese}  {ā}  {digi-taki}  {koya}  {gā.}\\
      \textsc{det}  Josese  \textsc{pst}  choose-\textsc{tr}  him  \textsc{emph}   \\
    \glt‘Josese voted for himself.’ \citep[775]{Park2013}
\z


\subsection{Reflexive pronominoids}\label{sec:Haspelmath:6.5}
In some languages, reflexive nominals are similar to independent personal pronouns in that they not only lack noun-specific features like articles and adpossessive person indexes, but also share idiosyncratic properties of personal pronouns. This is clearest in western Indo-European languages such as Slavic and Germanic. \tabref{tab:Haspelmath:2} shows a personal pronoun (‘you.\textsc{sg’}) and the reflexive pronoun in both Polish and Icelandic.

\begin{table}
\begin{tabular}{ccccc}
\lsptoprule
    & Polish ‘you’ & Polish ‘self’ & Icelandic ‘you’ & Icelandic ‘self’ \\
    \hline 
\textsc{nom} & \textit{ty}           & –             & \textit{þú}              & -                \\
\textsc{gen} &\textit{ciebie}       & \textit{siebie}        & \textit{þín}             & \textit{sín}              \\
\textsc{dat} & \textit{tobie}        & \textit{sobie}         & \textit{þér}             & \textit{sér}              \\
\textsc{acc}& \textit{ciebie}       & \textit{siebie}        & \textit{þig}             & \textit{sig}       \\
\lspbottomrule
\end{tabular}
\caption{Examples of personal pronouns and reflexive pronominoids}\label{tab:Haspelmath:2}
\end{table}

The inflectional patterns are so similar that there is no question that the reflexive pronouns belong to the same paradigm as the personal pronouns. But it should be noted that such \textsc{reflexive} \textsc{pronominoids} are apparently quite rare in the world’s languages.\footnote{ {I deliberately introduce the strange term} {\textit{reflexive} \textit{pronominoid}} {here in order to highlight the fact that such forms are unusual, even though they are very familiar to many linguists from European languages (Latin also has such pronominoids). Using the term} {\textit{pronoun}} {for the unusual forms in contrast to} {\textit{noun}} {for the forms in \sectref{sec:Haspelmath:6.2} would not have the same effect. (From \sectref{sec:Haspelmath:7} onwards, I will us the term} {\textit{reflexive} \textit{pronoun}} {for any kind of reflexive nominal, because this term is more familiar from the literature.)}}

  Another language which has reflexive pronominoids, in a much richer way, is Ingush (a Nakh-Dagestanian language of Russia; \citealt[\S9.1]{Nichols2011}). A small part of the paradigm is listed in \tabref{tab:Haspelmath:3}.


\begin{table}
    \centering
    \begin{tabular}{ccccccc}
     \lsptoprule
      & \textsc{1sg} & \textsc{1sg.refl} & \textsc{2sg}  \textsc{2sg.refl} & \textsc{3pl} & &\textsc{3pl.refl}\\
\hline 
\textsc{nom} &   \textit{so}  &\textit{sie}  \textit{hwo} & \textit{hwie} & \textit{yzh} & \textit{shoazh}  \\
\textsc{gen}   &   \textit{sy} & \textit{sei} & \textit{hwa}&  \textit{hwaai} & \textit{caar} & \textit{shoi}\\
\textsc{dat} &     \textit{suona}&  \textit{seina} & \textit{hwuona} & \textit{hwaaina}  &\textit{caana} & \textit{shoazhta}\\
\textsc{erg}    &  \textit{aaz}&  \textit{eisa} & \textit{wa} & \textit{waaixa}&  \textit{caar} & \textit{shoazh}\\
  \lspbottomrule
    \end{tabular}
     \caption{Personal pronouns and reflexive pronominoids in Ingush}\label{tab:Haspelmath:3}
\end{table}

       


\section{The rank of antecedent and reflexive pronoun}\label{sec:Haspelmath:7}

In this and the next few sections, we will consider syntactic conditions under which reflexive pronouns can be used, as well as some technical terms that are associated with these conditions.

  According to the definition given in (\ref{ex:Haspelmath:1}), a reflexive pronoun must occur in the same clause as its antecedent.\footnote{ {Note that this is not the same as saying that the antecedent and the reflexive pronoun must be clausemates: Clausemates are elements occurring in the same minimal clause, but a reflexive pronoun need not be a clausemate (see \sectref{sec:Haspelmath:9}). I could have said} {\textit{sentence}} {instead of} {\textit{clause}} {here, but the difference does not matter here (a sentence is a maximal clause, and maximality is irrelevant in the present context).}} However, there is generally an additional syntactic restriction: The antecedent must be a subject of the same clause or of a superordinate clause. Thus, (\ref{ex:Haspelmath:30a}) with a subject antecedent is possible, while (\ref{ex:Haspelmath:30b}) is not possible.

\ea%30
    \label{ex:Haspelmath:30}
    \ea  \label{ex:Haspelmath:30a}
   {My} {friend} {praised} \textbf{{herself}}.
    \ex  \label{ex:Haspelmath:30b}
     {*}\textbf{{Herself}} {praised} {my} {friend.}
    \z
\z 

And in (\ref{ex:Haspelmath:31}), the adpossessive reflexive pronoun must be coreferential with the subject, not with the dative object. 

\ea%31
   Russian     \label{ex:Haspelmath:31}\\
    \gll  {Ona\textsubscript{1}}  {dala}  {bratu\textsubscript{2}}  \textbf{{svoj}}{\textsubscript{1/*2}}  {zont.}\\
      she  gave  brother.\textsc{dat}  self’s  umbrella  \\
    \glt ‘She\textsubscript{1} gave her\textsubscript{1} (NOT: his\textsubscript{2}) umbrella to her brother\textsubscript{2}.’
    \z

In some languages (such as English), the conditions are less strict, in that it is also possible for the antecedent may additionally be the object and the reflexive pronoun an oblique argument, as illustrated in (\ref{ex:Haspelmath:32a}). But the opposite is impossible, as seen in (\ref{ex:Haspelmath:32b}).

\ea%32
    \label{ex:Haspelmath:32}
    \ea \label{ex:Haspelmath:32a}
    {Jane} {told} {James} {about} {himself.}
    \ex  \label{ex:Haspelmath:32b}
     {*Jane} {told} {himself} {about} {James.}
    \z
\z 

To describe the difference between Russian \textit{svoj} and English \textit{himself}, we say that svoj is \textsc{subject-oriented}, while \textit{himself} does not show this restriction. (Actually, there should be a special term for reflexive pronouns like \textit{himself}, because most reflexive pronouns seem to be subject-oriented, and the English case is apparently less usual.)

In some languages, the antecedent may be in the matrix clause and the reflexive pronoun in the embedded clause, as illustrated by (\ref{ex:Haspelmath:33}). (More such examples will be seen in \sectref{sec:Haspelmath:9} below.)

\ea%33
   Japanese \label{ex:Haspelmath:33}\\
    \gll  {Taroo-wa\textsubscript{1}}  [{Hanako-ga}  \textbf{{zibun-ni}}{\textsubscript{1}}  {kasi-te}  {kure-ta}]  {okane-o}  {tukat-te}  {simat-ta.}\\
      Taro-\textsc{top}  Hanako-\textsc{nom}  self-\textsc{dat}  lend-\textsc{cvb}  give-\textsc{pst}  money-\textsc{acc}  spend-\textsc{cvb}  end.up-\textsc{pst}   \\
    \glt ‘Taro has spent all the money that Hanako had lent him.’ (\citealt[635]{Kuno1977})
\z


Again, the reverse situation (with the reflexive \textit{zibun} in the matrix clause and the antecedent in the embedded clause) would not be possible here.

  While there is no systematic cross-linguistic research, it appears from the rich literature on many different languages that given the rank scale in (\ref{ex:Haspelmath:34}),\footnote{ {A scale of this kind was proposed by \citet[266]{PollardSag1992}, but they only discuss English. Other authors that have proposed similar rank scales are \citet[212]{Bresnan2001} and \citet[\S7.5]{Vanvalin1997}, and yet others have proposed to explain the restrictions in terms of a semantic role scale (\citealt{Jackendoff1972}: Ch. 4) or a in terms of a case scale (\citealt{Kiss1991}). None of these language-particular proposals are incompatible with the cross-linguistic claim of (\ref{ex:Haspelmath:33}).}} almost all languages restrict the relation between the antecedent and the reflexive pronoun in such a way that (\ref{ex:Haspelmath:35}) is observed. 

\ea%34
   rank scale of syntactic positions \label{ex:Haspelmath:34}\\
   subject > object > oblique > within nominal, within embedded clause
\z

\ea%35
    antecedent–reflexive asymmetry\label{ex:Haspelmath:35}\\
   The antecedent must be higher on the rank scale of syntactic positions than the reflexive pronoun.
\z

Note that this additional restriction is not definitional, but is an empirical generalization. The reason we can be fairly confident that (\ref{ex:Haspelmath:35}) is true is that a violation of (\ref{ex:Haspelmath:35}) would be very salient, and linguists would have discussed such cases more often. \citet{Forker2014} discusses a number of cases that have been mentioned in the literature, but she does not find many clear instances where the reflexive pronoun is unexpectedly in subject position. An exception to (\ref{ex:Haspelmath:35}) is found in Georgian, as illustrated in (\ref{ex:Haspelmath:36}).

\ea%36
    \label{ex:Haspelmath:36}
    \gll  \textbf{{šen-ma}}  \textbf{{tav-ma}}  {gac′ama}    {(šen).} \\
     {your-\textsc{erg}}  {head-\textsc{erg}}   {he.tormented.you}  {you.\textsc{nom}} \\
    \glt‘It was yourself that tormented you.’ \citep{Amiridze2003}
\z

In most languages, the occurrence of reflexive pronouns is actually still more restricted than is implied by (\ref{ex:Haspelmath:34}) and (\ref{ex:Haspelmath:35}), though the various language-particular regularities are difficult to generalize over, and nobody has tried to compare all the languages studied so far in a comprehensive way. Since \citet{Chomsky1981} and \citet{Reinhart1983a}, it has often been thought that a notion of “c-command” is necessary to describe the occurrence of reflexive pronouns (and nonreflexive anaphoric pronouns) in English, and it has been assumed without much argument that such a notion is universally applicable. However, even for English, c-command fails in many cases (e.g. \citealt{BarssLasnik1986}; \citealt{Pesetsky1987}; \citealt{Bruening2014}), and many of the proposals in the literature are highly speculative.\footnote{ {Many authors have proposed modifications of the constituent structure in order to accommodate recalcitrant cases, e.g. \citet[81]{Reinhart1983b}, \citet{Pesetsky1987}, and, most blatantly, \citet{Larson1988} (as discussed and criticized by \citealt[\S2.1.3]{CulicoverJackendoff2005}}} The general usefulness of “c-command”, while widely assumed by authors working in the Reinhart-Chomsky tradition, is therefore far from established knowledge, and even for particular languages, descriptions in terms of rank scales may be preferable (see n. 16).

In addition to the contrast between subjects, objects and obliques in (\ref{ex:Haspelmath:34}), many languages also allow experiencers which are objects or obliques to be antecedents of reflexive pronouns, as illustrated in (\ref{ex:Haspelmath:37}) from Italian (\citealt{BellettiRizzi1988}).

\ea%37
    \label{ex:Haspelmath:37}
    \gll  {Questi}  {pettegolezzi}  {su}  {di}  {sé}  {preoccupano}  {Gianni} {più}  {di}  {ogni}  {altra}  {cosa.} \\
       these  rumours  about  of  himself  worry  Gianni more  than  any  other  thing \\
    \glt‘These rumours about himself worry Gianni more than anything else.’
\z

  

This is also possible in English to some extent (\citealt[81]{Reinhart1983}; \citealt[127]{Pesetsky1987}), and in many other languages. These cases show that the rank scale in (\ref{ex:Haspelmath:34}) (let alone a notion of c-command) is not sufficient to account for the distribution of reflexive pronouns.


\section{Domains: Autopathic, oblique and adpossessive reflexive constructions}\label{sec:Haspelmath:8}
When the form that marks the reflexive construction is a reflexive pronoun, there are often interesting variations with respect to the \textsc{antecedent} \textsc{domain} (often called “binding domain”), i.e. the “syntactic distance” between the antecedent and the reflexive nominal. In this section, I distinguish between an autopathic domain, an oblique domain, and an adpossessive domain, because these are the most important distinctions. In the next section (\sectref{sec:Haspelmath:9}), we will see domains going beyond the minmal clause.

The \textsc{autopathic} \textsc{domain} is the relation between the subject and the object (or the A-argument and the P-argument) in a monotransitive clause, as in \textit{She} \textit{saw} \textit{herself;} \textit{He} \textit{painted} \textit{himself;} \textit{They} \textit{hit} \textit{themselves}. This is Faltz’s (1977: 3) “archetypal” reflexive context, Kemmer’s (1993: 41) “direct reflexive” situatio, and it describes what \citet{Reinhart1993} call “reflexive predicates”. We need the new term \textit{autopathic} for this domain, because the term \textit{reflexive} is generally used in the wider sense of \sectref{sec:Haspelmath:1}, and because this domain is so important that it deserves its own label.\footnote{ {The Greek term for ’reflexive’ is} {\textit{autopathēs}}{, deriving from} {\textit{auto-}} {’self, same’ and} {\textit{path-}} {’patient’ (i.e. literally it means ’domain in which the patient is the same’). The term} {\textit{autopathic}} {in this sense is thus very transparent etymologically.}} As far as I am aware, reflexive voice markers in the world’s languages are primarily used in the autopathic domain. Moreover, many languages have a short reflexive pronoun and a long reflexive pronoun, and not uncommonly, the longer pronoun is required (or preferred) in the autopathic domain. This is apparently due to the fact that coreference is particularly unlikely in this domain, at least with extroverted verbs \citep{Koenig2004, Haspelmath2008}.

 The \textsc{oblique} \textsc{domain} refers to the relation between the subject and an oblique-marked participant of the same minimal clause. In this domain, some languages can use a nonreflexive pronoun, e.g. French and English.

\ea%38
    French\label{ex:Haspelmath:38}\\
    {Pierre} {est} {fier} {de} \textbf{{lui}}.\\
   ‘Pierre is proud of him /of himself.’
\z

\ea%39
    English\label{ex:Haspelmath:39}\\
   \ea \label{ex:Haspelmath:39a}{Jane} {saw} {a} {snake} {near} \textbf{{her}} {/} {near} \textbf{{herself}}.
    \ex \label{ex:Haspelmath:39b} {John} {left} {his} {family} {behind} {him} {(/*himself).} \citep[43]{Kiparsky2002}
    \z
\z 


The precise conditions vary (in a complicated way, cf. \citealt{Zribi1995} for French), but the fact that the anaphoric position is an oblique argument (rather than a direct object, or P-argument) seems to play an important role in a number of languages.\footnote{ {A related notion is that of} {\textsc{coargument} \textsc{domain}} {\citep{Kiparsky2002}, which includes P-arguments and oblique arguments, but not modifying participants.}}   Another language that is similar to French and English is Malayalam \citep{Jayaseelan2000}, where the simple reflexive pronoun \textit{taan/tann-} cannot be used in an autopathic situation (which requires the complex form \textit{awan-} \textit{tanne}), but can be used when the reflexive is in an oblique position:

\ea%40
   Malayalam (\citealt[121,126]{Jayaseelan2000})\\ \label{ex:Haspelmath:40}
    \ea \label{ex:Haspelmath:40a}
    \gll  {Raaman}  \textbf{{awan-e}} \textbf{{tanne}} {aṭiccu.}    ({*Raaman} \textbf{{tann-e}} {aṭiccu}.)\\
      Raman  he-\textsc{acc}  self  hit  \\
    \glt ‘Raman hit himself.’
    \ex \label{ex:Haspelmath:40b}
    \gll {Raaman}  \textbf{{tan}}{-te}  {munn-il}  {oru}  {aana-ye}  {kaṇḍu.}\\
    Raman  self-\textsc{gen}  front-\textsc{loc}  one  elephant-\textsc{acc}  saw\\
    \glt ‘Raman saw an elephant in front of him(self).’
    \z
\z 

And in Homeric Greek, a complex reflexive pronoun \textit{hé-} \textit{+} \textit{autó-} must be used in the autopathic domain (\ref{ex:Haspelmath:41a}), while the oblique domain allows the bare reflexive \textit{hé-} (\ref{ex:Haspelmath:41b}) \citep[86-87]{Kiparsky2012}.

\ea%41
    Homeric Greek\label{ex:Haspelmath:41}
    \ea \label{ex:Haspelmath:41a}
    \gll {Heè}  {d’}  {autò-n}  {epotrún-ei}  {makésa-sthai.}\\
       \textsc{refl}.\textsc{acc}  \textsc{prt}  self-\textsc{acc}  rouse.\textsc{3sg}  fight.\textsc{aor}{}-\textsc{inf}  \\
    \glt‘And he rouses himself to fight.’ ({Il.} 20.171)
    \ex \label{ex:Haspelmath:41b}
    \gll  {Aspíd-a}  {taureíē-n}  {skhéth’}  {apò}  {héo.}\\
       shield-\textsc{acc}  bull.hided-\textsc{acc}  held.\textsc{3sg}  from  \textsc{refl}.\textsc{gen} \\
    \glt ‘He held the shield of bull hide away from him(self).’
    \z
\z 
          
An important further domain that is less often discussed is the \textsc{adpossessive} \textsc{domain}, where the coreferential anaphoric form is the adnominal possessor (= adpossessor) of the object or some other nonsubject participant. The West Germanic and Romance languages use nonreflexive possessive forms in this domain, which can be used subject-coreferentially or with disjoint reference (English \textit{She\textsubscript{1}} \textit{forgot} \textit{her\textsubscript{1/2} }\textit{umbrella}, French \textit{Elle\textsubscript{1}} \textit{a} \textit{oublié} \textit{son\textsubscript{1/2} }\textit{parapluie}). By contrast, many other languages make an obligatory distinction between subject-coreferential and subject-disjoint adpossessive pronouns. Examples come from Polish (\ref{ex:Haspelmath:42}) and Evenki (\ref{ex:Haspelmath:43}).

\ea%42
    Polish \label{ex:Haspelmath:42}\\
    \ea \label{ex:Haspelmath:42a}
    \gll {Ona\textsubscript{1}}   {jest}  {w}  \textbf{{swoim}}{\textsubscript{1}} {pokoju.}\\
       she  is  in  self’s    room.  \\
    \glt  ‘She is in her (own) room.’
     \ex \label{ex:Haspelmath:42b}
    \gll {Ona\textsubscript{1}}  {jest}  {w}  {jej\textsubscript{2}}  {pokoju.}\\
       she  is  in  her  room  \\
    \glt  ‘She is in her room (= another person’s room).’
    \z
\z 
         
   

While Polish has an independent reflexive possessive pronoun (\ref{ex:Haspelmath:42a}) contrasting with an independent nonreflexive one (\ref{ex:Haspelmath:42b}), Evenki has possessive person indexes (= bound person forms), both reflexive (\ref{ex:Haspelmath:43a}) and nonreflexive (\ref{ex:Haspelmath:43b}).

\ea%43
    Evenki \citep[103]{Nedjalkov1997}\label{ex:Haspelmath:43}\\
    \ea \label{ex:Haspelmath:43a}
    \gll {Nungan\textsubscript{1}}  {asi-}\textbf{{vi}}{\textsubscript{1}}  {iche-re-n.}\\
        he  wife-\textsc{refl.poss}  see-\textsc{nfut-3sg} \\
    \glt‘He saw his (own) wife.’
    \ex \label{ex:Haspelmath:43b}
    \gll {Nungan\textsubscript{1}}  {asi-va-n\textsubscript{2}}  {iche-re-n.}\\
       he  wife-\textsc{acc-3sg.poss}  see-\textsc{nfut-3sg} \\
    \glt‘He saw his wife (= another person’s wife).’
    \z
\z 
          

\section{Domains: Clausemate and long-distance reflexive constructions}\label{sec:Haspelmath:9}

From the point of view of a language like German, where the reflexive pronoun \textit{sich} must have a \textsc{clausemate} antecedent (i.e. the antecedent must be an argument of the same minimal clause, or coargument), the most surprising phenomenon is the existence of \textsc{long-distance} \textsc{reflexive} \textsc{pronouns} (generally shortened to \textit{long-distance} \textit{reflexives}, because there are no long-distance voice markers). A long-distance reflexive is a reflexive pronoun that can occur in a subordinate clause and take its antecedent in the matrix clause, as in (\ref{ex:Haspelmath:44a}-\ref{ex:Haspelmath:44c}).\footnote{ {Note that the opposite, a reflexive pronoun in the matrix clause and its antecedent in the subordinate clause, is excluded by antecedent-reflexive asymmetry in (\ref{ex:Haspelmath:33}).}} (We already saw an example from Japanese in (\ref{ex:Haspelmath:33}) above.)

\ea%44
    \label{ex:Haspelmath:44}
    \ea  Italian \label{ex:Haspelmath:44a} \\
    \gll {Gianni\textsubscript{1}}  {pensava}  [{che}  {quella}  {casa}    {appartenesse}  {ancora}  {alla}  \textbf{{propria}}{\textsubscript{1} } {famiglia}].\\
       Gianni  thought  that  that  house  belonged  still    to  self’s    family  \\
    \glt  ‘Gianni thought that that house still belonged to his (own) family.’  \citep[314]{Giorgi1984}
    \ex    Mandarin Chinese \label{ex:Haspelmath:44b}\\
    \gll {Zhangsan\textsubscript{1}}  {renwei}  [{Lisi\textsubscript{2}}  {zhidao}  [{Wangwu\textsubscript{3}}  {xihuan}  \textbf{{ziji}}{\textsubscript{1/2/3}}]].\\
        Zhangsan  think  Lisi  know  Wangwu  like  self \\
    \glt ‘Zhangsan thinks that Lisi knows that Wangwu likes him.’ (\citealt[22]{Cole2006})
    \ex  Ingush\\
    \gll {Aaz}  \textbf{{shiiga}}{\textsubscript{1}}  {telefon}  {tiexacha,}  {Muusaa\textsubscript{1}}  {chy-vaxar.}\\
       \textsc{1sg.erg}  \textsc{3sg.refl.all}  phone  do.\textsc{cvb}  Musa  in-go.\textsc{pst} \\
    \glt‘When I phoned him\textsubscript{1} (lit. ‘himself’), Musa\textsubscript{1} went home.’ \citep[645]{Nichols2011}
    \ex  Avar \label{ex:Haspelmath:44c}\\
    \gll {Maħmud}  {bož-ula}  [\textbf{{žiw}}  {łik’aw}  {či}  {w-uk’-inal-da}].\\
      Mahmud  believe-\textsc{prs}  self.\textsc{m}  good.\textsc{m}  man  \textsc{m}{}-be-\textsc{msd}{}-\textsc{loc}  \\
    \glt ‘Mahmud\textsubscript{1} believes that he\textsubscript{1/*2} is a good man.’ \citep[155]{Rudnev2017} 
    \z
\z


We can call this the \textsc{long-distance} \textsc{domain}, contrasting it with the \textsc{clausemate} \textsc{domain,} where the antecedent must be an argument of the same minimal clause.\footnote{ {The clausemate domain is often simply called “local domain“ (even though locality is generally a relative notion), or sometimes “clause-bound(ed)“ (e.g. \citealt[393]{Vanvalin1997}). It should also be noted that the term} {\textit{clause}} {is very different from ’minimal clause’, because a clause is generally taken to include all of its subordinate clauses. This is why the definition in (\ref{ex:Haspelmath:1}) talks about clauses, not sentences (though the latter would not have been wrong, because a sentence is generally understood as a maximal clause, and the difference between clauses and sentences is irrelevant in the context of (\ref{ex:Haspelmath:1})).} }

In some languages, especially Indo-Europan languages of Europe, long-distance-reflexives are limited to infinitival clauses. This is the case, for example, in Polish, where the counterparts of (\ref{ex:Haspelmath:44a}-\ref{ex:Haspelmath:44c}) would not be possible, but in (\ref{ex:Haspelmath:45}), the reflexive pronoun \textit{siebie} can be coreferential with the matrix subject (or alternatively with the understood infinitival subject). Likewise in Avar, the reflexive pronoun \textit{žiw-go} can only be used in the clausemate domain and the non-finite long-distance domain, while in finite subordinate clauses, the form \textit{žiw} must be used (\citealt[\S2.1]{Rudnev2017}).

\ea%45
    \label{ex:Haspelmath:45}
    \ea Polish \label{ex:Haspelmath:45a}\\
    \gll {Renata\textsubscript{1}}   {kazała}  {Piotrowi\textsubscript{2}}  [{zbudować}  {dom}  {dla}  \textbf{{siebie}}{\textsubscript{1/2}}].\\
        Renata.\textsc{nom}  ordered  Piotr.\textsc{dat}  build.\textsc{inf}  house.\textsc{acc}  for  self.\textsc{gen}\\
    \glt‘Renata ordered Piotr to build a house for her (OR: for himself).' \citep[195]{Siewierska2004}
    \ex  Avar \label{ex:Haspelmath:45b}\\
    \gll {Ebelal-da\textsubscript{1}}  {b-ix-ana}    [{Malik-ica\textsubscript{2}}  \textbf{{žindie-go}}{\textsubscript{1/2}}  {ruq‘}  {b-ale-b}].\\
    mother-\textsc{loc}  \textsc{n}{}-see-\textsc{pst}  Malik-\textsc{erg}  self.\textsc{dat}{}-\textsc{emph}  house  \textsc{n}{}-build-\textsc{n}\\
    \glt‘Mother saw Malik building a house for her (OR: for himself).’ \citep[159]{Rudnev2017}
    \z
 \z 

     
      

Perhaps one could distinguish different subdomains within the long-distance domain, but “finite” vs. “nonfinite” \citep{Kiparsky2002} does not work, because there is no cross-linguistically applicable definition of “(non)finite”.


\section{Obviative and nonobviative anaphoric pronouns}\label{sec:Haspelmath:10}

In many (or perhaps most) languages, nonsubject anaphoric personal pronouns are \textsc{obviative} (\citealt{Kiparsky2002}; 2012), i.e. they cannot be coreferential with a coargument. This is illustrated in (46a-b).

\ea%46
    \label{ex:Haspelmath:46}
    \ea English\\
     {The} {dogs\textsubscript{1}} {bit} {them\textsubscript{2/*1}}.
    \ex Mandarin Chinese\\
    \gll {Mali\textsubscript{1}}  {hai-le}  {ta\textsubscript{2/*1}}.\\
      Mali  hurt-\textsc{pfv}  her\\
    \glt‘Mali hurt her (*herself).’ (\citealt{Cole2015}: 142)
    \z
\z   

     

As noted earlier (\sectref{sec:Haspelmath:4}), many languages (such as English and Mandarin) must use reflexive pronouns rather than (nonreflexive) personal pronouns when coreference is intended (\textit{themselves}, \textit{ta-ziji}). This complementarity of personal pronouns and reflexive pronouns has often been noted and has been taken as a starting point for larger explanatory claims, but it is useful to have a separate term for anaphoric forms that cannot be used coreferentially with the subject.

In some languages, the use of reflexive pronouns is optional. This has been reported, for example, for Hausa:

\ea%47
    Hausa \citep[524]{Newman2000} \label{ex:Haspelmath:47}
    \ea  \label{ex:Haspelmath:47a}
    \gll {Tàlá}  {táa}  {gán}  {tà}  {à}  {màdùubîn.}\\
         Tala  \textsc{3sg.pst}  see  her  in  mirror \\
    \glt ‘Tala saw her/herself in the mirror.’
    \ex 
    \gll  {Tàlá}  {táa}  {gá}  {kântà}  {à}  {màdùubîn.}\\
     Tala  \textsc{3sg.pst}  see  herself  in  mirror\\
     \glt ‘Tala saw herself in the mirror.’
    \z
\z 

Thus, Hausa \textit{tà} is not obviative, unlike English \textit{her}, even though it is a nonreflexive pronoun, like English \textit{her}. The complementarity between nonreflexive and reflexive pronouns that we see in English textbook examples is by no means necessary (and it is not complete in Engish either, as seen in (\ref{ex:Haspelmath:39a})). Another interesting case is Turkish, which has three types of third person anaphoric pronouns: an obviative nonreflexive pronoun \textit{on-}, a nonobviative nonreflexive pronun \textit{kendisi}, and a reflexive pronoun \textit{kendi} \citep{Kornfilt2001}:

\ea%48
    \label{ex:Haspelmath:48}
    \gll  {Ahmet\textsubscript{1}}  {onu\textsubscript{2}/} {kendini\textsubscript{1/2}/} {kendini\textsubscript{1}}  {çok}  {beğeniyormuş.}\\
      Ahmet  him      him(self)    himself  much  admires  \\
    \glt‘Ahmet admires him / him(self) / himself very much.’ \citep[200]{Kornfilt2001}
\z


Like Hausa, Turkish shows no complementary distribution of reflexive and obviative anaphoric pronouns, and it is clear that it must be specified that \textit{on-} is obviative (i.e. that this cannot be derived from a general principle).

In addition to nonreflexive pronouns like \textit{him/her/them} that are obviative, some languages also have reflexive pronouns that are obviative. Examples are Swedish \textit{sig} and Malayalam \textit{taan}, which are long-distance reflexives but cannot be coreferential in the autopathic domain, as illustrated by (\ref{ex:Haspelmath:49a}-\ref{ex:Haspelmath:49b}).

\ea%49
    \label{ex:Haspelmath:49}
    \ea  Swedish\label{ex:Haspelmath:49a}\\
    \gll {Generalen\textsubscript{1}}  {tvingade}  {översten\textsubscript{2}}  {att}  {hjälpa}  \textbf{{sig}}{\textsubscript{1/*2}}.\\
       the.general  forced  the.colonel  to  help  \textsc{refl} \\
    \glt‘The general\textsubscript{1} forced the colonel\textsubscript{2} to help him\textsubscript{1}.’ (\citealt[26]{Kiparsky2002})
    \ex Malayalam \label{ex:Haspelmath:49b}(cf. \ref{ex:Haspelmath:40a})\\
    \gll {Raaman\textsubscript{1}}  {wicaariccu}  [{Siita\textsubscript{2}}  \textbf{{tann-e}}{\textsubscript{1/*2}}  {kaṇḍu}  {ennə}].\\
     Raman  thought  Sita  self-\textsc{acc}  saw  \textsc{comp}\\
     \glt  ‘Raman thought that Sita saw him.’ (NOT: ‘... Sita saw herself’)
      \citep[129]{Jayaseelan2000} 
    \z
\z 

It is very common for nonreflexive personal pronouns to be obviative (and demonstrative-derived anaphoric pronouns are apparently always obviative), but as we saw in (\ref{ex:Haspelmath:8}-\ref{ex:Haspelmath:10}) in \sectref{sec:Haspelmath:4}, some languages have anaphoric pronouns which are not obviative.


\section{Coexpression patterns of reflexivizers}\label{sec:Haspelmath:11}
The next topic to be covered briefly here is coexpression patterns, i.e. the use of a single form in a language for several meanings or functions that other languages distinguish. Such patterns are often described in terms of “polysemy”, but the term {coexpression} is more neutral in that it does not entail that the form actually has multiple (related) meanings in a language.


\subsection{Reflexive voice markers}\label{sec:Haspelmath:11.1}
It has been well-known at least since \citet{Faltz1977},  \citet[ch1]{Geniusiene1987} and \citet{Kemmer1993} that across languages, reflexive voice markers often have other uses, in addition to the reflexive meaning, and that the different meanings tend to recur. \citet[917]{Kazenin2001} notes that such markers are “normally polysemous”, and it is indeed hard to find a reflexive voice marker that has no nonreflexive uses. For example, reflexive voice markers commonly have \textsc{reciprocal} uses, as in (\ref{ex:Haspelmath:50}).

\ea%50
   Kuuk Thaayorre (Pama-Nyungan; \citealt{Gaby2008}: 260) \label{ex:Haspelmath:50}\\
    \ea  
    \gll {Ngay}  {nhaanhath-}\textbf{{e}}.\\
        \textsc{1sg.nom}  watch-\textsc{refl}\\
    \glt  ‘I am looking at myself.’
    \ex 
    \gll {Pul}  {runc-}\textbf{{e}}{{}-r.}\\
      \textsc{2du.nom}  collide-\textsc{refl-pst}\\
      \glt ‘They two collided with one another.’
    \z
\z 

When the verb denotes an action that is usually performed on inanimate objects, the reflexive voice marker often has an \textsc{anticausative} use, as in (\ref{ex:Haspelmath:51}).

\ea%51
  Polish \citep{Janic2020}  \label{ex:Haspelmath:51}\\
    \ea 
    \gll  {Gotuję}  {wodę.}\\
      boil.\textsc{1sg}  water.\textsc{acc}   \\
    \glt  ‘I am boiling water.’
    \ex 
    \gll  {Woda}  {gotuje}  \textbf{{się}}  {bardzo}  {szybko.}\\
      water  boil.\textsc{3sg}  \textsc{refl}  very  quickly  \\
    \glt ‘The water boils very quickly.’
    \z
\z 
         
Other meanings that are sometimes coexpressed with reflexive voice markers are nontranslational motion middles (e.g. German \textit{sich} \textit{umdrehen} ‘to turn around (intr.)’), passives (e.g. Russian \textit{opisyvat’s-sja} ‘be described’), and antipassives (e.g. French \textit{se} \textit{saisir} \textit{de} ‘seize’; \citealt{Janic2016}: 192).


\subsection{Reflexive pronouns}\label{sec:Haspelmath:11.2}
Reflexive pronouns are often identical to nouns with meanings such as ‘body’ or ‘head’, evidently because they originate in a metonymy process. \citet{Schladt1999} studied reflexive pronouns in 150 languages worldwide and found that over half of them have reflexive pronouns derived from body-part terms. In many languages, these behave like nouns in a variety of ways, which is evidently due to their relatively recent origin in nouns. 

  More intriguing is the fact that reflexive pronouns are very often identical or closely related to self-intensifiers (as in \textit{The} \textit{queen} \textit{came} \textit{herself}). In their sample of 168 languages, \citet{KoenigEtAl2005} found 94 languages with identity of reflexive pronouns and self-intensifiers, and 74 languages where the two are different forms. \citet{KoenigSiemund1999} and \citet{KoenigGast2006} propose an explanation for this overlap, by noting that the meanings of self-intensifiers are similar to the meanings of reflexive pronouns, and they can thus explain that reflexive pronouns typically derive from (or are made up of) self-intensifiers. However, \citet{GastSiemund2006} also note that the direction of change is sometimes the opposite, with reflexive pronoun uses preceding intensifier uses.

\section{Coreference constructions that are not reflexive constructions}\label{sec:Haspelmath:12}
Grammatical systems often specify coreference in constructions that are never called reflexive constructions. Two examples were already given in \sectref{sec:Haspelmath:2} above. This section gives a few more illustrations.

  In some languages, a construction with an anaphoric adpossessor modifying the object is necessarily interpreted as coreferential with the subject. For Finnish, this is reported by van \citet[232]{Steenbergen1991}, for Halkomelem, by \citealt[\S6]{DéchaineWiltschko2017a}), and for Chol, by \citet[53--54]{Coon2011}. The constructions (a) examples below entail coreference between the subject and the object adpossessor.

\ea%52
    Finnish\label{ex:Haspelmath:52}\\
   \ea 
    \gll {Pekka\textsubscript{1}}  {luki}  {kirjaa-nsa\textsubscript{1}}.\\
     Pekka  read  book-\textsc{3sg.poss}   \\
    \glt‘Pekka read his (own) book.’
    \ex {Pekka\textsubscript{1}}  {luki}  {hän-en\textsubscript{2}}  {kirjaa-nsa\textsubscript{2}}.\\
    Pekka  read  he-\textsc{gen}  book-\textsc{3sg.poss}\\
    \glt ‘Pekka read his book (i.e. another person’s book).’
    \z
\z 

\ea%53
   Halkomelem (Salishan) \label{ex:Haspelmath:53}\\
    \ea 
    \gll {Th’exw-xál-em}  {te}  {Strang.}\\
      wash-foot-\textsc{intr}  \textsc{det}  Strang  \\
    \glt’Strang washed his (own) feet.’
    \ex 
    \gll  {Th’exw-t-es}  {te}  {Strang}  {te}  {sxele-s.}\\
      wash-\textsc{tr-3sg}  \textsc{det}  Strang  \textsc{det}  foot-\textsc{3.poss}\\
      \glt ’Strang\textsubscript{1} washed his\textsubscript{1}/his\textsubscript{2} feet.’
    \z
\z 

\ea%54
    Chol (Mayan)\label{ex:Haspelmath:54}\\
    \ea 
    \gll {Tyi}  {i-boño}  {y\textsubscript{1}-otyoty}  jiñi  wiñik {\textsubscript{1}}.\\
      \textsc{pfv}  3.\textsc{erg}{}-paint  3.\textsc{poss}{}-house  \textsc{det}  man   \\
    \glt‘The man painted his (own) house.’
    \ex 
    \gll {Tyi}  {i-boñ-be}  {y\textsubscript{2}-otyoty}  jiñi  wiñik {\textsubscript{1}}.\\
      \textsc{pfv}  3.\textsc{erg}{}-paint-\textsc{appl}  3.\textsc{poss}{}-house  \textsc{det}  man\\
    \glt ‘The man painted his/her house (i.e. another person’s house).’
    \z
\z 


In all three languages, an additional form (a kind of \textsc{antireflexive} marking) is required to allow (or even force) a disjoint interpretation. In Finnish and Halkomelem, this is the nonreflexive anaphoric person form, and in Chol, it is the applicative suffix {{}-be} on the verb.

 Coreference constructions are also widespread in clause combining, e.g. in certain complement clauses (see (\ref{ex:Haspelmath:4}) in \sectref{sec:Haspelmath:1}), in infinitival purposive clauses (e.g. German \textit{Sie} \textit{kam,} \textit{um} \textit{zu} \textit{helfen} [she came for to help] ‘She came to help’), and in relative clauses (e.g. English \textit{the} \textit{people} \textit{[living} \textit{next} \textit{door]} \textit{are} \textit{our} \textit{friends}). Special same-subject (SS) and different-subject (DS) constructions are widely used for clause combining patterns of various kinds in the world’s languages (when the SS/DS constructions are formally symmetrical, the term \textit{switch} \textit{reference} is sometimes used, e.g. \citealt{Vangijn2016}). These constructions also help with reference tracking, and some authors have tried to consider both clause-combining constructions and reflexive markers together (e.g. \citealt{Matić2014} ). But so far, there is little work that attempts a comprehensive picture of coreference constructions of diverse types.

\section{Two kinds of coreference: Discourse-referential and co-varying interpretations}\label{sec:Haspelmath:13}
Since the 1960s, it has been recognized that there are often two interpretations of coreferential anaphoric forms, which are best called the \textsc{discourse-referential} \textsc{intepretation} and the \textsc{co-varying} \textsc{interpretation} (often called bound-variable anaphora, e.g. \citealt{Reinhart1983a}; \citealt{DéchaineWiltschko2017b}). The contrast can be illustrated by (\ref{ex:Haspelmath:55a}-\ref{ex:Haspelmath:55b}). In (\ref{ex:Haspelmath:55a}), the dog is owned by a particular woman who can be identified in the discourse. But in (\ref{ex:Haspelmath:55b}), there is no particular woman, and no particular dog.

\ea%55
    \label{ex:Haspelmath:55}
    \ea \label{ex:Haspelmath:55a}
    discourse-referential: \\ {Ibrahim\textsubscript{1}} {loves} {her\textsubscript{2}} {dog.}
    \ex co-varying: \label{ex:Haspelmath:55b} \\
   {Every} {woman\textsubscript{1}} {loves} {her\textsubscript{1}} {dog.} (every woman {x}: {x} loves {x}’s dog)
    \z
\z 

Rather (\ref{ex:Haspelmath:55b}) says that the interpretation of \textit{her} varies with the interpretation of the quantified expression \textit{every} \textit{woman}. In logic, this is traditionally expressed by saying that there is a variable \textit{x} that is \textsc{bound} by the quantifier ‘every’ that has scope over it. The anaphoric pronoun \textit{her} can be thought of as corresponding to the bound variable \textit{x} in (\ref{ex:Haspelmath:55b}), rather than denoting a discourse referent.

  In a tradition going back to \citeauthor{Reinhart1983a} (\citeyear{Reinhart1983a, Reinhart1983b}), some authors have referred to this distinction as “coreference vs. binding” (e.g. \citealt[\S9.1]{HeimKratzer1998}; \citealt[\S 1.6.1]{Reuland2011}), but this terminology is confusing, because \textit{coreference} has long been used for the meaning underlying reflexive constructions, and is still widely used in this way. Thus, it is better to keep the term \textit{coreference} for the meaning underlying reflexive constructions, and to distinguish between two subtypes of coreference: discourse referential coreference and co-varying coreference.\footnote{ {I would thus say that two participants are coreferential (i) if they have the same referent or (ii) if their reference covaries. Authors who prefer to use} {\textit{coreference}} {in a narrow sense (only for referent identity) have proposed alternative cover terms, e.g.} {\textit{coconstrual}} {\citep{Safir2005} or} {\textit{covaluation}} {(e.g. \citealt{Reinhart2006}), but these terms have not been widely adopted.}}

  The distinction is somewhat relevant for reflexive constructions, because it appears that some reflexive constructions only allow a co-varying interpretation, while others also allow a discourse-referential intepretation of the reflexive pronoun. In many cases, anaphoric pronouns can be interpreted in both ways when they are coreferential with the subject, as illustrated in (\ref{ex:Haspelmath:50}) \citep[127--128]{Sag1976}. These two interpretations are usually called \textsc{strict} \textsc{reading} and \textsc{sloppy} \textsc{reading}.

\ea%56
    \label{ex:Haspelmath:56}
    {Betsy\textsubscript{1}} {loves} {her\textsubscript{1}} {dog,} {and} {Sandy\textsubscript{2}} {does,} {too.}\footnote{ {There is also a third reading of this sentence:} {\textit{Betsy}}{\textit{\textsubscript{1}}}{ \textit{loves} \textit{her}}{\textit{\textsubscript{3}}}{ \textit{dog,} \textit{and} \textit{Sandy} \textit{does,} \textit{too}}{. Here the anaphoric pronoun is not coreferential with the subject. Its reference is not syntactically limited, and in the right context, it may be coreferential with} {\textit{Sandy}} {(this is clearer in an example like} {\textit{Betsy} \textit{loves} \textit{his} \textit{dog,} \textit{and} \textit{Ibrahim} \textit{does,} \textit{too}}.)}
    \ea strict reading  (= Sandy also loves Betsy’s dog)\\
     Betsy\textsubscript{1} \textit{x}:   \textit{x} loves her\textsubscript{1} dog

    \& Sandy \textit{y}:  \textit{y} loves her\textsubscript{1} dog  (discourse-referential)
   \ex  sloppy reading (= Sandy also loves her (own) dog)  \\
   Betsy \textit{x}:   \textit{x} loves \textit{x}’s dog \& Sandy \textit{y}:  \textit{y} loves \textit{y}’s dog    (co-varying)
    \z
\z           

Reflexive coreferential pronouns are often said to force a sloppy reading (i.e. a co-varying interpretation), not allowing a strict reading. Thus, it seems that (\ref{ex:Haspelmath:57}) says that Sandy also looked at herself in the mirror. But on the other hand, (\ref{ex:Haspelmath:58}) can apparently also mean that Ben’s boss does not admire Ben so much (i.e. can have not only the sloppy reading, but also the strict reading).

\ea%57
    \label{ex:Haspelmath:57}
    {Betsy} {looked} {at} {herself} {in} {the} {mirror,} {and} {so} {did} {Sandy.} (co-varying) 
\z

          

\ea%58
    \label{ex:Haspelmath:58}
    {Ben} {admires} {himself} {more} {than} {his} {boss} {does.} (co-varying or discourse-referential)
    \z

         

The relevance of the co-varying/discourse-referential distinction for reflexive constructions seems clearest with adpossessive reflexives. For Russian, \citet[106]{Dahl1973} reported the contrast between (\ref{ex:Haspelmath:59a}), with the reflexive adpossessive \textit{svoj}, and (\ref{ex:Haspelmath:59b}), with the nonreflexive first person singular adpossessive \textit{moj}. The contrast in (\ref{ex:Haspelmath:60a}--\ref{ex:Haspelmath:60b}) is completely analogous.

\ea%59
    \label{ex:Haspelmath:59}
    \ea co-varying\label{ex:Haspelmath:59a}\\
    \gll {Ja}  {ljublju}  {svoju}  {ženu,}  {i}  {Ivan}  {tože.}\\
       I.\textsc{nom}  love  \textsc{refl.poss}  wife.\textsc{acc}  and  Ivan.\textsc{nom}  too \\
    \glt‘I love my wife, and so does Ivan (= Ivan loves his (own) wife).’
    \ex discourse-referential\label{ex:Haspelmath:59b}\\
    \gll  {Ja}  {ljublju}  {moju}  {ženu,}  {i}  {Ivan}    {tože.}  \\
    I.\textsc{nom}  love  my  wife  and  Iva.\textsc{nom}  too\\
    \glt  ‘I love my wife, and so does Ivan (= Ivan loves my wife).’
    \z
\z     

      

\ea%60
    \label{ex:Haspelmath:60}
    \ea co-varying\label{ex:Haspelmath:60a}\\
    \gll {Tol’ko}  {ja}  {ljublju}  {svoju}  {ženu.}\\
     only  I.\textsc{nom}  love  \textsc{refl.poss}  wife.\textsc{acc}   \\
    \glt‘Only I love my wife (= nobody else loves his wife).’
    \ex 
     discourse-referential\label{ex:Haspelmath:60b}\\
      {Tol’ko} {ja} {ljublju} {moju} {ženu.}\\
      ‘Only I love my wife (= nobody else loves my wife).’
    \z
\z 

It seems that when the coreference is not expressed by an anaphoric pronoun but is implicit in the construction (as in the cases in \sectref{sec:Haspelmath:12}), we only get the co-varying interpretation. Sentences such as \textit{He} \textit{undressed,} \textit{and} \textit{so} \textit{did} \textit{she}, are unambiguous (she did not undress him), just like sentences such as \textit{He} \textit{wanted} \textit{to} \textit{sing,} \textit{and} \textit{so} \textit{did} \textit{she} (this cannot mean that she wanted him to sing). Likewise, when the reflexivizer is a verbal marker, we seem to get only the co-varying interpretation, as in (\ref{ex:Haspelmath:61a}) from Russian, which contrasts with (\ref{ex:Haspelmath:61b}).\footnote{ {However, surprisingly, \citet{Kapitonov2014} finds that the Imbabura Quichua verbal voice suffix} {\textit{{}-ri-}} {can be used with both readings:} {\textit{Jusi-lla-mi} \textit{tsijni-ri-n}} {[José-}{\textsc{lim-val}} {hate-}{\textsc{refl-3sg}}{] ’Only José hates himself = \ref{ex:Haspelmath:1} nobody else self-hates (\ref{ex:Haspelmath:2}) nobody else hates José’.}}

\ea%61
    \label{ex:Haspelmath:61}
    \ea \label{ex:Haspelmath:61a}
    \gll  \textit{Saša}  {posmotrela-s’}  {v}  {zerkalo,}    {i}  {ja}  {tože.}\\
       Sasha  looked-\textsc{refl}  in  mirror    and  I  too \\
    \glt‘Sasha looked at herself in the mirror, and so did I.’ (co-varying only)
    \ex \label{ex:Haspelmath:61b}
    \gll {Saša}  {posmotrela}  {na}  {sebja}  {v}  {zerkalo,}    {i}  {ja}  {tože.}\\
    Sasha  looked  at  self  in  mirror    and  I  too\\
    \glt   ‘Sasha looked at herself in the mirror, and so did I.’ (co-varying or discourse-referential)
    \z
\z 
 

Thus, there are certain situations where the contrast between discourse-referential and co-varying coreference is relevant to grammatical coding, but there is no systematic cross-linguistic research on this aspect of grammatical expression.

\section{Conclusion}\label{sec:Haspelmath:14}
This concludes the survey of reflexive and related constructions, which I combined with a survey of ch1 terms for general linguistics that are useful for comparing languages and identifying shared traits. The wide range of diverse reflexive constructions makes it difficult to get a broad view of the big picture, and due to the language-particular focus of the great majority of research papers, it is not easy to focus on what is general and what is particular in this domain. Much of the literature on reflexive pronouns has taken the conditions on English reflexive pronouns as a starting point, but it seems that a broader perspective is more promising when we try to identify general traits of human languages.

The three appendixes that follow contain (A\ref{sec:Haspelmath:a}) a tentative list of universal generalizations, (B\ref{sec:Haspelmath:b}) a list of technical terms as used in this paper, and (C\ref{sec:Haspelmath:c}) a list of other terms that have been used in the literature but that seem less suitable to me because they cannot be defined clearly, at least not independently of larger controversial claims.

\section*{Acknowledgements}

I am grateful to Volker Gast, Ekkehard Koenig, Katarzyna Janic, and Karsten Schmidtke-Bode for useful comments on earlier versions of this paper, and to Eric Reuland and Yakov Testelets for general discussion.



\section*{Appendix A: Some universals of reflexive constructions}\label{sec:Haspelmath:a}

\textbf{Universal I} 
If a language has a reflexive voice marker or a reflexive argument marker, one of its uses is for autopathic coreference (agent-patient).

\paragraph*{Universal II}
If a language uses different constructions for agent-patient coreference for different verb types, then it uses shorter coding for introverted verbs than for extroverted verbs. (\citealt{KoenigVezzosi2004}; \citealt[44]{Haspelmath2008})

\paragraph*{Universal III}
In all languages, the usual coding of disjoint anaphoric reference is at least as short as the usual coding of agent-patient coreference. \citep[48]{Haspelmath2008}

\paragraph*{Universal IV} 

If an anaphoric pronoun may also be used as a demonstrative, it is always obviative in the autopathic domain.

\paragraph*{Universal V} 

If a language has nonreflexive object indexes (= bound object person forms), these cannot be used subject-coreferentially in the autopathic domain.

\paragraph*{Universal VI} 

If a language has a reflexive voice marker, it also has a voice marker for reciprocal constructions \citep[141]{Dixon2012}.

\paragraph*{Universal VII} 

If a language has a reflexive adpossessive pronoun, it also has a reflexive object pronoun. \citep[50]{Haspelmath2008}

\paragraph*{Universal VIII} 

If a language has a reflexive pronoun in locative phrases, it also has a reflexive pronoun in object position. \citep[55]{Haspelmath2008}

\paragraph*{Universal IX}

If a language has a reflexive pronoun in the long-distance domain, it also has a reflexive pronoun in the autopathic domain. \citep[58]{Haspelmath2008}

\paragraph*{Universal X}

If a language has different reflexive pronouns in the autopathic and the and long-distance domain, the autopathic reflexive pronoun is at least as long as the long- distance reflexive. (\citealt{Pica1987}; \citealt{Haspelmath2008}: 55)

\paragraph*{Universal XI: Antecedent-reflexive assymmetry} 

In all languages, the antedecent is higher on the rank scale of syntactic positions than the reflexive pronoun: subject > object > oblique > within nominal/within embedded clause (see \sectref{sec:Haspelmath:7} above; \citealt{Dixon2012}: 152) 

\paragraph*{Universal XII} 

If a language has a prenominal definite article, it does not have a reflexive adpossessive pronoun. \citep{Despić2015}

\paragraph*{Universal XIII} 

If a reflexivizer and a reciprocalizer are formally related to each other, then the reflexivizer is formally simpler. \citep[153]{Dixon2012}


\section*{Appendix B: Technical terminology used in this paper}\label{sec:Haspelmath:b}

\paragraph*{anaphora} Anaphora is the use of linguistic forms or constructions to signal coreference within the discourse or within a clause.

\paragraph*{anaphoric form} An anaphoric form is a form that stands for a referent which is coreferential with another referent (an antecedent) in discourse. (Typical anaphoric forms are anaphoric pronouns.)

\paragraph*{antecedent} In an anaphoric relationship, the antecedent of an anaphoric form or of an unexpressed anaphoric referent is the referent which determines its reference.

\paragraph*{clause} A clause is a combination of a predicate (full verb or nonverbal predicate) and its arguments plus modifiers.

\paragraph*{endophoric use} An endophoric use of a pronoun is an anaphoric use within a sentence or the discourse, as opposed to an exophoric use. 

\paragraph*{exophoric use} An endophoric use of a pronoun is a use for a referent that was not mentioned earlier in the discourse but is present in the context.

\paragraph*{obviative pronoun} An obviative pronoun is an anaphoric pronoun that cannot be coreferential with a coargument.

\paragraph*{reflexive argument marker} A reflexive argument marker is a grammatical marker that occurs on a transitive verb and that exhibits striking similarities with nonreflexive object indexes, especially with respect to its position.

\paragraph*{reflexive construction} A reflexive construction is a grammatical construction that can only be used when two participants of a clause are coreferential and that contains a special form that signals this coreference.

\paragraph*{reflexive pronoun = reflexive nominal} A reflexive pronoun a form that can be used in the position of a full nominal and that signals coreference with an antecedent in the same clause (subtypes: reflexive pronominoid...)

\paragraph*{reflexive pronominoid} A reflexive pronominoid is a reflexive pronoun that shares striking similarities with independent personal pronouns and is strikingly different from the nouns in the language


\paragraph*{reflexive voice marker} A reflexive voice marker is a grammatical marker that occurs on a transitive verb and indicates that its agent is coreferential with its patient, without exhibiting similarities to argument indexes.

\paragraph*{reflexivizer} A reflexivizer is a reflexive pronoun or a reflexive voice marker.

\paragraph*{self-intensifier} A self-intensifier is a form that accompanies a nominal and indicates that the nominal’s referent is the central referent in a centre-periphery configuation.


\section*{Appendix C: Other terms used elsewhere in the literature}\label{sec:Haspelmath:c}
\textbf{anaphor}
The term \textit{anaphor} became well-known through \citet{Chomsky1981}, and its generally understood as meaning ‘reflexive pronoun or reciprocal pronoun’,\footnote{ {More transparently, one would of course use} {\textit{anaphor}} {for ’anaphoric form’ (or more specifically, ’anaphoric pronoun’). The term is indeed sometimes used in this sense by computational linguists in the context of anaphora resolution (e.g. \citealt{Mitkov2002}).} } but it is rarely defined explicitly in this way (but cf. \citealt{Forker2014}: 52, n. 1). Some authors define anaphor as an ‘interpretatively dependent element’ (cf. \citealt[82]{Reuland2018}), which seems to mean that it cannot be used exophorically. However, as noted by \citet{Kiparsky2002,Kiparsky2012}, many languages have anaphoric forms that must be used endophorically (he calls them “discourse anaphors”), e.g. English \textit{it}. Calling such forms, too, “anaphors” is confusing. Moreover, some authors have invoked a completely different criterion for distinguishing anaphors from pronominals: “pronouns can have split antecedents, and anaphors cannot” (\citealt{Volkova2017}: 178; following \citealt{Giorgi1984}: 310).

\paragraph*{binding theory} “Binding theory” (or sometimes “Binding Theory”) is the name for three general rules of English grammar formulated by \citet{Chomsky1981} (following \citealt{Reinhart1976,Reinhart1983a}): (A) Anaphors must be bound in their local domain; (B) Pronominals must be free in their local domain; (C) Other nominals must always be free; where “X binds Y” means that X is coindexed (and thus coreferential) with Y and c-commands it. These rules or principles have typically been thought to be universal, though they were established entirely on the basis of English. Since the 1990s, it has been universally recognized that the 1981 formulation does not work (even for English), and many alternative versions have been proposed \citep{Everaert2003}, but always as claims about the regularities of particular languages (possibly rooted in innate knowledge), not as readily testable claims about cross-linguistic distributions.

\paragraph*{controller} The term \textit{controller} is sometimes used in the same sense as \textit{antecedent} (e.g. \citealt{Dixon2012}).

\paragraph*{pronominal} In the Reinhart-Chomsky tradition, “anaphors” are typically contrasted with “pronominals”, illustrated by English personal pronouns such as \textit{her,} \textit{him,} \textit{them}. Like \textit{anaphor}, the term \textit{pronominal} is rarely defined, and it has never been clear whether nonobviative personal pronouns like Jambi Malay \textit{dio} (see \ref{ex:Haspelmath:10}) in \sectref{sec:Haspelmath:4}) should be considered “pronominals”. (In my terminology, English \textit{him} is an obviative-nonreflexivethird-person pronoun, while Jambi Malay \textit{dio} is a nonobviative-nonreflexive third-person pronoun.)

\paragraph*{reflexive} The noun \textit{reflexive} is often used vaguely in the sense ‘reflexive construction’, or ‘reflexive element’, or “reflexive pronoun’ (e.g. \citealt{Geniusiene1987, Frajzyngier1999, Kazenin2001, KoenigGast2008, DéchaineWiltschko2017a}). The context sometimes makes it sufficiently clear what intended, but when the term is used in a book title, there is no context. I avoid such abreviated terms in formal contexts. (Though I do abbreviate \textit{long-distance} \textit{reflexive} \textit{pronoun} to \textit{long-distance} \textit{reflexive}, as noted in \sectref{sec:Haspelmath:9})

\paragraph*{reflexivity} The term \textit{reflexivity} is sometimes used collectively for the domain of refkexive constructions, and in this sense, there is no problem with it (cf. similar terms such as \textit{ergativity,} \textit{transitivity,} \textit{coordination}). But it is sometimes also used as if it were a semantic notion, and linguists talk about “encoding of reflexivity” (e.g. \citealt[63]{DéchaineWiltschko2017a}). For the semantic notion, I find \textit{coreference} a better term (or maybe \textit{autopathic} \textit{coreference}, if agent-patient coreference is intended), because it is best to have different terms for constructions and the meanings they express (see n. 2).

\paragraph*{reflexivization} This term from the 1960s originally referred to the creation of a reflexive construction as a grammatical operation, but more recently it has sometimes been used in a more restricted sense, referring specifically to the creation of “reflexive predicates” (or verbs). Much of this corresponds to reflexive voice marking, but authors such as \citet[399]{Reinhart2005} and \citet[197]{Everaert2013} include constructions like \textit{Max} \textit{undressed}, which are not regarded as reflexive here (see note 2).

\paragraph*{SE anaphor vs. SELF anaphor} The distinction between “simplex expression” (or SE) anaphors (Dutch \textit{zich}, Swedish \textit{sig}) and complex SELF anaphors (English \textit{himself}, Dutch \textit{zichzelf}, Swedish \textit{sig} \textit{själv}) became well-known through \citet{Reinhart1993}, but these authors did not give clear definitions of these terms. It seems that they thought that reflexive pronominoids of the European type (see \sectref{sec:Haspelmath:6.4}) and self-intensified anaphoric pronouns (see \sectref{sec:Haspelmath:6.3}) are typical of reflexive pronouns in general, but it has been known since \citet{Faltz1977} that other types of reflexive nominals are more common in the world’s languages.

{\sloppy\printbibliography[heading=subbibliography,notkeyword=this]}
\end{document}
