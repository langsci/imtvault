\documentclass[output=paper]{langscibook}

\author{Alice Gaby\affiliation{Monash University}}
\title{Reflexive constructions in  Kuuk Thayorre} 

\abstract{Kuuk Thaayorre is an Aboriginal language of the west coast of Cape York Peninsula, Australia. Self-directed events may be described by a number of distinct Kuuk Thaayorre constructions, which may include one or more of the following forms: a reflexive voice suffix to the verb (\textit{-e}); a reciprocal voice suffix to the verb (-\textit{rr}); a reflexive pronoun, a self-intensifier pronoun; or an inherently reflexive verb. Alternatively, overt marking of reflexivity may be absent, with the self-directedness of the event left to inference. In addition to providing an overview of the various forms of reflexive marking in Kuuk Thaayorre, this chapter surveys the range of event types encoded by these forms. For example, the verbal reflexivizer often signals that multiple argument roles map to the subject argument, whether agent + patient, agent + beneficiary, agent + causer, and more. In many cases, it is only a subset (e.g. a body part) or the agent-subject that is acted upon, so the coreference of agent and patient roles is incomplete. In other cases, reflexive forms are used to signal the subject argument’s heightened involvement in and/or affectedness by the event, whether or not they are agent of that event. Lastly, just as reciprocal morphology is found in the description of some reflexive events, so too is the verbal reflexivizer employed to describe some reciprocal events. This curious pattern of polyfunctionality may find its origins in the deep history of these forms; the final section of this chapter considers possible cognates and the semantic ranges of reflexive forms across the Pama-Nyungan family and the Australian continent more broadly.}

\IfFileExists{../localcommands.tex}{%hack to check whether this is being compiled as part of a collection or standalone
  \input{../localpackages}
  %Copy this to localcommands.tex

\usepackage[english]{babel}
\usepackage{amsmath}
\usepackage{amssymb,amsfonts,textcomp}
\usepackage{array}
\usepackage{hhline}
\usepackage{hyperref}

\newenvironment{styleStandard}{}{}
\newenvironment{stylelsAbstract}{}{}
\newenvironment{stylelsSectioni}{}{}
\newenvironment{stylelsSectionii}{}{}
\newenvironment{stylelsBulletList}{}{}
\newenvironment{styleBibliographyi}{}{}
\newenvironment{listWWNumxxvleveli}{}{}
\newenvironment{listWWNumxxvlevelii}{}{}
\newenvironment{listWWNumxxvleveliii}{}{}
\newenvironment{listWWNumxxvleveliv}{}{}
\newenvironment{listWWNumixleveli}{}{}
\newenvironment{listWWNumixlevelii}{}{}
\newenvironment{listWWNumixleveliii}{}{}
\newenvironment{listWWNumixleveliv}{}{}

\newcommand\textstyleListLabelxvi[1]{#1}
\newcommand\labellistWWNumxxvleveli{\thelistWWNumxxvleveli.}
\newcommand\labellistWWNumxxvlevelii{\thelistWWNumxxvlevelii.}
\newcommand\labellistWWNumxxvleveliii{\thelistWWNumxxvleveliii.}
\newcommand\labellistWWNumxxvleveliv{\thelistWWNumxxvleveliv.}
\newcommand\labellistWWNumixleveli{[F0B7?]}
\newcommand\labellistWWNumixlevelii{\textstyleListLabelxvi{o}}
\newcommand\labellistWWNumixleveliii{[F0A7?]}
\newcommand\labellistWWNumixleveliv{[F0B7?]}

\newcounter{listWWNumxxvleveli}
\newcounter{listWWNumxxvlevelii}[listWWNumxxvleveli]
\newcounter{listWWNumxxvleveliii}[listWWNumxxvlevelii]
\newcounter{listWWNumxxvleveliv}[listWWNumxxvleveliii]
\newcounter{itemize} 
  \input{../localhyphenation}
  \bibliography{localbibliography}
  %\togglepaper[5]
}{}


\begin{document}
\maketitle


\section{Introduction}\label{sec:Gaby:1}

Kuuk Thaayorre is the language of the Thaayorre people, whose lands include the Aboriginal Shire of Pormpuraaw on the west coast of Australia’s Cape York Peninsula (see \figref{fig:Gaby:1}). It belongs to the Paman subgroup of the Pama-Nyungan language family. While some children are presently growing up as fluent speakers of Kuuk Thaayorre, the various policies promoting English through church, school and government services since colonization have had a dramatic impact upon language transmission. The number of people for whom Kuuk Thaayorre is a language of daily communication is declining, currently estimated at {\textasciitilde}200. This chapter draws on narratives, recorded conversations and elicited data, which I compiled in collaboration with more than 30 Kuuk Thaayorre language experts between 2002 and 2008. It also draws on the rich example sentences included in Allen Hall’s theses \citeyear{Hall1968; Hall1972} and a dictionary created by Hall and (Kuuk Thaayorre language expert and teacher) Tom Foote \citep{FooteHall1992}. Dictionary examples were neither glossed nor translated in the original; glosses and translations are provided by the present author, and some examples modified to align with the orthography and morphological analysis adopted throughout this chapter. The transcriptions and glosses of examples from the theses of Allen Hall have been likewise modified for orthographic and/or analytical consistency.

%%[Warning: Draw object ignored]
  
%%please move the includegraphics inside the {figure} environment

 

\begin{figure}
    \caption{Map of Australia, showing location of Pormpuraaw}
    %%\includegraphics[width=\textwidth]{figures/a21GabyKuukThaayorre-img001.png}
\end{figure}


Kuuk Thaayorre is a predominantly dependent-marking language. The core syntactic functionsare signalled by the case-marking of noun phrases, and optionally by pronominal enclitics to the verb. The redundant apposition of coreferential pronouns and noun phrases is a common rhetorical device. Conversely, core arguments are freely and frequently elided; neither subject nor object neest to be overtly realized in the clause. Word order is not employed to distinguish grammatical relations; subject, object and verb may occur in any order. Order within the noun phrase is more fixed, and only the final\footnote{Except where the final constituent is an adnominal demonstrative, in which case the penultimate constituent inflects for case.} constituent of an NP inflects for case. For nouns, the ergative case distinguishes transitive subjects from unmarked (absolutive) transitive objects and intransitive subjects. For pronouns, the accusative form of transitive objects is distinguished from the nominative (transitive and intransitive) subject form. 

\sectref{sec:Gaby:2} begins with an overview of the Kuuk Thaayorre personal pronominal paradigm (\sectref{sec:Gaby:2.1}), before detailing the forms of the reflexive pronouns (\sectref{sec:Gaby:2.2}). \sectref{sec:Gaby:2.3} presents the Kuuk Thaayorre self-intensifier pronouns which, while they do not encode reflexivity as such, feature in clauses that would be translated by reflexive clauses in some other languages. The next section (\sectref{sec:Gaby:3}) moves to consider verbal coding of reflexivity. Kuuk Thaayorre verbs obligatorily inflect for tense/aspect, in addition to hosting optional person-number enclitics cross-referencing the subject and/or object. Derivational morphology on the verb includes a valency-increasing (causative/applicative) morpheme, two associated motion morphemes, the reflexive voice suffix (\sectref{sec:Gaby:3.1}), and the reciprocal voice suffix (\sectref{sec:Gaby:3.2}). \sectref{sec:Gaby:4} follows with an overview of the range of event types coded by reflexive forms, including seemingly reciprocal events (\sectref{sec:Gaby:4.6}). \sectref{sec:Gaby:5} considers the converse; the use of reciprocal morphology to code self-directed events. Reflexive marking can be said to be optional, inasmuch as there is no one-to-one correspondence between reflexive semantics and the verbal reflexivizer and/or reflexive pronoun. \sectref{sec:Gaby:6} concludes the chapter with a consideration of Kuuk Thaayorre reflexive constructions in the comparative context of Australian Aboriginal languages more broadly.

\section{Pronouns}\label{sec:Gaby:2}

\subsection{Personal pronouns}\label{sec:Gaby:2.1}

The Kuuk Thaayorre paradigm of personal pronouns distinguishes first/second/third person and singular/dual/plural number. Inclusive/exclusive first person pronouns are distinguished for nonsingular numbers. The nominative forms of each person/number/clusivity combination are presented in \tabref{tab:Gaby:1}.

\begin{table}
\begin{tabularx}{0.78\textwidth}{p{1.2cm}p{3cm}p{2cm}p{2cm}} 
\lsptoprule
& {1st} {Person} & {2nd} {Person} & {3rd} {Person}\\
\hline
{Singular} & {ngay} & {nhunt} & {nhul}\\
{Dual} & {ngal} (inclusive) {ngali} (exclusive) & {nhip} & {pul}\\
{Plural} & {ngamp} (inclusive) & {nhurr} & {peln}\\
& {ngancn} (exclusive) &  & \\
\lspbottomrule
\end{tabularx}
\caption{Kuuk Thaayorre personal pronouns (nominative case forms)}\label{tab:Gaby:1}
\end{table}

Personal pronouns inflect for case, reflecting the role of the pronoun in the clause. As mentioned in \sectref{bkm:Ref34811992}, the same case form (nominative) is used for both intransitive and transitive subjects. This and the further four case forms are illustrated for the first person singular pronoun in \tabref{tab:Gaby:2}.

\begin{table}
\begin{tabularx}{0.5\textwidth}{XX}
\lsptoprule
nominative & {ngay}\\
accusative & {nganh}\\
genitive & {ngathn}\\
dative & {ngathun}\\
ablative & {ngathnma}\\
\lspbottomrule
\end{tabularx}
\caption{First person singular pronouns (all cases)}\label{tab:Gaby:2}
\end{table}

\subsection{Reflexive pronouns}\label{sec:Gaby:2.2}
The etymology of reflexive pronouns is evident from their formal resemblance to the corresponding possessive and nominative pronoun forms, as demonstrated in \tabref{tab:Gaby:3}:

\begin{table}
\begin{tabularx}{0.7\textwidth}{XXXX}
\lsptoprule
& {Reflexive} & {Possessive} & {Nominative}\\
\hline
\textsc{1sg} & {ngathnay} {{\textasciitilde} ngathney} & {ngathn} & {ngay}\\
\textsc{2sg} & {nhangknunt} & {nhangkn} & {nhunt}\\
\textsc{3sg} & {nhangnul} & {nhangn} & {nhul}\\
\lspbottomrule
\end{tabularx}
\caption{Singular pronouns (nominative, genitive and reflexive forms)}\label{tab:Gaby:3}
\end{table}




\tabref{tab:Gaby:3} represents the full set of attested reflexive pronouns, which can be said to distinguish number only inasmuch as they require a singular interpretation. There are no nonsingular (dual or plural) reflexive pronouns; reflexive clauses with nonsingular subjects are coded as such by other means (e.g. the verbal reflexivizer, verbal reciprocalizer, a lexical reflexive verb, but not a non-reflexive object pronoun), or left to inference (see, e.g., Haspelmath, this volume). (In special cases, a singular number reflexive pronoun appears in a clause with a non-singular subject, see \sectref{sec:Gaby:5} discussion.)

Reflexive pronouns may be the sole marker of a reflexive clause \REF{ex:Gaby:1}, or they may combine with the verbal reflexivizer (or reciprocalizer, \sectref{sec:Gaby:5}) to reinforce the reflexive meaning \REF{ex:Gaby:2}:

\ea 
    \label{ex:Gaby:1}
    \gll ngay wash-m rirk-r ngathney\\
   \textsc{\textsc{1sg}}(\textsc{nom})  wash-\textsc{tr}  \textsc{do}{}-\textsc{pst.pfv} \textsc{1sg}.\textsc{refl} \\
    \glt  ‘I washed myself.’ [GJ25Oct2002 Elicitation]
\z
    
    
\ea 
    \label{ex:Gaby:2}
    \gll ngay yup ngathnay rinth-e-nha\\
    1sg\textsc{(nom)}  soon  1\textsc{sg}.\textsc{refl}  squeeze-\textsc{refl}{}-\textsc{sbjv}\\
    \glt ‘I want to squeeze my [blackhead].’ \citep[121]{Hall1972}
\z
 


Indeed, due to the frequent repetition of noun phrases noted above, it is not unusual for the same reflexive pronoun to appear more than once in the same clause, with or without a co-occurrent verbal reflexivizer, as in \REF{ex:Gaby:3}:

\ea 
    \label{ex:Gaby:3}
    \gll nhul nhangnul nhaanhath-e nhangnul koowmiing\\
    3\textsc{sg}(\textsc{nom}) 3\textsc{sg}.\textsc{refl}  look.at-\textsc{refl:npst}  3\textsc{sg}.\textsc{refl} face(\textsc{abs})\\
    \glt ‘he is looking at his face (in a mirror).’ \citep[379]{Hall1972}
   \z
  
The combination of Kuuk Thaayorre’s flexible constituent order, the ambivalent transitivity of reflexive clauses (\sectref{sec:Gaby:3.1}), the frequent apposition of coreferential noun phrases, and /or argument ellipsis, makes it difficult to establish whether reflexive pronouns occupy subject or object (/oblique) position. Their formal resemblance to the nominative case personal pronouns might suggest they are apposed to the subject argument. However, this is insufficient reason to posit Kuuk Thaayorre as an exception to the strong typological tendency for reflexive pronouns to fill the object or oblique position.

\subsection{Self-intensifier pronouns}\label{sec:Gaby:2.3}
Kuuk Thaayorre intensifier pronouns are formed by the reduplication of the nominative case form of the corresponding personal pronoun plus suffixation of a rhotic (either a retroflex approximant or tap/trill, depending on the pronoun). These pronouns do not encode reflexivity as such. However, events that are described by means of reflexive clauses in other languages may be described by clauses including an self-intensifier pronoun in Kuuk Thaayorre. In particular, self-intensifiers may be used to stress the lack of involvement of any external agent in (‘extroverted’) events that typically involve transfer of energy. For example, \REF{ex:Gaby:4} was uttered in the description of an elicitation video stimulus that presented a piece of cloth spontaneously tearing down the middle, as if by magic (note: the translation is given in the original speaker’s own words):

 

\ea 
    \label{ex:Gaby:4}
    \gll {mimp}  {ith}  {nhulnhulr}  {thaariic-r}\\
    cloth(\textsc{abs})  that  3\textsc{sg}\textsc{emph}  tear-\textsc{pst.pfv}\\
    \glt  ‘that piece of material is tearing up itself.’ [EN03Dec2002 Elicitation Cut\& Break8]
    \z


While the inanimate piece of cloth does not truly act upon itself in a canonical reflexive sense, it is notable that the speaker employs a reflexive clause in her English description of the same clip.

\section{Verbs}\label{sec:Gaby:3}
The primary means of coding reflexivity are the reflexive pronouns (introduced above; \sectref{sec:Gaby:2.2}) and the verbal reflexivizer (\sectref{sec:Gaby:3.1}). However, self-directed actions may be also encoded by reciprocal morphology (\sectref{sec:Gaby:3.2}) and both non-reflexive and inherently reflexive verbs (\sectref{sec:Gaby:3.3}). 

\subsection{Verbal reflexivizer}
\label{sec:Gaby:3.1}
The Kuuk Thaayorre reflexive voice marker (or “reflexivizer”) takes the form of a derivational suffix with allophones -\textit{e} and \textit{{}-ey.} It occupies the same position in the verb as the reciprocalizer and valency-increasing morpheme, immediately following the verb root (plus verbalizer, for denominal verbs):


\ea 
    \label{ex:Gaby:5}
    \gll {yuur}  {yak-ey-r} \\
    hand(\textsc{abs})  cut-\textsc{refl-pst.pfv}\\
    \glt ‘[she] cut [her] hand.’\footnote{Note that the body part apposition construction, in which the body part \textit{yuur} ‘hand’ is here apposed to the whole / possessor / subject in the same case, is analogous to the ‘external possession’ constructions of other languages. This construction is considered further in \sectref{sec:Gaby:4.3}.} [EF15Dec2002 Elicitation Cut\& Break18]
    \z


The reflexivizer must precede any tense/aspect/mood inflection and, where present, the associated motion suffixes. The reflexivizer and valency-increasing morpheme may combine in either order \REF{ex:Gaby:21}, but the reflexivizer and reciprocalizer do not co-occur in the same verbal word (see \citealt[292 ff.]{GabyExperts2017}):

\ea 
    \label{ex:Gaby:6}
    \gll  {ngay}  {ngathney} {mungka-n-ey-r}  {merrethen} \\
     \textsc{1sg}(\textsc{nom})  \textsc{1sg}.\textsc{refl}  consume-\textsc{v}\^{}-\textsc{refl}{}-\textsc{pst.pfv}  medicine(\textsc{abs})\\
    \glt  ‘I made myself swallow the medicine.’ \citep[392]{Hall1972}
    \z

The effect of reflexive derivation on clausal transitivity is not straightforward. In transitive clauses, a subject NP receives ergative case-marking. Following reflexivization, this subject NP is typically in unmarked absolutive case, as in the following example: 

\ea 
    \label{ex:Gaby:7}
    \gll  {kuta}  {ngith}  {pathath-e}\\
    dog(\textsc{abs})  that  bite:\textsc{rdp-refl:npst}\\
    \glt ‘that dog is biting himself.’[AC21Aug2002 Conversation]
    \z
  
However, in other cases the ergative marking is retained. This is usually the case when the clause includes an overt object (as in \ref{ex:Gaby:8}) or instrument (as in 9, note that instrumental adjuncts also receive ergative case-marking in Kuuk Thaayorre). 

\ea 
    \label{ex:Gaby:8}
    \gll  {John-i}  {yuur}  {theerng-ey-r}\\
    John-\textsc{erg}  hand(\textsc{abs)}  hit-\textsc{refl-pst.pfv}\\
    \glt ‘John hit himself on the hand.’[GJ11Jan2004 Elicitation]
    \z


\ea 
    \label{ex:Gaby:9}
    \gll {John-i}  {yuur-u}  {theerng-ey-r} \\
    John-\textsc{erg}  hand-\textsc{erg}  hit-\textsc{refl-pst.pfv}\\
    \glt ‘he hit himself with [his own] hand.’ [GJ11Jan2004 Elicitation]]
    \z


However, it is not as simple as the presence /absence of an overt object argument determining the presence /absence of ergative marking on the subject. In examples like \REF{ex:Gaby:10}, the subject is in absolutive case in spite of the presence of an absolutive-case object NP, {yiin} ‘itch’. 

\ea 
    \label{ex:Gaby:10}
    \gll  {kuta} {ith} {yiin=p} {pathath-e}\\
    dog(\textsc{abs}) that itch(\textsc{abs})=\textsc{prag}  bite:\textsc{rdp-refl:npst}\\
    \glt  ‘that dog keeps biting itself.’ [AC21Aug2002 Conversation]
    \z

Mixed indicators of transitivity are found in reciprocal as well as reflexive clauses in Kuuk Thaayorre and many other Australian Aboriginal languages, as discussed in detail by \citet{EvansEtAl2007}. ‘Optional ergativity’—i.e. the ergative marking of some intransitive subjects and/or lack of ergative marking of some transitive subjects—is also a broader feature of Kuuk Thaayorre grammar (see \citealt{Gaby2008Discourse, Gaby2010}).

\subsection{Verbal reciprocalizer}\label{sec:Gaby:3.2}
The reciprocalizer, \textit{-rr}, occurs in the same position in the verb as the reflexivizer (\sectref{sec:Gaby:3.1}). While it primarily functions to encode symmetric (“reciprocal”) events, such as \REF{ex:Gaby:11}, the verbal reciprocalizer is also found in the description of self-directed (“reflexive”) events, such as \REF{ex:Gaby:12} and \REF{ex:Gaby:13}. (Note that \REF{ex:Gaby:13} involved a trivalent verb, where the reciprocalizer binds the direct object, what is shown, rather than the recipient, to whom it is shown).

\ea 
    \label{ex:Gaby:11}
    \gll Harry pul Micki-n melnkelnkarr nhaath-rr-nan\\
    Harry  \textsc{du}(\textsc{nom})  Micky-\textsc{erg}  tomorrow  see-\textsc{rcp-go\&} \\
    \glt ‘Harry and Micky will see each other tomorrow.’ [EF15Dec2002 Elicitation]
    \z

 
  

\ea 
    \label{ex:Gaby:12}
    \gll {peln}  {nhangnma}  {koowmiing}  {reepon-rr-nam}\\
    \textsc{3pl}(\textsc{nom})  \textsc{3sg}\textsc{abl}  face(\textsc{abs})  hide-\textsc{rcp-p.ipfv} \\
    \glt ‘they were hiding their faces from him.’ \citep[392]{Hall1972}
    \z

 
\ea 
    \label{ex:Gaby:13}
    \gll  {iirra}  {thakrwuthurr}  {yarr}  {meeren-rr-nan}  {pam}  {pork-a}\\
   to.there  promptly  go-\textsc{imp}  show-\textsc{rcp-go\&} man  big-\textsc{dat} \\
    \glt ‘go and report to the boss immediately.’ (\citealt{FooteHall1992}: 333)
    \z
    

Further, a number of lexicalized, semantically reflexive verbs exhibit what appears to be a relic of the verbal reciprocalizer. Compare, for example, \textit{wothoth} ‘wipe’ versus \textit{wothothrr} ‘wash oneself’, or \textit{puunm} ‘pity’ versus \textit{puunmrr} ‘grieve’ (\citealt{FooteHall1992}: 310). \sectref{sec:Gaby:5} describes the range of self-directed actions marked by the reciprocalizer. 

\subsection{Non-reflexive and inherently reflexive verbs}\label{sec:Gaby:3.3}
Introverted verbs expressing autopathic actions are typically expressed by non-reflexive clauses. For example, descriptions of dressing are never marked by either the pronominal or verbal reflexivizer, but rather by means of the verb \textit{rok} ‘enter’. In its basic usage, this verb’s argument structure comprises an intransitive subject (encoding the theme) and dative-marked oblique argument (encoding the goal), as seen in example \REF{ex:Gaby:14} below. In descriptions of dressing and adornment, however, the agent-theme is encoded as intransitive subject (in nominative/absolutive case) but the article of clothing/adornment is encoded as direct object (in absolutive case). This verb is used even in cases such as \REF{ex:Gaby:15}, where the agent-subject does not truly ‘enter’ the glasses in the way that one enters a dress or coat:

\ea 
    \label{ex:Gaby:14}
    \gll {hall-ak}  {ulp=okun}  {rok-nhan}\\
    hall-\textsc{dat}  \textsc{dem:adr.prox}=\textsc{dub}  enter-\textsc{go\&} \textsc{:npst}\\
    \glt ‘they might go into the hall [in a cyclone].’ [GJ03Apr2004 Conversation]
    \z
    


  

\ea 
    \label{ex:Gaby:15}
    \gll {nhul}  {meer+kay}  {rok-r}\\
    \textsc{3sg}\textsc{nom}  eye+metal(\textsc{abs)}  enter-\textsc{pst.pfv}\\
    \glt ‘he put on glasses.’ [GJ18Jan2004 Elicitation]
    \z

In conservative varieties of Kuuk Thaayorre, autopathic events of washing are described by means of the non-reflexive compound verbs \textit{koo+munth} ‘wash one’s own face’ (cf. \textit{koow} ‘nose, upper face’, \textit{munth} ‘sink’) and \textit{minc+munth} ‘wash oneself’ (cf. \textit{minc} ‘body’). Younger speakers, however, commonly employ the English loan \textit{wash} (paired with the light verb obligatory for loan verbs) and a reflexive pronoun, as in \REF{ex:Gaby:1} above. Kuuk Thaayorre possesses several “inherently reflexive” verbs; extroverted and introverted verb roots ending in /\textit{e}/, which is likely a frozen relic of the verbal reflexivizer. For example, \textit{koope} ‘wait’; \textit{kongke} ‘copulate’ (\citealt{FooteHall1992}: 218); \textit{ngaathe} ‘feel, experience’; and \textit{wene} ‘become’.

\section{Reflexive functions}\label{sec:Gaby:4}
Across all its uses, the reflexive pronoun marks that the agent-subject is affected by their own actions. No such monosemous definition is available for the verbal reflexivizer, which spans a range of functions as outlined below. The reflexive pronoun appears with many of these functions, too, but always contributing the meaning that the agent is affected by their own actions. We begin in \sectref{sec:Gaby:4.1} with a consideration of the core reflexive function; indicating coreference of agent-subject and patient. We then move to consider cases of coreference between the agent-subject and three different oblique argument roles; the beneficiary/maleficiary (\sectref{sec:Gaby:4.2.1}), causer (\sectref{sec:Gaby:4.2.2}), and recipient (\sectref{sec:Gaby:4.2.3}). \sectref{sec:Gaby:4.3} details the reflexive coding of events in which the agent-subject moves their whole body, or acts upon one body part in particular. \sectref{sec:Gaby:4.4} considers the use of reflexive marking to emphasize that the agent-subject is intensely involved in and/or affected by the action they take. \sectref{sec:Gaby:4.5} surveys the passive-like function of reflexive marking to foreground an affected patient (in the absence of any syntactic passive operation). Lastly, \sectref{sec:Gaby:4.6} demonstrates the use of reflexive morphology to encode apparently reciprocal events. 

\subsection{Coreference of agent-subject and patient}\label{sec:Gaby:4.1}
Under the working definition of reflexivity adopted here, the core function of reflexive marking is to indicate coreference of the two argument roles that would map to the subject and direct object of the non-reflexivized verb respectively. Thus, in example \REF{ex:Gaby:16}, the one man both touches and is touched, and in \REF{ex:Gaby:17} the one man both pities and is pitied:

\ea 
    \label{ex:Gaby:16}
    \gll {nhul}  {yarriy}  {katp-ey-r,}  {‘iitharrkoo,}  {kam}  {inh!’}\\
    \textsc{3sg}(\textsc{nom})  thus  grasp-\textsc{refl-pst.pfv}  wow  blood  \textsc{dem:sp.prx}\\
    \glt ‘he touched himself like this [and realized] ‘hey, there’s blood here!’’ 
[GJ03Feb2004 Narrative DarwinTrip]
    \z



\ea 
    \label{ex:Gaby:17}
    \gll {nhangnul}  {puunm-rr-r}\\
    \textsc{3sg}.\textsc{refl}  pity\textsc{{}-rcp-pst.pfv} \\
    \glt  ‘[he] pitied himself.’ (\citealt{FooteHall1992}: 310)
    \z

This subject-object coreference may be coded by the verbal reflexivizer \REF{ex:Gaby:16}, reflexive pronoun (\ref{ex:Gaby:22} above), both (\ref{ex:Gaby:2} above) or some other combination of markers, such as reflexive pronoun plus verbal reciprocalizer \REF{ex:Gaby:17}. 

\subsection{Coreference of agent and an oblique role}\label{sec:Gaby:4.2}
\subsubsection{Agent + beneficiary / maleficiary}\label{sec:Gaby:4.2.2}
As well as signalling the coreference of agent and patient arguments, the reflexive pronoun may be used to indicate that the agent-subject is also the beneficiary of their own action. This function, illustrated by example \REF{ex:Gaby:18}, is only attested for the reflexive pronoun, not the verbal suffix:

\ea 
    \label{ex:Gaby:18}
    \gll {ngay}  {ngok}  {mi’irr}  {ngathaney}\\
    \textsc{1sg}(\textsc{erg})  water(\textsc{acc})  pick.up.\textsc{pst.pfv}  \textsc{1sg}.\textsc{refl}\\
    \glt ‘I got myself some water.’ \citep[379]{Hall1972}
    \z
 
Attribution of the converse maleficiary role to the agent-subject may likewise be indicated by the pronominal, but not verbal, reflexivizer:

\ea 
    \label{ex:Gaby:19}
    \gll {plate}  {ulp}  {nhangnul}  {thiika-rr}\\
    plate\textsc{(acc})  dem:ad.prx  \textsc{3sg}.\textsc{refl}  break-\textsc{pst.pfv}\\
    \glt ‘[that kid] broke his own plate.’
    [GJ12Jan2004 Elicitation]
    \z

 
  



On first glance, the reflexive pronoun in \REF{ex:Gaby:19} might be thought to mark the agent-subject as the possessor of the patient, rather than a negatively-affected maleficiary. However, the coreference of agent-subject and possessor of the patient-object is not signalled by reflexive marking where the agent is not also beneficiary or maleficiary. Such possessors are encoded by the standard possessive pronouns, as in \REF{ex:Gaby:20}: 

\ea 
    \label{ex:Gaby:20}
    \gll {pamthaaw}  {nhangn}  {kaar}  {nhaawr}  {nhul?}\\
     friend  \textsc{3sg}\textsc{pos}(\textsc{abs)}  \textsc{neg}  see:\textsc{pst.pfv}  \textsc{3sg}\textsc{(nom)}\\
    \glt ‘didn't he see his friends?’ \citep[65]{Hall1972} 
    \z

\subsubsection{Agent + causer}\label{sec:Gaby:4.2.3}
The Kuuk Thaayorre valency-increasing morpheme (glossed “V\^{}”) makes intransitive verb roots transitive and transitive verb roots ditransitive by adding an argument with a semantic role determined by the verb root. Where it is a causer that is introduced, the causer (now encoded as subject) may be marked as coreferential with the causee-agent by means of the verbal reflexivizer, as in \REF{ex:Gaby:21}:

\ea 
    \label{ex:Gaby:21}
    \gll {ngay}  {ngathney}  {mungka-n-ey-r}   {merrethen}\\
    \textsc{1sg}(\textsc{erg})  \textsc{1sg}.\textsc{refl}  consume-\textsc{v}\^{}-\textsc{refl}{}-\textsc{pst.pfv}  medicine(\textsc{acc})\\
    \glt ‘I made myself swallow the medicine.’ \citep[392]{Hall1972}
    \z

  
\subsubsection{Agent + recipient}
\label{sec:Gaby:4.2.4}
My corpus contains no examples of the verbal reflexivizer being used to mark coreference of agent-subject and recipient, this is a function generally achieved by means of the reflexive pronoun, as seen in the following example:

\ea 
    \label{ex:Gaby:22}
    \gll {nhul}  {nhangnul}  {riiranmrr}  {yik-r}\\
    \textsc{3sg}(\textsc{nom})  \textsc{3sg}.\textsc{refl}  alone  say-\textsc{pst.pfv}\\
    \glt ‘he talks [sic] to himself alone.’ \citep[503]{Hall1972}
    \z

 
\subsection{Body-part and whole body actions}
\label{sec:Gaby:4.3}
Many reflexive clauses describe the agent-subject acting upon a part of their own body. There is thus only partial coreference of agent and patient arguments in examples like \REF{ex:Gaby:23}:

\ea 
    \label{ex:Gaby:23}
    \gll {pam-al}  {ith}  {koow}  {katpatp-e}\\
    man-\textsc{erg}  that  nose(\textsc{abs})  grasp\textsc{:rdp-refl:npst}\\
    \glt ‘that man is holding [his] nose.’ [FT10Feb2004 RcpPilot8]
    \z
    
    
The body part may have the role of instrument (aligning with the agent), rather than patient, as seen in examples \REF{ex:Gaby:8}--\REF{ex:Gaby:9} above and \REF{ex:Gaby:24} below. As with core reflexive clauses, the partial coreference of agent and body part patient /instrument may be signalled by a range of forms (alone or in combination), including the verbal \REF{ex:Gaby:23} and pronominal \REF{ex:Gaby:24} reflexivizers:

\ea 
    \label{ex:Gaby:24}
    \gll {parr-an}  {nhul}  {yangkar}  {wiiyth}  {thaa+rinthi-rr}  {yuur-u}  {nhangnul}\\
    child\textsc{{}-erg}  \textsc{3sg}(\textsc{nom})  leg(\textsc{abs})  sore(\textsc{abs})  mouth+squeeze\textsc{{}-pst.pfv}  hand-\textsc{erg}  \textsc{3sg}.\textsc{refl}\\
    \glt   ‘the boy squeezed himself on the leg with his finger.’ \citep[379]{Hall1972}
    \z


Events which may be framed as an agent’s reflexively acting upon their own body part in other languages are instead expressed via same-case apposition of the noun phrases representing whole and part in Kuuk Thaayorre. Thus in \REF{ex:Gaby:25}, both the whole dog (as agent) and his head part (as theme) are encoded by distinct noun phrases in the nominative case assigned to the subject of \textit{rok} ‘enter’ (see \textsc{\sectref{sec:Gaby:3.3}} for further discussion of this verb). (For a detailed description of part-whole apposition in Kuuk Thaayorre, see \citealt[277 ff.]{GabyExperts2017}):

\ea 
    \label{ex:Gaby:25}
    \gll {kuta}  {nhul}  {paant}  {glass-ak}  {rok-r}\\
    dog(\textsc{nom})  \textsc{3sg}(\textsc{nom})  head(\textsc{nom})  glass-\textsc{dat}  enter-\textsc{pst.pfv}\\
    \glt ‘the dog put his head into the jar.’ (MF17Sep2002 Narrative FrogStory)
    \z
    
Actions affecting a body part are not always clearly distinguishable from actions affecting the whole body. For example, the subject participant of \REF{ex:Gaby:16} above only touched one part of himself (his arm), but this body part was not explicitly mentioned. In other cases, such as \REF{ex:Gaby:26}, it is clearly, though implicitly, the agent’s whole body that is affected by their action:

\ea 
    \label{ex:Gaby:26}
    \gll  {nhunt}  {koorrkorr}  {thaat}  {pirk-rr}  {ngathun}\\
    \textsc{2sg}(\textsc{nom})  behind\textsc{:rdp}  wide  push-\textsc{rcp:imp}  \textsc{1sg}\textsc{dat}\\
    \glt ‘move yourself along there a bit for me.’ \citep[446]{Hall1972}
    \z
    

\subsection{Intensification / affected agent}
\label{sec:Gaby:4.4}
The etymological connection between self-intensifiers and reflexive pronouns has been well documented \citep{Faltz1985, KoenigSiemund2000}, though we have already seen that the Kuuk Thaayorre reflexive pronouns are distinct from the paradigm of self-intensifier pronouns (\sectref{sec:Gaby:2.2} and \sectref{sec:Gaby:2.3}). The verbal reflexivizer, however, can be used with an effect of intensification. Compare, for example, \REF{ex:Gaby:27} below with the English reflexive expression ‘to eat oneself sick’:
\ea 
    \label{ex:Gaby:27}
    \gll {nhunt}  {thaaw+murm}  {paath-ey-r}  {may-im}
\\
     \textsc{2sg}(\textsc{nom)}  mouth+sink  bite-\textsc{refl-pst.pfv}  \textsc{veg-abl}\\
    \glt  ‘You really hogged into that food [i.e. ate greedily] you did.’ \citep[504]{Hall1972}
    \z
    
\subsection{Medio-passive}
\label{sec:Gaby:4.5}
Unsurprisingly, given the ergative-absolutive case frame of noun-headed NPs and the free ellipsis of core arguments, Kuuk Thaayorre has no syntactic operation akin to a passive. However, the reflexive voice may be employed to foreground the affected patient in a passive-like construction \citep{Geniuisiene1987,  Kemmer1993}, as seen in examples \REF{ex:Gaby:28}--\REF{ex:Gaby:29}:

\ea 
    \label{ex:Gaby:28}
    \gll {nhul}  {Jesus}  {werngka}  {yongk-e-nham}\\
    \textsc{3sg}(\textsc{nom})  Jesus(\textsc{abs})  middle  hang-\textsc{refl-pst.ipfv}\\
    \glt ‘Jesus was hanging in the middle.’ \citep[137]{Hall1972}
    \z
    
 
\ea 
    \label{ex:Gaby:29}
    \gll {yangan}  {kaal-ak}  {kath-ey-r}\\
     hair(\textsc{abs})  ear-\textsc{dat}  bind-\textsc{refl-pst.pfv}\\
    \glt ‘[his] hair was tied over [his] ears.’ [GJ15Oct2002 Elicitation BowPed46)
    \z

Note that the (unmarked) absolutive case form of \textit{yangan} ‘hair’ in \REF{ex:Gaby:29} permits two syntactic analyses. Under the first, \textit{yangan} is the direct object representing the patient affected by the actions of some unmentioned agent (i.e ‘[someone] tied [his] hair over [his] ears’). Under the second, \textit{yangan} is either the intransitive subject, possibly in apposition to an elided NP representing the whole agent-subject (i.e. [he] tied [his own] hair over [his] ears’). 

\subsection{Reciprocal use of the reflexivizer}
\label{sec:Gaby:4.6}
Where a reflexive-marked verb combines with a non-singular subject, it is not necessary that both agent and patient roles be ascribed to each participant encoded as subject, it is sufficient that just one participant is both agent and patient of the action described. Example \REF{ex:Gaby:30}, for instance, could describe a single woman painting both herself and her sister, according to the language expert consulted:

\ea 
    \label{ex:Gaby:30}
    \gll  {ngali}  {muul-thurr}  {werk-ey-r}\\
   1du:excl(\textsc{nom})  white.ochre-\textsc{erg}  rub-\textsc{refl-pst.pfv} \\
    \glt ‘we two painted ourselves and/or each other with white ochre.’
[EF15Dec2002 Elicitation – modelled on \citealt{Hall1972}]
    \z


This ‘collective reflexive’ usage may be a bridging context for the use of the verbal reflexivizer to describe apparently reciprocal events. In these events, while each subject participant is both agent and patient of the action described, they are not patient of the same subevent of which they are agent. That is to say, they do not act upon themselves, but rather one another. In \REF{ex:Gaby:31}, for example, each of the two people encoded by the dual subject pronoun leans upon the other, not upon him/herself: 

\ea 
    \label{ex:Gaby:31}
    \gll pul mut-u thaayooyongk-e\\
   3\textsc{du}(\textsc{nom)}  back-\textsc{dat}  lean\textsc{:rdp-refl:npst}  \\
    \glt ‘they are leaning (on each other) back to back.’ [FT10Feb2004 Elicitation RCP12]
    \z
 

\ea 
    \label{ex:Gaby:32}
    \gll  {pul}  {runc-ey-r}\\
    \textsc{du}(\textsc{nom})  impact {}-\textsc{refl-pst.pfv} \\
    \glt ‘they two collided with one another.’\footnote{The verb root \textit{runc} denotes a broad range of events of impact, including stabbing, kicking, crushing, falling to the ground, and more.} [FT10Feb2004 Elicitation RCP22]
    \z

Such events typically receive reflexive marking only where one or more of the following conditions are met: (i) there is close physical contact between participants, or intimate non-physical contact (such as staring into one another’s eyes); (ii) participants synchronize or closely coordinate their actions; (iii) there is a blurring of roles (it is unimportant and/or unspecified which of the subject participant(s) play the roles of agent and/or patient). For further examples and discussion see \citet{Gaby2011}. 

\section{Self-directed actions marked by the reciprocalizer}\label{sec:Gaby:5}
As mentioned in \sectref{sec:Gaby:4.5}, the verbal reflexivizer may appear in clauses in which an external agent is backgrounded, unimportant or absent. It is perhaps for this reason that self-directed events in which the agent-subject acts upon themself with unexpected or heightened agency are marked by reciprocal, rather than reflexive, voice marking on the verb. For example, a woman’s breaking her foot by accidentally treading in a hole or knocking a heavy object onto it might be described by means of the verbal reflexivizer. But a woman taking her foot in her hands and deliberately breaking it is described by pairing the verbal reciprocalizer with the reflexive pronoun:

 \ea 
    \label{ex:Gaby:33}
    \gll {paanth-u}  {thamr}  {nhangnul}  {thiik-rr-r} \\
woman\textsc{-erg}  foot(\textsc{abs)}  \textsc{3sg}.\textsc{refl}  break-\textsc{rcp-pst.pfv}\\
    \glt ‘the woman broke her own feet.’ [GJ11Jan2004 Elicitation]
    \z
 
Similarly, a typical event of falling or lying down would be described by intransitive verb forms such as \textit{wont} ‘fall’ or \textit{wun} ‘lie’, but a pragmatically unusual event of throwing oneself to the ground receives reciprocal marking paired with the reflexive pronoun:

\ea 
    \label{ex:Gaby:34}
    \gll  {nhunt}  {nhangknunt}  {thunp-rr}  {ii-rr-kop}  {raak-un}\\
     \textsc{2sg}(\textsc{nom})  \textsc{2sg}.\textsc{refl}  throw-\textsc{rcp:imp}  there-towards-below  ground-\textsc{dat}\\
    \glt ‘throw yourself down onto the ground!’ (\citealt{FooteHall1992}: 360)
    \z
It is possible that revealing one’s own weaknesses (as in 35) is more unusual than pointing out the weaknesses of others. However, in the Bible translation work conducted by Allen Hall and Tom Foote (on which Hall’s two theses and \citealt{FooteHall1992} were based), at least, the reciprocal verb stem \textit{kunanpunrr} is relatively high frequency, with meanings including ‘testify’, ‘repent’, ‘confess’, ‘admit’, and ‘give an account of’:

\ea 
    \label{ex:Gaby:35}
    \gll  {ngamp}  {yiirryirram}  {nhanganul}  {kunanpun-rr-nan}  {nhangun}\\
    1\textsc{pl:incl}\textsc{(nom)}  severally  \textsc{\textsc{3sg}}.\textsc{refl}  reveal-\textsc{rcp-go\&} \textsc{:npst}  \textsc{3sg}\textsc{dat}\\
    \glt ‘We each will give an account of ourselves to Him.’ \citep[392]{Hall1972}
    \z


It may not be possible to motivate each instance of reciprocal marking of self-directed events in terms of pragmatic unusualness, as some such verb stems (such as \textit{kunanpunrr}) seem at least partially lexicalized. 

Example \REF{ex:Gaby:35} is interesting for another reason. It was mentioned in \sectref{sec:Gaby:2.2} that reflexive pronouns—which are exclusively singular in number—may combine with non-singular subject NPs. In \REF{ex:Gaby:35}, the plural subject pronoun (\textit{ngamp} ‘we’) combines with a singular reflexive pronoun (\textit{nhangnul} ‘him/herself’) to stress that each participant is both agent and patient of a single subevent. The inclusion of the reflexive pronoun is necessary not only to stress the strict, individual self-directness of the event described, but also to differentiate the intended reflexive reading from the basic reciprocal meaning suggested by the verbal morphology.

See \citet{Gaby2008Reciprocals, Gaby2011} for a more detailed consideration of the relationship between the verbal reflexivizer and verbal reciprocalizer and the events they encode.

\section{Kuuk Thaayorre in context}
\label{sec:Gaby:6}
The forms and functions of reflexive marking in Kuuk Thaayorre are not unusual in the local typological context. Among Australian Aboriginal languages, it is commonplace for verbal reflexivizers to be highly polysemous, frequently including reciprocity within their semantic range (Gaby In press). It is likewise common for a reflexive pronoun to optionally combine with a verbal reflexivizer to force a strictly reflexive interpretation. These similarities are not attributable to shared inheritance, however. Though a proto-Australian “reflexive/intransitivizer suffix” with the form *-DHirri-y\footnote{The digraph ‘DH’ here represents a laminal stop at either dental or palatal place of articulation.} was proposed by \citet[447]{Dixon1980}, a common ancestor to Australia’s 250+ Indigenous languages remains elusive and controversial (see, e.g., \citealt{HarveyMailhammer2017}). The Kuuk Thaayorre verbal reciprocalizer, \textit{{}-rr}, might be a reflex of *DHirri-y, along with any number of synchronic reflexive and/or reciprocalizers that include the tap/trill segment (for example Arrernte \textit{{}-rre}; Bāgandji \textit{{}-dhirri}; Bininj Gun-Wok \textit{{}-rr}; Djabugay \nobreakdash-\textit{nydyirri;}; Jiwarli \textit{\nobreakdash-rri}; Martuthunira \textit{\nobreakdash-yarri}; Panyjima \nobreakdash-\textit{rri}; Rembarrnga \textit{\nobreakdash-rroe}; Wirangu \textit{\nobreakdash-ri}). So too might the Kuuk Thaayorre verbal reflexivizer, -\textit{e}, ultimately derive from the high vowel + palatal glide at the end of Dixon’s reconstructed form, along with reflexive morphemes in other Australian languages which take the form of a high front vowel/glide (for example, Bunuba \textit{\nobreakdash-iy}; Djabugay \textit{\nobreakdash-yi}; Guugu Yimidhirr \textit{-:yi;} Kunbarlang -\textit{yi;} Ndjébbana \textit{\nobreakdash-yi}; Ngandi \textit{\nobreakdash-i} \textit{{\textasciitilde} -yi}). In the absence of a detailed account of sound change and genealogical relatedness of the relevant languages, though, there is insufficient support for asserting the cognacy of these mono- or bi-segmental morphemes.  

The etymological source of the Kuuk Thaayorre reflexive pronoun is more transparent. As outlined in \sectref{sec:Gaby:2.2}, these pronouns are formally related to both the nominative-form personal pronouns and the possessive pronominal paradigm. This may suggest an original bridging context in which the agent-subject encoded by the nominative-form pronoun acts upon a part or the whole of their body, encoded (at least in part) by the possessive pronoun, or in which the agent-subject is coreferential with the recipient/beneficiary/maleficiary of their action, coded as possessor. Unlike other reflexive pronouns (both in Australia and beyond), the contrast with a distinct set of self-intensifier pronouns rules out the latter as an etymological source. The Kuuk Thaayorre reflexive pronouns are also notable for their being limited to singular number, apparently linked to their strictly entailing that each agent-subject participant acts upon him/herself individually.

\section{Acknowledgments}

I gratefully acknowledge the collaboration of the many Kuuk Thaayorre language experts who contributed to the data cited and to my understanding of the underlying grammar analysed herein. The particular examples presented in this chapter were contributed by the following language experts, whose initials are provided in the source code beside the relevant example: Alfred Charlie; Elizabeth Foote; Myrtle Foote; Albert Jack; Gilbert Jack; Elizabeth Norman; Freddy Tyore. I am also grateful to the editors and reviewers for their helpful comments. 

\section*{Abbreviations}

\begin{tabularx}{.45\textwidth}{lQ}
1 & first person\\ 
2 & second person\\ 
3 & third person\\ 
\textsc{abl} & ablative\\ 
\textsc{abs} & absolutive\\ 
\textsc{acc} & accusative\\ 
\textsc{v\^{}}& valence increasing suffix \\ \textsc{dat} & dative\\ 
\textsc{dem:adr.prox} & addressee-proximal adnominal demonstrative\\ 
\textsc{dem:sp.prx} & speaker-proximal adnominal demonstrative\\
\textsc{du} & dual\\ 
\textsc{emph} & emphatic\\ 
\end{tabularx}
\begin{tabularx}{.45\textwidth}{lQ}
\textsc{erg} & ergative\\ 
\textsc{go\&}  & associated motion verbal suffix\\
\textsc{imp} & imperative\\ 
\textsc{incl} & inclusive of addressee\\
\textsc{nom} & nominative\\ 
\textsc{npst} & nonpast\\ 
\textsc{p.ipfv} & past imperfective\\ 
\textsc{pl} & plural\\ 
\textsc{pos} & possessive pronoun’ \\
\textsc{p.pfv} & past perfective\\ 
\textsc{prag} & pragmatic enclitic\\ 
\textsc{rcp} reciprocal\\
\textsc{rdp} & reduplicated\\ 
\textsc{rfl} & reflexive\\ 
\textsc{sg} & singular\\
\textsc{tr} & transitive suffix (to loan coverbs)\\
\end{tabularx}

The following abbreviations are employed in glossing example sentences: 
 

Generic nouns (e.g. \textit{may} ‘\textsc{veg} vegetable’) and the light verb \textit{rirk} ‘\textsc{do}’ are glossed in small caps.

\end{document}


