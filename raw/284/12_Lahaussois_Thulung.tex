\documentclass[output=paper]{langscibook}

\author{Aimée Lahaussois\affiliation{Laboratoire {Histoire} {des} {Théories} {Linguistiques}, CNRS, Université de Paris}}
\title{Reflexive constructions in Thulung} 
\abstract{In this contribution, I present reflexive constructions in Thulung (Sino-Tibetan, Nepal).  After introducing the language and its basic morphosyntax, I describe the primary reflexive strategy, which is the reflexive voice marker \textit{{}-siʈ}, as well as the other uses of the same voice marker and the unclear status of the emphatic nominal \textit{twap} in reflexivization.  I then discuss the expression of coreference with different verb types, and with different semantic roles, before describing the difficulties of expressing partial coreference. I close the chapter with examples of long-distance coreference, a relatively simple situation in Thulung, which can embed reported discourse (or thought) only as direct speech.}
\IfFileExists{../localcommands.tex}{
 \addbibresource{localbibliography.bib}
 \input{../localpackages}
 %Copy this to localcommands.tex

\usepackage[english]{babel}
\usepackage{amsmath}
\usepackage{amssymb,amsfonts,textcomp}
\usepackage{array}
\usepackage{hhline}
\usepackage{hyperref}

\newenvironment{styleStandard}{}{}
\newenvironment{stylelsAbstract}{}{}
\newenvironment{stylelsSectioni}{}{}
\newenvironment{stylelsSectionii}{}{}
\newenvironment{stylelsBulletList}{}{}
\newenvironment{styleBibliographyi}{}{}
\newenvironment{listWWNumxxvleveli}{}{}
\newenvironment{listWWNumxxvlevelii}{}{}
\newenvironment{listWWNumxxvleveliii}{}{}
\newenvironment{listWWNumxxvleveliv}{}{}
\newenvironment{listWWNumixleveli}{}{}
\newenvironment{listWWNumixlevelii}{}{}
\newenvironment{listWWNumixleveliii}{}{}
\newenvironment{listWWNumixleveliv}{}{}

\newcommand\textstyleListLabelxvi[1]{#1}
\newcommand\labellistWWNumxxvleveli{\thelistWWNumxxvleveli.}
\newcommand\labellistWWNumxxvlevelii{\thelistWWNumxxvlevelii.}
\newcommand\labellistWWNumxxvleveliii{\thelistWWNumxxvleveliii.}
\newcommand\labellistWWNumxxvleveliv{\thelistWWNumxxvleveliv.}
\newcommand\labellistWWNumixleveli{[F0B7?]}
\newcommand\labellistWWNumixlevelii{\textstyleListLabelxvi{o}}
\newcommand\labellistWWNumixleveliii{[F0A7?]}
\newcommand\labellistWWNumixleveliv{[F0B7?]}

\newcounter{listWWNumxxvleveli}
\newcounter{listWWNumxxvlevelii}[listWWNumxxvleveli]
\newcounter{listWWNumxxvleveliii}[listWWNumxxvlevelii]
\newcounter{listWWNumxxvleveliv}[listWWNumxxvleveliii]
\newcounter{itemize}  
 \input{../localhyphenation} 
 \togglepaper[1]%%chapternumber
}{}

\begin{document}
\maketitle


\section{Introduction to Thulung}
\label{sec:Lahaussois:1}

Thulung is a language of the Kiranti subgroup of Sino-Tibetan/Trans-Himalayan, spoken by several thousand speakers in Solukhumbu district in Eastern Nepal (across the villages of Mukli, Deusa, Kangel, Lokhim, Jubu, Panchan, Salle, Necha); see \figref{fig:Lahaussois:1}.\footnote{I am grateful to Guillaume Jacques, to the editors and to an anonymous reviewer for very helpful comments on an earlier version of this chapter.  I also thank Chandrakala Rai for her continued help and patience.}  The language is exclusively oral, although missionary efforts over the past twenty years have resulted in the translation of the Old Testament, transcribed in an adaptation of Devanagari.\footnote{The main adaptations concern the phonemes /ʉ/ and /ɵ/; vowel length is not transcribed.}  The data discussed herein comes from fieldwork I have carried out on Thulung since 1999.

Like other Kiranti languages, Thulung is in close contact with Nepali (Indo-Aryan), the national language of Nepal, resulting in a number of calqued constructions.


\begin{figure}
\includegraphics[width=\textwidth]{figures/Nepal.pdf}
\caption{Map of the Kiranti-speaking area, adapted from \citet{Schlemmer2019}}
\label{fig:Lahaussois:1}
\end{figure}


\section{Basics of Thulung morphosyntax}
\label{sec:Lahaussois:2}

While a minimal Thulung sentence can consist of a single finite verb, arguments are often present in the form of pronouns or full noun phrases. Arguments are identified on the basis of case-marking and indexation.

This section presents the personal pronouns of Thulung \sectref{sec:Lahaussois:2.1}, and case-marking and indexing of core arguments \sectref{sec:Lahaussois:2.2}, both important preliminaries to understanding the language's reflexive constructions.

\subsection{Personal pronouns}
\label{sec:Lahaussois:2.1}

Personal pronouns exhibit person, number (singular, dual, plural), clusivity and formality contrasts. The paradigm is shown in \tabref{tab:Lahaussois:1}.


\begin{table}
\begin{tabularx}{0.73\textwidth}{lp{2.5cm}p{2.5cm}p{2cm}} 
\lsptoprule
& \textsc{sg} & \textsc{du} & \textsc{pl}\\
\midrule
1 & go & gutsi (\textit{incl.})

gutsuku (\textit{excl}.) & gui (\textit{incl}.)

guku (\textit{excl}.)\\
2 & gana (\textit{inform}.)

gani (\textit{formal}) & gatsi & gani(mim)\\
3 & gu (\textit{inform}.)

gumi (\textit{formal}) & gutsi & gumi(mim)\\
\lspbottomrule
\end{tabularx}
%%please move \begin{table} just above \begin{tabular
\caption{Personal pronouns of Thulung}
\label{tab:Lahaussois:1}
\end{table}

The formality contrast in 2nd and 3rd person singular pronouns is a relatively new phenomenon.  An earlier description \citep{Allen1975} reports 2\textsc{sg} \textit{gana}, 2\textsc{PL} \textit{gani}, 3\textsc{sg} \textit{gu} and 3\textsc{PL} \textit{gumi}.  It appears that new formal pronouns were created for the 2\textsc{sg} and 3\textsc{sg} by coopting the 2\textsc{PL} and 3\textsc{PL} pronouns, respectively, some time between Allen's fieldwork and the start of my own in 1999.  The creation of new plural pronouns through suffixation of the nominal pluralizer -\textit{mim} filled the resulting gap in the pronoun system, even though speakers currently tend to use both new and old plural forms with about equal frequency \citep{Lahaussois2003}.

\subsection{Case-marking and indexing of core arguments}
\label{sec:Lahaussois:2.2}

Core arguments are identified through case-marking and through argument indexation, which are conditioned by the referential hierarchy (e.g. \citealt{Silverstein1976}; \citealt{DeLancey1981}) in \REF{ex:Lahaussois:1}:

\ea%1
    \label{ex:Lahaussois:1}
    1>2>3>human>non-human animate>inanimate
    \z

           

  Thulung has a split ergative case-marking system, with the split occurring within the person section of the hierarchy. When acting as A arguments, two case-marking possibilities exist: 1st singular, 2nd singular, 2nd dual persons are nominative-marked (i.e. unmarked); this is what is seen in \REF{ex:Lahaussois:3} and \REF{ex:Lahaussois:5} below. Other A arguments, namely 2nd plural, 3rd persons and other NPs, are ergative-marked (with -\textit{ka}),\footnote{The unusual position of the split, within the 2nd person, can be explained through the creation of new plural pronouns with suffixation of the nominal pluralizer -\textit{mim}.  Presumably, -\textit{mim}, previously only found with 3rd person-like NPs, triggered ergative-marking on the new 2\textsc{PL} form \textit{gani-mim} (through analogy with other -\textit{mim}{}-marked NPs).}  as is seen in \REF{ex:Lahaussois:4} and \REF{ex:Lahaussois:6} below. 

  Object arguments also have differential marking, with the split occurring within the animacy part of the referential hierarchy. The dative marker -\textit{lai} (glossed \textsc{dat}), borrowed from Nepali, appears on primary objects ("an indirect object in a ditransitive clause or a direct object in a monotransitive clause", \citealt[808]{Dryer1986}) characterized by animacy: it is generally found with high-status humans (see \REF{ex:Lahaussois:5} and \REF{ex:Lahaussois:6} below), and only optionally with low status humans (e.g. children) and occasionally animals ('dog' is unmarked in \REF{ex:Lahaussois:3} and \REF{ex:Lahaussois:4}).  Inanimate objects are almost never marked. (For some discussion of primary object marking across Tibeto-Burman and its semantic nature, see \citealt{LaPolla1992}).

  A few alignment patterns illustrate the marking of core arguments, encoded as follows: 'S-ø V-s' translates to mean that the S argument is unmarked, and the verb (V) takes indexation for the single S argument; Similarly, with transitive scenarios, 'A-ø P-ø V-a>p' is to be translated as two unmarked A and P arguments and a verb with indexation for A and P.

  

\ea%2
    \label{ex:Lahaussois:2}
    S-ø V-s:\\
    \gll gu      khor\\
        3\textsc{sg}[{}-ø]   snore[.3\textsc{sg}]\\
    \glt ‘He snores.'
    \z

  

\ea%3
    \label{ex:Lahaussois:3}
    A-ø P-ø V-a>p:\\
    \gll go khlea jal-u\\
        1\textsc{sg[}{}-ø]   dog[-ø]   strike-\textsc{1sg>3sg}\\
    \glt ‘I strike the dog.'
    \z

           

  

  

  

\ea%4
    \label{ex:Lahaussois:4}
    A-ka P-ø V-a>p:\\
    \gll gu-ka khlea jal-ʉ\\
         3\textsc{sg-erg} dog[-ø] strike-\textsc{3sg>3sg}\\
    \glt ‘He strikes the dog.'
    \z

\ea%5
    \label{ex:Lahaussois:5}
    A-ø P-lai V-a>p:\\
    \gll go me  mʉtsʉ-lai jal-u\\
        1\textsc{sg[}{}-ø]    \textsc{dem}   man\textsc{{}-dat}   strike-\textsc{1sg>3sg}\\
    \glt ‘I strike that man.'
    \z


  

\ea%6
    \label{ex:Lahaussois:6}
    A-ka P-lai V-a>p:\\
    \gll gu-ka go-lai jal-ŋi\\
        \textsc{3sg-erg} \textsc{1sg-dat}   strike\textsc{{}-3sg>1sg}\\
    \glt ‘He strikes me.'
    \z
    
Thulung verbs index up to two arguments\footnote{These are the A and P in a monotransitive scenario, and either the A and R (for secundative verbs) or the A and T (for indirective verbs) for ditransitive scenarios.} on verbs, with a series of intransitive person indexes and a series of transitive indexes. Verbs are often labile, with the same root occurring with either transitive or intransitive indexes, and bringing about changes to argument structure and semantics.


\section{Reflexive strategy and uses}
\label{sec:Lahaussois:3}

This section will present the reflexive voice marker \sectref{sec:Lahaussois:3.1}, additional uses of the same marker \sectref{sec:Lahaussois:3.2}, and raise the question of the role of the emphatic nominal in reflexive constructions \sectref{sec:Lahaussois:3.3}.

\subsection{Reflexive voice marking}
\label{sec:Lahaussois:3.1}

Thulung has a reflexive voice marker, -\textit{siʈ} (and allomorphs -\textit{si}, -\textit{sin}, -\textit{sik}), which is the primary strategy for expressing agent-patient coreference. It occurs in a specific slot of the verbal template, and can been reconstructed to proto-Kiranti \textit{*-nši} (\citealt{VanDriem1990}: 47).  The reflexive voice marker has a number of functions: the central one is the reduction of the valency of the verb,\footnote{Note however that in Thulung, as in related Khaling, intransitive verbs can sometimes be reflexivized (see (\citealt{Lahaussois2016}: §3.3.5; \citealt{JacquesEtAl2016}§4.2.2; \citealt{Jacques2015}).} which can thereby only take intransitive indexes. This brings about the interpretation of the agent and patient, neither of which is necessarily overtly expressed (although the agent is overtly expressed in \REF{ex:Lahaussois:7}--\REF{ex:Lahaussois:9}), as coreferential.

\ea%7
    \label{ex:Lahaussois:7}
    \gll go \textbf{sɵl-si-ŋu}{}-mim tsʌŋra tel-ka \textbf{klʌ:-si-ŋu}\\
        \textsc{1sg} \textbf{wash-\textsc{refl-1sg}}\textsc{{}-nmlz} after oil-\textsc{instr} \textbf{rub-\textsc{refl-1sg}}\\
    \glt ‘After I wash, I rub myself with oil.'
    \z

           
\ea%8
    \label{ex:Lahaussois:8}
    \gll go oram-nuŋ \textbf{tseŋ-si-ŋu}\\
        \textsc{1sg} \textsc{dem.prox-com}   \textbf{hang-\textsc{refl-1sg}}\\
    \glt ‘I will hang (myself) onto this.' [holding onto to a monkey's tail to escape a prison]
    \z

\ea%9
    \label{ex:Lahaussois:9}
    \gll meram mʉtsʉ u-twap-ŋa \textbf{sen-s-ta}\\
        \textsc{dem}   man   \textsc{3sg.poss}{}-self-\textsc{int}  \textbf{kill-\textsc{refl-3sg.pst}}\\
    \glt ‘The man killed himself.' (elicited)
    \z
    

Sentence \REF{ex:Lahaussois:9} could equally well be formulated with a 3\textsc{sg} pronoun subject as in \REF{ex:Lahaussois:10}:



\ea%10
    \label{ex:Lahaussois:10}
    \gll gu u-twap-ŋa sen-s-ta\\
        \textsc{3sg} \textsc{3sg.poss}{}-self-\textsc{int}   kill-\textsc{refl-3sg.pst}\\
    \glt ‘He killed himself.' 
    \z

It is interesting to contrast this with the expression of the object pronoun in situations of disjoint reference, which in this case would yield \REF{ex:Lahaussois:11}: 

\ea%11
    \label{ex:Lahaussois:11}
    \gll gu-ka meram-lai seʈ-ɖʉ \\
        \textsc{3sg-erg} \textsc{dem-dat} kill-\textsc{3sg>3sg.pst}\\
    \glt ‘She killed him.'  
    \z

Note that the use of the distal demonstrative \textit{meram} as the object pronoun makes it clear that this is a case of disjoint reference, with the distal deixis establishing otherness.


While there is a single reflexive voice marker, which is obligatory in reflexive constructions and occurs in full paradigms (see \tabref{tab:Lahaussois:2}), with no restrictions as to person/number and tense, there is an older, no-longer productive reflexive marker, the reflex of which is found in many verbs with middle semantics. This older reflexive marker only surfaces in partial paradigms, as an \textit{{}-s} on the verb stems that occur with 1\textsc{pi} and 3\textsc{sg} forms (see grey cells in \tabref{tab:Lahaussois:3}), and verbs where it appears are now considered to form an inflectional class (the s-stem class). Interestingly, this class is not limited to intransitive verbs: transitive verbs are also found (bearing transitive indexes), though in considerably lower numbers than intransitives.

  The two paradigms are contrasted below:




\begin{table}
\begin{tabularx}{0.5\textwidth}{lll}
\lsptoprule
& \textsc{npst} & \textsc{pst}\\
\midrule
\textsc{1sg} & khlo:-si-ŋu & khlo:-si-ŋro\\
\textsc{1di} & khlo:-si-tsi & khlo:-siʈ-tsi\\
\textsc{1de} & khlo:-si-tsuku & khlo:-siʈ-tsoko\\
\textsc{1pi} & khlo:-sir-i & khlo:-siʈ-ɖi\\
\textsc{1pe} & khlo:-sin-ku & khlo:-siʈ-toko\\
\textsc{2sg} & khlo:-si-na & khlo:-siʈ-na\\
\textsc{2du} & khlo:-si-tsi & khlo:-siʈ-tsi\\
\textsc{2pl} & khlo:-si-ni & khlo:-siʈ-ni\\
\textsc{3sg} & khlo:-si & khlo:-siʈ-ɖa\\
\textsc{3du} & khlo:-si-tsi & khlo:-siʈ-tsi\\
\textsc{3pl} & khlo:-si-mi & khlo:-si-mri\\
\lspbottomrule
\end{tabularx}
%%please move \begin{table} just above \begin{tabular
\caption{Paradigm for verb \textit{khlo:simu}, 'return', with the reflexive voice marker -\textit{si} (or allomorphs) in all forms}
\label{tab:Lahaussois:2}
\end{table}




\begin{table}
\begin{tabularx}{0.37\textwidth}{lll} 
\lsptoprule
& \textsc{npst} & \textsc{pst}\\
\midrule
\textsc{1sg} & se-ŋu & se-ŋro\\
\textsc{1di} & se-tsi & se-ttsi\\
\textsc{1de} & se-tsuku & se-ttsoko\\
\textsc{1pi} & ses-i & ses-ti\\
\textsc{1pe} & se-ku & se-ttoko\\
\textsc{2sg} & se-na & se-nna\\
\textsc{2du} & se-tsi & se-ttsi\\
\textsc{2pl} & se-ni & se-nni\\
\textsc{3sg} & se & ses-ta\\
\textsc{3du} & se-tsi & se-ttsi\\
\textsc{3pl} & se-mi & se-mri\\
\lspbottomrule
\end{tabularx}
%%please move \begin{table} just above \begin{tabular
\caption{Paradigm for verb \textit{semu}, ‘fart', with -\textit{s} only surfacing in forms in grey cells}
\label{tab:Lahaussois:3}
\end{table}

However, because the productive reflexive voice marker is optionally phonologically reduced to -\textit{s}, this sometimes leads to identical forms between the paradigms of reflexively-marked verbs and the s-stem class verbs of \tabref{tab:Lahaussois:3}, namely in the 1\textsc{pi} and 3\textsc{sg} forms. An example of the variant form of the reflexive voice marker is seen in \REF{ex:Lahaussois:9} above: instead of the expected \textit{sen-siʈ-ɖa} (kill-\textsc{refl}-3\textsc{sg}.\textsc{pst}), we have \textit{sensta.} While this form may look like it belongs to a paradigm such as that in \tabref{tab:Lahaussois:3}, it is in fact a variant form of an otherwise well-behaved reflexively-derived verb. (For more detailed discussion, see \citealt{Lahaussois2011, Lahaussois2016}).

\subsection{Other uses of the reflexive voice marker}
\label{sec:Lahaussois:3.2}

The reflexive voice marker has a number of other uses: it can also mark reciprocal, middle, antipassive and anticausative functions. I retain ‘\textsc{refl}' as a gloss for the marker across its different uses, as an indication of what I consider to be the core function. 

With a reciprocal function, the utterance must contain a non-singular subject.  Semantics is important to interpretation; in \REF{ex:Lahaussois:13}, without the reduplicated emphatic nominal, the utterance would be ambiguous as to a reciprocal vs a reflexive interpretation (which would be something like ‘twist themselves up'). 

\ea%12
    \label{ex:Lahaussois:12}
    \gll  mɵr-tsip mamtha \textbf{phwa-sit-ʦi}\\
        that-\textsc{du} last.year \textbf{separate-\textsc{refl-3du.pst}}\\
    \glt ‘They separated last year.'
    \z

\ea%13
    \label{ex:Lahaussois:13}
    \gll mɵmim twap-twap \textbf{bal-si-mi}\\
        3\textsc{pl} self-\textsc{red} \textbf{wind-\textsc{refl-3pl}}\\
    \glt ‘They are tangled together.'
    \z

Example \REF{ex:Lahaussois:13} can be contrasted with \REF{ex:Lahaussois:14}, which features a reflexive form of the same verb.

\ea%14
    \label{ex:Lahaussois:14}
    \gll memlo u-lʉ-ɖra u-mam-ku sem \textbf{bal-sik-}pa mini-ka lwas-tʉ ʔe\\
        then \textsc{3sg}\textsc{\textsubscript{2}}\textsc{.poss}{}-tooth-\textsc{loc} \textsc{3sg}\textsc{\textsubscript{1}}\textsc{.poss}{}-mother-\textsc{gen} hair \textbf{wind-\textsc{refl-}}\textsc{ptcp}   human\textsubscript{1}{}-\textsc{erg}   see-\textsc{3sg}\textsc{\textsubscript{1}}\textsc{>3sg.pst} \textsc{hs}\\
    \glt ‘Then the human\textsubscript{1} saw his\textsubscript{1} mother's hair that had wound itself around his\textsubscript{2} tooth.'
    \z

The line between a reflexive and a middle interpretation can be a fine one, but the following examples are of "situations where there is no clear distinction between the ‘doer' and the one ‘being done to'" (\citealt{LaPolla2003}: 36); see also \citealt[Chapter3]{Kemmer1993}), and are considered middles. Example \REF{ex:Lahaussois:15} is an example of non-translational motion, \REF{ex:Lahaussois:16}, of change in body posture.

\ea%15
    \label{ex:Lahaussois:15}
    \gll a-rɵm nɵ-ra-ma go \textbf{ki-si-ŋro}\\
        \textsc{1sg.poss}{}-body hurt-\textsc{3sg.pst-conj} \textsc{1sg} \textbf{pull.tight-\textsc{refl-1sg.pst}}\\
    \glt ‘My body hurt and I stretched.'
    \z

\ea%16
    \label{ex:Lahaussois:16}
    \gll lamʦoko-ra ʦɵtʦɵ-mim \textbf{ther-si-mri}\\
        door-\textsc{loc} child-\textsc{plu} \textbf{lean-\textsc{refl-3pl.pst}}\\
    \glt ‘The children were leaning on the door.'
    \z

With an antipassive use, the patient argument of the underived sentence becomes an oblique argument, a fact which is reflected in the case markers it takes on after derivation: comitative -\textit{nuŋ} or ablative \textit{{}-ram} \REF{ex:Lahaussois:17b}, or locative -\textit{ra} \REF{ex:Lahaussois:18b}. In the underived examples with the same base verbs in \REF{ex:Lahaussois:17a} and \REF{ex:Lahaussois:18a}, \textit{go} and \textit{mandir} are patient arguments.  The change in case-marking is accompanied by a change in the indexes on the verb, which are intransitive, indexing the S, after derivation.

\ea%17
    \label{ex:Lahaussois:17}
    \ea
    \label{ex:Lahaussois:17a}
    \gll gu-ka go-lai ghram-ŋi\\
        \textsc{3sg-erg} \textsc{1sg-dat} feel.disgust-\textsc{3sg>1sg}\\
    \glt ‘He is disgusted by me.'
    
    \ex
    \label{ex:Lahaussois:17b}
    \gll gumi bira-nuŋ/-ram \textbf{ghram-si-mi}\\
        3\textsc{pl} leech-\textsc{com/-abl} \textbf{feel.disgust-\textsc{refl-3pl}}\\
    \glt ‘They are disgusted by leeches.'
\z
\z

\ea%18
    \label{ex:Lahaussois:18}
    \ea
    \label{ex:Lahaussois:18a}
    \gll gu-ka mandir khir-ʉ\\
        \textsc{3sg-erg} temple circumambulate-3\textsc{sg>3sg}\\
    \glt ‘He circles the temple.'
    
    \ex
    \label{ex:Lahaussois:18b}
    \gll gu mandir-ra \textbf{khir-si}\\
        \textsc{3sg} temple-\textsc{loc} \textbf{circumambulate-\textsc{refl.3sg}}\\
    \glt ‘He circles around at the temple.'
\z
\z


When there is no clear external cause for the action, an anticausative interpretation results. This is the case with the reflexive-marked verb in \REF{ex:Lahaussois:19}. 

\ea%19
    \label{ex:Lahaussois:19}
    \gll dʌksa \textbf{ʦar-siʈ-ɖa}\\
        tree \textbf{make.fall-\textsc{refl-3sg.pst}}\\
    \glt ‘The tree fell.'
    \z



\subsection{Reflexive or emphatic nominal?}
\label{sec:Lahaussois:3.3}

While the primary reflexivization strategy in Thulung is clearly verbal, the language has an emphatic nominal, \textit{twak} or \textit{twap,} which is optionally used in some reflexive constructions, as in \REF{ex:Lahaussois:20}. 

\ea%20
    \label{ex:Lahaussois:20}
    \gll (u-twap tsʌi) thʌ-s-ta\\
        \textsc{(3sg.poss}{}-self \textsc{contr)} hide-\textsc{refl-3sg.pst}\\
    \glt ‘He hid (himself).'
    \z

This nominal, which can be translated as ‘self', often takes possessive indexes, as in the following paradigm.




\begin{table}
\begin{tabularx}{0.7\textwidth}{lp{1cm}p{1.5cm}p{1.5cm}} 
\lsptoprule
& \textsc{sg} & \textsc{du} & \textsc{pl}\\
\midrule
1 & a-twap & atsi-twap (incl.)

itsi-twap (excl.) & aki-twap (excl.)

iki-twap (incl.)\\
2 & i-twap & itsi-twap & ini-twap\\
3 & u-twap & u-tsitwap & uni-twap\\
\lspbottomrule
\end{tabularx}
%%please move \begin{table} just above \begin{tabular
\caption{Emphatic nominal paradigm (possessive pronoun + ‘self')}
\label{tab:Lahaussois:4}
\end{table}

There is an additional set of adnominal possession markers: the possessive indexes in \tabref{tab:Lahaussois:4} combine with a nominalizer -\textit{ma}, generating a full set with person/number/clusivity contrasts; these nominalized forms are used attributively, preceding the noun they modify.  We thus have \textit{ama} \textit{twap}, 1\textsc{sg}.\textsc{poss} self, ‘my self' as well as \textit{a-twap}, 1\textsc{sg}.\textsc{poss}-self, ‘myself' used interchangeably.

  Emphatic nominals are not obligatory with most reflexive constructions, and are often found in scenarios where there is no coreference, as in \REF{ex:Lahaussois:21}:

\ea%21
    \label{ex:Lahaussois:21}
    \gll u-twak-ka dwak-ʉ{}-m-num bia bo-m-sa-mu\\
        \textsc{3sg.poss}{}-self-\textsc{erg} like-\textsc{3sg>3sg.nmlz-com} marriage do-\textsc{inf-appl-inf}\\
    \glt ‘They should marry her to someone she herself likes.'
    \z

Nonetheless, in certain reflexive-voice-marked scenarios, the emphatic nominal can used as well. This is the case with \REF{ex:Lahaussois:22} below.  

\ea%22
    \label{ex:Lahaussois:22}
    \gll me kʌlʌs-ram ku-ka twap prʌn-si-mu ba:si\\
       \textsc{dem} Kales-\textsc{abl} water-\textsc{instr} self sprinkle-\textsc{refl-inf} must \\
    \glt ‘Each person must sprinkle himself with water from the Kales.'
    \z

\section{Coreference with different verb types}
\label{sec:Lahaussois:4}

This section explores the expression of coreference with different verb types: body care and grooming verbs \sectref{sec:Lahaussois:4.1}, and extroverted verbs \sectref{sec:Lahaussois:4.2}.

\subsection{Body care/grooming verbs}
\label{sec:Lahaussois:4.1}

Verbs of grooming and body care can be divided into those affecting only part of the body, and those affecting the whole body.

Body-part actions can be expressed either by means of reflexivized verbs or transitive constructions. Example \REF{ex:Lahaussois:23} illustrates two body-part actions expressed through reflexivized verbs. 

\ea%23
    \label{ex:Lahaussois:23}
    \gll \textbf{hur-si-ri}-mim tsʌŋra  bui-ɖʌ:la tel-ka \textbf{klʌ:-si-mu} ba:si\\
        \textbf{wash.head-\textsc{refl-1pi}}\textsc{{}-nmlz} after head-on oil-\textsc{instr} \textbf{apply-\textsc{refl-inf}} must\\
    \glt ‘After we wash our hair, we must apply oil [to our heads].'
    \z

Transitive constructions, with the object possessively marked or not, can also be used. Example \REF{ex:Lahaussois:24} illustrates this alternative construction with the same (first) verb as in \REF{ex:Lahaussois:23}.

\ea%24
    \label{ex:Lahaussois:24}
    \gll go a-sem hur-pu-ma ɖʉʈ-pu\\
        \textsc{1sg} \textsc{1sg.poss}{}-hair wash-\textsc{1sg>3sg-conj}   comb.hair-\textsc{1sg>3sg}\\
    \glt ‘I wash my hair and comb it.'
    \z

Example \REF{ex:Lahaussois:25} shows a transitive construction used for a body-part action, without possessive marking on the body part; the equivalent whole-body action can be seen in \REF{ex:Lahaussois:7}, with obligatory reflexive voice marking.

\ea%25
    \label{ex:Lahaussois:25}
    \gll go lwa ʣɵmka sɵl-pu\\
        1\textsc{sg} hand carefully wash-1\textsc{sg}>3\textsc{sg}\\
    \glt ‘I wash my hands carefully.'
    \z

Thulung also expresses some body-part actions through the following deponent verbs (as per \citealt[22]{Kemmer1993}), for which no base verb currently exists: \textit{hi:simu}, ‘turn body or head'; \textit{khusimu}, ‘wear on head'; \textit{khlʉsimu}, ‘wear on feet'.

Whole-body actions, typically dressing and bathing, are always reflexively marked, as illustrated in examples \REF{ex:Lahaussois:26}--\REF{ex:Lahaussois:28}.

\ea%26
    \label{ex:Lahaussois:26}
    \gll to:si-ra tshɵm-ra lʌ:-mu-lai bwapme-mim tshɵm \textbf{blwa-siʈ-miri}\\
        Tosi.festival-\textsc{loc} dance-\textsc{loc} go-\textsc{inf-dat} housewife-\textsc{plu} much \textbf{dress.up-\textsc{refl-3pl.pst}} \\
    \glt ‘To go to dance at Tosi, the housewives dressed themselves up a lot.'
    \z

\ea%27
    \label{ex:Lahaussois:27}
    \gll mʉ:sʉ ku-gui \textbf{plʉm-siʈ-ɖa}\\
        buffalo water-into \textbf{immerse-\textsc{refl-3sg.pst}}\\
    \glt ‘The buffalo immersed itself in the water.'
    \z

\ea%28
    \label{ex:Lahaussois:28}
    \gll go nepsuŋ-ra \textbf{blaŋ-siŋ-ro}\\
    1\textsc{sg} sun-\textsc{loc} \textbf{dry-\textsc{refl-1sg.pst}}\\
    \glt ‘I dried myself in the sun.'
    \z


\subsection{Extroverted verbs}
\label{sec:Lahaussois:4.2}

With reflexivized extroverted verbs ("those which denote an action typically performed on others", \citealt[73]{Haiman1998}), subjects are nominative case-marked and verbs take intransitive indexes and are reflexively marked.  Additionally, they tend to include the emphatic nominal, as seen in \REF{ex:Lahaussois:29}, as well as in \REF{ex:Lahaussois:9}, which includes another extroverted verb.

\ea%29
    \label{ex:Lahaussois:29}
    \gll khlea u-twap-ŋa khren-si\\
        dog \textsc{3sg.poss}{}-self-\textsc{int}   bite-\textsc{refl.3sg}\\
    \glt ‘The dog bites itself.' (elicited)
    \z


\subsection{Coreference of subject with different semantic roles}
\label{sec:Lahaussois:5}
\subsection{Possessors}
\label{sec:Lahaussois:5.1}

Thulung uses the same coding system for possessors, whether or not there is coreference between the subject and the possessor: possession is marked with a possessive index on the possessed noun and/or a genitive case marker on the possessor.\footnote{This yields the following possibilities: \textit{[mam-ku]} \textit{(u-)khel,} [mother-\textsc{gen}] (3\textsc{sg}.\textsc{poss}-)leg, [mother's] (her-)leg}  

Examples \REF{ex:Lahaussois:30} and \REF{ex:Lahaussois:31} illustrate the same adnominal possession marking (prefix \textit{u}{}-, for 3\textsc{sg.poss}) used to mark possession which is coreferential with the subject (in the first occurrence in each sentence), and coreferential with the patient (in the second occurrence in each sentence). This shows quite clearly that Thulung has no special adnominal possessor form for coreference with the subject. 

\ea%30
    \label{ex:Lahaussois:30}
    \gll \textbf{u-bʌdzʌi}{}-lai thoŋ-koʈ-ɖʉ ʔe me thʌŋki-ka \textbf{u-kʌl}{}-bʌri\\
        \textbf{\textsc{3sg}}\textbf{\textsc{\textsubscript{1}}}\textbf{\textsc{.poss}}\textbf{{}-grandmother}{}-\textsc{dat} \textsc{ideo}{}-spray-\textsc{3sg}\textsc{\textsubscript{1}}\textsc{>3sg}\textsc{\textsubscript{2}}\textsc{.pst} \textsc{hs} \textsc{dem} resin-\textsc{instr} \textbf{\textsc{3sg}}\textbf{\textsc{\textsubscript{2}}}\textbf{\textsc{.poss}}\textbf{{}-face}{}-all.over\\
    \glt ‘He sprayed \textbf{his} \textbf{grandmother} suddenly with resin, all over \textbf{her} \textbf{face}.'
    \z

The first instance of \textit{u-} \textit{(u-bʌdzai,} ‘his grandmother') is coreferential with the subject (not overtly expressed, but present in the discourse and indexed on the verb), whereas the second (\textit{u-kʌl}, \textsc{3sg.poss}{}-face) refers instead to the grandmother as possessor, and is thus coreferential with the object. The coreference is indicated with subscript numbers in the glosses.

A similar situation is found in \REF{ex:Lahaussois:31}, although it is made up of two sequential utterances:

\ea%31
    \label{ex:Lahaussois:31}
    \gll me \textbf{u-khel} tsʌi hoŋka ɖʌs-tʉ ʔe me khola-go-jʉ     tsobethaʈ-ɖʉ ʔe ; me ŋo-ka ne me \textbf{u-khel} khreʈ-ɖa geʈ-ɖa retsʌ ʔe\\
        \textsc{dem}   \textbf{\textsc{3sg}}\textbf{\textsc{\textsubscript{1}}}\textbf{\textsc{.poss-}}\textbf{leg}  \textsc{contr}  like.this  move-\textsc{3sg}\textsc{\textsubscript{1}}\textsc{>3sg.pst}  \textsc{hs}   \textsc{dem} river-inside-\textsc{low.loc}  dip-\textsc{3sg}\textsc{\textsubscript{1}}\textsc{>3sg.pst}  \textsc{hs} ; \textsc{dem} fish-\textsc{erg} \textsc{top}   \textsc{dem} \textbf{\textsc{3sg}}\textbf{\textsc{\textsubscript{1}}}\textbf{\textsc{.poss-}}\textbf{leg} bite-\textsc{purp}   come-\textsc{3sg}\textsc{\textsubscript{2}}\textsc{.pst}  it.seems  \textsc{hs}\\
    \glt ‘He moved \textbf{his} \textbf{legs} like this, he dipped them into the river, and that fish came to bite   \textbf{his} \textbf{legs}.'
    \z

The two relevant possessed nouns in \REF{ex:Lahaussois:31} are the two occurrences of \textit{khel} ‘leg': the first occurrence is coreferential with the (unexpressed, but indexed on the verb) subject of the verb \textit{ɖʌstʉ}; in the second occurrence, the subject is the (overtly expressed and ergative-marked) \textit{ŋo} ‘fish', and there is no coreference between the subject and the possessor of \textit{khel} ‘leg'.

 It might be suspected that the possession of body parts and kin terms in \REF{ex:Lahaussois:30} and \REF{ex:Lahaussois:31} potentially has an impact on the possessive index, but this is not the case: in \REF{ex:Lahaussois:32}, \textit{u}{}- alone marks possession by the subject of the utterance.

\ea%32
    \label{ex:Lahaussois:32}
    \gll \textbf{u-ʈa:rbar} khjarerere thʉʈ-to jokta ʔe\\
        \textbf{\textsc{3sg}}\textbf{\textsc{\textsubscript{1}}}\textbf{\textsc{.poss}}\textbf{{}-machete} scraping.sound pull-\textsc{sim.cvb} go.down-\textsc{3sg}\textsc{\textsubscript{1}}\textsc{.pst}  \textsc{hs}\\
    \glt ‘He went down, pulling \textbf{his} \textbf{machete} with a scraping sound.'
    \z

In situations where coreference between the subject and possessor must be definitively established, the emphatic nominal \textit{twap} is used, in which case no ambiguity remains.  Thus while \REF{ex:Lahaussois:33} can be used for both situations with coreference and disjoint reference between the subject pronoun and the nominal adpossessor, \REF{ex:Lahaussois:34} can only be interpreted as coreferential.

\ea%33
    \label{ex:Lahaussois:33}
    \gll gu-ka uma khe:sa seʈ-ɖʉ\\
       3\textsc{sg}\textsc{\textsubscript{1}}\textsc{{}-erg} \textsc{3sg}\textsc{\textsubscript{1/2}}\textsc{.poss} lover   kill-\textsc{3sg>3sg.pst} \\
    \glt ‘She\textsubscript{1} killed her\textsubscript{1/2} lover.'
    \z



\ea%34
    \label{ex:Lahaussois:34}
    \gll gu-ka uma twak-ku khe:sa seʈ-ɖʉ\\
        3\textsc{sg}\textsc{\textsubscript{1}}\textsc{{}-erg} 3\textsc{sg}\textsc{\textsubscript{1}}\textsc{.poss} self-\textsc{gen} lover kill-\textsc{3sg}\textsc{\textsubscript{1}}\textsc{>3sg.pst}\\
    \glt ‘She killed her own lover.'
    \z 

The coding of possessors is also relevant to the expression of coreference between two non-subject arguments of a single clause: because there is no special possessive marker for coreference, such situations are also potentially ambiguous (and can be disambiguated using an emphatic nominal), as in the following:

\ea%35
    \label{ex:Lahaussois:35}
    \gll jeluŋ-ka bala-nuŋ uma (twak-ku) du:tham sɵ-ʉrʉ\\
        Jeluŋ-\textsc{erg} Bala-\textsc{com} \textsc{3sg.poss} \textsc{(3sg.poss}{}-self-\textsc{gen)} about tell-\textsc{3sg>3sg.pst}\\
    \glt ‘Yelung told Bala about herself.'
    \z


\subsection{Beneficiaries}
\label{sec:Lahaussois:5.2}

Coreference between agent and beneficiary (which I have referred to as ‘auto-benefactive' elsewhere \citealt{Lahaussois2016,JacquesEtAl2016}) is also expressed through reflexive voice marking on the verb.  This is illustrated in \REF{ex:Lahaussois:36} and \REF{ex:Lahaussois:37}.  

\ea%36
    \label{ex:Lahaussois:36}
    \gll go a-khe:sa mal-si-ŋro\\
        \textsc{1sg} \textsc{1sg.poss}{}-lover search-\textsc{refl-1sg.pst}\\
    \glt ‘I searched for a lover for myself.'
    \z


\ea%37
    \label{ex:Lahaussois:37}
    \gll go ama la:gi ko:-le humje bhre-ŋ-si-ŋro\\
        \textsc{1sg} \textsc{1sg.poss} sake \textsc{-cl} shawl buy-\textsc{1sg-refl-1sg.pst}\\
    \glt ‘I bought myself a shawl.'
    \z

The phrase \textit{ama} \textit{la:gi} ‘for my sake' in \REF{ex:Lahaussois:37} functions here like an emphatic, but is by no means necessary for the expression of coreference.  Note that it is a strategy for introducing a beneficiary in cases of non-coreference as well.

When the beneficiary is not coreferential with the agent, the additional non-agentive argument is usually brought into the argument structure through an applicative marker on the verb; the indexes on the verb are for the agent and the beneficiary argument. There are a few options to mark the beneficiary: dative marker -\textit{lai} (used for primary objects) as in \REF{ex:Lahaussois:38}, with the phrase \textit{{}-ku/-kam la:gi}, ‘for the sake of', where the beneficiary is the possessor, or through possessive marking on the theme, as in \REF{ex:Lahaussois:39}.

\ea%38
    \label{ex:Lahaussois:38}
    \gll gu-ka lwak-lai phadzi bhre-saʈ-ɖʉ\\
        \textsc{3sg-erg} younger.sibling-\textsc{dat} bag buy-\textsc{appl-3sg>3sg.pst}\\
    \glt ‘He bought a bag for his brother.'
    \z

\ea%39
    \label{ex:Lahaussois:39}
    \gll uma sʌŋ phar-saʈ-toko\\
        \textsc{3sg.poss} wood collectively.cut-\textsc{appl-1pe>3sg}\\
    \glt ‘We collectively cut his wood for him.'
    \z

Verbs which are not applicative-marked are however also found, and use the same strategies for coding the beneficiary, as in \REF{ex:Lahaussois:40} where both the applicativized and non-applicativized forms are found to be acceptable.

\ea%40
    \label{ex:Lahaussois:40}
    \gll mam-ka tsɵttsɵ-lai dzam khok-sa-mri/khok-tʉ\\
        mother-\textsc{erg} child\textsc{{}-dat} rice   cook-\textsc{appl-3pl>3sg.pst}/cook-\textsc{3sg>3sg.pst}\\
    \glt ‘Mother cooked rice for the child.'
    \z

\subsection{Recipients}
\label{sec:Lahaussois:5.3}

The expression of coreference between an agent and a recipient appears to be quite unnatural in Thulung.  Utterances can be produced during elicitation, but my corpus contains not one spontaneously produced example. 

Example \REF{ex:Lahaussois:41a} is contrasted with an equivalent example without subject-recipient coreference in \REF{ex:Lahaussois:41b}.

\ea
\label{ex:Lahaussois:41}
    \ea
    \label{ex:Lahaussois:41a}
    \gll mesem u-twap upʌhar gwa:-si\\
        girl \textsc{3sg.poss}{}-self gift give-\textsc{refl.3sg}\\
    \glt ‘The girl gives herself a present.' (elicited)
    
    \ex
    \label{ex:Lahaussois:41b}
    \gll mesem-ka ŋopsɵ-lai upʌhar gwak-ʉ\\
    girl-\textsc{erg} friend-\textsc{dat} present give-\textsc{3sg>3sg}\\
    \glt ‘The girl gives her friend a present.'
\z
\z

In \REF{ex:Lahaussois:41a}, the verb is detransitivized with \textit{{}-siʈ}, as expected, takes intransitive 3\textsc{sg} indexation, and the agent and recipient (expressed overtly through the emphatic nominal) take nominative case-marking, as they would in an intransitive scenario. Yet while sentences expressing coreference between a subject and a recipient are able to be produced in elicitation, in some cases they involve ergative-marked subjects with reflexivized verbs, and thus appear to be marginal. 

\section{Exact vs partial coreference}
\label{sec:Lahaussois:6}

In partial coreference, there is incomplete overlap between the agent and patient, a situation brought about when the reference involves a first or second person and one of the arguments encompasses a larger set than the other ("I see us"; "we (incl.) see you"). Because the main strategy for establishing agent-patient coreference in Thulung is the use of the reflexive voice marker, entailing detransitivization and the use of intransitive indexes, the expression of partial coreference is not possible: partial coreference would need to index both arguments, something that cannot be done with intransitive indexes.

  While some ditransitive verbs may look like they express partial coreference, this is in fact the result of the verb in question being an indirective and indexing the A and the T (rather than the R, as in secundative verbs). This is seen in \REF{ex:Lahaussois:42}, in which the verb indexes the subject (1\textsc{sg}) and \textit{dzam} (3\textsc{sg}) meaning ‘food', and not the recipient (1\textsc{di}).

\ea%42
    \label{ex:Lahaussois:42}
    \gll go gutsi-lai dzam pheʈ-pu\\
        \textsc{1sg} \textsc{1di-dat} food serve-\textsc{1sg>3sg}\\
    \glt ‘I will serve us (incl.) both food.'
    \z

Attempts at eliciting situations involving partial coreference result in a number of strategies:

a) altering the scenario to involve exact coreference: 

\ea%43
    \label{ex:Lahaussois:43}
    \gll gutsi ko:le-ŋa je hum-sin-tsi\\
        \textsc{1di} one-\textsc{cl-int} cloth wrap-\textsc{refl-1di}\\
    \glt ‘We wrap ourselves in the same blanket.' [intended: Wrap us (incl.) in the same cloth]\footnote{Examples \REF{ex:Lahaussois:43} and \REF{ex:Lahaussois:44} were inspired by examples provided in \citet{BickelEtAl2010}.}
    \z


\ea%44
    \label{ex:Lahaussois:44}
    \gll gutsi chatta-ka rim-sin-tsi\\
        \textsc{1di} umbrella-\textsc{instr} cover-\textsc{refl-1di}\\
    \glt ‘We cover ourselves with the umbrella.' [intended: You cover us (incl.) with the umbrella]
    \z


b) using a semantically-related intransitive to express the intended scenario

\ea%45
    \label{ex:Lahaussois:45}
    \gll gana me-dzɵpa ga:ri thɵn-na-ma:la gui si-i\\
        \textsc{2sg} \textsc{neg}{}-good car drive-\textsc{2sg>3sg-cond} \textsc{1pi} die-\textsc{1pi}\\
    \glt ‘If you drive the car badly, we will die.' [intended: You will kill us (incl.) all (scenario: driver driving dangerously)]
    \z

c) paraphrasing the scenario:

\ea%46
    \label{ex:Lahaussois:46}
    \gll gana go-lai wakha lamdi-beʈ-ŋi\\
        \textsc{2sg} \textsc{1sg-dat} slow walk-\textsc{caus-2sg>1sg.pst}\\
    \glt ‘You made me walk slowly.' [intended: you slowed us (incl.) down (by walking slowly)]
    \z

The examples above, illustrating Thulung strategies for solving problems of partial coreference, show that the language can only express exact coreference (reflexive voice marking and intransitive indexes) or completely disjoint reference (transitive indexes and appropriate case-marking for distinct argument roles). 

\section{Long-distance coreference}
\label{sec:Lahaussois:7}

Thulung uses direct speech as a means of embedding any quoted material, and this applies both to speech and to thinking. As a result, the establishment of coreference of the subject across clauses does not need to be expressed in such complement clauses: a 1st person form of a verb within the direct speech clause establishes coreference; any other person expresses disjoint reference in utterances. For a similar situation in Chantyal, see \citet{Noonan2006}.

\ea%47
    \label{ex:Lahaussois:47}
    \gll go mi-bi-ŋu rwak-ta\\
        \textsc{1sg} \textsc{neg}{}-come-\textsc{1sg} say-\textsc{3sg.pst}\\
    \glt ‘He said he wouldn't come.' [lit. He said "I won't come."]
    \z


\ea%48
    \label{ex:Lahaussois:48}
    \gll gu-ka ne seʈ-to rwak-pa mim-ɖʉ{}-m ba-ira \\
        \textsc{3sg-erg} \textsc{top} kill-\textsc{1sg>3sg.pst} say-\textsc{act.ptcp} think-\textsc{3sg>3sg.pst-nmlz} be-\textsc{pst} \\
    \glt ‘She had thought she killed him.' [lit. She had thought, saying "I killed him"]
    \z

These examples can be contrasted with a scenario where the use of non-1st person marking in the embedded clause firmly establishes disjoint reference between the subjects of the two clauses.

\ea%49
    \label{ex:Lahaussois:49}
    \gll [me:sem u-lwak ne seʈ-ɖʉ] wostsɵ u-wa:-ka mem rwak-ta-m\\
        female \textsc{3sg.poss}{}-younger.sibling \textsc{top} kill-\textsc{3sg>3sg.pst} male \textsc{3sg.poss}{}-older.sibling-\textsc{erg} \textsc{dem}  say-\textsc{3sg.pst-nmlz}\\
    \glt ‘The older brother thought that she [an ogre] had killed his younger sister.' [lit. "She   killed my younger sister" said the older brother.]
    \z


\section{Conclusion}
\label{sec:Lahaussois:8}

Thulung has a primary strategy for establishing agent-patient coreference: the use of a reflexive voice marker -\textit{siʈ}. Its use on a verb triggers intransitive subject indexes and nominative case-marking on the subject. In addition to reflexivity, the reflexive voice marker is used to mark reciprocity, middle scenarios, antipassives and anticausatives.  

A possessive indexed emphatic nominal can be optionally used to reinforce the expression of coreference. The use of the emphatic nominal is notably found in situations where coreference cannot be established through any specialized markers, such as with the marking of possession: adnominal possessive indexes are neutral as to coreference or disjoint reference with other arguments. One also finds the emphatic nominal in sentences with extroverted verbs, suggesting that such situation types require additional coding of the coreference.  It is also possible that Thulung is undergoing change with respect to its reflexive-voice-only strategy, and that the use of the emphatic nominal is on the rise, calquing the situation in Nepali.

  Additional interesting features of Thulung are the impossibility of marking partial coreference, and the fact that the language's embedding of direct speech makes the marking of coreference across complement clauses unnecessary.

\section*{Abbreviations}
\begin{tabularx}{.45\textwidth}{lQ}
\textsc{act.ptcp} & active participle\\
\textsc{appl} &   applicative \\
\textsc{caus} &   causative \\
\textsc{cl} &   classifier \\
\textsc{com} &   comitative \\
\textsc{cond} &   conditional \\
\textsc{conj} &   conjunction \\
\textsc{contr} & contrastive focus \\
\textsc{dat} &   dative \\
\textsc{di} &   dual inclusive \\
\textsc{de} &   dual exclusive \\
\textsc{dem} &   demonstrative (distal) \\
\textsc{dem.prox} &  proximal demonstrative\\
\textsc{du} &  dual\\
\textsc{erg} &  ergative \\
\textsc{gen} &   genitive\\
\textsc{hs} &   hearsay \\
\textsc{inf} &   infinitive\\
\textsc{instr} &   instrumental \\
\textsc{int} &   intensifier\\
\textsc{loc} &   locative\\
\textsc{low.log} & low-locative\\
\textsc{neg} &   negative\\
\textsc{nmlz} &   nominalizer\\
\textsc{pe} &   plural exclusive\\
\textsc{pi} &   plural inclusive\\\\
\textsc{plu} &   plural\\
\end{tabularx}
\begin{tabularx}{.45\textwidth}{lQ}
\textsc{poss} &   possessive\\
\textsc{pst} &   past\\
\textsc{purp} &   purposive\\
\textsc{red} &   reduplication\\
\textsc{refl} &   reflexive\\
\textsc{sg} &   singular\\
\textsc{sim.cvb} & simultaneous converb\\
\textsc{temp} &   temporal\\
\textsc{top} &   topic
\end{tabularx}



%\section*{Acknowledgements}
%\citet{Nordhoff2018} is useful for compiling bibliographies.

{\sloppy\printbibliography[heading=subbibliography,notkeyword=this]}
\end{document}
