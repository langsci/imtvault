\documentclass[output=paper]{langscibook}

\author{Nicoletta Puddu \lastand Katarzyna Janic
} 


\title{Introducing reflexive constructions}

 

\abstract{Reflexive constructions vary from language to language in the way they encode the coreference meaning between the agent and patient arguments of the minimal clause. While some languages employ a dedicated form called reflexivizer, the others make use of a non-reflexive form that under some conditions may perform a reflexive function. There is also much variation in the types of reflexivizers employed by languages such as reflexive nominals, voice markers, argument markers, and whether they occur in the variant or invariant form. Additionally, they may display various co-expression patterns related to self-intensification, auto-benefaction or other voice-changing operations. Interesting variations also occur in the antecedent domain between the antecedent and the reflexive nominal. }

\begin{document}
\maketitle

\section{Aims of the volume}
Reflexivity has been widely studied under different approaches and perspectives. This is due, in our opinion, to two main reasons. On the one hand, the presence of a dedicated marker to express coreference between two participants of the clause is a common phenomenon, so that many scholars hypothezise that reflexivity is a universal concept (see \citealt{HeineMiyashita2008}). On the other hand, reflexive markers are highly polifunctional (see \citealt{Frajzyngier2000}). As \citealt{KittiläZuñiga2019} point out, reflexivizers are often syncretic with self intensifiers on the one hand, and with other valency-reduction operators, on the other hand.

The majority of the scholars agree that the main function of reflexive constructions is to express coreference (but see contra \citealt{Frajzyngier2000}). However, the diversity of theoretical approaches, determined that coreference has been defined on different bases: more at the semantic level (see for instance \citealt{Lazard2007}, \citealt{Geniusiene1987}, \citealt{KittiläZuñiga2019}) or at the syntactic level as in \citealt{Schladt1999} and \citealt{Everaert2013}).
Moreover, as \citet{Frajzyngier2000} points out, such a widespread interest for reflexivity often lead to ambiguous uses of the term "reflexives" alternatively interpreted as both referring to the form or to the function.  
This volume aims at investigating how reflexivity is marked in the world's languages starting from the definition of \textsc{reflexive} \textsc{construction} given in Haspelmath \ref{chapter 2}. 
To do so, we contacted several experts in different languages and provided them with the position paper (ch.2), the questionnaire \ref{Appendix}) and a sample chapter \ref{chapter 11} on Polish, all available in this volume. Moreover, we made reference to \citealt{Pudduinpress} which discusses some issues in the traditional distinction between "nominal" and "verbal" reflexives.

The questionnaire aims at investigating some key points often addressed in the literature on reflexive constructions, both at the formal and at the functional level.  As is well known, languages tend to have a dedicated marker in order to express coreference between subject and object. Since "the most natural situation is for the arguments of a predicate to be non-coreferential” (Comrie 1999: 341) coreference between the two arguments needs to be marked. However, this does not exclude that some languages do not have an explicit reflexive markers: this is the case of Old English and of several Austronesian languages. Consequently, first of all the contributors were requested to determine whether the language has one or more "reflexivizer" (see \ref{chapter1}), i.e. "the specialized form used by the language in order to express coreference between an agent subject and a patient object".
A second very well known point is that reflexivezers can variate In terms of their locus of appearance. They can be nominal (argumental) or verbal. However, as Kazenin (2001) points out, the consistently separated treatment of the two types of reflexives is misleading because a historical relation between them is obvious in a large number of cases”. Although this distinction is widely accepted in the literature, I show in Puddu (forthcoming) that it poses several issues at both a general and at a language specific level.
Secondly, the basic uses of the reflexivizer are described, and in case of presence of more than one reflexivizers, contributor were requested to explain the reason for the presence of one of the two function (for instance introverted vs. extroverted). A relevant part of the questionnaire is finally dedicated to the possibilities of coreference expression.
The approach is strictly sincronic, however, in several papers, some diachronic considerations emerge. Moreover, we have a chapter which documented an extincted language, Early Vedic.

In the choice of the sample, we aimed especially at documenting underrepresented languages. We collected a total of (28) studies of languages all over the world, which have been then subdivided into seven areas: Africa (chapters 3-7), Eurasia (8-12), Papunesia (13-17), Australia  (18-20), North America (21-24) and South America (25-27). While aiming at a general uniformity by using an homogeneous questionnaire, the chapters show some variation on several aspects.
First of all, even if all the chapters are organized according to the questions posed in the questionnaire,  the authors used different methods on order to elicitate data. The two main distinction is between elicitation and use of corpus data. The advantages and disadvantages of the two methods are well known. In some chapters (like for instance chapter 08 on Mano) the benefits of using both approaches is made explicit, since some of the occurences disapproved in elicitation appear in corpus data. 

 
2. In terms of their morphological complexity, reflexive markers can be simplex or complex. If a language has both simple and complex reflexive markers, their use will depend on syntactic, semantic and pragmatic criteria.
3. In terms of their person and number, they can be variable or invariable. Languages can use the same form for all person, or an form inflected for person and number. A well known implicational universals states that if a language has a first and second person reflexive pronoun, it also has a third person reflexive pronoun (Comrie 1989: 6-7, 28; 1999:337). 
 It has been claimed that this implicational hierarchy has also a diachronic value, suggesting that, if a reflexive extends from the third person to the first and the second, it extends first to the second person and then to the first person. Actually, as several scholars pointed out (like de Benito Moreno and Benincà and Poletto), the extension of reflexive from third person to second and third person involves the interaction of the person hierarchy and number hierarchy
I will treat in more detail the morphological level since, in my opinion, reflexives have been extensively studied at the syntactic and semantic level, while much less attention has been devoted to the morphological properties of reflexives, even if morphological features have been invoked in order to explain some syntactic or semantic properties of reflexives. Several morphological classification are “taken for granted”, but they deserve further study.
At the morphological level, reflexive markers can variate
1. In terms of their locus of appearance. They can be nominal (argumental) or verbal. However, as Kazenin (2001) points out, the consistently separated treatment of the two types of reflexives is misleading because a historical relation between them is obvious in a large number of cases”. Although this distinction is widely accepted in the literature, I show in Puddu (forthcoming) that it poses several issues at both a general and at a language specific level. 
2. In terms of their morphological complexity, reflexive markers can be simplex or complex. If a language has both simple and complex reflexive markers, their use will depend on syntactic, semantic and pragmatic criteria.
Reflexivity has been studied extensively from a syntactic and a semantic point of view,
At the syntactic level, reflexive construction have been studied especially in the generative framework in relation to the admissible grammatical relations of the antecedent and to the possible domains of coreference
At the semantic level reflexives have been distinguished between direct and indirect reflexives and in relation to the number of participants involved
It goes without saying that a reflexive phenomenon has been approached from different theoretical angles. 
Generative approach, which focuses mainly on syntactic effects related to reflexive morphology played an important role in the reflexive domain.
The crowning achievement of this research programme is Binding theory by Chomsky 
who proposed three simple binding principle 
where ‘X binds Y’ means that X is coindexed (and thus coreferential) with Y and c-commands it. 
These principles have typically been argued to be universal, though they were established entirely on the basis of English. 
even if the empirical data have been extended by Romance, Germanic…

it turned out that tripartition of nominals is insufficient and a range of non-trivial exceptions have been found

Since the 1990s, it has been recognized that the 1981 formulation is problematic (even for English), 

hence many alternative versions have been proposed 

Although these frameworks opened up for a new understanding of reflexivity from a syntactic point of view that could go beyond a specific language, the peculiarities of various understudied languages still have challenged the typological validity of this approach.
generative model designed by Reinhart \& Reuland (1993), differentiates between reflexive predicates, and predicates with reflexive markers
with the aim to explain the co-existence of cases such as Johni pulled the blanket over himi/j/himselfi/*j. 
(While with reflexive predicates the two arguments of a verb are co-indexed, with predicates with a reflexive marker one of the arguments must be himself.) 
Reinhart \& Reuland (1993) apply two parameters, the reflexive function and referential dependence, and 
and distinguish between SELF- and SE- anaphors saying that
although both are referentially deficient, only SE- anaphors are subject-oriented...
Cognitive grammar approaches a reflexive phenomenon holistically where semantics is inseparable from syntax, viewed as a continuum including also morphology \& discourse structure. 

Cognitive grammar defines transitivity 
building on the multifactorial model designed by Hopper and Thompson (1980) 
and the way the event is conceptualized by the speaker.

Within cognitive approach, a prototypical transitive event is characterized as having

has two silent participants, TRAJECTOR (subject) and LANDMARK (object), conceived as being asymmetrically related in an activity;
interaction between them is unidirectional;
LANDMARK is directly affected by the contact instigated by the TRAJECTOR


And where a Prototypical transitive construction is viewed as a chain of action

Orqueda (2018): although this is an interesting proposal in the field of linguistics,  but it remains unclear how to define ‘transmission of energy’ 
Research within the functional-typological approach highlights the role of semantics in our understanding a reflexive phenomenon as a linguistic function and not just as an operation 

This can be for instance observed in the semantically-based definition by Givon according to which reflexives are defined in the following way:

Functional typological approach to reflexive constructions starts with Fatlz’s  (1977) pioneer work ‘Reflexivization: A Study in Universal Syntax’, in which the author recognizes the coexistence of two primary and secondary reflexive strategies in the same language period and classifies languages according to the primary strategy.
This distinction, which was subsequently discussed by many scholars 
contributed substantially to our knowledge on reflexive marking and the way… 
For instance, Haspelmath has observed that 
Fatlz argues, however, that... 
this observation reflects the core of the research programme adopted along the  functional typological line 
that seeks to investigate the nature of a reflexive phenomenon
A range of linguists working on reflexive constructions provide comprehensive insights formulated in terms of general statements (or universals).

to mention few of them

In addition Fatlz, reflexive constructions became a subject of thorough investigation by many other linguists, including …

to add the following orally (to improve it).
they base their analysis highlighting different aspects of grammar focus more on pragmatics 
they have different approaches, analyze reflexive constructions in different languages at different level
through … (Greenberg and much subsequent work). 
It provides a fundamental characterization of a reflexive phenomenon, often formulated as general statements or universals, which do not support 					





{\sloppy\printbibliography[heading=subbibliography,notkeyword=this]}
\end{document}
