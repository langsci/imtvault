\documentclass[output=paper]{langsci/langscibook} \label{AppendixA}
\title{Appendix: Questionnaire on reflexive constructions in the world's languages} \label{Appendix}
\abstract{}
\author{Martin Haspelmath \lastand Katarzyna Janic} 

\begin{document}
\maketitle




\textit{A short notice on our terminology: the term} \textbf{\textit{reflexivizer} }\textit{refers to any specialized form that expresses coreference within a clause. By} \textbf{\textit{specialized form} we understand a form which at least in certain conditions necessarily expresses the coreference meaning (even if} {\textit{non-coreference meanings are possible elsewhere).} Reflexivizers can be dependent or non-dependent forms like reflexive (pro)nouns, reflexive argument markers, or reflexive voice markers. Languages that have not developed a specialized reflexivizer express coreference with the help of other linguistic forms, e.g. personal pronouns. In this context, we prefer to talk about a} \textbf{\textit{non-reflexive form}}.


\section{Basic uses of reflexivizers} 

\begin{itemize}
\item 
Describe the \textbf{personal} \textbf{pronouns} and \textbf{reflexive} \textbf{pronouns} of the language. If the language also has \textbf{verbal} \textbf{reflexivizers} (reflexive argument markers or reflexive voice markers), give a brief description of the relevant verbal marking patterns. 
\item 
If the language uses a \textbf{reflexive} \textbf{pronoun} to express coreference, does it have distinctions such as the following?
a. person    b. case    c. number   d. obviation    e. gender 

\item 
How is \textbf{coreference} \textbf{of} \textbf{the} \textbf{agent} \textbf{subject} \textbf{with} \textbf{the} \textbf{patient} \textbf{referent} \textbf{in} \textbf{object} \textbf{function} expressed in the language? Give examples like (a)-(d) and indicate the form (if any) expressing coreference.
\end{itemize}

a.  I saw myself in the mirror.\footnote{Feel free to change the provided examples here and elsewhere in the questionnaire, if for some reasons they are problematic in your language.}  



b.  My friend hates himself. 



c.  She praised herself. 



d.  The man killed himself. 


\begin{itemize}
\item
If the language uses a specialized reflexivizer to express coreference, is this form \textbf{obligatory} \textbf{or} \textbf{optional}? 

\item
If the language has \textbf{several} \textbf{different} \textbf{reflexivizers} used under different conditions, what determines their distribution? 

\end{itemize}

a.  If they are in a complementary distribution, define the conditions under which each reflexivizers is selected. 



b.  If the forms can occur in the same environments, can you think of any context in which one form is preferred over another? 


\textit{Some languages use a range of different forms to express agent-patient coreference. For instance, Dutch employs reflexive pronouns only in the third person; coreference with the first and second person is expressed by ordinary personal pronouns.}

\begin{itemize}
\item
Is the use of reflexivizers subject to specific conditions relating to person or number? If it is, define them and provide relevant examples. 

\end{itemize}
\section{Specialized reflexive form in other functions} 
\begin{itemize}
\item
Does the reflexive form have other uses? Specify and provide relevant examples. 

\end{itemize}
\section{Contrast between introverted and extroverted verbs} 

\textit{Transitive verbs that allow a human object can be divided into introverted and extroverted classes (\citealt{Haiman1980}: 803; \citealt{KönigSiemund1999}: 61). Extroverted actions express socially antagonistic events such as ‘kill’, ‘kick’, ‘attack’, ‘hate’ and ‘criticize’, whereas introverted actions include body care (or grooming) actions exemplified by ‘wash’, ‘shave’, ‘dress’, ‘bathe’, and a few others such as ‘defend oneself’.}


\begin{itemize}
\item
How are autopathic actions with extroverted verbs expressed in the language? Give examples like (a)-(c) and indicate the form (if any) responsible for the coreference interpretation. 

\end{itemize}

a.  The dog bit itself. 



b.  The girl hates herself. 



c.  The politician criticized himself. 


\begin{itemize}
\item
How are autopathic actions with introverted verbs expressed in your language? Translate the examples (a)-(c) and indicate the form (if any) responsible for the coreference interpretation. 

\end{itemize}

a.  The dog was washing himself. 



b.  The girl washed. 



c.  He shaved. 


\section{Contrast between body-part and whole-body actions} 

\textit{Some languages encode body-part actions (combing hair, brushing teeth, clipping nails) similarly to those involving whole-body actions (wash, bathe, get tented) i.e. with the help of the same reflexive form (e.g. French ‘se peigner’ ‘to comb one’s hair’ vs. ‘se laver’ ‘to wash’). In other languages, body-part and whole-body actions are treated apart, the former being expressed through a transitive construction with the body-part expressed as object (e.g. English: ‘I comb my hair.’ vs. ‘I washed.’). Moreover, some languages specify the body-part object in addition to the reflexive form (e.g. French: ‘Il se lave les mains’ ‘he washes his hands’). If your language contrast body-part actions with whole-body actions in coding, proceed to point 20, otherwise skip it.}


\begin{itemize}
\item
How are body-part actions expressed in your language: (i) through coreference, or (ii) through a transitive construction with the body-part expressed as object? Translate (a)-(c). 

\end{itemize}

a.  The men shaved their beard. 



b.  She scratched her back. 



c.  He brushed his teeth. 


\begin{itemize}
\item
If your language employs a specialized reflexive form to express body-part actions, can the body-part be expressed as well? 

\item
How are whole-body actions expressed in your language? Translate the examples (a)-(c) and indicate the form (if any) responsible for the coreference interpretation. 

\end{itemize}

a.  The men got dressed.



b. She washed. 



c. I bathed. 


\section{Reflexive pronoun in subject position} 
\begin{itemize}
\item
Except for a few cases (e.g. Georgian), languages do not allow reflexive pronouns in subject function. Does your language support this crosslinguistic observation? If it does not, provide a relevant example. 

\end{itemize}
\subsection{Coreference of the subject with various semantic roles} 
\begin{itemize}
\item
(a)  \textbf{\textit{Possessor}}. How is coreference of the subject with a possessor referent expressed in your language? Translate (a)-(c) and indicate the form (if any) triggering the coreference meaning. 

\end{itemize}

a. She\textsubscript{1} took her\textsubscript{1} umbrella. 



b. John\textsubscript{1} reads his\textsubscript{1} book. 



c. The women\textsubscript{1} swept their\textsubscript{1} rooms. 



14. (b) Can you contrast examples from \REF{ex:key:14a} with those provided in \REF{ex:key:14b} wherein the referent of possessor is not co-referred with a subject? 



a. She1 took her2 umbrella. 



b. John1 reads his2 book. 



c. The women1 swept their2 rooms. 


\begin{itemize}
\item
(a) \textbf{\textit{Locative}}. How is coreference of the subject with a spatial referent expressed in your language? Translate (a)-(c) and indicate the form (if any) triggering the coreference meaning. 

\end{itemize}

a. She\textsubscript{1} saw a snake beside her\textsubscript{1}. 



b. John\textsubscript{1} put a book next to him\textsubscript{1}. 



c. She\textsubscript{1} left the traces behind her\textsubscript{1}. 



15. (b) Contrast examples from \REF{ex:key:15a} with those in provided \REF{ex:key:15b} wherein the spatial referent is not co-referred with a subject. 



a. She\textsubscript{1} saw a snake beside her\textsubscript{2}. 



b. John\textsubscript{1} put a book next to him\textsubscript{2}. 



c. She\textsubscript{1} left the traces behind her\textsubscript{2}. 


\begin{itemize}
\item
(a) \textbf{\textit{Benefactive}}. How is coreference of the subject with a beneficiary referent expressed in your language? Translate (a)-(c) and indicate the form (if any) triggering coreference. 

\end{itemize}

a. She bought a book for herself. 



b. The boy cooked a dinner for himself. 



c. They built a house for themselves. 



16. (b) Contrast examples from \REF{ex:key:16a} with those provided in \REF{ex:key:16b} wherein the referent of beneficiary is not co-referred with a subject. 



a. She bought a book for her. 



b. He cooked a dinner for him. 



c. You built a house for them. 


\begin{itemize}
\item
(a) \textbf{\textit{Recipient}}. How is coreference of the subject with a recipient referent expressed in your language? Translate (a)-(c) and indicate the form (if any) triggering the coreference meaning. 

\end{itemize}

a. John talked to himself. 



b. They sent a postcard to themselves. 



c. The girl gave herself a present.



17. (b) Contrast examples from \REF{ex:key:17a} with those provided in \REF{ex:key:17b} wherein the referent of recipient is not co-referred with a subject. 



a. John talked to him. 



b. They sent a postcard to them. 



c. The girl gave a present to her. 


\subsection{Coreference between non-subject arguments} 
\begin{itemize}
\item
How is coreference between two non-subject arguments expressed in a single clause? Translate (a)-(c) and indicate the form (if any) responsible for the coreference interpretation. 

\end{itemize}

a. She told us\textsubscript{1} about ourselves\textsubscript{1}. 



b. He spoke with John\textsubscript{1} about himself\textsubscript{1}. 



c. John showed Mary\textsubscript{1} a picture of herself\textsubscript{1}. 


\subsection{Contrast between coreference and disjoint reference} 
\begin{itemize}
\item
Contrast the subject-coreference pronoun in object position (examples’) with disjoint reference pronoun in object position (examples’’). Which pronouns does your language use to code these two types of situations? 

\end{itemize}

a' The man saw himself.     vs.   a” The man saw him. 



b' The woman criticized herself.   vs.   b” The woman criticized her. 



c' He admired himself.     vs.   c” He admired him. 


\subsection{Contrast between object and nominal adpossessor} 
\begin{itemize}
\item
Contrast the subject-coreferential pronoun in object position (examples‘) with the subject-coreferential pronouns in adnominal possessive position (examples’’). Which pronouns does your language use to code these two types of situations? 

\end{itemize}

a’ She\textsubscript{1} killed herself\textsubscript{1}.    vs.   a’’ She\textsubscript{1} killed her\textsubscript{1/2} lover. 



b' He\textsubscript{1} admires himself\textsubscript{1}.  vs.   b’’ He\textsubscript{1} admires his\textsubscript{1/2} boss. 



c' She\textsubscript{1} saw herself\textsubscript{1}.    vs.   c’’ She\textsubscript{1} saw her\textsubscript{1/2} sister. 


\subsection{Contrast between exact and inclusive coreference} 
\begin{itemize}
\item
Contrast exact coreference (examples‘) with inclusive coreference (examples”). Which pronouns does your language use to code these two types of situations? 

\end{itemize}

a’ She\textsubscript{1} admires herself\textsubscript{1}.   vs.   a’’ She\textsubscript{1} admires herself and the others\textsubscript{1+X}. 



b’ He\textsubscript{1} criticized himself\textsubscript{1}.   vs.   b” He\textsubscript{1} criticized himself and the others\textsubscript{1+X}. 


c’ He\textsubscript{1} defended himself\textsubscript{1}.   vs.   c” He\textsubscript{1} defended himself and the others\textsubscript{1+X}. 

\subsection{Long-distance coreference} 
\begin{itemize}
\item
How is coreference of the subject across clauses expressed in your language? Translate (a)-(c) and indicate the form (if any) responsible for the coreference interpretation. 

\end{itemize}

a. She\textsubscript{1} thought that she\textsubscript{1} had enough money. 



b. The boy\textsubscript{1} said that he\textsubscript{1} must go home. 



c. We\textsubscript{1} said that we\textsubscript{1} worked the whole day. 


\sloppy\printbibliography[heading=subbibliography,notkeyword=this]
\end{document} 
