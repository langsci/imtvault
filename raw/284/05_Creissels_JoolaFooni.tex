\documentclass[output=paper]{langscibook}

\author{%
  Denis Creissels%
  \affiliation{Laboratoire Dynamique Du Langage (CNRS \& Université Lyon 2, UMR 5596)}
  \lastand%
  Alain Bassene%
  \affiliation{Université Cheikh Anta Diop, Dakar}
}
\title{Reflexive Constructions in Jóola Fóoñi}

\abstract{%
  The reflexive constructions of Jóola Fóoñi (an Atlantic language spoken in
  south western Senegal) are characterized by a~sharp distinction between
  subject--object coreference, which requires the use of derived forms of the
  verb, and other possible coreference relationships within the clause, which
  are not treated differently from coreference in discourse.  Three verbal
  suffixes are involved in the coding of subject--object coreference, none of
  which is specialized in reflexive function: \textit{‑ɔɔr} (productive in
  reciprocal function, very marginally involved in reflexivization), \textit{‑ɔ}
  (productive in decausative and quasi-reflexive function, also used to encode
  reflexivization with body-care verbs), and \textit{‑ɔɔrɔ} (the default marker
  of subject--object coreference, also used to mark self-intensification of the
  subject).%
}

\IfFileExists{../localcommands.tex}{%hack to check whether this is being compiled as part of a collection or standalone
  \input{../localpackages}
  %Copy this to localcommands.tex

\usepackage[english]{babel}
\usepackage{amsmath}
\usepackage{amssymb,amsfonts,textcomp}
\usepackage{array}
\usepackage{hhline}
\usepackage{hyperref}

\newenvironment{styleStandard}{}{}
\newenvironment{stylelsAbstract}{}{}
\newenvironment{stylelsSectioni}{}{}
\newenvironment{stylelsSectionii}{}{}
\newenvironment{stylelsBulletList}{}{}
\newenvironment{styleBibliographyi}{}{}
\newenvironment{listWWNumxxvleveli}{}{}
\newenvironment{listWWNumxxvlevelii}{}{}
\newenvironment{listWWNumxxvleveliii}{}{}
\newenvironment{listWWNumxxvleveliv}{}{}
\newenvironment{listWWNumixleveli}{}{}
\newenvironment{listWWNumixlevelii}{}{}
\newenvironment{listWWNumixleveliii}{}{}
\newenvironment{listWWNumixleveliv}{}{}

\newcommand\textstyleListLabelxvi[1]{#1}
\newcommand\labellistWWNumxxvleveli{\thelistWWNumxxvleveli.}
\newcommand\labellistWWNumxxvlevelii{\thelistWWNumxxvlevelii.}
\newcommand\labellistWWNumxxvleveliii{\thelistWWNumxxvleveliii.}
\newcommand\labellistWWNumxxvleveliv{\thelistWWNumxxvleveliv.}
\newcommand\labellistWWNumixleveli{[F0B7?]}
\newcommand\labellistWWNumixlevelii{\textstyleListLabelxvi{o}}
\newcommand\labellistWWNumixleveliii{[F0A7?]}
\newcommand\labellistWWNumixleveliv{[F0B7?]}

\newcounter{listWWNumxxvleveli}
\newcounter{listWWNumxxvlevelii}[listWWNumxxvleveli]
\newcounter{listWWNumxxvleveliii}[listWWNumxxvlevelii]
\newcounter{listWWNumxxvleveliv}[listWWNumxxvleveliii]
\newcounter{itemize} 
  \input{../localhyphenation}
  \bibliography{localbibliography}
  %\togglepaper[5]
}{}


\begin{document}
\maketitle

\section{Introduction}%

Jóola Fóoñi (a.\,k.\,a.\ Diola-Fogny), spoken in south western Senegal by
approximately half a~million speakers, belongs to the Bak group of languages
included in the Atlantic family.\footnote{ Jóola languages can be divided into
  Central Jóola, a~dialect continuum within the limits of which it is difficult
  (if not impossible) to decide what is a~language and what is a~dialect, and
  peripheral Jóola varieties whose status as separate languages is hardly
  disputable, in spite of their close relationship to Central Jóola, such as
  Karon, Kwaataay, Mulomp-North, or Bayot.  Jóola Fóoñi is part of the Central
  Jóola dialect continuum. } Three overall presentations of Jóola Fóoñi grammar
are available: \citet{Weiss1939}, \citet{Sapir1965}, and \citet{Hopkins1995},
but none of them includes a~discussion of reflexive constructions.  The
available documentation on the reflexive constructions of Jóola Fóoñi is
limited to a~few examples of the use of the verbal suffixes \textit{‑ɔ} and
\textit{‑ɔɔrɔ}, designated by \citet[51]{Sapir1965} as `reflexive-descriptive'
and `strong reflexive', respectively.  In particular, a~crucial property of the
suffix \textit{‑ɔɔrɔ}, namely the possibility of a~non-reflexive use in which it
marks self-intensification of the subject, has never been acknowledged before.

In a~general way, we base our analysis of Jóola Fóoñi on a~corpus of more than
twelve hours of recorded naturalistic texts of various genres.\footnote{The
  texts have been transcribed by Boubacar Sambou (a~graduate student in
  linguistics who is also a~native speaker of the language), and analyzed by
  Alain Christian Bassène and Denis Creissels with the help of Boubacar
  Sambou. } However, in contrast to other morphosyntactic phenomena on which we
already worked, and for which our corpus provided abundant data, it turns out
that reflexive constructions are very rare in our corpus, and the analysis of
reflexive constructions presented in this paper would not have been possible
without systematic recourse to elicitation.  In fact, most of the examples we
quote have been elicited.

The article is organized as follows.  \sectref{sec:Creissels:background} provides
background information on Jóola Fóoñi morphosyntax.
\sectref{sec:Creissels:coref-within-clause} describes the general principles
underlying the expression of coreference within the clause in Jóola Fooñi,
characterized by a~sharp distinction between subject--object coreference, which
requires verbal marking, and other configurations, which are not treated
differently from coreference in discourse.  \sectref{sec:Creissels:verbal-marking} is
on the reflexive and non-reflexive uses of the verbal suffixes involved in the
coding of subject--object coreference.  \sectref{sec:Creissels:refl-and-self-intens}
gives additional precisions on the relationship between reflexivization and
self-intensification, which constitutes a~particularly original aspect of Jóola
Fóoñi.  \sectref{sec:Creissels:conclusion} summarizes the main conclusions.

\section{Background information on Jóola Fóoñi morphosyntax}%
\label{sec:Creissels:background}

\subsection{Clause structure}%
\label{sec:Creissels:clause-struct}

\subsubsection{Transitive--intransitive alignment}%
\label{sec:Creissels:transitive-alignment}

Like most of the languages of Subsaharan Africa, Jóola Fóoñi has
a~straightforward `nominative-accusative' alignment system making it possible to define a~grammatical relation `subject' on the basis of a~set of properties shared by~A in the basic transitive construction and the sole argument of semantically monovalent verbs, and a~grammatical relation `object' on the basis of a~set of properties that distinguish the P~phrase in the basic transitive construction from noun phrases fulfilling other roles.

\subsubsection{Subjects, objects and obliques}%
\label{sec:Creissels:subj-obj-obl}

Subjects and objects are equally unflagged.  The most obvious property that distinguishes them is that subjects are indexed by means of verbal prefixes, whereas objects are indexed by means of verbal suffixes.  Moreover, as illustrated in~(\ref{ex:Creissels:1}), with the verb forms lending themselves to subject indexation, the prefixed subject index is obligatory (even in the presence of a~subject NP) whereas object indexation, conditioned by topicality, is equally optional with all verb forms.\footnote{ Our transcription of the Jóola Fóoñi examples is a~broad phonetic transcription that coincides with the official orthography as regards the notation of consonants, but departs from it in the notation of vowels, for which we follow the IPA conventions.
This choice is motivated by the fact that the official orthography uses the
  acute accent to distinguish $+$ATR vowels from their $-$ATR counterparts,
  which may be confusing since accents are more commonly used to indicate word
  stress or tone.  Phonological processes are responsible for variation in the
  form of some formatives.  In particular, ATR harmony is responsible for
  variation in the vowels of most affixes, as illustrated by the non-subject
  index of class~A, which depending on the context may surface as \textit{‑ɔɔl},
  \textit{‑ool}, \textit{‑ɔɔ}, or \textit{‑oo}.  Consonants in coda position are also
  often affected by phonological processes, as in \textit{εrʊnrʊm}
  (\ref{ex:Creissels:1a}), where the final consonant of \textit{rʊm} `bite' is modified
  in contact with the initial consonant of the reduplicative suffix, or in
  \textit{nιjʊjʊk} (\ref{ex:Creissels:2a}), where the final consonant of \textit{jʊk}
  `see' is deleted for the same reasons. }

\ea \label{ex:Creissels:1}
  
    \ea\label{ex:Creissels:1a}
    
      \textit{Ɛyɛnɛy ɛrʊnrʊm añɩɩlaw.} \\
      \gll ɛ-yɛn-ɛ-y ɛ-rʊn-rʊm a-ñɩɩl-a-w \\
      \textsc{sg}-dog(E)-\textsc{d}-clE sI:clE-bite-\textsc{rdpl} \textsc{sg}-child(A)-\textsc{d}-clA \\
      \glt `The dog bit the child.'


    \ex\label{ex:Creissels:1b}
    
      \textit{Ɛrʊmɔɔrʊm.} \\
      \gll ɛ-rʊm-ɔɔ-rʊm\\
      sI:clE-bite-I:clA-\textsc{rdpl}\\
      \glt `It (the dog) bit him (the child).'

  \z
\z

Subject NPs consistently precede the verb.  The unmarked position for object
NPs and obliques is after the verb, as in~(\ref{ex:Creissels:2a}), but in case of
focalization (marked by the use of special verb forms also used in relative
clauses), they move to clause-initial position, as in~(\ref{ex:Creissels:2b}).

\ea
  \label{ex:Creissels:2}
  
    \ea\label{ex:Creissels:2a}
    
      \textit{Nɩjʊjʊk Musaa.} \\
      \gll n-ɩ-jʊ-jʊk Musaa \\
      \textsc{ppf}-sI:\textsc{1sg}-see-\textsc{rdpl} Moussa(A) \\
      \glt `I~saw Moussa.'


    \ex\label{ex:Creissels:2b}
    
      \textit{Musaa nɩjʊkʊm.}\\
      \gll Musaa n-ɩ-jʊk-ʊ-m\\
      Moussa(A) \textsc{ppf}-sI:\textsc{1sg}-see-\textsc{ep-act$_1$}\\
      \glt `It is Moussa that I~saw.'

  \z
\z

Obliques are easy to distinguish from objects when they are introduced by
a~preposition, but unflagged obliques are relatively common in Jóola Fooñi.
However, their indexation properties distinguish them from objects: some
obliques cannot be indexed at all, and for those lending themselves to
indexation, contrary to objects, the choice of the index is not sensitive to
the gender-number of the NP in oblique role, but only to its semantic role.
For example, in~(\ref{ex:Creissels:3a}), \textit{esukey} `the village' is not flagged,
which could suggest analyzing it as an object, but if it were the case, it
should be possible to substitute the class~E index \textit{‑yɔ} for it, since
\textit{esuk} `village' governs class~E agreement.  The fact that, in this
sentence, esukey can only be represented by the locative class index
\textit{‑bɔ}, as in~(\ref{ex:Creissels:3b}), shows that it must be analyzed as an
unflagged oblique rather than as an object.

\ea
  \label{ex:Creissels:3}
  
    \ea\label{ex:Creissels:3a}
    
      \textit{Kʊjajaw esukey.} \\
      \gll kʊ-ja-jaw e-suk-e-y\\
      sI:clBK-go-\textsc{rdpl} \textsc{sg}-village(E)-\textsc{d}-clE\\
      \glt `They went to the village.'


    \ex\label{ex:Creissels:3b}
    
      \textit{Esukey, kʊjajawbɔ.}\\
      \gll e-suk-e-y kʊ-ja-jaw-bɔ\\
      \textsc{sg}-village(E)-\textsc{d}-clE sI:clBK-go-\textsc{rdpl}-I:clB\\
      \glt `The village, they went there.'

  \z
\z

In Jóola Fóoñi, there is no strict relative ordering of objects and obliques.

\subsubsection{Transitivity prominence}%
\label{sec:Creissels:transitivity-prominence}

Given that, in Jóola Fóoñi, transitivity is crucial in the conditioning of the
expression of coreference within the clause, it is important to mention here
that one of the salient typological characteristics of Jóola Fóoñi is its
extremely high degree of transitivity prominence (i.\,e., a~very strong
tendency to extend transitive coding to verbs whose meaning departs from
prototypical transitivity).

% TODO citations
Creissels (In preparation) proposes a~questionnaire consisting of 30 verb
meanings specially designed to evaluate the cross-linguistic variation in
transitive prominence.  The verb meanings that constitute this questionnaire
are neither among those expressed by transitive verbs in (almost) all the
languages for which the relevant data have been checked, nor among those that
have a~marked tendency to be expressed by verbs assigning other types of coding
to their arguments.

Within the limits of this questionnaire, the ratio of transitive coding and
other types of coding is for example 29.5 vs.\ 0.5 for Tswana (Bantu), 23 vs.\
7 for Italian, 21 vs.\ 9 for Yoruba (Benue-Congo), 20.5 vs.\ 9.5 for Mandinka
(Mande), 17 vs.\ 13 for Basque, 15.5 vs.\ 14.5 for Russian, 13 vs.\ 17 for
Koroboro Senni (Songhay), and 3 vs.\ 27 for Akhvakh (Nakh-Daghestanian).  Jóola
Fóoñi, with a~ratio of 29 vs.\ 1, is among the languages for which this
questionnaire indicates an extremely high degree of transitivity prominence.

\subsubsection{Multiple-object constructions and the coding of beneficiaries}%
\label{sec:Creissels:beneficiaries-coding}

Another salient characteristic of Jóola Fooñi, which has consequences for the
productivity of voice markers in the expression of coreference within the
clause, is the remarkable productivity of multiple-object constructions.  In
particular, double-object constructions are used not only for semantically
trivalent verbs such as \textit{sɛn} `give' or \textit{yisen} `show', but also for
bivalent verbs to which an NP with the semantic role of beneficiary is added.

In contrast to most Atlantic languages, Jóola Fóoñi does not use the
applicative strategy to encode beneficiaries,\footnote{ Jóola Fooñi has
  a~single applicative marker (\textit{‑um}) exclusively used to license applied
  phrases with a~prolative, instrumental, causal, motivative or mediative
  semantic role, which constitutes a~typologically unusual situation. } and
does not have a~benefactive adposition either: in Jóola Fóoñi, beneficiaries
are simply encoded as objects that nothing distinguishes from the objects
representing the P~argument of transitive verbs.  This results in the
possibility of transitive constructions of semantically monovalent verbs, as
in~(\ref{ex:Creissels:4}) with \textit{jɔn} `set (speaking of the sun)', and of
double-object constructions of semantically bivalent verbs, as
in~(\ref{ex:Creissels:5}) with \textit{wɔnk} `call'.

\ea
  \label{ex:Creissels:4}
  
    \ea \label{ex:Creissels:4a}
    
      \textit{Bɐgunɐb di bɔɔnɛ: ``Ɛɛnʊjaa pan bʊjɔn, pan bʊjɔn, barɛ ɛɛnʊjaa lɛɛ bʊjɔn, lɛɛ bʊjɔn.''}\\
      \gll bɐ-gun-ɐ-b dɩ b-ɔɔnɛ ɛɛn-ʊ-jaa pan bʊ-jɔn pan bʊ-jɔn barɛ ɛɛn-ʊ-jaa lɛɛ bʊ-jɔn lɛɛ bʊ-jɔn\\
      \textsc{sg}-genius(B)-\textsc{d}-clB \textsc{seq} sI:clB-say sI:\textsc{1sg}.say-\textsc{ep-hyp} \textsc{fut} sI:clB-set \textsc{fut} sI:clB-set but sI:\textsc{1sg}.say-\textsc{ep-hyp} \textsc{fut.neg} sI:clB-set \textsc{fut.neg} sI:clB-set\\
      \glt `Then the genius said: ``If I say that it will set (balaab `the sun(B)'), it will set, but if I say that it will not set, it will not set.''\,'
    

    \ex\label{ex:Creissels:4b}
    % FIXME should be capitalised
    \textit{ɐmoorɐɐw naanoo: ``Añɩɩlaw ʊmɛ, lɛɛ bʊjɔnɔɔ.''}\\
    \gll ɐ-moori-ɐ-w n-aan-ɔɔ a-ñɩɩl-a-w ʊ-m-ɛ lɛɛ bʊ-jɔn-ɔɔ\\
    \textsc{sg}-marabout(A)-\textsc{d}-clA \textsc{ppf}-sI:clA.say-I:clA
    \textsc{sg}-child(A)-\textsc{d}-clA \textsc{dem}-clA-\textsc{prox}
    \textsc{fut.neg} sI:clB-set-I:clA\\
    \glt `Then the marabout told her: ``This child, he will die by this evening.''\,' \\
    lit.\ `it (balaab `the sun(B)') will not set (for) him.'
  \z
 \z
 
  \ea\label{ex:Creissels:5}
  
    \ea\label{ex:Creissels:5a}
    \textit{Nɩwɔnwɔnk Musaa añiilaw.}\\
    \gll n-ɩ-wɔn-wɔnk Musaa a-ñɩɩl-a-w\\
    \textsc{ppf}-sI:\textsc{1sg}-call-\textsc{rdpl} Moussa(A) \textsc{sg}-child(A)-\textsc{d}-clA\\
    \glt `I called the child for Moussa.'

    \ex\label{ex:Creissels:5b}
    
      \textit{Nɩwɔnkɔɔlɔɔwɔnk}\\
      \gll n-ɩ-wɔnk-ɔɔl-ɔɔ-wɔnk\\
      \textsc{ppf}-sI:\textsc{1sg}-call-I:clA-I:clA-\textsc{rdpl}\\
      \glt `I called him (for) him.'

  \z
\z

Given that non-specific P~arguments may simply be left unexpressed, the coding
of beneficiaries as objects may give rise to ambiguities of the type
illustrated in~(\ref{ex:Creissels:6}).

\ea
  \label{ex:Creissels:6}
  
    \textit{Pan ɩpɔsɔɔl.}\\
    \gll pan ɩ-pɔs-ɔɔl\\
    \textsc{fut} sI:\textsc{1sg}-wash-I:clA\\
    \glt `I'll wash him.' \textit{or} `I'll do the washing for him.'

\z

\subsection{Nouns and noun phrases}%
\label{sec:Creissels:nouns-nps}

Jóola Fóoñi has a~gender system of the type commonly found in Niger-Congo
languages, especially among Bantu and Atlantic languages, characterized by
a~close relationship (which however does not boil down to a~straightforward
one-to-one correspondence) between the division of nouns into subsets according
to the way they express the singular vs.\ plural distinction, and their
division into subsets according to the agreement marks they control on their
modifiers or on the pronouns that resume them.

In Jóola Fooñi, each noun \textsc{form} is associated with one of thirteen
possible agreement patterns, and genders can be defined as sets of nominal
\textsc{lexemes} that are associated with the same agreement pattern both in
the singular and the plural.  Agreement patterns and genders are conventionally
designated here by capital letters that evoke the phonological form of the
agreement markers.  For example, `dog' as a~lexeme belongs to gender E/S, which
means that the singular form \textit{ɛ-yɛn} `dog' is associated with the
agreement pattern~E (cf.\ \textit{ɛ-yɛn ɛ-cɛɛn} `some dog', \textit{ɛ-yɛn ɛ-cɩla}
`the aforementioned dog', etc., to be compared for example with agreement
pattern~K in \textit{ka-laak kʊ-cɛɛn} `some field', \textit{ka-laak kʊ-cɩla} `the
aforementioned field', etc.), whereas the corresponding plural form
\textit{sɩ-yɛn} is associated with the agreement pattern~S (cf.\ \textit{sɩ-yɛn
  sɩ-cɛɛn} `some dogs', \textit{sɩ-yɛn sɩ-cɩla} `the aforementioned dogs', etc.,
to be compared with agreement pattern~U in \textit{ʊ-laak ʊ-cɛɛn} `some fields',
\textit{ʊ-laak ʊ-cɩla} `the aforementioned fields', etc.).

In our terminology, the term `class' refers exclusively to cells in the
morphological paradigm of adnominals and pronouns that can be the target of an
agreement mechanism.\footnote{ For a~detailed criticism of the way the term
  `class' is traditionally used in descriptions of Niger-Congo languages, the
  reader is referred to \citet{GüldemannFiedler2017}. } For example,
\textit{ɛ-cɛɛn} is the class~E form of the determiner \textit{‑cɛɛn} `some', and
\textit{sɩ-cɛɛn} is the class~S form of the same determiner.

The inflectional paradigm of adnominals and pronouns consists of 15 cells.  13
of them are involved in one of the 13 possible agreement patterns for noun
forms (and are labeled by means of the same capital letter), The remaining two
(class~D and class~N) are only used pronominally or adverbially with meanings
that do not refer to any possible controller: vague reference to things for
class~D, and time for class~N.  For example, 13 of the 15 possible forms of the
indefinite determiner \textit{‑cɛɛn} `some' are found in constructions in which
their prefix can be analyzed as an agreement marker (\textit{ɛ-yɛn ɛ-cɛɛn} `some
dog', \textit{a-ñiil a-cɛɛn} `some child', \textit{u-bɐɐr ʊ-cɛɛn} `some trees',
\textit{kɐ-rumbɐ kʊ-cɛɛn} `some pot' etc.), but the morphological paradigm of
\textit{‑cɛɛn} also includes two forms that do not correspond to any noun that
could trigger their choice in an agreement mechanism, and can only be used
pronominally (\textit{dɩ-cɛɛn} `something') or adverbially (\textit{nɩ-cɛɛn}
`sometimes').

Gender A/BK (agreement pattern~A in the singular, BK in the plural) coincides
almost perfectly with the set of nouns denoting humans.  The other genders are
semantically heterogeneous.

Jóola Fóoñi has an enclitic definite article expressing class agreement with
the noun to which it attaches.\footnote{ Depending on the stem to which it
  attaches, the first formative of the enclitic definite article may surface as
  \textit{‑a‑}, \textit{‑ɐ‑}, \textit{‑ε‑}, or \textit{­‑e‑}. } As illustrated
in~(\ref{ex:Creissels:7}), attributive adjectives agree with their head in
definiteness.

\ea
  \label{ex:Creissels:7}
  
    \ea\label{ex:Creissels:7a}
    %
      \textit{bubɐɐr bɐɐmɐk} / \textit{bubɐɐrɐb bɐɐmɐkɐb}\\
      \gll bu-bɐɐr b-ɐɐmɐk / bu-bɐɐr-ɐ-b b-ɐɐmɐk-ɐ-b\\
      \textsc{sg}-tree(B) clB-big { } \textsc{sg}-tree(B)-\textsc{d}-clB clB-big-\textsc{d}-clB\\
      \glt `big tree' / `the big tree'


    \ex\label{ex:Creissels:7b}
    %
      \textit{fal fɐɐmɐk} / \textit{falaf fɐɐmɐkɐf}\\
      \gll f-al f-ɐɐmɐk / f-al-a-f f-ɐɐmɐk-ɐ-f\\
      \textsc{sg}-river(F) clF-big { } \textsc{sg}-river(F)-\textsc{d}-clF clF-big-\textsc{d}-clF\\
      \glt `big river' / `the big river'

  \z
\z

Within noun phrases, the general rule is that modifiers follow their head.
However, adnominal possessors differ from the other noun modifiers in that they
may optionally precede their head.  As illustrated in \REF{ex:Creissels:8},
adnominal possessors that follow their head are usually introduced by the
genitive linker \textit{‑ati} expressing the gender and number of the head,
whereas adnominal possessors preceding their head are obligatorily resumed by
an index suffixed to their head.

\ea
  \label{ex:Creissels:8}
  
    \ea\label{ex:Creissels:8a}
    
      \textit{asɛɛkaw ati Musaa}\\
      \gll a-sɛɛk-a-w $\varnothing$-ati Musaa\\
      \textsc{sg}-woman(A)-\textsc{d}-clA clA-\textsc{gen} Moussa(A)\\
      \glt `Moussa's wife', lit.\ `the wife that-of Moussa'


    \ex\label{ex:Creissels:8b}
    
      \textit{Musaa asɛɛkɔɔl}\\
      \gll Musaa a-sɛɛk-ɔɔl\\
      Moussa(A) \textsc{sg}-woman(A)-I:clA\\
      \glt `Moussa's wife' lit.\ `Moussa his wife'

  \z
\z

\subsection{Verb forms}%
\label{sec:Creissels:verb-forms}

With the exception of the imperative, in which the 2nd person prefix may
optionally be deleted, the verb forms of Jóola Fóoñi consist minimally of
a~stem and a~prefix.  The stem may be a~root (irreducible lexical element), or
a~root enlarged by one or more derivational suffixes.

According to the nature of their obligatory prefix, verb forms can be
characterized \textsc{morphologically} as finite or non-finite:

\begin{itemize}
\item in finite verb forms, the obligatory prefix preceding the root is
  a~subject index expressing the person (and in the third person, the gender
  and number) of the subject argument;
\item on-finite verb forms do not include a~subject index, and their obligatory
  prefix characterizes them as belonging to one of the following three types of
  non-finite forms: infinitive, participle, or converb.
\end{itemize}

However, this morphological distinction does not coincide with the syntactic
distinction between independent and dependent verb forms.

On the one hand, the relative verb forms, whose use is restricted to relative
clauses and clauses in which a~noun phrase or adverb is focalized, also include
an obligatory subject index.  They differ from the independent verb forms in
the details of their TAM and polarity inflection.  Independent verb forms may
include TAM markers preceding the subject index, whereas the inflection of
relative verb forms is purely suffixal, and includes a~special paradigm of
three `actualizers' (glossed \textsc{act}) that have no equivalent in the
inflection of independent verb forms.\footnote{ he actualizers characterize the
  event to which the relative verb form refers as irrealis (\textsc{act}$_0$),
  realis (\textsc{act}$_1$), or having a~close relationship with the time of
  utterance (\textsc{act}$_0$).  The \textsc{act}$_2$ marker \textit{‑ñaa}
  results from the grammaticalization of the adverb \textit{ñaa} `now'.  In its
  presence, the incompletive aspect is interpreted as expressing present
  progressive, and the completive aspect is interpreted as expressing recent
  past. }

On the other hand, the non-finite verb forms as defined above, in addition to
uses that justify the labels we use to designate them (infinitive, participle,
and converb), can also be used by themselves (i.\,e., without having to combine
with an auxiliary) as the nucleus of independent assertive clauses expressing
TAM values distinct from those expressed by morphologically finite verb forms.

\subsection{Personal pronouns and indexes}%
\label{sec:Creissels:pers-pronouns}

The inventory of personal pronouns and indexes is given in
Tables~\ref{tab:Creissels:pronouns-1-2} and~\ref{tab:Creissels:pronouns-3}.  There is
a~single morphological paradigm of free pronouns, but two distinct paradigms of
indexes.  The forms given in these two tables are those that can be considered
basic; depending on the contexts in which they occur, they may be modified by
regular morphophonological processes.\footnote{ In particular, in combination
  with $+$ATR stems, all the indexes whose underlying form includes a~$-$ATR
  vowel undergo ATR harmony. }

\begin{table}[ht]
  \centering
  \begin{tabular}{llll}
    \lsptoprule
    & {free} & {subject} & {non-subject}\\
    & {pronouns} & {indexes} & {indexes} \\
    \hline
    \textsc{1sg} & {inje} & {ɩ‑} & {‑ɔm} $\sim$ {‑aam} \\
    \textsc{2sg} & {aw} & {ʊ‑} & {‑ɩ} \\
    \textsc{1pl.excl} & {uli} $\sim$ {oli} & {ʊ‑} & {‑oli} \\
    \textsc{1pl.incl} & {walaal} $\sim$ {ɔlaal} & {ʊ‑\dots‑aal} & {‑ɔlaal} \\
    \textsc{2pl} & {mʊyʊʊl} $\sim$ {mɩyʊʊl} & -- & {‑ʊʊl} \\
    \lspbottomrule
  \end{tabular}
  \caption{First and second person pronouns and indexes}%
  \label{tab:Creissels:pronouns-1-2}
\end{table}

\begin{table}[ht]
  \centering
  \begin{tabular}{llll}
    \lsptoprule
    & {free} & {subject} & {non-subject} \\
    & {pronouns} & {indexes} & {indexes} \\
    \hline
    \textsc{cl.\ A} & {ɔɔ} & {a‑} & {‑ɔɔl} \\
    \textsc{cl.\ BK} & {k-ɔɔ $\sim$ bʊk-ɔɔ} & {k‑} & {‑ɩɩl} \\
    \textsc{cl.\ E} & {y-ɔɔ} & {ɛ‑} & {‑yɔ} \\
    \textsc{cl.\ S} & {s-ɔɔ} & {s‑} & {‑sɔ} \\
    \textsc{cl.\ F} & {f-ɔɔ} & {f‑} & {‑fɔ} \\
    \textsc{cl.\ K} & {k-ɔɔ} & {k‑} & {‑kɔ} \\
    \textsc{cl.\ B} & {b-ɔɔ} & {b‑} & {‑bɔ} \\
    \textsc{cl.\ Ñ} & {ñ-ɔɔ} & {ñ‑} & {‑ñɔ} \\
    \textsc{cl.\ U} & {w-ɔɔ} & {ʊ‑} & {‑wɔ} \\
    \textsc{cl.\ J} & {j-ɔɔ} & {j‑} & {‑jɔ} \\
    \textsc{cl.\ M} & {m-ɔɔ} & {m‑} & {‑mɔ} \\
    \textsc{cl.\ T} & {t-ɔɔ} & {t‑} & {‑tɔ} \\
    \textsc{cl.\ Dʹ} & {d-oo $\sim$ r-oo} & {d‑ $\sim$ r‑} & {‑do $\sim$ ‑ro} \\
    \textsc{cl.\ D} & {d-ɔɔ $\sim$ r-ɔɔ} & $\varnothing$‑ & {‑dɔ $\sim$ ‑rɔ} \\
    \textsc{cl.\ N} & {n-ɔɔ} & -- & {‑nɔ} \\
    \lspbottomrule
  \end{tabular}
  \caption{Third person pronouns and indexes}%
  \label{tab:Creissels:pronouns-3}
\end{table}

Note that:

\begin{itemize}
\item There is no dedicated subject index of second person plural.  Second
  person plural subjects are indexed by means of the class~J index (\textit{j‑}),
  which can also be used optionally to index first person plural subjects
  instead of the dedicated first person plural index \textit{ʊ‑}.  We do not know
  the historical explanation of the use of the class~J index to represent
  speech act participants.
\item The lack of subject index of class~N is due to the fact that there is no
  noun triggering class~N agreement, and class~N forms are exclusively used as
  adverbs.
\item Non-subject indexes can be suffixed not only to verbs, but also to nouns
  (as possessive indexes), to some adnominal particles, and to some
  adpositions.
\item Non-subject indexes suffixed to verbs can index not only objects, but
  also some obliques.  There is however an important distinction: as object
  indexes, they agree in gender-number with their antecedent, whereas oblique
  indexes are determined by the function of the oblique phrase they represent
  (for example, in object function, \textit{esukey} `the village' --~gender
  E/S~-- is indexed by the E~class index \textit{‑yɔ}, whereas the spatial
  adjunct \textit{dɩ esukey} `in the village' is indexed by the class~B index
  \textit{‑bɔ}).
\end{itemize}

As illustrated in~(\ref{ex:Creissels:9b}) (to be compared with the adpossessive
construction in~(\ref{ex:Creissels:9a})), with the exception of the two unanalysable
stems \textit{‑umbɐ(ɐm)} (1st person singular possessive) and \textit{‑ɩɩya} (2nd
person singular possessive), possessive pronouns (also used as possessive
determiners) consist of a~class prefix marking agreement with their antecedent
or head (the possessee), a~stem \textit{‑ɔɔl‑} (glossed \textsc{poss}), and
a~suffixed index representing the possessor.

\ea
  \label{ex:Creissels:9}
    \ea\label{ex:Creissels:9a}
    
      \textit{ʊwɔsaw watɩ fujicelɐf}\\
      \gll ʊ-wɔs-a-w w-ati fu-jicel-ɐ-f\\
      \textsc{pl}-ear(U)-\textsc{d}-clU clU-\textsc{gen} \textsc{sg}-male.goat(F)-\textsc{d}-clF\\
      \glt `the ears of the male goat'


    \ex\label{ex:Creissels:9b}
    
      
        \textit{wɔɔlʊfɔ}\\
        \gll w-ɔɔl-ʊ-fɔ\\
        clU-\textsc{poss-ep}-I:clF\\
        \glt lit.\ `those of it' (possessee of class~U, possessor of class~F)
  

  \z
\z


\section{Coreference within the clause: general principles}%
\label{sec:Creissels:coref-within-clause}

Jóola Fóoñi does not have reflexive pronouns or indexes, and does not have
logophorics or long-distance reflexives either.  This means that coreference
relationships within the clause that do not require verbal marking are not
treated differently from coreference in discourse, and the same applies to
coreference relationships across clause boundaries in complex constructions.

In the examples of coreference within the clause that we have been able to find
in our corpus or to elicit, the subject is most of the time one of the two
terms of the clause involved in the coreference relationship, and in all cases,
the fact that the subject is involved in a~coreference relationship with
another term of the clause (object, oblique, or adpossessor) has no incidence
on its coding.

The most salient characteristic of Jóola Fóoñi with respect to the expression
of coreference within the clause is a~particularly clearcut distinction between
subject--object coreference and all other possible configurations, including
sub\-ject--ob\-lique coreference and subject--adpossessor coreference:

\begin{itemize}
\item Subject--object coreference obligatorily implies verbal marking by means
  of voice markers that reduce by one the number of objects with which the verb
  can combine.
\item In all the other possible configurations, there is no verbal marking, and
  one of the two terms of the clause involved in the coreference relationship
  is encoded in the same way as if it resumed a~referent to be retrieved from
  a~previous sentence.
\end{itemize}

All the voice markers that may be involved in reflexivization have possible
functions other than the marking of reflexivization, and the verb forms they
are part of may be ambiguous between a~reflexive reading and other
interpretations.

Note that, given the very high degree of transitivity prominence of Jóola Fóoñi
and the extensive use of multiple-object constructions, subject-object
coreference in Jóola Fóñi often corresponds to other syntactic types of
coreference in other languages.  In particular, with ditransitive verbs,
agent--theme coreference and agent-goal coreference are just particular cases
of subject--object coreference, and when semantically plausible, are not
treated differently from agent--patient coreference with monotransitive verbs.
Moreover, the fact that beneficiaries are simply encoded as objects (see
\sectref{sec:Creissels:clause-struct}) results in that, in Jóola Fóoñi,
auto-benefaction (or agent--beneficiary coreference) is also a~particular case
of subject--object coreference (see \sectref{sec:Creissels:ooro-agent-benef}).

Subject--object coreference will be described in detail in
\sectref{sec:Creissels:verbal-marking}.  For the moment, we limit ourselves to
illustrating the following two principles:

\begin{itemize}
\item in all the possible coreference relationships within the clause other
  than subject--object coreference, one of the two terms is encoded by means of
  pronouns or indexes that are not specialized in the expression of coreference
  within the clause;
\item the possibility of interpreting non-subject pronouns or indexes as having
  an antecedent within the clause is conditioned by the syntactic hierarchy
  subject $>$ object $>$ oblique.
\end{itemize}

In all the examples quoted in the remainder of this section to illustrate
coreference relationships within the clause other than subject--object
coreference, the pronoun coreferential with another term of the clause can in
principle be also interpreted as resuming a~referent to be retrieved from the
context, although semantically, this latter interpretation is not always
equally plausible, and is sometimes totally excluded for semantic reasons.

Example~(\ref{ex:Creissels:10}) shows that a~possessive pronoun modifying an object
may have the subject as its antecedent.  The same possibility exists for
adpossessor indexes attached to objects.

\ea
  \label{ex:Creissels:10}
  
    \textit{Nawanwañ kalaakak kɔɔla.}\\
    \gll n-a-wan-wañ ka-laak-a-k k-ɔɔl-a\\
    \textsc{ppf}-sI:clA-cultivate-\textsc{rdpl} \textsc{sg}-field(K)-\textsc{d}-clK clK-\textsc{poss}-I:clA\\
    \glt `He$_i$ cultivated his$_{i/j}$ field.'
\z

This configuration (with an object modified by an adpossessor index or
possessive pronoun resuming the subject) is in particular the configuration
found with body-care verbs in constructions in which the object noun specifies
the body part directly affected by the action, and the affected person is
encoded as an adpossessor, as in~(\ref{ex:Creissels:11}).  In such constructions, if
the affected person is represented by a~possessive pronoun or adpossessor index
whose person-gender-number features coincide with those of subject, nothing
indicates whether it must be understood as coreferential with the subject, or
as resuming a~referent to be retrieved from the context:\footnote{ The object
  of body-care verbs may also represent the affected person, and in that case,
  as can be expected from the general rules, coreference with the subject
  requires reflexive marking on the verb -- see \sectref{sec:Creissels:isolated-o}. }

\ea
  \label{ex:Creissels:11}
  
    \ea\label{ex:Creissels:11a}
    
      \textit{Nɐciiciik fʊlɛmpɔɔl.}\\
      \gll n-ɐ-cii-ciik fʊ-lɛmp-ɔɔl\\
      \textsc{ppf}-sI:clA-shave-\textsc{rdpl} \textsc{sg}-beard(F)-I:clA\\
      \glt `He$_i$ shaved his$_{i/j}$ beard.'


    \ex\label{ex:Creissels:11b}
    
      \textit{Nakɔkɔf ukamunool.}\\
      \gll n-a-kɔ-kɔf u-kɐmun-ool\\
      \textsc{ppf}-sI:clA-scratch-\textsc{rdpl} \textsc{sg}-leg(U)-I:clA\\
      \glt `He$_i$ scratched his$_{i/j}$ legs.'

  \z
\z

Example (\ref{ex:Creissels:12}) shows that an adpossessor index attached to an
oblique may have the subject as its antecedent.  In this particular example,
a~non-coreferential reading is ruled out by semantic considerations.

\ea
  \label{ex:Creissels:12}
  
    \textit{Najʊkɩɩjʊkʊbɔ dɩ kucilool.}\\
    \gll n-a-jʊk-ɩɩ-jʊk-ʊ-bɔ dɩ ku-cil-ool\\
    \textsc{ppf}-sI:clA-see-I:clBK-\textsc{rdpl-ep}-I:clB \textsc{prep} \textsc{pl}-eye(K)-I:clA \\
    \glt `He$_i$ saw them there with his$_i$ own eyes.'
  
\z

Examples~(\ref{ex:Creissels:13}) and~(\ref{ex:Creissels:14}) illustrate the possibility
that in multiple-object constructions, an adpossessor index or possessive
pronoun included in an object phrase has another object as its antecedent:

\ea\label{ex:Creissels:13}
    \textit{Kat añɩɩlaw ɩñaayɔɔl!}\\
    \gll kat a-ñɩɩl-a-w ɩñaay-ɔɔl\\
    leave \textsc{sg}-child(A)-\textsc{d}-clA mother(A)-I:clA\\
    \glt `Leave the child$_i$ to his$_{i/j}$ mother.'
\z

 \ea\label{ex:Creissels:14}
    \textit{Nɩsancɛnɔɔsancɛn mɔɔla}\\
    \gll n-ɩ-sancɛn-ɔɔ-sancɛn m-ɔɔl-a\\
    \textsc{ppf}-sI:\textsc{1sg}-speak-I:clA-\textsc{rdpl} clM-\textsc{poss}-I:clA\\
    \glt `I spoke with him about him(self).' lit.\ `I told him$_i$ his$_{i/j}$ (matter).'
\z


Example~(\ref{ex:Creissels:15}) illustrates the possibility that a~possessive pronoun
modifying an oblique has an object as its antecedent:

\ea
  \label{ex:Creissels:15}
  
    \textit{Nɩjʊjʊk ɐkuutɐɐw di ɛlʊʊpɛy yɔɔla.}\\
    \gll n-ɩ-jʊ-jʊk ɐ-kuutɐ-ɐ-w dɩ ɛ-lʊʊp-ɛ-y y-ɔɔl-a\\
    \textsc{ppf}-sI:\textsc{1sg}-see-\textsc{rdpl} \textsc{sg}-thief(A)-\textsc{d}-clA \textsc{prep} \textsc{sg}-house(E)-\textsc{d}-clE clE-\textsc{poss}-I:clA\\
    \glt `I saw the thief$_i$ in his$_{i/j}$ house.'

\z

Example~(\ref{ex:Creissels:16}) shows that an oblique may be the antecedent of
a~possessive pronoun modifying another oblique.

\ea
  \label{ex:Creissels:16}
  
    \textit{Nayaboyabo dɩ ɐniinɐɐw ʊmɛ matɩ sikoorɐɐs sɔɔla}\\
    \gll n-a-yabɔ-yabɔ dɩ ɐ-niinɐ-ɐ-w ʊ-m-ɛ matɩ si-koori-ɐ-s s-ɔɔl-a\\
    \textsc{ppf}-sI:clA-get.married-\textsc{rdpl} \textsc{prep} \textsc{sg}-man(A)-\textsc{d}-clA \textsc{dem}-clA-\textsc{prox} because.of \textsc{pl}-money(S)-\textsc{d}-clS clS-\textsc{poss}-I:clA\\
    \glt `She got married with this man$_i$ because of his$_i$ money.'

\z

Finally, examples~(\ref{ex:Creissels:17}) and~(\ref{ex:Creissels:18}) illustrate
subject--oblique coreference.

\ea
  \label{ex:Creissels:17}
  
    \textit{Najʊjʊk ɛwɛla bajandɩ ɔɔ.}\\
    \gll n-a-jʊ-jʊk ɛ-wɛla bajandɩ $\varnothing$-ɔɔ\\
    \textsc{ppf}-sI:clA-see-\textsc{rdpl} \textsc{sg}-snake(E) near clA-\textsc{pro}\\
    \glt `He$_i$ saw a~snake near him(self)$_{i/j}$.'
\z

 \ea\label{ex:Creissels:18}
  
    \textit{Najanjam kawɛl bɐlɐmukool.}\\
    \gll n-a-jan-jam ka-wɛl bɐlɐmuk-ool\\
    \textsc{ppf}-sI:clA-hear-\textsc{rdpl} \textsc{sg}-noise(K) behind-I:clA\\
    \glt `He$_i$ heard a~noise behind himself$_i$.'

\z

We have not been able to find examples of coreference relationship involving
two objects in a~multiple object construction, and speakers seem to avoid this
configuration, as illustrated for example by the fact that `X~showed~Y to self
(in the mirror)' is rendered literally as `X~showed~Y his/her face (in the
mirror)', with the second term of the coreference relationship encoded as an
adpossessor.


\section{The verbal marking of coreference within the clause}%
\label{sec:Creissels:verbal-marking}

As already mentioned in \sectref{sec:Creissels:coref-within-clause}, in Jóola Fooñi,
verbal marking by means of voice markers is obligatory in case of
subject--object coreference, whereas no verbal marking can be observed in the
other possible configurations.  In \sectref{sec:Creissels:voice-markers}, we briefly
illustrate the reflexive use of the three voice markers involved in reflexive
marking (\textit{‑ɔɔrɔ}, \textit{‑ɔ} and \textit{‑ɔɔr}).  A~detailed description of
the reflexive and non-reflexive uses of each of them is provided in
\sectref{sec:Creissels:ooro} (\textit{‑ɔɔrɔ}), \sectref{sec:Creissels:o} (\textit{‑ɔ}) and
\sectref{sec:Creissels:oor} (\textit{‑ɔɔr}).

\subsection{The voice markers involved in reflexive marking}%
\label{sec:Creissels:voice-markers}

óola Fóoñi has six verbal suffixes involved in the marking of valency
operations, and three of them are involved in reflexive marking:\footnote{ The
  other three are \textit{‑ɛn} `causative', \textit{‑um} `applicative', and
  \textit{‑ɩ} `passive'. }

\begin{itemize}
\item \textit{‑ɔɔrɔ} ($\sim$ \textit{‑ooro} in combination with $+$ATR verb roots),
  labeled `strong reflexive' by \citet{Sapir1965};
\item \textit{‑ɔ} ($\sim$ \textit{‑o} in combination with $+$ATR verb roots),
  labeled `reflexive-descriptive' by \citet{Sapir1965};
\item \textit{‑ɔɔr} ($\sim$ \textit{‑oor} in combination with $+$ATR verb roots),
  labeled `reciprocal' by \citet{Sapir1965}.
\end{itemize}

The behavior of these three suffixes in the inflected forms of the verb
including the reduplicative suffix provides decisive evidence that they must be
analyzed as voice markers forming part the verb stem, rather than reflexive
indexes.\footnote{ he reduplicative suffix cannot be analyzed as carrying
  a~particular TAM value by itself, but it is an obligatory element of two
  non-relative forms of the verb expressing completive aspect and habitual
  aspect, respectively.  It disappears in the corresponding relative forms, for
  example \textit{kʊ-rɛ-rɛg} `they said' / \textit{kʊ-rɛg-ɛ-rɛg} `they say
  (habitually)' vs.\ \textit{kʊ-rɛg-ʊ-m} `that they said' (where \textit{‑ʊ‑} is an
  epenthetic vowel) / \textit{kʊ-rɛg-ɛ-m} `that they say'. } The point is that,
as illustrated in~(\ref{ex:Creissels:19c}) for \textit{‑ɔɔrɔ}, they are systematically
repeated in the inflected forms of the verb including the reduplicative suffix,
like other derivational suffixes forming part of the stem, whereas object
indexes occur just once (\ref{ex:Creissels:19}a--b), either before the reduplicative
suffix (human object indexes) or after it (non-human object indexes).\footnote{
  In the presentation of the examples, the gloss \textsc{refl} is avoided,
  because it might be a~source of confusion, given that each of the suffixes
  involved in reflexivization also has non-reflexive uses. }

\ea
  \label{ex:Creissels:19}
  
    \ea\label{ex:Creissels:19a}
    
      \textit{Ɛyɛnɛy ɛrʊnrʊmʊfɔ.}\\
      \gll ɛ-yɛn-ɛ-y ɛ-rʊn-rʊm-ʊ-\textbf{fɔ}\\
      \textsc{sg}-dog(E)-\textsc{d}-clE sI:clE-bite-\textsc{rdpl-ep}-I:clF\\
      \glt `The dog bit it (\textit{fujicelɐf} `the male goat').'


    \ex\label{ex:Creissels:19b}
    
      \textit{Ɛyɛnɛy ɛrʊmɔɔrʊm.}\\
      \gll ɛ-yɛn-ɛ-y ɛ-rʊm-\textbf{ɔɔ}-rʊm\\
      \textsc{sg}-dog(E)-\textsc{d}-clE sI:clE-bite-I:clA-\textsc{rdpl}\\
      \glt `The dog bit him (\textit{añɩɩlaw} `the child').'


    \ex\label{ex:Creissels:19c}
    
      \textit{Ɛyɛnɛy ɛrʊmɔɔrɔrʊmɔɔrɔ.}\\
      \gll ɛ-yɛn-ɛ-y ɛ-rʊm-\textbf{ɔɔrɔ}-rʊm\textbf{ɔɔrɔ}\\
      \textsc{sg}-dog(E)-\textsc{d}-clE sI:clE-bite-\textsc{ɔɔrɔ-rdpl}\\
      \glt `The dog bit itself.'

  \z
\z

In their reflexive function, these three suffixes equally reduce by one the
number of objects compatible with the verb in its underived form, and the
semantic correlate of this reduction is that the participant roles fulfilled by
the subject and (one of) the object(s) in the construction of the base verb are
cumulated by a~single participant, encoded as the subject of the derived verb.
This is illustrated in~(\ref{ex:Creissels:20}) for \textit{‑ɔɔrɔ}, in~(\ref{ex:Creissels:21})
for \textit{‑ɔ}, and in~(\ref{ex:Creissels:22}) for \textit{‑ɔɔr}.

\ea
  \label{ex:Creissels:20}
  
    \ea\label{ex:Creissels:20a}
    
      % FIXME lack of capital ɐ
      \textit{ɐniinɐɐw basɔfɛ epimbeney m'aabʊj ɐkuutɐɐw.}\\
      \gll ɐ-niine-ɐ-w ba-sɔf-ɛ e-pimben-e-y man a-bʊj ɐ-kuutɐ-ɐ-w\\
      \textsc{sg}-man(A)-\textsc{d}-clA \textsc{cvb}-catch-\textsc{cvb} \textsc{sg}-gun(E)-\textsc{d}-clE \textsc{csc} sI:clA-kill \textsc{sg}-thief(A)-\textsc{d}-clA\\
      \glt `The man took the gun and killed the thief.'


    \ex\label{ex:Creissels:20b}
    
      % FIXME lack of capital ɐ
      \textit{ɐniinɐɐw baraanɛ bɛɛbɛn m'aabʊjɔɔrɔ.}\\
      \gll ɐ-niine-ɐ-w ba-raan-ɛ b-ɛɛbɛn man a-bʊj-ɔɔrɔ\\
      \textsc{sg}-man(A)-\textsc{d}-clA \textsc{cvb}-drink-\textsc{cvb} \textsc{sg}-poison(B) \textsc{csc} sI:clA-kill-\textsc{ɔɔrɔ}\\
      \glt `The man committed suicide (lit.\ `killed himself') by drinking poison.'

  \z
  \z

  \ea\label{ex:Creissels:21}
  
    \ea\label{ex:Creissels:21a}
      \textit{Nan alaañʊlɔm bɩɩtab, asɛɛkaw napɔs añɩɩlaw.}\\
      \gll n-an a-laañ-ʊlɔ-m b-ɩɩt-a-b a-sɛɛk-a-w n-a-pɔs a-ñɩɩl-a-w.\\
      clN-\textsc{rel} sI:clA-return-\textsc{ven-act$_1$} \textsc{sg}-rice.field(B)-\textsc{d}-clB \textsc{sg}-woman(A)-\textsc{d}-clA \textsc{ppf}-sI:clA-wash \textsc{sg}-child(A)-\textsc{d}-clA \\
      \glt `After returning from the rice field, the woman washed the child.'


    \ex\label{ex:Creissels:21b}
      \textit{Nan alaañʊlɔm bɩɩtab, asɛɛkaw napɔsɔ.}\\
      \gll n-an a-laañ-ʊlɔ-m b-ɩɩt-a-b a-sɛɛk-a-w n-a-pɔs-ɔ\\
      clN-\textsc{rel} sI:clA-return-\textsc{ven-act$_1$} \textsc{sg}-rice.field(B)-\textsc{d}-clB \textsc{sg}-woman(A)-\textsc{d}-clA \textsc{ppf}-sI:clA-wash-\textsc{ɔ}\\
      \glt `After returning from the rice field, the woman washed (herself).'

  \z
  \z

  \ea\label{ex:Creissels:22}
 
    \ea\label{ex:Creissels:22a}
      \textit{Ŋar fʊmbanjaf man uguuy añɩɩlaw.}\\
      \gll ŋar fʊ-mbanj-a-f man u-guuy a-ñɩɩl-a-w\\
      take \textsc{sg}-blanket(F)-\textsc{d}-clF \textsc{csc} sI:\textsc{2sg}-cover \textsc{sg}-child(A)-\textsc{d}-clA\\
      \glt `Take the blanket and cover the child (with it).'


    \ex\label{ex:Creissels:22b}
    
      \textit{Ŋar fʊmbanjaf man uguuyoor.}\\
      \gll ŋar fʊ-mbanj-a-f man u-guuy-oor\\
      take \textsc{sg}-blanket(F)-\textsc{d}-clF \textsc{csc} sI:\textsc{2sg}-cover-\textsc{ɔɔr}\\
      \glt `Take the blanket and cover yourself (with it).'

  \z
\z

However, none of these three suffixes is specialized in reflexive marking.
Moreover, as reflexivizers, they are not interchangeable, and their
non-reflexive uses are very different.

\subsection{Reflexive and non-reflexive uses of \textit{‑ɔɔrɔ}}%
\label{sec:Creissels:ooro}

\subsubsection{\textit{‑ɔɔrɔ} as the default marker of subject--object coreference}%
\label{sec:Creissels:ooro-default}

The verbal suffix \textit{‑ɔɔrɔ} can be analyzed as the default marker of
subject--object coreference, freely available for the transitive verbs that do
not belong to a~restricted semantic class of transitive verbs that regularly
mark subject--object coreference by means of \textit{‑ɔ} (see
\sectref{sec:Creissels:o}), and are not lexically specified as marking
subject--object coreference by means of \textit{‑ɔ} or \textit{‑ɔɔr} (see
\sectref{sec:Creissels:o} and \sectref{sec:Creissels:oor}).

\subsubsection{\textit{‑ɔɔrɔ} and the coding of A--P coreference}%
\label{sec:Creissels:ooro-ap}

The reflexive use of \textit{‑ɔɔrɔ} has already been illustrated above with
\textit{rʊm-ɔɔrɔ} `bite oneself' in~(\ref{ex:Creissels:19c}) and \textit{bʊj-ɔɔrɔ} `kill
oneself' in~(\ref{ex:Creissels:20b}).  Examples~(\ref{ex:Creissels:23})
to~(\ref{ex:Creissels:25}) further illustrate the use of \textit{‑ɔɔrɔ} to derive
intransitive verbs expressing agent--patient coreference, and more generally,
coreference between the core arguments of monotransitive verbs.

\ea
  \label{ex:Creissels:23}
  
    \textit{An acɛsɔɔrʊtɔɔl, ɔɔ acɛsɔɔrɔɔrɔɛ.}\\
    \gll an a-cɛsɔɔr-ʊt-ɔɔl $\varnothing$-ɔɔ a-cɛsɔɔr-ɔɔrɔ-ɛ\\
    person(A) sI:clA-chase.away-\textsc{neg}-I:clA clA-\textsc{pro} sI:clA-chase.away-\textsc{ɔɔrɔ-cpl}\\
    \glt `Nobody chased him away, he himself decided to leave.'
    lit.\ `it's him who chased himself away'
\z

  \ea\label{ex:Creissels:24}
    \textit{Naŋɔɔlɛnŋɔɔlɛn apacɛn bʊkaakʊ, barɛ aŋɔɔlɛnʊt apacɛnɔɔrɔ.}\\
    \gll n-a-ŋɔɔlɛn-ŋɔɔlɛn a-pacɛn bʊk-aa-kʊ barɛ a-ŋɔɔlɛn-ʊt a-pacɛn-ɔɔrɔ\\
    \textsc{ppf}-sI:clA-be.able-\textsc{rdpl} sI:clA-save clBK-other-clBK but sI:clA-be.able-\textsc{neg} sI:clA-save-\textsc{ɔɔrɔ}\\
    \glt `He was able to save the others, but not to save himself.'
\z

  \ea\label{ex:Creissels:25}
  
    \textit{Anɔɔsan atɛbɛnɔɔrɔm, Atɩjamɩt pan awalɛnɔɔl,
    barɛ anɔɔsan awalɛnɔɔrɔm, Atɩjamɩt panatɛbɛnɔɔl}\\
    \gll $\varnothing$-anɔɔsan a-tɛbɛn-ɔɔrɔ-m Atɩjamɩt pan a-walɛn-ɔɔl
    barɛ $\varnothing$-anɔɔsan a-walɛn-ɔɔrɔ-m Atɩjamɩt pan a-tɛbɛn-ɔɔl\\
    clA-any sI:clA-lift.up-\textsc{ɔɔrɔ-act$_1$} God(A) \textsc{fut} sI:clA-bring.down-I:clA
    but clA-any sI:clA-bring.down-\textsc{ɔɔrɔ-act$_1$} God(A) \textsc{fut} sI:clA-lift.up-I:clA\\
    \glt `Whoever exalts himself, God will humble him, but whoever humbles himself, God will exalt him.' (from the Jóola Fóoñi translation of the New Testament)
\z

In our data, the coding of A--P coreference by means of \textit{‑ɔɔrɔ} is
attested by the verbs listed in Table~\ref{tab:Creissels:ooro-ap}:\footnote{ In the
  case of polysemous verbs, the translation of the base verb given in this
  table is that corresponding to the meaning of the reflexive derivate attested
  in our data. }

\begin{table}[ht]
  \centering
  \begin{tabular}{ll}
    \lsptoprule
  {Reflexive verb} & {Base verb} \\
    \hline
    \textit{bɛbɛn-ɔɔrɔ} & \textit{bɛbɛn} `calm s.o.\ down' \\
    \textit{buken-ooro} & \textit{buken} `hurt s.o.' \\
    \textit{bʊj-ɔɔrɔ} & \textit{bʊj} `kill s.o.' \\
    \textit{bʊnt-ɔɔrɔ} & \textit{bʊnt} `fool s.o.' \\
    \textit{cɛsɔɔr-ɔɔrɔ} & \textit{cεsɔɔr} `chase s.o.\ away' \\
    \textit{gamɛn-ɔɔrɔ} & \textit{gamɛn} `judge s.o.' \\
    \textit{jɛl-ɔɔrɔ} & \textit{jɛl} `insult s.o.' \\
    \textit{jʊk-ɔɔrɔ} & \textit{jʊk} `see s.o.' \\
    \textit{jʊʊr-ɔɔrɔ} & \textit{jʊcɛr} `look at s.o.' \\
    \textit{kambɛn-ɔɔrɔ} & \textit{kambɛn} `lock s.o.' \\
    \textit{kɐɐnum-ooro} & \textit{kɐɐnum} `take care of s.o.' \\
    \textit{lat-ɔɔrɔ} & \textit{lat} `hate s.o.' \\
    \textit{manj-ɔɔrɔ} & \textit{manj} `know s.o.' \\
    \textit{maŋ-ɔɔrɔ} & \textit{maŋ} `like s.o.' \\
    \textit{nag-ɔɔrɔ} & \textit{nag} `hit s.o.' \\
    \textit{pacɛn-ɔɔrɔ} & \textit{pacεn} `save s.o.' \\
    \textit{pɐjul-ooro} & \textit{pɐjul} `separate s.o.\ out' \\
    \textit{rambɛn-ɔɔrɔ} & \textit{rambɛn} `help s.o.' \\
    \textit{rʊm-ɔɔrɔ} & \textit{rʊm} `bite s.o.' \\
    \textit{sal-ɔɔrɔ} & \textit{sal} `praise s.o.' \\
    \textit{tɛbɛn-ɔɔrɔ} & \textit{tεbεn} `glorify s.o.' \\
    \textit{walɛn-ɔɔrɔ} & \textit{walεn} `humiliate s.o.' \\
    \lspbottomrule
  \end{tabular}
  \caption{Verbs attesting the use of \textit{‑ɔɔrɔ} to mark A--P coreference}%
  \label{tab:Creissels:ooro-ap}
\end{table}

\subsubsection{\textit{‑ɔɔrɔ} and the coding of agent--goal coreference}%
\label{sec:Creissels:ooro-agent-goal}

Given the extensive use of multiple-object constructions in Jóola Fóoñi, with
ditransitive verbs, agent--goal coreference is a~particular case of
subject--object coreference, and is consequently productively encoded by means
of \textit{‑ɔɔrɔ} -- examples~(\ref{ex:Creissels:26}) and~(\ref{ex:Creissels:27}).

\ea
  \label{ex:Creissels:26}
    \textit{Ʊjʊk, aw faŋɩ nʊlanlaañ umɐrulooro.}\\
    \gll ʊ-jʊk aw faŋ-ɩ n-ʊ-lan-laañ u-mɐrul-ooro\\
    sI:\textsc{2sg}-look \textsc{2sg.pro} self-\textsc{2sg} \textsc{ppf}-sI:\textsc{2sg}-return-\textsc{rdpl} sI:\textsc{2sg}-answer-\textsc{ɔɔrɔ}\\
    \glt `Look, you answered your own question yourself again.'
    lit.\ `Look, you answered yourself again.'
\z

 \ea\label{ex:Creissels:27}
    \textit{Nɩmammaŋ man ʊsɛnɔɔrɔ kʊnak kufeeji man ʊwɔnɔɔr jak.}\\
    \gll n-ɩ-mam-maŋ man ʊ-sɛn-ɔɔrɔ kʊ-nak ku-feeji man ʊ-wɔnɔɔr jak\\
    \textsc{ppf}-sI:\textsc{1sg}-want-\textsc{rdpl} \textsc{csc} sI:2SG-give-\textsc{ɔɔrɔ}
    \textsc{pl}-day(K) clK-three \textsc{csc} sI:\textsc{2sg}-think well\\
    \glt `I would like you to give yourself three days to think about it well.'
\z

\subsubsection{\textit{‑ɔɔrɔ} and the coding of agent--beneficiary coreference}%
\label{sec:Creissels:ooro-agent-benef}

The suffix \textit{‑ɔɔrɔ} is also productively used to encode autobenefaction
(i.\,e., agent--beneficiary coreference) -- examples~(\ref{ex:Creissels:28})
and~(\ref{ex:Creissels:29}).  This is consistent with the fact that, in Jóola Fóoñi,
beneficiaries are simply encoded as objects.

\ea
  \label{ex:Creissels:28}
  
    \ea\label{ex:Creissels:28a}
  
      \textit{Nɩnɔɔmɛ asɛɛkom ewoto.}\\
      \gll n-ɩ-nɔɔm-ɛ a-sɛɛk-ɔm e-woto\\
      \textsc{ppf}-sI:\textsc{1sg}-buy-\textsc{cpl} \textsc{sg}-woman(A)-I:\textsc{1sg} \textsc{sg}-car(E)\\
      \glt `I bought a car for my wife.'


    \ex\label{ex:Creissels:28b}
    
      \textit{Nɩnɔɔmɔɔrɔɛ ewoto.}\\
      \gll n-ɩ-nɔɔm-ɔɔrɔ-ɛ e-woto\\
      \textsc{ppf}-sI:\textsc{1sg}-buy-\textsc{ɔɔrɔ-cpl} \textsc{sg}-car(E)\\
      \glt `I bought a car for myself.'

  \z
  \z

  \ea\label{ex:Creissels:29}
  
    \ea\label{ex:Creissels:29a}
    
      \textit{Fɔk ɩñɛs añɔɔlɔm asɛɛk.}\\
      \gll fɔk ɩ-ñɛs a-ñɔɔl-ɔm a-sɛɛk\\
      \textsc{oblg} sI:\textsc{1sg}-look.for \textsc{sg}-child(A)-I:\textsc{1sg} \textsc{sg}-woman(A)\\
      \glt `I must look for a wife for my son.'


    \ex\label{ex:Creissels:29b}
    
      \textit{Fɔk ɩñɛsɔɔrɔ asɛɛk.}\\
      \gll fɔk ɩ-ñɛs-ɔɔrɔ a-sɛɛk\\
      \textsc{oblg} sI:\textsc{1sg}-look.for-\textsc{ɔɔrɔ} \textsc{sg}-woman(A)\\
      \glt `I must look for a wife for myself.'

  \z
\z

\subsubsection{\textit{‑ɔɔrɔ} as a~marker of self-intensification of the subject}%
\label{sec:Creissels:ooro-intens}

As a~valency operator, \textit{‑ɔɔrɔ} is exclusively used to encode
reflexivization in one of the configurations illustrated in the previous
sections.  However, in addition to its use as the default marker of
subject--object coreference, \textit{‑ɔɔrɔ} is also productively used as a~marker
of \textsc{self-intensification of the subject}.  In this use, \textit{‑ɔɔrɔ} has
no incidence on the valency properties of the verb to which it attaches, and
its contribution to the meaning of the clause corresponds to that more commonly
expressed cross-linguistically by free intensifying self-forms having scope on
the subject, as in \textit{John HIMSELF came} or \textit{John came
  HIMSELF}.\footnote{ On the general question of the relationship between
  self-intensification and reflexivization in typological perspective, readers
  are referred to \citet{GastSiemund2006}, \citet{KoenigGast2006}, and
  references therein. }

In its function of self-intensification of the subject, \textit{‑ɔɔrɔ} can attach
to intransitive verbs, as in~(\ref{ex:Creissels:30}).

\ea\label{ex:Creissels:30}

    \textit{Inje ɩjawɔɔrɔɛ bɛɛbɔ.}\\
    \gll inje ɩ-jaw-ɔɔrɔ-ɛ bɛɛ-bɔ\\
    \textsc{1sg.pro} sI:\textsc{1sg}-go-\textsc{ɔɔrɔ-cpl} \textsc{all}-I:clB\\
    \glt `It's me who went there in person.'

\z

With intransitive verbs, the only possible ambiguity is between
self-in\-ten\-si\-fi\-ca\-tion of the subject and autobenefaction.

With transitive verbs, the choice between the possible interpretations of
\textit{‑ɔɔrɔ} (coreference between the subject and another core argument,
autobenefaction, or self-intensification of the subject) is partly conditioned
by the presence vs.\ absence of object NPs or indexes.  However, the choice
between an autobenefactive reading and a~self-intensification reading can only
rely on the context, since in the autobenefactive use of \textit{ɔɔrɔ}-verbs, the
valency operation is not apparent.  For example, in~(\ref{ex:Creissels:31a}),
\textit{kambɛn-ɔɔrɔ} is interpreted as encoding agent--patient coreference (`lock
self').  In~(\ref{ex:Creissels:31b}), the presence of the object index \textit{‑kɔ}
excludes this possibility, but the first part of the sentence is decisive for
the choice between the two possible readings `close s.th.\ for self'
(autobenefactive) and `close s.th.\ self' (self-intensification of subject).

\ea
  \label{ex:Creissels:31}
  
    \ea\label{ex:Creissels:31a}
    
      \textit{Jaw ʊkambɛnɔɔrɔ dɩ kalɩmbɩsak man ʊwʊraŋ.}\\
      \gll jaw ʊ-kambɛn-ɔɔrɔ dɩ ka-lɩmbɩs-a-k man ʊ-wʊraŋ\\
      go sI:\textsc{2sg}-close-\textsc{ɔɔrɔ} \textsc{prep} \textsc{sg}-room(K)-\textsc{d}-clK
      \textsc{csc} sI:\textsc{2sg}-undress\\
      \glt `Go and lock yourself in the room to change your clothes.'


    \ex\label{ex:Creissels:31b}
    
      \textit{Nɛɛnɔɔ akambɛn kajʊnkʊtak, naanɛ ɩjaw ɩkambɛnɔɔrɔkɔ.}\\
      \gll n-ɛɛn-ɔɔ a-kambɛn ka-jʊnkʊt-a-k n-aanɛ ɩ-jaw ɩ-kambɛn-ɔɔrɔ-kɔ\\
      \textsc{ppf}-sI:\textsc{1sg}.tell-I:clA sI:clA-close \textsc{sg}-door(K)-\textsc{d}-clK
      \textsc{ppf}-sI:clA.tell sI:\textsc{1sg}-go sI:\textsc{1sg}-close-\textsc{ɔɔrɔ}-I:clK\\
      \glt `I told him to close the door, and he told me to go and close it myself.'

  \z
\z

In the case or \textit{rɛg-ɔɔrɔ} $<$ \textit{rɛg} `tell', the possibility of
a~reflexive interpretation is widely attested in our data (in Jóola Fóoñi, as
in many other languages, `think' or `imagine' can be expressed as lit.\ `tell
to self'), but in~(\ref{ex:Creissels:32}), the context is hardly compatible with the
agent--addressee coreference reading, leaving self-intensification as the only
plausible reading.

\ea
  \label{ex:Creissels:32}
  
    \textit{Ʊcɛɛŋɔɔl, ɔɔ lɛt añɩɩl, pan aŋoolɛn arɛgɔɔrɔ.}\\
    \gll ʊ-cɛɛŋ-ɔɔl $\varnothing$-ɔɔ $\varnothing$-lɛt a-ñɩɩl pan a-ŋɔɔlɛn a-rɛg-ɔɔrɔ\\
    sI:\textsc{2sg}-ask-I:clA clA-\textsc{pro} sI:clD-not.to.be \textsc{sg}-child(A)
    \textsc{fut} sI:clA-be.able sI:clA-tell-\textsc{ɔɔrɔ}\\
    \glt `Ask him, he is not a~child, he will be able to tell (it) himself.'

\z

Examples~(\ref{ex:Creissels:33}) to~(\ref{ex:Creissels:36}) provide further illustrations
of the role of the context in the interpretation of \textit{‑ɔɔrɔ} as expressing
autobenefaction or self-intensification of the subject.

\ea
  \label{ex:Creissels:33}
  
    \ea\label{ex:Creissels:33a}
    
      \textit{Anaw, tanɔɔsan ʊlakɔɛ, ñɛs man ukaanɔɔrɔ karɛɛs kajakɛ.}\\
      \gll an-a-w t-anɔɔsan ʊ-lakɔ-ɛ ñɛs man ʊ-kaan-ɔɔrɔ ka-rɛɛs k-a-jak-ɛ\\
      person(A)-\textsc{d}-clA clT-any sI:\textsc{2sg}-be-\textsc{act$_0$} try
      \textsc{csc} sI:\textsc{2sg}-make-\textsc{ɔɔrɔ} \textsc{sg}-name(K)
      clK-\textsc{ptcp}-be.good-\textsc{act$_0$}\\
      \glt `Wherever you may be, try to build a~good reputation (for yourself).'


    \ex\label{ex:Creissels:33b}
    
      \textit{Nan ʊbajʊt arambɛna, fɔk ʊkaanɔɔrɔ bʊrɔkab bɩɩya.}\\
      \gll nan ʊ-baj-ʊt a-rambɛna fɔk ʊ-kaan-ɔɔrɔ bʊ-rɔk-a-b b-ɩɩya\\
      if sI:\textsc{2sg}-have-\textsc{NEG} \textsc{sg}-helper(A) \textsc{oblg}
      sI:\textsc{2sg}-make-\textsc{ɔɔrɔ} \textsc{sg}-work(B)-\textsc{d}-clB clB-your\\
      \glt `If you have nobody to help you, you must do your work yourself.'

  \z
  \z

  \ea\label{ex:Creissels:34}
  
    \ea\label{ex:Creissels:34a}
    
      \textit{Fɔk ɩñɛsɔɔrɔ asɛɛk.}\\
      \gll fɔk ɩ-ñɛs-ɔɔrɔ a-sɛɛk\\
      \textsc{oblg} sI:\textsc{1sg}-look.for-\textsc{ɔɔrɔ} \textsc{sg}-woman(A)\\
      \glt `I must look for a~wife (for) myself.'


    \ex\label{ex:Creissels:34b}
    
      \textit{Ampaɔm naanɛ man añɛsɔm asɛɛk, barɛ inje nɛɛnɔɔ  pan ɩñɛsɔɔrɔ.}\\
      \gll a-mpa-ɔm n-aanɛ man a-ñɛs-ɔm a-sɛɛk barɛ inje n-ɛɛn-ɔɔ pan ɩ-ñɛs-ɔɔrɔ\\
      \textsc{sg}-father(A)-I:\textsc{1sg} \textsc{ppf}-sI:clA.say \textsc{csc}
      sI:clA-look.for-I:\textsc{1sg} \textsc{sg}-woman(A) but \textsc{1sg.pro}
      \textsc{ppf}-sI:\textsc{1sg}.say-I:clA \textsc{fut} sI:\textsc{1sg}-look.for-\textsc{ɔɔrɔ}\\
      \glt `My father said he would look for a~wife for me,
      but I told him that I will look for (a~wife) myself.'

  \z
  \z

  \ea\label{ex:Creissels:35}
  
    \textit{Nan asɛɛkɔm ɐsumutum, inje kɐsiilooro.}\\
    \gll n-an a-sɛɛk-ɔm ɐ-sumut-u-m inje kɐ-siil-ooro\\
    clN-\textsc{rel} \textsc{sg}-woman(A)-I:\textsc{1sg}
    sI:clA-be.sick-\textsc{ep-act$_1$} \textsc{1sg.pro}
    \textsc{inf}(K)-cook-\textsc{ɔɔrɔ}\\
    \glt `When my wife is sick, I do the cooking myself.'
\z

  \ea\label{ex:Creissels:36}
  
    \textit{Ɛlʊʊpɛy yati ɩñaam umbɐ, inje ɩtɛɛpɔɔrɔɛyɔ.}\\
    \gll ɛ-lʊʊp-ɛ-y y-atɩ ɩñaam $\varnothing$-umbɐ inje ɩ-tɛɛp-ɔɔrɔ-ɛ-yɔ\\
    \textsc{sg}-house(E)-\textsc{d}-clE clE-\textsc{gen} mother(A) clA-my
    \textsc{1sg.pro} sI:\textsc{1SG}-build-\textsc{ɔɔrɔ-cpl}-I:clE\\
    \glt `My mother's house, I built it myself.'

\z

However, it may also happen that the lexical meaning of the verb helps to solve
the ambiguity.  For exemple \textit{manj-ɔɔrɔ} $<$ \textit{manj} `know' may be used
with the reflexive reading `know oneself', but in~(\ref{ex:Creissels:37}), the
presence of a~complement clause excludes this possibility, and the
self-intensification reading is the only one really available, since
semantically, an autobenefactive interpretation is difficult to imagine.

\ea
  \label{ex:Creissels:37}
  
    \textit{Inje nɩmanjɔɔrɔmanjɔɔrɔ man ɩjɛɛm b'ɛɛkaanɛy.}\\
    \gll inje n-ɩ-manj-ɔɔrɔ-manjɔɔrɔ m-an ɩ-ja-ɛ-m bɛɛ ɛ-kaan-ɛ-y\\
    \textsc{1sg.pro} \textsc{ppf}-sI:\textsc{1sg}-know-\textsc{ɔɔrɔ-rdpl}
    clM-\textsc{rel} sI:\textsc{1sg}-go-\textsc{icpl-act$_1$} \textsc{all}
    \textsc{inf}(E)-do-\textsc{d}-clE\\
    \glt `I myself know how I will do (that).'

\z

\subsubsection{The lexicalization of \textit{ɔɔrɔ}-derivates}%
\label{sec:Creissels:ooro-lexicalization}

As a~rule, \textit{ɔɔrɔ}-derivates are semantically transparent.  In this
respect, \textit{‑ɔɔrɔ} behaves very differently from the other two suffixes
involved in the expression of reflexivization, which have a~marked tendency
toward lexicalization.

There are, however, a~few \textit{ɔɔrɔ}-derivates with a~lexicalized meaning.
For example, \textit{sɔf-ɔɔrɔ} is attested with two meanings, `strive to do
s.th.'  and `keep from doing s.th.', which cannot be straightforwardly
predicted from the meaning of the base verb \textit{sɔf} `catch', although it is
not very difficult to imagine how they developed from `catch self'.

\textit{Tɛb-ɔɔrɔ} `invite oneself' (in the sense of `go s.wh.\ without having
been called') $<$ \textit{tɛb} `carry' is another example of lexicalization of an
\textit{ɔɔrɔ}-derivate.

\subsection{Reflexive and non-reflexive uses of \textit{‑ɔ}}%
\label{sec:Creissels:o}

\subsubsection{The reflexive use of \textit{‑ɔ} with body-care verbs}%
\label{sec:Creissels:o-bodycare}

As already illustrated by example~(\ref{ex:Creissels:21}) (reproduced here
as~(\ref{ex:Creissels:38})), the verbal suffix \textit{‑ɔ} is used to express
a~reflexive meaning with body-care verbs, if no particular body part is
mentioned and the object represents the person affected by the action (for
body-care verbs with a~body-part noun in object role, see
\sectref{sec:Creissels:coref-within-clause}).

\ea
  \label{ex:Creissels:38}
    \ea \label{ex:Creissels:38a}
      \textit{Nan alaañʊlɔm bɩɩtab, asɛɛkaw napɔs añɩɩlaw.}\\
      \gll n-an a-laañ-ʊlɔ-m b-ɩɩt-a-b a-sɛɛk-a-w n-a-pɔs a-ñɩɩl-a-w.\\
      clN-\textsc{rel} sI:clA-return-\textsc{ven-act$_1$} \textsc{sg}-rice.field(B)-\textsc{d}-clB \textsc{sg}-woman(A)-\textsc{d}-clA \textsc{ppf}-sI:clA-wash \textsc{sg}-child(A)-\textsc{d}-clA \\
      \glt `After returning from the rice field, the woman washed the child.'


    \ex \label{ex:Creissels:38b}
      \textit{Nan alaañʊlɔm bɩɩtab, asɛɛkaw napɔsɔ.}\\
      \gll n-an a-laañ-ʊlɔ-m b-ɩɩt-a-b a-sɛɛk-a-w n-a-pɔs-ɔ\\
      clN-\textsc{rel} sI:clA-return-\textsc{ven-act$_1$} \textsc{sg}-rice.field(B)-\textsc{d}-clB \textsc{sg}-woman(A)-\textsc{d}-clA \textsc{ppf}-sI:clA-wash-\textsc{ɔ}\\
      \glt `After returning from the rice field, the woman washed (herself).'
  \z
\z

In our data, the reflexive use of \textit{‑ɔ} with body-care verbs is attested by
verbs listed in Table~\ref{tab:Creissels:o-verbs}.

\begin{table}[ht]
  \centering
  \begin{tabular}{ll}
    \lsptoprule
    {Reflexive verb} & {Base verb} \\
    \hline
    \textit{bʊŋ-ɔ} `braid (self)' & \textit{bʊŋ} `braid (s.th., or s.o.\ else)' \\
    \textit{ciik-o} `shave (self)' & \textit{ciik} `shave (s.th., or s.o.\ else)' \\
    \textit{kaan-ɔ} `put (clothes) on self' & \textit{kaan} `put (clothes) on s.o.\ else' \\
    \textit{kɔk-ɔ} `tie cloth around the waist' & \textit{kɔk} `tie (s.th.)' \\
    \textit{ñaaw-ɔ} `bathe (self)' & \textit{ñaaw} `bathe (s.o.\ else)' \\
    \textit{pɔs-ɔ} `wash (self)' & \textit{pɔs} `wash (s.th., or s.o.\ else)' \\
    \lspbottomrule
  \end{tabular}
  \caption{Body-care verbs attesting the reflexive use of \textit{‑ɔ}}%
  \label{tab:Creissels:o-verbs}
\end{table}

In all cases, it is also possible to have \textit{‑ɔɔrɔ} instead of \textit{‑ɔ}, as in (\ref{ex:Creissels:39}).

\ea
  \label{ex:Creissels:39}

    \textit{Añɩɩlaw nɐrindiiŋ tembe yatɩ kapɔsɔɔrɔ.}\\
    \gll a-ñɩɩl-a-w n-ɐ-rin-diiŋ tembe y-atɩ ka-pɔs-ɔɔrɔ\\
    \textsc{sg}-child(A)-\textsc{d}-clA \textsc{ppf}-sI:clA-reach-\textsc{rdpl} time(E)
    clE-\textsc{gen} \textsc{inf}(K)-wash-\textsc{ɔɔrɔ}\\
    \glt `The child is old enough to be able to wash himself.'

\z

The use of \textit{‑ɔɔrɔ} instead of \textit{‑ɔ} adds to the reflexive meaning an
intensifying nuance.  Consequently, a~possible analysis is that \textit{‑ɔ} is
still present underlyingly, but for phonetic reasons, the addition of
\textit{‑ɔɔrɔ} to mark self-intensification of the subject makes it invisible.

\subsubsection{The reflexive use of \textit{‑ɔ}: isolated cases}%
\label{sec:Creissels:isolated-o}

In addition to body-care verbs, for which the suffixation of \textit{‑ɔ} is the
regular and semantically unmarked way to encode subject--object coreference,
\textit{‑ɔ} is used in reflexive function, without any obvious explanation, in
the two verb pairs given in Table~\ref{tab:Creissels:o-more-verbs}, one of which is
formally an equipollent pair.

\begin{table}[ht]
  \centering
  \begin{tabular}{ll}
    \lsptoprule
    Reflexive verb & Corresponding non-reflexive verb \\
                           
    \hline
    \textit{lɩb-ɔ} `cut self' & \textit{lɩb} `cut' \\
    \textit{rɔɔk-ɔ} `carry on one's head' & \textit{rɔɔk-ɛn} `load s.th.\ onto s.th.' \\
    \lspbottomrule
  \end{tabular}
  \caption{Other verbs attesting the reflexive use of \textit{‑ɔ}}%
  \label{tab:Creissels:o-more-verbs}
\end{table}

There may be other similar cases, but these are the only ones we came across.

\subsubsection{The quasi-reflexive use of \textit{‑ɔ}}%
\label{sec:Creissels:quasi-refl-o}

Jóola Fóoñi has several verbs of spontaneous motion that are formally related
to a~verb of caused motion in one of the following two ways:

\begin{itemize}
\item either the spontaneous-motion verb derives from the caused-motion verb
  via the addition of \textit{‑ɔ} (Table~\ref{tab:Creissels:o-spontaneous-motion}),
\item or the spontaneous-motion and caused-motion verbs share a~root not
  attested by itself as a~verb stem, the spontaneous-motion verb being derived
  from this root via the addition of \textit{‑ɔ}, and the caused-motion verb via
  the addition of the causative suffix \textit{‑ɛn}
  (Table~\ref{tab:Creissels:o-spontaneous-motion-en}\footnote{ In the case of
    \textit{wal-ɛn} `set down', it is interesting to observe the contrast between
    \textit{wal-ɔ} `get down' and \textit{wal-ɛn-ɔɔr} lit.\ `bring self down' $>$
    `humble self', as in example~(\ref{ex:Creissels:25}). }).
\end{itemize}

\begin{table}[ht]
  \centering
  \begin{tabular}{ll}
    \lsptoprule
    {spontaneous-motion verb} & {caused-motion verb} \\
    \hline
    \textit{fim-o} `turn over on one's stomach' & \textit{fim} `turn over (pot)' \\
    \textit{jup-o} `embark' & \textit{jup} `load s.th.up, insert into' \\
    \textit{lak-ɔ} `sit down' & \textit{lak} `put (a~pot) on the fire' \\
    \textit{rup-o} `emerge from' & \textit{rup} `pull s.th.\ up' \\
    \textit{wɛɛt-ɔ} `lie on back' & \textit{wɛɛt} `spread out' \\
    \lspbottomrule
  \end{tabular}
  \caption{Spontaneous-motion verbs derived from the corresponding
    caused-motion verb via the addition of \textit{‑ɔ}}%
  \label{tab:Creissels:o-spontaneous-motion}
\end{table}

\begin{table}[ht]
  \centering
  \begin{tabular}{ll}
    \lsptoprule
    {spontaneous-motion verb} & {caused-motion verb} \\
    \hline
    \textit{fɩnt-ɔ} `lie down' & \textit{fɩnt-ɛn} `make lie down' \\
    \textit{ñɩt-ɔ} `climb' & \textit{ñɩt-ɛn} `hoist up' \\
    \textit{pan-ɔ} `move aside (intr.) & \textit{pan-ɛn} `move aside (tr.)' \\
    \textit{tink-o} `lean (intr.)' & \textit{tink-en} `tilt, bow' \\
    \textit{wal-ɔ} `get down' & \textit{wal-ɛn} `set down' \\
    \textit{yɩt-ɔ} `get up, stand up' & \textit{yɩt-ɛn} `lift s.th.\ up' \\
    \lspbottomrule
  \end{tabular}
  \caption{Spontaneous-motion verbs marked by \textit{‑ɔ} corresponding to
    caused-motion verbs marked by \textit{‑ɛn}}%
  \label{tab:Creissels:o-spontaneous-motion-en}
\end{table}

This use of \textit{‑ɔ} can be deemed \textsc{quasi-reflexive}, since the
relationship between caused motion and spontaneous motion shares important
characteristics with the relationship between two-participant events and the
corresponding reflexive events, but nevertheless differs from it in some
respects.  For example, a~person who is standing up cannot be described as
performing on him/herself the same action as when raising another person or an
object.  However, the use of lit.\ `raise oneself' in the sense of `stand up'
is attested in a~number of unrelated languages, and this extension of reflexive
marking to verbs of spontaneous motion has a~clear semantic motivation in that
a~person who is standing up is the instigator of an event whose manifestations
concern exclusively his/her own body.

This can be captured by introducing the notion of \textsc{quasi-reflexivity}
(or \textsc{autocausativity} in \citeauthor{Geniušienė1987}'s
(\citeyear{Geniušienė1987}) terminology) for the following type of relationship
between verbs encoding one- and two-participant events:

\begin{itemize}
\item the action performed by the unique participant in the one-participant
  event manifests itself in the same way and has the same result as if it were
  the affected participant in the two-participant event;
\item the unique participant in the one-participant event acts consciously and
  voluntarily, but in a~way that cannot be assimilated to the action performed
  by the agentive participant in the two-participant event.
\end{itemize}

In addition to the motion verbs listed above, the use of \textit{‑ɔ} in the verb
pairs in Table~\ref{tab:Creissels:o-quasi-refl} meets the definition of
quasi-reflexivity.

\begin{table}[ht]
  \centering
  \begin{tabular}{ll}
    \lsptoprule
   {Quasi-reflexive verb} & {Base verb} \\
    \hline
    \textit{yɔf-ɔ} `hide self' & \textit{yɔf} `hide s.th.\ or s.o.' \\
    \textit{yokul-o} `take a~rest' & \textit{yokul} `allow s.o.\ to rest' \\
    \lspbottomrule
  \end{tabular}
  \caption{Quasi-reflexive uses of \textit{‑ɔ} with verbs that are not motion verbs}%
  \label{tab:Creissels:o-quasi-refl}
\end{table}

\subsubsection{The decausative use of \textit{‑ɔ}}%
\label{sec:Creissels:decausative-o}

As illustrated by the verb pairs in Table~\ref{tab:Creissels:decausative-o}, in
addition to its reflexive and quasi-reflexive use, the verbal suffix \textit{‑ɔ}
is fully productive in decausative (or `anticausative') function.

\begin{table}[ht]
  \centering
  \begin{tabular}{ll}
    \lsptoprule
    {Decausative verb} & {Base verb} \\
    \hline
    \textit{bɐmbul-o} `open (intr.)' & \textit{bɐmbul} `open (tr.)' \\
    \textit{bɛnɛn-ɔ} `increase (intr.)' & \textit{bɛnɛn} `increase (tr.)' \\
    \textit{fʊm-ɔ} `break (intr.)' & \textit{fʊm} `break (tr.)' \\
    \textit{gumbul-o} `diminish (intr.)' & \textit{gumbul} `diminish (tr.)' \\
    \textit{jɩs-ɔ} `tear (intr.)' & \textit{jɩs} `tear (tr.)' \\
    \textit{liw-o} `wake up' & \textit{liw} `wake s.o.\ up' \\
    \textit{loopul-o} `come off' & \textit{loopul} `take off' \\
    etc. & \\
    \lspbottomrule
  \end{tabular}
  \caption{Examples of verbs attesting the decausative use of \textit{‑ɔ}}%
  \label{tab:Creissels:decausative-o}
\end{table}

In Jóola Fóoñi, \textit{‑ɔ} is not used productively in passive or resultative
function, but it is possible to find sporadic cases of transitive verbs whose
\textit{ɔ}-derivate has a~passive or resultative rather than decausative meaning.
Those we came across are listed in Table~\ref{tab:Creissels:passive-o}.

\begin{table}[ht]
  \centering
  \begin{tabular}{ll}
    \lsptoprule
    {Passive or resultative verb} & {Base verb} \\
    \hline
    \textit{wot-o} `be known' & \textit{wot} `know' \\
    \textit{kʊr-ɔ} `be well-mannered' & \textit{kʊr} `educate' \\
    \textit{yab-ɔ} `get married (speaking of a woman) & \textit{yab} `marry' \\
    \lspbottomrule
  \end{tabular}
  \caption{Verbs attesting a~passive or resultative use of \textit{‑ɔ}}%
  \label{tab:Creissels:passive-o}
\end{table}

\subsubsection{Lexicalized uses of \textit{‑ɔ}}%
\label{sec:Creissels:lexicalized-o}

The lexicalization of \textit{ɔ}-derivates can be illustrated by the verb pairs
in Table~\ref{tab:Creissels:lexicalized-o}:

\begin{table}[ht]
  \centering
  \begin{tabular}{ll}
    \lsptoprule
   {Lexicalized derivate} & {Base verb} \\
    \hline
    \textit{baj-ɔ} `exist' or `have time' & \textit{baj} `have' \\
    \textit{jam-ɔ} `be famous' & \textit{jam} `hear' \\
    \textit{bɔl-ɔ} `burn self' & \textit{bɔl} `grill, roast' \\
    \textit{kɔk-ɔ} `be unlucky' & \textit{kɔk} `tie s.th.\ up' \\
    etc. & \\
    \lspbottomrule
  \end{tabular}
  \caption{Examples of verbs attesting lexicalized uses of \textit{‑ɔ}}%
  \label{tab:Creissels:lexicalized-o}
\end{table}

\subsection{Reflexive and non-reflexive uses of \textit{‑ɔɔr}}%
\label{sec:Creissels:oor}

\subsubsection{The reflexive and quasi-reflexive use of \textit{‑ɔɔr}}%
\label{sec:Creissels:refl-oor}

The reflexive use of \textit{‑ɔɔr} has been illustrated by
example~(\ref{ex:Creissels:22}), reproduced here as~(\ref{ex:Creissels:40}):

\ea
  \label{ex:Creissels:40}
  
    \ea\label{ex:Creissels:40a}
      \textit{Ŋar fʊmbanjaf man uguuy añɩɩlaw.}\\
      \gll ŋar fʊ-mbanj-a-f man u-guuy a-ñɩɩl-a-w\\
      take \textsc{sg}-blanket(F)-\textsc{d}-clF \textsc{csc} sI:\textsc{2sg}-cover \textsc{sg}-child(A)-\textsc{d}-clA\\
      \glt `Take the blanket and cover the child (with it).'

    \ex\label{ex:Creissels:40b}
      \textit{Ŋar fʊmbanjaf man uguuyoor.}\\
      \gll ŋar fʊ-mbanj-a-f man u-guuy-oor\\
      take \textsc{sg}-blanket(F)-\textsc{d}-clF \textsc{csc} sI:\textsc{2sg}-cover-\textsc{ɔɔr}\\
      \glt `Take the blanket and cover yourself (with it).'

  \z
\z

However, \textit{‑ɔɔr} is used only marginally as a~reflexive marker, and the possibility of marking subject--object coreference by means of \textit{‑ɔɔr} (rather than \textit{‑ɔɔrɔ} or \textit{‑ɔ}) can only be analyzed as a~lexically
specified property of a~handful of verbs that do not constitute a~natural
semantic class.  In all cases, the \textit{ɔɔr}‑verb can also express
a~reciprocal meaning.  We also came across an equipollent pair in which the
\textit{ɔɔr}-verb has a~reflexive meaning, and three pairs in which the use of
\textit{‑ɔɔr} can be analyzed as quasi-reflexive (a~type of meaning more commonly
encoded by means of the suffix \textit{‑ɔ} -- see
\sectref{sec:Creissels:quasi-refl-o}).  The list of the \textit{ɔɔr}-verbs attested
with a~reflexive or quasi-reflexive meaning in our data is given in
Table~\ref{tab:Creissels:refl-oor}.

\begin{table}[ht]
  \centering
  \begin{tabular}{ll}
    \lsptoprule
    Reflexive or quasi-reflexive \textit{ɔɔr}-verb & Corresponding non reflexive verb \\
    \hline
    \textit{guuy-oor} `cover self' & \textit{guuy} `cover s.o.\ or s.th.' \\
    \textit{kɔf-ɔɔr} `scratch self' & \textit{kɔf} `scratch s.o.' \\
    \textit{kɔɔrɛn-ɔɔr} `heal self with inhalation' & \textit{kɔɔrɛn} `smoke an enclosed place' \\
    \textit{naan-ɔɔr} `smear one's body' & \textit{naan-ɛn} `smear' \\
    \textit{bɔɔñ-ɔɔr} `curl up' & \textit{bɔɔñ} `roll up, fold up' \\
    \textit{raaw-ɔɔr} `stretch self (arms, legs, etc.)' & \textit{raaw} `stretch' \\
    \textit{tiiw-oor} `turn self over, turn self around' & \textit{tiiw} `turn s.th.\ over, turn s.th.\ around' \\
    \lspbottomrule
  \end{tabular}
  \caption{Reflexive or quasi-reflexive \textit{ɔɔr}-verbs}%
  \label{tab:Creissels:refl-oor}
\end{table}

In the case of \textit{sɔntɛn} `heal s.o.', the addition of \textit{‑ɔɔr} gives
a~reflexive-causative meaning (\textit{sɔntɛn-ɔɔr} `get treatment for self', cf.\
French \textit{se faire soigner}), whereas the plain reflexive meaning `heal
self' is regularly expressed as \textit{sɔntɛn-ɔɔrɔ}.  Our data include no other
verb with the possibility of a~similar contrast between \textit{‑ɔɔr} and
\textit{‑ɔɔrɔ}.

\subsubsection{Other uses of \textit{‑ɔɔr}}%
\label{sec:Creissels:other-oor}

As illustrated in~(\ref{ex:Creissels:41}), \textit{‑ɔɔr} is fully productive as
a~reciprocal marker.

\ea
  \label{ex:Creissels:41}
  
    \ea\label{ex:Creissels:41a}
      \textit{Nan asaafolim, naanoli: ``Mɩyʊʊ bɛy jɩjɛɛ bɛɛt?''}\\
      \gll n-an a-saaf-oli-m n-aan-oli mɩyʊʊ b-ɛy jɩ-ja-ɛ bɛɛt\\
      clN-\textsc{rel} sI:clA-greet-I:\textsc{1pl.excl-act$_1$}
      \textsc{ppf}-sI:clA.say-I:\textsc{1pl.excl} \textsc{2pl.pro}
      clB-which sI:clJ-go-\textsc{act$_0$} \textsc{all}\\
      \glt `After greeting us, he asked us: ``Where are you going?''\,'


    \ex\label{ex:Creissels:41b}
      \textit{Nan ʊsaafoorʊm, naanɔm: ``Karɛɛsɩ bʊʊ?''}\\
      \gll n-an ʊ-saaf-ɔɔr-ʊ-m n-aan-ɔm ka-rɛɛs-ɩ bʊʊ\\
      clN-\textsc{rel} sI:\textsc{1pl}-greet-\textsc{ɔɔr-ep-act$_1$}
      \textsc{ppf}-sI:clA.say-I:\textsc{1sg} SG-name(K)-I:\textsc{2sg} how\\
      \glt `After we greeted each other, he asked me: ``What's your name?''\,'

  \z
\z

\textit{‑ɔɔr} is also productively used with intransitive verbs to express joint
action (as in \textit{jaw-ɔɔr} `go together' $<$ \textit{jaw} `go', or
\textit{cɩn-ɔɔr} `live together as neighbors' $<$ \textit{cɩn} `live at a~place').

Finally, the suffix \textit{‑ɔɔr} distinguishes itself by the very high
proportion of lexicalized verb pairs in which a~verb which seems to have been
derived via the addition of \textit{‑ɔɔr} expresses a~meaning whose relationship
to that of the base verb is more or less opaque in the present state of the
language (see Table~\ref{tab:Creissels:lexicalized-oor}).

\begin{table}[ht]
  \centering
  \begin{tabular}{ll}
    \lsptoprule
    {lexicalized derivate} &  {base verb} \\
    \hline
    \textit{baj-ɔɔr} `be in conflict' & \textit{baj} `have' \\
    \textit{gam-ɔɔr} `lament' & \textit{gam} `advise' \\
    \textit{gɔr-ɔɔr} `move (intr.)' & \textit{gɔr} `touch' \\
    \textit{kaan-ɔɔr} `quarrel' & \textit{kaan} `do, make' \\
    etc. & \\
    \lspbottomrule
  \end{tabular}
  \caption{Examples of verbs attesting lexicalized uses of \textit{‑ɔɔr}}%
  \label{tab:Creissels:lexicalized-oor}
\end{table}

\subsection{Summary}%
\label{sec:Creissels:verb-marker-summary}

Table~\ref{tab:Creissels:verb-marker-summary} summarizes the possible uses of the
three verbal suffixes of Jóola Fóoñi variously involved in the coding of
reflexivization:

\begin{table}[ht]
  \centering
  \begin{tabular}{lccc}
    \lsptoprule
    & {‑ɔɔrɔ} & {‑ɔ} & {‑ɔɔr} \\
    \hline
    {reflexive (other than body care)} & $+$ & $(+)$ & $(+)$ \\
    {reflexive (body care)} & $-$ & $+$ & $-$ \\
    {quasi-reflexive} & $-$ & $+$ & $(+)$ \\
    {decausative} & $-$ & $+$ & $-$ \\
    {passive, resultative} & $-$ & $(+)$ & $-$ \\
    {reciprocal} & $-$ & $-$ & $+$ \\
    {joint action} & $-$ & $-$ & $+$ \\
    {self-intensification} & $+$ & $-$ & $-$ \\
    \lspbottomrule
  \end{tabular}
  \caption{The possible uses of \textit{‑ɔɔrɔ}, \textit{‑ɔ} and \textit{‑ɔɔr}}%
  \label{tab:Creissels:verb-marker-summary}
\end{table}


\section{Reflexivization and self-intensification}%
\label{sec:Creissels:refl-and-self-intens}

As already discussed above, Jóola Fóoñi has the cross-linguistically
exceptional particularity of marking self-intensification of the subject by
means of a~verbal suffix also acting as a~reflexive voice marker.

Jóola Fooñi also has free forms available to express self-intensification of
NPs irrespective of their syntactic role, but our data include no example in
which one of these self-intensifiers, either alone or combined with a~pronoun,
could be analyzed as acting as a~reflexive pronoun.

\subsection{The adnominal self-intensifier \textit{faŋ}}%
\label{sec:Creissels:fang}

Jóola fooñi has a~noun \textit{f-aŋ} (gender F/K), glossed `personnalité, le moi
/ personality, the self' in \citeauthor{SapirEtAl1993}'s
(\citeyear{SapirEtAl1993}) Jóola-French-English dictionary, which is the
obvious source of the adnominal self-intensifier \textit{faŋ} illustrated
in~(\ref{ex:Creissels:42}).  Note that \textit{faŋ} as an adnominal self-intensifier is
optionally suffixed by an index resuming the noun or pronoun it
intensifies.\footnote{The optional suffixation of indexes is also found with other adnominal particles such as \textit{cɛb} `only' or \textit{buroom} `all'. }

\ea
  \label{ex:Creissels:42}
   \ea\label{ex:Creissels:42a}
      \textit{Kuliinool kɔɔkʊ dɩ bulokɐb, ampaɔɔl ɔɔmʊ dɩ bulokɐb, ɔɔ faŋɔɔl Inaa ɔɔmʊ dɩ bulokɐb.}\\
      \gll ku-liin-ool k-ɔɔ-kʊ dɩ bu-lok-ɐ-b ampa-ɔɔl $\varnothing$-ɔɔ-mʊ dɩ bu-lok-ɐ-b $\varnothing$-ɔɔ faŋ-ɔɔl Ɩnaa $\varnothing$-ɔɔ-mʊ dɩ bu-lok-ɐ-b\\
      \textsc{pl}-different.sex.sibling(BK)-I:clA clBK-\textsc{lcop}-clBK
      \textsc{prep} \textsc{sg}-tears(B)-\textsc{d}-clB father(A)-I:clA
      clA-\textsc{lcop}-clA \textsc{prep} \textsc{sg}-tears(B)-\textsc{d}-clB
      clA-\textsc{pro} self-I:clA Inaa clA-\textsc{lcop}-clA \textsc{prep}
      \textsc{sg}-tears(B)-\textsc{d}-clB\\
      \glt `Her brothers are in tears, her father is in tears,
      and Inaa herself is in tears.'

    \ex\label{ex:Creissels:42b}
      \textit{Inje faŋ, mɔɔ nilɐkoumum taatɛ.}\\
      \gll inje faŋ m-ɔɔ n-i-lɐko-um-u-m t-aa-t-ɛ\\
      \textsc{1sg.pro} self clM-\textsc{pro}
      \textsc{ppf}-sI:\textsc{1sg}-stay-\textsc{appl-ep-act$_1$}
      clT-\textsc{dem}-clT-\textsc{prox}\\
      \glt `As for myself, this is why I stayed here.'

    \ex\label{ex:Creissels:42c}
      \textit{Ʊjɔɔnɛn ʊwʊ jak, jakʊm ʊgaalɛn an, jakʊm ʊgaalɛn aw faŋɩ}\\
      \gll ʊ-jɔɔnɛn ʊ-wʊ jak jakʊm ʊ-gaalɛn an jakʊm ʊ-gaalɛn aw faŋ-ɩ\\
      sI:\textsc{2sg}-fix clU-\textsc{dem} well \textsc{proh} sI:clU-disturb
      person(A) \textsc{proh} sI:clU-disturb \textsc{2sg.pro} self-I:\textsc{2sg}\\
      \glt `Fix that properly, so that it doesn't disturb anybody, and it doesn't
      disturb yourself.'

  \z
\z

Interestingly, in (\ref{ex:Creissels:42c}), in spite of the fact that the subject index of class~U is homonymous with the subject index of 2nd person singular, a reflexive interpretation of \textit{jakʊm ʊgaalɛn aw faŋɩ} (that is, `don't disturb yourself') is excluded, since if it were the case, the verb form should
include the suffix \textit{‑ɔɔrɔ}.

\subsection{The noun \textit{fu-ko} `head' in self-intensifier function}%
\label{sec:Creissels:fuko}

In Jóola Fóoñi, the noun \textit{fu-ko} `head' has a~grammaticalized use as an
adnominal self-intensifier.  In this use, \textit{fu-ko} obligatorily combines
with an index resuming the noun or pronoun it intensifies, and may optionally
be introduced by the comitative-instrumental-locative preposition \textit{dɩ}.

Among Atlantic languages, it is common that the noun `head' combined with
a~possessive index or pronoun acts not only as a~self-intensifier, as
in~(\ref{ex:Creissels:43a}), but also as a~reflexive pronoun, as
in~(\ref{ex:Creissels:43b}).

\ea
  \label{ex:Creissels:43}
  Wolof (Atlantic)\footnote{The two sentences quoted in this example are from \citeauthor{Diouf2003}'s (\citeyear{Diouf2003}) Wolof-French dictionary. }
    \ea\label{ex:Creissels:43a}
      \textit{Waxal ko sa bopp!}\\
      \gll wax-al ko sa bopp\\
      say-\textsc{imp} I:\textsc{3sg} \textsc{poss.2sg} head(B)\\
      \glt `Say it yourself
      
    \ex\label{ex:Creissels:43b}
      \textit{Wuude bi du ëwal boppam.}\\
      \gll wuude b-i du ëw-al bopp-am\\
      shoemaker(B) clB-\textsc{d}.\textsc{prox} \textsc{neg}.sI:\textsc{3sg}
      sew.leather-\textsc{appl} head(B)-I:\textsc{3sg}\\
      \glt `The shoemaker does not sew leather for himself.'

  \z
\z

The case of Jóola Fooñi is very different, since `\textit{fu-ko} `head' $+$
possessive index or pronoun' in its grammaticalized use can only occur as an
emphatic self-intensifier, never as a~reflexive pronoun.  In all the examples
we have in our data, `\textit{fu-ko} `head' $+$ possessive index or pronoun'
combines with a~verb form marked by the suffix \textit{‑ɔɔrɔ} in
self-intensifying function, and just adds some additional emphasis, as in (\ref{ex:Creissels:44b}).

\ea\label{ex:Creissels:44}
    \ea[ ]{
       \textit{Jaw usiilooro!}\\
        \gll jaw u-siil-ooro \\
        go sI:\textsc{2sg}-cook-\textsc{ɔɔrɔ}\\
        \glt `Go and do the cooking for yourself!' (autobenefaction)\\
      \textit{or} `Go and do the cooking yourself!' (self-intensification) 
      \label{ex:Creissels:44a}
    }
    \ex[]{
        \textit{Jaw usiilooro fukoi!}\\
        \gll jaw u-siil-ooro fu-ko-i\\
        go sI:\textsc{2sg}-cook-\textsc{ɔɔrɔ} \textsc{sg}-head(F)-I:\textsc{2sg}\\
      \glt `Go and do the cooking YOURSELF!' (emphatic self-intensification)
      \label{ex:Creissels:44b}
}
    \ex[*]{
       \textit{Jaw usiil fukoi!}\\
          \gll jaw u-siil fu-ko-i\\
        go sI:\textsc{2sg}-cook-\textsc{ɔɔrɔ} \textsc{sg}-head(F)-I:\textsc{2sg}\\
        \label{ex:Creissels:44c}
    }
  \z
\z


\section{Conclusion}%
\label{sec:Creissels:conclusion}

In this article, basing ourselves on a~corpus of naturalistic texts of various
genres completed by elicitation because of the relative scarcity of reflexive
constructions in the corpus, we have analyzed the way Jóola Fóoñi codes
coreference within the clause, as well as the non-reflexive uses of the verbal
suffixes that have the ability of acting as reflexive voice markers.  The main
conclusions are as follows:

\begin{itemize}
\item Jóola Fóoñi does not have reflexive pronouns or indexes.
\item Subject--object coreference requires verbal marking by means of
  derivational suffixes that reduce by one the number of objects compatible
  with the verb and imply that a~single participant, encoded as the subject,
  cumulates the semantic roles assigned to the subject and (one of) the
  object(s) in the construction of the base verb.
\item Due to the extensive use of multiple-object constructions, and to the
  fact that beneficiaries as simply coded as objects, in Jóola Fóoñi,
  agent--goal coreference and agent--beneficiary coreference are just
  particular cases of subject--object coreference.
\item Coreference relationships within the clause other than subject--object
  coreference are not treated differently from coreference in discourse.
\item Three verbal suffixes may be found in constructions in which they act as
  reflexive voice markers, but all three also have non-reflexive uses:
  \textit{‑ɔɔr}, whose use in reflexive constructions is marginal, is
  productively used as a~reciprocal voice marker; \textit{‑ɔ}, used as
  a~reflexive voice marker with body-care verbs, is also productive in
  quasi-reflexive (or auto-causative) and decausative function; \textit{‑ɔɔrɔ},
  analyzable as the default reflexive voice marker, is also fully productive as
  a~marker of self-intensification of the subject.
\item Jóola Fóoñi shows that the co-expression of self-intensification and
  reflexivization, very common cross-linguistically for free self-forms, may
  also characterize verbal derivational suffixes.
\end{itemize}


%\section*{Acknowledgements}
% TODO acknowledgements


\section*{Abbreviations}

Capital letters between parentheses immediately after the lexical gloss of
nouns (for example, `woman(A)', or `dog(E)') or after the gloss \textsc{inf}
(`infinitive') indicate the agreement pattern associated to the form in
question.

The other abbreviations are as follows:

\begin{center}
  \begin{tabularx}{.45\textwidth}[t]{lQ}
    \textsc{all}  & allative \\
    \textsc{appl} & applicative \\
    \textsc{cpl}  & completive \\
    \textsc{act}  & actualizer \\
    clX           & class X \\
    \textsc{csc}  & consecutive \\
    \textsc{cvb}  & converb \\
    \textsc{d}    & definite \\
    \textsc{dem}  & demonstrative \\
    \textsc{ep}   & epenthetic vowel \\
    \textsc{excl} & exclusive \\
    \textsc{fut}  & future \\
    \textsc{gen}  & genitive \\
    \textsc{hyp}  & hypothetical \\
    I             & index (other than subject index cf.\ sI) \\
    \textsc{icpl} & incompletive \\
    \textsc{incl} & inclusive \\
    \textsc{imp}  & imperative \\
  \end{tabularx}
  \begin{tabularx}{.45\textwidth}[t]{lQ}
    \textsc{inf}  & infinitive \\
    \textsc{lcop} & locational copula \\
    \textsc{neg}  & negation \\
    \textsc{oblg} & obligative \\
    \textsc{pl}   & plural \\
    \textsc{poss} & possessive \\
    \textsc{ppf}  & pre-prefix \\
    \textsc{proh} & prohibitive \\
    \textsc{prox} & proximal \\
    \textsc{prep} & preposition \\
    \textsc{pro}  & pronoun \\
    \textsc{ptcp} & participle \\
    \textsc{rdpl} & reduplicative suffix \\
    \textsc{rel}  & relativizer \\
    \textsc{seq}  & sequential \\
    \textsc{sg}   & singular \\
    sI            & subject index \\
    \textsc{ven}  & ventive \\
  \end{tabularx}
\end{center}

{\sloppy\printbibliography[heading=subbibliography,notkeyword=this]}
\end{document}
