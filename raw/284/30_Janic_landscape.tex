\documentclass[output=paper]{langsci/langscibook} 

\author{Katarzina Janic\lastand Nicoletta Puddu{}\orcid{}} 
%\ORCIDs{}

\title{The landscape of reflexive constructions: Form and function} 


\abstract{Reflexive constructions vary from language to language in the way they encode the coreference meaning between the agent and patient arguments of the minimal clause. While some languages employ a dedicated form called reflexivizer, the others make use of a non-reflexive form that under some conditions may perform a reflexive function. There is also much variation in the types of reflexivizers employed by languages such as reflexive nominals, voice markers, argument markers, and whether they occur in the variant or invariant form. Additionally, they may display various co-expression patterns related to self-intensification, auto-benefaction or other voice-changing operations. Interesting variations also occur in the antecedent domain between the antecedent and the reflexive nominal. This study summarizes these cross-linguistic variations which emerged from the contributions to this volume.}

\begin{document}
\maketitle

\section{Introduction} 
Let’s discuss first all what is related to ‘FORM’ and then move to ‘FUNCTION’. Ideally, it would be good to include in this chapter also these pieces of information which were not mentioned in Martin’s chapter or those which add additional clarification to his investigation.


\section{How is co-reference in a minimal clause between Agent and Patient coded?}
Let’s start with this point. First, we provide the definition by Martin. Then we state that the co-reference in a minimal clause between Agent and Patient can be coded differently in languages. Then we provide relevant examples showing this diversity.
As we mentioned in the introduction the languages across the word generally have a specialized marker in order to signal coreference. However, there are some exceptions. Our sample goes in the predicted directions: the majority of the languages have a specialized marker. However, also in other sample, we have two exceptions, Kazym Khanty (ch. 10) and Zenzontepec Chatino.
In Kazym Khanty verbal agreement seem to play a role for a bound interpretation, at least for some speaker, while in 
\section{ Type of reflexivizers}

\begin{itemize}
\item Reflexive nominals
\item Reflexive voice markers
\item Reflexive argument markers
\item Others
\end{itemize}

This point is about the form. Likewise, let’s provide examples showing the diversity of the types of reflexivizers. This point corresponds to one of the points from Martin’s chapter. Hence, the “others” D would be the least repetitive and, in this sense, the most interesting. Again, each a, b, c, d, should be accompanied with the relevant example.
Many languages have two strategies, one nominal and one verbal (Ex. Walman, 

\section{Are reflexive variant or invariant?}
\begin{itemize}
    \item Invariant reflexive form for all persons
\item reflexive form only for third person 
\item Variant reflexive form for all person (cumulative reflexive form?)
\item Other
\end{itemize}

Point 3 relates to 2nd point as it is also about the ‘form’. Its content can be integrated while discussing a, b, c, d, etc. from point 2. 


\section{Co-expression patterns of reflexivizers}
This point which corresponds to Martin’s 11 point seems to be worth developing more as it also tackles the cases that was not discussed by Martin, for instance the use of the reflexive form to express autobenefactive. This was discussed on several occasions in different papers.

\begin{itemize}
    \item Reflexive pronoun vs. Self-intensification form
\item Reflexive voice marker vs. Other voice reducing valency  changing operations
\item Reflexive pronoun vs. Autobenefactive (Creissels)
\end{itemize}
\section{Morphological features}
\begin{itemize}
\item Verbal vs. nominal reflexives
\begin{itemize}
\item Verbal reflexives do not confirm to be less frequent in general. All the languages from the South American sample have verbal reflexives
Some languages, like Abaz (Arkadiev and Durneva) show verbal and nominal reflexives in competition. In this case the introverted/extroverted opposition  
\item The main criterion proposed by Faltz (1985) in order to distinguish verbal from nominal reflexives is the fact that the reflexive marker may appear in non-object position alone or after a preposition. This criterion seems to be relevant to distinuish verbal from nominal reflexives in  Abaza (Arkadiev and Durneva). In Oneida (Michelson) phrases with locative meaning introduced by prepositions are not possible, so this criterion cannot be used.
Describe the interaction with ergativity. In Abaz (Arkadiev and Durneva)  verbal absolutive reflexive čə
\item In verbs with highly synthetic verb morphology, we expect to have a “verbal reflexive” and this is the case of Oneida (Michelson) and Hoocak (Helmbrecht). However, in Waray, the language employs a full nominal expression 'kalugaríngon' (Payne and Oyzon) and in Abaz (Arkadiev and Durneva) we have both
\item Variant vs. invariant (see map)
\end{itemize}
\end{itemize}

\section{Self-benefactive coding}

Self-benefactive (or 'auto-benefactive') events are characterised by the fact that the agent participant benefits from the action denoted by the verb. The coreference between the agent and the beneficiary participants is encoded in various ways across languages in the world. Agurana (Chicham) displays a commonly attested pattern in which the applicative voice marker co-occurs together with the reflexive voice marker. The function of the applicative suffix \emph{-hu} or \emph{-tu} is to introduce an object argument assuming typically a beneficiary (or maleficiary) semantic role. The reflexive marker \emph{-ma}, which occupies the object slot in the verbal paradigm in in (\ref{ex:ch1:1}), signals that the introduced beneficiary argument co-refers with the agent, as shown in (1).


\ea \label{ex:ch1:1}
        Aguaruna (Overall 2017: 617) \\
\gll    yupichu	    hu-hu-ma-tayamɨ  ̃  \\
         easy	    take-APPL-REFL-NORM\\
\glt    'We easily take it away (for ourselves).' 
\z

The self-benefactive event can also be encoded only by the middle marker. This can be observed in Kambaata. Attached to the verbal form \emph{min- }'build', the multi-functional middle morpheme \emph{-aqq / -’} signals that the subject of the clause is the beneficiary of the event expressed by the verb. Treis (this volume) notes that this is the most productive interpretation of the middle morpheme in Kambaata and all other East Cushitic languages (cf. Mous 2004).


\ea\label{ex:treis:2} Kambaata (Afro-Asiatic, Treis, this volume) \\
\gll Gizz-á hoolam-á ir-á xáaz-z qú’mm=\textbf{eecc}-ít min-í         \textbf{mi’nn}-itóo’u\\
     money-\textsc{m.acc} much-\textsc{m.acc} time-\textsc{m.acc} gather-\textsc{3f.pfv.cv} gather=do.\textsc{mid}\textsc{-3f.pfv.cv} house-\textsc{m.acc} build.\textsc{mid}-\textsc{3f.pfv}\\
\glt ‘After having saved money for many years, they could build a house for their own benefit.’
\z

A comparable situation is observed in Jóola  Fóoñi, an Atlantic language. Among three verbal suffixes encoding subject-object coreference, the morpheme \emph{-ɔɔrɔ}, which is a default subject-object coreference marker, serves as self-benefactivizer. Like in Cushitic languages, it shows a high degree of productivity in expressing the self-benefactive meaning. 

\vspace{1cm} %1mm vertical space

\ea\label{ex:treis:3} Jóola Fóoñi (Atlantic-Congo, Creissels \& Bassène, this volume) \\
\gll n-ɩ-nɔɔm-ɔɔrɔ-ɛ        e-woto \\
      PPF-sI:1SG-buy-ƆƆRƆ-CPL  SG-car(E) \\
\glt ‘I bought a car for myself.’
\z

In (3), the subject argument is indexed on the verb by means of the subject index which is always prefix \emph{ɩ}.

\begin{itemize}
    \item co-expression of the reflexive verbal marker which expresses self-benefactive and self-intensification 
\end{itemize}


In many languages, self-benefactive meaning 'do something for oneself' and self-intensification 'do something oneself' are expressed by means of the same linguistic tool because of the close semantic affinity [add REF]. This can be observed in Jóola  Fóoñi, in which \emph{-ɔɔrɔ} constructions are frequently ambiguous expressing both self-benefactive and self-intensification interpretation, as shown in (4). In most cases, the difference can be resolved by the context.

\ea\label{ex:treis:4} Jóola Fóoñi (Atlantic-Congo, Creissels & Bassène, this volume) \\
\gll jaw  u-siil-ooro \\                             
    go  sI:2SG-cook-ƆƆ \\                             
\glt ‘Go and do the cooking for yourself! (autobenefaction) \\
\glt ‘Go and do the cooking yourself!’ (self-intensification) 





  


\section{The expression of self-intensification}

\section{Coding of introverted and extroverted verbs}
Crosslinguistically, reflexive  voice  markers tend to be employed  in the autopathic domain. Many scholars (e.g. König & Vezzosi 2004; Haspelmath 2008, this volume: 17) argue that the coreference is particularly unlikely in this domain with extroverted verbs. 







\section{Domains: Autopathic, oblique and adpossessive reflexive constructions}
Here we move more towards ‘FUNCTION’ or maybe better ‘USAGE’. It seems to me that we should also discuss this point (as many authors investigated it) with the aim of showing crosslinguistic variation;

\section{Less trivial uses of reflexive constructions}
A less trivial use of the reflexive is attested in Abaza (Arkadiev and Durneva) only in combination with the morphological causative and involves the meaning of simulation or pretence
This will be a tiny point but worth mentioning. However, I do not have (yet?) any specific example.

% this part has been copied pasted to google doc

\section{Socio-cultural factors which can shape the notion of reflexivity}Socio-cultural factors which can shape the notion of reflexivity

\begin{itemize}
    \item Coreference is particularly unlikely in the autopathic domain, at least with extroverted verbs (Haspelmath, 16-17). As a consequence 
    \begin{itemize}
        \item Reflexivizers can be particularly “heavy” 
\begin{itemize}
\item Payne and Oyzon. In Waray the reflexive form, kalugaríngon, is phonologically large and morphologically complex nominal, which is rather surprising as 'Most languages, it seems, have well structuralized and phonologically reduced patterns for expressing reflexive ideas. One may especially expect languages with highly synthetic verb morphology, such as Waray, to have some verb or verb-phrase element that expresses at least some varieties of reflexivity.' Instead, the language employs a full nominal expression 'kalugaríngon'.
They hypothesize by saying that 'Our speculation on this topic is grounded in the observation that Philippine cultures, Waray in particular, are very communal societies. Acting together with others is a high cultural value. Consequently, it is often unusual, and rather aberrant that someone should act exclusively on one's own, or upon oneself.'\end{itemize}
\item Some reflexive constructions can be “impossible”
\begin{itemize}
    \item Brooks. In Chini "Autopathic construals of certain events sensical in English are absurd in the Chini sociocultural world (e.g. ‘give a gift to oneself’)"
\item Michelson. In Oneida, one cannot give or send something to oneself; but one can give things to one another, as with the reciprocal of the verb -awi-/-u- ‘give’ in (27). For ‘talk to oneself’ a speaker provided the circumlocution in (28). Here, a form of the emphatic pronoun -ulhaʔ- (see additional examples in Section 6) meaning ‘I am all alone’ is followed by a verb that asserts I am talking; indeed this is perhaps a more careful interpretation of what it means to say ‘talk to oneself’, namely, ‘there I am all alone, and still (nevertheless) I am talking’.
\end{itemize}
\item Ritual speech determines an increase in expression of agentivity. The difference between introverted and extroverted verbs can be different in ritual speech. 
\begin{itemize}
    \item Orqueda and Pooth “introverted events are infrequent in Early Vedic, probably due to the liturgical nature of the preserved texts. […] For determining verbs as introverted in Early Vedic we cannot simply look for what is introverted in other languages. Thus, for instance, vas- ‘to cloth’ is a peculiar verb as in general it refers to the preparation for the ritual. Consquently vas- is mostly found with a nominative agent and an optional accusative Theme
\item Campbell. In Zenzontepec Chatino “In performative ritual speech, notions that would typically be expressed in an intransitive clause, like the prohibitive, can be creatively cast in a causative, reflexive expression in order to emphasize the agency and responsibility of the referent”
% the end of copy_paste
\end{itemize}

Short reflexives as expressing the point of view of the subject 
Fraizynigier (2000) suggests that the so-called short reflexives represent the event from the point of view of the subject
Oneida (Michelson) “The semireflexive verb just indicates some sort of physical or perceived proximity. In fact, for many verbs, it would be odd to think of the entity as being owned. The semi-reflexive can be used in the context of the narrator’s grandmother making baskets, which she sold or traded for goods. the sense is making baskets herself for her own purpose”. Some of the “unpredictable” meanings derived by addition of the semi-reflexive could be seen as focusing on the point of view of the subject (see for instance -hloli- ‘tell someone something’, -athloli- ‘talk about someone or something’). This is just an idea, to be tested.
\end{itemize}
\end{itemize}

\section{Diachronic development}

% the below text is copy_pasted to google doc
Only some papers explicitly mention diachronic development
Abaz (Arkadiev and Durneva): “the etymology of the absolutive reflexive čə- is unclear, however, comparative data from Abkhaz  indicates that it goes back to a noun with possessive prefix incorporated into the absolutive slot of the verb”. This confirms that verbal reflexives derive from nominal reflexives (Kazenin 2001)
It is true that not many people discuss it but let’s make a short summary of possible sources of the reflexive form.
% the end of copy paste
\section{Maps}
\section{Tables} 

\begin{table}
\begin{tabularx}{\textwidth}{XXX}
\lsptoprule
Language & Primary reflexive construction & Form \\
\midrule
\textbf{Africa} & & \\
Bangime & 0/Nominal & POSS+déɡé \\
Hausa & Nominal & POSS+kai \\
Kambaata & Nominal & gag-á, verbal (-aaq-) \\
Jóola Fóoñi & Verbal & -ɔɔrɔ \\
Mano & Nominal & ē \\
\tablevspace
\textbf{Eurasia}\\
Abaza & Verbal / Nominal & čə- /qa \\
Khanty & - & - \\
Polish & & \\
Thulung & Verbal & -siʈ \\
Vedic & Nominal & tanu- \\
\tablevspace
\textbf{Papunesia}\\
Nungon & Nominal & PRO+intensifier \\
Waray & Nominal & kalugaríngon  \\
Komnzo & verbal & ? \\
Chini & Verbal & nij \\
\tablevspace
\textbf{North America} \\
Hoocąk & Verbal& kii-    \\
Yaqui & Nominal & emo \\
Oneida & Verbal & -atat-/-atate- \\
Zenzontepec Chatino & - & - \\
\tablevspace
\textbf{South America} \\
Aguaruna & Verbal & -m(a) or -mam(a)\\
Kakataibo & Verbal & -akat (and more) \\
Mojeño Trinitario & Verbal & -wo \\
\tablevspace
\textbf{Australia} \\
Anindilyakwa & Verbal & -jungwV- \\
Jaminjung/Ngaliwurru & Verbal & -ji \\
Kuuk Thaayorre & Nominal & ngathnay \\
Warlpiri & Nominal & (=nyanu) \\
\lspbottomrule
\end{tabularx}
\caption{Morphological typology}
\label{tab:chapterhandle:keytotable}
\end{table}

Morphological typology
Language
Primary reflexive construction

Nominal (kalugaríngon)

































{\sloppy\printbibliography[heading=subbibliography,notkeyword=this]}
\end{document}
