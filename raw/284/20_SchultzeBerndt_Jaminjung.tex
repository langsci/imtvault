\documentclass[output=paper,colorlinks,citecolor=brown]{langscibook} 


\author{Eva Schultze-Berndt\affiliation{University of Manchester}}
\title{The reflexive construction in Jaminjung/Ngaliwurru}
\abstract{Jaminjung/Ngaliwurru, a language of the Western subgroup of the Australian Mirndi family, has a single reflexive construction which is marked by a derivational affix on the verb; there are no reflexive pronouns in the language. This paper provides an overview of the formal and functional characteristics of this construction, in a comparative perspective. Reflexive marking renders a verb morphologically and syntactically intransitive. Unlike in a number of other Australian languages, the construction does not have a general detransitivising function. Rather, it is restricted to encoding both volitional and non-volitional self-directed actions. Specific subtypes discussed here are actions affecting an inalienable part of the subject, expressed by means of an external possession construction, and certain autocausatives of position and spatial arrangement. As is common among languages of Australia with verbal reflexive marking, the reflexive construction also has a reciprocal function.}

\IfFileExists{../localcommands.tex}{
 \addbibresource{localbibliography.bib}
 \input{../localpackages}
 %Copy this to localcommands.tex

\usepackage[english]{babel}
\usepackage{amsmath}
\usepackage{amssymb,amsfonts,textcomp}
\usepackage{array}
\usepackage{hhline}
\usepackage{hyperref}

\newenvironment{styleStandard}{}{}
\newenvironment{stylelsAbstract}{}{}
\newenvironment{stylelsSectioni}{}{}
\newenvironment{stylelsSectionii}{}{}
\newenvironment{stylelsBulletList}{}{}
\newenvironment{styleBibliographyi}{}{}
\newenvironment{listWWNumxxvleveli}{}{}
\newenvironment{listWWNumxxvlevelii}{}{}
\newenvironment{listWWNumxxvleveliii}{}{}
\newenvironment{listWWNumxxvleveliv}{}{}
\newenvironment{listWWNumixleveli}{}{}
\newenvironment{listWWNumixlevelii}{}{}
\newenvironment{listWWNumixleveliii}{}{}
\newenvironment{listWWNumixleveliv}{}{}

\newcommand\textstyleListLabelxvi[1]{#1}
\newcommand\labellistWWNumxxvleveli{\thelistWWNumxxvleveli.}
\newcommand\labellistWWNumxxvlevelii{\thelistWWNumxxvlevelii.}
\newcommand\labellistWWNumxxvleveliii{\thelistWWNumxxvleveliii.}
\newcommand\labellistWWNumxxvleveliv{\thelistWWNumxxvleveliv.}
\newcommand\labellistWWNumixleveli{[F0B7?]}
\newcommand\labellistWWNumixlevelii{\textstyleListLabelxvi{o}}
\newcommand\labellistWWNumixleveliii{[F0A7?]}
\newcommand\labellistWWNumixleveliv{[F0B7?]}

\newcounter{listWWNumxxvleveli}
\newcounter{listWWNumxxvlevelii}[listWWNumxxvleveli]
\newcounter{listWWNumxxvleveliii}[listWWNumxxvlevelii]
\newcounter{listWWNumxxvleveliv}[listWWNumxxvleveliii]
\newcounter{itemize}  
 \input{../localhyphenation} 
 \togglepaper[1]%%chapternumber
}{}

\begin{document}
\maketitle

\section{Introduction} 
\label{sec:SchultzeBernd:1}
This paper provides an overview of the form and functions of the reflexive construction in Jaminjung/Ngaliwurru, a language of the Western subgroup of the Australian Mirndi family \citep{Chadwick1997,Harvey2008}. Jaminjung and Ngaliwurru are two named varieties of a single language, i.e. they are mutually intelligible. The main differences between these are of a lexical nature; there is no difference between the two varieties as far as the reflexive construction and its uses are concerned. 

Taken together, Jaminjung and Ngaliwurru are spoken today by fewer than 50 people in the areas of Katherine, Timber Creek, and Kununurra in Northern Australia. The approximate location of the area for which Jaminjung and Ngaliwurru people are traditional owners is shown in Figure \ref{fig:map}. The first language of younger people in these communities, and the language mostly used in day-to-day communication, is Kriol (also known as Roper River Kriol), an English-lexified creole language \citep{Harris1986,SchultzeBerndt2013}.

\begin{figure}
  %\includegraphics[height=.3\textheight]{Jam_Ngali_map.png}
  \caption{Approximate location of Jaminjung and Ngaliwurru traditional country}
  \label{fig:map}
\end{figure}

The discussion of reflexives is based on fieldwork spanning more than 25 years by the author and collaborators. The resulting documentation corpus \cite{ComrieSchultzeBerndt2016} includes various genres ranging from narratives, procedural texts, and fictive and actual dialogue to elicitation by translation or (verbal or visual) scenarios.  References accompanying each example indicate the file name and transcript line number under which it is archived in the DoBeS Endangered Languages Archive (http://dobes.mpi.nl/research/; access upon request).

Jaminjung/Ngaliwurru has a single construction that meets the definition of a reflexive construction in \citet{Haspelmath_thisvolume} \todo{this volume} in that it has a marker with the specialised function of indicating coreference between two participants in a clause. The reflexive marker is an invariable verbal suffix; there are no reflexive pronouns in the language (for a list of free pronouns, see the Appendix). It is also used in reciprocal function, but has no additional (e.g. detransitivising) functions. 

Following an introduction to the main relevant grammatical properties of Jaminjung/Ngaliwurru in \sectref{sec:SchultzeBernd:2}, the formal properties of the construction are described in \sectref{sec:SchultzeBernd:3}. \sectref{sec:SchultzeBernd:4} provides more detail on the semantic range of the reflexive construction, while the reciprocal function will be briefly discussed in \sectref{sec:SchultzeBernd:5}. \sectref{sec:SchultzeBernd:6} provides a brief overview and illustration of cases of coreferentiality of subjects and non-objects, none of which license the reflexive construction.
The main features of the reflexive/reciprocal construction are summarised and discussed in an areal and typological perspective in \sectref{sec:SchultzeBernd:7}.

\section{Grammatical background} \label{sec:SchultzeBernd:2}
In terms of its morphological type, Jaminjung/Ngaliwurru can be characterised as agglutinative to fusional. Verbs are the most complex class morphologically, since they are obligatorily marked for person and number of subjects and (for transitive verbs) objects, and for tense, aspect and modality. Verbs fall into two non-overlapping transitivity classes in terms of their paradigm of pronominal prefixes, as illustrated in (\ref{ex:IV-trans}). Morphologically intransitive inflecting verbs only have a subject index; morphologically transitive verbs mark both subject and object. 

\ea Examples of intransitive and transitive inflecting verb forms\\
\label{ex:IV-trans}
\ea
\gll ga-ruma-ny\\
\textsc{3min}.come-\textsc{pst.pfv}\\ 
\glt `he/she/it came'

\ex
\gll {gani-wa}\\
\textsc{3min>3min}-bite.\textsc{pst.pfv}\\ 
\glt `it bit him/her'
\z
\z

All ditransitive predicates are morphologically transitive \citep{SchultzeBerndt2010}. Usually, the recipient rather than the theme is cross-referenced by the object prefix on the verb, but this depends on the relative animacy of recipient and theme.

Most morphologically transitive verbs have a reflexive counterpart which follows the intransitive paradigm (for details see \sectref{sec:SchultzeBernd:3}). This is the only detransitivising morpheme (in fact, the only valency-changing morpheme) in Jaminjung/Ngaliwurru.

Tense and aspect are marked by suffixation or stem suppletion; as is common cross-linguistically, an inflectional aspectual distinction (perfective vs. imperfective) is only made in the past tense. Modality is marked by prefixation (with a distinction between imperative, potential/future, and irrealis forms). The structure of inflecting verb forms, already illustrated in (\ref{ex:IV-trans}), is schematically represented in (\ref{ex:IVstructure}). 

%\protectedex{
\ea{Inflecting Verb Structure}\\
\textsc{(imp/irr-)subj-obj.min-(pot-)[obj.aug/ua-]}\textbf{root}\textsc{(-refl)(-tense/aspect)}\\
\label{ex:IVstructure}
\z
%}

The combination of cross-referencing on the verb and (optional) case-marked noun phrases makes Jaminjung/Ngaliwurru a double-marking language in the terminology of \citet{Nichols1986}. Case marking is by phrase-level enclitics. The alignment system for core arguments is ergative-absolutive (at a morphological, not a syntactic level); however, Jaminjung/Ngaliwurru exhibits “optional” (fluid) ergativity in that the presence of case marking on agents depends on a variety of factors including person and information structure \citep{SchultzeBerndt2017,SchultzeBerndtMeakinsinprep}.
Constituent order is pragmatically conditioned and does not serve to mark grammatical roles, and noun phrases can be freely omitted if their referent can be retrieved from discourse.

Throughout this paper, the terms 'subject' and 'object' will be used in a semantic sense, as a shorthand for core participants with the macro-roles of Actor and Undergoer, respectively. Example (\ref{ex:bul ganarrga}) illustrates a transitive clause where the subject is represented by an ergative-marked noun phrase and the first prefix slot on the verb, and the object by an absolutive noun phrase and the second prefix slot on the verb. 

\ea 
\gll {Gumurrinji=ni=biyang} {bul} {gan-arrga} {gudarlg.}\\
emu\textsc{=erg=seq} emerge \textsc{3min>3min}-approach.\textsc{pst.pfv} brolga\\ 
\glt `The emu then approached the brolga.' [ES96-A01-01.022]
\label{ex:bul ganarrga}
\z

An important characteristics of Jaminjung/Ngaliwurru – shared with a number of unrelated languages of the area – is the existence of two distinct predicative parts of speech. Verbs of the obligatorily inflecting type discussed above form a closed class with approximately 30 members (depending on the variety and speaker). They encode semantically generic events or states. In addition, there is an open class of semantically specific items restricted to predicative function but incompatible with inflectional marking, and therefore termed uninflecting verbs; other terms used in the literature are "coverb" and "preverb" (see \citealt{SchultzeBerndt2003}, \citealt{SchultzeBerndt2017} and \citealt{McGregor2002} for further discussion).

Inflecting verbs can occur as simple predicates in independent clauses, or combine with one or (rarely) two uninflecting verbs. The resulting combinations meet the widely accepted definition of complex predicates as monoclausal constructions, found in a single intonation unit, where two or more predicative constituents jointly contribute to the argument structure of the clause, share at least one semantic argument, and share values for tense, aspect, modality and polarity (see e.g. \citealt[108]{Butt1997}, \citealt{Butt2010}). In terms of both their lexical semantics and their argument structure they form nuclear junctures, in the terminology of Role and Reference Grammar \citep{FoleyOlson1985}.

The semantic valency or argument structure of a complex predicate, in most instances, matches the morphological transitivity of its inflecting verb, but there are exceptions (see also \citealt[1126-1128]{SchultzeBerndt2015}). For example, the combination of uninflecting verb \textit{bul} 'emerge' and inflecting verb \textit{-ma-} 'hit', with a meaning of 'appear', is syntactically intransitive: it takes a single core argument in the absolutive (unmarked) case, as illustrated in (\ref{ex:barangan}).

\ea
\gll{Barangan} {bul} {gani-ma-m.}\\
moon emerge \textsc{3min>3min}-hit.\textsc{prs}\\ 
\glt `The moon comes out.' (CS11-A103-01.057)
\label{ex:barangan}
\z

However, this syntactically monovalent status is not reflected in the morphological transitivity of the inflecting verb, which retains the transitive paradigm of pronominal indexing (with an invariable 3rd person singular, non-referential object prefix). Semantically intransitive complex predicates formed with a transitive inflecting verb therefore differ from (simple or complex) reflexive predicates in their morphological transitivity (see \sectref{sec:SchultzeBernd:3}).

It follows from the above that the existence of a reflexive counterpart of a given transitive complex predicate depends on its semantics and not purely on the morphological possibilities of the inflecting verb involved.

Another grammatical feature relevant for the discussion of reflexives is the grammatical treatment of inalienably possessed body parts and other part-whole relationships in Jaminjung/Ngaliwurru. As is cross-linguistically common, the preferred way of expressing the involvement of a (body) part in a state-of-affairs is to treat the whole (or possessor) as a core argument in what has been termed an external possessor construction: the possessor is indexed on the verb and optionally (and rarely) represented by a noun phrase as well, while the body part is represented as an additional noun phrase which agrees in case with the possessor expression. This is illustrated in (\ref{ex:lurr gardbany wirlga}) and (\ref{ex:Jarlig wuju gulyu}) for an intransitive and a transitive verb, respectively. 

\ea
\gll {Lurr} {ga-rdba-ny} {wirlga}. \\
pierce \textsc{3min}-fall-\textsc{pst.pfv} foot \\
\glt `She pierced her foot.' (lit: `She fell such that she got pierced, with respect to her foot.') [ES97-n02-Jam.020]
\label{ex:lurr gardbany wirlga}
\z

\ea
\gll {Jarlig} {wuju} {gulyu} {ba-ngu} {juwiya}! \\
child small wash \textsc{imp}-get/handle nose \\
\glt `Wipe the little child's nose!' (lit: `Wipe the little child (with respect to his/her) nose.') (ES12-N01-3Lgs.008)
\label{ex:Jarlig wuju gulyu}
\z

Inalienable relations in Jaminjung include not only body parts, but also items in the personal sphere, such as a shadow or a name (see example \ref{ex:jinij} below), but not kinship relations. What is important in the present context is that the presence of the part expression does not increase the number of arguments of the predicate; it is licensed exclusively by the presence of a semantic argument representing the 'whole'. 


\section{Formal properties of the reflexive construction} \label{sec:SchultzeBernd:3}


In Jaminjung/Ngaliwurru, coreferentiality of subject and object is obligatorily marked by a verbal suffix \textit{-ji} which immediately follows the verb root, as shown in the verbal structure template in (\ref{ex:IVstructure}) and illustrated in (\ref{ex:IV-refl}).

\ea Example of a reflexive verb form \\
\gll {ga-wirri-ja}\\
\textsc{3min}-bite.\textsc{refl.pst.pfv}\\ 
\glt `he/she/it bit himself/herself/itself'
\label{ex:IV-refl}
\z

The suffix is identical in form to the free 3rd person singular pronoun \textit{ji}, but it is unclear whether they are etymologically related, as reflexive and other intransitivising suffixes cognate with \textit{-ji} are widespread in Australian languages \citep[321]{Dixon2002}. As the template in (\ref{ex:IVstructure}) and the comparison of examples (\ref{ex:IV-refl}) and (\ref{ex:IV-trans}) shows, in the verbal template the reflexive suffix is found in a different slot from any object indexing prefix. This distinguishes Jaminjung/Ngaliwurru and its close relative Nungali from more distantly related languages within the Mirndi family such as Wambaya \citep{Nordlinger1998} and Jingulu \citep{Pensalfini2003}, which mark the reflexive by means of an invariable object prefix. It thus appears that the reflexive construction of Jaminjung/Ngaliwurru is an innovation at the level of the Western Mirndi (Yirram) subgroup.
Jaminjung/Ngaliwurru does not have reflexive pronouns or any other reflexive marker (see the pronoun table in the Appendix).

Reflexive marking interacts with tense/aspect marking: except for the past imperfective, the productive tense/aspect suffixes on non-reflexive verbs are not employed on reflexive verbs, and portmanteau forms are used instead, listed in Table \ref{tab:1:reflexive and TA}. These portmanteau suffixes appear even on verbs that otherwise mark tense/aspect distinctions by suppletion. They are clearly described, with the same forms as in the recent corpora, in \citet{Cleverly1968} and \citet{Bolt1971}.

\begin{table}
\caption{Reflexive and tense-aspect marking}
\label{tab:1:reflexive and TA}
 \begin{tabular}{lll}
  \lsptoprule
            & reflexive+tense/aspect & other tense/aspect allomorphs\\ 
  \midrule
  untensed  &   -ji &    –\\
  \textsc{prs}  &   -ji &   -m, -ya\\
  \textsc{pst.pfv}  &   -ja &   -\emptyset, -ny\\
  \textsc{pst.ipfv}  &   -ji-na &   -na, -nyi\\
  \lspbottomrule
 \end{tabular}
\end{table}

The reflexive suffix also has the function of reciprocal marking (see further \sectref{sec:SchultzeBernd:5}). Since it changes the morphological transitivity of the verb, it is analysed here as a derivational affix, although the construction also corresponds to what \citet{haspelmath_thisvolume} \todo{this volume} terms reflexive voice. It has a detransitivising effect in that it results in otherwise morphologically transitive verbs taking the intransitive paradigm of person prefixes (see \sectref{sec:SchultzeBernd:2}). However, rather than having a general detransitivising function, it is semantically restricted to marking reflexive or reciprocal action (see further \sectref{sec:SchultzeBernd:4} and \sectref{sec:SchultzeBernd:5}).

Syntactically, likewise, a reflexive verb is intransitive: it is not compatible with an ergative-marked argument representing the subject (although an instrumental phrase, marked with the same ergative/instrumental case marker, can be added). The morphological and syntactic contrast between the transitive verb \textit{-angga-} and its (suppletive) reflexive version is illustrated in (\ref{ex:gulyu gananggam}) and (\ref{ex:gulyu ngabiliji}).

\ea
\gll ({Marlayi=ni}) {gulyu=biyang} {gan-angga-m} {gugu=ni}. \\
woman\textsc{=erg/inst} wash\textsc{=seq} \textsc{3min>3min}-get/handle-\textsc{prs} water\textsc{=erg/inst} \\
\glt `(The woman) is washing it with water.' [ES12-A02-02.203]
\label{ex:gulyu gananggam}
\z

\ea
\gll ({Ngayug}) {gulyu} {nga-bili-ji} {gugu=ni}. \\
\textsc{1min} wash \textsc{1min-pot:}get/handle-\textsc{refl} water=\textsc{erg/inst} \\
\glt `I will wash myself with water.' \citep[53]{Bolt1971}
\label{ex:gulyu ngabiliji}
\z

For a few morphologically transitive inflecting verbs, e.g. \textit{-muwa} `have', no reflexive/reciprocal forms are attested in the data, due to semantic incompatibility or at least implausibility. In one case, a different verb systematically substitutes for the non-attested reflexive form of a morphologically transitive verb: For reflexively or reciprocally addressed speech, the ditransitive verb \textit{-ngarna} `give' replaces \textit{-junggu} `say/do', the usual speech-framing verb (which also has a range of other uses including both syntactically transitive and intransitive ones; see \citealt{SchultzeBerndt2008}). As a speech-framing verb, \textit{-junggu} `say/do' combines with a quotation in place of a theme object and with an oblique-marked recipient, as illustrated in (\ref{ex:bamanggu nami-ngunyi}). Since only direct objects, not obliques, are accessible to reflexive marking, \textit{-ngarna} `give' – which allows for the encoding of the recipient as a direct object – is used instead, as shown in (\ref{ex:gangarnaja}) (for an example of its reciprocal use, see \ref{ex:bunyngarnaja babiny-majawari}).

\ea
\gll "{Ba-manggu} {nami=ngunyi}!" {burru-yu=ngunggu} {jarlig=ni=gun}, {ngih}? \\
\textsc{imp}-hit \textsc{2min=abl} \textsc{3aug>3min-}say/do.\textsc{pst.pfv=3min.obl} child=\textsc{erg/inst=emph} \textsc{tag} \\
\glt `"Kill it yourself!" the children said to you, didn't they?!'
\label{ex:bamanggu nami-ngunyi}
\z

\ea
\gll "{Wanaja=warra} {nga-wu-yu?}" {ga-ngarna-ja}, {ji=wung} {warladbari}. \\
do.what\textsc{=dubit} \textsc{1min>3min-pot}-say/do \textsc{3min}-give-\textsc{refl.pst.pfv} \textsc{3min=restr} old.man \\
\glt `"What am I going to do? (...)" he said to himself, (being) by himself, the old man.' (lit. `gave himself') (Bowerbird and Crow story, recorded by J. Bolt; partially printed in \citealt[130--134]{Bolt1971}; glossing by ESB)
\label{ex:gangarnaja}
\z

As example (\ref{ex:gangarnaja}) also shows, reflexive marking reduces the valency of a ditransitive predicate to two core arguments. Usually, as in (\ref{ex:gangarnaja}), it indicates coreference between the subject and the recipient object; only in the case of a theme object which is animate or a natural force, coreference of the subject with a theme object is also attested, as in (\ref{ex:yurrg garraji ngurrgban}). This variable reference of the reflexive suffix reflects the similarly variable reference of the object prefix in the non-reflexive usage of ditransitive predicates \citep{SchultzeBerndt2010}.

\ea
\gll {Yurrg} {ga-rra-ji} {ngurrgban}. \\
show \textsc{3min}-put-\textsc{refl.prs} rainbow \\
\glt `A rainbow shows (after the rain).'
\label{ex:yurrg garraji ngurrgban}
\z

As already pointed out in \ref{sec:SchultzeBernd:2}, Jaminjung/Ngaliwurru does not have any valency-changing derivational morphology apart from the reflexive construction, nor does it exhibit voice marking. The possibility of combining the same uninflecting verb with different inflecting verbs (illustrated in \ref{ex:bul ganarrga} compared with \ref{ex:barangan}) can fulfil the same functions as applicative markers, causativisers, and other valency-changing morphology in many other languages \citep[1132-1145]{SchultzeBerndt2015}. As discussed in more detail in \sectref{sec:SchultzeBernd:4} below, the reflexive construction is restricted to encoding self-directed (autopathic) actions. 


\section{Function of the reflexive construction} \label{sec:SchultzeBernd:4}

The reflexive construction is a semantic reflexive in the terminology of \citet[27]{Geniusiene1987}: it is restricted to expressing coreference of subject and object in what \citet{Comrie1998} calls a 'local domain', i.e. when they are arguments of a single predicate, and it is used to encode autopathic actions, i.e. the subject referent acting upon him-/her-/itself. No instances of partial coreferentiality are attested, i.e. cases where the subject referent is a subset of the object referent.

This section discusses the three attested (not strictly delineated) subtypes of this use of the reflexive: prototypical reflexive actions where the object is presented as fully affected by the action of the coreferential subject (\sectref{sec:SchultzeBernd:4.1}), reflexive actions affecting an (explicitly mentioned) part of the object (\sectref{sec:SchultzeBernd:4.2}), and reflexive expressions of placement and position (\sectref{sec:SchultzeBernd:4.3}); a summary of reflexive uses is provided in \sectref{sec:SchultzeBernd:4.4}. For the reciprocal function of the same construction, see \sectref{sec:SchultzeBernd:5}.

\subsection{Full affectedness of object} \label{sec:SchultzeBernd:4.1}

 The reflexive construction is used with both "extroverted" and "introverted" predicates. Typical examples are (\ref{ex:jiwuly ngabaji birl-birl}) to (\ref{ex:warn gan-ngangu}). These illustrate the use of the reflexive for deliberate self-directed actions, including self-grooming (\ref{ex:warrwarr gamiliji=rndi} and \ref{ex:dirrma gamaja}), as well as for events of accidental self-harm (\ref{ex:jibug yanthijjaji} and \ref{ex:warn gan-ngangu}).

\ea
\gll {Jiwuly} n{ga-ba-ji} {birl-birl}. \\
cool \textsc{1min-pot}:hit-\textsc{refl} \textsc{rdp-}fan \\
\glt 'I will cool myself by fanning.' [ES97-A01-02.130]
\label{ex:jiwuly ngabaji birl-birl}
\z

\ea
\gll {Nginyju=biya} {mugurn} {ga-yu}, {janyung} {warr-warr} {ga-mili-ji=rndi}. \\
\textsc{prox=seq}  sleep \textsc{3min}-be.\textsc{prs}  another \textsc{rdp-}scratch \textsc{3min}-get/handle-\textsc{refl.prs=ego} \\
\glt `This one is sleeping, the other is scratching itself.' [CS11-a102-01.007]
\label{ex:warrwarr gamiliji=rndi}
\z

\ea
\gll {Dirrma} {ga-ma-ja:::}, {malinygalg=gug}. \\
paint.up \textsc{3min}-hit-\textsc{refl.pst.pfv} pretty\textsc{=limit}  \\
\glt `He painted himself, until he looked beautiful.' [ES09-A02-01.016]
\label{ex:dirrma gamaja}
\z

\ea
\gll {Majani=biya} {gunbarr} {yanthi-muwa}, {jibug} {yanth-ijja-ji}. \\
maybe\textsc{=seq} sore \textsc{irr:2min>3min}-have bust \textsc{irr:2min}-poke-\textsc{refl} \\
\glt `You might get a sore, you might poke yourself.' [ES97-A01-01.301]
\label{ex:jibug yanthijjaji}
\z

\ea
\gll {Nganthan=warra} {warn} {gan-ngangu}, {wardba=biyang} {nga-mili=ja}, {nga-rdba-ny}. \\
what\textsc{=dubit} get.hooked \textsc{3min>1min}-get/handle.\textsc{pst.pfv} entangle\textsc{=seq} \textsc{1min}-get/handle-\textsc{refl.pst.pfv} \textsc{1min}-fall-\textsc{pst.pfv} \\
\glt `Something (I don't know what) hooked me, I  entangled myself, and fell.' [ES03-A01-04.201]
\label{ex:warn gan-ngangu}
\z


\subsection{Reflexive actions affecting a part of the object} \label{sec:SchultzeBernd:4.2}

Self-directed actions frequently only affect an (inalienable) part of the object; the use of reflexive constructions in these instances is termed "partitive object reflexives" by \citet[195--196]{Geniusiene1987} (see also \citealt{Gabyinpress}). In Jaminjung/Ngaliwurru, the affected part is usually specified, in the form of an external possessor construction: an additional absolutive noun phrase representing the part is licensed if the whole (possessor) is indexed as a core argument. As pointed out in \sectref{sec:SchultzeBernd:2}, the presence of this additional noun phrase does not change the transitivity of the clause, i.e. it is not a syntactic object argument.

Typically, the part expression represents the body part of an animate which is impacted upon by the self-directed action, as in (\ref{ex:juwiya waipim nga-bili-ji}), (\ref{ex:juwiya murrb ga-mili-ji}), and (\ref{ex:thed nganth-inama-ny}), or towards which attention is directed, as in (\ref{ex:mung bangayiji jurruny}).

\ea
\gll {Ngulgul=wu} {bana} {reg}, {juwiya} {waipim} {nga-bili-ji}. \\
snot=\textsc{dat} \textsc{imp:2min>1min}-give rag(Kriol) nose wipe:\textsc{tr}(Kriol) \textsc{1min-pot}:get/handle-\textsc{refl} \\
\glt `Give me a rag for my snot, I want to wipe my nose.' [ES99-N01-Jam.073]
\label{ex:juwiya waipim nga-bili-ji}
\z

\ea
\gll {Juwiya} {murrb} {ga-mili-ji}. \\
nose covered.up \textsc{3min}-get/handle-\textsc{refl.prs} \\
\glt `He is covering his nose.' [ES97-A03-01.162]
\label{ex:juwiya murrb ga-mili-ji}
\z

\ea
\gll {Thed} {nganth-inama-ny}, {gad} {na-wirri-ja} {jarra}. \\
trip.over \textsc{2min>3min}-kick/step-\textsc{pst.pfv} cut \textsc{2min}-bite-\textsc{refl.pst.pfv} mouth \\
\glt `You tripped over something, and you bit yourself on the lip.' [ES97-A03-06.144]
\label{ex:thed nganth-inama-ny}
\z

\ea
\gll {Mung} {ba-ngayi-ji} {jurruny}. \\
look.out \textsc{imp}-see-\textsc{refl} hand \\
\glt `Watch your hands!' (to avoid getting burnt) [ES15-N01-Ngar-Ngali.005]
\label{ex:mung bangayiji jurruny}
\z

Body parts, most prominently \textit{burru} 'stomach', also represent emotions or mental states, giving rise to the use of the reflexive in combination with the external possession construction to express a self-directed emotion or mental effort, as in (\ref{ex:warlyang ngarraja}).

\ea
\gll {Burru} {ngarrgina} {warlyang} {nga-rra-ja}, {burru} {jarlag} {ng-agba.} \\
stomach \textsc{1min.poss} ahead \textsc{1min}-put-\textsc{refl.pst.pfv} stomach good \textsc{1min}-be.\textsc{pst.pfv} \\
\glt `I had mentally prepared myself, I was calm.' (lit. `I put my belly ahead, my belly was good') (before an operation) [CS15-A014-18.012]
\label{ex:warlyang ngarraja}
\z

Finally, as in other Australian languages \citep{Gabyinpress}, items other than body parts can be treated as inalienably possessed parts in an external possessor construction, as long as they are considered intimately associated with the whole, such as clothing, a sore (\ref{ex:gunbarr=gayi}), a shadow, or a name (\ref{ex:jinij}).

\ea
\gll {Yurr} {burra-ma-ji-na} {gunbarr=gayi}. \\
rub \textsc{3aug}-hit-\textsc{refl-pst.ipfv} sore=also \\
\glt `They also used to rub themselves (with it) on a sore.' [ES96-A18-01.265]
\label{ex:gunbarr=gayi}
\z

\ea
\gll {Jinij} {nij} {ga-bili-ji=yinyag}. \\
name say.name \textsc{3min-pot}:get/handle-\textsc{refl=1ua.obl} \\
\glt `She should tell us two her name.' (lit. `She should name herself (by way of her) name to us two.') [ES08-A08-02.046]
\label{ex:jinij}
\z

\subsection{Reflexive expressions of placement and position} \label{sec:SchultzeBernd:4.3}

The final function of the reflexive construction in Jaminjung/Ngaliwurru is to form inchoatives of placement or position. This function is labelled "autocausative" by \citet[196--197]{Geniusiene1987}, defined as “an action performed with one’s body and resulting in motion or change of position". In Jaminjung/Ngaliwurru, however, only a subset of potentially autocausative meanings are expressed using the reflexive construction. It is not used at all for either translational or non-translational motion, and for most positionals, inchoatives  are formed with the intransitive verb {-irdba} 'fall; assume a position', as illustrated in (\ref{ex:ngamang ngawirdbaj}) (see \citealt{SchultzeBerndt2015} for further discussion). 

\ea
\gll {Ngamang} {nga-w-irdbaj} {motika=ni.} \\
be.astride \textsc{1min-pot}-fall car=\textsc{loc} \\
\glt `I will get in the car.’ [ES16-A07-06.004]
\label{ex:ngamang ngawirdbaj}
\z

However, the reflexive of the inflecting verb \textit{-arra} 'put; place in a position' is used in expressions of a subject simultaneously bringing about and entering a spatial arrangement (\ref{ex:bad-bad ga-rra-ja}) and for changes of position which require energy to maintain the position (\ref{ex:diddi garraji}).

\ea
\gll {Gurlbinyji=ni} {bad-bad} {ga-rra-ja.} \\
paperbark=\textsc{erg/inst} \textsc{rdp}-covered \textsc{3min}-put-\textsc{refl.pst.pfv} \\
\glt `He covered himself with paperbark.’ [ES03-A03-01.024]
\label{ex:bad-bad ga-rra-ja}
\z

\ea
\gll {Diddi} {ga-rra-ji}. \\
lean \textsc{3min}-put-\textsc{refl.prs} \\
\glt `He is leaning over.' (ES97-A03-01.219)
\label{ex:diddi garraji}
\z

The uninflecting verb \textit{jubard} 'enclosed, shut in' has the semantics of a positional. Consequently, with the reflexive of the verb \textit{-arra} 'put; place in a position', \textit{jubard} encodes shutting oneself away, as in (\ref{ex:jubard ga-rra-ji warnda-ni}).

\ea
\gll {Garnmungul} {warnda} {walthub} {gan-antha}, {jubard} {ga-rra-ji} {warnda=ni}. \\
bandicoot grass inside \textsc{3min>3min}-take.\textsc{prs}  shut.in \textsc{3min}-put-\textsc{refl.prs} grass=\textsc{erg/inst} \\
\glt `The bandicoot takes grass inside (a hole) and shuts itself up with the grass.' [ES03-N01-JAM.057]
\label{ex:jubard ga-rra-ji warnda-ni}
\z

The position entered can be restricted to a body part, which can be specified by means of an external possessor construction as already discussed in \sectref{sec:SchultzeBernd:2} and \sectref{sec:SchultzeBernd:4.2}. An example is (\ref{ex:thandarlng garraji}).

\ea
\gll {Thandarlng} {ga-rra-ji} {jurruny}. \\
straight \textsc{3min}-put-\textsc{refl.prs} hand \\
\glt `She is straightening her arm.' (lit. 'She is putting herself straight [with respect to her] arm.') [ES96-A08-03.304]
\label{ex:thandarlng garraji}
\z


\subsection{Function of the reflexive construction: summary} \label{sec:SchultzeBernd:4.4}

As the discussion in this section has shown, the reflexive construction in Jaminjung/Ngaliwurru, in its reflexive function, always encodes a self-directed action, including actions affecting only an inalienable part of the subject. In the case of positionals, this may not be obvious from the most idiomatic English translations, but taking into account the semantics of the generic verb \textit{-arra} 'put; cause to be in a position' and the semantics of the accompanying uninflecting verbs of position, these expressions are fully covered by the description of the Jaminjung/Ngaliwurru reflexive construction as being restricted to the function of core reflexivity, with coreferentiality of an agentive subject and an affected object. In other words, it does not have a more general inchoative/anticausative function, and neither does it have any other detransitivising function such as antipassive. 

Given this characterisation, it is not surprising that verbs that allow for the reflexive construction (in its reflexive rather than reciprocal function, for which see \sectref{sec:SchultzeBernd:5}) are found towards the top end of a hierarchy \citep{Wichmann2015} generalised from the reflexive valency alternations in 16 languages in the Leipzig Valency database \citep{HartmannEtAl2013}. In this hierarchy, the Jaminjung/Ngaliwurru equivalents attested in reflexive construal are marked in boldface in (\ref{ex:verb-hierarchy}). Meanings shown in brackets only have reflexive uses in particular contexts; for `\textsc{give}' this is the speech framing use discussed in \sectref{sec:SchultzeBernd:3}; for `\textsc{break}' this is the use in expressions of `breaking a limb' as illustrated in (\ref{ex:football-nyunga}). 

\ea
\gll {Football-nyunga} {bag} {ga-rra-ja} {marnal.} \\
football-\textsc{orig} break \textsc{3min}-put-\textsc{refl.pst.pfv} ankle \\
\glt 'From (playing) football he broke his ankle.’
\label{ex:football-nyunga}
\z

Some of the gaps at the top end of the hierarchy in (\ref{ex:verb-hierarchy}), e.g. '\textsc{hear}', may be accidental gaps in the data; however, `\textsc{hide} (oneself)' and `\textsc{dress}' are encoded with an intransitive and a (non-reflexive) transitive predicate, respectively. 

%\protectedex{
\ea
\textsc{\textbf{wash, cover, shave, show, cut, see}, hide, dress, \textbf{[give]}, \textbf{touch > look at}, hear, \textbf{[put], [beat]}, hug, smell, \textbf{tie}, throw, \textbf{hit, kill}, like, fear, \textbf{wipe} > know, push, \textbf{ask for}, tear, \textbf{name}, help > search for, think, teach, take, say, carry, tell, \textbf{[break]}, send > frighten, talk, load > build, steal > bring, peel, cook, follow, eat > fill, meet, grind, sing, burn, dig, be sad, pour, roll > shout at, be dry, scream, laugh, run, play, feel pain, leave, go > jump, sit, blink, boil, be a hunter > live, rain, sink, be hungry, die, feel cold, climb > sit down > cough} \\
(adapted from \citealt[169]{Wichmann2015})
\label{ex:verb-hierarchy}
\z
%}

\section{Reciprocal function} \label{sec:SchultzeBernd:5}

As already indicated, the Jaminjung/Ngaliwurru reflexive marker also has a reciprocal use, a cross-linguistically frequent overlap (see e.g. \citealt[17]{Nedjalkov2007Overview}; \citealt{MaslovaNedjalkov2013}). The reciprocal interpretation of the construction requires a non-singular subject; however, a reflexive interpretation (i.e. multiple agents engaged in reflexive action) is often also possible. For example, (\ref{ex:mud-mud}) could also mean 'the dogs are (each) biting holes in themselves'. The interpretation is usually clear from context. 

\ea
\gll {Mud-mud} {burru-wiri-ji} {wirib} \textit{thanthu}. \\
\textsc{rdp-}make.hole \textsc{3aug}-bite-\textsc{refl.prs} dog \textsc{dem} \\
\glt `Those dogs are biting holes in each other.' [ES97-A03-06.078]
\label{ex:mud-mud}
\z

The reflexive/reciprocal form of the inflecting verb sec:SchultzeBernd:7{-ma} 'hit' with a non-singular subject is conventionally used to express 'fighting', as in (\ref{ex:yirrginy}). The equally conventionalised reflexive/reciprocal speech framing verb \textit{-ngarna} 'give' is discussed in \sectref{sec:SchultzeBernd:3}; its reciprocal function is illustrated in (\ref{ex:bunyngarnaja babiny-majawari}).

\ea
\gll {Yangarra} {buny-ma-ji} {yirrginy=jirram}, {jurruny=ni}. \\
kangaroo \textsc{3ua}-hit-\textsc{refl.prs} do.reciprocally=two lower.arm=\textsc{erg/inst} \\
\glt 'The two kangaroos fight one another with their paws.' [ES96-A04-03]
\label{ex:yirrginy}
\z

%\protectedex{
\ea
\gll "{Mindi-wardagarra-m} {ngiya} {gurang}!" {buny-ngarna-ja} {babiny-majawari}. \\
\textsc{1+2min}-follow-\textsc{prs}  \textsc{prox} old.man \textsc{3ua}-give-\textsc{refl.pst} sister-\textsc{dyad} \\
\glt `"Let's follow this old man" the two sisters said to each other.' [ES08-A04-02.106]
\label{ex:bunyngarnaja babiny-majawari}
\z
%}

A reciprocal interpretation can be available even with predicates encoding inherently symmetrical events such as 'split up, scatter' in (\ref{ex:larrarra yirrimilija}) and with other state-of-affairs that are semantically incompatible with a reflexive interpretation, such as 'look back at' in (\ref{ex:mung=jirram bunyngayija wib}). 

\ea
\gll {Mirdang} {nyanying=biya} {gani-yu}, {larrarra} {yirri-mili-ja}, {buru-buru} {yagbali-bina}.\\
night proper=\textsc{seq} \textsc{3min>3min}-say/do.\textsc{pst.pfv} scatter \textsc{1aug}-get/handle-\textsc{refl.pst.pfv} \textsc{rdp}-return place-\textsc{all} \\
\glt `It really became night then, we scattered, (going) back to our places.' [CS15-A014-17.173-175]
\label{ex:larrarra yirrimilija}
\z

\ea
\gll {Mung=jirram} {buny-ngayi-ja} {wib}. \\
watch=two \textsc{3ua}-see-\textsc{refl.pst.pfv} look.back \\
\glt `The two looked back (over their shoulders) at each other.' [ES01-A03-07.104]
\label{ex:mung=jirram bunyngayija wib}
\z

The reciprocal function can optionally be made explicit by the addition of an adverb \textit{yirrginy} 'reciprocally, in return', as shown in (\ref{ex:yirrginy}). This adverb is  also compatible with predicates not marked with the reflexive/reciprocal suffix, e.g. in a biclausal reciprocal expression.

The Jaminjung reflexive/reciprocal does not have any of the additional functions cross-linguistically associated with reciprocal marking (\citealt{Nedjalkov2007Encoding}; \citealt[9]{KoenigGast2006}) such as collective/joint action (Nedjalkov's 'sociative'), iterativity, or distributivity.\todo{Koenig Gast 2006 or 2008}


\section{Coreference of subject with non-objects} \label{sec:SchultzeBernd:6}

The reflexive construction cannot be used in the case of co-reference of the subject with any non-object. Such coreference is not encoded at all in Jaminjung/Ngaliwurru; rather, coreference is just one of the possible interpretations of free or enclitic pronouns. The following examples illustrate such pronouns in the functions of recipient/addressee (\ref{ex:gani-yu ji-wuwung}), recipient/beneficiary (\ref{ex:burringami=burrag}), and possessor (\ref{ex:mung ganamana}), with an interpretation of coreference with the subject. Outside the contexts for these specific examples, a non-coreferential interpretation is equally possible, as indicated in the translations.

\ea
\gll "{Wanaja=warra} {nga-wu-yu}" {gani-yu}, {ji=wu=wung}. \\
do.what\textsc{=dubit} \textsc{1min:3min-pot}-say/do \textsc{3min:3min}-say/do.\textsc{pst.pfv} 3min\textsc{=dat=restr}\\
\glt `"I don't know what to do" he said to himself.' (or: `just to him') (Bowerbird and Crow story, recorded by J. Bolt; transcription and glossing by ESB)
\label{ex:gani-yu ji-wuwung}
\z

\ea
\gll {Majani} {malinygalg} {nganjan} {burri-ngami=burrag}. \\
maybe pretty what \textsc{3aug>3min}-see.\textsc{prs=3aug:obl} \\
\glt `Maybe they see something nice for themselves.' (or: `for them') (describing the behaviour of thieves) [ES99-V01-06a.282]
\label{ex:burringami=burrag}
\z

\ea
\gll {Mung} {gana-ma-na} {gujarding}  {nuwina}. \\
watch \textsc{3min>3min}-have-\textsc{pst.ipfv}  mother \textsc{3min:poss} \\
\glt `She was looking after her own mother.' (or: `... after his/her mother') [ES15-A03-10.030]
\label{ex:mung ganamana}
\z

Cross-linguistically, intensifiers in adverbial function are frequently formally identical to reflexive pronouns \citep {Koenig2000, Koenig2013}. In Jaminjung/Ngaliwurru, which lacks a reflexive pronoun, this function is fulfilled either by an absolutive pronoun followed by the restrictive enclitic \textit{=wung} \citep{SchultzeBerndt2002}, or by a possessive pronoun with the agentive adverbialising suffix \textit{-man}. These are illustrated in (\ref{ex:jiwung}) and (\ref{ex:nuwinaman}), respectively. No clear examples of pronouns as adnominal intensifiers have been found.

\ea
\gll {Bugu} {ji=wung} {ngilijja} {ga-ngga}. \\
just \textsc{3min=restr} cry \textsc{3min}-go.\textsc{prs} \\
\glt `Just himself (i.e. without a cause) he is crying!' [ES97-A01-05.145]
\label{ex:jiwung}
\z

\ea
\gll {Ga-rdba-ny}, {nuwina-man}. \\
\textsc{3min}-fall-\textsc{pst.pfv} \textsc{3min.poss-adv} \\
\glt `He fell, by himself (by his own mistake).' [ES96-A09-02.216]
\label{ex:nuwinaman}
\z

In sum, this section has provided further evidence that the function of the Jaminjung/Ngaliwurru reflexive construction does not extend beyond encoding coreference of agent and patient (subject and object) in expressions of self-directed action, as discussed in \sectref{sec:SchultzeBernd:4}.


\section{Concluding discussion} \label{sec:SchultzeBernd:7}

The preceding sections provided an overview of the formal and functional properties of the Jaminjung/Ngaliwurru reflexive construction. Jaminjung/Ngaliwurru does not have reflexive pronouns; the reflexive is marked by an invariable derivational suffix on inflecting verbs which renders the verb morphologically and syntactically intransitive (see \sectref{sec:SchultzeBernd:3}). In \sectref{sec:SchultzeBernd:4} it was shown that the construction – even in its uses with positionals and in metaphorical uses – is restricted to indicating coreference between (semantic) subjects and objects, i.e. in what \citet{Comrie1998} calls the "most local domain", and that it always encodes self-directed (autopathic) actions. It cannot be used for co-reference of subjects with non-objects (\sectref{sec:SchultzeBernd:6}), and it does not have any general detransitivising function such as inchoative/anticausative, nor is it used in nontranslational motion expressions such as `turn, swivel'. 

The same construction, with nonsingular subjects only, functions as a reciprocal construction (see \sectref{sec:SchultzeBernd:5}). Here it is restricted to events where the same participants simultaneously fulfil the role of agents and patients, and it does not have any additional functions such as collective/joint action, iterativity, or distributivity. 

It should be pointed out that the Jaminjung/Ngaliwurru reflexive/reciprocal construction, while representative of a substantial subset of the languages in the Australian linguistic area (\citealt[320--321]{Dixon2002}; \citealt{Gabyinpress}), is by no means typical of all Australian languages. First, not all Australian languages have verbal reflexive marking; invariable reflexive pronouns are also found in a number of languages, for example in the Ngumpin-Yapa group which includes Warlpiri (Laughren, this volume), Djaru \citep{Tsunoda2007} and Bilinarra \citep[235--238]{MeakinsNordlinger2013}. In these languages, moreover, reflexive clauses are formally transitive and allow for ergative-marked agents. Second, the formal overlap between reflexive and reciprocal marking is common (found in 28 of the 55 languages surveyed by \citealt{Gabyinpress}), but even among those languages that have verbal reflexive marking, distinct verbal affixes for reflexive and reciprocal are found; examples are Kuuk Thayorre (\citealt{Gaby2008} and Gaby this volume) and Warrungu \citep{Tsunoda2007}. Third, functional extensions beyond the core reflexive and reciprocal meanings, attested in some Australian languages, are not found in Jaminjung/Ngaliwurru. For example, the reflexive/reciprocal verbal markers in Bininj Gun-wok \citep[495--497]{Evans2003} and (more marginally) in Nyikina \citep[114]{McGregor2000} also have a collective interpretation; other extensions of reciprocal marking found in Australian languages are pluractional, habitual, and continuative \citep{Gabyinpress}. The reflexive construction has a more general detransitivising function in a number of Australian languages, e.g. an anti-passive function in Yidinj \citep[532]{Dixon2002} and in several other Pama-Nyungan languages discussed by \citet{Terrill1997} and \citet[165--167]{Janic2010}. Even in a close neighbour of Jaminjung/Ngaliwurru, Wardaman, two of the closed-class verbs in their reflexive form function as inchoative markers \citep[208--210]{Merlan1994}, and there are mediopassive verbs which exhibit the reflexive/reciprocal suffix but do not encode action on self \citep[191]{Merlan1994}. Conversely, some languages, unlike Jaminjung, use different constructions for volitional and non-volitional reflexive action \citep{Gabyinpress}. Finally, some Australian languages, for example the Jarragan languages – western neighbours of the Western Mirndi languages – have a paradigm of middle verbs in addition to reflexive forms of transitive verbs (e.g. \citealt[60]{KofodPalmer2007} for Gajirrabeng). Thus, Australian languages exhibit considerable diversity in their encoding of reflexive meanings as well as the functional range of reflexive and reciprocal constructions.

To conclude, I will address the question of whether the findings for Jaminjung/Ngaliwurru support a number of universals that have been postulated for reflexive constructions (summarised in \citealt{haspelmath_thisvolume}; the numbering of the universals below follows Haspelmath’s list). Only those universals that are applicable in this language will be considered (omitting any that only apply to reflexive pronouns, or only if the language has more than one reflexive construction, for example).

Jaminjung/Ngaliwurru confirms the universal [Universal I] that if a language has a verbal derivational/reflexive voice marker, one of its uses is for autopathic coreference (agent-patient). As the discussion in this paper has shown, this is in fact the only use of the Jaminjung/Ngaliwurru construction apart from the reciprocal use. This reflexive marker is in fact obligatory for autopathic (self-directed) action, and only co-occurs with the intransitive (subject-indexing) person paradigm. This confirms the proposed universal [Universal V] that if a language has non-reflexive bound object person forms, these cannot be used coreferentially with the subject.

\citet[141]{Dixon2012} proposes the generalisation [Universal VI] that if a language has a verbal reflexive marker, it also has a verbal marker for reciprocal constructions. This is also confirmed for Jaminjung/Ngaliwurru: as we have seen, in fact the two markers are identical. 

An issue which is not as straightforwardly addressed is whether Jaminjung/Ngaliwurru also confirms Universal III from Haspelmath's (this volume) list. Haspelmath's formulation of this universal is as follows (in the version in \citealt[48]{Haspelmath2008}: "In all languages, the primary reflexive-marking strategy is at least as long as the primary disjoint-reference-marking strategy."

This universal is based on assumptions about a universal asymmetry in the frequency of coreference (infrequent) as opposed to disjoint reference (frequent) between subject and object. The relative frequencies of reflexive and non-reflexive transitive verbs in a subset of the Jaminjung/Ngaliwurru corpora (247 files, comprising 16149 annotation units) certainly confirms these assumptions: out of 4610 potentially transitive verbs, only 168 (3.6\%) were reflexive in form; this figure includes verbs in both reflexive and reciprocal interpretations. The principle of economy therefore demands that the more frequent disjoint-reference-marking strategy should be encoded by shorter, or at least not longer, forms than the less frequent reflexive-marking strategy. A proper application of this universal to Jaminjung/Ngaliwurru would involve a comparison of the length of all disjoint-reference verb forms with that of their corresponding reflexive verb forms while also taking into account the frequency of the respective verbs, which is beyond the scope of this paper. A brief glance at some common forms (\ref{ex:IV-refl-hit}) suggests that the reflexive verb forms are not longer, but also not shorter than the corresponding non-reflexive forms: reflexive marking involves the addition of a suffix, but removes any object-indexing prefix (see \sectref{sec:SchultzeBernd:3} for details). 

\ea Comparison of verb forms encoding disjoint reference and coreference (3rd person minimal / augmented)
\label{ex:IV-refl-hit}
\ea
\gll {gani-ma}\\
\textsc{3min>3min}-hit.\textsc{pst.pfv}\\ 
\glt `he/she hit him/her'

\ex
\gll {ganurru-ma}\\
\textsc{3min>3aug}-hit.\textsc{pst.pfv}\\ 
\glt `he/she hit them'

\ex
\gll {burrurru-ma}\\
\textsc{3aug>3aug}-hit.\textsc{pst.pfv}\\ 
\glt `they hit them'

\ex
\gll {ga-ma-ja}\\
\textsc{3min}-hit-\textsc{refl.pst.pfv}\\ 
\glt `he/she hit himself/herself'

\ex
\gll {burru-ma-ja}\\
\textsc{3aug}-hit-\textsc{refl.pst.pfv}\\ 
\glt `they hit themselves (or: each other)'
\z
\z

If we also include the presence of argument noun phrases in the evaluation (which are however optional and often not present in coherent discourse), an expression involving both a subject and object noun phrase is obviously longer than a syntactically intransitive reflexive expression which only allows for a subject noun phrase. Thus Jaminjung/Ngaliwurru at least does not present a counter-example to the above generalisation.

Still, it is of interest to also consider the original formulation of this universal in \citet[338]{Comrie1998}, which is more straightforwardly confirmed by Jaminjung/Ngaliwurru: "Languages are likely to have special marked forms that indicate coreference within the most local domain (the predicate and its arguments), possibly extending to more expanded domains."

If the expression "special marked forms" is understood not in terms of length and thus (mechanical) economy of speech production, but rather in terms of the presence of a distinct, specialised construction that signals the unexpected state of affairs, the reflexive suffix in Jaminjung/Ngaliwurru clearly confirms this prediction. One therefore might consider whether the frequency–economy correlation cannot also be met by a conception of economy that is more subtle than mere length of forms, but rather also accounts for the processing load for a distinct construction.

\section*{Acknowledgements}

My deepest appreciation goes to the speakers of Jaminjung and Ngaliwurru and their families who have worked with me over the years. I gratefully acknowledge the financial support received for fieldwork from the Max Planck Society and from the Volkswagen Foundation (DoBeS Programme, Grants 82957 and 85598). For helpful comments on earlier versions on this paper, I wish to thank Alice Gaby and the editors of this volume.

\section*{Appendix: Jaminjung/Ngaliwurru pronouns}

\begin{table}
\caption{Jaminjung/Ngaliwurru free and clitic pronouns (minimal-augmented system)}
\label{tab:pronouns}
 \begin{tabularx}{\textwidth}{XXXX} 
  \lsptoprule
            & \textsc{abs} & \textsc{obl} & \textsc{poss}\\ 
  \midrule
\textsc{1min}	&	ngayug	&	ngarrgu &	ngarrgina \\ 
\textsc{1+2min}&	mindi	&	mindag	&	mindajgina \\ 
\textsc{2min}	&	nami	&	(ngu)nggu	&	ngunggina \\ 
\textsc{3min}	&	ji	&	nu	&	nuwina \\ 
\textsc{1ua}	&	yirrinyi	&	yinyag	&	yinyajgina \\ 
\textsc{1+2ua} &	yurrinyi	&	yunyag	&	yunyajgina \\ 
\textsc{2ua}	&	gurrinyi	&	gunyag	&	gunyajgina \\ 
\textsc{3ua}	&	burrinyi	&	bunyag	&	bunyajgina \\ 
\textsc{1aug}	&	yirri	&	yirrag	&	yirrajgina \\ 
\textsc{1+2aug}	&	yurri	&	yurrag	&	yurrajgina \\ 
\textsc{2aug}	&	gurri	&	gurrag	&	gurrajgina \\ 
\textsc{3aug}	&	burri	&	burrag	&	burrajgina \\ 
  \lspbottomrule
 \end{tabularx}
\end{table}


\section*{Abbreviations}



\begin{tabularx}{.45\textwidth}{lQ}
1, 1+2, 2, 3 & 1st/1st+2nd/2nd/3rd person \\
> & subject acting on object \\
\textsc{abl} & ablative (‘[moving] from’) \\
\textsc{all} & allative \\
\textsc{aug} & augmented \\
\textsc{coll} & collective ‘all together’  \\
\textsc{dat} & dative  \\
\textsc{dem} & demonstrative (general, distance-neutral) \\
\textsc{dist} & distal \\
\textsc{dubit} & dubitative \\
\textsc{dyad} & kinship dyad \\
\textsc{emph} & emphatic assertion \\
\textsc{erg/inst} & ergative/instrumental\\
\textsc{ego} & speaker authority \\
\textsc{imp} & imperative \\
\textsc{ipfv} & imperfective \\
\textsc{iter} & iterative \\
\textsc{irr} & irrealis  \\
\textsc{limit} & limitative ('until') \\
\textsc{loc} & locative  \\
\textsc{min} & minimal  \\
\textsc{neg} & negative  \\
\textsc{obl} & oblique pronominal    \\
\textsc{orig} & origin, source, cause  \\
\end{tabularx}
\begin{tabularx}{.45\textwidth}{lQ}
\textsc{poss} & possessor \\
\textsc{pot} & potential \\
\textsc{prox} & proximal   \\
\textsc{pfv} & perfective\\
\textsc{prs} & present   \\
\textsc{pst} & past  \\
\textsc{rdp}  & reduplication    \\
\textsc{refl} & reflexive / reciprocal \\
\textsc{restr} & restrictive (‘just x, still v’)\\
\textsc{seq} & sequential (‘then’)  \\
\textsc{tag} & tag \\
\textsc{tr} & transitive marker  \\
\textsc{ua} & unit augmented \\
\end{tabularx}

%\begin{table}
%\caption{Abbreviations in interlinear glosses)}
% \begin{tabularx}{\textwidth}{L{1.7cm}QQ} 
%  \lsptoprule
%1, 1+2, 2, 3 & 1st/1st+2nd/2nd/3rd person & pronominal category \\
%> & subject acting on object & in glosses for bound pronouns \\
%\textsc{abl} & ablative (‘[moving] from’) & case clitic \\
%\textsc{all} & allative & case clitic \\
%\textsc{aug} & augmented & number (pronouns) \\
%\textsc{coll} & collective ‘all together’ & number marking enclitic \\
%\textsc{dat} & dative & case clitic \\
%\textsc{dem} & demonstrative (general, distance-neutral) & demonstrative \\
%\textsc{dist} & distal & demonstrative \\
%\textsc{dubit} & dubitative (‘don’t know wh-‘)  & enclitic  \\
%\textsc{dyad} & kinship dyad & derivational suffix \\
%%\textsc{erg/inst} & ergative/instrumental & case clitic \\
%\textsc{ego} & speaker authority & epistemic clitic \\
%\textsc{imp} & imperative  & modal prefix \\
%\textsc{ipfv} & imperfective  & tense/aspect suffix \\
%\textsc{iter} & iterative  & suffix/clitic  \\
%prefix \\
%\textsc{limit} & limitative ('until') & derivational suffix \\
%\textsc{loc} & locative  & case clitic \\
%\textsc{min} & minimal & number (pronouns) \\
%\textsc{neg} & negative & polarity particle \\
%\textsc{obl} & oblique pronominal  & clitic pronoun category  \\
%\textsc{orig} & origin, source, cause  & case clitic \\
%\textsc{poss} & possessor  & pronominal form \\
%\textsc{pot} & potential  & modal prefix \\
%\textsc{prox} & proximal  & demonstrative \\
%\textsc{pfv} & perfective & tense/aspect suffix \\
%\textsc{prs} & present  & tense/aspect suffix \\
%\textsc{pst} & past  & tense/aspect suffix  \\
%\textsc{rdp}  & reduplication  &  \\
%\textsc{refl} & reflexive / reciprocal & derivational suffix on verbs  \\
%\textsc{restr} & restrictive (‘just x, still v’) & clitic  \\
%\textsc{seq} & sequential (‘then’) & temporal/discourse clitic  \\
%\textsc{tag} & tag & particle \\
%\textsc{tr} & transitive marker & suffix on Kriol loan verbs \\
%\textsc{ua} & unit augmented & number (pronouns) \\
%  \lspbottomrule
% \end{tabularx}
%\end{table}


\printbibliography[heading=subbibliography,notkeyword=this]

\end{document}
