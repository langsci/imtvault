\documentclass[output=paper]{../langscibook}
\author{Eric W. Campbell\affiliation{University of California, Santa Barbara}}
\title{Coreference constructions in Zenzontepec Chatino}
\abstract{This chapter describes reflexive and other coreference constructions in Zenzontepec Chatino, an Otomanguean language of southern Mexico, based on a corpus of naturalistic language use. It is shown that the language has no specific reflexive marker or reflexive construction and that reflexivity and coreference are expressed in the same ways that non-coreference between clausal participants is expressed: coreferential coding devices occupy the multiple grammatical relations that share the coreference. While intensifiers may co-occur with and reinforce coreference for emphasis or disambiguation, they are neither necessary nor sufficient for expressing reflexivity on their own. As domains of grammar that in many languages share formal content or functional overlap with reflexives, the language’s Reciprocal construction and correlate of middle voice are also briefly described; they do not overlap with reflexive expressions in Zenzontepec Chatino. While most languages display a specialized construction for expressing reflexives, Zenzontepec Chatino provides a clear and interesting exception to this cross-linguistic tendency}
\IfFileExists{../localcommands.tex}{
  \addbibresource{localbibliography.bib}
  \input{../localpackages}
  %Copy this to localcommands.tex

\usepackage[english]{babel}
\usepackage{amsmath}
\usepackage{amssymb,amsfonts,textcomp}
\usepackage{array}
\usepackage{hhline}
\usepackage{hyperref}

\newenvironment{styleStandard}{}{}
\newenvironment{stylelsAbstract}{}{}
\newenvironment{stylelsSectioni}{}{}
\newenvironment{stylelsSectionii}{}{}
\newenvironment{stylelsBulletList}{}{}
\newenvironment{styleBibliographyi}{}{}
\newenvironment{listWWNumxxvleveli}{}{}
\newenvironment{listWWNumxxvlevelii}{}{}
\newenvironment{listWWNumxxvleveliii}{}{}
\newenvironment{listWWNumxxvleveliv}{}{}
\newenvironment{listWWNumixleveli}{}{}
\newenvironment{listWWNumixlevelii}{}{}
\newenvironment{listWWNumixleveliii}{}{}
\newenvironment{listWWNumixleveliv}{}{}

\newcommand\textstyleListLabelxvi[1]{#1}
\newcommand\labellistWWNumxxvleveli{\thelistWWNumxxvleveli.}
\newcommand\labellistWWNumxxvlevelii{\thelistWWNumxxvlevelii.}
\newcommand\labellistWWNumxxvleveliii{\thelistWWNumxxvleveliii.}
\newcommand\labellistWWNumxxvleveliv{\thelistWWNumxxvleveliv.}
\newcommand\labellistWWNumixleveli{[F0B7?]}
\newcommand\labellistWWNumixlevelii{\textstyleListLabelxvi{o}}
\newcommand\labellistWWNumixleveliii{[F0A7?]}
\newcommand\labellistWWNumixleveliv{[F0B7?]}

\newcounter{listWWNumxxvleveli}
\newcounter{listWWNumxxvlevelii}[listWWNumxxvleveli]
\newcounter{listWWNumxxvleveliii}[listWWNumxxvlevelii]
\newcounter{listWWNumxxvleveliv}[listWWNumxxvleveliii]
\newcounter{itemize}  
  \input{../localhyphenation} 
  \togglepaper[1]%%chapternumber
}{}

\begin{document}
\maketitle 
%\shorttitlerunninghead{}%%use this for an abridged title in the page headers


\section{Introduction}
\label{sec:campbell:1}

In their typological survey on reflexive and reciprocal constructions in 150 languages, \citet[172]{HeineMiyashita2008} state that “reflexivity and reciprocity are universal concepts insofar as all languages can be expected to have some grammaticalized expression for both.” In fact, they cite only one language, the Portuguese-based Creole of São Tomé, as having “no productive means of expressing reflexivity” (ibid.). These findings echo those of \citet[24]{Kemmer1993}, who states that most languages have a \textsc{reflexive} \textsc{marker}: “a special marker to indicate that the Agent and Patient (or analogous semantic roles) in an event ordinarily involving two such roles are the same entity.” In a similar vein, for the purposes of cross-linguistic comparison, Haspelmath (this volume)\todo{this volume / define later} defines a \textsc{reflexive} \textsc{construction} as a construction that (i) is used only when (at least) two participants in a clause are coreferential and (ii) also includes some marker indicating that there is such coreference.



The present chapter explores coreference and reflexivity in Zenzontepec Chatino (czn), an Otomanguean language of Southern Mexico, which does not have any reflexive marker, nor does it display any specific construction dedicated to expressing reflexives. Although it is typologically uncommon for a language to lack a reflexive construction, it is not the case that the language has “no productive means of expressing reflexivity.” Rather, reflexivity is expressed in the language by using the same constructions that are used when there is no coreference: the typical referential coding devices (NPs, independent or dependent pronouns, anaphoric zero) are used in their canonical positions for expressing each of the coreferential roles, supported by the usual semantic and contextual factors and disambiguation strategies that help language users maintain referential coherence in discourse. A pair of examples illustrate a transitive clause with non-coreferential agent and patient \REF{ex:campbell:1a} and a canonical reflexive expression with coreferential agent and patient \REF{ex:campbell:1b};\footnote{The orthography used here differs from the IPA as follows: \textit{r}  = [ɾ], \textit{ty} = [tʲ], \textit{ly} = [lʲ], \textit{ny}  = [nʲ], \textit{ch} = [ʧ], \textit{x} = [ʃ], y = [j], j = [h], V̨ = nasal vowel, VV = long vowel, \={V} = mid tone, \'{V} = high tone, ‘+’ = compound boundary. Grammatical abbreviations beyond the Leipzig Glossing Rules are listed at the end of the chapter.} in both cases, the agent is expressed by a pronoun that encliticizes to the verb, and the patient, whether a lexical NP or an enclitic pronoun, is flagged by the device \textit{jiʔį}.


\ea\label{ex:campbell:1}
\ea
\label{ex:campbell:1a}
Non-coreferential agent and patient\\
\gll ntē-naʔa+tiʔi=kāʔá\textbf{=na}  jiʔį  \textbf{ya.jnii}\\
     \textsc{prog}{}-see+pain=also=\textsc{1incl}  \textsc{obj}  plant\\
\glt ‘we are also making the plants suffer’ [familia 4:18]

\ex\label{ex:campbell:1b}
{Coreferential agent and patient (reflexive)}\\
\gll Ntē-naʔa+tiʔi\textbf{=na}  jiʔį=\textbf{na}.\\
     \textsc{prog}{}-see+pain=\textsc{1incl}  \textsc{obj}{}-\textsc{1incl}\\
\glt ‘We are making ourselves suffer.’ [ntelinto itza7 17:22]
\z
\z


After presenting some basic information about Zenzontepec Chatino and the data and methods used in this study \sectref{sec:campbell:2}, the basic syntax, grammatical relations, and referential coding devices of the language are described in \sectref{sec:campbell:3}. With those details outlined, the heart of the chapter \sectref{sec:campbell:4} examines the language’s correlate of a canonical reflexive construction, and then other expressions of coreference, both within and across clauses. From there, \sectref{sec:campbell:5} describes Zenzontepec Chatino grammar in the domains that most often overlap with reflexive expressions in languages of the world: middle voice, the Reciprocal construction, and intensifiers. The Zenzontepec Chatino correlate of middle voice is rarely used, and, like reflexives, it has no dedicated construction. The Reciprocal construction, on the other hand, does have a dedicated marker, which is grammaticalized from the noun ‘companion’, and thus does not overlap with the expression of reflexives. The language’s two intensifiers may be used to reinforce coreference and reflexivity but are themselves never necessary or sufficient for expressing reflexives. Some final remarks conclude the chapter in \sectref{sec:campbell:6}.


\section{Language and research context}
\label{sec:campbell:2}

The municipality of Santa Cruz Zenzontepec is situated in the Sierra Madre del Sur mountains in the southwestern part of Oaxaca state, Mexico (\figref{fig:campbell:1}). The 2010 national census \citep{INEGI2010} reports a municipal population of about 18,000 residents, some 11,000 of which speak an Indigenous language (in most cases Zenzontepec Chatino), and of which about 4,000 are monolingual Indigenous language speakers. These numbers reflect a palpable and progressing language shift to Spanish. There is ongoing and significant migration out of the traditional community, and the language is also currently spoken in diaspora communities in other parts of Mexico and the United States, especially in California, where most Chatino people work in the state’s large agricultural industry.


\begin{figure}
\caption{Map of the Chatino region in Oaxaca, Mexico (Zenzontepec is represented by ZEN)}
%\includegraphics[width=\textwidth]{figures/CampbellChatino-img001.png}
\label{fig:campbell:1}
\end{figure}

Zenzontepec Chatino is the most divergent extant language of the Chatino language group, a cluster of about 17 language varieties (Ezéquiel Vásquez, in \citealt{Boas1913}; \citealt{Campbell2013Diversification}; \citealt{CruzWoodbury2014}: 265; \citealt{Sullivant2016}). Chatino is coordinate with the larger and more diversified Zapotec language group \citep{Mechling1912}, and the two together form the Zapotecan group. Zapotecan, in turn, is a major subgroup of the Otomanguean language family \citep{Rensch1976}, which is an ancient and diverse family spread across the Mesoamerican cultural and linguistic area (\citealt{Kirchhoff1967}[1943]; \citealt{CampbellEtAl1986}; \citealt{SmithStark1988}).



The data used in this study were selected from a corpus of about 21 hours of recorded, transcribed, and translated naturalistic discourse of varied genres that has been developed collaboratively with Zenzontepec community members since 2007. Examples of language use by about a dozen men and women of varied ages are included in the present work, drawn from about 30 different recorded events. Each example’s translation is followed by a tag in square brackets that includes keywords and time points within the source texts, most of which can be consulted and appreciated by registered users of the Endangered Language Archive \citep{Campbell2013Zenzontepec}.\footnote{The reference for the collection refers to the archival depositor, and not the owner of the copyright or intellectual property right of the material in the collection, which remain with community-member participants.}


\section{Basic syntax and reference in Zenzontepec Chatino}
\label{sec:campbell:3}
This section provides a sketch of Zenzontepec Chatino basic syntax \sectref{sec:campbell:3.1}, grammatical relations \sectref{sec:campbell:3.2}, and referential coding devices \sectref{sec:campbell:3.3}, all of which must be understood in order to characterize and understand coreference and reflexivity in the language.


\subsection{Basic syntax and alignment}
\label{sec:campbell:3.1}

Zenzontepec Chatino verbs obligatorily inflect for aspect-mood via complex combinations of prefixes (with some fusion) and tonal alternations (\citealt{Campbell2016,Campbell2019}). The syntax prefers head-initial structures. Basic constituent order in intransitive clauses is VS, as shown in \REF{ex:campbell:2}, where the sole participant in the event is a lexical NP that follows the intransitive verb. An adjectival predication whose sole participant is expressed by a dependent (enclitic) pronoun is shown in \REF{ex:campbell:3}. The enclitic attaches directly to the predicate.

\ea
\label{ex:campbell:2}
{Intransitive verbal clause}\\
\gll Nku-tiyaą  tsaka  máʔa.\\
     \textsc{pfv}{}-arrive.here  one  lady\\
\glt ‘A lady arrived.’ [laa nka sa7ne 1:01]
\z

\ea\label{ex:campbell:3}
{Adjectival predication}\\
\gll Tyāʔ  luʔu=ya.\\
     still  alive=\textsc{1incl}\\
\glt ‘We’re still alive.’ [historia1 7:19]
\z


Basic constituent order in transitive clauses is VAO. In \REF{ex:campbell:4} a lexical NP agent immediately follows the verb, and the patient NP follows the agent. If the patient-like participant is topical, it is preceded by the flagging device \textit{jiʔį} (\citealt{DalrympleNikolaeva2011}; \citealt{Campbell2015}), as shown in \REF{ex:campbell:5}. Thus, the language displays information structure-based differential object marking.


\ea\label{ex:campbell:4}
{Transitive clause with non-topical patient}\\
\gll Kākʷá  to.niʔi=ri  nka-sāʔ\'{ą}=kāʔá  nkʷítsą=Vʔ  kiiʔ.\\
     near  house=only  \textsc{pfv-caus}.be.attached=also  child=\textsc{ana}  fire\\
\glt ‘Right near the house the child also set a fire.’ [ntetakan7 jute7 0:24]
\z

\ea\label{ex:campbell:5}
{Transitive clause with topical patient} \\
\gll Nka-s-atīʔ  tī  niʔ  kūʔwí=Vʔ  \textbf{jiʔį̄}  na  lateʔ  chaja=Vʔ.\\
     \textsc{pfv}.\textsc{caus}{}-\textsc{trn}{}-get.untied  \textsc{tplz}  \textsc{3rsp}  drunk=\textsc{ana}  \textsc{obj}  \textsc{def}  cloth  tortilla=\textsc{ana}\\
\glt ‘The drunk untied the tortilla cloth.’ [kwini7 laja 9:23]
\z


Dependent pronouns in S role \REF{ex:campbell:6} and A role \REF{ex:campbell:7} encliticize to the predicate, while in O role \REF{ex:campbell:8} they always encliticize to the object marker \textit{jiʔį}.


\ea\label{ex:campbell:6}
{1sg pronoun in S role}\\
\gll Yākʷá  nk-ula\textbf{=\={ą}ʔ}.\\
     there  \textsc{pfv}{}-be.born=\textsc{1sg}\\
\glt ‘I was born there.’ [antes aparatos 0:53]
\z

\ea\label{ex:campbell:7}
{1sg pronoun in A role}\\
\gll jā  tala  k-ājá\textbf{=\={ą}ʔ}  jiʔį̄=chúʔ\\
     \textsc{conj}  for.sure  \textsc{pot}{}-get=\textsc{1sg}  \textsc{obj}=\textsc{3sg.f}\\
\glt ‘because for sure I am going to get her’ [burro zopilotes 1:20]
\z

\ea\label{ex:campbell:8}
{1sg pronoun in O role}\\
\gll Nka-lōó=yu  jiʔį̄\textbf{=\={ą}ʔ}.\\
     \textsc{pfv}{}-take.out=\textsc{3sg.m}  \textsc{obj}{}-\textsc{1sg}\\
\glt ‘He took me out.’ [kwini7 laja 11:46]
\z


Dependent pronouns in S and A functions never encliticize to the marker \textit{jiʔį̄}, and dependent pronouns in O function never encliticize to the predicate. Thus, the language displays accusative alignment. Note that there is only one set of dependent (enclitic) pronouns, and one parallel set of independent (emphatic) pronouns in the language (see \sectref{sec:campbell:3.3}). The syntactic function of any NP is encoded by its position in the clause (or its host in the case of enclitic pronouns) as well as semantic, pragmatic, and contextual factors.



The language has flexible constituent order in discourse. For example, the clause in \REF{ex:campbell:9} displays OVA constituent order in a construction in which the theme is in focus.


\ea\label{ex:campbell:9}
{OVA constituent order}\\
\gll Nkuti  chojo  nch-ujwiʔ  tī  kʷaa.\\
     seed  squash  \textsc{prog}{}-sell  \textsc{tplz}  \textsc{1incl}\\
\glt ‘We were selling squash seeds.’ [naten7 michen 5:36]
\z

\subsection{Grammatical relations}
\label{sec:campbell:3.2}
The examples in \REF{ex:campbell:6}--\REF{ex:campbell:8} show that Zenzontepec Chatino alignment is accusative, and the language has a Subject grammatical relation that includes arguments in S and A functions. In ditransitive constructions, the theme-like participant (T) is unflagged if non-topical \REF{ex:campbell:10} and flagged by \textit{jiʔį} if topical \REF{ex:campbell:11}, while the recipient-like participant (R) is obligatorily flagged by \textit{jiʔį}, as shown in both examples. Note that the flagging device \textit{jiʔį} often reduces to \textit{jį} or even contracts to \textit{j-\textsuperscript{M}} in natural speech.


%\setcounter{itemize}{9}
\ea\label{ex:campbell:10}
{Ditransitive with non-topical theme}\\
\gll Nt-u-tūʔú=yu  \textbf{kiiʔ}  j-nuw\={ę}ʔ.\\
     \textsc{hab-caus-trn.}be.inside=3\textsc{sg.m}  fire  \textsc{dat}{}-\textsc{3ana}\\
\glt ‘He put fire to that.’ [juan oso 9:32]
\z

%\setcounter{itemize}{1}
\ea\label{ex:campbell:11}
{Ditransitive with topical theme}\\
\gll Tyāá  tī  ntyūsé  \textbf{j-nuwęʔ}  jiʔį́  nī  ná  k-ūts\={ę}.\\
     \textsc{pot}.\textsc{iter}.give  \textsc{tplz}  god  \textsc{obj}{}-\textsc{3ana}  \textsc{dat}.\textsc{2sg}  now  \textsc{neg}  \textsc{pot}{}-be.afraid.\textsc{2sg}\\
\glt ‘God will give that back to you, don’t be afraid.’ [historia2 21:31]
\z


Beneficiary (and maleficiary) participants are also obligatorily flagged by \textit{jiʔį}, regardless of their topicality \REF{ex:campbell:12}. Thus, recipients, beneficiaries, and maleficiaries pattern together as an Indirect Object grammatical relation (obligatorily flagged by \textit{jiʔį}, glossed as \textsc{dat}), while patients and themes pattern alike as a Direct Object grammatical relation (flagged by \textit{jiʔį} only if topical or pronominal, glossed as \textsc{obj}). The language thus displays indirective alignment in ditransitives \citep{MalchukovEtAl2010}.


\ea\label{ex:campbell:12}
{Beneficiary} \\
\gll Liwrū  k-u-jnyā=yu  \textbf{jiʔį̄}  \textbf{kitsę}.\\
     book  \textsc{pot-caus-}move=3\textsc{sg.m}  \textsc{dat}  village\\
\glt ‘He’s going to make a book for the village.’ [historia1 30:22]
\z


Instruments \REF{ex:campbell:13} and comitatives \REF{ex:campbell:14} are preceded by the flagging device \textit{lóʔō} ‘with’, and together form an adjunct grammatical relation: Oblique \textit{lóʔō} (glossed as \textsc{with)}.


\ea\label{ex:campbell:13}
{Instrument}\\
\gll Nti-ʔnya=ūʔ  kela  j-\'{\k{u}}ʔ  \textbf{lóʔō}  \textbf{jlyekʷā}.\\
     \textsc{hab-}clear=\textsc{3pl}  corn.field  \textsc{gen}{}-3\textsc{pl}  \textsc{with}  hoe\\
\glt ‘They would clear their corn fields with hoes.’ [cambios 1:09]
\z

\ea\label{ex:campbell:14}
{Comitative}\\
\gll Nk-yánō  na  nkʷīts\'{ą}  kíʔyū=Vʔ  \textbf{lóʔō}  \textbf{juti}.\\
     \textsc{pfv-}stay  \textsc{def}  child  male=\textsc{ana}  \textsc{with}  father(.3)\\
\glt ‘the child stayed with his father.’ [nkwitzan ti7i 0:32]
\z


Adnominal grammatical relations include Inalienable Possessor and Alienable Possessor. Inalienable possession is expressed by juxtaposition; the possessor NP follows the possessum NP, as shown in \REF{ex:campbell:15}. If the Inalienable Possessor is encoded by a dependent pronoun, then it encliticizes to the possessum NP. In the Alienable Possession construction, the Alienable Possessor is—similar to the Indirect Object adverbal relation—obligatorily flagged by \textit{jiʔį} (glossed as \textsc{gen}), following the possessum \REF{ex:campbell:16}.


\ea\label{ex:campbell:15}
{Inalienable Possessor}\\
\gll Lēʔ  nkʷí-cha\={ą}  nyáʔa  \textbf{na}  \textbf{sēné=Vʔ}.\\
     then  \textsc{pfv}{}-arrive.here  mother  \textsc{def}  toad\textsc{=ana}\\
\glt ‘Then the toads’ mother arrived.’ [kwentu sene 1:19]
\z

\ea\label{ex:campbell:16}
{Alienable Possessor}\\
\gll Létā  tsoʔō  nte-chaʔne  wātá  \textbf{jiʔį=yu.}\\
     very  well  \textsc{prog-}reproduce  livestock  \textsc{gen-3sg.m}\\
\glt ‘His livestock are reproducing very well.’ [vaquero 0:59]
\z


Zenzontepec Chatino lacks productive voice alternations such as passives or antipassives that would rearrange argument structure, but instead displays a variety of lexicalized and not widely productive derivational valence alternations \citep{Campbell2015}. The pair of examples in \REF{ex:campbell:17} illustrates an equipollent Intransitive-Causative alternation. The intransitive verb takes the intransitivizing prefix \textit{y-} \REF{ex:campbell:17a}, which alternates with the transitivizing prefix \textit{t-} and the Causative prefix \textit{u-} in the transitive causative verb \REF{ex:campbell:17b}, but note that the Causative marker \textit{u-} is elided by the aspect prefix vowel in vowel hiatus in this instance. The Subject of the intransitive clause becomes the Direct Object of the Causative construction, which has an added agent.


\ea\label{ex:campbell:17}
{An equipollent Intransitive-Causative alternation}\\
\ea
\label{ex:campbell:17a}
{ Intransitive}\\
\gll Nk-\textbf{y-}ak\={ę}  na  liwrū=Vʔ.\\
     \textsc{pfv}{}-\textsc{itrn}{}-burn  \textsc{def}  book=\textsc{ana}\\
\glt ‘The books were burned.’ [historia1 31:05]


\ex\label{ex:campbell:17b}
{  Causative}\\
\gll Lūwíʔ  nka-(\textbf{u-})\textbf{t}{}-āk\'{ę}=\={\k{u}}ʔ  j-nā  liwrū=Vʔ.\\
     then  \textsc{pfv-(caus}{}-)\textsc{trn}{}-burn=\textsc{3pl}  \textsc{obj}{}-\textsc{def}  book=\textsc{ana}\\
\glt ‘Then they burned the books.’ [historia1 31:47]
\z
\z


There are only a few detransitivizing valence alternations in the language, and they apply to relatively few verbs \citep{Campbell2015}. Thus, the language displays a strong transitivizing preference, as most simplex verb stems are monovalent, and most valence alternations increase transitivity. The preference for low transitivity in the verbal lexicon is paralleled in (or the result of) patterns of language use in which events are often expressed in constructions that convey low agency, as shown in \REF{ex:campbell:18}.


\ea\label{ex:campbell:18}
{De-agentive expression}\\
\gll Y-aa  ntsukʷāʔ  jiʔį̄,  tsa  majlyā.\\
     \textsc{pfv}{}-go  corn  \textsc{gen(.3)}  one  almud\\
\glt ‘They took him some corn, one almud ({\textasciitilde}4kg).’ (lit. ‘His corn went, one almud.’) [historia3 6:25]
\z

\subsection{Referential coding devices}
\label{sec:campbell:3.3}

In order to understand how coreference is expressed in Zenzontepec Chatino, it is necessary to understand how reference is established and tracked in discourse. The sequence in \REF{ex:campbell:19} introduces the protagonist of a narrative with the noun ‘person’ preceded by the indefinite article (the numeral ‘one’). In the following clause, the same referent is the Subject of the verb, but since it is topical and the only possible agent of the verb, it is not overtly referred to; this is anaphoric zero \citep{Givon1983}. Zero anaphora— glossed as “(.3)”—is common in Zenzontepec Chatino discourse but is only allowed for third person referents. The Direct Object in the second clause is ‘corn’, an indefinite mass noun, and since it is non-topical it occurs with no article or flagging. The third clause is also transitive, with anaphoric zero Subject (the protagonist), and another new referent, ‘granary’, encoded as an indefinite D.O. The D.O. is introduced into the discourse in an Alienable Possession construction, in which the Alienable Possessor (the protagonist, coreferential with the Subject) is flagged by the Genitive marker \textit{jiʔį̄}, but is again otherwise unexpressed (anaphoric zero).


%\setcounter{itemize}{18}
%original did not have abc
\ea\label{ex:campbell:19}
{Introducing and establishing referents in discourse}\\
\ea
\label{ex:campbell:19a}
\gll Nk-ā+tāká  \textbf{tsaka}  \textbf{nyat\={ę}}  ʔne  jnyá.\\
     \textsc{pfv}{}-be+exist  \textbf{one}  \textbf{person}  \textsc{hab}.do  work\\
\glt ‘There was a person that worked.’


\ex\label{ex:campbell:19b}
\gll Nt-u-tūkʷá  \textbf{ntsukʷāʔ}.\\
     \textsc{hab}{}-\textsc{caus}{}-be.inside\textbf{(.3)}  \textbf{corn}\\
\glt ‘He planted corn.’


\ex\label{ex:campbell:19c}
\gll Wiʔ  niī  nkʷ-ise+toǫ  \textbf{jaʔwa}  jiʔį̄.\\
     \textsc{ana}  now  \textsc{pfv}{}-turn+stand\textbf{(.3)}  granary  \textsc{gen}\textbf{(.3)}\\
\glt ‘From there, he built his granary.’ [kwiten7 nkatzen 0:32]
\z
\z


The referring expressions in \REF{ex:campbell:19} provide examples of what (\citealt{Givon1983}, \citealt[6]{Givon2017}) considers devices that express low referential continuity (indefinite lexical NPs) and the highest referential continuity (anaphoric zero) in discourse. Intermediate on the scale from lower to higher referential continuity are definite NPs (see e.g. ‘the book’ in the examples in \REF{ex:campbell:17}, independent (emphatic) pronouns, as in the 3\textsuperscript{rd} person Anaphoric pronoun in Subject function in \REF{ex:campbell:20}, and dependent (enclitic) pronouns like the 3\textsuperscript{rd} person Feminine pronoun in \REF{ex:campbell:21}.


%\setcounter{itemize}{1}
\ea\label{ex:campbell:20}
{Independent pronoun}\\
\gll Ná  tsaka  ntaʔą  nka-su  tī  \textbf{nuwęʔ}.\\
     \textsc{neg}  one  fresh.ear.corn  \textsc{pfv.caus-}come.loose  \textsc{tplz}  3\textsc{ana}\\
\glt ‘Not even one fresh ear of corn did \textit{she} cut.’ [kwiti7yu 12:34]
\z

\ea\label{ex:campbell:21}
{Dependent pronoun}\\
\gll Nka-sāʔ\'{ą}\textbf{=chūʔ}  chaja  telā.\\
     \textsc{pfv.caus-}be.attached=\textsc{3sg.f}  tortilla  night\\
\glt ‘She made tortillas at night.’ [historia3 6:31]
\z


\tabref{tab:campbell:1} presents the full paradigm of independent pronouns in the language. The first and second person forms have unique roots, with a distinction between singular and plural, and a clusivity distinction in the first person plural. Third person independent pronouns consist of elements which mostly resemble the dependent pronouns fused with demonstratives (Proximal, Distal, or Anaphoric), as appropriate for the context. For example, in \REF{ex:campbell:20} the subject is expressed with a form consisting of the general subordinator \textit{nu} and the Anaphoric demonstrative \textit{wiʔ}.


%%please move \begin{table} just above \begin{tabular
\begin{table}
\caption{Zenzontepec Chatino independent pronouns}
\label{tab:campbell:1}

\begin{tabularx}{0.4\textwidth}{p{0.3cm}p{0.7cm}XX} 
\lsptoprule
&  & {\bfseries \textmd{\textsc{sg}}} & {\bfseries \textmd{\textsc{pl}}}\\
\hline
1 & \textsc{excl} & \textit{nāáɁ} & \textit{(ya)kʷaa}\\
& \textsc{incl} &  & \textit{naa}\\
2 &  & \textit{nuʔu} & \textit{kʷaʔą}\\
3 & \textsc{neut} & \textit{nu-}\textsc{dem} & \\
& \textsc{m} & \textit{yu-}\textsc{dem} & \textit{ų́ʔ}{}-\textsc{dem}\\
& \textsc{f} & \textit{chūɁ-}\textsc{dem} & \\
\lspbottomrule
\end{tabularx}
\end{table}

\tabref{tab:campbell:2} presents the dependent pronouns of the language, which also display the clusivity distinction, as well as additional third person singular distinctions (Non-specific and Respectful). As mentioned earlier, the language has only these two parallel sets of pronouns, independent and dependent, which can serve the function of any grammatical relation in the language. The syntactic functions are expressed not by different forms for Subject, Object, etc., but solely by position (or host), along with semantic and contextual factors.


%%please move \begin{table} just above \begin{tabular
\begin{table}
\caption{Zenzontepec Chatino dependent pronouns}
\label{tab:campbell:2}


\begin{tabularx}{0.37\textwidth}{p{0.2cm}p{0.6cm}XX}

\lsptoprule
&  & {\bfseries \textmd{\textsc{sg}}} & {\bfseries \textmd{\textsc{pl}}}\\
\hline
1 & \textsc{excl} & \textit{=ą̄Ɂ} & \textit{=ya}\\
& \textsc{incl} & {}-{}- & \textit{=na} {\textasciitilde} \textit{=ą}\\
2 &  & TONE & \textit{=wą}\\
3 & \textsc{nspc} & \textit{=ūɁ} & \textit{=ūɁ}\\
%\hhline%%replace by cmidrule{~~~-} & \textsc{m} & \textit{=yu} & \\
& \textsc{f} & \textit{=chūɁ} & \\
& \textsc{rsp} & \textit{=niɁ} & \\
%\hhline%%replace by cmidrule{---~}
\lspbottomrule
\end{tabularx}
\end{table}

With the preceding sketch of Zenzontepec Chatino basic syntax, grammatical relations, and referential coding devices now provided, the patterns of expressing coreference in the language are presented next.


\section{Reflexives and other coreference constructions}
\label{sec:campbell:4}
\subsection{Canonical reflexive constructions}
\label{sec:campbell:4.1}

As outlined in the preceding discussion, Zenzontepec Chatino does not have distinct pronouns for different syntactic functions: the same set of pronouns (dependent pronouns and their corresponding independent pronouns) serves all syntactic functions. Coreference within a clause is expressed by simply using the same pronoun (or another referential coding device for the same referent) in the appropriate positions for the multiple syntactic functions with shared reference. For example, a “canonical reflexive” construction, in which the Subject is coreferential with the Direct Object \citep[268]{Kulikov2013}, contains the coreferential coding devices in the Subject and D.O. positions and is otherwise formally equivalent to a canonical transitive clause without Subject and D.O. coreference. That is, there is no reflexive marker. The examples in \REF{ex:campbell:22} illustrate canonical reflexive expressions for first person Inclusive, third person Masculine, and second person singular referents.


%\setcounter{itemize}{21}
\ea\label{ex:campbell:22}
{Canonical reflexives} \\
\ea\label{ex:campbell:22a}
{   First person inclusive}\\
\gll Ntē-naʔa+tiʔi\textbf{=na}  j\textbf{{}-nā}.\\
     \textsc{prog}{}-see+pain=\textsc{1incl}  \textsc{obj}{}-\textsc{1incl}\\
\glt ‘We are punishing ourselves.’ [ntelinto itza7 17:22]
\ex\label{ex:campbell:22b}
{   Third person masculine (singular)}\\
\gll Nte-ʔne+kaya\textbf{=yu}  j\textbf{{}-yū}.\\
     \textsc{prog}{}-do+coward=\textsc{3sg}.\textsc{m}  \textsc{obj}{}-\textsc{3sg}.\textsc{m}\\
\glt ‘He is making himself a coward.’ [ntelinto itza7 15:00]

\ex\label{ex:campbell:22c}
{   Second person singular}\\
\gll T\textbf{ō}t\textbf{ī}ʔ  jiʔ\textbf{į́}.\\
     \textsc{pot}.take.care.of.\textsc{2sg}  \textsc{obj}.\textsc{2sg}\\
\glt ‘Take care of yourself!’ [muchacha ixtayutla 1:48]
\z
\z


The example in \REF{ex:campbell:23} illustrates a canonical reflexive expression coordinated with an intransitive clause with coreferential Subject. The referent is encoded by the Masculine (singular) dependent pronoun as Subject and D.O. of the transitive clause (A and O roles) as well as Subject of the intransitive clause (S role).\\

%\setcounter{itemize}{1}
\ea\label{ex:campbell:23}
{Canonical reflexive and coreferential Subject in coordinate clause}\\
\gll Lēʔ.nu  nka-jnyā\textbf{=yu}  j\textbf{{}-yū}  lēʔ  nchaa\textbf{=yu}.\\
     then  \textsc{pfv}{}-make=\textsc{3sg}.\textsc{m}  \textsc{obj}{}-\textsc{3sg}.\textsc{m}  then  \textsc{prog}.go=\textsc{3sg}.\textsc{m}\\
\glt ‘Then he made himself (dressed himself up fancy), and he went.’ [un rico 4:02]
\z


Reflexive expressions like \REF{ex:campbell:22b} and \REF{ex:campbell:23} that have coreferential third person referents (Masculine in these cases) may raise the question of how reflexive expressions would be disambiguated from similar transitive expressions with non-coreferential participants of the same type. While discourse context is usually sufficient for the intended meaning to be understood, if there is potential referential ambiguity, speakers can employ an independent demonstrative pronoun, for emphasis, thereby cuing the non-coreference \REF{ex:campbell:2}. As an alternative to signaling non-coreference this way, another disambiguation strategy is to reinforce coreference by using intensifiers (see \sectref{sec:campbell:5.3}).


\ea\label{ex:campbell:24}
{Non-coreferential Subject and Direct Object}\\
\gll y-akwiʔ=yu  j-nuwę̄ʔ\\
     \textsc{pfv}{}-speak=\textsc{3sg}.\textsc{m}  \textsc{obj-3.ana}\\
\glt ‘he\textsubscript{i} spoke to him\textsubscript{j} (that less topical aforementioned one)’ [ku7wi lo jo7o 9:59]\\
*‘he spoke to himself’
\z


The examples in \REF{ex:campbell:25} display reflexive expressions involving complex verbal predications. The first can be analyzed either as a single complex verbal predicate meaning ‘make one(self) Spanish’ (i.e. non-Indigenous) or as a transitive verb with something like a resultative secondary predicate meaning ‘Spanish’. The second example contains a verbal lexeme meaning ‘cure’ or ‘heal’ consisting of a light verb ‘do’ and the non-compounded nominal element \textit{kʷítī} ‘medicine’. The second example can also be analyzed as an indirect reflexive (see \sectref{sec:campbell:4.2.1}).


\ea\label{ex:campbell:25}
{Reflexive complex verbal predications}\\
\ea\label{ex:campbell:25a}
{  Resultative-like}\\
\gll Tatīyá  tī  úʔw\={ą}  tsáʔ.jlyā  nka-jnyā\textbf{=\={\k{u}}ʔ}  j\textbf{{}-\'{\k{u}}ʔ}.\\
     all  \textsc{tplz}  \textsc{3pl.dist}  Spanish  \textsc{pfv.caus-}move=3\textsc{pl}  \textsc{obj-3pl}\\
\glt ‘All of them turned (made themselves) Spanish.’ [historia1 34:14]

\ex\label{ex:campbell:25b}
{  Complex verbal lexeme}\\
\gll nu.jā  ʔne=kāʔá\textbf{=ą}  kʷítī  j\textbf{{}-nā}\\
     but  \textsc{hab.}do=also=1\textsc{incl}  medicine  \textsc{obj/dat-1incl}\\
\glt ‘but we also cure ourselves’ [historia1 21:33]
\z
\z

In all of the canonical reflexive examples presented so far, the coreferential arguments are coded with the same type of device: dependent pronouns. While this is the most common structure found in reflexive expressions in discourse, combinations of other types of referential device are also possible. For example, in the first clause in \REF{ex:campbell:26}, the referent is encoded with a topicalized independent pronoun in Subject function and anaphoric zero in Direct Object function.\footnote{The second clause is similar, with topicalized independent pronoun as Subject and anaphoric zero as inalienable possessor.}


\ea\label{ex:campbell:26}
{Reflexive with independent pronoun and anaphoric zero}\\
\gll Nkʷ-i-jnya+kíʔyū  tī  \textbf{nuwęʔ}  jį̄  \\
     \textsc{pfv-iter-}make+man  \textsc{tplz}  \textsc{3ana}  \textsc{obj}\textbf{\textsc{(.3)}}\\
\glt ‘He made himself manly (put on fancy manly clothes),’

\gll nkʷ-i-tyúkʷa+kiyaʔ  tī  \textbf{nuwęʔ}  sapatū  tsoʔō.\\
     \textsc{pfv-iter-}put+foot  \textsc{tplz}  \textsc{3ana}  shoe  good\textbf{(.3)} \\
\glt ‘and he put on his good shoes.’ [cuento DSF 6:59]
\z


In performative ritual speech, notions that would typically be expressed in an intransitive clause, like the prohibitive in \REF{ex:campbell:27}, can be creatively cast in a causative reflexive expression in order to emphasize the agency and responsibility of the referent, as illustrated twice in the passage in \REF{ex:campbell:28}.


\ea\label{ex:campbell:27}
{Intransitive prohibitive}\\
\gll Ná  kutsę\textbf{=wą}.\\
     \textsc{neg}  \textsc{pot-}be.afraid=\textsc{2pl}\\
\glt ‘Don’t be afraid!’ [kela ke kwiten7 3:10]
\z

%original no abc
\ea\label{ex:campbell:28}
{Causative prohibitives expressing canonical reflexives}\\
\ea
\gll K-aka\textbf{=wą}  chujlyāʔ.jyná.\\
     \textsc{pot-}be=\textsc{2pl}  authority\\
\glt ‘You all will be community authorities.’


\ex
\gll Ná   k-e+k-utsę\textbf{=wą}   jiʔį̄\textbf{=wą}.\\
     \textsc{neg}  \textsc{pot}{}-\textsc{caus}{}-\textsc{pot}{}-be.afraid=\textsc{2pl}  \textsc{obj}=\textsc{2pl}\\
\glt ‘Don’t be afraid!’ (lit. don’t frighten yourselves!).


\ex
\gll Ná  k-e+k-ulaʔ\textbf{=wą}  jiʔį̄\textbf{=wą}.\\
     \textsc{neg}  \textsc{pot}{}-\textsc{caus}{}-\textsc{pot}{}-be.cold=\textsc{2pl}  \textsc{obj}=\textsc{2pl}\\
\glt ‘Don’t be threatened!’ (lit. don’t make yourselves cold!).’ [ntelinto itza7 3:10]
\z
\z


Some verbs whose semantics are inherently reflexive, such as ‘bathe oneself’ and ‘dress oneself’ are not expressed as canonical reflexives and can be considered to be merely lexically reflexive, as illustrated in \REF{ex:campbell:29}. Note that the NP ‘his old clothes’ is coded as an adjunct Locative NP and not a Direct Object.


\ea\label{ex:campbell:29}
{Lexical reflexives}\\
\gll \textbf{Y-ata=yu}  tsoʔō  lēʔ  \textbf{nkʷ-i-tyuʔu=yu}  sateʔ  la-wíī=yu.\\
     \textsc{pfv-}bathe=\textsc{3sg.m}  well  then  \textsc{pfv-iter-}be.inside=\textsc{3sg.m}  clothes  \textsc{adjz-}get.cleaned=3\textsc{sg.m}\\
\glt ‘He bathed (himself) well and then got (himself) dressed in his clean clothes.’ [santaru tikela 6:50]
\z

\subsection{Other coreference constructions}
\label{sec:campbell:4.2}

The preceding discussion focused on canonical reflexive constructions in which there is coreference between the Subject and the Direct Object within a clause (and some cases of coreference across coordinated or sequential clauses). It was shown that Zenzontepec Chatino has no construction particular to reflexives that mark them as such. Instead, the basic transitive construction is used, with the coreferential coding devices occurring in their typical positions. The same is true for other types of coreference beyond direct reflexives. The following sections discuss coreference in various non-direct reflexive constructions (\sectref{sec:campbell:4.2.1}) and coreference across matrix and embedded clauses (\sectref{sec:campbell:4.2.2}).


\subsubsection{Non-direct reflexive expressions}
\label{sec:campbell:4.2.1}
The example in \REF{ex:campbell:30} illustrates something like an indirect reflexive \citep[74]{Kemmer1993} in which both the Subject and a recipient-like participant are coreferential.\footnote{An alternative analysis of this example is as a monotransitive clause with transitive verb of the schema [-\textbf{speak}] SUBJ [\textbf{work}] OBJ.}


%\setcounter{itemize}{29}
\ea\label{ex:campbell:30}
{Indirect reflexive} \\
 Nchakʷeʔ\={ę}ʔ jnyá jy\'{ą}ʔ.\\
    \gll  Nch-akʷiʔ\textbf{=ąʔ}  jnyá  jiʔį\textbf{=\={ą}ʔ}.\\
     \textsc{prog}{}-speak=1\textsc{sg}  work  \textsc{dat=1sg}  \\
\glt ‘I am directing myself.’ [vaquero 5:09]
\z


The example in \REF{ex:campbell:31} shows an alternation between a direct reflexive, with a 2\textsc{sg} Subject and D.O., and a parallel expression with coreference between the same Subject and an Oblique \textit{lóʔō} comitative. As usual, all of the 2\textsc{sg} pronominal inflection is conveyed by tonal alternation.


%original had no abc
\ea\label{ex:campbell:31}
{Subject-comitative coreference}\\
\ea
\gll Nk\textbf{ā}{}-\textbf{lī}nt\textbf{ō}  jiʔ\textbf{į́}  nakʷę.\\
     \textsc{pfv.caus}{}-go.to.waste.\textsc{2sg}  \textsc{obj}.\textsc{2sg}  say.3\\
\glt ‘You wasted yourself, he said.’

\ex
\gll Nk\textbf{ā}{}-l\textbf{ī}nt\textbf{ō}  jy=\'{ą}ʔ  l\textbf{ō}ʔ\textbf{ō}.\\
     \textsc{pfv.caus}{}-go.to.waste.\textbf{\textsc{2sg}}  \textsc{obj}=\textsc{1sg}  with.\textbf{\textsc{2sg}}\\
\glt ‘You wasted me with you.’ [ku7wi lojo7o 17:17]
\z
\z


The example in \REF{ex:campbell:32} shows coreference of the Subject with an Alienable Possessor of the D.O. and Alienable Possessor of an instrumental Oblique \textit{lóʔō} in the same clause. The example in \REF{ex:campbell:33} shows coreference of the Subject with Inalienable Possessors of two coordinated comitative Obliques.


\ea\label{ex:campbell:32}
{Coreference of Subject and Alienable Possessors of D.O. and Oblique}\\
\gll Lō  laa  n-tāá  tī  \textbf{nāáʔ}    \\
     like.so  be  \textsc{hab}{}-give  \textsc{tplz}  \textsc{1sg}    \\
\gll j-nā  ītsáʔ  ntē  jy\textbf{{}-\'{ą}ʔ}\\
     \textsc{obj}{}-\textsc{def}  word  \textsc{prox}  \textsc{gen}{}-\textsc{1sg}\\
\gll lóʔō  nu.tii  jy\textbf{{}-\'{ą}ʔ}.\\
     \textsc{with}  thought  \textsc{gen}{}-\textsc{1sg}\\
\glt ‘Like so, I tell this story of mine with my thoughts.’ [familia 10:46]
\z

\ea\label{ex:campbell:33}
{Coreference of Subject and Inalienable Possessors of Obliques}\\
\gll Ta  nkā-naʔa\textbf{=ūʔ}  titsę  lóʔō  jniʔ\textbf{=ūʔ}  lóʔō  lyoʔo\textbf{=ūʔ}.\\
     already  \textsc{pfv}{}-see=\textsc{3pl}  badly  \textsc{with}  offspring=\textsc{3pl}  \textsc{with}  spouse=\textsc{3pl}  \\
\glt ‘They have already seen the bad with their children and with their spouses.’ [familia 12:26]
\z


The next set of examples illustrate coreference of an Inalienable Possessor of the Subject with a D.O. \REF{ex:campbell:34} and with an Indirect Object \REF{ex:campbell:35}.


\ea\label{ex:campbell:34}
{Coreference of Inalienable Possessor and D.O.}\\
\gll Laaʔ  nkā-naʔa+tīkáʔā  tī  nyáʔa\textbf{=yu}  j\textbf{{}-yū}.\\
     like.so  \textsc{pfv}{}-see+cherished  \textsc{tplz}  mother=3\textsc{sg.m}  \textsc{obj-3sg.m}\\
\glt \textsc{‘}His mother took care of him like that.’ [santa maria2 14:41]
\z

\ea\label{ex:campbell:35}
{Coreference of Inalienable Possessor and I.O.}\\
\gll Ntyūsé  n-tyōtíʔ  nakʷę  lyoʔo\textbf{=yu}  j\textbf{{}-yū}.\\
     god  \textsc{hab}{}-know(.3)  say  spouse=\textsc{3sg}.\textsc{m}  \textsc{dat}{}-\textsc{3sg}.\textsc{m}\\
\glt “God knows’, his wife said to him.’ [choo kwe7en 0:50]
\z

\subsubsection{Coreference in embedded contexts}
\label{sec:campbell:4.2.2}

Similar to intra-clausal coreference expressions, coreference between main and embedded clauses is achieved by simply using the appropriate referential coding device in the appropriate syntactic positions in each clause. For example, in \REF{ex:campbell:36} the Subject of the matrix clause is coreferential with the Subject of the purpose adverbial clause in the first line of the passage. In the second line of the passage, the Alienable Possessor of the questioned Subject is coreferential with the Subject of the following relative clause.


%\setcounter{itemize}{35}
%original had no a, b
\ea\label{ex:campbell:36}
\ea
{Coreference in embedded adverbial clause and relative clause}\\
\gll Tyūkʷá\textbf{=ą}  [k-ako\textbf{=ą}  chaja  lóʔō  tī  nūw\'{ą}].\\
     \textsc{pot.}sit=1\textsc{incl}  \textsc{pot}{}-eat=\textsc{1incl}  tortilla  \textsc{with}  \textsc{tplz}  \textsc{3dist}\\
\glt ‘Let’s sit down to eat with him.’

\ex
\gll Tukʷi  tāká  jiʔį̄\textbf{=ą}  [k-ako\textbf{=ą}  lóʔō].\\
     what  exist  \textsc{gen=1incl}  \textsc{pot}{}-eat=\textsc{1incl}  \textsc{with(.3)}\\
\glt ‘What do we have that we can eat with him?’ [cuento DSF 7:27]
\z
\z


Example \REF{ex:campbell:37} contains a relative construction in which the head (‘good medicine’) is the relative clause Subject and matrix clause D.O. The Alienable Possessor of the Alienable Possessor (a possession chain) of the relative clause Subject is coreferential with the matrix clause Subject.


%\setcounter{itemize}{1}
\ea\label{ex:campbell:37}
{Coreference of Subject with Alienable Possessor in a relative clause}\\
\gll Nte-lāstí\textbf{=na}  kʷítī  tsoʔō  [nu  nk-yuʔu   jį̄  nyat\={ę}  kusūʔ  jiʔį̄\textbf{=na}.\\
     \textsc{prog}{}-abandon=1\textsc{incl}  medicine  good  \textsc{sbd}  \textsc{pfv}{}-be.inside  \textsc{gen}  person  elder  \textsc{gen=}1\textsc{incl}\\
\glt ‘We are abandoning the good medicine that our ancestors had.’ [familia 11:05]
\z


The Subject of the matrix clause in \REF{ex:campbell:38} is coreferential with the Inalienable Possessor of the Subject in the preposed Object complement clause.


\ea\label{ex:campbell:38}
{Coreference in matrix and complement clause}\\
\gll Titsę  laa  ntoo\textbf{=chúʔ}  nkʷ-ii\textbf{=chúʔ}.\\
     badly  be  face=\textsc{3sg.f}  \textsc{pfv}{}-feel=\textsc{3sg.f}\\
\glt ‘She felt that her face was very bad.’ [bruja barbona 4:23]
\z


A more complex example is illustrated in \REF{ex:campbell:39}. The matrix clause verb with Subject enclitic occurs in final position. The Object complement is a nominal predication construction. A light-headed relative clause is the nominal predicate and it is juxtaposed with a headless relative clause that functions as its Subject. The Subject of the matrix clause verb ‘want’ is coreferential with both the Subject of the relative clause in the nominal predicate and the beneficiary in the relative clause that is the Subject of the nominal predication.


\ea\label{ex:campbell:39}
{Coreference in multiple embeddings}\\
\gll [[Tatīyá  nu  k-aku\textbf{=ą̄ʔ}]  [nu  tyúʔu  jy\textbf{{}-\'{ą}ʔ}]]  nch-ātíʔ\textbf{=ą̄ʔ}.\\
     all  \textsc{sbd}  \textsc{pot}{}-eat=\textsc{1sg}  \textsc{sbd}  \textsc{pot}.go.out(.3)  \textsc{dat}{}-\textsc{1sg}  \textsc{prog}{}-want=\textsc{1sg}\\
\glt ‘I want \textit{what I harvest} to be \textit{all that I eat}.’ [kuna7a kusu7 5:44]
\z


The preceding examples illustrate that speakers of Zenzontepec Chatino use complex constructions in which multiple clausal embeddings may occur, and coreference is tracked through these structures the same way that coreference is expressed intra-clausally: coreferential coding devices occur in the relevant positions in the same way non-coreferential coding devices would.


\section{Middles, reciprocals, and intensifiers}
\label{sec:campbell:5}

Cross-linguistically, reflexive markers show tendencies to overlap with the markers for middle voice \citep{Kemmer1993}, reciprocal expressions (\citealt{Maslova2008}; \citealt{HeineMiyashita2008}), and what are referred to as intensifiers (\citealt{KoenigSiemund1999}). In Zenzontepec Chatino, none of these constructions share a marker with reflexive expressions because reflexives have no dedicated marker in the first place.


\subsection{Middle voice}
\label{sec:campbell:5.1}

The correlate of a middle voice construction in Zenzontepec Chatino at least shares with reflexive expressions the fact that it is an \textsc{uncoded} valency pattern (\citealt{HaspelmathHartmann2015}: 65), that is, there is no specific marker on the verb, or any marker at all, that indicates that the construction is a middle voice one. Like most of the valence alternations in the language \citep{Campbell2015}, it is highly unproductive, only occurring with a few verbs: the verbs of ingestion, like ‘eat’ \REF{ex:campbell:40} and ‘drink’ \REF{ex:campbell:41}.


%\setcounter{itemize}{39}
\ea\label{ex:campbell:40}
{Middle with ‘eat’}\\
\gll jā  nu.ntē  tsoʔō  \textbf{ntaku}\\
     \textsc{conj}  \textsc{3prox}  good  \textsc{hab-}eat(.3)\\
\glt ‘this (fruit) is tasty’ (lit. ‘this (fruit) eats well’) [familia 12:58]
\z

%\setcounter{itemize}{1}
\ea\label{ex:campbell:41}
{Middle with ‘drink’}\\
\ea
\gll Nt-u-nuʔu  jiʔį̄  na  lúkʷī=Vʔ.  \\
     \textsc{hab-caus-}get.ruined(.3)  \textsc{obj}  \textsc{def}  mezcal=\textsc{ana}  \\
\glt ‘It (water with no sweetness) ruins the mezcal.’

\ex
\gll Lēʔ  yoōʔ  \textbf{ntī-ʔyó}  tiʔī  chini.\\
     then  disgusting  \textsc{hab-}drink(.3)  flavor  smoke\\
\glt ‘Then it (the mezcal) drinks disgustingly with a smoke flavor.’ [lukwi proceso 6:18]
\z
\z

\subsection{The Reciprocal construction}
\label{sec:campbell:5.2}

Unlike reflexives, the Reciprocal construction in Zenzontepec Chatino has a distinct marker. In the Reciprocal construction, the form \textit{tyáʔā} ‘companion, relative’ is encliticized to the verb base, and there is no longer a grammatical Direct Object, that is, the clause becomes syntactically intransitive. A basic transitive clause with the verb ‘kill’ is shown in \REF{ex:campbell:42}, followed by a Reciprocal construction involving the same verb.


%\setcounter{itemize}{41}
\ea\label{ex:campbell:42}
{Reciprocal alternation}\\
\ea\label{ex:campbell:42a}
{  Basic transitive clause} \\
\gll Nt-ujwi  tī  kʷaa  j-nuw\={ę}ʔ  lóʔō  kētǫ.\\
     \textsc{hab-}kill  \textsc{tplz}  \textsc{1incl}  \textsc{obj-3ana}  \textsc{with}  rifle\\
\glt ‘We would kill those (macaws) with rifles.’ [animales desaparecidos 0:46]

\ex\label{ex:campbell:42b}
{  Reciprocal construction}\\
\gll Nku-tyejnā  nt-ujwi=\textbf{tyáʔā}  tī  \'{\k{u}}ʔwiʔ.\\
     \textsc{pfv-}begin  \textsc{hab-}kill=\textsc{recp}  \textsc{tplz}  \textsc{3pl.ana}\\
\glt ‘They started to kill each other.’ [maldicion 1:10]
\z
\z

The form \textit{tyáʔā} whence the Reciprocal marker has grammaticalized is an inalienably possessed noun, meaning ‘companion’, ‘family’,  or ‘sibling’, a cross-linguistically relatively common source for reciprocal markers (\citealt{HeineMiyashita2008}: 178). The example in \REF{ex:campbell:43} shows the form as a noun in a comitative noun phrase, and the example in \REF{ex:campbell:44} shows the noun in Direct Object function, a bridging context in which the clause can be interpreted either with disjoint reference of the D.O. and Subject or with reciprocal reference.


%\setcounter{itemize}{1}
\ea\label{ex:campbell:43}
{The form \textit{tyáʔā} ‘companion’ as a noun}\\
\gll Ná  kʷēyáʔ  xī  ntetāk\'{ą}ʔ=na  lóʔō  kʷitiʔ  lóʔō  \textbf{tyáʔā}=na.\\
     \textsc{neg}  measure  \textsc{sbd}  \textsc{prog-}suffer=\textsc{1incl}  \textsc{with}  brother  and  companion=1\textsc{incl}\\
\glt ‘What we are suffering with our brothers and companions is immeasurable. [ntelinto itza7 2:43]
\z

\ea\label{ex:campbell:44}
{Likely bridging context of grammaticalization of Reciprocal =\textit{tyáʔā}}\\
\gll Nkā-sāʔ\'{ą}=ya  jį̄  \textbf{tyáʔā}=ya.\\
     \textsc{pfv.caus}{}-be.attached=1\textsc{excl}  \textsc{obj}  companion=\textsc{1excl}\\
\glt ‘We take responsibility for our companions.’\\
‘We take responsibility for each other.’ [ntelinto itza7 22:01]
\z

\subsection{Intensifiers}
\label{sec:campbell:5.3}

Zenzontepec Chatino has two forms that may function as intensifiers. These are not part of the canonical reflexive construction of the language \sectref{sec:campbell:4.1}, but they may co-occur with or reinforce reflexives. The first is \textit{lákʷiʔ}, an adjective-like form that can mean ‘one’s \textit{own}’\REF{ex:campbell:45} or ‘the same ones’ \REF{ex:campbell:46}, as well as having an intensifier function in which a particular—either surprising or especially important—referent is indicated \REF{ex:campbell:47}.


%\setcounter{itemize}{44}
\ea\label{ex:campbell:45}
{\textit{lákʷiʔ} used to mean one’s ‘own’}\\
\gll Keję  tsǫʔ  \textbf{lákʷiʔ=yu}  nti-nījnyá=yu  nu  nt-una=yu  kitse.\\
     skin  back  own=\textsc{3sg.m}  \textsc{hab-}use=\textsc{3sg.m}  \textsc{sbd}  \textsc{hab-}twist=3\textsc{sg.m}  fiber\\
\glt ‘He would use \textbf{his} \textbf{own} leather when he would twist maguey fiber (into twine).’ [maclovio 2:09]
\z

%\setcounter{itemize}{1}
\ea\label{ex:campbell:46}
{\textit{lákʷiʔ} used to mean ‘the same ones’}\\
\gll \textbf{Lákʷiʔ=j\={\k{u}}ʔ}  ntyeʔę=j\={\k{u}}ʔ  nti-ka+kūʔwí=j\={\k{u}}ʔ.\\
     same=3\textsc{pl}  \textsc{hab}.be.located=3\textsc{pl}  \textsc{hab-}be+drunk=3\textsc{pl}\\
\glt ‘They are the same ones that are there and get drunk.’ [ntelinto itza7 12:04]
\z

\ea\label{ex:campbell:47}
{\textit{lákʷiʔ} used as an intensifier}\\
\gll wī  ntyōtíʔ  tī  \textbf{lákʷiʔ=ūʔ}  tula  nakʷę  lóʔō  x-ītsáʔ=\textbf{ūʔ}\\
     and  \textsc{hab}.know  \textsc{tplz}  \textsc{int=3pl}  what  say(.3)  \textsc{with}  \textsc{poss-}word=\textsc{3pl}\\
\glt ‘and \textbf{they} \textbf{themselves} know how to say (it) with their language’ [historia3 15:30]
\z


The passage in \REF{ex:campbell:48} illustrates the use of the intensifier \textit{lákʷiʔ} to provide contrast and reinforce the coreference of an otherwise canonical reflexive expression.


\ea\label{ex:campbell:48}
{\textit{lákʷiʔ} as an intensifier reinforcing a reflexive construction}\\
\gll Nyáʔa=yu  nkā-línto  j-yū\\
     mother=\textsc{3sg.m}  \textsc{pfv.caus-}go.to.waste(.3)  \textsc{obj-3sg.m}\\
\glt ‘So his mother killed him?...


\gll ʔa  nu  \textbf{lákʷiʔ=yu}  nkā-línto\textbf{=yu}  j\textbf{{}-yū}.\\
     \textsc{q}  \textsc{sbd}  \textsc{int=3sg}.\textsc{m}  \textsc{pfv}.\textsc{caus}{}-go.to.waste=3\textsc{sg.m}  \textsc{obj}{}-\textsc{3sg}.\textsc{m}\\
\glt …or \textbf{he} \textbf{himself} killed himself?’ [santa maria2 14:34]
\z


The other Zenzontepec Chatino intensifier is \textit{kʷiʔya} ‘alone’, which on its own may function as an adjective, as in the negated adjectival predication in \REF{ex:campbell:49} and the depictive secondary predicate in \REF{ex:campbell:50}. It may also function as an adverb encliticized to a verb, either as a manner adverb \REF{ex:campbell:51} or an intensifier reinforcing a reflexive expression \REF{ex:campbell:52}.


\ea\label{ex:campbell:49}
{\textit{kʷiʔya} as adjectival predicate}\\
\gll Nāxíʔi  \textbf{kʷiʔya}=ūʔ  nka-ʔne=ūʔ  jį̄  nkā.\\
     is.not  alone=\textsc{3pl}  \textsc{pfv-}do=3\textsc{pl}  \textsc{obj(.3)}  \textsc{pst}\\
\glt ‘It was not alone that they did it (the work) before.’ [antes aparatos 11:05]
\z

\ea\label{ex:campbell:50}
{Unbound \textit{kʷiʔya} reinforcing a reflexive}\\
\gll \textbf{Kʷiʔya=}ri=ą  nte-líntoo=ą  j-nā.\\
     alone=only=\textsc{1incl}  \textsc{prog}{}-go.to.waste=\textsc{1incl}  \textsc{obj}{}-\textsc{1incl}\\
\glt ‘Alone one is wasting oneself away.’ [historia3 22:19]
\z

\ea\label{ex:campbell:51}
{\textit{kʷiʔya} as adverbial enclitic on the verb}\\
\gll Nka-ʔne+tsáʔ\={ą}=\textbf{kʷiʔya}=ri  tī  úʔwiʔ  lō.laa  nte-ʔne  tselā.yuu.\\
     \textsc{pfv-}do+study=alone=only  \textsc{tplz}  \textsc{3pl.ana}  how  \textsc{prog-}do  world\\
\glt ‘They just studied alone what nature was doing.’ [luna siembra 2:56]
\z

\ea\label{ex:campbell:52}
{Enclitic \textit{kʷiʔya} reinforcing a reflexive}\\
\gll Nte-ʔne+lóʔō=\textbf{kʷiʔya}=ri=ą  j-nā.\\
     \textsc{prog-}do+\textsc{with=int}=only=\textsc{1incl}  \textsc{obj}{}-\textsc{1incl}\\
\glt ‘We ourselves are making ourselves suffer.’ [familia 0:51]
\z


As \citet[68]{KoenigSiemund1999} point out, “intensifiers may be completely identical to reflexives, they may provide the source for the development of reflexives, and they may combine with reflexives” in different languages. In Zenzontepec Chatino, the intensifiers are not part of the expression of canonical reflexives, nor are they the source of any reflexive marker. They may, however, reinforce coreference, but that is only one of a range of functions displayed by each of the two intensifiers.


\section{Conclusion}
\label{sec:campbell:6}

Zenzontepec Chatino presents a typologically interesting case for the cross-linguistic study of reflexives and coreference. While most languages display a reflexive construction in which the “co-referential direct object is not repeated in the sentence but is either (i) replaced by the reflexive pronoun […] or (ii) removed from the original structure” \citep[268]{Kulikov2013}, the Zenzontepec Chatino correlate of a canonical reflexive expression employs neither of these strategies. In fact, Zenzontepec Chatino has no reflexive construction that meets the cross-linguistic comparative concept as defined by Haspelmath (this volume) in which at least two clausal participants are coreferential, and in which some grammatical marker signals that there is such coreference. In lieu of a specific reflexive construction or reflexive marker, Zenzontepec Chatino expresses reflexives and other types of intra- and inter-clausal coreference by means of the language’s standard referential coding devices (NPs, independent pronouns, dependent pronouns, anaphoric zero) in the relevant grammatical relations that share the coreference. This uncommon strategy is also reported for a couple of varieties of the related Zapotec languages (\citealt{AntonioRamos2015}: 53; \citealt{Lee2003}: 88). Where referential ambiguity could arise, speakers can use a demonstrative emphatic pronoun for signaling disjoint reference or an intensifier for reinforcing coreference.



While in many languages middle voice, reciprocal constructions, and intensifiers overlap with or share features with reflexive constructions, this is not the case in Zenzontepec Chatino. The language only sparsely uses a likewise uncoded middle voice alternation, and it presents a distinct and specialized Reciprocal construction with the marker \textit{tyáʔā} ‘companion’, which has cognate structures and markers in related Chatino \citep[71]{Rasch2002} and Zapotec (\citealt{Lee1999}: 91; \citealt{Munro2015}) languages, as well as more distantly-related Mixtec varieties (\citealt{Shields1988}: 344; \citealt{Zylstra1991}: 47). Zenzontepec Chatino has two forms that function as intensifiers, among other functions, neither of which has grammaticalized into any reflexive marker, but which may be used to reinforce unexpected or important coreference relations in otherwise canonical reflexive expressions.


\section*{Acknowledgements}

Most sincere thanks to Tranquilino Cavero Ramírez and Flor Cruz Ortiz for the collaboration in recording, transcribing, and translating the corpus of texts used for this study, some of which was funded in part by grants MDP0153 and IGS0080 from the Hans Rausing Endangered Language Programme (ELDP) to the University of Texas at Austin. Other work for this chapter was supported in part by a UC Santa Barbara Academic Senate Faculty Research Grant. Special thanks to the editors Katarzyna Janic, Nicoletta Puddu, and Martin Haspelmath for the opportunity to participate in this volume. Helpful comments by the editors and one anonymous reviewer have improved the chapter, and any remaining imperfections are solely the author’s responsibility.


\section*{Abbreviations}

\begin{tabularx}{.45\textwidth}{lQ}
\textsc{adjz} &  adjectivizer\\
\textsc{ana}  &  anaphoric demonstrative\\
\textsc{caus}  &  causative\\
\textsc{conj}  &  conjunction\\
\textsc{dat}  &  dative\\
\textsc{def}  &  definite article\\
\textsc{dist}  &  distal demonstrative\\
\textsc{excl}  &  exclusive (1\textsc{pl)})\\
\textsc{f}  &  feminine\\
\textsc{gen}  &  genitive\\
\textsc{hab}  &  habitual aspect\\
\textsc{incl} &  inclusive (\textsc{1pl})\\
\textsc{int}  &  intensifier\\
\textsc{iter}  &  iterative\\
\textsc{itrn}  &  intransitivizer\\
\textsc{m}  &  masculine\\
\textsc{neg}  &  negative\\
\textsc{neut}  &  neutral gender\\
\textsc{obj}  &  direct object\\
\textsc{pfv}  &  perfective aspect\\
\textsc{pl}  &  plural\\
\textsc{prog}  &  progressive aspect\\
\textsc{prox}  &  proximal demonstrative\\
\textsc{pot}  &  potential mood\\
\textsc{q}  &  question particle (yes/no)\\
\textsc{rsp}  &  respectful (pronoun)\\
\textsc{sbd}  &  subordinator\\
\textsc{sg}  &  singular\\
\textsc{tplz}  &  topicalizer\\
\textsc{trn}  &  transitivizer\\
\textsc{with} &  oblique (comitative or instrument)\\
\end{tabularx}


\begin{verbatim}%%move bib entries to  localbibliography.bib


\end{verbatim}
\sloppy\printbibliography[heading=subbibliography,notkeyword=this]
\end{document}
