\addchap{\lsPrefaceTitle}
 
This book starts out with the idea of modeling human language varieties as information-theoretic variables, and proceeds to define a conditional independence relation between sets of them. The conditional independence relationships are then used to infer two types of directed networks over language varieties which have all the properties of causal graphs, as defined by \cite{pearl2009}. Such a graph can be interpreted as a parsimonious explanation of how the lexicon of the investigated varieties was shaped by inheritance and contact. This type of directed phylogenetic network is more general than the types which were previously discussed in the literature on tractable phylogenetic network inference, as covered e.g. in the book-length overview by \cite{morrison2011}.

After a summary of the necessary background in historical linguistics (Chapter 2) and causal inference (Chapter 3), Chapter 4 describes the many preparatory steps which were necessary to arrive at good test data for these methods. Since none of the existing lexical databases has all the characteristics necessary for automatic computation of lexical overlaps across language family boundaries, a new deep-coverage lexical database of Northern Eurasia was compiled as part of the project which gave rise to this book. This NorthEuraLex database contains data for an unusually large list of more than a thousand concepts, and is the first database to cover the languages of a large continuous geographic area with more than 20 language families in a unified phonetic format. For four interesting areas of language contact (the Baltic Sea, the Uralic languages, Siberia, and the Caucasus), the literature on language contacts is surveyed at the end of this chapter to build a gold-standard of contact events which we would expect an automated method to be able to extract from the database.

Since network inference builds on a similarity measure which is based on measuring lexical overlap, the word forms need to be grouped into sets of etymologically related words in a preparatory step. While this clustering into ``cognate'' sets could be done manually by experts in the linguistic history of the respective region, recent developments in computational historical linguistics have made it possible to infer approximate cognate judgments by automated means. These approaches still misclassify many non-cognate pairs as cognates and vice versa, but the number of errors is low enough for much of the relevant signal to persist on the language level, which makes it possible to apply statistical methods. By introducing a new phonetic form alignment method called Information-Weighted Sequence Alignment (IWSA), this book shows that established methods for automated cognacy detection are refinable in such a way that they work on phonetically transcribed dictionary forms, making it unnecessary to manually reduce all words to their stems before running cognate detection on them. The method shows its strength especially in the unusual scenario of cross-family cognate detection, where it does not pay off to assume cognacy of similar forms as much as on the single-family datasets commonly used in the literature on automated cognate detection.

The central contribution of this book, laid out in Chapter 6, is the derivation of a consistent information measure for sets of language varieties which is based on cognate set overlaps. The resulting measure of conditional mutual information quantifies a notion of lexical flow, where the lexical material needs to be distributed via paths connecting varieties in order to explain the overlap in their lexicons. Standard causal inference algorithms can then be applied to conditional independence constraints arising from vanishing mutual information. The result is a network which is minimal in the number of lateral connections while still being able to explain the cognate overlap patterns in the observed varieties.

In Phylogenetic Lexical Flow Inference (PLFI), the simpler of the two algorithms introduced by this book, proto-languages are modeled explicitly as sources of overlaps in the inherited lexicon of related varieties. This requires the use of a guide tree defining the proto-languages, on which existing ancestral state reconstruction methods from bioinformatics are used to reconstruct the presence or absence of each cognate set at each node. The resulting flow network adds directed lateral links to the guide tree, each of which represents some lexical material that was inferred to be transmitted from the donor to the recipient language by borrowing. The framework is general enough to infer directional contact among proto-languages, which means that the output structures are fully general evolutionary networks.

In contrast, Contact Lexical Flow Inference (CLFI), which is described and evaluated in Chapter 7, does not explicitly model the proto-languages, but instead conceptualizes them as unobserved sources of shared lexical material. The contact flow network only features the varieties included in the data, and different arrow types distinguish directional contact from common inheritance. From the statistical point of view, the proto-languages become latent confounders which cause spurious dependencies between the observable language variables. The presence of such hidden common causes is not necessarily a problem for causal inference, since the most advanced algorithms can in principle distinguish dependence relations that are due to common causes from those that are a product of direct causal relationships.

For both algorithms, the discrete and unreliable nature of the cognate data makes it necessary to develop alternative methods for the different stages of causal inference, with the purpose of increasing robustness against erroneous cognacy judgments. This is achieved by a combination of re-analyzing the intuition behind the PC algorithm for causal inference in order to quantify and balance conflicting signals arising from different three-variable configurations, and putting further consistency restrictions on edge deletion decisions via a connectedness criterion on the level of individual cognate sets.

Both methods are evaluated on the lexical database as well as large amounts of simulated cognacy data. Chapter 5 describes the model used to generate the simulated data, which is based on a simple evolutionary process that mimics language change by lexical replacement and borrowing on the level of individual words. This model is shown to produce realistic cognate data which will also be of use in validating other methods for inferring evolutionary networks from cognacy-encoded language data, whether expert-annotated or based on automated cognate detection.
