\chapter{Topic relations in ZAI}\label{topicchapter}

The chapter discusses the linguistic resources available to ZAI speakers for expressing \isi{topic} relations. This discussion of \isi{topic} relations will set the stage for the analysis of a very commonly used topic-marking strategy involving the \isi{discourse particle} \textsc{la}.

In this discussion, I follow \citet{lambrecht1994} and use the term \textbf{topic} or \textbf{\isi{topic} referent} to describe the referent or entity which the proposition is about. As such, the \isi{topic} or \isi{topic} referent is the referent or entity which bears a \isi{topic} relation to the proposition. It is not to be confused with ``old'' information, which refers to the cognitive status of a referent. From this perspective, information which performs the role of \isi{topic} in a given proposition may have a cognitive status that is either ``old'' or ``new''. On the \isi{givenness hierarchy} discussed in \sectref{nomforms}, \isi{topic} referents must be identifiable in the mind of the speaker and hearer, and continuous topics are usually also activated and familiar, but this is not a pre-requisite for topic-hood. Instead, it is the relation that the \isi{topic} referent or entity bears to the rest of the proposition that is significant. By contrast, the terms \textbf{\isi{topic} constituent} or \textbf{\isi{topic} NP} refer to the corresponding linguistic expression and not the referent or entity to which that expression refers. 

Again, as was mentioned in the previous chapter, it is important to bear in mind that stress and pauses play a critical structural function in ZAI \isi{prosody} (see \sectref{prosody}). Pitch accents, however, do not play a role in the marking of \isi{topic} or \isi{focus} relations in ZAI.\footnote{We may keep in mind, as \citet[15]{crocco2009} states, that ``the actual realization of the prosodic marking of topicality may vary according to the different positions occupied by the \isi{topic} with respect to the prosodic nucleus of the utterance."} 

 

\section{Topic constructions}\label{topicconstructionssection}


In Chapter \ref{paschapter} we saw that the cognitive status of discourse referents has observable and direct correlates in ZAI grammar in terms of nominal forms and the grammatical roles -- A, S, or O -- in which they tend to occur. The cognitive status of referents correlates highly with the pragmatic acceptability of sentences in other ways as well. For example, because insufficiently accessible \isi{topic} referents are more difficult for hearers to interpret, \isi{topic} referents tend to have a certain degree of pragmatic \isi{accessibility}. \citet[165]{lambrecht1994} expresses this correlation in terms of a ``Topic Acceptability Scale'' by which more acceptable topics are coded by linguistic expressions that are higher on a cognitive status scale, such as the Givenness Hierarchy in \tabref{givennesshierarchy}, and less acceptable topics are coded by expressions which are lower on this scale. For ZAI, therefore, we would predict that the most acceptable topics would be coded by subject clitics, while the least acceptable topics would be coded by indefinite NPs or bare nouns.

In addition, we will see that there is also a correlation between the \isi{information structure} of certain types of constructions and the cognitive status of the \isi{topic} referents involved. In particular, in \isi{focus} or activated referents do not occur in presentational or event-reporting constructions, and type-identifiable referents do not occur in ``\isi{marked topic}'' or detachment constructions involving the particle \textsc{la}. In other words, NPs in presentational constructions are never pronominal forms and NPs in detached, \textsc{la}-marked constructions are never indefinite. 



\subsection{Presentational constructions}\label{presentationalsection}

Cross-linguistically, statements about the weather tend to be thetic constructions.\footnote{Constructions such as these are also labeled ``\isi{sentence focus}''; see \sectref{sfsection}. They are sometimes also referred to as `out-of-the-blue' sentences.} An example is presented in (\ref{thetic}):
\ea\label{thetic} 
\glll cayaba nisaguie \\
ca-yaba nisa-guie \\
\textsc{prog}-fall water-stone  \\
\glt `Rain falls.' 
\z
The construction is verb-initial and the lexical, subject NP is a bare noun. The subject is not topical and the \isi{focus domain} is the entire sentence.

The following example from a Pear Story narrative shows an event-reporting construction with a presentational function:
\ea\label{thetic2} 
\glll rihuinni t\'{i} r\'{i}gola  \\
ri-huinni\textsuperscript{LH} ti ri\textsuperscript{H}gola  \\
\textsc{hab}-appear one man  \\
\glt `A man appears.' 
\z
The construction, used to introduce a new participant into a discourse, is also verb-initial and here the subject is a lexical, indefinite NP. Again, there is no topical subject, the \isi{focus domain} is the entire sentence, and it lacks a presupposed \isi{topic}. In other words, it is thetic, i.e. the whole sentence is asserted. 

In the Pear Story corpus, new referents are always introduced as lexical NPs, most often in the O role, followed by the S role, and much more rarely in the A role (see \tabref{newreferents}). When we take into account \isi{animacy}, however, new referents are introduced at a higher rate in the S role than the O role (see \tabref{newhumanreferents}). That is, the majority of human referents in the Pear Story corpus are introduced using presentational constructions of the type in (\ref{thetic2}). New referents introduced in the O role are introduced using \isi{topic-comment} sentences, which I discuss in \sectref{topiccommentsection}.


\subsection{Topic-comment}\label{topiccommentsection}

In the following example from a Pear Story narrative, the subject in line 2 is the \isi{topic}, and the predicate is a comment or assertion about the subject-\isi{topic}. 
\ea\label{TC} (\textit{Pear Stories}, M: l.4)
\begin{itemize}
\item[01]
\glll m\'{a} bihuinni t\'{i} se\~{n}\v{o}r \\
ma'\textsuperscript{H} bi-huinni\textsuperscript{LH} ti se\~{n}o\textsuperscript{LH}r \\
already \textsc{compl}-appear one man \\
\glt `A man appeared.' 


\item[02]
\glll cuchuugube p\v{e}ra \\
cu-chuugu'=be\textsuperscript{LH} pe\textsuperscript{LH}ra \\
\textsc{prog}-cut=\textsc{3sg} pear \\
\glt `He (was) cutting pears.' 

\end{itemize}
\z
The narrator uses a presentational clause in line 1 to introduce the man and, in the second line, uses a \isi{topic-comment construction} to predicate a property (i.e. that he was cutting pears) about that man, an already established referent. The subject-\isi{topic} in line 2 appears as an enclitic on the verb. 

The subject NP, when topical, appears as an enclitic on the verb. In rare cases, such as in a transitive clause with a topical object, the subject NP may occur as a lexical NP. Invariably, however, like event-reporting constructions, \isi{topic-comment} constructions in ZAI are always verb-initial (except in cases of \isi{topicalization} or `marked' topics). Therefore, because the \isi{verb-initial construction} is compatible with other pragmatic construals, such as event-reporting or identificational constructions, we can consider the verb-initial \isi{topic-comment construction} the unmarked type. I discuss identificational constructions next.


\subsection{Identificational constructions}\label{identificationalsection}

Also referred to as an \isi{argument focus} construction (cf. \sectref{afsection}), an \isi{identificational construction} contains a topical argument and the \isi{focus domain} is a single constituent. This focused constituent may occur in the O role, as in (\ref{identificational1}), a response to the question ``What did he cut?'':

\ea\label{identificational1} 
{Q: What did he cut?} \\
\glll p\v{e}ra cuchuugube \\
pe\textsuperscript{LH}ra cu-chuugu'=be\textsuperscript{LH} \\
pear \textsc{prog}-cut=\textsc{3sg} \\
\glt `He was cutting PEARS.'
\z
Here, the subject-\isi{topic} in the A role appears as an enclitic on the verb and the focused NP in the O role is placed in pre-verbal position. It is just as acceptable and common, however, in the same communicative context, to respond with a \isi{verb-initial construction} with the object in clause-final position, as in (\ref{identificational2}):
\ea\label{identificational2} 
(Q: What did he cut?) \\
\glll cuchuugube p\v{e}ra  \\
cu-chuugu'=be\textsuperscript{LH} pe\textsuperscript{LH}ra \\
\textsc{prog}-cut=\textsc{3sg} pear  \\
\glt `He was cutting PEARS.'
\z
Out of context, the construction in (\ref{identificational2}) is formally ambiguous between an \isi{identificational construction} and a \isi{topic-comment construction}. While the \isi{verb-initial construction} can be interpreted as either, the object-initial construction can only be interpreted as an \isi{identificational construction}.

In identificational constructions, the single focused constituent may also be an adjunct. As above, the adjunct may appear clause-initially (\ref{identificational3}) or clause-finally (\ref{identificational4}):

\ea\label{identificational3} 
(Q: How did he finish?) \\
\glll nagu\v{e}end\'{a} b\'{i}luxeb\v{e}  \\
na-guee\textsuperscript{LH}nda\textsuperscript{LH} bi-luxe=be\textsuperscript{LH}  \\
\textsc{stat}-fast \textsc{compl}-finish=\textsc{3.hum}  \\
\glt `He finished FAST.'
\z

\ea\label{identificational4} 
(Q: How did he finish?) \\
\glll biluxebe n\'{a}gu\v{e}end\v{a}  \\
bi-luxe=be\textsuperscript{LH} na-guee\textsuperscript{LH}nda\textsuperscript{LH}  \\
\textsc{compl}-finish=\textsc{3.hum} \textsc{stat}-fast   \\
\glt `He finished FAST.'
\z
In (\ref{identificational3}), the focused constituent is an adverb and appears in pre-verbal position and the subject-\isi{topic} again appears as an enclitic on the verb. In contrast, in (\ref{identificational4}), the subject-\isi{topic} again appears as an enclitic on the verb but the focused constituent appears in clause-final position. 

Finally, the single focused constituent in an \isi{identificational construction} may also be a subject. Again, the focused subject can appear pre-verbally (\ref{identificational5}) or post-verbally (\ref{identificational6}): 

\ea\label{identificational5} 
{Q: Who fell?} \\
\glll badu que bi\'{a}ba \\
badu que\textsuperscript{LH} bi-aba  \\
boy \textsc{dist} \textsc{compl}-fall   \\
\glt `THE BOY fell.'
\z

\ea\label{identificational6}  
{Q: Who fell?} \\
\glll biaba badu qu\v{e}  \\
bi-aba badu que\textsuperscript{LH}  \\
\textsc{compl}-fall boy \textsc{dist}  \\
\glt `The boy fell.'
\z

If, however, the subject is coded as a pronominal NP, it may only appear pre-verbally as an independent form, as in (\ref{identificational7}). Unlike dependent pronouns, independent pronouns are always stressed.
\ea\label{identificational7}  
{Q: Who fell?} \\
\glll laabe bi\'{a}ba   \\
laa=be\textsuperscript{LH} bi-aba  \\
\textsc{base}=\textsc{3.hum} \textsc{compl}-fall  \\
\glt `HE fell.'
\z
The focused subject cannot appear as an enclitic, as shown in (\ref{identificational8}). 
\ea\label{identificational8}  
{Q: Who fell?} \\
\glll {\#}biabab\v{e}  \\
bi-aba=be\textsuperscript{LH}\\
\textsc{compl}-fall=\textsc{3.hum}  \\
\glt `He fell.'  
\z
As an unaccented pronominal form, it is unsurprising that the subject enclitic cannot function as a focused constituent. This can be seen in transitive environments as well, where focused pronominal subjects in the A role must occur as independent pronouns in pre-verbal positions, as in (\ref{identificational12}):

\ea\label{identificational12} 
{Q: Who cut the pears?} \\
\glll laabe b\'{i}chuugu ca p\v{e}r\'{a} qu\v{e}  \\
laa=be\textsuperscript{LH} bi-chuugu' ca pe\textsuperscript{LH}ra que\textsuperscript{LH}  \\
base=\textsc{3.hum} \textsc{compl}-cut \textsc{pl} pear \textsc{dist}  \\
\glt `HE cut the pears.'
\z
The semantically equivalent form with a pronominal subject enclitic is pragmatically inappropriate in the same context: 
\ea\label{identificational13}  
{Q: Who cut the pears?} \\
\glll ?bichuugube ca p\v{e}r\'{a} qu\v{e}  \\
bi-chuugu'=be\textsuperscript{LH} ca pe\textsuperscript{LH}ra que\textsuperscript{LH}  \\
\textsc{compl}-cut=\textsc{3.hum} \textsc{pl} pear \textsc{dist}  \\
\glt `He cut the pears.'
\z

In transitive constructions with a topical object, the focused subject constituent must appear before the verb, as in (\ref{identificational9}). 
\ea\label{identificational9}  
{Q: Who cut the pears?} \\
\glll r\'{i}gola que b\'{i}chuugu ca p\v{e}r\'{a} qu\v{e}  \\
ri\textsuperscript{H}gola que\textsuperscript{LH} bi-chuugu' ca pe\textsuperscript{LH}ra que\textsuperscript{LH}  \\
man \textsc{dist} \textsc{compl}-cut \textsc{pl} pear \textsc{dist}  \\
\glt `THE MAN cut the pears.'
\z
Here, the object-\isi{topic} appears as a bare NP in post-verbal position and the focused subject appears pre-verbally. If the subject appears as a lexical NP in the position immediately after the verb, the construction can only be interpreted as an event-reporting construction:

\ea\label{identificational10} 
\glll bichuugu r\'{i}gola que p\v{e}r\'{a} qu\v{e}  \\
bi-chuugu' ri\textsuperscript{H}gola que\textsuperscript{LH} pe\textsuperscript{LH}ra que\textsuperscript{LH}  \\
\textsc{compl}-cut man \textsc{dist} pear \textsc{dist}  \\
\glt `The man cut the pears.'
\z
This construction would not be used as an answer to the question ``Who cut the pears?''. The only way for a lexical NP functioning as a focused subject in the A role to appear after the verb would be for the object NP to appear as an independent pronominal form, as in (\ref{identificational11}):

\newpage
\ea\label{identificational11}  
{Q: Who cut the pears?} \\
\glll bichuugu r\'{i}gola que la\'{a}c\'{a}n\v{i}  \\
bi-chuugu' ri\textsuperscript{H}gola que\textsuperscript{LH} laa=ca=ni\textsuperscript{LH}  \\
\textsc{compl}-cut man \textsc{dist} \textsc{base}=\textsc{pl}=\textsc{3}  \\
\glt `THE MAN cut them.'
\z
While acceptable, such a construction is not considered common or natural by the ZAI speakers with whom I worked and was produced only in elicitation settings.

In summary, based on the above discussion, two factors can be observed to interact closely in the expression of \isi{topic} relations in ZAI: \isi{constituent order} and nominal form. Verb-initial clauses are compatible with the widest range of pragmatic construals as they can be employed in event-reporting, \isi{topic-comment}, and identificational constructions. Lexical NPs in any of these three construction types typically signal a constituent that forms part of the \isi{focus domain}. Independent pronominal forms, for their part, may signal topical or focal material, depending on position and on context. Meanwhile, dependent forms, i.e. subject enclitics, are used exclusively for subject-topics. Pre-verbal constituents, whether subjects, objects, or adjuncts, are almost exclusively focused constituents of identificational constructions. One exception to this is the \isi{topicalization construction}, which I turn to next.


\subsection{Topicalization}\label{topicalizationsection}

Arguments that appear immediately before the verb form part of the \isi{focus domain} (\sectref{identificationalsection}). This is the case in an \isi{identificational construction}, where the focused constituent can be an object (\ref{identificational1}), an adjunct (\ref{identificational3}), or a subject (\ref{identificational12}). In a \isi{topicalization construction}, however, a pre-verbal subject is followed by a resumptive subject enclitic on the verb, as in the following example:

\ea\label{topicalization}
\glll laabe b\'{i}chuugube p\v{e}ra  \\
laa=be\textsuperscript{LH} bi-chuugu'=be\textsuperscript{LH} pe\textsuperscript{LH}ra  \\
base=\textsc{3sg} \textsc{compl}-cut=\textsc{3sg} pear  \\
\glt `He cut pears.' 
\z
In contrast to (\ref{identificational12}) where the pre-verbal pronoun functions as a focused constituent, here the pronoun in pre-verbal position functions as a subject-\isi{topic}, as signaled by the co-indexed subject clitic. The predicate is a comment on that \isi{topic}. 

Topicalization constructions typically occur with referents that have already been introduced. In the following example, the definite NP in pre-verbal position in line 4 refers to an already introduced referent (\ref{topicalization1}):

\ea\label{topicalization1}  (\textit{Pear Stories}, T: l.25--27)
\begin{itemize}
\item[01]
\glll huaxa neza ze xcu\'{i}di que l\'{a},  \\
huaxa neza ze xcui\textsuperscript{H}di que\textsuperscript{LH} la\textsuperscript{H}  \\
but path \textsc{part}.go boy \textsc{dist} \textsc{la}  \\
\glt `But on the path that the boy went la,'


\item[02]
\glll m\'{a}l\'{a}s\'{i} b\'{i}dxaagab\'{e} t\'{i} badudxaapahuiini  \\
ma\textsuperscript{H}lasi bi-dxaaga\textsuperscript{LH}=be\textsuperscript{LH} ti badudxaapa-huiini  \\
suddenly \textsc{compl}-cross-3\textsc{sg} \textsc{indef} girl-\textsc{dim}  \\
\glt `Suddenly he encountered a little girl.'


\item[03]
\glll dx\'{i}'ba sti b\'{i}c\'{i}cl\'{e}ta  \\
dxi'\textsuperscript{H}ba=$\varnothing$ sti\textsuperscript{LH} bicicle\textsuperscript{H}ta  \\
\textsc{part}.climb=\textsc{3} other bicycle  \\
\glt `(She was) on another bicycle.'


\item[04]
\glll badudxaapahuiini que g\'{u}xha zi\~{n}a band\'{a} nuu \'{i}qu\'{e}b\v{e}  \\
badudxaapa-huiini que\textsuperscript{LH} gu-xha=$\varnothing$ zi\~{n}a banda'\textsuperscript{H} n-uu\textsuperscript{LH} ique=be\textsuperscript{LH}  \\
girl-\textsc{dim} \textsc{dist} \textsc{compl}-knock=\textsc{3} palm shade \textsc{stat}-be head-3\textsc{sg}  \\
\glt `The little girl knocked off the hat that was on his head.' 

\end{itemize}
\z
A new participant in the discourse, the bike girl, is introduced in line 2 as an indefinite, lexical NP in the O role, \textit{ti badudxaapahuiini} `a little girl'. This referent appears again in pre-verbal position in line 4, as a definite NP in pre-verbal position, and coincides with a change in subject from the previous clause. This is not an identification construction, however, but a \isi{topicalization construction} in which the bike girl is promoted to \isi{topic}.\footnote{There is, in fact, no difference in formal marking between the \isi{zero form} and no subject enclitic. For this reason, the contrast between the two constructions can only be elicited in discursive contexts and then discussed with native speaker consultants who, in my experience, are then readily able to recognize the appropriate interpretation.} 

There are two elements that permit the analysis of this construction as a \isi{topicalization construction} rather than an identificational one. First, whereas in an \isi{identificational construction} the predicate forms part of the \isi{presupposition}, here the predicate is a comment on the \isi{topic}. There is nothing in the context that ties the predicate as already part of the discourse. Second, as we saw in the previous chapter, the zero \isi{third person} pronominal enclitic form is commonly used by speakers to signal the bike girl as the less thematic participant. This is true in this particular narration of the Pear Story as well. In fact, the zero \isi{third person form} was assigned to the bike girl in the previous \isi{intonation unit}, in line 3. Line 4 is thus a \isi{topic-comment construction} about the bike girl.



The following example further illustrates a similar \isi{topicalization construction}, again from a Pear Story narrative:

\ea\label{topicalization2} (\textit{Pear Stories}, M: l.61--64)

\begin{itemize}


\item[01] 
\glll iza'na sombr\v{e}ru que r\'{a} n\v{u}ub\v{e}  \\
gu-iza'na=$\varnothing$ sombre\textsuperscript{LH}ru que\textsuperscript{LH} ra n-uu\textsuperscript{LH}=be\textsuperscript{LH}  \\
\textsc{compl}-took=\textsc{3sg} hat \textsc{dist} \textsc{loc} \textsc{stat}-be=\textsc{3.hum}  \\
\glt `(He) took the hat to where he (the boy) was.'


\item[02]
\glll laabe b\'{i}siga'debe l\'{a}a chonna p\v{e}ra  \\
laa=be\textsuperscript{LH} bi-si-ga'de=be\textsuperscript{LH} laa=$\varnothing$ chonna\textsuperscript{LH} pe\textsuperscript{LH}ra  \\
\textsc{base}=\textsc{3.hum} \textsc{compl}-\textsc{caus}-give=\textsc{3.hum} \textsc{base}=\textsc{3sg} three pear  \\
\glt `He (the boy) gave him three pears.' 

\end{itemize}
\z
In line 1, the narrator uses a \isi{topic-comment construction} to tell how one of the three boys, the boy with the paddleball, takes the hat to where the bike boy is. The boy with the paddleball functions as the subject-\isi{topic} and is encoded using the zero \isi{third person enclitic}. In line 2, the bike boy is promoted to \isi{topic} through the \isi{topicalization construction}. We see the use of the independent pronominal form in pre-verbal position which is followed by the resumptive subject enclitic. We also see the use of the zero \isi{third person form} in this line to refer to the boy with the paddleball. 



\subsection{Detached or \textsc{la}-marked constructions}

One final sub-class of \isi{topic} phrases is found with the particle \textsc{la} where, similar to a \isi{topicalization construction}, the NP appears before the verb and is co-indexed by a subject enclitic on the verb:

\ea\label{laphrase} 
\glll laabe l\'{a}, cuchuugube p\'{e}ra  \\
laa=be\textsuperscript{LH} la\textsuperscript{H} cu-chuugu'=be\textsuperscript{LH} pe\textsuperscript{LH}ra  \\
base=\textsc{3sg} \textsc{dem} \textsc{prog}-cut=\textsc{3sg} pear  \\
\glt `As for him, he was cutting pears.' 
\z
Constructions such as that in (\ref{laphrase}) were addressed briefly above in \sectref{markedtopics}. In contrast to the similar, semantically equivalent constructions in (\ref{identificational12}) and (\ref{topicalization}), here the NP is set off in a separate \isi{intonation unit} marked by the particle \textsc{la} and accompanied by an audible pause. In some contexts such here in (\ref{laphrase}), \textsc{la}-marked phrases have a \isi{topic} promoting function similar to a \isi{topicalization construction}. In other contexts, however, \textsc{la}-marked phrases can have additional discourse functions. What are the main functions of the \textsc{la} construction, how does it compare cross-linguistically, and what are its uses in spontaneous conversation? This is the \isi{focus} of the rest of this chapter.



\section{Topic relations and the \textsc{la} particle in discourse}\label{laparticle}

The \textsc{la} particle is used widely in ZAI discourse and does not have referential meaning, but interacts with \isi{constituent order} and \isi{intonation}. It carries a High \isi{tone} and invariably appears at the end of an IU, followed by a pause (never anywhere else). In this section, I review the range of constructions in which \textsc{la} occurs, including \isi{adverbial}, conditional, and left-detached clauses, and assess its possible status as a \isi{topic marker}. I conclude by exploring and commenting on the functions of \textsc{la} in extended discourse and conversation. 

\textsc{la} is used consistently in temporal clauses that advance or give information about the sequence of events in a narrative, as in (\ref{whenpear}) and (\ref{temporal2}):

\ea\label{whenpear} (\textit{Pear Stories}, T: l.28--29)
\begin{itemize}
\item[01]
\glll \v{o}ra bidxiguetalube b\'{i}iyabe b\'{a}dudxaapa que \textbf{l\'{a}},  \\
o\textsuperscript{LH}ra bi-dxiguetalu=be\textsuperscript{LH} bi-uuya=be\textsuperscript{LH} badudxaapa que\textsuperscript{LH} la\textsuperscript{H}  \\
when \textsc{compl}-turn=3\textsc{sg.anim} \textsc{compl}-see=3\textsc{sg.anim} girl \textsc{dist} \textsc{la}  \\
\glt `WWhen he turned and saw that girl \textbf{la},'


\item[02]
\glll bidxelasaa bicicl\'{e}taneb\'{e} t\'{i} guieroo'ba  \\
bi-dxela-saa bicicle\textsuperscript{H}ta-ne\textsuperscript{LH}=be\textsuperscript{LH} ti guie-roo'ba  \\
\textsc{compl}-find-\textsc{recip} bicycle-with=3\textsc{sg.anim} one stone-\textsc{aug}  \\
\glt `He crashed his bike against the rock.' 

\end{itemize}
\z

\newpage 
\ea\label{temporal2} (\textit{Pear Stories}, Ts: l.8--9)
\begin{itemize}
\item[01]
\glll raque m\'{a} zeeda t\'{i} xcu\'{i}dihuiini \textbf{l\'{a}},  \\
raque\textsuperscript{LH} ma'\textsuperscript{H} zeeda\textsuperscript{LH} ti xcui\textsuperscript{H}di-huiini la\textsuperscript{H}  \\
then already \textsc{part}.come \textsc{indef} boy-\textsc{dim} \textsc{la}  \\
\glt `Then as a little boy arrives \textbf{la}'  


\item[02]
\glll biiyabe r\'{a} cuchuugu p\v{e}r\'{a} qu\v{e}  \\
bi-iya=be\textsuperscript{LH} ra cu-chuugu'=$\varnothing$ pe\textsuperscript{LH}ra que\textsuperscript{LH}  \\
\textsc{compl}-see=3\textsc{sg.anim} when \textsc{prog}-pick=3 pear \textsc{dist}  \\
\glt `He saw he (the man) was cutting the pears.' 

\end{itemize}
\z
This use in temporal clauses is extremely common and, despite the fact that speakers do not deem it obligatory, it is rare to find cases in spontaneous speech in which \textsc{la} is absent.\footnote{A tentative hypothesis in this regard may be that this use could be related to the lack of temporal or tense information in the verb. ZAI verbs obligatorily take aspectual prefixes, although it is an open question to what extent those prefixes convey tense or mood information (cf. \sectref{verbalmorphology}). More detailed study is required in this direction to determine whether this is the case.}

It is also possible to use \textsc{la} discourse-initially:

\ea\label{initial2} (\textit{Lexu ne gueu})
\begin{itemize}
\item[01]
\glll Ni chig\"{u}eni\'{a}' laatu d\'{i} \textbf{l\'{a}}  \\
Ni chig\"{u}e-ne\textsuperscript{LH}=a'\textsuperscript{H} laa=tu\textsuperscript{LH} di'\textsuperscript{H}  la\textsuperscript{H}  \\
\textsc{rel} \textsc{pot}.say-with=1\textsc{sg} base=3\textsc{pl.anim} \textsc{dem} \textsc{la}  \\
\glt  `This that I will tell you \textbf{la}'


\item[02]
\glll bizaacani m\'{a} xadxi  \\
bi-zaaca=ni\textsuperscript{LH} ma'\textsuperscript{H} xadxi  \\
\textsc{compl}-happen=3\textsc{sg.inan} already time  \\
\glt  `it happened some time ago.'

\end{itemize}
\z
This discourse-initial use of \textsc{la} has a similar function to the use of \textsc{la} with temporal clauses mentioned above as it presents background knowledge or links elements of the discourse with the setting. The \textsc{la} particle also appears consistently at the end of the initial phrase of conditionals, as in (\ref{if}):

\ea\label{if}
\glll Pa gui\'{a}ba nisaguie guix\'{i} \textbf{la}, qu\'{e} zia\'{a}'  \\
pa\textsuperscript{LH} gui\textsuperscript{LH}-aba nisa-guie guixi'\textsuperscript{H}  la\textsuperscript{H} que\textsuperscript{H} zi\textsuperscript{LH}-e=a'  \\
if \textsc{pot}-fall water-stone tomorrow \textsc{la} \textsc{neg} \textsc{fut}-go=1\textsc{sg}  \\
\glt `If it rains tomorrow \textbf{la}, I won't go.' \hfill \citep[109]{pickett1998}
\z
Both \isi{adverbial} and conditional clauses are known to be explicitly marked in other languages as well (see \citealt[292]{thompson2007}). For example, in Hua (Papuan) topics, interrogatives, conditionals are marked with \textit{ve} \citep{haiman1978}. In \ili{Turkish}, a conditional suffix also marks topics \citep{kerslake1996}. Such adverbials and conditionals are not the only clauses to be marked as topics, as it is extremely common to find various types of \isi{adverbial} clauses functioning as topics. Concession, reason, time and condition clauses in Chinese may all occur with the four \isi{topic}/interrogative particles \citep[293]{thompson2007}. In Godi\'{e} (Kru (Ivory Coast)), a non-final morpheme occurs at the ends of \isi{adverbial} clauses functioning as topics and single nouns which function as topics may also be similarly marked \citep{marchese1977,marchese1987}. In \ili{Lisu} (Tibeto-Burman), \isi{adverbial} clauses functioning as topics are marked with the same marker \textit{nya} which is used for NP topics \citep[294]{thompson2007}. In \ili{Karbi} (Tibeto-Burman), the additive particle marks contrastive topics \citep{konnerth2013}. The same is true in Central Kurdish, where the additive particle marks topics as well as temporal, spatial clauses \citep{opengin2013}.

The question, therefore, is whether we can assume \textsc{la} is a \isi{topic marker}. According to \citet[50]{chafe1976} (see also \citealt{li1976}), topics may have the following characteristics: a) they appear in sentence-initial position; b) they are discourse dependent; c) they need not be arguments of the main predication; d) they are definite; and e) they set a ``spatial, temporal, or individual framework within which the main predication holds." 

These facts fit with an analysis in which \textsc{la} is involved in the marking of topical information. This does, in fact, appear to be the case, as \textsc{la} can appear with topical NPs, but never with focused initial NPs:

\ea
\glll {?`}tu b\'{i}'ni' n\v{i}? Tom\v{a}s (*la) bi'ni n\v{i} \\
tu\textsuperscript{LH} bi-uni ni\textsuperscript{LH} Toma\textsuperscript{LH}s { } bi-uni ni\textsuperscript{LH}  \\
who \textsc{compl}-do 3\textsc{sg.inan} Tom\'{a}s { } \textsc{compl}-do 3\textsc{sg.inan}  \\
\glt 'Who did it? Tom\'{a}s (*la) did it.'
\z
There are several reasons why it is common for topical \isi{adverbial} or conditional clauses to play this discourse cohesion role. First, background temporal or spatial clauses may function as a ``scene-setting" \isi{topic} for the matrix clause \citep[125]{lambrecht1994}. Second, their main function is to link the preceding clause with the clause to which they are attached and, at the same time, set a framework within which the following predication holds \citep[294]{thompson2007}. Third, they serve to recapitulate already-mentioned material, i.e. to establish common ground between interlocutors. Finally, there is often a H pitch that appears on the end of the first \isi{intonation unit}, then falling on the second. This helps bind the information into a couplet structure which allows for interpretation together (cf. \sectref{chiasmus}; see also \citet[126--127]{sicoli2007}.\footnote{In this contrasting and textual cohesion function, the ZAI morpheme appears to have characteristics similar to the \ili{Somali} morpheme  \textit{baa} reported in \citet[138-140]{matic2013}.}


\subsection{Left-detachment constructions}

The topic-marking function of \textsc{la} can be seen in left-detached constructions as well. In a left-detached construction, an active or accessible lexical or pronominal NP is set off from the matrix clause without a verb by the \textsc{la} particle and a pause, and is then taken up again in the following matrix clause by a co-indexed element. In (\ref{markedIPR2}), line 3, taken from a Pear Story narrative, the narrator uses an \isi{independent pronoun} followed by \textsc{la} as well as by a pause in the \isi{intonation}: 

\ea\label{markedIPR2}  (\textit{Pear Stories}, Ts: l.30--33)
\begin{itemize}
\item[01]
\glll biabantaab\v{e}  \\
bi-abantaa=be\textsuperscript{LH}  \\
\textsc{compl}-fall.hard=3\textsc{sg.anim}  \\
\glt `He fell.'


\item[02]
\glll bireeche dxumi p\v{e}ra stib\v{e}  \\
bi-reeche dxumi\textsuperscript{LH} pe\textsuperscript{LH}ra sti\textsuperscript{LH}=be\textsuperscript{LH}  \\
\textsc{compl}-spill basket pear \textsc{poss}-3\textsc{sg.anim}  \\
\glt `His basket of pears spilled.'


\item[03]
\glll \textbf{laabe} \textbf{l\'{a}},  \\
laa=be\textsuperscript{LH} la\textsuperscript{H}  \\
\textsc{base}=3\textsc{sg.anim} \textsc{la}  \\
\glt `He \textbf{la},'


\item[04]
\glll biiyadxisibe b\'{a}dudxaapahuiini qu\v{e}  \\
bi-uuyadxisi=be\textsuperscript{LH} badudxaapa-huiini que\textsuperscript{LH}  \\
\textsc{compl}-look=3\textsc{sg.anim} girl-\textsc{dim} \textsc{dist}  \\
\glt `He looked at that little girl.' 

\end{itemize}
\z
The use of \textsc{la} at the end of the \isi{intonation unit} marks the referent of the \isi{independent pronoun}, the bike boy, as the \isi{topic} of the subsequent clause. This is also a different \isi{topic} referent than the \isi{topic} referent of line 2.

The signaling of a different main-clausal subject (or object), as well as a different \isi{topic}, from the previous clause is an extremely common use of \textsc{la}.  Below is another example, this time from casual conversation:

\ea\label{alternatives2} (\textit{20070730{\_}TVA})
\begin{itemize}
\item[01]
\glll xaguet\'{e} nisa runidxi binn\v{i}  \\
xaguete\textsuperscript{H} nisa ru-nidxi binni\textsuperscript{LH}  \\
under water \textsc{hab}-dive person  \\
\glt `Under the water people dive.'


\item[02]
\glll ne l\'{u} nisa \textbf{l\'{a}}, \\
ne\textsuperscript{LH} lu nisa la\textsuperscript{H}  \\
and face water \textsc{la}  \\
\glt `And above water \textbf{la},'


\item[03]
\glll rixuubacab\v{e} \\
ri-xuuba'=ca-be\textsuperscript{LH}  \\
\textsc{hab}-swim=\textsc{pl}-3\textsc{sg.anim}  \\
\glt `they swim.' 

\end{itemize}
\z
After offering one alternative in line 1 to what people may do under the water, the speaker switches the \isi{topic} in line 2, marked by the use of \textsc{la}, to what people may do above water. In this way, the left-detachment construction marked by \textsc{la} is often used to mark a shift in attention from one to another of two or more already topical referents. 

To summarize briefly, we have observed thus far that the \textsc{la} particle serves the following two main discourse functions: 1) it consistently appears at the end of sentence-initial \isi{adverbial} clauses and conditionals, i.e. in a frame-setting or delimiting function, and 2) it may signal changes in \isi{topic} or boundaries of topical units, i.e. as a contrastive \isi{topic marker}. In this way, constructions with \textsc{la} form part of the background presuppositions which, as \citet[292]{thompson2007} note, ``establish a framework within which to proceed with a discourse, in the same way a question does." In fact, all of the constructions involving \textsc{la} that we have reviewed so far share a common morphology with yes/no questions. 


\subsection{Yes/no questions}

Yes/no questions in ZAI are formed by the addition of a question marker that has the exact same form as a sentence-initial \isi{adverbial} clause or conditional (also carries a H \isi{tone}):

\ea\label{yes/no}
\glll {?`}riuuladxu' Lul\'{a} \textbf{l\'{a}}? \\
ri=yuu-ladxi=lu' Lula'\textsuperscript{H} la\textsuperscript{H}  \\
\textsc{hab}=enter-gut=2\textsc{sg} Oaxaca \textsc{la}  \\
\glt `Do you like Oaxaca?' 
\z
There are three principal reasons to think this is the same morpheme as the \isi{discourse particle} \textsc{la}. First, as we saw in \sectref{wordorder}, it is uncommon in V-initial languages for question particles to occur in clause-final position \citep{payne1990}. Second, common morphology has been found cross-linguistically between interrogatives and conditionals (cf. \citealt{haiman1978}). Finally, conditional markers are known to consistently develop out of interrogative particles \citep[296]{konig2007}.

A possible reason for the existence of such a connection in ZAI is that the \textsc{la} particle is used by ZAI speakers as a resource in interaction for managing the common ground. More specifically, \textsc{la} can be seen as a ``try-marking" device \citep{sacks1974}. \citet{sacks1974} define a ``try-marker" as the use of an accessible form, with upward \isi{intonation} contour, followed by a short pause, possibly searching for confirmation of the referent from other participants (cf. \citealt{pekarek2011}). One way to think about this is to think of sentences that are marked with \textsc{la} as similar to ``mini-conversations" \citep[292]{thompson2007}. For example, the \isi{conditional construction} in (\ref{if}) is semantically similar to (\ref{if2}):

\ea\label{if2}
A: \textit{{?`}chi guiaba nisaguie guix\'{i}' \textbf{la}?}  
 `Is it going to rain tomorrow?'

B: \textit{ziaba} 
 `It will.'

A: \textit{que zia\'{a}'} 
`I won't go.' 

\z

Here, Speaker A uses a \textsc{la}-marked phrase (similar to the protasis in the corresponding \isi{conditional construction} in (\ref{if})) to seek confirmation from B in the form of a yes/no response. In this case, B's explicit response provides a shared ground within which A can proceed to effectively convey the main propositional content (the apodosis in the corresponding \isi{conditional construction}), i.e. that he won't go. 

The \isi{conditional construction}, therefore, has a very similar interactional function, the main difference lying in the lack of an explicit response from an addressee after the protasis. It is an open question, however, to what extent ZAI speakers do or do not signal degrees of awareness of common ground through non-verbal means during conversation, as this varies cross-culturally. This is an important question to explore in future work.\footnote{From a usage-based perspective, this analysis suggests the notion of (action and grammatical) projection (cf. \citealt{auer2005}), in the sense that the use of a \textsc{la} foreshadows a range of possible upcoming actions or constructions.} In both cases, \textsc{la} is used to mark the speaker's turn as a procedure for securing referential common ground with the addressee(s). 

The use of \textsc{la} with the function of securing referential common ground can also be seen in cases in which a speaker is constructing a list. An example is given in (\ref{list}), taken from a casual conversation between three male adults. Here, \textsc{la} is used in lines 2, 4, and 5. 

\ea\label{list} (\textit{20120318{\_}C{\_}TVA}: 5:44-5:54)
\begin{itemize}
\item[01]
\glll p\'{e}ru ti dxi \v{a}nte \\
pe\textsuperscript{LH}ru ti dxi a\textsuperscript{LH}nte \\
but one day before \\
\glt  `But one day before,'


\item[02]
\glll vi\v{e}rne huaxhinni que \textbf{l\'{a}} \\
vie\textsuperscript{LH}rne huaxhinni que\textsuperscript{LH} la\textsuperscript{H} \\
Friday evening \textsc{dem} \textsc{la} \\
\glt  `that Friday evening \textbf{la}'


\item[03]
\glll uxudxid\v{u} \\
gu=xudxi=du\textsuperscript{LH} \\
\textsc{compl}=drink=1\textsc{pl.excl} \\
\glt  `we got drunk.'


\item[04]
\glll laabe \textbf{l\'{a}} \\
laa=be\textsuperscript{LH} la\textsuperscript{H} \\
base=3\textsc{sg.anim} \textsc{la} \\
\glt  `Him (pointing) \textbf{la}'


\item[05]
\glll Vidal \textbf{l\'{a}} \\
Vidal la\textsuperscript{H} \\
Vidal \textsc{la} \\
\glt  `Vidal \textbf{la}'


\item[06]
\glll ne n\'{a}a \\
ne\textsuperscript{LH} naa \\
and 1\textsc{sg} \\
\glt  `and I.'


\item[07]
\glll bide'du jm\'{a} c\'{a}gu\v{a}ma \\
bi-de'=du\textsuperscript{LH} jma\textsuperscript{H} cagua\textsuperscript{LH}ma \\
\textsc{compl}-drink=1\textsc{pl.excl} much beer \\
\glt  `We drank lots of beer.'  

\end{itemize}
\z
\largerpage

The \textsc{la} particle appears in line 2 at the end of an \isi{adverbial} clause similar to the uses discussed above in (\ref{whenpear}) and (\ref{temporal2}). In line 4, the speaker uses the \isi{third person} \isi{independent pronoun} followed by \textsc{la} to refer to one of his interlocutors (which he reiterates by simultaneously pointing). In the immediately following line, line 5, he refers to yet another \isi{third person} referent (not a participant) using his first name followed by \textsc{la}. He adds one final referent, himself, in line 6, without the use of \textsc{la}. Those three individuals make up a group, established over three \isi{intonation} units, who together function as the subject-\isi{topic} in line 7 referred using the 1\textsc{pl.excl} enclitic. In this way, the \textsc{la} particle is used by the speaker to help the addressee identify the individuals in question, i.e. secure common ground, prior to the predication (cf. Principle of the Separation of Reference and Role, \citealt{lambrecht1994}).

In addition to \isi{topic} marking and \isi{topic promotion}, then, the use of \textsc{la} should be seen as a resource for organizing talk and for making that organization recognizable to the speech participants. This section has shown that an analysis of the multifunctional nature of \textsc{la} depends on the analysis of spontaneous speech and, especially, of conversation. It may be useful to investigate the use of \textsc{la} as a resource in the co-construction of talk, in floor-holding, in turn-taking, in turn entry points, etc. and, more generally, as a window into the ways in which listeners orient to speech and conversation. Because listeners in different speech communities may orient in different ways, the relevant question thus becomes: how might the use of the \textsc{la} particle be tied to local conversational strategies and conversational norms? From this perspective, it is likely that a characterization of \textsc{la} in terms of notions like \isi{topic} and \isi{focus} is insufficient, and that insight into its functions can be better understood through an analysis of talk-in-interaction, i.e. of the kinds of interactional work that are being done in conversation and how.


\section{Summary and conclusions}

This chapter has presented an analysis of the strategies available to ZAI speakers to mark various types of topics and \isi{topic} relations. It explored the relationship between pragmatic or cognitive status and topic-hood and found that it is not a pre-requisite, but that \isi{topic} referents usually have a certain degree of pragmatic \isi{accessibility}, where more acceptable topics are higher on a cognitive status scale (i.e., the Topic Accessibility Scale,  \citealt{lambrecht1994}). Because insufficiently accessible \isi{topic} referents are more difficult to interpret, the most acceptable topics in ZAI were found to be clitics and the least acceptable to be indefinite NPs and bare nouns.

\largerpage 
Two main factors, \isi{constituent order} and nominal form, were observed to interact closely in the expression of \isi{topic} relations in ZAI. Verb-initial clauses are compatible with the widest range of pragmatic construals as they can be employed in event-reporting, \isi{topic-comment}, and identificational constructions. Lexical NPs in any of these three construction types typically signal a constituent that forms part of the \isi{focus domain}. Independent pronominal forms, for their part, may signal topical or focal material, depending on position and on context. Meanwhile, dependent forms, i.e. subject enclitics, are used exclusively for subject-topics. Pre-verbal constituents, whether subjects, objects, or adjuncts, are almost exclusively focused constituents of identificational constructions. One exception to this is the \isi{topicalization construction}. In \textsc{topicalization} constructions, the pre-verbal constituent is a subject-\isi{topic} with a co-referring enclitic on the verb. These are used typically in cases of \isi{topic promotion}.

A correlation was identified between \isi{information structure} and certain types of constructions and the cognitive status of the referents involved. For example, \textsc{in focus} \citep{gundel1993} or \textsc{activated} referents do not occur in presentational or event-reporting constructions. Also, \textsc{type identifiable} referents do not occur in ``\isi{marked topic}", detachment constructions involving the particle \textsc{la}. Therefore, for ZAI, NPs in presentational constructions are never pronominal forms, and NPs in detached, \textsc{la}-marked phrases are never indefinite.

It is important to note that the analysis of spontaneous speech and, specifically, of conversation makes possible a multifunctional analysis of \textsc{la}. Through this analysis, we saw too that \textsc{la}-marked constructions can have a topic-promoting function, but also mark topical information, set the spatial, temporal, or individual framework within which the predication holds, and play a discourse cohesion role. They mark phrases that function as ``scene-setting topics" that have a frame-setting or delimiting function. \textsc{la}-marked constructions also mark contrastive topics, indicating changes in topics or boundaries of topical units. 

Furthermore, constructions with \textsc{la} form part of the background presuppositions, and establish a framework within which to proceed with the discourse, in the same way a question does. \textsc{la} is, in fact, used in yes/no questions to secure referential common ground with the addressee(s). As such, \textsc{la} can be seen not only as a resource for marking various types of topical information, but more generally as a resource for organizing talk and interaction. 


