\documentclass[output=paper]{langscibook}
\ChapterDOI{10.5281/zenodo.5483096}

\author{Ljudmila Geist\orcid{0000-0001-7907-4958}\affiliation{University of Cologne; University of Stuttgart}}
\title[New developments in the semantics of noun phrases in Slavic languages]
      {New developments in the semantics of noun phrases in Slavic languages}
\abstract{The paper gives a general overview of the theoretical approaches to the semantics and syntax of nominal phrases. It shows how the recent work on this topic in formal Slavistics has contributed to the further development of the theory. The following issues are addressed: What counts as reliable evidence for the assumption of the DP-layer in articleless Slavic languages? How do Slavic languages express the distinction between strong definiteness based on anaphoricity and weak definiteness based on situational uniqueness? What is the semantic concept behind definiteness contributed by NPs in the topic position? What is the meaning of special collective nouns such as Czech \textit{dvojice} ‘a group of two people’ and Russian complex numerical measure nouns such as \textit{strogrammovka} ‘a 100-gram glass’? What do nominal roots in Slavic languages denote before they enter different syntactic environments and how do different syntactic environments determine their interpretation? Is there evidence for the assumption of the functional projections NumP and ClassifierP in addition to NP and DP in Slavic languages?

\keywords{DP syntax, DP semantics, definiteness}
}


\begin{document}
\maketitle

%see or s. ?

\section{Introduction}
The goal of this article is twofold: At a general level, its aim is to give an overview of the development of theoretical approaches to the semantics (and syntax) of nominal phrases since \citet{Abney1987} and to determine the current state of the art in this particular field. A second, more specific task is to set the scene for the contributions by the participants of the ``Semantics of Noun Phrases'' Workshop held on December 6, 2018 at the University of Göttingen as a part of the 13th Conference on ``Formal Description of \ili{Slavic} Languages'' (FDSL 13). The workshop focused on nominal categories and their interpretation and formal representation. As under the principle of compositionality the meaning of the whole is determined by the meanings of its syntactic parts, the papers address not only semantics but also the syntax of noun phrases. In this article I want to identify the main questions in the current research on noun phrases in \ili{Slavic} but also in other languages and show how the papers by the workshop participants can contribute to answering some of these questions.

Since the formulation of the DP-hypothesis in \citet{Abney1987} and the introduction of D as a functional category for determiners, various functional projections between D and NP have been added to integrate nominal categories such as number and host numerals and attributive adjectives cross-linguistically (see \citealt{Alexiadou.Haegeman2007,Borer2005,Cheng.Sybesma1999,Cheng.Heycock2017,Zamparelli2000}, a.o). Many researchers agree upon at least the DP layers depicted in \figref{fig:1}.

\begin{figure}
\caption{DP-layers}
   \begin{forest}
   [DP [D] [NumP [Num] [CIP [Classifier] [NP] {\draw (.east) node[right]{$\longrightarrow$ descriptive content};}]{\draw (.east) node[right]{$\longrightarrow$ countability};}]{\draw (.east) node[right]{$\longrightarrow$ quantization};}]{\draw (.east) node[right]{$\longrightarrow$ reference};}
    \end{forest}
    \label{fig1}
\label{fig:1}
\end{figure}

Each layer is a host for a particular element that is endowed with a particular semantic function. The head D maps the whole phrase into an argument. The DP-layer hosts strong determiners such as definite articles and demonstratives. The numeral phrase NumP is responsible for quantization, i.e. it is the place where cardinals and other weak determiners and quantifiers can merge. The analysis of \citet{Borer2005}, but also \citet{Cheng.Heycock2017} and \citet{Cheng.Sybesma1999}, among others, posits a Classifier head responsible for countability. It is the host for classifiers in classifier languages and for plural morphology in languages without classifiers. The lowest layer, the NP, is projected by the noun introducing descriptive content.

The internal semantic and syntactic architecture of DPs has traditionally been a topic of research in \ili{Slavic} languages as well. Although most \ili{Slavic} languages have no articles, the instantiation of definiteness in the D-layer has received a lot of discussion. Numeral and classifier layers have been assumed as well.

In what follows, I will go through the layers of the DP and mention some current topics of debate which serve as connecting points for contributions in this volume. We start with the highest layer, the DP.

\section{DP}
\subsection{DP-layer: Yes or no?}

Since many \ili{Slavic} languages lack articles, the availability of the DP projection in those languages has been hotly disputed. The question is whether nominals in articleless \ili{Slavic} languages are DPs, as in the \ili{Germanic} or \ili{Romance} languages, or bare NPs (or possibly intermediate structures). There are three views: (i) According to the so-called universal DP approach, adopted by \citet{Longobardi1994} and \citet{Matthewson1998}, among others, the structure of noun phrases in languages without articles is the same as in languages with articles such as English and \ili{German}: argument noun phrases are projected fully as DPs in both types. (ii) The proponents of the so-called parameterized DP approach, among others \citet{Chierchia1998} and \citet{Baker2003}, claim that the structure of noun phrases in languages without articles differs radically from that of languages with articles: in the former type of language, noun phrases do not project a DP. \tabref{table:1} from \citet{Veselovska2014} lists the proponents of each theory in \ili{Slavic} linguistics.

\begin{table}
\begin{tabularx}{\textwidth}{lQQ}
\lsptoprule
& Universal QP/DP/NP structure & Parameterized QP/DP/NP structure \tabularnewline
\midrule
\ili{Czech} & \citet{Veselovska1995,Veselovska2001} & \citet{Corver1990} \tabularnewline
\ili{Russian} & \citet{Pereltsvaig2007,Pereltsvaig2013} &  \citet{Boskovic2005,Boskovic2007,Boskovic2009} \tabularnewline
\ili{Serbo-Croatian} & \citet{Progovac1998}, \citet{Basic2004}, \citet{Caruso2012,Caruso2013} & \citet{Zlatic1997,Zlatic1998} \tabularnewline
\ili{Polish} & \citet{Rutkowski2002} & \tabularnewline
\lspbottomrule
\end{tabularx}
\caption{Universal vs. parameterized nominal projection \citep[13]{Veselovska2014}\label{table:1}}
\end{table}

(iii) \citet{Pereltsvaig2006} develops a new view on the structure of noun phrases: she assumes that verbs take arguments of various semantic types and syntactic sizes. In addition to generalized quantifiers of type $\stb{\stb{e, t}, t}$ and referential DPs of type $e$, they are able to take arguments of type $\stb{e, t}$ as well. Some heads, such as the \ili{Russian} cumulative prefix \textit{na-}, select only arguments that are NPs or NumPs of predicate type $\stb{e, t}$.

Further evidence for the type $\stb{e, t}$ is the use of NPs as predicative complements of the copula verb \textit{be}. According to \citet{Partee1987}, constituent conjunction requires identical semantic types, and as adjectives are treated as type $\stb{e, t}$, the predicative NP \textit{millioner} `millionaire' in \REF{ex:1} must also be of type $\stb{e, t}$, i.e. a nonreferential predicative expression.

\ea \label{ex:1}
\gll On chotja i millioner, no očen' skromnyj. \\
    he although and millionaire but very modest\\
\glt `Although he is a millionaire, he is very modest.' \hfill (\ili{Russian})
\z

\noindent There is also a semantic argument in favor of a DP layer for some occurrences of noun phrases in \ili{Russian}. Normally, the DP is identified as the locus of referentiality. \citet{Borer2005}, for example, states that only DPs have referential indices and can be interpreted as arguments. In my work \citep{Geist2010}, I have shown that bare NPs in the topic position in \ili{Russian} are always referential and definite, see \REF{ex:2}. In the first clause, \textit{mal'čik} `boy' and \textit{devočka} `girl' introduce new discourse referents and the topical noun phrase \textit{devočka} in the second clause anaphorically picks out the same individual girl introduced in the first clause. If a non-definite use is intended, the NP \textit{devočka} must be accompanied by the indefiniteness marker \textit{odin} `one'.

\ea \label{ex:2}
\gll Ja uvidela mal'čika i devočku. Devočka nesla korzinku. \\
    I saw boy and girl girl bore basket\\
\glt `I saw a boy and a girl. The girl bore a basket.'\hfill (\ili{Russian})
\z

\noindent Provided that DPs are the locus of referentiality and bare nouns in \ili{Russian} can be used referentially at least in the topic position, it must be assumed that they may project a DP.

Besides semantic evidence in favor of a DP level in \ili{Slavic}, there is some syntactic support in the literature (see the overview in \citealt{Pereltsvaig2013}). At least the following arguments have been mentioned: (i) a rigid order of prenominal adjectives, (ii) a split between light and heavy adjectival modifiers, and (iii) maximal interpretation of prenominal possessives. We will not discuss all these arguments but will look only at the last one, since it was addressed in the workshop.

\citet{Kagan.Pereltsvaig2014} observe that the syntactic position of the possessive adjective relative to the numeral has an impact on the interpretation of the whole phrase, see \REF{ex:3}. In the unmarked order \REF{ex:3a}, where the possessive follows the numeral, the phrase is neither interpreted as maximal nor exhaustive: Dima may have more than five books. \citet{Kagan.Pereltsvaig2014} discuss the possible alternative marked order \REF{ex:3b} where the possessive precedes the numeral. Unlike \REF{ex:3a}, this phrase can only receive a maximal or exhaustive interpretation and presupposes that Dima has exactly five books. \citet{Kagan.Pereltsvaig2014} assume that the maximal interpretation in \REF{ex:3b} comes about as a result of the placement of the possessive in a syntactically high position in the DP-domain above the numeral in the NumP. The possessive adjective in \REF{ex:3a}, however, is placed low, in the NP-domain which is below NumP.

\ea \label{ex:3}
\ea
\gll pjat' Diminyx knig \\
    five Dima.\textsc{gen.pl} books\\ \hfill not maximal
\glt `Dima’s five books' \label{ex:3a}
\ex \gll Diminyx pjat' knig \\
 Dima.\textsc{gen.pl} five books\\ \hfill maximal/exhaustive
\glt `Dima's five books' \hfill (\ili{Russian}) \label{ex:3b}
\z \z

\noindent But there is evidence that even NPs preceded by possessive adjectives without numerals can project full DPs. As \REF{ex:4} shows, nouns occurring with possessive adjectives can be used anaphorically: \textit{Petin kollega} in the second clause in \REF{ex:4} picks up the colleague introduced in the previous clause. Since anaphoric NPs must be DPs following \citet{Kagan.Pereltsvaig2014}, we would assume that the possessive adjective \textit{Petin} in \textit{Petin kollega} `Petja's colleague' is a modifier that applies at the high DP-level and hence indicates the presence of a zero D-head. \citeposst{Pereltsvaig2007} position is more radical, she analyzes the possessive adjective as a D-element.


\ea \label{ex:4}
\gll U Peti novyj kollega i u Niny tože. Petin kollega očen' molod. \\
    with Petja new colleague and with Nina too Petja's colleague very young\\
\glt `Petja has a new colleague and Nina, too. Petja's colleague is very young.'\\
\glt \hfill (\ili{Russian})
\z

\noindent \textcitetv{chapters/gepner} investigates the morphological and syntactic properties of possessives but also demonstratives and the quantifier \textit{každyj} in \ili{Russian}. She examines whether these expressions can provide evidence for a DP-layer. She shows, based on their morphological and syntactic properties, that prenominal possessives and demonstratives behave as adjectives rather than determiners and argues that they do not fulfill the criteria for a D-element. NPs accompanied by them can occur in predicate positions as shown in \REF{ex:5} and in existential sentences, where typical DPs such as proper names are excluded.

\ea \label{ex:5}
\gll Ivan byl petinym kollegoj. \\
    Ivan was Peter's.\textsc{ins} colleague.\textsc{ins}\\
\glt `Ivan was Peter’s colleague.' \hfill (\ili{Russian})
\z

\noindent According to Gepner, the interpretation of the possessive adjective in the predicate NP such as \REF{ex:5} does not differ from the interpretation of possessive adjective in an argument NP such as \REF{ex:4}. She assumes that in both cases possessive adjectives modify the noun within NP, e.g., are always placed low in the structure. An exception is the quantifier \textit{každyj}. Despite patterning morphologically with adjectives, it has the syntax and semantics of a quantifier and behaves like a functional element outside the NP. Gepner leaves open in which functional layer \textit{každyj} is hosted. But is the interpretation of the possessive NP in \REF{ex:5} really the same as the interpretation of the possessive NP in \REF{ex:4}?

There is an old observation that the interpretation of possessive NPs depends on their use as arguments or as predicates. \citet{Jespersen1965} discusses example \REF{ex:6} in English:

\ea \label{ex:6} \ea The captain of the vessel was my brother. \label{ex:6a}
\ex My brother was captain of the vessel.\label{ex:6b}
\z \z

\noindent Jespersen says that in \REF{ex:6a} \textit{my brother} in the predicate position means `one of my brothers', or leaves it unspecified whether the speaker has more than one brother, whereas \textit{my brother} in the argument position in \REF{ex:6b} has a maximal\slash unique or exhaustive interpretation `the speaker’s only brother'. To explain this difference in interpretation, \citet{Kagan.Pereltsvaig2014} would assume that the possessive adjective \textit{my} in \REF{ex:6b} is placed in the ``high'' DP-domain, which excludes the non-ex\-haustive interpretation. The placement of the possessive in the ``low'' NP-domain as in \REF{ex:6a} would only specify the relational meaning of \textit{brother} and have no restriction on the unique/non-unique interpretation. Under Gepner's analysis, however, \textit{my} would be integrated low in the NP-domain in both cases. The lack of non-maximal interpretation of the possessive NP in the argument position in \REF{ex:6b} should then be explained in a different way. The validity of both analyses should be compared in the future research.

\subsection{DP-layer: Two types of definites}

Many languages such as \ili{German} and \ili{Mauritian} Creole differentiate between two types of definites \citep{Lobner2011,Schwarz2009,Schwarz2013,Jenks2015}. Definites that have a more complex form indicate an anaphoric link and are called ``strong definites''. Definites with a simpler form, the so-called ``weak definites'', express definiteness based on situational \textsc{uniqueness}. In standard \ili{German}, the contrast between the two forms is reflected in cases where a definite can contract with a preceding preposition. The contracted form is the weak form, indicating situational uniqueness while the non-contracted one is a strong form indicating an anaphoric use.


\ea \label{ex:7}
\multicolsep=.25\baselineskip
\begin{multicols}{2}\raggedcolumns
\ea
\gll Hans ging zu dem Haus. \\
    Hans went to the\textsubscript{strong} house\\ %\hfill strong
\glt `Hans went to the house.' \label{ex:7a}
\columnbreak\ex \gll Hans ging zum Haus. \\
 Hans went to-the\textsubscript{weak} house\\ %\hfill weak
\glt `Hans went to the house.' \hfill (\ili{German}; \citealt[12]{Schwarz2009}) \label{ex:7b}
\z \end{multicols}\z

\noindent \citet{Czardybon2017} shows that a similar distinction between anaphoric and situational definiteness is made in \ili{Polish}: the demonstrative \textit{ten} as a strong determiner can optionally be used to signal anaphoric definiteness, while weak bare NPs are used if definiteness is based on situational uniqueness. In \REF{ex:8a}, \textit{mȩżczyzna} `man' in its second occurrence is used anaphorically and is accompanied by \textit{ten}. In \REF{ex:8b}, the situationally unique NPs \textit{odległość} `distance', \textit{sufit} `ceiling’, and \textit{podłoga} `floor' are used without \textit{ten}, i.e., have a weak form.\largerpage

% % %     \largerpage[-2] % To get ex. (8b) on next page

\ea \label{ex:8}
\ea
\gll Widziałem jak do pokoju wchodził mȩżczyzna. Kiedy wszedłem obaczyłem że przy ornie stoi ten mȩżczyzna. \\
    see.\textsc{pst}.1\textsc{sg} how in room.\textsc{loc} enter.\textsc{pst} man when enter.\textsc{pst}.1\textsc{sg} see.\textsc{pst}.1\textsc{sg} that at window.\textsc{loc} stand.\textsc{prs} \textsc{det} man\\
\glt `I saw a man go into the room. When I entered I saw that the man was standing at the window.' \glt \hfill (\ili{Polish}; \citealt[96--97]{Szwedek1976}, cited from \citealt[50]{Czardybon2017})\label{ex:8a}
\ex \gll [\dots] odległość od sufitu do podłogi wynosi 2,85 metra \\
 {} distance from ceiling.\textsc{gen} to floor.\textsc{gen} amount.3\textsc{sg.prs} 2.85 meter\\
\glt `the distance from the ceiling to the floor amounts to 2.85 meters.' \\
\glt \hfill (\ili{Polish}; \citealt[74]{Czardybon2017}) \label{ex:8b}
\z
\z

\noindent Thus, \ili{Polish} obeys the standard correspondence between the form of the definite (strong vs. weak) and the use of definite descriptions (situationally unique vs. anaphoric).

\begin{sloppypar}
\textcitetv{chapters/simik} studies the two types of definiteness in \ili{Czech}. \ili{Czech} also uses bare NPs and NPs combined with a demonstrative for definite reference. However, as Šimík shows, strong demonstrative NPs are also able to refer to situationally unique objects in addition to weak bare NPs, unlike \ili{Polish}. To explain the division of labor between weak bare NPs and strong demonstrative NPs he distinguishes between two types of situational uniqueness: accidental uniqueness and inherent uniqueness. An object is inherently unique if it is unique in all relevant situations that are ``like'' the mentioned situation. An accidentally unique object is unique in the mentioned situation but need not be unique in other similar situations. Šimík discusses an example with the noun \textit{tabule} `blackboard'. The object referred to by this NP is typically unique in all classroom situations. By contrast, the object denoted by the NP \textit{book} can be unique in a particular situation but it need not be unique in other situations in which books are typically involved. Thus, the referent of \textit{the book} can only be accidentally unique. Accidentally unique objects in \ili{Czech} are referred to by strong demonstrative NPs, while inherently unique objects are referred to by weak bare NPs, see \tabref{table:2}.
\end{sloppypar}

\begin{table}
\centering
\begin{tabularx}{\textwidth}{llQll}
\lsptoprule
\ili{Polish} & use &  anaphoric  &  \multicolumn{2}{c}{situationally unique}\\\cmidrule(lr){4-5}
 & form  &    SF: demonstrative NP       & \multicolumn{2}{c}{WF: bare NP} \\\midrule
\ili{Czech} & use &    anaphoric       &      accidentally uniq.     & inherently uniq. \\\cmidrule(lr){3-4}
 & form & \multicolumn{2}{c}{SF: demonstrative NP} & WF: bare NP \\
\lspbottomrule
\end{tabularx}
\caption{Two types of definiteness in Polish and Czech. Source for Polish: \citet{Czardybon2017}. Source for Czech: \textcitetv{chapters/simik}. WF: weak form. SF: strong form.\label{table:2}}
\end{table}

Šimík uses situation semantics and proposes an analysis in which inherent unique\-ness is taken to be a property of topic situations and accidental unique\-ness a property of demonstratives. He shows how other types of NPs such as generic, anaphoric, and non-specific indefinite NPs can be analyzed within this framework.

\subsection{DP-layer: Semantics of definiteness}

If definiteness is what the DP-layer may contribute, the question is what the semantic concept or notion behind it is. Definiteness is often considered to correspond to \textsc{familiarity}:  The individual referred to by the definite expression has often been assumed to be familiar to the speaker and hearer, e.g. if the NP is used anaphorically \citep{Christophersen1939,Heim1982}. In the philosophical tradition, definiteness is assumed to correspond to uniqueness: a definite description conveys that there is exactly one individual in the situation that satisfies the description (\citealt{Chierchia1998,Dayal2004}; a.o.). Besides familiarity and uniqueness there are other less prominent notions of definiteness that we will not consider here. The two main notions of definiteness are in competition if we want to explain the use of the definite article in languages such as \ili{German} or English: most uses can be explained by both theories, but some occurrences receive a better account in the familiarity theory and the others by uniqueness.

The common tenet is that languages without definite articles can convey the same meaning as definite descriptions do in languages with articles, albeit with different formal means. According to the classical view, bare NPs as themes obligatorily receive a definite interpretation in articleless \ili{Slavic} languages. In my work \citet{Geist2010} I explain and formalize this traditional belief using the notion of aboutness topic instead of theme, see \REF{ex:9}.

%\ea \label{ex:9} \textit{Context:} `I saw a boy and a girl.'\\
%\gll Devočka	 vošla   	v     dom. \\
%    {(the)/(*a) girl}  came  	into {(the/a) house}\\
%\glt `The girl entered the house.'\hfill (\ili{Russian}; \citealt[193]{Geist2010})
%\z

\ea \label{ex:9} \textit{Situation:} I saw a boy and a girl.\\
\gll Devočka	 vošla   	v     dom. \\
    girl  came  	into house\\
\glt `The (*a) girl entered \{the/a\} house.'\hfill (\ili{Russian}; \citealt[193]{Geist2010})
\z

\noindent Given the situation in \REF{ex:9}, we can utter \textit{devočka vošla v dom}, where \textit{devočka} can only receive a definite referential interpretation; an indefinite interpretation (that it was another girl, not anaphorically related to the previously mentioned girl) is not available. In the topical use of the bare NP in \REF{ex:9}, familiarity and uniqueness coincide and it cannot be decided which notion of definiteness can better capture the definite interpretation.

In very recent work, \citet{Simik.Demian2020} provide experimental evidence that bare singular NPs as topics in \ili{Russian} do not convey uniqueness. They test two scenarios via pictures: In the first picture there is a locomotive and a unique disconnected carriage. In the second picture there is a locomotive and two carriages, one of them is disconnected.

The authors show that \ili{Russian} speakers can use sentence \REF{ex:10} with \textit{vagon} `carriage' as topic to describe both pictures, although the second picture violates uniqueness.

\ea \label{ex:10}
\gll Vagon    otcepilsja \\
    carriage  disconnected\\
\glt `The carriage got disconnected.'\hfill (\ili{Russian}; \citealt[15]{Simik.Demian2020})
\z

\noindent From this they conclude that definiteness contributed by topical definiteness is not based on uniqueness.

The investigation by \textcitetv{chapters/seres} is in line with \citeposst{Simik.Demian2020} observations. They have the intuition that alleged uniqueness contributed by bare NPs as topics can be overridden in appropriate contexts such as \REF{ex:11}. However, definiteness conveyed by the definite article for topical definites in English contributes strong uniqueness and cannot be overridden.

\ea \label{ex:11}
\ea
\gll Direktor našej školy pojavilsja v tok-šou. \\
    director.\textsc{nom} our school.\textsc{gen} appeared in talkshow\\
\glt `The director of our school appeared in a talkshow.'
\ex
\gll Drugoj direktor \minsp{(} našej) školy vystupil na radio. \\
 other director.\textsc{nom} {} our school.\textsc{gen} spoke on radio\\
\glt  `The other director (of our school) spoke on the radio.`\\
\glt \hfill (\ili{Russian}; \textcitetv{chapters/seres})
\z\z

\noindent To account for the difference between \ili{Russian} and English, \citeauthor{chapters/seres} assume that the kind of definiteness expressed by bare nominals in \ili{Russian} is better captured in terms of pragmatic strengthening than the uniqueness presupposition. While uniqueness contributed by the definite article is semantic in nature and can be formally represented by the iota operator, this representation is not appropriate for the purely pragmatic definiteness contributed by topicality of bare NPs in articleless languages. Following \citet{Heim2011}, \citeauthor{chapters/seres} propose that bare nominal phrases in articleless \ili{Russian} are born indefinite. Definiteness can be achieved by pragmatic strengthening of an indefinite and can have different sources: ``ontological'' (or ``situational'') uniqueness, topicality and\slash or familiarity\slash anaphoricity. All these can be seen as sources for the familiarity of the object. Thus, familiarity rather than uniqueness in the narrow sense underlies definiteness contributed by the topical use of bare NPs.

To conclude, the experimental findings by \citet{Simik.Demian2020} and the investigation of uniqueness by \citeauthor{chapters/seres} suggest that languages differ not only in the means that contribute to the expression of definiteness, but also in the type of concept of definiteness. Definite articles do not contribute the same type of definiteness as topicality in articleless languages.
%%%%%%%%%%%%%%%%%%%%%%%%%%%%%%%%%%%%%%%%%%%%%%%%%%%%%%%%%%%
\section{Numeral phrase: Numerals and collectivity}

In addition to ordinary numerals, \ili{Slavic} languages have a special class, the so-called collective numerals. Collective numerals can be nominalized and denote groups of $n$ members of $x$, see \ili{Czech} \textit{dvojice} `twosome = a group of two people'. Since such collectives range over sets they have been called \textsc{set collectives}. \textcitetv{chapters/docekal} address the behavior of set collectives in comparison to collectives denoted by collective nouns such as \textit{skupina} `group' in \ili{Czech}. The latter type of collectives ranges over atomic entities and has been called \textsc{atom collectives}.

Although collections are composed of a plurality in both types of collectives, they differ in the accessibility of the members of that plurality. The difference becomes apparent in combinations with the determiner \textit{each} and the binominal \textit{each}. The two uses of \textit{each} are illustrated in \REF{ex:12}.

\ea \label{ex:12} \ea Each [\textsubscript{\textsc{pp}} of the three girls] has bought three books. \hfill (determiner \textit{each})\label{ex:12a}
\ex Two girls have bought [\textsubscript{\textsc{np}} three books] each. \hfill (binominal \textit{each})\label{ex:12b}
\z \z

\noindent \citeauthor{chapters/docekal} show that the determiner \textit{každý} `each' cannot distribute over the members of collectives regardless of type. Binominal \textit{každý}, on the other hand, can combine with set collectives yielding distribution over members of the collection, while it is excluded with atom collectives.

\begin{table}
\centering
\begin{tabularx}{\textwidth}{Qcc}
\lsptoprule
 & \textit{skupina sportovců} & \textit{dvojice sportovců} \tabularnewline
 & `group of athlets' & `a group of two athletes' \tabularnewline
 & (atom collectives) & (set collectives) \tabularnewline
\midrule
distribution over members with determiner \textit{každý} & $*$ & $*$ \tabularnewline
\midrule
distribution over members with binominal \textit{každý}  & $*$ & \ding{51} \tabularnewline
\lspbottomrule
\end{tabularx}
\caption{Atom collectives and set collectives}
\label{table:3}
\end{table}

Thus, binominal \textit{každý} serves as a diagnostic to test the accessibility of the members of collections and to distinguish between the two types of collectives: while the individual members of set collectives are at least weakly accessible, members of atom collectives are completely inaccessible and atomic from the outside.

The authors model the complex interaction of determiner \textit{každý} and binominal \textit{každý} with set and atom collectives within the plural compositional discourse representation theory (PCDRT). The main idea of the formalization is this: while determiner \textit{každý} distributes over both the restrictor and the nuclear scope, binominal \textit{každý} only distributes over the distributive share denoted by the NP it is attached to, remaining neutral with respect to the collectivity and cumulativity of the material outside of its scope. This explains its compatibility with set collectives and its incompatibility with atom collectives.

\section{Classifier phrase}
\subsection{Types of classifiers}

The typological literature on the mass/count distinction commonly distinguishes between classifier and non-classifier languages. In classifier languages such as \ili{Chinese}, nouns cannot be directly combined with a numeral and need the help of a classifier, the so-called individual or natural unit classifier \citep{Krifka1989, Krifka1995}, as in \REF{ex:13}.

\ea \label{ex:13}
\gll san zhi bi \\
    three \textsc{cl} pencil\\
\glt `three pencils'\hfill (\ili{Chinese})
\z

\noindent According to \citet{Cheng.Sybesma1999}, count classifiers in \ili{Chinese} primarily serve to name the unit in which the entity denoted by the noun naturally occurs. Classifier languages are contrasted with non-classifier languages such as \ili{Slavic} languages, which have count nouns that can be directly combined with numerals, as in \REF{ex:14}.

\ea \label{ex:14}
\gll pjat' stolov \\
    five tables.\textsc{gen.pl}\\
\glt `five tables'\hfill (\ili{Russian})
\z

\noindent But non-classifier languages can also use a type of classifier which occurs in combination with mass nouns, see \REF{ex:15}:

\ea \label{ex:15}
\gll tri 	litra 	vina \\
    three	liter.\textsc{gen.sg}	wine.\textsc{gen.sg}\\
\glt `three liters of wine'\hfill (\ili{Russian})
\z

\noindent However, classifiers of the type in \REF{ex:15} considerably differ from \ili{Chinese} classifiers in their status (lexical vs. grammatical) and function (measuring vs. counting). \citet{Cheng.Sybesma1999} and \citet{Li2013} argue that individual classifiers in \ili{Chinese} have the status of a grammatical category. Their function is that of counting, which has to be distinguished from measuring. According to \citet{Rothstein2010}, ``Counting puts entities (which already count as `one') in correspondence with the natural numbers, while measuring assigns a (plural) individual a value on a dimensional scale'' \citep[386]{Rothstein2010}. The numeral `five' in the counting context \REF{ex:14} provides a property of a plural entity in the denotation of N, expressing how many atomic units the plurality has. Rothstein argues that measure classifiers such as in \REF{ex:15} should rather be considered a lexical category for measuring. In \REF{ex:15} `liter' combines with a numeral and together they form a measure predicate. In the syntactic composition, this predicate applies to sets of quantities expressed by the mass noun `wine' and assigns a value to it on a measure scale calibrated in liters.

Theories of the mass/count distinction suggest that languages have grammatical classifiers only if they have no number morphology. Thus, count nouns and grammaticalized classifiers should be in complementary distribution (e.g., \citealt{Borer2005, Chierchia2010}). \citet{Khrizman2016} shows that this complementarity does not hold in \ili{Russian}. In addition to number morphology, \ili{Russian} has three grammaticalized classifiers \textit{štuka} `item', \textit{čelovek} `person', and \textit{golova} `head', which optionally occur in numeral constructions with plural, see \REF{ex:16}.

\ea \label{ex:16}
\gll pjat' \minsp{(} štuk) jaic\\
    five {} item.\textsc{gen.pl} egg.\textsc{gen.pl}\\
\glt `five eggs'\hfill (\ili{Russian}; \citealt{Khrizman2016})
\z

\noindent According to \citet{Khrizman2016}, such classifiers differ from \ili{Chinese}-type individual classifiers and should rather be analyzed as a special class of measure words. They denote functions that map quantities of entities onto the value on a scale calibrated in natural units in the sense of \citet{Krifka1989,Krifka1995}.



\subsection{Diminutive suffixes as classifiers}

So far we have characterized classifiers that are morphologically free morphemes. However, in some languages the classifying function can also be performed by suffixes as bound morphemes.  As \citet{DeBelder2008} shows, the diminutive suffix in \ili{Dutch} turns mass nouns into count nouns, hence it functions as a classification device, yet it is compatible with overt morphological plural marking, see \REF{ex:17}.

\ea \label{ex:17}
\gll veel brod-je-s\\
   many bread-\textsc{dim-pl} \\
\glt `many rolls'\hfill (\ili{Dutch}; \citealt[2]{DeBelder2008})
\z

\noindent In \ili{Russian}, diminutive suffixes such as -\textit{ka} may also perform a classifier function if combined with a mass noun, see \REF{ex:18}.

\ea \label{ex:18}
\ea
\gll železo -- železka -- dve železki \\
iron {} iron.\textsc{dim.f} {} two iron.\textsc{dim.pl}\\
\glt `iron -- a piece of iron -- two pieces of iron'
\ex
\gll  šokolad -- šokoladka -- dve šokoladki\\
chocolate {}	chocolate.\textsc{dim.f} {} two chocolate.\textsc{dim.pl}\\
\glt `chocolate -- a bar of chocolate -- two bars of chocolate' \hfill (\ili{Russian})
\z \z


\noindent In \REF{ex:18}, -\textit{ka} has a function identical to the function of unit classifiers in \ili{Chinese}. First,  it turns an uncountable noun into a countable one: while \textit{železo} and \textit{šokolad} are mass nouns, \textit{železka} and \textit{šokoladka} denote countable units, which are compatible with numerals and plural formation. Second, being a suffix, -\textit{ka} has the status of a grammatical morpheme. Third, besides determining countability, -\textit{ka} also triggers a gender shift of the noun: the noun becomes feminine. This feature qualifies \textit{ka}- for being a syntactic functional head, the Classifier head.\largerpage

\textcitetv{chapters/khrizman} addresses other formations with the suffix -\textit{ka}, complex numerical measure nouns in \ili{Russian} such as \textit{stogrammovka} `a 100-gram glass' or \textit{dvuxlitrovka} `a two-liter-jar'. In colloquial \ili{Russian}, such morphologically complex nouns are productively constructed out of a numeral and a measure noun as shown in \REF{ex:19}.

\ea \label{ex:19}
\gll sto- 	grammov- 	ka 		vodki\\
  hundred.\textsc{nom}-	gram.\textsc{gen.pl}-	ka 					vodka.\textsc{gen} \\
\glt `a 100-gram glass of vodka'\hfill (\ili{Russian}; \citetv{chapters/khrizman})
\z

\noindent \citeauthor{chapters/khrizman} shows that complex measure nouns with -\textit{ka} are count nouns as they can be pluralized and modified by numerals. Such measure nouns denote containers, i.e., actual objects. For example, \textit{stogrammovka} in \REF{ex:19} refers to objects which weigh 100 grams. The nature of the object is determined by context (a 100-ml bottle/tube, a 100-gram package/bar etc.). Khrizman analyzes such nouns within \citeposst{Rothstein2017} theory of counting and measuring. This approach treats complex measure nouns as predicates denoting sets of discrete entities with certain measure properties, e.g. properties of having a value on a dimensional scale calibrated in certain units. For instance, \textit{stogrammovka} denotes a set of disjoined entities (jars, bottles etc.), which have the property of having the value 100 on a weight scale calibrated in gram units. The suffix -\textit{ka} in the formation of measure nouns contributes a shift from a measure interpretation to a container interpretation, thus its function is similar to container nouns (e.g. glass).

\sloppypar{\citeauthor{chapters/khrizman}'s analysis has an important implication for the theory of noun phrases. It has been argued in the literature that all count nouns originate from mass nouns and bare count nouns should be derived from mass nouns via lexically concealed individuating operators \citep{Krifka1989, Krifka1995, Rothstein2017, Sutton.Filip2016}. Diminutive suffixes like -\textit{ka} could then be seen as a morphological realization of such operators.}
%%%%%%%%%%%%%%%%%%%%%%%%%%%%%%%%%%%%%%%%%%%%%%%%%%%%%%%%%%%%%%%%
\section{NP}

Now we move on to the lowest layer of the DP, the NP-layer. What does the head of the NP denote? There has been a surge of interest in this question in the literature that has led to many different views. According to \citet{Chierchia1998}, languages vary in what their NPs are able to denote. The syntax-semantics mapping is not universally fixed and, in some languages, nouns can denote kinds (or masses), in others they denote objects, but there are also languages where some nouns denote objects and others denote kinds. This view was questioned in \citet{Borer2005}. She argues that the basic interpretation of a noun crosslinguistically is a non-countable interpretation as mass (sometimes also interpreted as kind). A non-countable noun can achieve countability by combining with functional heads in the syntax. This basic idea was further developed by \citet{Borik.Espinal2012, Borik.Espinal2015} and applied to \ili{Russian} in \citet{Borik.Espinal2012}. They assume that bare nouns in \ili{Russian} as in \REF{ex:20} primarily denote properties of kinds of individuals that share the property denoted by the noun. If they occur in an argument position as topics they are interpreted as definite and form a DP that refers directly to a kind.

\ea \label{ex:20}
\gll Slon skoro budet zanesen v Krasnuju Knigu esli na nego ne perestanut ochotit'sja.\\
   elephant soon will listed in red book if on him not stop hunt\\
\glt `The elephant will soon be listed in the IUCN Red List if people don’t stop hunting it.'\hfill (\ili{Russian}; \citealt[137]{Borik.Espinal2012})
\z

\noindent Syntactically, the kind-referring DP has the simple structure in \REF{ex:21a}. \citeauthor{Borik.Espinal2012} argue that kind-denoting nouns are definite and numberless. In their syntactic structure, D is the locus of the iota operator. If the bare noun is used to refer to a concrete individual as in \REF{ex:22}, it must be shifted into the object domain and receive number. This shift from kinds to objects is performed by the realization operator \cnst{r} of \citet{Carlson1977}. \cnst{r} is specified by number in NumP, see \REF{ex:21b}. No NumP is involved in the composition of a definite kind interpretation.

\ea \label{ex:21}
\ea  {[\textsubscript{DP} D   [\textsubscript{NP} N]]}   \hfill kind \label{ex:21a}
\ex   {[\textsubscript{DP} D  [\textsubscript{NumP} NUM\textsubscript{[–PL]} [\textsubscript{NP} N]]]} \hfill individual object \label{ex:21b}
\z \ex \label{ex:22}
\gll Slon podošel k vode.\\
  elephant came to water\\
\glt `The elephant came to the water.'\hfill (\ili{Russian})
\z

\noindent \textcitetv{chapters/kwapiszewski} use the work of \citeauthor{Borik.Espinal2012} as the point of departure for their analysis of the DP-structure in \ili{Polish}. They argue that bare noun counterparts of \textit{slon} in \REF{ex:20} are definite and numberless in \ili{Polish}, just as in other languages. Following \citeauthor{Borik.Espinal2012}, the authors assume that bare NPs in \ili{Polish} denote properties of kinds, which must be bound by the iota operator in D to license direct reference to kinds. Number projection is not available in their syntactic representation. In noun phrases referring to object instances of kinds as in \REF{ex:22}, the number projection is available and it is responsible for the derivation of individual instances of kinds.

However, \citeauthor{chapters/kwapiszewski} show that \citeauthor{Borik.Espinal2012}'s approach is incompatible with the theory of intersective kind modification by \citet{McNally.Boleda2004}, who analyze modifiers such as \textit{Bengal} in \REF{ex:23} as intersective modifiers of kinds. Since \citeauthor{Borik.Espinal2012} consider nouns to be singleton sets of kinds, such a treatment of modification is impossible.

\ea \label{ex:23}
\gll Tygrys bengalski jest na skraju wymarcia.\\
 tiger.\textsc{nom} Bengal.\textsc{m} is on verge extinction.\textsc{gen} \\
\glt `The Bengal tiger is on the verge of extinction.'
\glt \hfill (\ili{Polish}; \citetv{chapters/kwapiszewski})
\z

\noindent To solve this problem, the authors introduce a subkind operator (\cnst{sk}) into the semantics and link it to the functional head Classifier in the syntax. Thus \textit{tygrys} `tiger' in \REF{ex:23} has the following structure:

\ea \label{ex:24} [\textsubscript{DP} \textsc{$+$def} [\textsubscript{NumberP} \textsc{$-$plural} [\textsubscript{ClassifierP}   \cnst{sk} [\textsubscript{NP} tygrys ]]]]   \hfill subkind reading
\z

\noindent \citeauthor{chapters/kwapiszewski} assume that \textit{tygrys} in \REF{ex:24} refers to a subkind rather to an object, the classifier head is specified as a \cnst{sk}. Thus, the classifier head can have different functions:  deriving subkinds of a kind by the \cnst{sk} or deriving object instances of a kind by the realization operator (\cnst{r}). The authors propose the following structure for definite object-denoting and subkind-denoting NPs in \ili{Polish}:

\begin{figure}[ht]
\caption{The structure of a DP in Polish \parencitetv{chapters/kwapiszewski}}
   \begin{forest}
   [DP [\textsc{[$+$def]}] [NumP [\textsc{[$-$plural]}] [ClassifierP [{[}\cnst{r}{]}/{[}\cnst{sk}{]}] [NP] ]]]
    \end{forest}
    \label{fig2}
\end{figure}

If the \cnst{sk} is introduced in the classifier head, the NP denotes a set of subkinds and a kind-modifying adjective such as \textit{bengalski} `Bengal' can intersectively modify the subkinds. All in all, this work substantiates the assumption of the functional layers DP, NumP, and ClassifierP besides NP in \ili{Polish}.

\section{Conclusions}\largerpage

To conclude, the contributions in this volume address different theoretical issues which have been under controversial discussion in the literature. The contributors develop and improve the theory of NP structure, relativize some previous assumptions, and show how languages without articles specify the NP structure, assumed to be universal in natural language. The main findings can be summarized as follows:

\begin{itemize}
    \item Although prenominal possessive adjectives in \ili{Russian} are not determiners, they have been assumed to be placed in the high DP-domain if the NP occurs in an argument position. Since NPs with possessives can also occur in predicate positions, where referential DPs are normally excluded, the question arises whether the possessive is hosted lower in the structure in this case. Alternatively, it can be assumed that possessives are always integrated low in the NP, see \textcitetv{chapters/gepner}. Under this analysis, the exhaustive interpretation of the NP with possessive in an argument position requires a different explanation. These two analyses should be compared in future research.
    \item Generally, weak and strong definiteness has been assumed to correspond to anaphoric vs. situational uniqueness, respectively. The example of \ili{Czech} shows that the boundary between the two types of definiteness may alternatively lie within situational uniqueness dividing situational uniqueness into accidental and inherent uniqueness, see \textcitetv{chapters/simik}.
    \item Languages without articles have been assumed to express definiteness by topicality. However, definiteness contributed by topicality seems to be different from definiteness contributed by the definite article in languages that have it. While topicality indicates familiarity, the definite article indicates uniqueness, see  \textcitetv{chapters/seres}.
    \item In the formation of collectives in \ili{Polish}, we have to distinguish between two types with respect to the accessibility of its members: set collectives formed of collective numerals and atom collectives formed of collective nouns such as \textit{group}. While the individual members of the set collectives are at least weakly accessible, the members of the atom collectives are completely inaccessible and atomic from the outside. The binominal \textit{každý} is sensitive to this distinction, see \textcitetv{chapters/docekal}.
    \item The nominalizing suffix -\textit{ka} in \ili{Russian}, also used as a diminutive suffix, can serve as a classifier turning non-countable expressions such as measure expressions but also mass nouns into countable nouns. This function renders it similar to classifiers in \ili{Chinese}, see \textcitetv{chapters/khrizman}.
    \item Nouns in articleless \ili{Slavic} languages, in particular \ili{Polish}, can be analyzed as being numberless and denoting properties of kinds. They can refer to a kind if combined with a iota operator in D. But they can be turned into object level denotation or subkind denotation by the classifier head and then be combined with a numeral in NumP, see \textcitetv{chapters/kwapiszewski}.
\end{itemize}

The workshop contributors present their generalizations and analyses developed for single languages: \ili{Russian}, \ili{Polish}, or \ili{Czech}. Future research should show whether these generalizations extend to other \ili{Slavic} languages as well and what implications this has for the theory of the universal structure of NPs.

\section*{Abbreviations}

\begin{tabularx}{.5\textwidth}{@{}lX}
\textsc{cl}&{classifier}\\
\textsc{det}&{determiner}\\
\textsc{dim}&{diminutive}\\
\textsc{f}&{feminine}\\
\textsc{gen}&{genitive}\\
\textsc{ins}&{instrumental}\\
\textsc{loc}&{locative}\\
\textsc{m}&{masculine}\\
\end{tabularx}%
\begin{tabularx}{.5\textwidth}{lX@{}}
\textsc{nom}&{nominative}\\
\textsc{pl}&{plural}\\
\textsc{prs}&{present}\\
\textsc{pst}&{past}\\
\textsc{sg}&{singular}\\
\textsc{1}&{first person}\\
\textsc{3}&{third person}\\
&\\
\end{tabularx}

\section*{Acknowledgments}
I am grateful to the organizers of the FDSL\,13 conference in 2018 in Göttingen, especially Uwe Junghanns, for giving me the opportunity to hold the workshop on the semantics of noun phrases. The research for this paper was funded by the \ili{German} Research Foundation (DFG), via the project grant number GE 2136/3-1, project ``The fine structure of the \ili{Russian} noun phrase: A comparative perspective'' (\url{https://gepris.dfg.de/gepris/projekt/445439335}).

{\sloppy\printbibliography[heading=subbibliography,notkeyword=this]}

\end{document}
