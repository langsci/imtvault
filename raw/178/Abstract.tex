% !TEX root = ../main.tex
\addchap{Abstract}\label{Abstract}
\begin{refsection}

This book investigates the associations between \isi{information structure}
and linguistic forms in spoken Japanese
mainly by analyzing spoken corpora.
It proposes multi-dimensional annotation and analysis procedures for spoken corpora and
explores the relationships between \isi{information structure} on the one hand
and particles, \isi{word order}, and intonation on the other.

Particles, \isi{word order}, and intonation in spoken Japanese have been investigated separately in different frameworks and in different subfields of the literature;
there was no unified theory accounting for the all the phenomena.
This book provides a unified investigation of all the phenomena in question, %investigated the phenomena as a whole in a consistent way
by annotating all target expressions according to the same criteria and
by investigating them all from the same analytical framework. %employing the same analytical framework.
Chapter \ref{Introduction} outlines the questions to be investigated in the study and introduces the methodology of the book.
Chapter \ref{Background} reviews the literature on Japanese linguistics
as well as the literature on \isi{information structure} in different languages.
Chapter \ref{Framework} proposes the analytical framework of the book.
Major findings are discussed in Chapters \ref{Particles}, \ref{WordOrder}, and \ref{Intonation}.

Chapter \ref{Particles} analyzes the distributions of \isi{topic} and case particles.
It is made clear that so-called \isi{topic} particles 
(\ci{wa}, zero particles, \ci{toiuno-wa}, and \ci{kedo/ga} preceded by \isi{copula}) are mainly sensitive to the given-new taxonomy,
whereas case particles (\ci{ga}, \ci{o}, and the zero particles) are sensitive to both focushood and grammatical function.
While the distinction between \ci{wa} and \ci{ga} has attracted much attention in traditional Japanese linguistics,
this book analyzes the distribution of different kinds of \isi{topic} and case particles, including zero particles.%,
%are analyzed in this book.

Chapter \ref{WordOrder} studies \isi{word order}; more specifically, clause-initial, pre-predicate, and post-pred\-i\-cate noun phrases. %
%%%ISSUE OUT OF ALIGNMENT
Topical NPs appear either clause-initially or post-pred\-i\-cat\-ively,
while focal NPs appear pre-predicatively.
Clause-initial and post-predicate NPs differ from each other mainly in their status in the given-new taxonomy.
The previous literature investigated clause-initial, pre-predicate, and post-predicate constructions from different frameworks;
however, there was no unified account of \isi{word order} in Japanese.
The book outlines an account of \isi{word order} in spoken Japanese within a unified framework.

\newpage 
Chapter \ref{Intonation} investigates intonation.
While the previous literature mainly concentrated on \isi{contrastive focus},
this book discusses intonation from the perspective of both \isi{topic} and focus. %in terms of both \isi{topic} and focus.
It is argued that intonation corresponds to a unit of processing and that \isi{information structure} influences the form of the intonation units.

Chapter \ref{Discussion} discusses the theoretical implications of these findings.
Finally, Chapter \ref{Conclusion} summarizes the book and points out some remaining issues and possible future studies.

\end{refsection}
