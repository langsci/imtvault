% !TEX root = ../main.tex
\chapter{Conclusion}\label{Conclusion}


%%----------------------------------------------------
%%----------------------------------------------------
\section{Summary}

This study attempted to partially answer a larger question of
how Japanese speakers communicate with each other
through assumptions regarding the mental state of other people.
It revealed that Japanese speakers employ a variety of cues to express the speaker's assumptions about the \isi{hearer}'s mental state.

While a substantial portion of the literature has discussed the distinction between \ci{wa} and \ci{ga},
the relationships among other kinds of particles have not been discussed as thoroughly.
Chapter \ref{Particles} in this study revealed the differences between \ci{wa} and other \isi{topic} particles such as \ci{toiuno-wa} and \ci{kedo/ga} preceded by \isi{copula},
as well as the distribution of case markers,
by drawing a \isi{semantic map} of these particles.
It also investigated the distribution of zero particles and their associations with \isi{information structure}.

The previous literature investigated clause-initial, pre-predicate, and post-pred\-i\-cate constructions %
%%%ISSUE post-predicate out of alignment
independently in different frameworks;
however, there was no unified account of \isi{word order} in Japanese.
In Chapter \ref{WordOrder},
I described \isi{word order} in spoken Japanese in a unified framework.

Chapter \ref{Intonation} investigated intonation.
While the previous literature mainly concentrates on \isi{contrastive focus},
this study discussed both \isi{topic} and focus.
I investigated intonation as a unit of processing and
argued that \isi{information structure} influences the form of intonation units.

To the best of my knowledge,
particles, \isi{word order}, and intonation in Japanese have been investigated separately in the literature;
there was no unified theory to account for the all of these phenomena.
This study investigated the phenomena as a whole in a consistent way
by annotating the same information for all linguistic expressions and
by employing the same analytical framework for all of them.



%%----------------------------------------------------
%%----------------------------------------------------
\section{Theoretical and methodological implications}

This section discusses theoretical and methodological implications of
this study.
First, I proposed that \isi{topic} and focus are multidimensional
rather than homogeneous: they are interpreted as a bundle of features, where feature is scalar rather than binary.
Different languages are sensitive to different features to different degrees.
Even within a language,
different linguistic expressions are sensitive to different features to various extents.
Moreover, it is often the case that
a single linguistic expression is sensitive to multiple features.
As outlined in Chapter \ref{Background},
different authors discuss different kinds of \isi{topic} and focus,
which is a confusing situation.
I argue that linguistic research would be clearer if one asks
``what feature(s) is/are sensitive to what linguistic expression(s)?'',
instead of asking
``which feature best predicts the distribution of some linguistic expressions?''

Second,
I proposed methods of annotation and analysis
that are cross-lin\-guis\-tically applicable.
I did not annotate all the features proposed in \ref{ISFeatures} in \S \ref{FrameworkDefinition};
however, all the features can be defined independent of language-specific categories and can be applied universally.
Some features such as specificity and \isi{definiteness} are hard to annotate, and it is highly likely that different annotators have different intuitions about the expression in question.
I argue that this is not a problem.
In real life,
some people might interpret some expression to be definite,
while other people might interpret the same expression to be \isi{indefinite}.
This is a source of linguistic variation, and there is no single right answer.
Ideally, a statistically sufficient number of annotators annotate the same corpus, and all the annotations are used in analyses.

Third,
I point out the importance of qualitative analysis in addition to quantitative analysis.
In \S \ref{TopPar}, for example,
I concluded that \ci{toiuno-wa} and \ci{wa} attach to elements with different statuses of the given-new taxonomy by examining each example,
even though the difference was not visible from the raw numbers.
This is because my annotation is not fine-grained enough to capture the subtle difference between these markers.
Of course, it is necessary to run statistical tests in the future.
However, it is also important to examine each example to make sure that
the quantitative results do not contradict other observations.
%However, it is also important to examine each example to make sure that
%the results do not contradict observations.



%%----------------------------------------------------
%%----------------------------------------------------
\section{Remaining issues}

This study has left several issues open for future investigation. %many remaining issues to be investigated in the future.
In this section, I discuss two of these issues.


%%----------------------------------------------------
\subsection{Predication or judgement types}

As discussed in Chapter \ref{Background},
traditional Japanese linguistics scholars have paid attention to
\isi{predication} types or judgement types.
Predication or judgement types include the distinctions between \isi{thetic} vs.~categorical judgements and between attribute vs.~phenomenon judgements
\cite{matsushita28,yamada36,mio48,kuroda72,masuoka08,kageyama12}.
Although this study focused on the distinction between nominal types such as \isi{topic} and focus,
the findings of this study can be integrated into theories of \isi{predication} or judgement types.
This implies that \isi{information structure} is not only related to properties of NPs;
rather, it is also associated with properties of predicates.
Especially, grammatical categories such as tense, aspect, modality, and evidentiality are highly likely to be related to different \isi{information structure} types.
For example, as \citeA{masuoka12} points out,
the \isi{topic} marker \ci{toiuno-wa} cannot be used in event \isi{predication} (or stage-level \isi{predication});
it can only be used in property \isi{predication} (or individual-level \isi{predication}).%
 \footnote{
 See \S \ref{Back:GeneralChar:Toiunowa} in Chapter \ref{Background}
 for the distinction between property vs.~event \isi{predication}.
 }
This is shown in the contrast between \Next[a] and \Next[b].
\Next[a], where \ci{toiuno-wa} is used in event \isi{predication} with the simple past tense,
is unacceptable.
\Next[b],
on the other hand, where \ci{toiuno-wa} is used in property \isi{predication},
is acceptable.
%
\ex.
 \ag. *sachiko-\EM{toiuno-wa} uso-o tui-ta \\
      Sachiko-\ci{toiuno-wa} lie-\ci{o} spit-\ab{past} \\
      `Regarding Sachiko, she lied.'
     \hfill{\cite[p.~96]{masuoka12}}
 \bg. sachiko-\EM{toiuno-wa} uso-tuki-da \\
      Sachiko-\ci{toiuno-wa} lie--spitter-\ab{cop} \\
      `Regarding Sachiko, she is a liar.'
     \hfill{(Constructed)}

\citeA{masuoka12} concludes that
\ci{toiuno-wa} is used only for property \isi{predication}.

Moreover, it is well known that
the interpretations of \ci{wa} and \ci{ga} change
depending on predicate type \cite{kuroda72,kuno73}.
In property \isi{predication},
\ci{wa} is the default marker, and
\ci{ga} tends to be interpreted as exhaustive listing.
As exemplified in \Next[a-b], both of which are copular sentences (i.e., property \isi{predication}),
the sentence with \ci{wa} \Next[a] is considered to have a common topic-comment structure,
while the sentence with \ci{ga} \Next[b] is considered to focus only John.
Specifically, \Next[b] is interpreted as the answer to the question `who is a student?'
In Kuno's terminology,
\ci{ga} is interpreted as marking exhaustive listing.
%\ci{ga} is interpreted to be exhaustive listing.
%
\largerpage
\ex.
 \ag. zyon-\EM{wa} gakusei-desu \\
      John-\ci{wa} student-\ab{cop} \\
      `John is a student.'
 \bg. zyon-\EM{ga} gakusei-desu \\
      John-\ci{ga} student-\ab{cop} \\
      `JOHN is a student. (it is John who is a student.)'
      \hfill{\cite[38]{kuno73}}

In event \isi{predication},
on the other hand,
\ci{ga} is the default marker and is interpreted as involving a neutral description %and is interpreted to be a neutral description
while \ci{wa} tends to be interpreted as contrastive.
In \Next[a-b], which involve event \isi{predication},
the NP followed by \ci{wa} in \Next[a] is interpreted to be contrastive,
while the whole sentence including the NP with \ci{ga} in \Next[b] is interpreted to have a \isi{broad focus} structure;
as above, in Kuno's terminology,
\ci{ga} is considered to be a neutral description.
%
\ex.
 \ag. ame-\EM{wa} hut-te i-masu-ga... \\
      rain-\ci{wa} fall-and \ab{prog}-\ab{plt}-though \\
      `Though it IS raining...'
 \bg. ame-\EM{ga} hut-te i-masu \\
      rain-\ci{ga} fall-and \ab{prog}-\ab{plt} \\
      `It is raining.'
      \hfill{(ibid.)}


I am aware of only a few studies investigating the question of
why the sentences of particular \isi{information structure} types are associated with
specific \isi{predication} types.



%%----------------------------------------------------
\subsection{Genres}

Genres are also an important factor influencing the phenomena investigated in this study.
As pointed out in \S \ref{BackSubSubZero},
for example,
the choice between zero vs.~overt particles is sensitive to
styles (casual vs.~formal).
However, it is not clear why the formal style requires overt particles more often than the casual style.

Further, I have argued that
post-predicate constructions are more frequent in conversations than in monologues.
Although I suggested a few possible suggestions as to why this is the case (\S \ref{WOPostPreEles}),
there is still no clear answer.
Since there is a corpus of conversations annotated in the same way as the corpus used in this study \cite{nakagawaden12},
it could be useful to compare the two corpora.

It is likely that in monologues like the ones employed here \isi{predicate-focus structure}s appears more frequently than in usual conversations;
\chd{in narratives, the speaker usually talks about what s/he did or what happened to him/her, which fixes a \isi{topic} (typically the speaker) -- and fixing a \isi{topic} elicits a \isi{predicate-focus structure}.}
\chd{Moreover, because of the absence of hearers who ask \ci{wh}-questions and who misunderstand what the speaker means,} the speaker has to answer \ci{wh}-questions or correct the hearer less frequently, which is what typically elicits an \isi{argument-focus structure}.
This is another reason why it is important to investigate other genres of spoken language. %It is important also for this reason to investigate other genres of spoken language.



