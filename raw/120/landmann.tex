\documentclass[output=paper,modfonts]{langscibook} 
\title{Focus marking in Kuria} 
\author{Meredith Landman\affiliation{Pomona College}\lastand
Rodrigo Ranero\affiliation{University of Maryland, College Park}
}
\ChapterDOI{10.5281/zenodo.1251754}
\abstract{This paper examines focus marking in Kuria. We propose an account of the syntax/semantics of the prefix /ne-/. This prefix displays a varied syntactic distribution, posing a puzzle as to what semantically unifies all of its uses. In focus constructions, /ne-/ obligatorily appears on a fronted (i.e. focused) phrase, whereas in simple declaratives, /ne-/ obligatorily appears pre-verbally. Following previous analyses of similar markers in Bantu (\citealt{Schwarz2007} for Kikuyu and \citealt{AbelsMuriungi2008} for Kiitharaka), we analyze /ne-/ uniformly as a focus marker that arises in a focus phrase in the left periphery. We support this account of /ne-/ by presenting novel data that suggest that even when /ne-/ occurs pre-verbally, it still marks focus (VP focus or sentential focus.) We also show how /ne-/ differs syntactically from similar markers in other languages. For example, Kuria allows for the focus marker to appear internal to the focused constituent, in contrast with data from Kiitharaka \citep{AbelsMuriungi2008} and other languages (see \citealt{HartmannZimmermann2009} for Guruntum). This paper thus discusses a range of data patterns relating to the Kuria prefix /ne-/, offering insight into a syntax/semantic puzzle as well as cross-linguistic variation in the realization of focus.
}
\begin{document}
\maketitle
 
\title{Focus marking in Kuria}
 
% \todo{subexamples}
% \todo{examples in footnotes roman numbers}

 
\section{Introduction} 

\ili{Kuria} (Narrow Bantu E.43) is an understudied language spoken in \isi{Kenya} and \isi{Tanzania}. In this paper, we investigate the syntax and semantics of the \ili{Kuria} morpheme /ne-/, which in the literature has been labeled a \isi{focus} marker (\citealt{Cammenga2004,Mwita2008}). In declaratives, this morpheme obligatorily occurs pre-verbally, as in \REF{ex:landmann:1}:\footnote{The morpheme /ne-/ displays predictable allomorphy, surfacing as either [ne] or as [n], and it is possible that it is /n-/ that is in fact underlying.}\textsuperscript{,}\footnote{In the declaratives presented in this paper, /ne-/ is obligatory only in the remote past and remote future tenses; /ne-/ does not appear in the immediate past or immediate future tenses—see also \citealt{Mwita2008}, who reports that /ne-/ must appear in 11 out of 22 TAM combinations. It is unclear to us why /ne-/ is absent in certain tenses; this may be due to phonological, syntactic, or semantic/pragmatic conditioning, and we do not attempt an account of this pattern here.}


\ea\label{ex:landmann:1}
  \gll Ichi-ng’iti *(\textbf{n-})cha-a-it-ir-e ege-toocho.\\
     10-hyena (\textsc{\textbf{foc}}\textbf{-})\textsc{10sa}-\textsc{pst}-kill-\textsc{prf}-\textsc{fv} 7-rabbit\\
\glt ‘The hyenas killed the rabbit.’
\z

Previous analyses of \ili{Kuria} /ne-/, as well as similar morphemes in other Bantu languages, treat these items in different ways: Some view them as \isi{focus} markers (see \citealt{Cammenga2004} and \citealt{Mwita2008} for \ili{Kuria}, \citealt{Schwarz2007} for \ili{Kikuyu}, and \citealt{AbelsMuriungi2008} for \ili{Kiitharaka}), while others see them as assertion or polarity markers (see \citealt{Bergvall1987} for \ili{Kikuyu} and \citealt{Eslinger2013} for \ili{Kuria}). We argue here, based on original data, that /ne-/ is in fact a \isi{focus} marker, and we propose an extension of \citegen{Schwarz2007} syntactic account of \ili{Kikuyu} to \ili{Kuria}, by which /ne-/ heads a Focus Phrase projection in the left periphery of the \isi{clause}.\footnote{\citet{Mwita2008} suggests that some cases involving pre-verbal /ne-/ are due to grammaticalization, and therefore lose the \isi{focus} meaning. We argue here that because some instances of pre-verbal /ne-/ involve \isi{focus}, the simplest synchronic analysis is to take all instances of pre-verbal /ne-/ as instantiating \isi{focus}.} Thus, our aims in this paper are twofold: first, empirically, to document an array of patterns related to \ili{Kuria}  /ne-/, which illustrate the language’s strategy for \isi{focus} realization, and, second, from a theoretical perspective, to show that an extension of a previous syntactic analysis of a similar morpheme in a related language (i.e., Schwarz's account of \ili{Kikuyu}) can capture the \ili{Kuria} data.

The remainder of this paper is organized as follows. In §2, we present the basic syntactic distribution of /ne-/. In §3, we present evidence for analyzing /ne-/ semantically as a \isi{focus} marker. In §4, we propose a syntactic account of /ne-/, extending \citet{Schwarz2007}’s account of \ili{Kikuyu} to \ili{Kuria}. In §5, we compare \isi{focus} marking in \ili{Kuria} with what has been observed for \isi{focus} marking in other languages, discussing how \ili{Kuria} fits into a crosslinguistic typology of \isi{focus} marking. In §6, we document a number of additional patterns regarding /ne-/, articulating several issues regarding /ne-/ for future research. Finally, §7 concludes our paper.



\section{The basic distribution of /ne-/}
\largerpage[-1]
 
\ili{Kuria} displays default SVO \isi{word order}, with some freedom of object ordering in \isi{ditransitive} and tritransitive constructions. As shown in \REF{ex:landmann:1}, in declaratives, /ne-/ obligatorily appears pre-verbally and only once per \isi{clause} (parallel to \ili{Kikuyu}; see \citealt[142]{Schwarz2007}).\footnote{%
  This also holds for embedded declaratives, e.g. clauses embedded by a bridge \isi{verb} such as ‘say’:

  \ea
  \gll N-eng’we a-a-gamb-er-e iga gati *(\textbf{n-})a-a-ha-y-e umw-igia ege-tabo.\\
  \textsc{foc}-who \textsc{sa}-\textsc{pst}-say-\textsc{prf}-\textsc{fv} \textsc{comp} 1.Gati \textsc{\textbf{foc}}\textbf{-}\textsc{3sg}.\textsc{sa}-\textsc{pst}-give-\textsc{appl}.\textsc{prf}-\textsc{fv} 1-teacher 5-book\\
  \glt ‘Who said that Gati gave the teacher a book?’
  \zlast
}% 
$^,$\footnote{See, however, \sectref{sec:landman:5.1}, in which we observe some phrase-internal instances of /ne-/.}

\newpage 
In \textit{wh-}questions and \isi{focus} constructions, that is, constructions that involve a fronted constituent, /ne-/ obligatorily precedes the fronted constituent. For example, in the \textit{wh-}questions in \REF{ex:landmann:2}, /ne-/ precedes the fronted \textit{wh-}phrase; note that here /ne-/ cannot also occur pre-verbally:

\ea\label{ex:landmann:2}
\ea 
\gll *(\textbf{N-})ke (*n-)ge-it-ir-e ege-toocho?\\
     (\textsc{\textbf{foc}}\textbf{-})what (\textsc{foc}-)\textsc{sa}-kill-\textsc{prf}-\textsc{fv} 7-rabbit\\
\glt ‘What killed the rabbits?’

\ex
\gll  *(\textbf{N-})ke ichi-ng’iti (*n-)cha-a-it-ir-e?\\
     (\textsc{\textbf{foc}}\textbf{-})what 10-hyena (\textsc{foc}-)\textsc{10sa}-\textsc{pst}-kill-\textsc{prf}-\textsc{fv}\\
\glt ‘What did the hyenas kill?’
\z
\z

Similarly, in (information) \isi{focus} constructions, e.g. answers to \textit{wh-}questions, as in \REF{ex:landmann:3}, /ne-/ appears on the fronted, focused phrase; here too, /ne-/ cannot also appear pre-verbally:

\ea\label{ex:landmann:3}
\ea 
  \gll   *(\textbf{N-})ichi-ng’iti (*n-)cha-a-it-ir-e ege-toocho.\\
     (\textsc{\textbf{foc}}\textbf{-})10-hyena (\textsc{foc}-)\textsc{10sa}-\textsc{pst}-kill-\textsc{prf}-\textsc{fv} 7-rabbit\\
\glt ‘THE HYENAS killed the rabbit.’
\ex
\gll  *(\textbf{N-})ege-toocho ichi-ng’iti (*n-)cha-a-it-ir-e.\\
     (\textsc{\textbf{foc}}\textbf{-})7-rabbit 10-hyena (\textsc{foc}-)\textsc{10sa}-\textsc{pst}-kill-\textsc{prf}-\textsc{fv}\\
\glt ‘The hyenas killed THE RABBIT.’
\z
\z

Note that alongside (\ref{ex:landmann:3}a), the alternative \isi{word order} in \REF{ex:landmann:4} is also possible, where the subject precedes the constituent marked with /ne-/; as far as we can tell, there is no difference in interpretation or contextual appropriateness between (\ref{ex:landmann:3}a) and \REF{ex:landmann:4}.

\ea\label{ex:landmann:4}
  \gll Ichi-ng’iti *(\textbf{n-})ege-toocho (*n-)cha-a-it-ir-e.\\
     10-hyena (\textsc{\textbf{foc}}\textbf{-})7-rabbit (\textsc{foc}-)\textsc{10sa}-\textsc{pst}-kill-\textsc{prf}-\textsc{fv}\\
\glt ‘The hyenas killed THE RABBIT.’
\z

It is not possible for /ne-/ to appear post-verbally, thus, attempting to \isi{focus} an \textit{in-situ} object is ungrammatical:

\ea
  \gll Ichi-ng’iti cha-a-it-ir-e (*\textbf{n-})ege-toocho.\\
     10-hyena \textsc{10sa}-\textsc{pst}-kill-\textsc{prf}-\textsc{fv} (\textsc{\textbf{foc}}\textbf{-})7-rabbit\\
\glt (Intended meaning: ‘The hyenas killed THE RABBIT.’)
\z

To summarize this section, in declaratives, /ne-/ obligatorily appears pre-verbally, while in \textit{wh-}questions and \isi{focus} constructions, /ne-/ obligatorily precedes the fronted constituent.\footnote{{A reviewer asks whether /ne-/ also appears in copular constructions, as in \ili{Kikuyu} \citep{Schwarz2007}. A nasal morpheme does indeed appear in copular constructions, as in (ii) below, but investigating the distribution of this morpheme and whether it is the same as /ne-/ goes beyond the scope of this paper:}

\ea
\gll Gati \textbf{n-}omo-reri.\\
1.Gati \textsc{\textbf{n}}\textbf{-}1-doctor\\
\glt ‘Gati is a doctor.’
\z

Note also that the negative counterpart to /ne-/, /te-/, which we discuss in \sectref{sec:landmann:6.2}, also appears in copular constructions:

\ea
\gll Gati \textbf{t-}omo-reri.\\
1.Gati \textsc{\textbf{t-}}1-doctor\\
\glt ‘Gati is not a doctor.’
\z
} 
We see the varied distribution of this morpheme as raising two questions for a uniform account of its different uses. First, is /ne-/ truly a \isi{focus} marker? Second, where does /ne-/ originate syntactically? We put forward answers to these questions in the next two sections. Specifically, in \sectref{sec:landmann:3}, we present evidence that /ne-/ semantically is uniformly a \isi{focus} marker, and in \sectref{sec:landman:4}, we show that \citegen{Schwarz2007} syntactic account of \ili{Kikuyu} can be extended to \ili{Kuria}, so that in all cases, /ne-/ heads a Focus Phrase projection in the left periphery of the \isi{clause}.


\section{Diagnosing focus}\label{sec:landmann:3}


In this section, we present evidence that /ne-/ behaves like a \isi{focus} marker, across its different uses. Following the alternative semantics approach to \isi{focus} (\citealt{Rooth1985,Rooth1992}), we assume that ‘\isi{focus} indicates the presence of alternatives that are relevant for the interpretation of linguistic expressions’ \citep{Krifka2008}. This definition encompasses different types of \isi{focus}, e.g. signaling new information, correction, contrast, etc. We elicited data from three types of contexts in which \isi{focus} marking would be expected, and found that /ne-/ consistently marks focused constituents, across syntactic categories. Specifically, following in part \citet{HartmannZimmermann2009}’s work on \isi{focus} marking in \ili{Gùrùntùm}, we looked at the following four \isi{focus} contexts in \ili{Kuria}, all of which involve morphological marking with /ne-/:
(i) question-answer congruence, 
(ii) corrective \isi{focus},
(iii) contrastive \isi{focus}, and 
(iv) association with \isi{focus} sensitive operators such as \ili{Kuria} \textit{bene} ‘only’. In the following subsections, we consider each type of \isi{focus} context in turn.

\subsection{Question-answer congruence}
\largerpage[-2]


In felicitous answers to \textit{wh-}questions, /ne-/ obligatorily appears on the phrase corresponding to the \textit{wh-}phrase. Consider, e.g. the object \textit{wh-}question in \REF{ex:landmann:6}. A felicitous answer to this question is one where the object bears /ne-/, as in \REF{ex:landmann:7}; answers in which /ne-/ appears on the subject, as in \REF{ex:landmann:8}, are grammatical but infelicitous in this context.\footnote{%
    We have conflicting judgments from our speaker regarding whether pre-verbal /ne-/ as in (iv) below is felicitous in contexts where we would expect the object to bear /ne-/. For example, (iv) below is sometimes judged as infelicitous and sometimes as felicitous as an answer to \REF{ex:landmann:6}; however, \REF{ex:landmann:7} is consistently offered by our speaker as the first and best answer to the question in \REF{ex:landmann:6}.

    \ea
      \gll A3: \#Ichi-ng’iti \textbf{n-}cha-a-it-ir-e ege-toocho.\\
    ~ 10-hyena \textsc{\textbf{foc}}\textbf{-}\textsc{10sa}-\textsc{pst}-kill-\textsc{prf}-\textsc{fv} 7-rabbit\\
    \glt ‘The hyenas killed the rabbit.’
    \z
}

\ea\label{ex:landmann:6}
\gll Q: \textbf{N-}ke ichi-ng’iti cha-a-it-ir-e?\\
    ~ \textsc{\textbf{foc}}\textbf{-}what 10-hyena \textsc{10sa}-\textsc{pst}-kill-\textsc{prf}-\textsc{fv}\\
\glt ‘What did the hyenas kill?’
\z

\ea\label{ex:landmann:7}
\gll A1: \textbf{N-}ege-toocho ichi-ng’iti cha-a-it-ir-e.\\
     ~ \textsc{\textbf{foc}}\textbf{-}7-rabbit 10-hyena \textsc{10sa}-\textsc{pst}-kill-\textsc{prf}-\textsc{fv}\\
\glt ‘The hyenas killed THE RABBIT.’
\z

\ea\label{ex:landmann:8}
\gll A2: \#\textbf{N-}ichi-ng’iti cha-a-it-ir-e ege-toocho.\\
     ~ \textsc{\textbf{foc}}\textbf{-}10-hyena \textsc{10sa}-\textsc{pst}-kill-\textsc{prf}-\textsc{fv} 7-rabbit\\
\glt (Intended meaning: ‘They hyenas killed THE RABBIT.’)
\z

 
Consider also subject \textit{wh-}questions and their answers, as in \REF{ex:landmann:7}. A felicitous answer to a subject \textit{wh-}question exhibits a fronted subject DP bearing /ne-/, as in A1; if other phrases bear /ne-/, the answer is infelicitous, as in \REF{ex:landmann:10}.

\ea\label{ex:landmann:9}
    Q: Who ate mangoes?\\
\gll A1: \textbf{N-}omo-onto a-a-rey-e ama-yembe.\\
    ~ \textsc{\textbf{foc}}\textbf{-}1-person \textsc{1sa}-\textsc{pst}-eat.\textsc{prf}-\textsc{fv} 6-mango\\
\glt ‘SOMEONE ate mangoes.’
\z

\ea\label{ex:landmann:10}
\gll A2: \#\textbf{N-}ama-yembe omo-onto a-a-rey-e.\\
    ~ \textsc{\textbf{foc}}\textbf{-}6-mango 1-person \textsc{1sa}-\textsc{pst}-eat.\textsc{prf}-\textsc{fv}\\
\glt (Intened meaning: ‘SOMEONE ate mangoes.’)
\z

\newpage 
/Ne-/ marking in answers to \textit{wh-}questions holds not only for object and subject \textit{wh-}questions, but also for adjuncts, such as PPs and adverbials. For example, \REF{ex:landmann:11} illustrates a focused PP, and \REF{ex:landmann:12} illustrates a focused AdvP.\footnote{%
  PP adjuncts canonically appear at the end of the sentence and manner adverbs are relatively free in their positioning.}\textsuperscript{,}%
\footnote{%
A reviewer asks whether in this case the PP S O V ordering is obligatory. We note that a postverbal object is also possible, as in (v):

    \ea

    Q: Did you see the frog on the table or on the floor?

    \gll A: \textbf{N-}ko-mesa naa-mah-er-e i-kjoora.\\
    ~ \textsc{\textbf{foc}}\textbf{-}on-17.table \textsc{1sg}.\textsc{sa}-\textsc{pst}-see-\textsc{prf}-\textsc{fv} 9-frog\\
    \glt ‘I saw the frog ON THE TABLE.’
    \z
    The PP S O V ordering in \REF{ex:landmann:8} above might be evidence for a TopP below FocP, although this matter needs further investigation.
}

\ea\label{ex:landmann:11}
Q: Where will Gati see the owl?\\
\gll A: \textbf{N-}ko-mesa gati umw-iti a-ra-maah-e.\\
   ~  \textsc{\textbf{foc}}\textbf{-}on-17.table 1.Gati 3-owl \textsc{3sg}.\textsc{sa}-\textsc{fut}-see-\textsc{fv}\\
\glt ‘Gati will see the owl ON THE TABLE.’
\z

\ea\label{ex:landmann:12}
Q: How did Chacha drink the chai?\\
\gll A: \textbf{M-}bongo chacha a-a-nyoy-e i-chaahe.\\
     ~ \textsc{\textbf{foc}}\textbf{-}quickly 1.Chacha \textsc{3sg}.\textsc{sa}-\textsc{pst}-drink.\textsc{prf}-\textsc{fv} 8-chai\\
\glt ‘Chacha drank the chai QUICKLY.’
\z

In answers to VP-oriented \textit{wh-}questions, /ne-/ must occur pre-verbally, as the question-answer pair in \REF{ex:landmann:13}-\REF{ex:landmann:14} shows; we take this to indicate that pre-verbal /ne-/ is also a \isi{focus} marker, in this case marking VP \isi{focus}.\footnote{%
  The example in (vi) below, in which the object bears /ne-/, is occasionally judged by our speaker as a felicitous reply to the VP-oriented question in \REF{ex:landmann:13}; thus, objects appear to project \isi{focus} to VP in \ili{Kuria}, as has been observed for \ili{English} \citep{Selkirk1984}:

  \ea
  \gll ?\textbf{N-}i-chaahe a-a-nyoy-e.\\
  \textsc{\textbf{foc}}\textbf{-}8-chai \textsc{3sg}.\textsc{sa}-\textsc{pst}-drink.\textsc{prf}-\textsc{fv}\\
  \glt ‘He DRANK CHAI.’
  \zlast
}\textsuperscript{,}%
\footnote{
  Note that while fronting the nominalized version of a \isi{verb} is grammatical in \ili{Kuria}, as in (vii) below, this construction is infelicitous as an answer to a VP \textit{wh-}question:

  \ea
  \gll \#\textbf{N-}oko-ria ama-ako b-a-rey-e.\\
  \textsc{\textbf{foc}}\textbf{-}15-eat 6-fruit \textsc{3pl}.\textsc{sa}-\textsc{pst}-eat.\textsc{prf}-\textsc{fv}\\
  \glt (Intended meaning: ‘They ATE FRUITS.’)
  \z
  This contrasts with data reported for \ili{Kikuyu} (see \citealt{Schwarz2007}); see \sectref{sec:landmann:5.2} below.
}

\ea\label{ex:landmann:13}
\gll Q: \textbf{N-}ke gati a-a-korr-e.\\
     ~ \textsc{\textbf{foc}}\textbf{-}what 1.Gati \textsc{3sg}.\textsc{sa}-\textsc{pst}-do.\textsc{prf}-\textsc{fv}\\
\glt ‘What did Gati do?’
\z

\ea\label{ex:landmann:14}
\gll A: \textbf{N-}a-a-nyoy-e i-chaahe.\\
    ~ \textsc{\textbf{foc}}\textbf{-}\textsc{3sg}.\textsc{sa}-\textsc{pst}-drink.\textsc{prf}-\textsc{fv} 8-chai\\
\glt ‘He DRANK CHAI.’
\z

Answers to questions in which sentential \isi{focus} is expected, e.g. answers to questions such as ‘What happened?’, also require pre-verbal /ne-/, as \REF{ex:landmann:15} shows; thus, preverbal /ne-/ also marks sentential \isi{focus}.\footnote{A reviewer notes that this instance of sentential \isi{focus} looks identical to what we called a declarative in \REF{ex:landmann:1}. We wish to emphasize that this is a context where sentential \isi{focus} would be expected, and so we take /ne-/ to be indicating \isi{focus} here.} Attaching /ne-/ to any other constituent, e.g. the object, as in \REF{ex:landmann:16}, would be infelicitous:

\ea\label{ex:landmann:15}
Q: What happened?\\
\gll A1: Gati \textbf{n-}a-a-it-ir-e ama-siisi.\\
    ~ Gati \textsc{\textbf{foc}}\textbf{-}\textsc{3sg}.\textsc{sa}-\textsc{pst}-kill-\textsc{prf}-\textsc{fv} 6-ant\\
\glt ‘Gati killed ants.’
\z

\ea\label{ex:landmann:16}
\gll A2: \#\textbf{N-}ama-siisi gati a-a-it-ir-e.\\
    ~ \textsc{\textbf{foc}}\textbf{-}6-ant 1.Gati \textsc{3sg}.\textsc{sa}-\textsc{pst}-kill-\textsc{prf}-\textsc{fv}\\
\glt (Intended meaning: ‘Gati killed ants.’)
\z

Summarizing this subsection: (i) /ne-/ consistently marks \isi{focus} expressions in answers to \textit{wh-}questions across categories; and (ii) pre-verbal /ne-/ marks VP \isi{focus} as well as sentential \isi{focus}.


\subsection{Corrective focus}



Corrective \isi{focus} contexts are those in which a (focused) phrase serves as a correction to a like phrase already introduced into the discourse. Consider, e.g. the dialogue in \REF{ex:landmann:17} below; in \REF{ex:landmann:18}, speaker B corrects the VP from speaker A’s utterance. In this context, /ne-/ may not occur on any phrase other than the corrected VP, as the infelicitous \REF{ex:landmann:19} shows:\footnote{A reviewer asks if something is focused in \REF{ex:landmann:17}. We assume that the appearance of /ne-/ in this instance shows sentential \isi{focus}.}


\ea\label{ex:landmann:17}
  \gll A: \textbf{M-}ba-a-gurr-i i-nyamu.\\
     ~ \textsc{\textbf{foc}}\textbf{-}\textsc{3pl}.\textsc{sa}-\textsc{pst}-sell.\textsc{prf}-\textsc{fv} 9-cat\\
\glt ‘They sold the cat.’
\z

\ea\label{ex:landmann:18}
\gll B: Aʔa, \textbf{m-}ba-a-gi-sirr-i.\\
    ~ no, \textsc{\textbf{foc-}}\textsc{3pl}.\textsc{sa}-\textsc{pst}-\textsc{9om}-lose.\textsc{prf}-\textsc{fv}\\
\glt ‘No, they LOST it.’
\z

\ea\label{ex:landmann:19}
\gll C: \#Aʔa, \textbf{n-}i-nyamu ba-a-sirr-i.\\
    ~ no, \textsc{\textbf{foc-}}9-cat \textsc{3pl}.\textsc{sa}-\textsc{pst}-lose.\textsc{prf}-\textsc{fv}\\
\glt (Intended meaning: ‘No, they LOST the cat.’)
\z

Compare the dialogue in \REF{ex:landmann:20} and \REF{ex:landmann:21}, where, in \REF{ex:landmann:21}, B corrects the object DP from A’s utterance in \REF{ex:landmann:20}; in this case, the corrected DP object bears /ne-/:\footnote{
As with question-answer congruence, we have conflicting judgments regarding whether pre-verbal /ne-/ is felicitous in object-focused contexts. Thus, (viii) below is occasionally judged as felicitous for corrective \isi{focus} on the object:

\ea
\gll Aʔa, \textbf{n-}cha-a-it-ir-e in-chage.\\
no, \textsc{\textbf{foc-}}\textsc{10sa}-\textsc{pst}-kill-\textsc{prf}-\textsc{fv} 9-zebra\\
\glt ‘No, they killed a ZEBRA.’
\z
}  

\ea\label{ex:landmann:20}
\gll A: Ichi-ng’iti \textbf{n-}cha-a-it-ir-e ege-toocho.\\
    ~ 10-hyena \textsc{\textbf{foc-}}\textsc{10sa}-\textsc{pst}-kill-\textsc{prf}-\textsc{fv} 7-rabbit\\
\glt ‘The hyenas killed the rabbit.’
\z

\ea\label{ex:landmann:21}
\gll B: Aʔa, \textbf{n-}in-chage cha-a-it-ir-e.\\
    ~ no, \textsc{\textbf{foc-}}9-zebra \textsc{10sa}-\textsc{pst}-kill-\textsc{prf}-\textsc{fv}\\
\glt ‘No, they killed the ZEBRA.’
\z

Corrective \isi{focus} contexts thus provide further evidence that phrases bearing /ne-/ are focused.


\subsection{Contrastive focus}



Contrastive \isi{focus} contexts are those in which a phrase is presented in contrast with one or more like phrases already introduced into the discourse. Consider for example \REF{ex:landmann:22}, where the contrastively focused VP in the conjoined \isi{clause} bears /ne-/:\footnote{A reviewer asks how we can tell that /ne-/ in the second conjunct in \REF{ex:landmann:22} marks VP \isi{focus}. The alternative would be to claim that nothing is marking VP \isi{focus} in the second conjunct and it is just another case of sentential \isi{focus} (i.e., a declarative). Given that in this contrastive context we expect VP \isi{focus} in the second conjunct, we conclude that it is /ne-/ that marks \isi{focus}.}


\ea\label{ex:landmann:22}
\gll Chacha n-a-a-gorr-e i-indwi, na gati \textbf{n-}a-a-gurr-i i-indwi.\\
     1.Chacha \textsc{foc}-\textsc{3sg}.\textsc{sa}-\textsc{pst}-buy.\textsc{prf}-\textsc{fv} 9-lion and 1.Gati \textsc{\textbf{foc-}}\textsc{3sg}.\textsc{sa}-\textsc{pst}-sell.\textsc{prf}-\textsc{fv} 9-lion\\
\glt ‘Chacha bought a lion and Gati SOLD a lion.’
\z

Similarly, in \REF{ex:landmann:23}, the contrastively focused DP in the conjoined \isi{clause} bears /ne-/:\footnote{A reviewer asks whether the appearance of /ne-/ in the first conjunct means that it is an instance of \isi{focus}. We assume that all sentences in this tense bear \isi{focus}, and that the first conjunct is an instance of default or sentential \isi{focus}.}

\ea\label{ex:landmann:23}
  \gll Chacha n-a-a-gorr-e i-nyamu, na gati \textbf{n-}i-indwi\textbf{} a-a-gorr-e.\\
     1.Chacha \textsc{foc}-\textsc{3sg}.\textsc{sa}-\textsc{pst}-buy.\textsc{prf}-\textsc{fv} 9-cat and 1.Gati \textsc{\textbf{foc-}}9-lion \textsc{3sg}.\textsc{sa}-\textsc{pst}-buy.\textsc{prf}-\textsc{fv}\\
\glt ‘Chacha bought a cat and Gati bought a LION.’
\z

Since /ne-/ appears on a contrastively focused phrase in these examples, they thus also indicate that /ne-/ is a \isi{focus} marker.


\subsection{Focus sensitive operators}



Finally, we consider \isi{focus} sensitive operators analogous to \ili{English} \textit{only}, which have been shown to associate with phrases bearing \isi{focus} (see e.g. \citealt{Rooth1992,Rooth1996}). In \ili{Kuria}, \isi{focus} sensitive operators such as \textit{bene} `only' associate with phrases bearing /ne-/. Consider e.g. \REF{ex:landmann:24} below, which is felicitous given the context:



\ea\label{ex:landmann:24} 
{Context: \textit{We are discussing the methods used by students to protest dining hall food two weeks ago at school. There was only one method they used to protest and I specify it as such.}}\\
\gll \textbf{M-}ba-a-tan-er-a g-oko-rekeera ama-geena bene.\\
     \textsc{\textbf{foc}}\textbf{-}\textsc{3pl}.\textsc{sa}-\textsc{pst}-protest-\textsc{prf}-\textsc{fv} by-15-throw 6-stone only\\
\glt ‘They only PROTESTED BY THROWING ROCKS.’
\z

\textit{Bene} may also associate with a focused DP, in which case \textit{bene} displays \isi{noun class} agreement morphology, as evidenced by \REF{ex:landmann:25} (which is felicitous in a different context):\footnote{%
  A reviewer asks whether this is the only reading for this sentence or whether any scope interactions exist; this is indeed the only reading available for the sentence.
}

\ea\label{ex:landmann:25}
  \gll \textbf{N-}i-nswi i-nyene ba-a-ta-rey-e.\\
     \textsc{\textbf{foc-}}9-fish 9-only \textsc{3pl}.\textsc{sa}-\textsc{pst}-\textsc{neg}-eat.\textsc{prf}-\textsc{fv}\\
\glt ‘They only didn’t eat FISH.’
\z

That \ili{Kuria} \textit{bene} associates with phrases bearing /ne-/ also indicates that /ne-/ is a \isi{focus} marker.


\subsection{Summary of focus diagnostics}



Summarizing this section, we have shown that /ne-/ appears on just those phrases that are in \isi{focus}, based on four types of \isi{focus} constructions: (i) question-answer congruence; (ii) corrective \isi{focus}; (iii) contrastive \isi{focus}; and (iv) association of phrases bearing /ne-/ with \isi{focus} sensitive operators.

\section{Analysis}\label{sec:landman:4}


Having established that /ne-/ is semantically a \isi{focus} marker, we turn now to its syntax. Extending \citet{Schwarz2007}’s account of \ili{Kikuyu} to \ili{Kuria}, we analyze /ne-/ as heading a Focus Phrase (FocP) projection in the left periphery of CP \citep{Rizzi1997}, as in \REF{ex:landmann:26}.\footnote{A TopP projection is observed in the tree, since we will argue for this position below.}

% \todo{pgf bug}
\ea\label{ex:landmann:26}
\small
{Position of \textit{ne}}\nopagebreak
\\
\begin{forest}
[CP, for tree={nice empty nodes} [] [C'
	  [C] [TopP
	    [] [Top'
	     [Top] [FocP
	       [] [Foc'
	        [Foc\\\textit{ne}-,align=left,base=top] [TP [] [] ]
	       ]
	     ]	  
	  ]
        ]	  
    ]
] 
\end{forest}
\z 

\hspace*{10mm}

Following Schwarz, we derive the different constructions that /ne-/ occurs in via \isi{movement} of phrases bearing /ne-/ into [Spec, FocP].\footnote{We, like \citet{Schwarz2007}, are noncommittal with respect to how /ne-/ combines morphologically with the constituent in [Spec, FocP]. \citet[144]{Schwarz2007} notes two possibilities for \ili{Kikuyu} \textit{ne}: (i) \textit{ne} itself heads FocP, and cliticizes to whatever is in the specificer of FocP, and (ii) the \isi{focus} feature is spelled out phonologically as \textit{ne}, and surfaces to the left edge of whatever occupies [Spec, FocP]. For concreteness, we adopt the first possibility, and position /ne-/ as heading the \isi{focus} phrase in the trees throughout.}  For example, in a sentence with a fronted object, the object raises to [Spec, FocP], as \REF{ex:landmann:27} illustrates.\footnote{A reviewer asks what the trigger for \isi{movement} is in our analysis. While we do not spell out in detail the mechanics of \isi{movement}, our account is compatible with a Minimalist analysis (\citealt{Chomsky2000,Chomsky2004}), whereby A-bar \isi{movement} is the result of an Agree operation between a head bearing an uninterpretable feature (such as [wh], [Foc]) and a Goal which carries a matching feature. Furthermore, when the Probe has the [EPP] property, this property is satisfied via internal merge of the Goal, which merges in the Spec position of the Probe. While extending the Agree operation to account for A-bar \isi{movement} in addition to A \isi{movement} is not uncontroversial—see the discussion in \citealt{Horvath2007} for instance—the present analysis is compatible with accounts that take Agree and [Foc] features to drive syntactic \isi{movement}.} Here, topicalization of the subject is optional.

\ea\label{ex:landmann:27} 
{Object focus}\\{}
[{\textsubscript{CP}} 
 [{\textsubscript{TopP}} 
  [{\textsubscript{FocP}} \textsc{object}{\textsubscript{1}} 
   [{\textsubscript{Foc}}{\textsubscript{'}} 
    [{\textsubscript{Foc}} ne- ]{} 
    [{\textsubscript{TP}} …\textit{t}{\textsubscript{1}}…]{}
   ]
  ]
 ]
]
\z

\newpage 
In the case of sentential \isi{focus}, again following \citet{Schwarz2007}, TP moves up into [Spec, FocP], while the subject is obligatorily topicalized, as in \REF{ex:landmann:28}.\footnote{A reviewer asks whether \isi{movement} of the subject out of the TP, which itself has moved to [Spec, FocP] in \REF{ex:landmann:28} constitutes a violation of \citealt{Rizzi2010}'s Criterial Freezing: “In a criterial configuration, the Criterial Goal is frozen in place.” We believe that it does not, on the following grounds. Rizzi states that “In the criterial configuration, only the element carrying the crucial feature is frozen in place, while the other elements of the phrase pied-piped to the Spec of the Criterial Probe remain available for \isi{movement}, and can be subextracted, if no other syntactic principle is violated…” Although Rizzi does not specifically discuss examples in which a TP moves to the \isi{specifier} of a Criterial Probe, we hypothesize that in these cases the subject is still available for \isi{movement}, as observed in example \REF{ex:landmann:28}.}

\ea\label{ex:landmann:28} 
{Sentential focus}\\{}
     [\textsubscript{CP} [\textsubscript{TopP} \textsc{subject}\textsubscript{2} [\textsubscript{Top}\textsubscript{'} [\textsubscript{FocP} [\textsubscript{TP1} ...\textit{t}\textsubscript{2}...] [\textsubscript{Foc}\textsubscript{'} [\textsubscript{Foc} \textit{ne-} ] [\textsubscript{TP} ...\textit{t}\textsubscript{1}...]]]]]\\
\z

In the case of VP \isi{focus}, VP moves to [Spec, FocP], while the subject is obligatorily topicalized, again following Schwarz:

\ea\label{ex:landmann:29}\label{exx:} 
{VP focus}\\{}
 [{\textsubscript{CP}} [{\textsubscript{TopP}} \textsc{subject}{\textsubscript{2}} [{\textsubscript{Top}}{\textsubscript{'}} [{\textsubscript{FocP}} [{\textsubscript{VP1}}{ ...}\textit{t}{\textsubscript{2}}{...] [}{\textsubscript{Foc}}{\textsubscript{'}} [{\textsubscript{Foc}} ne-{ ] [}{\textsubscript{TP}}{ ...}\textit{t}{\textsubscript{1}}{...]]]]]]}\\
\z
 
Summarizing the analysis, we have extended Schwarz's account of \ili{Kikuyu} to \ili{Kuria}, so that in all of its uses, /ne-/ heads a Focus Phrase in the left periphery of the \isi{clause}.   We turn first to a comparison of /ne-/ to morphological marking of \isi{focus} in other languages (\sectref{sec:landman:5}), and then to further data regarding the distribution of /ne-/ (\sectref{sec:landman:6}).\footnote{{An 
  alternative account of /ne-/ might analyze it as a cleft construction. For example, \citet{Bergvall1987} treats \ili{Kikuyu} /ne-/ as an assertion marker that heads TP, which would correctly account for the position of pre-verbal /ne-/. To account for clauses involving fronted phrases, Bergvall proposes a bi-clausal cleft analysis: The matrix \isi{clause} has a null expletive subject, and /ne-/ appears before a null copula, followed by a \isi{relative clause}. We see this type of account as problematic for \ili{Kuria} on several counts. First, if these sentences are clefts, we might expect that idiomatic readings should not be possible (e.g. in \ili{English} \textit{It was the bucket that he kicked} lacks the idiomatic reading). However, idiomatic readings are preserved in \ili{Kuria}, as the example in (ix) illustrates:}

  \ea
  \gll \textbf{N-}i-bara chonesi a-a-gey-e.\\
  \textsc{\textbf{foc-}}9-wild 1.Johnes \textsc{3sa}-\textsc{pst}-go.\textsc{prf}-\textsc{fv}\\
  \glt ‘Johnes went INTO THE WILD.’ {\textasciitilde} ‘Johnes did something completely unexpected.’
  \z

  Second, as \citet{Schwarz2007} points out, clefts are not expected to occur in multiple \textit{wh-}questions, such as \textit{Who killed what?} However, we do find such data in \ili{Kuria}:

  \ea 
  \gll \textbf{N-}eng’we a-it-ir-e (*n-)ke?\\
  \textsc{\textbf{foc-}}who \textsc{sa}-kill-\textsc{prf}-\textsc{fv} \textsc{foc}\textsc{-}what\\
  \glt ‘Who killed what?’
  \z

  Third, /ne-/ appears in non-assertive contexts, e.g. yes/no questions, which differ tonally from declaratives but are otherwise identical (see \citealt{Mwita2008} for a discussion of \ili{Kuria} tonology).  
}\textsuperscript{,}\footnote{A reviewer 
  asks whether an analysis along the lines of \citet{Horvath2007}, according to which an exhaustivity operator merges with some phrase and then is attracted by a higher head, would also account for the \ili{Kuria} pattern. While the full consequences of extending Horvath's analysis to \ili{Kuria} is an issue for future research (especially with regards to phrase-internal \isi{focus} marking; see \sectref{sec:landman:5.1}), we do not see how Horvath's account can be extended to \ili{Kuria} for the following reason: Horvath justifies the exhaustivity operator based on the empirical observation that information \isi{focus} remains in-situ in \ili{Hungarian}, while contrastive/identificational \isi{focus} undergoes overt \isi{movement}. Since no such asymmetry exists in \ili{Kuria} (i.e., both types of \isi{focus} are expressed in the same way, with /ne-/), we do not see that positing an exhaustivity operator is justified for \ili{Kuria}.}


 
\section{Crosslinguistic comparison}\label{sec:landman:5}


In this section, we compare \isi{focus} marking in \ili{Kuria} to \isi{focus} marking in other languages, thereby placing the \ili{Kuria} data in a crosslinguistic context.

 
\subsection{Phrase-internal focus}\label{sec:landman:5.1}



In contrast with related Bantu languages (\citealt{AbelsMuriungi2008} on \ili{Kiitharaka}) and other language families (\citealt{HartmannZimmermann2009} for \ili{Gùrùntùm}), \ili{Kuria} permits phrase-internal \isi{focus} marking. Consider first the following examples, which show that /ne-/ can appear on the head \isi{noun} in a fronted DP, \REF{ex:landmann:30}, or on the determiner, \REF{ex:landmann:31}:


 
\ea\label{ex:landmann:30}
   Q: Which owls did Johnes see, these owls or those owls?\\
\gll A: \textbf{N-}imi-iti ge-no chonesi a-a-roch-e.\\
     ~ \textsc{\textbf{foc-}}4-owl 4-this 1.Johnes \textsc{3sg}.\textsc{sa}-\textsc{pst}-see.\textsc{prf}-\textsc{fv}\\
\glt ‘Johnes saw THESE OWLS.’
\z

\ea\label{ex:landmann:31}
    Q: Which owls did Johnes see, these owls or those owls?\\
\gll A: Imi-iti \textbf{n-}ge-no chonesi a-a-roch-e.\\
    ~ 4-owl \textsc{\textbf{foc-}}4-this 1.Johnes \textsc{3sg}.\textsc{sa}-\textsc{pst}-see.\textsc{prf}-\textsc{fv}\\
\glt ‘Johnes saw THESE OWLS.’
\z

In contrast with \REF{ex:landmann:30} and \REF{ex:landmann:31}, if the determiner is kept constant and the nouns in the two possible answers provided by the question differ, marking the determiner with /ne-/ becomes infelicitous:

\ea\label{ex:landmann:32}
 Q: What did the children like, this lion or this rhinoceros?\\
\gll A1: \textbf{N-}i-huuburia e-no b-a-tanch-er-e.\\
    ~ \textsc{\textbf{foc-}}9-rhino 9-this \textsc{3pl}.\textsc{sa}-\textsc{pst}-like-\textsc{prf}-\textsc{fv}\\
\glt ‘They liked this RHINOCEROS.’
\z

\ea\label{ex:landmann:33}
\gll A2: \#I-huuburia \textbf{n-}e-no ba-a-tanch-er-e.\\
   ~ 9-rhino \textsc{\textbf{foc-}}9-this \textsc{3pl}.\textsc{sa}-\textsc{pst}-like-\textsc{prf}-\textsc{fv}\\
\glt (Intended meaning: ‘They liked this RHINOCEROS.’)
\z

As a reply to questions that require a possessive DP as an answer, /ne-/ may also appear phrase internally. Possessive DPs exhibit the following surface structure:

\ea\label{ex:landmann:34} 
{[Possessed DP Associative Marker Possessor DP]}\\
\z

The morpheme can attach to the possessed DP on the left edge of the phrase, or on the associative marker. Consider first \REF{ex:landmann:35} below, where a question offers two alternative answers in which the possessor DP is kept constant. Attaching /ne-/ to the left-edge of the fronted phrase is the only felicitous reply in this context:

\ea\label{ex:landmann:35}
Q: What did Boke eat, Gati’s ugali or Gati’s chapati?\\
\gll A1: \textbf{N-}iri-chabati re gati a-a-rey-e.\\
    ~ \textsc{\textbf{foc-}}5-chapati 5.\textsc{assoc} 1.Gati \textsc{3sg}.\textsc{sa}-\textsc{pst}-eat.\textsc{prf}-\textsc{fv}\\
\glt ‘He ate Gati’s CHAPATI.’
\z

\ea\label{ex:landmann:36}
\gll A2: \#Iri-chabati \textbf{ne-}re gati a-a-rey-e.\\
    ~ 5-chapati \textsc{\textbf{foc-}}5.\textsc{assoc} 1.Gati \textsc{3sg}.\textsc{sa}-\textsc{pst}-eat.\textsc{prf}-\textsc{fv}\\
\glt (Intended meaning: ‘He ate Gati’s CHAPATI.’)
\z

However, compare \REF{ex:landmann:35} with \REF{ex:landmann:37} below. Here, the question asks for the identity of the possessor. As the second answer, \REF{ex:landmann:38}, shows, only attaching /ne-/ to the associative marker is felicitous in this context:

\ea\label{ex:landmann:37}
Q: Whose ugali did Sammy eat?\\
\gll A1: \#\textbf{N-}ubu-kima bo gati a-a-rey-e.\\
    ~ \textsc{\textbf{foc-}}14-ugali 14.\textsc{assoc} 1.Gati \textsc{3sg}.\textsc{sa}-\textsc{pst}-eat.\textsc{prf}-\textsc{fv}\\
\glt (Intended meaning: ‘He ate GATI’s ugali.’)
\z

\ea\label{ex:landmann:38}
\gll A2: Ubu-kima \textbf{n-}obo gati a-a-rey-e.\\
    ~ 14-ugali \textsc{\textbf{foc-}}14.\textsc{assoc} 1.Gati \textsc{3sg}.\textsc{sa}-\textsc{pst}-eat.\textsc{prf}-\textsc{fv}\\
\glt ‘He ate GATI’S ugali.’
\z

As a reply to the question in \REF{ex:landmann:37} above, we might have expected that marking the possessor DP with /ne-/ would have been the only felicitous reply. However, it is ungrammatical (in any context) to mark the possessor DP with /ne-/, as \REF{ex:landmann:39} shows:\footnote{The ban on this type of construction was also noted for \ili{Kiitharaka} by \citet{AbelsMuriungi2008}. We hypothesize that the correct analysis of the surface structure of a possessive DP is one in which the associative marker and the possessor DP form a single morphophonological unit that cannot be broken up by the /ne-/ morpheme. Therefore, the ban against marking the possessor DP with /ne-/ might not be related to semantics at all.}

\ea\label{ex:landmann:39}
\gll *Ubu-kima bo \textbf{n-}gati a-a-rey-e.\\
     14-ugali 14.\textsc{assoc} \textsc{\textbf{foc-}}1.Gati \textsc{3sg}.\textsc{sa}-\textsc{pst}-eat.\textsc{prf}-\textsc{fv}\\
\glt (Intended meaning: ‘He ate GATI’s ugali.’)
\z

\largerpage
Based on these examples, we wish to highlight the fact that /ne-/ may attach to different subconstituents of a focused phrase, depending on the context. Therefore, a potential analysis treating the appearance of /ne-/ on different subconstituents of a focused phrase to a semantically vacuous morphological operation would be undesirable. For now, we leave the mechanics by which the \isi{focus} marker appears phrase internally for future research, although we can shed doubt on /ne-/ being base-generated phrase internally given examples like \REF{ex:landmann:40} below, which show that marking /ne-/ on both the possessed DP and the associative phrase is ungrammatical:

\ea\label{ex:landmann:40}
\gll *\textbf{N-}ubu-kima n-obo gati a-a-rey-e.\\
     \textsc{\textbf{foc-}}14-ugali \textsc{foc}-14.\textsc{assoc} 1.Gati \textsc{3sg}.\textsc{sa}-\textsc{pst}-eat.\textsc{prf}-\textsc{fv}\\
\glt (Intended meaning: ‘He ate Gati’s ugali.’)
\z


\subsection{VP fronting}\label{sec:landmann:5.2}



As reported by \citet{Schwarz2007}, VP \isi{focus} in \ili{Kikuyu} involves fronting an \isi{infinitival} \isi{verb}, while a fully inflected form remains in base position (see his example 15B). This construction is also possible in \ili{Kuria}, but does not trigger VP \isi{focus}. Instead, fronting an \isi{infinitival} \isi{verb} is felicitous only as a reply to a question expressing incredulity. Therefore, this construction can only be used to express a \textit{verum} interpretation. Consider first the following question-answer pairing, in which marking the VP with /ne-/ triggers VP \isi{focus}, as expected:


\ea\label{ex:landmann:41}
 Q: What did they do, eat fruits or drink water?\\
\gll A1: \textbf{M-}ba-a-rey-e ama-ako.\\
    ~ \textsc{\textbf{foc-}}\textsc{3pl}.\textsc{sa}-\textsc{pst}-eat.\textsc{prf}-\textsc{fv} 6-fruit\\
\glt ‘They ATE FRUITS.’
\z

\ea\label{ex:landmann:42}
\gll A2: \#\textbf{N-}oko-ria ama-ako ba-a-rey-e.\\
    ~ \textsc{\textbf{foc-}}15-eat 6-fruit \textsc{3pl}.\textsc{sa}-\textsc{pst}-eat-\textsc{prf}-\textsc{fv}\\
\glt (Intended meaning: ‘They ATE FRUITS.’)
\z

However, as a reply to the follow-up question observed below, only example \REF{ex:landmann:43} is felicitous:

\ea\label{ex:landmann:43}
Q: Did they really?\\
\gll A: E, \textbf{n-}oko-ria ama-ako ba-a-rey-e.\\
     ~ yes, \textsc{\textbf{foc-}}15-eat 6-fruit \textsc{3pl}.\textsc{sa}-\textsc{pst}-eat.\textsc{prf}-\textsc{fv}\\
\glt ‘Yes, they DID eat fruits.’
\z

 
These data indicate that in \ili{Kuria}, \textit{verum} \isi{focus} in the remote past is marked via fronting of an \isi{infinitival} \isi{verb} and \isi{predicate} doubling. Further, these data show that \ili{Kuria} differs from \ili{Kikuyu} with respect to the use of preverbal /ne-/ and the expression of \textit{verum} \isi{focus}: In \ili{Kikuyu}, preverbal /ne-/ is used for \textit{verum} \isi{focus}, while fronting the \isi{infinitival} \isi{verb} and \isi{predicate} doubling is akin to the use of preverbal /ne-/ in \ili{Kuria} (what \citealt{Schwarz2007} calls “narrow \isi{focus} on the \isi{verb}”). Although we leave an analysis of constructions such as \REF{ex:landmann:31} for future research, we note the typological difference between \ili{Kikuyu} and \ili{Kuria} with regards to the realization of VP and \textit{verum} \isi{focus}.\footnote{The examples in \REF{ex:landmann:42} and \REF{ex:landmann:43} are parallel to data from other languages discussed in \citet{Aboh2006} and  \citet{AbohDyakanova2009}, where they are called “\isi{predicate} fronting with doubling”. We refer the reader to these papers for further examples of this type of construction crosslinguistically and thank an anonymous reviewer for bringing these papers to our attention.}


 
\section{Additional patterns regarding /ne-/}\label{sec:landman:6}


In this section, we document some additional patterns regarding the syntactic distribution of /ne-/, and in doing so articulate some puzzles for future research with respect to \isi{focus} marking in \ili{Kuria}. Specifically, in \sectref{sec:landmann:6.1}, we observe certain contexts which appear to involve \isi{focus}, but lack /ne-/, and in \sectref{sec:landmann:6.2}, we discuss the \isi{complementary distribution} between /ne-/ and the negative marker /te-/.

 
\subsection{Focus but no /ne-/}\label{sec:landmann:6.1}



We have identified at least three contexts in which a phrase is semantically focused, while /ne-/ is absent, which we discuss in the following subsections.

\subsubsection{TAM}


In certain TAM combinations, /ne-/ does not appear pre-verbally in certain contexts (as also observed by \citealt{Mwita2008}). Consider, e.g. the question-answer pairing in the immediate past below (cf. the remote past, in which /ne-/ does appear, as in, e.g. \REF{ex:landmann:1}):


\ea\label{ex:landmann:44}
Q: What did Gati and Johnes just do?\\
\gll A: Ba-rey-e omo-gate.\\
     ~ \textsc{3pl}.\textsc{sa}-eat.\textsc{prf}-\textsc{fv} 3-bread\\
\glt ‘They just ATE BREAD.’
\z

The morpheme may only appear pre-verbally in this tense as a reply to questions showing incredulity, expressing a \textit{verum} \isi{focus} interpretation. This context, similar to the previously discussed VP \isi{nominalization} cases in the remote past, is exemplified below in \REF{ex:landmann:45}:

\ea\label{ex:landmann:45} 
{Follow up Q: \textit{Did they really?}}\\
\gll A: \textbf{M-}ba-rey-e omo-gate.\\
     ~ \textsc{\textbf{foc-}}\textsc{3pl}.\textsc{sa}-eat-\textsc{prf}.\textsc{fv} 3-bread\\
\glt ‘They DID just eat bread.’
\z

We presently have no explanation for why /ne-/ only appears in \textit{verum} \isi{focus} contexts in certain TAM combinations, as illustrated above, and why \textit{verum} \isi{focus} in certain TAM combinations is not expressed via fronting of the \isi{infinitival} \isi{verb} and doubling.


\subsubsection{Focus in relative clauses} 
Relative clauses provide another context in which /ne-/ does not appear pre-verbally. Even in answer to an echo question context that forces a \isi{focus} interpretation on the \isi{verb}, /ne-/ may not appear, as the question-answer pair in \REF{ex:landmann:46} shows:


\ea\label{ex:landmann:46}
Q: The teacher who did WHAT drank water?\\
\gll A: Umw-arimu ora (*\textbf{n-})a-a-giy-er-e.\\
     ~ 1-teacher 1.who (\textsc{\textbf{foc-}})\textsc{3sg}.\textsc{sa}-\textsc{pst}-march-\textsc{prf}-\textsc{fv}\\
\glt ‘The teacher who MARCHED.’
\z

We might account for the restriction on /ne-/ in this context to the absence of a FocP projection in the left periphery of relative clauses. However, it remains a puzzle how phrases with no morphological marking are interpreted as being focused. We leave this issue for future research.\footnote{A reviewer notes that a fuller paradigm related to the unavailability of /ne-/ in relative clauses might show this is due to island effects.}


 \subsubsection{Multiple \textit{wh-}phrases}


Since we assume that congruent answers to \textit{wh-}questions involve semantic \isi{focus}, we would expect all replies to \textit{wh-}questions to bear /ne-/; this, however, is not the case. Consider the multiple \textit{wh-}question-answer pairing below; notice that only the phrase that answers the \textit{wh-}phrase marked with /ne-/ in the question can bear /ne-/ in the answer; the in-situ constituent cannot also bear /ne-/:


\ea\label{ex:landmann:47}
 \gll Q: \textbf{N-}eng’we a-it-ir-e (*n-)ke?\\
    ~ \textsc{\textbf{foc-}}who \textsc{sa}-kill-\textsc{prf}-fv (\textsc{foc}-)what\\
\glt ‘Who killed what?’
\z

\ea\label{ex:landmann:48}
\gll A1: \textbf{M-}boke a-it-ir-e igi-siisi.\\
    ~ \textsc{\textbf{foc-}}1.Boke \textsc{3sg}.\textsc{sa}-kill-\textsc{prf}-\textsc{fv} 7-ant\\
\glt ‘Boke killed ANTS.’
\z

\ea\label{ex:landmann:49}
\gll A2: \#\textbf{N-}igi-siisi boke a-it-ir-e.\\
    ~ \textsc{\textbf{foc-}}7-ant 1.Boke \textsc{3sg}.\textsc{sa}-kill-\textsc{prf}-\textsc{fv}\\
\glt (Intended meaning: ‘BOKE killed ANTS.’)
\z

Examples like these again raise the question of how semantic \isi{focus} is expressed and interpreted in \ili{Kuria}; if structural restrictions ban the appearance of multiple instances of /ne-/ in a \isi{clause}, how are phrases with no morphological marking interpreted as being focused? We leave further discussion of this matter for future research.

 
\subsection{The complementary distribution of /ne-/ and /te-/} \label{sec:landmann:6.2}
While /ne-/ appears in positive sentences, its apparent negative counterpart, /te-/, appears in exactly the same position in negative sentences:


\ea\label{ex:landmann:50}
 \gll Aba-saacha \textbf{te-}ba-a-mah-er-e eng’-ombe.\\
     2-man \textsc{\textbf{neg.foc-}}\textsc{3pl}.\textsc{sa}-\textsc{pst}-see-\textsc{prf}-\textsc{fv} 9-cow\\
\glt ‘The men did not SEE THE COW.’
\z

The two are in \isi{complementary distribution}, never co-occurring on the same phrase or in the same \isi{clause}. /Te-/ appears to express both \isi{negation} as well as \isi{focus}. Like /ne-/, /te-/ attaches to focused phrases, such as the question-answer pairing below:

\ea\label{ex:landmann:51}
  Q: Was it the men who saw the cow?\\
\gll A: A:, *(\textbf{t-})aba-saacha (*m-)ba-a-mah-er-e eng’-ombe.\\
    ~ no, \textsc{\textbf{neg.foc-}}2-man (\textsc{foc}-)\textsc{3pl}.\textsc{sa}-\textsc{pst}-see-\textsc{prf}-\textsc{fv} 9-cow\\
\glt ‘It was not THE MEN who saw the cow.’
\z

Also like /ne-/, \isi{focus} sensitive semantic operators such as \textit{bene} ‘only’ may associate with /te-/. The example below is felicitous if the speaker is commenting on the methods used by student protesters:

\ea\label{ex:landmann:52}
  \gll \textbf{Te-}ba-a-giy-ir-e bene.\\
     \textsc{\textbf{neg-foc-}}\textsc{3pl}.\textsc{sa}-\textsc{pst}-march-\textsc{prf}-\textsc{fv} only\\
\glt ‘They did not only MARCH.’
\z

The \isi{complementary distribution} of /ne-/ and /te-/ in \ili{Kuria} suggests to us that the two are closely related in their syntax and semantics, though we do not presently have an explanation for this connection. \citet{Mwita2008} attributes the incompatibility of /ne-/ with /te-/ to \isi{negation} itself being “inherently focused”. However, this approach would not account for the fact that /ne-/ can co-occur with a different \isi{negation} marker, /ta-/, which immediately precedes the \isi{verb} \isi{root} in contexts such as the following:\footnote{/Te-/ and /ta-/ may not both appear on the \isi{verb} stem.}

\ea\label{ex:landmann:53}
  Q: What won’t they do?\\
  \gll A: \textbf{M-}ba-taa-r-e ege-eki.\\
    ~ \textsc{\textbf{foc-}}\textsc{3pl}.\textsc{sa}-\textsc{neg.fut}-eat-\textsc{fv} 5-cake\\
\glt ‘They will not EAT THE CAKE.’
\z

A connection between the morphological marking of \isi{focus} and \isi{negation} has not, to our knowledge, been studied in detail in the Bantu literature on \isi{focus} realization. While \citet{Schwarz2007} and \citet{AbelsMuriungi2008} acknowledge the incompatibility of the \isi{focus} morpheme with a \isi{negation} morpheme in \ili{Kikuyu} and \ili{Kiitharaka}, respectively, they do not propose an account of the pattern.\footnote{A similar connection between \isi{focus} and \isi{negation} manifests in \ili{English} do-support, suggesting a cross-linguistic connection between \isi{focus} and \isi{negation}. E.g. in \ili{English}, \textit{do-}support is obligatory in negated sentences (e.g. \textit{They *(did) not leave}) as well as \textit{verum} \isi{focus} sentences (e.g. \textit{They *(DID) leave}).} We leave the issue for future research as well.


 \section{Conclusion}


In this paper, we have presented a range of novel data regarding the syntax and semantics of the \ili{Kuria} morpheme /ne-/. We have argued that our data support analyzing the morpheme /ne-/ as a \isi{focus} marker, and we have shown that an analysis in the spirit of \citet{Schwarz2007} captures the syntactic distribution of /ne-/. This paper thus contributes to our understanding of \isi{focus} strategies in Bantu specifically, as well as across languages more generally. Furthermore, we have articulated the following puzzles regarding the realization of \isi{focus}, not only for \ili{Kuria} but across the Bantu family. First, we have documented patterns that involve phrase-internal \isi{focus} marking, which have not previously been reported in the Bantu literature. Second, we have shown how VP \isi{focus} differs in \ili{Kuria} in comparison with other Bantu languages like \ili{Kikuyu}. Finally, we have observed the \isi{complementary distribution} of \isi{focus} and \isi{negation} in \ili{Kuria}, a connection that may have implications for the analysis of \isi{focus} in \ili{Kuria} specifically, as well as \isi{focus} across languages more generally.

\section*{Acknowledgements}

Authors are listed alphabetically. Special thanks to Johnes Kitololo for his judgments and patience, to Michael Diercks for extensive input and advice, and to \ili{Adele} Eslinger, whose paper on \ili{Kuria} /ne-/ for the Spring 2013 Field Methods course at Pomona College, taught by Michael Diercks, provided a starting point for this work. Thanks also to Mary Paster, Jesse Harris, the participants of the 2014 mini-symposium on African languages at Pomona, the participants of ACAL 2014, and two anonymous reviewers for many helpful questions and comments. This paper builds on the second author’s undergraduate thesis at Pomona College. Any mistakes are our responsibility.

\section*{Abbreviations}
Glosses are as follows:

\bigskip
\begin{tabularx}{.9\textwidth}{lQlQ}
 \textsc{foc} &  {focus} &       \textsc{sa} &  {subject agreement}\\
 \textsc{pst} & past &        \textsc{neg} &  {negation}\\
 \textsc{prf} &  {perfective} &  \textsc{comp} &  {complementizer}\\
 \textsc{fut} & future &      \textsc{assoc} & associative marker\\
 \textsc{fv} & final vowel &  \textsc{appl} &  {applicative}.
\end{tabularx}
\bigskip

In the orthographic conventions used throughout, an intervocalic <b> represents a voiced bilabial \isi{fricative} [β], an intervocalic <g> a voiced \isi{velar} \isi{fricative} [ɣ],  <ng’> a \isi{velar} nasal [ŋ], <ny> a \isi{palatal} nasal [ɲ], <y> a \isi{palatal} glide [j], <r> an \isi{alveolar} tap [ɾ], <rr> a voiced \isi{alveolar} trill [r], and <ch> a voiced alveo-\isi{palatal} \isi{affricate} [ʧ]. Numbers indicate Bantu \isi{noun class}. For ease of comprehension, when one morpheme’s presence is to be noticed in an utterance, it appears in bold; when two morphemes are to be noticed, the second one is underlined. We do not transcribe tone in our data; see \citet{Mwita2008} and \citet{MarloEtAl2014kuriamelody,MarloEtAl2014kuriaH} for discussion of \ili{Kuria} tone.

  
{\sloppy
\printbibliography[heading=subbibliography,notkeyword=this]
}
\end{document}