\documentclass[output=paper,
modfonts
]{langscibook} 
% \bibliography{localbibliography}
\ChapterDOI{10.5281/zenodo.1251718}

  
\title{Linguistic imperialism and language decolonisation in Africa through documentation and preservation} 
\shorttitlerunninghead{Linguistic imperialism and language decolonisation in Africa}
\author{Kofi Agyekum \affiliation{The University of Kansas}}
 

\abstract{This paper addresses the politics of language use in African nations and societies. It highlights the role of power and economics in the choice of language. It discusses linguistic imperialism and language shift, and how they lead to language endangerment. The paper also discusses linguistic decolonization whereby societies resist linguistic domination and endangerment and embark on language maintenance. It touches on the methods employed in language decolonisation, namely language revitalisation, resistance, maintenance, documentation and preservation. Attention will be on lexicology, terminology and the role of radio and TV. We argue that as a society tries to redeem itself from linguistic imperialism through decolonisation, certain stronger politico-economic factors push it back into linguistic imperialism. We will find out that some of the indigenous people themselves kick against language decolonisation. The paper hinges on the theoretical base of language endangerment. Examples are taken from African and Ghanaian languages with emphasis on Akan.}

\begin{document}
\maketitle
 

% \begin{abstract}
% Key terms: Language shift, \isi{endangerment}, \isi{linguistic imperialism}, language \isi{decolonisation}, documentation.
% \end{abstract}

\section{Introduction and definition of linguistic imperialism}

According to \citet[780]{Phillipson2009} “\isi{linguistic imperialism} is the notion that certain languages dominate internationally on others. It is the way nation-states privileged one language, and often sought actively to eradicate others, forcing their speakers to shift to the \isi{dominant language}.” The expressions, \textit{actively} eradicating and \textit{forcing} speakers are too strong because despite the policies employed by the colonial masters in their former colonies, they cannot be said to be actively seeking to eradicate African languages. According to \citet[158]{Mous2003} “Contrary to the situation in \ili{Latin} America, in Africa, colonial languages are not a \isi{major} factor in language loss.” In the context of this paper, our working definition and conception of \isi{linguistic imperialism} is “a linguistic situation where the indigenous people are gradually conscientised to shun their indigenous languages and adopt foreign languages because of the benefits they expect to derive from them. They are made to believe that their languages cannot be used in any transaction in education, economics, science and technology but instead a foreign language is the best.

In \isi{linguistic imperialism}, there is a greater relationship between political and economic dependence. Even though many African and Asian countries have won political independence, there is still economic reliance and dependence on the developed and industrial world. This has culminated in linguistic and economic neo-colonialism. The European languages are still the dominant languages in science and technology, medicine, engineering and ICT. The \isi{linguistic dominance} of the \ili{English} language in science and technology has marginalized other languages in this field.

\citet[780]{Phillipson2009} further argues that “\isi{linguistic imperialism} entails unequal resource allocation and communicative rights between people, defined in terms of their competence in specific languages, with unequal benefits as a result.” Linguistic imperialism has deepened in African countries to the extent that many families in the cities communicate with their children in \ili{English}, \ili{French}, \ili{Portuguese} and \ili{Arabic}, and the children cannot speak their mother tongue. People bear western names and have shelved their African names. Attitudes towards African languages in school and at home are very negative, and people are ashamed of speaking their languages.

Linguistic imperialism shows that language is one of the manifestations of European imperial expansion in Africa, Asia and \ili{Latin} America. After the European Struggle and Partition of Africa, new languages of communication emerged patterned on the colonial masters’ languages. Linguistic imperialistic ideologies glorify the dominant languages like \ili{Arabic}, \ili{French}, Hebrew, \ili{Latin}, \ili{English}, as the languages of religion and civilization.\footnote{\citet[782]{Phillipson2009} posits that \isi{linguistic imperialism} is the maintenance of injustice and inequality by means of language policies, and it is strongly connected to policies in commerce, science, international affairs, education, culture and the media. }

% \section{\biberror{title missing}}
%2.1.
\section{Language shift and language endangerment}

This section discusses the theoretical notions of \isi{language shift} and \isi{language endangerment}. In \isi{language contact}, \isi{language imperialism} leads to \isi{language shift}, whereby a community of speakers effectively abandons one language and shifts to another \citep[63]{Garret2006}.\footnote{\citet[83]{O’Shannessy2011} states that “in \isi{language shift}, members of the community stop speaking the pre-contact language habitually and mostly speak the post-contact language, which comes to be the language of the next generation.” }{ }Language shift is the gradual replacement of the communicative functions of one language by another which the user considers to serve the maximum linguistic, political and social benefits (see \citealt{Agyekum2009}). The basic determining factor in \isi{language shift} relates to particular benefits to be derived from the use of the target language; especially economic benefits (see \citealt{Mufwene2006}, \citeyear[175]{Mufwene2002}, \citealt[160]{Mous2003}). \citet[116]{Mufwene2006} states that “\ili{English} is spreading around the world because there are more and more people who hope to find better jobs, to travel, to distant places, to be read by more and more scholars, etc.” In \isi{language shift}, the functions of the \isi{minority language} diminishes in relation to the more “prestigious and politically powerful language” whose functions become expanded to cater for previous functions of the minor language. Speakers find the dominant languages more advantageous in all spheres of their lives. We can say that \isi{language shift} comes from two sources (1) \isi{linguistic imperialism} involving forcing or conscientisation, and (2) speakers making the change from the \isi{indigenous language} to another language whether a foreign or a \isi{major} \isi{indigenous language} for socio-economic benefits.

\citet[213--214]{Pandharipande2002} adopts the term \textit{functional load} and asserts that the higher the \isi{functional load}, the more powerful the language is perceived to be. An increase in the \isi{functional load} implies the promotion and sustenance of the \isi{dominant language} (colonial or national language) while the trend of “linguistic diminishing returns” in the \isi{functional load} is serious. An uncontrolled \isi{language shift} can lead to \isi{language decay}, \isi{endangerment}, extinction and death or attrition. \citet[151]{Derhemi2002a} avers that “An \isi{endangered language} is a language that may soon vanish, ceasing to be used as a vehicle of communication, perhaps even disappearing completely from human history.” In \isi{Ghana}, the Togo-Mountain languages can be considered as examples of endangered languages.

%2.2.
% \subsection{Factors that contribute to language shift and endangerment }

Language shift and death are engineered by globalization, language ideologies, language policies, language attitudes and everyday communicative practices in bilingual or multilingual communities. Theoretically, communities that have positive attitudes about their own language and community try to embark on \isi{language revitalisation} and maintenance. In looking at the positive attitude towards \isi{language survival} and maintenance \citet[307]{Adegbija2001} posits that “ As long as speakers of a language have a deep stake in its survival and a high emotional involvement and commitment to its existence, all the \isi{language shift} agents and triggers in this world will not be able to kill their resolve.” By contrast, negative attitude to one’s language results in \isi{language shift}, culminating into \isi{language death}. Minority languages can die when their speakers are indifferent and lack the effective interventional means to protect them (see \citealt[249]{Derhemi2002a}). 

The other \isi{major} causes of \isi{language shift} and \isi{endangerment} are social status of the language, educational and cultural factors, power relations of the languages that are in contact, economic, migratory and settlement patterns, exogamous marriages, religion, language policies and globalisation \citep[5--6]{Austin2011}.


% \subsection{\biberror{title missing}}
%2.3.1.
\subsection{Language shift based on globalisation and international language policies}

Some language shifts are caused by globalisation while some are based on language policies coming from colonial masters or national governments.

%2.3.2.
\subsubsection{Globalisation}

Globalisation is one of the key factors of \isi{language endangerment} (see \citealt[21]{Austin2011}). In \isi{language shift} within the context of globalisation, very few languages benefit from international trade and international exchange. \citet[446]{Brenzinger2009} asserts that “The international exchange of knowledge and world trade are conducted in only a few world languages, and some scholars, seem to expect that, given this tendency, a world culture, based on one common language will finally emerge from these development.” In the language of globalisation, the selected languages have wider communication and they are also easily accessed (see \citealt[5]{Bamgbose2011}).

Some imperial languages are used as supra-regional or global languages of communication in international organisations like UN, Council of Europe, NATO, EU, AU, and ECOWAS. Imperial languages are used in negotiations between two or more states, languages of diplomacy, international agreements and pacts. \ili{English} is increasingly becoming the world’s international \isi{lingua franca}.\footnote{The integration of the new European Union member states accelerates the spread of \ili{English} as the European \isi{lingua franca}. }{ }\citet[160]{Mous2003} records that “In \isi{Nigeria} there is the rebirth of Nigerian Pidgin \ili{English} as a consequence of globalisation. Other examples of language rebirth under globalisation in Africa are cases of urban slang losing its stigma and serving the function of bridging ethnicity and taking over all communication situations in the big city.” Examples are Iscamto in Johannesburg, South Africa, and the use of Ghanaian Pidgin \ili{English} in secondary schools, universities and public work places in \isi{Ghana}. 

Linguistic imperialism and education, and the expansion of the \ili{English} language have provided a market and economic force for the teaching of \ili{English} as a \isi{second language} for foreigners. The \ili{English} Language Teaching and Learning business is a \isi{major} pillar for the British economy.{ }\footnote{\citet[781]{Phillipson2009} records that the British economy benefits by £11billion directly and a further £11billion indirectly from their intake of foreign students in Higher education. He continues that over half a million foreign students attend language schools in Britain each year, spending time in learning \ili{English}.} \ili{English} language has become a global commodity sold in the international market. The socio-economic factors in \isi{language shift} to a language like \ili{English} for economic integration, cultural shifts, international networks and employment makes language a more marketable commodity than a marker of identity (see \citealt[21]{Austin2011}). At the University of \isi{Ghana}, students from Equatorial \isi{Guinea} are sponsored heavily by their government to learn \ili{English} at the Language Centre. At the department of Linguistics, Legon, a full-fee Sandwich Masters programme in TESL is oversubscribed. The situation is different for an MA sandwich in Ghanaian Languages at the University of Education Winneba. Very few students are ready to pay full-fees to study Ghanaian languages. The Presbyterian University of \isi{Ghana} has a Ghanaian languages department but has not been able to enroll a single student to pay full fees. 

%2.3.3.
\subsubsection{Language policies }

In some cases legislation is passed declaring that a language is the only one that individuals are allowed to use in both public and non-public domains (see \citealt[677]{Lewis2013}). Some aspects of \isi{language shift} deals with language of official communication. Most African countries use the former colonial masters’ languages as the official languages in national administration, governance, education, legislation, judiciary, media, etc. Politically, very few of the masses in African countries speak the colonial languages; it is limited to the few elites. For instance, in \isi{Ghana}, the official parliamentary language is \ili{English} and nobody can become an MP without speaking \ili{English}. Undoubtedly, no matter how brilliant one is, if s/he cannot communicate in \ili{English} s/he cannot take part in any serious and meaningful political deliberations, and cannot be a parliamentarian. What happens in parliament is an evidence of disenfranchising of the masses by a few small classes of educated elites \citep{Bamgbose2000}. In \isi{Ghana}, all formal and official interactions at the governmental levels are done in \ili{English}. The constitution and public documents are all in \ili{English}. 

Within the same nation, some languages are instrumentalised by governments, and they receive official and national support and recognition. In North Africa, \ili{Arabic} varieties are expanding very fast. \ili{Kiswahili} is established as a nationwide language for communication in \isi{Tanzania}, and is threatening more than 130 other Tanzanian languages (see \citealt[159]{Mous2003}). The same phenomenon prevails in \isi{Botswana} where \ili{Setswana} language is dominating about 30 other languages in the country. 

%2.4.
\subsection{Education and language shift}

Language shift occurs due to the type of \isi{language policy} in education (see \citealt{Agyekum2009}). Where a different language (colonial or African) is the language of education there is a shift from the indigenous languages. In some schools in the urban centres in \isi{Ghana}, children from other indigenous languages have no option but to use, study and pass in the local language. For example, all pupils who attend school in the \ili{Ashanti} region study \ili{Asante} \ili{Twi} for their Basic Education Certificate Examination (BECE). In the schools in Accra, the capital city of \isi{Ghana}, pupils are exposed to only three languages, namely \ili{Twi}, \ili{Ewe} and Ga. Here too, the \ili{Twi} language dominates and most children from the three northern territories are forced to study \ili{Twi}. This is \textit{subtractive bilingualism} and can lead to loss of self-confidence and lower achievements (see \citealt[9]{Austin2011}). 

In \isi{Ghana}, one cannot enter any high school or tertiary institution without passing \ili{English} language. Many families thus forbid their children to learn or communicate in the Ghanaian languages even at home so as to be fluent in \ili{English}.  Most private basic schools prohibit their pupils from speaking any of the Ghanaian languages apart from having them as courses on the timetable. Pupils who flout this rule are punished. Teachers hang placards around pupils’ necks reading, “I have spoken vernacular”, or “Vernacular is prohibited”. These trample on the pupils’ linguistic human rights and freedom of speech.\footnote{\citet[285]{Adegbija2001} records that “Ex-colonial languages have been functionally placed on a pedestal, being attitudinally extolled and being seen as inevitable both at the individual and societal levels.” He further stated that “In \isi{Nigeria}, \isi{Ghana}, Uganda, \isi{Kenya} and many other African countries colonized by the British, for instance, to be educated is virtually synonymous with knowing and being able to use \ili{English}.”}{ }

%2.5.
\subsection{Media and language shift}

Linguistic imperialism goes with media imperialism. Many news agencies in Africa carry their information mostly in \ili{English}, \ili{French} or \ili{Portuguese}. Politically, the world's information channels are dominated by these few languages, and to follow the global trends of information, one needs to know one or more of these languages.\footnote{Unequal benefits in language also lead to cultural imperialism where one society spreads its culture unto others such that they will throw away their cultures. Most African countries have fallen into this trap.}{ }Most programmes, classified advertisements on jobs, funeral announcement, conferences and other important announcements in the print media, radio and TV are mainly in the colonial languages. 

There is a \isi{social stratification} based on the degree of access to communication where speakers of the indigenous languages are not able to use their own \textit{languages in broadcasting}. They are compelled to use the dominant languages, and this shift can lead to \isi{language endangerment} and marginalisation of the masses. In \isi{Ghana}, after independence, there were a few newspapers in the local languages, including, Nkwantabisa in \ili{Asante} \ili{Twi} and \ili{Ewe}. Unfortunately, there is not a single newspaper printed in any Ghanaian language now, The Daily Graphic, the most widely circulated Ghanaian newspaper tried a pull up in its Tuesday edition and captioned it \textit{\isi{Ghana} }\textit{Ns\textipa{E}mpa} ‘\isi{Ghana}’s Good News’ but it did not go well so they have stopped. Until the proliferation of many FM radio stations that broadcast in the local languages, many people who listened to \isi{Ghana} Broadcasting Corporation (GBC), BBC, and VOA were elites. The current use of indigenous languages in radio and TV is a situation of language \isi{decolonisation} where the media shields the local languages from \isi{linguistic imperialism}. In Accra most FM radio stations broadcast in \ili{Akan}, and the local language Ga, was succumbing to the \ili{Akan} language until the government established Ogbonu FM that broadcasts only in Ga to shield the Ga language. 

%2.6.
\subsection{Religion and language shift}

Every religion has a political undertone especially when it is a foreign religion. It can change the ideology and attitude of the adherents. Adherents to new religious faiths must learn the language associated with it, and shift from their indigenous languages during worship \citep[447--448]{Brenzinger2009}. For example, \ili{Arabic}, \ili{Hausa} and Dyula languages have spread in West Africa along with Islam. All good Moslems, especially the Sheiks, are expected to be proficient and knowledgeable in the \ili{Arabic} language and culture.\footnote{According to the literature, \ili{Arabic} was strongly attached to the widespread of Islam because people firmly believed that it was very difficult if not impossible to translate the Quran into another language (see \citealt[208]{SavilleTroike1989}). This myth of the intranslatability of the Quran has now been broken since there are \ili{English} and \ili{Akan} versions of the Quran.}  

%2.7.
\subsection{Economic occupational and migratory language shift}

Economic and occupational survival are crucial factors in language learning and shift either from a minority to a majority language or vice versa (see \citealt[194]{Salzmann2004}, \citealt[58]{Holmes2001}).\footnote{\citet[485--486]{Philips2006} asserts that ‘The political economic position of a group determines its attitude towards the codes in a group’s multilingual repertoire, the group’s code choices, and the ultimate survival of the codes being spoken. The inequality of languages originates in economic equality.’} Africans in Europe, Asia and American learn the host language so as to be employed. Theoretically, notwithstanding how strongly people are attached to their languages and cultures, the socio-economic pressures in modern global world make it difficult for them to glue themselves to ancestral languages and culture, they have to shift to another language (see \citealt[190]{Mufwene2002}). Lack of language and cultural practices gradually lead to \isi{language attrition} and death.

Economic factors can raise a language to the status of a \isi{lingua franca} and many non-native speakers will have the urge and zeal to learn and speak it. This supports an assertion by \citet[130]{Mufwene2006} that “A particular language can be preferred because it functions both as vernacular and as a \isi{lingua franca}, or because it is associated with money- making or better paying jobs, or because it affords its speakers a better social status, or because it is more widely spoken demographically and/or geographically.” 

\citet[286]{Adegbija2001} asserts that “The more powerful and functionally the \isi{dominant language}, the more pressing its attraction and pull and the greater the tendency to shift towards it given the fact that the pressure for social vertical mobility is virtually irresistible.” In view of this notion, many dominant languages in Africa are replacing the smaller ones. Examples are the spread of the \ili{Somali} language in \ili{Somali}, Bamana or \ili{Jula} in \ili{Mali}, and Burkina Faso, \ili{Hausa} in \isi{Nigeria} and \isi{Niger}, Yoruba and \ili{Igbo} in \isi{Nigeria}, \ili{Lingala} in \isi{Congo}, and \ili{Wolof} in \isi{Senegal}, \ili{Kikuyu} in \isi{Kenya}, \ili{Akan} and \ili{Ewe} in \isi{Ghana}. These replacement are normally based on urbanisation, economic benefits and prestige (see \citealt[160]{Mous2003}, \citealt[286]{Adegbija2001})

In \isi{Ghana}, the \ili{Akan} language is dominating the other indigenous languages very fast.\footnote{In \isi{Ghana}, \ili{Akan} is spoken in six out of the ten regions and by 47.5\% of the entire population as native speakers and also by 44\% of the remaining Ghanaians as non-native speakers.}{ } Non-\ili{Akan} speakers naturally find it more advantageous to learn \ili{Akan} so as to operate easily and extensively in \isi{Ghana}. It is used mostly in the non-\ili{Akan} cities and \isi{major} towns like Accra, Bolgatanga, Yendi, Tamale, Navrongo, Wa, as the preferred \isi{lingua franca} in religion, trade and commerce, music and the arts, artisanship and many aspects of the society. The same situation prevails in the Volta region where most of the \isi{Togo} Mountain minority languages are being dominated by the \ili{Ewe} language. In the \ili{Akan} and \ili{Ewe} examples, we can argue that the \isi{language shift} from the minority languages is not caused by Indo-European languages but by African languages that the groups choose based on socio-political benefits (see \citealt{Mous2003}, \citealt{Brenzinger1998}, \citealt{Lüpke2013}).

Social mobility, migration and settlement patterns result in \isi{language contact} and \isi{language shift} (see \citealt{Agyekum2009}). Some people move to urban centres and abandon their languages for the ‘so-called’ prestigious ones. However, large groups of immigrants often manage to preserve their languages. 


% \section{\biberror{title missing}}
%3.1.
\section{Language decolonisation}

The term \textit{language decolonisation} is derived from colonisation and \isi{decolonisation}. \citet[534]{Jaffe2009} states that 

\begin{quote}
Linguistic \isi{decolonisation} (LD) describes both the actions taken in postcolonial contexts to undo the social, political, and cultural effects of the dominance of colonial languages and a philosophical challenge to the Western language ideologies that underpinned the colonial project and that have persisted in the postcolonial period.
\end{quote}

%3.2.
\subsection{Strategies and tools for language decolonisation}

Language decolonisers employ documentary approaches, sociological, ideological and pragmatic concepts to fight against \isi{language shift}, \isi{endangerment}, extinction and death. In language \isi{decolonisation}, states and societies take nationalist interventions, efforts and action plans to legitimate their languages and identities that have been suppressed by \isi{linguistic imperialism}. 

They use multifaceted avenues and tools for \isi{language revitalization}, and maintenance. These include educational curriculum, \isi{literacy} and awareness of \isi{language death}, media, especially radio, \isi{linguistic pluralism}, \isi{language documentation} and preservation. Language \isi{decolonisation}, documentary projects and interventions redress the linguistic inequality, manipulation, repression and cultural oppression created by the replacement of the functions of their language by the majority and colonial languages. 


% \subsection{\biberror{title missing}}
%3.3.1.
\subsection{Language decolonisation through radio, TV and pop-culture}

The media can be a strong tool for language \isi{decolonisation}; it serves as a platform for the use, documentation and development of the endangered languages.\footnote{\citet[95]{Spitulnik2001} records that “there are several cases in which phrases and discourse styles are extracted from radio broadcasting and then recycled and reanimated in everyday usage outside of the contexts of radio listening”.} To \citet[447]{Moriarty2011} {\textquotedbl}Endangered media and pop-culture can help raise the status of the relevant language, aid corpus planning through the dissemination of new terminology, and encourage \isi{language acquisition} by increasing language exposure in both the public and private domains.” In \isi{Ghana}, the mass media, especially radio record most of their programmes in the Ghanaian languages. Mass media serves as a mechanism for the storage of expressions, reservoirs and reference points for the circulation of words, phrases and discourse, proverbs and other aspects of Ghanaian language and popular culture (see \citealt[96]{Spitulnik2001}). The media as a tool for \isi{language decolonization} and \isi{language maintenance} serves as an unplanned and surest area where language changes and practical language policies can be employed spontaneously by the speech community (see \citealt{Agyekum2010radio}, \citealt[158]{Derhemi2002b}, \citealt[447]{Moriarty2011}). 

%3.3.2.
\subsection{Recycling, recontextualisation and reinterpretation of media discourse}

Oral media discourse in Africa is an essential aspect of the people’s life and a new social interconnection across socio-structural, ethnic, religious, professional and party groupings. African societies have public and private FM stations that have adopted African languages \isi{policy}. Some do 50\%, 70\% and even 100\% African languages programmes. In \isi{Ghana}, examples are Ogbonu FM in Tema (Ga), Ada FM (Ada), Peace FM. Nhyira, Adom, Oman, and FM, (\ili{Akan}). These policies are means of revitalising, maintaining, preserving and documenting their languages as part of the process of language \isi{decolonisation}. 

%3.3.3.
\subsection{Linguistic impact}

Linguistically, African language programmes on radio and TV have immensely contributed  in African societies. Listeners learn a lot of things on language such as new vocabularies for modern concepts in politics, medicine, health, education, administration, economics and science from African programmes. They get new terms, idiomatic expressions, etymologies of words, proverbs, archaisms, appellations, etc. of indigenous languages (see \citealt{Agyekum2010radio} on radio). 

One of the \isi{major} functions of the African language programmes on \textsc{radio and TV talk-shows} is language modernization, development and elaboration of terms to cater for most aspects of human life. Radio is one of the most powerful tools in the dissemination, interpretation and \isi{recontextualisation} of discourse (see \citealt{Agyekum2000}, \citeyear{Agyekum2010radio}). Coined terms and phrases commonly used on African language radio programmes are picked up by the people and accepted for use outside radio. Below are some \ili{Akan} (\isi{Ghana}) terms popularly used, they include: 

\ea\label{ex:agyekum:1}
{dumsɔ is from dum, ‘to switch off’ and sɔ, ‘switch on’, implying ‘power outage’.\footnote{Dumsɔ, ‘power outage’ has gained international recognition and has been used in Wikipedia and in Google.}}\\
\z

\ea\label{ex:agyekum:2}
{\textit{ɔ}\textit{mampanin}, is made up of \textit{ɔman,} ‘state’, \textit{panin,} ‘elder’ and hence ‘president’. }\\
\z

\ea\label{ex:agyekum:3}
{\textit{ɔ}\textit{soafo}\textit{ɔ}\textit{,} ‘minister’ is made up of \textit{soa}, ‘to carry’ and \isi{agentive} prefix \textit{ɔ{}- }and suffix \textit{{}-foɔ.} The minister is the carrier of the “load” and responsibility of the ministry.}\\
\z

\ea\label{ex:agyekum:4}
{\textit{mmarahy}\textsc{\textit{ɛ}}\textit{badwa,} ‘parliament’, the components are mmara, ‘laws’, \textit{hy}\textsc{\textit{ɛ}}\textit{,} ‘to fix’ and \textit{badwa}, a gathering of people who are tasked to make laws.}\\
\z

\ea\label{ex:agyekum:5}
{\textit{ankor}\textsc{\textit{ɛ}}\textit{ankor}\textsc{\textit{ɛ}}\textit{ ky}\textsc{\textit{ɛ}}\textit{p}\textsc{\textit{ɛ}}\textit{n, }for ‘human rights’ the adjective \textit{kor\textipa{E}} means singular and the derived \isi{compound} \textit{ankor}\textsc{\textit{ɛ}}\textit{ankor}\textsc{\textit{ɛ}} means individuals, \textit{ky}\textsc{\textit{ɛ}}\textit{p}\textsc{\textit{ɛ}}\textit{n i}s made up of \textit{ky}\textsc{\textit{ɛ}}\textit{,} ‘to share’ and \textit{p}\textsc{\textit{ɛ}}\textit{n} means level or equal. The composite term thus means equal distribution of entities to people (see \citealt{Agyekum2006} for other terms).}\\
\z

African \isi{oral literature} genres appear on radio because they are less available in urbanized areas where the heterogeneous nature of the society is making people lose their \isi{cultural heritage}. The radio and TV are becoming repositories (and innovators) for African languages especially in areas of orality. Radio culture is gradually replacing the traditional “village” oral traditions that were provided by the elders. Until the advent of the missionaries, the African languages were purely oral. Storytelling and \isi{proverb} sections that were earlier on provided by the elders in the villages are now done effectively on African radio and TV programmes for a wider audience (see \citealt{Agyekum2000}). The air waves allow us to reach “the villages”, the radio is wider than the reach of the village \isi{oral literature} sessions.

%3.4.
\subsection{Linguistic pluralism and official multilingualism as language decolonisation}

One of the surest theoretical approaches to \isi{language maintenance} and decolonization in African multilingual societies is the \isi{policy} of \textit{\isi{linguistic pluralism} and official multilingualism} whereby various languages are decolonised and liberated from \isi{language imperialism} and \isi{endangerment}. Linguistic pluralism or official \isi{multilingualism} is a system that promotes the coexistence of different languages and allows them to operate on equitable basis. The same opportunity for development and usage are given to more languages irrespective of their number of speakers. Linguistic pluralism is a democratic way of dealing with linguistic varieties in a multilingual nation. 

\isi{Eritrea} is one of the recognised linguistic pluralistic nations in Africa that gives all the nine indigenous languages the same opportunity in the mass media and in schools.\footnote{The nine languages used in \isi{Eritrea} are 
\href{http://en.wikipedia.org/wiki/Afar_language}{Afar}, \href{http://en.wikipedia.org/wiki/Arabic_language}{Arabic}, 
\href{http://en.wikipedia.org/wiki/Beja_language}{Beja}, 
\href{http://en.wikipedia.org/wiki/Blin_language}{Blin}, 
\href{http://en.wikipedia.org/wiki/Kunama_language}{Kunama}, 
\href{http://en.wikipedia.org/wiki/Nara_language}{Nara}, 
\href{http://en.wikipedia.org/wiki/Saho_language}{Saho}, 
\href{http://en.wikipedia.org/wiki/Tigre_language}{Tigre} and 
\href{http://en.wikipedia.org/wiki/Tigrinya_language}{Tigrinya}. 
\isi{Ghana} embarks on partial \isi{linguistic pluralism}, and offers eleven languages in schools at the Junior High School (JHS), Senior High School (SHS) and universities. } Societies that practise \textit{official multilingualism} include Canada (\ili{French}, \ili{English}) Switzerland (\ili{French}, Swiss, and \ili{German}) and South Africa (11 languages).\footnote{The South Africa 1996 Constitution has a new \isi{language policy} that accords official status to eleven languages also referred to as ‘de facto’ national languages spoken by \isi{major} populations within the country. The aim is to avoid linguistic conflicts. 
}

The most obvious problem with \isi{linguistic pluralism} is economic, we use more money for printing books, training personnel, employing examiners, etc. A cost benefit analysis will prove that money spent in operating \isi{linguistic pluralism} would be far less than money spent in peacekeeping if conflicts and wars emerge out of linguistic imposition and imperialism.

%3.5.
\subsection{Language resistance and maintenance}
\largerpage
Let us continue our discussion with \isi{language resistance} and maintenance as tools of language \isi{decolonisation}. It is possible for a community to undertake \isi{language revitalization} and maintenance during the time that the shift is taking place. This happens in a situation where the speakers of the language undergoing a shift are loyal to their language and are bent on maintaining it. \citet{Nahir2003} states that 

\begin{quote}
Language maintenance is the preservation of the use of a group’s native language, as a first or even as a \isi{second language}, where political, social, economic, educational, or other pressures threaten or cause (or are perceived to threaten or cause) a decline in the status of the language as a means of communication, a cultural medium, or a symbol of group or national identity. \citep[439]{Nahir2003}.
\end{quote}

\citet[673]{Lewis2013} also posits “Language maintenance is the effort to arrest and reverse the process of \isi{language shift}: an effort is made to ensure that a vulnerable language does not decline and eventually disappear, but rather it continues to be spoken by a sustainable community of people.” Language maintenance is the situation where a language group that is bound to shift to other languages and become endangered under linguistic imperialistic circumstances, rather holds to their language and expands its \isi{functional load} (see \citealt[157]{Coulmas2005}). 

Most Africans adhere to various practical ways to maintain their languages and culture. Culture and language are inextricably interwoven. The loss of a language through \isi{language shift} and \isi{endangerment} logically results in the loss or reduction of cultural practices \citep[153]{Derhemi2002b}. These include \isi{oral literature} genres like proverbs, folktales, myths, legends, historical narratives, folk songs and poetry (see \citealt{Crystal2000}). There are good reasons to protect languages and cultures from dying. 

Speakers of a particular language group themselves can maintain their language instead of shifting to the \isi{dominant language}. An extended form of this type of \isi{language maintenance} is found in African communities in Europe and North America. African ethnic groups congregate from different parts of the cities in which they live to meet regularly and chat in their languages, eat African food, and wear African clothes. They have African musical groups that meet occasionally to perform. They have chiefs, queens and sub-chiefs, and they continue to practice their cultures and languages. They have established African churches where the worship is conducted in African languages and they sometimes invite African pastors from Africa to preach. These types of gatherings help to promote African languages and cultures in the developed world.


\section{Language documentation}
\largerpage[1.5]
This section discusses basic concepts, techniques, and challenges of \isi{language documentation}.

%4.1.
\subsection{Basic concepts about language documentation and preservation}

Let us finally look at decolonising language through documentation and preservation of aspects of the language including the phonology, syntax, semantics, ethnography, culture, and \isi{oral literature}. 

To salvage situations of \isi{language death} and \isi{endangerment}, linguists counteract \isi{language death} through \textit{\isi{language preservation}, revitalization }and \textit{documentation. }The linguistic documentation of dying languages is sometimes labelled as \textbf{“}salvage linguistics” \citep[257]{Craig1997}. The documentation of language and oral art is the \isi{major} technique that acts as the catalyst for its preservation and \isi{language decolonization}, because it is a way for putting out materials in the form of books, CD Rom, and all kinds of electronic forms. These sources can be consulted for specific aspects of language teaching, learning and research. 

Language documentation is a proactive process and involves actions that are meant to record, maintain and preserve a language for the future; it is also geared towards protecting it from \isi{language shift} and death (see \citealt{Agyekum2012}). From this standpoint, \isi{language documentation} precedes preservation.\footnote{\citet[3140]{Trilsbeek2006} identify three agents in the archiving of language and these are (a) the depositors, (b) the users, and (c) the archivists. The depositors record, and create different sorts of material and hand these over to the archivists (documentation). The users are those who use the materials, and the archivists solve the long-term preservation problems (preservation).}{ }To \citet[1]{Himmelmann2006}, “Language documentation is a lasting, multipurpose record of language.” The researcher should rely on the \textit{linguistic knowledge }of the native speakers on flora and fauna, kinship systems, artefacts, food items, religion, narratives and \isi{oral literature} genres, etc. If native speakers are actively involved in the creation of the document, the researcher gets firsthand information about the language, and this increases the authenticity of the information for documentation and preservation. 

Before independence, \isi{language documentation} in \isi{Ghana} was carried out by missionaries like the Presbyterian and the Methodist who also translated the Bible into \ili{Twi}, Fante, \ili{Ewe} and Ga. Since independence, documentation in the country has been undertaken by the Bureau of Ghanaian Languages, \isi{Ghana} Publishing Corporation and \isi{Ghana} Institute of Literacy and Bible Translation. The Linguistic and Ghanaian language departments of the universities research into Ghanaian languages by producing long essays, theses, dissertations, primers, dictionaries and journal articles (see \citealt{Agyekum2012}). 

%4.2.
\subsection{Techniques in language documentation}

The \isi{major} strategies of \isi{language documentation} are the recording, processing, preservation and dissemination of the primary data (see \citealt[159]{Woodbury2011}). It involves audio and/or video recording of all communicative events such as folktales, myths, proverbs, folksongs, dirges, appellations, ordinary conversations and all other observable linguistic behaviour. In documentation, we collect the language materials and put them into \isi{archival database}, and disseminate the information for public use (see \citealt[106]{Salffner2006}). The scope of \isi{language documentation} covers a large set of primary data that provides strong evidence of language use in their natural sociocultural settings such as funerals, traditional rituals, greetings, requests, apology, thanking, political discourse and chieftaincy (see \citealt{Agyekum2012}). The language documenter elicits information from people and takes notes from earlier documents like literature books, history and religion or recorded court proceedings and notes from the Bible where they exist.

Language documentation should include many varied records and everything that is related to language. To \citet[3]{Himmelmann2006}, “The goal of \isi{language documentation} is to create a record of language in the sense of a \textit{comprehensive corpus of the primary data} which leaves nothing to be desired by later generations wanting to explore whatever aspect of the language they are interested in.” No matter how thoroughly we record and document it, there would be areas that are undocumented. 

In specific subject \isi{language documentation}, the researcher could set some limits based on his focus and needs. There is the danger of producing what \citet[4]{Himmelmann2006} calls \textit{“data graveyards”.} This refers to large heaps of data with little or no use to anyone. However, if we collect very limited data, the documented information may not reflect the real nature and state of the language under study (see \citealt{Agyekum2012}). 

\largerpage[-2]
Data is of no importance if it is not well formatted and catalogued for easy accessibility (see \citealt[26]{Agyekum2012}). Any data should be accessible to the compilers of the document, and also be user friendly and accessible to a broad range of users including children, researchers, and foreigners who want to learn the language in future.\footnote{The documents should provide information about; participants, purpose, setting (when and where the data was recorded), contents, structure and quality of the data, general information about the speech community and the language, the data collection methods, links and references to other resources (books and articles), \citep[11--12]{Himmelmann2006}. The materials should be stored in hardcopy, softcopy, backup, pen drives, CDs, DVD, CD Roms, e-mail accounts, etc.}

%4.3.
\subsection{Factors militating against language decolonisation and documentation:}

Most linguistic decolonisers do not have the requisite resources and materials needed to develop and raise their languages to the levels that can replace the status and functions of the colonial languages in all domains. The colonial masters are politically and economically powerful, and their languages resist the challenges of decolonization, and they embark on official monolingualism (see \citealt{Lewis2013}). \citet[13]{Fishman2001} states that “the resources available to threatened languages are often quite meager and constantly fewer than those available to their Big Brother rivals and competitors…. Threatened languages often have no \textit{outside support of any optimal significance} to fall back upon.” For instance, at the University of \isi{Ghana}, most of the funds for research are granted to the sciences and social sciences and not to Linguistics, Languages and Performing arts.

Another reason for the resistance to Language Decolonisation is that language planning policies favour foreign languages and majority national languages; this phenom\-e\-non makes language \isi{decolonisation} difficult. This unfortunate situation prevails in  \is{Ghana}{Gha\-na} where the official \isi{language policy} states that the medium of instruction for the first three years should be the indigenous Ghanaian languages, but the “international and preparatory” schools in the urban areas rather use the \ili{English} language as the medium from day one especially when the pupils come from diverse linguistic backgrounds.

The development of the national and the minority languages for education and other modern domains creates a division between the pure and mixed codes of the various languages (see \citealt[534]{Jaffe2009}). If the minority and threatened languages are only taught as school subjects and not used in homes there is a disconnection, and that retards the progress of the language. This is what normally happens in the urban centres, where the \isi{minority language} becomes a \isi{second language} (see also \citealt[14--15]{Fishman2001}). The use of the colonial languages brings about elitism and \isi{social stratification} that help the elite to lord over the non-elite. They therefore want to do everything possible to thwart the efforts of the linguistic decolonisers.\footnote{\citet[535]{Jaffe2009} concludes that “Postcolonial linguistic agents are often faced with a double-bind: if they use the colonial language, they are seen as traitors to their cultural/ethnic group; if they use the dominated language, their \isi{voice} has a more limited power and reach.” }

Some scholars argue that \isi{language preservation} and documentation are very costly and not cost effective. \citet[162, 191]{Mufwene2002} asserts that \isi{language development} should take a natural path. The changes occurring in linguistic behaviour of speakers are simply an adaptation to changes in the socio-economic conditions of the speech communities driven by interests related to costs and profits that come along with language use, and therefore must be respected. Theoretically, \isi{language shift}, \isi{endangerment} and death are all part of language evolution and must be allowed to take their natural path based on human behaviour. 

Generally, if speakers of an \isi{endangered language} have more pressing socio-economic concerns and prospects to think about, they are less worried about the fates of their languages and how to salvage them from \isi{linguistic imperialism}. Despite this conundrum facing Language \isi{decolonisation}, it is prudent to forge ahead since \isi{linguistic imperialism} tramples on peoples’ “linguistic human rights”, and \isi{language endangerment} is a machine that drives languages into the morgue.

Problems that militate against research, documentation, and publication of indigenous languages include lack of cooperation from the informants, especially on issues relating to the secret elements of oral art, filming materials or the performance itself\textbf{. }Translating \isi{oral literature} materials is not an easy task, many words and concepts are archaic and obsolete, some flora, fauna and particular names are untranslatable. 

Many African universities have a negative attitude towards teaching and research in African languages, \isi{oral literature} and cultural studies, and will not provide funds to promote the development of African languages. There are difficulties in finding outlets and funds for publication of research materials in \isi{oral literature}, \isi{lexicology} and other aspects of African languages. 

%4.4.
\subsection{Recommendations for the development of African languages }

All languages should be given the opportunity to survive so as to maintain their \isi{cultural heritage} (see also \citealt[305--306]{Adegbija2001}). Based on the points raised in this paper, we make the following recommendations and conclusions for the promotion, development and \isi{decolonisation} of African languages.


\begin{itemize}
\item 
The provision of vocabulary and appropriate terminology for the teaching and learning of African languages should be made an important aspect of \isi{language development}. To cope with language modernisation and elaboration, \isi{lexicology} and terminology should be integral parts of the language curriculum at the teachers’ training colleges, language centres and the linguistics departments of our universities. This will help us to produce word lists, terms and special dictionaries, glossary books on all aspects of African linguistics, \ili{English}- African reference books for African language students, broadcasters, translators, etc. 

\item 
We must develop and elevate the status of the African languages and literatures to be used in education to teach our languages, mathematics, science, and technology in all our schools, universities and technical institutions. 

\item 
At the training colleges, the teaching and research into African languages should be intensified to produce well-trained knowledgeable and adequately qualified teachers to teach our African languages (see \citealt[9]{Prah1993}, \citealt[305]{Adegbija2001}). 

\item 
More books and articles should be written on African languages. The more we document and preserve the language, the better we are able to revitalise, maintain and decolonize our African languages from \isi{linguistic imperialism}. We could translate some of the African, Caribbean, and Western literature into the African languages to expand our textbooks and supplementary readers (see \citealt{Agyekum2003}). 

\item 
A branch of \isi{language planning} and modernisation should be established within the Academy of Arts and Sciences in African countries. Our departments of languages at the universities should teach courses on \isi{language planning} and policies especially, MTE. 

\item 
African languages, linguistics and literature institutes should establish language planning branches. We should also have national interdisciplinary lexicographical and terminological centres to deal with \isi{language documentation} in the various African languages. Africans should have explicit, practicable, implementable and firm language policies and planning on MTE rather than the theory and “skip-hopping policies”.

\item 
We should establish language media; radio, TV and newspapers and popular culture domains that the youth will be involved in and enjoy. The media can shape and change their negative attitudes and ideologies about African indigenous languages.

\item 
African agencies like the AU, ECOWAS and zonal groups should equip Bureaux of African Languages in our countries with personnel, technical know-how, and funds to develop our own languages (see \citealt{Agyekum2012}, \citeyear{Agyekum2003}). The 21\textsuperscript{st} century should be dubbed “Afrolingual Independence and Development”. We should be proactive to develop our mother tongues and decolonise them from \isi{linguistic imperialism}, \isi{language endangerment} and death. The overall result of positive linguistic intervention will help our social, economic, religious, political, scientific, and technological development. 

\end{itemize}
\section{Conclusions}

In this paper, we have seen that \isi{language shift} is part of language and politics and it is found in \isi{language policy}, politics, language of education, religion, media, economics, immigration and settlement patterns. Naturally, a language with limited instrumental utility to catch up with modernisation and globalisation is replaced with the one that serves the current needs of the people. The international exchanges of knowledge and world trade are conducted in only a few world languages. We opined that \isi{language imperialism} occurs in two folds (a) conscientisation of a people to gradually shift from their language, and (2) shift of a language for socio-economic and political benefits.

We paid attention to Linguistic \isi{decolonisation} as a process where a society reinstates its language status and reverses \isi{language shift}. Decolonising agencies in African redeem and maintain their languages across geographical frontiers. Unfortunately, while language decolonisers work very hard, globalisation, language attitude by some elites, modernisation and technological influence, borrowing and diffusion hamper their progress. Problems relating to the collection, documentation and publication include the uncooperative and negative attitude of informants and speakers, lack of funds, problems of translation, and difficulty in finding publishers (see \citealt[5]{Bamgbose2011}). 

Theoretically, every language can become a vehicle of modern civilization to meet the demands of her people. The non-use of a language in both informal and formal set ups contribute to \isi{language shift}, \isi{endangerment}, decay and loss, and restricts its functional usage, but the documentation of a language is a big step in the process of \isi{language survival} and restoration (see \citealt[256]{Derhemi2002a}). 

The survival of every language theoretically depends on whether it is appropriate to be used for all contexts of communication identified by the society such as marketing, occupation, religion, music, etc. A language begins to lose its prestige and eventually decays if its \isi{functional load} is restricted to only informal usage. Our languages should be integrated into ICT. It is only when our languages are able to cope with most of our societal needs that Africa’s independence can be said to be complete. 

Finally, the paper has argued that if we develop African languages, it will have a corresponding impact on African culture, art, politics, economics, agriculture, industrialisation, technology, medicine, primary health, commerce and trade, environmental studies, media and education. If the modern ideas and information are disseminated in the indigenous African languages, their impact will be greater and help the majority of the masses to understand them better. \textit{Political independence without linguistic independence is partial independence.} In such a situation, our pace in modernisation will be slow, and we will still be under \isi{linguistic imperialism} and \isi{language endangerment} \citep{Agyekum2003}.

 
{\sloppy
\printbibliography[heading=subbibliography,notkeyword=this]
}
\end{document}
